%        File: poincare_inequality.tex
%     Created: Mon Feb 22 09:00 AM 2021 W EST
% Last Change: Mon Feb 22 09:00 AM 2021 W EST
%
\documentclass[a4paper]{article}

\usepackage{amsmath,amsthm,amssymb}
\usepackage{times}
\usepackage{enumerate}
\usepackage{graphicx}
\usepackage{dsfont}
\usepackage{subfig}

\RequirePackage[numbers]{natbib}
\RequirePackage[colorlinks,citecolor=blue,linkcolor=blue,urlcolor=blue]{hyperref}

\pagestyle{myheadings}
\def\titlerunning#1{\gdef\titrun{#1}}
\makeatletter
\def\author#1{\gdef\autrun{\def\and{\unskip, }#1}\gdef\@author{#1}}
\def\address#1{{\def\and{\\\hspace*{18pt}}\renewcommand{\thefootnote}{}%
  \footnote {#1}}%
\markboth{\autrun}{\titrun}}
\makeatother
\def\email#1{e-mail: #1}
\def\subjclass#1{{\renewcommand{\thefootnote}{}%
\footnote{\emph{Mathematics Subject Classification (2020):} #1}}}
\def\keywords#1{\par\medskip
\noindent\textbf{Keywords.} #1}

\newtheorem{theorem}{Theorem}[section]
\newtheorem{corollary}[theorem]{Corollary}
\newtheorem{lemma}[theorem]{Lemma}
\newtheorem{proposition}[theorem]{Proposition}

\theoremstyle{definition}
\newtheorem{definition}[theorem]{Definition}
\newtheorem{remark}[theorem]{Remark}
\newtheorem{example}[theorem]{Example}

\DeclareMathOperator*{\esssup}{ess\,sup}
\DeclareMathOperator*{\essinf}{ess\,inf}
\DeclareMathOperator{\supp}{supp}
\DeclareMathOperator{\law}{Law}
\DeclareMathOperator{\sgn}{sgn}
\DeclareMathOperator{\divv}{div}
\DeclareMathOperator{\leb}{Leb}
\DeclareMathOperator{\pr}{pr}
\DeclareMathOperator{\Var}{Var}
\DeclareMathOperator{\diag}{diag}
\DeclareMathOperator{\Vol}{Vol}
\DeclareMathOperator{\Leb}{Leb}
\DeclareMathOperator{\Surf}{Surf}

\numberwithin{equation}{section}

\frenchspacing

\textwidth=15cm
\textheight=22cm
\parindent=16pt
\oddsidemargin=0.7cm
\evensidemargin=0.8cm
\topmargin=-0.5cm

\begin{document}

\newcommand{\R}{\mathbb{R}}
\newcommand{\I}{\mathbb{I}}
\newcommand{\Q}{\mathbb{Q}}
\newcommand{\Z}{\mathbb{Z}}
\newcommand{\N}{\mathbb{N}}
\newcommand{\eps}{\varepsilon}
\newcommand{\p}{\mathbb{P}}
\newcommand{\F}{\mathcal{F}}
\newcommand{\e}{\mathcal{E}}
\newcommand{\Oe}{\mathcal{E}^{OU}\!}
\newcommand{\E}{\mathbb{E}}
\newcommand{\mC}{\mathcal{C}}
\newcommand{\B}{\mathcal{B}}
\newcommand{\cdl}{c\`{a}dl\`{a}g }
\newcommand{\D}{\mathrm{D}}
\newcommand{\Li}{L_2^{\uparrow}}
\newcommand{\FC}{\mathcal{FC}}
\newcommand{\OXi}{\Xi^{OU}\!}
\newcommand{\OLambda}{\Lambda^{OU}}
\newcommand{\Deriv}{\mathop D}
\newcommand{\bDeriv}{{\mathop D}^\tau}
\renewcommand{\L}{\mathrm{L}}
\newcommand{\Vitalii}{\tt\color{blue} Vitalii: }
\newcommand{\Max}{\tt\color{red} Max: }
\newcommand{\Victor}{\tt\color{green} Victor: }
\newcommand{\Cint}{K_{\Sigma,\Omega}}
\newcommand{\neumann}{\Sigma_{\operatorname{N}}}
\newcommand{\mus}{\lambda_\Sigma}
\newcommand{\mun}{\lambda_{\operatorname{N}}}
\newcommand{\DomE}{\mathcal {D}}
\newcommand{\Ric}{\operatorname{Ric}}
\newcommand{\id}{\operatorname{id}}

\def\one{\mathbb I}
\def\Bdelta{\Delta^\tau}
\def\bnabla{\nabla^\tau}

\baselineskip=17pt

\title{Spectral gap estimates for Brownian motion on domains with sticky-reflecting boundary diffusion}

\titlerunning{Spectral gap for domains with boundary diffusion}

\author{Vitalii Konarovskyi$^{*\dagger}$, Victor Marx$^*$, and Max von Renesse$^*$}

\date{\today}

\maketitle

\address{$*$ Universit\"{a}t Leipzig, Fakult\"{a}t f\"{u}r Mathematik und Informatik, Augustusplatz 10, 04109 Leipzig, Germany; $\dagger$ Universit\"{a}t Hamburg, Bundesstraße 55, 20146 Hamburg, Germany; Institute of Mathematics of NAS of Ukraine, Tereschenkivska st. 3, 01024 Kiev, Ukraine
konarovskyi@gmail.com, marx@math.uni-leipzig.de, renesse@uni-leipzig.de }

\subjclass{Primary 
26D10, % Inequalities involving derivatives and differential and integral operators
35A23, %Inequalities applied to PDEs involving derivatives, differential and integral operators, or integrals
34K08 %Spectral theory of functional-differential operators
; Secondary 
46E35, %Sobolev spaces and other spaces of “smooth” functions, embedding theorems, trace theorems
53B25, %Local submanifolds
60J60, %Diffusion processes
47D07%Markov semigroups and applications to diffusion processes 
.}

\begin{abstract} Introducing  an interpolation method we estimate the spectral gap for Brownian motion on general domains with sticky-reflecting boundary diffusion associated to the first nontrivial eigenvalue for the Laplace operator with corresponding Wentzell-type boundary condition. In the manifold case our proofs involve  novel applications of the celebrated Reilly formula. \end{abstract}


\section{Introduction and statement of main results}
Brownian motion on smooth domains with sticky-reflecting  diffusion along the boundary has a long history, dating back at least to  Wentzell \cite{MR121855}. As a  prototype consider a diffusion on the closure $\overline \Omega$ of a  smooth domain $ \Omega$ with Feller generator $(\mathcal D(A), A)$ 
\begin{equation}
\label{eq:gendef}
    \begin{aligned} \mathcal D(A) =  \{ f \in C_0(\overline \Omega) \, | \, Af \in C_0(\overline \Omega) \} \\
 A f =  \Delta f \one_{\Omega} + ( \beta  \Bdelta f - \gamma \frac {\partial f}{\partial \nu })\one_{\partial \Omega} \end{aligned}
 \end{equation}
where 
$\frac{\partial}{\partial \nu}$ is the outer normal derivative, $\Bdelta$ is the Laplace-Beltrami operator on the boundary $\partial \Omega$ and $\beta >0, \gamma \in \R$. The case of pure sticky reflection but no diffusion along the boundary corresponds to the regime $\beta =0$; models with $\beta >0$ have appeared recently in
interacting particle systems with singular boundary or zero-range pair interaction \cite{MR4096131,MR2198199,MR2215623,konarovskyi2017reversible,nonnenmacher2018overdamped}.  The first rigorous process constructions on special domains  $\Omega$ were given in  \cite{MR126883,MR929208,MR287612} and were later extended to jump-diffusion processes  on general domains   \cite{MR245085} cf. \cite{MR4176673}. An efficient construction in symmetric cases 
was given by Grothaus and Vo\ss{}hall via Dirichlet forms in \cite{grothaus}. Qualitative regularity properties of the associated semigroups were studied  e.g.\ in \cite{MR4065110}. In this note we address the problem of estimating the spectral gap for such processes, which is a natural question also in algorithmic applications. To our knowledge this question has been considered only for  $\beta=0$ by Kennedy \cite{MR2448584} and Shouman \cite{MR3951758}. However, for $\beta >0$ the properties of the process change significantly,  which is indicated by the fact that the energy form of $A$ now also contains a boundary part and which also constitutes the main difference to the closely related work  \cite{kolesnikov}.

In the sequel we treat the case when $\gamma > 0$ which corresponds to an inward sticky reflection at $\partial \Omega$. Our ansatz to estimate the spectral gap is based on a simple interpolation idea. To this aim assume that $\Omega$ and $\partial \Omega$ have finite (Hausdorff) measure so that we may choose   $\alpha \in (0,1)$ for which
\[ \frac \alpha {1-\alpha} \frac{|\partial \Omega|}{|\Omega|}=\gamma.\]
Introducing $\lambda_\Omega$ and $\lambda_\partial$
as normalized volume and Hausdorff measures on $\Omega$ and $\partial \Omega$ and  setting 
\[
\lambda_\alpha = \alpha \lambda_\Omega + (1-\alpha) \lambda_\partial,\]
we find that  $-A$ is $\lambda_\alpha$-symmetric with first nonzero eigenvalue/spectral gap characterized by the Rayleigh quotient 
\[ \sigma_{\alpha,\beta} = \inf_{\substack{f \in C^1 (\overline \Omega)\\ \Var_{\lambda_\alpha}(f)>0}} \frac{\mathcal E_{\alpha, \beta}(f)}{\Var_{\lambda_\alpha}f},\]
where 
\[ \Var_{\lambda_\alpha}f =\int_\Omega f^2d\lambda_\alpha-\left(\int_\Omega fd\lambda_\alpha\right)^2\]
and 
\[ \mathcal E_{\alpha, \beta}(f) = \alpha \int_{\Omega} \|\nabla f\|^2 d \lambda_\Omega + (1- \alpha)  \int_{\partial \Omega} \beta \|\bnabla f\|^2 d \lambda_\partial, \]
 and $\bnabla$ denotes the tangential derivative operator on $\partial \Omega$.  
 
This representation of $\sigma_{\alpha,\beta}$  formally interpolates between the two extremal cases of the spectral gap for reflecting Brownian motion on $\Omega$ when $\alpha=1$ and for  Brownian motion on the surface $\partial \Omega$ when $\alpha=0$. As our main result,  in  Proposition~\ref{prop:interp_method} we propose a simple method to estimate $\sigma_{\alpha,\beta}$ from below using only $\sigma_0$ and $\sigma_1$  and estimates for certain bulk-boundary interaction terms  which are independent of $\alpha$. The method can lead to quite good results which is  illustrated by the example when $\Omega=B_1 \subset \R^d$ is a $d$-dimensional unit ball. When $d=2$ and $\beta =1$, for instance, it yields the estimate  
\begin{align*}
\sigma_\alpha \geq
\frac{8(1+\alpha)\sigma_{\Omega} }{8 (1-\alpha)\sigma_{\Omega}+16\alpha+3\alpha(1-\alpha)\sigma_{\Omega}}
 \mbox{ with }  \alpha = \frac \gamma {2+\gamma},
\end{align*}
where $\sigma_0\approx 3.39$ is the spectral gap for the Neumann Laplacian on the 2-dimensional unit ball, c.f.\  Section~\ref{subsec:ball_full}. --
In case when $\Omega$ is a $d$-dimensional manifold with Ricci curvature bounded from below by $k_R>0$ and with boundary $\partial \Omega $ whose second fundamental  form $\mathop{\rm II}_{\partial \Omega}$ is bounded from below by $k_2>0$ we obtain  (again with  $\beta =1$, for simplicity) that 
\begin{align*}
    \sigma_\alpha \geq 
\min \left( \frac{d k_R}{
C_\Omega d k_R + (1-\alpha)(d-1) },\frac{d k_R}{C_{\partial\Omega}}
% \cdot
\frac{2(1-\alpha)+\alpha  k_2 C_{\partial\Omega}}{2(1-\alpha) dk_R +\alpha d k_2 k_R C_\Omega +\alpha(1-\alpha)(d-1)k_2 } \right),
\end{align*}
where $C_\Omega$ and $C_{\partial \Omega} $ are the usual (Neumann) Poincaré constants of $\Omega$ and $\partial \Omega$ respectively. To derive this result we combine Escobar's lower bound \cite{escobar} on the first Steklov eigenvalue \cite{MR3662010,Nazarov} of $\Omega$ with a novel estimate on the optimal zero mean trace Poincaré constant  of $\Omega$  \cite{Matculevich_Repin,Nazarov_Repin}, for which we obtain that
\[
\int_\Omega f^ 2  dx \leq \frac {d-1}{d  k_R} \int_\Omega  |\nabla f|^2 , \]
for all  $f\in C^1(\Omega)$ with  $\int_{\partial \Omega} f dS =0$, and which is of independent interest. The proof is based on a novel application of Reilly's formula \cite{reilly} which is also used  for a  complementary lower bound of $\sigma$ independent of the interpolation approach stating that 
\[ \sigma_{\alpha} \geq \min\left(\frac {d k_2 }{3 d -1 }\frac \alpha {1-\alpha} \frac{|\partial \Omega|}{|\Omega|},\frac {d}{d-1}k_R\right),  \]
but which is generally weaker for small values of $\alpha$, c.f.\ Section~\ref{sec:smooth_manifolds}.



The interpolation approach also yields a sufficient condition for the continuity of $\sigma_\alpha$ at $\alpha\in \{0,1\}$, which in general may fail. In Section~\ref{subsection:cont_section} we present sufficient conditions for continuity and discontinuity of $\sigma_\alpha$ at $\{0,1\}$ which hints towards a phase transition in the associated family of variational problems. 

We conclude
with the  discussion of two applications of the method in non-standard or singular situations, c.f.\ Sections~\ref{subsec:partial_ball} and~\ref{subsec:needle}. 


\section{An interpolation approach}
\label{sec:interp_method}

 \subsection{Generalized framework}
 \label{subsec:general_approach}

It will be convenient to work with a slight generalisation of the setup above. To this aim let $\Omega$ be an  open domain   in $\R^d$ or a Riemannian manifold with a piecewise smooth boundary  $\partial \Omega$. 
Let $\Sigma$ be a smooth compact and connected subset of $\partial \Omega$. We denote by $\partial \Sigma$ the boundary of $\Sigma$ in the space $\partial \Omega$, \textit{i.e.} $\partial \Sigma=\Sigma \cap \overline{\partial \Omega \backslash \Sigma}$. 
We consider  two probability measures $\lambda_\Omega$ and $\lambda_\Sigma$ with support $\Omega$ and $\Sigma$, which are absolutely continuous with respect to the Lebesgue and  the Hausdorff measure on $\Omega$ and  $\Sigma$, respectively. 


Let $\Deriv: C^1(\Omega) \mapsto \Gamma^0(\Omega)$ and $\bDeriv: C^1(\partial \Omega) \mapsto \Gamma^0(\partial \Omega)$ denote given first order gradient operators mapping differentiable functions into (tangential) vector fields  on $\Omega$ and  on $\partial \Omega$, respectively, and for $\alpha \in [0,1]$ let \begin{align*}
\lambda_\alpha&:= \alpha \lambda_\Omega + (1-\alpha) \lambda_\Sigma, \\
\mathcal E_\alpha (f) &:= \alpha \int_\Omega \| \Deriv f \|^2 d\lambda_\Omega+ (1-\alpha)  \int_{\Sigma} \| \bDeriv f \|^2 d\lambda_\Sigma, \quad  f \in \DomE_0,
\end{align*}
where $\DomE_0 \subset   \mC^1(\overline \Omega)$ is  dense in $C_0(\Omega)$. We assume that  for $\alpha \in [0,1]$ the quadratic form $(\mathcal E_\alpha,\DomE_0)$ is a pre-Dirichlet form on $L^2(\overline \Omega,\lambda_\alpha)$ whose closure we shall denote by $(\mathcal E_\alpha,\DomE)$, c.f. \cite{grothaus} for details. We wish to estimate from above $\sigma_\alpha^{-1}=C_\alpha$, where $C_\alpha$ is the optimal Poincaré constant given by
\begin{equation} %constant C alpha
  \label{equ_constant_c_alpha}
  C_\alpha:= \sup_{\substack {f \in \DomE_0 \\ \mathcal E_\alpha(f)>0}}  \frac{\Var_{\lambda_\alpha} f}{\mathcal E_\alpha (f)} .
\end{equation}
In the interpolation method presented below  it is assumed that $C_\alpha$ are known or can be estimated at the two extremals  $\alpha\in \{0,1\}$. For instance, when $\Deriv= \nabla$, $\bDeriv = \bnabla$ are the standard gradient resp.\ tangential gradient operators and $\lambda_\Omega$ and $\lambda_\Sigma$ are normalized Lebesgue resp. Hausdorff measures on $\Omega$ and $\Sigma \subset \partial \Omega$,   $C_\Omega:=C_1$ is the optimal Poincaré constant associated to the Laplace operator on $\Omega$ with Neumann boundary conditions, 
whereas  $C_\Sigma:=C_0$ is  the optimal Poincaré constant associated to the Laplace-Beltrami operator on $\Sigma$ with Neumann boundary conditions on $\partial\Sigma$. 

\smallskip 

The following proposition establishes an estimate of $C_\alpha$ in terms of $C_\Omega$ and $C_\Sigma$. 
\begin{proposition}
\label{prop:interp_method}
Assume  there exists constants $\Cint$, $K_1, K_2$ such that for any $f \in \DomE_0$
\begin{align}
\Var_{\lambda_\Sigma} f
\leq  \Cint \int_\Omega \| \Deriv f \|^2 d\lambda_\Omega ,
\label{ineq Cint}
\end{align}
and 
\begin{align}
\left( \int_\Omega f d \lambda_\Omega - \int_\Sigma f d \lambda_\Sigma \right)^2
\leq K_1 \int_\Omega \| \Deriv f \|^2 d\lambda_\Omega + K_2 \int_\Sigma \| \bDeriv f \|^2 d\lambda_\Sigma,
\label{ineq K}
\end{align}
then it holds for any  $\alpha \in (0,1)$, 
\begin{align}
C_\alpha \leq
\max \left(
C_\Omega + (1-\alpha)K_1 ,
\alpha K_2 ,
\frac{(1-\alpha)\Cint C_\Sigma +\alpha C_\Omega C_\Sigma+\alpha(1-\alpha)(\Cint K_2+C_\Sigma K_1)}{(1-\alpha) \Cint + \alpha C_\Sigma} \right).
\label{inequality_interpolation}
\end{align}
\end{proposition}




\begin{proof}
By definition of $C_\Sigma$ and by~\eqref{ineq Cint}, for any $f \in \DomE_0$
\begin{align*}
\Var_{\lambda_\Sigma} f \leq t \Cint \int_\Omega \| \Deriv f \|^2 d\lambda_\Omega + (1-t) C_\Sigma \int_\Sigma \| \bDeriv f \|^2 d\lambda_\Sigma,
\end{align*}
for any $t \in [0,1]$. 
Let $\alpha \in (0,1)$. For any $f  \in \DomE_0$ and any $t \in [0,1]$
\begin{align*}
\Var_{\lambda_\alpha} f 
&= \alpha \Var_{\lambda_\Omega} f + (1-\alpha) \Var_{\lambda_\Sigma} f
+\alpha(1-\alpha) \left( \int_\Omega f d \lambda_\Omega - \int_\Sigma f d \lambda_\Sigma \right)^2 \\
&\leq \left( C_\Omega +\frac{(1-\alpha)t}{\alpha} \Cint + (1-\alpha) K_1 \right) \alpha \int_\Omega \| \Deriv f \|^2 d\lambda_\Omega \\
&\quad+ \left( (1-t) C_\Sigma +  \alpha K_2 \right) (1-\alpha)\int_\Sigma \| \bDeriv f \|^2 d\lambda_\Sigma.
\end{align*}
Therefore,
\begin{align*}
C_\alpha \leq \inf_{t \in [0,1]} \max\left(C_\Omega +\frac{(1-\alpha)t}{\alpha} \Cint + (1-\alpha) K_1 ,(1-t) C_\Sigma +  \alpha K_2 \right).
\end{align*}
For any positive constants $a,b,c,d$, we have
\begin{align*}
\inf_{t \in [0,1]} \max\left(a+bt,c-dt\right)
= 
\begin{cases}
a &\text{if } c-a<0,\\
c-d &\text{if } c-a>b+d,\\
\frac{bc+ad}{b+d} &\text{if } 0\leq c-a \leq b+d.
\end{cases}
\end{align*}
Therefore
\begin{align*}
C_\alpha \leq
\begin{cases}
C_\Omega + (1-\alpha)K_1 &\text{\ \ if }  \alpha K_2-(1-\alpha)K_1 + C_\Sigma-C_\Omega <0,\\
\alpha K_2 &\text{\ \ if }  \alpha K_2-(1-\alpha)K_1 -C_\Omega >  \frac{1-\alpha}{\alpha}\Cint,\\
\frac{(1-\alpha)\Cint C_\Sigma +\alpha C_\Omega C_\Sigma+\alpha(1-\alpha)(\Cint K_2+C_\Sigma K_1)}{(1-\alpha) \Cint + \alpha C_\Sigma} &\begin{array}{r}
     \text{if } 0 \leq \alpha K_2-(1-\alpha)K_1 + C_\Sigma-C_\Omega\quad\quad   \\
     \leq C_\Sigma + \frac{1-\alpha}{\alpha}\Cint.
\end{array}
\end{cases}
\end{align*}
The last term is equivalent to the announced result. 
\end{proof}



\subsection{Continuity of \texorpdfstring{$C_\alpha$}{C alpha}}
\label{subsection:cont_section} 
In general, the  function $\alpha\mapsto C_\alpha$ 
might have discontinuities at  $\alpha\in \{0,1\}$ in which cases an upper bound for $C_\alpha$ which interpolates continuously between $C_0$ and $C_1$ cannot exist. For example, when $\Omega=(0,b) \times (0,1)\subset \R^2$ and $\Sigma=[0,b]\times\{0\}$, straightforward  computations yield \[
  \lim_{ \alpha \to  0 }C_{\alpha}= \max\left\{ C_\Sigma, \frac{4}{ \pi^2 } \right\},
\] 
where $C_\Sigma= \frac{b^2}{\pi^2}$. Hence $\alpha \mapsto C_{\alpha}$  is discontinuous at $\alpha=0$   if and only if $b< 2$.  
-- To generalize this to the framework of Section~\ref{subsec:general_approach}  let 
   $\mC^1_0(\overline{\Omega})=\{f \in \mC^1(\overline{\Omega}) : f=0 \mbox{ on }\Sigma\}$  
and 
  \[
    \tilde{C}_0:=\sup_{\substack {f \in \mC^1_0(\overline \Omega) \\ f \text{ non constant}}}  \frac{\int_\Omega f^2d\lambda_\Omega}{\int_\Omega \| \Deriv f\|^2d\lambda_\Omega}.
  \]
  (If $\Deriv = \nabla$, $\tilde C_0$ is the inverse of the spectral gap for Brownian motion on $\Omega$ with killing on $\Sigma$ and normal reflection at $\partial \Omega \setminus \Sigma$. )
We can then record the following statement as a partial corollary to Proposition~\ref{prop:interp_method}.

\begin{proposition} 
\label{prop:continuity_condition}
In the setting of proposition \ref{prop:interp_method} it holds that  \[
    \varliminf_{ \alpha \to 0 }C_{\alpha}\geq \tilde{C}_0.
  \]
In particular,  if $C_\Sigma < \tilde C_0$, then $\alpha \mapsto C_\alpha$ is discontinuous at $\alpha=0$. Conversely, if $C_\Sigma \geq C_\Omega+K_1$ then $\alpha \mapsto C_\alpha$ is continuous at 0. If $C_{\Omega} \geq K_2$ continuity at 1 holds.
\end{proposition}

\begin{proof} 
To prove the second statement, take a non constant function $g \in \mC^1_{0}(\overline\Omega)$ and estimate 
  \begin{align*}
    \varliminf_{ \alpha \to 0 }C_{\alpha}&= \varliminf_{ \alpha \to 0 }\sup_{\substack {f \in \mC^1(\overline \Omega) \\ f \text{ non constant}}}  \frac{\Var_{\lambda_\alpha} f}{\e_\alpha (f)}\geq \varliminf_{ \alpha \to 0 }\frac{\Var_{\lambda_\alpha} g}{\e_\alpha (g)}\\
    &= \varliminf_{ \alpha \to 0 } \frac{ \alpha\Var_{\lambda_{\Omega}}g+(1-\alpha)\Var_{\lambda_{\Sigma}}g+\alpha(1-\alpha)\left( \int_{ \Omega }   gd \lambda_{\Omega}-\int_{ \Sigma }   gd \lambda_{\Sigma}   \right)^2 }{ \alpha \int_{ \Omega }   \|\Deriv g\|^2 d\lambda_{\Omega}+(1-\alpha)\int_{ \Sigma}   \|\bDeriv g\|^2 d \lambda_{\Sigma }   }.
  \end{align*}
 Since $g=0$ on $\Sigma$, we obtain
  \begin{align*}
    \varliminf_{ \alpha \to 0 }C_{\alpha}&\geq \varliminf_{ \alpha \to 0 } \frac{ \alpha\Var_{\lambda_{\Omega}}g+\alpha(1-\alpha)\left( \int_{ \Omega }   gd \lambda_{\Omega}  \right)^2 }{ \alpha \int_{ \Omega }   \|\Deriv g\|^2 d \lambda_{\Omega}  } = \frac{ \int_\Omega g^2 d\lambda_\Omega }{ \int_{ \Omega }   \|\Deriv g\|^2 d \lambda_{\Omega}  } .
  \end{align*}
Taking the supremum over $g \in \mC^1_0(\overline\Omega)$  yields the first statement. 
  
To prove the second assertion  note that $ \alpha \mapsto C_\alpha$ is the pointwise supremum of a family of continuous functions and therefore lower semi continuous. 
Thus $C_\Sigma=C_0 \leq \varliminf_{\alpha \to 0} C_\alpha$. 
If $C_\Sigma \geq C_\Omega+K_1$, the r.h.s. of inequality~\eqref{inequality_interpolation} converges to $C_\Sigma$ as $\alpha$ goes to 0, which implies that  $ \varlimsup_{\alpha \to 0} C_\alpha  \leq C_\Sigma$. Similarly, if $C_{\Omega} \geq K_2$, the r.h.s. of~\eqref{inequality_interpolation} converges to $C_\Omega$ as $\alpha$ goes \end{proof}



\begin{remark}  For smooth enough boundary 
%   (c.f.\ Section~\ref{subsec:needle}), 
the constant $K_2$ can always be taken equal to zero, hence by proposition~\ref{prop:continuity_condition} continuity at $\alpha=1$ holds. An example where a phase transition appears at $\alpha=0$ is given in section~\ref{subsec:partial_ball}.  In section~\ref{subsec:needle} we present an example where  $C_\Omega<K_2$ but continuity of at $\alpha =1$ can  be established via Mosco-convergence \cite{MR1283033} of the associated Dirichlet forms, see also \cite{MR3154581}.
\end{remark}


\section{Examples}
\subsection{Brownian motion on balls with sticky  boundary diffusion}
\label{subsec:ball_full}

As our first example let $\Omega:= B_1$ be the unit ball in $\R^d$, $\Sigma=\partial \Omega$ and  $\Deriv=\nabla$ and $\bDeriv=\sqrt \beta \, \bnabla$ with $\DomE_0=C^1(\overline \Omega)$.

\begin{proposition}
\label{prop:example_full_sphere}
In the case when $\Omega=B_1\subset \R^d$ the optimal  Poincaré constant of the generator~\eqref{eq:gendef} is bounded from above by 
\begin{align}
\label{example_full_sphere}
C_\alpha \leq
\max \left(
C_\Omega + (1-\alpha)\frac{d+1}{4d^2},
\frac{4(1-\alpha) d  +4\alpha d^2 C_\Omega +\alpha(1-\alpha)(d+1) }{4d(\alpha d + (1-\alpha) \beta (d-1)) }
\right),
\end{align}
where $\alpha=\frac{\gamma}{d+\gamma}$ and $C_\Omega$ is the optimal Poincaré constant for reflecting Brownian motion on $B_1\subset \R^d$.
\end{proposition}

\begin{proof}
In order to apply Proposition~\ref{prop:interp_method}, it is sufficient to compute the constants $C_\Sigma$,  $\Cint$, $K_1$ and $K_2$. We claim that inequalities~\eqref{ineq Cint} and~\eqref{ineq K} holds with
\[ C_\Sigma =\frac{1}{\beta(d-1)}, \quad \Cint=\frac{1}{d}, \quad K_1= \frac{d+1}{4d^2}, \quad K_2=0. \]

First, according to~\cite[Theorem 22.1]{Shubin}, the first eigenvalue of the Laplace-Beltrami operator on the unit sphere of dimension $d-1$ is equal to $d-1$, thus $C_\Sigma=\frac{1}{\beta (d-1)}$. 

Moreover, according to~\cite[Theorem~4]{Beckner:1993}, for every $f \in \mC^1(\partial \Omega)$ one has 
\begin{equation*} %sobolev inequality on B
%  \label{equ_sobolev_inequality_on_O}
  \left( \int_{ \partial \Omega  }   |f|^q d\lambda_{\Sigma }  \right)^{ \frac{2}{ q }}\leq \frac{ q-2 }{ d } \int_{ \Omega}   \|\nabla u\|^2 d\lambda_\Omega+\int_{ \partial \Omega  }   f^2 d\lambda_{\Sigma },  
\end{equation*}
for $2\leq q<\infty$ if $d=2$ and $2\leq q< \frac{ 2d-2 }{ d-2 }$ if $d\geq 3$, where $u$ is the harmonic extension of $f$ to  the unit ball $\Omega$. 
It implies the logarithmic Sobolev inequality $ \operatorname{Ent}_{\lambda_\Sigma} (f^2)\leq  \frac{2}{ d }\int_{ \Omega }   \|\nabla u\|^2 d\lambda_\Omega$.
Repeating the proof of Proposition~5.1.3 in~\cite{Bakry:2014}, we get $
\Var_{\lambda_{\Sigma }} f\leq  \frac{1}{ d }\int_{ \Omega }   \|\nabla u\|^2 d\lambda_\Omega$.
Moreover, since the harmonic extension of $f$ is minimizing the energy functional $\mathcal E_1$ under any function with boundary condition $f$, the last inequality implies for any $f \in \mC^1(\overline \Omega)$
\begin{align}
\label{equ_poincare_inequality_on_s}
\Var_{\lambda_{\Sigma }} f\leq  \frac{1}{ d }\int_{ \Omega }   \|\nabla f\|^2 d\lambda_\Omega,
\end{align}
which implies $\Cint=\frac{1}{d}$.

Furthermore, note that $\int_{\partial \Omega} f(y)  \lambda_\Sigma(dy)=\int_\Omega f(\pi_x)  \lambda_\Omega (dx)$, where $\pi_x= \frac{x}{ \|x\| }$, $x \not= 0$. Hence, using  Jensen's inequality and polar coordinates
\begin{align*}
 \left( \int_\Omega f d \lambda_\Omega - \int_{\partial \Omega} f d \lambda_\Sigma \right)^2&\leq  \int_{\Omega} (f(x)-f(\pi_x))^2  \lambda_\Omega (dx)\\
       &=  \frac{ 1 }{ |\Omega| }\int_{ \partial \Omega } \int_{ 0 }^{ 1 }  \left(f(ry)-f(y)\right)^2r^{d-1}drdy \\
    &=  \frac{ 1 }{|\Omega|} \int_{ \partial \Omega }   \int_{ 0 }^{ 1 }  \left( \int_{ r }^{ 1 } \frac{ d }{ ds }f(sy)ds\right)^2r^{d-1}drdy
\end{align*} %zzzz
\begin{align*}
  &\leq \frac{ 1 }{ |\Omega| } \int_{ \partial \Omega }    \int_{ 0 }^{ 1 } (1-r)\left( \int_{ r }^{ 1 } \left(\frac{d}{ds}f(sy)\right)^2ds  \right)r^{d-1} drdy\\
    &= \frac{1}{ |\Omega| }\int_{ \partial \Omega }   \int_{ 0 }^{ 1 } \left[ \int_{ 0 }^{ s } (1-r) r^{d-1}dr  \right] \left( \frac{d}{ ds }f(sy) \right)^2 dsdy.
  \end{align*}
  We separately estimate
  \begin{align*}
  \int_{ 0 }^{ s } (1-r) r^{d-1}dr =  \left( \frac{s}{d}-\frac{s^2}{d+1}  \right)s^{d-1}
  \leq \frac{d+1}{4d^2}s^{d-1}.
  \end{align*}
  for any $s \in [0,1]$.
 Hence,
  \begin{align}
  \left( \int_\Omega f d \lambda_\Omega - \int_{\partial \Omega} f d \lambda_\Sigma \right)^2&\leq \frac{ d+1 }{ 4d^2|\Omega| }\int_{ \partial \Omega }    \int_{ 0 }^{ 1 }\left( \nabla f(sy) \cdot y\right)^2s^{d-1} ds \notag\\
  &= \frac{ d+1 }{ 4d^2|\Omega| } \int_{ \partial \Omega }    \int_{ 0 }^{ 1 } \left\| \nabla f(sy)\right\|^2 s^{d-1} dsdy \notag\\
  &= \frac{ d+1 }{ 4d^2 }\int_{ \Omega }   \|\nabla f(x)\|^2 \lambda_\Omega(dx). 
  \label{ineq_interpolation_ball}
  \end{align}
  which implies  $K_1= \frac{d+1}{4d^2}$ and $K_2=0$. 
\end{proof}

For illustration, in $d=2, $ we compare the bound from Proposition~\ref{prop:example_full_sphere} for $\beta=1,\gamma>0$ to the  optimal constant $C_\alpha$  which will be computed numerically. To evaluate the bound \eqref{example_full_sphere}, note that  in this case   \begin{equation}
\label{def:gamma_star}
C_\Omega=\frac{1}{\sigma_{\Omega}}\approx\frac{1}{3.39}, 
\end{equation}
where $\sigma_{\Omega}$ is the smallest positive eigenvalue of the Laplace operator with Neumann boundary condition on the circle. It is given as the minimal positive solution to the equation $J_m'(\sqrt{\gamma})=0$, $m\in\N_0$, where $J_m$ is the Bessel function of the first kind of parameter $m$,  defined  by $J_m(x)=\frac{1}{\pi} \int_0^\pi\cos (mt-x \sin t) dt$, $x \geq 0$.  As a consequence, 
 inequality~\eqref{example_full_sphere} becomes  
\begin{align}
\label{example_full_sphere_dim2}
C_\alpha \leq
\frac{8 (1-\alpha)\sigma_{\Omega}+16\alpha+3\alpha(1-\alpha)\sigma_{\Omega}}
{8(1+\alpha)\sigma_{\Omega} }.
\end{align}

For the numerical computation of  $C_\alpha$ one notes that  the generator $A_\alpha$ associated with $\e_\alpha$ is defined on $ D(A_\alpha) \subset \mC^2(\overline{\Omega})$ as  
\[
  A_\alpha f =  \I_{\Omega}\Delta f+\I_{\partial \Omega}\left( \Bdelta f- \frac{ 2 \alpha }{ 1-\alpha }\frac{\partial f}{\partial \nu} \right),
\]
where $\Bdelta$  and $\frac{\partial}{\partial \nu}$ denote the Laplace-Beltrami operator  and the outer normal derivative on the circle $\partial \Omega$.
Hence, an eigenvector of $-A_\alpha$ for eigenvalue $\lambda \geq 0$ is a function $f\in D(A_\alpha)$ such that
\[
  A_\alpha f=-\lambda f \quad \mbox{in}\ \ \Omega.
\]
This equation is equivalent to the system of partial differential equations
\begin{equation*}
  \label{equ_equation_for_eigenvalues}
  \begin{cases}
    \Delta f= -\lambda f & \mbox{in}\ \ \Omega,\\
    \Bdelta f-\frac{2\alpha}{1-\alpha} \frac{\partial f}{\partial \nu}= -\lambda f & \mbox{on}\ \ \partial \Omega,
  \end{cases}
\end{equation*}
which by the continuity of $f$ can be rewritten  as
\begin{equation*}
  \label{equ_equation_for_eigenvalues_with_boundary_conditions}
  \begin{cases}
    \Delta f= -\lambda f & \mbox{in}\ \ \Omega,\\
    \Delta f=\Bdelta f-\frac{2\alpha}{1-\alpha} \frac{\partial f}{\partial \nu} & \mbox{on}\ \ \partial \Omega.
  \end{cases}
\end{equation*}
Passing to polar coordinates  $(x_1,x_2)=(r\cos\theta,r\sin\theta)\in\Omega$ in $d=2$ and separating variables, we obtain the set of eigenfunctions $\{f_{m,l}^c,f_{m,l}^s\}_{m,l \in \N_0}$,
\[
f_{m,l}^c(x_1,x_2)=J_m(\sqrt{\lambda_{m,l}}r)\cos(m\theta), \quad m,l \in\N_0,
\]
\[
f_{m,l}^s(x_1,x_2)=J_m(\sqrt{\lambda_{m,l}}r)\sin(m\theta), \quad m \in\N,\ \ l \in\N_0,
\]
where  $\lambda_{m,l}$, $l \in\N_0$, are countable family of positive solutions to the  equation
\begin{equation} %equation for gamma
  \label{equ_equation_for_gamma}
  \sqrt{\lambda} J_m''(\sqrt{ \lambda })+\frac{ 1+ \alpha }{ 1-\alpha }J_m'(\sqrt{ \lambda })=0
\end{equation}
for every $m \in \N_0$. Since the family $\{f_{m,l}^c,\ m,l \in\N_0\}\cup\{f_{m,l}^s,\ m\in\N_0,\ l\in\N_0\}$ is dense in $L_2(\Omega,\lambda_\alpha)$  and the operator $A_\alpha$ is symmetric, the standard argument implies 
\begin{equation}
\label{optimal_constant}
C_\alpha=\frac{1}{\lambda_{\alpha,\star}},
\end{equation}
where $\lambda_{\alpha,\star}=\min\limits_{m,l\in\N_0}\lambda_{m,l}$. The resulting curves  are plotted in Figure~\ref{fig:full_sphere}.

%\begin{figure}[b!]\label{fig:graphs}
\begin{figure}[h]\label{fig:graphs}
\centering
  \includegraphics[width=80mm]{007n.pdf}
  \caption{The blue curve represents $\alpha \mapsto C_\alpha$ the optimal Poincaré constant when $\Omega$ is the unit ball of $\R^2$ with full boundary diffusion. The red curve is the upper estimate given by~\eqref{example_full_sphere_dim2}.}
  \label{fig:full_sphere}
\end{figure}


\subsection{Smooth manifold with boundary}
\label{sec:smooth_manifolds}

Let $\Omega$ be a smooth compact Riemannian manifold of dimension $d$ with piecewise smooth boundary $\partial \Omega$. We denote by $\Ric$ the Ricci curvature of $\Omega$ and by $\mathrm{II}$ the second fundamental form on the boundary $\partial \Omega$. Assume in this section that:
\begin{align*}
\text{Assumption (M)}: \quad \quad \quad  \quad \quad \exists k_r>0, k_2>0, \quad
\Ric|_{\Omega}\geq k_R \id \quad \text{and} \quad \mathrm{II}|_{\partial \Omega} \geq k_2 \id. 
\end{align*}
As before we consider $\Sigma=\partial\Omega$, $\Deriv=\nabla$ and $\bDeriv= \bnabla$ with $\DomE_0=C^1(\overline \Omega)$.


\begin{proposition}
Under assumption (M), it holds that
\begin{align}
\label{def_M1}
 C_\alpha \leq
\max \left(
C_\Omega + \frac{(1-\alpha)(d-1)}{d k_R} ,
\frac{C_\Sigma}{d k_R}\cdot \frac{2(1-\alpha) dk_R +\alpha d k_2 k_R C_\Omega +\alpha(1-\alpha)(d-1)k_2 }{2(1-\alpha)+\alpha  k_2 C_{\Sigma}} \right)=:M_1.
\end{align}
\end{proposition}



This statement is obtained via Proposition~\ref{prop:interp_method} and the two statements below.


\begin{proposition} \label{prop:K_one_manif}
Under assumption (M), inequality~\eqref{ineq K} is satisfied with $K_2=0$ and
\begin{align*}
K_1 = \frac{d-1}{d k_R}. 
\end{align*}
\end{proposition}

\begin{proof}
Our goal is to obtain an lower bound of
\begin{align*}
\inf_{\substack {f \in C^1(\overline \Omega) }}  \frac{\int_\Omega \| \nabla f \|^2 d\lambda_\Omega}{\left( \int_\Omega f d \lambda_\Omega - \int_\Sigma f d \lambda_\Sigma \right)^2} ,
\end{align*}
where we recall that $\Sigma=\partial \Omega$. 
We note  that
\begin{align*}
\inf_{\substack {f \in C^1(\overline \Omega) }}  \frac{\int_\Omega \| \nabla f \|^2 d\lambda_\Omega}{\left( \int_\Omega f d \lambda_\Omega - \int_\Sigma f d \lambda_\Sigma \right)^2} 
=\inf_{\substack {f \in C^1(\overline \Omega)\\ \int_\Sigma f d \lambda_\Sigma=0  }}  \frac{\int_\Omega \| \nabla f \|^2 d\lambda_\Omega}{\left( \int_\Omega f d \lambda_\Omega  \right)^2} 
\geq
\inf_{\substack {f \in C^1(\overline \Omega)\\ \int_\Sigma f d \lambda_\Sigma=0  }}  \frac{\int_\Omega \| \nabla f \|^2 d\lambda_\Omega}{ \int_\Omega f^2 d \lambda_\Omega } =:\sigma. 
\end{align*}
Let $f \in C^1(\overline \Omega)$  be a minimizer for $\sigma$. Then $\int_\Sigma f d \lambda_\Sigma=0$ and 
\begin{align*}
\int_\Omega \nabla f \cdot \nabla \xi d\lambda_\Omega
=\sigma \int_\Omega f \xi d\lambda_\Omega
\end{align*}
for each $\xi \in C^1(\overline \Omega)$ with $\int_\Sigma \xi d \lambda_\Sigma=0$. By integration by parts, the latter equality is equivalent to
\begin{align*}
-\int_\Omega \Delta f   \xi d\lambda_\Omega + \frac{|\Sigma|}{|\Omega|} \int_\Sigma \frac{\partial f}{\partial \nu} \xi d\lambda_\Sigma
=\sigma \int_\Omega f \xi d\lambda_\Omega
\end{align*}
for each $\xi \in C^1(\overline \Omega)$ satisfying $\int_\Sigma \xi d \lambda_\Sigma=0$.
In particular, choosing $\xi \in C^\infty_0(\Omega)$ (which obviously satisfies $\int_\Sigma \xi d \lambda_\Sigma=0$), we get that $f$ should satisfy $-\Delta f=\sigma f$ in $\Omega$. Hence $ \int_\Sigma \frac{\partial f}{\partial \nu} \xi d\lambda_\Sigma=0$ for each $\xi$ with zero mean, so it follows that $\int_\Sigma \frac{\partial f}{\partial \nu} \left(\xi-\int_\Sigma \xi d \lambda_\Sigma\right) d\lambda_\Sigma=0$ for every $\xi \in C^1(\overline \Omega)$, which is equivalent to
\begin{align*}
\int_\Sigma \left( \frac{\partial f}{\partial \nu}-\int_\Sigma \frac{\partial f}{\partial \nu} d \lambda_\Sigma\right)\xi d\lambda_\Sigma=0
\end{align*}
for every $\xi \in C^1(\overline \Omega)$. It follows that $\frac{\partial f}{\partial \nu}$ is constant on $\Sigma$. Therefore, $f$ satisfies
\begin{equation} 
\label{minimizer}
  \begin{cases}
    \Delta f= -\sigma f & \mbox{in}\ \ \Omega,\\
    \frac{\partial f}{\partial \nu} \equiv c & \mbox{on}\ \ \partial \Omega,\\
    \int_\Sigma f d\lambda_\Sigma=0,    
  \end{cases}
\end{equation}
for some constant $c$. 

Moreover, recall Reilly's formula (see~\cite{reilly})
\begin{equation}\label{reilly}
\begin{aligned}
\int_\Omega \left((\Delta f)^2 - \|\nabla^2 f \|^2 \right) dx
&=\int_\Omega \Ric (\nabla f, \nabla f) dx \\
&\quad
+%\frac{|\Sigma|}{|\Omega|}
\int_\Sigma \left( H ( \frac{\partial f}{\partial \nu})^2   +\mathrm{II} (\bnabla f, \bnabla f) +2 \Bdelta  f \frac{\partial f}{\partial \nu}\right)  dS
\end{aligned}
    \end{equation}
where $dx$ and $dS$ denote the Riemannian volume resp. surface measure on $\Omega$ and $\partial \Omega$, $\nabla^2 f$ is the Hessian of $f$ and $H$ is the mean curvature of $\Sigma$ (\textit{i.e.} the trace of $\mathrm{II}$). 
Since $f$ satisfies~\eqref{minimizer}, 
\begin{align*}
\int_\Omega (\Delta f)^2 dx
= -\sigma \int_\Omega f\Delta f dx
&=\sigma \int_\Omega \|\nabla f\|^2 dx
-\sigma 
% \frac{|\Sigma|}{|\Omega|} 
\int_\Sigma \frac{\partial f}{\partial \nu} f dS \\
&=\sigma \int_\Omega \|\nabla f\|^2 dx
-\sigma 
% \frac{|\Sigma|}{|\Omega|}
c \int_\Sigma  f dS =\sigma \int_\Omega \|\nabla f\|^2 dx,
\end{align*}
because $\int_\Sigma  f dS=|\Sigma| \int_\Sigma f d\lambda_\Sigma=0$.
Furthermore, note that $\|\nabla^2 f \|^2=\sum_{i,j} (\partial_{ij}^2 f)^2\geq \sum_{i=1}^d (\partial_{ii}^2 f)^2\geq \frac{1}{d}(\sum_{i=1}^d \partial_{ii}^2 f)^2=\frac{1}{d} (\Delta f)^2$. Therefore, the l.h.s. of~\eqref{reilly} is bounded by
\begin{align*}
\int_\Omega \left((\Delta f)^2 - \|\nabla^2 f \|^2 \right) dx
\leq \frac{d-1}{d}\int_\Omega (\Delta f)^2  dx
\leq \frac{d-1}{d} \sigma \int_\Omega \|\nabla f\|^2 dx. 
\end{align*}
On the other hand, by assumption (M), $H \geq 0$, $\mathrm{II} (\bnabla f, \bnabla f)\geq 0$ and 
\begin{align*}
\int_\Omega \Ric (\nabla f, \nabla f) dx 
\geq k_R \int_\Omega \|\nabla f\|^2 dx. 
\end{align*}
Since 
\begin{align*}
\int_\Sigma  \Bdelta  f \frac{\partial f}{\partial \nu} dS=c\int_\Sigma  \Bdelta  f  dS=0
\end{align*}
the r.h.s.\ of~\eqref{reilly} is bounded from below by $k_R \int_\Omega \|\nabla f\|^2 dx$. It turns out that
\begin{align*}
\frac{d-1}{d} \sigma \int_\Omega \|\nabla f\|^2 dx
\geq k_R \int_\Omega \|\nabla f\|^2 dx,
\end{align*}
which implies that $\sigma\geq \frac{d}{d-1} k_R$. It follows that inequality~\eqref{ineq K} holds with $K_1= \frac{d-1}{d k_R}$. 
\end{proof}

\begin{remark} Instead of using $K_1$ from Proposition~\ref{prop:K_one_manif} another admissible choice  is \[K_1'  = \frac{|\Omega|}{|\partial \Omega|}B^2  (1+C_\Omega)<\infty,\] where $B$ is  the {optimal Sobolev trace constant} of $\Omega$, i.e.\ the norm of the embedding $H^{1,2}(\Omega)\hookrightarrow L^2 (\partial \Omega)$.
 $B^{-2}$ is the first nontrivial eigenvalue of a  Steklov-type eigenvalue problem 
 \begin{equation*}
\begin{cases}
  -\Delta f + f =0 &  \mbox{ in } \Omega \\
 \frac{\partial f}{\partial \nu}  = \sigma f  & \mbox{ on } \partial \Omega,
    \end{cases}\end{equation*}
% However,  
for which however explicit lower bounds in terms of the geometry of $\Omega$ seem yet unknown \cite{MR1978428,MR1971310, MR3299025,MR1443055,MR2384747}. 
\end{remark}


\begin{proposition} \label{prop:Escobar_ineq}
Under assumption (M), inequality~\eqref{ineq Cint} holds with $\Cint=\frac{2}{k_2}$. 
\end{proposition}

\begin{proof} The optimal choice for $\Cint$ is $\sigma^{-1}$, where $\sigma$ given by
\[
\sigma=\inf_{\substack {f \in C^1(\overline \Omega)\\ \int_\Sigma f d \lambda_\Sigma=0  }}  \frac{\int_\Omega \| \nabla f \|^2 d\lambda_\Omega}{\left( \int_\Sigma f^2 d \lambda_\Sigma  \right)^2}\]
is the first nontrivial eigenvalue of the Steklov-problem  c.f.\ \cite{MR3662010}
\begin{equation*}
\begin{cases}
 \Delta f = 0 &  \mbox{ in } \Omega, \\
 \frac{\partial f}{\partial \nu} = \sigma f & \mbox{ on } \partial \Omega.
    \end{cases}\end{equation*}
Escobar \cite{escobar}  showed  $\sigma \geq \frac {k_2} 2 $ in this case.
\end{proof}

Alternatively, we obtain another upper bound for $C_\alpha$ by a direct application of  Reilly's formula.

\begin{proposition}
Under assumption (M) it holds that 
\begin{align}
    \label{def_M2}
    C_\alpha \leq \max\left(\frac{(3d-1)(1-\alpha) }{d\alpha k_2}\frac{|\Omega|}{|\partial \Omega |}, \frac{d-1}{d k_R}\right)=:M_2.
\end{align}
\end{proposition}

\begin{proof} We estimate equivalently from below the first nontrivial eigenvalue  $\sigma=C_\alpha^{-1}$ for the problem  \[
\begin{cases}
\Delta f + \sigma f = 0 & \mbox{ in } \Omega\\
\Bdelta f - \gamma \frac{\partial f}{\partial \nu} + \sigma f =0 & \mbox{ on } \partial \Omega,
\end{cases}\]
where $\gamma = \frac \alpha {1-\alpha} \frac{|\partial \Omega|}{|\Omega|}$. As in the proof of Proposition~\ref{prop:K_one_manif} we apply  Reilly's formula~\eqref{reilly} to the corresponding eigenfunction $f$. In this case, for the l.h.s. we estimae 
\begin{align*}
\int_\Omega \left((\Delta f)^2 - \|\nabla^2 f \|^2 \right) dx
& \leq \frac{d-1}{d} \int_\Omega (\Delta f)^2 dx= - \frac {d-1}{d}\sigma \int_\Omega f \Delta f dx \\
& = \frac{d-1}{d}\sigma \int_\Omega \|\nabla f\|^2   dx - \frac {d-1}{d}\sigma \int_\Sigma  \frac{\partial f}{\partial \nu}f dS  \\
& = \frac{d-1}{d}\sigma \int_\Omega \|\nabla f\|^2   dx -
 \frac {d-1}{d}\frac{\sigma}{\gamma} \int_\Sigma (\Bdelta f +  {\sigma} f) f dS\\
& = \frac{d-1}{d}\sigma \int_\Omega \|\nabla f\|^2   dx + 
 \frac {d-1}{d}\frac{\sigma}{\gamma} \int_\Sigma \|\bnabla f\|^2  dS -  \frac {d-1}{d}\frac{\sigma^2 }{\gamma} \int_\Sigma  f^2   dS\\
 & \leq   \frac{d-1}{d}\sigma \int_\Omega \|\nabla f\|^2   dx + 
 \frac {d-1}{d}\frac{\sigma}{\gamma} \int_\Sigma \|\bnabla f\|^2  dS. \end{align*}
Since  
\begin{align*}
\int_\Sigma \frac{\partial f}{\partial \nu} \Bdelta f dS & = \frac  1 \gamma \int_\Sigma (\Bdelta f + \sigma f) \Bdelta f dS\\
& = \frac 1  \gamma \int_\Sigma(\Bdelta f)^2 dS - \frac \sigma \gamma \int_\Sigma \|\bnabla f \|^2 dS
\geq - \frac \sigma \gamma \int_\Sigma \|\bnabla f \|^2 dS
\end{align*}
the r.h.s. of~\eqref{reilly} is bounded from below by 
\begin{align*} 
% \mbox{r.h.s.} & \geq 
 k_R \int_\Omega  \|\nabla f\|^2 dx    &- \frac {2 \sigma}  \gamma \int_\Sigma \|\bnabla f\|^2 dS + \int_\Sigma h |\frac{\partial f}{\partial \nu}|^2 dS + k_2  \int_\Sigma \|\bnabla f\|^2 dS
\\
& \geq k_R \int_\Omega \|\nabla f\|^2 dx   - \frac {2 \sigma}  \gamma \int_\Sigma \|\bnabla f\|^2 dS + k_2 \int_\Sigma  \|\bnabla f\|^2 dS. 
 \end{align*}
Combining the two bounds  for \eqref{reilly} yields 
\[ \left(\frac {d-1}{d } \sigma - k_R \right) \int_\Omega \|\nabla f\|^2 dx \geq \left(k_2-\frac {3d-1}{d} \frac \sigma \gamma \right) \int_\Sigma \|\bnabla f\| ^2 dS,\]
which implies that either 
\[ k_2 - \frac {3d-1}d \frac {\sigma}\gamma \leq 0,\quad \mbox{ i.e. }\quad  \sigma \geq \frac {d k_2 \gamma }{3 d -1 }\]
or 
\[ \frac{d-1}{d}\sigma - k_R \geq 0,\quad \mbox{ i.e. }\quad\sigma  \geq  \frac {d}{d-1}k_r.\]
Consequently, 
\[ \sigma \geq \min\left(\frac {d k_2 \gamma }{3 d -1 },\frac {d}{d-1}k_R\right). \]
\end{proof}

\begin{corollary}
Under assumption  (M), it holds that
\begin{align*}
    C_\alpha \leq \min (M_1,M_2),
\end{align*}
where $M_1=M_1(\alpha)$ and $M_2=M_2(\alpha)$ are defined by~\eqref{def_M1} and~\eqref{def_M2}, respectively. 
\end{corollary}

When $\alpha$ goes to 0, $M_1$ tends to $\max(C_\Omega,\frac{d-1}{dk_R},C_\Sigma)$ and $M_2$ tends to $+\infty$, so the estimation via the interpolation method is always stronger. 
When $\alpha$ goes to $1$, $M_1$ tends to $C_\Omega$ and $M_2$ tends to $\frac{d-1}{dk_R}$, so the relative strength of each method depends on the values of $C_\Omega$, $d$ and $k_R$. 



\subsection{Brownian motion on balls with partial sticky reflecting boundary diffusion}
\label{subsec:partial_ball}

As in Section~\ref{subsec:ball_full}, let  $\Omega:= B_1$ be the unit ball of $\R^2$. Now, define for a fixed 
 $\delta \in (0,1)$  
\[
\Sigma=\{ (\cos \theta, \sin \theta) \in \partial\Omega: -\delta \pi \leq \theta \leq \delta \pi\}, \quad \quad \neumann:=\partial \Omega \backslash \Sigma.
\]

\begin{proposition}
\label{prop_partial_sphere} It holds that 
\begin{align}
\label{estim_partial}
C_\alpha \leq
\max \left( 
C_\Omega + (1-\alpha)K_1(\delta), 
\frac{4(1-\alpha) \delta^2+8 \alpha \delta^3 C_\Omega +8 \alpha (1-\alpha) \delta^3 K_1(\delta)  }{(1-\alpha) + 8 \alpha \delta^3 }
\right),
\end{align}
where $C_\Omega = \frac{1}{\sigma_{\Omega}}\approx\frac{1}{3.39}$  and $K_1(\delta)= \left( \sqrt{1-\delta}\pi + \frac{1}{4}\sqrt{\frac{3}{\delta}} \right)^2$.
\end{proposition}

As previously, we will start by computing the needed constants $C_\Omega$, $C_\Sigma$, $\Cint$, $K_1$ and $K_2$. 
The first constant, $C_\Omega=\frac{1}{\sigma_{\Omega}}\approx \frac{1}{3.39}$, remains unchanged.
\begin{lemma} The following inequalities hold true
\begin{align}
\label{lem_ine_1}
\Var_{\mus} f &\leq C_\Sigma \int_{\Sigma} \|\bnabla f\|^2 d \mus, \\
\Var_{\mus} f &\leq \Cint \int_{\Omega} \|\nabla f\|^2 d\lambda_\Omega,
\label{lem_ine_2}
\end{align}
where $C_\Sigma=4\delta^2$ and $\Cint=\frac{1}{2\delta}$. 
\end{lemma}


\begin{proof}
Inequality~\eqref{lem_ine_1} corresponds to the Poincaré inequality of the Laplacian on the one-dimensional interval $[-\delta \pi, \delta \pi]$ with Neumann boundary conditions. It is well known (see~\cite[Prop. 4.5.5]{Bakry:2014}) that the optimal Poincaré constant is given by $C_\Sigma=4\delta^2$.

Moreover, let us decompose the normalized Hausdorff measure $\lambda_\partial$ on the sphere $\partial \Omega$ into the normalized Hausdorff measure $\mus$ on $\Sigma$  and the normalized Hausdorff measure $\mun$ on $\neumann$: $\lambda_\partial = \delta \mus + (1-\delta) \mun$. 
Therefore
\begin{align*}
\Var_{\lambda_{\partial }} f = \delta\Var_{\mus} f + (1-\delta)\Var_{\mun} f
  +  \delta(1-\delta) \left( \int_{\Sigma} f d\mus- \int_{\neumann} f d\mun \right)^2
  \geq \delta\Var_{\mus} f,
\end{align*}
Furthermore, recall that by inequality~\eqref{equ_poincare_inequality_on_s}, for any $f \in \mathcal C^1(\overline \Omega)$, $
\Var_{\lambda_{\partial }} f\leq  \frac{1}{ 2 }\int_{ \Omega }   \|\nabla f\|^2 d\lambda_\Omega$. 
It implies~\eqref{lem_ine_2}.
\end{proof}



\begin{lemma}
It holds that 
\begin{align*}
\left( \int_\Omega f d \lambda_\Omega - \int_{\Sigma} f d \mus \right)^2
\leq K_1(\delta) \int_\Omega \| \nabla f \|^2 d\lambda_\Omega
\end{align*}
with  $K_1(\delta)= \left( \sqrt{1-\delta}\pi + \frac{1}{4}\sqrt{\frac{3}{\delta}} \right)^2$.
\end{lemma}

\begin{proof}
For every $x \in \Omega\backslash \{0\}$ with polar coordinates $(r, \theta)$, $r \in (0,1)$, $\theta \in (-\pi,\pi]$, denote by $p_x$ the point of coordinates $(1, \delta \theta)$ on $\Sigma$. Obviously, $\int_{\Sigma} f(y)  \mus (dy)= \int_\Omega f(p_x)  \lambda_\Omega (dx)$ and by Jensen's inequality
\begin{align*}
I:=\left( \int_\Omega f d \lambda_\Omega - \int_{\Sigma} f d \mus \right)^2
\leq  \int_\Omega \left(f(x) -f(p_x)\right)^2  \lambda_\Omega (dx).
\end{align*}
Define $g(r, \theta):=f(r \cos(\theta), r \sin (\theta))$. Then
\begin{align}
\label{triangle}
I &\leq \frac{1}{\pi} \int_0^1 \int_{-\pi}^\pi (g(r, \theta)-g(1,\delta \theta))^2 r dr d\theta \leq  ( \sqrt{J_1} +  \sqrt{J_2})^2 ,
\end{align}
where $J_1=\frac{1}{\pi} \int_0^1 \int_{-\pi}^\pi (g(r, \theta)-g(r,\delta \theta))^2 r dr d\theta$ and $J_2=\frac{1}{\pi} \int_0^1 \int_{-\pi}^\pi (g(r, \delta\theta)-g(1,\delta \theta))^2 r dr d\theta$. 
On the one hand
\begin{align}
J_1 &= \frac{1}{\pi} \int_0^1 \int_{-\pi}^\pi \left(\int_{\delta \theta} ^\theta \frac{\partial g}{\partial \theta} (r,u) du\right)^2 r dr d\theta \leq \frac{1-\delta}{\pi} \int_0^1 \int_{-\pi}^\pi |\theta| \int_{-\pi}^\pi  \left(\frac{\partial g}{\partial \theta} \right)^2(r,u) du \; r dr d\theta \notag\\
&\leq (1-\delta)\pi^2  \frac{1}{\pi}\int_0^1  \int_{-\pi}^\pi  \left(\frac{1}{r}\frac{\partial g}{\partial \theta} \right)^2(r,u) du \; r dr 
\leq (1-\delta)\pi^2 \int_\Omega \| \nabla f \|^2 d\lambda_\Omega.
\label{ineq_J1}
\end{align}
On the other hand
\begin{align*}
J_2 
&\leq \frac{1}{\pi} \int_0^1 \int_{-\pi}^\pi (1-r)  \int_r ^1 \left(\frac{\partial g}{\partial r}\right)^2 (s,\delta \theta) ds \; r dr d\theta \leq \frac{1}{\pi} \int_0^1 \int_{-\pi}^\pi  \left(\frac{\partial g}{\partial r}\right)^2 (s,\delta \theta) \int_0^s (1-r) r dr ds d\theta.
\end{align*}
For every $s \in [0,1]$, 
$\int_0^s (1-r) r dr = \frac{s^2}{2}-\frac{s^3}{3}\leq \frac{3s}{16}$, thus
\begin{align}
J_2
&\leq \frac{3}{16\delta \pi} \int_0^1 \int_{-\delta \pi}^{\delta\pi}  \left(\frac{\partial g}{\partial r}\right)^2 (s,u) s ds du \leq \frac{3}{16\delta} \int_\Omega \| \nabla f \|^2 d\lambda_\Omega.
\label{ineq_J2}
\end{align}
The proof of the lemma is completed by putting together~\eqref{triangle}, \eqref{ineq_J1} and~\eqref{ineq_J2}.
\end{proof}


\begin{proof}[Proof of Proposition~\ref{prop_partial_sphere}]
We apply Proposition~\ref{prop:interp_method} with $C_\Omega = \frac{1}{\sigma_{\Omega}}$, $C_\Sigma=4\delta^2$, $\Cint=\frac{1}{2\delta}$, $K_1(\delta)= \left( \sqrt{1-\delta}\pi + \frac{1}{4}\sqrt{\frac{3}{\delta}} \right)^2$ and $K_2=0$. 
\end{proof}


\begin{figure}[ht]
\centering
\subfloat[$\delta=0.5$]{\includegraphics[width=70mm]{101n.pdf}\label{delta1}}
\subfloat[$\delta=0.9$]{\includegraphics[width=70mm]{102n.pdf}\label{delta2}}
\caption{The above two figures show the upper estimate given by the r.h.s of~\eqref{estim_partial}. 
In the case $\delta=0.9$ (Figure~\ref{delta2}), the curve interpolates between the extremal constants $C_\Sigma$ and $C_\Omega$, as opposed to the half-sphere case (Figure~\ref{delta1}).}
\label{fig:partial ball}
\end{figure}

For $\delta$ sufficiently large, the map $\alpha \mapsto C_\alpha$ is continuous at $\alpha=0$. Indeed, by Proposition~\ref{prop:continuity_condition}, a sufficient condition is $C_\Sigma(\delta) > C_\Omega + K_1(\delta)$, that is
\[4 \delta^2 > \frac{1}{\sigma_{\Omega}}+ \left( \sqrt{1-\delta}\pi + \frac{1}{4}\sqrt{\frac{3}{\delta}} \right)^2, \]
which is satisfied for any $\delta \geq 0.862$.

\subsection{Ball with a needle}
\label{subsec:needle}

 Our final example is the  unit ball $\Omega=B_1$  of $\R^2$ with a needle $\mathcal L$   of length $L$  attached to one point of the boundary, i.e.\  $\mathcal L:=\{ (x,0): 1\leq x \leq L+1\}$, see Figure~\ref{fig:needle}.
The attachment point and the endpoint of the needle are denoted by $x_0:=(1,0)$ and $x_L=(L+1,0)$, respectively.

\begin{figure}[ht]
\centering
  \includegraphics[width=70mm]{103n.pdf}
  \caption{The ball (in green) is denoted by $\Omega$, the boundary of the ball is denoted by $\partial \Omega$ and the needle (in blue) is denoted by $\mathcal L$.}
  \label{fig:needle}
\end{figure}


In that setting, we define $\overline \Omega = \overline{ B_1 }\cup \mathcal L $, $\Sigma = \partial B_1 \cup \mathcal L$ and 
\[
\lambda_\alpha = \alpha \lambda_\Omega + (1-\alpha) \lambda_\Sigma,\]
where $\lambda_\Omega$ is as previously the normalized Lebesgue measure on $\Omega$ and $\lambda_\Sigma=\frac{2 \pi}{2 \pi+L} \lambda_\partial + \frac{L}{2\pi +L} \lambda_\mathcal L$, with $\lambda_\partial$ and $\lambda_\mathcal L$ being the normalized Hausdorff measures on $\partial \Omega$ and $\mathcal L$, respectively. We choose \[
 \DomE_0 =\left\{ f \in C_0(\overline{\Omega})) \cap C^1(\overline \Omega \setminus \{x_0\} ) \,| \,   \frac{\partial f}{\partial e_1}+\frac{\partial f}{\partial e_2}+\frac{\partial f}{\partial e_3}= 0 \mbox{ at } x_0\right\},\]
 where $e_1=(0,1)$, $e_2=(0,-1)$ and $e_3=(1,0)$ are the three "tangent" vectors to $\Sigma$ at point $x_0$,
 and  $\Deriv := \nabla$, $\bDeriv:=\sqrt{\beta}\nabla^\tau$, which is well defined  in $\Sigma \setminus \{x_0\}$. With this choice, for $\alpha \in [0,1]$ $(\mathcal E_\alpha, \DomE_0)$ is a pre-Dirichlet form on $L^2(\overline \Omega, \lambda_\alpha) $, whose closure generates Brownian motion on $\Omega$ with sticky boundary diffusion on $\Sigma$, i.e.\ whose generator is given by  \[A_\alpha(f) =  \Delta f \one_{\Omega} + \beta \Delta_\Sigma f\one_{\Sigma} -\frac{\alpha}{1-\alpha}\frac{2\pi+L}{\pi} \frac {\partial f}{\partial \nu }\one_{\partial \Omega },\]
with $\Delta_\Sigma$ being the generator of the canonical diffusion on $\Sigma$ with reflecting boundary condition at $x_L$. As before, the optimal Poincaré constant $C_\alpha$  for $A_\alpha$ is given by
\begin{align*}
C_\alpha:=\sup_{\substack {f \in \DomE_0 \\ \mathcal E_\alpha(f)>0}}  \frac{\Var_{\lambda_\alpha} f}{\mathcal E_\alpha (f)},
\end{align*}
and  let $C_\Omega:=C_1$ and $C_\Sigma:=C_0$.  In this case the following estimate is obtained.
\begin{proposition}
\label{prop:ball_needle}
\begin{align*}
    C_\alpha \leq \max \left( \frac{1}{\sigma_{\Omega}} + \frac{3}{8}(1-\alpha), \frac{1}{\beta\gamma_L} + \alpha \frac{L^2(\pi+L)}{\beta(2\pi + L)} \right),
\end{align*}
where $\gamma_L >0$ is the smallest positive solution to 
  \begin{align}
  \label{first eigenvalue needle}
 2 \cos (\sqrt{\gamma}L) (1- \cos (\sqrt{\gamma}2 \pi)) +\sin (\sqrt{\gamma}L)\sin (\sqrt{\gamma}2 \pi)=0.
 \end{align}
Note that $\gamma_L \leq 1$ for any $L>0$ and if  $L=2\pi$,  $\gamma_{2\pi}= \left(\frac{ \arccos(-1/3)}{2\pi}\right)^2 \approx 0.0925$. 
\end{proposition}

Let us compute the constants needed to apply Proposition~\ref{prop:interp_method}. 
As we do not expect an inequality of type~\eqref{ineq Cint} to hold in that case, we set $\Cint:=+\infty$. Moreover, $C_\Sigma$ can be computed exactly as follows.

\begin{lemma}
In this case, $C_\Sigma=\frac{1}{\beta \gamma_L}$.
\end{lemma}

\begin{proof}
The constant $\frac{1}{C_\Sigma}$ is the smallest non-zero eigenvalue $\gamma$ of the  following problem:
\begin{align*}
  \begin{cases}
   \beta \Bdelta f = -\gamma f & \mbox{on}\ \ \Sigma \backslash\{x_0\},\\
    \frac{\partial f}{\partial \nu} = 0 & \mbox{at point }x_L,\\
    \frac{\partial f}{\partial e_1}+\frac{\partial f}{\partial e_2}+\frac{\partial f}{\partial e_3}= 0 & \mbox{at point }x_0,
  \end{cases}
\end{align*}
where $\Bdelta$ is the Laplace-Beltrami operator on $\partial \Omega $ and $\mathcal L$.
A general solution to that boundary value problem is given by
\begin{align*}
f(x)=\begin{cases}
 A \cos (\sqrt{\frac{\gamma}{\beta}}y)+B \sin (\sqrt{\frac{\gamma}{\beta}}y)  & \mbox{if } x=(y,0) \in \mathcal L, \\
C \cos (\sqrt{\frac{\gamma}{\beta}}\theta) + D \sin(\sqrt{\frac{\gamma}{\beta}}\theta)  & \mbox{if } x=(\cos \theta, \sin \theta) \in \partial \Omega,
\end{cases}
\end{align*}
where $A$, $B$, $C$ and $D$ have to satisfy the continuity assumption of $f$ at point $x_0$ and  both boundary conditions, that is:
\begin{align*}
\left\{ \begin{aligned}
A &= C = C \cos (\sqrt{\frac{\gamma}{\beta}} 2 \pi) + D \sin(\sqrt{\frac{\gamma}{\beta}}2 \pi) ,\\
0 &= -A \sin  (\sqrt{\frac{\gamma}{\beta}}L) + B \cos (\sqrt{\frac{\gamma}{\beta}}L), \\
0 &= B + D + C \sin (\sqrt{\frac{\gamma}{\beta}}2 \pi) - D \cos (\sqrt{\frac{\gamma}{\beta}}2 \pi).
\end{aligned}
\right.
\end{align*}
A short computation shows that this system has a non-trivial solution if and only if $\frac{\gamma}{\beta}$ solves~\eqref{first eigenvalue needle}. Therefore, $\frac{1}{C_\Sigma}=\beta \gamma_L$. Obviously, $\gamma=1$ is a solution to~\eqref{first eigenvalue needle}, thus $\gamma_L \leq 1$. 
 \end{proof}

Next, we look for the constants $K_1$ and $K_2$.

\begin{lemma}
Inequality~\eqref{ineq K} holds with $K_1=\frac{3}{8}$ and $K_2=\frac{L^2(\pi+L)}{\beta(2\pi + L)}$. 
\end{lemma}

\begin{proof}
Recall that $\Sigma=\partial \Omega \cup \mathcal L$. 
Let us insert the average of $f$ over $\partial \Omega$ as follows:
\begin{align*}
\left( \int_\Omega f d \lambda_\Omega - \int_\Sigma f d \lambda_\Sigma \right)^2
&\leq
2 \left( \int_\Omega f d \lambda_\Omega - \int_{\partial \Omega} f d \lambda_\partial \right)^2
+ 2 \left( \int_{\partial \Omega} f d \lambda_\partial - \int_\Sigma f d \lambda_\Sigma \right)^2 \\
&\leq \frac{3}{8} \int_\Omega \| \nabla f \|^2 d\lambda_\Omega
+ 2 \left( \int_{\partial \Omega} f d \lambda_\partial - \int_\Sigma f d \lambda_\Sigma \right)^2,
\end{align*}
where the second inequality follows directly from~\eqref{ineq_interpolation_ball}. 
Moreover, recalling that $\lambda_\Sigma=\frac{2 \pi}{2 \pi+L} \lambda_\partial + \frac{L}{2\pi +L} \lambda_\mathcal L$
\begin{align*}
\left( \int_{\partial \Omega} f d \lambda_\partial - \int_\Sigma f d \lambda_\Sigma \right)^2
 &=  \frac{L^2}{(2\pi +L)^2} \left( \int_{\partial \Omega} f d \lambda_\partial - \int_{\mathcal L} f d \lambda_\mathcal L \right)^2 .
\end{align*}
For every $x=(\cos\theta, \sin \theta) \in \partial \Omega$, with $\theta \in (-\pi,\pi]$, we denote by $p_x$ the point of $\mathcal L$ with coordinates $(1+L -\frac{|\theta| L}{\pi},0)$. It follows that
\begin{align*}
\left( \int_{\partial \Omega} f d \lambda_\partial - \int_{\mathcal L} f d \lambda_\mathcal L \right)^2 
= \left( \int_{\partial \Omega} (f(x)-f(p_x)) d \lambda_\partial  \right)^2 
\leq \int_{\partial \Omega} (f(x)-f(p_x))^2 d \lambda_\partial . 
\end{align*}
Denoting by $\lambda_\partial^+$ and $\lambda_\partial^-$  the normalized Hausdorff measures on $\partial \Omega^+:=  \{(x,y) \in \partial \Omega:  y>0\}$ and $\partial \Omega^-:=  \{(x,y) \in \partial \Omega:  y<0\}$, respectively, 
\begin{align*}
 \int_{\partial \Omega} (f(x)-f(p_x))^2 d \lambda_\partial 
 = \frac{1}{2}   \int_{\partial \Omega^+} (f(x)-f(p_x))^2 d \lambda^+_\partial 
 +\frac{1}{2} \int_{\partial \Omega^-} (f(x)-f(p_x))^2 d \lambda^-_\partial .
\end{align*}
Moreover, for  
any $\mC^1$-function $g:[-\pi, L] \to \R$, 
\begin{align*}
\frac{1}{\pi} \int_0^\pi \left|g(-\theta)-g(L-\textstyle \frac{\theta L}{\pi})\right|^2 d\theta 
\leq \frac{\pi+L}{2} \int_{-\pi}^L |g'(t)|^2  dt ,
\end{align*}
so we deduce, identifying $\partial \Omega^+$ with $[-\pi,0]$ and $\mathcal L$ with $[0,L]$, that
\begin{align*}
\int_{\partial \Omega^+} (f(x)-f(p_x))^2 d \lambda^+_\partial 
\leq \frac{\pi+L}{2} \left( \pi \int_ {\partial \Omega^+} \|\bnabla f\|^2 d \lambda^+_\partial  + L \int_\mathcal L  \|\bnabla f\|^2 d \lambda_\mathcal L \right)
\end{align*}
and using symmetry to deal with $\partial \Omega^-$, we obtain
\begin{align*}
 \int_{\partial \Omega} (f(x)-f(p_x))^2 d \lambda_\partial 
 &\leq \frac{\pi+L}{4}\left( \pi \int_ {\partial \Omega^+} \|\bnabla f\|^2 d \lambda^+_\partial +\pi \int_ {\partial \Omega^-} \|\bnabla f\|^2 d \lambda^-_\partial  + 2L \int_\mathcal L  \|\bnabla f\|^2 d \lambda_\mathcal L \right) \\
  &\leq \frac{(\pi+L)(2\pi+L)	}{2} \int_\Sigma \|\bnabla f\|^2 d \lambda_\Sigma.
\end{align*}
Putting together the above inequalities, we get
\begin{align*}
\left( \int_\Omega f d \lambda_\Omega - \int_\Sigma f d \lambda_\Sigma \right)^2
&\leq \frac{3}{8} \int_\Omega \| \nabla f \|^2 d\lambda_\Omega
+ 2\frac{L^2}{(2\pi +L)^2} \frac{(\pi+L)(2\pi+L)	}{2\beta} \int_\Sigma \beta\|\bnabla f\|^2 d \lambda_\Sigma 
\end{align*}
which leads to inequality~\eqref{ineq K} with $K_1=\frac{3}{8}$ and $K_2=\frac{L^2(\pi+L)}{\beta(2\pi + L)}$. 
\end{proof}

\begin{proof}[Proof of Proposition~\ref{prop:ball_needle}]
Since $\Cint=\infty$, we immediately get from Proposition~\ref{prop:interp_method} that 
\begin{align*}
C_\alpha \leq
\max \left(
C_\Omega + (1-\alpha)K_1 ,
\alpha K_2 ,
C_\Sigma + \alpha K_2 \right)
= \max \left(
C_\Omega + (1-\alpha)K_1 ,
C_\Sigma + \alpha K_2 \right).
\end{align*}
Therefore,  
\begin{align}
\label{needle_with_beta}
    C_\alpha \leq \max \left( \frac{1}{\sigma_{\Omega}} + \frac{3}{8}(1-\alpha), \frac{1}{\beta\gamma_L} + \alpha \frac{L^2(\pi+L)}{\beta(2\pi + L)} \right),
\end{align}
where $\sigma_{\Omega} \approx 3.39$. 
\end{proof}

\begin{remark}
  If $\beta$ is large enough, that is if the diffusion velocity  is larger on $\Sigma$ than on $\Omega$, then the first term in~\eqref{needle_with_beta} dominates. Precisely, if $\beta \geq \sigma_\Omega \left( \frac{1}{\gamma_L} +  \frac{L^2(\pi+L)}{2\pi + L} \right)$, then \eqref{needle_with_beta} rewrites for any $\alpha$
  \[C_\alpha \leq \frac{1}{\sigma_{\Omega}} + \frac{3}{8}(1-\alpha).
  \]
  Conversely, if $\beta \leq \frac{1}{\gamma_L} \left(\frac{1}{\sigma_{\Omega}} + \frac{3}{8}\right)^{-1}$, then  \eqref{needle_with_beta} rewrites for any $\alpha$
  \[C_\alpha \leq \frac{1}{\beta\gamma_L} + \alpha \frac{L^2(\pi+L)}{\beta(2\pi + L)}.
  \]
\end{remark}





\nocite{MR2255233,MR2255233,MR1988680,MR3498008,MR2029716,MR2277314}




%\bibliography{ref}
%\bibliographystyle{amsplain}%amsalpha


\providecommand{\bysame}{\leavevmode\hbox to3em{\hrulefill}\thinspace}
\providecommand{\MR}{\relax\ifhmode\unskip\space\fi MR }
% \MRhref is called by the amsart/book/proc definition of \MR.
\providecommand{\MRhref}[2]{%
  \href{http://www.ams.org/mathscinet-getitem?mr=#1}{#2}
}
\providecommand{\href}[2]{#2}
\begin{thebibliography}{10}

\bibitem{MR4096131}
Alexander Aurell and Boualem Djehiche, \emph{Behavior near walls in the
  mean-field approach to crowd dynamics}, SIAM J. Appl. Math. \textbf{80}
  (2020), no.~3, 1153--1174. \MR{4096131}

\bibitem{Bakry:2014}
Dominique Bakry, Ivan Gentil, and Michel Ledoux, \emph{Analysis and geometry of
  {M}arkov diffusion operators}, Grundlehren der Mathematischen Wissenschaften
  [Fundamental Principles of Mathematical Sciences], vol. 348, Springer, Cham,
  2014. \MR{3155209}

\bibitem{Beckner:1993}
William Beckner, \emph{Sharp {S}obolev inequalities on the sphere and the
  {M}oser-{T}rudinger inequality}, Ann. of Math. (2) \textbf{138} (1993),
  no.~1, 213--242. \MR{1230930}

\bibitem{MR1978428}
Rodney~Josu\'{e} Biezuner, \emph{Best constants in {S}obolev trace
  inequalities}, Nonlinear Anal. \textbf{54} (2003), no.~3, 575--589.
  \MR{1978428}

\bibitem{MR1971310}
Juli\'{a}n~Fern\'{a}ndez Bonder, Julio~D. Rossi, and Ra\'{u}l Ferreira,
  \emph{Uniform bounds for the best {S}obolev trace constant}, Adv. Nonlinear
  Stud. \textbf{3} (2003), no.~2, 181--192. \MR{1971310}

\bibitem{MR245085}
Jean-Michel Bony, Philippe Courr\`ege, and Pierre Priouret, \emph{Semi-groupes
  de {F}eller sur une vari\'{e}t\'{e} \`a bord compacte et probl\`emes aux
  limites int\'{e}gro-diff\'{e}rentiels du second ordre donnant lieu au
  principe du maximum}, Ann. Inst. Fourier (Grenoble) \textbf{18} (1968),
  no.~fasc. 2, 369--521 (1969). \MR{245085}

\bibitem{MR2198199}
Jean-Dominique Deuschel, Giambattista Giacomin, and Lorenzo Zambotti,
  \emph{Scaling limits of equilibrium wetting models in {$(1+1)$}-dimension},
  Probab. Theory Related Fields \textbf{132} (2005), no.~4, 471--500.
  \MR{2198199}

\bibitem{MR2029716}
Klaus-Jochen Engel, \emph{The {L}aplacian on {$C(\overline\Omega)$} with
  generalized {W}entzell boundary conditions}, Arch. Math. (Basel) \textbf{81}
  (2003), no.~5, 548--558. \MR{2029716}

\bibitem{escobar}
Jos\'{e}~F. Escobar, \emph{The geometry of the first non-zero {S}tekloff
  eigenvalue}, J. Funct. Anal. \textbf{150} (1997), no.~2, 544--556.
  \MR{1479552}

\bibitem{MR3498008}
Torben Fattler, Martin Grothaus, and Robert Vo\ss{}hall, \emph{Construction and
  analysis of a sticky reflected distorted {B}rownian motion}, Ann. Inst. Henri
  Poincar\'{e} Probab. Stat. \textbf{52} (2016), no.~2, 735--762. \MR{3498008}

\bibitem{MR3299025}
Vincenzo Ferone, Carlo Nitsch, and Cristina Trombetti, \emph{On a conjectured
  reverse {F}aber-{K}rahn inequality for a {S}teklov-type {L}aplacian
  eigenvalue}, Commun. Pure Appl. Anal. \textbf{14} (2015), no.~1, 63--82.
  \MR{3299025}

\bibitem{MR3662010}
Alexandre Girouard and Iosif Polterovich, \emph{Spectral geometry of the
  {S}teklov problem (survey article)}, J. Spectr. Theory \textbf{7} (2017),
  no.~2, 321--359. \MR{3662010}

\bibitem{MR2215623}
Gis\`ele~Ruiz Goldstein, \emph{Derivation and physical interpretation of
  general boundary conditions}, Adv. Differential Equations \textbf{11} (2006),
  no.~4, 457--480. \MR{2215623}

\bibitem{MR4065110}
Gis\`ele~Ruiz Goldstein, Jerome~A. Goldstein, Davide Guidetti, and Silvia
  Romanelli, \emph{Maximal regularity, analytic semigroups, and dynamic and
  general {W}entzell boundary conditions with a diffusion term on the
  boundary}, Ann. Mat. Pura Appl. (4) \textbf{199} (2020), no.~1, 127--146.
  \MR{4065110}

\bibitem{grothaus}
Martin Grothaus and Robert Vo\ss{}hall, \emph{Stochastic differential equations
  with sticky reflection and boundary diffusion}, Electron. J. Probab.
  \textbf{22} (2017), Paper No. 7, 37. \MR{3613700}

\bibitem{MR126883}
Nobuyuki Ikeda, \emph{On the construction of two-dimensional diffusion
  processes satisfying {W}entzell's boundary conditions and its application to
  boundary value problems}, Mem. Coll. Sci. Univ. Kyoto Ser. A. Math.
  \textbf{33} (1960/61), 367--427. \MR{126883}

\bibitem{MR2448584}
James Kennedy, \emph{An isoperimetric inequality for the second eigenvalue of
  the {L}aplacian with {R}obin boundary conditions}, Proc. Amer. Math. Soc.
  \textbf{137} (2009), no.~2, 627--633. \MR{2448584}

\bibitem{kolesnikov}
Alexander~V. Kolesnikov and Emanuel Milman, \emph{Brascamp-{L}ieb-type
  inequalities on weighted {R}iemannian manifolds with boundary}, J. Geom.
  Anal. \textbf{27} (2017), no.~2, 1680--1702. \MR{3625169}

\bibitem{konarovskyi2017reversible}
Vitalii Konarovskyi and Max von Renesse, \emph{Reversible
  coalescing-fragmentating {W}asserstein dynamics on the real line}, 2017.

\bibitem{Nazarov}
Nikolay Kuznetsov and Alexander Nazarov, \emph{Sharp constants in the
  {P}oincar\'{e}, {S}teklov and related inequalities (a survey)}, Mathematika
  \textbf{61} (2015), no.~2, 328--344. \MR{3343056}

\bibitem{MR1443055}
Yanyan Li and Meijun Zhu, \emph{Sharp {S}obolev trace inequalities on
  {R}iemannian manifolds with boundaries}, Comm. Pure Appl. Math. \textbf{50}
  (1997), no.~5, 449--487. \MR{1443055}

\bibitem{Matculevich_Repin}
Svetlana Matculevich and Sergey Repin, \emph{Explicit constants in
  {P}oincar\'{e}-type inequalities for simplicial domains and application to a
  posteriori estimates}, Comput. Methods Appl. Math. \textbf{16} (2016), no.~2,
  277--298. \MR{3483617}

\bibitem{MR1283033}
Umberto Mosco, \emph{Composite media and asymptotic {D}irichlet forms}, J.
  Funct. Anal. \textbf{123} (1994), no.~2, 368--421. \MR{1283033}

\bibitem{MR3154581}
Delio Mugnolo, Robin Nittka, and Olaf Post, \emph{Norm convergence of sectorial
  operators on varying {H}ilbert spaces}, Oper. Matrices \textbf{7} (2013),
  no.~4, 955--995. \MR{3154581}

\bibitem{MR2255233}
Delio Mugnolo and Silvia Romanelli, \emph{Dirichlet forms for general
  {W}entzell boundary conditions, analytic semigroups, and cosine operator
  functions}, Electron. J. Differential Equations (2006), No. 118, 20.
  \MR{2255233}

\bibitem{Nazarov_Repin}
A.~I. Nazarov and S.~I. Repin, \emph{Exact constants in {P}oincar\'{e} type
  inequalities for functions with zero mean boundary traces}, Math. Methods
  Appl. Sci. \textbf{38} (2015), no.~15, 3195--3207. \MR{3400329}

\bibitem{nonnenmacher2018overdamped}
Andreas Nonnenmacher and Martin Grothaus, \emph{Overdamped limit of generalized
  stochastic hamiltonian systems for singular interaction potentials}, 2018.

\bibitem{reilly}
Robert~C. Reilly, \emph{Applications of the {H}essian operator in a
  {R}iemannian manifold}, Indiana Univ. Math. J. \textbf{26} (1977), no.~3,
  459--472. \MR{474149}

\bibitem{MR2384747}
Julio~D. Rossi, \emph{First variations of the best {S}obolev trace constant
  with respect to the domain}, Canad. Math. Bull. \textbf{51} (2008), no.~1,
  140--145. \MR{2384747}

\bibitem{MR3951758}
Abdolhakim Shouman, \emph{Generalization of {P}hilippin's results for the first
  {R}obin eigenvalue and estimates for eigenvalues of the bi-drifting
  {L}aplacian}, Ann. Global Anal. Geom. \textbf{55} (2019), no.~4, 805--817.
  \MR{3951758}

\bibitem{Shubin}
M.~A. Shubin, \emph{Pseudodifferential operators and spectral theory}, second
  ed., Springer-Verlag, Berlin, 2001, Translated from the 1978 Russian original
  by Stig I. Andersson. \MR{1852334}

\bibitem{MR4176673}
Kazuaki Taira, \emph{Boundary value problems and {M}arkov processes}, third
  ed., Lecture Notes in Mathematics, vol. 1499, Springer, Cham, [2020]
  \copyright 2020, Functional analysis methods for Markov processes.
  \MR{4176673}

\bibitem{MR929208}
Satoshi Takanobu and Shinzo Watanabe, \emph{On the existence and uniqueness of
  diffusion processes with {W}entzell's boundary conditions}, J. Math. Kyoto
  Univ. \textbf{28} (1988), no.~1, 71--80. \MR{929208}

\bibitem{MR121855}
A.~D. Ventcel, \emph{On boundary conditions for multi-dimensional diffusion
  processes}, Theor. Probability Appl. \textbf{4} (1959), 164--177. \MR{121855}

\bibitem{MR1988680}
Hendrik Vogt and J\"{u}rgen Voigt, \emph{Wentzell boundary conditions in the
  context of {D}irichlet forms}, Adv. Differential Equations \textbf{8} (2003),
  no.~7, 821--842. \MR{1988680}

\bibitem{MR2277314}
Mahamadi Warma, \emph{The {R}obin and {W}entzell-{R}obin {L}aplacians on
  {L}ipschitz domains}, Semigroup Forum \textbf{73} (2006), no.~1, 10--30.
  \MR{2277314}

\bibitem{MR287612}
Shinzo Watanabe, \emph{On stochastic differential equations for
  multi-dimensional diffusion processes with boundary conditions. {II}}, J.
  Math. Kyoto Univ. \textbf{11} (1971), 545--551. \MR{287612}

\end{thebibliography}



\end{document} 



