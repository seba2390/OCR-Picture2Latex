\section{Reduction to Kronecker Quiver}\label{reduction}

Through Proposition \ref{cshrunksubrep}, if a representation $W$ is not $\sigma$-semi-stable, we can still measure its closeness to $\sigma$-semi-stability by finding a subrepresentation $W'$ with $\sigma(\dimvec(W'))$ maximal. In \cite{CK21B}, this is called the \emph{discrepancy} of $(W,\sigma)$. Note that we are now no longer limited to the Kronecker quiver --- we can now ask this question for any acyclic quiver $W$, for any~$\sigma$. We call a subrepresentation $W'$ which maximizes $c=\sigma(\dimvec(W'))$ an optimal $\sigma$-witness. When $\sigma$ is understood, we call this $W'$ an \emph{optimal witness}. We can generalize Lemma \ref{combinecshrunk} for subrepresentations.

\begin{proposition}
If $W_1,W_2$ are optimal $\sigma$-witnesses of $W$, then so are $W_1\cap W_2$ and $W_1+W_2$. In particular, there is a minimal and maximal optimal $\sigma$-witness.
\end{proposition}
\begin{proof}
Let $c$ be the discrepancy of $(W,\sigma)$. Let $\sigma_+(x)=\max\{0,\sigma(x)\}$, and similarly, $\sigma_-(x)=-\min\{0,\sigma(x)\}$. For $i=1,2$, by assumption $$\sigma(\dimvec(W_i))=\sum_{x\in Q_0} \big(\sigma_+(x)-\sigma_-(x)\big)\dim W_i(x)=c.$$ Let $W_3=W_1\cap W_2$, $W_4=W_1 + W_2$. We then have:
\begin{multline*}
c+c\geq \sum \big(\sigma_+(x)-\sigma_-(x)\big)\dim W_3(x)+\big(\sigma_+(x)-\sigma_-(x)\big)\dim W_4(x)=\\=
\sum \sigma_+(x)\dim W_3(x)-\sigma_-(x)\dim W_3(x)+\sigma_+(x)\dim W_4(x)-\sigma_-(x)\dim W_4(x)=\\=
\sigma_+(x)\big(\dim W_1(x)+\dim W_2(x)\big)-\sigma_-(x)\big(\dim W_1(x)+\dim W_2(x)\big)=c+c.
\end{multline*}
We conclude that  $\sigma(\dimvec(W_3))=\sigma(\dimvec(W_4))=c,$ as $c$ is maximal.
Therefore, $W_3$ and $W_4$ are optimal $\sigma$-witnesses. We can find a minimal optimal $\sigma$-witness by taking the intersection of all optimal $\sigma$-witnesses, and similarly find a maximal optimal $\sigma$-witness by taking the sum of all optimal $\sigma$-witnesses.
\end{proof}

We would like to extend the techniques in \cite{IQS17} in order to find an optimal $\sigma$-witness. To do this, we reduce any acyclic quiver to the Kronecker quiver. We use the construction described in \cite{DM18}, but provide an altered set up, using presentations as in \cite{DF15}. Let $P_x$ be the indecomposable representation of $Q$ with basis given by all paths starting at vertex ~$x$. 

Let $P_x$ be the indecomposable projective representation corresponding to vertex $x$. So, $P_x(y)=e_y\F Qe_x$, with basis given by paths from $x$ to $y$. Let $\mathbf{P_1}:=\bigoplus_{x\in Q_0}P_x\,^{\sigma_-(x)}$, and $\mathbf{P_0}:=\bigoplus_{x\in Q_0}P_x\,^{\sigma_+(x)}$. Consider all possible morphisms $\varphi$ between the quiver representations $$\varphi:\mathbf{P_1}\rightarrow\mathbf{P_0}.$$
To our above set of morphisms, apply $\Hom(\cdot,W)$ to get
$$A(\varphi):\Hom(\mathbf{P_0},W)\rightarrow\Hom(\mathbf{P_1},W),$$
where $A(\varphi):=\Hom(\varphi,W)$. We can consider a subspace $\Hom(P_x\,^{\sigma_+(x)},W')$ as $Z_+(x)\otimes W'(x)$ for some $Z_+(x)=\F\,^{\sigma_+(x)}$. Notice $\Hom(\mathbf{P_0},W)$ is a right $\End(\mathbf{P_0})$-module by precomposition. Let $x,y$ be so that both $\sigma_+(x)$ and $\sigma_+(y)$ are positive. Note $\End(\mathbf{P_0})$ contains $H=\prod_{x\in Q_0}\GL(\sigma_+(x))$, a reductive group, which acts on the $Z_+(x)$, leaving the $W(x)$ alone. So, an $H$-subrepresentation of $\bigoplus_{x\in Q_0} Z_+(x)\otimes W(x)$ must be of the form $\bigoplus_{x\in Q_0}Z_+(x)\otimes W'(x)$ for some subspaces $W'(x)$ of each $W(x)$. Our set of maps can also be considered between the spaces 

$$A(\varphi):\bigoplus_{x\in Q_0} W(x)\,^{\sigma_+(x)}\rightarrow\bigoplus_{x\in Q_0}W(x)\,^{\sigma_-(x)}.$$

Now, we have a matrix space $\A$ consisting of all $A(\varphi)$. This is the space of block matrices with blocks mapping $W(x)$ to $W(y)$ given by a linear combination of $W(p)$, where $p$ is a path from $x$ to $y$. For this new Kronecker Quiver, we may run the algorithm in \cite{IQS17} to get the minimal $c-$shrunk subspace of $\bigoplus_{x\in Q_0} W(x)\,^{\sigma_+(x)}$, $U$. 

\begin{lemma}
The minimal $c$-shrunk subspace, $U\subseteq\Hom(\mathbf{P_0},W)$, is a left $\End(\mathbf{P_0})$ module, and $\sum_\varphi A(\varphi) U$ is a left $\End(\mathbf{P_1})$ module.
\end{lemma}\label{fixedlem}
\begin{proof}
First, we prove that given any $c$-shrunk subspace, $U$, and invertible $T$ in $\End(\mathbf{P_0})$, $T\cdot U$ is also $c$-shrunk. We have the image of $T \cdot U$:

$$
\sum_\varphi A(\varphi) (T \cdot U)=\sum_\varphi A(\varphi \cdot T)U =\sum_\varphi A(\varphi) U. 
$$
Here the sum is taken over all morphisms $\varphi$ as above. 
It follows that
$$
\dim \sum_\varphi A(\varphi) (T \cdot U)=\dim \sum_{\varphi} A(\varphi) U.
$$
As $T$ is an automorphism, we also have $\dim T \cdot U=\dim U$, so $T \cdot U$ is $c$-shrunk. If $U$ is the minimal $c$-shrunk subspace, $T \cdot U$ is also $c$-shrunk and of the same dimension, so $T \cdot U=U$. As $\End(\mathbf{P_0})$ is spanned by invertible elements, this shows that the minimal $c$-shrunk subspace $U$ is a left $\End(\mathbf{P_0})$ module. Similarly, given $S$ in $\End(\mathbf{P_1})$, we see that \[S\cdot\sum_\varphi A(\varphi) (U)=\sum_\varphi A(S\cdot\varphi)U =\sum_\varphi A(\varphi) U.\]
\end{proof}

\begin{theorem}
Given the minimal $c-$shrunk subspace for the set of linear maps
$$A(\varphi):\bigoplus_{x\in Q_0} W(x)\,^{\sigma_+(x)}\rightarrow\bigoplus_{x\in Q_0}W(x)\,^{\sigma_-(x)},$$
we can construct a subrepresentation of $W$, $W'$, so that $\sigma(\dimvec(W'))$ is maximal. Furthermore, $\sigma(\dimvec(W'))=c$. 
\end{theorem}

\begin{proof}
Considered as a subspace of $\bigoplus Z(x)\otimes W(x)$, the minimal $c$-shrunk $U$ is of the form $\bigoplus Z(x)\otimes W'(x)$, for some subspaces $W'(x)$ of $W(x)$. For $y$ so that $\sigma_+(y)=0$, define $\displaystyle W'(y)=\sum_{a:x\rightarrow y}W(a)W'(x)$. This ensures we have a subrepresentation. Note that $c\leq \sum\dim(W'(x)^{\sigma_+(x)})-\sum\dim(W'(x)^{\sigma_-(x)})$, but $c$ is maximal, so $\sigma(\dimvec(W'))=c$. We note that the $W'(y)$ are similarly closed under the action of $\End(\mathbf{P_1})$.

If there were a subrepresentation $W''$ with $\sigma(\dimvec(W''))$ less than $c$, Note that $U'=\displaystyle\bigoplus_{x\in Q_0} W''(x)^{\sigma_+(x)}$ is a shrunk subspace, with $\dim(U')-\dim(\A( U'))>c$, so $c$ would not be maximal.
\end{proof}

\subsection{Algorithms}

After using this reduction of a quiver representation to a Kronecker quiver, we can employ any previous algorithms or other techniques for finding a $c-$shrunk subspace. 
 If we successfully find a $c-$shrunk subspace, $U$, that is not minimal, we can construct a $c-$shrunk subspace that is fixed under the action of $\End(P_1)$ by taking instead $$\bigcap_{T\in \End(\mathbf{P_1})}T\cdot U.$$ Such a subspace will give a optimal $\sigma$ witness. We may use a basis of $\End(\mathbf{P_1})$ to get this subspace in polynomial time. 
 
 In \cite{IKQS15}, Wong sequences, originally defined by Kai-Tek Wong \cite{Wong74}, are used in certain cases to find a $c-$shrunk subspace. In \cite{IQS17}, blow-ups are used to extend the original algorithm to find a $c-$shrunk subspace for any collection of matrices. The algorithm takes a matrix $A$ in $\A$, constructing a sequence starting with $W_0=0$, and letting $W_{i+1}=\A A^{-1}(W_i)$. This sequence stabilizes to some subspace, $W^*$. In the case that $W^*$ is contained in $\im{A}$, $A^{-1}(W^*)$ is a $c-$shrunk subspace with $c$ maximal and equal to $n-\rk{A}$. In this case, where the algorithm returns a $c-$shrunk subspace, we claim the subspace is minimal.

The minimal shrunk subspace, $U$, is the intersection of all $c-$shrunk subspaces, so $U\subseteq A^{-1}(W^*)$. The limit of the sequence $W^*$ is the smallest subspace $Z$ so that $\displaystyle\bigcup_{i=1}^m A_i^{-1}(Z)$ contains $A^{-1}(Z)$. So by minimality, $U$ is returned when this sequence terminates with $W^*$ contained in $\im(A)$. In the case where blow-ups are invoked to find a $c-$shrunk subspace, the same sequence is used in the larger space, finding a $cd-$shrunk subspace. This by the same reasoning must be minimal, so when pulled back to a $c-$shrunk subspace in the original space, it must remain minimal.  

Let $n=\min\{\sum \sigma_+(x)\dim W(x),\sum \sigma_-(y)\dim W(y)\}$. For sufficiently large fields, ($|\F|>2n$) there is a randomized algorithm to find a $c-$shrunk subspace \cite[Corollary 1.5]{IQS17}. This randomized algorithm is much simpler and typically must faster than the deterministic algorithm. In the context of representations, this algorithm immediately after reduction, blow up by a sufficiently large \cite{IQS18,DM18} factor, $d\geq n-1$. In this blow-up, randomly choose a matrix
$$A:\bigoplus_{x\in Q_0} W(x)\,^{d\sigma_+(x)}\rightarrow\bigoplus_{x\in Q_0}W(y)\,^{d\sigma_-(y)},$$
where $A$ is in $\A^{\{d\}}:=M(d,\F)\otimes\A$.
Through the Schwarz-Zippel-DeMillo-Lipton lemma \cite{Schw80,Zip79,DL78}, if a field is large enough, evaluating a non-zero polynomial over that field at a randomly chosen point is likely to give a non-zero result. Taking the determinant of minors of a matrix in the blow-up, we are likely to have $\rk A= \ncrk \A^{\{d\}}$. Thus, running the Wong sequence on this $A$ will result in the return of a $cd-$shrunk subspace \cite[Lemma 9]{IKQS15}. From this $cd$-shrunk subspace in the blow-up, we can find a $c-$shrunk subspace of $\bigoplus_{x\in Q_0} W(x)\,^{\sigma_+(x)}$, constructing a subrepresentation as above.

The deterministic Wong sequence algorithm for finding non-commutative rank, introduced in \cite{IKQS15}, uses a sequence of subspaces, testing its limit, $W^*$ for evidence of a $c$-shrunk subspace. In the quiver representation context, we would like to instead use a sequence of subrepresentations. In this deterministic setting, we only need $|\F|>n$.

To do this, we again start with a random matrix $A$ in the blow-up, as above. Next, find a pseudo-inverse of $A$, a matrix $B$ so that $B$'s restriction to $\im(A)$ is the inverse to $A$'s restriction to a direct complement of $\ker(A)$. Note that $B$ is a block matrix as well, with blocks mapping each $W(y)$ for $\sigma(y)<0$ to each $W(x)$ with $\sigma(x)>0$. Let $I_x$ index the $|d\sigma(x)|$ copies of $W(x)$. Let $\pi_{x,i}:\bigoplus_{x\in Q_0} W(x)\,^{d\sigma_+(x)}\rightarrow W(x)$ be the projection to the $i$th copy of $W(x)$. Each projection can be thought of as coming from the action of $\End(\mathbf{P_0})$. Similarly, define this for vertices $y$ with $\sigma_{-}(y)>0$. 

For each block, take the projection $\pi_{y,i} B \pi_{x,j}$. This gives a linear map from $W(x)$ to $W(y)$. Construct a new quiver representation, $W^{+}$, on a new quiver $Q^{+}$ by adding arrows $p:y\rightarrow x$ for each block in the pseudo-inverse, with each $W^{+}(p)$ defined as $\pi_{y,i} B \pi_{x,j}$. 

Define a subspace at vertices $x$ with $\sigma(x)>0$ of $W^{+}$: 
$$K(x):=\sum_{i\in I_{x}} \pi_{x,i}\ker(A).$$
For all other vertices, define $K(y)=0.$ Let $W'$ be the smallest subrepresentation of $W^{+}$ containing each $K(x)$. Note that $W'$ must also be a subrepresentation of our original $W$. 

\begin{proposition}
For $W'$ as defined above, $\bigoplus_{x\in Q_0} W'(x)\,^{d\sigma_+(x)}$ is $cd$-shrunk, with image (under $\A^{[d]}$) $\bigoplus_{x\in Q_0} W'(x)\,^{d\sigma_-(x)}$. Thus, $W'$ is an optimal $\sigma$ witness.
\end{proposition}

First, we claim that $\bigoplus_{x\in Q_0} W'(x)\,^{d\sigma_+(x)}$ is the minimal $cd$-shrunk subspace of $\A^{\{d\}}$. By construction, the Wong sequence algorithm returns the smallest subspace containing $\ker(A)$, and closed under $\A^{\{d\}}$ and our pseudo-inverse $B$. The $K(x)$ must remain inside the minimal shrunk subspace, as the projections come from $\End(\mathbf{P_0})$. Similarly, the new maps in $W^{+}$ come from the action of $\End(\mathbf{P_0})\bigoplus\End(\mathbf{P_1})$, so in finding the smallest subrepresentation, we must still remain in the minimal shrunk subspace (at positive vertices). So in finding the minimal representation of $W^+$ that contains each $K(x)$, $W'$, we get the smallest subspace $\bigoplus_{x\in Q_0} W'(x)\,^{d\sigma_+(x)}$ containing $\ker(A)$ and closed under $\A^{\{d\}}$ and $B$, i.e. the minimal $cd$-shrunk subspace. 

\begin{proposition}
Given a quiver representation $W$, a weight vector $\sigma$, $|\F|>n$, letting $n_x:=\dim(W(x))$, and $N=\sum_{x\in Q_0} n_x$, there is an algorithm polynomial time in the $n_x$ to find an optimal $\sigma$ witness.
\end{proposition}

Recalling the above discussion, we first construct $Q^+$ and $W^+$. To do this, we chose a random matrix in the $d=N-1$ blowup, $A$, and find its pseudo-inverse, $B$, which takes polynomial time ($\leq(dN)^3$). We then construct new linear maps for each of the $d^2\sigma_+\sigma_-$ blocks in $B$ by composing $B$ with projection maps. This composition is matrix multiplication, which can be done in polynomial time. Next we contruct $K(x)$ at each vertex $x$, which is the sum over the $d\sigma(x)$ projections of $\ker(A)$. We can find a basis for $\ker(A)$ itself in polynomial time using row reduction. Last, we use Algorithm \ref{findwitness} to loop through all our arrows $N$ times, to find the optimal $\sigma$ witness, $W'$ from the $K(x)$. This algorithm will stabilize at the $N$th loop or shorter, as each iteration of the outside loop will either raise the dimension of the current $W'$, or will not (in which case, we are done, we have found the final $W'$). We can increase the dimension at most $N$ times, so this must be a correct bound for the number of times to run the outer loop. 

\begin{alg}\label{findwitness}
Algorithm for finding $W'$.\\
\emph{Input:} Quiver $Q$, Representation $W$ of $Q$, subspaces $K(x)\subseteq W(x)$ for all vertices $x$.\\
\emph{Output:} Smallest subrepresentation $W'$ so that $K(x)\subseteq W'(x)$ for all vertices $x$.
\begin{algorithmic}[1]
\State $W'(x) = K(x)$ for all $x$;
\For{$i=1$ \textbf{ to } $N$}
    \For{$a \in Q_1$}
       \State $W'(ha)= W'(ha)+W(a)W'(ta)$;
    \EndFor
\EndFor
\end{algorithmic}
\end{alg}


