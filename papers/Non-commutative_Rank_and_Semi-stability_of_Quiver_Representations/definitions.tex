\section{Non-commutative Rank}\label{ncrkdefs}
We will be concerned with the free skew field, made up of non-commuting polynomials, $\F\langle x_1,\ldots,x_m\rangle$, their inverses, and then enlarged to contain all sums, products, and inverses. The free skew field was first defined by Amitsur, \cite{Amit66}. In the free skew field, there is no standardized way to express elements, and elements may need to be defined with nested inverses. For example, $(x+yz^{-1}w)^{-1}$ can not be written without a nested inverse \cite{HW14}. 

Given a matrix, $A(\underline{x})$, with homogeneous linear polynomials in $\F\langle x_1,\ldots,x_m\rangle$, the non-commutative analogue of the Edmonds' problem asks to determine the rank of $A(\underline{x})$ over the free skew field. We denote this rank by $\ncrk(A)$. Similarly, the NCFullRank problem asks whether $A(\underline{x})$ has full rank over the free skew field. 
For example, we row reduce the following skew symmetric matrix over the free skew field to get:
\begin{equation}\label{OurExample}
T=\begin{bmatrix}
0 & x_1 & x_2\\ 
-x_1 & 0 &x_3 \\ 
-x_2 & -x_3  & 0
\end{bmatrix} \sim \begin{bmatrix}
0 & x_1 & x_2\\ 
-x_1 & 0 &x_3 \\ 
0 & 0  & x_3x_1^{-1}x_2-x_2x_1^{-1}x_3
\end{bmatrix}.
\end{equation}
Unfortunately, by the nature of the free skew field, it is hard to determine polynomial identities --- so it is not immediately clear if this matrix has non-commutative rank $2$ or $3$. For this reason, we explore additional equivalent formulations of non-commutative rank. For sketches on their equivalence, see \cite{IQS17}. 

We note here that many aspects of rank carry over to the non-commutative rank, for instance, the non-commutative row rank and column rank of $A(\underline{x})$ equal the $\ncrk(A)$ and we must have a minor with full rank equal to $\ncrk(A)$. We must still be careful, as other aspects do not: naively finding the ``determinant'' of $A(\underline{x})$, and comparing it to zero will not tell us whether the non-commutative rank is full (in fact, even how to define a single determinant in this context is unclear) \cite{GR91}.


\subsection{Blow-ups}
If $T=x_1A_1+\ldots x_mA_m$, let $\A=\spn{\{A_1,\ldots,A_m\}}$. The $d$th \emph{tensor blow-up} of $\A$ is $$\A^{\{d\}}:=M(d,\F)\otimes\A \subseteq M(dn,\F).$$ 
The rank of a matrix space, $\rk{\A}$, is the maximal $r$ so that there is a matrix with rank $r$ in $\A$. When $\F$ is large enough, $d$ divides the rank of $\A^{\{d\}}$  \cite{IQS17}. We have $$\ncrk(A)=\lim_{d\rightarrow\infty}\frac{\rk{\A^{\{d\}}}}{d}.$$

We may also write $\ncrk(\A)$ instead of $\ncrk(A)$. The value of $(\rk{\A^{\{d\}}})/d$ is increasing as $d$ increases, and is bounded by $n$. Derksen and Makam proved that if $\A$ has maximal non-commutative rank, then taking $d\geq n-1$ ensures $\rk{\A^{\{d\}}}=nd$ \cite{DM18}. 
If $\ncrk(A)=r<n$, then restricting to a full rank $r\times r$ submatrix of $A(\underline{x})$, we see
that $\rk{\A^{\{d\}}}=nd$ for $d\geq r-1$.
So we always have  $\rk{\A^{\{d\}}}=nd$ for $d\geq n-1$.


For our example~(\ref{OurExample}), take $d=2$. We then look for $2\times 2$ matrices $D_1,D_2,D_3$, so that $A_1\otimes D_1+A_2\otimes D_2+A_3\otimes D_3$ has max rank. Letting 
$$D_1=\begin{bmatrix}
1 & 0\\ 
0 & 0
\end{bmatrix}\!,\; D_2=\begin{bmatrix}
0 & 0\\ 
0 & 1 \\ 
\end{bmatrix}\!,\; D_3=\begin{bmatrix}
0 & 1 \\ 
1 & 0 \\ 
\end{bmatrix},$$

We find $$\rk{\left[
\begin{array}{c|c|c}
0 & D_1 & D_2\\
\hline
-D_1 & 0 & D_3 \\
\hline
-D_2 & -D_3 & 0
\end{array}
\right]}=6,$$
which must be maximal, and so $\ncrk(T)=3$.


\subsection{$c$-shrunk subspaces}
A subspace $U\subseteq \F^n$ is a \emph{$c$-shrunk subspace} of $\A$ if there exists a subspace $W\subseteq \F^n$ with $\dim(W)\leq\dim(U)-c$, and for every $A$ in $\A$, $A(U)\subseteq W$. The NCFullRank problem is equivalent to determining whether $\A$ has no $c$-shrunk subspace for $c>0$~\cite{Cohn95}. More generally \cite{FR04}, 
$$\ncrk(\A)=n-\max\{c \mid \text{there is a $c$-shrunk subspace of $\A$}\}.$$

Throughout the rest of this paper, we let $c=n-\ncrk(\A)$, i.e. all $c$-shrunk subspaces discussed are so that $c$ is maximal. 

\begin{lemma}\label{combinecshrunk}  Let $c=n-\ncrk(\A)$. If $U_1,U_2$ are $c$-shrunk subspaces of $\A$, then so are $U_1\cap U_2$ and $U_1+U_2$.
\end{lemma}
\begin{proof} By assumption $\dim U_i-\dim \A(U_i)=c$. Let $U_3=U_1\cap U_2$, $U_4=U_1+U_2$. We then have:
\begin{multline*}
c+c\geq ( \dim U_3-\dim \A(U_3))+(\dim U_4-\dim \A(U_4))=\\ = ( \dim(U_1\cap U_2)+\dim(U_1+U_2)) -(\dim(\A(U_1)\cap \A(U_2))+\dim(\A(U_1)+\A(U_2)))\geq\\ \geq
(\dim U_1+\dim U_2)-(\dim \A(U_1)+\dim \A(U_2))=\\=(\dim U_1-\dim \A(U_1))+(\dim U_2-\dim \A(U_2))=c+c.
\end{multline*}
We conclude that  $\dim U_3-\dim \A(U_3)=\dim U_4-\dim \A(U_4)=c,$ as $c$ is maximal.
Therefore, $U_3$ and $U_4$ are $c$-shrunk subspaces.
\end{proof}

In particular, there is a unique $c$-shrunk subspace of the lowest dimension, namely, the intersection of all $c$-shrunk subspaces. A recent similar discussion can be found in \cite{IMQ21}. In our skew-symmetric example~(\ref{OurExample}), although any matrix in $\A$ has rank $2$, the image of any subspace $U$ of $\F^3$ has the same dimension as $U$. In this case $c=n-\ncrk(\A)=3-3=0$, and the minimal $c$-shrunk subspace is the zero subspace.


\subsection{Semi-stability of Kronecker quiver}
A quiver $Q$ is a directed graph, with vertex set denoted $Q_0$ and arrow set denoted $Q_1$. A representation $W$ of a quiver $Q$, is an assignment of finite dimensional $\F$ vector spaces $W(x)$ to each $x$ in $Q_0$, and an assignment of linear maps $W(a)$ to each $a$ in $Q_1$. We let $ha$ and $ta$ denote the head and tail vertices of the arrow $a$ respectively. A dimension vector is a function $\alpha:Q_0\to {\mathbb N}=\{0,1,2,\dots\}$.
The dimension vector $\underline{\dim} W$
of a representation $W$ is defined by $(\underline{\dim} W)(x)=\dim W(x)$.
Fixing $Q$ and a dimension vector $\alpha$, there is an action on quiver representations by $\GL(\alpha):=\prod_{x\in Q_0} \GL(\alpha(x))$. The action of $(Y(x),x\in Q_0)$ takes $W(a)$ to $Y(ha)W(a)Y(ta)^{-1}$ for all $a\in Q_1$, and leaves each $W(x)$ with $x\in Q_0$ unchanged. For a path $p=a_ja_{j-1}\cdots a_1$, we denote by $W(p)$ the composition of linear maps $W(a_j)W(a_{j-1})\cdots W(a_1)$. The empty path from vertex $x$ to itself is denoted by $e_x$ and $W(e_x)$ is defined as the identity map of $W(x)$.

The representations of $Q$ with dimension vector $\alpha$ (indexed by the vertices) is denoted $\Rep_\alpha(Q)$. A representation is semi-stable if its orbit closure does not contain the zero representation; representations that are not semi-stable define the \emph{nullcone}. No acyclic quiver representations are semi-simple. Instead, for a weight $\sigma$ in $\Z^{Q_0}$, we additionally use the $1$-dimensional representation $\chi_\sigma$, a \emph{character} with action of $\GL(\alpha)$ given by multiplication by
$$\chi_\sigma(Y(x),x\in Q_0)=\prod_{x\in Q_0}\det(Y(x))^{\sigma(x)}.$$
A representation $W$ is $\sigma$ semi-stable if $(W,1)$ is semi-stable in $\Rep_\alpha(Q)\oplus \chi_\sigma$.

The NCFullRank problem for $T=x_1A_1+\ldots,x_mA_m$ is equivalent to determining whether the quiver representation $W$,

\begin{center}
\begin{tikzcd}
\F^n \arrow[r, draw=none, "\raisebox{+1.5ex}\vdots" description] \arrow[r,shift left=1.7ex,"A_1"]  \arrow[r,shift right=1.2ex,swap,"A_m"] & \F^n
\end{tikzcd}\end{center}

is $\sigma$-semistable, for $\sigma=(1,-1)$. In our skew-symmetric matrix example~(\ref{OurExample}), we would like to determine whether the above quiver with
$$A_1=\begin{bmatrix}
0 & 1 & 0\\ 
-1 & 0 & 0\\ 
0 & 0 & 0
\end{bmatrix}\!,\; A_2=\begin{bmatrix}
0 & 0 & 1\\ 
0 & 0 & 0\\ 
-1 & 0 & 0
\end{bmatrix}\!,\; A_3=\begin{bmatrix}
0 & 0 & 0\\ 
0 & 0 & 1\\ 
0 & -1 & 0
\end{bmatrix}$$
is $(1,-1)$-semi-stable.

We would like to be able to relate quiver representations to the non-commutative rank, rather than just to NCFullRank. To do this, we need a way of measuring how far a representation $V$, is from being $\sigma$-semistable. For this, we use King's Criterion \cite{King94}. For a representation $W$, let $\sigma(\dimvec(W))=\sum \dim(W(x))\sigma(x)$.

\begin{proposition}[King's Criterion, \cite{King94}]
A representation $W$ in $\Rep_\alpha(Q)$ is $\sigma$-semi-stable if and only if $\sigma(\alpha)=0$ and $\sigma(\dimvec(W))\leq 0$ for all subrepresentations $W'$ of $W$.
\end{proposition}

\begin{proposition}\label{cshrunksubrep}
    Given $\A=\spn{\{A_1,\ldots,A_m\}}$, and $c$ the maximum $\sigma(\dimvec(W'))$ over all subrepresentations $W'$ of the Kronecker quiver with maps $\{A_i\}$, the $\ncrk(\A)=n-c$.
\end{proposition} 

\begin{proof}
If $W$ is a Kronecker quiver, and $\sigma=(1,-1)$, let $W'$ be a subrepresentation with $c:=\sigma(\dimvec(W'))$ maximal. Then, since $\A(W'(x))$ is contained in $W'(y)$, $W'(x)$ is a $c$-shrunk subspace. On the other hand, if instead we start with a $c$-shrunk subspace $U$,  $W'(x):=U$, and $W'(y):=\sum_{i=1}^{m}A_i\, U$. This defines a subrepresentation $W'$, where $\sigma(\dimvec(W'))=c$. So for Kronecker quivers, $c$-shrunk subspaces give us subrepresentations $W'$ with $\sigma(\dimvec(W'))$ maximal, and vice-versa. 
\end{proof}

So, the non-commutative rank of $\A$ is equal to the maximum of $\sigma(\dimvec(W'))$ over all subrepresentations $W'$ of the Kronecker quiver with maps $A_1,\ldots,A_n$.


