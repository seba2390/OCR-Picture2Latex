\documentclass[12pt]{amsart}
\usepackage[utf8]{inputenc}

\title{Non-commutative Rank and Semi-stability\\ of Quiver Representations}
\author{Alana Huszar}

\usepackage{amsmath}								%math
\usepackage{amssymb}
\usepackage{amsthm}
\usepackage{amscd}
\usepackage{amsfonts}
\usepackage{tikz}
\usetikzlibrary{matrix}
\usetikzlibrary{cd}
\usetikzlibrary{babel}
\usepackage{algpseudocode}
\usepackage{algorithm}

\usepackage[pdftitle={Quivers and Non-Commuative Rank},pdfauthor={Alana Huszar},ocgcolorlinks,linkcolor=linkblue,citecolor=linkred,urlcolor=linkred]{hyperref}
\usepackage{color}
\definecolor{linkred}{rgb}{0.6,0,0}
\definecolor{linkblue}{rgb}{0,0,0.6}

\usepackage{fullpage}
\usepackage{enumerate}

\linespread{1.1}
\setlength{\parindent}{.5cm}

\newcommand{\ds}{\displaystyle}
\newcommand{\Mat}{\operatorname{Mat}}


\newtheorem{theorem}{Theorem}[section]
\newtheorem{proposition}[theorem]{Proposition}
\newtheorem{lemma}[theorem]{Lemma}
\newtheorem{corr}[theorem]{Corollary}
\newtheorem{alg}[theorem]{Algorithm}
\theoremstyle{definition}
\newtheorem{claim}[theorem]{Claim}
\newtheorem{question}[theorem]{Question}
\newtheorem{conjecture}[theorem]{Conjecture}
\newtheorem{definition}[theorem]{Definition}
\newtheorem{example}[theorem]{Example}

\newcommand{\rk}{\operatorname{rk}}
\newcommand{\Span}{\operatorname{span}}
\newcommand{\Hom}{\operatorname{Hom}}
\newcommand{\Aut}{\operatorname{Aut}}
\newcommand{\ext}{\operatorname{ext}}
\newcommand{\Rep}{\operatorname{Rep}}
\newcommand{\ncext}{\operatorname{ncext}}
\newcommand{\nchom}{\operatorname{nchom}}
\newcommand{\dimvec}{\underline{\rm dim}}
\newcommand{\im}{\operatorname{Im}}

\newcommand{\A}{\mathcal{A}}
\newcommand{\SL}{\operatorname{SL}}
\newcommand{\GL}{\operatorname{GL}}
\newcommand{\R}{{\mathbb R}}
\newcommand{\N}{{\mathbb N}}
\newcommand{\C}{{\mathbb C}}
\newcommand{\F}{{\mathbb F}}
\newcommand{\Z}{{\mathbb Z}}
\newcommand{\Q}{{\mathbb Q}}
\newcommand{\crk}{\operatorname{crk}}
\newcommand{\ncrk}{\operatorname{ncrk}}
\newcommand{\spn}{\operatorname{Span}}
\newcommand{\Ext}{\operatorname{Ext}}
\newcommand{\End}{\operatorname{End}}
\newcommand{\fmap}[2]{f_{#1}^{#2}}


\begin{document}

\begin{abstract}
Fortin and Reutenauer defined the non-commutative rank for a matrix with entries that are linear functions. The  non-commutative rank is related to stability in invariant theory,
non-commutative arithmetic circuits, and Edmonds' problem. We will generalize the  non-commutative rank to the representation theory of quivers and define 
non-commutative Hom and Ext spaces. We will relate these new notions  to King's criterion for $\sigma$-stability
of quiver representations, and the general Hom and Ext spaces studied by Schofield. We discuss polynomial time algorithms that compute
the non-commutative Homs and Exts and find an optimal witness for the $\sigma$-semi-stability
of a quiver representation.

\end{abstract}

\maketitle

\section{Introduction}
Given $A_1,A_2,\dots,A_m$, $n\times n$ matrices over a field $\F$, and $x_1,x_2,\dots,x_m$, variables in the free skew-field defined by Cohn in \cite{Cohn95}, Fortin and Reutenauer \cite{FR04} defined the non-commutative rank $\ncrk(A(\underline{x}))$ as the rank of 
the matrix of linear functions $A(\underline{x})=x_1A_1+x_2A_2+\cdots+x_mA_m$ over the free skew field. The non-commutative rank $\ncrk(A(\underline{x}))$ is also equal to the maximal value of 
%$\rk(X_1\otimes A_1+X_2\otimes A_2+\cdots+X_n\otimes A_n)/d$ 
$$\frac{\rk(X_1\otimes A_1+X_2\otimes A_2+\cdots+X_n\otimes A_n)}{d}
$$
where $d$ is a positive integer and $X_1,X_2,\dots,X_n$ are $d\times d$ matrices. A third characterization of non-commutative rank is in terms of shrunk subspaces.
Non-commutative rank is related to the notion of stability in geometric invariant theory. Consider the action of $\SL_n\times \SL_n$ on the space $\Mat_{n,n}^m$ of $m$-tuples of $n\times n$ matrices by left-right multiplication.
Then $\underline{A}=(A_1,A_2,\dots,A_m)$ is semi-stable with respect to this action if and only if $A(\underline{x})$ has full non-commutative rank, i.e., $\ncrk(A(\underline{x}))$ is equal to the maximal value $n$.

The example of $m$-tuples of $n\times n$ matrices fits in the broader framework of representation theory of quivers. A quiver $Q$ is just a directed graph,
with vertex set $Q_0$ and arrow set $Q_1$.
If we consider the generalized Kronecker quiver $Q$ with two vertices, say $x$ and $y$, and $m$ arrows from $x$ to $y$
and we choose the dimension vector $\alpha=(n,n)$, then the representation space $\Rep_\alpha(Q)$ of $\alpha$-dimensional representations of $Q$ is equal to $\Mat_{n,n}^m$. To construct a moduli space of $\alpha$-dimensional representations one needs to quotient out the $\SL_n\times \SL_n$ action. The goal of this paper is to generalize the notion of non-commutative rank and its properties to arbitrary quivers and dimension vectors. Geometric invariant theory for quiver representations was studied by King in \cite{King94}. For every quiver $Q$, dimension vector $\alpha\in \Z_{\geq 0}^{Q_0}$ and weight $\sigma\in \Z^{Q_0}$
King constructed a quotient for $\alpha$-dimensional representations of $Q$ with respect to the weight $\sigma$.
So for every weight $\sigma$ there is a notion of semi-stability for quiver representations and King gave a criterion  $\sigma$-semi-stability of a representation of~$Q$. 

In this paper, we connect non-commutative rank to $\sigma$-semi-stability of quiver representations. Through King's Criterion \cite{King94}, we discuss the importance of special subrepresentations that are optimal in witnessing the $\sigma$-semi-stability. We provide a framework to use existing algorithms to find these optimal $\sigma$-witnesses, as well as provide an algorithm using a sequence of subrepresentations. We then generalize work of Schofield on pairs of general representations, general ext, and general hom to pairs with one representation fixed, non-commutative ext, and non-commutative hom. We conclude by using non-commutative rank methods to demonstrate useful inequalities for the non-commutative ext. 

The Edmonds' problem, posed in 1967, asks to determine the rank of the $n\times n$ matrix $A(\underline{x})$, with homogeneous linear polynomials in $\Z[x_1,\ldots,x_m]$ over $\Q(x_1,\ldots,x_m)$ \cite{Edm67}. The decision version of this question, asking whether $A(\underline{x})$ has full rank or not, is known as the symbolic determinant identity testing problem (SDIT). We are considering instead the rank of $A(\underline{x})$ over the free skew field, and so the question of finding $\ncrk(A)$ is the \emph{non-commutative Edmonds' problem}, and the relaxation in simply deciding whether $A(\underline{x})$ has full non-commutative rank is the \emph{non-commutative full rank problem} (NCFullRank). Letting $\A=\Span\{A_1,A_2,\ldots,A_n\}$, we alternatively denote the non-commutative rank of $A(\underline{x})$ by $\ncrk(\A)$. Ivanyos, Qiao, and Subrahmanyam give equivalent formulations and history of NCFullRank in \cite{IQS17}. We are interested in the $c$-shrunk subspace, tensor blow-up, and particularly the nullcone formulations, which are discussed in Section~\ref{ncrkdefs}.

Lots of work from different angles has been done on this non-commutative rank. Cohn and Reutenauer proved NCFullRank was in PSPACE (can be solved using polynomial space) \cite{CR99}. Fortin and Reutenauer connected non-commutative rank explicitly to $c$-shrunk subspaces \cite{FR04}. Coming from studying non-commutative arithmetic circuits with divisions, Hrubes and Wigderson proved that non-commutative rank was equivalent to rank for large enough tensor blow-ups \cite{HW14}. Garg, Gurvitz, Oliveira, and Wigderson provide a polynomial time algorithm of non-commutative rank for fields of characteristic zero. In \cite{IKQS15}, for certain matrix spaces, Karpinski, Ivanyos, Subrahmanyam, and Qiao use Wong sequences to calculate the non-commutative rank. Building on this using blow-ups, the latter three authors provide an algorithm for finding the non-commutative rank of any matrix space \cite{IQS17}. Utilizing results on bounds from \cite{DM18b}, in \cite{IQS18}, they give a deterministic polynomial time algorithm.  

This problem can be expanded to that of finding a subrepresentation of a quiver representation $W$ that demonstrates the representation's semi-stability. This \emph{optimal $\sigma$-witness}, $W'$, can be found using algorithms finding a $c$-shrunk subsapce of a certain matrix space. Work by Chindris and Kline connect this problem to that of simultaneous robust subspace recovery (SRSR) \cite{CK20a}, and provide an algorithm for certifying the semi-stability of $W$ \cite{CK21B}.

In \cite{Scho92}, Schofield explored the dimension of $\Hom(V,W)$ and $\Ext(V,W)$ for a fixed quiver representation $V$, and generic quivers representation $W$ and $V$. This work was generalized by Crawley-Boevey, who described the asymptotic behavior of these dimensions when one of $W$ or $V$ is fixed (rather than both generic) \cite{Craw96}. We re-prove many of these results using insights from non-commutative rank methods, ultimately leading us to a bound on the asymptotic behavior.

\newcommand\UU{\mathcal{U}}
\newcommand\SC{\mathcal{S}}
\newcommand\MM{\mathbb{M}}
\newcommand\MO{\mathbb{M}\mkern-4mu\downarrow}
\newcommand\Idx{\mathsf{Idx}\,}
\newcommand\Cns{\mathsf{Cns}\,}
\newcommand\Typ{\mathsf{Typ}\,}
\newcommand\Pos{\mathsf{Pos}\,}
\newcommand\Decor{\mathsf{Decor}\,}

\newcommand\Set{\mathcal{S}et}

\newcommand\Id{\mathsf{Id}}
\newcommand\Pb{\mathsf{Pb}\,}
\newcommand\Slice{\mathsf{Slice}\,}
\newcommand\Pd{\mathsf{Tree}\,}
\newcommand\Free{\mathsf{Free}\,}
\newcommand\Slc{\mathsf{Slc}\,}

\newcommand{\ooGrp}{\infty\mhyphen\mathsf{Grp}}
\newcommand{\preooCat}{\mathsf{pre}\mhyphen\infty\mhyphen\mathsf{Cat}}
\newcommand{\ovr}{\mkern-8mu\downarrow}
\newcommand{\smovr}{\mkern-4mu\downarrow}

\newcommand{\dsum}[1]{\textstyle{\sum_{(#1)}}\,}
\newcommand{\dprod}[1]{\textstyle{\prod_{(#1)}}\,}
\mathchardef\mhyphen="2D

\newcommand\refl{\mathsf{refl}}
\newcommand\ttt{\mathsf{tt}}
\newcommand\ctr{\mathsf{ctr}\,}
\newcommand\wit{\mhyphen\mathsf{wit}}
\newcommand\coh{\mhyphen\mathsf{coh}}
\newcommand\alg{\mhyphen\mathsf{alg}}

% Antoine's over commands

\newcommand{\da}{{\downarrow}}

\newcommand\MMd{\mathbb{M\da}}
\newcommand\Idxd{\mathsf{Idx}\da\,}
\newcommand\Cnsd{\mathsf{Cns}\da\,}
\newcommand\Typd{\mathsf{Typ}\da\,}
\newcommand\Posd{\mathsf{Pos}\da\,}

\newcommand\upetad{\upeta\da}
\newcommand\upmud{\upmu\da}

\newcommand\lfd{\operatorname{lf\da}}
\newcommand\ndd{\operatorname{nd\da}}

\newcommand\Idd{\mathsf{Id\da}\,}
\newcommand\Pbd{\mathsf{Pb\da}\,}
\newcommand\Sliced{\mathsf{Slice\da}\,}
% end 

\newcommand\Unit{\top}
\newcommand\Empty{\bot}
\newcommand\botelim{\bot\mhyphen\mathsf{elim}}
\newcommand\etapos{\upeta\mhyphen\mathsf{pos}\,}
\newcommand\etaposelim{\upeta\mhyphen\mathsf{pos}\mhyphen\mathsf{elim}\,}
\newcommand\etadec{\upeta\mhyphen\mathsf{dec}\,}
\newcommand\etadecd{\upeta\mhyphen\mathsf{dec}\da\,}
\newcommand\mupos{\upmu\mhyphen\mathsf{pos}\,}
\newcommand\muposfst{\upmu\mhyphen\mathsf{pos}\mhyphen\mathsf{fst}\,}
\newcommand\mupossnd{\upmu\mhyphen\mathsf{pos}\mhyphen\mathsf{snd}\,}
\newcommand\gammaposinl{\upgamma\mhyphen\mathsf{pos}\mhyphen\mathsf{inl}\,}
\newcommand\gammaposinr{\upgamma\mhyphen\mathsf{pos}\mhyphen\mathsf{inr}\,}
\newcommand\gammaposelim{\upgamma\mhyphen\mathsf{pos}\mhyphen\mathsf{elim}\,}

\newcommand\lf{\mathsf{lf}\,}
\newcommand\nd{\mathsf{nd}\,}
\newcommand\fst{\mathsf{fst}\,}
\newcommand\snd{\mathsf{snd}\,}
\newcommand\inl{\mathsf{inl}\,}
\newcommand\inr{\mathsf{inr}\,}

\newcommand\iscontr{\mathsf{is}\mhyphen\mathsf{contr}\,}
\newcommand\ismult{\mathsf{is}\mhyphen\mathsf{mult}\,}
\newcommand\isfibrant{\mathsf{is}\mhyphen\mathsf{fibrant}\,}
\newcommand\carmult{\mathsf{car}\mhyphen\mathsf{is}\mhyphen\mathsf{mult}\,}
\newcommand\relfib{\mathsf{rel}\mhyphen\mathsf{is}\mhyphen\mathsf{fibrant}\,}
\newcommand\isalgebraic{\mathsf{is}\mhyphen\mathsf{algebraic}\,}

\newcommand\OpType{\mathsf{OpetopicType}\,}
\newcommand\OvrOpType{\da\mathsf{OpType}\,}
\newcommand\Car{\mathcal{C}\,}
\newcommand\Rel{\mathcal{R}\,}

\newcommand\lflf{\operatorname{lf-lf}}
\newcommand\ndlf{\operatorname{nd-lf}}

\newcommand{\commentt}[1]{}


%%% Local Variables:
%%% mode: latex
%%% TeX-master: "lics-article"
%%% End:


To demonstrate that collecting network data can reduce the posterior uncertainty about the parameters of the population model,
we consider a population consisting of $K=3$ subpopulations.
The $K=3$ subpopulations correspond to 
\bi
\item a low-degree subpopulation of size 127 with degree parameter $\gamma_1 = -3.5$;
\item a moderate-degree subpopulation of size 50 with degree parameter $\gamma_2 = -1.5$;
\item a high-degree subpopulation of size 10 with degree parameter $\gamma_3 = .5$.
\ei
% The degree parameters of population members $i$ are given by $\theta_i = \bZ_i^\top\, \bgamma$ ($i = 1, \dots, N$).
We generate 1,000 ego-centric samples of sizes $n = 25$,\, $50$,\, $75$,\, $100$,\, $125$,\, $150$,\, $187$ from the population of size $N = 187$.
We then estimate the population model from each sample of contacts along with observations of the exposure, infectious, and removal times of infected population members.
In addition,
we estimate the population model without observations of contacts,
which corresponds to a sample size of $n=0$,
using observations of the exposure, infectious, and removal times of infected population members.
% As pointed out in Section 5.2,
% collecting network data sampling helps infer the unobserved sources of infections,
% which in turn helps infer the population model.
To assess how much the posterior uncertainty about the parameters of the population model is reduced by sampling contacts,
we use the mean squared error (MSE) of the posterior median and mean of the parameters.


\section{Non-commutative General Ext and Hom}

In \cite{Scho92}, Schofield studied the minimal dimension of $\Ext(V,W)$ for representations $V$ and $W$ of dimension vectors $\alpha$ and $\beta$. Define
$$Z^t(\alpha,\beta):=\{(V,W)\in \Rep_\alpha(Q)\times\Rep_\beta(Q)|\dim\Hom_Q(V,W)\geq t\}.$$
Each of these subsets of $\Rep_\alpha(Q)\times\Rep_\beta(Q)$ are closed. Take $t$ the minimal positive value of $\dim\Hom_Q(V,W)$. Then, $Z^{t+1}(\alpha,\beta)$ is a proper closed subset. We call the pair $(V,W)$ $(\alpha,\beta)$-general if they are in the (open and dense) complement of $Z^{t+1}(\alpha,\beta)$. On this complement, $\dim\Hom_Q(V,W)$ is constant, as is $\dim\Ext_Q(V,W)$. Schofield calls these generic hom and ext respectively:
\begin{eqnarray*}
\hom(\alpha,\beta)&=&\dim(\Hom_Q(V,W)), \text{ and}\\
\ext(\alpha,\beta)&=&\dim(\Ext_Q(V,W)).
\end{eqnarray*}

Crawley-Boevey generalized the generic hom in \cite{Craw96}, which we show along with the generalization of generic ext. To do this, fix a representation $W$ of $Q$, with dimension vector $\beta$. Define now 
$$Z^t(\alpha,W):=\{V\in \Rep_\alpha(Q)|\dim\Hom_Q(V,W)\geq t\}.$$
Again, each of these subsets are closed, and we take $t$ minimal so that $\dim\Hom_Q(V,W)$ is positive. We call $V$ $(\alpha,W)$-general if it is in the complement of $Z^{t+1}(\alpha,W)$.  

\begin{definition}
Let $V$ be an $(\alpha,W)$-general representation. We define the $(\alpha,W)$-general hom and ext as
\begin{eqnarray*}
\hom(\alpha,W)&=&\dim(\Hom_Q(V,W)), \text{ and}\\
\ext(\alpha,W)&=&\dim(\Ext_Q(V,W)).
\end{eqnarray*}
\end{definition}


\begin{lemma} We have 
$$
\ext(\alpha,W)\geq \max\{-\langle \alpha, \dimvec W'\rangle \mid \mbox{$W'$ factor representation of $W$}\}.
$$
\end{lemma}
\begin{proof}
If $V$ is a general representation of dimension $\alpha$, then applying $\Hom_Q(V,\cdot)$ to 
$$0\to W''\to W\to W'\to 0$$
gives
an exact sequence
$$
\cdots\to \Ext_Q(V,W)\to \Ext_Q(V,W')\to 0,
$$
so $\dim \Ext_Q(V,W')\leq \dim \Ext_Q(V,W)$ and $\ext(\alpha,W')\leq \ext(\alpha,W)$.
We get
$$
-\langle \alpha,\dimvec W'\rangle=\ext(\alpha,W')-\hom(\alpha,W')\leq \ext(\alpha,W')\leq \ext(\alpha,W).
$$
\end{proof}

\begin{definition}\label{nchomdef} The non-commutative ext and hom are defined by the following limits of ext and hom:
\begin{eqnarray*}
\ncext(\alpha,W)&=&\lim_{d\to\infty}\frac{\ext(d\alpha,W)}{d}\\
\nchom(\alpha,W)&=&\lim_{d\to\infty}\frac{\hom(d\alpha,W)}{d}.
%\ncext(V,\beta)&=&\lim_{d\to\infty}\frac{\ext(V,d\beta)}{d}\\
%\nchom(V,\beta)&=&\lim_{d\to\infty}\frac{\hom(V,d\beta)}{d}
\end{eqnarray*}
\end{definition}
Note that for every representation $W$ of dimension $\beta$, we have $\nchom(\alpha,W)-\ncext(\alpha,W)$ equal to $\langle\alpha,\beta\rangle$. These limits were originally studied in \cite{Craw96}, though we give them a name to highlight their connection to non-commutative rank, as seen in the next discussion and proposition.

We have a map $$\fmap{W}{\alpha}:\Rep_\alpha \longrightarrow\Hom\Big(\bigoplus_{x\in Q_0}\Hom\big(\F^{\alpha(x)},W(x)\big)\rightarrow\bigoplus_{a\in Q_1}\Hom\big(\F^{\alpha(ta)},W(ha)\big) \Big)$$
given by sending a representation $V$ to the map $\fmap{W}{\alpha}(V)$, which takes the set of $\varphi(x)$ from $\Hom(\F^{\alpha(x)},W(x))$ over all vertices $x$ to the set of maps $\varphi(ha)V(a)-W(a)\varphi(ta)$ over all arrows $a$. Note that the kernel of each $\fmap{W}{\alpha}(V)$ is $\Hom_Q(V,W)$, and and the cokernel is $\Ext_Q(V,W)$. From this point forward, we will refer to the image of $\fmap{W}{\alpha}$ (the set of $\fmap{W}{\alpha}(V)$ over all $V$), as simply $\fmap{W}{\alpha}$ itself. 

Next, we note that we can consider $\Rep_{d\alpha}$ as the blow-up of $\Rep_\alpha$ as follows. Each $Z$ in $\Rep_{d\alpha}$ is so that $Z(x)\cong \F^{\alpha(x)}\otimes U(x)$ for $U(x)\cong\F^d$. At the arrows, we have $Z(a)\cong\sum V_i(a)\otimes U_i(a)$, a finite sum where each $U_i(a)$ is a $d\times d$ matrix over $\F$, and each $V_i$ is from $\Rep_\alpha$. Now, given a $\overline{V}$ in $\Rep_{d\alpha}$, we get a map:
$$\Hom\Big(\bigoplus_{x\in Q_0}\Hom\big(\F^{d\alpha(x)},W(x)\big)\xrightarrow{\fmap{W}{d\alpha}(\overline{V})}\bigoplus_{a\in Q_1}\Hom\big(\F^{d\alpha(ta)},W(ha)\big) \Big).$$
Notice that we can find $\ncrk{(\fmap{W}{\alpha})}$ using $\ncrk{(\fmap{W}{d\alpha})}$ and dividing by $d$ since $\fmap{W}{d\alpha}$ is the $d$th blow-up of $\fmap{W}{\alpha}$. 

\begin{proposition}\label{samed}
The rank and non-commutative rank of $\fmap{W}{d\alpha}$ are equal if and only if $\ds\nchom(\alpha,W)=\frac{\hom(d\alpha,W)}{d}.$
\end{proposition}

For a $(d\alpha,W)$-general $\overline{V}$ in $\Rep_{d\alpha}$, the kernel of $\fmap{W}{d\alpha}(\overline{V})$ is of minimal dimension. So, $\rk(\fmap{W}{d\alpha})=\sum d\alpha(x)\beta(x)-\hom(d\alpha,W)$. We get $$\frac{\rk \fmap{W}{d\alpha}}{d}=\sum\alpha(x)\beta(x)-\frac{\hom(d\alpha,W)}{d},$$
showing that $d$ which maximizes the left-side (giving us the non-commutative rank), maximizes the right side (minimizing $\frac{\hom(d\alpha,W)}{d}$, giving us the non-commutative hom).

\begin{corr}
Given dimension vector $\alpha$, and a representation $W$ of dimension $\beta$, the $d$ in the limit of definition \ref{nchomdef} can be chosen to be $$\min\Big\{\sum_{x\in Q_0}\alpha(x)\beta(x)-1,\sum_{a\in Q_1}\alpha(ta)\beta(ha)-1\Big\}$$.
\end{corr}
Recall the bound for non-commutative rank blow-ups from \cite{DM18} is $n-1$, where $n$ is the dimension of both the domain and co-domain. We may not have a space of square matrices, so a large enough $d$ will be found when we first reach either $\sum_{x\in Q_0}\alpha(x)\beta(x)-1$ or $\sum_{a\in Q_1}\alpha(ta)\beta(ha)-1$. 

\begin{theorem}\label{ncext}
We have
$$
\ncext(\alpha,W)=\max\{-\langle\alpha,\dimvec W'' \rangle \mid \mbox{$W''$ factor representation of $W$}\}.
$$
\end{theorem}
\begin{proof}

Choose $d$ so that $\ncext(d\alpha,W)$ equals $\frac{\ext(d\alpha,W)}{d}$. Look at the set of maps:
$$\Hom\Big(\bigoplus_{x\in Q_0}\Hom\big(\F^{\alpha(x)},W(x)\big)\xrightarrow{\fmap{W}{d\alpha}(\overline{V})}\bigoplus_{a\in Q_1}\Hom\big(\F^{\alpha(ta)},W(ha)\big) \Big),$$
for all representations $\overline{V}$ in $\Rep_{d\alpha}$. By Proposition \ref{samed}, this set of maps has non-commutative rank equal to its rank. So we can find the minimal $c$-shrunk subspace, which:
\begin{enumerate}
    \item has the form $\bigoplus_{x\in Q_0}\Hom\big(\F^{d\alpha(x)},W'(x)\big)$, for some subrepresentation $W'$ of $W$ (from discussion in section \ref{reduction}), and
    \item has image of the form $\bigoplus_{a \in Q_1}\Hom\big(\F^{d\alpha(ta)},W'(ha)\big)$.
\end{enumerate}
So we get $c=d\sum\alpha(x)\dim(W'(x))-d\sum\alpha(ta)\dim(W'(ha))=\langle d\alpha,\dimvec(W')\rangle$, but $c$ is the non-commutative rank, so also can be found by $\sum d\alpha(x)\beta(x)-\rk(\fmap{W}{d\alpha})=\hom(d\alpha,W).$ This leaves us with $\frac{\hom(d\alpha,W)}{d}=\langle\alpha,\dimvec(W')\rangle$ after dividing by $d$.
As for non-commutative ext, we then get $\ncext(\alpha,W)=\nchom(\alpha,W)-\langle\alpha,\beta\rangle$, finally leaving us with $\ncext(\alpha,W)=-\langle\alpha,\dimvec W'' \rangle$, for $W''=W/W'.$

\end{proof}

We note that we can dually fix a representation $V$, and look at $\hom(V,\beta)$ and $\ext(V,\beta)$ to define $\nchom(V,\beta)$ and $\ncext(V,\beta)$. 
\begin{definition} The non-commutative ext and hom are defined by the following limits of ext and hom:
\begin{eqnarray*}
\ncext(V,\beta)&=&\lim_{d\to\infty}\frac{\ext(V,d\beta)}{d}\\
\nchom(V,\beta)&=&\lim_{d\to\infty}\frac{\hom(V,d\beta)}{d}
\end{eqnarray*}
\end{definition}
\begin{theorem}
We have
$$
\ncext(V,\beta)=\max\{-\langle \dimvec V', \beta \rangle \mid \mbox{$V'$ subrepresentation of $V$}\}.
$$
\begin{proof}
The proof follows from duality of theorem \ref{ncext}. We note that this can also be seen by using Corollary 1 from \cite{Craw96}, by subtracting $\langle \dimvec V, \beta \rangle$.
\end{proof}
\end{theorem}

\begin{corr}
For large enough $|\F|$, there are both deterministic and randomized algorithms for calculating $\ncext(\alpha,W),\nchom(\alpha,W),\ncext(V,\beta),$ and $\nchom(V,\beta)$.
\begin{proof}
We can apply any of the algorithms used to find $c$-shrunk subspaces to the set of maps $\fmap{W}{\alpha}({V})$ or $\fmap{\beta}{V}(W)$ respectively, and use the dimension of $c$ to calculate the non-commutative ext and hom.
\end{proof}
\end{corr}


%\section{Discussion on Approximation \textit{vs} Stability and Recovery}\label{sec:approx-stability}


In the world of approximation algorithms, for a maximization problem for which an algorithm outputs $S$ and the optimum is $S^*$, what we typically try to prove is that
$w(S)\ge w(S^*)/\alpha$, even in the worst case; this \textit{approximation inequality} means that the algorithm at hand is an $\alpha$-approximation, so it is a \textit{good} algorithm. Though one might be quick to say that recovery of $\alpha$-stable instances immediately follows from the approximation inequality, this is not true because of the intersection $S\cap S^*$; if we have no intersection, then recovery indeed follows. 

What the research on stability and exact recovery suggests, is that we should try to understand if some of our already known approximation algorithms have the stronger property $w(S\setminus S^*)\ge w(S^*\setminus S)/\alpha$ or at least if they have it on stable instances. We refer to the latter as the \textit{recovery inequality}. This would directly imply an exact recovery result for $\alpha$-stable instances because we could $\alpha$-perturb only the $S\setminus S^*$ part of the input and get: 
\[
\noindent w(S\setminus S^*)\ge w(S^*\setminus S)/\alpha \implies \alpha\cdot w(S\setminus S^*) +w(S\cap S^*) \ge w(S^*\setminus S) +w(S\cap S^*) = w(S^*)
\] thus violating the fact we were given an $\alpha$-stable instance, unless $S\setminus S^* = \emptyset$.

This would mean that the algorithm successfully retrieved $S^*$ and could potentially explain why many approximation algorithms behave far better in practice than in theory. Furthermore, from a theory perspective, it would mean that many results from the well-studied area of approximation algorithms could be translated in terms of stability and recovery.

As a concluding remark, we want to point out that even though one might think that an $\alpha$-approximation algorithm needs at least $\alpha$-stability for recovery, this is not true as the somewhat counterintuitive result from \cite{balcan2015k} tells us: asymmetric $k$-center cannot be approximated to any constant factor, but can be solved optimally on 2-stable instances. This was the
first problem that is hard to approximate to any constant factor in the worst case, yet can be optimally
solved in polynomial time for 2-stable instances. The other direction (having an $\alpha$-approximation algorithm that cannot recover arbitrarily stable instances) is also true. These findings suggest that there are interesting connections between stability, exact recovery and approximation.


\section{Acknowledgements}
The author was supported by NSF grant DGE 1256260, and would additionally like to thank Harm Derksen for direction and conversation, and Calin Chindris and Daniel Kline for comments on an earlier draft of this paper.

\bibliography{bb}{}
\bibliographystyle{alpha}
\end{document}