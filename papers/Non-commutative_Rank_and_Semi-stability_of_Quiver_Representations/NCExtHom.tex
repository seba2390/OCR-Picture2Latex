\section{Non-commutative General Ext and Hom}

In \cite{Scho92}, Schofield studied the minimal dimension of $\Ext(V,W)$ for representations $V$ and $W$ of dimension vectors $\alpha$ and $\beta$. Define
$$Z^t(\alpha,\beta):=\{(V,W)\in \Rep_\alpha(Q)\times\Rep_\beta(Q)|\dim\Hom_Q(V,W)\geq t\}.$$
Each of these subsets of $\Rep_\alpha(Q)\times\Rep_\beta(Q)$ are closed. Take $t$ the minimal positive value of $\dim\Hom_Q(V,W)$. Then, $Z^{t+1}(\alpha,\beta)$ is a proper closed subset. We call the pair $(V,W)$ $(\alpha,\beta)$-general if they are in the (open and dense) complement of $Z^{t+1}(\alpha,\beta)$. On this complement, $\dim\Hom_Q(V,W)$ is constant, as is $\dim\Ext_Q(V,W)$. Schofield calls these generic hom and ext respectively:
\begin{eqnarray*}
\hom(\alpha,\beta)&=&\dim(\Hom_Q(V,W)), \text{ and}\\
\ext(\alpha,\beta)&=&\dim(\Ext_Q(V,W)).
\end{eqnarray*}

Crawley-Boevey generalized the generic hom in \cite{Craw96}, which we show along with the generalization of generic ext. To do this, fix a representation $W$ of $Q$, with dimension vector $\beta$. Define now 
$$Z^t(\alpha,W):=\{V\in \Rep_\alpha(Q)|\dim\Hom_Q(V,W)\geq t\}.$$
Again, each of these subsets are closed, and we take $t$ minimal so that $\dim\Hom_Q(V,W)$ is positive. We call $V$ $(\alpha,W)$-general if it is in the complement of $Z^{t+1}(\alpha,W)$.  

\begin{definition}
Let $V$ be an $(\alpha,W)$-general representation. We define the $(\alpha,W)$-general hom and ext as
\begin{eqnarray*}
\hom(\alpha,W)&=&\dim(\Hom_Q(V,W)), \text{ and}\\
\ext(\alpha,W)&=&\dim(\Ext_Q(V,W)).
\end{eqnarray*}
\end{definition}


\begin{lemma} We have 
$$
\ext(\alpha,W)\geq \max\{-\langle \alpha, \dimvec W'\rangle \mid \mbox{$W'$ factor representation of $W$}\}.
$$
\end{lemma}
\begin{proof}
If $V$ is a general representation of dimension $\alpha$, then applying $\Hom_Q(V,\cdot)$ to 
$$0\to W''\to W\to W'\to 0$$
gives
an exact sequence
$$
\cdots\to \Ext_Q(V,W)\to \Ext_Q(V,W')\to 0,
$$
so $\dim \Ext_Q(V,W')\leq \dim \Ext_Q(V,W)$ and $\ext(\alpha,W')\leq \ext(\alpha,W)$.
We get
$$
-\langle \alpha,\dimvec W'\rangle=\ext(\alpha,W')-\hom(\alpha,W')\leq \ext(\alpha,W')\leq \ext(\alpha,W).
$$
\end{proof}

\begin{definition}\label{nchomdef} The non-commutative ext and hom are defined by the following limits of ext and hom:
\begin{eqnarray*}
\ncext(\alpha,W)&=&\lim_{d\to\infty}\frac{\ext(d\alpha,W)}{d}\\
\nchom(\alpha,W)&=&\lim_{d\to\infty}\frac{\hom(d\alpha,W)}{d}.
%\ncext(V,\beta)&=&\lim_{d\to\infty}\frac{\ext(V,d\beta)}{d}\\
%\nchom(V,\beta)&=&\lim_{d\to\infty}\frac{\hom(V,d\beta)}{d}
\end{eqnarray*}
\end{definition}
Note that for every representation $W$ of dimension $\beta$, we have $\nchom(\alpha,W)-\ncext(\alpha,W)$ equal to $\langle\alpha,\beta\rangle$. These limits were originally studied in \cite{Craw96}, though we give them a name to highlight their connection to non-commutative rank, as seen in the next discussion and proposition.

We have a map $$\fmap{W}{\alpha}:\Rep_\alpha \longrightarrow\Hom\Big(\bigoplus_{x\in Q_0}\Hom\big(\F^{\alpha(x)},W(x)\big)\rightarrow\bigoplus_{a\in Q_1}\Hom\big(\F^{\alpha(ta)},W(ha)\big) \Big)$$
given by sending a representation $V$ to the map $\fmap{W}{\alpha}(V)$, which takes the set of $\varphi(x)$ from $\Hom(\F^{\alpha(x)},W(x))$ over all vertices $x$ to the set of maps $\varphi(ha)V(a)-W(a)\varphi(ta)$ over all arrows $a$. Note that the kernel of each $\fmap{W}{\alpha}(V)$ is $\Hom_Q(V,W)$, and and the cokernel is $\Ext_Q(V,W)$. From this point forward, we will refer to the image of $\fmap{W}{\alpha}$ (the set of $\fmap{W}{\alpha}(V)$ over all $V$), as simply $\fmap{W}{\alpha}$ itself. 

Next, we note that we can consider $\Rep_{d\alpha}$ as the blow-up of $\Rep_\alpha$ as follows. Each $Z$ in $\Rep_{d\alpha}$ is so that $Z(x)\cong \F^{\alpha(x)}\otimes U(x)$ for $U(x)\cong\F^d$. At the arrows, we have $Z(a)\cong\sum V_i(a)\otimes U_i(a)$, a finite sum where each $U_i(a)$ is a $d\times d$ matrix over $\F$, and each $V_i$ is from $\Rep_\alpha$. Now, given a $\overline{V}$ in $\Rep_{d\alpha}$, we get a map:
$$\Hom\Big(\bigoplus_{x\in Q_0}\Hom\big(\F^{d\alpha(x)},W(x)\big)\xrightarrow{\fmap{W}{d\alpha}(\overline{V})}\bigoplus_{a\in Q_1}\Hom\big(\F^{d\alpha(ta)},W(ha)\big) \Big).$$
Notice that we can find $\ncrk{(\fmap{W}{\alpha})}$ using $\ncrk{(\fmap{W}{d\alpha})}$ and dividing by $d$ since $\fmap{W}{d\alpha}$ is the $d$th blow-up of $\fmap{W}{\alpha}$. 

\begin{proposition}\label{samed}
The rank and non-commutative rank of $\fmap{W}{d\alpha}$ are equal if and only if $\ds\nchom(\alpha,W)=\frac{\hom(d\alpha,W)}{d}.$
\end{proposition}

For a $(d\alpha,W)$-general $\overline{V}$ in $\Rep_{d\alpha}$, the kernel of $\fmap{W}{d\alpha}(\overline{V})$ is of minimal dimension. So, $\rk(\fmap{W}{d\alpha})=\sum d\alpha(x)\beta(x)-\hom(d\alpha,W)$. We get $$\frac{\rk \fmap{W}{d\alpha}}{d}=\sum\alpha(x)\beta(x)-\frac{\hom(d\alpha,W)}{d},$$
showing that $d$ which maximizes the left-side (giving us the non-commutative rank), maximizes the right side (minimizing $\frac{\hom(d\alpha,W)}{d}$, giving us the non-commutative hom).

\begin{corr}
Given dimension vector $\alpha$, and a representation $W$ of dimension $\beta$, the $d$ in the limit of definition \ref{nchomdef} can be chosen to be $$\min\Big\{\sum_{x\in Q_0}\alpha(x)\beta(x)-1,\sum_{a\in Q_1}\alpha(ta)\beta(ha)-1\Big\}$$.
\end{corr}
Recall the bound for non-commutative rank blow-ups from \cite{DM18} is $n-1$, where $n$ is the dimension of both the domain and co-domain. We may not have a space of square matrices, so a large enough $d$ will be found when we first reach either $\sum_{x\in Q_0}\alpha(x)\beta(x)-1$ or $\sum_{a\in Q_1}\alpha(ta)\beta(ha)-1$. 

\begin{theorem}\label{ncext}
We have
$$
\ncext(\alpha,W)=\max\{-\langle\alpha,\dimvec W'' \rangle \mid \mbox{$W''$ factor representation of $W$}\}.
$$
\end{theorem}
\begin{proof}

Choose $d$ so that $\ncext(d\alpha,W)$ equals $\frac{\ext(d\alpha,W)}{d}$. Look at the set of maps:
$$\Hom\Big(\bigoplus_{x\in Q_0}\Hom\big(\F^{\alpha(x)},W(x)\big)\xrightarrow{\fmap{W}{d\alpha}(\overline{V})}\bigoplus_{a\in Q_1}\Hom\big(\F^{\alpha(ta)},W(ha)\big) \Big),$$
for all representations $\overline{V}$ in $\Rep_{d\alpha}$. By Proposition \ref{samed}, this set of maps has non-commutative rank equal to its rank. So we can find the minimal $c$-shrunk subspace, which:
\begin{enumerate}
    \item has the form $\bigoplus_{x\in Q_0}\Hom\big(\F^{d\alpha(x)},W'(x)\big)$, for some subrepresentation $W'$ of $W$ (from discussion in section \ref{reduction}), and
    \item has image of the form $\bigoplus_{a \in Q_1}\Hom\big(\F^{d\alpha(ta)},W'(ha)\big)$.
\end{enumerate}
So we get $c=d\sum\alpha(x)\dim(W'(x))-d\sum\alpha(ta)\dim(W'(ha))=\langle d\alpha,\dimvec(W')\rangle$, but $c$ is the non-commutative rank, so also can be found by $\sum d\alpha(x)\beta(x)-\rk(\fmap{W}{d\alpha})=\hom(d\alpha,W).$ This leaves us with $\frac{\hom(d\alpha,W)}{d}=\langle\alpha,\dimvec(W')\rangle$ after dividing by $d$.
As for non-commutative ext, we then get $\ncext(\alpha,W)=\nchom(\alpha,W)-\langle\alpha,\beta\rangle$, finally leaving us with $\ncext(\alpha,W)=-\langle\alpha,\dimvec W'' \rangle$, for $W''=W/W'.$

\end{proof}

We note that we can dually fix a representation $V$, and look at $\hom(V,\beta)$ and $\ext(V,\beta)$ to define $\nchom(V,\beta)$ and $\ncext(V,\beta)$. 
\begin{definition} The non-commutative ext and hom are defined by the following limits of ext and hom:
\begin{eqnarray*}
\ncext(V,\beta)&=&\lim_{d\to\infty}\frac{\ext(V,d\beta)}{d}\\
\nchom(V,\beta)&=&\lim_{d\to\infty}\frac{\hom(V,d\beta)}{d}
\end{eqnarray*}
\end{definition}
\begin{theorem}
We have
$$
\ncext(V,\beta)=\max\{-\langle \dimvec V', \beta \rangle \mid \mbox{$V'$ subrepresentation of $V$}\}.
$$
\begin{proof}
The proof follows from duality of theorem \ref{ncext}. We note that this can also be seen by using Corollary 1 from \cite{Craw96}, by subtracting $\langle \dimvec V, \beta \rangle$.
\end{proof}
\end{theorem}

\begin{corr}
For large enough $|\F|$, there are both deterministic and randomized algorithms for calculating $\ncext(\alpha,W),\nchom(\alpha,W),\ncext(V,\beta),$ and $\nchom(V,\beta)$.
\begin{proof}
We can apply any of the algorithms used to find $c$-shrunk subspaces to the set of maps $\fmap{W}{\alpha}({V})$ or $\fmap{\beta}{V}(W)$ respectively, and use the dimension of $c$ to calculate the non-commutative ext and hom.
\end{proof}
\end{corr}
