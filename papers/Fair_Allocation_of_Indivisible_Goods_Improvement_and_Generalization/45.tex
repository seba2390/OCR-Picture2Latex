\section{A $4/5$-$\MMS$ Allocation for Four Agents}\label{45}
%\color{magenta}
In this section we propose an algorithm to find a $4/5$-$\MMS$ for $n=4$ in the additive setting. Since the number of agents is exactly 4, we assume $\agents = \{\agent_1, \agent_2, \agent_3, \agent_4\}$. Again, for simplicity we assume $\MMS_i = 1$ for every $\agent_i \in \agents$. In general, our algorithm is consisted of  three main steps: first, $\agent_1$ optimally partitions the items into $4$ bundles with values at least 1 to him. Next, $\agent_2$ selects three of the bundles and repartitions them. Finally, we satisfy one of $\agent_1$ or $\agent_2$ with a bundle and solve the problem for remaining agents and items via Lemma ~\ref{recurse}. Without loss of generality, we assume the valuation of every agent $\agent_i$ for every bundle in his optimal $n$-partitioning is exactly equal to 1. Therefore, from here on, we assume that the summation of the values of the items within the same bundle for every agent is at most $1$. In addition, we suppose that the problem is $4/5$-irreducible, since Lemma \ref{reducibility} narrows down the problem into such instances. Thus, by Lemma \ref{remove1}, the value of every item is less than $4/5$ to any agent. 
We begin this section by stating a number of definitions and observations. In this section, we use the term \emph{bundle} to refer to a set of items.
\begin{definition}
\label{perfect}
A set $S$ of bundles is perfect for a set $T$ of agents, if $(i)$ $|S| = |T|$ and $(ii)$ there exists an allocation of the bundles in $S$ to the agents of $T$ such that all the agents in $T$ are satisfied by their allocated bundle.
\end{definition}



\begin{observation}\label{firstobs}
\label{sum}
Let $\agent_i$ be an agent and $S$ be a set of items where $\valu_i(\{\ite_j\}) \leq v$ for every item $\ite_j \in S$. If $V_i(S) > v$, then there exists a subset $S' \subseteq S$ of items such that $ v \leq \valu_i(S') < 2v$.
\end{observation}
\begin{proof}
We begin with an empty set $S'$ and add the items of $S$ to $S'$ one by one, until $\valu_i(S')$ exceeds $v$. Before adding the last element to $S'$, the valuation of $\agent_i$ for $S'$ was no more than $v$ and every item alone is of value less than $v$ to $\agent_i$. Therefore, after adding the last item to $S’$, its value is less than $2v$ to $\agent_i$.
\end{proof}

\begin{definition}
\label{core}
For a bundle $B$ of items that satisfies $\agent_i$, the core of $B$ with respect to agent $\agent_i$, denoted by $C_i(B)$, is defined as follows: let $m_1,m_2,..,m_k$ be the items of $B$ in the increasing order of their values to $\agent_i$. Then $C_i(B) = \{m_j,m_{j+1},...,m_{k}\}$ , where $j$ is the highest index, such that set of items $\{m_j,m_{j+1},...,m_k\}$ satisfies $\agent_i$.
\end{definition}

Note that for every subset $B$ with $\valu_i(B) \geq 4/5$, $C_i(B)$ is a subset of $B$ with the minimum size that satisfies $\agent_i$. Since the items in $C_i(B)$ satisfy $\agent_i$, we have $\valu_i(C_i(B))\geq 4/5$. On the other hand, by the fact that $|C_i(B)|$ is minimal, removing any item from $C_i(B)$ results in a subset that no longer satisfies $\agent_i$. Thus, Observation \ref{eps2} holds.

\begin{observation}\label{eps2} 
If $\valu_i(C_i(B))  = 4/5 + \beta$, then the value of every item in $C_i(B)$ is more than $\beta$ for $\agent_i$.
\end{observation}

By the fact the value of every item in $B$ is less than $4/5$ we have Observation \ref{cre}.

\begin{observation}\label{cre}
For every agent $\agent_i$ and any subset $B$ of items, $\valu_i(C_i(B))<8/5$.
\end{observation}


\begin{lemma}
\label{3p}
Suppose that $S = \{X,Y,Z\}$ is a 3-partitioning of a set of items with the following properties for an agent $\agent_i$: 

\begin{minipage}[t]{\linegoal}
\begin{enumerate}[leftmargin=*]
\item $\valu_i(X) < 4/5$ and $\valu_i(Y)<4/5$. 

\item $V_i(X\cup Y \cup Z) >16/5$. 
\end{enumerate}
\end{minipage}
\\

Then we can move some items from $Z$ to an arbitrary bundle of $\{X, Y\}$, such that, both $Z$ and the corresponding bundle will be worth at least $4/5$ to $\agent_i$. 

\end{lemma}

\begin{proof}
Since $V_i(X \cup Y \cup Z ) > 16/5$, $V_i(Z) > 8/5$ holds. Moreover, by Observation \ref{cre}, we have $\valu_i(C_i(Z)) \leq 8/5$. Considering $Z' = Z \setminus C_i(Z)$, we have   
$$\valu_i(X \cup Y \cup Z' ) \geq 8/5.$$
According to the fact that $\valu_i(X)<4/8$ and $\valu_i(Y)<4/5$, we have 
$$\valu_i(X \cup Z' ) \geq 4/5$$
and 
$$\valu_i(Y \cup Z' ) \geq 4/5.$$


\end{proof}

\begin{lemma}
\label{recurse}
Let $S= \{X,Y,Z\}$ be a set of three bundles of items, such that 

\begin{minipage}[t]{\linegoal}
\begin{enumerate}[leftmargin=*]
 \item $V_1(X) \ge 4/5 , V_1(Y) \ge 4/5, V_1(Z) \ge 4/5 $. 
\item $V_2(X \cup Y \cup Z) > 16/5$.
\item $V_3(X \cup Y \cup Z) \ge 3 $.
 
\end{enumerate}
\end{minipage}
\\

Then a $4/5$-$\MMS$ allocation of $X \cup Y \cup Z$ to the agents $\agent_1,\agent_2,\agent_3$ is possible.
\end{lemma}
\begin{proof}
If $\agent_2$ can be satisfied with two different bundles, then trivially $S$ is perfect. Otherwise, $\agent_2$ is satisfied with only one bundle, say $Z$. By Lemma ~\ref{3p}, $\agent_2$ can transfer some items from $Z$ to $Y$, such that both bundles satisfy him.  After moving the items, both $Y$ and $Z$ satisfy $\agent_2$, and bundle $X$ and $Y$, satisfy $\agent_1$. One the other hand,   since $V_3(X \cup Y \cup Z) \ge 3 $, $\agent_3$ is satisfied with at least one bundle. Its easy to observe that for any valuation of bundles $X,Y,Z$ for $\agent_3$, the set of bundles is perfect.
\end{proof}


\begin{lemma}
\label{core}
Let $S=\{X,Y,Z,T\}$ be a 4-partitioning of $\cal M$ and $\agent_i$ be an arbitrary agent. Then $\agent_i$ can select $3$ bundles and re-partition them into three new bundles in such a way that each bundle will be worth at least $4/5$ to $\agent_i$.
\end{lemma}
\begin{proof}
Consider bundles $X,Y,Z,T$. If more than two of them satisfy $\agent_i$, then the selection is trivial. Furthermore, if only one bundle satisfies $\agent_i$, then by Lemma ~\ref{3p}, we can move some items from the satisfying bundle to another bundle, such that both bundles satisfy $\agent_i$. Thus, without loss of generality, we assume that bundles $Z$ and $T$ satisfy $\agent_i$. 


Let $Z' = Z \setminus C_i(Z)$ and $T'= T \setminus C_i(T)$. Without loss of generality, we assume $V_i(X) \geq V_i(Y)$ and let $X' = X \cup Z' \cup T'$. If $V_i(X') \ge 4/5$, then the proof is trivial. Thus, suppose that $V_i(X') < 4/5$.

Consider the value of bundles as $$V_i(X’)=4/5 - \epsilon_1,$$ $$V_i(Y)=4/5 - \epsilon_2,$$ $$V_i(C_i(Z))=6/5 + \epsilon_3,$$ and $$V_i(C_i(T))=6/5 + \epsilon_4$$ where $\epsilon_1 \leq \epsilon_2$ and $\epsilon_3 \leq \epsilon 4$. Note that $\epsilon_3$ can be negative. By the fact that total value of the items equals $4$ for all of the agents, we assume that $$\epsilon_1 + \epsilon_2 = \epsilon_3 + \epsilon_4$$ and hence $$\epsilon_1 \leq \epsilon_4.$$  


Now, we explore the properties of the items in $C_i(T)$. Regarding Observation \ref{eps2}, every item in $C_i(T)$ is worth more than $2/5+\epsilon_4$ to $\agent_i$. Hence, $C_i(T)$ cannot contain more than 2 items, since value of every pair of items in $C_i(T)$ is more than $4/5$. Moreover, $C_i(T)$ cannot contain one item and hence, $C_i(T)$ contains exactly two items. Let $\ite_1$ and $\ite_2$ be these two items. Since $\valu_i(T) = 6/5 + \epsilon_4$, at least one of these two items, say $\ite_2$ is worth at least $3/5+\epsilon_4/2$ to $\agent_i$. Thus, in summary, $C_i(T)$ contains two items $\ite_1$ and $\ite_2$ with

$$\valu_i(\{\ite_1\}) >2/5 + \epsilon_4,$$
$$\valu_i(\{\ite_2\}) \geq 3/5+\epsilon_4/2.$$

Next, we characterise the items in $X'$. For Bundle $X'$, let $B$ be the set of items with a value less than $1/5 - \epsilon_4/2$. If $V_i(B)\geq 1/5 - \epsilon_4/2$, then Observation ~\ref{sum} states that there exists a subset $B'$ of $B$, such that:
$$1/5 - \epsilon_4/2 \leq V_i(B') < 2/5 - \epsilon_4.$$

Therefore, Bundles $B' \cup \{b_2\}$ and $(X' \setminus B') \cup \{b_1\}$ satisfy $\agent_i$. These two bundles together with  $C_i(Z)$ result in three bundles that satisfy $\agent_i$. Thus, $V_i(B) < 1/5 - \epsilon_4/2$. 

Finally, regarding the fact that $V_i(B) < 1/5 - \epsilon_4/2$, we have 
$$\valu_i(X' \setminus B)> 3/5  -\epsilon_1 + \epsilon_4/2 $$

For this case, we show that $X'\setminus B$ contains exactly one item. Otherwise, at least one of these items, say $\ite_3$, is worth less than $3/10 - \epsilon_1/2 + \epsilon_4/4$ and therefore, for the bundles $\{\ite_3\} \cup \{\ite_2\}$ and $(X' \setminus \{\ite_3\}) \cup \{\ite_1\}$  we have:
$$\valu_i(\{\ite_3\} \cup \{\ite_2\}) \geq 1/5 - \epsilon_4/2 + 3/5+\epsilon_4/2 \geq 4/5$$
and 
$$\valu_i((X' \setminus \{\ite_3\}) \cup \{\ite_1\}) \geq 4/5 - \epsilon_1 - 3/10 + \epsilon_1/2- \epsilon_4/4+ 2/5 + \epsilon_4 \geq 4/5$$
respectively. These two bundles along with $C_i(Z)$ form $3$ bundles that satisfy $\agent_i$. Therefore, we conclude that $X' \setminus B$ contains an item $\ite_3$ with a value more than $3/5-\epsilon_1  + \epsilon_4/2$ to $\agent_i$. The rest of the items in $X'$ belong to $B$ that are in total worth less than $1/5-\epsilon_4/2$.

Note that $X \subseteq X'$. Therefore, consider the $4-$partitioning of $\agent_i$ and remove the bundle containing $\ite_3$. Also, remove the items with value less than $1/5 - \epsilon_4/2$ to $\agent_i$ in $X$, from their corresponding bundles. Three bundles with value of each to $\agent_i$ more than $$1 - 1/5 + \epsilon_4/2 \geq 4/5,$$ with all of their items from $Y,Z$ and $T$ remain. Thus, $\agent_i$ can make three satisfying bundles with items in $Y,Z,T$. 
\end{proof}
Based on what we showed so far, we prove Theorem \ref{45main}.
\begin{theorem}
\label{45main}
A $4/5$-$\MMS$ allocation for $n=4$ is possible in the additive setting.
\end{theorem}
\begin{proof}
Consider the optimal $4$-partitioning of $\cal M$ with respect to $\agent_1$. Now,  ask $\agent_2$ to select $3$ bundles and re-partition them, such that he can be satisfied with all the three bundles. Based on Lemma \ref{core}, such a repartitioning is always possible. Due to the Pigeonhole principle, at least one of these three bundles still satisfies $\agent_1$. Let $S = \{X,Y,Z,T\}$ be the resulting bundles and without loss of generality, suppose that bundles $X,Y$ satisfy $\agent_1$ and bundles $Y,Z,T$ satisfy $\agent_2$.

Now, consider agents $\agent_3$ and $\agent_4$ and let $\phi$ be the set of bundles that satisfy $\agent_3$ or $\agent_4$. There are only two cases, in which $S$ is not perfect 
(recall definition \ref{perfect}):

\begin{enumerate}
\item $\phi \subseteq \{X,Y\}$ : $\agent_1$, selects three bundles $X,Y$ and one of $Z$ or $T$, say $Z$ and re-partitions them to three satisfying bundles. Now, give bundle $T$ to $\agent_2$. According to Lemma  ~\ref{recurse}, items of $X,Y,Z$ can satisfy the remaining three agents.

\item $|\phi|=1, \phi \notin \{X,Y\}$ : give $X$ to $\agent_1$ and allocate the items of $Y \cup Z \cup T$ to $\agent_2,\agent_3,\agent_4$, using Lemma ~\ref{recurse}.
\end{enumerate}
\end{proof}
\color{black}