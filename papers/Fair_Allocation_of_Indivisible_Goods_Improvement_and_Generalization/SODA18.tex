\documentclass[letterpaper,11pt]{article}
%\usepackage{amsmath, amsthm, amssymb}
\usepackage{amsmath, amssymb}
\usepackage{color}
\usepackage{fullpage}
\usepackage[numbers]{natbib}
\usepackage[noend]{algorithmic}
\usepackage{tikz}
\usepackage{enumerate}
\usepackage{hyperref}
\usepackage{graphicx}
\usepackage{caption}
\usepackage{hyperref}
\usepackage{subcaption}
\usepackage{zerosum,csquotes}
\usepackage{enumitem,linegoal}
\usepackage[linesnumbered,ruled,vlined]{algorithm2e}
\usepackage{comment}	
\usepackage{ctable}					
\newtheorem{claim}{Claim}[section]
\newtheorem{fact}{Fact}[section]
\newtheorem{observation}{Observation}[section]
\usepackage{mathtools}
\usepackage[flushleft]{threeparttable}
\usepackage[normalem]{ulem}


\usepackage{footnote}
\makesavenoteenv{tabular}
\makesavenoteenv{table}

%\usepackage{cite}

\usepackage[margin=1in]{geometry}

\newtheorem{example}{Example}

% NON-SODA:
 \newtheorem{theorem}{Theorem}[section]
 \newtheorem{lemma}[theorem]{Lemma}
 \newtheorem{proposition}[theorem]{Proposition}
 \newtheorem{corollary}[theorem]{Corollary}
 \newtheorem{conj}[theorem]{Conjecture}
 \newtheorem{definition}[theorem]{Definition}
 
%\theoremstyle{definition}
 \newtheorem{remark}[theorem]{Remark}
\makeatletter
\def\GrabProofArgument[#1]{ #1: \egroup\ignorespaces}
\def\proof{\noindent\textbf\bgroup Proof%
	\@ifnextchar[{\GrabProofArgument}{. \egroup\ignorespaces}}
\def\endproof{\hspace*{\fill}$\Box$\medskip}
\makeatother

% SODA:
%\newtheorem{claim}{Claim}[section] 
%\newtheorem{conj}{Conjecture}[section] 




\newcommand{\figref}[1]{Fig.~\ref{#1}}
\newcommand{\tblref}[1]{Table~\ref{#1}}
\newcommand{\secref}[1]{Section~\ref{#1}}
\renewcommand{\eqref}[1]{Equation~(\ref{#1})}

\def\availableat{\url{url-published-on-acceptance}}

\newcommand{\todo}[1]{{\color{red} TODO: {#1}}}
\newcommand{\newstuff}[1]{{\color{red} CHECK: {#1}}}
%\newcommand{\todo}[1]{{}}

\newcommand{\ckp}[2]{$CK_{#1}P_{#2}$}
\newcommand{\cext}{\ckp{8}{16}$ext$}
\newcommand{\cfin}{$F_{CK_{X}P_{Y}}$}
\newcommand{\cray}{\ckp{8}{8}$ray$}
\newcommand{\csin}{$SK_{8}P_{8}$}
\newcommand{\casin}{$SK_{combined}$}
%\renewcommand{\cext}{$SK_{8}K_{8}P_{8}$}
\newcommand{\ckpnl}[2]{$CK_{#1}P_{#2}nl$}


\makeatletter
\newcommand{\Spvek}[2][r]{%
	\gdef\@VORNE{1}
	\left(\hskip-\arraycolsep%
	\begin{array}{#1}\vekSp@lten{#2}\end{array}%
	\hskip-\arraycolsep\right)}

\def\vekSp@lten#1{\xvekSp@lten#1;vekL@stLine;}
\def\vekL@stLine{vekL@stLine}
\def\xvekSp@lten#1;{\def\temp{#1}%
	\ifx\temp\vekL@stLine
	\else
	\ifnum\@VORNE=1\gdef\@VORNE{0}
	\else\@arraycr\fi%
	#1%
	\expandafter\xvekSp@lten
	\fi}
\makeatother
\newcommand*\samethanks[1][\value{footnote}]{\footnotemark[#1]}
\DeclareMathOperator{\CC}{CC}

\newcounter{proccnt}
\newenvironment{process}[1][]{\refstepcounter{proccnt}\par\medskip\noindent%
\textbf{%
\vskip\medskipamount % or other desired dimension
\leaders\vrule width \textwidth\vskip0.4pt % or other desired thickness
\vskip\medskipamount % ditto
\nointerlineskip
\noindent Subroutine~\theproccnt. [#1]
\vskip\medskipamount % or other desired dimension
\leaders\vrule width \textwidth\vskip0.4pt % or other desired thickness
\vskip\medskipamount % ditto
\nointerlineskip
} \rmfamily}{%\medskip
\vskip\medskipamount % or other desired dimension
\leaders\vrule width \textwidth\vskip0.4pt % or other desired thickness
\vskip\medskipamount % ditto 	
\nointerlineskip
}

\newcommand{\QEDA}{\hfill\ensuremath{\blacksquare}}

\newcommand{\konote}[1]{}

\title{Fair Allocation of Indivisible Goods: Improvement and Generalization
%\footnote{In our submission to STOC, despite receiving strong support from the reviewers regarding the importance of the problem, novelty, technical depth, and presentation of the paper, apparently one of the reviewers had convinced the PC to reject the paper based on the following comment: ``The main negative is that the main contribution is technical". Since we do not fully understand what exactly this critique means, we sincerely ask the reviewers to be more precise about their objections. To make the paper more understandable, we clearly outline the $3/4$-$\MMS$ algorithm in Section \ref{overview}. We show how it has been derived from previous works and how it can be further improved to achieve even better bounds.Also, more explanation is added for the algorithms and the ideas (see \textit{e.g.} Section \ref{overview} and Figure \ref{go}). In addition, we have created a website at \href{https://www.cs.umd.edu/\~saeedrez/fair.html}{https://www.cs.umd.edu/$\sim$saeedrez/fair.html} for the implemented algorithm and all related materials.}
}
\author{
	Mohammad Ghodsi \thanks{Sharif University of Technology} 
	\and MohammadTaghi HajiAghayi \thanks{University of Maryland}
	\thanks{Supported in part by NSF CAREER award CCF-1053605,  NSF BIGDATA grant IIS-1546108, NSF AF:Medium grant CCF-1161365, DARPA GRAPHS/AFOSR grant FA9550-12-1-0423, and another DARPA SIMPLEX grant.
		Portions of this research were completed while the first and the second authors were visitors at the Simons Institute for the Theory of Computing.}
	\and Masoud Seddighin \samethanks[1]
	\and Saeed Seddighin \samethanks[2] \samethanks[3]
	\and Hadi Yami \samethanks[2] \samethanks[3]
}


%\def\baselinestretch{1}

\begin{document}
	\newcommand{\ignore}[1]{}
\renewcommand{\theenumi}{(\roman{enumi}).}
\renewcommand{\labelenumi}{\theenumi}
\sloppy

%\setlength{\abovecaptionskip}{0.1ex}
% \setlength{\belowcaptionskip}{0.1ex}
% \setlength{\floatsep}{0.1ex}
% \setlength{\textfloatsep}{0.1ex}
% 
% 
% \abovedisplayskip.30ex
%   \belowdisplayskip.30ex
%   \abovedisplayshortskip.30ex
%   \belowdisplayshortskip.30ex

\date{}

\maketitle
\textit{``Being good is easy, what is difficult is being just."} \cite{hugo2000miserables} (Victor Hugo, 1862)

\thispagestyle{empty}


\begin{abstract}
\begin{abstract}
\label{sec:abstract}

%% 1. what is the problem 
Scientific applications that run on leadership computing facilities often face the challenge 
of being unable to fit leading science cases onto accelerator devices due to memory constraints 
(memory-bound applications).
%
% 2. what is your solution 
In this work, the authors studied one such US Department of Energy mission-critical condensed matter 
physics application, Dynamical Cluster Approximation (DCA++), and this paper discusses how device memory-bound challenges were successfully reduced  by proposing an effective 
``all-to-all'' communication method---a ring communication algorithm. 
%
This implementation takes advantage of acceleration on GPUs and remote direct memory access (RDMA) for fast data exchange between GPUs. 
%
\\Additionally, the ring algorithm was optimized with sub-ring communicators
and multi-threaded support to further reduce communication overhead and 
expose more concurrency, respectively.
%
% 3. What's the cherry-picked evaluation result you want to mention
The computation and communication were also analyzed 
by using the Autonomic Performance Environment for Exascale 
(APEX) profiling tool,  and this paper further discusses the 
performance trade-off for the ring algorithm implementation. 
%
The memory analysis on the ring algorithm shows that the allocation size for the authors' most 
memory-intensive data structure per GPU is now reduced to $1/p$ of the original size, where $p$ is the number of GPUs in the ring communicator.
%
The communication analysis suggests that 
the distributed Quantum Monte Carlo execution time grows linearly as sub-ring size increases, and the cost of messages passing through the network interface connector could be a limiting factor.


%
% \todoRed{Ronnie: Next sentence needs rewrite, too much information about Green's function that no one knows in the abstract; recommend generalizing.} \emph {However, DCA++ is currently facing memory-bound challenge as 
% a larger device array $G_t$ is limited by device memory size, where
% $G_t$ is a two-particle Green's function that allows condensed matter
% scientists to explore larger and more complex (higher fidelity)
% physics cases.}

\end{abstract}

\keywords{DCA++, Quantum Monte Carlo, GPU Remote Direct Memory Access, memory-bound issue, exascale machines}

\end{abstract}
Reinforcement learning has achieved great success in areas such as Game-playing \citep{silver2018general,vinyals2019grandmaster}, robotics \cite{kober2013reinforcement}, large language models \citep{ouyang2022training}, etc.
However, due to safety concerns or physical limitations, in some real-world reinforcement learning problems, we must consider additional constraints that may influence the optimal policy and the learning process \citep{garcia2015comprehensive}.
% For example, a robotic arm must not take actions that may cause harm to itself or the environments.
A standard framework to handle such cases is the constrained Markov Decision Process (CMDP) \citep{altman1999constrained}.
Within the CMDP framework, the agent has to maximize
the expected cumulative reward while
obeying a finite number of constraints, which are usually in the form of expected cumulative cost criteria.

However, we are sometimes concerned with the problem with a continuum of constraints.
For example,
the constraints we meet might be time-evolving or subject to uncertain parameters, which
cannot be formulated as an ordinary CMDP
(see Examples \ref{Example_Time_Evolving} and  \ref{Example_Uncertain}).
In this paper we would study a generalized CMDP  
to address the above problem.  Because the constraints are not only infinite-number but also lie
in a continuous set,
the generalization is not trivial. Fortunately, we find that we can borrow the idea behind semi-infinite programming (SIP) \citep{remez1934determination, hettich1993semi} to deal with the semi-infinite constraints.
Accordingly, we propose \emph{semi-infinitely constrained Markov decision processes} (SICMDPs)
as a novel complement to the ordinary CMDP framework.
%More specifically,  an SICMDP model %, we consider 
%contains a continuum of constraints whereas an ordinary CMDP contains a finite number of constraints. 

%This generalization is natural but not trivial. However, we can brows the idea  
%The idea is quite natural and can be backtracked
%to the practice of extending linear programming to linear semi-infinite programming (LSIP) %\cite{remez1934determination, GobernaLSIO1998}.
%In addition, 
%As a complementary approach to the ordinary CMDP framework, 
%SICMDP can be used to model these problems  which cannot be described by a finite number of constraints
%that are not covered by .
%For example,
%the restrictions we consider can be time-evolving or subject to uncertain parameters
%, thus
%cannot be described by a finite number of constraints but a continuum of constraints 
%(see Examples \ref{Example_Time_Evolving} and  \ref{Example_Uncertain}).

We also present two reinforcement learning algorithms to solve SICMDPs called SI-CRL and SI-CPO, respectively.
SI-CRL is a model-based reinforcement learning algorithm designed for tabular cases, and SI-CPO is a policy optimization algorithm for non-tabular cases.
% and analyze its performance both theoretically and empirically.
The main challenge is that we need to deal with a continuum of constraints, thus reinforcement learning algorithms for ordinary CMDPs do not work anymore.
In SI-CRL, we tackle this difficulty by first transforming the reinforcement learning problem to an equivalent LSIP problem, which can then be solved using methods in the LSIP literature like the dual exchange methods \citep{Hu1990,reemtsen1998numerical}.
In SI-CPO, we resort to the idea of cooperative stochastic approximation developed in \cite{lan2020algorithms, wei2020comirror}.
As far as we know, we are the first to introduce tools from semi-infinitely programming (SIP) into the reinforcement learning community for solving constrained reinforcement learning problems.

% To the best of our knowledge, we are the first to apply tools from semi-infinitely programming (SIP) to solve reinforcement learning problems.
Furthermore, we give theoretical analysis for both SI-CRL and SI-CPO.
We decompose the error of SI-CRL into two parts: the statistical error from approximating the true SICMDP with an offline dataset and the optimization error due to the fact that the solution of the LSIP problem obtained by the dual exchange method is inexact.
On the optimization side, we show that the iteration complexity of SI-CRL is $O\left(\left\{\mathrm{diam}(Y)L\sqrt{|\gS|^2|\gA|m}/\left[(1-\gamma)\epsilon\right]\right\}^m\right)$.
On the statistical side, we show that the sample complexity of SI-CRL is $\widetilde O\left(\frac{|S|^2|A|^2}{\epsilon^2(1-\gamma)^3}\right)$ if the offline dataset is generated by a generative model, and $\widetilde O\left(\frac{|S||A|}{\nu_{\min} \epsilon^2(1-\gamma)^3}\right)$ if the dataset is generated by a probability measure $\nu$ as considered in \cite{chen2019information}.
Here $\widetilde O$ means that all logarithm terms are discarded.
For SI-CPO, things become a little more complicated because other than the statistical error and the optimization error, we also need to consider the function approximation error, which comes from imperfect policy parametrizations.
It is shown if the function approximation error can be controlled to $O(\epsilon)$ order, the iteration complexity of SI-CPO is $\widetilde{O}\left(\frac{1}{\epsilon^2(1-\gamma)^6}\right)$ and the sample complexity of SI-CPO is $\widetilde{O}(\frac{1}{\epsilon^4(1-\gamma)^{10}})$.
Here our iteration complexity bound is equivalent to a typical $\widetilde O(1/\sqrt{T})$ global convergence rate.

We perform a set of numerical experiments to illustrate the SICMDP model and validate our proposed algorithms.
Specifically, we examine two numerical examples, namely the discharge of sewage and ship route planning.
Through the discharge of sewage example, we show the advantage of the SICMDP framework over the CMDP baseline obtained by naive discretization in modeling realistic sequential decision-making problems.
Moreover, we demonstrate the effectiveness of the SI-CRL and SI-CPO algorithms in such tabular environments. 
In the ship route planning example, we illustrate the benefits of the SICMDP framework and the ability of the SI-CPO algorithm to address complex continuous control tasks involving continuous state spaces with modern deep reinforcement learning techniques.

% In summary, our contributions are listed as follows.
% First, we present the SICMDP model, which can be viewed as a generalization of the ordinary CMDP model.
% Second, we propose an algorithm to perform reinforcement learning for SICMDPs, which is called SI-CRL, and we believe that we are the first to apply tools from SIP
% to solve reinforcement learning problems.
% Third, we give a theoretical analysis of SI-CRL and identify both its sample complexity and iteration complexity.
% In addition, we perform numerical experiments to illustrate the SICMDP model and validate the SI-CRL algorithm.
% \{This paragraph can be removed!!! \}





\section{Preliminaries}\label{chpt:preliminiaries}
In this chapter we will introduce some of the mathematical background and notation needed for this thesis. In particular, we will shortly introduce the differential geometric description of spacetime in Section \ref{sec:spacetime_geometry} and give an introduction to the notion of global hyperbolicity and its connection to Green- and normally-hyperbolic operators in Section \ref{sec:global_hyperbolicity}. In a bit more detail, we will introduce the notion of differential forms and give explicit definitions, also in terms of an index based notation, in Section \ref{sec:differential_forms}. For completeness, in Section \ref{sec:cat-theory}, we present basic definitions of category theory. The reader familiar with these topics can safely skip this chapter and refer to it when interested in the chosen conventions.
%
%
%
%
%%%%%%
%%SPACTIME GEOMETRY
%%%%%
%
%
%
\subsection{Spacetime geometry}\label{sec:spacetime_geometry}
In GR, the universe is mathematically described as a four dimensional \emph{spacetime}, consisting of a smooth, four dimensional manifold \gls{M} (assumed to be Hausdorff, connected, oriented, time-oriented and para-compact) and a Lorentzian metric $g$. We will assume the signature of the Lorentzian metric $g$ to be $(-,+,+,+)$. The Levi-Civita connection on $(\M,g)$ is as usual denoted by \gls{nabla}.
Throughout this thesis, we treat spacetime as fixed, implementing a gravitational background determined classically by Einstein's field equations. Hence, we neglect any back-reaction of the fields on the metric, both in the quantum and the classical case. In that sense, we treat the fields as \emph{test fields}.\par
For the basic mathematical theory regarding Lorentzian manifolds, we refer to the literature: An introduction to the topic with an emphasis on the physical application in GR is for example given in \cite{wald_GR} and \cite{carroll_spacetime-and-gr}.
Here, we will shortly recap the notion of a tangent space and tangent bundle and generalize to the notion of a vector bundle which we will use in the general description of normally hyperbolic operators and differential forms.
In the following, we generalize the setting to an arbitrary smooth manifold $\N$ of dimension $N$ with either Lorentzian or Riemannian metric $k$.\par
%
%
A \emph{tangent vector} $v_x$ at point $x \in \N$ is a linear map $v_x : C^\infty(\N , \IR) \to \IR$ that obeys the Leibniz rule, that is, for $f,g \in C^\infty (\N,\IR)$ it holds $v_x(fg) = f(x)v_x(g) + v_x(f)g(x)$.
We define the \emph{tangent space} \gls{TxN} of $\N$ at $x$ as the real $N$-dimensional vector space of all tangent vectors at point $x$.
The disjoint union of all tangent spaces is called the \emph{tangent bundle} \gls{TN} of $\N$ and is itself a manifold of dimension $2N$. A \emph{vector field} is a map $v: \N \to T\N$ such that $v(x) \in T_x\N$.
The respective dual spaces, that is the space of all linear functionals, the \emph{co-tangent space} and the \emph{co-tangent bundle}, are denoted by \gls{TsxN} and \gls{TsN} respectively.\par
%
For Lorentzian manifolds, we call a tangent vector $v$ at $x \in \N$ \emph{timelike} if $k_{\mu \nu} v^\mu v^\nu < 0$, \emph{spacelike} if $k_{\mu \nu} v^\mu v^\nu > 0$ and \emph{null} (or lightlike) if $k_{\mu \nu} v^\mu v^\nu = 0$. At every point $x \in \N$, we define the set of all \emph{causal}, that is, either timelike or null, tangent vectors in the tangent space at $x$. This set is called the \emph{light cone} at $x$ and it is split up into two distinct parts, one that we call the future light cone, and one that we call the past light cone at $x$. Since we assume the manifold to be time orientable, there exists a smooth vector field $t$ that is timelike at every $x \in \N$. Given this time orientation, we identify the future (past) light cone with the set of tangent vectors $v \in T_x\N$ such that $k_{\mu\nu} v^\mu t^\nu < 0$ (respectively $> 0$). Therefore, a tangent vector $v$ at $x$ is called \emph{future directed} (past directed) if it lies in the future (past) light cone at $x$.\\
Accordingly, a curve $\gamma : I \to \N$ is called timelike (spacelike, null, causal, future or past directed) if its tangent vector $\dot{\gamma}$ is timelike (spacelike, null, causal, future or past directed) at every $x \in \N$.  For every point $x \in \N$ we define the \emph{causal future/past} \gls{causalfuturepast} of $x$ as the set of all points $q \in \N$ that can be reached by a future directed causal curve originating in $x$. For any subset $S \in \N$ we define $J^\pm (S) = \bigcup_{x \in S} J^\pm(x)$ and $J(S) = J^+(S) \cup J^- (S)$. Finally, the future/past domain of dependence $\gls{futurepastdomainofdependence}$ of a set $S \subset \N$ is the set of all points $x \in \N$ such that every inextendible causal curve through $x$ intersects $S$. The \emph{domain of dependence} \gls{domainofdependence} of $S$ is the union of the future and past domain of dependence of the set $S$.
For more details on the causal structure of spacetime we refer to for example \cite[Chapter 8]{wald_GR}.\par
%
%
%
The notion of tangent bundles can be generalized to the notion of a vector bundle. Instead of ``attaching'' the vector spaces $T_x \N$ to every point $x$ of the manifold, we allow for the occurrence of arbitrary vector spaces, called the fibres of the vector bundle. A vector bundle then consists of the base manifold, in our case $\N$, the total space and a map $\pi$ from the total space to the base manifold, that can be locally trivialized. At each point of the base manifold, the pre-image of $\pi$ is the fibre of the vector bundle. To be precise we define, following \cite{rudolph_schmidt}:
\begin{definition}[Vector bundle]
	A smooth \emph{vector bundle} over $\N$ is a tuple $\gls{vectorbundle} = (E,\N, \pi)$, where $E$ is a smooth manifold and $\pi : E \to \N$ is a smooth surjective map satisfying:
	\begin{enumerate}
		\item For every $x \in \N$, $\pi^{-1}(x)$ is a vector space, called the fibre of the bundle at point $x$.
		\item There exists a finite dimensional vector space $F$, an open covering $\left\{ U_\alpha\right\}_\alpha$ of $\N$ and a family of diffeomorphisms $\chi_\alpha : \pi^{-1}(U_\alpha) \to U_\alpha \times F$ such that for all $\alpha$ it holds $\chi_\alpha \comp \text{pr}_1 =  \restr{\pi}{\pi^{-1}(U_\alpha)}$ and for every $x \in \N$ the map $\text{pr}_2 \comp \restr{\chi_\alpha}{\pi^{-1}(x)} : \pi^{-1}(x) \to F$ is linear.
	\end{enumerate}
\end{definition}
Here, the maps $\text{pr}_1$ and $\text{pr}_2$ denote the projection onto the first respectively second component of an element in $U_\alpha \times F$. The properties graphically mean that \emph{locally}, the vector bundle ``looks like" the product of the base manifold with the fibre. The tuples $(U_\alpha, \chi_\alpha)$ are called \emph{local trivializations} of the vector bundle. Like for vector spaces, we can define the sum and product of vector bundles, by using the according vector space definitions on the fibres of the bundle.\par
Let $\mathfrak{X}, \mathfrak{Y}$ be vector bundles over $\N$ with fibres $X_x$ and $Y_x$ at $x \in \N$. We denote by \gls{whitneysum} the \emph{Whitney sum} of the two vector bundles - the vector bundle over $\N$ whose fibres are given by the direct sum $X_x \oplus Y_x$. Similarly, one obtains the local trivializations of the Whitney sum from the trivializations of $\mathfrak{X}, \mathfrak{Y}$ and direct sums.\par
Accordingly, let $\mathfrak{X}, \mathfrak{Y}$ be vector bundles over $\N$ and $\widetilde{\N}$, with fibres $X_x$ and $Y_{\tilde{x}}$ at $x \in \N$, $\tilde{x} \in \widetilde{\N}$ respectively. We denote by \gls{outerproductbundle} the \emph{outer product} of the two vector bundles - the vector bundle over $\N \times \widetilde{\N}$ whose fibres are given by the tensor products $X_x \otimes Y_x$. Similarly, one obtains the local trivializations of the outer product from the trivializations of $\mathfrak{X}, \mathfrak{Y}$ and tensor products. \par
%
Finally, we generalize the notion of vector fields:
\begin{definition}[Sections of vector bundles]
Let $\mathfrak{X}=(E,\N,\pi)$ be a vector bundle with fibres $X_x=\pi^{-1}(x)$ at $x \in \N$. A \emph{smooth section} of the vector bundle is a smooth map $\gamma : \N \to E$ such that $\gamma(x) \in X_x$ for all $x \in \N$. The \emph{vector space of smooth sections} of $\mathfrak{X}$ is denoted by \gls{gammax}, the one with compactly supported sections is as usual denoted by \gls{gammaxzero}.
\end{definition}
In this language, a vector field $v$ is just a smooth section of the tangent bundle of a manifold, $v \in \Gamma(T\N)$. One may therefore identify the physical notion of fields with smooth sections of vector bundles. This point of view will be used to define the notion of differential forms in Section \ref{sec:differential_forms}.\par
In this thesis, we usually are interested in complex valued functions (or sections in general). Therefore, we view all occurring vector bundles as complex, in the sense that we take two distinct copies of the vector bundle, one representing the real, one the imaginary part of the bundle. A section of that complex vector bundle is just a pair of two sections of the real vector bundle under consideration. From now, if not specified explicitly, we will view all vector bundles, including the tangent bundle $T\N$, as complex vector bundles. Accordingly, smooth sections of those bundles will in general be complex valued.
%
%
%
%
%
%
%
%
%%%%%%%
%%PARTIAL DIFFERENTIAL OPERATORS AND GLOBAL HYPERBOLICITY
%%%%%%%
%
%
%
\subsection{Partial differential operators and global hyperbolicity}\label{sec:global_hyperbolicity}
When dealing with field theories, whether classical or quantum, one is, of course, interested in the dynamics of the fields. These are usually described by some partial differential equation, often of second order. In the following, we give a short introduction to the theory of certain partial differential operators acting on smooth sections of a vector bundle over the spacetime $(\M,g)$.\par
%
As we have seen, these smooth sections are generalizations of the notion of a field.  In the following, let $\mathfrak{X}$ denote a vector bundle over the manifold $\M$ and let $P: \Gamma(\mathfrak{X}) \to \Gamma(\mathfrak{X})$ be a partial differential operator acting on smooth sections of the bundle. As in the case of flat spacetime, we are interested in basic questions regarding the differential equation $Pf = j$, for example: Can we formulate a (globally) well posed initial value problem? Does the differential equation possess (unique) solutions? To answer these questions, we will now restrict to the case where $P$ is linear and of second order, as it is often the case in physical applications. One can show that for a certain class of such operators, namely normally hyperbolic partial differential operators of second order, we can rigorously treat these questions.\par
Choosing local coordinates $x=(x_\mu)$ on $\M$ and a local trivialization of $\mathfrak{X}$, a linear partial differential operator of second order is called \emph{normally hyperbolic} if it takes the form
\begin{align}
	P = - \sum_{\mu,\nu} g^{\mu \nu} \partial_\mu \partial_\nu + \sum_{\alpha} A_\alpha (x) \partial_\alpha + B(x) \formspace,
\end{align}
where $A_\alpha$ and $B$ are matrix-valued coefficients depending smoothly on the coordinate $x$ (see. \cite[Chapter 1.5]{baer_ginoux_pfaeffle}). One can also formulate a coordinate independent definition in terms of the principal symbol, which we will not present here (see for example \cite[Section 1.5]{baer_ginoux_pfaeffle} ). \par
%
Normally hyperbolic operators possess unique fundamental solutions (see for example the fundamental solutions to the wave operator as noted in Lemma \ref{lem:fundamental_solution_wave_operator}). These fundamental solutions fulfill certain physically important properties, such as a finite propagation speed smaller than the speed of light. Furthermore, specifying the initial data on some space-like hypersurface $X \in  \M$ specifies a unique solution on the domain of dependence $D(X)$ of $X$. Due to these properties, one often calls normally hyperbolic operators just \emph{wave operators}. But to state a \emph{globally} well posed initial value problem for a wave equation, we need to restrict the class of spacetimes $\M$ under consideration to those that possess space-like hypersurfaces $X$ whose domain of dependence is all of the spacetime, $D(X) = \M$. This leads to the notion of \emph{globally hyperbolic} spacetimes:
\begin{definition}[Global Hyperbolicity]
	A spacetime $\M$ is called \emph{globally hyperbolic} if there exists a Cauchy surface $\gls{sigma}$ in $\M$.
\end{definition}
\noindent Here, a Cauchy surface is a space-like hypersurface $\Sigma \subset \M$ such that every inextendible causal curve $\gamma$ intersects $\Sigma$ exactly once. One can show that Cauchy surfaces fulfill the desired property mentioned above, that is,  $D(\Sigma) = \M$. Furthermore, one can show that any globally hyperbolic spacetime $\M$ is foliated by a one-parameter family $\left\{ \Sigma_t \right\}_t$ of Cauchy surfaces (see for example \cite[Theorem 8.3.14]{wald_GR}). \par
In physical applications, one often finds the dynamics of a theory to be described by wave operators. Most prominently, the Klein-Gordon operator $(\square + m^2)$ acting on scalar fields, or its generalization, the wave operator acting on differential forms introduced in Section \ref{sec:differential_forms}, is normally hyperbolic. But there are also important physical field theories that are not described by wave operators, such as the Proca field treated in this thesis. It turns out that the Proca operator (see Definition \ref{def:proca_operator}) is a so called \emph{Green-hyperbolic} operator. These are again partial differential operators $P$ of second order acting on smooth sections of some vector bundle, such that $P$ (and its dual $P'$) posses fundamental solutions. Obviously, normally hyperbolic operators are Green-hyperbolic, but the opposite is not true. One can generalize some results obtained by studying normally hyperbolic operators to Green-hyperbolic operators. An introduction to this topic is given in \cite{baer_green-hyperbolic}, where it is also shown that the Proca operator is Green-hyperbolic but not normally hyperbolic.\par
For our application, the notion of Green-hyperbolicity is not of vast importance, but it is worth mentioning that there exists a more detailed mathematical background on the treatment of such operators.
A very detailed description of normally hyperbolic operators on Lorentzian manifolds, including proofs of the above statements regarding the initial value problem and the existence of fundamental solutions, is given in \cite{baer_ginoux_pfaeffle}, also with an overview of quantization. A shorter introduction to the topic is for example treated in \cite{baer-ginoux_classical-and-quantum-fields}, also with a description of quantization.
%
%
%
%
%
%
%%%
%
%
%
%%
%%%%%%%%%
%%%DIFFERENTIAL FORMS
%%%%%%%%
%
%
%
\subsection{Differential forms}\label{sec:differential_forms}
%
%
Differential forms provide an elegant, coordinate independent description of calculus on smooth manifolds. In particular, they generalize the notion of line- and volume-integrals that are known from analysis. Differential forms play a remarkable role in physics, as one can argue that they indeed describe fundamental physical entities. As an example, instead of viewing a classical force as a vector, one can think of it, more closely related to experiments, as a differential one-form that assigns a scalar to a tangent vector of a curve. This scalar is the (infinitesimal) work associated with the force along the curve. Also, differential forms allow for an elegant geometric description of field theories, for example the Maxwell and Proca field theories that we encounter in this thesis. In Maxwell's classical theory of electromagnetism, instead of viewing the electric and magnetic field (which are conceptually just forces) as the fundamental physical entities, one introduces the \emph{vector potential}, a one-form, consisting of the scalar electric potential and the vector potential associated with the magnet field. Experiments like the Aharonov-Bohm experiment allow for an interpretation of the vector potential as the fundamental physical object, rather than the associated electromagnetic field. \\
Even more fundamentally, the two main theories of physics, General Relativity and the Standard Model of particle physics, are field theories. They are deeply connected to a geometric interpretation and can be elegantly described using differential forms. \par
%
%
Despite of all this, differential forms are usually not part of the standard curriculum of physicists. We shall therefore introduce the basic aspects and definitions regarding differential forms that are used in this thesis. For a more detailed introduction we refer to the literature: For example \cite[Chapter 2 and 4]{rudolph_schmidt} or \cite[Appendix B]{wald_GR} provide introductions to the topic.\par
%
%
In the following, let $\N$ denote a smooth $N$-dimensional manifold, assumed to be Hausdorff, connected, oriented and para-compact, with either Lorentzian or Riemannian metric $k$ and Levi-Civita connection $\nabla$. For a Lorentzian manifold we use the sign convention $(-,+,\dots,+)$ of the metric $k$. The number of negative eigenvalues of $k$ is denoted by $s$, so $s=0$ for a Riemannian manifold and, in our convention, $s=1$ for a Lorentzian manifold.
Later, we will specify to a four dimensional (globally hyperbolic) spacetime consisting of a four dimensional manifold $\M$ with Lorentzian metric $g$ and Cauchy surface $\Sigma$ with induced Riemannian metric $h$.
%
We define:
\begin{definition}[Differential form]
	Let $p\in \{0,1,\dots,N\}$. A \emph{differential form} $\omega$ of degree $p$, or $p$-form for short, on the manifold $\N$ is an anti-symmetric tensor field of rank $(0,p)$. That is, at every point $x \in \N$, $\omega_x$ is an anti-symmetric multi-linear map
	\begin{align}
	\omega_x : \underbrace{T_x \N \times T_x \N \times \cdots \times T_x \N}_{p\text{-times}} \to \IR \formspace.
	\end{align}
	We denote the vector space\footnote{Naturally, addition and scalar multiplication are defined point-wise.} of $p$-forms on $\N$ by $\gls{omegap}$, the space with compactly supported ones by \gls{omegapz}.
\end{definition}
As an example, a zero-form $f \in \Omega^0(\N)$ is just a $C^\infty$-function from $\N$ to $\IR$, hence we can identify $\Omega^0(\N) = C^\infty (\N, \IR)$. A one-form $A \in \Omega^1(\N)$ is nothing more than a co-vector field and in a physical context usually denoted in local coordinates by $A_\mu$. Note, that alternatively one can directly define a $p$-form as a smooth section of the $p$-th exterior product of the co-tangent bundle and hence identify $\Omega^p(\N) = \Gamma \big( \largewedge^k T^*\N\big)$. As mentioned in Section \ref{sec:spacetime_geometry}, we view the tangent bundle as a complex bundle. Therefore, the sections of that bundle will be complex valued functionals. In that fashion, we will usually view the spaces $\Omega^p(\N)$ as complex valued differential forms.\par
%
Next we define the basic operations, besides addition and scalar multiplication, that one can perform on differential forms.
%
\begin{definition}[Exterior product]
	Let $A \in \Omega^p(\N)$ be a $p$-form and  $B\in \Omega^q(\N)$ a $q$-form on $\N$. \\
	The \emph{exterior product} $\gls{wedge}:\Omega^p(\N) \times \Omega^q(\N) \to \Omega^{p+q} (\N)$ is defined by
	\begin{align}
	(A \wedge B)_{\mu_1\dots\mu_p \nu_1\dots\nu_q} = \frac{(p+q)!}{p!q!}\, A_{[\mu_1 \dots \mu_p} B_{\nu_1\dots\nu_q]} \formspace,
	\end{align}
	where the anti-symmetrization of a tensor $T$ is given through
	\begin{align}
	T_{[\mu_1\dots\mu_p]} = \frac{1}{p!} \sum\limits_{\sigma\in S_N }\textrm{sgn}(\sigma) T_{\sigma(\mu_1)\dots\sigma(\mu_p)} \formspace.
	\end{align}
\end{definition}
Here, $S_N$ denotes the symmetric group\footnote{Usually the symmetric group is defined as the set of permutations of $\{1,2,\dots,N\}$ but we chose the index to run over $\{0,1,\dots,N-1\}$, identifying the time component with zero rather then one.} of degree $N$, consisting of permutations of the set $\{0,1,\dots,N-1\}$.
With this notion of multiplication, point-wise addition and scalar multiplication, the space $\gls{omega} \coloneqq \bigoplus_{p = 0}^\infty \Omega^p(\N) = \bigoplus_{p = 0}^N \Omega^p(\N)$ becomes an algebra, usually called the Grassmann- or \emph{exterior algebra} of differential forms on $\N$. We have used that obviously $\Omega^k(\N) =0$ for $k >N$ due to the anti-symmetrization.\par
Furthermore, we find a notion of how to \emph{pullback} differential forms on manifolds to another manifold, for example the pullback of a differential form on the spacetime $\M$ to differential forms on its Cauchy surface $\Sigma$. Given a $C^\infty$-map $\psi: \widetilde{\N} \to \N$, where $\N, \widetilde{\N}$ are manifolds, we can naturally define the pullback of a function $f \in \Omega^0(\N)$ to a function $(\psi^* f) \in \Omega^0(\widetilde{\N})$ by composing $f$ with $\psi$:
\begin{align}
\psi^* f \coloneqq f \comp \psi \formspace.
\end{align}
\newpage
With the pullback of functions defined, we can define how to \emph{push forward}, or carry along, vector fields on $\widetilde{\N}$ to vector fields on $\N$: Let $f\in \Omega^0(\N)$ and $\tilde{v} \in \Gamma(T\widetilde{\N})$ and $\tilde{x} \in \widetilde{\N}$. Then
\begin{align}
(\psi_* \tilde{v})_{\psi(\tilde{x})} (f) \coloneqq \tilde{v}_{\tilde{x}}(\psi^* f)
\end{align}
defines the vector field $(\psi_* v) \in \Gamma(T\N)$. With these basic operations at hand, we can generalize to define the pullback of differential forms:
\begin{definition}[Pullback]\label{def:pullback}
	Let $\N, \widetilde{\N}$ be manifolds of dimension $N,\widetilde{N}$ respectively, and let $\psi: \widetilde{\N} \to \N$ be a smooth map. Then, $\psi$ defines an algebra homomorphism $\psi^* : \Omega(\N) \to  \Omega(\widetilde{\N})$,
	called the \emph{pullback} of differential forms. For $\omega \in \Omega^p(\N)$, $\tilde{x} \in \widetilde{\N}$ and $\tilde{v}_i \in T_x \widetilde{\N}$, $i=1,2,\dots,p$, it is defined by
	\begin{align}
	\left( \psi^* \omega \right)_{\tilde{x}}  (\tilde{v}_1,\tilde{v}_2,\dots,\tilde{v}_p) \coloneqq \omega_{\psi(\tilde{x})} (\psi_* \tilde{v}_1, \dots , \psi_* \tilde{v}_p) \formspace.
	\end{align}
\end{definition}
%
%
%
%
On the exterior algebra we find a duality, provided by the Hodge operator:
\begin{definition}[Hodge dual]
	The hodge star operator $\gls{hodge}: \Omega^p(\N) \to \Omega^{N-p}(\N)$ is defined through
	\begin{align}
	B \wedge *A = \frac{1}{p!} B^{\mu_1\dots\mu_p}A_{\mu_1\dots\mu_p} \dvolk \formspace,
	\end{align}
	which yields the coordinate representation
	\begin{align}
	(*A)_{\mu_{p+1}\dots\mu_N} = \frac{\detk}{p!} \, \epsilon_{\mu_1\dots\mu_N} A^{\mu_1\dots\mu_p} \formspace.
	\end{align}
\end{definition}
Here, \gls{levicivita} denotes the fully antisymmetric tensor of rank $N$ (Levi-Civita symbol) satisfying $\epsilon_{12,\dots,N} =1$ and the \emph{volume element} \gls{dvolk} is defined by
\begin{align}
\left( \gls{dvolk} \right)_{\alpha_1\dots\alpha_N} = \detk \, \epsilon_{\alpha_1\dots\alpha_N} \formspace.
\end{align}
In a sense, the volume element describes how the curvature of the manifold deforms a unit volume.
The duality follows from the important property of the Hodge operator as stated in the following lemma:
\begin{lemma}
	Let $*$ denote the Hodge star operator on the exterior algebra $\Omega(\N) $. It holds that
	\begin{align}
	** = (-1)^{s+p(N-p)} \, \mathbbm{1} \formspace,
	\end{align}
	which is trivially equivalent to $*^{-1} = (-1)^{s+p(N-p)} \, *$.
\end{lemma}
\begin{proof}
	Let $A \in \Omega^p(\N)$ be a $p$-form on $\N$. Then:
	\begin{align}
	(*{*A})_{\mu_1 \dots \mu_p}
	&= \frac{\detk \, \detk}{p! \, (N-p)!} \; \epsilon_{\alpha_{p+1}\dots\alpha_N \mu_1 \dots \mu_p}\;\epsilon^{\alpha_{1}\dots\alpha_N}\;A_{\alpha_1\dots\alpha_p} \notag\\
	&= (-1)^{p(N-p)} \frac{\detk \, \detk}{p! \, (N-p)!} \; \epsilon_{\alpha_{p+1}\dots\alpha_N \mu_1 \dots \mu_p}\;\epsilon^{\alpha_{p+1}\dots\alpha_{N}\alpha_1\dots\alpha_p}\;A_{\alpha_1\dots\alpha_p}  \notag\\
	&= (-1)^{s+p(N-p)} \delta\indices{^{[\alpha_{1}}_{\mu_{1}}}\, \dots \, \delta\indices{^{\alpha_p ] }_{\mu_p}} \;A_{\alpha_1\dots\alpha_p} \notag\\
	&=  (-1)^{s+p(N-p)}\;A_{\mu_1\dots\mu_p} \formspace
	\end{align}
	We have used Lemma \ref{lem:epsilon_contraction} and, in the last step, that the anti-symmetrization is absorbed by contraction because $A$ is antisymmetric.
\end{proof}
%
%
%
%
%
Furthermore, we can equip the exterior algebra with a differentiable structure, introducing the notion of the exterior derivative.
\begin{definition}[Exterior derivative]
	The \emph{exterior derivative} $\gls{d}:\Omega^p(\N) \to \Omega^{p+1} (\N)$ is defined by the following properties:
	\begin{enumerate}
		\item $d$ is linear
		\item $d$ obeys a graded Leibniz rule: Let $A \in \Omega^p(\N)$ and  $B\in \Omega^q(\N)$, then
		\begin{align}
		d(A \wedge B) = dA \wedge B + (-1)^p \, A \wedge dB
		\end{align}
		\item $d$ is nilpotent, that is,  $d^2 = 0$.
	\end{enumerate}
	In local coordinates, this is equivalent to the representation
	\begin{align}
	(dA)_{\mu \alpha_1\dots\alpha_p} = (p+1)\, \nabla_{[\mu}A_{\alpha_1\dots\alpha_p]} \formspace.
	\end{align}
\end{definition}
An important property of the exterior derivative is that it commutes (or rather intertwines its action) with pullbacks (see \cite[Proposition 4.1.7]{rudolph_schmidt}).
A $p$-form $\omega \in \Omega^p(\N)$ is called \emph{exact} if there is a $(p-1)$-form $\alpha \in \Omega^{p-1}(\N)$ such that $\omega = d\alpha$. We call $\omega$ \emph{closed} if $d \omega =0$. Accordingly, the space of closed $p$-forms is denoted by \gls{omegapd}, the space of exact ones by \gls{domegap}. As usual, the ones with compact support are denoted by a subscript zero. Note, that every exact form is closed, using that $d$ is by definition nilpotent, but the reverse is in general not true. It does hold, however, on certain manifolds with trivial topology, such as Minkowski spacetime. This is expressed in the so called Poincar\'e-Lemma (see for example \cite[Chapter 4]{bott_tu}) based on the study of de Rham cohomology.\par
%
Moreover, $N$-forms can naturally be integrated. Using local coordinates and a partition of unity, we define the integral of $N$-forms via the well known integration on $\IR^N$:
\begin{definition}[Integration on manifolds]
	Let $\left\{U_\alpha, \psi_\alpha\right\}_\alpha$ be an atlas of the manifold $\N$ and $\left\{\chi_\alpha\right\}_\alpha$ a partition of unity subordinate to the locally finite open cover $\left\{U_\alpha\right\}_\alpha$. Let $x^\mu_{(\alpha)}$ be a coordinate basis of $\psi$ on $U_\alpha$. For any $N$-form $\omega \in \Omega^N_0(\M)$ we define the integral
	\begin{align}
	\int\limits_{\N} \omega &\coloneqq \sum_{\alpha} \int\limits_{\psi_\alpha (U_\alpha)} w(x_{(\alpha)}^0,\dots,x_{(\alpha)}^1)\; dx_{(\alpha)}^0 \cdots dx_{(\alpha)}^{N-1} \formspace,
	\end{align}
	where $w$ are the components of $\omega$ in the coordinates $x_{(\alpha)}^\mu$, that is $\omega = w dx_{(\alpha)}^0 \wedge \cdots \wedge dx_{(\alpha)}^{N-1}$.
	This definition is independent of the choice of the atlas and the partition of unity (see \cite[Proposition 3.3]{bott_tu}).
\end{definition}
With integration at our disposal, we present an important theorem regarding the integration of exact differential forms:
\begin{theorem}[Stoke's Theorem]\label{thm:stokes}
	Let $\N$ be an oriented manifold of dimension $N$ and let its boundary $\partial \N$ be endowed with the induced orientation. Let $\gls{inclusionmap} : \partial \N \hookrightarrow \N$ be the inclusion operator.
	Let $\omega \in \Omega^{N-1}_0(\N)$ be a compactly supported $(N-1)$-form on $\N$. Then it holds
	\begin{align}
	\int\limits_\N d\omega = \int\limits_{\partial \N} i^*\omega \formspace.
	\end{align}
\end{theorem}
\begin{proof}
	A proof is given in most of the introductory literature on differential geometry (see for example \cite[Chapter 17, Theorem 2.1]{lang}).
	Note that one can equivalently formulate Stoke's theorem on a \emph{compact} manifold but for {arbitrary} (that is, in general not compactly supported) $(N-1)$-forms on the manifold (see for example \cite[Theorem 4.2.14]{rudolph_schmidt}). This will be of importance in later calculations.
\end{proof}
%
Furthermore, we can define a bilinear map on $\Omega^p(\N)$ using the integration of $N$-forms:
\begin{definition}
	Let $A,B \in \Omega^p(\N)$ such that their supports have a compact intersection. Define the bilinear map $\gls{innerprod} : \Omega^p(\N) \times \Omega^p(\N) \to \IC$ by
	\begin{align}
	\langle A, B \rangle_\N \coloneqq  \int_{\N } A \wedge * B = \int_{\N } A_{\mu_1 \dots \mu_p}B^{\mu_1 \dots \mu_p}\,\dvolk \formspace.
	\end{align}
\end{definition}
Since by definition $A \wedge * B$ is a compactly supported $N$-form, this is well defined. We may sometimes refer to $\langle \cdot , \cdot \rangle_\N$ as an inner product for simplicity, even though it is not positive definite.
%
%
%
%
%
Using the exterior derivative, we define the interior or co-derivative:
\begin{definition}[Interior derivative]
	The \emph{interior derivative} $\gls{delta} : \Omega^p(\N) \to \Omega^{p-1}(\N)$ is defined by
	\begin{align}
	\delta \coloneqq (-1)^{s+1+N(p-1)}\, {*{d*}} \formspace.
	\end{align}
	From the defining properties of $d$ and $*$ it follows $\delta^2 =0$.
\end{definition}
Here, $s$ again denotes the number of negative eigenvalues of the metric $k$ of $\N$. In accordance with our nomenclature, we call a $p$-form $\omega$ co-exact if there exists a $\alpha \in \Omega^{p+1}(\N)$ such that $\omega = \delta \alpha$ and co-closed if $\delta \omega = 0$. Accordingly, the spaces of co-closed and co-exact $p$-forms are denoted by \gls{omegapdelta} and \gls{deltaomegap} respectively.\par
Using the exterior and interior derivative we define the partial differential operator:
\begin{definition}[D'Alembert Operator]
	The d'Alembert (or Laplace - de Rham) operator $\gls{dalembert}: \Omega^p(\N) \to \Omega^{p}(\N)$ is defined by
	\begin{align}
	\square \coloneqq \delta d +d \delta \formspace.
	\end{align}
\end{definition}
By definition of the exterior and interior derivative, it is easy to show that $\square$ commutes with both $d$ and $\delta$:
\begin{align}
\square d &= (\delta d + d \delta )d \notag \\
&= d \delta d \notag \\
&= d (\delta d + d \delta) \formspace,
\end{align}
and analogously for $\delta$.
The d'Alembert operator, and its generalization to $(\square + m^2)$ for some constant $m > 0$, are important examples for a normally hyperbolic differential operators (see Section \ref{sec:global_hyperbolicity}) and we may therefore sometimes just refer to them as \emph{wave operators}.\par
The sign convention in the definition of the exterior derivative is chosen such that on any Lorentzian or Riemannian manifold the interior derivative is formally adjoint to the exterior derivative, that is,  for $A \in \Omega^{p}(\N)$ and $B \in \Omega^{p+1}(\N)$ it holds that
\begin{align}
\langle dA , B \rangle_{\N} = \langle A , \delta B \rangle_\N \formspace,
\end{align}
which leads to a representation in local coordinates of the Manifold given by:
\begin{align}
(\delta A)_{\mu_2\dots\mu_p} = - \nabla^{\mu_1}A_{\mu_1\dots\mu_p} \formspace.
\end{align}
To see that this is consistent, let $A \in \Omega^{p-1}(\N)$ and $B \in \Omega^{p}(\N)$ such that their supports have compact intersection.
We obtain, using Stoke's Theorem \ref{thm:stokes}:
\begin{align}
0 &= \int \limits_{\partial \N} i^* (A \wedge *B) \notag\\
&= \int \limits_{\N} d(A \wedge *B)  \notag\\
&= \int \limits_{\N} dA \wedge *B + (-1)^{p-1} A \wedge d{*B} \notag\\
&= \int \limits_{\N} dA \wedge *B + (-1)^{p-1} A \wedge *{*^{-1}}\underbrace{d{*B}}_{\textrm{is a } (N-p+1) \textrm{ form.}} \notag\\
&= \int \limits_{\N} dA \wedge *B + (-1)^{p-1}(-1)^{s+(N-p+1)(N-N+p-1)} A \wedge *{*d{*B}} \notag\\
&= \int \limits_{\N} dA \wedge *B + (-1)^{p+(1-p)(p-1)} A \wedge *\delta B \formspace.
\end{align}
It can easily be proven by induction that $\big(p+(1-p)(p-1)\big)$ is odd for any $p \in \IN$, which yields the result
\begin{align}
\langle dA , B \rangle_{\N} = \langle A , \delta B \rangle_\N \formspace.
\end{align}
The definitions stated above thus fulfill the requirement of formal adjointness of the exterior and interior derivate on an arbitrary Lorentzian or Riemannian manifold $\N$.
In local coordinates we use a partial integration to obtain
\begin{align}
\langle dA , B \rangle_\N &= \int \limits_{\N} dA \wedge * B \notag\\
%&= \int \limits_{\N} \frac{1}{p!} (dA)^{\alpha_1\dots\alpha_p}\,B_{\alpha_1 \dots \alpha_p} \, \dvolk \notag\\
&= \int \limits_{\N}  \frac{p}{p!} \nabla^{[\alpha_1}A^{\alpha_2\dots\alpha_p]}\,B_{\alpha_1 \dots \alpha_p} \, \dvolk \notag\\
&= \int \limits_{\N}  \frac{1}{(p-1)!} \nabla^{\alpha_1}A^{\alpha_2\dots\alpha_p}\,B_{\alpha_1 \dots \alpha_p} \, \dvolk \notag\\
&= - \int \limits_{\N}  \frac{1}{(p-1)!} A^{\alpha_2\dots\alpha_p}\, \nabla^{\alpha_1}B_{\alpha_1 \dots \alpha_p} \, \dvolk \notag\\
&= \langle A, \delta B \rangle_\N \formspace,
\end{align}
which yields
\begin{align}
-\nabla^{\alpha_1}B_{\alpha_1 \dots \alpha p} = (\delta B)_{\alpha_2 \dots \alpha_p}\formspace.
\end{align}
On the four dimensional spacetime $(\M,g)$ the definitions of the Hodge star operator and the interior derivative simplify, such that
\begin{align}
*_{(\M)}*_{(\M)} &= (-1)^{p+1} \mathbbm{1} \\
\delta_{(\M)} &= *_{(\M)}{d_{(\M)}*_{(\M)}} \formspace ,
\end{align}
holds on the spacetime $(\M,g)$ and
\begin{align}
*_{(\Sigma)}*_{(\Sigma)} &= \mathbbm{1} \\
\delta_{(\Sigma)} &= (-1)^p *_{(\Sigma)}{d_{(\Sigma)}*_{(\Sigma)}}
\end{align}
holds on  $(\Sigma,h)$. In the following we will drop the subscript ${(\M)}$, since we will perform all the calculations on a four dimensional spacetime, except when explicitly noted (for example with a subscript $(\Sigma)$).
%
%
%
%
%
%
%
%
%%%%%%
%%CATEGORY THEORY
%%%%%%
\subsection{Category theory}\label{sec:cat-theory}
The description of Quantum Field Theory on Curved Spacetimes (QFTCS) in the framework of \name{Brunetti}, \name{Fredenhagen} and \name{Verch} \cite{Brunetti_Fredenhagen_Verch} is based on category theory. In this thesis, we will not go into detail on those categorical aspects, however we will need some basic definitions to formulate the theory rigorously, that is namely the notion of a category and that of covariant functors, since, in the used framework, the generally covariant QFTCS is a functor.\par
Here, we present definitions given in \cite[Appendix A.1]{baer_ginoux_pfaeffle} and refer to the appropriate literature for details. We define:
\begin{definition}[Category]
	A \emph{category} $\mathsf{Cat}$ consists of the following:
	\begin{enumerate}
		\item a class $\mathsf{Obj}_\mathsf{Cat}$ whose members are called \emph{objects},
		\item a set $\mathsf{Mor}_\mathsf{Cat}(A,B)$, for any two objects $A,B \in \mathsf{Obj}_\mathsf{Cat}$, whose elements are called \emph{morphisms},
		\item for any three objects $A,B,C \in \mathsf{Obj}_\mathsf{Cat}$ there is a map
		\begin{align}
\mathsf{Mor}_\mathsf{Cat}(B,C) \times \mathsf{Mor}_\mathsf{Cat}(A,B) &\to \mathsf{Mor}_\mathsf{Cat}(A,C) \notag\\
(\psi,\phi) &\mapsto \psi \comp \phi
		\end{align}
		called the composition of morphisms subject to the relations:\vspace{4mm}
		\begin{enumerate}[label=(\arabic*)]
			\item for non equal pairs $(A,B)$, $(A',B')$ of objects, the sets $\mathsf{Mor}_\mathsf{Cat}(A,B)$ and $\mathsf{Mor}_\mathsf{Cat}(A',B')$ are disjoint,
			\item for every object $A$ there exists a morphism $\text{id}_A \in \mathsf{Mor}_\mathsf{Cat}(A,A)$ such that it holds for all objects $B$, morphisms $\psi \in \mathsf{Mor}_\mathsf{Cat}(B,A)$ and $\phi \in \mathsf{Mor}_\mathsf{Cat}(A,B)$
			\begin{align}
				\text{id}_A \comp \psi &= \psi \quad \text{and}\\
				\phi \comp \text{id}_A &= \phi \quad,
			\end{align}
			\item the composition law is associative, that is for an objects $A,B,C,D$ and any morphisms $\psi \in \mathsf{Mor}_\mathsf{Cat}(A,B)$, $\phi \in \mathsf{Mor}_\mathsf{Cat}(B,C)$ and $\chi \in \mathsf{Mor}_\mathsf{Cat}(C,D)$ it holds
			\begin{align}
				(\chi \comp \phi) \comp \psi = \chi \comp (\phi \comp \psi) \formspace.
			\end{align}
		\end{enumerate}
	\end{enumerate}
\end{definition}
%
%
%
\begin{definition}[Functor]
	Let $\mathsf{Cat1}$ and $\mathsf{Cat2}$ be categories. A \emph{covariant functor} $\mathscr{A}: \mathsf{Cat1} \to \mathsf{Cat2}$ consists of the map $\mathscr{A} : \mathsf{Obj}_\mathsf{Cat1} \to \mathsf{Obj}_\mathsf{Cat2}$ and maps $\mathscr{A}: \mathsf{Mor}_\mathsf{Cat1}(A,B) \to \mathsf{Mor}_\mathsf{Cat2}\big(\mathscr{A}(A),\mathscr{A}(B)\big)$ for any two objects $A,B \in \mathsf{Obj}_\mathsf{Cat1}$ such that
	\begin{enumerate}
		\item {the composition is preserved, that is for all objects $A,B,C \in \mathsf{Obj}_\mathsf{Cat1}$ and for any morphisms $\psi \in \mathsf{Mor}_\mathsf{Cat1}(A,B)$ and $\phi \in \mathsf{Mor}_\mathsf{Cat1}(B,C)$ it holds
		\begin{align}
			\mathscr{A}(\phi \comp \psi) = \mathscr{A}(\phi) \comp \mathscr{A}(\psi) \formspace,
		\end{align}}
		\item{
			$\mathscr{A}$ maps identities to identities, that is for any object $A \in \mathsf{Obj}_\mathsf{Cat1}$ it holds
			\begin{align}
				\mathscr{A}(\text{id}_\mathsf{A}) = \text{id}_{\mathscr{A}(A)} \formspace.
			\end{align}
			}
	\end{enumerate}
\end{definition}
%
%
%
%
%
%
%
%
%
%
%
%
%%%%%%
%%SIGN CONVENTIONS
%%%%%%
%
%
\subsection{Sign conventions}\label{sec:sign_conventions}
At certain points throughout this chapter we have had a freedom of choice regarding the signs of some entities, in particular the sign of the signature of the Lorentzian metric $g$ and that of the interior derivative $\delta$. Though at this stage the choice can be made arbitrarily, we want to make it in a way that in the end allows us to make certain physical interpretations on some parameters. More precisely, we want to interpret the parameter $m$ of the Klein-Gordon equation\footnote{or its generalization on $p$-forms} $(\square + m^2) f = 0$ for a zero-form $f \in \Omega^0(\M)$ as a mass in the physical sense. With the chosen sign convention for $\delta$ we find, using ${\delta}f = 0$:
\begin{align}
	\square f
	&= (\delta d + d \delta) f \notag\\
	&= \delta d f \notag\\
	&= - \nabla^\mu \nabla_\mu f \formspace.
\end{align}
In the following heuristic (local) argument we see
\begin{align}
	\square + m^2
	&= -\nabla^\mu \nabla_\mu + m^2 \notag\\
	&\sim \partial_t^2 + \sum_i \partial_i^2 + m^2\notag\\
	&\sim -E^2 + \abs{\vector{p}}^2 + m^2
\end{align}
which yields the correct relativistic relation of energy, momentum and mass according to $E^2 = \abs{\vector{p}}^2 + m^2$.
A similar calculation holds for the Klein-Gordon operator generalized to act on one-forms. If we had found a ``wrong'' relation between energy, momentum and mass, we would have had to adapt the chosen signs. Usually one chooses the sign of the metric and the interior derivative such that they are in some sense mathematically convenient (although one might disagree with another one's choice). We have made the choice of the metric, such that the Cauchy surfaces become Riemannian rather that ``anti-Riemannian'' (with an all minus signature), which seems more natural to some. Also, a lot of the used references on spacetime geometry (in particular the book by \name{Wald} \cite{wald_GR}) use this sign convention, which makes the application of certain formulas easier. As mentioned, the sign of the interior derivative was chosen such that it is formally adjoint to the exterior derivative (with respect the specified inner product) on all Lorentzian and Riemannian manifolds. It seemed convenient for the actual calculations to fix the sign regardless of the signature of the metric of the underlying manifold. One could equivalently have fixed the opposite sign, yielding the two derivatives to be skew-adjoint, which is also done in the literature. However, in the end, one has one freedom left to make the energy-momentum-mass relation work: that is the sign in front of the mass in the Klein-Gordon equation and all other wave equations accordingly. Hence, one regularly also finds the Klein-Gordon equation to be defined with a flipped sign of the mass term. But for our case, we want the mass $m$ in any wave equation to appear with a positive sign.
%
%

\section{Additive Agents\protect\footnote{We have created a website at \href{https://www.cs.umd.edu/\~saeedrez/fair.html}{https://www.cs.umd.edu/$\sim$saeedrez/fair.html} for the implemented algorithm and all related materials.}}}\label{additive}
In this section we study the fair allocation problem in the additive setting. We present a proof to the existence of a $3/4$-$\MMS$ allocation when the agents are additive. This improves upon the work of \procacciafirst ~\cite{Procaccia:first} wherein the authors prove a $2/3$-$\MMS$ allocation exists for any combination of additive agents.  As we show, our proof is constructive; given an algorithm that determines the $\MMS$ of an additive set function within a factor $\alpha$, we can implement an algorithm that finds a $3/(4 \alpha)$-$\MMS$ allocation in polynomial time. This married with the PTAS algorithm of \epsteinefficient ~\cite{epstein2014efficient} for finding the $\MMS$ values, results is an algorithm that finds a $3/(4+\epsilon)$-$\MMS$ allocation in polynomial time.

The main idea behind the $3/4$-$\MMS$ allocation is \textit{clustering} the agents. Roughly speaking, we categorize the agents into three clusters, namely $\cone$, $\ctwo$, and $\cthree$. We show that the valuation functions of the agents within each cluster show similar behaviors. Along the clustering process, we allocate the heavy items (the items that have a valuation of at least $1/4$ to some agents) to the agents. By Observation \ref{reducibility}, proving a $3/4$-$\MMS$ guarantee can be narrowed down to only $3/4$-irreducible instances. The $3/4$-irreducibility of the problem guarantees that after the clustering process, the remaining items are light. This enables us to run a $\bagfilling$ process to satisfy the agents. In order to prove the correctness of the algorithm, we take advantage of the properties of each cluster separately.

The organization of this section is summarized in the following: we start by a brief and abstract explanation of the ideas in Section \ref{overview}. In Section \ref{additive:observations} we study the properties of the additive setting and state the main observations that later imply the correctness of our algorithm. Next, in Section \ref{additive:clusters} we discuss a method for clustering the agents and in Section \ref{additive:allocation} we show how we allocate the items to the agents of each cluster to ensure a $3/4$-$\MMS$ guarantee. Finally, in Section \ref{additive:algorithm} we explain the implementation details and prove a polynomial running time for the proposed algorithm.

Throughout this section, we assume $\MMS_i = 1$ for all agents $\agent_i \in \agents$. This is without loss of generality for the existential proof since one can scale the valuation functions to impose this constraint. However, the computational complexity of the allocation will be affected by this assumption since determining the $\MMS$ of an additive function is NP-hard~\cite{epstein2014efficient}. That said, we show in Section \ref{additive:algorithm} that this challenge can be overcome by incurring an additional $1+\epsilon$ factor to the approximation guarantee.  

For brevity, we defer the proofs of Sections \ref{additive:observations}, \ref{additive:clusters}, \ref{additive:allocation}, and \ref{additiveproofs} to Appendices \ref{additiveobservationsproof},\ref{clusteringappendix},\ref{clustering2appendix}, and \ref{additiveproofappendix}, respectively.
\section{System Overview} \label{sec:overview}

In this section, we give an overview of our keyword search system. Our approach can be summarized as a two-phase framework, as illustrated in Figure \ref{fig:overview}. 

\begin{figure} [h]
	\centering
	\scalebox{1.0}[1.0]
	{
		\resizebox{\linewidth}{!}
		{
			\includegraphics[scale=0.5]{visio_pics/approach_overview_simp.pdf}
		}
	}
	\vspace{-0.3in}
	\caption{An Overview of Our Approach.}
	\label{fig:overview}
	\vspace{-0.2in}
\end{figure}

\subsection{Phase-I: Segmentation and Annotation} \label{sec:phase1}
The first phase is to segment the keyword token sequence $RQ = \{k_1, k_2, ..., k_m\}$ into several \emph{terms} and each term is annotated with one of the three characters $\{$entity, class, relation$\}$. The converted query is called \emph{annotated query}. Formally, we denote an \textit{annotated query} as $AQ= \{t_1:c_1, t_2:c_2, ..., t_l:c_l\}$, where each $t_i$ is a term and  $c_i \in \{$entity, class, relation$\}$. Note that the first phase (i.e., segmentation and annotation) is not the focus of this paper, as it has been studied extensively \cite{hua2015short,han2011generative, li2011faerie, nakashole2012patty,cai2013large}. We briefly describe the implementation of the first phase as follows. 

For each continuous subsequence $s$ in $RQ$, we check whether it could be matched to an entity, a class, or a relation of the RDF dataset, by employing the existing techniques of entity linking \cite{han2011generative,li2011faerie, ratinov2011local} and relation paraphrasing \cite{nakashole2012patty,cai2013large}. If $s$ is matched, we regard $s$ as a \emph{candidate term} $t_i$, and annotate it with the corresponding character (entity, class, or relation).
We may find that two candidate terms $t_i$ and $t_j$ \emph{overlap} with each other. We say $t_i$ overlaps with $t_j$ if and only if they have at least one common token. Obviously, if two terms overlap, they cannot occur at the same segmentation result. For example, ``university'' and ``university locate USA'' cannot occur in the same segmentation result. We build a \emph{candidate term graph} to describe the mutually exclusive relations: (1) each candidate term $t_i$ is represented a vertex; (2) there is an edge between $t_i$ and $t_j$ if and only of there is \emph{no} overlapping tokens between $t_i$ and $t_j$. Thus each maximal clique in the candidate term graph stands for a possible segmentation result. To obtain top-$N$ best $AQ$, we employ the maximal clique algorithm \cite{bron1973algorithm}, and adopt the pairwise metrics in \cite{hua2015short} to rank the segmentation result.
In out example, we get the top-2 $AQ$ as shown in Figure \ref{fig:overview}.

In the first phase, we have converted keyword token sequence $RQ$ into top-$N$ $AQ$ by some off the shelf techniques. Furthermore, these terms in $AQ$ have been matched to some elementary query graph building blocks (i.e., entity/class vertices and predicate edges). Specifically, if a term $t_i$ is annotated with ``entity'' or ``class'', it will be matched to candidate entity/class vertices in RDF graph.
%We do not distinguish entity vertex or class vertex until generating SPARQL, since we have the same operation on both of them.
If a term $t_i$ is annotated with ``relation'', it will be matched to a set of candidate predicates.

\vspace{-0.05in}
\begin{example}\label{example:aq}
	Given a keyword token sequence $RQ = \{$scientist, graduate, from, university, locate, USA$\}$, we obtain the annotated query $AQ=\{$``scientist'':class, ``graduate from'':relation, ``university'':class, ``locate'':relation, ``USA'':entity $\}$, where ``scientist'' is matched to $\{$dbo:Scientist$\}$, ``university'' is matched to $\{$dbo:University$\}$, ``USA'' is matched to two possible entities $\{$res:USA$\_$Today, res:United$\_$\\States$\}$ due to the ambiguity. Also, the relations ``graduate from'' and ``locate'' also match to two candidate predicates $\{$dbo:almaMater, dbo:education$\}$ and $\{$dbo:country, dbo:location$\}$
\end{example}

\vspace{-0.1in}
\subsection{Phase-II: Query Graph Assembly}
In the second phase, we concentrate on how to assemble a query graph $Q$ based on these elementary building blocks. Formally, the query graph assembly problem is defined as follows:

\vspace{-0.05in}
\begin{definition} \textbf{(Query Graph Assembly Problem)}\label{def:querygraphassembling}
	Given $n$ terms $t^{v}_i$ ($i=1,...,n$) annotated with ``entity'' or ``class'', and $m$ terms $t_j^{e}$ ($j=1,...,m$) annotated with ``relation'', each term $t^{v}_i$ is matched to a set $V_i$ of candidate entity/class vertices and each $t^{e}_j$ is matched to a set $E_j$ of candidate predicate edges. Let $\Upsilon=\{V_1, ..., V_n\}$ and $\Gamma=\{E_1, ..., E_m\}$. A valid assembly query graph is denoted as $Q (V_Q,E_Q)$, which satisfies the following constraints:
	\begin{enumerate}
		\item $|V_Q|=n$, and $\forall V_i \in \Upsilon, V_Q \cap V_i \not= \phi$; 
		\emph{/*each entity or class vertex set $V_i$ has exactly one vertex in  $Q$*/}
		\item $|E_Q|=m$, and $\forall E_j \in \Gamma, E_Q \cap E_j \not= \phi$. 
		\emph{/*each predicate edge set $E_j$ has exactly one edge in $Q$*/}
	\end{enumerate}
	Each edge $e(\langle v_1, v_2\rangle,p) \in E_Q$ connects a pair of vertices $\langle v_1, v_2\rangle \in V_Q$ by a predicate $p$.
	The assembly cost of $Q$ is
	\begin{equation} \label{equ:cost}
	cost(Q)=\sum_{e(\langle v_1, v_2\rangle,p) \in E_Q}{w(\langle v_1, v_2\rangle,p)}
	\end{equation}
	where  $w(\langle v_1, v_2\rangle,p)$ denotes the triple assembly cost. 
	
	
	The \emph{query graph assembly} (QGA for short) problem is to construct a valid graph $Q$ with the minimum assembly cost. 
\end{definition} 

%\vspace{-0.06in}

There are two aspects that should be explained for QGA.

\textbf{\emph{1. Constraints:}}
The two constraints in Definition \ref{def:querygraphassembling} mean that each term $t^{v}_i$ ($1\leq i \leq n$) only corresponds to a single entity/class vertex in $Q$. For example, although ``USA'' may match two candidate entities dbo:USA$\_$Today and dbo:United$\_$States, in the final query graph $Q$, ``USA'' only matches a single entity (dbo:United$\_$States). It is analogue for the relation term $t_j^{e}$ ($1\leq j \leq m$). 

\textbf{\emph{2. Disengaged Edges:}}
A \emph{predicate edge} $e(\langle \cdot,\cdot \rangle,p)$ (in $E_j$) does not have two fixed endpoints but its edge label is fixed to predicate $p$. Thus, a predicate edge can be also called a \emph{disengaged} edge. The triple assembly cost $w(\langle v_1, v_2\rangle,p)$ measures the goodness of assembling $\langle v_1, v_2\rangle$ and $p$ into an edge in $Q$. Then the goal of the QGA problem is to determine the endpoints of $e(\langle \cdot,\cdot \rangle,p)$ to minimize the overall $cost(Q)$.

After finding the optimal $Q$ with minimum $cost(Q)$, we can translate it to SPARQL statements naturally, as illustrated in Figure \ref{fig:overview}.

\subsection{Graph Embedding Cost Model}
Note that the triple assembly cost $w(\langle v_1, v_2\rangle,p)$ can be any positive cost function, which does not affect the hardness of QGA. In other words, the QGA problem is a general computing framework to interpret the input keywords as SPARQL, which does not depend on any specific triple assembly cost function.

\begin{figure} [b]
	\centering
	\scalebox{0.55} [0.50]
	{
		\resizebox{\linewidth}{!}
		{
			\includegraphics[scale=1.0]{visio_pics/transe_visualizing.pdf}
		}
	}
	\caption{Visualizing the Intuition of Graph Embedding.}
	\label{fig:transe_visualizing}
	\vspace{-0.2in}
\end{figure}

The only thing affected by the selection of triple assembly cost function is the system's accuracy. A good cost function can guide to assemble correct query graph $Q$ that implies users' query intention. The process of assembling $\langle v_1, v_2\rangle$ and $p$ into a triple is analogue to ``link prediction'' problem in the RDF knowledge graph \cite{miller2009nonparametric}. Given two entity/class vertices $v_1$ and $v_2$, the link prediction is to ``predict'' the predicate $p$ between $v_1$ and $v_2$, and $w(\langle v_1,v_2\rangle, p)$ is a \emph{measure} of the prediction. Recent research show that the graph embedding technique is superior to other traditional approaches, such as \cite{miller2009nonparametric,nickel2011three,jenatton2012latent}. In the graph embedding model, all subjects (s), objects (o) and predicates (p) are encoded as multi-dimensional vectors $\overrightarrow{s}$, $\overrightarrow{p}$ and $\overrightarrow{o}$ such that $\overrightarrow s  + \overrightarrow p  \approx \overrightarrow o $ if $\langle s,p,o \rangle \in G$ (i.e., $\langle s,p,o\rangle$ is a triple in RDF graph); while $\overrightarrow s  + \overrightarrow p$ should be far away from $\overrightarrow o$ otherwise. Figure \ref{fig:transe_visualizing} visualizes the intuition. From the intuition, the structural information among entities, classes and relations in RDF graph is embedded into vectors. Therefore, we define the triple assembly cost based on graph embedding vectors as follows.


\begin{definition}\textbf{ (Triple Assembly Cost) }. \label{def:tripleassemblycost}
	Given two entity/class vertices $v_1$ and $v_2$ and a predicate edge $p$, the \emph{cost} of assembly triple $(v_1,p,v_2)$ is denoted as follows:
	\begin{equation}\label{equ:tripleassembly}
	w(\langle v_1, v_2\rangle, p) = MIN(| \overrightarrow{v_1} + \overrightarrow p  - \overrightarrow {v_2 } |, | \overrightarrow{v_2} + \overrightarrow p  - \overrightarrow {v_1 } |)
	\end{equation}
	where $\overrightarrow{v_1}$, $\overrightarrow {v_2}$ and $\overrightarrow {p}$ are the encoded multi-dimensional vectors  of $v_1$, $v_2$ and $p$, respectively. 
	%where $|x|_+$ denotes the positive part of $x$. 
\end{definition}  

\begin{figure} [h]
	\centering
	\scalebox{1.0}
	{
		\resizebox{\linewidth}{!}
		{
			\includegraphics[scale=1.0]{visio_pics/query_graph_elements_example2.pdf}
		}
	}
%	\caption{Candidate Entity/Class Vertices and Predicate Edges.}
	\caption{Elementary Query Graph Building Blocks.}
	\label{fig:graph_elements_exp2}
	\vspace{-0.3in}
\end{figure}

\begin{figure} [h]
	\newcommand{\mywidth}{0.23\textwidth}
	\centering
	\begin{subfigure}[t]{\mywidth}
		\centering
		\resizebox{\linewidth}{!}
		{
			\includegraphics{visio_pics/query_assembly_graph_q1.pdf}
			
		}
		\caption{$cost(Q_1)=1.76$}
		\label{fig:assembly_query_graph_q1}
	\end{subfigure}
	\begin{subfigure}[t]{\mywidth}
		\centering
		\resizebox{1.0\linewidth}{!}
		{
			\includegraphics{visio_pics/query_assembly_graph_q2.pdf}
		}
		\caption[font=\small]{$cost(Q_2)=2.46$}
		%        \vspace{0.1in}
		\label{fig:assembly_query_graph_q2}
	\end{subfigure}
	\caption{Possible Assembly Query Graphs.}
	%    \vspace{-0.1in}
	\label{fig:assembly_query_graph}
	\vspace{-0.1in}
\end{figure}

\begin{example} In our example, there are three entity/class terms ``scientist'', ``university'', ``USA'' and two relation terms ``graduate from''  and ``locate''. Their corresponding entity/class vertices and predicate edges are shown in Figure \ref{fig:graph_elements_exp2}. There are two different assembly query graph $Q_1$ and $Q_2$ in Figure \ref{fig:assembly_query_graph}, among which $cost(Q_1)<cost(Q_2)$. Thus, the QGA problem result is $Q_1$ (Figure \ref{fig:assembly_query_graph_q1}).
%It means that we interpret the keywords as a query graph $Q$ and evaluate $Q$ using SPARQL query engine to return answers to users. 
\end{example}
\subsection{General Definitions and Observations}\label{additive:observations}
Throughout this section we explore the properties of the fair allocation problem with additive agents.
\subsubsection{Consequences of Irreducibility}
 Since the objective is to prove the existence of a $3/4$-$\MMS$ allocation, by Observation \ref{reducibility}, it only suffices to show every $3/4$-irreducible instance of the problem admits a $3/4$-$\MMS$ allocation. Therefore, in this section we provide several properties of the $3/4$-irreducible instances. We say a set $S$ of items \textit{satisfies} an agent $\agent_i$ if and only if $\valu_i(S) \geq 3/4$. Perhaps the most important consequence of irreducibility is a bound on the valuation of the agents for every item. In the following we show if the problem is $3/4$-irreducible, then no agent has a value of $3/4$ or more for an item. 

\begin{lemma}\label{remove1} 
For every $\alpha$-irreducible instance of the problem we have 
$$\forall \agent_i \in \agents, \ite_j \in \items \hspace{1cm} \valu_i(\ite_j) < \alpha.$$
\end{lemma}

In other words, Lemma \ref{remove1} states that in a $3/4$-irreducible instance of the problem, no item alone can satisfy an agent. 

It is worth mentioning that the proof for Lemma \ref{remove1} does not rely on additivity of the valuation functions and holds as long as the valuations are monotone. Thus, regardless of the type of the valuation functions, one can assume that in any $\alpha$-irreducible instance, value of any item is less than $\alpha$ for any agent. Hence the statement carries over to the submodular, XOS, and subadditive settings. 



As a natural generalization of Lemma \ref{remove1}, we show a similar observation for every pair of items. However, this involves an additional constraint on the valuation of the other agents for the pertinent items. In contrast to Lemma \ref{remove1}, Lemmas \ref{remove2} and \ref{remove3} are restricted to additive setting and their results  do not hold in more general settings.


\begin{lemma}
\label{remove2}
If the problem is $3/4$-irreducible and
$\valu_i(\{\ite_j,\ite_k\}) \geq 3/4$
holds for an agent $\agent_i \in \agents$ and items $\ite_j, \ite_k \in \items$, then there exists an agent $\agent_{i'} \neq \agent_i$ such that
$$\valu_{i'}(\{\ite_j,\ite_k\}) > 1$$
\end{lemma}

According to Lemma \ref{remove2}, in every $3/4$-irreducible instance of the problem, for every agent $\agent_i$ and items $\ite_j,\ite_k$, either $\valu_i(\{\ite_j,\ite_k\}) < 3/4$ or there exists another agent $\agent_{i'} \neq \agent_i$, such that $\valu_{i'}(\{\ite_j,\ite_k\}) > 1$. Otherwise, we can reduce the problem and find a $3/4$-$\MMS$ allocation recursively. More generally, let $S = \{\ite_{j_1},\ite_{j_2},\ldots,\ite_{j_{|S|}}\}$ be a set of items in $\cal{M}$ and $T =\{\agent_{i_1},\agent_{i_2},\ldots,\agent_{i_{|T|}}\}$ be a set of agents such that
\begin{description}
 \item (i) $|S| = 2|T|$
 \item (ii) For every $\agent_{i_a} \in T$ we have $\valu_{i_a}(\{\ite_{j_{2a-1}},\ite_{j_{2a}}\}) \geq 3/4$.
 \item (iii) For every $\agent_{i} \notin T$ we have $\valu_{i}(\{\ite_{j_{2a-1}},\ite_{j_{2a}}\}) \leq 1$ for every $1 \leq a \leq |T|$.
\end{description}
then the problem is $3/4$-reducible.
\begin{lemma}\label{remove3}
In every $3/4$-irreducible instance of the problem, for every set $T =\{\agent_{i_1},\agent_{i_2},\ldots,\agent_{i_{|T|}}\}$ of agents and set $S = \{\ite_{j_1},\ite_{j_2},\ldots,\ite_{j_{|S|}}\}$ of items at least one of the above conditions is violated.
\end{lemma}

% Note that by applying Lemma \ref{remove2} $k$ times, there is no decrease in the maxmin share of the agents in ${\cal{N}}\setminus P$ for the rest of items.  
\subsubsection{Modeling the Problem with Bipartite Graphs}\label{MtPwBG}
In our algorithm we subsequently make use of classic algorithms for bipartite graphs. Let $G = \langle V(G),E(G)\rangle$ be a graph representing the agents and the items. Moreover, let $V(G) = \itemsv \cup \agentsv$ where $\agentsv$ corresponds to the agents and $\itemsv$ corresponds to the items. More precisely, for every agent $\agent_i$ we have a vertex $\agentv_i \in \agentsv$ and every item $\ite_j$ corresponds to a vertex $\itemv_j \in \itemsv$. For every pair of vertices $\agentv_i \in \agentsv$ and $\itemv_j \in \itemsv$, there exists an edge $(\itemv_j,\agentv_i) \in E(G)$ with weight $w(\itemv_j,\agentv_i) = \valu_i(\{\ite_j\})$. We refer to this graph as \emph{the value graph}.

We define an operation on the weighted graphs which we call \textit{filtering}. Roughly speaking, a filtering is an operation that receives a weighted graph as input and removes all of the edges with weight less than a threshold from the graph. Next, we remove all of the isolated\footnote{A vertex is called isolated if no edge is incident to that vertex.} vertices and report the remaining as the filtered graph. In the following we formally define the notion of filtering for weighted graphs.
\begin{definition}
A $\beta$-filtering of a weighted graph $H\langle V(H),E(H)\rangle$, denoted by $H_{\beta}\langle V_\beta(H),E_\beta(H)\rangle$, is a subgraph of $H$ where $V_\beta(H)$ is the set of all vertices in $V(H)$ incident to at least one edge of weight $\beta$ or more and 
$$E_\beta(H) = \{(u,v) \in E(H)| w(u,v) \geq \beta\}.$$ 
\end{definition}
For the case of the value graph, we also denote by $\agentsv_\beta$ and $\itemsv_\beta$ the sets of agents and items corresponding to vertices of $V_\beta(G)$.
\begin{figure}[t!]
    \centering
    \includegraphics[scale=0.8]{figs/filtering}
    \caption{An example of $\beta$-filtering on a graph. After removing the edges with a value smaller than $\beta$, some vertices may become isolated. All such vertices are removed from the filtered graph.}
    \label{fig:filtering}
\end{figure}
%\begin{example}
Figure \ref{fig:filtering} illustrates an example of a graph $H$, together with $H_{0.4}$ and $H_{0.5}$. Note that none of the vertices in $H_{0.4}$ or $H_{0.5}$ are isolated. 
% 
%\end{example}
%%%%%%%%%%%%%%%%

Denote by a maximum matching, a matching that has the highest number of edges in a graph. In definition \ref{FG}, we introduce our main tool for clustering the agents. 
\begin{definition}
\label{FG}
Let $H\langle V(H),E(H)\rangle$ be a bipartite graph with $V(H) = \partone \cup \parttwo$ and let $M$ be a maximum matching of $H$. Define $\parttwo_1$ as the set of the vertices in $\parttwo$ that are not saturated by $M$. Also, define $\parttwo_2$ as the set of vertices in $\parttwo$ that are connected to $\parttwo_1$ by an alternating path and let $\partone_2 = M(\parttwo_2)$, where  $M(\parttwo_2)$ is the set of vertices in $\partone$ that are matched with the vertices of $\parttwo_2$ in $M$. We define $F_{H}(M,\partone)$ as the set of the vertices in $\partone \setminus \partone_2$. 
\end{definition}

For a better understanding of Definition \ref{FG}, consider Figure \ref{fig:FG}. By the definition of alternating paths, there is no edge between the saturated vertices of $F_H(M,\partone)$ and $ \parttwo_1 \cup \parttwo_2$. On the other hand, since $M$ is maximum, the graph doesn't have any augmenting path. Thus, there is no edge between unsaturated vertices in $F_H(M,\partone)$ and $ \parttwo_1 \cup \parttwo_2$. As a result, there is no edge between $F_H(M,\partone)$ and $ \parttwo_1 \cup \parttwo_2$. Furthermore, $F_H(M,\partone)$ has another important property: there exists a matching from $N(F_H(M,\partone))$ to $F_H(M,\partone)$, that saturates all the vertices in $N(F_H(M,\partone))$, where $N(F_H(M,\partone))$ is the set of neighbors of  $F_H(M,\partone)$.

\begin{figure}
\centering
\includegraphics[scale=0.8]{figs/matching}
\caption{Definition of $F_H$}
\label{fig:FG}
\end{figure}

In Lemmas \ref{iff} and \ref{rem}, we prove two remarkable properties for bipartite graphs. As a consequence of these two lemmas, Corollary \ref{remcol} holds for every bipartite graph. We leverage the result of Corollary \ref{remcol} in the clustering phase.
 
\begin{lemma}
\label{rem}
Let $H(V,E)$ be a bipartite graph with $V = \partone \cup \parttwo$ and let $M$ be a maximum matching of $H$. Then, for every set $T \subseteq \partone \setminus F_H(M,\partone)$ we have $|N(T)| > |T|$, where $N(T)$ is the set of neighbors of $T$. 
\end{lemma}

\begin{lemma}
\label{iff}
For a bipartite graph $H(V,E)$ with $V = \partone \cup \parttwo$, $F_H(M,\partone) = \emptyset$ holds, if and only if
for all $T \subseteq \partone$ we have  $|N(T)| > |T|$, where $N(T)$ is the set of neighbors of $T$.

\end{lemma}

\begin{corollary}[of Lemmas \ref{iff} and \ref{rem}]
\label{remcol}
Let $H(V,E)$ be a bipartite graph with $V = \partone \cup \parttwo$ and let $M$ be a maximum matching of $H$. 
Furthermore, let  $H'(V',E')$ be the induced sub-graph of $H$, with $V' = \partone' \cup \parttwo'$, where $\partone' = \partone \setminus F_H(M,\partone)$ and $\parttwo' = \parttwo \setminus N(F_H(M,\partone))$. Then, for any maximum matching $M'$ of $H'$, $F_{H'}(M',\partone') = \emptyset$ holds. 
\end{corollary}
%%%%%%%%%%%%%%%%
\subsubsection{Cycle-envy-freeness and $\MCMWM$}\label{additive:cef}

In the algorithm, we satisfy each agent in two steps. More precisely, we allocate each agent two sets of items that are together of worth at least $3/4$ to him. We denote the first set of items allocated to agent $\agent_i$ by $\firstset_i$ and the second set by $\secondset_i$. Moreover, we attribute the agents with labels \textit{satisfied}, \textit{unsatisfied}, and \textit{semi-satisfied} in the following way:
\begin{enumerate}
	\item An agent $\agent_i$ is satisfied if $\valu_i(\firstset_i \cup \secondset_i) \geq 3/4$.
	\item An agent $\agent_i$ is semi-satisfied if $\firstset_i \neq \emptyset$ but $\secondset_i = \emptyset$. In this case we define $\epsilon_i = 3/4-\valu_i(\firstset_i)$.
	\item An agent $\agent_i$ is unsatisfied if $\firstset_i = \secondset_i = \emptyset$.
\end{enumerate}
As we see, the algorithm maintains the property that for every semi-satisfied agent $\agent_i$, $\valu_i(\firstset_i) \geq 1/2$ holds and hence, $\epsilon_i < 1/4$. 

To capture the competition between different agents, we define an attribution for an ordered pair of agents. We say a semi-satisfied agent envies another semi-satisfied agent, if he prefers to switch sets with the other agent. 

\begin{definition}
\label{winloose}
Let $T$ be a set of semi-satisfied agents. An agent $\agent_i \in T$ envies an agent $\agent_j \in T$, if $\valu_i(\firstset_j) \geq \valu_i(\firstset_i)$. Also, we call an agent $ \agent_i \in T$ a winner of $T$, if $\agent_i$ envies no other agent in $T$. Similarly, we call an agent $\agent_i$ a loser of $T$, if no other agent in $T$ envies $\agent_i$.
\end{definition}

Note that it could be the case that an agent $\agent_i$ is both a loser and a winner of a set $T$ of agents. Based on Definition \ref{winloose}, we next define the notion of \textit{cycle-envy-freeness}.

\begin{definition}
 We call a set $T$ of semi-satisfied agents cycle-envy-free, if every non-empty subset of $T$ contains at least one winner and one loser. 
\end{definition}

Let $C$ be a cycle-envy-free set of semi-satisfied agents. Define the representation graph of $C$ as a digraph $G_C(V(G_C),\overrightarrow{E}(G_C))$, such that for any agent $\agent_i \in C$, there is a vertex $v_i$ in $V(G_C)$ and there is a directed edge from $v_i$ to $v_j$ in $\overrightarrow{E}(G_C)$, if $\agent_i$ envies $\agent_j$. In Lemma \ref{dag}, we show that $G_C$ is acyclic.
\begin{lemma}
\label{dag}
For every cycle-envy-free set of semi-satisfied agents $C$, $G_C$ is a DAG. 
\end{lemma}
\begin{definition}
 A topological ordering of a cycle-envy-free set $C$ of semi-satisfied agents, is a total order $\prec_O$ corresponding to the topological ordering of the representation graph $G_C$. More formally, for the agents $\agent_i,\agent_j \in C$ we have $\agent_i \prec_O \agent_j$ if and only if $v_i$ appears before $v_j$, in the topological ordering of $G_C$.  
\end{definition}

Note that in the topological ordering of a cycle-envy-free set $C$ of semi-satisfied agents, if $\agent_i \in C$ envies $\agent_j \in C$, then $\agent_i \prec_O \agent_j$. 


\begin{observation}
\label{epsofcluster}
Let $C$ be a cycle-envy-free set of semi-satisfied agents. Then, for every agent $\agent_i \in C$ such that $\agent_j \prec_O \agent_i$, we have:
$$\valu_i(\firstset_j) \leq 3/4 - \epsilon_i.$$ 

\end{observation}

We define a maximum cardinality maximum weighted matching of a weighted graph as a matching that has the highest number of edges and among them the one that has the highest total sum of edge weights. For brevity we call such a matching an $\MCMWM$. In Lemma \ref{wm}, we show that an $\MCMWM$ of a weighted bipartite graph has certain properties that makes it useful for building cycle-envy-free clusters. 

\begin{lemma}
\label{wm}
Let $H\langle V(H),E(H)\rangle$ be a weighted bipartite graph with $V(H) = \partone \cup \parttwo$ and let $M = \{(\vone_1,\vtwo_1),...,(\vone_k,\vtwo_k)\}$ be an $\MCMWM$ of $H$. Then, for every subset $T \subseteq \{\vtwo_1,\vtwo_2, \ldots,\vtwo_k\}$, the following conditions hold:

\begin{minipage}[t]{\linegoal}
\begin{enumerate}[leftmargin=*]
 \item There exists a vertex $\vtwo_j \in T$ which is a winner in $T$, i.e.,  $w(\vone_j,\vtwo_{j}) \geq w(\vone_i,\vtwo_j)$, for all $\vone_i \in M(T)$ and  $(\vone_i,\vtwo_j) \in E(H)$. 
\item There exists a vertex $ \vtwo_j \in T$ which is a loser in $T$, i.e.,  $w(\vone_i,\vtwo_i)  \geq w(\vone_j,\vtwo_i) $, for all $\vtwo_i \in T$ and $(\vone_j,\vtwo_i) \in E(H)$.
\item For any vertex $\vtwo_i \in T$ and any unsaturated vertex $\vone_j \in \partone$ such that $(\vone_j,\vtwo_i) \in E(H)$, $w(\vone_i,\vtwo_i) \geq w(\vone_j,\vtwo_i)$. 
\end{enumerate}
\end{minipage}
\\[6pt]
where $M(T)$ is the set of vertices which are matched by the vertices of $T$ in $M$.
\end{lemma}


Notice the similarities of the first and the second conditions of Lemma \ref{wm} with the conditions of the winner and loser in 
 Definition \ref{winloose}. In Section \ref{additive:clusters}, we assign items to the agents based on an $\MCMWM$ of the value-graph. Lemma \ref{wm} ensures that such an assignment results in a cycle-envy-free set of semi-satisfied agents.   

%\begin{definition}
%\label{mp}
%For a set $M$ of items and a set $P$ of semi-satisfied agents, we define $M_P$ as the set of all items in $M$ with 
%the property that at least one agent in $P$ can be satisfied by this item. 
%More formally, $M_P$ contains all the items %$\ite_j \in M$, such that:

%$$\exists \agent_i \in P \mbox{   }s.t \mbox{   %}\valu_i(\firstset_i \cup \{\ite_j\}) \geq 3/4$$
%\end{definition}


%subsection{Definitions for Section 3}
%\begin{definition}
%For a bundle $Q$ of items that satisfies $\agent_i$, the core of $Q$ with respect to agent $\agent_i$, denoted as $C_i(Q)$ is defined as follows: let $\ite_1,\ite_2,..,\ite_k$ be the items of $Q$ in the increasing order of their values for $\agent_i$. Then $C_i(Q) = \{\ite_j,\ite_{j+1},...,\ite_{k}\}$ , where $j$ is the highest index, such that the bundle with items $\{\ite_j,\ite_{j+1},...,\ite_k\}$ satisfies $\agent_i$.
%\end{definition}
 


%\begin{definition}
%\label{satisfaction-graph}
%Given a set $P = \{\agent_1, \agent_2, \ldots, \agent_n\}$ of the semi-satisfied agents and a set $S = \{b_1,b_2,\ldots,b_m\}$, whose elements are bundles of items. The satisfaction-graph with respect to $S$ and $P$, is a bipartite graph denoted by $G^-(\agentsv,\itemsv)$. For each bundle $b_i$ in $S$, there is a related vertex $\agentv_i$ in $\agentsv$ and for each agent $\agent_j$ in $P$, there is a related vertex $\itemv_j$ in $\itemsv$. There is an edge between $\agentv_i$ and $\itemv_j$, if the agent $\agent_j$ can be satisfied by the items in $b_i$ .i.e. $\valu_j( \firstset_j \cup b_i) \geq 3/4$.  
%\end{definition}

\subsection{Phase 1: Building the Clusters}\label{additive:clusters}
In this section, we explain our method for clustering the agents. Intuitively, we divide the agents into three clusters $\cone,\ctwo$ and $\cthree$. As mentioned before, during the algorithm, two sets of items $\firstset_i,\secondset_i$ are allocated to each agent $\agent_i$. Throughout this section, we prove a set of lemmas that are labeled as \emph{value-lemma}. In these lemmas we bound the value of $f_i$ and $g_i$ allocateed to any agent for other agents. A summary of these lemmas is shown in Tables \ref{table0}, \ref{table4} and \ref{table1}. 


After constructing each cluster, we refine that cluster. In the refinement phase of each cluster, we target a certain subset of the remaining items. If any item in this subset could satisfy an agent in the recently created cluster, we allocate that item to the corresponding agent. The goal of the refinement phase is to ensure that the remaining items in the targeted subset are light enough for the agents in that cluster, i.e., none of the remaining items can satisfy an agent in this cluster.

We denote by $\satagents$, the set of satisfied agents. In addition, denote by $\satagents_1, \satagents_2$, and $\satagents_3$ the subsets of $\satagents$, where $\satagents_i$ refers to the agents of $\satagents$ that previously belonged to ${\mathcal C}_i$. Furthermore, we use $\satagents_1^r$ and $\satagents_2^r$ to refer to the agents of $\satagents_1$ and $\satagents_2$ that are satisfied in the refinement phases of $\cone$ and $\ctwo$, respectively.
\subsubsection{Cluster $\cone$} \label{cluster1:building}
Consider the filtering $G_{1/2}\langle V_{1/2}(G),E_{1/2}(G) \rangle$ of the value-graph $G$ and let $M$ be an $\MCMWM$ of $G_{1/2}$. We define Cluster $\cone$ as the set of agents whose corresponding vertex is in $N(F_{G_{1/2}}(M,\itemsv_{1/2}))$. 

For brevity, denote by $V_{\cone}$ the set of vertices in $V(G)$ that correspond to the agents of $\cone$. In other words:
$$V_{\cone} = N(F_{G_{1/2}}(M,\itemsv_{1/2})).$$



 
Also, let $F_{G_{1/2}}(M,\itemsv_{1/2}) $ be $U_1 \cup S_1$, where $U_1$ is the set of unsaturated vertices in $F_{G_{1/2}}(M,\itemsv_{1/2})$ and $S_1$ is the set of the saturated vertices. For each edge $(\itemv_j,\agentv_i) \in M$ such that $\itemv_j \in S_1$, we allocate  item $\ite_j$ to agent $\agent_i$. More precisely, we set $\firstset_i = \{\ite_j \}$. Since $w(\itemv_j,\agentv_i)\geq {1/2}$, we have:
$$\forall \agent_k \in \cone \qquad V_k(f_k) \geq {1/2}.$$
According to the definition of $\epsilon_i$, we have
\begin{equation}
\forall \agent_k \in \cone \qquad \epsilon_k \leq {1/4}.
\end{equation} 

By the definition of $F_{G_{1/2}}$, for every agent which is not in $\cone$, the condition of Lemma \ref{forc2c3} holds. Note that all the agents that are not in $\cone$, belong to either $\ctwo$ or $\cthree$.


\begin{lemma}[value-lemma]
\label{forc2c3}
For all $\agent_i \in \ctwo \cup \cthree$ we have: \[ \forall \agent_j \in \cone \qquad \valu_i(\firstset_j) < 1/2. \]
\end{lemma}

For each vertex $\agentv_i \in V_{\cone}$, denote by $N_{\agentv_i}$ the set of vertices $\itemv_j  \in \itemsv \setminus \itemsv_{1/2}$, where $w(\itemv_j,\agentv_i) \geq \epsilon_i$ and let \[W_1 = U_1 \cup \bigcup_{\agentv_i \in V_{\cone}} N_{\agentv_i}.\]

Note that by definition, for any vertex $\itemv_j \in U_1$ and $\agentv_i \notin V_{\cone}$, there is no edge between $\itemv_j$ and $\agentv_i$ in $G_{1/2}$ and hence $w(\itemv_j,\agentv_i)<1/2$. Also, since the rest of the vertices in $W_1$ are from $\itemsv \setminus \itemsv_{1/2}$, for any vertex $\agentv_i$ and $\itemv_j \in (W_1 \setminus U_1)$, $w(\itemv_j,\agentv_i)<1/2$ holds. Thus, we have the following observation:

\begin{observation}
\label{w1small}
For every item $\ite_j$ with $\itemv_j \in W_1$ and every agent $\agent_i$ with $\agentv_i \notin V_{\cone}$, $\valu_i(\{\ite_j\})<1/2$.
\end{observation}

Now, define $\itemsv'$ and $ \agentsv' $ as follows:

$$\itemsv' = \itemsv \setminus (W_1 \cup S_1),$$ $$\agentsv' = \agentsv \setminus V_{\cone}.$$ 

Let $G'\langle V(G'),E(G')\rangle $ be the induced subgraph of $G$ on $V(G') = \agentsv' \cup \itemsv'$. We use graph $G'$ to build Cluster $\ctwo$. 



\subsubsection{Cluster $\cone$ Refinement} Before building Cluster $\ctwo$, we satisfy some of the agents in $\cone$ with the items corresponding to the vertices of $W_1$. Consider the subgraph $G_1 \langle V(G_1),E(G_1) \rangle$ of $G$ with $V(G_1) = W_1 \cup V_{\cone}$. In $G_1$, There is an edge between $\agentv_i \in V_{\cone}$ and $\itemv_j \in W_1$, if $V_i(\{\ite_j\}) \geq \epsilon_i$. Note that $G_1 \langle V(G_1),E(G_1) \rangle$ is not necessarily an induced subgraph of $G$. We use $G_1$ to satisfy a set of agents in $\cone$. To this end, we first show that $G_1$ admits a special type of matching, described in Lemma \ref{nicematch}.

\begin{lemma}
\label{nicematch}
There exists a matching $M_1$ in $G_1$, that saturates all the vertices of $W_1$ and for any edge $(\itemv_i,\agentv_j) \in M_1$ and any unsaturated vertex $\agentv_k \in N(\itemv_i)$, $\agent_k$ does not envy $\agent_j$. 
\end{lemma}


Let $M_1$ be a matching of $G_1$ with the property described in Lemma \ref{nicematch}. For every edge $(\agentv_i,\itemv_j) \in M_1$, we allocate  item $\ite_j$ to agent $\agent_i$ i.e., we set $\secondset_i = \{\ite_j\}$. By the definition, $\agent_i$ is now satisfied. Thus, we remove $\agent_i$ from $\cone$ and add it to $\cal S$. Note that, after refining $\cone$, none of the items whose corresponding vertex is in $\itemsv' \setminus \itemsv'_{1/2}$ can satisfy any remaining agent in $\cone$. Thus, Observation \ref{fsmallc1} holds.



\begin{observation}
\label{fsmallc1}
For every item $\ite_j$ such that $\itemv_j \in \itemsv'$, either $\itemv_j \in \itemsv'_{1/2}$ or for all $\agent_i \in \cone$, $V_i(\{\ite_j\}) < \epsilon_i$.
\end{observation}

At this point, all the agents of $\satagents$ belong to $\satagents_1^r$. Each one of these agents is satisfied with two items, i.e., for any agent $\agent_j \in \satagents_1^r$, $|\firstset_j| = |\secondset_j| = 1$. In Lemma \ref{gsmallc1r} we give an upper bound on $\valu_i(\secondset_j)$ for every agent $\agent_j \in \satagents_1^r$ and every agent $\agent_i$ in $\ctwo \cup \cthree$.  

\begin{lemma}[value-lemma]
\label{gsmallc1r}
For every agent $\agent_i \in \ctwo \cup \cthree$, we have
$$ \forall \agent_j \in \satagents_1^r \qquad \valu_i(\secondset_j)< 1/2.$$
\end{lemma}

Lemmas \ref{gsmallc1r} and  \ref{forc2c3}  state that for every agent $\agent_i \in \ctwo \cup \cthree$  and every agent $\agent_j \in \satagents_1^r$, $\valu_i(\firstset_j)$ and $\valu_i(\secondset_j)$ are upper bounded by $1/2$. This, together with the fact that $|\firstset_j| = |\secondset_j|=1$, results in Lemma \ref{forc2}.
\begin{lemma}
\label{forc2}
For all $\agent_i \notin \cone$, we have
\[ \MMS_{\valu_i}^{|\agents \setminus \satagents_1^r|} ( {\items} \setminus \bigcup_{\agentv_j \in \satagents_1^r} \firstset_j \cup \secondset_j) \geq 1.\]
\end{lemma}

\subsubsection{Cluster $\ctwo$}
Recall graph $G' \langle V(G') , E(G') \rangle$ as described in the last part of Section \ref{cluster1:building} and let $G'_{1/2}\langle V_{1/2}(G'), E_{1/2}(G') \rangle$ be a $1/2$-filtering of $G'$. Lemma \ref{rem} states that the size of the maximum matching between $\itemsv'_{1/2}$ and $\agentsv'_{1/2}$ is $|\itemsv'_{1/2}|$. Also, according to Corollary \ref{remcol}, for any maximum matching $M'$ of $G'_{1/2}$, $F_{G'_{1/2}}(M',\itemsv'_{1/2})$ is empty. In what follows, we increase the size of the maximum matching in  $G'_{1/2}$ by merging the vertices of $\itemsv' \setminus \itemsv'_{1/2}$ as described in Definition \ref{merge}.

\begin{figure}[t!]
    \centering
    \includegraphics[scale=1]{figs/merge}
    \caption{Merging $\itemv_1$ and $\itemv_2$}
    \label{fig:merge}
\end{figure}


\begin{definition}
\label{merge}
For merging vertices $\itemv_i,\itemv_j$ of $G'(\itemsv',\agentsv')$, we create a new vertex labeled with $\itemv_{i,j}$. Next, we add $\itemv_{i,j}$ to $\itemsv'$ and for every vertex $\agentv_k \in \agentsv'$, we add an edge from $\agentv_k$ to $\itemv_{i,j}$ with weight $w(\agentv_k,\itemv_i) + w(\agentv_k,\itemv_j)$. Finally we remove vertices $\itemv_i$ and $\itemv_j$ from $\itemsv$. See Figure ~\ref{fig:merge}.
\end{definition}

In Lemmas \ref{c1small2} and \ref{pairsmall}, we give upper bounds on the value of the pair of items corresponding to a merged vertex. In Lemma \ref{c1small2}, we show that the value of a merged vertex is less than $2\epsilon_i$ to every agent $\agent_i \in \cone$. This fact is a consequence of Observation \ref{fsmallc1}. Also, in Lemma \ref{pairsmall}, we prove that the value of the items corresponding to a merged vertex is less than $3/4$ to any agent. Lemma \ref{pairsmall} is a direct consequence of $3/4$-irreducibility. In fact, we show that if the condition of Lemma \ref{pairsmall} does not hold, then the problem can be reduced. 

\begin{lemma}
\label{c1small2}
For any agent $\agent_k \in \cone$ and any pair of vertices $\itemv_i, \itemv_j \in \itemsv' \setminus \itemsv'_{1/2}$, $\valu_k(\{\ite_i,\ite_j\}) < 2\epsilon_k$ holds. In particular, total value of the items that belong to a merged vertex is less than $2\epsilon_k$ for $\agent_k$.
\end{lemma}

\begin{lemma}
\label{pairsmall}
 For any pair of vertices $\itemv_i , \itemv_j \in \itemsv' \setminus \itemsv'_{1/2}$ and any vertex $\agentv_k \in \agentsv$, we have $V_k(\{\ite_i,\ite_j\}) < {3/4}$.
\end{lemma}

\begin{corollary} [of Lemma \ref{pairsmall}]
\label{forc2small}
For any agent $\agent_i$ with $\agentv_i \in \agentsv$, there is at most one item $\ite_j$, with $\itemv_j \in \itemsv' \setminus \itemsv'_{1/2}$ and $\valu_i(\{\ite_j\}) \geq {3/8}$.
\end{corollary}

Consider the vertices in $\itemsv' \setminus \itemsv'_{1/2}$. We call a pair $(\itemv_i,\itemv_j)$ of distinct vertices in $\itemsv' \setminus \itemsv'_{1/2}$ \textit{desirable} for $\agentv_k \in \agentsv'$, if $w(\agentv_k,\itemv_i) + w(\agentv_k,\itemv_j) \geq {1/2}$. With this in mind, consider the process described in Algorithm \ref{addvertex}. 

In each step of this process, we find an $\MCMWM$ $M'$ of $G'_{1/2}$. Note that $M'$ changes after each step of the algorithm. Next, we find a pair $(\itemv_i,\itemv_j)$ of the vertices in $\itemsv' \setminus \itemsv'_{1/2}$ that is desirable for at least one agent in $T = \agentsv' \setminus N(F_{G'_{1/2}}(M',\itemsv'_{1/2}))$. If no such pair exists, we terminate the algorithm. Otherwise, we select an arbitrary desirable pair $(\itemv_i,\itemv_j)$ and merge them to obtain a vertex $\itemv_{i,j}$. According to the definition of $T$ in Algorithm \ref{addvertex}, merging a pair  $(\itemv_i,\itemv_j)$ results in an augmenting path in $G'_{1/2}$. Hence, the size of the maximum matching in $G'_{1/2}$ is increased by one. Note that after the termination of Algorithm \ref{addvertex}, either $T = \emptyset$ or  no pair of vertices in $\itemsv' \setminus \itemsv'_{1/2}$ is desirable for any vertex in $T$. 

\begin{lemma} 
\label{sizeeq}
After running Algorithm \ref{addvertex}, we have
$$|F_{G'_{1/2}}(M',\itemsv'_{1/2})| = |N(F_{G'_{1/2}}(M',\itemsv'_{1/2}))|.$$  
\end{lemma}

\begin{algorithm}[t!]
 \KwData{$G'(V(G'),E(G'))$}
 \While{True}{
  $M' = \MCMWM \mbox{  of } G'_{1/2}$\; 
  Find $F_{G'_{1/2}}(M',\itemsv'_{1/2})$\;
  $T = \agentsv' \setminus N(F_{G'_{1/2}}(M',\itemsv'_{1/2}))$\;
   $Q = $ Set of all desirable pairs in $\itemsv' \setminus \itemsv'_{1/2}$ for the agents in $T$\;
  \eIf{$ Q = \emptyset$ }{
   STOP\;
   }{
   Select an arbitrary pair $\itemv_i,\itemv_j$ from $Q$\;
   Merge($\itemv_i,\itemv_j$)\;
  }
 }
 \caption{Merging vertices in $G'$}
 \label{addvertex}
\end{algorithm}

 




We define Cluster $\ctwo$ as the set of agents that correspond to the vertices of $N(F_{G'_{1/2}}(M',\itemsv'_{1/2}))$. Also, denote by $V_{\ctwo}$ the vertices in $N(F_{G'_{1/2}}(M',\itemsv'_{1/2}))$. For each agent $\agent_i \in \ctwo$, we allocate the item corresponding to $M'(\agentv_i)$ (or pair of items in case  $M'(\agentv_i)$ is a merged vertex) to $\agent_i$.


Note that we put the rest of the agents in Cluster $\cthree$. Therefore, Lemma \ref{forc3} holds for all the agents of $\cthree$.

\begin{lemma}[value-lemma]
\label{forc3}
For all $\agent_i \in \cthree$ we have \[ \forall \agent_j \in \ctwo, \valu_i(\firstset_j) < 1/2. \]
\end{lemma}

\subsubsection{Cluster $\ctwo$ Refinement}
The refinement phase of $\ctwo$, is semantically similar to the refinement phase of $\cone$. In the refinement phase of $\ctwo$, we satisfy some of the agents of $\ctwo$ by the items with vertices in $\itemsv' \setminus \itemsv'_{1/2}$. Note that none of the vertices in $\itemsv' \setminus \itemsv'_{1/2}$ is a merged vertex.


The refinement phase of $\ctwo$ is presented in Algorithm \ref{c2ref}. Let $\agent_{i_1}, \agent_{i_2}, \ldots, \agent_{i_k}$ 
\begin{comment}
$\agentv_{i_1}, \agentv_{i_2}, \ldots, \agentv_{i_k}$ 
\end{comment}
be the topological ordering of the agents in $\ctwo$ as described in Section \ref{additive:cef}
\begin{comment}
with respect to their representation graph
\end{comment}
. In Algorithm \ref{c2ref}, We start with $\agentv_{i_1}$ and $W_2 = \emptyset$ and check whether there exists a vertex $\itemv_j \in \itemsv' \setminus (\itemsv'_{1/2} \cup W_2)$ such that $V_{i_1}(\{\ite_j\}) \geq \epsilon_{i_1}$. If so, we add $\itemv_j$ to $W_2$ and satisfy $\agent_{i_1}$ by allocating $\ite_j$ to $\agent_{i_1}$. Next, we repeat the same process for $\agentv_{i_2}$ and continue on to $\agentv_{i_k}$. Note that at the end of the process, $W_2$ refers to the vertices whose corresponding items are allocated to the agents that are satisfied during the refinement step of $\ctwo$. For convenience, let $S_2 = F_{G'_{1/2}}(M',\itemsv'_{1/2})$ and define $\itemsv''$ and $\agentsv''$ as follows:
$$\itemsv'' = \itemsv' \setminus (W_2 \cup S_2),$$
$$\agentsv'' = \agentsv' \setminus V_{\ctwo}.$$

Let $G'' \langle V(G''),E(G'') \rangle$ be the induced subgraph of $G'$ on $V(G'') = \itemsv'' \cup \agentsv''$. We use $G''$ to build Cluster $\cthree$.


\begin{algorithm}[t!]
 \KwData{$G'(V(G'),E(G'))$}
 \KwData{$\agent_{i_1},\agent_{i_2},\ldots,\agent_{i_k}$ = Topological ordering of agents in $\ctwo$}
  \For{$l:1\rightarrow k$}{
	\If{$ \exists \itemv_j \in \itemsv' \setminus (\itemsv'_{1/2} \cup W_2)$ s.t. $V_{i_1}(\{\ite_j\}) \geq \epsilon_{i_l}$)}
	{
		$\secondset_{i_l} = \ite_j$ \;
		$W_2 = W_2 \cup \itemv_j$\;
		$\ctwo = \ctwo \setminus \agent_{i_l}$\;
		${\satagents} = {\satagents} \cup \agent_{i_l}$\;
	}
  }
 \caption{Refinement of $\ctwo$}
 \label{c2ref}
\end{algorithm}


\begin{observation} 
\label{fsmallc2}
After running Algorithm \ref{c2ref}, For every item $\ite_j$ with $\itemv_j \in \itemsv'' \setminus \itemsv''_{1/2} $ and every agent $ \agent_i \in \ctwo$, we have $V_i(\{\ite_j\}) < \epsilon_i$. 
\end{observation}


In the following two lemmas, we give upper bounds on the value of $\secondset_i$ for every agent $\agent_i \in \satagents_2^r$. First, in Lemma \ref{cr2smallc1}, we show that for every agent $\agent_j \in \cone$, $\valu_j(\secondset_i)$ is upper bounded by $\epsilon_j$. Furthermore, by the fact that the agents that are not selected for Clusters $\cone$ and $\ctwo$ belong to Cluster $\cthree$, we show that $\valu_j(\secondset_i)$ is upper bounded by $1/2$ for every agent $\agent_j \in \cthree$. 
\begin{lemma}[value-lemma]
\label{cr2smallc1}
Let $\agent_i \in \satagents_2^r$ be an agent that is satisfied in the refinement phase of Cluster $\ctwo$ and $\agent_j$ be an  agent in $\cone$. Then, $\valu_j(\secondset_i)<\epsilon_j$.
\end{lemma}


\begin{lemma}[value-lemma]
\label{cr2smallc3}
Let $\agent_i \in \satagents_2^r$ be an agent that is satisfied in the refinement phase of Cluster $\ctwo$ and $\agent_j$ be an agent in $\cthree$. Then, $\valu_j(\secondset_i)< 1/2$.
\end{lemma}

\subsubsection{Cluster $\cthree$.} Finally, Cluster $\cthree$ is defined as the set of agents corresponding to the vertices of $\agentsv''$. Let $M''$ be an $\MCMWM$ of $G''_{1/2}$. Note that by Lemma \ref{rem}, all the vertices in $\itemsv''_{1/2}$ are saturated by $M''$. 
For each vertex $\agentv_i$ that is saturated by $M''$, we allocate the item (or pair of items in a case that $M''(\agentv_i)$ is a merged vertex) corresponding to $M''(\agentv_i)$ to $\agent_i$. Unlike the previous clusters, this allocation is temporary. A semi-satisfied agent $\agent_i$ in $\cthree$ may \emph{lend} his $f_i$ to the other agents of $\cthree$. Therefore, we have three type of agents in $\cthree$: 
\begin{enumerate}
    \item \textbf{The semi-satisfied agents}: we denote the set of semi-satisfied agents in $\cthree$ by $\cthree^s$
    \item \textbf{The borrower agents}: the agents that may borrow from a semi-satisfied agent. An agent $\agent_j$ in $\cthree$ is a borrower, if $\agent_j \notin \cthree^s$ and $\max_{\agent_i \in \cthree^S} V_j(f_i) \geq {1/2}$. We denote the set of borrower agents in $\cthree$ by $\cthree^b$.
    \item \textbf{The free agents}: the remaining agents in $\cthree$. We denote the set of free agents by $\cthree^f$.
\end{enumerate}
So far, the agents corresponding to unsaturated vertices in $\agentsv''_{1/2}$ belong to $\cthree^b$ and the agents corresponding to the vertices in $\agentsv'' \setminus \agentsv''_{1/2}$ are in $\cthree^f$. As we see, during the second phase, agents in $\cthree$ may change their type. For example, an agent in $\cthree^s$ may move to $\cthree^f$ or vice versa.  For convenience, for every agent $\agent_i \in \cthree^b$, we define $\epsilon_i$ as follows: 
\begin{equation}
\label{borrowers}
 3/4 - \max_{\agent_j \in \cthree^s}\valu_i(\firstset_j)
\end{equation} 
Note that by the definition, $\epsilon_i \leq 1/4$ holds for every agent of $\cthree^b$.


\begin{figure}[t!]
\centering
\includegraphics[scale=0.5]{figs/overview}
\caption{Overview on the state of the algorithm}
\label{fig:overview}
\end{figure}

In Lemma \ref{lsmall_c3}, we show that the remaining items are not \emph{heavy} for the agents in $\cthree$. The main reason that Lemma \ref{lsmall_c3} holds, is the fact that no pair of vertices is desirable for any agents in $\cthree$ at the end of Algorithm \ref{addvertex}. 
\begin{lemma}
\label{lsmall_c3}
For all $\agent_i \in \cthree$ and $\itemv_j,\itemv_k \in \itemsv'' \setminus \itemsv''_{1/2}$, we have  $V_i(\{\ite_j,\ite_k\}) < {1/2}$.
\end{lemma}

\begin{corollary}[of Lemma \ref{lsmall_c3}]
\label{small_c3}
For any agent $\agent_i \in \cthree$, there is at most one vertex $\itemv_j \in \itemsv'' \setminus \itemsv''_{1/2}$, such that $V_i(\{\ite_j\}) \geq {1/4}$.
\end{corollary}


\subsection{Phase 2: Satisfying the Agents}\label{additive:allocation}
\subsubsection{An Overview on the State of the Algorithm}
Before going through the second phase, we present an overview of the current state of the agents and items. In Figure \ref{fig:overview}, for every agent $\agent_i \in \cone \cup \ctwo \cup \satagents$, $\firstset_i$ is shown by a gray rectangle and for every agent $\agent_i \in \satagents$, $\secondset_i$  is shown by a hatched rectangle. 

Currently, we know that every agent in $\satagents$ belongs to $\satagents_1^r$ or $\satagents_2^r$. These agents are satisfied in the refinement phases of $\cone$ and $\ctwo$. The rest of the agents will be satisfied in the second phase. For brevity, for $i \leq 2$ we use $\satagents_i^s$ to refer to the agents in $\satagents_i$ that are satisfied in the second phase. More formally, $$\mbox{ for }i=1,2 \qquad \satagents_i^s = \satagents_i \setminus \satagents_i^r .$$

Since we didn't refine Cluster $\cthree$, all the agents in the Cluster $\cthree$ are satisfied in the second phase. As mentioned in the previous section, the item allocation to the semi-satisfied agents in $\cthree$ is temporary; That is, we may alter such allocations later. Therefore, in Figure \ref{fig:overview} we illustrate such allocations by dashed lines. 

 

In this section, we denote the set of free items (the items corresponding to the vertices in $\itemsv''\setminus \itemsv''_{1/2}$ at the end of the first phase) by $\fitems$. By Observations \ref{fsmallc1}, \ref{fsmallc2} and Corollary \ref{small_c3}, we know that the items in $\fitems$ have the following properties:
\begin{enumerate}
\item For every agent $\agent_i$ in $\cone$, $\valu_i(\{\ite_j\}) < \epsilon_i$ holds for all $\ite_j \in \fitems$ (Observation \ref{fsmallc1}).
\item For every agent $\agent_i$ in $\ctwo$, $\valu_i(\{\ite_j\}) < \epsilon_i$ holds for all $\ite_j \in \fitems$ (Observation \ref{fsmallc2}).
\item For every agent $\agent_i$ in $\cthree$, there is at most one item $\ite_j \in \fitems$, such that $\valu_i(\{\ite_j\}) \geq 1/4$ (Corollary \ref{small_c3}).
\end{enumerate}


\begin{table}[t]
	\caption{Summary of value lemmas for $f_i$}
	\label{table0} 
	\begin{center}
\begin{tabular}{|c|c|c|c|}
\hline
	& $\forall \agent_i \in \cone$&  $\forall \agent_i \in \ctwo$ & $\forall \agent_i \in \cthree$\\
\hline
$\forall \agent_j \in \cone$&	- & $\valu_i(\firstset_j) < 1/2$ ($\star$)& $\valu_i(\firstset_j) < 1/2$ ($\star$) \\
\hline
$\forall \agent_j \in \ctwo$ & $ \valu_i(\firstset_j) < 3/4 $ ($\ddagger$)  & - & $\valu_i(\firstset_j) < 1/2$ ($\dagger$)\\ 
\hline
$\forall \agent_j \in \cthree^s$ & $ \valu_i(\firstset_j) < 3/4 $($\ddagger$)  & $\valu_i(\firstset_j) <3/4$($\ddagger$)  &  - \\ 
\hline

\end{tabular}
\end{center}
$\hspace{110pt} \star$: Lemma \ref{forc2c3} $\hspace{10pt} \dagger$: Lemma \ref{forc3} $\hspace{10pt} \ddagger$: Lemma \ref{general}\\
\end{table}


\begin{table}[t]
	\caption{Summary of value lemmas for the agents in $\satagents_i^r$}
	\label{table4} 
\begin{center}
	\begin{tabular}{|c|c|c|c|}
\hline
	&  $\forall \agent_i \in \cone $ & $\forall \agent_i \in \ctwo$ & $\forall \agent_i \in \cthree$\\
\hline
$\forall \agent_j \in \satagents_1^r $	 & - & $\valu_i(\secondset_j)<1/2$ ($\star$)& $\valu_i(\secondset_j)<1/2$ ($\star$)\\
\hline
$\forall \agent_j \in \satagents_2^r $	 & $\valu_i(\secondset_j)<\epsilon_i (\dagger)$ & - & $\valu_i(\secondset_j)<1/2$ ($\ddagger$) \\
\hline
\end{tabular}
\end{center}
$\hspace{105pt}$ $\star$: Lemma \ref{gsmallc1r} $\hspace{10pt}$ $\dagger$: Lemma \ref{cr2smallc1} $\hspace{10pt}$ $\ddagger$: Lemma \ref{cr2smallc3}\\

\end{table}

In summary, items of $\fitems$ are small enough, therefore we can run a process similar to the $\bagfilling$ algorithm described earlier to allocate them to the agents. Recall that our clustering and refinement methods preserve the conditions stated in Lemmas \ref{forc2c3}, \ref{gsmallc1r}, \ref{forc3}, \ref{cr2smallc1} and \ref{cr2smallc3}. In addition to this, we state Lemma \ref{general} as follows.

\begin{lemma}[value-lemma]
\label{general}
For every agent $\agent_i \in \cone \cup \ctwo \cup \cthree^s$, we have
$$\forall \agent_j \in \cone \cup \ctwo \cup \cthree \qquad \valu_j(\firstset_i)<3/4.$$
\end{lemma}  
A brief summary of Lemmas \ref{forc2c3}, \ref{gsmallc1r}, \ref{forc3}, \ref{cr2smallc1}, \ref{cr2smallc3} and \ref{general} is illustrated in Tables \ref{table0} and \ref{table4}. Moreover, since sets $\cone,\ctwo$ and $\cthree^s$ are cycle-envy-free, Observation \ref{epsofcluster} holds for these sets. 



\subsubsection{Second Phase: $\bagfilling$}
We begin this section with some definitions. In the following, we define the notion of feasible subsets and, based on that, we define $\phi(S)$ for a feasible subset $S$ of items.
\begin{definition}
A subset $S$ of items in $\fitems$ is feasible, if at least one of the following conditions are met:
\begin{minipage}[t]{\linegoal}
\begin{enumerate}[leftmargin=30pt]
    \item There exists an agent $\agent_i \in \cthree^f $ such that  $\valu_i(\{S\}) \geq {1/2}$. 
    \item There exists an agent $\agent_i \in \cone \cup \ctwo \cup \cthree^s \cup \cthree^b$ such that  $\valu_i(\{S\}) \geq \epsilon_i$.
\end{enumerate}
\end{minipage}
\end{definition}

\begin{definition}
For a feasible set $S$, we define $\Phi(S)$ as the set of agents, that set $S$ is feasible for them. 
\end{definition}

Recall the notion of cycle-envy-freeness and the topological ordering of the agents in a cycle-envy-free set of semi-satisfied agents. Based on this, we define a total order $\prec_{pr}$ to prioritize the agents in the $\bagfilling$ algorithm. 

\begin{definition}
\label{priority}
Define a total order $\prec_{pr}$ on the agents of $\cone \cup \ctwo \cup \cthree$ with the following rules: 
\begin{minipage}[t]{\linegoal}
	
\begin{enumerate}[leftmargin=50pt]
    \item $\agent_{i_5} \prec_{pr} \agent_{i_1} \prec_{pr} \agent_{i_2} \prec_{pr} \agent_{i_3}  \prec_{pr} \agent_{i_4} \qquad$  $\forall \agent_{i_1} \in \cone, \agent_{i_2} \in \ctwo, \agent_{i_3} \in \cthree^s, \agent_{i_4} \in \cthree^b, \agent_{i_5} \in \cthree^f$
    \item $\agent_i \prec_{pr} \agent_j \Leftrightarrow \agent_i \prec_o \agent_j \hspace{85pt}$ $\forall \agent_i, \agent_j \in \cone \cup \ctwo \cup \cthree^s,  \agent_i ,\agent_j \mbox{ in the same cluster }$
	\item $\agent_i \prec_{pr} \agent_j \Leftrightarrow i < j \hspace{100pt}$ $\forall \agent_i,\agent_j \in \cthree^b \vee \agent_i,\agent_j \in \cthree^f$
\end{enumerate}
\end{minipage}
\end{definition}    

\begin{comment}
\begin{array}{cll}
(I)&\agent_{i_5} \prec_{pr} \agent_{i_1} \prec_{pr} \agent_{i_2} \prec_{pr} \agent_{i_3}  \prec_{pr} \agent_{i_4}& \forall \agent_{i_1} \in \cone, \agent_{i_2} \in \ctwo, \agent_{i_3} \in \cthree^s, \agent_{i_4} \in \cthree^b, \agent_{i_5} \in \cthree^f\\[6pt]
(II)&\agent_i \prec_{pr} \agent_j \Leftrightarrow \agent_i \prec_o \agent_j  & \forall \agent_i, \agent_j \in \cone \cup \ctwo \cup \cthree^s, \qquad \agent_i ,\agent_j \mbox{ in the same cluster }\\[6pt]

(III)&\agent_i \prec_{pr} \agent_j \Leftrightarrow i < j & \agent_i,\agent_j \in \cthree^b \vee \agent_i,\agent_j \in \cthree^f\\[6pt]
\end{array}.

\end{comment}


Recall that $\prec_o$ refers to the topological ordering of a semi-satisfied set of agents. Roughly speaking, for the semi-satisfied agents in the same cluster, $\prec_{pr}$ behaves in the same way as $\prec_{o}$. Furthermore, for the agents in different clusters, agents in $\cthree^f , \cone , \ctwo, \cthree^s , \cthree^b$ have a lower priority, respectively. Finally, the order of the agents in $\cthree^b$ and $\cthree^f$ is determined by their index, i.e., the agent with a lower index has a lower priority.


The second phase consists of several rounds and every round has two steps. Each of these two steps is described below. We continue running this algorithm until $\fitems$ is no longer feasible for any agent.
\begin{itemize}
\item \textbf{Step1}: In the first step, we run a process very similar to the $\bagfilling$ algorithm described in Section \ref{introduction}. That is, we find a feasible subset $S \subseteq \fitems$, such that $|S|$ is minimal. Such a subset can easily be found, using a slight modification of the $\bagfilling$ process (see Section \ref{sphase}).  

\item \textbf{Step2}: In the second step, we choose an agent to allocate set $S$ to him. In contrast to the $\bagfilling$ algorithm, we do not select an arbitrary agent. Instead, we select the agent in $\Phi(S)$ with the lowest priority regarding $\prec_{pr}$, i.e., smallest element in $\Phi(S)$ regarding $\prec_{pr}$. Let $\agent_i$ be the selected agent. We consider three cases separately:

\begin{minipage}[t]{\linegoal}
\begin{enumerate}[leftmargin=50pt]
    \item $\agent_i \in \cthree^f$: temporarily allocate $S$ to $\agent_i$, i.e., set $\firstset_i = S$. 
    \item $\agent_i \in \cthree^b$: let $\agent_j$ be the agent that $\valu_i(\firstset_j) = {3/4} - \epsilon_j$. 
Take back $\firstset_j$ from $\agent_j$ and allocate $\firstset_j \cup S$ to $\agent_i$ i.e. set $\firstset_i = \firstset_j$, $\firstset_j=\emptyset$ and $\secondset_i = S$. Remove $\agent_i$ from $\cthree$ and add it to $\satagents$.
    \item $\agent_i \in \cone \cup \ctwo \cup \cthree^s$: satisfy agent $\agent_i$ by $S$, i.e., set $\secondset_i = S$ and remove $\agent_i$ from its corresponding cluster and add it to $\satagents$. 
\end{enumerate}
\end{minipage}

By the construction of $\cthree^s,\cthree^b$, and $\cthree^f$, the above process may cause agents in $\cthree$ to move in between $\cthree^s,\cthree^b$ and $\cthree^f$. For example, if the first case happens, then $\agent_i$ is moved from $\cthree^f$ to $\cthree^s$. In addition, all other agents in $\cthree^f$ for which $S$ is feasible are moved to $\cthree^b$. For the second case, $\agent_j$ is moved to one of $\cthree^f$ or $\cthree^b$, based on $\valu_j(\firstset_k)$ for every $\agent_k \in \cthree^s$; that is, if there exists an agent $\agent_k \in \cthree^s$ such that $\valu_j(\firstset_k) \geq 1/2$, $\agent_j$ is moved to $\cthree^b$. Otherwise, $\agent_j$ is moved to $\cthree^f$. For both the second and the third cases, some of the agents in $\cthree^b$ may move to $\cthree^f$. 
\end{itemize}
The second phase terminates, when $\fitems$ is no longer feasible for any agent. More details about the second phase can be found in Algorithm \ref{second-phase}.  In Algorithm \ref{second-phase}, we use $Update(\cthree)$ to refer the process of moving agents among $\cthree^s, \cthree^b$ and $\cthree^f$.
 

\begin{algorithm}[t!]
 \KwData{$\fitems, \cone,\ctwo,\cthree$}
  \While{$\fitems$ is feasible}{
	$S$ = a minimal feasible subset of $\fitems$ \;
	$\agent_i = $ agent in $\Phi(S)$ with lowest order regarding  $\prec_{pr}$\;
	\If{$\agent_i \in C_3^f$}
	{
		$\firstset_i = S$ \;
		$Update(\cthree)$ \;
	}
	\If{$\agent_i \in \cthree^b$}
	{
		Let $\agent_j$ be the agent that $\valu_i(\firstset_j) = 3/4 - \epsilon_i$ \;
		$\firstset_i = \firstset_j$ \;
		$\secondset_i = S$ \;
		$\satagents = \satagents \cup \agent_i$ \;
		$\firstset_j = \emptyset$\;
		$\cthree = \cthree \setminus \agent_i$ \;
		$Update(\cthree)$ \;
	}
	\If {$\agent_i \in \cthree^s$}
	{
		$\secondset_i = S$\;
		$\satagents = \satagents \cup \agent_i$\;
		$\cthree = \cthree \setminus \agent_i$ \;
		$Update(\cthree)$ \;
	}
	\If {$\agent_i \in \cone \cup \ctwo$}
	{
		$\secondset_i = S$\;
		remove $\agent_i$ from its corresponding cluster \;
		$\satagents = \satagents \cup \agent_i$\;

	}
}
 \caption{The Second Phase}
 \label{second-phase}
\end{algorithm}

In each round of the second phase, either an agent is satisfied or an agent in $\cthree^f$ becomes semi-satisfied. In Lemma \ref{c3fsmall}, we show that if an agent $\agent_i \in \cthree^f$ is selected in some round of the second phase, then $\valu_j(\firstset_i)$ is upper bounded by $2\epsilon_j$ for every agent $\agent_j \in \cthree \cup \ctwo \cup \cone^s \cup \cone^b$. As a consequence of Lemma \ref{c3fsmall}, in Lemma \ref{cef} we show that sets $\cone,\ctwo$ and $\cthree$ remain cycle-envy-free during the second phase. For convenience, we use $\mathbb{R}_z$ to refer to the $z$'th round of the second phase. 

\begin{lemma}
\label{c3fsmall}
Let $\mathbb{R}_z$ be a round of the second phase that an agent $\agent_i \in \cthree^f$ is selected. Then, for every agent $\agent_j \in \cthree \cup \ctwo \cup \cone^s \cup \cone^b$, we have $\valu_j(\firstset_i)<2\epsilon_j<3/4$.
\end{lemma}



\begin{lemma}
\label{cef}
During the second phase, the $\cone,\ctwo$ and $\cthree^s$ maintain the property of cycle-envy-freeness. 
\end{lemma}

Finally, for the rounds that an agents $\agent_i$ is satisfied, Lemmas \ref{prvalue} and \ref{m_1} give upper bounds on the value of $\secondset_i$ for remaining agents in different clusters. 

\begin{lemma}[value-lemma]
\label{prvalue}
Let $\agent_i \in \satagents$ be an agent that is satisfied in the second phase. Then, for every other agent $\agent_j \in \cone \cup \ctwo$ we have:

\begin{minipage}[t]{\linegoal}
\begin{enumerate}[leftmargin=30pt]
\item If $\agent_j \prec_{pr} \agent_i$, then $\valu_j(\secondset_i) < \epsilon_j$.
\item If $\agent_i \prec_{pr} \agent_j$, then $\valu_j(\secondset_i) < 2\epsilon_j$.
\end{enumerate}
\end{minipage}
\end{lemma}

\begin{lemma}[value-lemma]
\label{m_1}
Let $\agent_i$ be an agent in $\satagents_1^s \cup \satagents_2^s$. Then, for every agent $\agent_j \in \cthree$, we have $\valu_j(\secondset_i) < {1/2}$.
\end{lemma}

The results of Lemmas \ref{prvalue} and \ref{m_1} are summarized in Table \ref{table1}.





\begin{table}[htbp]
\centering
\begin{tabular}{c|c|cccc|cccc}
                     &            & \multicolumn{4}{c|}{$n=50$}                       & \multicolumn{4}{c}{$n=200$}                      \\ \hline
Method               & Evaluation & $\alpha_1$ & $\alpha_2$ & $\alpha_3$ & $\alpha_4$ & $\alpha_1$ & $\alpha_2$ & $\alpha_3$ & $\alpha_4$ \\ \hline
\multirow{4}{*}{MM1}& Bias ($10^{-2}$)  & 1.79 & 4.08 & 4.01 & 2.43 & 0.23 & 0.33 & -0.29 & 0.31 \\ 
& MSE ($10^{-1}$)  & 0.78 & 0.78 & 0.76 & 0.84 & 0.17 & 0.17 & 0.17 & 0.18 \\ 
& MAPE ($10^{-1}$)  & 2.23 & 2.18 & 2.11 & 2.25 & 1.04 & 1.05 & 1.01 & 1.04 \\ 
& Coverage (\%)  & 94.7 & 93.9 & 95.1 & 93.0 & 94.4 & 93.5 & 94.1 & 93.2 \\ 
\hline 
\multirow{4}{*}{MM2}& Bias ($10^{-2}$)  & 2.12 & 4.32 & 4.31 & 2.68 & 0.41 & 0.52 & -0.09 & 0.46 \\ 
& MSE ($10^{-1}$)  & 0.63 & 0.6 & 0.59 & 0.66 & 0.13 & 0.13 & 0.13 & 0.13 \\ 
& MAPE ($10^{-1}$)  & 2.01 & 1.91 & 1.87 & 2.02 & 0.9 & 0.91 & 0.9 & 0.9 \\ 
& Coverage (\%)  & 93.0 & 93.8 & 94.3 & 92.4 & 94.4 & 94.3 & 95.3 & 95.5 \\ 
\hline 
\multirow{4}{*}{MM3}& Bias ($10^{-2}$)  & 2.12 & 4.32 & 4.31 & 2.68 & 0.41 & 0.52 & -0.09 & 0.46 \\ 
& MSE ($10^{-1}$)  & 0.63 & 0.6 & 0.59 & 0.66 & 0.13 & 0.13 & 0.13 & 0.13 \\ 
& MAPE ($10^{-1}$)  & 2.01 & 1.91 & 1.87 & 2.02 & 0.9 & 0.91 & 0.9 & 0.9 \\ 
& Coverage (\%)  & 93.8 & 94.0 & 94.4 & 92.8 & 95.2 & 94.4 & 94.7 & 95.5 \\ 
\hline 
\multirow{4}{*}{MM4}& Bias ($10^{-2}$)  & 0.88 & 3.06 & 3.04 & 1.44 & 0.11 & 0.22 & -0.39 & 0.16 \\ 
& MSE ($10^{-1}$)  & 0.61 & 0.57 & 0.56 & 0.63 & 0.13 & 0.13 & 0.13 & 0.13 \\ 
& MAPE ($10^{-1}$)  & 1.98 & 1.87 & 1.82 & 1.98 & 0.89 & 0.91 & 0.9 & 0.9 \\ 
& Coverage (\%)  & 93.3 & 93.8 & 94.8 & 93.1 & 95.1 & 95.1 & 95.1 & 95.2 \\ 
\hline 
\multirow{4}{*}{BE1}& Bias ($10^{-2}$)  & 1.91 & 3.71 & 3.65 & 2.27 & 0.53 & 0.44 & -0.14 & 0.59 \\ 
& MSE ($10^{-1}$)  & 0.45 & 0.44 & 0.42 & 0.46 & 0.11 & 0.11 & 0.11 & 0.11 \\ 
& MAPE ($10^{-1}$)  & 1.67 & 1.65 & 1.58 & 1.7 & 0.81 & 0.84 & 0.83 & 0.83 \\ 
& Coverage (\%)  & 94.6 & 95.9 & 96.1 & 95.2 & 95.3 & 94.4 & 94.8 & 95.5 \\ 
\hline 
\multirow{4}{*}{BE2}& Bias ($10^{-2}$)  & 0.7 & 2.51 & 2.45 & 1.07 & 0.24 & 0.14 & -0.43 & 0.3 \\ 
& MSE ($10^{-1}$)  & 0.44 & 0.44 & 0.41 & 0.46 & 0.11 & 0.11 & 0.11 & 0.11 \\ 
& MAPE ($10^{-1}$)  & 1.67 & 1.64 & 1.58 & 1.7 & 0.82 & 0.84 & 0.83 & 0.83 \\ 
& Coverage (\%)  & 94.6 & 95.9 & 96.1 & 95.2 & 95.3 & 94.4 & 94.8 & 95.5 \\ 
\end{tabular}
\caption{\label{tab:alpha-experiment1}Estimate of bias, MSE, MAPE and Coverage for each of the six methods when the true value of $\alpha$ of the generative process is $\alpha = (1,1,1,1)$ and the number of samples is $n=50$ or $n=200$. 
The estimates are calculated using Monte Carlo with $1,000$ iterations, as described in \autoref{sec:recovering-bivariate-beta}.}
\end{table}

\subsection{The Algorithm Finds a $3/4$-$\MMS$ Allocation}\label{additiveproofs}
In the rest of this section, we prove that the algorithm finds a $3/4$-$\MMS$ allocation. For the sake of contradiction, suppose that the second phase is terminated, which means $\fitems$ is not feasible anymore, but not all agents are satisfied. Such an unsatisfied agent belongs to one of the Clusters $\cone$ or $\ctwo$, or $\cthree$. In Lemmas \ref{c3null}, \ref{c1null}, and \ref{c2null}, we separately rule out each of these possibilities. This implies that all the agents are satisfied and contradicts the assumption. For brevity the proofs are omitted and included in Appendix \ref{additiveproofappendix}. We begin with Cluster $\cthree$.
\begin{lemma}
	\label{c3null}
	At the end of the algorithm we have $\cthree = \emptyset$.
\end{lemma}
%\input{c3proof}
To prove Lemma \ref{c3null} we consider two cases separately. If $\cthree \neq \emptyset$, either there exists an agent $\agent_i \in \cthree^s \cup \cthree^b$ or all the agents of $\cthree$ are in $\cthree^f$. If the former holds, we show $\cthree^s$ is non-empty and assume $\agent_i$ is a winner of $\cthree^s$. We bound the total value of $\agent_i$ for all the items dedicated to other agents and show the value of the remaining items in $\fitems$ is at least $\epsilon_i$ for $\agent_i$. This shows set $\fitems$ is feasible for $\agent_i$ and contradicts the termination of the algorithm. In case all agents of $\cthree$ are in $\cthree^f$, let $\agent_i$ be an arbitrary agent of $\cthree^f$. With a similar argument we show that the value of $\agent_i$ for the remaining unassigned items is at least $3/4$ and conclude that $\fitems$ is feasible for $\agent_i$ which again contradicts the termination of the algorithm.

Next, we prove a similar statement for $\cone$. 
\begin{lemma}
	\label{c1null}
	At the end of the algorithm we have $\cone = \emptyset$.
\end{lemma}
Proof of Lemma \ref{c1null} follows from a coloring argument. Let $\agent_i$ be a winner of $\cone$. We color all items in either blue or white. Roughly speaking, blue items are in a sense \textit{heavy}, i.e., they may have a high valuation to $\agent_i$ whereas white items are somewhat \textit{lighter} and have a low valuation to $\agent_i$. Next, via a double counting argument, we show that $\agent_i$'s value for the items of $\fitems$ is at least $\epsilon_i$ and thus $\fitems$ is feasible for $\agent_i$. This contradicts $\cone = \emptyset$ and shows at the end of the algorithm all agents of $\cone$ are satisfied.

Finally, we show that all the agents in Cluster $\ctwo$ are satisfied by the algorithm.
\begin{lemma}
	\label{c2null}
	At the end of the algorithm we have $\ctwo = \emptyset$.
\end{lemma}
The proof of Lemma \ref{c2null} is a similar to both proofs of Lemmas \ref{c3null} and \ref{c1null}. Let $\agent_i$ be winner of Cluster $\ctwo$. We consider two cases separately. (i) $\epsilon_i \geq 1/8$ and (ii) $\epsilon_i < 1/8$.
In case $\epsilon_i \geq 1/8$, we use a similar argument to the proof of Lemma \ref{c3null} and show $\fitems$ is feasible for $\agent_i$. If $\epsilon_i < 1/8$ we again use a coloring argument, but this time we color the items with 4 different colors. Again, via a double counting argument we show $\fitems$ is feasible for $\agent_i$ and hence every agent of $\ctwo$ is satisfied when the algorithm terminates. 
\begin{theorem}
	\label{34main}
	All the agents are satisfied before the termination of the algorithm.
\end{theorem}
\begin{proof}
	By Lemmas \ref{c3null}, \ref{c1null}, and \ref{c2null}, at the end of the algorithm all agents are satisfied which means each has received a subset of items which is worth at least $3/4$ to him.
\end{proof}

\subsection{Algorithm}\label{additive:algorithm}
In this section, we present a polynomial time algorithm to find a $(3/4-\epsilon)$-$\MMS$ allocation in the additive setting. More precisely, we show that our method for proving the existence of a $3/4$-$\MMS$ allocation can be used to find such an allocation in polynomial time. 
Recall that our algorithm consists of two main phases: The clustering phase and the $\bagfilling$ phase. In Sections \ref{algcluster} and \ref{sphase} we separately explain how to implement each phase of the algorithm in polynomial time. Given this, there are still a few computational issues that need to be resolved. First, in the existential proof, we assume $\MMS_i = 1$ for every agent $\agent_i \in \agents$.  Second, we assume that the problem is $3/4$-irreducible. Both of these assumptions are without loss of generality for the existential proof due to Observation \ref{reducibility} and the fact that one can scale the valuation functions to ensure $\MMS_i =1$ for every agent $\agent_i$. However, the computational aspect of the problem will be affected by these assumptions.  
The first issue can be alleviated by incurring an additional $1+\epsilon$ factor to the approximation guarantee. \epsteinefficient ~\cite{epstein2014efficient} show that for a given additive function $f$, $\MMS_f^n$ can be approximated within a factor $1+\epsilon$ for constant $\epsilon$ in time $\poly(n)$. Thus, we can scale the valuation functions to ensure $\MMS_i = 1$ while losing a factor of at most $1+\epsilon$. Therefore, finding a $(3/4-\epsilon)$-$\MMS$ allocation can be done in polynomial time if the problem is $3/4$-irreducible. Finally, in Section \ref{irre} we show how to reduce the $3/4$-reducible instances and extend the algorithm to all instances of the problem. The algorithm along with the reduction yields Theorem \ref{addpoly}

\begin{theorem}
	\label{addpoly}
	For any $\epsilon > 0$, there exists an algorithm that finds a $(3/4 - \epsilon)$-$\MMS$ allocation in polynomial time. 
\end{theorem}

\begin{comment}
\subsubsection{Computing the value of $\MMS_i$ in polynomial time}\label{mmsi}
As described in the beginning of Section \ref{additive}, finding the exact value of $\MMS_i$ for an agent is $NP-hard$ and there is no polynomial time algorithm for this problem, unless $P=NP$. In \cite{epstein2014efficient}, this problem is studied in the context of job scheduling. In addition to the hardness proof, they proposed a PTAS for finding a $(1+\epsilon)$ approximation of $\MMS_i$ which yields to a ploynomial time algorithm for finding $\MMS_i$, for constant $\epsilon$.  

Thus, in the beginning of the algorithm, we can compute $\MMS_i$ for every agent $\agent_i$. Considering the fact the the problem of finding $\MMS_i$ is a maximization problem, we know that the value obtained for $\MMS_i$ by method in \cite{epstein2014efficient}, is at least $\MMS_i (1 - \epsilon)$. Considering this value as $\MMS_i$ has no effect on the correctness of the algorithm except that the final result would be a $(3/4-\epsilon)$-$\MMS$ allocation. Thus, assuming that the rest of the algorithm can be implemented in polynomial time, we have a polynomial time $(3/4-\epsilon)$-$\MMS$ allocation algorithm for contant $\epsilon$. 
\end{comment}
\subsubsection{The Clustering Phase}\label{algcluster}
Recall that in the clustering phase we cluster the agents into three sets $\cone$,$\ctwo$, and $\cthree$. In order to build Cluster $\cone$, we find an $\MCMWM$ of the $1/2$-filtering of the value graph. This can be trivially done in polynomial time since finding an $\MCMWM$ is polynomially tractable~\cite{west2001introduction}. However, the refinement phase of Cluster $\cone$ requires finding $F_G(\itemsv,M)$ for a giving graph $G$ and a matching $M$. In what follows, we show this problem can also be solved in polynomial time.

\begin{comment}
First, note that an $\MCMWM$ of $G$ can be found in polynomial time using standard methods for finding minimum cost maxmimum flow in networks. For this, create a networks as follows: Orient every edge $(\itemv_j,\agentv_i)$ from $\agentv_i$ to $\itemv_j$, with cost $-w(\itemv_j,\agentv_i)$ and capacity $1$. Also, add a source node $s$ and connect it to all the vertices in $\parttwo$ with cost $0$ and capacity $1$ and add a sink node $t$ and connect every vertex in $\partone$ to $t$, with cost $0$ and capacity $1$. It is easy to observe that the edges between $\partone$ and $\parttwo$ with non-zero flow in a Min Cost Maximum Flow from $s$ to $t$ in this network form a maximum cardinality matching $M$. In addition to this, since the maximum flow was minimum cost, 
$$\sum_{(\itemv_j,\agentv_i) \in M} -w(\itemv_j,\agentv_i)$$
is minimized, which means 
$$\sum_{(\itemv_j,\agentv_i) \in M} w(\itemv_j,\agentv_i)$$
is maximized in $M$. So, $M$ is a $\MCMWM$ of $H$. 
\end{comment}

Notice that finding an $\MCMWM$ of $G$ can be done in polynomial time~\cite{west2001introduction}. Therefore, in order to determine $F_H(M,\partone)$, it only suffices to find the vertices of $\partone$ that are reachable from the unmatched vertices of $\parttwo$ by an alternating path. Let $\hat{X}$ be the set of these vertices. We can find $\hat{X}$ using a depth-first-search from the unmatched vertices of $\parttwo$. By definition, $F_H(M,\partone) = \parttwo \setminus \hat{X}$. Therefore, $F_H(M,\partone)$ can be found in polynomial time.

In addition to $F_G(\itemsv,M)$, we also need to find a matching of the graph which satisfies the conditions of Lemma \ref{nicematch}. We show in the following that this problem also can be solved in polynomial time. First, note that in Lemma \ref{v1size} we prove that $G_1$ has a matching that saturates all the vertices of $W_1$. Now, let $p_{\agent_k}$ be the position of $\agent_k$ in the topological ordering of $\cone$, as described in the proof of Lemma \ref{nicematch}. Furthermore, Let $M_1$ be a matching that minimizes the following expression.
$$ \sum_{(x_j,y_i) \in M_1} p_i.$$ Recall that in the proof Lemma \ref{nicematch}, we show that $M_1$ satisfies the condition described in Lemma \ref{nicematch}. Here, we show that $M_1$ can be found in polynomial time. To this end, we model this with a network design problem. 

Orient every edge $(x_j,y_i) \in G_1$ from $y_i$ to $x_j$ and set the cost of this edge to $p_{a_i}$. Also, add a source node $s$ and add a directed edge from $s$  to every vertex of $V_{\cone}$ with cost $0$. Furthermore, add a sink node $t$ and add directed edges from the vertices of $W_1$ to $t$ with cost $0$. Finally, set the capacity of all edges to $1$. 

One can observe that in a minimum cost maximum flow from $s$ to $t$ in this network, the edges with non-zero flow between $V_{\cone}$ and $W_1$ form a maximum matching $M_1$. In addition to this, since the cost of the flow is minimal, $\sum_{(x_j,y_i) \in M_1} cost(x_j,y_i)$ is minimized. Therefore, in this matching, 
$\sum_{(x_j,y_i) \in M_1} p_i$
is minimized. Thus, the matching with desired properties of Lemma \ref{nicematch} can be found in polynomial time.

The same algorithms can be used to compute Cluster $\ctwo$. Finally, we put the rest of the agents in Cluster $\cthree$.


\subsubsection{The $\bagfilling$ Phase}\label{sphase}
In each round of the second phase, we iteratively find a minimal feasible subset of $\fitems$ and allocate its items to the agent with the lowest priority in $\Phi(S)$.  Note that for a feasible set $S$, one can trivially find the agent with lowest priority in $\Phi(S)$ in polynomial time. Thus, it only remains to show that we can find a minimal feasible subset of $\fitems$ in polynomial time. 

Consider the following algorithm, namely \emph{reverse $\bagfilling$ algorithm}: Start with a bag containing all the items of $\fitems$ and so long as there exists an item $\ite_j$ in the bag such that after removing $\ite_j$, the set of items in the bag is still feasible, remove $\ite_j$ from the bag. After this process, the remaining items in the bag  form  a minimally feasible subset of $\fitems$. Therefore, this phase can be run in polynomial time.

\subsubsection{Reducibility}\label{irre}
The most challenging part of our algorithm is dealing with the $3/4$-irreducibility assumption. The catch is that, in order to run the algorithm, we don't necessarily need the $3/4$-irreducibility assumption. Recall that we leverage the following three consequences of irreducibility to prove the existential theorem.
\begin{itemize}
	\item The value of every item in $\items$ is less that $3/4$ to every agent.
	\item Every pair of items in $\itemsv'' \setminus \itemsv''_{1/2}$ is in total worth less than $3/4$ to any agent.
	\item The condition of Lemma \ref{v1size} holds.
\end{itemize}
 Therefore, the algorithm works so long as the mentioned conditions hold. Note that, although it is not clear whether determining if an instance of the problem is $3/4$-reducible is polynomially tractable, all of the above conditions can be validated in polynomial time. This is trivial for the first two conditions; we iterate over all items or pairs of items and check if the condition holds for these items. The last condition, however, is harder to validate.
 %Therefore, it only suffices to show that we can operate in a way that our problem preserves these three conditions.

%For the first condition, consider the following process: While there exists an agent $\agent_i$ and an item $\ite_j$ with $\valu_i(\{\ite_j\})\geq 3/4$, assign $\ite_j$ to $\agent_i$ and solve the problem recursively for the rest of the agents and items. Trivially, this process can be implemented in polynomial time. Furthermore, after this process, all the items are worth less than $3/4$ to any agent and the first condition holds. 

The condition of Lemma \ref{v1size} holds if for all $S \subseteq W_1$, $|N(S)| > |S|$. Recall that in the proof of Lemma \ref{v1size} we showed that if this condition does not hold, then $F_{G_1}(M,\itemsv)$ is non-empty. Next, we showed that if $F_{G_1}(M,\itemsv)$ is non-empty, then we can reduce the problem via satisfying every agents of $F_{G_1}(M,\itemsv)$ by his matched item in $M$. Therefore, on the computational side, we only need to find whether $F_{G_1}(M,\itemsv)$ is empty which indeed can be determined in polynomial time. 
%In this case, we operate as follows: %
%If the condition in Lemma \ref{v1size} does not hold, we reduce the problem by satisfying the agents in $F_{G_1}(M,\itemsv))$. After this, regarding Lemma \ref{iff}, the condition in Lemma \ref{v1size} holds. As described in section \ref{cphase}, $F_{G_1}(M,\itemsv))$ can be found in polynomial time.

%For the third condition, we can operate in the same way as the first condition: While there exists an agent $\agent_i$ and a pair of items $\ite_j,\ite_k$ such that   $\valu_i(\{\ite_j,\ite_k\})\geq 3/4$, reduce the problem by assigning $\ite_j,\ite_k$ to $\agent_i$ and removing them from $\items$ and $\agents$, respectively. 

Note that every time we reduce the problem, $|\agents|$ is decreased by at least $1$, which implies the number of times we reduce the problem is no more than $n$. Moreover, our reduction takes a polynomial time. Thus, the running time of the algorithm is polynomial. %Thus, in polynomial time, you can either satisfty all the agents, or obtain the problem instance that preserves all three conditions.

\section{Recurrent Submodular Welfare}
Let $f(S): 2^{\A} \rightarrow \mathbb{R}_{\geq 0}$ be a monotone submodular function over a universe $\A$ of $k$ elements, such that $f(\emptyset) = 0$. In the {\em blocking} setting, each element $i \in \A$ is associated with a known deterministic {\em delay} $d_i \in \mathbb{N}_{>0}$, such that once the arm is played at some round $t$, it becomes unavailable for the next $d_i-1$ rounds, namely, in the interval $\{t, \dots, t+d_i-1\}$. At each round $t \in [T]$, the player chooses a subset $\A_t$ of available (i.e., non-blocked) elements and collects a reward $f(\A_t)$. The goal is to maximize the total reward collected, i.e., $\sum_{t \in [T]} f(\A_t)$, within an unknown time horizon $T$. 

Before we present our algorithm, we provide ``bad'' instances for two natural approaches to \rsm.

\begin{remark} \label{rem:greedy}
The greedy approach of choosing $\A_t$ to be the set of all available elements at round $t \in [T]$ can be as bad as a $\frac{1}{k}$-approximation. In order to see that, consider the monotone (budget-additive) submodular function $f(S) = \min\{|S|, 1\}$. Let $k$ be the number of elements with delay $d_i = k$ for each $i \in \A$. Assuming an infinite time horizon, the optimal strategy collects an average reward of $1$, simply by choosing one element at a time in a round-robin manner. However, the average reward of the greedy approach in this case is $\frac{1}{k}$.
\end{remark}

\begin{remark}
The independent randomized sampling approach of adding each arm $i$ to $\A_t$ independently with probability $\frac{1}{d_i}$, if available, can be as bad as a $(1 - \frac{1}{\sqrt{e}} )$-approximation. Consider the same setting as in Remark \ref{rem:greedy}, where for $T \to \infty$ the optimal average reward is $1$. However, the average expected reward of the independent randomized sampling strategy is $1 - (1 - p)^k$, where $p = \frac{1}{2k-1}$ is the probability that each element is selected at each round (in stationarity). For $k \to \infty$, we have that $1 - (1 - p)^k \to 1- e^{-\frac{1}{2}} \approx 0.393$.

\end{remark}
We provide an efficient randomized $\left(1-\frac{1}{e}\right)$-approximation algorithm for \rsm. Informally, the algorithm starts by considering, for each element $i \in \A$, a sequence of rational numbers of the form $\{t\cdot \frac{1}{d_i}\}_{t \in [T]}$. Then, these sequences are {\em interleaved} by randomly adding an {\em offset} $r_i$, drawn uniformly at random from $[0,1]$, for each $i \in \A$ to the corresponding sequence. At every round $t \in [T]$, the algorithm chooses a set $\A_t$, consisting only of elements for which the (perturbed) interval $L_{i,t} = [t\cdot \frac{1}{d_i}+ r_i, (t+1)\cdot \frac{1}{d_i}+ r_i )$ contains an integer.

\begin{algorithm}[\is (\IS)]
For each element $i \in \A$, let $r_i \sim U[0,1]$ be a random {\em offset} drawn uniformly from $[0,1]$. 
At every round $t = 1, 2, \dots$,  let $\A_t \subseteq \A$ be the subset of elements such that for any $i \in \A_t$, the interval $L_{i,t} = [t\cdot \frac{1}{d_i} + r_i, (t+1) \cdot \frac{1}{d_i} + r_i)$ contains an integer. Choose the elements $\A_t$ and collect the reward $f(\A_t)$.
\end{algorithm}


\subsection{Correctness and approximation guarantee.} 
We first show the algorithm is correct, namely, that the elements chosen at each round respect the blocking constraints. The correctness is established by the following simple observation:

\begin{restatable}{fact}{restatefactalwaysavailable}\label{inter:fact:alwaysavailable}
At any $t \in [T]$, all the elements in $\A_t$ are available (i.e., not blocked).
\end{restatable}

In order to prove the competitive guarantee of our algorithm, we first construct a convex programming (CP)-based (approximate) upper bound on the optimal reward. Although our algorithm never computes an optimal solution to this CP, this step allows us to prove our guarantee, leveraging results on the correlation gap of submodular functions. For $\bm{d}^{-1} \in \mathbb{R}^k$ such that $(\bm{d}^{-1})_i = \frac{1}{d_i}, \forall i \in [k]$, consider the following formulation based on the concave closure $f^+$ of $f$:
\begin{align}
\maximize_{\z \in \mathbb{R}^k}~~ T \cdot f^+(\z)~~\textbf{s.t.}~~ \bm{0} \preceq \z \preceq \bm{d}^{-1}. \tag{\textbf{CP}} \label{cp:CP}
\end{align}

In \eqref{cp:CP}, each variable $z_{i}$ can be thought of as the fraction of rounds where element $i\in \A$ is chosen. Intuitively, the constraints indicate the fact that, due to the blocking, each element $i \in \A$ can be played at most once every $d_i$ steps. 
In order to derive \eqref{cp:CP}, we start from a non-convex integer program (IP) with 0-1 variables $\{x_{i,t}\}_{i \in \A, t \in [T]}$, each indicating whether element $i \in \A$ is used at round $t \in [T]$. The objective is to maximize $\sum_{t \in [T]} \sum_{S \subseteq \A} f(S) \prod_{i \in S} x_{i,t} \prod_{i \notin S}(1 - x_{i,t})$ subject to natural blocking constraints. For integral solutions, the above objective is equivalent to $\sum_{t \in [T]} f^+(\x_t)$ (where $(\x_t)_i = x_{i,t}$) and, thus, the above relaxation is simply the result of averaging over time the variables and constraints of this IP. By using the concavity of $f^+$, we are able to show that \eqref{cp:CP} yields an (approximate) upper bound on the optimal solution of \rsm, while the approximation becomes exact as $T$ increases.

\begin{restatable}{lemma}{restateStructuralCP}\label{lem:structural:CP}
Let $\Rew^{CP}(T)$ be the optimal solution to \eqref{cp:CP} and $\OPT(T)$ be the optimal solution over $T$ rounds. We have
$
\Rew^{CP}(T) \geq \OPT(T) - \mathcal{O}(d_{\max} f(\A)),
$ where $d_{\max} = \max_{i \in \A}\{d_i\}$.
\end{restatable}

\begin{remark}
By replacing $f^+(\z)$ in \eqref{cp:CP} with the multi-linear extension $F(\z)$, the formulation no longer yields an upper bound on the optimal reward (not even asymptotically). Indeed, consider a function $f$ over a ground set $\A=\{1,2\}$ with $d_1 = d_2 = 2$, such that $f(\emptyset) = 0$, $f(\{1\}) = f(\{2\}) = 2$ and $f(\{1,2\}) = 3$. For $T \to \infty$, the optimal average reward is $2$, simply by choosing the two elements interchangeably. However, the formulation based on $F(\z)$ in that case would be to maximize $2z_1(1-z_2) + 2z_2(1-z_1) + 3 z_1 z_2$ subject to $z_1,z_2 \leq \frac{1}{2}$, which has a global maximum of $\frac{7}{4} < 2$.
\end{remark}


Before we complete the proof of our first main result, we first compute the probability that $i \in \A_t$, i.e., an element $i \in \A$ is sampled at round $t \in [T]$:

\begin{restatable}{fact}{restatefactsampling}\label{inter:fact:sampling}
For any $i \in \A$ and $t \in [T]$, we have
$\Pro{i \in \A_t} = \Pro{L_{i,t} \cap \mathbb{N} \neq \emptyset } = \frac{1}{d_i}.
$
\end{restatable}

\noindent{\em Proof of Theorem \ref{thm:interleavedSubmodular}.} 
Let us denote by $S \sim {\bf p}$ with ${\bf p} \in [0,1]^k$ the random set $S \subseteq \A$, where each element $i$ participates in $S$ independently with probability equal to $p_i$. 
By Fact~\ref{inter:fact:sampling} and due to the randomness of the offsets $\{r_i\}_{i \in \A}$, we have that $\A_t \sim {\bf d}^{-1}$ for each $t \in [T]$. Let $\z^*$ be an optimal solution to \eqref{cp:CP}. By monotonicity of $f$ and the fact that $\z^* \preceq \bm{d}^{-1}$, for the expected value of $f(\A_t)$ at any round $t \in [T]$, we know that $\Ex{\A_t \sim \bm{d}^{-1}}{f(\A_t)} \geq \Ex{\A_t \sim \z^*}{f(\A_t)}$. Moreover, by definition of the multi-linear extension, we have that $\Ex{\A_t \sim \z^*}{f(\A_t)} = F(\z^*)$, while by Lemma~\ref{lem:correlationgap} (the correlation gap of submodular functions), we have that, $F(\z) \geq \left(1 - \frac{1}{e}\right)f^+(\z)$ for any vector $\z \in [0,1]^k$. By combining the above facts, we can see that
\begin{align*}
\Rew^{IS}(T) = \sum_{t \in [T]} \Ex{\A_t \sim \bm{d}^{-1}}{f(\A_t)} \geq 
\sum_{t \in [T]} F(\z^*) \geq \left(1 - \frac{1}{e}\right)T\cdot f^+(\z^*) = \left(1 - \frac{1}{e}\right) \Rew^{CP}(T).
\end{align*}
Therefore, by Lemma~\ref{lem:structural:CP}, we can conclude that $\Rew^{IS}(T) \geq \left(1 - \frac{1}{e}\right)\OPT(T) - \mathcal{O}(d_{\max} f(\A))$.
\qed
\newline

In Appendix \ref{appendix:hardness}, we provide a $\left(1-\frac{1}{e}\right)$-hardness result for \rsm, thus proving that the guarantee of Theorem~\ref{thm:interleavedSubmodular} is asymptotically tight. This result, which holds even for the special case where $d_{\max} = o(T)$ (that is when the delays are significantly smaller than the time horizon), is proved via a reduction from the SWM problem with identical utilities, in a way that the constructed \rsm instance accepts w.l.o.g. solutions of a simple periodic structure.

\begin{restatable}{theorem}{restateSubmodularHardness}\label{thm:submodular:hardness}
For any $\epsilon>0$, there exists no polynomial-time $\left(1-\frac{1}{e} + \epsilon \right)$-approximation algorithm for the \rsm problem, unless ${\bf P}={\bf NP}$, even in the special case where $d_{\max} = o(T)$.
\end{restatable}
\section{XOS Agents}\label{xos}
Class of fractionally subadditive (XOS) set functions is a super class of submodular functions. These functions too, have been subject of many studies in recent years \cite{christodoulou2008bayesian, bhawalkar2011welfare, feige2009maximizing,blumrosen2007welfare, syrgkanis2012bayesian,feldman2013simultaneous,fu2012conditional,feldmancombinatorial,milchtaich1996congestion}. Similar to sub-modular functions, in this section we show a $1/5$-$\MMS$ allocation is possible when all agents have XOS valuations. Furthermore, we complement our proof by providing a polynomial algorithm to find a $1/8$-$\MMS$ allocation in Section \ref{xosalgorithm}.

 %In this section we study the fair  allocation problem with XOS agents. To the best of our knowledge, this is the first work that studies this problem with non-additive valuation functions. We give a proof to the existence of a $1/5$-$\MMS$ allocation for XOS agents. This is followed by an algorithm that finds such an allocation in polynomial time. This is surprising since even finding the $\MMS$ of an XOS function is NP-hard and cannot be implemented in polynomial time unless P=NP. Since every submodular function is also fractionally subadditive, our result carries over to submodular functions as well. For brevity, we defer the proof of some lemmas to Appendix \ref{xosappendix} and just mention the high-level ideas.
 
 
\subsection{Existential Proof}\label{ep}
In this section we show every instance of the fair allocation problem with XOS agents admits a $1/5$-$\MMS$ allocation. 
Without loss of generality, we assume $\MMS_i = 1$ for every agent $\agent_i$. Recall the definition of ceiling functions.
\begin{definition}\label{fxfunction}
Given a set function $f(.)$, we define $\ceil{f}{x}(.)$ as follows:
$$\ceil{f}{x}(S) =
\begin{cases}
	f(S), & \text{if }f(S) \leq x \\
	x, & \text{if }f(S) > x.
\end{cases}$$

\end{definition}

As stated in Lemma \ref{ceilingfunctions}, for every XOS function and every real number $x \geq 0$, $f^x$ is also XOS. The proof of this section is similar to the result of Section \ref{submodularalg}. However, the details are different since XOS functions do not adhere to the nice structure of submodular functions. For every allocation $\mathcal{B}$, we define $\mathsf{ex}^{2/5}(\mathcal{B})$ as follows:
 
$$\mathsf{ex}^{2/5}(\mathcal{B}) = \sum_{\agent_i \in \agents} \ceil{\valu_i}{2/5}(B_i).$$

Now Let $\mathcal{A} = \langle A_1, A_2, \ldots, A_n\rangle$ be an allocation of items to the agents that maximizes $\mathsf{ex}^{2/5}$. Provided that the problem is $1/5$-irreducible, we show $\mathcal{A}$ is a $1/5$-$\MMS$ allocation. Before we proceed to the main proof, we state Lemmas  \ref{xos2lemma}, and \ref{2nsets} as auxiliary observations. 
%Based on Definition \ref{fxfunction}, one can show that given an XOS function $f(.)$, $\ceil{f}{x}(.)$ is also XOS for any $x \geq 0$.

\begin{lemma}\label{xos2lemma}
Let $f(.)$ be an XOS set function and  $f(S) = \beta$ for a set $S \subseteq \domp(f)$. If we divide $S$ into $k$ (possibly empty)  sets $S_1, S_2, \ldots, S_k$ then 
$$\sum_{i=1}^k \Big(f(S) - f(S\setminus S_i)\Big) \leq f(S).$$
\end{lemma}
The complete proof of Lemma \ref{xos2lemma} is included in Appendix \ref{xosappendix}. Roughly speaking, the proof follows from the fact that for at least one of the additive set functions in the representation of $f$, we have $g_j(S) = \beta$. The rest of the proof is trivial by the additive properties of $g_j$.

By Lemma \ref{remove1}, we know that in every $1/5$-irreducible instance of the problem, the value of every item for a person is bounded by $1/5$. 
\begin{comment}
\begin{lemma}\label{xosreducible}
In every $1/5$-irreducible instance of the problem we have $\valu_i(\{\ite_j\}) \leq 1/5$ for every $\agent_i \in \agents$ and $\ite_j \in \items$.
\end{lemma}
\end{comment}
For XOS functions, we again, leverage the reducibility principal to show another important property of the $1/5$-irreducible instances of the problem. 
\begin{lemma}
\label{2nsets}
In a $1/5$-irreducible instance of the problem, for a given agent $\agent_i$ we can divide the items into $2n$ sets $S_1, S_2, \ldots, S_{2n}$ such that
$$\valu_i(S_i) \geq 2/5$$
for every $1 \leq i \leq 2n$.
\end{lemma}
We first apply Lemma \ref{remove1} and show in such instances of the problem the valuation of every agent for every item is bounded by $1/5$. We remark that for every agent $\agent_i$, one can split the items into $n$ partitions such that each partition is worth at least $1$ to $\agent_i$. Combining the two observations, we conclude that such a decomposition is possible for every agent $\agent_i$. The full proof of this lemma is included in Appendix \ref{xosappendix}. Next we prove the main theorem of this section.
\begin{theorem}\label{xosproof}
The fair allocation problem with XOS agents admits a $1/5$-$\MMS$ allocation.
\end{theorem}
\begin{proof}
Similar to what we did in Section \ref{additive}, we only prove this for $1/5$-irreducible instances of the problem. By Observation \ref{reducibility}, we can extend this result to all instances of the problem.

Consider an allocation $\mathcal{A} = \langle A_1, A_2, \ldots, A_n\rangle$ of items to the agents that maximizes $\mathsf{ex}^{2/5}$.
We show that such an allocation is $1/5$-$\MMS$. Suppose for the sake of contradiction that there exists an agent $\agent_i$ who receives a set of items which are together of worth less than $1/5$ to him. More precisely,
$$\valu^{2/5}_i(A_i) = \valu_i(A_i) < 1/5.$$ 
Since the problem is $1/5$-irreducible, by Lemma \ref{2nsets}, we can divide the items into $2n$ sets $S_1, S_2, \ldots, S_{2n}$ such that $\valu_i(S_j) \geq 2/5$ for every $1 \leq j \leq 2n$. Note that in this case, $\valu^{2/5}_i(S_j) = 2/5$ follows from the definition. Moreover by monotonicity, $\valu^{2/5}_i(S_j \cup A_i) = 2/5$ holds for every $j$.

Now consider $2n$ allocations $\mathcal{A}^1, \mathcal{A}^2, \ldots, \mathcal{A}^{2n}$ such that 
$$\mathcal{A}^{j} = \langle A^j_1, A^j_2 \ldots, A^j_n\rangle$$ for every $1 \leq j \leq 2n$ where 

$$A^j_k =
\begin{cases}
A_k \cup S_j, & \text{if }k = i \\
A_k \setminus S_j, & \text{if }k \neq i.
\end{cases}$$
We show at least one of these allocations has a higher for $\mathsf{ex}^{2/5}$ than $\mathcal{A}$.
Since $\valu^{2/5}_i$ is XOS, by Lemma \ref{xos2lemma} we have
\begin{equation*}
\sum_{j=1}^{2n}\Big(\valu^{2/5}_k(A_k) - \valu^{2/5}_k(A_k \setminus S_j)\Big) \leq \valu^{2/5}_k(A_j)
\end{equation*}
for every $\agent_k \neq \agent_i$ and thus
\begin{equation}\label{gulu1}
\begin{split}
\sum_{j=1}^{2n} \valu^{2/5}_k(A^j_k) &= \sum_{j=1}^{2n} \valu^{2/5}_k(A_j \setminus S_j)\\
 &\geq 2n \valu^{2/5}_k(A_k) - \valu^{2/5}_k(A_k)\\
&= (2n-1)\valu^{2/5}_k(A_k)
\end{split}
\end{equation}
Moreover, since $\valu^{2/5}_i(A_i) < 1/5$, we have
\begin{equation}\label{gulu2}
\begin{split}
\sum_{\agent_j \neq \agent_i} \ceil{\valu_j}{2/5}(A_j) & > \sum_{\agent_j \in \agents} \ceil{\valu_j}{2/5}(A_j) - 1/5\\
&=\mathsf{ex}^{2/5}(\mathcal{A})-1/5.
\end{split}
\end{equation}
Furthermore, since $\valu_i^{2/5}(S_j \cup A_i) = 2/5$ for every $1 \leq j \leq 2n$, we have
\begin{equation}\label{gulu3}
\begin{split}
\sum_{\agent_k \neq \agent_i} \valu^{2/5}_k(A^j_k) &= \sum_{\agent_k \in \agents} \valu^{2/5}_k(A^j_k) - 2/5\\
&= \mathsf{ex}^{2/5}(\mathcal{A}^j) - 2/5
\end{split}
\end{equation}
Finally, by combining Inequalities \eqref{gulu1}, \eqref{gulu2}, and \eqref{gulu3} we have
\begin{equation*}
	\begin{split}
	\sum_{j=1}^{2n} \mathsf{ex}^{2/5}(\mathcal{A}^j) & = \sum_{j=1}^{2n}(2/5 +\sum_{\agent_k \neq \agent_i} \valu^{2/5}_k(A^j_k)) \\
	& = 4n/5 + \sum_{j=1}^{2n}\sum_{\agent_k \neq \agent_i} \valu^{2/5}_k(A^j_k)\\  
	&\geq 4n/5 + \sum_{\agent_k \neq \agent_i} (2n-1) \valu^{2/5}_k(A_k)\\
	&\geq 4n/5 + (2n-1)(\mathsf{ex}^{2/5}(\mathcal{A}) - 1/5)\\
	&\geq 2n \cdot \mathsf{ex}^{2/5}(\mathcal{A}) + (4n-2n+1)/5 - \mathsf{ex}^{2/5}(\mathcal{A})\\
	&\geq 2n \cdot \mathsf{ex}^{2/5}(\mathcal{A}) + (2n+1)/5 - \mathsf{ex}^{2/5}(\mathcal{A})
	\end{split} 
\end{equation*}
Now notice that since $\valu^{2/5}_k(A_k) \leq 2/5$, we have
\begin{equation*}
	\begin{split}
	\mathsf{ex}^{2/5}(\mathcal{A}) & = \sum_{k=1}^n \valu^{2/5}_k(A_k)\\
	& \leq \sum_{k=1}^n 2/5\\
	& \leq 2n/5. 
	\end{split}
\end{equation*}
and thus 
\begin{equation*}
	\begin{split}
	\sum_{j=1}^{2n} \mathsf{ex}^{2/5}(\mathcal{A}^j) & \geq 2n \cdot \mathsf{ex}^{2/5}(\mathcal{A}) + (2n+1)/5 - \mathsf{ex}^{2/5}(\mathcal{A})\\
	&\geq 2n \cdot \mathsf{ex}^{2/5}(\mathcal{A}) + (2n+1)/5 - 2n/5\\
	&\geq 2n \cdot \mathsf{ex}^{2/5}(\mathcal{A}) + 1/5.
	\end{split}
\end{equation*}
Therefore, $\mathsf{ex}^{2/5}(\mathcal{A}^j) > \mathsf{ex}^{2/5}(\mathcal{A})+1/10n$ holds for at least one $\mathcal{A}^j$ which contradicts the maximality of $\mathcal{A}$.
\end{proof}



\subsection{Algorithm}\label{xosalgorithm}
In this section we provide a polynomial time algorithm for finding a $1/8$-$\MMS$ allocation for the fair allocation problem with XOS agents. The algorithm is based on a similar idea that we argued for the proof of Theorem \ref{xosproof}. Remark that our algorithm only requires access to demand and XOS oracles. It does \textit{not} have any additional information about the maxmin values. This makes the problem computationally harder since computing the maxmin values is NP-hard~\cite{epstein2014efficient}. We begin by giving a high-level intuition of the algorithm and show the computational obstacles can be overcome by combinatorial tricks. Consider the pseudo-code described in Algorithm \ref{xosalg}.
\begin{algorithm}%[t!]
	\KwData{$\agents, \items, \langle \valu_1, \valu_2, \ldots, \valu_n\rangle$}
	For every $\agent_j$, scale $\valu_j$ to  ensure $\MMS_j = 1$\;
	\While{there exist an agent $\agent_i$ and an item $\ite_j$ such that $\valu_i(\{\ite_j\}) \geq 1/8$}{
		Allocate $\{\ite_j\}$ to $\agent_i$\;
		$\items = \items \setminus \ite_j$\;
		$\agents = \agents \setminus \agent_i$\;
	}
	$\mathcal{A} = $ an arbitrary allocation of the items to the agents\;
	\While{$\min \valu^{1/4}_j(A_j) < 1/8$}{
		$i = $ the agent who receives the lowest value in allocation $\mathcal{A}$\; 
		Find a set $S$ such that:\label{find}
		$\mathsf{ex}^{1/4}(\langle A_1 \setminus S, A_2 \setminus S, \ldots, A_{i-1} \setminus S, A_i \cup S, A_{i+1} \setminus S, \ldots, A_n \setminus S\rangle) \geq \mathsf{ex}^{1/4}(\mathcal{A})+1/12n$\;
		$\mathcal{A} = \langle A_1 \setminus S, A_2 \setminus S, \ldots, A_{i-1} \setminus S, A_i \cup S, A_{i+1} \setminus S, \ldots, A_n \setminus S\rangle$\;
	}
	For every $\agent_i \in \agents$ allocate $A_i$ to $\agent_i$\;
\caption{Algorithm for finding a $1/8$-$\MMS$ allocation}
\label{xosalg}
\end{algorithm}

As we show in Section \ref{former}, Command \ref{find} of the algorithm is always doable. More precisely, there always exists a set $S$ that holds in the condition of Command \ref{find}. Notice that in every step of the algorithm, $\mathsf{ex}^{1/4}(\mathcal{A})$ is increased by at least $1/12n$ and this value is bounded by $1/4\cdot n = n/4$. Therefore the algorithm terminates after at most $3n^2$ steps and the allocation is guaranteed to be $1/8$-$\MMS$.

That said, there are two major computational obstacles in the way of running Algorithm \ref{xosalg}. Firstly, finding a set $S$ that holds in the condition of Command \ref{find} can not be trivially done in polynomial time. Second, scaling the valuation functions to ensure $\MMS_i = 1$ for all agents is NP-hard and cannot be done in polynomial time unless P=NP. To overcome the former, in Section \ref{former} we provide an algorithm for finding such a set $S$ in polynomial time. Next, in Section \ref{latter}, we present a combinatorial trick to run the algorithm in polynomial time without having to deal with NP-hardness of scaling the valuation functions.

\subsubsection{Executing Command \ref{find} in Polynomial Time}\label{former}
In this section we present an algorithm to execute Command \ref{find} of Algorithm \ref{xosalg}. We show that such a procedure can be implemented via demand oracles.

Let for every $\ite_j \notin A_i$, $c_j$ be the amount of contribution that $\ite_j$ makes to $\mathsf{ex}^{1/4}(\mathcal{A})$. We set $p_e = 3(n/(n-1))c_e$ and ask the demand oracle of $\valu_i$ to find a set $S$ that maximizes $\valu_i(S) - \sum_{\ite_j \in S} p_j$. Via a trivial calculation, one can show that $\valu_i(S) - \sum_{\ite_j \in S} p_j \geq 1/4$ holds for at least one set of items. The reason this is correct is that one can divide the items into $n$ partitions where each is worth at least $1$ to $\agent_i$. Moreover, the summation of prices for the items is bounded by $3n/(n-1)\cdot (\sum_{j\neq i} \valu_j^{1/4} (A_j) )\leq 3n/4$. Therefore, for at least one of those partitions  $\valu_i(S) - \sum_{\ite_j \in S} p_j$ is at least $1/4$. Thus, the set that the oracle reports is worth at least $1/4$ to $\agent_i$. 

Now, let $S^*$ be the set that the oracle reports and for every $\ite_j \in S^*$, $c^*_j$ be the contribution of $\ite_j$ to $\valu_i(S^*)$. We sort the items of $S^*$ based on $c^*_j - p_j$ in non-increasing order. Next, we start with an empty bag and add the items in their order to the bag until the total value of the items in the bag to $\agent_i$ reaches $1/4$. Since the value of every item alone is bounded by $1/8$, the total value of the items in the bag to $\agent_i$ is bounded by $3/8$. Thus the contribution of those items to $\mathsf{ex}^{1/4}(\mathcal{A})$ is at most $(3/8) / (3n/(n-1)) \leq 1/8 - 1/(10n)$. Therefore, removing items of the bag from other allocations and adding them to $A_i$, increases $\mathsf{ex}^{1/4}(\mathcal{A})$ by at least $1/10n$.

Remark that one can use the same argument to prove this even if $\MMS_i \geq 1/(1+1/10n)$.

\subsubsection{Running Algorithm \ref{xosalg} in Polynomial Time}\label{latter}
As aforementioned, scaling valuation functions to ensure $\MMS_i = 1$ for every agent $\agent_i$ is an NP-hard problem since determining the maxmin values is hard even for additive agents~\cite{Procaccia:first}. Therefore, unlike Section \ref{former}, in this section we massage the algorithm to make it executable in polynomial time.

Suppose an oracle gives us the maxnmin values of the agents. Provided that we can run Command \ref{find} of Algorithm \ref{xosalg} in polynomial time, we can find a $1/8$-$\MMS$ allocation in polynomial time. Therefore, in case the oracle reports the actual maxmin values, the solution is trivial. However, what if the oracle has an error in its calculations? There are two possibilities: (i) Algorithm \ref{xosalg} terminates and finds an allocation which is $1/8$-$\MMS$ with respect to the reported maxmin values. (ii) The algorithm fails to execute Command \ref{find}, since no such set $S$ holds in the condition of Command \ref{find}. The intellectual merit of this section boils down to investigation of the case when algorithm fails to execute Command \ref{find}. We show, this only happens due to an overly high misrepresentation of the maxmin value for agent $\agent_i$. Note that $\agent_i$ is the agent who receives the lowest value in the last cycle of the execution.

\begin{observation}\label{beautiful}
	Given $\langle d_1, d_2, \ldots, d_n\rangle$ as an estimate for the maxmin values, if Algorithm \ref{xosalg} fails to execute Command \ref{find} for an agent $\agent_i$, then we have
	$$d_i \geq (1+1/10n) \MMS_i.$$
\end{observation} 
Proof of Observation \ref{beautiful} follows from the argument of Section \ref{former}. More precisely, as mentioned in Section \ref{former}, such a set $S$ exists, if $\MMS_i \geq 1/(1+1/10n)$. Thus, given that the procedure explained in Section \ref{former} fails to find such a set, one can conclude the the reported value for $\MMS_i$ is at least $(1/(1+1/10n))$ times its actual value. Based on Observation \ref{beautiful}, we propose Algorithm \ref{oracle} for implementing a maxmin oracle.

\begin{algorithm}%[t!]
	\KwData{$\agents, \items, \langle \valu_1, \valu_2, \ldots, \valu_n\rangle$}
	\For {every $\agent_i \in \agents$}{
		$d_i \leftarrow \valu_i(\items)$\;
	}
	\While{true}{
		Run Algorithm \ref{xosalg} assuming maxmin values are $d_1, d_2, \ldots, d_n$\;
		\If {the Algorithm fails to run Command \ref{find} for an agent $\agent_i$}{
			$d_i \leftarrow d_i / (1+1/10n)$\; 
		}\Else{
			Report the allocation and terminate the algorithm\;
		}
	}
	\caption{Implementing a maxmin oracle}
	\label{oracle}
\end{algorithm}

Note that in the beginning of the algorithm, we set $d_i = \valu_i(\items)$ which is indeed greater than or equal to $\MMS_i$. By Lemma \ref{beautiful}, every time we decrease the value of $d_i$ for an agent $\agent_i$, we preserve the condition $d_i \geq \MMS_i$ for that agent. Therefore, in every step of the algorithm, we have $d_i \geq \MMS_i$ and thus the reported allocation which is $1/8$-$\MMS$ with respect to $d_i$'s is also $1/8$-$\MMS$ with respect to true maxmin values. Thus, the algorithm provides a correct $1/8$-$\MMS$ allocation in the end. All that remains is to show the running time of the algorithm is polynomial.

Notice that every time we decrease $d_i$ for an agent $\agent_i$, we multiply this value by $1/(1+1/10n)$, hence the number of such iterations is polynomial in $n$, unless the valuations are super-exponential in $n$. Since we always assume the input numbers are represented by $\mathsf{poly}(n)$ bits, the number of iterations is bounded by $\mathsf{poly}(n)$ and hence the algorithm terminates after a polynomial number of steps.

\begin{theorem}\label{xa}
	Given access to demand and XOS oracles, there exists a polynomial time algorithm that finds a $1/8$-$\MMS$ allocation for XOS agents.
\end{theorem}

An elegant consequence of Theorem \ref{xa} is a $8$-approximation algorithm for determining the maxmin value of an XOS function with $r$ partitions.
\begin{corollary}
	Given an XOS function $f$, an integer number $r$, and access to demand and XOS oracles of $f$, there exists a $8$-approximation polytime algorithm for determining $\MMS^r_f$.
\end{corollary}
\begin{proof}
	We construct an instance of the fair allocation problem with $r$ agents, all of whom have a valuation function equal to $f$. We find a $1/8$-$\MMS$ allocation of the items to the agents in polynomial time and report the minimum value that an agent receives as output.
	
	The $1/8$ guarantee follows from the fact that every agent receives a subset of values that are worth $1/8$-$\MMS_i$ to him, and since $\MMS_i$ is exactly equal to $\MMS^r_f$, every partition has a value of at least $\MMS^r_f/8$.
\end{proof}
\begin{remark}
	A similar procedure can also be used to overcome the challenge of computing the maxmin values for the algorithm described in Section \ref{submodularalgorithm}.
\end{remark}
\section{Subadditive Agents}\label{subadditive}
In this section we present a reduction from subadditive agents to XOS agents. More precisely, we show for every subadditive set function $f(.)$, there exists an XOS function $g(.)$, where $g$ is dominated by $f$ but the maxmin value of $g$ is within a logarithmic factor of the maxmin value of $f$. We begin by an observation. Suppose we are given a subadditive function $f$ on set $\domp(f)$, and we wish to approximate $f$ with an additive function $g$ which is dominated by $f$. In other words, we wish to find an additive function $g$ such that 
$$\forall S \subseteq \domp(f) \hspace{1cm} g(S) \leq f(S)$$
and $g(\domp(f))$ is maximized. One way to formulate $g$ is via a linear program. Suppose $\domp(f)=\{\ite_1,\ite_2,\ldots,\ite_m\}$ and let $g_1, g_2, \ldots, g_m$ be $m$ variables that describe $g$ in the following way:
$$\forall S \subseteq \domp(f) \hspace{1cm} g(S) = \sum_{\ite_i \in S} g_i.$$
Based on this formulation, we can find the optimal additive function $g$ by LP \ref{lp1}.
\begin{alignat}{3}\label{lp1}
\text{maximize: }& \hspace{0.5cm} &  \sum_{\ite_i \in \domp(f)}g_i & &\\
\text{subject to: }& & \sum_{\ite_i \in S}g_i \leq f(S) & \hspace{1cm}&\forall S \subseteq \domp(f)\nonumber\\
& & g_i \geq 0 & &\forall \ite_i \in \domp(f)\nonumber
\end{alignat}
We show the objective function of LP \ref{lp1} is lower bounded by $f(\domp(f)) / \log m$. The basic idea is to first write the dual program and then based on a probabilistic method, lower bound the optimal value of the dual program by $f(\domp(f))/ \log m$. 
\begin{lemma}\label{jj1}
	The optimal solution of LP \ref{lp1} is at least $f(\domp(f))/ \log m$.
\end{lemma}
\begin{proof}
To prove the lemma, we write the dual of LP \ref{lp1} as follows:
\begin{alignat}{3}\label{lp2}
\text{minimize: }& \hspace{0.5cm}& \sum_{S \subseteq \domp(f)} \alpha_S f(S)   & &\\
\text{subject to: }& & \sum_{S \ni \ite_i} \alpha_S \geq 1 & \hspace{1cm}&\forall \ite_i \in \domp(f)\nonumber\\
& & \alpha_S \geq 0 & &\forall S \subseteq \domp(f)\nonumber
\end{alignat}
By the strong duality theorem, the optimal solutions of LP \ref{lp1} and LP \ref{lp2} are equal~\cite{bachem1992linear}. Next, based on the optimal solution of LP \ref{lp2}, we define a randomized procedure to draw a set of elements: We start with an empty set $S^*$ and for every set $S \subseteq \domp(f)$ we add \textit{all} elements of $S$ to $S^*$ with probability $\alpha_S$. Since $f$ is subadditive, the marginal increase of $f(S^*)$ by adding elements of a set $S$ to $S^*$ is bounded by $f(S)$ and thus the expected value of $f(S^*)$ is bounded by the objective of LP \ref{lp2}. In other words:
\begin{equation}\label{jef0}
\mathbb{E}[f(S^*)] \leq \sum_{S \subseteq \domp(f)} \alpha_S f(S)
\end{equation}
Remark that we repeat this procedure for all subsets of $\domp(S)$ independently and thus for every $\ite_i \in \domp(f)$, $\sum_{S \ni \ite_i} \alpha_S \geq 1$ holds we have
\begin{equation}\label{jef1}
\mathsf{PR}[\ite_i \in S^*] \geq 1-1/e \simeq 0.632121 > 1/2
\end{equation}
for every element $\ite_i \in \domp(s)$. Now, with the same procedure, we draw $\lceil \log m \rceil + 2$ sets $S^*_1, S^*_2, \ldots, S^*_{\lceil \log m \rceil + 2}$ \textit{independently}. We define $\hat{S} = \bigcup S^*_i$. By Inequality \eqref{jef1} and the union bound we show
\begin{equation*}
\begin{split}
\mathsf{PR}[\hat{S} = \domp(f)] & \geq 1- \sum_{\ite_i \in \domp(i)} \mathsf{PR}[\ite_i \notin \hat{S}]\\
& = 1- \sum_{\ite_i \in \domp(i)} \mathsf{PR}[\ite_i \notin S^*_1 \text{ and } \ite_i \notin S^*_1 \text{ and } \ldots \text{ and } \ite_i \notin S^*_{\lceil \log m \rceil + 2}]\\
& = 1- \sum_{\ite_i \in \domp(i)} \prod_{j=1}^{\lceil \log m \rceil + 2} \mathsf{PR}[\ite_i \notin S^*_j]\\
& \geq 1- \sum_{\ite_i \in \domp(i)} \prod_{j=1}^{\lceil \log m \rceil + 2} 1/2\\
& = 1- \sum_{\ite_i \in \domp(i)} \prod_{j=1}^{\lceil \log m \rceil + 2} \mathsf{PR}[\ite_i \notin S^*_j]\\
& \geq 1- \sum_{\ite_i \in \domp(i)} 1/4m\\
& = 1- 1/4\\
& = 3/4\\
\end{split}
\end{equation*}
and thus $\mathbb{E}[f(\hat{S})] \geq 3/4 f(\domp(f))$. On the other hand, by the linearity of expectation and the fact that $f$ is subadditive we have:
\begin{equation*}
\begin{split}
\mathbb{E}[f(\hat{S})] &= \mathbb{E}[f(\bigcup S^*_i)]\\
& \leq \mathbb{E}[\sum f(S^*_i)]\\
& \leq (\lceil \log m \rceil + 2) (\sum_{S \subseteq \domp(f)} \alpha_S f(S))
\end{split}
\end{equation*}
Therefore $\sum_{S \subseteq \domp(f)} \alpha_S f(S) \geq 3/4 f(\domp(f)) / (\lceil \log m \rceil + 2)$, which means $$\sum_{S \subseteq \domp(f)} \alpha_S f(S) \geq f(\domp(f)) / (2\lceil \log m \rceil)$$ for big enough $m$. This shows the optimal solution of LP \ref{lp1} is lower bounded by $f(\domp(f)) / (2\lceil \log m \rceil)$ and the proof is complete.
\end{proof}

In what follows, based on Lemma \ref{jj1}, we provide a reduction from subadditive agents to XOS agents. An immediate corollary of Lemma \ref{jj1} is the following:
\begin{corollary}[of Lemma \ref{jj1}]\label{kk}
For any subadditive function $f$ and integer number $n$, there exists an XOS function $g$ such that
$$g(S) \leq f(S) \qquad \forall S \subseteq \domp(f)$$
and 
$$\MMS_g^n \geq \MMS_f^n/2\lceil \log n \rceil.$$
\end{corollary} 
\begin{proof}
	By definition, we can divide the items into $n$ disjoint sets such that the value of $f$ for every set is at least $\MMS_f^n$. Now, based on Lemma \ref{jj1}, we approximate $f$ for each set with an additive function $g_i$ wile losing a factor of at most $\lceil 2 |\log \domp(f)|\rceil$ and finally we set $g = \max g_i$. Based on Lemma \ref{jj1}, both conditions of this lemma are satisfied by $g$.
\end{proof}

Based on Theorem \ref{xosproof} and Lemma \ref{kk} one can show that a $1/10\lceil \log m \rceil$-$\MMS$ allocation is always possible for subadditive agents.
\begin{theorem}\label{subadditiveproof}
	The fair allocation problem with subadditive agents admits a $1/10\lceil \log m \rceil$-$\MMS$ allocation.
\end{theorem}
\section{Acknoledgment}
We would like to thank the anonymouse reviewers for their thoughtful comments and direction.
%This document was prepared by the LArIAT collaboration using the resources of the Fermi National Accelerator Laboratory (Fermilab), a U.S. Department of Energy, Office of Science, HEP User Facility. Fermilab is managed by Fermi Research Alliance, LLC (FRA), acting under Contract No. DE-AC02-07CH11359. We also gratefully acknowledge the support of the National Science Foundation; Brazil CNPq grant number 233511/2014-8, Coordena\c{c}\~ao de Aperfei\c{c}oamento de Pessoal de N\'ivel Superior - Brazil (CAPES) - Finance Code 001, S\~ao Paulo Research Foundation - FAPESP (BR) grant number 16/22738-0; Polish National Science Centre grant Dec-2013/09/N/ST2/02793; the Science and Technology Facilities Council (STFC), part of the United Kingdom Research and Innovation; The Royal Society (United Kingdom); and the JSPS grant-in-aid (Grant Number 25105008), Japan. The collaboration extends a special thank you to the coordinators and technicians of the Fermilab Test Beam Facility, without whom none of this work would have been possible.

%The LArIAT collaboration uses the resources of the Fermi National Accelerator Laboratory (Fermilab), a U.S. Department of Energy, Office of Science, HEP User Facility. Fermilab is managed by Fermi Research Alliance, LLC (FRA).%, acting under Contract No. DEAC02-07CH11359.This research was supported by the U.S. Department of Energy; the U.S. National Science Foundation.%; the Department of Science and Technology, India; the European Research Council; the MSMT CR, GA UK, Czech Republic; the RAS, RMES, and RFBR, Russia; CNPq and FAPEG, Brazil. 



%%\section{A Note for The Reviewers}\label{defend}
We received some controversial critiques, among many positive comments, from FOCS'16 that we would like to respond to.
 \textit{''The complexity of the algorithm was seen as a minus and it was hard to get any new insights from it''}: such comments are very rarely (if ever) given especially in venues such as FOCS and STOC. Of course the solution is not trivial, nonetheless we do not believe it is fair to see this as a minus for a paper. In Section \ref{overview}, we clearly outline the algorithm, how it has been derived from previous works, and how it can be further improved to achieve better bounds.
 With regard to the presentation, we would like to note that we received positive feedback on an earlier version of the paper submitted to FOCS'16: \textit{''Although it is very complex, the authors make a good job in explaining the ideas and I like them."}. However, we further improved the presentation style (abstract, intro, and technical parts), based on the comments we received from other reviewers.
 Following on a comment on the applicability of our algorithms, we would like to note that our algorithms are efficient and easy to implement. What makes the paper slightly hard to understand is the analysis of the algorithms. You can find an implemented version of the algorithms at \href{}{text}.

%
\vspace{-3mm}
\section{Conclusion}
\label{sec:conc}
\vspace{-2mm}

%In this paper, we bring a human in the loop to  facilitate the learning process of an image captioning model. 
In this paper, we enable a human teacher to provide feedback to the learning agent in the form of natural language. We focused on the problem of image captioning. 
We proposed a hierarchical phrase-based RNN as our captioning model, which allowed natural integration with human feedback. We crowd-sourced feedback for a snapshot of our model, and showed how to incorporate it in Policy Gradient optimization. We showed  that by exploiting descriptive feedback our model learns to perform better than when given independently written captions. %In the future, we aim to explicitly deal with the annotator 

%There are several exciting avenues for future work. In particular, when dealing with crowd-sourced human reward one needs to take into account annotator noise. We further plan to explore self-criticism, where the agent automatically decides when to seek human advice, possibly in a dialog-like setting. 

\vspace{-3mm}
\section*{Acknowledgment}
\vspace{-3mm}
\begin{small}
We gratefully acknowledge the support from NVIDIA for their donation of the GPUs used for this research. This work was partially  supported by NSERC. We also thank Relu Patrascu for infrastructure support. \end{small}



	
%	\bibliographystyle{alpha}
%	\bibliographystyle{cell}
	
\bibliographystyle{abbrv}
	
\bibliography{frugal}
\newpage

\appendix
\section{A $4/5$-$\MMS$ Allocation for Four Agents}\label{45}
%\color{magenta}
In this section we propose an algorithm to find a $4/5$-$\MMS$ for $n=4$ in the additive setting. Since the number of agents is exactly 4, we assume $\agents = \{\agent_1, \agent_2, \agent_3, \agent_4\}$. Again, for simplicity we assume $\MMS_i = 1$ for every $\agent_i \in \agents$. In general, our algorithm is consisted of  three main steps: first, $\agent_1$ optimally partitions the items into $4$ bundles with values at least 1 to him. Next, $\agent_2$ selects three of the bundles and repartitions them. Finally, we satisfy one of $\agent_1$ or $\agent_2$ with a bundle and solve the problem for remaining agents and items via Lemma ~\ref{recurse}. Without loss of generality, we assume the valuation of every agent $\agent_i$ for every bundle in his optimal $n$-partitioning is exactly equal to 1. Therefore, from here on, we assume that the summation of the values of the items within the same bundle for every agent is at most $1$. In addition, we suppose that the problem is $4/5$-irreducible, since Lemma \ref{reducibility} narrows down the problem into such instances. Thus, by Lemma \ref{remove1}, the value of every item is less than $4/5$ to any agent. 
We begin this section by stating a number of definitions and observations. In this section, we use the term \emph{bundle} to refer to a set of items.
\begin{definition}
\label{perfect}
A set $S$ of bundles is perfect for a set $T$ of agents, if $(i)$ $|S| = |T|$ and $(ii)$ there exists an allocation of the bundles in $S$ to the agents of $T$ such that all the agents in $T$ are satisfied by their allocated bundle.
\end{definition}



\begin{observation}\label{firstobs}
\label{sum}
Let $\agent_i$ be an agent and $S$ be a set of items where $\valu_i(\{\ite_j\}) \leq v$ for every item $\ite_j \in S$. If $V_i(S) > v$, then there exists a subset $S' \subseteq S$ of items such that $ v \leq \valu_i(S') < 2v$.
\end{observation}
\begin{proof}
We begin with an empty set $S'$ and add the items of $S$ to $S'$ one by one, until $\valu_i(S')$ exceeds $v$. Before adding the last element to $S'$, the valuation of $\agent_i$ for $S'$ was no more than $v$ and every item alone is of value less than $v$ to $\agent_i$. Therefore, after adding the last item to $S’$, its value is less than $2v$ to $\agent_i$.
\end{proof}

\begin{definition}
\label{core}
For a bundle $B$ of items that satisfies $\agent_i$, the core of $B$ with respect to agent $\agent_i$, denoted by $C_i(B)$, is defined as follows: let $m_1,m_2,..,m_k$ be the items of $B$ in the increasing order of their values to $\agent_i$. Then $C_i(B) = \{m_j,m_{j+1},...,m_{k}\}$ , where $j$ is the highest index, such that set of items $\{m_j,m_{j+1},...,m_k\}$ satisfies $\agent_i$.
\end{definition}

Note that for every subset $B$ with $\valu_i(B) \geq 4/5$, $C_i(B)$ is a subset of $B$ with the minimum size that satisfies $\agent_i$. Since the items in $C_i(B)$ satisfy $\agent_i$, we have $\valu_i(C_i(B))\geq 4/5$. On the other hand, by the fact that $|C_i(B)|$ is minimal, removing any item from $C_i(B)$ results in a subset that no longer satisfies $\agent_i$. Thus, Observation \ref{eps2} holds.

\begin{observation}\label{eps2} 
If $\valu_i(C_i(B))  = 4/5 + \beta$, then the value of every item in $C_i(B)$ is more than $\beta$ for $\agent_i$.
\end{observation}

By the fact the value of every item in $B$ is less than $4/5$ we have Observation \ref{cre}.

\begin{observation}\label{cre}
For every agent $\agent_i$ and any subset $B$ of items, $\valu_i(C_i(B))<8/5$.
\end{observation}


\begin{lemma}
\label{3p}
Suppose that $S = \{X,Y,Z\}$ is a 3-partitioning of a set of items with the following properties for an agent $\agent_i$: 

\begin{minipage}[t]{\linegoal}
\begin{enumerate}[leftmargin=*]
\item $\valu_i(X) < 4/5$ and $\valu_i(Y)<4/5$. 

\item $V_i(X\cup Y \cup Z) >16/5$. 
\end{enumerate}
\end{minipage}
\\

Then we can move some items from $Z$ to an arbitrary bundle of $\{X, Y\}$, such that, both $Z$ and the corresponding bundle will be worth at least $4/5$ to $\agent_i$. 

\end{lemma}

\begin{proof}
Since $V_i(X \cup Y \cup Z ) > 16/5$, $V_i(Z) > 8/5$ holds. Moreover, by Observation \ref{cre}, we have $\valu_i(C_i(Z)) \leq 8/5$. Considering $Z' = Z \setminus C_i(Z)$, we have   
$$\valu_i(X \cup Y \cup Z' ) \geq 8/5.$$
According to the fact that $\valu_i(X)<4/8$ and $\valu_i(Y)<4/5$, we have 
$$\valu_i(X \cup Z' ) \geq 4/5$$
and 
$$\valu_i(Y \cup Z' ) \geq 4/5.$$


\end{proof}

\begin{lemma}
\label{recurse}
Let $S= \{X,Y,Z\}$ be a set of three bundles of items, such that 

\begin{minipage}[t]{\linegoal}
\begin{enumerate}[leftmargin=*]
 \item $V_1(X) \ge 4/5 , V_1(Y) \ge 4/5, V_1(Z) \ge 4/5 $. 
\item $V_2(X \cup Y \cup Z) > 16/5$.
\item $V_3(X \cup Y \cup Z) \ge 3 $.
 
\end{enumerate}
\end{minipage}
\\

Then a $4/5$-$\MMS$ allocation of $X \cup Y \cup Z$ to the agents $\agent_1,\agent_2,\agent_3$ is possible.
\end{lemma}
\begin{proof}
If $\agent_2$ can be satisfied with two different bundles, then trivially $S$ is perfect. Otherwise, $\agent_2$ is satisfied with only one bundle, say $Z$. By Lemma ~\ref{3p}, $\agent_2$ can transfer some items from $Z$ to $Y$, such that both bundles satisfy him.  After moving the items, both $Y$ and $Z$ satisfy $\agent_2$, and bundle $X$ and $Y$, satisfy $\agent_1$. One the other hand,   since $V_3(X \cup Y \cup Z) \ge 3 $, $\agent_3$ is satisfied with at least one bundle. Its easy to observe that for any valuation of bundles $X,Y,Z$ for $\agent_3$, the set of bundles is perfect.
\end{proof}


\begin{lemma}
\label{core}
Let $S=\{X,Y,Z,T\}$ be a 4-partitioning of $\cal M$ and $\agent_i$ be an arbitrary agent. Then $\agent_i$ can select $3$ bundles and re-partition them into three new bundles in such a way that each bundle will be worth at least $4/5$ to $\agent_i$.
\end{lemma}
\begin{proof}
Consider bundles $X,Y,Z,T$. If more than two of them satisfy $\agent_i$, then the selection is trivial. Furthermore, if only one bundle satisfies $\agent_i$, then by Lemma ~\ref{3p}, we can move some items from the satisfying bundle to another bundle, such that both bundles satisfy $\agent_i$. Thus, without loss of generality, we assume that bundles $Z$ and $T$ satisfy $\agent_i$. 


Let $Z' = Z \setminus C_i(Z)$ and $T'= T \setminus C_i(T)$. Without loss of generality, we assume $V_i(X) \geq V_i(Y)$ and let $X' = X \cup Z' \cup T'$. If $V_i(X') \ge 4/5$, then the proof is trivial. Thus, suppose that $V_i(X') < 4/5$.

Consider the value of bundles as $$V_i(X’)=4/5 - \epsilon_1,$$ $$V_i(Y)=4/5 - \epsilon_2,$$ $$V_i(C_i(Z))=6/5 + \epsilon_3,$$ and $$V_i(C_i(T))=6/5 + \epsilon_4$$ where $\epsilon_1 \leq \epsilon_2$ and $\epsilon_3 \leq \epsilon 4$. Note that $\epsilon_3$ can be negative. By the fact that total value of the items equals $4$ for all of the agents, we assume that $$\epsilon_1 + \epsilon_2 = \epsilon_3 + \epsilon_4$$ and hence $$\epsilon_1 \leq \epsilon_4.$$  


Now, we explore the properties of the items in $C_i(T)$. Regarding Observation \ref{eps2}, every item in $C_i(T)$ is worth more than $2/5+\epsilon_4$ to $\agent_i$. Hence, $C_i(T)$ cannot contain more than 2 items, since value of every pair of items in $C_i(T)$ is more than $4/5$. Moreover, $C_i(T)$ cannot contain one item and hence, $C_i(T)$ contains exactly two items. Let $\ite_1$ and $\ite_2$ be these two items. Since $\valu_i(T) = 6/5 + \epsilon_4$, at least one of these two items, say $\ite_2$ is worth at least $3/5+\epsilon_4/2$ to $\agent_i$. Thus, in summary, $C_i(T)$ contains two items $\ite_1$ and $\ite_2$ with

$$\valu_i(\{\ite_1\}) >2/5 + \epsilon_4,$$
$$\valu_i(\{\ite_2\}) \geq 3/5+\epsilon_4/2.$$

Next, we characterise the items in $X'$. For Bundle $X'$, let $B$ be the set of items with a value less than $1/5 - \epsilon_4/2$. If $V_i(B)\geq 1/5 - \epsilon_4/2$, then Observation ~\ref{sum} states that there exists a subset $B'$ of $B$, such that:
$$1/5 - \epsilon_4/2 \leq V_i(B') < 2/5 - \epsilon_4.$$

Therefore, Bundles $B' \cup \{b_2\}$ and $(X' \setminus B') \cup \{b_1\}$ satisfy $\agent_i$. These two bundles together with  $C_i(Z)$ result in three bundles that satisfy $\agent_i$. Thus, $V_i(B) < 1/5 - \epsilon_4/2$. 

Finally, regarding the fact that $V_i(B) < 1/5 - \epsilon_4/2$, we have 
$$\valu_i(X' \setminus B)> 3/5  -\epsilon_1 + \epsilon_4/2 $$

For this case, we show that $X'\setminus B$ contains exactly one item. Otherwise, at least one of these items, say $\ite_3$, is worth less than $3/10 - \epsilon_1/2 + \epsilon_4/4$ and therefore, for the bundles $\{\ite_3\} \cup \{\ite_2\}$ and $(X' \setminus \{\ite_3\}) \cup \{\ite_1\}$  we have:
$$\valu_i(\{\ite_3\} \cup \{\ite_2\}) \geq 1/5 - \epsilon_4/2 + 3/5+\epsilon_4/2 \geq 4/5$$
and 
$$\valu_i((X' \setminus \{\ite_3\}) \cup \{\ite_1\}) \geq 4/5 - \epsilon_1 - 3/10 + \epsilon_1/2- \epsilon_4/4+ 2/5 + \epsilon_4 \geq 4/5$$
respectively. These two bundles along with $C_i(Z)$ form $3$ bundles that satisfy $\agent_i$. Therefore, we conclude that $X' \setminus B$ contains an item $\ite_3$ with a value more than $3/5-\epsilon_1  + \epsilon_4/2$ to $\agent_i$. The rest of the items in $X'$ belong to $B$ that are in total worth less than $1/5-\epsilon_4/2$.

Note that $X \subseteq X'$. Therefore, consider the $4-$partitioning of $\agent_i$ and remove the bundle containing $\ite_3$. Also, remove the items with value less than $1/5 - \epsilon_4/2$ to $\agent_i$ in $X$, from their corresponding bundles. Three bundles with value of each to $\agent_i$ more than $$1 - 1/5 + \epsilon_4/2 \geq 4/5,$$ with all of their items from $Y,Z$ and $T$ remain. Thus, $\agent_i$ can make three satisfying bundles with items in $Y,Z,T$. 
\end{proof}
Based on what we showed so far, we prove Theorem \ref{45main}.
\begin{theorem}
\label{45main}
A $4/5$-$\MMS$ allocation for $n=4$ is possible in the additive setting.
\end{theorem}
\begin{proof}
Consider the optimal $4$-partitioning of $\cal M$ with respect to $\agent_1$. Now,  ask $\agent_2$ to select $3$ bundles and re-partition them, such that he can be satisfied with all the three bundles. Based on Lemma \ref{core}, such a repartitioning is always possible. Due to the Pigeonhole principle, at least one of these three bundles still satisfies $\agent_1$. Let $S = \{X,Y,Z,T\}$ be the resulting bundles and without loss of generality, suppose that bundles $X,Y$ satisfy $\agent_1$ and bundles $Y,Z,T$ satisfy $\agent_2$.

Now, consider agents $\agent_3$ and $\agent_4$ and let $\phi$ be the set of bundles that satisfy $\agent_3$ or $\agent_4$. There are only two cases, in which $S$ is not perfect 
(recall definition \ref{perfect}):

\begin{enumerate}
\item $\phi \subseteq \{X,Y\}$ : $\agent_1$, selects three bundles $X,Y$ and one of $Z$ or $T$, say $Z$ and re-partitions them to three satisfying bundles. Now, give bundle $T$ to $\agent_2$. According to Lemma  ~\ref{recurse}, items of $X,Y,Z$ can satisfy the remaining three agents.

\item $|\phi|=1, \phi \notin \{X,Y\}$ : give $X$ to $\agent_1$ and allocate the items of $Y \cup Z \cup T$ to $\agent_2,\agent_3,\agent_4$, using Lemma ~\ref{recurse}.
\end{enumerate}
\end{proof}
\color{black}
\section{Omitted Proofs of Section \ref{additive:observations}}\label{additiveobservationsproof}

\begin{proof}[of Lemma \ref{remove1}]
The key idea is that given $\MMS_i \geq 1$ for an agent $\agent_i$, then for every item $\ite_j \in \items$ we have 
$\MMS_i^{n-1}(\items\setminus\ite_j) \geq 1$. This holds since removing an item from $\items$ will diminish the value of at most one partition in the optimal $n$ partitioning of the items. Therefore, at least $n-1$ partitions have a value of $1$ or more to $\agent_i$ and thus $\MMS_i^{n-1}(\items\setminus\ite_j) \geq 1$.
The rest of the proof follows from the definition of $\alpha$-irreducibility. If the valuation of an item $\ite_j$ to an agent $\agent_i$ is at least $\alpha$, then the problem is $\alpha$-reducible since if we allocate $\ite_j$ to $\agent_i$, we have 
$$\MMS_{\valu_k}^{n-1}(\items \setminus \{\ite_j\}) \geq 1$$
for every agent $\agent_k \neq \agent_i$. This contradicts with the $\alpha$-irreducibility assumption.
\end{proof}

\begin{proof}[of Lemma \ref{remove2}]
Suppose for the sake of contradiction that for every agent $\agent_{i'} \neq \agent_i$ we have $\valu_{i'}(\{\ite_j,\ite_k\}) \leq 1$. By this assumption, we show 
\begin{equation}
 \MMS_{i'}^{n-1}(\items\setminus \{\ite_j,\ite_k\}) \geq 1 \label{saeed1}
\end{equation} holds. This is true since removing two items $\ite_j$ and $\ite_k$ from $\items$ decreases the value of at most two partitions of the optimal partitioning of $\items$ for $\MMS_{i'}$. If $n-1$ partitions remain intact, then Inequality \eqref{saeed1} trivially holds. If not, merging the two partitions that initially contained $\ite_j$ and $\ite_k$ results in a partition with value at least $1$ to $\agent_i$. This partition together with the $n-2$ remaining partitions result in a desirable partitioning of $\items$ into $n-1$ partitions. Therefore, Inequality \eqref{saeed1} holds for any agent $\agent_{i'}$, and this implies that by allocating $S = \{\ite_j, \ite_k\}$ to $\agent_i$, not only does $\valu_i(S) \geq 3/4$ hold, but also for every $\agent_{i'} \neq \agent_i$ we have 
$$\MMS_{i'}^{n-1}(\items\setminus \{\ite_j,\ite_k\}) \geq 1$$
which means the problem is $3/4$-reducible, and it contradicts our assumption.
\end{proof}

\begin{proof}[of Lemma \ref{remove3}]
The proof for this lemma is obtained by applying Lemma \ref{remove2}, $|T|$ times. Consider an agent $\agent_i \notin T$. According to the argument in Lemma \ref{remove2}, if we assign $\ite_{{j_1}}$ and $\ite_{{j_2}}$ to $\agent_{i_1}$, $\agent_i$ can partition the items in $\items \setminus \{ \ite_{{j_1}}$ $\ite_{{j_2}}\}$ into $n-1$ partitions with value at least 1 to $\agent_i$, i.e.
$$\MMS_{i}^{n-1}(\items\setminus \{\ite_{j_1},\ite_{j_2}\}) \geq 1.$$
By the same deduction, after assigning $\ite_{{j_3}}$ and $\ite_{{j_4}}$ to $\agent_{i_2}$, we have
$$\MMS_{i}^{n-2}(\items\setminus \{\ite_{j_1},\ite_{j_2},\ite_{j_3},\ite_{j_4}\}) \geq 1.$$
By repeating above argument $|T|$ times, we have:
$$\MMS_{i}^{n-|T|}(\items\setminus S) \geq 1.$$

On the other hand, by condition $(II)$, every agent $\agent_{i_k}$ satisfies with items $\ite_{j_{2k-1}}$ and $\ite_{j_{2k}}$. This means that we can reduce the instance by satisfying the agents in $T$ by the items in $S$, which is a contradiction by the irreducibility assumption.


\begin{comment}
We show that if all the conditions are met, the instance is $3/4$-reducible. To show this, let $\mathcal{A}$ be an allocation of $S$ to the agents in $T$ such that all of the conditions are met. Suppose that $\mathcal{A} = \langle A_{i_1}, A_{i_2}, \ldots, A_{i_{|T|}}\rangle$ where $A_{i_a} = \{\ite_{j_{2a-1}},\ite_{j_{2a}}\}$ for every $\agent_{i_a} \in T$. According to  condition (ii), after we allocate the items, every $\agent_{i_a} \in T$ has a value of at least $3/4$ for the items assigned to him. Now, if we prove that for every $\agent_i \notin T$ we have $\MMS_{V_i}^{n-|T|} (\items \setminus S)\geq 1$, our problem will be $3/4$-reducible which contradicts our assumption. \\
Suppose there is at least one agent $\agent_i \notin T$ such that $\MMS_{V_i}^{n-|T|} (\items \setminus S) < 1$. 
Before removing $S$, $\agent_i$ can divide $\items$ into $n$ parts ${\cal P} =\langle P_1, P_2, \ldots, P_n \rangle$ such that $V_i(P_k) \geq 1$ for every $k$. Suppose that $\ite_{j_1}$ and $\ite_{j_2}$ are in parts $P_{k_1}$ and $P_{k_2}$ respectively. As a first step, we start a procedure of removing the items of $S$ with $\ite_{j_1}$ and $\ite_{j_2}$ from $P_{k_1}$ and $P_{k_2}$. If $k_1 = k_2$, after this removal, the other $n-1$ parts will have a value of at least $1$ to $\agent_i$. Also, we add the rest of the items of $P_{k_1}$ to another arbitrary part. 

If $k_1 \neq k_2$, we add all of the remaining items in $P_{k_1}$ to $P_{k_2}$. According to the third condition we have $\valu_{i}(\{\ite_{j_{1}},\ite_{j_{2}}\}) \leq 1$. Therefore, after adding the rest of the items of $P_{k_1}$ to $P_{k_2}$, $\valu_{i}(P_{k_2})$ is at least $1$. Hence, after the first step we have $n-1$ parts with value at least $1$ to $\agent_i$. \\
We do the same in all of the $|T|$ steps. Before starting $q$-th step, we have $n-q+1$ remaining parts. In the $q$-th step we remove $\ite_{j_{2q-1}}$ and $\ite_{j_{2q}}$ from their corresponding parts, which are $P_{k_{2q-1}}$ and $P_{k_{2q}}$ respectively. If $k_{2q-1} = k_{2q}$, we add the rest of the items of $P_{k_{2q-1}}$ to another arbitrary part. Otherwise, we add the remaining items of $P_{k_{2q-1}}$ to $P_{k_{2q}}$. Since $\valu_{i}(\{\ite_{j_{2q-1}},\ite_{j_{2q}}\}) \leq 1$, after the $q$-th step, we have $n-q$ parts with a value of at least $1$ to $\agent_i$. Hence, after the $|T|$-th step, we have removed all of the items of $S$, and all of the $n-|T|$ remaining parts have a value of at least $1$ to $\agent_i$. Therefore, $\MMS_{V_i}^{n-|T|} (\items \setminus S) \ge 1$ which is a contradiction.
\end{comment}
\end{proof}


\begin{proof}[of Lemma \ref{rem}]
We define $\parttwo_1$ as the set of vertices in $\parttwo$ that are not saturated by $M$, and $\parttwo_2$ as the set of vertices in $\parttwo$ that are connected to $\parttwo_1$ by an alternating path. Moreover, let $\partone_2 = M(\parttwo_2)$. By definition, $F_{H}(M,\partone) = \partone \setminus \partone_2$ (See Figure \ref{fig:FG}). As discussed before, all the vertices in $\partone_2$ are saturated by $M$. Consequently, all the vertices of $T$ are saturated by $M$ and $|N(T)| \geq |T|$. 

Let $M(T)$ be the set of vertices which are matched to the vertices of $T$ in $M$. We know that every vertex of $T$ is present in at least one of the alternating paths which connect $\parttwo_1$ to $\parttwo_2$. Let $$P = \langle \hat{y}_0, \hat{x}_1, \hat{y}_1, \hat{x}_2, \hat{y}_2, \ldots, \hat{x}_k, \hat{y}_k \rangle$$ be one of these paths that includes at least one of the vertices of $T$. Since $P$ is an alternating path which connects $\parttwo_1$ to $\parttwo_2$, $\hat{y}_0 \in \parttwo_1$ (see Figure \ref{fig:FG4}). In addition, according to the definition of alternating path, every edge $(\hat{x}_j,\hat{y}_j)$ of $P$ belongs to $M$ and every edge $(\hat{x}_j,\hat{y}_{j-1})$ does not belong to $M$. 

Let $\hat{x}_i$ be the first vertex of $T$ that appears in $P$. We know that the edge $(\hat{x}_i,\hat{y}_{i-1})$ does not belong to $M$. On the other hand, since $\hat{x}_i$ is the first vertex of $T$ in $M$, $\hat{x}_{i-1} \notin T$. Note that $\hat{y}_{i-1}$ does not belong to $M(T)$, since every vertex of $M(T)$ is matched with a vertex of $T$ in $M$ and $(\hat{x}_{i-1},\hat{y}_{i-1})$ is in $M$.  The fact that $\hat{y}_{i-1} \notin M(T)$ means $N(T)$ contains at least one vertex that is not in $M(T)$. Since all the vertices in $M(T)$ are in $N(T)$, $|N(T)|>|M(T)|$ and hence, $|N(T)|>|T|$.
\begin{figure}
\centering
\includegraphics[scale=0.6]{figs/matching2}
\caption{Alternating path $P$ connects ${\hat{\cal Y}}_1$ to ${\hat{\cal Y}}_2$ and intersects $T$}
\label{fig:FG4}
\end{figure}
\begin{comment}
Suppose $x_i$ is the first vertex of $T$ in $P$. Since $x_1 \in \parttwo_1$, $x_i \neq x_1$. By the assumption that $x_{i}$ is the first vertex of $T$ in $P$, $x_{i-1}$ does not belong to $T$. In addition, $x_{i-1}$ is connected to $x_i$, since $(x_i,x_{i-1})$ is an edge of $P$. Thus, the vertices in $T$ have at least one neighbour, which is not in $M(T)$. Therefore, $N(T) = M(T)$ cannot hold, and we have $|N(T)| > |M(T)|$ which yields $|N(T)| > |T|$.
\end{comment}
\end{proof}


\begin{proof}[of Lemma \ref{iff}]
If $F_H(M, \partone) = \emptyset$, according to Lemma \ref{rem}, $$\forall T \subseteq \partone \qquad |N(T)| > |T|.$$

On the other hand, suppose that for all $T \subseteq \partone$ we have $|N(T)| > |T|$. We show that $F_H(M, \partone) = \emptyset$. For the sake of contradiction, assume that $F_H(M, \partone) \neq \emptyset$ and let $T = F_H(M, \partone)$. Since there exists a matching from $T$ to $N(T)$ that saturates all the vertices of $N(T)$, we have $|T| \geq |N(T)|$, which is a contradiction. Hence, $F_H(M, \partone) = \emptyset$. 
\end{proof}



\begin{proof}[of Lemma \ref{dag}]
Consider a cycle $L$ in $G_C$. For each vertex $v_j \in L$, there is at least one vertex $v_i \in L$ such that $\agent_i$ envies $\agent_j$. Therefore, Considering $S$ as the set of agents with vertices in $L$, none of the agents of $S$ is a loser. By the same deduction, none of the agents of $S$ is a winner. But this contradicts the fact that the set $C$ is cycle-envy-free.
\end{proof}


\begin{proof}[of Lemma \ref{wm}] We describe the proof for the first condition in more details. The proof for the second condition is almost the same as the first condition. 

\textbf{The first condition}: Suppose that there exists no such vertex. Our goal is to find a new matching of $H$ with the same cardinality, but with more weight. To this end, we construct a directed graph $H'$ from $H$ as follows: for each $\vtwo_j \in T$ we consider a vertex $v_j$ in $V(H')$. Furthermore, there is a directed edge from $v_j$ to $v_i$ in $H'$, if and only if $w(\vone_j,\vtwo_{j}) < w(\vone_i,\vtwo_j)$ in $H$. 

If there exists a vertex $v_j$ with out-degree zero in $H'$, then $\vtwo_j$ is the desired winner in $T$, since
$$ \forall \vtwo_i \in H, w(\vone_j,\vtwo_{j}) \geq w(\vone_i,\vtwo_j).$$
 Otherwise, the out-degree of every vertex in $T$ is non-zero. Therefore, $H'$ has at least one cycle $L = \langle v_{l_1}, v_{l_2}, \ldots, v_{l_{|L|}}\rangle$. Now, if we change matching $M$ by removing the set of edges $$ \{(\vtwo_{l_1},\vone_{l_1}), (\vtwo_{l_2},\vone_{l_2}), \ldots, (\vtwo_{l_{|L|}},\vone_{l_{|L|}})\} $$
from $M$ and adding 
$$\{(\vtwo_{l_1},\vone_{l_2}), (\vtwo_{l_2},\vone_{l_3}),\ldots,(\vtwo_{l_{|L|}},\vone_{l_1})\}$$ 
to $M$, the weight of our matching will be increased. Note that by the definition of an edge in $H'$, we have $$w(\vone_{l_2}, \vtwo_{l_1}) > w(\vone_{l_1}, \vtwo_{l_1}),w(\vone_{l_3}, \vtwo_{l_2}) > w(\vone_{l_2}, \vtwo_{l_2}),\ldots,w(\vone_{l_1}, \vtwo_{l_{|L|}}) > w(\vone_{l_{|L|}}, \vtwo_{l_{|L|}}).$$ But this contradicts the fact that  $M$ was $\MCMWM$ of $H$.


\textbf{The second condition}:
\begin{comment}
The proof of this part is very similar to the proof of the first condition. For the sake of contradiction suppose that for every vertex $\vtwo_j \in T$, there is at least one vertex $\vtwo_i \in T$ where $w(\vone_j,\vtwo_{i}) > w(\vone_i,\vtwo_i)$ in $H$ with $(\vone_j,\vtwo_i) \in E(H)$.
\end{comment}
Similar to the proof of the first condition, we construct a new directed graph $H'$ from $H$ where we have a vertex $v_j$ in $H'$ for each vertex $\vtwo_j$ in $T$. For every pair $\vtwo_i$ and $\vtwo_j$ which are members of $T$ we connect $v_i$ to $v_j$ with a directed edge in $H'$ if 
$$w(\vone_j,\vtwo_{i}) > w(\vone_i,\vtwo_i)$$ in $H$ and $(\vone_j,\vtwo_i) \in E(H)$. Note that if $H'$ contains a vertex $v_i$ with in-degree equal to zero, then $\vtwo_i$ is the desired loser in $T$. Thus, suppose that no vertex in $H'$ has in-degree zero and hence, $H'$ has a directed cycle.  Let $L = \langle \vtwo_{l_1}, \vtwo_{l_2}, \ldots, \vtwo_{l_{|L|}}\rangle$ be a directed cycle in $H'$. Similar to the proof of the previous condition, we leverage $L$ to alter $M$ to a new matching with more weight, which is a contradiction by the maximality of $M$.  

\textbf{The third condition}: If $w(\vone_i,\vtwo_i) < w(\vone_j,\vtwo_i)$, we can replace the edge between $\vone_i$ and $\vtwo_i$ by $(\vone_j, \vtwo_i)$ in $M$ which yields a matching with a greater weight. This contradicts the maximality of $M$.
\end{proof}

\section{Omitted Proofs of Section \ref{additive:clusters}}\label{clusteringappendix}

\begin{proof}[of Lemma \ref{forc2c3}]
By definition, there is no edge between the vertices of $F_{G_{1/2}}(M,\itemsv_{1/2})$ and $\agentsv_{1/2} \setminus N(F_{G_{1/2}}(M,\itemsv_{1/2}))$ in $G_{1/2}$. Furthermore, all the items are in worth less than $1/2$ for the agents corresponding to the vertices in $\agentsv \setminus \agentsv_{1/2}$. Thus, for every agent $\agent_i$ and every item $\ite_j$ with $\agentv_i \in \agentsv \setminus N(F_{G_{1/2}}(M,\itemsv_{1/2}))$ and $\itemv_j \in F_{G_{1/2}}(M,\itemsv_{1/2})$, we have $\valu_i(\ite_j)<1/2$. According to the fact that the agents that are not selected in the clustering of $\cone$ either belong to $\ctwo$ or $\cthree$, we have:
\[ \forall \agent_j \in \cone \qquad \valu_i(\firstset_j) < 1/2. \]
\end{proof}

\begin{proof}[of Lemma \ref{nicematch}]
First, we prove Lemma \ref{v1size}. This lemma ensures that there exists a matching in $G_1$ that saturates all the vertices in $W_1$. Lemma \ref{v1size} is a consequence of irreducibility. In fact, we show that if the condition in Lemma \ref{v1size} does not hold, the instance is reducible.
\begin{lemma}
\label{v1size}
For graph $G_1$, we have $$ \forall R \subseteq  W_1, \qquad |N(R)| > |R|.$$ 
\end{lemma}
\begin{proof}
Let $M_1$ a matching with the maximum number of edges in $G_1$ . Regarding Lemma \ref{iff}, it only suffices to show that $F_{G_1}(M_1,W_1)$ is empty. For the sake of contradiction, suppose that $F_{G_1}(M_1,W_1)$ is not empty. As mentioned before, there exists a matching between $F_{G_1}(M_1,W_1)$ and $N(F_{G_1}(M_1,W_1))$ that saturates all the vertices in $N(F_{G_1}(M_1,W_1))$. Let 
$$M_S = \{(\itemv_{j_1},\agentv_{i_1}),(\itemv_{j_2},\agentv_{i_2}),\ldots,(\itemv_{j_k},\agentv_{i_k})\}$$
be this matching. We show that the set of agents $$T= \{\agent_{i_1}, \agent_{i_2}, \ldots, \agent_{i_k}\}$$ and the set of items $$S = \{\firstset_{i_1}, \ite_{j_1},\firstset_{i_2},\ite_{j_2},\ldots, \firstset_{i_k},\ite_{j_k} \}.$$
have all three conditions in Lemma \ref{remove3} (Note that $\firstset_{i_l}$ contains exactly one item).  The first condition is trivial: $|S| = 2|T|$. Regarding the definition of an edge in $G_1$, we know that $\firstset_{i_l} \cup \{\ite_{j_l}\}$ satisfy $\agent_{i_l}$ and hence, the second condition is held as well.  
For the third condition, we should prove that for every agent $\agent_{i_l}$ in $T$, 
$$\valu_{i'}(\firstset_{i_l} \cup \{\ite_{j_l}\}) < 1 \qquad  \forall \agent_{i'} \notin T .$$
To show this, we consider two cases separately. First, if $\agent_{i'} \notin C_1$, by Lemma \ref{forc2c3}, $\valu_{i'}(\firstset_{i_l})<1/2$ and by Observation \ref{w1small}, $\valu_{i'}(\{\ite_{j_l}\})<1/2$, which means $\valu_{i'}(\firstset_{i_l} \cup \{\ite_{j_l}\}) < 1$.

Moreover, consider the case that $\agent_{i'} \in C_1$. Note that since $\agent_{i'} \notin T$, it's corresponding vertex $\agentv_{i'}$ is not in $N(F_{G_1}(M_1,W_1))$, which means:
$$\agentv_{i'} \in V_{C_1} \setminus N(F_{G_1}(M_1,W_1)).$$
By the definition of $N(F_{G_1}(M_1,W_1)$, there is no edge between $\agentv_{i'}$ and $\itemv_{j_l}$ and hence, $\valu_{i'}(\{\ite_{j_l}\})< \epsilon_{i'} \leq 1/4$. On the other hand, by the irreducibility assumption and the fact that $\firstset_{i_l}$ contains exactly one item,  $\valu_{i'}(\firstset_{i_l}) < 3/4$. Thus, $\valu_{i'}(\firstset_{i_l} \cup \{\ite_{j_l}\}) < 1$. 

As a result, $\valu_{i'}(\firstset_{i_l} \cup \{\ite_{j_l}\}) < 1$ for every agent $\agent_{i'} \notin T$ which means the third condition of Lemma \ref{remove3} is held as well. Thus, regarding Lemma \ref{remove3}, the instance is reducible. But this contradicts the irreducibility assumption.
\end{proof}

The rest of the proof of Lemma \ref{nicematch} is as follows. Since we used $\MCMWM$ to build cluster $\cone$, regarding Lemma \ref{wm}, $\cone$ is cycle-envy-free. Consider the topological ordering of $\cone$ and let $p_{a_i}$ be the position of $\agent_i$ in this ordering. More precisely, $p_{\agent_i} = k$ if $\agent_i$ is the $k$-th agent in the topological ordering of $\cone$.

According to Lemma \ref{v1size}, the condition of Hall’s Theorem holds for graph $G_1$ and as a result there exists a matching in $G_1$ that saturates all the vertices in $W_1$. Among all possible maximum matchings of $G_1$, let $M_1$ be a maximum matching that minimizes $$p_{M_1} = \sum_{\agentv_i \in M_1} p_{a_i}.$$ We claim that $M_1$ is the desired matching described in Lemma \ref{nicematch}. To prove our claim, we must show that for any edge $(\itemv_i, \agentv_j) \in M_1$ and any unsaturated vertex $\agentv_k \in N(\itemv_i)$, $\agent_j$ is a loser for the set $\{\agent_j, \agent_k\}$, which means $\agent_k$ does not envy $\agent_j$. Note that if $\agent_k$ envies $\agent_j$, $\agent_k$ appears before $\agent_j$ in the topological ordering of $\cone$ which means $p_{\agent_k} < p_{\agent_j}$. Therefore, if we replace $(\itemv_i, \agentv_j)$ by $(\itemv_i, \agentv_k)$ in $M_1$, $p_{M_1}$ will be decreased that contradicts the minimality of $p_{M_1}$.
\end{proof}

\begin{proof}[of Lemma \ref{gsmallc1r}]
Let $\ite_k$ be the item assigned to $\agent_j$ in the refinement of $\cone$. Since $\itemv_k \in W_1$, according to Observation \ref{w1small}, $\valu_i(\secondset_j)< 1/2$. 
\end{proof}

\begin{proof}[of Lemma \ref{forc2}]
Let $\agent_j$ be an agent in $\satagents_1^r$. First, note that $|\firstset_j| = |\secondset_j| = 1$. Lemma \ref{forc2c3} together with Observation \ref{w1small} state that $\valu_i(\firstset_j \cup \secondset_j)<1$. According to Inequality (\ref{saeed1}), we have  
\begin{equation}
\label{eq100}
\MMS_{\valu_i}^{|\agents \setminus \agent_j|} ( {\items} \setminus \firstset_j \cup \secondset_j) \geq 1.
\end{equation}
 Note that Equation (\ref{eq100}) holds for every agent in $\satagents_1^r$. Applying Equation (\ref{eq100}) to all the agents of $\satagents_1^r$ yields
 \[ \MMS_{\valu_i}^{|\agents \setminus \satagents_1^r|} ( {\items} \setminus \bigcup_{\agentv_i \in \satagents_1^r} \firstset_i \cup \secondset_i) \geq 1.\] 


\end{proof}

\begin{proof}[of Lemma \ref{c1small2}]
According to Observation \ref{fsmallc1}, for any agent $a_k \in \cone$ and for every $\itemv_j \in \itemsv' \setminus \itemsv'_{1/2}$ we have $V_k(\{b_j\}) < \epsilon_k$. By additivity assumption, for any $\agent_k \in \cone$  we have 
$$ \forall {x_i, x_j \in \itemsv' \setminus \itemsv'_{1/2}} \qquad V_k(\{b_i, b_j\}) < 2\epsilon_k.$$
\end{proof}

\begin{proof}[of Lemma \ref{pairsmall}]
Suppose for the sake of contradiction that the problem is $3/4$-irreducible, and there exists a vertex $y_k \in \agentsv$ such that $V_k(\{b_i, b_j\}) \geq 3/4$. According to Lemma \ref{remove2} there exists an agent $a_{k'} \neq a_k$ such that $$V_{k'}(\{b_i, b_j\}) \geq 1.$$
Since the valuations are additive, we know that one of the inequalities $V_{k'}(\{b_i\}) \geq 1/2$ or $V_{k'}(\{b_j\}) \geq 1/2$ are held, which is contradiction, since we know both $\itemv_i$ and $\itemv_j$ belong to $\itemsv' \setminus \itemsv'_{1/2}$.
\end{proof}

\begin{proof}[of Lemma \ref{sizeeq}]
We prove Lemma \ref{sizeeq} in two steps. Firstly, we show that
\begin{equation}
\label{ineqhadi2}
|F_{G'_{1/2}}(M',\itemsv'_{1/2})| \leq |N(F_{G'_{1/2}}(M',\itemsv'_{1/2}))|.
\end{equation}
Furthermore, we prove 
\begin{equation}
\label{ineqhadi1}
|F_{G'_{1/2}}(M',\itemsv'_{1/2})| \geq |N(F_{G'_{1/2}}(M',\itemsv'_{1/2}))|.
\end{equation}
Inequalities \eqref{ineqhadi1} and \eqref{ineqhadi1} yields
\begin{equation}
\label{ineqhadi3}
|F_{G'_{1/2}}(M',\itemsv'_{1/2})| = |N(F_{G'_{1/2}}(M',\itemsv'_{1/2}))|.
\end{equation}


\textbf{To show Inequality \eqref{ineqhadi2},} argue that before Algorithm \ref{addvertex} starts, we have $$F_{G'_{1/2}}(M',\itemsv'_{1/2}) = \emptyset$$ and  $$N(F_{G'_{1/2}}(M',\itemsv'_{1/2})) = \emptyset$$ and all the vertices in $\itemsv'_{1/2}$ are saturated by $M'$. In each step of Algorithm \ref{addvertex}, we add a new vertex to $\itemsv'_{1/2}$, and the size of the maximum matching $M'$ is increased by one. Therefore, after each step of Algorithm \ref{addvertex}, all of the vertices in $\itemsv'_{1/2}$ remain saturated by $M'$. Since $F_{G'_{1/2}}(M',\itemsv'_{1/2}) \subseteq \itemsv'_{1/2}$, all the vertices of $F_{G'_{1/2}}(M',\itemsv'_{1/2})$ are also saturated by $M'$, which means
$$|F_{G'_{1/2}}(M',\itemsv'_{1/2})| \leq |N(F_{G'_{1/2}}(M',\itemsv'_{1/2}))|.$$

\textbf{To prove Inequality \eqref{ineqhadi1}}, note that by definition, $F_{G'_{1/2}}(M',\itemsv'_{1/2})$ has a property that there exists a matching from $F_{G'_{1/2}}(M',\itemsv'_{1/2})$ to $N(F_{G'_{1/2}}(M',\itemsv'_{1/2}))$ that saturates all the vertices of $N(F_{G'_{1/2}}(M',\itemsv'_{1/2}))$. Therefore, we have
$$
|F_{G'_{1/2}}(M',\itemsv'_{1/2})| \geq |N(F_{G'_{1/2}}(M',\itemsv'_{1/2}))|.
$$
This completes the proof.
\end{proof}

\begin{proof}[of Lemma \ref{forc3}] 
Firstly, we clarify what agents are in $\cthree$. Roughly speaking, the agents that are not selected for Clusters $\cone$ and $\ctwo$ are in $\cthree$. Thus, the agents in $\cthree$ correspond to the vertices in  
$$\agentsv' \setminus N(F_{G'_{1/2}}(M',\itemsv'_{1/2}))$$
$$=\big(\agentsv' \setminus \agentsv'_{1/2}\big) \cup  \big(\agentsv'_{1/2} \setminus N(F_{G'_{1/2}}(M',\itemsv'_{1/2}))\big).$$

\textbf{The term $ \agentsv' \setminus \agentsv'_{1/2} $ } refers to the vertices that are filtered in $G'_{1/2}$ which means no edge with weight at least $1/2$ is incident to any of these vertices.  
On the other hand, for every agent $\agent_j \in \ctwo$, $\firstset_j$ corresponds to a vertex in $F_{G'_{1/2}}(M',\itemsv'_{1/2})$.  Hence, for every agent $\agent_j \in \ctwo$  and every agent $\agent_i$ with corresponding vertex in $\agentsv' \setminus \agentsv'_{1/2}$ we have $\valu_i(f_j) <1/2$


\textbf{Next, consider the term $\agentsv'_{1/2} \setminus N(F_{G'_{1/2}}(M',\itemsv'_{1/2}))$.} By definition, the vertices of $F_{G'_{1/2}}(M',\itemsv'_{1/2})$ are only incident to the vertices of $N(F_{G'_{1/2}}(M',\itemsv'_{1/2}))$ in $G'_{1/2}$. Regarding the definition of an edge in $G'_{1/2}$, for every agent $\agent_j \in \ctwo$ and agent $\agent_i$ with $\agentv_i \in \agentsv'_{1/2} \setminus N(F_{G'_{1/2}}(M',\itemsv'_{1/2}))$ we have $\valu_i(f_j) <1/2$.

Therefore, for all $\agent_i \in \cthree$ we have: 
 $$\forall \agent_j \in \ctwo \qquad \valu_i(\firstset_j) < 1/2.$$
\end{proof}

\begin{proof}[of Lemma \ref{cr2smallc1}]
Regarding Observation \ref{fsmallc1}, after refinement of $\cone$, all the items with vertex in $\itemsv' \setminus \itemsv'_{1/2}$ are in worth less than $\epsilon_j$ for every agent $\agent_j \in \cone$. Furthermore, note that for every agent $\agent_i \in \satagents_2^r$, $\secondset_i$ is a single item with vertex in $\itemsv' \setminus \itemsv'_{1/2}$. Thus, $\valu_j(\secondset_i)< \epsilon_j$ for every agent $\agent_j \in \cone$.
\end{proof}

\begin{proof}[of Lemma \ref{cr2smallc3}]
According to Algorithm \ref{c2ref}, for any agent $\agent_i \in \satagents_2^r$, the corresponding vertex of the only member of $g_i$ is in $\itemsv' \setminus \itemsv'_{1/2}$. Therefore, for any agent $a_j \notin \cone \cup \ctwo$ we have $V_j(g_i) < 1/2$. Finally, note that the remaining agents that are not in $\cone$ and $\ctwo$ belong to $\cthree$.
\end{proof}

\begin{proof}[of Lemma \ref{lsmall_c3}] The algorithm \ref{addvertex} terminates when there is no desirable pair for the agents in $T = \agentsv' \setminus N(F_{G'_{1/2}}(M',\itemsv'_{1/2})).$ Furthermore, by definition, for every agent  $\agent_i \in \cthree$ we have  $$\agentv_i \in \agentsv' \setminus N(F_{G'_{1/2}}(M',\itemsv'_{1/2})).$$ But at the end of Algorithm \ref{addvertex}, no pair of vertices is desirable for $\agent_i$ which means for every $\itemv_j,\itemv_k \in \itemsv'' \setminus \itemsv''_{1/2}$, we have  $V_i(\{\ite_j,\ite_k\}) < {1/2}$ (note that $\itemsv'' \setminus \itemsv''_{1/2} \subseteq \itemsv' \setminus \itemsv'_{1/2}$).
\end{proof}


\section{Omitted Proofs of Section \ref{additive:allocation}}\label{clustering2appendix}
\begin{proof}[of Lemma \ref{general}]
At this point, for every agent $\agent_i \in \cone \cup \ctwo \cup \cthree^s$, $|\firstset_j| \leq 2$. If $|\firstset_i| = 1$ holds, then according to Lemma \ref{remove1}, value of the item in $\firstset_i$ is less than $3/4$ to all other agents. Moreover, if $|\firstset_i| = 2$, then $\firstset_i$ corresponds to a merged vertex. In this case, by Lemmas \ref{c1small2} and \ref{pairsmall}, value of $\firstset_i$ is less than $3/4$ to all other agents. 
\end{proof}

\begin{proof}[of Lemma \ref{c3fsmall}]
According to Lemma \ref{lsmall_c3}, value of every pair of items in $\fitems$ is less than $1/2$ to $\agent_i$. Therefore, $\firstset_i$ contains at least three items. Let $\ite_k$ be an arbitrary item in $\firstset_i$. Since $|\firstset_i| \geq 3$, $\firstset_i \setminus \{\ite_k\}$ is non-empty. On the other hand, $S$ is minimal and hence, none of the sets $\firstset_i \setminus \ite_k$ and $\{\ite_k\}$ is feasible for any agent. According to the definition of feasibility for the agents of $\cone \cup \ctwo \cup \cthree^s \cup \cthree^b$, we have
$$ \forall \agent_j \in \cone \cup \ctwo \cup \cthree^s \cup \cthree^b \qquad \valu_j(\firstset_i \setminus \{\ite_k\})< \epsilon_j $$ 
and 
$$\forall \agent_j \in \cone \cup \ctwo \cup \cthree^s \cup \cthree^b \qquad \valu_j(\{\ite_k\})< \epsilon_j$$
which means
$$ \forall \agent_j \in \cone \cup \ctwo \cup \cthree^s \cup \cthree^b \qquad \valu_j(\firstset_i)< 2\epsilon_j. $$ 

\end{proof}

\begin{proof}[of Lemma \ref{cef}]
The Lemma trivially holds for $\cone$ and $\ctwo$, since removing an agent from a cycle-envy-free set preserves this property. For $\cthree^s$, there may be multiple rounds that an agent is added to $\cthree^s$. We show that adding an agent to $\cthree^s$ preserves cycle-envy-freeness as well.

For the sake of contradiction, let $\mathbb{R}_z$ be the first round in which adding an agent $\agent_i$ to $\cthree^s$ results in a set, that is no longer cycle-envy-free. Since $\cthree^s \setminus \{\agent_i\}$ is cycle-envy-free, every subset of $\cthree^s \setminus \{\agent_i\}$ contains at least one winner and one loser. Moreover, by Lemma \ref{c3fsmall} we have:
\begin{equation}
\label{inec1}
\forall \agent_j \in \cthree^s, j \neq i, \qquad \valu_j( \firstset_i )< 2\epsilon_j.
\end{equation}

Note that $\agent_i$ previously belonged to $\cthree^f$.  By definition of $\cthree^f$  
\begin{equation}
\label{inec2}
\forall \agent_j \in \cthree^s , j \neq i, \qquad \valu_i(\firstset_j)< 1/2 .
\end{equation}

Inequalities (\ref{inec1}) and (\ref{inec2}) together imply that $\agent_i$ is both a winner and a loser for every subset of $\cthree^s$ that contains $\agent_i$. This means that every subset of $\cthree^s$ contains at least one winner and one loser, which is a contradiction.

\end{proof}

\begin{proof}[of Lemma \ref{prvalue}]

If $\agent_j \prec_{pr} \agent_i$, then $\secondset_i$ is not feasible for $\agent_j$, since the agent with the lowest 
priority is satisfied in each round of the second phase. Thus, $\valu_j(\secondset_i) < \epsilon_j$. For the case where $\agent_i \prec_{pr} \agent_j$, let $\ite_k$ be an arbitrary item of $\secondset_i$. According to the fact that $\secondset_i$ is minimal, $\secondset_i \setminus \{\ite_k\}$ is not feasible for any agent. Hence, $\valu_j(\secondset_i \setminus \{\ite_k\})< \epsilon_j$. On the other hand, by Observations \ref{fsmallc1} and \ref{fsmallc2}, $\valu_j(\{\ite_k\})<\epsilon_j $. Therefore, $\valu_j(\secondset_i)<2\epsilon_j$. 
\end{proof}

\begin{proof}[of Lemma \ref{m_1}]
Let $\mathbb{R}_z$ be the round, in which $\agent_i$ is satisfied. At that point, if $\agent_j \in \cthree^f$ then $\valu_j(\secondset_i) < {1/2}$ trivially holds. Since in round $\mathbb{R}_z$, $\agent_j \prec_{pr} \agent_i$ holds, $\secondset_i$ was not feasible for $\agent_j$ in the first place. Recall that in each round, the agent with lowest order in $\Phi(S)$ is selected. 

Furthermore, if in round $\mathbb{R}_z$, $\agent_j$ was in $\cthree^s \cup \cthree^b$, according to Observations \ref{fsmallc1} and \ref{fsmallc2}, $|S| \geq 2$, since no item alone can satisfy $\agent_i$. If $|S|=2$, then by Observation \ref{lsmall_c3}, $\valu_j(\secondset_i)<1/2$. For the case of $|S|>2$, let $\ite_k$ be the item in $S$ with the minimum value to $\agent_j$. According to Corollary \ref{small_c3}, $\valu_j(\{\ite_k\})<{1/4}$. Also, since $S$ is minimal, $S \setminus \{\ite_k\}$ is not feasible for any agent and hence, $\valu_j(S \setminus \{\ite_k\}) < \epsilon_j \leq {1/4}$. Thus, $\valu_j(S) < {1/2}$.
\end{proof}




\color{black}	 
\section{Omitted Proofs of Section \ref{additiveproofs}}\label{additiveproofappendix}
Before proceeding to the proof of Lemma \ref{c3null}, we show Lemmas (\ref{m_2}, \ref{c3bssmall} and \ref{c3sat}). 

\begin{lemma}
\label{m_2}
Let $\agent_i$ be an agent in $\satagents_3$ and let ${\mathbb R}_z$ be the round of the second phase in which $\agent_i$ is satisfied. Then, for any other agent $\agent_j$ that is  in $\cthree^f$ in ${\mathbb R}_z$, $\valu_j(\secondset_i) < 1/2$ holds.
\end{lemma}

\begin{proof}
In ${\mathbb R}_z$, $\agent_i$ either belongs to $\cthree^s$ or $\cthree^b$. Thus, $\agent_j \prec_{pr} \agent_i$, and thus $\secondset_i$ is not feasible for $\agent_j$ in that round. Therefore, $\valu_j(\secondset_i)< 1/2$.
\end{proof}

\begin{lemma}
\label{c3bssmall}
Let $\agent_i \in \satagents_3$ be a satisfied agent and let ${\mathbb R}_z$ be the round in which $\agent_i$ is satisfied. Then, for every other agent $\agent_j$ that belongs to $\cthree^s \cup \cthree^b$ in that round, either $\valu_j(\secondset_i) < \epsilon_j$ or $\valu_j(\firstset_i) \leq 3/4-\epsilon_j$.

\end{lemma}
\begin{proof}
If $\secondset_i$ is not feasible for $\agent_j$, then the condition trivially holds. Moreover, by the definition, the statement is correct for the agents of $\cthree^b$. Therefore, it only suffices  to  consider the case that $\agent_j \in \cthree^s$ and $\secondset_i$ is feasible for $\agent_j$. Due to the priority rules for satisfying the agents in the second phase, $\agent_i \prec_{pr} \agent_j$ and hence, $\agent_i$ cannot be in  $\cthree^b$. Thus, $\agent_i \in \cthree^s$. According to Observation \ref{epsofcluster} and the fact that $\prec_{pr}$ is equivalent to $\prec_{o}$ for the agents in $\cthree^s$, we have $\valu_j(\firstset_i) \leq 3/4 - \epsilon_j$.
\end{proof}



\begin{lemma}
\label{c3sat}
During the second phase, for any agent  $\agent_i$ in  $\cthree$, we have: $$\sum_{\agent_j \in \satagents_3} \valu_i(\firstset_j \cup \secondset_j)< |\satagents_3| + 1/4.$$ 
\end{lemma}

\begin{proof}
To show Lemma \ref{c3sat}, we show that for all the agents $\agent_j \in \satagents_3$ except at most one agent, $\valu_i(\firstset_j \cup \secondset_j)<1$ holds. To show this, let ${\mathbb R}_z$ be an arbitrary round of the second phase, in which an agent $\agent_j \in \cthree$ is satisfied. First, note that in ${\mathbb R}_z$, $\agent_j$ belongs to $\cthree^s \cup \cthree^b$. Also, in round ${\mathbb R}_z$, $\agent_i$ belongs to one of $\cthree^s, \cthree^b$, or $\cthree^f$.  
 
If $\agent_i \in \cthree^f$, then by Lemma \ref{m_2}, $\valu_i(\secondset_j)<1/2$ holds. On the other hand, by definition, $\valu_i(\firstset_j)<1/2$ and hence, $\valu_i(\firstset_j \cup \secondset_j)<1$. 

Now, consider the case, where $\agent_i \in \cthree^b \cup \cthree^s$. Note that by Lemma \ref{c3bssmall}, either $\valu_i(\firstset_j) \leq 3/4-\epsilon_i$ or $\valu_i(\secondset_j) < \epsilon_i$. If $\valu_i(\secondset_j) < \epsilon_i$, then by Lemmas \ref{general} and \ref{c3fsmall}, we know $\valu_i(\firstset_j) < 3/4$ and hence, $\valu_i(\firstset_j \cup \secondset_j)<3/4 + \epsilon_i < 1$. 

For the case where $\valu_i(\firstset_j) \leq 3/4-\epsilon_i$, let $\ite_l$ be the item in $\secondset_j$ with the maximum value to $\agent_i$. By minimality of $\secondset_j$, $\secondset_j \setminus \{\ite_l\}$ is not feasible for any agent, including  $\agent_i$ and thus, $\valu_i(\secondset_j\setminus \{\ite_l\}) < \epsilon_i$. Recall that by Corollary \ref{small_c3}, there is at most one item $\ite_k$ in $\fitems$, such that $\valu_i(\ite _k) \geq 1/4$. In addition to this, $\valu_i(\ite_k) < 1/2$ trivially holds, since $\ite_k$ is not assigned to any agent during the clustering phase. If $\ite_l \neq \ite_k$, $\valu_i(\secondset_j)< 1/4 + \epsilon_i$ holds and hence, $$\valu_i(\firstset_j \cup \secondset_j) < 3/4-\epsilon_i + 1/4 + \epsilon_i<1.$$ Moreover, If $\ite_l = \ite_k$, $\valu_i(\secondset_j)< 1/2 + \epsilon_i$ holds and thus, $\valu_i(\firstset_j \cup \secondset_j) < 3/4-\epsilon_i + 1/2 + \epsilon_i<5/4$. But, this can happen at most one round. Therefore, for all the agents $\agent_j \in \satagents_3$ except at most one, $\valu_i(\firstset_j \cup \secondset_j)<1$. Also, for at most one agent $\agent_j \in \satagents_3$, $\valu_i(\firstset_j \cup \secondset_j)<5/4$. Thus, 
$$\sum_{\agent_j \in \satagents_3} \valu_i(\firstset_j \cup \secondset_j)< |\satagents_3| + 1/4.$$   
\end{proof}



\begin{proof}[of Lemma \ref{c3null}]
Suppose for the sake of contradiction that $\cthree \neq \emptyset$.  Note that, by the definition of $\cthree^b$, if $\cthree^s = \emptyset$ holds, then consequently $\cthree^b = \emptyset$. Therefore, since we have $\cthree = \cthree^s \cup \cthree^b \cup \cthree^f$, if $\cthree$ is non-empty, at least either of the two sets $\cthree^s$ or $\cthree^f$ is non-empty. In case $\cthree^s$ is non-empty, let $\agent_i$ be a winner of $\cthree^s$, otherwise let $\agent_i$ be an arbitrary agent of $\cthree^f$.

According to Lemma \ref{m_1}, for every agent $\agent_j \in \satagents_1^s \cup \satagents_2^s$, $\valu_i(\secondset_j) < {1/2}$ holds. Also, by Lemmas \ref{gsmallc1r} and \ref{cr2smallc3}, for every agent  $\agent_j \in \satagents_1^r \cup \satagents_2^r$, we have $\valu_i(\secondset_j) < {1/2}$. Therefore, 
$$\forall \agent_j \in \satagents_1 \cup \satagents_2 \qquad \valu_i(\secondset_j) < {1/2}.$$

Also, by Lemmas \ref{forc2c3} and \ref{forc3} we know that $\valu_i(\firstset_j)<{1/2}$ for every $\agent_j \in \satagents_1 \cup \satagents_2$. Thus, for every satisfied agent $\agent_j \in \satagents_1 \cup \satagents_2$, $\valu_i(\firstset_j \cup \secondset_j) <1$ holds, and hence 
\begin{equation}\label{eq1}
\sum_{\agent_j \in \satagents_1 \cup \satagents_2} \valu_i (\firstset_j \cup \secondset_j) < |\satagents_1 \cup \satagents_2|.
\end{equation}


Moreover, by Lemma \ref{c3sat}, the total value of items assigned to the agents in $\satagents_3$ to $\agent_i$ is less than $|\satagents_3| + 1/4$. More precisely,
\begin{equation}\label{eq2}
\sum_{\agent_j \in \satagents_3} \valu_i(\firstset_j \cup \secondset_j) \leq |\satagents_3| + 1/4.
\end{equation}
Inequality \eqref{eq1} along with Inequality \eqref{eq2} implies: 
\begin{equation}
\begin{split}
\sum_{\agent_j \in \satagents} \valu_i (\firstset_j \cup \secondset_j) & = \sum_{\agent_j \in \satagents_1 \cup \satagents_2} \valu_i (\firstset_j \cup \secondset_j) + \sum_{\agent_j \in \satagents_3} \valu_i (\firstset_j \cup \secondset_j)\\
& < |\satagents_1 \cup \satagents_2| + |\satagents_3| + 1/4\\
 & = |\satagents |+1/4
\end{split}
\end{equation}

Recall that the total sum of the item values for $\agent_i$ is equal to $n$. In addition to this, since every agent belongs to either of the Clusters $\cone$, $\ctwo$, $\cthree$, or $\satagents$ we have $$|\satagents| + |\cone| + |\ctwo| + |\cthree| = n.$$ Furthermore, every item $\ite_j \in \items$ either belongs to $\fitems$ or one of the sets $\firstset_{j'}$ and $\secondset_{j'}$ for an agent $\agent_{j'}$. More precisely,
$$\fitems = \items \setminus \Big[\bigcup_{\agent_j \in \satagents \cup \cone \cup \ctwo \cup \cthree^s} \firstset_j \cup \bigcup_{\agent_j \in \satagents} \secondset_j\Big].$$ 
 Therefore
\begin{equation}\label{eq5}
\begin{split}
\sum_{\agent_j \in \cone} \valu_i(\firstset_j) + \sum_{\agent_j \in \ctwo} \valu_i(\firstset_j) + \sum_{\agent_j \in \cthree^s} \valu_i(\firstset_j) + \valu_i({\fitems}) & = \valu_i(\items) - \sum_{\agent_j \in \satagents} \valu_i(\firstset_j \cup \secondset_j)\\
&= n - \sum_{\agent_j \in \satagents} \valu_i(\firstset_j \cup \secondset_j)\\
&\geq n - (|\satagents| + 1/4)\\
&= |\cone| + |\ctwo| + |\cthree|-1/4
\end{split}
\end{equation}

According to Lemmas \ref{forc2c3} and \ref{forc2},  
\begin{equation}\label{eq5.1}
\sum_{\agent_j \in \cone} \valu_i(\firstset_j) < {1/2}|\cone|
\end{equation}
 and 
\begin{equation}\label{eq5.2}
\sum_{\agent_j \in \ctwo} \valu_i(\firstset_j)< {1/2} |\ctwo|
\end{equation}
hold. Inequalities \eqref{eq5}, \eqref{eq5.1}, and \eqref{eq5.2} together prove
\begin{equation}\label{eq6}
\begin{split}
\valu_i({\fitems}) &\geq |\cone| + |\ctwo| + |\cthree|-1/4 - \big[\sum_{\agent_j \in \cone} \valu_i(\firstset_j) + \sum_{\agent_j \in \ctwo} \valu_i(\firstset_j) + \sum_{\agent_j \in \cthree^s} \valu_i(\firstset_j)\big]\\
&\geq |\cone| + |\ctwo| + |\cthree|-1/4 - \big[1/2|\cone| + 1/2|\ctwo| + \sum_{\agent_j \in \cthree^s} \valu_i(\firstset_j)\big]\\
&\geq 1/2 |\cone| + 1/2 |\ctwo| + |\cthree| -1/4 - \sum_{\agent_j \in \cthree^s} \valu_i(\firstset_j).
\end{split}
\end{equation}
Now, we consider two cases separately: (i) $\agent_i \in \cthree^s$ and  (ii) $\agent_i \in \cthree^f$.


\textbf{In case $\agent_i \in \cthree^s$}, since $\agent_i$ is a winner of $\cthree^s$, we have 
\begin{equation}
\begin{split}
\sum_{\agent_j \in \cthree^s} \valu_i(\firstset_j) & \leq \sum_{\agent_j \in \cthree^s}  \valu_i(\firstset_i)\\
& = \sum_{\agent_j \in \cthree^s} 3/4 - \epsilon_i\\
& = ({3/4}-\epsilon_i) |\cthree^s|.
\end{split}
\end{equation}
This combined with Inequality \eqref{eq6} concludes
\begin{equation*}
\begin{split}
 \valu(\fitems) &\geq  1/2 |\cone| + 1/2 |\ctwo| + |\cthree| -1/4 - \sum_{\agent_j \in \cthree^s} \valu_i(\firstset_j)\\
 & \geq 1/2 |\cone| + 1/2 |\ctwo| + |\cthree| - 1/4 - ({3/4}-\epsilon_i) |\cthree^s|\\
 & \geq 1/2 |\cone| + 1/2 |\ctwo| + (1/4 + \epsilon) |\cthree| - 1/4.
\end{split}
\end{equation*}
On the other hand, since $\agent_i \in \cthree^s$, $|\cthree| \geq 1$ and hence, $\valu_i({\fitems}) \geq {1/4} + \epsilon_j - {1/4} = \epsilon_j$. This means that $\fitems$ is feasible for $\agent_i$, which contradicts the termination of the algorithm. 

\textbf{In case $\agent_i \in \cthree^f$}, by the definition of $\cthree^f$ we know that $\sum_{\agent_j \in \cthree^s} \valu_i(\firstset_j) < {1/2} |\cthree^s|$, which by Inequality \eqref{eq6} implies:

$$\valu_i({\fitems}) > {1/2}|\cthree^s| + |\cthree^b| + |\cthree^f| + {1/2}|\ctwo| + {1/2}|\cone|-1/4.$$

Since $\agent_i \in \cthree^f$, we have $|\cthree^f| \geq 1$ and hence, $\valu_i({\fitems}) > 3/4$. Again, this contradicts the termination of the algorithm since $\fitems$ is feasible for $\agent_i$.  
\end{proof}



\begin{proof}[of Lemma \ref{c1null}]
By Lemma \ref{c3null}, we already know $\cthree = \emptyset$. Now, let $\agent_i$ be a winner of the remaining agents in $\cone$. For convenience, we color the items in either blue or white. Intuitively, blue items may have a high value for $\agent_i$ whereas white items are always of lower value to $\agent_i$. Initially, all items are colored in white. For each $\agent_j  \in \agents$, if $|\firstset_j|=1$, then we color the item in $\firstset_j$ in blue. Moreover, for every $\agent_j \in \satagents$, if $|\secondset_j|=1$ and $\valu_i(\secondset_j) \geq \epsilon_i$, then we color the item in $\secondset_j$ in blue. 


Now, let $\cal P$ $= \langle P_1, P_2, \ldots, P_n \rangle$ be the optimal $n$-partitioning of the items in $\items$ for $\agent_i$, that is, the value of every partition $P_k$ to $\agent_i$ is at least $1$. Based on the coloring procedure, we have three types of partitions in $\cal P$:
\begin{itemize}
    \item $B_2$: the set of partitions with at least two blue items
    \item $B_1$: the set of partitions with exactly one blue item
    \item $B_0$: the set of partitions without any blue items
\end{itemize}
Note that every partition in $\cal P$ belongs to one of $B_0,B_1$ or $B_2$. Hence,
\begin{equation}
\label{sumba}
|B_0| + |B_1| + |B_2| = n
\end{equation}
As declared, all the items in the partitions of $B_0$ are white. The total value of these items to $\agent_i$ is at least $|B_0| \geq 4 \epsilon_i |B_0|$, which is
\begin{equation}\label{pq1}
\sum_{P_k \in B_0}\sum_{\ite_j \in P_k} \valu_i(\ite_j) \geq 4 \epsilon_i |B_0|.
\end{equation}
 Also, each partition in $B_2$ has at least two blue items, each of which is singly assigned to another agent. We decompose the partitions of $B_1$ into two disjoint sets, namely $\hat{B_1}$ and $\tilde{B_1}$. More precisely, let $\hat{B_1}$ be the partitions in $B_1$, in which the blue item is worth more than $\valu_i(\firstset_i)$ to $\agent_i$ and $\tilde{B_1} = B_1 \setminus \hat{B_1}$. As such, for each partition $P_k \in \tilde{B_1}$, 
the white items in $P_k$ are worth at least 
\begin{equation*}
\begin{split}
1- \valu_i(\firstset_i) &= 1-(3/4-\epsilon_i)\\
&= {1/4}+ \epsilon_i\\
&\geq 2\epsilon_i
\end{split}
\end{equation*}
to $\agent_i$. Therefore,
\begin{equation}\label{pq2}
\sum_{P_k \in \tilde{B_1}}  \valu_i(\wcal(P_k)) \geq 2 |\tilde{B_1}|\epsilon_i
\end{equation}
where $\wcal(S)$ stands for the set of white items in a set $S$ of items.
 On the other hand, since the problem is $3/4$-irreducible, by Lemma \ref{remove1}, no item alone is of worth $3/4$ to $\agent_i$ and thus for each partition $P_k \in \hat{B_1}$, the white items in $P_k$ have a value of at least ${1/4} \geq \epsilon_i$ to $\agent_i$. This implies that
\begin{equation}\label{pq4}
\sum_{P_k \in \hat{B_1}} \valu_i(\wcal(P_k)) \geq |\hat{B_1}|  \epsilon_i.
\end{equation}
By Inequalities \eqref{sumba},\eqref{pq1}, \eqref{pq2}, and \eqref{pq4} we have
\begin{equation}\label{enough}
\begin{split}
\valu_i(\wcal(\items)) &= \sum_{P_j \in B_0} \valu_i(\wcal(P_j)) + \sum_{P_j \in B_1} \valu_i(\wcal(P_j)) + \sum_{P_j \in B_2} \valu_i(\wcal(P_j)) \\
&\geq \sum_{P_j \in B_0} \valu_i(\wcal(P_j)) + \sum_{P_j \in B_1} \valu_i(\wcal(P_j)) \\
&\geq \sum_{P_j \in B_0} \valu_i(\wcal(P_j)) + \sum_{P_j \in \hat{B_1}} \valu_i(\wcal(P_j)) + \sum_{P_j \in \tilde{B_1}} \valu_i(\wcal(P_j)) \\
&\geq |B_0|  4\epsilon_i + |\hat{B_1}|  \epsilon_i + |\tilde{B_1}|  2\epsilon_i \\
&\geq |B_0|  4\epsilon_i + |B_1| 2\epsilon_i - |\hat{B_1}|  \epsilon_i  \\ 
&\geq |B_0|  4\epsilon_i + |B_1| 4\epsilon_i + |B_2|4\epsilon_i - |B_1| 2\epsilon_i - |B_2|4\epsilon_i- |\hat{B_1}|  \epsilon_i \\
&= (2n-2|B_2| - |B_1| -  |\hat{B_1}|)2\epsilon_i + (|\hat{B_1}|)\epsilon_i
\end{split}
\end{equation}
%Since every element in $\fitems$ is colored in white, we have
%\begin{equation}
%\valu_i(\wcal(\items \setminus \fitems)) = \sum_{\agent_j \in \agents}\valu_i(\firstset_j \cup \secondset_j) - \valu_i(\items \setminus \wcal(\items))
%\end{equation}
%Note that for every agents $\agent_j \in \agents \setminus \satagents$, $g_j = \emptyset$.
Note that the total value of white items that are assigned to the agents during the algorithm is equal to $\valu_i(\wcal(\items \setminus \fitems))$. The rest of the white items are still in $\fitems$. Thus, we have 
\begin{equation}
\label{fsum}
\valu_i(\wcal(\items)) = \valu_i(\wcal(\items \setminus \fitems)) + \valu_i(\fitems)
\end{equation}
Now, we provide an upper bound on the value of $\valu_i(\wcal(\items \setminus \fitems))$. As a warm up, one can trivially prove an upper bound of $2\epsilon_i (2n - 1 - |B_1| - 2|B_2|)$ on $\valu_i(\wcal(\items \setminus \fitems))$. This follows from the fact that two sets of items are assigned to any agent and hence we have a total of $2n$ disjoint sets. Among these $2n$ sets, at least one of them is empty (since $\secondset_i = \emptyset$) and at least $|B_1| + 2|B_2|$ of the sets contain a single blue item. On the other hand, by Lemmas \ref{c1small2}, \ref{cr2smallc1}, \ref{c3fsmall} and \ref{prvalue} every set with white items is of worth at most $2\epsilon_i$ to $\agent_i$. Therefore, the total value of the white items in $\items \setminus \fitems$ to $\agent_i$ is less than $2\epsilon_i (2n - 1 - |B_1| - 2|B_2|)$ and thus $$\valu_i(\wcal(\items \setminus \fitems)) \leq 2\epsilon_i (2n - 1 - |B_1| - 2|B_2|).$$ 

However, in order to complete the proof, we need a stronger upper bound on $\valu_i(\wcal(\items \setminus \fitems))$. To this end, we provide the following auxiliary lemma.
 
\begin{lemma}
\label{eps}
Let $\agent_j$ be an agent such that $|\firstset_j| = 1$ and $\valu_i(\firstset_j) > \valu_i(\firstset_i)$. Then, $\valu_i(\secondset_j) < \epsilon_i$.
\end{lemma} 
\begin{proof}
First, note that if $\agent_j$ is not satisfied yet, then $\secondset_j = \emptyset$ and therefore $\valu_i(\secondset_j) < \epsilon_i$. Otherwise, we argue that agent $\agent_j$ is either satisfied in the second phase, or in the refinement phases of $\cone$ or $\ctwo$.

Consider the case that $\agent_j$ is satisfied in the second phase. If $\agent_j \in \satagents_2^s \cup \satagents_3$, then by Lemma \ref{prvalue}, $\valu_i(\secondset_j) < \epsilon$ holds. Also, if $\agent_j \in \satagents_1^s$, considering the fact that $\agent_i$ envies $\agent_j$, $\agent_i \prec_{pr} \agent_j$. Thus, by Lemma \ref{prvalue}, we have $\valu_i(\secondset_j)< \epsilon_i$.

Next, consider the case that $\agent_j$ is in $\satagents_1^r \cup \satagents_2^r$. Note that the matching of the refinement phase of $\cone$ preserves the property described in Lemma \ref{nicematch}. Hence, if $\agent_j$ belongs to $\satagents_1^r$, then either $\agent_j \prec_{pr} \agent_i$ or there is no edge between $\agentv_i$ and $M_1(\agentv_j)$ in $G_1$, where $M_1(\agentv_j)$ is the vertex matched with $\agentv_j$ in $M_1$. If $\agent_j \prec_{pr} \agent_i$, according to Observation \ref{epsofcluster}, $\valu_i(\firstset_j) \leq 3/4-\epsilon_i$ holds. On the other hand, by the definition, if no edge exists between $\agentv_i$ and  $M_1(\agentv_j)$ in $G_1$, $\valu_i(\secondset_j)<\epsilon_i$. 
In addition to this, if $\agent_j$ belongs to $\satagents_2^r$, according to Lemma \ref{cr2smallc1}, $\valu_i(\secondset_j)< \epsilon_i$ holds. 
Therefore, Lemma \ref{eps} holds for the agents in $\satagents_1^r \cup \satagents_2^r$.

\end{proof}

Note that since matching $M$ of $G_{1/2}$ for building Cluster $\cone$ is $\MCMWM$ according to condition (iii) of Lemma \ref{wm}, there exists no agent $\agent_k$, such that $|\secondset_k| = 1$ and $\valu_i(\secondset_k) > 3/4 - \epsilon_i$. Otherwise, by assigning the item in $\secondset_j$ to $\agent_i$ instead of the item in $\firstset_i$, we can increase the total weight of the matching, that contradicts the maximality of $M$.

According to Lemma \ref{eps}, for all the agents $\agent_j$ with the property that $f_j$ is a blue item that belongs to a partition in $\hat{B_1}$, $\valu_i(\secondset_j)< \epsilon_i$ holds. 
The number of such agents is at least $|\hat{B_1}|$. Therefore, the total value of $\valu_i(\wcal(\items \setminus \fitems))$ is less than $2\epsilon_i \cdot (2n - 1 - |B_1| - 2|B_2|- |\hat{B_1}|) + \epsilon_i \cdot |\hat{B_1}|$. Combining the bounds obtained in Observation \ref{enough} and Lemma \ref{eps} by Inequality \eqref{fsum}, we have:
$$ \valu_i({\fitems}) \geq 2\epsilon_i \cdot(2n -2|B_2| - |B_1| -  |\hat{B_1}|) + \epsilon_i \cdot(|\hat{B_1}|) -  2\epsilon_i \cdot (2n - 1 - |B_1| - 2|B_2|- |\hat{B_1}|) - \epsilon_i \cdot |\hat{B_1}|$$
That is:
$$ \valu_i({\fitems}) \geq 2\epsilon_i$$

This contradicts the fact that the set $\fitems$ is not feasible for $\agent_i$.


\begin{comment}
Now, we observe the algorithm, from the viewpoint of $\agent_i$. For each agent $\agent_j$, consider $\firstset_j$ and $\secondset_j$ as two slots that must be filled by proper set of items during the algorithm. In the viewpoint of $\agent_i$, during the algorithm, each slot is filled, either with a single blue item, or a set of white items with value at most $\epsilon$ for $\agent_i$ ( if the set $\secondset_j$ is not feasible for $\agent_i$ ) or with a set of items that is worth at most $2\epsilon_i$ for $\agent_i$ ( if the set $S$ was feasible for $\agent_i$ in the second phase, or the slots was filled in the clustering phase by merging two vertices). 

In the deadlock, we know that at least $2|B_2| + |B_1|$ of the slots is filled by blue items. other $r = 2n - 2|B_2| - |B_1|$ slots must be filled with white items. By lemma\ref{eps}, at most $r - |\tilde{B_1}|$ of the slots must be filled with items that are worth at most $2\epsilon$. Other $|B_1|$ slots must be filled with items that are worth at most $\epsilon$ for $\agent_i$. Regarding Observation \ref{enough}, value of the white items ($W$) is enough for filling all  the slots. So, in the deadlock situation, no slot can be empty. But while $\agent_i$ is semi-satisfied, the slot $\secondset_i$ is empty and hence, $\valu_i({\fitems}) \geq 2\epsilon$ that contradicts by the deadlock situation.
\end{comment} 

\end{proof}


\begin{proof}[of Lemma \ref{c2null}]
Lemmas \ref{c3null} and \ref{c1null} state that at the end of the algorithm, $\cone = \cthree = \emptyset$. Now, let $\agent_i$ be a winner of $\ctwo$. We consider two cases separately: $\epsilon_i \geq {1/8}$ and $\epsilon_i < {1/8}$.

If $\epsilon_i \geq{1/8}$, the proof follows from a similar argument we used to prove Lemma \ref{c1null}. 

\begin{lemma}
\label{c2rem}
If $\epsilon_i \geq 1/8$, then the following inequality holds:
$$ \sum_{\agent_j \in \satagents} \valu_i( \firstset_j \cup \secondset_j) \leq |\satagents| + 1/8.$$
\end{lemma}
\begin{proof}
We know $\satagents = \satagents_1 \cup \satagents_2 \cup \satagents_3$. For every agent $\agent_j$ in $\satagents_3$, by Lemmas \ref{general} and \ref{c3fsmall}, we know $\valu_i(\firstset_j)<3/4$. Also, according to Lemma \ref{prvalue}, $\valu_i(\secondset_j)< \epsilon_i \leq 1/4$. Therefore,
\begin{equation}
\label{bow}
\sum_{\agent_j \in \satagents_3} \valu_i(\firstset_j \cup \secondset_j) \leq \sum_{\agent_j \in \satagents_3} (3/4 + 1/4) = |\satagents_3|.
\end{equation} 
Now, consider an agent $\agent_j \in \satagents_1$. Note that by Lemma \ref{forc2c3}, $\valu_i(\firstset_j)< 1/2$. Also, remark that either $\agent_j \in \satagents_1^r$  or $\agent_j \in \satagents_1^s$ . If $\agent_j \in \satagents_1^r$ then according to Lemma \ref{gsmallc1r}, $\valu_i(\secondset_j)< 1/2$ holds and hence $\valu_i(\firstset_j \cup \secondset_j)< 1$. Also, If $\agent_j \in \satagents_1^s$, then according to Lemma \ref{prvalue}, $\valu_i(\secondset_j)<2\epsilon_i<1/2$. 
Thus, in both cases, $\valu_i(\firstset_j \cup \secondset_j)< 1$ and hence:
\begin{equation}
\label{hogh}
\sum_{\agent_j \in \satagents_1} \valu_i(\firstset_j \cup \secondset_j) \leq \sum_{\agent_j \in \satagents_1} 1 = |\satagents_1|.
\end{equation} 

Finally consider a satisfied agent $\agent_j \in \satagents_2$. Again, remark that either $\agent_j \in \satagents_2^r$  or $\agent_j \in \satagents_2^s$ holds. 

Consider the case that $\agent_j \in \satagents_2^s$. If $\agent_j \prec_{pr} \agent_i$, then by Observation \ref{epsofcluster}, $\valu_i(\firstset_j) \leq 3/4 - \epsilon_i$ and by Lemma \ref{prvalue}, $\valu_i(\secondset_j) < 2\epsilon_i \leq 1/4 + \epsilon_i$ which means $\valu_i(\firstset_j \cup \secondset_j) < 1$. Moreover, if $\agent_i \prec_{pr} \agent_j$, according to Lemmas \ref{general} and \ref{prvalue}, $\valu_i(\firstset_j \cup \secondset_j) < 3/4 + \epsilon_i \leq 1 $. Thus, we have:

\begin{equation}
\label{f}
\sum_{\agent_j \in \satagents_2^s} \valu_i(\firstset_j \cup \secondset_j) \leq \sum_{\agent_j \in \satagents_2^s} 1 = |\satagents_1|
\end{equation}
It only remains to investigate the case where $\agent_j \in \satagents_2^r$. Note that since $\agent_i$ is not satisfied in the refinement phase of $\ctwo$, if $\agent_i \prec_{pr} \agent_j$, then $\valu_i(\secondset_j)< \epsilon_i \leq 1/4$. Otherwise, we could assign the item in $\secondset_j$ to $\agent_i$ in the refinement phase of $\ctwo$. Also, by Lemma \ref{general}, $\valu_i(\firstset_j)<3/4$ holds which yields $\valu_i(\firstset_j \cup \secondset_j)<1$.  

Finally, if $\agent_j \prec_{pr} \agent_i$, by Observation \ref{epsofcluster} $\valu_i(\firstset_j) \leq 3/4 - \epsilon_i$ holds. Corollary \ref{forc2small} states that there is at most one item $\ite_k$ with $\items_k \in \itemsv' \setminus \itemsv'_{1/2}$ and $\valu_i(\ite_k) \geq 3/8$. Also, note that since $\ite_k$ belongs to  $\itemsv' \setminus \itemsv'_{1/2}$, $\valu_i(\{\ite_k\})<1/2$ holds. For agent $\agent_j$, let $\ite_l$ be the item that is assigned to $\agent_j$ in the refinement of $\ctwo$, i.e., $\secondset_j = \{\ite_l\}$. We have $$\valu_i(\firstset_j \cup \secondset_j) \leq 3/4 - \epsilon_i + \valu_i(\{\ite_l\}).$$  
If $\ite_l \neq \ite_k$, $\valu_i(\firstset_j \cup \secondset_j) \leq 3/4 - \epsilon_i + 3/8$ holds which by the fact that $\epsilon_i \geq{1/8}$, implies $\valu_i(\firstset_j \cup \secondset_j) \leq 3/4 -1/8 + 3/8 \leq 1$. In addition to this, If $\ite_l = \ite_k$, 
$\valu_i(\firstset_j \cup \secondset_j) \leq 3/4 - 1/8 + 4/8 \leq 1+1/8$. But this can happen for at most one agent. Thus, for every agent $\agent_j$ in $\satagents_2^r$, $\valu_i(\firstset_j \cup \secondset_j) \leq 1$ holds and for at most one agent $\agent_j \in \satagents_2^r$, $\valu_i(\firstset_j \cup \secondset_j) \leq 1+1/8$. Thus, we have

\begin{equation}
\label{u}
\sum_{\agent_j \in \satagents_2^r} \valu_i(\firstset_j \cup \secondset_j) \leq |\satagents_2^r| + 1/8.
\end{equation}
Inequality (\ref{u}) together with Inequality (\ref{f}) yields
\begin{equation}
\label{oh}
\sum_{\agent_j \in \satagents_2} \valu_i(\firstset_j \cup \secondset_j) \leq |\satagents_2| + 1/8.
\end{equation}
Furthermore, by Inequalities (\ref{bow}), (\ref{hogh}) and (\ref{oh}) we have
\begin{equation}
\begin{split}
\sum_{\agent_j \in \satagents} \valu_i( \firstset_j \cup \secondset_j)  & = \sum_{\agent_j \in \satagents_1} \valu_i( \firstset_j \cup \secondset_j) + \sum_{\agent_j \in \satagents_2} \valu_i( \firstset_j \cup \secondset_j) + \sum_{\agent_j \in \satagents_3} \valu_i( \firstset_j \cup \secondset_j)\\
& \leq |\satagents_1| + |\satagents_2|+ 1/8 + |\satagents_3| \\
& \leq |\satagents|+1/8. \\
\end{split}
\end{equation}
 
\end{proof}


By Lemma \ref{c2rem}, value of agent $\agent_i$ for the items assigned to the satisfied agents is less than $|\satagents| + 1/8$. Recall that $\ctwo = \cthree = \emptyset$ and hence $|\satagents| = n - |\ctwo|$. Therefore,
\begin{equation}
\sum_{\agent_j \in \satagents} \valu_i( \firstset_j \cup \secondset_j) \leq n - |\ctwo| + 1/8.
\end{equation}
Since $\agent_i$ is a winner of $\ctwo$, for all $\agent_j \in \ctwo$, we have  $\valu_i(\firstset_j)\leq \valu_i(\firstset_i)$. On the other hand, since the total value of all items for $\agent_i$ is equal to $n$ we have

\begin{equation}\label{yy1}
\begin{split}
\valu_i({\fitems}) & = \valu_i(\items) - \sum_{\agent_j \in \ctwo} {\valu_i(\firstset_j)} - \sum_{\agent_j \in \satagents}\valu_i(\firstset_j \cup \secondset_j)\\
& = n - \sum_{\agent_j \in \ctwo} {\valu_i(\firstset_j)} - \sum_{\agent_j \in \satagents}\valu_i(\firstset_j \cup \secondset_j)\\
& \geq n - \sum_{\agent_j \in \ctwo} {\valu_i(\firstset_j)} - \big[ n - |\ctwo| + 1/8 \big] \\
& = |\ctwo| - 1/8 - \sum_{\agent_j \in \ctwo} {\valu_i(\firstset_j)}.
\end{split}
\end{equation}

Also, $\valu_i(\firstset_i) = {3/4} - \epsilon_i$ holds and $\valu_i(\firstset_j) \leq \valu_i(\firstset_i)$ for any $\agent_j \in \ctwo$ follows from the fact that $\agent_i$ is a winner of $\ctwo$. Therefore by Inequality (\ref{yy1}) we have
\begin{equation*}
\begin{split}
\valu_i({\fitems}) & \geq |\ctwo| - 1/8 - \sum_{\agent_j \in \ctwo} {\valu_i(\firstset_j)}\\
& \geq |\ctwo| - 1/8 - \sum_{\agent_j \in \ctwo} {\valu_i(\firstset_i)}\\
& = |\ctwo| - 1/8 - |\ctwo| {\valu_i(\firstset_i)}\\
& = |\ctwo| - 1/8 - |\ctwo| ({3/4} - \epsilon_i)\\
& = |\ctwo| ({1/4} + \epsilon_i) - 1/8.
\end{split}
\end{equation*}
%Furthermore since $\epsilon_i \leq 1/4$, $$\valu_i({\fitems}) > |\ctwo|({1/4}+\epsilon_i) - 1/8.$$ 
Recall that by the assumption $\epsilon_i \geq 1/8$ holds. Moreover, $\epsilon_i\leq 1/4$, and thus
\begin{equation*}
\begin{split} 
\valu_i(\fitems) & \geq |\ctwo| ({1/4} + \epsilon_i) - 1/8\\
& \geq |\ctwo| 2\epsilon_i - 1/8\\
& \geq |\ctwo|  2\epsilon_i - \epsilon_i
\end{split}
\end{equation*}
and since $|\ctwo| \geq 1$, 
\begin{equation*}
\begin{split}
\valu_i(\fitems) &\geq |\ctwo|  2\epsilon_i - \epsilon_i\\
& \geq 2\epsilon_i - \epsilon_i\\
& \geq \epsilon_i
\end{split}
\end{equation*}
 and thus $\fitems$ is feasible for $\agent_i$. This contradicts the termination of the algorithm. 

Next, we investigate the case where $\epsilon < {1/8}$. Our proof for this case is similar to the one for $\cone$. Let $\satagents^r_{1}$ be the agents in $\satagents_1$ that are satisfied in the refinement phase and let $$\items^r_1 = \bigcup_{\agent_j \in \satagents^r_1} \firstset_j \cup \secondset_j.$$ Lemma \ref{forc2} states that the maxmin value of the agents in $\ctwo \cup \cthree$ for the items in $\items' = \items \setminus \items^r_1$ is at least $1$. More precisely for every $\agent_j \in \ctwo$:
\begin{equation}
\label{c1refine}
 \MMS_{j}^{n- |\satagents^r_{1}|} ( {\items} \setminus \items^r_{1}) \geq 1 
 \end{equation}  

We color the items of $\items'$ in one of four colors blue, red, green, or white. Initially, all the items are colored in white. For each agent $\agent_j \in \agents \setminus \satagents^r_1$, if $|\firstset_j|=1$, then we color the item in $\firstset_j$ in blue. Also, if $|\firstset_j|=2$ (which means $\firstset_j$ is corresponding to a merged vertex), color both the elements of $\firstset_j$ in red. In addition to this, if $|\secondset_j|=1$ then color the item in $\secondset_j$ in green. For any set $S \subseteq \items$, we denote the subset of blue, red, green, and white items in $S$ by $\mathcal{B}(S)$,$\mathcal{R}(S)$, $\mathcal{G}(S)$, and $\mathcal{W}(S)$, respectively. Recall that by Lemma \ref{pairsmall}, every pair of items in red or green are worth less that $3/4$ in total to $\agent_i$. In other words,
\begin{equation*}
\valu_i(\{\ite_j, \ite_k\}) \leq 3/4.
\end{equation*}
for any two different items $\ite_j, \ite_k \in \mathcal{B}(\items) \cup \mathcal{G}(\items)$.
  Also, according to Lemmas \ref{prvalue} and \ref{c3fsmall}, every set including white items is worth less than $2\epsilon_i < 1/4$ to $\agent_i$. 

Now, let $n' = n - |\satagents^r_1|$. Let $\cal P$ $= \langle P_1, P_2, \ldots, P_{n'} \rangle$ be the optimal $n'-$partitioning of $\items' $ for $\agent_i$. Recall that by Inequality \eqref{c1refine} the value of every partition in $\cal P$ is at least $1$ for $\agent_i$. Based on the number of blue and red items in every partition, we define three sets of partitions:
\begin{itemize}
    \item $B_{00}:$ Partitions with no red or blue items.
    \item $B_{10}:$ Partitions with blue items, but without any red items.
    \item $B_{01}:$ Partitions that contain at least one red item.
\end{itemize}

Next we prove Lemmas \ref{lowerwhite1} and \ref{lowerwhite2} to be used later in the proof. 
\begin{lemma}
\label{lowerwhite1}
Let $|{\mathcal{G}}(B_{00})|$ be the number of green items in the partitions of $B_{00}$. Then, $$\valu_i(\wcal(B_{00})) \geq (3|B_{00}| - |{\mathcal{G}}(B_{00})|)\cdot 1/4.$$
\end{lemma}
\begin{proof}
Let $B_{00}^j$ be the set of partitions in $B_{00}$ that contain exactly $j$ green items. We have:
\begin{equation}
\label{gbound}
|\mathcal{G}(B_{00})| = \sum_{1 \leq j < \infty} j  |B_{00}^j| \geq |B_{00}^1| + 2|B_{00}^2| + \sum_{3 \leq j < \infty} 3  |B_{00}^j| 
\end{equation}
Also, we have:
\begin{equation}
\label{bbound}
3 |B_{00}| = \sum_{0 \leq j < \infty} 3  |B_{00}^j| = 3  |B_{00}^0| + 3  |B_{00}^1| + 3  |B_{00}^2| + \sum_{3 \leq j < \infty} 3  |B_{00}^j|
\end{equation}
Finally, we argue that the value of white items in $B_{00}$ is at least $|B_{00}^0| + |B_{00}^1|\cdot 1/2 + |B_{00}^3|\cdot 1/4$. This follows from the fact that every green item in $P_k \in B_{00}^1$ has a value less than $1/2$ and by Lemma \ref{pairsmall}, every pair of green items in $P_k \in B_{00}^2$ are worth less than $3/4$ to $\agent_j$. According to the fact that the value of every partition $P_k$ is at least 1, we have:

\begin{equation}
\label{wbound}
\valu_i({\cal{W}}(B_{00})) \geq |B_{00}^0| + |B_{00}^1|\cdot 1/2 + |B_{00}^3|\cdot 1/4 = \big( 4|B_{00}^0| + 2|B_{00}^1| + |B_{00}^2| \big)\cdot 1/4
\end{equation}  

According to Equations \eqref{gbound} and \eqref{bbound}, we have:
\begin{equation}
\label{tbound}
3 |B_{00}| - |\mathcal{G}(B_{00})| \leq 3|B_{00}^0| + 2|B_{00}^1| + |B_{00}^2| \leq 4|B_{00}^0| + 2|B_{00}^1| + |B_{00}^2|
\end{equation}

Next we combine Equations \eqref{wbound} and \eqref{tbound} to obtain: 
\begin{equation}
\valu_i({\cal{W}}(B_{00})) \geq \big( 3 |B_{00}| - |\mathcal{G}(B_{00})| \big)\cdot 1/4
\end{equation}

\end{proof}



\begin{lemma}
\label{lowerwhite2}
$\valu_i(\wcal(B_{10})) \geq (2|B_{10}| - |{\mathcal{B}}(B_{10}) | - |{\mathcal{G}}(B_{10})| )\cdot 1/4$
\end{lemma}
\begin{proof}
First, note that every partition in $B_{10}$ contains at least one blue item. Let $B_{10}^{w}$ be the partitions in $B_{10}$ that contains exactly one blue item and no green item. The other items in each partition of $B_{10}^{w}$, are white. Since the problem is $3/4$-irreducible, the value of every blue item to $\agent_i$ is less than $3/4$ and therefore we have:
$$\valu_i(\wcal(B_{10})) \geq |B_{10}^{w}|\cdot 1/4$$
or 
\begin{equation}
\label{wbound2}
4\valu_i(\wcal(B_{10})) \geq |B_{10}^{w}|.
\end{equation}
Moreover, let $B_{10}^{\bar{w}} = B_{10} \setminus B_{10}^w$. Since every partition in $B_{10}$ contains at least one blue item, every partition in $B_{10}^{\bar{w}}$ contains at least two items with colors blue or green. Thus, we have:
\begin{equation}
\label{gbound2}
 |{\mathcal{G}}(B_{10}^{\bar{w}})| + |{\mathcal{B}}(B_{10}^{\bar{w}})| \geq 2|B_{10}^{\bar{w}}|
\end{equation}
Summing up Equations \eqref{wbound2} and \eqref{gbound2} results in 
$$
4\valu_i(\wcal(B_{10})) +  |{\mathcal{G}}(B_{10}^{\bar{w}})| + |{\mathcal{B}}(B_{10}^{\bar{w}})| \geq 2|B_{10}^{\bar{w}}| + |B_{10}^{w}|
$$
which means:
\begin{equation}
\label{fbound}
4\valu_i(\wcal(B_{10})) \geq 2|B_{10}^{\bar{w}}| -  |{\mathcal{G}}(B_{10}^{\bar{w}})| - |{\mathcal{B}}(B_{10}^{\bar{w}})|  + |B_{10}^{w}|.
\end{equation}

Morover, we have $|{\mathcal{B}}(B_{10})| = |{\mathcal{B}}(B_{10}^{w})|+|{\mathcal{B}}(B_{10}^{\bar{w}})|$. According to the fact that every partition in $B_{10}^w$ contains exactly one blue item, $|{\mathcal{B}}(B_{10}^{w})| = |B_{10}^w|$ and hence, $|{\mathcal{B}}(B_{10})| = |B_{10}^w|+|{\mathcal{B}}(B_{10}^{\bar{w}})|$. By Equation \eqref{fbound}, we have:
$$ 4\valu_i(\wcal(B_{10})) \geq 2|B_{10}^{\bar{w}}| -  |{\mathcal{G}}(B_{10}^{\bar{w}})| -|{\mathcal{B}}(B_{10})|+ |B_{10}^w|  + |B_{10}^{w}|. $$
Finally by the fact that $2|B_{10}^{w}| + 2|B_{10}^{\bar{w}}| = 2|B_{10}|$, we have:
$$ 4\valu_i(\wcal(B_{10})) \geq 2|B_{10}| -  |{\mathcal{G}}(B_{10}^{\bar{w}})| -|{\mathcal{B}}(B_{10})| $$
which is:
$$ \valu_i(\wcal(B_{10})) \geq \big(2|B_{10}| -|{\mathcal{B}}(B_{10})|-  |{\mathcal{G}}(B_{10}^{\bar{w}})| \big) \cdot 1/4$$
\end{proof}


For the partitions in $B_{01}$, we construct a graph $G_{01} \langle V_{01},E_{01} \rangle$, where every vertex $v_j \in V_{01}$ corresponds to a partition $P_j \in B_{01}$. Consider an agent $\agent_j$ such that $\firstset_j$ consists of a pair of red items $\ite_k,\ite_{k'}$ and let $\ite_k \in P_l$ and $\ite_{k'} \in P_{l'}$. We add an edge $(v_l,v_{l'})$ to $E_{01}$. By the definition of $B_{01}$, $P_l,P_{l'} \in B_{01}$ holds. Note that $\ite_k$ and $\ite_{k'}$ might belong to the same partition, i.e., $P_l = P_{l'}$. In this case, we add a loop to $G_{01}$. Furthermore, for every item $\ite_k \in {\mathcal B}(B_{01})$, we add a loop to the vertex $v_l$, where $\ite_k \in P_l$. 

Next, define $R_j$ as the set of partitions in $B_{01}$, such that the degree of their corresponding vertices in $V_{01}$ are equal to $j$. In other words:
$$P_k \in R_j \iff d(v_k)=j$$

Next we prove Lemma \ref{lowerwhite3}.

\begin{lemma}
\label{lowerwhite3}
For $R_1$, we have: $$\valu_i(\wcal(R_1)) \geq  (2|R_1| - |{\mathcal{G}}(R_1)|)\cdot 1/4 $$
\end{lemma}
\begin{proof}
Consider a partition $P_j \in R_1$. Since $d(v_j)=1$, $P_j$ contains exactly one red item and no blue item. Thus, other items in $P_j$ are either green or white. We show that 
\begin{equation}
\label{newway}
|{\mathcal G}(P_j)| + 4.\valu_i(\wcal(P_j)) \geq 2.
\end{equation}
First, argue that if $|{\mathcal G}(P_j)| \geq 2$, then Inequality \eqref{newway} holds. Also, if $|{\mathcal G}(P_j)|=0$, then $\valu_i(\wcal(P_j)) \geq 1/2$, because the value of the red item in $P_j$ is less than $1/2$ (recall that all the red items correspond to the vertices in $\itemsv' \setminus \itemsv'_{1/2}$). This immediately implies the fact that $4.\valu_i(\wcal(P_j)) \geq 2$. Finally, if $|{\mathcal G}(P_j)|=1$, then by Lemma \ref{pairsmall}, the total value of the green and red items in $P_j$ is less than $3/4$ and hence, $\valu_i(\wcal(P_j)) \geq 1/4$ which means $|{\mathcal G}(P_j)| + 4.\valu_i(\wcal(P_j)) \geq 2$.

Since Inequality \eqref{newway} holds for every partition $P_j \in R_1$, we have:

$$\sum_{P_j \in R_1} \big(|{\mathcal G}(P_j)| + 4.\valu_i(\wcal(P_j))\big) \geq 2 |R_1|$$
Therefore,
 $$|{\mathcal G}(R_1)| + 4.\valu_i(\wcal(R_1)) \geq 2|R_1|$$ and hence, $$\valu_i(\wcal(R_1)) \geq  (2|R_1| - |{\mathcal{G}}(R_1)|)\cdot 1/4 $$
\end{proof}

\begin{lemma}
\label{lowerwhite4}
For $R_2$, we have: $$ \valu_i(\wcal(R_2)) \geq (|R_2| - |{\mathcal{G}}(R_2)|)\cdot 1/4 $$
\end{lemma}
\begin{proof}
Let $P_j$ be a partition in $R_2$. First, we show the following inequality holds:
\begin{equation}
\label{gwbound2}
4\valu_i(\wcal(P_j)) + |{\mathcal G}(P_j)| \geq 1
\end{equation}

By the definition of $R_2$, degree of $v_j$ is $2$. Therefore, $P_j$ contains two red items. Note that the degree of the partitions in $B_{01}$ that contain blue items is at least $3$. Thus, $P_j$ contains no blue items. By Lemma \ref{pairsmall}, the total value of the red items in $P_j$ is less than $3/4$. The rest of the items in $P_j$ are either green or white. If $P_j$ contains a green item, then Inequality \eqref{gwbound2} holds. On the other hand, if $P_j$ contains no green items, then $\valu_i(\wcal(P_j)) \geq 1/4$ and hence, $4\valu_i(\wcal(P_j)) \geq 1$. Therefore, Inequality \eqref{gwbound2} holds in both cases. 

Summing up Inequality \eqref{gwbound2} for all the partitions in $R_2$, we have:

$$
\sum_{P_j \in R_2} 4\valu_i(\wcal(P_j)) + |{\mathcal G}(P_j)| \geq |R_2|
$$
which means:
$$
4\valu_i(\wcal(R_2)) + |{\mathcal G}(R_2)| \geq |R_2|
$$
That is:
$$
\valu_i(\wcal(R_2)) \geq \big(|R_2| - |{\mathcal G}(R_2)|\big)\cdot 1/4  
$$
\end{proof}

Putting together Lemmas \ref{lowerwhite1},\ref{lowerwhite2},\ref{lowerwhite3}, and \ref{lowerwhite4} we obtain the following lower bound on the valuation of $\agent_i$ for all white items:
\begin{equation}\label{w1}
\begin{split}
\valu_i(\wcal(\items')) & = \valu_i(\wcal(B_{00})) + \valu_i(\wcal(B_{01})) + \valu_i(\wcal(B_{10}))\\
& \geq \bigg(3|B_{00}| - |{\mathcal{G}}(B_{00})|\bigg)\cdot 1/4 + \bigg(2|B_{10}| - |{\mathcal{B}}(B_{10})| - |{\mathcal{G}}(B_{10})|\bigg)\cdot 1/4 \\ & \hspace{10pt}+ \bigg(2|R_1| - |{\mathcal{G}}(R_1)|\bigg)\cdot 1/4 + \bigg(|R_2| - |{\mathcal{G}}(R_2)|\bigg)\cdot 1/4\\
&= \bigg((3|B_{00}| + 2|B_{10}| + 2|R_1| + |R_2| - |{\mathcal{B}}(B_{10})|) - \big(|{\mathcal{G}}(B_{00})| + |{\mathcal{G}}(B_{10})| + |{\mathcal{G}}(R_1)| + |{\mathcal{G}}(R_2)|\big) \bigg)\cdot 1/4\\
& \geq \bigg((3|B_{00}| + 2|B_{10}| + 2|R_1| + |R_2| )- |{\mathcal{B}}(B_{10})| -  |{\mathcal{G}}(\items')| \bigg)\cdot 1/4
\end{split}
\end{equation}
where $|{\mathcal{G}}(\items')|$ is the total number of green items. 



The items in $\wcal(\items')$ are either allocated to an agent during the second phase, or are still in $\fitems$. Let $\wcal_2$ be the white items that are allocated to an agent during the second phase. We have: 
\begin{equation}
\valu_i(\wcal(\items')) = \valu_i(\wcal_2) + V_i(\fitems)
\end{equation}
Now, we present an upper bound on the value of $\valu_i(\wcal_2)$. First, note that the number of agents in $\satagents \setminus \satagents^r_1$ is $n'$. Each of these $n'$ agents has two sets $\firstset_j$ and $\secondset_j$, that leaves us $2n'$ sets. Since $\secondset_i = \emptyset$ we know that at least one of these sets is empty. Moreover, of all these $|{\mathcal{G}}(\items')|$ sets contain a single green item and $|{\mathcal{B}}(B_{10})| + |E_{01}|$ of the sets contain either a single blue item, or a pair of red items (recall that each edge of $G_{01}$ refers to a blue item or a pair of red items). Therefore, the number of the sets that contain only white items is at most:
$$2n' - 1 - |{\mathcal{G}}(\items')| - |{\mathcal{B}}(B_{10})| - |E_{01}|$$

By Lemmas \ref{prvalue} and \ref{c3fsmall}, the value of every set with white items to $\agent_i$ is less than $2\epsilon_i<1/4$ and hence:
\begin{equation}
\label{w_2}
\valu_i(\wcal_2) \leq (2n' - 1 - |{\mathcal{G}}(\items')| - |{\mathcal{B}}(B_{10})| - |E_{01}|)\cdot 1/4
\end{equation}
Subtracting the lower bound obtained for $\valu_i(\wcal(\items'))$ in \eqref{w1} from the upper bound for $\valu_i(\wcal_2)$ in $\eqref{w_2}$ gives us a lower bound on the value of $\fitems$:
\begin{equation}\label{bachekhoshgel}
\begin{split}
\valu_i(\fitems) &= \valu_i(\wcal(\items')) - \valu_i(\wcal_2)\\
 &\geq \bigg((3|B_{00}| + 2|B_{10}| -|{\mathcal{B}}(B_{10})| + 2|R_1| + |R_2|) -  |{\mathcal{G}}(\items')| \bigg)\cdot 1/4 - \valu_i(\wcal_2)\\
 &\geq \bigg((3|B_{00}| + 2|B_{10}| - |{\mathcal{B}}(B_{10})| + 2|R_1| + |R_2|) -  |{\mathcal{G}}(\items')| \bigg)\cdot 1/4\\
 & \qquad - \bigg(2n' - 1 - |{\mathcal{G}}(\items')| - |{\mathcal{B}}(B_{10})| - |E_{01}|\bigg)\cdot 1/4 \\
&= \bigg(3|B_{00}|+2|B_{10}| + 2|R_1| + |R_2| -2n' + 1 + |E_{01}| \bigg)\cdot 1/4 \\
&= \bigg(2|B_{00}|+2|B_{10}| + |B_{00}| + |E_{01}| + 2|R_1| + |R_2| - 2n' + 1\bigg) \cdot 1/4 
\end{split}
\end{equation} 
Next we provide Lemmas \ref{ebound}, \ref{B00size}, and \ref{Esize} to complete the proof.
\begin{lemma}
\label{ebound}
$|B_{00}| \geq |E_{01}| - |B_{01}|$ 
\end{lemma}
\begin{proof}
First, note that $|B_{00}| + |B_{10}| + |B_{01}|=n'$. Moreover we have $|{\mathcal B}(B_{10})| + |E_{01}| \leq n'$. To show this Lemma, note that each edge in $G_{01}$ corresponds to the first set of an agent in $\satagents \setminus \satagents_1^r$. Also, every blue item in $B_{10}$ corresponds to the first set of an agent in $\satagents \setminus \satagents_1^r$. Therefore, the total number of the agents must be more than this number. By the definition of $B_{10}$, we know that $|{\mathcal B}(B_{10})| \geq |B_{10}|$. Therefore, we have: 

\begin{equation}
\begin{split}
|B_{00}| + |B_{10}| + |B_{01}| &\geq  |{\mathcal B}(B_{10})| + |E_{01}|\\
 &\geq |(B_{10})| + |E_{01}|\\
\end{split}
\end{equation} 
This means:
$$ |B_{00}| \geq |E_{01}| - |B_{01}| $$
\end{proof}

\begin{lemma}
\label{Esize}
$|E_{01}| \geq 3/2 \sum_{j \geq 3}|R_j| + |R_2| + |R_1|/2$ 
\end{lemma}
\begin{proof}
$|E_{01}| = \frac{\sum_{v_j \in V_{01}} d(v_j)}{2} = \frac{\sum_j j|R_j|}{2} \geq 3/2 \sum_{j \geq 3}|R_j| + |R_2| + |R_1|/2.$
\end{proof}

\begin{lemma}
\label{B00size}
$|B_{00}| \geq \frac{\sum_{j \geq 3}|R_j| - |R_1|}{2}$
\end{lemma}
\begin{proof}
By Lemma \ref{ebound}, $|B_{00}| \geq |E_{01}| - |B_{01}| $. Furthermore, by Lemma \ref{Esize}, $$|E_{01}| \geq 3/2 \sum_{j \geq 3}|R_j| + |R_2| + |R_1|/2.$$ By these two inequalities, we have:
\begin{equation}
\label{Ebd}
|B_{00}| \geq 3/2 \sum_{j \geq 3}|R_j| + |R_2| + |R_1|/2 - |B_{01}| 
\end{equation}
Also, since there is a one-to-one correspondence between $B_{01}$ and $V_{01}$, $|B_{01}| = |V_{01}|$ holds. By the definition of $R_j$, we have: 
\begin{equation}
\label{Vbd}
|V_{01}| = \sum_j |R_j| 
\end{equation}

By replacing the value obtained for $B_{01}$ from \eqref{Vbd} into Inequality \eqref{Ebd}, we have:
\begin{equation}
\begin{split}
|B_{00}| &\geq 1/2 \sum_{j \geq 3}|R_j| - |R_1|/2 \\
& = \frac{\sum_{j \geq 3}|R_j| - |R_1|}{2}.
\end{split} 
\end{equation}
  
\end{proof}



By applying Lemmas \ref{B00size} and \ref{Esize} to Inequality \eqref{bachekhoshgel} we have:
\begin{equation*}
\begin{split}
\valu_i(\fitems) & = \bigg(2|B_{00}|+2|B_{10}| + |B_{00}| + |E_{01}| + 2|R_1| + |R_2| - 2n' + 1\bigg) \cdot 1/4\\
&\geq \bigg(2|B_{00}|+2|B_{10}| + \frac{\sum_{j \geq 3}|R_j| - |R_1|}{2} + 3/2 \sum_{j \geq 3}|R_j| + |R_2| + |R_1|/2 + 2|R_1| + |R_2| - 2n' + 1\bigg)\cdot 1/4 \\
&= \bigg(2|B_{00}|+2|B_{10}| + \sum_{j \geq 3} 2|R_j| + 2|R_2| + 2|R_1| - 2n' + 1 \bigg)\cdot 1/4
\end{split}
\end{equation*} 
Finally, note that $\sum_{j \geq 3} 2|R_j| + 2|R_2| + 2|R_1| = 2|V_{01}| = 2|B_{01}|$. This, together with the fact that $|B_{00}| + |B_{01}| + |B_{10}| = n'$, yields $\valu_i(\fitems) \geq (2n' - 2n' + 1)\cdot 1/4$. This means $\valu_i(\fitems) \geq 1/4$ which is a contradiction since $\fitems$ is feasible for $\agent_i$.
\begin{comment}
\begin{lemma}
\label{capability}
Total items in $cal M$ is enough to fill in $2n$ slots. 
\end{lemma}

\begin{proof}

The blue items, can singly fill in a slot. Also, each pair of red items can fill a slot. For simplicity, we assume that each red item can fill a half slot. 

Now, consider the items from the viewpoint of $\agent_i$. Each partition in $B_{10}$ is enough to fill in $2|B_{10}|$ slots. Also, all the partitions $b_i \in B_{01}$ with $b_i \in R_1$ is enough to fill in $ 2.5$ slots. Furthermore every partition $b_i \in B_{01}$ with $b_i \in R_3$ is capable of filling $1.5$ slots. By lemma \ref{B00size}, we know that the number of such partitions is less than $\frac{B_{00}}{2}+ R_1$. On the other hand, each partition in $B_{00}$ is capable of filling $3$ slots. So, total number of slots that can be filled by $\cal M$ is at least:
\begin{align*} 
2|B_{10}| + 3|B_{00}| + 2.5|R_1| + 1.5|R_3| \\ > 2|B_{10}|+ 2|B_{00}| + 2|R_1| + 2 |R_3| - (2*\frac{|R_3|}{2} - \frac{|B_{00}|}{2} + \frac{|R_1|}{2}) \\ > 2|B_{10}|+ 2|B_{00}| + 2|R_1| + 2 |R_3| = 2n
\end{align*}

 So, total items in $\cal M$ is capable of filling $2n$ slots, from the viewpoint of $\agent_i$. 
\end{proof}

So, by lemma \ref{capability}, $\agent_i$ can fill in $2n$ slots, that contradicts the deadlock situation. 
\end{comment}
\end{proof}


\section{Omitted Proofs of Section \ref{submodular}}\label{submodular-appendix}

\begin{observation}\label{obs_E1}
$f^x(S) \leq x$ for every given $S$.
\end{observation}

\begin{observation}\label{obs_E2}
$f^x(S) \leq f(S)$ for every given $S$.
\end{observation}

\begin{proof}[Of Lemma \ref{ceilingfunctions}]
\\
\textbf{First Claim:} By definition of submodular functions, for given sets $A$ and $B$ we have:
$$ f(A \cup B) \leq f(A) + f(B) - f(A \cap B)  $$
We prove that $f^x(.)$ is a submodular function in three different cases:\\

First Case: Let both $f(A)$ and $f(B)$ be at least $x$. According to Observation \ref{obs_E1}, $f^x(A \cup B)$ and $f^x(A \cap B)$ are bounded by $x$. Therefore, $f^x(A \cup B) + f^x(A \cap B) \leq 2x$, which yields: 
$$f^x(A \cup B) + f^x(A \cap B) \leq f^x(A) + f^x(B)$$

Second Case: In this case one of $f(A)$ and $f(B)$ is at least $x$. We have $f(A \cup B) \geq x$ and $f(A \cap B)$ is no more than max $\{f(A), f(B)\}$. As a result $f^x(A \cup B)$ and one of $f^x(A)$ or $f^x(B)$ are equal to $x$ which yields:
$$f^x(A \cup B) + f^x(A \cap B) \leq f^x(A) + f^x(B)$$

Third Case: In this case both $f(A)$ and $f(B)$ are less than $x$, and $f(A \cap B)$ is less than $x$ too. Since $f^x(A) = f(A)$, $f^x(B) = f(B)$, $f^x(A \cap B) = f(A \cap B)$, according to Observation \ref{obs_E2}, $f^x(A \cup B) \leq f(A \cup B)$ holds. Since $f(.)$ is a submodular function, we conclude that:
$$f^x(A \cup B) \leq f^x(A) + f^x(B) - f^x(A \cap B).$$\\
\textbf{Second Claim:} Since $f(.)$ is an XOS set function, by definition, there exists a finite set of additive functions  $\{f_1, f_2, \ldots, f_{\alpha}\}$ such that $$f(S) = \max_{i=1}^{\alpha} f_i(S)$$ for any set $S \subseteq \domp(f)$. With that in hand, for a given real number $x$, we define an XOS set function $g(.)$, and show $g(.)$ is equal to $f^{x}(.)$.

We define $g(.)$ on the same domain as $f(.)$. Moreover, based on $\{f_1, f_2, \ldots, f_{\alpha}\}$, we define a finite set of additive functions $\{g_1, g_2, \ldots, g_{\beta}\}$ that describe $g$. More precisely, for each set $S$ in domain of $f(.)$ we define a new additive function like $g_{\gamma}$ in $g(.)$ as follows: Without loss of generality let $f_{\delta}$ be the function which maximizes $f(S)$. For each $b_i \notin S$ let $g_{\gamma}(b_i) = 0$. Furthermore, for each $b_i \in S$ if $f(S) \leq x$ let $g_{\gamma}(b_i) = f_{\delta}(b_i)$, and otherwise let $g_{\gamma}(b_i) = \frac{x}{f(S)} f_{\delta}(b_i)$. 

We claim that $g(.)$ is equivalent to $f^x(.)$, which implies $f^x(.)$ is an XOS function. $g(.)$ and $f^x(.)$ are two functions which have equal domains. First, we prove that $g(S) \leq f(S)$ for any given set $S$. According to construction of $g(.)$, for each additive function in $g(.)$ such $g_{\gamma}$, there is at least one additive function in $f(.)$ such $f_{\delta}$ where $g_{\gamma}(b_i) \leq f_{\delta}(b_i)$ for each $b_i \in \items$. Therefore, for any given set $S$ we have:
 \begin{equation}\label{hineq1} g(S) \leq f(S)  \end{equation} 
Now, according to the construction of $g(.)$, for any given set $S$ where $f(S) \leq x$, we have a function $g_{\gamma}(S) = f(S)$, and where $f(S) > x$, we have a function $g_{\gamma}(S) = x$. Therefore, we can conclude that: 
 \begin{equation}\label{hineq2} g(S) \geq f^x(S)  \end{equation} 
 
For any given set $S$ where $f(S) \leq x$, according to the definition of $f^x(.)$, $f(S) = f^x(S)$, and using Inequalities \eqref{hineq1} and \eqref{hineq2} we argue that $f^x(S) = g(S)$. Moreover, according to the construction of $g(.)$, $g(S) \leq x$ for any given set $S$. Therefore, for any given set $S$ where $f(S) > x$, according to the definition of $f^x(.)$ and Inequality \eqref{hineq2}, $f^x(S) = g(S) = x$. As a result, by considering these two cases we argue that $f^x(.)$ and $g(.)$ are equivalent, which shows $f^x(.)$ is an XOS function.\\
\textbf{Third Claim:} In this claim, we use a similar argument to the first claim. By definition of subadditive functions for any given sets $A$ and $B$, we have:
$$f(A \cup B) \leq f(A) + f(B)$$
We prove that $f^x(.)$ meets the definition of subadditive functions by considering two different cases. In the first case at least one of $f(A)$ and $f(B)$ is at least $x$, and in the second case both $f(A)$ and $f(B)$ is less than $x$.\\

First Case: In this case $f^x(A) + f^x(B)$ is at least $x$, and since $f^x(S) \leq x$ for any given set $S$, $f^x(A \cup B) \leq x$. Therefore, 
$$f^x(A \cup B) \leq f^x(A) + f^x(B)$$

Second Case: Since $f^x(A \cup B) \leq f(A \cup B)$, $f(A \cup B) \leq f(A) + f(B)$, $f(A) = f^x(A)$, and $f(B) = f^x(B)$, we have:
$$f^x(A \cup B) \leq f^x(A) + f^x(B)$$
 
\end{proof}

\begin{comment}
\begin{proof}[Of Lemma \ref{submodularsefr}]
Suppose for the sake of contradiction that there is at least one item $b_j$ where $V_k(\{b_j\}) \geq 1/3$ for $a_k$. Now, let $T = \{a_k\}$, $S = \{b_j\}$, and $\mathcal{A} = \langle A_1, A_2, \ldots, A_n\rangle$ be an allocation of $S$ to agents of $T$ where $A_k = \{b_j\}$, and $A_i = \emptyset$ for any $i \neq k$. Since $V_k(\{b_j\}) \geq 1/3$ and $b_j$ is the only member of $A_k$, we have $V_k(A_k) \geq 1/3$. Therefore: 
\begin{equation}\label{hineq3} \forall \agent_i \in T \hspace{3cm} V_i(A_i) \geq 1/3  \end{equation} 

According to the definition of $\MMS$, any agent $a_i$ can allocate $\items$ to all $n$ agents in such a way that the value of each of these sets is at least $1$ for $a_i$. Since $|S| = 1$, the unique member of $S$ in of only one of these $n$ sets for each agent $a_i$. Therefore, any agent $a_i$ can divide $\items \setminus S$ to at least $n-1$ sets in order the value of each of these $n-1$ sets be at least one for $a_i$. Using allocation $A$:
\begin{equation}\label{hineq4} \forall \agent_i \notin T \hspace{1cm} \MMS_{V_i}^{n-|T|} (\items \setminus S)\geq 1  \end{equation} 

Inequalities \eqref{hineq3} and \eqref{hineq4} imply that the problem is $1/3$-reducible.
\end{proof}
\end{comment}
\begin{proof}[Of Lemma \ref{submodularaval}]
Since $f(.)$ is submodular, according to the definition of submodular functions, for every given sets $X$ and $Y$ in domain of $f(.)$ with $X \subseteq Y$ and every $x \in \items \setminus Y$ we have:
\begin{equation} \label{hineq5} f(X \cup \{x\}) - f(X) \geq f(Y \cup \{x\}) - f(Y) \end{equation}

Let $S_i = \{e_1, e_2, \ldots, e_{\alpha}\}$, $T_0 = \emptyset$, and $T_j = \{e_1, e_2, \ldots, e_j\}$, for every $1 \leq j \leq \alpha$. Since $T_j \subseteq S_i$ for each $0 \leq j \leq \alpha$ and $f_i$ is a submodular function, according to Inequality \eqref{hineq5} we have:
\begin{equation} \label{hineq6} \sum_{1 \leq j \leq \alpha} f_i(S_i \setminus T_{j-1}) - f_i(S_i \setminus T_j) \geq \sum_{1 \leq j \leq \alpha} f_i(S_i) - f_i(S_i - e_j) \end{equation}

Since $f_i(S_i) = \sum_{1 \leq j \leq \alpha} f_i(S_i \setminus T_{j-1}) - f_i(S_i \setminus T_j)$, we can rewrite Inequality \eqref{hineq6} for every $1 \leq i \leq k$ as follows:
\begin{equation} \label{hineq7} f_i(S_i) \geq \sum_{e \in S_i} f_i(S_i) - f_i(S_i - e) \end{equation}

For every $1 \leq i \leq k$ we can rewrite Inequality \eqref{hineq7} as follows:
\begin{equation} \label{hineq8} \sum_{e \in s_i} f_i(S_i-e) \geq (|S_i| - 1) f_i(S_i) \end{equation}

By adding $(|\bigcup S_i| - |S_i|) f_i(S_i)$ to the both sides of Inequality \eqref{hineq8}, we have:
\begin{equation} 
\label{hineq9} 
 \begin{split}
 (|\bigcup S_i| - |S_i|) f_i(S_i) + \sum_{e \in S_i} f_i(S_i - e) &= \sum_{e \in \bigcup S_i} f_i(S_i \setminus \{e\}) \\
 &\geq (|\bigcup S_i| - 1) f_i(S_i)
 \end{split}
\end{equation}

Since Inequality \eqref{hineq9} holds for every $1 \leq i \leq k$, we can sum up both sides of Inequality \eqref{hineq9} as follows:

\begin{equation} \label{hineq10} \sum_{1 \leq i \leq k}\sum_{e \in \bigcup S_i} f_i(S_i - e) \geq \sum_{1 \leq i \leq k} (|\bigcup S_i| - 1) f_i(S_i) \end{equation}

By dividing both sides of Inequality \eqref{hineq10} over $1/|\bigcup S_i|$ we obtain:

\begin{equation}
\label{hineq11} 
\begin{split}
    \frac{1}{|\bigcup S_i|}(\sum_{e \in \bigcup S_i} \sum_{1 \leq i \leq k} f_i(S_i - e)) &= \mathbb{E}[\sum_{1 \leq i \leq k} f_i(S^*_i)]\\
    &\geq \sum_{1 \leq i \leq k} f_i(S_i)\frac{|\bigcup S_i| -1}{|\bigcup S_i|}.
\end{split}
\end{equation}

\end{proof}

\begin{proof}[Of Lemma \ref{submodulardovom}]
Similar to the proof of Lemma \ref{submodularaval}, we use Inequality \eqref{hineq5} as a definition of submodular functions. Let $S'_i = S_i \setminus S = \{e_1, e_2, \ldots, e_{\alpha}\}$, $T_0 = S$, and $T_j = S \cup \{e_1, e_2, \ldots, e_j\}$ for $1 \leq j \leq \alpha$. According to $f(S) < 1/3$, $f(S \cup S'_i) \geq 1$, and Inequality \eqref{hineq5} as a definition of sub-modular functions, we have:

\begin{equation}
\label{hineq12} 
\begin{split}
    2/3 &< f(S \cup S') - f(S)\\
    &= \sum_{1 \leq j \leq \alpha} f(T_{j-1} \cup \{e_j\}) - f(T_{j-1})\\
    &\leq \sum_{e \in S'_i} f(S \cup \{e\}) - f(S)
\end{split}
\end{equation}

Similar to Inequality \eqref{hineq10}, we can rewrite Inequality \eqref{hineq12} with a summation, since Inequality \eqref{hineq12} holds for any $1 \leq i \leq k$.

\begin{equation}
    \label{hineq13}
    2k/3 < \sum_{1 \leq i \leq k} \sum_{e \in S'_i} f(S \cup \{e\}) - f(S)
\end{equation}

By dividing both sides of Inequality \eqref{hineq13} over $1/ |\bigcup S_i \setminus S|$ we have:

\begin{equation}
    \label{hineq14}
    \begin{split}
    \frac{2k/3}{|\bigcup S_i \setminus S|} &< \frac{1}{|\bigcup S_i \setminus S|}(\sum_{1 \leq i \leq k} \sum_{e \in S'_i} f(S \cup \{e\}) - f(S))\\
    &= \mathbb{E}[f(S \cup \{e\}) - f(S)]
    \end{split}
\end{equation}
\end{proof}
\section{Omitted Proofs of Section \ref{xos}}\label{xosappendix}

\begin{proof}[of Lemma \ref{xos2lemma}]
According to the definition of XOS function, $f(.)$ is an XOS function with a finite set of additive functions $\{g_1, g_2, \ldots, g_{\alpha}\}$ where $f(S) = \max_{i=1}^{\alpha} g_i(S)$ for any set $S \in \domp(f)$. Let $g_j(.)$ be the additive function which maximizes $S$. Let $g_j(S_1) = \alpha_1, g_j(S_2) = \alpha_2, \ldots, g_j(S_k) = \alpha_k$, which yields $\beta = \sum \alpha_i$. Since $g_j(S_i) = \alpha_i$, $f(S \setminus S_i) \geq \beta - \alpha_i$. Therefore, we have:

\begin{equation}
    \label{hineq15}
    \begin{split}
    \sum f(S) - f(S \setminus S_i) &\leq \sum \beta - (\beta - \alpha_i)\\
    &= \beta\\
    &= f(S)
    \end{split}
\end{equation}


%Since $f(.)$ is an XOS set function, at least for one of the additive functions of $f$ like $f_j$ we have $f_j(S) = \beta$. So according the definition of an additive function it means that $f_j(S_1) + f_j(S_2) + \ldots + f_j(S_k) = \beta$. Therefore, there is a set like $S_i$ such that $f_j(S_i) \ge \frac{\beta}{k}$, which yields that $f(S_i) \ge \frac{\beta}{k}$.
\end{proof}




%Any agent $a_{i}$ has an XOS valuation function $V_{i}(.)$, and we have relaxed this valuation function in a way $\MMS$ of any agent is equal to one. In this subsection we want to describe an existential proof for $1/6$ allocation in XOS valuations. We do not go through the finding of the allocation in this subsection, and we will talk about the algorithm in the next subsection.
\begin{comment}
\begin{proof}[ of Lemma \ref{xosreducible}]
%Suppose for the sake of contradiction that we have at least one item $b_j$ with value equal or more than $1/6$ for $a_k$. Now, suppose that $T = \{a_k\}$, $S = \{b_j\}$, and $A = \langle A_1, A_2, \ldots, A_n\rangle$ is an allocation of $S$ to agents of $T$ where $A_k = (b_j, a_k)$, and $A_i = \emptyset$ for $i \neq k$. Now, $\forall \agent_i \in T$ we have $V_i(A_i) \geq 1/6$. Before allocation $\MMS$ of each agent was equal to one. By the definition of $\MMS$ we know that each agent can divide $\items$ to $n$ sets such that the value of each set be at least one for him. Since $|S| = 1$ the lonely member of $S$ is present in only one of these $n$ sets for each agent. So, we have $n-1$ sets with value at least one for each agent after the allocation, and we have satisfied one of the agents. Therefore, $\forall \agent_i \notin T$ we have $\MMS_{V_i}^{n-|T|} (\items \setminus S)\geq 1$. According to Definition \ref{d1} our instance of problem is $1/6-reducible$ which contradicts our assumption.
Similar to the proof of Lemma \ref{submodularsefr}, suppose that there is an item $b_j$ where $V_k(\{b_j\}) \geq 1/5$ for $a_k$. Let $T = \{a_k\}$, $S = \{b_j\}$, and $\mathcal{A} = \langle A_1, A_2, \ldots, A_n\rangle$ is an allocation of $S$ to agents of $T$ where $A_k = (b_j, a_k)$, and $A_i = \emptyset$ for any $i \neq k$. Similar to the proof of Lemma \ref{submodularsefr} we can prove that using allocation $\mathcal{A}$ our instance of the problem is $1/5$-reducible.
\end{proof}
\end{comment}
%According to Lemma \ref{xosreducible} we can suppose that there is no item with value equal or more than $1/6$ for any agent.
%Now, suppose an allocation which maximizes $\sum_{i=1}^{i=n} V_{i}^{1/3}(S_{i})$, where $S(i)$ is the set of allocated items to $a_{i}$. If after this allocation for all of the agents like $a_{i}$ we have $V_{i}^{1/3}(S_{i}) \ge 1/6$ all of the agents are satisfied by the allocation. Therefore, suppose that there is at least one agent like $a_{i}$ such that $V_{i}^{1/3}(S_{i}) < 1/6$. Now, we want to divide $\items$ to $2n$ sets such that each set has at least $1/3\MMS_{i}$ value for $a_{i}$.


\begin{proof}[of Lemma \ref{2nsets}]
According to the definition of $\MMS$, we know that $a_{i}$ can divide items to $n$ sets ${\cal P} = \langle P_1, P_2, \ldots, P_n \rangle$ such that $V_i(P_j) \geq 1$ for any $P_j$. The catch is that $a_i$ can divide each of these $n$ sets to two disjoint sets such that the value of each of these new sets be at least $2/5$ to him. Let $T = \{b_1, b_2, \ldots, b_\gamma\}$ be one of these $n$ sets, and $g_j(.)$ be an additive function which maximizes $V_i(T)$. Let $T_k = \{b_1, b_2, \ldots, b_k\}$ for any $1 \leq k \leq \gamma$. According to Lemma \ref{remove1}, since the problem is $1/5$-irreducible, the value of any item is less than $1/5$ to $a_i$. Therefore, there is a set $T_k$ among $T_1$ to $T_\gamma$ where $2/5 \leq g_j(T_k) < 3/5$. Since $g_j(.)$ is one of additive functions of XOS function $V_i$, we have $V_i(T_k) \geq 2/5$. Moreover, since $g_j(T_k) < 3/5$, $g_j(T \setminus T_k) \geq 2/5$, which yields $V_i(T \setminus T_k) \geq 2/5$. As a conclusion, we can divide each of $n$ sets to two disjoint sets with at least $2/5$ value to $a_i$.

\end{proof}

 
%\begin{proof}[of Lemma \ref{beautiful}]
%\end{proof}


\end{document}