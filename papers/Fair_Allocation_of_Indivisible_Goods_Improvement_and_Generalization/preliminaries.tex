\section{Preliminaries}\label{prelim}

Throughout this paper we assume the set of agents is denoted by $\agents$ and the set of items is referred to by $\items$. Let $|\agents| = n$ and $|\items| = m$, we refer to the agents by $\agent_i$ and to the items by $\ite_i$, i.e., $\agents = \{\agent_1, \agent_2, \ldots, \agent_n\}$ and $\items = \{\ite_1, \ite_2, \ldots,\ite_m\}$. We denote the valuation of agent $\agent_i$ for a set $S$ of items by $\valu_i(S)$. Our interest is in valuation functions that are monotone and non-negative. More precisely, we assume $\valu_i(S) \geq 0$ for every agent $a_i$ and set $S \subseteq \items$, and for every two sets $S_1$ and $S_2$ we have
$$\forall \agent_i \in \agents\hspace{2cm} \valu_i(S_1 \cup S_2) \geq \max\{\valu_i(S_1), \valu_i(S_2)\}.$$

Due to obvious impossibility results for the general valuation functions\footnote{If the valuation functions are not restricted, no approximation guarantee can be achieved. For instance consider the case where we have two agents and 4 items. Agent $\agent_1$ has value 1 for sets $\{\ite_1,\ite_2\}$ and $\{\ite_3,\ite_4\}$ and 0 for the rest of the sets. Similarly, agent $\agent_2$ has value 1 for sets $\{\ite_1,\ite_3\}$ and $\{\ite_2,\ite_4\}$ and 0 for the rest of the sets. In this case, no allocation can provide both of the agents with sets which are of non-zero value to them.}, we restrict our attention to four classes of set functions:
\begin{itemize}
	\item \textbf{Additive}: A set function $V(.)$ is additive if $V(S_1) + V(S_2) = V(S_1 \cup S_2) - V(S_1 \cap S_2)$ for every two sets $S_1,S_2 \in \domp(V)$.
	\item \textbf{Submodular}: A set function $V(.)$ is submodular if $V(S_1) + V(S_2) \geq V(S_1 \cup S_2) - V(S_1 \cap S_2)$ for every two sets $S_1,S_2 \in \domp(V)$.
	\item \textbf{Fractionally Subadditive (XOS)}: An XOS set function $V(.)$ can be shown via a finite set of additive functions $\{V_1, V_2, \ldots, V_{\alpha}\}$ where $V(S) = \max_{i=1}^{\alpha} V_i(S)$ for any set $S \subseteq \domp(V)$. 
	\item \textbf{Subadditive}: A set function $V(.)$ is subadditive if $V(S_1) + V(S_2) \geq V(S_1 \cup S_2)$ for every two sets $S_1,S_2 \subseteq \domp(V)$. 
\end{itemize}

For additive functions, we assume the value of the function for every element is given in the input. However, representing other classes of set functions requires access to oracles. For submodular functions, we assume we have access to \textit{query oracle} defined below. Query oracles are great identifier for submodular functions, however, they are too weak when it comes to XOS and subadditive settings. For such functions, we use a stronger oracle which is called \textit{demand oracle}. It is shown that for some functions, such as gross substitutes, a demand oracle can be implemented via a query oracle in polynomial time~\cite{leme2014gross}. In addition to this, we consider a special oracle for XOS functions which is called \textit{XOS oracle}. Access to query oracles for submodular functions, XOS oracle for XOS functions, and demand oracles for XOS and subadditive functions are quite common and have been very fruitful in the literature~\cite{dobzinski2005approximation,feige2009maximizing,feige2011maximizing,feige2006approximation,feldmancombinatorial,leme2014gross,vondrak2008optimal}. In what follows, we formally define the oracles:
\begin{itemize}
	\item \textbf{Query oracle:} Given a function $f$, a query oracle $\mathcal{O}$ is an algorithm that receives a set $S$ as input and computes $f(S)$ in time $O(1)$.
	\item \textbf{Demand oracle:} Given a function $f$, a demand oracle $\mathcal{O}$ is an algorithm that receives a sequence of prices $p_1, p_2, \ldots, p_n$ as input and finds a set $S$ such that
	$$f(S) - \sum_{e \in S} p_e$$ is maximized. We assume the running time of the algorithm is $O(1)$.  
	\item \textbf{XOS oracle:} (defined only for an XOS functions $f$) Given a set $S$ of items, it returns the additive representation of the function that is maximized for $S$. In other words, it reveals the contribution of each item in $S$ to the value of $f(S)$. 
\end{itemize}


Let $\Pi_r$ be the set of all partitionings of $\items$ into $r$ disjoint subsets. For every $r$-partitioning $P^* \in \Pi_r$, we denote the partitions by $P^*_1,P^*_2,\ldots,P^*_r$. For a set function $f(.)$, we define $\MMS_f^r(\items)$ as follows:
$$ \MMS_f^r(\items) = \max_{P^* \in {\Pi_r}}  \min_{1 \leq j \leq r} f(P^*_j).$$
For brevity we refer to $\MMS_{f_i}^n(\items)$ by $\MMS_{i}$.



%\textit{Optimal r-partition} of $\cal M$ with respect to player $p_i$, is the best partitioning that player $p_i$ can make with the items of $\cal M$, i.e. the $r$-partition that defines $\mbox{MMS}_i^r({\cal M})$. Specifically we are interested in $\mbox{MMS}_i^n({\cal M})$. Here, for the sake of simplicity, we assume that the values of the players for items are normalized, so that $\mathrm{MMS}_{i}^n({\cal{M}}) = 1 $. 

An allocation of items to the agents is a vector $\mathcal{A} = \langle A_1, A_2, \ldots, A_n\rangle$ where $\bigcup A_i = \items$ and $A_i \cap A_j = \emptyset$ for every two agents $\agent_i, \agent_j \in \agents$. An allocation $\mathcal{A}$ is  $\alpha$-$\MMS$, if every agent $a_i$ receives a subset of the items whose value to that agent is at least $\alpha$ times $\MMS_i$. More precisely, $\mathcal{A}$ is $\alpha$-$\MMS$ if and only if
$\valu_i(A_i) \geq \alpha\MMS_i$
for every agent $\agent_i \in \agents$.
%, value of allocated items to $p_i$ is at least $\alpha$. We name $\alpha$ as the \textit{satisfaction parameter}. For the rest of the definitions, we assume that a value $\alpha$ is fixed.

We define the notion of \textit{reducibility} for an instance of the problem as follows.

\begin{definition}\label{d1}
	We say an instance of the problem is $\alpha$-reducible, if there exist a set $T \subset \agents$ of agents, a set $S$ of items, and an allocation $\mathcal{A} = \langle A_1, A_2, \ldots, A_n\rangle$ of $S$ to agents of $T$ such that 
	$$\forall \agent_i \in T \hspace{3cm} V_i(A_i) \geq \alpha \MMS_i$$
	and
	$$\forall \agent_i \notin T \hspace{1cm} \MMS_{V_i}^{n-|T|} (\items \setminus S)\geq \MMS_i.$$
\end{definition}
Similarly, we call an instance $\alpha$-\textit{irreducible} if it is not $\alpha$-reducible. The intuition behind Definition \ref{d1} is the following: In order to prove the existence of an $\alpha$-$\MMS$ allocation for every instance of the problem, it only suffices to prove this for the $\alpha$-irreducible instances.

\begin{observation}\label{reducibility}
	Every instance of the fair allocation problem admits an $\alpha$-$\MMS$ allocation if this holds for all $\alpha$-irreducible instances. 
\end{observation}
\begin{proof}
	Suppose for the sake of contradiction that all $\alpha$-irreducible instances of the problem admit an $\alpha$-$\MMS$ allocation, but there exists an $\alpha$-reducible instance of the problem which does not admit any $\alpha$-$\MMS$ allocation. Among all such instances, we consider the one with the lowest number of agents. Since this instance is $\alpha$-reducible, there exists a subset $T$ of agents and a subset $S$ of items such that an allocation of $S$ to agents of $T$ provides each of them with a valuation of at least $\alpha \MMS_i$. Moreover, the rest of the items and agents make an instance of the problem with a smaller $n$. Thus, an $\alpha$-$\MMS$ allocation can satisfy the rest of the agents and hence the instance admits an $\alpha$-$\MMS$ allocation. This contradicts the assumption. 
\end{proof}

The reducibility argument plays an important role in both the existential proofs and algorithms that we present in the paper. As we see in Sections \ref{additive:observations} and \ref{xos}, irreducible instances of the problem exhibit several desirable properties for additive and non-additive agents. We take advantage of these properties to improve the approximation guarantee of the problem for different classes of set functions.
