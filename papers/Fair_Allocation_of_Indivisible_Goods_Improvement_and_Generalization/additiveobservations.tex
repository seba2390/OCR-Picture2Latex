\subsection{General Definitions and Observations}\label{additive:observations}
Throughout this section we explore the properties of the fair allocation problem with additive agents.
\subsubsection{Consequences of Irreducibility}
 Since the objective is to prove the existence of a $3/4$-$\MMS$ allocation, by Observation \ref{reducibility}, it only suffices to show every $3/4$-irreducible instance of the problem admits a $3/4$-$\MMS$ allocation. Therefore, in this section we provide several properties of the $3/4$-irreducible instances. We say a set $S$ of items \textit{satisfies} an agent $\agent_i$ if and only if $\valu_i(S) \geq 3/4$. Perhaps the most important consequence of irreducibility is a bound on the valuation of the agents for every item. In the following we show if the problem is $3/4$-irreducible, then no agent has a value of $3/4$ or more for an item. 

\begin{lemma}\label{remove1} 
For every $\alpha$-irreducible instance of the problem we have 
$$\forall \agent_i \in \agents, \ite_j \in \items \hspace{1cm} \valu_i(\ite_j) < \alpha.$$
\end{lemma}

In other words, Lemma \ref{remove1} states that in a $3/4$-irreducible instance of the problem, no item alone can satisfy an agent. 

It is worth mentioning that the proof for Lemma \ref{remove1} does not rely on additivity of the valuation functions and holds as long as the valuations are monotone. Thus, regardless of the type of the valuation functions, one can assume that in any $\alpha$-irreducible instance, value of any item is less than $\alpha$ for any agent. Hence the statement carries over to the submodular, XOS, and subadditive settings. 



As a natural generalization of Lemma \ref{remove1}, we show a similar observation for every pair of items. However, this involves an additional constraint on the valuation of the other agents for the pertinent items. In contrast to Lemma \ref{remove1}, Lemmas \ref{remove2} and \ref{remove3} are restricted to additive setting and their results  do not hold in more general settings.


\begin{lemma}
\label{remove2}
If the problem is $3/4$-irreducible and
$\valu_i(\{\ite_j,\ite_k\}) \geq 3/4$
holds for an agent $\agent_i \in \agents$ and items $\ite_j, \ite_k \in \items$, then there exists an agent $\agent_{i'} \neq \agent_i$ such that
$$\valu_{i'}(\{\ite_j,\ite_k\}) > 1$$
\end{lemma}

According to Lemma \ref{remove2}, in every $3/4$-irreducible instance of the problem, for every agent $\agent_i$ and items $\ite_j,\ite_k$, either $\valu_i(\{\ite_j,\ite_k\}) < 3/4$ or there exists another agent $\agent_{i'} \neq \agent_i$, such that $\valu_{i'}(\{\ite_j,\ite_k\}) > 1$. Otherwise, we can reduce the problem and find a $3/4$-$\MMS$ allocation recursively. More generally, let $S = \{\ite_{j_1},\ite_{j_2},\ldots,\ite_{j_{|S|}}\}$ be a set of items in $\cal{M}$ and $T =\{\agent_{i_1},\agent_{i_2},\ldots,\agent_{i_{|T|}}\}$ be a set of agents such that
\begin{description}
 \item (i) $|S| = 2|T|$
 \item (ii) For every $\agent_{i_a} \in T$ we have $\valu_{i_a}(\{\ite_{j_{2a-1}},\ite_{j_{2a}}\}) \geq 3/4$.
 \item (iii) For every $\agent_{i} \notin T$ we have $\valu_{i}(\{\ite_{j_{2a-1}},\ite_{j_{2a}}\}) \leq 1$ for every $1 \leq a \leq |T|$.
\end{description}
then the problem is $3/4$-reducible.
\begin{lemma}\label{remove3}
In every $3/4$-irreducible instance of the problem, for every set $T =\{\agent_{i_1},\agent_{i_2},\ldots,\agent_{i_{|T|}}\}$ of agents and set $S = \{\ite_{j_1},\ite_{j_2},\ldots,\ite_{j_{|S|}}\}$ of items at least one of the above conditions is violated.
\end{lemma}

% Note that by applying Lemma \ref{remove2} $k$ times, there is no decrease in the maxmin share of the agents in ${\cal{N}}\setminus P$ for the rest of items.  
\subsubsection{Modeling the Problem with Bipartite Graphs}\label{MtPwBG}
In our algorithm we subsequently make use of classic algorithms for bipartite graphs. Let $G = \langle V(G),E(G)\rangle$ be a graph representing the agents and the items. Moreover, let $V(G) = \itemsv \cup \agentsv$ where $\agentsv$ corresponds to the agents and $\itemsv$ corresponds to the items. More precisely, for every agent $\agent_i$ we have a vertex $\agentv_i \in \agentsv$ and every item $\ite_j$ corresponds to a vertex $\itemv_j \in \itemsv$. For every pair of vertices $\agentv_i \in \agentsv$ and $\itemv_j \in \itemsv$, there exists an edge $(\itemv_j,\agentv_i) \in E(G)$ with weight $w(\itemv_j,\agentv_i) = \valu_i(\{\ite_j\})$. We refer to this graph as \emph{the value graph}.

We define an operation on the weighted graphs which we call \textit{filtering}. Roughly speaking, a filtering is an operation that receives a weighted graph as input and removes all of the edges with weight less than a threshold from the graph. Next, we remove all of the isolated\footnote{A vertex is called isolated if no edge is incident to that vertex.} vertices and report the remaining as the filtered graph. In the following we formally define the notion of filtering for weighted graphs.
\begin{definition}
A $\beta$-filtering of a weighted graph $H\langle V(H),E(H)\rangle$, denoted by $H_{\beta}\langle V_\beta(H),E_\beta(H)\rangle$, is a subgraph of $H$ where $V_\beta(H)$ is the set of all vertices in $V(H)$ incident to at least one edge of weight $\beta$ or more and 
$$E_\beta(H) = \{(u,v) \in E(H)| w(u,v) \geq \beta\}.$$ 
\end{definition}
For the case of the value graph, we also denote by $\agentsv_\beta$ and $\itemsv_\beta$ the sets of agents and items corresponding to vertices of $V_\beta(G)$.
\begin{figure}[t!]
    \centering
    \includegraphics[scale=0.8]{figs/filtering}
    \caption{An example of $\beta$-filtering on a graph. After removing the edges with a value smaller than $\beta$, some vertices may become isolated. All such vertices are removed from the filtered graph.}
    \label{fig:filtering}
\end{figure}
%\begin{example}
Figure \ref{fig:filtering} illustrates an example of a graph $H$, together with $H_{0.4}$ and $H_{0.5}$. Note that none of the vertices in $H_{0.4}$ or $H_{0.5}$ are isolated. 
% 
%\end{example}
%%%%%%%%%%%%%%%%

Denote by a maximum matching, a matching that has the highest number of edges in a graph. In definition \ref{FG}, we introduce our main tool for clustering the agents. 
\begin{definition}
\label{FG}
Let $H\langle V(H),E(H)\rangle$ be a bipartite graph with $V(H) = \partone \cup \parttwo$ and let $M$ be a maximum matching of $H$. Define $\parttwo_1$ as the set of the vertices in $\parttwo$ that are not saturated by $M$. Also, define $\parttwo_2$ as the set of vertices in $\parttwo$ that are connected to $\parttwo_1$ by an alternating path and let $\partone_2 = M(\parttwo_2)$, where  $M(\parttwo_2)$ is the set of vertices in $\partone$ that are matched with the vertices of $\parttwo_2$ in $M$. We define $F_{H}(M,\partone)$ as the set of the vertices in $\partone \setminus \partone_2$. 
\end{definition}

For a better understanding of Definition \ref{FG}, consider Figure \ref{fig:FG}. By the definition of alternating paths, there is no edge between the saturated vertices of $F_H(M,\partone)$ and $ \parttwo_1 \cup \parttwo_2$. On the other hand, since $M$ is maximum, the graph doesn't have any augmenting path. Thus, there is no edge between unsaturated vertices in $F_H(M,\partone)$ and $ \parttwo_1 \cup \parttwo_2$. As a result, there is no edge between $F_H(M,\partone)$ and $ \parttwo_1 \cup \parttwo_2$. Furthermore, $F_H(M,\partone)$ has another important property: there exists a matching from $N(F_H(M,\partone))$ to $F_H(M,\partone)$, that saturates all the vertices in $N(F_H(M,\partone))$, where $N(F_H(M,\partone))$ is the set of neighbors of  $F_H(M,\partone)$.

\begin{figure}
\centering
\includegraphics[scale=0.8]{figs/matching}
\caption{Definition of $F_H$}
\label{fig:FG}
\end{figure}

In Lemmas \ref{iff} and \ref{rem}, we prove two remarkable properties for bipartite graphs. As a consequence of these two lemmas, Corollary \ref{remcol} holds for every bipartite graph. We leverage the result of Corollary \ref{remcol} in the clustering phase.
 
\begin{lemma}
\label{rem}
Let $H(V,E)$ be a bipartite graph with $V = \partone \cup \parttwo$ and let $M$ be a maximum matching of $H$. Then, for every set $T \subseteq \partone \setminus F_H(M,\partone)$ we have $|N(T)| > |T|$, where $N(T)$ is the set of neighbors of $T$. 
\end{lemma}

\begin{lemma}
\label{iff}
For a bipartite graph $H(V,E)$ with $V = \partone \cup \parttwo$, $F_H(M,\partone) = \emptyset$ holds, if and only if
for all $T \subseteq \partone$ we have  $|N(T)| > |T|$, where $N(T)$ is the set of neighbors of $T$.

\end{lemma}

\begin{corollary}[of Lemmas \ref{iff} and \ref{rem}]
\label{remcol}
Let $H(V,E)$ be a bipartite graph with $V = \partone \cup \parttwo$ and let $M$ be a maximum matching of $H$. 
Furthermore, let  $H'(V',E')$ be the induced sub-graph of $H$, with $V' = \partone' \cup \parttwo'$, where $\partone' = \partone \setminus F_H(M,\partone)$ and $\parttwo' = \parttwo \setminus N(F_H(M,\partone))$. Then, for any maximum matching $M'$ of $H'$, $F_{H'}(M',\partone') = \emptyset$ holds. 
\end{corollary}
%%%%%%%%%%%%%%%%
\subsubsection{Cycle-envy-freeness and $\MCMWM$}\label{additive:cef}

In the algorithm, we satisfy each agent in two steps. More precisely, we allocate each agent two sets of items that are together of worth at least $3/4$ to him. We denote the first set of items allocated to agent $\agent_i$ by $\firstset_i$ and the second set by $\secondset_i$. Moreover, we attribute the agents with labels \textit{satisfied}, \textit{unsatisfied}, and \textit{semi-satisfied} in the following way:
\begin{enumerate}
	\item An agent $\agent_i$ is satisfied if $\valu_i(\firstset_i \cup \secondset_i) \geq 3/4$.
	\item An agent $\agent_i$ is semi-satisfied if $\firstset_i \neq \emptyset$ but $\secondset_i = \emptyset$. In this case we define $\epsilon_i = 3/4-\valu_i(\firstset_i)$.
	\item An agent $\agent_i$ is unsatisfied if $\firstset_i = \secondset_i = \emptyset$.
\end{enumerate}
As we see, the algorithm maintains the property that for every semi-satisfied agent $\agent_i$, $\valu_i(\firstset_i) \geq 1/2$ holds and hence, $\epsilon_i < 1/4$. 

To capture the competition between different agents, we define an attribution for an ordered pair of agents. We say a semi-satisfied agent envies another semi-satisfied agent, if he prefers to switch sets with the other agent. 

\begin{definition}
\label{winloose}
Let $T$ be a set of semi-satisfied agents. An agent $\agent_i \in T$ envies an agent $\agent_j \in T$, if $\valu_i(\firstset_j) \geq \valu_i(\firstset_i)$. Also, we call an agent $ \agent_i \in T$ a winner of $T$, if $\agent_i$ envies no other agent in $T$. Similarly, we call an agent $\agent_i$ a loser of $T$, if no other agent in $T$ envies $\agent_i$.
\end{definition}

Note that it could be the case that an agent $\agent_i$ is both a loser and a winner of a set $T$ of agents. Based on Definition \ref{winloose}, we next define the notion of \textit{cycle-envy-freeness}.

\begin{definition}
 We call a set $T$ of semi-satisfied agents cycle-envy-free, if every non-empty subset of $T$ contains at least one winner and one loser. 
\end{definition}

Let $C$ be a cycle-envy-free set of semi-satisfied agents. Define the representation graph of $C$ as a digraph $G_C(V(G_C),\overrightarrow{E}(G_C))$, such that for any agent $\agent_i \in C$, there is a vertex $v_i$ in $V(G_C)$ and there is a directed edge from $v_i$ to $v_j$ in $\overrightarrow{E}(G_C)$, if $\agent_i$ envies $\agent_j$. In Lemma \ref{dag}, we show that $G_C$ is acyclic.
\begin{lemma}
\label{dag}
For every cycle-envy-free set of semi-satisfied agents $C$, $G_C$ is a DAG. 
\end{lemma}
\begin{definition}
 A topological ordering of a cycle-envy-free set $C$ of semi-satisfied agents, is a total order $\prec_O$ corresponding to the topological ordering of the representation graph $G_C$. More formally, for the agents $\agent_i,\agent_j \in C$ we have $\agent_i \prec_O \agent_j$ if and only if $v_i$ appears before $v_j$, in the topological ordering of $G_C$.  
\end{definition}

Note that in the topological ordering of a cycle-envy-free set $C$ of semi-satisfied agents, if $\agent_i \in C$ envies $\agent_j \in C$, then $\agent_i \prec_O \agent_j$. 


\begin{observation}
\label{epsofcluster}
Let $C$ be a cycle-envy-free set of semi-satisfied agents. Then, for every agent $\agent_i \in C$ such that $\agent_j \prec_O \agent_i$, we have:
$$\valu_i(\firstset_j) \leq 3/4 - \epsilon_i.$$ 

\end{observation}

We define a maximum cardinality maximum weighted matching of a weighted graph as a matching that has the highest number of edges and among them the one that has the highest total sum of edge weights. For brevity we call such a matching an $\MCMWM$. In Lemma \ref{wm}, we show that an $\MCMWM$ of a weighted bipartite graph has certain properties that makes it useful for building cycle-envy-free clusters. 

\begin{lemma}
\label{wm}
Let $H\langle V(H),E(H)\rangle$ be a weighted bipartite graph with $V(H) = \partone \cup \parttwo$ and let $M = \{(\vone_1,\vtwo_1),...,(\vone_k,\vtwo_k)\}$ be an $\MCMWM$ of $H$. Then, for every subset $T \subseteq \{\vtwo_1,\vtwo_2, \ldots,\vtwo_k\}$, the following conditions hold:

\begin{minipage}[t]{\linegoal}
\begin{enumerate}[leftmargin=*]
 \item There exists a vertex $\vtwo_j \in T$ which is a winner in $T$, i.e.,  $w(\vone_j,\vtwo_{j}) \geq w(\vone_i,\vtwo_j)$, for all $\vone_i \in M(T)$ and  $(\vone_i,\vtwo_j) \in E(H)$. 
\item There exists a vertex $ \vtwo_j \in T$ which is a loser in $T$, i.e.,  $w(\vone_i,\vtwo_i)  \geq w(\vone_j,\vtwo_i) $, for all $\vtwo_i \in T$ and $(\vone_j,\vtwo_i) \in E(H)$.
\item For any vertex $\vtwo_i \in T$ and any unsaturated vertex $\vone_j \in \partone$ such that $(\vone_j,\vtwo_i) \in E(H)$, $w(\vone_i,\vtwo_i) \geq w(\vone_j,\vtwo_i)$. 
\end{enumerate}
\end{minipage}
\\[6pt]
where $M(T)$ is the set of vertices which are matched by the vertices of $T$ in $M$.
\end{lemma}


Notice the similarities of the first and the second conditions of Lemma \ref{wm} with the conditions of the winner and loser in 
 Definition \ref{winloose}. In Section \ref{additive:clusters}, we assign items to the agents based on an $\MCMWM$ of the value-graph. Lemma \ref{wm} ensures that such an assignment results in a cycle-envy-free set of semi-satisfied agents.   

%\begin{definition}
%\label{mp}
%For a set $M$ of items and a set $P$ of semi-satisfied agents, we define $M_P$ as the set of all items in $M$ with 
%the property that at least one agent in $P$ can be satisfied by this item. 
%More formally, $M_P$ contains all the items %$\ite_j \in M$, such that:

%$$\exists \agent_i \in P \mbox{   }s.t \mbox{   %}\valu_i(\firstset_i \cup \{\ite_j\}) \geq 3/4$$
%\end{definition}


%subsection{Definitions for Section 3}
%\begin{definition}
%For a bundle $Q$ of items that satisfies $\agent_i$, the core of $Q$ with respect to agent $\agent_i$, denoted as $C_i(Q)$ is defined as follows: let $\ite_1,\ite_2,..,\ite_k$ be the items of $Q$ in the increasing order of their values for $\agent_i$. Then $C_i(Q) = \{\ite_j,\ite_{j+1},...,\ite_{k}\}$ , where $j$ is the highest index, such that the bundle with items $\{\ite_j,\ite_{j+1},...,\ite_k\}$ satisfies $\agent_i$.
%\end{definition}
 


%\begin{definition}
%\label{satisfaction-graph}
%Given a set $P = \{\agent_1, \agent_2, \ldots, \agent_n\}$ of the semi-satisfied agents and a set $S = \{b_1,b_2,\ldots,b_m\}$, whose elements are bundles of items. The satisfaction-graph with respect to $S$ and $P$, is a bipartite graph denoted by $G^-(\agentsv,\itemsv)$. For each bundle $b_i$ in $S$, there is a related vertex $\agentv_i$ in $\agentsv$ and for each agent $\agent_j$ in $P$, there is a related vertex $\itemv_j$ in $\itemsv$. There is an edge between $\agentv_i$ and $\itemv_j$, if the agent $\agent_j$ can be satisfied by the items in $b_i$ .i.e. $\valu_j( \firstset_j \cup b_i) \geq 3/4$.  
%\end{definition}
