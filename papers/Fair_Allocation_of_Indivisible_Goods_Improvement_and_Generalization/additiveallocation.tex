\subsection{Phase 2: Satisfying the Agents}\label{additive:allocation}
\subsubsection{An Overview on the State of the Algorithm}
Before going through the second phase, we present an overview of the current state of the agents and items. In Figure \ref{fig:overview}, for every agent $\agent_i \in \cone \cup \ctwo \cup \satagents$, $\firstset_i$ is shown by a gray rectangle and for every agent $\agent_i \in \satagents$, $\secondset_i$  is shown by a hatched rectangle. 

Currently, we know that every agent in $\satagents$ belongs to $\satagents_1^r$ or $\satagents_2^r$. These agents are satisfied in the refinement phases of $\cone$ and $\ctwo$. The rest of the agents will be satisfied in the second phase. For brevity, for $i \leq 2$ we use $\satagents_i^s$ to refer to the agents in $\satagents_i$ that are satisfied in the second phase. More formally, $$\mbox{ for }i=1,2 \qquad \satagents_i^s = \satagents_i \setminus \satagents_i^r .$$

Since we didn't refine Cluster $\cthree$, all the agents in the Cluster $\cthree$ are satisfied in the second phase. As mentioned in the previous section, the item allocation to the semi-satisfied agents in $\cthree$ is temporary; That is, we may alter such allocations later. Therefore, in Figure \ref{fig:overview} we illustrate such allocations by dashed lines. 

 

In this section, we denote the set of free items (the items corresponding to the vertices in $\itemsv''\setminus \itemsv''_{1/2}$ at the end of the first phase) by $\fitems$. By Observations \ref{fsmallc1}, \ref{fsmallc2} and Corollary \ref{small_c3}, we know that the items in $\fitems$ have the following properties:
\begin{enumerate}
\item For every agent $\agent_i$ in $\cone$, $\valu_i(\{\ite_j\}) < \epsilon_i$ holds for all $\ite_j \in \fitems$ (Observation \ref{fsmallc1}).
\item For every agent $\agent_i$ in $\ctwo$, $\valu_i(\{\ite_j\}) < \epsilon_i$ holds for all $\ite_j \in \fitems$ (Observation \ref{fsmallc2}).
\item For every agent $\agent_i$ in $\cthree$, there is at most one item $\ite_j \in \fitems$, such that $\valu_i(\{\ite_j\}) \geq 1/4$ (Corollary \ref{small_c3}).
\end{enumerate}


\begin{table}[t]
	\caption{Summary of value lemmas for $f_i$}
	\label{table0} 
	\begin{center}
\begin{tabular}{|c|c|c|c|}
\hline
	& $\forall \agent_i \in \cone$&  $\forall \agent_i \in \ctwo$ & $\forall \agent_i \in \cthree$\\
\hline
$\forall \agent_j \in \cone$&	- & $\valu_i(\firstset_j) < 1/2$ ($\star$)& $\valu_i(\firstset_j) < 1/2$ ($\star$) \\
\hline
$\forall \agent_j \in \ctwo$ & $ \valu_i(\firstset_j) < 3/4 $ ($\ddagger$)  & - & $\valu_i(\firstset_j) < 1/2$ ($\dagger$)\\ 
\hline
$\forall \agent_j \in \cthree^s$ & $ \valu_i(\firstset_j) < 3/4 $($\ddagger$)  & $\valu_i(\firstset_j) <3/4$($\ddagger$)  &  - \\ 
\hline

\end{tabular}
\end{center}
$\hspace{110pt} \star$: Lemma \ref{forc2c3} $\hspace{10pt} \dagger$: Lemma \ref{forc3} $\hspace{10pt} \ddagger$: Lemma \ref{general}\\
\end{table}


\begin{table}[t]
	\caption{Summary of value lemmas for the agents in $\satagents_i^r$}
	\label{table4} 
\begin{center}
	\begin{tabular}{|c|c|c|c|}
\hline
	&  $\forall \agent_i \in \cone $ & $\forall \agent_i \in \ctwo$ & $\forall \agent_i \in \cthree$\\
\hline
$\forall \agent_j \in \satagents_1^r $	 & - & $\valu_i(\secondset_j)<1/2$ ($\star$)& $\valu_i(\secondset_j)<1/2$ ($\star$)\\
\hline
$\forall \agent_j \in \satagents_2^r $	 & $\valu_i(\secondset_j)<\epsilon_i (\dagger)$ & - & $\valu_i(\secondset_j)<1/2$ ($\ddagger$) \\
\hline
\end{tabular}
\end{center}
$\hspace{105pt}$ $\star$: Lemma \ref{gsmallc1r} $\hspace{10pt}$ $\dagger$: Lemma \ref{cr2smallc1} $\hspace{10pt}$ $\ddagger$: Lemma \ref{cr2smallc3}\\

\end{table}

In summary, items of $\fitems$ are small enough, therefore we can run a process similar to the $\bagfilling$ algorithm described earlier to allocate them to the agents. Recall that our clustering and refinement methods preserve the conditions stated in Lemmas \ref{forc2c3}, \ref{gsmallc1r}, \ref{forc3}, \ref{cr2smallc1} and \ref{cr2smallc3}. In addition to this, we state Lemma \ref{general} as follows.

\begin{lemma}[value-lemma]
\label{general}
For every agent $\agent_i \in \cone \cup \ctwo \cup \cthree^s$, we have
$$\forall \agent_j \in \cone \cup \ctwo \cup \cthree \qquad \valu_j(\firstset_i)<3/4.$$
\end{lemma}  
A brief summary of Lemmas \ref{forc2c3}, \ref{gsmallc1r}, \ref{forc3}, \ref{cr2smallc1}, \ref{cr2smallc3} and \ref{general} is illustrated in Tables \ref{table0} and \ref{table4}. Moreover, since sets $\cone,\ctwo$ and $\cthree^s$ are cycle-envy-free, Observation \ref{epsofcluster} holds for these sets. 



\subsubsection{Second Phase: $\bagfilling$}
We begin this section with some definitions. In the following, we define the notion of feasible subsets and, based on that, we define $\phi(S)$ for a feasible subset $S$ of items.
\begin{definition}
A subset $S$ of items in $\fitems$ is feasible, if at least one of the following conditions are met:
\begin{minipage}[t]{\linegoal}
\begin{enumerate}[leftmargin=30pt]
    \item There exists an agent $\agent_i \in \cthree^f $ such that  $\valu_i(\{S\}) \geq {1/2}$. 
    \item There exists an agent $\agent_i \in \cone \cup \ctwo \cup \cthree^s \cup \cthree^b$ such that  $\valu_i(\{S\}) \geq \epsilon_i$.
\end{enumerate}
\end{minipage}
\end{definition}

\begin{definition}
For a feasible set $S$, we define $\Phi(S)$ as the set of agents, that set $S$ is feasible for them. 
\end{definition}

Recall the notion of cycle-envy-freeness and the topological ordering of the agents in a cycle-envy-free set of semi-satisfied agents. Based on this, we define a total order $\prec_{pr}$ to prioritize the agents in the $\bagfilling$ algorithm. 

\begin{definition}
\label{priority}
Define a total order $\prec_{pr}$ on the agents of $\cone \cup \ctwo \cup \cthree$ with the following rules: 
\begin{minipage}[t]{\linegoal}
	
\begin{enumerate}[leftmargin=50pt]
    \item $\agent_{i_5} \prec_{pr} \agent_{i_1} \prec_{pr} \agent_{i_2} \prec_{pr} \agent_{i_3}  \prec_{pr} \agent_{i_4} \qquad$  $\forall \agent_{i_1} \in \cone, \agent_{i_2} \in \ctwo, \agent_{i_3} \in \cthree^s, \agent_{i_4} \in \cthree^b, \agent_{i_5} \in \cthree^f$
    \item $\agent_i \prec_{pr} \agent_j \Leftrightarrow \agent_i \prec_o \agent_j \hspace{85pt}$ $\forall \agent_i, \agent_j \in \cone \cup \ctwo \cup \cthree^s,  \agent_i ,\agent_j \mbox{ in the same cluster }$
	\item $\agent_i \prec_{pr} \agent_j \Leftrightarrow i < j \hspace{100pt}$ $\forall \agent_i,\agent_j \in \cthree^b \vee \agent_i,\agent_j \in \cthree^f$
\end{enumerate}
\end{minipage}
\end{definition}    

\begin{comment}
\begin{array}{cll}
(I)&\agent_{i_5} \prec_{pr} \agent_{i_1} \prec_{pr} \agent_{i_2} \prec_{pr} \agent_{i_3}  \prec_{pr} \agent_{i_4}& \forall \agent_{i_1} \in \cone, \agent_{i_2} \in \ctwo, \agent_{i_3} \in \cthree^s, \agent_{i_4} \in \cthree^b, \agent_{i_5} \in \cthree^f\\[6pt]
(II)&\agent_i \prec_{pr} \agent_j \Leftrightarrow \agent_i \prec_o \agent_j  & \forall \agent_i, \agent_j \in \cone \cup \ctwo \cup \cthree^s, \qquad \agent_i ,\agent_j \mbox{ in the same cluster }\\[6pt]

(III)&\agent_i \prec_{pr} \agent_j \Leftrightarrow i < j & \agent_i,\agent_j \in \cthree^b \vee \agent_i,\agent_j \in \cthree^f\\[6pt]
\end{array}.

\end{comment}


Recall that $\prec_o$ refers to the topological ordering of a semi-satisfied set of agents. Roughly speaking, for the semi-satisfied agents in the same cluster, $\prec_{pr}$ behaves in the same way as $\prec_{o}$. Furthermore, for the agents in different clusters, agents in $\cthree^f , \cone , \ctwo, \cthree^s , \cthree^b$ have a lower priority, respectively. Finally, the order of the agents in $\cthree^b$ and $\cthree^f$ is determined by their index, i.e., the agent with a lower index has a lower priority.


The second phase consists of several rounds and every round has two steps. Each of these two steps is described below. We continue running this algorithm until $\fitems$ is no longer feasible for any agent.
\begin{itemize}
\item \textbf{Step1}: In the first step, we run a process very similar to the $\bagfilling$ algorithm described in Section \ref{introduction}. That is, we find a feasible subset $S \subseteq \fitems$, such that $|S|$ is minimal. Such a subset can easily be found, using a slight modification of the $\bagfilling$ process (see Section \ref{sphase}).  

\item \textbf{Step2}: In the second step, we choose an agent to allocate set $S$ to him. In contrast to the $\bagfilling$ algorithm, we do not select an arbitrary agent. Instead, we select the agent in $\Phi(S)$ with the lowest priority regarding $\prec_{pr}$, i.e., smallest element in $\Phi(S)$ regarding $\prec_{pr}$. Let $\agent_i$ be the selected agent. We consider three cases separately:

\begin{minipage}[t]{\linegoal}
\begin{enumerate}[leftmargin=50pt]
    \item $\agent_i \in \cthree^f$: temporarily allocate $S$ to $\agent_i$, i.e., set $\firstset_i = S$. 
    \item $\agent_i \in \cthree^b$: let $\agent_j$ be the agent that $\valu_i(\firstset_j) = {3/4} - \epsilon_j$. 
Take back $\firstset_j$ from $\agent_j$ and allocate $\firstset_j \cup S$ to $\agent_i$ i.e. set $\firstset_i = \firstset_j$, $\firstset_j=\emptyset$ and $\secondset_i = S$. Remove $\agent_i$ from $\cthree$ and add it to $\satagents$.
    \item $\agent_i \in \cone \cup \ctwo \cup \cthree^s$: satisfy agent $\agent_i$ by $S$, i.e., set $\secondset_i = S$ and remove $\agent_i$ from its corresponding cluster and add it to $\satagents$. 
\end{enumerate}
\end{minipage}

By the construction of $\cthree^s,\cthree^b$, and $\cthree^f$, the above process may cause agents in $\cthree$ to move in between $\cthree^s,\cthree^b$ and $\cthree^f$. For example, if the first case happens, then $\agent_i$ is moved from $\cthree^f$ to $\cthree^s$. In addition, all other agents in $\cthree^f$ for which $S$ is feasible are moved to $\cthree^b$. For the second case, $\agent_j$ is moved to one of $\cthree^f$ or $\cthree^b$, based on $\valu_j(\firstset_k)$ for every $\agent_k \in \cthree^s$; that is, if there exists an agent $\agent_k \in \cthree^s$ such that $\valu_j(\firstset_k) \geq 1/2$, $\agent_j$ is moved to $\cthree^b$. Otherwise, $\agent_j$ is moved to $\cthree^f$. For both the second and the third cases, some of the agents in $\cthree^b$ may move to $\cthree^f$. 
\end{itemize}
The second phase terminates, when $\fitems$ is no longer feasible for any agent. More details about the second phase can be found in Algorithm \ref{second-phase}.  In Algorithm \ref{second-phase}, we use $Update(\cthree)$ to refer the process of moving agents among $\cthree^s, \cthree^b$ and $\cthree^f$.
 

\begin{algorithm}[t!]
 \KwData{$\fitems, \cone,\ctwo,\cthree$}
  \While{$\fitems$ is feasible}{
	$S$ = a minimal feasible subset of $\fitems$ \;
	$\agent_i = $ agent in $\Phi(S)$ with lowest order regarding  $\prec_{pr}$\;
	\If{$\agent_i \in C_3^f$}
	{
		$\firstset_i = S$ \;
		$Update(\cthree)$ \;
	}
	\If{$\agent_i \in \cthree^b$}
	{
		Let $\agent_j$ be the agent that $\valu_i(\firstset_j) = 3/4 - \epsilon_i$ \;
		$\firstset_i = \firstset_j$ \;
		$\secondset_i = S$ \;
		$\satagents = \satagents \cup \agent_i$ \;
		$\firstset_j = \emptyset$\;
		$\cthree = \cthree \setminus \agent_i$ \;
		$Update(\cthree)$ \;
	}
	\If {$\agent_i \in \cthree^s$}
	{
		$\secondset_i = S$\;
		$\satagents = \satagents \cup \agent_i$\;
		$\cthree = \cthree \setminus \agent_i$ \;
		$Update(\cthree)$ \;
	}
	\If {$\agent_i \in \cone \cup \ctwo$}
	{
		$\secondset_i = S$\;
		remove $\agent_i$ from its corresponding cluster \;
		$\satagents = \satagents \cup \agent_i$\;

	}
}
 \caption{The Second Phase}
 \label{second-phase}
\end{algorithm}

In each round of the second phase, either an agent is satisfied or an agent in $\cthree^f$ becomes semi-satisfied. In Lemma \ref{c3fsmall}, we show that if an agent $\agent_i \in \cthree^f$ is selected in some round of the second phase, then $\valu_j(\firstset_i)$ is upper bounded by $2\epsilon_j$ for every agent $\agent_j \in \cthree \cup \ctwo \cup \cone^s \cup \cone^b$. As a consequence of Lemma \ref{c3fsmall}, in Lemma \ref{cef} we show that sets $\cone,\ctwo$ and $\cthree$ remain cycle-envy-free during the second phase. For convenience, we use $\mathbb{R}_z$ to refer to the $z$'th round of the second phase. 

\begin{lemma}
\label{c3fsmall}
Let $\mathbb{R}_z$ be a round of the second phase that an agent $\agent_i \in \cthree^f$ is selected. Then, for every agent $\agent_j \in \cthree \cup \ctwo \cup \cone^s \cup \cone^b$, we have $\valu_j(\firstset_i)<2\epsilon_j<3/4$.
\end{lemma}



\begin{lemma}
\label{cef}
During the second phase, the $\cone,\ctwo$ and $\cthree^s$ maintain the property of cycle-envy-freeness. 
\end{lemma}

Finally, for the rounds that an agents $\agent_i$ is satisfied, Lemmas \ref{prvalue} and \ref{m_1} give upper bounds on the value of $\secondset_i$ for remaining agents in different clusters. 

\begin{lemma}[value-lemma]
\label{prvalue}
Let $\agent_i \in \satagents$ be an agent that is satisfied in the second phase. Then, for every other agent $\agent_j \in \cone \cup \ctwo$ we have:

\begin{minipage}[t]{\linegoal}
\begin{enumerate}[leftmargin=30pt]
\item If $\agent_j \prec_{pr} \agent_i$, then $\valu_j(\secondset_i) < \epsilon_j$.
\item If $\agent_i \prec_{pr} \agent_j$, then $\valu_j(\secondset_i) < 2\epsilon_j$.
\end{enumerate}
\end{minipage}
\end{lemma}

\begin{lemma}[value-lemma]
\label{m_1}
Let $\agent_i$ be an agent in $\satagents_1^s \cup \satagents_2^s$. Then, for every agent $\agent_j \in \cthree$, we have $\valu_j(\secondset_i) < {1/2}$.
\end{lemma}

The results of Lemmas \ref{prvalue} and \ref{m_1} are summarized in Table \ref{table1}.





\begin{table}[htbp]
\centering
\begin{tabular}{c|c|cccc|cccc}
                     &            & \multicolumn{4}{c|}{$n=50$}                       & \multicolumn{4}{c}{$n=200$}                      \\ \hline
Method               & Evaluation & $\alpha_1$ & $\alpha_2$ & $\alpha_3$ & $\alpha_4$ & $\alpha_1$ & $\alpha_2$ & $\alpha_3$ & $\alpha_4$ \\ \hline
\multirow{4}{*}{MM1}& Bias ($10^{-2}$)  & 1.79 & 4.08 & 4.01 & 2.43 & 0.23 & 0.33 & -0.29 & 0.31 \\ 
& MSE ($10^{-1}$)  & 0.78 & 0.78 & 0.76 & 0.84 & 0.17 & 0.17 & 0.17 & 0.18 \\ 
& MAPE ($10^{-1}$)  & 2.23 & 2.18 & 2.11 & 2.25 & 1.04 & 1.05 & 1.01 & 1.04 \\ 
& Coverage (\%)  & 94.7 & 93.9 & 95.1 & 93.0 & 94.4 & 93.5 & 94.1 & 93.2 \\ 
\hline 
\multirow{4}{*}{MM2}& Bias ($10^{-2}$)  & 2.12 & 4.32 & 4.31 & 2.68 & 0.41 & 0.52 & -0.09 & 0.46 \\ 
& MSE ($10^{-1}$)  & 0.63 & 0.6 & 0.59 & 0.66 & 0.13 & 0.13 & 0.13 & 0.13 \\ 
& MAPE ($10^{-1}$)  & 2.01 & 1.91 & 1.87 & 2.02 & 0.9 & 0.91 & 0.9 & 0.9 \\ 
& Coverage (\%)  & 93.0 & 93.8 & 94.3 & 92.4 & 94.4 & 94.3 & 95.3 & 95.5 \\ 
\hline 
\multirow{4}{*}{MM3}& Bias ($10^{-2}$)  & 2.12 & 4.32 & 4.31 & 2.68 & 0.41 & 0.52 & -0.09 & 0.46 \\ 
& MSE ($10^{-1}$)  & 0.63 & 0.6 & 0.59 & 0.66 & 0.13 & 0.13 & 0.13 & 0.13 \\ 
& MAPE ($10^{-1}$)  & 2.01 & 1.91 & 1.87 & 2.02 & 0.9 & 0.91 & 0.9 & 0.9 \\ 
& Coverage (\%)  & 93.8 & 94.0 & 94.4 & 92.8 & 95.2 & 94.4 & 94.7 & 95.5 \\ 
\hline 
\multirow{4}{*}{MM4}& Bias ($10^{-2}$)  & 0.88 & 3.06 & 3.04 & 1.44 & 0.11 & 0.22 & -0.39 & 0.16 \\ 
& MSE ($10^{-1}$)  & 0.61 & 0.57 & 0.56 & 0.63 & 0.13 & 0.13 & 0.13 & 0.13 \\ 
& MAPE ($10^{-1}$)  & 1.98 & 1.87 & 1.82 & 1.98 & 0.89 & 0.91 & 0.9 & 0.9 \\ 
& Coverage (\%)  & 93.3 & 93.8 & 94.8 & 93.1 & 95.1 & 95.1 & 95.1 & 95.2 \\ 
\hline 
\multirow{4}{*}{BE1}& Bias ($10^{-2}$)  & 1.91 & 3.71 & 3.65 & 2.27 & 0.53 & 0.44 & -0.14 & 0.59 \\ 
& MSE ($10^{-1}$)  & 0.45 & 0.44 & 0.42 & 0.46 & 0.11 & 0.11 & 0.11 & 0.11 \\ 
& MAPE ($10^{-1}$)  & 1.67 & 1.65 & 1.58 & 1.7 & 0.81 & 0.84 & 0.83 & 0.83 \\ 
& Coverage (\%)  & 94.6 & 95.9 & 96.1 & 95.2 & 95.3 & 94.4 & 94.8 & 95.5 \\ 
\hline 
\multirow{4}{*}{BE2}& Bias ($10^{-2}$)  & 0.7 & 2.51 & 2.45 & 1.07 & 0.24 & 0.14 & -0.43 & 0.3 \\ 
& MSE ($10^{-1}$)  & 0.44 & 0.44 & 0.41 & 0.46 & 0.11 & 0.11 & 0.11 & 0.11 \\ 
& MAPE ($10^{-1}$)  & 1.67 & 1.64 & 1.58 & 1.7 & 0.82 & 0.84 & 0.83 & 0.83 \\ 
& Coverage (\%)  & 94.6 & 95.9 & 96.1 & 95.2 & 95.3 & 94.4 & 94.8 & 95.5 \\ 
\end{tabular}
\caption{\label{tab:alpha-experiment1}Estimate of bias, MSE, MAPE and Coverage for each of the six methods when the true value of $\alpha$ of the generative process is $\alpha = (1,1,1,1)$ and the number of samples is $n=50$ or $n=200$. 
The estimates are calculated using Monte Carlo with $1,000$ iterations, as described in \autoref{sec:recovering-bivariate-beta}.}
\end{table}

\subsection{The Algorithm Finds a $3/4$-$\MMS$ Allocation}\label{additiveproofs}
In the rest of this section, we prove that the algorithm finds a $3/4$-$\MMS$ allocation. For the sake of contradiction, suppose that the second phase is terminated, which means $\fitems$ is not feasible anymore, but not all agents are satisfied. Such an unsatisfied agent belongs to one of the Clusters $\cone$ or $\ctwo$, or $\cthree$. In Lemmas \ref{c3null}, \ref{c1null}, and \ref{c2null}, we separately rule out each of these possibilities. This implies that all the agents are satisfied and contradicts the assumption. For brevity the proofs are omitted and included in Appendix \ref{additiveproofappendix}. We begin with Cluster $\cthree$.
\begin{lemma}
	\label{c3null}
	At the end of the algorithm we have $\cthree = \emptyset$.
\end{lemma}
%Before proceeding to the proof of Lemma \ref{c3null}, we show Lemmas (\ref{m_2}, \ref{c3bssmall} and \ref{c3sat}). 

\begin{lemma}
\label{m_2}
Let $\agent_i$ be an agent in $\satagents_3$ and let ${\mathbb R}_z$ be the round of the second phase in which $\agent_i$ is satisfied. Then, for any other agent $\agent_j$ that is  in $\cthree^f$ in ${\mathbb R}_z$, $\valu_j(\secondset_i) < 1/2$ holds.
\end{lemma}

\begin{proof}
In ${\mathbb R}_z$, $\agent_i$ either belongs to $\cthree^s$ or $\cthree^b$. Thus, $\agent_j \prec_{pr} \agent_i$, and thus $\secondset_i$ is not feasible for $\agent_j$ in that round. Therefore, $\valu_j(\secondset_i)< 1/2$.
\end{proof}

\begin{lemma}
\label{c3bssmall}
Let $\agent_i \in \satagents_3$ be a satisfied agent and let ${\mathbb R}_z$ be the round in which $\agent_i$ is satisfied. Then, for every other agent $\agent_j$ that belongs to $\cthree^s \cup \cthree^b$ in that round, either $\valu_j(\secondset_i) < \epsilon_j$ or $\valu_j(\firstset_i) \leq 3/4-\epsilon_j$.

\end{lemma}
\begin{proof}
If $\secondset_i$ is not feasible for $\agent_j$, then the condition trivially holds. Moreover, by the definition, the statement is correct for the agents of $\cthree^b$. Therefore, it only suffices  to  consider the case that $\agent_j \in \cthree^s$ and $\secondset_i$ is feasible for $\agent_j$. Due to the priority rules for satisfying the agents in the second phase, $\agent_i \prec_{pr} \agent_j$ and hence, $\agent_i$ cannot be in  $\cthree^b$. Thus, $\agent_i \in \cthree^s$. According to Observation \ref{epsofcluster} and the fact that $\prec_{pr}$ is equivalent to $\prec_{o}$ for the agents in $\cthree^s$, we have $\valu_j(\firstset_i) \leq 3/4 - \epsilon_j$.
\end{proof}



\begin{lemma}
\label{c3sat}
During the second phase, for any agent  $\agent_i$ in  $\cthree$, we have: $$\sum_{\agent_j \in \satagents_3} \valu_i(\firstset_j \cup \secondset_j)< |\satagents_3| + 1/4.$$ 
\end{lemma}

\begin{proof}
To show Lemma \ref{c3sat}, we show that for all the agents $\agent_j \in \satagents_3$ except at most one agent, $\valu_i(\firstset_j \cup \secondset_j)<1$ holds. To show this, let ${\mathbb R}_z$ be an arbitrary round of the second phase, in which an agent $\agent_j \in \cthree$ is satisfied. First, note that in ${\mathbb R}_z$, $\agent_j$ belongs to $\cthree^s \cup \cthree^b$. Also, in round ${\mathbb R}_z$, $\agent_i$ belongs to one of $\cthree^s, \cthree^b$, or $\cthree^f$.  
 
If $\agent_i \in \cthree^f$, then by Lemma \ref{m_2}, $\valu_i(\secondset_j)<1/2$ holds. On the other hand, by definition, $\valu_i(\firstset_j)<1/2$ and hence, $\valu_i(\firstset_j \cup \secondset_j)<1$. 

Now, consider the case, where $\agent_i \in \cthree^b \cup \cthree^s$. Note that by Lemma \ref{c3bssmall}, either $\valu_i(\firstset_j) \leq 3/4-\epsilon_i$ or $\valu_i(\secondset_j) < \epsilon_i$. If $\valu_i(\secondset_j) < \epsilon_i$, then by Lemmas \ref{general} and \ref{c3fsmall}, we know $\valu_i(\firstset_j) < 3/4$ and hence, $\valu_i(\firstset_j \cup \secondset_j)<3/4 + \epsilon_i < 1$. 

For the case where $\valu_i(\firstset_j) \leq 3/4-\epsilon_i$, let $\ite_l$ be the item in $\secondset_j$ with the maximum value to $\agent_i$. By minimality of $\secondset_j$, $\secondset_j \setminus \{\ite_l\}$ is not feasible for any agent, including  $\agent_i$ and thus, $\valu_i(\secondset_j\setminus \{\ite_l\}) < \epsilon_i$. Recall that by Corollary \ref{small_c3}, there is at most one item $\ite_k$ in $\fitems$, such that $\valu_i(\ite _k) \geq 1/4$. In addition to this, $\valu_i(\ite_k) < 1/2$ trivially holds, since $\ite_k$ is not assigned to any agent during the clustering phase. If $\ite_l \neq \ite_k$, $\valu_i(\secondset_j)< 1/4 + \epsilon_i$ holds and hence, $$\valu_i(\firstset_j \cup \secondset_j) < 3/4-\epsilon_i + 1/4 + \epsilon_i<1.$$ Moreover, If $\ite_l = \ite_k$, $\valu_i(\secondset_j)< 1/2 + \epsilon_i$ holds and thus, $\valu_i(\firstset_j \cup \secondset_j) < 3/4-\epsilon_i + 1/2 + \epsilon_i<5/4$. But, this can happen at most one round. Therefore, for all the agents $\agent_j \in \satagents_3$ except at most one, $\valu_i(\firstset_j \cup \secondset_j)<1$. Also, for at most one agent $\agent_j \in \satagents_3$, $\valu_i(\firstset_j \cup \secondset_j)<5/4$. Thus, 
$$\sum_{\agent_j \in \satagents_3} \valu_i(\firstset_j \cup \secondset_j)< |\satagents_3| + 1/4.$$   
\end{proof}



\begin{proof}[of Lemma \ref{c3null}]
Suppose for the sake of contradiction that $\cthree \neq \emptyset$.  Note that, by the definition of $\cthree^b$, if $\cthree^s = \emptyset$ holds, then consequently $\cthree^b = \emptyset$. Therefore, since we have $\cthree = \cthree^s \cup \cthree^b \cup \cthree^f$, if $\cthree$ is non-empty, at least either of the two sets $\cthree^s$ or $\cthree^f$ is non-empty. In case $\cthree^s$ is non-empty, let $\agent_i$ be a winner of $\cthree^s$, otherwise let $\agent_i$ be an arbitrary agent of $\cthree^f$.

According to Lemma \ref{m_1}, for every agent $\agent_j \in \satagents_1^s \cup \satagents_2^s$, $\valu_i(\secondset_j) < {1/2}$ holds. Also, by Lemmas \ref{gsmallc1r} and \ref{cr2smallc3}, for every agent  $\agent_j \in \satagents_1^r \cup \satagents_2^r$, we have $\valu_i(\secondset_j) < {1/2}$. Therefore, 
$$\forall \agent_j \in \satagents_1 \cup \satagents_2 \qquad \valu_i(\secondset_j) < {1/2}.$$

Also, by Lemmas \ref{forc2c3} and \ref{forc3} we know that $\valu_i(\firstset_j)<{1/2}$ for every $\agent_j \in \satagents_1 \cup \satagents_2$. Thus, for every satisfied agent $\agent_j \in \satagents_1 \cup \satagents_2$, $\valu_i(\firstset_j \cup \secondset_j) <1$ holds, and hence 
\begin{equation}\label{eq1}
\sum_{\agent_j \in \satagents_1 \cup \satagents_2} \valu_i (\firstset_j \cup \secondset_j) < |\satagents_1 \cup \satagents_2|.
\end{equation}


Moreover, by Lemma \ref{c3sat}, the total value of items assigned to the agents in $\satagents_3$ to $\agent_i$ is less than $|\satagents_3| + 1/4$. More precisely,
\begin{equation}\label{eq2}
\sum_{\agent_j \in \satagents_3} \valu_i(\firstset_j \cup \secondset_j) \leq |\satagents_3| + 1/4.
\end{equation}
Inequality \eqref{eq1} along with Inequality \eqref{eq2} implies: 
\begin{equation}
\begin{split}
\sum_{\agent_j \in \satagents} \valu_i (\firstset_j \cup \secondset_j) & = \sum_{\agent_j \in \satagents_1 \cup \satagents_2} \valu_i (\firstset_j \cup \secondset_j) + \sum_{\agent_j \in \satagents_3} \valu_i (\firstset_j \cup \secondset_j)\\
& < |\satagents_1 \cup \satagents_2| + |\satagents_3| + 1/4\\
 & = |\satagents |+1/4
\end{split}
\end{equation}

Recall that the total sum of the item values for $\agent_i$ is equal to $n$. In addition to this, since every agent belongs to either of the Clusters $\cone$, $\ctwo$, $\cthree$, or $\satagents$ we have $$|\satagents| + |\cone| + |\ctwo| + |\cthree| = n.$$ Furthermore, every item $\ite_j \in \items$ either belongs to $\fitems$ or one of the sets $\firstset_{j'}$ and $\secondset_{j'}$ for an agent $\agent_{j'}$. More precisely,
$$\fitems = \items \setminus \Big[\bigcup_{\agent_j \in \satagents \cup \cone \cup \ctwo \cup \cthree^s} \firstset_j \cup \bigcup_{\agent_j \in \satagents} \secondset_j\Big].$$ 
 Therefore
\begin{equation}\label{eq5}
\begin{split}
\sum_{\agent_j \in \cone} \valu_i(\firstset_j) + \sum_{\agent_j \in \ctwo} \valu_i(\firstset_j) + \sum_{\agent_j \in \cthree^s} \valu_i(\firstset_j) + \valu_i({\fitems}) & = \valu_i(\items) - \sum_{\agent_j \in \satagents} \valu_i(\firstset_j \cup \secondset_j)\\
&= n - \sum_{\agent_j \in \satagents} \valu_i(\firstset_j \cup \secondset_j)\\
&\geq n - (|\satagents| + 1/4)\\
&= |\cone| + |\ctwo| + |\cthree|-1/4
\end{split}
\end{equation}

According to Lemmas \ref{forc2c3} and \ref{forc2},  
\begin{equation}\label{eq5.1}
\sum_{\agent_j \in \cone} \valu_i(\firstset_j) < {1/2}|\cone|
\end{equation}
 and 
\begin{equation}\label{eq5.2}
\sum_{\agent_j \in \ctwo} \valu_i(\firstset_j)< {1/2} |\ctwo|
\end{equation}
hold. Inequalities \eqref{eq5}, \eqref{eq5.1}, and \eqref{eq5.2} together prove
\begin{equation}\label{eq6}
\begin{split}
\valu_i({\fitems}) &\geq |\cone| + |\ctwo| + |\cthree|-1/4 - \big[\sum_{\agent_j \in \cone} \valu_i(\firstset_j) + \sum_{\agent_j \in \ctwo} \valu_i(\firstset_j) + \sum_{\agent_j \in \cthree^s} \valu_i(\firstset_j)\big]\\
&\geq |\cone| + |\ctwo| + |\cthree|-1/4 - \big[1/2|\cone| + 1/2|\ctwo| + \sum_{\agent_j \in \cthree^s} \valu_i(\firstset_j)\big]\\
&\geq 1/2 |\cone| + 1/2 |\ctwo| + |\cthree| -1/4 - \sum_{\agent_j \in \cthree^s} \valu_i(\firstset_j).
\end{split}
\end{equation}
Now, we consider two cases separately: (i) $\agent_i \in \cthree^s$ and  (ii) $\agent_i \in \cthree^f$.


\textbf{In case $\agent_i \in \cthree^s$}, since $\agent_i$ is a winner of $\cthree^s$, we have 
\begin{equation}
\begin{split}
\sum_{\agent_j \in \cthree^s} \valu_i(\firstset_j) & \leq \sum_{\agent_j \in \cthree^s}  \valu_i(\firstset_i)\\
& = \sum_{\agent_j \in \cthree^s} 3/4 - \epsilon_i\\
& = ({3/4}-\epsilon_i) |\cthree^s|.
\end{split}
\end{equation}
This combined with Inequality \eqref{eq6} concludes
\begin{equation*}
\begin{split}
 \valu(\fitems) &\geq  1/2 |\cone| + 1/2 |\ctwo| + |\cthree| -1/4 - \sum_{\agent_j \in \cthree^s} \valu_i(\firstset_j)\\
 & \geq 1/2 |\cone| + 1/2 |\ctwo| + |\cthree| - 1/4 - ({3/4}-\epsilon_i) |\cthree^s|\\
 & \geq 1/2 |\cone| + 1/2 |\ctwo| + (1/4 + \epsilon) |\cthree| - 1/4.
\end{split}
\end{equation*}
On the other hand, since $\agent_i \in \cthree^s$, $|\cthree| \geq 1$ and hence, $\valu_i({\fitems}) \geq {1/4} + \epsilon_j - {1/4} = \epsilon_j$. This means that $\fitems$ is feasible for $\agent_i$, which contradicts the termination of the algorithm. 

\textbf{In case $\agent_i \in \cthree^f$}, by the definition of $\cthree^f$ we know that $\sum_{\agent_j \in \cthree^s} \valu_i(\firstset_j) < {1/2} |\cthree^s|$, which by Inequality \eqref{eq6} implies:

$$\valu_i({\fitems}) > {1/2}|\cthree^s| + |\cthree^b| + |\cthree^f| + {1/2}|\ctwo| + {1/2}|\cone|-1/4.$$

Since $\agent_i \in \cthree^f$, we have $|\cthree^f| \geq 1$ and hence, $\valu_i({\fitems}) > 3/4$. Again, this contradicts the termination of the algorithm since $\fitems$ is feasible for $\agent_i$.  
\end{proof}


To prove Lemma \ref{c3null} we consider two cases separately. If $\cthree \neq \emptyset$, either there exists an agent $\agent_i \in \cthree^s \cup \cthree^b$ or all the agents of $\cthree$ are in $\cthree^f$. If the former holds, we show $\cthree^s$ is non-empty and assume $\agent_i$ is a winner of $\cthree^s$. We bound the total value of $\agent_i$ for all the items dedicated to other agents and show the value of the remaining items in $\fitems$ is at least $\epsilon_i$ for $\agent_i$. This shows set $\fitems$ is feasible for $\agent_i$ and contradicts the termination of the algorithm. In case all agents of $\cthree$ are in $\cthree^f$, let $\agent_i$ be an arbitrary agent of $\cthree^f$. With a similar argument we show that the value of $\agent_i$ for the remaining unassigned items is at least $3/4$ and conclude that $\fitems$ is feasible for $\agent_i$ which again contradicts the termination of the algorithm.

Next, we prove a similar statement for $\cone$. 
\begin{lemma}
	\label{c1null}
	At the end of the algorithm we have $\cone = \emptyset$.
\end{lemma}
Proof of Lemma \ref{c1null} follows from a coloring argument. Let $\agent_i$ be a winner of $\cone$. We color all items in either blue or white. Roughly speaking, blue items are in a sense \textit{heavy}, i.e., they may have a high valuation to $\agent_i$ whereas white items are somewhat \textit{lighter} and have a low valuation to $\agent_i$. Next, via a double counting argument, we show that $\agent_i$'s value for the items of $\fitems$ is at least $\epsilon_i$ and thus $\fitems$ is feasible for $\agent_i$. This contradicts $\cone = \emptyset$ and shows at the end of the algorithm all agents of $\cone$ are satisfied.

Finally, we show that all the agents in Cluster $\ctwo$ are satisfied by the algorithm.
\begin{lemma}
	\label{c2null}
	At the end of the algorithm we have $\ctwo = \emptyset$.
\end{lemma}
The proof of Lemma \ref{c2null} is a similar to both proofs of Lemmas \ref{c3null} and \ref{c1null}. Let $\agent_i$ be winner of Cluster $\ctwo$. We consider two cases separately. (i) $\epsilon_i \geq 1/8$ and (ii) $\epsilon_i < 1/8$.
In case $\epsilon_i \geq 1/8$, we use a similar argument to the proof of Lemma \ref{c3null} and show $\fitems$ is feasible for $\agent_i$. If $\epsilon_i < 1/8$ we again use a coloring argument, but this time we color the items with 4 different colors. Again, via a double counting argument we show $\fitems$ is feasible for $\agent_i$ and hence every agent of $\ctwo$ is satisfied when the algorithm terminates. 
\begin{theorem}
	\label{34main}
	All the agents are satisfied before the termination of the algorithm.
\end{theorem}
\begin{proof}
	By Lemmas \ref{c3null}, \ref{c1null}, and \ref{c2null}, at the end of the algorithm all agents are satisfied which means each has received a subset of items which is worth at least $3/4$ to him.
\end{proof}
