\section{Additive Agents\protect\footnote{We have created a website at \href{https://www.cs.umd.edu/\~saeedrez/fair.html}{https://www.cs.umd.edu/$\sim$saeedrez/fair.html} for the implemented algorithm and all related materials.}}}\label{additive}
In this section we study the fair allocation problem in the additive setting. We present a proof to the existence of a $3/4$-$\MMS$ allocation when the agents are additive. This improves upon the work of \procacciafirst ~\cite{Procaccia:first} wherein the authors prove a $2/3$-$\MMS$ allocation exists for any combination of additive agents.  As we show, our proof is constructive; given an algorithm that determines the $\MMS$ of an additive set function within a factor $\alpha$, we can implement an algorithm that finds a $3/(4 \alpha)$-$\MMS$ allocation in polynomial time. This married with the PTAS algorithm of \epsteinefficient ~\cite{epstein2014efficient} for finding the $\MMS$ values, results is an algorithm that finds a $3/(4+\epsilon)$-$\MMS$ allocation in polynomial time.

The main idea behind the $3/4$-$\MMS$ allocation is \textit{clustering} the agents. Roughly speaking, we categorize the agents into three clusters, namely $\cone$, $\ctwo$, and $\cthree$. We show that the valuation functions of the agents within each cluster show similar behaviors. Along the clustering process, we allocate the heavy items (the items that have a valuation of at least $1/4$ to some agents) to the agents. By Observation \ref{reducibility}, proving a $3/4$-$\MMS$ guarantee can be narrowed down to only $3/4$-irreducible instances. The $3/4$-irreducibility of the problem guarantees that after the clustering process, the remaining items are light. This enables us to run a $\bagfilling$ process to satisfy the agents. In order to prove the correctness of the algorithm, we take advantage of the properties of each cluster separately.

The organization of this section is summarized in the following: we start by a brief and abstract explanation of the ideas in Section \ref{overview}. In Section \ref{additive:observations} we study the properties of the additive setting and state the main observations that later imply the correctness of our algorithm. Next, in Section \ref{additive:clusters} we discuss a method for clustering the agents and in Section \ref{additive:allocation} we show how we allocate the items to the agents of each cluster to ensure a $3/4$-$\MMS$ guarantee. Finally, in Section \ref{additive:algorithm} we explain the implementation details and prove a polynomial running time for the proposed algorithm.

Throughout this section, we assume $\MMS_i = 1$ for all agents $\agent_i \in \agents$. This is without loss of generality for the existential proof since one can scale the valuation functions to impose this constraint. However, the computational complexity of the allocation will be affected by this assumption since determining the $\MMS$ of an additive function is NP-hard~\cite{epstein2014efficient}. That said, we show in Section \ref{additive:algorithm} that this challenge can be overcome by incurring an additional $1+\epsilon$ factor to the approximation guarantee.  

For brevity, we defer the proofs of Sections \ref{additive:observations}, \ref{additive:clusters}, \ref{additive:allocation}, and \ref{additiveproofs} to Appendices \ref{additiveobservationsproof},\ref{clusteringappendix},\ref{clustering2appendix}, and \ref{additiveproofappendix}, respectively.
\begin{figure}
\centering

\def\picScale{0.08}    % define variable for scaling all pictures evenly
\def\colWidth{0.5\linewidth}

\begin{tikzpicture}
\matrix [row sep=0.25cm, column sep=0cm, style={align=center}] (my matrix) at (0,0) %(2,1)
{
\node[style={anchor=center}] (FREEhand) {\includegraphics[width=0.85\linewidth]{figures/FREEhand.pdf}}; %\fill[blue] (0,0) circle (2pt);
\\
\node[style={anchor=center}] (rigid_v_soft) {\includegraphics[width=0.75\linewidth]{figures/FREE_vs_rigid-v8.pdf}}; %\fill[blue] (0,0) circle (2pt);
\\
};
\node[above] (FREEhand) at ($ (FREEhand.south west)  !0.05! (FREEhand.south east) + (0, 0.1)$) {(a)};
\node[below] (a) at ($ (rigid_v_soft.south west) !0.20! (rigid_v_soft.south east) $) {(b)};
\node[below] (b) at ($ (rigid_v_soft.south west) !0.75! (rigid_v_soft.south east) $) {(c)};
\end{tikzpicture}


% \begin{tikzpicture} %[every node/.style={draw=black}]
% % \draw[help lines] (0,0) grid (4,2);
% \matrix [row sep=0cm, column sep=0cm, style={align=center}] (my matrix) at (0,0) %(2,1)
% {
% \node[style={anchor=center}] {\includegraphics[width=\colWidth]{figures/photos/labFREEs3.jpg}}; %\fill[blue] (0,0) circle (2pt)
% &
% \node[style={anchor=center}] {\includegraphics[width=\colWidth, height=160pt]{figures/stewartRender.png}}; %\fill[blue] (0,0) circle (2pt);
% \\
% };

% %\node[style={anchor=center}] at (0,-5) (FREEstate) {\includegraphics[width=0.7\linewidth]{figures/FREEstate_noLabels2.pdf}};

% \end{tikzpicture}

\caption{\revcomment{2.3}{(a) A fiber-reinforced elastomerc enclosure (FREE) is a soft fluid-driven actuator composed of an elastomer tube with fibers wound around it to impose specific deformations under an increase in volume, such as extension and torsion. (b) A linear actuator and motor combined in \emph{series} has the ability to generate 2 dimensional forces at the end effector (shown in red), but is constrained to motions only in the directions of these forces. (b) Three FREEs combined in \emph{parallel} can generate the same 2 dimensional forces at the end effector (shown in red), without imposing kinematic constraints that prohibit motion in other directions (shown in blue).}}

% \caption{A fiber-reinforced elastomeric enclosure (FREE) (top) is a soft fluid-driven actuator composed of an elastomer tube with fibers wound around it to impose deformation in specific directions upon pressurization, such as extension and torsion. \revcomment{2.3}{In this paper we explore the potential of combining multiple FREEs in parallel to generate fully controllable multi-dimensional spacial forces}, such as in a parallel arrangement around a flexible spine element (bottom-left), or a Stewart Platform arrangement (bottom-right).}

\label{fig:overview}
\end{figure}

\subsection{General Definitions and Observations}\label{additive:observations}
Throughout this section we explore the properties of the fair allocation problem with additive agents.
\subsubsection{Consequences of Irreducibility}
 Since the objective is to prove the existence of a $3/4$-$\MMS$ allocation, by Observation \ref{reducibility}, it only suffices to show every $3/4$-irreducible instance of the problem admits a $3/4$-$\MMS$ allocation. Therefore, in this section we provide several properties of the $3/4$-irreducible instances. We say a set $S$ of items \textit{satisfies} an agent $\agent_i$ if and only if $\valu_i(S) \geq 3/4$. Perhaps the most important consequence of irreducibility is a bound on the valuation of the agents for every item. In the following we show if the problem is $3/4$-irreducible, then no agent has a value of $3/4$ or more for an item. 

\begin{lemma}\label{remove1} 
For every $\alpha$-irreducible instance of the problem we have 
$$\forall \agent_i \in \agents, \ite_j \in \items \hspace{1cm} \valu_i(\ite_j) < \alpha.$$
\end{lemma}

In other words, Lemma \ref{remove1} states that in a $3/4$-irreducible instance of the problem, no item alone can satisfy an agent. 

It is worth mentioning that the proof for Lemma \ref{remove1} does not rely on additivity of the valuation functions and holds as long as the valuations are monotone. Thus, regardless of the type of the valuation functions, one can assume that in any $\alpha$-irreducible instance, value of any item is less than $\alpha$ for any agent. Hence the statement carries over to the submodular, XOS, and subadditive settings. 



As a natural generalization of Lemma \ref{remove1}, we show a similar observation for every pair of items. However, this involves an additional constraint on the valuation of the other agents for the pertinent items. In contrast to Lemma \ref{remove1}, Lemmas \ref{remove2} and \ref{remove3} are restricted to additive setting and their results  do not hold in more general settings.


\begin{lemma}
\label{remove2}
If the problem is $3/4$-irreducible and
$\valu_i(\{\ite_j,\ite_k\}) \geq 3/4$
holds for an agent $\agent_i \in \agents$ and items $\ite_j, \ite_k \in \items$, then there exists an agent $\agent_{i'} \neq \agent_i$ such that
$$\valu_{i'}(\{\ite_j,\ite_k\}) > 1$$
\end{lemma}

According to Lemma \ref{remove2}, in every $3/4$-irreducible instance of the problem, for every agent $\agent_i$ and items $\ite_j,\ite_k$, either $\valu_i(\{\ite_j,\ite_k\}) < 3/4$ or there exists another agent $\agent_{i'} \neq \agent_i$, such that $\valu_{i'}(\{\ite_j,\ite_k\}) > 1$. Otherwise, we can reduce the problem and find a $3/4$-$\MMS$ allocation recursively. More generally, let $S = \{\ite_{j_1},\ite_{j_2},\ldots,\ite_{j_{|S|}}\}$ be a set of items in $\cal{M}$ and $T =\{\agent_{i_1},\agent_{i_2},\ldots,\agent_{i_{|T|}}\}$ be a set of agents such that
\begin{description}
 \item (i) $|S| = 2|T|$
 \item (ii) For every $\agent_{i_a} \in T$ we have $\valu_{i_a}(\{\ite_{j_{2a-1}},\ite_{j_{2a}}\}) \geq 3/4$.
 \item (iii) For every $\agent_{i} \notin T$ we have $\valu_{i}(\{\ite_{j_{2a-1}},\ite_{j_{2a}}\}) \leq 1$ for every $1 \leq a \leq |T|$.
\end{description}
then the problem is $3/4$-reducible.
\begin{lemma}\label{remove3}
In every $3/4$-irreducible instance of the problem, for every set $T =\{\agent_{i_1},\agent_{i_2},\ldots,\agent_{i_{|T|}}\}$ of agents and set $S = \{\ite_{j_1},\ite_{j_2},\ldots,\ite_{j_{|S|}}\}$ of items at least one of the above conditions is violated.
\end{lemma}

% Note that by applying Lemma \ref{remove2} $k$ times, there is no decrease in the maxmin share of the agents in ${\cal{N}}\setminus P$ for the rest of items.  
\subsubsection{Modeling the Problem with Bipartite Graphs}\label{MtPwBG}
In our algorithm we subsequently make use of classic algorithms for bipartite graphs. Let $G = \langle V(G),E(G)\rangle$ be a graph representing the agents and the items. Moreover, let $V(G) = \itemsv \cup \agentsv$ where $\agentsv$ corresponds to the agents and $\itemsv$ corresponds to the items. More precisely, for every agent $\agent_i$ we have a vertex $\agentv_i \in \agentsv$ and every item $\ite_j$ corresponds to a vertex $\itemv_j \in \itemsv$. For every pair of vertices $\agentv_i \in \agentsv$ and $\itemv_j \in \itemsv$, there exists an edge $(\itemv_j,\agentv_i) \in E(G)$ with weight $w(\itemv_j,\agentv_i) = \valu_i(\{\ite_j\})$. We refer to this graph as \emph{the value graph}.

We define an operation on the weighted graphs which we call \textit{filtering}. Roughly speaking, a filtering is an operation that receives a weighted graph as input and removes all of the edges with weight less than a threshold from the graph. Next, we remove all of the isolated\footnote{A vertex is called isolated if no edge is incident to that vertex.} vertices and report the remaining as the filtered graph. In the following we formally define the notion of filtering for weighted graphs.
\begin{definition}
A $\beta$-filtering of a weighted graph $H\langle V(H),E(H)\rangle$, denoted by $H_{\beta}\langle V_\beta(H),E_\beta(H)\rangle$, is a subgraph of $H$ where $V_\beta(H)$ is the set of all vertices in $V(H)$ incident to at least one edge of weight $\beta$ or more and 
$$E_\beta(H) = \{(u,v) \in E(H)| w(u,v) \geq \beta\}.$$ 
\end{definition}
For the case of the value graph, we also denote by $\agentsv_\beta$ and $\itemsv_\beta$ the sets of agents and items corresponding to vertices of $V_\beta(G)$.
\begin{figure}[t!]
    \centering
    \includegraphics[scale=0.8]{figs/filtering}
    \caption{An example of $\beta$-filtering on a graph. After removing the edges with a value smaller than $\beta$, some vertices may become isolated. All such vertices are removed from the filtered graph.}
    \label{fig:filtering}
\end{figure}
%\begin{example}
Figure \ref{fig:filtering} illustrates an example of a graph $H$, together with $H_{0.4}$ and $H_{0.5}$. Note that none of the vertices in $H_{0.4}$ or $H_{0.5}$ are isolated. 
% 
%\end{example}
%%%%%%%%%%%%%%%%

Denote by a maximum matching, a matching that has the highest number of edges in a graph. In definition \ref{FG}, we introduce our main tool for clustering the agents. 
\begin{definition}
\label{FG}
Let $H\langle V(H),E(H)\rangle$ be a bipartite graph with $V(H) = \partone \cup \parttwo$ and let $M$ be a maximum matching of $H$. Define $\parttwo_1$ as the set of the vertices in $\parttwo$ that are not saturated by $M$. Also, define $\parttwo_2$ as the set of vertices in $\parttwo$ that are connected to $\parttwo_1$ by an alternating path and let $\partone_2 = M(\parttwo_2)$, where  $M(\parttwo_2)$ is the set of vertices in $\partone$ that are matched with the vertices of $\parttwo_2$ in $M$. We define $F_{H}(M,\partone)$ as the set of the vertices in $\partone \setminus \partone_2$. 
\end{definition}

For a better understanding of Definition \ref{FG}, consider Figure \ref{fig:FG}. By the definition of alternating paths, there is no edge between the saturated vertices of $F_H(M,\partone)$ and $ \parttwo_1 \cup \parttwo_2$. On the other hand, since $M$ is maximum, the graph doesn't have any augmenting path. Thus, there is no edge between unsaturated vertices in $F_H(M,\partone)$ and $ \parttwo_1 \cup \parttwo_2$. As a result, there is no edge between $F_H(M,\partone)$ and $ \parttwo_1 \cup \parttwo_2$. Furthermore, $F_H(M,\partone)$ has another important property: there exists a matching from $N(F_H(M,\partone))$ to $F_H(M,\partone)$, that saturates all the vertices in $N(F_H(M,\partone))$, where $N(F_H(M,\partone))$ is the set of neighbors of  $F_H(M,\partone)$.

\begin{figure}
\centering
\includegraphics[scale=0.8]{figs/matching}
\caption{Definition of $F_H$}
\label{fig:FG}
\end{figure}

In Lemmas \ref{iff} and \ref{rem}, we prove two remarkable properties for bipartite graphs. As a consequence of these two lemmas, Corollary \ref{remcol} holds for every bipartite graph. We leverage the result of Corollary \ref{remcol} in the clustering phase.
 
\begin{lemma}
\label{rem}
Let $H(V,E)$ be a bipartite graph with $V = \partone \cup \parttwo$ and let $M$ be a maximum matching of $H$. Then, for every set $T \subseteq \partone \setminus F_H(M,\partone)$ we have $|N(T)| > |T|$, where $N(T)$ is the set of neighbors of $T$. 
\end{lemma}

\begin{lemma}
\label{iff}
For a bipartite graph $H(V,E)$ with $V = \partone \cup \parttwo$, $F_H(M,\partone) = \emptyset$ holds, if and only if
for all $T \subseteq \partone$ we have  $|N(T)| > |T|$, where $N(T)$ is the set of neighbors of $T$.

\end{lemma}

\begin{corollary}[of Lemmas \ref{iff} and \ref{rem}]
\label{remcol}
Let $H(V,E)$ be a bipartite graph with $V = \partone \cup \parttwo$ and let $M$ be a maximum matching of $H$. 
Furthermore, let  $H'(V',E')$ be the induced sub-graph of $H$, with $V' = \partone' \cup \parttwo'$, where $\partone' = \partone \setminus F_H(M,\partone)$ and $\parttwo' = \parttwo \setminus N(F_H(M,\partone))$. Then, for any maximum matching $M'$ of $H'$, $F_{H'}(M',\partone') = \emptyset$ holds. 
\end{corollary}
%%%%%%%%%%%%%%%%
\subsubsection{Cycle-envy-freeness and $\MCMWM$}\label{additive:cef}

In the algorithm, we satisfy each agent in two steps. More precisely, we allocate each agent two sets of items that are together of worth at least $3/4$ to him. We denote the first set of items allocated to agent $\agent_i$ by $\firstset_i$ and the second set by $\secondset_i$. Moreover, we attribute the agents with labels \textit{satisfied}, \textit{unsatisfied}, and \textit{semi-satisfied} in the following way:
\begin{enumerate}
	\item An agent $\agent_i$ is satisfied if $\valu_i(\firstset_i \cup \secondset_i) \geq 3/4$.
	\item An agent $\agent_i$ is semi-satisfied if $\firstset_i \neq \emptyset$ but $\secondset_i = \emptyset$. In this case we define $\epsilon_i = 3/4-\valu_i(\firstset_i)$.
	\item An agent $\agent_i$ is unsatisfied if $\firstset_i = \secondset_i = \emptyset$.
\end{enumerate}
As we see, the algorithm maintains the property that for every semi-satisfied agent $\agent_i$, $\valu_i(\firstset_i) \geq 1/2$ holds and hence, $\epsilon_i < 1/4$. 

To capture the competition between different agents, we define an attribution for an ordered pair of agents. We say a semi-satisfied agent envies another semi-satisfied agent, if he prefers to switch sets with the other agent. 

\begin{definition}
\label{winloose}
Let $T$ be a set of semi-satisfied agents. An agent $\agent_i \in T$ envies an agent $\agent_j \in T$, if $\valu_i(\firstset_j) \geq \valu_i(\firstset_i)$. Also, we call an agent $ \agent_i \in T$ a winner of $T$, if $\agent_i$ envies no other agent in $T$. Similarly, we call an agent $\agent_i$ a loser of $T$, if no other agent in $T$ envies $\agent_i$.
\end{definition}

Note that it could be the case that an agent $\agent_i$ is both a loser and a winner of a set $T$ of agents. Based on Definition \ref{winloose}, we next define the notion of \textit{cycle-envy-freeness}.

\begin{definition}
 We call a set $T$ of semi-satisfied agents cycle-envy-free, if every non-empty subset of $T$ contains at least one winner and one loser. 
\end{definition}

Let $C$ be a cycle-envy-free set of semi-satisfied agents. Define the representation graph of $C$ as a digraph $G_C(V(G_C),\overrightarrow{E}(G_C))$, such that for any agent $\agent_i \in C$, there is a vertex $v_i$ in $V(G_C)$ and there is a directed edge from $v_i$ to $v_j$ in $\overrightarrow{E}(G_C)$, if $\agent_i$ envies $\agent_j$. In Lemma \ref{dag}, we show that $G_C$ is acyclic.
\begin{lemma}
\label{dag}
For every cycle-envy-free set of semi-satisfied agents $C$, $G_C$ is a DAG. 
\end{lemma}
\begin{definition}
 A topological ordering of a cycle-envy-free set $C$ of semi-satisfied agents, is a total order $\prec_O$ corresponding to the topological ordering of the representation graph $G_C$. More formally, for the agents $\agent_i,\agent_j \in C$ we have $\agent_i \prec_O \agent_j$ if and only if $v_i$ appears before $v_j$, in the topological ordering of $G_C$.  
\end{definition}

Note that in the topological ordering of a cycle-envy-free set $C$ of semi-satisfied agents, if $\agent_i \in C$ envies $\agent_j \in C$, then $\agent_i \prec_O \agent_j$. 


\begin{observation}
\label{epsofcluster}
Let $C$ be a cycle-envy-free set of semi-satisfied agents. Then, for every agent $\agent_i \in C$ such that $\agent_j \prec_O \agent_i$, we have:
$$\valu_i(\firstset_j) \leq 3/4 - \epsilon_i.$$ 

\end{observation}

We define a maximum cardinality maximum weighted matching of a weighted graph as a matching that has the highest number of edges and among them the one that has the highest total sum of edge weights. For brevity we call such a matching an $\MCMWM$. In Lemma \ref{wm}, we show that an $\MCMWM$ of a weighted bipartite graph has certain properties that makes it useful for building cycle-envy-free clusters. 

\begin{lemma}
\label{wm}
Let $H\langle V(H),E(H)\rangle$ be a weighted bipartite graph with $V(H) = \partone \cup \parttwo$ and let $M = \{(\vone_1,\vtwo_1),...,(\vone_k,\vtwo_k)\}$ be an $\MCMWM$ of $H$. Then, for every subset $T \subseteq \{\vtwo_1,\vtwo_2, \ldots,\vtwo_k\}$, the following conditions hold:

\begin{minipage}[t]{\linegoal}
\begin{enumerate}[leftmargin=*]
 \item There exists a vertex $\vtwo_j \in T$ which is a winner in $T$, i.e.,  $w(\vone_j,\vtwo_{j}) \geq w(\vone_i,\vtwo_j)$, for all $\vone_i \in M(T)$ and  $(\vone_i,\vtwo_j) \in E(H)$. 
\item There exists a vertex $ \vtwo_j \in T$ which is a loser in $T$, i.e.,  $w(\vone_i,\vtwo_i)  \geq w(\vone_j,\vtwo_i) $, for all $\vtwo_i \in T$ and $(\vone_j,\vtwo_i) \in E(H)$.
\item For any vertex $\vtwo_i \in T$ and any unsaturated vertex $\vone_j \in \partone$ such that $(\vone_j,\vtwo_i) \in E(H)$, $w(\vone_i,\vtwo_i) \geq w(\vone_j,\vtwo_i)$. 
\end{enumerate}
\end{minipage}
\\[6pt]
where $M(T)$ is the set of vertices which are matched by the vertices of $T$ in $M$.
\end{lemma}


Notice the similarities of the first and the second conditions of Lemma \ref{wm} with the conditions of the winner and loser in 
 Definition \ref{winloose}. In Section \ref{additive:clusters}, we assign items to the agents based on an $\MCMWM$ of the value-graph. Lemma \ref{wm} ensures that such an assignment results in a cycle-envy-free set of semi-satisfied agents.   

%\begin{definition}
%\label{mp}
%For a set $M$ of items and a set $P$ of semi-satisfied agents, we define $M_P$ as the set of all items in $M$ with 
%the property that at least one agent in $P$ can be satisfied by this item. 
%More formally, $M_P$ contains all the items %$\ite_j \in M$, such that:

%$$\exists \agent_i \in P \mbox{   }s.t \mbox{   %}\valu_i(\firstset_i \cup \{\ite_j\}) \geq 3/4$$
%\end{definition}


%subsection{Definitions for Section 3}
%\begin{definition}
%For a bundle $Q$ of items that satisfies $\agent_i$, the core of $Q$ with respect to agent $\agent_i$, denoted as $C_i(Q)$ is defined as follows: let $\ite_1,\ite_2,..,\ite_k$ be the items of $Q$ in the increasing order of their values for $\agent_i$. Then $C_i(Q) = \{\ite_j,\ite_{j+1},...,\ite_{k}\}$ , where $j$ is the highest index, such that the bundle with items $\{\ite_j,\ite_{j+1},...,\ite_k\}$ satisfies $\agent_i$.
%\end{definition}
 


%\begin{definition}
%\label{satisfaction-graph}
%Given a set $P = \{\agent_1, \agent_2, \ldots, \agent_n\}$ of the semi-satisfied agents and a set $S = \{b_1,b_2,\ldots,b_m\}$, whose elements are bundles of items. The satisfaction-graph with respect to $S$ and $P$, is a bipartite graph denoted by $G^-(\agentsv,\itemsv)$. For each bundle $b_i$ in $S$, there is a related vertex $\agentv_i$ in $\agentsv$ and for each agent $\agent_j$ in $P$, there is a related vertex $\itemv_j$ in $\itemsv$. There is an edge between $\agentv_i$ and $\itemv_j$, if the agent $\agent_j$ can be satisfied by the items in $b_i$ .i.e. $\valu_j( \firstset_j \cup b_i) \geq 3/4$.  
%\end{definition}

\subsection{Phase 1: Building the Clusters}\label{additive:clusters}
In this section, we explain our method for clustering the agents. Intuitively, we divide the agents into three clusters $\cone,\ctwo$ and $\cthree$. As mentioned before, during the algorithm, two sets of items $\firstset_i,\secondset_i$ are allocated to each agent $\agent_i$. Throughout this section, we prove a set of lemmas that are labeled as \emph{value-lemma}. In these lemmas we bound the value of $f_i$ and $g_i$ allocateed to any agent for other agents. A summary of these lemmas is shown in Tables \ref{table0}, \ref{table4} and \ref{table1}. 


After constructing each cluster, we refine that cluster. In the refinement phase of each cluster, we target a certain subset of the remaining items. If any item in this subset could satisfy an agent in the recently created cluster, we allocate that item to the corresponding agent. The goal of the refinement phase is to ensure that the remaining items in the targeted subset are light enough for the agents in that cluster, i.e., none of the remaining items can satisfy an agent in this cluster.

We denote by $\satagents$, the set of satisfied agents. In addition, denote by $\satagents_1, \satagents_2$, and $\satagents_3$ the subsets of $\satagents$, where $\satagents_i$ refers to the agents of $\satagents$ that previously belonged to ${\mathcal C}_i$. Furthermore, we use $\satagents_1^r$ and $\satagents_2^r$ to refer to the agents of $\satagents_1$ and $\satagents_2$ that are satisfied in the refinement phases of $\cone$ and $\ctwo$, respectively.
\subsubsection{Cluster $\cone$} \label{cluster1:building}
Consider the filtering $G_{1/2}\langle V_{1/2}(G),E_{1/2}(G) \rangle$ of the value-graph $G$ and let $M$ be an $\MCMWM$ of $G_{1/2}$. We define Cluster $\cone$ as the set of agents whose corresponding vertex is in $N(F_{G_{1/2}}(M,\itemsv_{1/2}))$. 

For brevity, denote by $V_{\cone}$ the set of vertices in $V(G)$ that correspond to the agents of $\cone$. In other words:
$$V_{\cone} = N(F_{G_{1/2}}(M,\itemsv_{1/2})).$$



 
Also, let $F_{G_{1/2}}(M,\itemsv_{1/2}) $ be $U_1 \cup S_1$, where $U_1$ is the set of unsaturated vertices in $F_{G_{1/2}}(M,\itemsv_{1/2})$ and $S_1$ is the set of the saturated vertices. For each edge $(\itemv_j,\agentv_i) \in M$ such that $\itemv_j \in S_1$, we allocate  item $\ite_j$ to agent $\agent_i$. More precisely, we set $\firstset_i = \{\ite_j \}$. Since $w(\itemv_j,\agentv_i)\geq {1/2}$, we have:
$$\forall \agent_k \in \cone \qquad V_k(f_k) \geq {1/2}.$$
According to the definition of $\epsilon_i$, we have
\begin{equation}
\forall \agent_k \in \cone \qquad \epsilon_k \leq {1/4}.
\end{equation} 

By the definition of $F_{G_{1/2}}$, for every agent which is not in $\cone$, the condition of Lemma \ref{forc2c3} holds. Note that all the agents that are not in $\cone$, belong to either $\ctwo$ or $\cthree$.


\begin{lemma}[value-lemma]
\label{forc2c3}
For all $\agent_i \in \ctwo \cup \cthree$ we have: \[ \forall \agent_j \in \cone \qquad \valu_i(\firstset_j) < 1/2. \]
\end{lemma}

For each vertex $\agentv_i \in V_{\cone}$, denote by $N_{\agentv_i}$ the set of vertices $\itemv_j  \in \itemsv \setminus \itemsv_{1/2}$, where $w(\itemv_j,\agentv_i) \geq \epsilon_i$ and let \[W_1 = U_1 \cup \bigcup_{\agentv_i \in V_{\cone}} N_{\agentv_i}.\]

Note that by definition, for any vertex $\itemv_j \in U_1$ and $\agentv_i \notin V_{\cone}$, there is no edge between $\itemv_j$ and $\agentv_i$ in $G_{1/2}$ and hence $w(\itemv_j,\agentv_i)<1/2$. Also, since the rest of the vertices in $W_1$ are from $\itemsv \setminus \itemsv_{1/2}$, for any vertex $\agentv_i$ and $\itemv_j \in (W_1 \setminus U_1)$, $w(\itemv_j,\agentv_i)<1/2$ holds. Thus, we have the following observation:

\begin{observation}
\label{w1small}
For every item $\ite_j$ with $\itemv_j \in W_1$ and every agent $\agent_i$ with $\agentv_i \notin V_{\cone}$, $\valu_i(\{\ite_j\})<1/2$.
\end{observation}

Now, define $\itemsv'$ and $ \agentsv' $ as follows:

$$\itemsv' = \itemsv \setminus (W_1 \cup S_1),$$ $$\agentsv' = \agentsv \setminus V_{\cone}.$$ 

Let $G'\langle V(G'),E(G')\rangle $ be the induced subgraph of $G$ on $V(G') = \agentsv' \cup \itemsv'$. We use graph $G'$ to build Cluster $\ctwo$. 



\subsubsection{Cluster $\cone$ Refinement} Before building Cluster $\ctwo$, we satisfy some of the agents in $\cone$ with the items corresponding to the vertices of $W_1$. Consider the subgraph $G_1 \langle V(G_1),E(G_1) \rangle$ of $G$ with $V(G_1) = W_1 \cup V_{\cone}$. In $G_1$, There is an edge between $\agentv_i \in V_{\cone}$ and $\itemv_j \in W_1$, if $V_i(\{\ite_j\}) \geq \epsilon_i$. Note that $G_1 \langle V(G_1),E(G_1) \rangle$ is not necessarily an induced subgraph of $G$. We use $G_1$ to satisfy a set of agents in $\cone$. To this end, we first show that $G_1$ admits a special type of matching, described in Lemma \ref{nicematch}.

\begin{lemma}
\label{nicematch}
There exists a matching $M_1$ in $G_1$, that saturates all the vertices of $W_1$ and for any edge $(\itemv_i,\agentv_j) \in M_1$ and any unsaturated vertex $\agentv_k \in N(\itemv_i)$, $\agent_k$ does not envy $\agent_j$. 
\end{lemma}


Let $M_1$ be a matching of $G_1$ with the property described in Lemma \ref{nicematch}. For every edge $(\agentv_i,\itemv_j) \in M_1$, we allocate  item $\ite_j$ to agent $\agent_i$ i.e., we set $\secondset_i = \{\ite_j\}$. By the definition, $\agent_i$ is now satisfied. Thus, we remove $\agent_i$ from $\cone$ and add it to $\cal S$. Note that, after refining $\cone$, none of the items whose corresponding vertex is in $\itemsv' \setminus \itemsv'_{1/2}$ can satisfy any remaining agent in $\cone$. Thus, Observation \ref{fsmallc1} holds.



\begin{observation}
\label{fsmallc1}
For every item $\ite_j$ such that $\itemv_j \in \itemsv'$, either $\itemv_j \in \itemsv'_{1/2}$ or for all $\agent_i \in \cone$, $V_i(\{\ite_j\}) < \epsilon_i$.
\end{observation}

At this point, all the agents of $\satagents$ belong to $\satagents_1^r$. Each one of these agents is satisfied with two items, i.e., for any agent $\agent_j \in \satagents_1^r$, $|\firstset_j| = |\secondset_j| = 1$. In Lemma \ref{gsmallc1r} we give an upper bound on $\valu_i(\secondset_j)$ for every agent $\agent_j \in \satagents_1^r$ and every agent $\agent_i$ in $\ctwo \cup \cthree$.  

\begin{lemma}[value-lemma]
\label{gsmallc1r}
For every agent $\agent_i \in \ctwo \cup \cthree$, we have
$$ \forall \agent_j \in \satagents_1^r \qquad \valu_i(\secondset_j)< 1/2.$$
\end{lemma}

Lemmas \ref{gsmallc1r} and  \ref{forc2c3}  state that for every agent $\agent_i \in \ctwo \cup \cthree$  and every agent $\agent_j \in \satagents_1^r$, $\valu_i(\firstset_j)$ and $\valu_i(\secondset_j)$ are upper bounded by $1/2$. This, together with the fact that $|\firstset_j| = |\secondset_j|=1$, results in Lemma \ref{forc2}.
\begin{lemma}
\label{forc2}
For all $\agent_i \notin \cone$, we have
\[ \MMS_{\valu_i}^{|\agents \setminus \satagents_1^r|} ( {\items} \setminus \bigcup_{\agentv_j \in \satagents_1^r} \firstset_j \cup \secondset_j) \geq 1.\]
\end{lemma}

\subsubsection{Cluster $\ctwo$}
Recall graph $G' \langle V(G') , E(G') \rangle$ as described in the last part of Section \ref{cluster1:building} and let $G'_{1/2}\langle V_{1/2}(G'), E_{1/2}(G') \rangle$ be a $1/2$-filtering of $G'$. Lemma \ref{rem} states that the size of the maximum matching between $\itemsv'_{1/2}$ and $\agentsv'_{1/2}$ is $|\itemsv'_{1/2}|$. Also, according to Corollary \ref{remcol}, for any maximum matching $M'$ of $G'_{1/2}$, $F_{G'_{1/2}}(M',\itemsv'_{1/2})$ is empty. In what follows, we increase the size of the maximum matching in  $G'_{1/2}$ by merging the vertices of $\itemsv' \setminus \itemsv'_{1/2}$ as described in Definition \ref{merge}.

\begin{figure}[t!]
    \centering
    \includegraphics[scale=1]{figs/merge}
    \caption{Merging $\itemv_1$ and $\itemv_2$}
    \label{fig:merge}
\end{figure}


\begin{definition}
\label{merge}
For merging vertices $\itemv_i,\itemv_j$ of $G'(\itemsv',\agentsv')$, we create a new vertex labeled with $\itemv_{i,j}$. Next, we add $\itemv_{i,j}$ to $\itemsv'$ and for every vertex $\agentv_k \in \agentsv'$, we add an edge from $\agentv_k$ to $\itemv_{i,j}$ with weight $w(\agentv_k,\itemv_i) + w(\agentv_k,\itemv_j)$. Finally we remove vertices $\itemv_i$ and $\itemv_j$ from $\itemsv$. See Figure ~\ref{fig:merge}.
\end{definition}

In Lemmas \ref{c1small2} and \ref{pairsmall}, we give upper bounds on the value of the pair of items corresponding to a merged vertex. In Lemma \ref{c1small2}, we show that the value of a merged vertex is less than $2\epsilon_i$ to every agent $\agent_i \in \cone$. This fact is a consequence of Observation \ref{fsmallc1}. Also, in Lemma \ref{pairsmall}, we prove that the value of the items corresponding to a merged vertex is less than $3/4$ to any agent. Lemma \ref{pairsmall} is a direct consequence of $3/4$-irreducibility. In fact, we show that if the condition of Lemma \ref{pairsmall} does not hold, then the problem can be reduced. 

\begin{lemma}
\label{c1small2}
For any agent $\agent_k \in \cone$ and any pair of vertices $\itemv_i, \itemv_j \in \itemsv' \setminus \itemsv'_{1/2}$, $\valu_k(\{\ite_i,\ite_j\}) < 2\epsilon_k$ holds. In particular, total value of the items that belong to a merged vertex is less than $2\epsilon_k$ for $\agent_k$.
\end{lemma}

\begin{lemma}
\label{pairsmall}
 For any pair of vertices $\itemv_i , \itemv_j \in \itemsv' \setminus \itemsv'_{1/2}$ and any vertex $\agentv_k \in \agentsv$, we have $V_k(\{\ite_i,\ite_j\}) < {3/4}$.
\end{lemma}

\begin{corollary} [of Lemma \ref{pairsmall}]
\label{forc2small}
For any agent $\agent_i$ with $\agentv_i \in \agentsv$, there is at most one item $\ite_j$, with $\itemv_j \in \itemsv' \setminus \itemsv'_{1/2}$ and $\valu_i(\{\ite_j\}) \geq {3/8}$.
\end{corollary}

Consider the vertices in $\itemsv' \setminus \itemsv'_{1/2}$. We call a pair $(\itemv_i,\itemv_j)$ of distinct vertices in $\itemsv' \setminus \itemsv'_{1/2}$ \textit{desirable} for $\agentv_k \in \agentsv'$, if $w(\agentv_k,\itemv_i) + w(\agentv_k,\itemv_j) \geq {1/2}$. With this in mind, consider the process described in Algorithm \ref{addvertex}. 

In each step of this process, we find an $\MCMWM$ $M'$ of $G'_{1/2}$. Note that $M'$ changes after each step of the algorithm. Next, we find a pair $(\itemv_i,\itemv_j)$ of the vertices in $\itemsv' \setminus \itemsv'_{1/2}$ that is desirable for at least one agent in $T = \agentsv' \setminus N(F_{G'_{1/2}}(M',\itemsv'_{1/2}))$. If no such pair exists, we terminate the algorithm. Otherwise, we select an arbitrary desirable pair $(\itemv_i,\itemv_j)$ and merge them to obtain a vertex $\itemv_{i,j}$. According to the definition of $T$ in Algorithm \ref{addvertex}, merging a pair  $(\itemv_i,\itemv_j)$ results in an augmenting path in $G'_{1/2}$. Hence, the size of the maximum matching in $G'_{1/2}$ is increased by one. Note that after the termination of Algorithm \ref{addvertex}, either $T = \emptyset$ or  no pair of vertices in $\itemsv' \setminus \itemsv'_{1/2}$ is desirable for any vertex in $T$. 

\begin{lemma} 
\label{sizeeq}
After running Algorithm \ref{addvertex}, we have
$$|F_{G'_{1/2}}(M',\itemsv'_{1/2})| = |N(F_{G'_{1/2}}(M',\itemsv'_{1/2}))|.$$  
\end{lemma}

\begin{algorithm}[t!]
 \KwData{$G'(V(G'),E(G'))$}
 \While{True}{
  $M' = \MCMWM \mbox{  of } G'_{1/2}$\; 
  Find $F_{G'_{1/2}}(M',\itemsv'_{1/2})$\;
  $T = \agentsv' \setminus N(F_{G'_{1/2}}(M',\itemsv'_{1/2}))$\;
   $Q = $ Set of all desirable pairs in $\itemsv' \setminus \itemsv'_{1/2}$ for the agents in $T$\;
  \eIf{$ Q = \emptyset$ }{
   STOP\;
   }{
   Select an arbitrary pair $\itemv_i,\itemv_j$ from $Q$\;
   Merge($\itemv_i,\itemv_j$)\;
  }
 }
 \caption{Merging vertices in $G'$}
 \label{addvertex}
\end{algorithm}

 




We define Cluster $\ctwo$ as the set of agents that correspond to the vertices of $N(F_{G'_{1/2}}(M',\itemsv'_{1/2}))$. Also, denote by $V_{\ctwo}$ the vertices in $N(F_{G'_{1/2}}(M',\itemsv'_{1/2}))$. For each agent $\agent_i \in \ctwo$, we allocate the item corresponding to $M'(\agentv_i)$ (or pair of items in case  $M'(\agentv_i)$ is a merged vertex) to $\agent_i$.


Note that we put the rest of the agents in Cluster $\cthree$. Therefore, Lemma \ref{forc3} holds for all the agents of $\cthree$.

\begin{lemma}[value-lemma]
\label{forc3}
For all $\agent_i \in \cthree$ we have \[ \forall \agent_j \in \ctwo, \valu_i(\firstset_j) < 1/2. \]
\end{lemma}

\subsubsection{Cluster $\ctwo$ Refinement}
The refinement phase of $\ctwo$, is semantically similar to the refinement phase of $\cone$. In the refinement phase of $\ctwo$, we satisfy some of the agents of $\ctwo$ by the items with vertices in $\itemsv' \setminus \itemsv'_{1/2}$. Note that none of the vertices in $\itemsv' \setminus \itemsv'_{1/2}$ is a merged vertex.


The refinement phase of $\ctwo$ is presented in Algorithm \ref{c2ref}. Let $\agent_{i_1}, \agent_{i_2}, \ldots, \agent_{i_k}$ 
\begin{comment}
$\agentv_{i_1}, \agentv_{i_2}, \ldots, \agentv_{i_k}$ 
\end{comment}
be the topological ordering of the agents in $\ctwo$ as described in Section \ref{additive:cef}
\begin{comment}
with respect to their representation graph
\end{comment}
. In Algorithm \ref{c2ref}, We start with $\agentv_{i_1}$ and $W_2 = \emptyset$ and check whether there exists a vertex $\itemv_j \in \itemsv' \setminus (\itemsv'_{1/2} \cup W_2)$ such that $V_{i_1}(\{\ite_j\}) \geq \epsilon_{i_1}$. If so, we add $\itemv_j$ to $W_2$ and satisfy $\agent_{i_1}$ by allocating $\ite_j$ to $\agent_{i_1}$. Next, we repeat the same process for $\agentv_{i_2}$ and continue on to $\agentv_{i_k}$. Note that at the end of the process, $W_2$ refers to the vertices whose corresponding items are allocated to the agents that are satisfied during the refinement step of $\ctwo$. For convenience, let $S_2 = F_{G'_{1/2}}(M',\itemsv'_{1/2})$ and define $\itemsv''$ and $\agentsv''$ as follows:
$$\itemsv'' = \itemsv' \setminus (W_2 \cup S_2),$$
$$\agentsv'' = \agentsv' \setminus V_{\ctwo}.$$

Let $G'' \langle V(G''),E(G'') \rangle$ be the induced subgraph of $G'$ on $V(G'') = \itemsv'' \cup \agentsv''$. We use $G''$ to build Cluster $\cthree$.


\begin{algorithm}[t!]
 \KwData{$G'(V(G'),E(G'))$}
 \KwData{$\agent_{i_1},\agent_{i_2},\ldots,\agent_{i_k}$ = Topological ordering of agents in $\ctwo$}
  \For{$l:1\rightarrow k$}{
	\If{$ \exists \itemv_j \in \itemsv' \setminus (\itemsv'_{1/2} \cup W_2)$ s.t. $V_{i_1}(\{\ite_j\}) \geq \epsilon_{i_l}$)}
	{
		$\secondset_{i_l} = \ite_j$ \;
		$W_2 = W_2 \cup \itemv_j$\;
		$\ctwo = \ctwo \setminus \agent_{i_l}$\;
		${\satagents} = {\satagents} \cup \agent_{i_l}$\;
	}
  }
 \caption{Refinement of $\ctwo$}
 \label{c2ref}
\end{algorithm}


\begin{observation} 
\label{fsmallc2}
After running Algorithm \ref{c2ref}, For every item $\ite_j$ with $\itemv_j \in \itemsv'' \setminus \itemsv''_{1/2} $ and every agent $ \agent_i \in \ctwo$, we have $V_i(\{\ite_j\}) < \epsilon_i$. 
\end{observation}


In the following two lemmas, we give upper bounds on the value of $\secondset_i$ for every agent $\agent_i \in \satagents_2^r$. First, in Lemma \ref{cr2smallc1}, we show that for every agent $\agent_j \in \cone$, $\valu_j(\secondset_i)$ is upper bounded by $\epsilon_j$. Furthermore, by the fact that the agents that are not selected for Clusters $\cone$ and $\ctwo$ belong to Cluster $\cthree$, we show that $\valu_j(\secondset_i)$ is upper bounded by $1/2$ for every agent $\agent_j \in \cthree$. 
\begin{lemma}[value-lemma]
\label{cr2smallc1}
Let $\agent_i \in \satagents_2^r$ be an agent that is satisfied in the refinement phase of Cluster $\ctwo$ and $\agent_j$ be an  agent in $\cone$. Then, $\valu_j(\secondset_i)<\epsilon_j$.
\end{lemma}


\begin{lemma}[value-lemma]
\label{cr2smallc3}
Let $\agent_i \in \satagents_2^r$ be an agent that is satisfied in the refinement phase of Cluster $\ctwo$ and $\agent_j$ be an agent in $\cthree$. Then, $\valu_j(\secondset_i)< 1/2$.
\end{lemma}

\subsubsection{Cluster $\cthree$.} Finally, Cluster $\cthree$ is defined as the set of agents corresponding to the vertices of $\agentsv''$. Let $M''$ be an $\MCMWM$ of $G''_{1/2}$. Note that by Lemma \ref{rem}, all the vertices in $\itemsv''_{1/2}$ are saturated by $M''$. 
For each vertex $\agentv_i$ that is saturated by $M''$, we allocate the item (or pair of items in a case that $M''(\agentv_i)$ is a merged vertex) corresponding to $M''(\agentv_i)$ to $\agent_i$. Unlike the previous clusters, this allocation is temporary. A semi-satisfied agent $\agent_i$ in $\cthree$ may \emph{lend} his $f_i$ to the other agents of $\cthree$. Therefore, we have three type of agents in $\cthree$: 
\begin{enumerate}
    \item \textbf{The semi-satisfied agents}: we denote the set of semi-satisfied agents in $\cthree$ by $\cthree^s$
    \item \textbf{The borrower agents}: the agents that may borrow from a semi-satisfied agent. An agent $\agent_j$ in $\cthree$ is a borrower, if $\agent_j \notin \cthree^s$ and $\max_{\agent_i \in \cthree^S} V_j(f_i) \geq {1/2}$. We denote the set of borrower agents in $\cthree$ by $\cthree^b$.
    \item \textbf{The free agents}: the remaining agents in $\cthree$. We denote the set of free agents by $\cthree^f$.
\end{enumerate}
So far, the agents corresponding to unsaturated vertices in $\agentsv''_{1/2}$ belong to $\cthree^b$ and the agents corresponding to the vertices in $\agentsv'' \setminus \agentsv''_{1/2}$ are in $\cthree^f$. As we see, during the second phase, agents in $\cthree$ may change their type. For example, an agent in $\cthree^s$ may move to $\cthree^f$ or vice versa.  For convenience, for every agent $\agent_i \in \cthree^b$, we define $\epsilon_i$ as follows: 
\begin{equation}
\label{borrowers}
 3/4 - \max_{\agent_j \in \cthree^s}\valu_i(\firstset_j)
\end{equation} 
Note that by the definition, $\epsilon_i \leq 1/4$ holds for every agent of $\cthree^b$.


\begin{figure}[t!]
\centering
\includegraphics[scale=0.5]{figs/overview}
\caption{Overview on the state of the algorithm}
\label{fig:overview}
\end{figure}

In Lemma \ref{lsmall_c3}, we show that the remaining items are not \emph{heavy} for the agents in $\cthree$. The main reason that Lemma \ref{lsmall_c3} holds, is the fact that no pair of vertices is desirable for any agents in $\cthree$ at the end of Algorithm \ref{addvertex}. 
\begin{lemma}
\label{lsmall_c3}
For all $\agent_i \in \cthree$ and $\itemv_j,\itemv_k \in \itemsv'' \setminus \itemsv''_{1/2}$, we have  $V_i(\{\ite_j,\ite_k\}) < {1/2}$.
\end{lemma}

\begin{corollary}[of Lemma \ref{lsmall_c3}]
\label{small_c3}
For any agent $\agent_i \in \cthree$, there is at most one vertex $\itemv_j \in \itemsv'' \setminus \itemsv''_{1/2}$, such that $V_i(\{\ite_j\}) \geq {1/4}$.
\end{corollary}


\subsection{Phase 2: Satisfying the Agents}\label{additive:allocation}
\subsubsection{An Overview on the State of the Algorithm}
Before going through the second phase, we present an overview of the current state of the agents and items. In Figure \ref{fig:overview}, for every agent $\agent_i \in \cone \cup \ctwo \cup \satagents$, $\firstset_i$ is shown by a gray rectangle and for every agent $\agent_i \in \satagents$, $\secondset_i$  is shown by a hatched rectangle. 

Currently, we know that every agent in $\satagents$ belongs to $\satagents_1^r$ or $\satagents_2^r$. These agents are satisfied in the refinement phases of $\cone$ and $\ctwo$. The rest of the agents will be satisfied in the second phase. For brevity, for $i \leq 2$ we use $\satagents_i^s$ to refer to the agents in $\satagents_i$ that are satisfied in the second phase. More formally, $$\mbox{ for }i=1,2 \qquad \satagents_i^s = \satagents_i \setminus \satagents_i^r .$$

Since we didn't refine Cluster $\cthree$, all the agents in the Cluster $\cthree$ are satisfied in the second phase. As mentioned in the previous section, the item allocation to the semi-satisfied agents in $\cthree$ is temporary; That is, we may alter such allocations later. Therefore, in Figure \ref{fig:overview} we illustrate such allocations by dashed lines. 

 

In this section, we denote the set of free items (the items corresponding to the vertices in $\itemsv''\setminus \itemsv''_{1/2}$ at the end of the first phase) by $\fitems$. By Observations \ref{fsmallc1}, \ref{fsmallc2} and Corollary \ref{small_c3}, we know that the items in $\fitems$ have the following properties:
\begin{enumerate}
\item For every agent $\agent_i$ in $\cone$, $\valu_i(\{\ite_j\}) < \epsilon_i$ holds for all $\ite_j \in \fitems$ (Observation \ref{fsmallc1}).
\item For every agent $\agent_i$ in $\ctwo$, $\valu_i(\{\ite_j\}) < \epsilon_i$ holds for all $\ite_j \in \fitems$ (Observation \ref{fsmallc2}).
\item For every agent $\agent_i$ in $\cthree$, there is at most one item $\ite_j \in \fitems$, such that $\valu_i(\{\ite_j\}) \geq 1/4$ (Corollary \ref{small_c3}).
\end{enumerate}


\begin{table}[t]
	\caption{Summary of value lemmas for $f_i$}
	\label{table0} 
	\begin{center}
\begin{tabular}{|c|c|c|c|}
\hline
	& $\forall \agent_i \in \cone$&  $\forall \agent_i \in \ctwo$ & $\forall \agent_i \in \cthree$\\
\hline
$\forall \agent_j \in \cone$&	- & $\valu_i(\firstset_j) < 1/2$ ($\star$)& $\valu_i(\firstset_j) < 1/2$ ($\star$) \\
\hline
$\forall \agent_j \in \ctwo$ & $ \valu_i(\firstset_j) < 3/4 $ ($\ddagger$)  & - & $\valu_i(\firstset_j) < 1/2$ ($\dagger$)\\ 
\hline
$\forall \agent_j \in \cthree^s$ & $ \valu_i(\firstset_j) < 3/4 $($\ddagger$)  & $\valu_i(\firstset_j) <3/4$($\ddagger$)  &  - \\ 
\hline

\end{tabular}
\end{center}
$\hspace{110pt} \star$: Lemma \ref{forc2c3} $\hspace{10pt} \dagger$: Lemma \ref{forc3} $\hspace{10pt} \ddagger$: Lemma \ref{general}\\
\end{table}


\begin{table}[t]
	\caption{Summary of value lemmas for the agents in $\satagents_i^r$}
	\label{table4} 
\begin{center}
	\begin{tabular}{|c|c|c|c|}
\hline
	&  $\forall \agent_i \in \cone $ & $\forall \agent_i \in \ctwo$ & $\forall \agent_i \in \cthree$\\
\hline
$\forall \agent_j \in \satagents_1^r $	 & - & $\valu_i(\secondset_j)<1/2$ ($\star$)& $\valu_i(\secondset_j)<1/2$ ($\star$)\\
\hline
$\forall \agent_j \in \satagents_2^r $	 & $\valu_i(\secondset_j)<\epsilon_i (\dagger)$ & - & $\valu_i(\secondset_j)<1/2$ ($\ddagger$) \\
\hline
\end{tabular}
\end{center}
$\hspace{105pt}$ $\star$: Lemma \ref{gsmallc1r} $\hspace{10pt}$ $\dagger$: Lemma \ref{cr2smallc1} $\hspace{10pt}$ $\ddagger$: Lemma \ref{cr2smallc3}\\

\end{table}

In summary, items of $\fitems$ are small enough, therefore we can run a process similar to the $\bagfilling$ algorithm described earlier to allocate them to the agents. Recall that our clustering and refinement methods preserve the conditions stated in Lemmas \ref{forc2c3}, \ref{gsmallc1r}, \ref{forc3}, \ref{cr2smallc1} and \ref{cr2smallc3}. In addition to this, we state Lemma \ref{general} as follows.

\begin{lemma}[value-lemma]
\label{general}
For every agent $\agent_i \in \cone \cup \ctwo \cup \cthree^s$, we have
$$\forall \agent_j \in \cone \cup \ctwo \cup \cthree \qquad \valu_j(\firstset_i)<3/4.$$
\end{lemma}  
A brief summary of Lemmas \ref{forc2c3}, \ref{gsmallc1r}, \ref{forc3}, \ref{cr2smallc1}, \ref{cr2smallc3} and \ref{general} is illustrated in Tables \ref{table0} and \ref{table4}. Moreover, since sets $\cone,\ctwo$ and $\cthree^s$ are cycle-envy-free, Observation \ref{epsofcluster} holds for these sets. 



\subsubsection{Second Phase: $\bagfilling$}
We begin this section with some definitions. In the following, we define the notion of feasible subsets and, based on that, we define $\phi(S)$ for a feasible subset $S$ of items.
\begin{definition}
A subset $S$ of items in $\fitems$ is feasible, if at least one of the following conditions are met:
\begin{minipage}[t]{\linegoal}
\begin{enumerate}[leftmargin=30pt]
    \item There exists an agent $\agent_i \in \cthree^f $ such that  $\valu_i(\{S\}) \geq {1/2}$. 
    \item There exists an agent $\agent_i \in \cone \cup \ctwo \cup \cthree^s \cup \cthree^b$ such that  $\valu_i(\{S\}) \geq \epsilon_i$.
\end{enumerate}
\end{minipage}
\end{definition}

\begin{definition}
For a feasible set $S$, we define $\Phi(S)$ as the set of agents, that set $S$ is feasible for them. 
\end{definition}

Recall the notion of cycle-envy-freeness and the topological ordering of the agents in a cycle-envy-free set of semi-satisfied agents. Based on this, we define a total order $\prec_{pr}$ to prioritize the agents in the $\bagfilling$ algorithm. 

\begin{definition}
\label{priority}
Define a total order $\prec_{pr}$ on the agents of $\cone \cup \ctwo \cup \cthree$ with the following rules: 
\begin{minipage}[t]{\linegoal}
	
\begin{enumerate}[leftmargin=50pt]
    \item $\agent_{i_5} \prec_{pr} \agent_{i_1} \prec_{pr} \agent_{i_2} \prec_{pr} \agent_{i_3}  \prec_{pr} \agent_{i_4} \qquad$  $\forall \agent_{i_1} \in \cone, \agent_{i_2} \in \ctwo, \agent_{i_3} \in \cthree^s, \agent_{i_4} \in \cthree^b, \agent_{i_5} \in \cthree^f$
    \item $\agent_i \prec_{pr} \agent_j \Leftrightarrow \agent_i \prec_o \agent_j \hspace{85pt}$ $\forall \agent_i, \agent_j \in \cone \cup \ctwo \cup \cthree^s,  \agent_i ,\agent_j \mbox{ in the same cluster }$
	\item $\agent_i \prec_{pr} \agent_j \Leftrightarrow i < j \hspace{100pt}$ $\forall \agent_i,\agent_j \in \cthree^b \vee \agent_i,\agent_j \in \cthree^f$
\end{enumerate}
\end{minipage}
\end{definition}    

\begin{comment}
\begin{array}{cll}
(I)&\agent_{i_5} \prec_{pr} \agent_{i_1} \prec_{pr} \agent_{i_2} \prec_{pr} \agent_{i_3}  \prec_{pr} \agent_{i_4}& \forall \agent_{i_1} \in \cone, \agent_{i_2} \in \ctwo, \agent_{i_3} \in \cthree^s, \agent_{i_4} \in \cthree^b, \agent_{i_5} \in \cthree^f\\[6pt]
(II)&\agent_i \prec_{pr} \agent_j \Leftrightarrow \agent_i \prec_o \agent_j  & \forall \agent_i, \agent_j \in \cone \cup \ctwo \cup \cthree^s, \qquad \agent_i ,\agent_j \mbox{ in the same cluster }\\[6pt]

(III)&\agent_i \prec_{pr} \agent_j \Leftrightarrow i < j & \agent_i,\agent_j \in \cthree^b \vee \agent_i,\agent_j \in \cthree^f\\[6pt]
\end{array}.

\end{comment}


Recall that $\prec_o$ refers to the topological ordering of a semi-satisfied set of agents. Roughly speaking, for the semi-satisfied agents in the same cluster, $\prec_{pr}$ behaves in the same way as $\prec_{o}$. Furthermore, for the agents in different clusters, agents in $\cthree^f , \cone , \ctwo, \cthree^s , \cthree^b$ have a lower priority, respectively. Finally, the order of the agents in $\cthree^b$ and $\cthree^f$ is determined by their index, i.e., the agent with a lower index has a lower priority.


The second phase consists of several rounds and every round has two steps. Each of these two steps is described below. We continue running this algorithm until $\fitems$ is no longer feasible for any agent.
\begin{itemize}
\item \textbf{Step1}: In the first step, we run a process very similar to the $\bagfilling$ algorithm described in Section \ref{introduction}. That is, we find a feasible subset $S \subseteq \fitems$, such that $|S|$ is minimal. Such a subset can easily be found, using a slight modification of the $\bagfilling$ process (see Section \ref{sphase}).  

\item \textbf{Step2}: In the second step, we choose an agent to allocate set $S$ to him. In contrast to the $\bagfilling$ algorithm, we do not select an arbitrary agent. Instead, we select the agent in $\Phi(S)$ with the lowest priority regarding $\prec_{pr}$, i.e., smallest element in $\Phi(S)$ regarding $\prec_{pr}$. Let $\agent_i$ be the selected agent. We consider three cases separately:

\begin{minipage}[t]{\linegoal}
\begin{enumerate}[leftmargin=50pt]
    \item $\agent_i \in \cthree^f$: temporarily allocate $S$ to $\agent_i$, i.e., set $\firstset_i = S$. 
    \item $\agent_i \in \cthree^b$: let $\agent_j$ be the agent that $\valu_i(\firstset_j) = {3/4} - \epsilon_j$. 
Take back $\firstset_j$ from $\agent_j$ and allocate $\firstset_j \cup S$ to $\agent_i$ i.e. set $\firstset_i = \firstset_j$, $\firstset_j=\emptyset$ and $\secondset_i = S$. Remove $\agent_i$ from $\cthree$ and add it to $\satagents$.
    \item $\agent_i \in \cone \cup \ctwo \cup \cthree^s$: satisfy agent $\agent_i$ by $S$, i.e., set $\secondset_i = S$ and remove $\agent_i$ from its corresponding cluster and add it to $\satagents$. 
\end{enumerate}
\end{minipage}

By the construction of $\cthree^s,\cthree^b$, and $\cthree^f$, the above process may cause agents in $\cthree$ to move in between $\cthree^s,\cthree^b$ and $\cthree^f$. For example, if the first case happens, then $\agent_i$ is moved from $\cthree^f$ to $\cthree^s$. In addition, all other agents in $\cthree^f$ for which $S$ is feasible are moved to $\cthree^b$. For the second case, $\agent_j$ is moved to one of $\cthree^f$ or $\cthree^b$, based on $\valu_j(\firstset_k)$ for every $\agent_k \in \cthree^s$; that is, if there exists an agent $\agent_k \in \cthree^s$ such that $\valu_j(\firstset_k) \geq 1/2$, $\agent_j$ is moved to $\cthree^b$. Otherwise, $\agent_j$ is moved to $\cthree^f$. For both the second and the third cases, some of the agents in $\cthree^b$ may move to $\cthree^f$. 
\end{itemize}
The second phase terminates, when $\fitems$ is no longer feasible for any agent. More details about the second phase can be found in Algorithm \ref{second-phase}.  In Algorithm \ref{second-phase}, we use $Update(\cthree)$ to refer the process of moving agents among $\cthree^s, \cthree^b$ and $\cthree^f$.
 

\begin{algorithm}[t!]
 \KwData{$\fitems, \cone,\ctwo,\cthree$}
  \While{$\fitems$ is feasible}{
	$S$ = a minimal feasible subset of $\fitems$ \;
	$\agent_i = $ agent in $\Phi(S)$ with lowest order regarding  $\prec_{pr}$\;
	\If{$\agent_i \in C_3^f$}
	{
		$\firstset_i = S$ \;
		$Update(\cthree)$ \;
	}
	\If{$\agent_i \in \cthree^b$}
	{
		Let $\agent_j$ be the agent that $\valu_i(\firstset_j) = 3/4 - \epsilon_i$ \;
		$\firstset_i = \firstset_j$ \;
		$\secondset_i = S$ \;
		$\satagents = \satagents \cup \agent_i$ \;
		$\firstset_j = \emptyset$\;
		$\cthree = \cthree \setminus \agent_i$ \;
		$Update(\cthree)$ \;
	}
	\If {$\agent_i \in \cthree^s$}
	{
		$\secondset_i = S$\;
		$\satagents = \satagents \cup \agent_i$\;
		$\cthree = \cthree \setminus \agent_i$ \;
		$Update(\cthree)$ \;
	}
	\If {$\agent_i \in \cone \cup \ctwo$}
	{
		$\secondset_i = S$\;
		remove $\agent_i$ from its corresponding cluster \;
		$\satagents = \satagents \cup \agent_i$\;

	}
}
 \caption{The Second Phase}
 \label{second-phase}
\end{algorithm}

In each round of the second phase, either an agent is satisfied or an agent in $\cthree^f$ becomes semi-satisfied. In Lemma \ref{c3fsmall}, we show that if an agent $\agent_i \in \cthree^f$ is selected in some round of the second phase, then $\valu_j(\firstset_i)$ is upper bounded by $2\epsilon_j$ for every agent $\agent_j \in \cthree \cup \ctwo \cup \cone^s \cup \cone^b$. As a consequence of Lemma \ref{c3fsmall}, in Lemma \ref{cef} we show that sets $\cone,\ctwo$ and $\cthree$ remain cycle-envy-free during the second phase. For convenience, we use $\mathbb{R}_z$ to refer to the $z$'th round of the second phase. 

\begin{lemma}
\label{c3fsmall}
Let $\mathbb{R}_z$ be a round of the second phase that an agent $\agent_i \in \cthree^f$ is selected. Then, for every agent $\agent_j \in \cthree \cup \ctwo \cup \cone^s \cup \cone^b$, we have $\valu_j(\firstset_i)<2\epsilon_j<3/4$.
\end{lemma}



\begin{lemma}
\label{cef}
During the second phase, the $\cone,\ctwo$ and $\cthree^s$ maintain the property of cycle-envy-freeness. 
\end{lemma}

Finally, for the rounds that an agents $\agent_i$ is satisfied, Lemmas \ref{prvalue} and \ref{m_1} give upper bounds on the value of $\secondset_i$ for remaining agents in different clusters. 

\begin{lemma}[value-lemma]
\label{prvalue}
Let $\agent_i \in \satagents$ be an agent that is satisfied in the second phase. Then, for every other agent $\agent_j \in \cone \cup \ctwo$ we have:

\begin{minipage}[t]{\linegoal}
\begin{enumerate}[leftmargin=30pt]
\item If $\agent_j \prec_{pr} \agent_i$, then $\valu_j(\secondset_i) < \epsilon_j$.
\item If $\agent_i \prec_{pr} \agent_j$, then $\valu_j(\secondset_i) < 2\epsilon_j$.
\end{enumerate}
\end{minipage}
\end{lemma}

\begin{lemma}[value-lemma]
\label{m_1}
Let $\agent_i$ be an agent in $\satagents_1^s \cup \satagents_2^s$. Then, for every agent $\agent_j \in \cthree$, we have $\valu_j(\secondset_i) < {1/2}$.
\end{lemma}

The results of Lemmas \ref{prvalue} and \ref{m_1} are summarized in Table \ref{table1}.





\begin{table}[htbp]
\centering
\begin{tabular}{c|c|cccc|cccc}
                     &            & \multicolumn{4}{c|}{$n=50$}                       & \multicolumn{4}{c}{$n=200$}                      \\ \hline
Method               & Evaluation & $\alpha_1$ & $\alpha_2$ & $\alpha_3$ & $\alpha_4$ & $\alpha_1$ & $\alpha_2$ & $\alpha_3$ & $\alpha_4$ \\ \hline
\multirow{4}{*}{MM1}& Bias ($10^{-2}$)  & 1.79 & 4.08 & 4.01 & 2.43 & 0.23 & 0.33 & -0.29 & 0.31 \\ 
& MSE ($10^{-1}$)  & 0.78 & 0.78 & 0.76 & 0.84 & 0.17 & 0.17 & 0.17 & 0.18 \\ 
& MAPE ($10^{-1}$)  & 2.23 & 2.18 & 2.11 & 2.25 & 1.04 & 1.05 & 1.01 & 1.04 \\ 
& Coverage (\%)  & 94.7 & 93.9 & 95.1 & 93.0 & 94.4 & 93.5 & 94.1 & 93.2 \\ 
\hline 
\multirow{4}{*}{MM2}& Bias ($10^{-2}$)  & 2.12 & 4.32 & 4.31 & 2.68 & 0.41 & 0.52 & -0.09 & 0.46 \\ 
& MSE ($10^{-1}$)  & 0.63 & 0.6 & 0.59 & 0.66 & 0.13 & 0.13 & 0.13 & 0.13 \\ 
& MAPE ($10^{-1}$)  & 2.01 & 1.91 & 1.87 & 2.02 & 0.9 & 0.91 & 0.9 & 0.9 \\ 
& Coverage (\%)  & 93.0 & 93.8 & 94.3 & 92.4 & 94.4 & 94.3 & 95.3 & 95.5 \\ 
\hline 
\multirow{4}{*}{MM3}& Bias ($10^{-2}$)  & 2.12 & 4.32 & 4.31 & 2.68 & 0.41 & 0.52 & -0.09 & 0.46 \\ 
& MSE ($10^{-1}$)  & 0.63 & 0.6 & 0.59 & 0.66 & 0.13 & 0.13 & 0.13 & 0.13 \\ 
& MAPE ($10^{-1}$)  & 2.01 & 1.91 & 1.87 & 2.02 & 0.9 & 0.91 & 0.9 & 0.9 \\ 
& Coverage (\%)  & 93.8 & 94.0 & 94.4 & 92.8 & 95.2 & 94.4 & 94.7 & 95.5 \\ 
\hline 
\multirow{4}{*}{MM4}& Bias ($10^{-2}$)  & 0.88 & 3.06 & 3.04 & 1.44 & 0.11 & 0.22 & -0.39 & 0.16 \\ 
& MSE ($10^{-1}$)  & 0.61 & 0.57 & 0.56 & 0.63 & 0.13 & 0.13 & 0.13 & 0.13 \\ 
& MAPE ($10^{-1}$)  & 1.98 & 1.87 & 1.82 & 1.98 & 0.89 & 0.91 & 0.9 & 0.9 \\ 
& Coverage (\%)  & 93.3 & 93.8 & 94.8 & 93.1 & 95.1 & 95.1 & 95.1 & 95.2 \\ 
\hline 
\multirow{4}{*}{BE1}& Bias ($10^{-2}$)  & 1.91 & 3.71 & 3.65 & 2.27 & 0.53 & 0.44 & -0.14 & 0.59 \\ 
& MSE ($10^{-1}$)  & 0.45 & 0.44 & 0.42 & 0.46 & 0.11 & 0.11 & 0.11 & 0.11 \\ 
& MAPE ($10^{-1}$)  & 1.67 & 1.65 & 1.58 & 1.7 & 0.81 & 0.84 & 0.83 & 0.83 \\ 
& Coverage (\%)  & 94.6 & 95.9 & 96.1 & 95.2 & 95.3 & 94.4 & 94.8 & 95.5 \\ 
\hline 
\multirow{4}{*}{BE2}& Bias ($10^{-2}$)  & 0.7 & 2.51 & 2.45 & 1.07 & 0.24 & 0.14 & -0.43 & 0.3 \\ 
& MSE ($10^{-1}$)  & 0.44 & 0.44 & 0.41 & 0.46 & 0.11 & 0.11 & 0.11 & 0.11 \\ 
& MAPE ($10^{-1}$)  & 1.67 & 1.64 & 1.58 & 1.7 & 0.82 & 0.84 & 0.83 & 0.83 \\ 
& Coverage (\%)  & 94.6 & 95.9 & 96.1 & 95.2 & 95.3 & 94.4 & 94.8 & 95.5 \\ 
\end{tabular}
\caption{\label{tab:alpha-experiment1}Estimate of bias, MSE, MAPE and Coverage for each of the six methods when the true value of $\alpha$ of the generative process is $\alpha = (1,1,1,1)$ and the number of samples is $n=50$ or $n=200$. 
The estimates are calculated using Monte Carlo with $1,000$ iterations, as described in \autoref{sec:recovering-bivariate-beta}.}
\end{table}

\subsection{The Algorithm Finds a $3/4$-$\MMS$ Allocation}\label{additiveproofs}
In the rest of this section, we prove that the algorithm finds a $3/4$-$\MMS$ allocation. For the sake of contradiction, suppose that the second phase is terminated, which means $\fitems$ is not feasible anymore, but not all agents are satisfied. Such an unsatisfied agent belongs to one of the Clusters $\cone$ or $\ctwo$, or $\cthree$. In Lemmas \ref{c3null}, \ref{c1null}, and \ref{c2null}, we separately rule out each of these possibilities. This implies that all the agents are satisfied and contradicts the assumption. For brevity the proofs are omitted and included in Appendix \ref{additiveproofappendix}. We begin with Cluster $\cthree$.
\begin{lemma}
	\label{c3null}
	At the end of the algorithm we have $\cthree = \emptyset$.
\end{lemma}
%Before proceeding to the proof of Lemma \ref{c3null}, we show Lemmas (\ref{m_2}, \ref{c3bssmall} and \ref{c3sat}). 

\begin{lemma}
\label{m_2}
Let $\agent_i$ be an agent in $\satagents_3$ and let ${\mathbb R}_z$ be the round of the second phase in which $\agent_i$ is satisfied. Then, for any other agent $\agent_j$ that is  in $\cthree^f$ in ${\mathbb R}_z$, $\valu_j(\secondset_i) < 1/2$ holds.
\end{lemma}

\begin{proof}
In ${\mathbb R}_z$, $\agent_i$ either belongs to $\cthree^s$ or $\cthree^b$. Thus, $\agent_j \prec_{pr} \agent_i$, and thus $\secondset_i$ is not feasible for $\agent_j$ in that round. Therefore, $\valu_j(\secondset_i)< 1/2$.
\end{proof}

\begin{lemma}
\label{c3bssmall}
Let $\agent_i \in \satagents_3$ be a satisfied agent and let ${\mathbb R}_z$ be the round in which $\agent_i$ is satisfied. Then, for every other agent $\agent_j$ that belongs to $\cthree^s \cup \cthree^b$ in that round, either $\valu_j(\secondset_i) < \epsilon_j$ or $\valu_j(\firstset_i) \leq 3/4-\epsilon_j$.

\end{lemma}
\begin{proof}
If $\secondset_i$ is not feasible for $\agent_j$, then the condition trivially holds. Moreover, by the definition, the statement is correct for the agents of $\cthree^b$. Therefore, it only suffices  to  consider the case that $\agent_j \in \cthree^s$ and $\secondset_i$ is feasible for $\agent_j$. Due to the priority rules for satisfying the agents in the second phase, $\agent_i \prec_{pr} \agent_j$ and hence, $\agent_i$ cannot be in  $\cthree^b$. Thus, $\agent_i \in \cthree^s$. According to Observation \ref{epsofcluster} and the fact that $\prec_{pr}$ is equivalent to $\prec_{o}$ for the agents in $\cthree^s$, we have $\valu_j(\firstset_i) \leq 3/4 - \epsilon_j$.
\end{proof}



\begin{lemma}
\label{c3sat}
During the second phase, for any agent  $\agent_i$ in  $\cthree$, we have: $$\sum_{\agent_j \in \satagents_3} \valu_i(\firstset_j \cup \secondset_j)< |\satagents_3| + 1/4.$$ 
\end{lemma}

\begin{proof}
To show Lemma \ref{c3sat}, we show that for all the agents $\agent_j \in \satagents_3$ except at most one agent, $\valu_i(\firstset_j \cup \secondset_j)<1$ holds. To show this, let ${\mathbb R}_z$ be an arbitrary round of the second phase, in which an agent $\agent_j \in \cthree$ is satisfied. First, note that in ${\mathbb R}_z$, $\agent_j$ belongs to $\cthree^s \cup \cthree^b$. Also, in round ${\mathbb R}_z$, $\agent_i$ belongs to one of $\cthree^s, \cthree^b$, or $\cthree^f$.  
 
If $\agent_i \in \cthree^f$, then by Lemma \ref{m_2}, $\valu_i(\secondset_j)<1/2$ holds. On the other hand, by definition, $\valu_i(\firstset_j)<1/2$ and hence, $\valu_i(\firstset_j \cup \secondset_j)<1$. 

Now, consider the case, where $\agent_i \in \cthree^b \cup \cthree^s$. Note that by Lemma \ref{c3bssmall}, either $\valu_i(\firstset_j) \leq 3/4-\epsilon_i$ or $\valu_i(\secondset_j) < \epsilon_i$. If $\valu_i(\secondset_j) < \epsilon_i$, then by Lemmas \ref{general} and \ref{c3fsmall}, we know $\valu_i(\firstset_j) < 3/4$ and hence, $\valu_i(\firstset_j \cup \secondset_j)<3/4 + \epsilon_i < 1$. 

For the case where $\valu_i(\firstset_j) \leq 3/4-\epsilon_i$, let $\ite_l$ be the item in $\secondset_j$ with the maximum value to $\agent_i$. By minimality of $\secondset_j$, $\secondset_j \setminus \{\ite_l\}$ is not feasible for any agent, including  $\agent_i$ and thus, $\valu_i(\secondset_j\setminus \{\ite_l\}) < \epsilon_i$. Recall that by Corollary \ref{small_c3}, there is at most one item $\ite_k$ in $\fitems$, such that $\valu_i(\ite _k) \geq 1/4$. In addition to this, $\valu_i(\ite_k) < 1/2$ trivially holds, since $\ite_k$ is not assigned to any agent during the clustering phase. If $\ite_l \neq \ite_k$, $\valu_i(\secondset_j)< 1/4 + \epsilon_i$ holds and hence, $$\valu_i(\firstset_j \cup \secondset_j) < 3/4-\epsilon_i + 1/4 + \epsilon_i<1.$$ Moreover, If $\ite_l = \ite_k$, $\valu_i(\secondset_j)< 1/2 + \epsilon_i$ holds and thus, $\valu_i(\firstset_j \cup \secondset_j) < 3/4-\epsilon_i + 1/2 + \epsilon_i<5/4$. But, this can happen at most one round. Therefore, for all the agents $\agent_j \in \satagents_3$ except at most one, $\valu_i(\firstset_j \cup \secondset_j)<1$. Also, for at most one agent $\agent_j \in \satagents_3$, $\valu_i(\firstset_j \cup \secondset_j)<5/4$. Thus, 
$$\sum_{\agent_j \in \satagents_3} \valu_i(\firstset_j \cup \secondset_j)< |\satagents_3| + 1/4.$$   
\end{proof}



\begin{proof}[of Lemma \ref{c3null}]
Suppose for the sake of contradiction that $\cthree \neq \emptyset$.  Note that, by the definition of $\cthree^b$, if $\cthree^s = \emptyset$ holds, then consequently $\cthree^b = \emptyset$. Therefore, since we have $\cthree = \cthree^s \cup \cthree^b \cup \cthree^f$, if $\cthree$ is non-empty, at least either of the two sets $\cthree^s$ or $\cthree^f$ is non-empty. In case $\cthree^s$ is non-empty, let $\agent_i$ be a winner of $\cthree^s$, otherwise let $\agent_i$ be an arbitrary agent of $\cthree^f$.

According to Lemma \ref{m_1}, for every agent $\agent_j \in \satagents_1^s \cup \satagents_2^s$, $\valu_i(\secondset_j) < {1/2}$ holds. Also, by Lemmas \ref{gsmallc1r} and \ref{cr2smallc3}, for every agent  $\agent_j \in \satagents_1^r \cup \satagents_2^r$, we have $\valu_i(\secondset_j) < {1/2}$. Therefore, 
$$\forall \agent_j \in \satagents_1 \cup \satagents_2 \qquad \valu_i(\secondset_j) < {1/2}.$$

Also, by Lemmas \ref{forc2c3} and \ref{forc3} we know that $\valu_i(\firstset_j)<{1/2}$ for every $\agent_j \in \satagents_1 \cup \satagents_2$. Thus, for every satisfied agent $\agent_j \in \satagents_1 \cup \satagents_2$, $\valu_i(\firstset_j \cup \secondset_j) <1$ holds, and hence 
\begin{equation}\label{eq1}
\sum_{\agent_j \in \satagents_1 \cup \satagents_2} \valu_i (\firstset_j \cup \secondset_j) < |\satagents_1 \cup \satagents_2|.
\end{equation}


Moreover, by Lemma \ref{c3sat}, the total value of items assigned to the agents in $\satagents_3$ to $\agent_i$ is less than $|\satagents_3| + 1/4$. More precisely,
\begin{equation}\label{eq2}
\sum_{\agent_j \in \satagents_3} \valu_i(\firstset_j \cup \secondset_j) \leq |\satagents_3| + 1/4.
\end{equation}
Inequality \eqref{eq1} along with Inequality \eqref{eq2} implies: 
\begin{equation}
\begin{split}
\sum_{\agent_j \in \satagents} \valu_i (\firstset_j \cup \secondset_j) & = \sum_{\agent_j \in \satagents_1 \cup \satagents_2} \valu_i (\firstset_j \cup \secondset_j) + \sum_{\agent_j \in \satagents_3} \valu_i (\firstset_j \cup \secondset_j)\\
& < |\satagents_1 \cup \satagents_2| + |\satagents_3| + 1/4\\
 & = |\satagents |+1/4
\end{split}
\end{equation}

Recall that the total sum of the item values for $\agent_i$ is equal to $n$. In addition to this, since every agent belongs to either of the Clusters $\cone$, $\ctwo$, $\cthree$, or $\satagents$ we have $$|\satagents| + |\cone| + |\ctwo| + |\cthree| = n.$$ Furthermore, every item $\ite_j \in \items$ either belongs to $\fitems$ or one of the sets $\firstset_{j'}$ and $\secondset_{j'}$ for an agent $\agent_{j'}$. More precisely,
$$\fitems = \items \setminus \Big[\bigcup_{\agent_j \in \satagents \cup \cone \cup \ctwo \cup \cthree^s} \firstset_j \cup \bigcup_{\agent_j \in \satagents} \secondset_j\Big].$$ 
 Therefore
\begin{equation}\label{eq5}
\begin{split}
\sum_{\agent_j \in \cone} \valu_i(\firstset_j) + \sum_{\agent_j \in \ctwo} \valu_i(\firstset_j) + \sum_{\agent_j \in \cthree^s} \valu_i(\firstset_j) + \valu_i({\fitems}) & = \valu_i(\items) - \sum_{\agent_j \in \satagents} \valu_i(\firstset_j \cup \secondset_j)\\
&= n - \sum_{\agent_j \in \satagents} \valu_i(\firstset_j \cup \secondset_j)\\
&\geq n - (|\satagents| + 1/4)\\
&= |\cone| + |\ctwo| + |\cthree|-1/4
\end{split}
\end{equation}

According to Lemmas \ref{forc2c3} and \ref{forc2},  
\begin{equation}\label{eq5.1}
\sum_{\agent_j \in \cone} \valu_i(\firstset_j) < {1/2}|\cone|
\end{equation}
 and 
\begin{equation}\label{eq5.2}
\sum_{\agent_j \in \ctwo} \valu_i(\firstset_j)< {1/2} |\ctwo|
\end{equation}
hold. Inequalities \eqref{eq5}, \eqref{eq5.1}, and \eqref{eq5.2} together prove
\begin{equation}\label{eq6}
\begin{split}
\valu_i({\fitems}) &\geq |\cone| + |\ctwo| + |\cthree|-1/4 - \big[\sum_{\agent_j \in \cone} \valu_i(\firstset_j) + \sum_{\agent_j \in \ctwo} \valu_i(\firstset_j) + \sum_{\agent_j \in \cthree^s} \valu_i(\firstset_j)\big]\\
&\geq |\cone| + |\ctwo| + |\cthree|-1/4 - \big[1/2|\cone| + 1/2|\ctwo| + \sum_{\agent_j \in \cthree^s} \valu_i(\firstset_j)\big]\\
&\geq 1/2 |\cone| + 1/2 |\ctwo| + |\cthree| -1/4 - \sum_{\agent_j \in \cthree^s} \valu_i(\firstset_j).
\end{split}
\end{equation}
Now, we consider two cases separately: (i) $\agent_i \in \cthree^s$ and  (ii) $\agent_i \in \cthree^f$.


\textbf{In case $\agent_i \in \cthree^s$}, since $\agent_i$ is a winner of $\cthree^s$, we have 
\begin{equation}
\begin{split}
\sum_{\agent_j \in \cthree^s} \valu_i(\firstset_j) & \leq \sum_{\agent_j \in \cthree^s}  \valu_i(\firstset_i)\\
& = \sum_{\agent_j \in \cthree^s} 3/4 - \epsilon_i\\
& = ({3/4}-\epsilon_i) |\cthree^s|.
\end{split}
\end{equation}
This combined with Inequality \eqref{eq6} concludes
\begin{equation*}
\begin{split}
 \valu(\fitems) &\geq  1/2 |\cone| + 1/2 |\ctwo| + |\cthree| -1/4 - \sum_{\agent_j \in \cthree^s} \valu_i(\firstset_j)\\
 & \geq 1/2 |\cone| + 1/2 |\ctwo| + |\cthree| - 1/4 - ({3/4}-\epsilon_i) |\cthree^s|\\
 & \geq 1/2 |\cone| + 1/2 |\ctwo| + (1/4 + \epsilon) |\cthree| - 1/4.
\end{split}
\end{equation*}
On the other hand, since $\agent_i \in \cthree^s$, $|\cthree| \geq 1$ and hence, $\valu_i({\fitems}) \geq {1/4} + \epsilon_j - {1/4} = \epsilon_j$. This means that $\fitems$ is feasible for $\agent_i$, which contradicts the termination of the algorithm. 

\textbf{In case $\agent_i \in \cthree^f$}, by the definition of $\cthree^f$ we know that $\sum_{\agent_j \in \cthree^s} \valu_i(\firstset_j) < {1/2} |\cthree^s|$, which by Inequality \eqref{eq6} implies:

$$\valu_i({\fitems}) > {1/2}|\cthree^s| + |\cthree^b| + |\cthree^f| + {1/2}|\ctwo| + {1/2}|\cone|-1/4.$$

Since $\agent_i \in \cthree^f$, we have $|\cthree^f| \geq 1$ and hence, $\valu_i({\fitems}) > 3/4$. Again, this contradicts the termination of the algorithm since $\fitems$ is feasible for $\agent_i$.  
\end{proof}


To prove Lemma \ref{c3null} we consider two cases separately. If $\cthree \neq \emptyset$, either there exists an agent $\agent_i \in \cthree^s \cup \cthree^b$ or all the agents of $\cthree$ are in $\cthree^f$. If the former holds, we show $\cthree^s$ is non-empty and assume $\agent_i$ is a winner of $\cthree^s$. We bound the total value of $\agent_i$ for all the items dedicated to other agents and show the value of the remaining items in $\fitems$ is at least $\epsilon_i$ for $\agent_i$. This shows set $\fitems$ is feasible for $\agent_i$ and contradicts the termination of the algorithm. In case all agents of $\cthree$ are in $\cthree^f$, let $\agent_i$ be an arbitrary agent of $\cthree^f$. With a similar argument we show that the value of $\agent_i$ for the remaining unassigned items is at least $3/4$ and conclude that $\fitems$ is feasible for $\agent_i$ which again contradicts the termination of the algorithm.

Next, we prove a similar statement for $\cone$. 
\begin{lemma}
	\label{c1null}
	At the end of the algorithm we have $\cone = \emptyset$.
\end{lemma}
Proof of Lemma \ref{c1null} follows from a coloring argument. Let $\agent_i$ be a winner of $\cone$. We color all items in either blue or white. Roughly speaking, blue items are in a sense \textit{heavy}, i.e., they may have a high valuation to $\agent_i$ whereas white items are somewhat \textit{lighter} and have a low valuation to $\agent_i$. Next, via a double counting argument, we show that $\agent_i$'s value for the items of $\fitems$ is at least $\epsilon_i$ and thus $\fitems$ is feasible for $\agent_i$. This contradicts $\cone = \emptyset$ and shows at the end of the algorithm all agents of $\cone$ are satisfied.

Finally, we show that all the agents in Cluster $\ctwo$ are satisfied by the algorithm.
\begin{lemma}
	\label{c2null}
	At the end of the algorithm we have $\ctwo = \emptyset$.
\end{lemma}
The proof of Lemma \ref{c2null} is a similar to both proofs of Lemmas \ref{c3null} and \ref{c1null}. Let $\agent_i$ be winner of Cluster $\ctwo$. We consider two cases separately. (i) $\epsilon_i \geq 1/8$ and (ii) $\epsilon_i < 1/8$.
In case $\epsilon_i \geq 1/8$, we use a similar argument to the proof of Lemma \ref{c3null} and show $\fitems$ is feasible for $\agent_i$. If $\epsilon_i < 1/8$ we again use a coloring argument, but this time we color the items with 4 different colors. Again, via a double counting argument we show $\fitems$ is feasible for $\agent_i$ and hence every agent of $\ctwo$ is satisfied when the algorithm terminates. 
\begin{theorem}
	\label{34main}
	All the agents are satisfied before the termination of the algorithm.
\end{theorem}
\begin{proof}
	By Lemmas \ref{c3null}, \ref{c1null}, and \ref{c2null}, at the end of the algorithm all agents are satisfied which means each has received a subset of items which is worth at least $3/4$ to him.
\end{proof}

\subsection{Algorithm}\label{additive:algorithm}
In this section, we present a polynomial time algorithm to find a $(3/4-\epsilon)$-$\MMS$ allocation in the additive setting. More precisely, we show that our method for proving the existence of a $3/4$-$\MMS$ allocation can be used to find such an allocation in polynomial time. 
Recall that our algorithm consists of two main phases: The clustering phase and the $\bagfilling$ phase. In Sections \ref{algcluster} and \ref{sphase} we separately explain how to implement each phase of the algorithm in polynomial time. Given this, there are still a few computational issues that need to be resolved. First, in the existential proof, we assume $\MMS_i = 1$ for every agent $\agent_i \in \agents$.  Second, we assume that the problem is $3/4$-irreducible. Both of these assumptions are without loss of generality for the existential proof due to Observation \ref{reducibility} and the fact that one can scale the valuation functions to ensure $\MMS_i =1$ for every agent $\agent_i$. However, the computational aspect of the problem will be affected by these assumptions.  
The first issue can be alleviated by incurring an additional $1+\epsilon$ factor to the approximation guarantee. \epsteinefficient ~\cite{epstein2014efficient} show that for a given additive function $f$, $\MMS_f^n$ can be approximated within a factor $1+\epsilon$ for constant $\epsilon$ in time $\poly(n)$. Thus, we can scale the valuation functions to ensure $\MMS_i = 1$ while losing a factor of at most $1+\epsilon$. Therefore, finding a $(3/4-\epsilon)$-$\MMS$ allocation can be done in polynomial time if the problem is $3/4$-irreducible. Finally, in Section \ref{irre} we show how to reduce the $3/4$-reducible instances and extend the algorithm to all instances of the problem. The algorithm along with the reduction yields Theorem \ref{addpoly}

\begin{theorem}
	\label{addpoly}
	For any $\epsilon > 0$, there exists an algorithm that finds a $(3/4 - \epsilon)$-$\MMS$ allocation in polynomial time. 
\end{theorem}

\begin{comment}
\subsubsection{Computing the value of $\MMS_i$ in polynomial time}\label{mmsi}
As described in the beginning of Section \ref{additive}, finding the exact value of $\MMS_i$ for an agent is $NP-hard$ and there is no polynomial time algorithm for this problem, unless $P=NP$. In \cite{epstein2014efficient}, this problem is studied in the context of job scheduling. In addition to the hardness proof, they proposed a PTAS for finding a $(1+\epsilon)$ approximation of $\MMS_i$ which yields to a ploynomial time algorithm for finding $\MMS_i$, for constant $\epsilon$.  

Thus, in the beginning of the algorithm, we can compute $\MMS_i$ for every agent $\agent_i$. Considering the fact the the problem of finding $\MMS_i$ is a maximization problem, we know that the value obtained for $\MMS_i$ by method in \cite{epstein2014efficient}, is at least $\MMS_i (1 - \epsilon)$. Considering this value as $\MMS_i$ has no effect on the correctness of the algorithm except that the final result would be a $(3/4-\epsilon)$-$\MMS$ allocation. Thus, assuming that the rest of the algorithm can be implemented in polynomial time, we have a polynomial time $(3/4-\epsilon)$-$\MMS$ allocation algorithm for contant $\epsilon$. 
\end{comment}
\subsubsection{The Clustering Phase}\label{algcluster}
Recall that in the clustering phase we cluster the agents into three sets $\cone$,$\ctwo$, and $\cthree$. In order to build Cluster $\cone$, we find an $\MCMWM$ of the $1/2$-filtering of the value graph. This can be trivially done in polynomial time since finding an $\MCMWM$ is polynomially tractable~\cite{west2001introduction}. However, the refinement phase of Cluster $\cone$ requires finding $F_G(\itemsv,M)$ for a giving graph $G$ and a matching $M$. In what follows, we show this problem can also be solved in polynomial time.

\begin{comment}
First, note that an $\MCMWM$ of $G$ can be found in polynomial time using standard methods for finding minimum cost maxmimum flow in networks. For this, create a networks as follows: Orient every edge $(\itemv_j,\agentv_i)$ from $\agentv_i$ to $\itemv_j$, with cost $-w(\itemv_j,\agentv_i)$ and capacity $1$. Also, add a source node $s$ and connect it to all the vertices in $\parttwo$ with cost $0$ and capacity $1$ and add a sink node $t$ and connect every vertex in $\partone$ to $t$, with cost $0$ and capacity $1$. It is easy to observe that the edges between $\partone$ and $\parttwo$ with non-zero flow in a Min Cost Maximum Flow from $s$ to $t$ in this network form a maximum cardinality matching $M$. In addition to this, since the maximum flow was minimum cost, 
$$\sum_{(\itemv_j,\agentv_i) \in M} -w(\itemv_j,\agentv_i)$$
is minimized, which means 
$$\sum_{(\itemv_j,\agentv_i) \in M} w(\itemv_j,\agentv_i)$$
is maximized in $M$. So, $M$ is a $\MCMWM$ of $H$. 
\end{comment}

Notice that finding an $\MCMWM$ of $G$ can be done in polynomial time~\cite{west2001introduction}. Therefore, in order to determine $F_H(M,\partone)$, it only suffices to find the vertices of $\partone$ that are reachable from the unmatched vertices of $\parttwo$ by an alternating path. Let $\hat{X}$ be the set of these vertices. We can find $\hat{X}$ using a depth-first-search from the unmatched vertices of $\parttwo$. By definition, $F_H(M,\partone) = \parttwo \setminus \hat{X}$. Therefore, $F_H(M,\partone)$ can be found in polynomial time.

In addition to $F_G(\itemsv,M)$, we also need to find a matching of the graph which satisfies the conditions of Lemma \ref{nicematch}. We show in the following that this problem also can be solved in polynomial time. First, note that in Lemma \ref{v1size} we prove that $G_1$ has a matching that saturates all the vertices of $W_1$. Now, let $p_{\agent_k}$ be the position of $\agent_k$ in the topological ordering of $\cone$, as described in the proof of Lemma \ref{nicematch}. Furthermore, Let $M_1$ be a matching that minimizes the following expression.
$$ \sum_{(x_j,y_i) \in M_1} p_i.$$ Recall that in the proof Lemma \ref{nicematch}, we show that $M_1$ satisfies the condition described in Lemma \ref{nicematch}. Here, we show that $M_1$ can be found in polynomial time. To this end, we model this with a network design problem. 

Orient every edge $(x_j,y_i) \in G_1$ from $y_i$ to $x_j$ and set the cost of this edge to $p_{a_i}$. Also, add a source node $s$ and add a directed edge from $s$  to every vertex of $V_{\cone}$ with cost $0$. Furthermore, add a sink node $t$ and add directed edges from the vertices of $W_1$ to $t$ with cost $0$. Finally, set the capacity of all edges to $1$. 

One can observe that in a minimum cost maximum flow from $s$ to $t$ in this network, the edges with non-zero flow between $V_{\cone}$ and $W_1$ form a maximum matching $M_1$. In addition to this, since the cost of the flow is minimal, $\sum_{(x_j,y_i) \in M_1} cost(x_j,y_i)$ is minimized. Therefore, in this matching, 
$\sum_{(x_j,y_i) \in M_1} p_i$
is minimized. Thus, the matching with desired properties of Lemma \ref{nicematch} can be found in polynomial time.

The same algorithms can be used to compute Cluster $\ctwo$. Finally, we put the rest of the agents in Cluster $\cthree$.


\subsubsection{The $\bagfilling$ Phase}\label{sphase}
In each round of the second phase, we iteratively find a minimal feasible subset of $\fitems$ and allocate its items to the agent with the lowest priority in $\Phi(S)$.  Note that for a feasible set $S$, one can trivially find the agent with lowest priority in $\Phi(S)$ in polynomial time. Thus, it only remains to show that we can find a minimal feasible subset of $\fitems$ in polynomial time. 

Consider the following algorithm, namely \emph{reverse $\bagfilling$ algorithm}: Start with a bag containing all the items of $\fitems$ and so long as there exists an item $\ite_j$ in the bag such that after removing $\ite_j$, the set of items in the bag is still feasible, remove $\ite_j$ from the bag. After this process, the remaining items in the bag  form  a minimally feasible subset of $\fitems$. Therefore, this phase can be run in polynomial time.

\subsubsection{Reducibility}\label{irre}
The most challenging part of our algorithm is dealing with the $3/4$-irreducibility assumption. The catch is that, in order to run the algorithm, we don't necessarily need the $3/4$-irreducibility assumption. Recall that we leverage the following three consequences of irreducibility to prove the existential theorem.
\begin{itemize}
	\item The value of every item in $\items$ is less that $3/4$ to every agent.
	\item Every pair of items in $\itemsv'' \setminus \itemsv''_{1/2}$ is in total worth less than $3/4$ to any agent.
	\item The condition of Lemma \ref{v1size} holds.
\end{itemize}
 Therefore, the algorithm works so long as the mentioned conditions hold. Note that, although it is not clear whether determining if an instance of the problem is $3/4$-reducible is polynomially tractable, all of the above conditions can be validated in polynomial time. This is trivial for the first two conditions; we iterate over all items or pairs of items and check if the condition holds for these items. The last condition, however, is harder to validate.
 %Therefore, it only suffices to show that we can operate in a way that our problem preserves these three conditions.

%For the first condition, consider the following process: While there exists an agent $\agent_i$ and an item $\ite_j$ with $\valu_i(\{\ite_j\})\geq 3/4$, assign $\ite_j$ to $\agent_i$ and solve the problem recursively for the rest of the agents and items. Trivially, this process can be implemented in polynomial time. Furthermore, after this process, all the items are worth less than $3/4$ to any agent and the first condition holds. 

The condition of Lemma \ref{v1size} holds if for all $S \subseteq W_1$, $|N(S)| > |S|$. Recall that in the proof of Lemma \ref{v1size} we showed that if this condition does not hold, then $F_{G_1}(M,\itemsv)$ is non-empty. Next, we showed that if $F_{G_1}(M,\itemsv)$ is non-empty, then we can reduce the problem via satisfying every agents of $F_{G_1}(M,\itemsv)$ by his matched item in $M$. Therefore, on the computational side, we only need to find whether $F_{G_1}(M,\itemsv)$ is empty which indeed can be determined in polynomial time. 
%In this case, we operate as follows: %
%If the condition in Lemma \ref{v1size} does not hold, we reduce the problem by satisfying the agents in $F_{G_1}(M,\itemsv))$. After this, regarding Lemma \ref{iff}, the condition in Lemma \ref{v1size} holds. As described in section \ref{cphase}, $F_{G_1}(M,\itemsv))$ can be found in polynomial time.

%For the third condition, we can operate in the same way as the first condition: While there exists an agent $\agent_i$ and a pair of items $\ite_j,\ite_k$ such that   $\valu_i(\{\ite_j,\ite_k\})\geq 3/4$, reduce the problem by assigning $\ite_j,\ite_k$ to $\agent_i$ and removing them from $\items$ and $\agents$, respectively. 

Note that every time we reduce the problem, $|\agents|$ is decreased by at least $1$, which implies the number of times we reduce the problem is no more than $n$. Moreover, our reduction takes a polynomial time. Thus, the running time of the algorithm is polynomial. %Thus, in polynomial time, you can either satisfty all the agents, or obtain the problem instance that preserves all three conditions.
