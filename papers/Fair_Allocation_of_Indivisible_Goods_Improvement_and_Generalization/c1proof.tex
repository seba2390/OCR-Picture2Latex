
\begin{proof}[of Lemma \ref{c1null}]
By Lemma \ref{c3null}, we already know $\cthree = \emptyset$. Now, let $\agent_i$ be a winner of the remaining agents in $\cone$. For convenience, we color the items in either blue or white. Intuitively, blue items may have a high value for $\agent_i$ whereas white items are always of lower value to $\agent_i$. Initially, all items are colored in white. For each $\agent_j  \in \agents$, if $|\firstset_j|=1$, then we color the item in $\firstset_j$ in blue. Moreover, for every $\agent_j \in \satagents$, if $|\secondset_j|=1$ and $\valu_i(\secondset_j) \geq \epsilon_i$, then we color the item in $\secondset_j$ in blue. 


Now, let $\cal P$ $= \langle P_1, P_2, \ldots, P_n \rangle$ be the optimal $n$-partitioning of the items in $\items$ for $\agent_i$, that is, the value of every partition $P_k$ to $\agent_i$ is at least $1$. Based on the coloring procedure, we have three types of partitions in $\cal P$:
\begin{itemize}
    \item $B_2$: the set of partitions with at least two blue items
    \item $B_1$: the set of partitions with exactly one blue item
    \item $B_0$: the set of partitions without any blue items
\end{itemize}
Note that every partition in $\cal P$ belongs to one of $B_0,B_1$ or $B_2$. Hence,
\begin{equation}
\label{sumba}
|B_0| + |B_1| + |B_2| = n
\end{equation}
As declared, all the items in the partitions of $B_0$ are white. The total value of these items to $\agent_i$ is at least $|B_0| \geq 4 \epsilon_i |B_0|$, which is
\begin{equation}\label{pq1}
\sum_{P_k \in B_0}\sum_{\ite_j \in P_k} \valu_i(\ite_j) \geq 4 \epsilon_i |B_0|.
\end{equation}
 Also, each partition in $B_2$ has at least two blue items, each of which is singly assigned to another agent. We decompose the partitions of $B_1$ into two disjoint sets, namely $\hat{B_1}$ and $\tilde{B_1}$. More precisely, let $\hat{B_1}$ be the partitions in $B_1$, in which the blue item is worth more than $\valu_i(\firstset_i)$ to $\agent_i$ and $\tilde{B_1} = B_1 \setminus \hat{B_1}$. As such, for each partition $P_k \in \tilde{B_1}$, 
the white items in $P_k$ are worth at least 
\begin{equation*}
\begin{split}
1- \valu_i(\firstset_i) &= 1-(3/4-\epsilon_i)\\
&= {1/4}+ \epsilon_i\\
&\geq 2\epsilon_i
\end{split}
\end{equation*}
to $\agent_i$. Therefore,
\begin{equation}\label{pq2}
\sum_{P_k \in \tilde{B_1}}  \valu_i(\wcal(P_k)) \geq 2 |\tilde{B_1}|\epsilon_i
\end{equation}
where $\wcal(S)$ stands for the set of white items in a set $S$ of items.
 On the other hand, since the problem is $3/4$-irreducible, by Lemma \ref{remove1}, no item alone is of worth $3/4$ to $\agent_i$ and thus for each partition $P_k \in \hat{B_1}$, the white items in $P_k$ have a value of at least ${1/4} \geq \epsilon_i$ to $\agent_i$. This implies that
\begin{equation}\label{pq4}
\sum_{P_k \in \hat{B_1}} \valu_i(\wcal(P_k)) \geq |\hat{B_1}|  \epsilon_i.
\end{equation}
By Inequalities \eqref{sumba},\eqref{pq1}, \eqref{pq2}, and \eqref{pq4} we have
\begin{equation}\label{enough}
\begin{split}
\valu_i(\wcal(\items)) &= \sum_{P_j \in B_0} \valu_i(\wcal(P_j)) + \sum_{P_j \in B_1} \valu_i(\wcal(P_j)) + \sum_{P_j \in B_2} \valu_i(\wcal(P_j)) \\
&\geq \sum_{P_j \in B_0} \valu_i(\wcal(P_j)) + \sum_{P_j \in B_1} \valu_i(\wcal(P_j)) \\
&\geq \sum_{P_j \in B_0} \valu_i(\wcal(P_j)) + \sum_{P_j \in \hat{B_1}} \valu_i(\wcal(P_j)) + \sum_{P_j \in \tilde{B_1}} \valu_i(\wcal(P_j)) \\
&\geq |B_0|  4\epsilon_i + |\hat{B_1}|  \epsilon_i + |\tilde{B_1}|  2\epsilon_i \\
&\geq |B_0|  4\epsilon_i + |B_1| 2\epsilon_i - |\hat{B_1}|  \epsilon_i  \\ 
&\geq |B_0|  4\epsilon_i + |B_1| 4\epsilon_i + |B_2|4\epsilon_i - |B_1| 2\epsilon_i - |B_2|4\epsilon_i- |\hat{B_1}|  \epsilon_i \\
&= (2n-2|B_2| - |B_1| -  |\hat{B_1}|)2\epsilon_i + (|\hat{B_1}|)\epsilon_i
\end{split}
\end{equation}
%Since every element in $\fitems$ is colored in white, we have
%\begin{equation}
%\valu_i(\wcal(\items \setminus \fitems)) = \sum_{\agent_j \in \agents}\valu_i(\firstset_j \cup \secondset_j) - \valu_i(\items \setminus \wcal(\items))
%\end{equation}
%Note that for every agents $\agent_j \in \agents \setminus \satagents$, $g_j = \emptyset$.
Note that the total value of white items that are assigned to the agents during the algorithm is equal to $\valu_i(\wcal(\items \setminus \fitems))$. The rest of the white items are still in $\fitems$. Thus, we have 
\begin{equation}
\label{fsum}
\valu_i(\wcal(\items)) = \valu_i(\wcal(\items \setminus \fitems)) + \valu_i(\fitems)
\end{equation}
Now, we provide an upper bound on the value of $\valu_i(\wcal(\items \setminus \fitems))$. As a warm up, one can trivially prove an upper bound of $2\epsilon_i (2n - 1 - |B_1| - 2|B_2|)$ on $\valu_i(\wcal(\items \setminus \fitems))$. This follows from the fact that two sets of items are assigned to any agent and hence we have a total of $2n$ disjoint sets. Among these $2n$ sets, at least one of them is empty (since $\secondset_i = \emptyset$) and at least $|B_1| + 2|B_2|$ of the sets contain a single blue item. On the other hand, by Lemmas \ref{c1small2}, \ref{cr2smallc1}, \ref{c3fsmall} and \ref{prvalue} every set with white items is of worth at most $2\epsilon_i$ to $\agent_i$. Therefore, the total value of the white items in $\items \setminus \fitems$ to $\agent_i$ is less than $2\epsilon_i (2n - 1 - |B_1| - 2|B_2|)$ and thus $$\valu_i(\wcal(\items \setminus \fitems)) \leq 2\epsilon_i (2n - 1 - |B_1| - 2|B_2|).$$ 

However, in order to complete the proof, we need a stronger upper bound on $\valu_i(\wcal(\items \setminus \fitems))$. To this end, we provide the following auxiliary lemma.
 
\begin{lemma}
\label{eps}
Let $\agent_j$ be an agent such that $|\firstset_j| = 1$ and $\valu_i(\firstset_j) > \valu_i(\firstset_i)$. Then, $\valu_i(\secondset_j) < \epsilon_i$.
\end{lemma} 
\begin{proof}
First, note that if $\agent_j$ is not satisfied yet, then $\secondset_j = \emptyset$ and therefore $\valu_i(\secondset_j) < \epsilon_i$. Otherwise, we argue that agent $\agent_j$ is either satisfied in the second phase, or in the refinement phases of $\cone$ or $\ctwo$.

Consider the case that $\agent_j$ is satisfied in the second phase. If $\agent_j \in \satagents_2^s \cup \satagents_3$, then by Lemma \ref{prvalue}, $\valu_i(\secondset_j) < \epsilon$ holds. Also, if $\agent_j \in \satagents_1^s$, considering the fact that $\agent_i$ envies $\agent_j$, $\agent_i \prec_{pr} \agent_j$. Thus, by Lemma \ref{prvalue}, we have $\valu_i(\secondset_j)< \epsilon_i$.

Next, consider the case that $\agent_j$ is in $\satagents_1^r \cup \satagents_2^r$. Note that the matching of the refinement phase of $\cone$ preserves the property described in Lemma \ref{nicematch}. Hence, if $\agent_j$ belongs to $\satagents_1^r$, then either $\agent_j \prec_{pr} \agent_i$ or there is no edge between $\agentv_i$ and $M_1(\agentv_j)$ in $G_1$, where $M_1(\agentv_j)$ is the vertex matched with $\agentv_j$ in $M_1$. If $\agent_j \prec_{pr} \agent_i$, according to Observation \ref{epsofcluster}, $\valu_i(\firstset_j) \leq 3/4-\epsilon_i$ holds. On the other hand, by the definition, if no edge exists between $\agentv_i$ and  $M_1(\agentv_j)$ in $G_1$, $\valu_i(\secondset_j)<\epsilon_i$. 
In addition to this, if $\agent_j$ belongs to $\satagents_2^r$, according to Lemma \ref{cr2smallc1}, $\valu_i(\secondset_j)< \epsilon_i$ holds. 
Therefore, Lemma \ref{eps} holds for the agents in $\satagents_1^r \cup \satagents_2^r$.

\end{proof}

Note that since matching $M$ of $G_{1/2}$ for building Cluster $\cone$ is $\MCMWM$ according to condition (iii) of Lemma \ref{wm}, there exists no agent $\agent_k$, such that $|\secondset_k| = 1$ and $\valu_i(\secondset_k) > 3/4 - \epsilon_i$. Otherwise, by assigning the item in $\secondset_j$ to $\agent_i$ instead of the item in $\firstset_i$, we can increase the total weight of the matching, that contradicts the maximality of $M$.

According to Lemma \ref{eps}, for all the agents $\agent_j$ with the property that $f_j$ is a blue item that belongs to a partition in $\hat{B_1}$, $\valu_i(\secondset_j)< \epsilon_i$ holds. 
The number of such agents is at least $|\hat{B_1}|$. Therefore, the total value of $\valu_i(\wcal(\items \setminus \fitems))$ is less than $2\epsilon_i \cdot (2n - 1 - |B_1| - 2|B_2|- |\hat{B_1}|) + \epsilon_i \cdot |\hat{B_1}|$. Combining the bounds obtained in Observation \ref{enough} and Lemma \ref{eps} by Inequality \eqref{fsum}, we have:
$$ \valu_i({\fitems}) \geq 2\epsilon_i \cdot(2n -2|B_2| - |B_1| -  |\hat{B_1}|) + \epsilon_i \cdot(|\hat{B_1}|) -  2\epsilon_i \cdot (2n - 1 - |B_1| - 2|B_2|- |\hat{B_1}|) - \epsilon_i \cdot |\hat{B_1}|$$
That is:
$$ \valu_i({\fitems}) \geq 2\epsilon_i$$

This contradicts the fact that the set $\fitems$ is not feasible for $\agent_i$.


\begin{comment}
Now, we observe the algorithm, from the viewpoint of $\agent_i$. For each agent $\agent_j$, consider $\firstset_j$ and $\secondset_j$ as two slots that must be filled by proper set of items during the algorithm. In the viewpoint of $\agent_i$, during the algorithm, each slot is filled, either with a single blue item, or a set of white items with value at most $\epsilon$ for $\agent_i$ ( if the set $\secondset_j$ is not feasible for $\agent_i$ ) or with a set of items that is worth at most $2\epsilon_i$ for $\agent_i$ ( if the set $S$ was feasible for $\agent_i$ in the second phase, or the slots was filled in the clustering phase by merging two vertices). 

In the deadlock, we know that at least $2|B_2| + |B_1|$ of the slots is filled by blue items. other $r = 2n - 2|B_2| - |B_1|$ slots must be filled with white items. By lemma\ref{eps}, at most $r - |\tilde{B_1}|$ of the slots must be filled with items that are worth at most $2\epsilon$. Other $|B_1|$ slots must be filled with items that are worth at most $\epsilon$ for $\agent_i$. Regarding Observation \ref{enough}, value of the white items ($W$) is enough for filling all  the slots. So, in the deadlock situation, no slot can be empty. But while $\agent_i$ is semi-satisfied, the slot $\secondset_i$ is empty and hence, $\valu_i({\fitems}) \geq 2\epsilon$ that contradicts by the deadlock situation.
\end{comment} 

\end{proof}
