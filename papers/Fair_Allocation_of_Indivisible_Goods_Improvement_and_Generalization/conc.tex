\section{Conclusion}
We have studied the problem of allocating items to the players, and tried to guarantee an approximation of maxmin-share for each player. We gave a $\frac{4}{5}$-MMS allocation algorithm for $n=4$. One future work would be improving this approximations for general $n$ or special cases like $n=3,4$ , or giving an upper bound on the maximum possible value of $\alpha$ for any $\alpha$-MMS allocation. 

For general $n$, we provided a new $\frac{2}{3}$-MMS algorithm with less dependency to the definition of MMS and proved that regarding the \emph{weak linearity} condition, the algorithm can be turn into a $\frac{2}{3}$ proportional allocation. An extension here is to exactly determine how likely is it for an instance of a problem to satisfy \emph{weak linearity}, e.g. when the value of items are chosen uniformly from $[0,1]$. With the same deduction as Lemma 1 for proportionality, we can assume that all the items have values less than $\frac{2}{3}$. Regarding this fact, we think it is common for a set of items to preserve \emph{weak linearity} condition.

Another direction is extending the algorithm for more general valuations functions such as submodular valuations. In that case, using the queries mentioned in Section 3 to implement the algorithm would be much more convincing.

Finally, generalizing our method for classification of the players and items may help to give better approximation ratio than $\frac{2}{3}$. 