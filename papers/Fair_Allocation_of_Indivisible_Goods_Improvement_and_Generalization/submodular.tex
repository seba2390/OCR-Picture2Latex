\section{Submodular Agents}\label{submodular}
Previous work on the fair allocation problem was limited to the additive agents \cite{amanatidis2015approximation,Procaccia:first}. In real-world, however, valuation functions are usually more complex than additive ones. As an example, imagine an agent is interested in at most $k$ items. More precisely, he is indifferent between receiving $k$ items or more than $k$ items. Such a valuation function is called $k$-demand and cannot be modeled by additive functions. $k$-demand functions are a subclass of submodular set functions which have been extensively studied the literature of different contexts, e.g., optimization, mechanism design, and game theory \cite{buchbinder2012tight,buchbinder2015tight, fujishige2005submodular,gupta2010constrained,kim2011distributed,krause2010sfo,lee2009non,minoux1978accelerated,vondrak2008optimal}. 

 In this section, we study the fair  allocation problem where the valuations of agents are submodular. We begin by presenting an impossibility result; We show in Section \ref{submodular-upperbound} that the best guarantee that we can achieve for submodular agents is upper bounded by $3/4$. Next, we give a proof to the existence of a $1/3$-$\MMS$ allocation in this setting. This is followed by an algorithm that finds such an allocation in polynomial time. This is surprising since even finding the $\MMS$ of a submodular function is NP-hard and cannot be implemented in polynomial time unless P=NP ~\cite{epstein2014efficient}. In our algorithm, we assume we have access to query oracle for the valuation of agents; That is, for any set $S$ and any agent $\agent_i$, $\valu_i(S)$ can be computed via a given query oracle in time $O(1)$. 
 
\subsection{Upper Bound}\label{submodular-upperbound}
We begin by providing an upper bound. In this section, we show for some instances of the problem with submodular agents, no allocation can be better than $3/4$-$\MMS$. Our counter-example is generic; We show this result for any number of agents.
\begin{theorem}\label{subupperbound}
	For any $n \geq 2$, there exists an instance of the fair allocation problem with $n$ submodular agents where no allocation is better than $3/4$-$\MMS$.
\end{theorem}
\begin{proof}
	We construct an instance of the problem that does not admit any $3/4+\epsilon$-$\MMS$ allocation. To this end, let $n$ be the number of agents and $\items = \{\ite_1, \ite_2, \ldots,\ite_m\}$ where $m = 2n$. Furthermore, let $f:2^{\items} \rightarrow \mathbb{R}$ be as follows:
	
	$$f(S) =
	\begin{cases}
	0, & \text{if }|S| = \emptyset \\
	1, & \text{if }|S| = 1  \\
	2, & \text{if }|S| > 2 \\
	2, & \text{if }S = \{\ite_{2i}, \ite_{2i+1} \} \text{ for some }i \\
	3/2, & \text{if }|S| = 2 \text{ and }S \neq \{\ite_{2i}, \ite_{2i+1} \} \text{ for any }i. \\
	\end{cases}$$
	
	Notice that $\MMS_f^n = 2$. Moreover, in what follows we show that $f$ is submodular. To this end, suppose for the sake of contradiction that there exist sets $S$ and $S'$ such that $S \subseteq S'$ and for some element $\ite_i$ we have:
	\begin{equation}
	f(S' \cup \{\ite_i\}) - f(S') > f(S \cup \{\ite_i\}) - f(S). \label{gharch}
	\end{equation}
	Since $f$ is monotone and $S' \neq S$, $f(S' \cup \{\ite_i\}) - f(S') > 0$ holds and thus $S'$ cannot have more than two items. Therefore, $S'$ contains at most two items and thus $S$ is either empty or contains a single element. If $S$ is empty, then adding every element to $S$ has the highest increase in the value of $S$ and thus Inequality \eqref{gharch} doesn't hold. Therefore, $S$ contains a single element and $S'$ contains exactly two elements. Thus, $f(S) = 1$ and $f(S') \geq 3/2$. Therefore, f(S $\cup \{\ite_i\}) - f(S) \geq 1/2$ and $f(S' \cup \{\ite_i\}) - f(S') \leq 1/2$ which contradicts Inequality \eqref{gharch}.
	
	 Now, for agents $\agent_1, \agent_2, \ldots, \agent_{n-1}$ we set $\valu_i = f$ and for agent $\agent_n$ we set $\valu_n = f(\inc(S))$ where $\ite_i$ is in $\inc(S)$ if and only if either $i > 1$ and $\ite_{i-1} \in S$ or $i=1$ and $\ite_m \in S$. 
	
	The crux of the argument is that for any allocation of the items to the agents, someone receives a value of at most $3/2$. In case an agent receives fewer than two items, his valuation for his items would be at most $1$. Similarly, if an agent receives more than two items, someone has to receive fewer than $2$ items and the proof is complete. Therefore, the only case to investigate is where everybody receives exactly two items. We show in such cases, $\min \valu_i(A_i) = 3/2$ for all possible allocations. If all agents $\agent_1, \agent_2, \ldots, \agent_{n-1}$ receive two items whose value for them is exactly equal to $2$, then by the construction of $f$, the value of the remaining items is also equal to $2$ to them. Thus, $\agent_n$'s valuation for the items he receives is equal to $3/2$.
\end{proof}
 
Remark that one could replace function $f$ with an XOS function 
	$$g(S) =
	\begin{cases}
	0, & \text{if }|S| = \emptyset \\
	1, & \text{if }|S| = 1  \\
	2, & \text{if }|S| > 2 \\
	2, & \text{if }S = \{\ite_{2i}, \ite_{2i+1} \} \text{ for some }i \\
	1, & \text{if }|S| = 2 \text{ and }S \neq \{\ite_{2i}, \ite_{2i+1} \} \text{ for any }i. \\
	\end{cases}$$
and make the same argument to achieve a $1/2$-$\MMS$ upper bound for XOS and subadditive agents.

\begin{theorem}\label{xosupperbound}
	For any $n > 1$, there exists an instance of the fair allocation problem with $n$ XOS agents where no allocation is better than $1/2$-$\MMS$.
\end{theorem}

\subsection{Existential Proof}\label{submodularep}
In this section we provide an existential proof to a $1/3$-$\MMS$ allocation. Due to the algorithmic nature of the proof, we show in Section \ref{submodularalgorithm} that such an allocation can be computed in time $\mathsf{poly}(n,m)$. For simplicity, we scale the valuation functions to ensure $\MMS_i = 1$ for every agent $\agent_i$.

We begin by introducing the ceiling functions.
\begin{definition}\label{fxfunction}
	Given a set function $f(.)$, we define $\ceil{f}{x}(.)$ as follows:
	
	$$\ceil{f}{x}(S) =
	\begin{cases}
	f(S), & \text{if }f(S) \leq x \\
	x, & \text{if }f(S) > x.
	\end{cases}$$
\end{definition}
A nice property of the ceiling functions is that they preserve submodularity, fractionally subadditivity, and sub-additivity as we show in Appendix \ref{submodular-appendix}.

\begin{lemma}\label{ceilingfunctions}
	For any real number $x \geq 0$, we have:
	\begin{enumerate}
		\item Given a submodular set function $f(.)$, $f^x(.)$ is submodular.
		\item Given an XOS set function $f(.)$, $f^x(.)$ is XOS.
		\item Given an subadditive set function $f(.)$, $f^x(.)$ is also subadditive.
	\end{enumerate}
\end{lemma}

The idea behind the existence of a $1/3$-$\MMS$ allocation is simple: Suppose the problem is $1/3$-irreducible and let  $\mathcal{A} = \langle A_1, A_2, \ldots, A_n\rangle$ be an allocation of items to the agents that maximizes the following expression:
\begin{equation}\label{exex}
\sum_{\agent_i \in \agents} \ceil{\valu_i}{2/3}(A_i)
\end{equation}
We refer to Expression \eqref{exex} by $\mathsf{ex}^{(2/3)}(\mathcal{A})$. We prove $\valu_i(A_i) \geq 1/3$ for every agent $\agent_i \in \agents$. By the reducibility principal, it only suffices to show every $1/3$-irreducible instance of the problem admits a $1/3$-$\MMS$ allocation. The main ingredients of the proof are Lemmas \ref{remove1}, \ref{submodularaval} and \ref{submodulardovom}. For brevity we skip the proofs and include them in Appendix \ref{submodular-appendix}.

\begin{comment}
\begin{lemma} \label{submodularsefr}
	In a $1/3$-irreducible instance of the problem for every item $\ite_j$ and every agent $\agent_i$ we have
	$$\valu_i(\{\ite_j\}) < 1/3.$$
\end{lemma}

The proof of Lemma \ref{submodularsefr} follows from the definition of reducibility. We show if for an agent $\agent_i$ and an item $\ite_j$, $\valu_i(\{\ite_j\}) \geq 1/3$ holds then we can assign $\ite_j$ to $\agent_i$ and reduce the problem. This contradicts with the definition of reducibility.
\end{comment}
\begin{lemma}\label{submodularaval}
	Let $S_1, S_2, \ldots, S_k$ be $k$ disjoint sets and $f_1, f_2, \ldots, f_k$ be $k$ submodular functions. We remove an element $e$ from $\bigcup S_i$ uniformly at random to obtain sets $S^*_1 = S_1 \setminus \{e\}, S^*_2 = S_2 \setminus \{e\}, \ldots, S^*_k = S_k \setminus \{e\}$. In this case we have
	$$\mathbb{E}[\sum f_i(S^*_i)] \geq \sum f_i(S_i)\frac{|\bigcup S_i| -1}{|\bigcup S_i|}.$$
\end{lemma}

The high-level intuition behind the proof of Lemma \ref{submodularaval} is as follows: For submodular functions, the smaller the size of a set is, the higher the marginal values for adding items to that set will be. Based on that, we show the summation of marginal decreases for removing each element is bounded by the total value of the set and that completes the proof. A complete proof is included in Appendix \ref{submodular-appendix}. 

\begin{lemma}\label{submodulardovom}
	Let $f$ be a submodular function and $S_1, S_2, \ldots, S_k$ be $k$ disjoint sets such that $f(S_i) \geq 1$ for every set $S_i$. Moreover, let $S \subseteq \bigcup S_i$ be a set such that $f(S) < 1/3$. If we pick an element $\{e\}$ of $\bigcup S_i \setminus S$ uniformly at random, we have:
	$$\mathbb{E}[f(S \cup \{e\}) - f(S)] \geq \frac{2k/3}{|\bigcup S_i \setminus S|}.$$ 
\end{lemma}
The proof of Lemma \ref{submodulardovom} is very similar to that of Lemma \ref{submodularaval}. The main point is that in submodular functions, the marginal increase decreases as the sizes of sets grow.

Next, we show the fair allocation problem with submodular agents admits a $1/3$-$\MMS$ allocation\footnote{Almost one year after the first draft of our work, the existense of a $1/10$-$\MMS$ allocation in the submodular case along with an algorithm to find a $1/31$ approximation algorithm for the submodular case is also proved in ~\cite{barman2017approximation}. They also study the problem in the additive setting and present another $2/3$-$\MMS$ algorithm. This work is completely parallel to and independent of our paper. Moreover, their analysis is fundamentally different from our analysis and also their bounds are looser.}.
\begin{theorem}\label{submodulartheorem}
	The fair allocation problem with submodular agents admits a $1/3$-$\MMS$ allocation.
\end{theorem}
\begin{proof}
	By Lemma \ref{reducibility}, the problem boils down to the case of $1/3$-irreducible instances. Let the problem be $1/3$-irreducible and $\mathcal{A}$ be an allocation that maximizes $\mathsf{ex}^{(2/3)}$. Suppose for the sake of contradiction that $\valu_i(A_i) < 1/3$ for some agent $\agent_i$.
	In this case we select an item $\ite_r$ from $\items \setminus A_i$ uniformly at random to create a new allocation $\mathcal{A}^r$ as follows:
	 
	 $$A^r_j =
	 \begin{cases}
	 A_j \setminus \{\ite_r\}, & \text{if }i \neq j \\
	 A_j \cup \{\ite_r\} & \text{if }i = j.
	 \end{cases}$$
	
	In the rest we show $\mathbb{E}[\mathsf{ex}^{(2/3)}(\mathcal{A}^r)] > \mathsf{ex}^{(2/3)}(\mathcal{A})$ which contradicts the maximality of $\mathcal{A}$. 
	Note that by Lemma \ref{submodularaval} the following inequality holds: 
	\begin{equation}\label{rex1}
	\mathbb{E}[\sum_{j\neq i} \valu_j^{2/3}(A^r_j)] \geq \sum_{j\neq i} \valu_j^{2/3}(A_j) \frac{|\items \setminus A_i| - 1}{|\items \setminus A_i|}.
	\end{equation}
	Moreover, by Lemma \ref{submodulardovom} we have 
	\begin{equation}\label{rex2}
	\mathbb{E}[\valu_i(A^r_i) - \valu_i(A_i)] \geq \frac{2n/3}{|\items \setminus A_i|}.
	\end{equation}
	Inequality \eqref{rex1} along with Inequality \eqref{rex2} shows
	\begin{equation}
	\begin{split}
	\mathbb{E}[\mathsf{ex}^{(2/3)}(\mathcal{A}^r)] &= \mathbb{E}[\sum_{j\neq i} \valu_j^{2/3}(A^r_j)] + \mathbb{E}[\valu_i(A^r_i)]\\
	&\geq \sum_{j\neq i} \valu_j^{2/3}(A_j) \frac{|\items \setminus A_i| - 1}{|\items \setminus A_i|} + \mathbb{E}[\valu_i(A^r_i)]\\
	&\geq \sum_{j\neq i} \valu_j^{2/3}(A_j) \frac{|\items \setminus A_i| - 1}{|\items \setminus A_i|} + \frac{2n/3}{|\items \setminus A_i|} + \valu_i(A_i)\\
	&\geq \sum_{j\neq i} \valu_j^{2/3}(A_j) \frac{|\items \setminus A_i| - 1}{|\items \setminus A_i|} + \frac{2n/3}{|\items \setminus A_i|} + \valu_i^{(2/3)}(A_i)\\
	&\geq \sum_{j\neq i} \valu_j^{2/3}(A_j) \frac{|\items \setminus A_i| - 1}{|\items \setminus A_i|} + \frac{2n/3}{|\items \setminus A_i|} + \valu_i^{(2/3)}(A_i)\frac{|\items \setminus A_i| - 1}{|\items \setminus A_i|}\\
	&=  \mathsf{ex}^{(2/3)}(\mathcal{A}) \frac{|\items \setminus A_i| - 1}{|\items \setminus A_i|} + \frac{2n/3}{|\items \setminus A_i|}.\\
	\end{split}
	\end{equation}
Recall that by Lemma \ref{remove1}, the value of agent $\agent_i$ for any item alone is bounded by $1/3$ and thus $\mathbb{E}[\valu_i(A^r_i) - \valu_i(A_i)] = \mathbb{E}[\valu^{2/3}_i(A^r_i) - \valu^{2/3}_i(A_i)]$. 
Notice that by the definition, $\valu_j^{(2/3)}$ is always bounded by $2/3$ and also $\valu_i(A_i) < 1/3$, therefore, $\mathsf{ex}^{(2/3)}(\mathcal{A}) \leq 2n/3-1/3$ and thus
\begin{equation}
\begin{split}
\mathbb{E}[\mathsf{ex}^{(2/3)}(\mathcal{A}^r)] &\geq  \mathsf{ex}^{(2/3)}(\mathcal{A}) \frac{|\items \setminus A_i| - 1}{|\items \setminus A_i|} + \frac{2n/3}{|\items \setminus A_i|}\\
&\geq \mathsf{ex}^{(2/3)}(\mathcal{A}) + \frac{1/3}{|\items \setminus A_i|}\\
&\geq \mathsf{ex}^{(2/3)}(\mathcal{A}) + 1/3m.
\end{split}
\end{equation}
\end{proof}

\subsection{Algorithm}\label{submodularalgorithm}
In this section we give an algorithm to find a $1/3$-$\MMS$ allocation for submodular agents. We show our algorithm runs in time $\poly(n,m)$.

For simplicity, we assume for every agent $\agent_i$, $\MMS_i$ is given as input to the algorithm. However, computing $\MMS_i$ alone is an NP-hard problem. That said, we show in Section \ref{latter} that such a computational barrier can be lifted by a combinatorial trick. We refer the reader to Section \ref{latter} for a more detailed discussion. The procedure is illustrated in Algorithm \ref{submodularalg}:
\begin{algorithm}%[b!]
	\KwData{$\agents, \items, \langle \valu_1, \valu_2, \ldots, \valu_n\rangle, \langle \MMS_1, \MMS_2, \ldots, \MMS_n\rangle$}
	For every $\agent_j$, scale $\valu_j$ to  ensure $\MMS_j = 1$\;
	\While{there exist an agent $\agent_i$ and an item $\ite_j$ such that $\valu_i(\{\ite_j\}) \geq 1/3$}{
		Allocate $\{\ite_j\}$ to $\agent_i$\;
		$\items = \items \setminus \ite_j$\;
		$\agents = \agents \setminus \agent_i$\;
	}
	$\mathcal{A} = $ an arbitrary allocation of the items to the agents\;
	\While{$\min \valu^{2/3}_j(A_j) < 1/3$}{
		$i = $ the agent who receives the lowest value in allocation $\mathcal{A}$\; 
		Find an item $\ite_e$ such that:\label{find}
		$\mathsf{ex}(\langle A_1 \setminus \{\ite_e\}, A_2 \setminus \{\ite_e\}, \ldots, A_{i-1} \setminus \{\ite_e\}, A_i \cup \{\ite_e\}, A_{i+1} \setminus \{\ite_e\}, \ldots, A_n \setminus \{\ite_e\}\rangle) \geq \mathsf{ex}(\mathcal{A})+1/3m$\;
		$\mathcal{A} = \langle A_1 \setminus \{\ite_e\}, A_2 \setminus \{\ite_e\}, \ldots, A_{i-1} \setminus \{\ite_e\}, A_i \cup \{\ite_e\}, A_{i+1} \setminus \{\ite_e\}, \ldots, A_n \setminus \{\ite_e\}\rangle$\;
	}
	For every $\agent_i \in \agents$ allocate $A_i$ to $\agent_i$\;
	\caption{Finding a $1/3$-$\MMS$ allocation for submodular agents}
	\label{submodularalg}
\end{algorithm}
Based on Theorem \ref{submodulartheorem}, one can show that in every iteration of the algorithm value of $\mathsf{ex}^{2/3}(\mathcal{A})$ is increased by at least $1/3m$. Moreover, such an element $\ite_e$ can be easily found by iterating over all items in time $O(m)$. Furthermore, the number of iterations of the algorithm is bounded by $2nm$, since $\mathsf{ex}^{2/3}(\mathcal{A})$ is bounded by $2n/3$. Therefore, Algorithm \ref{submodularalg} finds a $1/3$-$\MMS$ allocation in time $\poly(n,m)$.

\begin{theorem}\label{subsubalg}
	Given access to query oracles, one can find a $1/3$-$\MMS$ allocation for submodular agents in polynomial time.
\end{theorem}

As a corollary of Theorem \ref{subsubalg}, one can show that the problem of finding the maxmin value of a submodular function admits a 3 approximation algorithm.

\begin{corollary}
	For a given submodular function $f$, we can in polynomial time split the elements of ground set into $n$ dijsoint sets $S_1, S_2, \ldots, S_n$ such that 
	$$f(S_i) \geq \MMS_f^n/3$$
	for every $1 \leq i \leq n$.
\end{corollary}