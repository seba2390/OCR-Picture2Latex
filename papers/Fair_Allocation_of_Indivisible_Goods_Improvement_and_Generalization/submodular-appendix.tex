\section{Omitted Proofs of Section \ref{submodular}}\label{submodular-appendix}

\begin{observation}\label{obs_E1}
$f^x(S) \leq x$ for every given $S$.
\end{observation}

\begin{observation}\label{obs_E2}
$f^x(S) \leq f(S)$ for every given $S$.
\end{observation}

\begin{proof}[Of Lemma \ref{ceilingfunctions}]
\\
\textbf{First Claim:} By definition of submodular functions, for given sets $A$ and $B$ we have:
$$ f(A \cup B) \leq f(A) + f(B) - f(A \cap B)  $$
We prove that $f^x(.)$ is a submodular function in three different cases:\\

First Case: Let both $f(A)$ and $f(B)$ be at least $x$. According to Observation \ref{obs_E1}, $f^x(A \cup B)$ and $f^x(A \cap B)$ are bounded by $x$. Therefore, $f^x(A \cup B) + f^x(A \cap B) \leq 2x$, which yields: 
$$f^x(A \cup B) + f^x(A \cap B) \leq f^x(A) + f^x(B)$$

Second Case: In this case one of $f(A)$ and $f(B)$ is at least $x$. We have $f(A \cup B) \geq x$ and $f(A \cap B)$ is no more than max $\{f(A), f(B)\}$. As a result $f^x(A \cup B)$ and one of $f^x(A)$ or $f^x(B)$ are equal to $x$ which yields:
$$f^x(A \cup B) + f^x(A \cap B) \leq f^x(A) + f^x(B)$$

Third Case: In this case both $f(A)$ and $f(B)$ are less than $x$, and $f(A \cap B)$ is less than $x$ too. Since $f^x(A) = f(A)$, $f^x(B) = f(B)$, $f^x(A \cap B) = f(A \cap B)$, according to Observation \ref{obs_E2}, $f^x(A \cup B) \leq f(A \cup B)$ holds. Since $f(.)$ is a submodular function, we conclude that:
$$f^x(A \cup B) \leq f^x(A) + f^x(B) - f^x(A \cap B).$$\\
\textbf{Second Claim:} Since $f(.)$ is an XOS set function, by definition, there exists a finite set of additive functions  $\{f_1, f_2, \ldots, f_{\alpha}\}$ such that $$f(S) = \max_{i=1}^{\alpha} f_i(S)$$ for any set $S \subseteq \domp(f)$. With that in hand, for a given real number $x$, we define an XOS set function $g(.)$, and show $g(.)$ is equal to $f^{x}(.)$.

We define $g(.)$ on the same domain as $f(.)$. Moreover, based on $\{f_1, f_2, \ldots, f_{\alpha}\}$, we define a finite set of additive functions $\{g_1, g_2, \ldots, g_{\beta}\}$ that describe $g$. More precisely, for each set $S$ in domain of $f(.)$ we define a new additive function like $g_{\gamma}$ in $g(.)$ as follows: Without loss of generality let $f_{\delta}$ be the function which maximizes $f(S)$. For each $b_i \notin S$ let $g_{\gamma}(b_i) = 0$. Furthermore, for each $b_i \in S$ if $f(S) \leq x$ let $g_{\gamma}(b_i) = f_{\delta}(b_i)$, and otherwise let $g_{\gamma}(b_i) = \frac{x}{f(S)} f_{\delta}(b_i)$. 

We claim that $g(.)$ is equivalent to $f^x(.)$, which implies $f^x(.)$ is an XOS function. $g(.)$ and $f^x(.)$ are two functions which have equal domains. First, we prove that $g(S) \leq f(S)$ for any given set $S$. According to construction of $g(.)$, for each additive function in $g(.)$ such $g_{\gamma}$, there is at least one additive function in $f(.)$ such $f_{\delta}$ where $g_{\gamma}(b_i) \leq f_{\delta}(b_i)$ for each $b_i \in \items$. Therefore, for any given set $S$ we have:
 \begin{equation}\label{hineq1} g(S) \leq f(S)  \end{equation} 
Now, according to the construction of $g(.)$, for any given set $S$ where $f(S) \leq x$, we have a function $g_{\gamma}(S) = f(S)$, and where $f(S) > x$, we have a function $g_{\gamma}(S) = x$. Therefore, we can conclude that: 
 \begin{equation}\label{hineq2} g(S) \geq f^x(S)  \end{equation} 
 
For any given set $S$ where $f(S) \leq x$, according to the definition of $f^x(.)$, $f(S) = f^x(S)$, and using Inequalities \eqref{hineq1} and \eqref{hineq2} we argue that $f^x(S) = g(S)$. Moreover, according to the construction of $g(.)$, $g(S) \leq x$ for any given set $S$. Therefore, for any given set $S$ where $f(S) > x$, according to the definition of $f^x(.)$ and Inequality \eqref{hineq2}, $f^x(S) = g(S) = x$. As a result, by considering these two cases we argue that $f^x(.)$ and $g(.)$ are equivalent, which shows $f^x(.)$ is an XOS function.\\
\textbf{Third Claim:} In this claim, we use a similar argument to the first claim. By definition of subadditive functions for any given sets $A$ and $B$, we have:
$$f(A \cup B) \leq f(A) + f(B)$$
We prove that $f^x(.)$ meets the definition of subadditive functions by considering two different cases. In the first case at least one of $f(A)$ and $f(B)$ is at least $x$, and in the second case both $f(A)$ and $f(B)$ is less than $x$.\\

First Case: In this case $f^x(A) + f^x(B)$ is at least $x$, and since $f^x(S) \leq x$ for any given set $S$, $f^x(A \cup B) \leq x$. Therefore, 
$$f^x(A \cup B) \leq f^x(A) + f^x(B)$$

Second Case: Since $f^x(A \cup B) \leq f(A \cup B)$, $f(A \cup B) \leq f(A) + f(B)$, $f(A) = f^x(A)$, and $f(B) = f^x(B)$, we have:
$$f^x(A \cup B) \leq f^x(A) + f^x(B)$$
 
\end{proof}

\begin{comment}
\begin{proof}[Of Lemma \ref{submodularsefr}]
Suppose for the sake of contradiction that there is at least one item $b_j$ where $V_k(\{b_j\}) \geq 1/3$ for $a_k$. Now, let $T = \{a_k\}$, $S = \{b_j\}$, and $\mathcal{A} = \langle A_1, A_2, \ldots, A_n\rangle$ be an allocation of $S$ to agents of $T$ where $A_k = \{b_j\}$, and $A_i = \emptyset$ for any $i \neq k$. Since $V_k(\{b_j\}) \geq 1/3$ and $b_j$ is the only member of $A_k$, we have $V_k(A_k) \geq 1/3$. Therefore: 
\begin{equation}\label{hineq3} \forall \agent_i \in T \hspace{3cm} V_i(A_i) \geq 1/3  \end{equation} 

According to the definition of $\MMS$, any agent $a_i$ can allocate $\items$ to all $n$ agents in such a way that the value of each of these sets is at least $1$ for $a_i$. Since $|S| = 1$, the unique member of $S$ in of only one of these $n$ sets for each agent $a_i$. Therefore, any agent $a_i$ can divide $\items \setminus S$ to at least $n-1$ sets in order the value of each of these $n-1$ sets be at least one for $a_i$. Using allocation $A$:
\begin{equation}\label{hineq4} \forall \agent_i \notin T \hspace{1cm} \MMS_{V_i}^{n-|T|} (\items \setminus S)\geq 1  \end{equation} 

Inequalities \eqref{hineq3} and \eqref{hineq4} imply that the problem is $1/3$-reducible.
\end{proof}
\end{comment}
\begin{proof}[Of Lemma \ref{submodularaval}]
Since $f(.)$ is submodular, according to the definition of submodular functions, for every given sets $X$ and $Y$ in domain of $f(.)$ with $X \subseteq Y$ and every $x \in \items \setminus Y$ we have:
\begin{equation} \label{hineq5} f(X \cup \{x\}) - f(X) \geq f(Y \cup \{x\}) - f(Y) \end{equation}

Let $S_i = \{e_1, e_2, \ldots, e_{\alpha}\}$, $T_0 = \emptyset$, and $T_j = \{e_1, e_2, \ldots, e_j\}$, for every $1 \leq j \leq \alpha$. Since $T_j \subseteq S_i$ for each $0 \leq j \leq \alpha$ and $f_i$ is a submodular function, according to Inequality \eqref{hineq5} we have:
\begin{equation} \label{hineq6} \sum_{1 \leq j \leq \alpha} f_i(S_i \setminus T_{j-1}) - f_i(S_i \setminus T_j) \geq \sum_{1 \leq j \leq \alpha} f_i(S_i) - f_i(S_i - e_j) \end{equation}

Since $f_i(S_i) = \sum_{1 \leq j \leq \alpha} f_i(S_i \setminus T_{j-1}) - f_i(S_i \setminus T_j)$, we can rewrite Inequality \eqref{hineq6} for every $1 \leq i \leq k$ as follows:
\begin{equation} \label{hineq7} f_i(S_i) \geq \sum_{e \in S_i} f_i(S_i) - f_i(S_i - e) \end{equation}

For every $1 \leq i \leq k$ we can rewrite Inequality \eqref{hineq7} as follows:
\begin{equation} \label{hineq8} \sum_{e \in s_i} f_i(S_i-e) \geq (|S_i| - 1) f_i(S_i) \end{equation}

By adding $(|\bigcup S_i| - |S_i|) f_i(S_i)$ to the both sides of Inequality \eqref{hineq8}, we have:
\begin{equation} 
\label{hineq9} 
 \begin{split}
 (|\bigcup S_i| - |S_i|) f_i(S_i) + \sum_{e \in S_i} f_i(S_i - e) &= \sum_{e \in \bigcup S_i} f_i(S_i \setminus \{e\}) \\
 &\geq (|\bigcup S_i| - 1) f_i(S_i)
 \end{split}
\end{equation}

Since Inequality \eqref{hineq9} holds for every $1 \leq i \leq k$, we can sum up both sides of Inequality \eqref{hineq9} as follows:

\begin{equation} \label{hineq10} \sum_{1 \leq i \leq k}\sum_{e \in \bigcup S_i} f_i(S_i - e) \geq \sum_{1 \leq i \leq k} (|\bigcup S_i| - 1) f_i(S_i) \end{equation}

By dividing both sides of Inequality \eqref{hineq10} over $1/|\bigcup S_i|$ we obtain:

\begin{equation}
\label{hineq11} 
\begin{split}
    \frac{1}{|\bigcup S_i|}(\sum_{e \in \bigcup S_i} \sum_{1 \leq i \leq k} f_i(S_i - e)) &= \mathbb{E}[\sum_{1 \leq i \leq k} f_i(S^*_i)]\\
    &\geq \sum_{1 \leq i \leq k} f_i(S_i)\frac{|\bigcup S_i| -1}{|\bigcup S_i|}.
\end{split}
\end{equation}

\end{proof}

\begin{proof}[Of Lemma \ref{submodulardovom}]
Similar to the proof of Lemma \ref{submodularaval}, we use Inequality \eqref{hineq5} as a definition of submodular functions. Let $S'_i = S_i \setminus S = \{e_1, e_2, \ldots, e_{\alpha}\}$, $T_0 = S$, and $T_j = S \cup \{e_1, e_2, \ldots, e_j\}$ for $1 \leq j \leq \alpha$. According to $f(S) < 1/3$, $f(S \cup S'_i) \geq 1$, and Inequality \eqref{hineq5} as a definition of sub-modular functions, we have:

\begin{equation}
\label{hineq12} 
\begin{split}
    2/3 &< f(S \cup S') - f(S)\\
    &= \sum_{1 \leq j \leq \alpha} f(T_{j-1} \cup \{e_j\}) - f(T_{j-1})\\
    &\leq \sum_{e \in S'_i} f(S \cup \{e\}) - f(S)
\end{split}
\end{equation}

Similar to Inequality \eqref{hineq10}, we can rewrite Inequality \eqref{hineq12} with a summation, since Inequality \eqref{hineq12} holds for any $1 \leq i \leq k$.

\begin{equation}
    \label{hineq13}
    2k/3 < \sum_{1 \leq i \leq k} \sum_{e \in S'_i} f(S \cup \{e\}) - f(S)
\end{equation}

By dividing both sides of Inequality \eqref{hineq13} over $1/ |\bigcup S_i \setminus S|$ we have:

\begin{equation}
    \label{hineq14}
    \begin{split}
    \frac{2k/3}{|\bigcup S_i \setminus S|} &< \frac{1}{|\bigcup S_i \setminus S|}(\sum_{1 \leq i \leq k} \sum_{e \in S'_i} f(S \cup \{e\}) - f(S))\\
    &= \mathbb{E}[f(S \cup \{e\}) - f(S)]
    \end{split}
\end{equation}
\end{proof}