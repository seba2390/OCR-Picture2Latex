\subsection{Phase 1: Building the Clusters}\label{additive:clusters}
In this section, we explain our method for clustering the agents. Intuitively, we divide the agents into three clusters $\cone,\ctwo$ and $\cthree$. As mentioned before, during the algorithm, two sets of items $\firstset_i,\secondset_i$ are allocated to each agent $\agent_i$. Throughout this section, we prove a set of lemmas that are labeled as \emph{value-lemma}. In these lemmas we bound the value of $f_i$ and $g_i$ allocateed to any agent for other agents. A summary of these lemmas is shown in Tables \ref{table0}, \ref{table4} and \ref{table1}. 


After constructing each cluster, we refine that cluster. In the refinement phase of each cluster, we target a certain subset of the remaining items. If any item in this subset could satisfy an agent in the recently created cluster, we allocate that item to the corresponding agent. The goal of the refinement phase is to ensure that the remaining items in the targeted subset are light enough for the agents in that cluster, i.e., none of the remaining items can satisfy an agent in this cluster.

We denote by $\satagents$, the set of satisfied agents. In addition, denote by $\satagents_1, \satagents_2$, and $\satagents_3$ the subsets of $\satagents$, where $\satagents_i$ refers to the agents of $\satagents$ that previously belonged to ${\mathcal C}_i$. Furthermore, we use $\satagents_1^r$ and $\satagents_2^r$ to refer to the agents of $\satagents_1$ and $\satagents_2$ that are satisfied in the refinement phases of $\cone$ and $\ctwo$, respectively.
\subsubsection{Cluster $\cone$} \label{cluster1:building}
Consider the filtering $G_{1/2}\langle V_{1/2}(G),E_{1/2}(G) \rangle$ of the value-graph $G$ and let $M$ be an $\MCMWM$ of $G_{1/2}$. We define Cluster $\cone$ as the set of agents whose corresponding vertex is in $N(F_{G_{1/2}}(M,\itemsv_{1/2}))$. 

For brevity, denote by $V_{\cone}$ the set of vertices in $V(G)$ that correspond to the agents of $\cone$. In other words:
$$V_{\cone} = N(F_{G_{1/2}}(M,\itemsv_{1/2})).$$



 
Also, let $F_{G_{1/2}}(M,\itemsv_{1/2}) $ be $U_1 \cup S_1$, where $U_1$ is the set of unsaturated vertices in $F_{G_{1/2}}(M,\itemsv_{1/2})$ and $S_1$ is the set of the saturated vertices. For each edge $(\itemv_j,\agentv_i) \in M$ such that $\itemv_j \in S_1$, we allocate  item $\ite_j$ to agent $\agent_i$. More precisely, we set $\firstset_i = \{\ite_j \}$. Since $w(\itemv_j,\agentv_i)\geq {1/2}$, we have:
$$\forall \agent_k \in \cone \qquad V_k(f_k) \geq {1/2}.$$
According to the definition of $\epsilon_i$, we have
\begin{equation}
\forall \agent_k \in \cone \qquad \epsilon_k \leq {1/4}.
\end{equation} 

By the definition of $F_{G_{1/2}}$, for every agent which is not in $\cone$, the condition of Lemma \ref{forc2c3} holds. Note that all the agents that are not in $\cone$, belong to either $\ctwo$ or $\cthree$.


\begin{lemma}[value-lemma]
\label{forc2c3}
For all $\agent_i \in \ctwo \cup \cthree$ we have: \[ \forall \agent_j \in \cone \qquad \valu_i(\firstset_j) < 1/2. \]
\end{lemma}

For each vertex $\agentv_i \in V_{\cone}$, denote by $N_{\agentv_i}$ the set of vertices $\itemv_j  \in \itemsv \setminus \itemsv_{1/2}$, where $w(\itemv_j,\agentv_i) \geq \epsilon_i$ and let \[W_1 = U_1 \cup \bigcup_{\agentv_i \in V_{\cone}} N_{\agentv_i}.\]

Note that by definition, for any vertex $\itemv_j \in U_1$ and $\agentv_i \notin V_{\cone}$, there is no edge between $\itemv_j$ and $\agentv_i$ in $G_{1/2}$ and hence $w(\itemv_j,\agentv_i)<1/2$. Also, since the rest of the vertices in $W_1$ are from $\itemsv \setminus \itemsv_{1/2}$, for any vertex $\agentv_i$ and $\itemv_j \in (W_1 \setminus U_1)$, $w(\itemv_j,\agentv_i)<1/2$ holds. Thus, we have the following observation:

\begin{observation}
\label{w1small}
For every item $\ite_j$ with $\itemv_j \in W_1$ and every agent $\agent_i$ with $\agentv_i \notin V_{\cone}$, $\valu_i(\{\ite_j\})<1/2$.
\end{observation}

Now, define $\itemsv'$ and $ \agentsv' $ as follows:

$$\itemsv' = \itemsv \setminus (W_1 \cup S_1),$$ $$\agentsv' = \agentsv \setminus V_{\cone}.$$ 

Let $G'\langle V(G'),E(G')\rangle $ be the induced subgraph of $G$ on $V(G') = \agentsv' \cup \itemsv'$. We use graph $G'$ to build Cluster $\ctwo$. 



\subsubsection{Cluster $\cone$ Refinement} Before building Cluster $\ctwo$, we satisfy some of the agents in $\cone$ with the items corresponding to the vertices of $W_1$. Consider the subgraph $G_1 \langle V(G_1),E(G_1) \rangle$ of $G$ with $V(G_1) = W_1 \cup V_{\cone}$. In $G_1$, There is an edge between $\agentv_i \in V_{\cone}$ and $\itemv_j \in W_1$, if $V_i(\{\ite_j\}) \geq \epsilon_i$. Note that $G_1 \langle V(G_1),E(G_1) \rangle$ is not necessarily an induced subgraph of $G$. We use $G_1$ to satisfy a set of agents in $\cone$. To this end, we first show that $G_1$ admits a special type of matching, described in Lemma \ref{nicematch}.

\begin{lemma}
\label{nicematch}
There exists a matching $M_1$ in $G_1$, that saturates all the vertices of $W_1$ and for any edge $(\itemv_i,\agentv_j) \in M_1$ and any unsaturated vertex $\agentv_k \in N(\itemv_i)$, $\agent_k$ does not envy $\agent_j$. 
\end{lemma}


Let $M_1$ be a matching of $G_1$ with the property described in Lemma \ref{nicematch}. For every edge $(\agentv_i,\itemv_j) \in M_1$, we allocate  item $\ite_j$ to agent $\agent_i$ i.e., we set $\secondset_i = \{\ite_j\}$. By the definition, $\agent_i$ is now satisfied. Thus, we remove $\agent_i$ from $\cone$ and add it to $\cal S$. Note that, after refining $\cone$, none of the items whose corresponding vertex is in $\itemsv' \setminus \itemsv'_{1/2}$ can satisfy any remaining agent in $\cone$. Thus, Observation \ref{fsmallc1} holds.



\begin{observation}
\label{fsmallc1}
For every item $\ite_j$ such that $\itemv_j \in \itemsv'$, either $\itemv_j \in \itemsv'_{1/2}$ or for all $\agent_i \in \cone$, $V_i(\{\ite_j\}) < \epsilon_i$.
\end{observation}

At this point, all the agents of $\satagents$ belong to $\satagents_1^r$. Each one of these agents is satisfied with two items, i.e., for any agent $\agent_j \in \satagents_1^r$, $|\firstset_j| = |\secondset_j| = 1$. In Lemma \ref{gsmallc1r} we give an upper bound on $\valu_i(\secondset_j)$ for every agent $\agent_j \in \satagents_1^r$ and every agent $\agent_i$ in $\ctwo \cup \cthree$.  

\begin{lemma}[value-lemma]
\label{gsmallc1r}
For every agent $\agent_i \in \ctwo \cup \cthree$, we have
$$ \forall \agent_j \in \satagents_1^r \qquad \valu_i(\secondset_j)< 1/2.$$
\end{lemma}

Lemmas \ref{gsmallc1r} and  \ref{forc2c3}  state that for every agent $\agent_i \in \ctwo \cup \cthree$  and every agent $\agent_j \in \satagents_1^r$, $\valu_i(\firstset_j)$ and $\valu_i(\secondset_j)$ are upper bounded by $1/2$. This, together with the fact that $|\firstset_j| = |\secondset_j|=1$, results in Lemma \ref{forc2}.
\begin{lemma}
\label{forc2}
For all $\agent_i \notin \cone$, we have
\[ \MMS_{\valu_i}^{|\agents \setminus \satagents_1^r|} ( {\items} \setminus \bigcup_{\agentv_j \in \satagents_1^r} \firstset_j \cup \secondset_j) \geq 1.\]
\end{lemma}

\subsubsection{Cluster $\ctwo$}
Recall graph $G' \langle V(G') , E(G') \rangle$ as described in the last part of Section \ref{cluster1:building} and let $G'_{1/2}\langle V_{1/2}(G'), E_{1/2}(G') \rangle$ be a $1/2$-filtering of $G'$. Lemma \ref{rem} states that the size of the maximum matching between $\itemsv'_{1/2}$ and $\agentsv'_{1/2}$ is $|\itemsv'_{1/2}|$. Also, according to Corollary \ref{remcol}, for any maximum matching $M'$ of $G'_{1/2}$, $F_{G'_{1/2}}(M',\itemsv'_{1/2})$ is empty. In what follows, we increase the size of the maximum matching in  $G'_{1/2}$ by merging the vertices of $\itemsv' \setminus \itemsv'_{1/2}$ as described in Definition \ref{merge}.

\begin{figure}[t!]
    \centering
    \includegraphics[scale=1]{figs/merge}
    \caption{Merging $\itemv_1$ and $\itemv_2$}
    \label{fig:merge}
\end{figure}


\begin{definition}
\label{merge}
For merging vertices $\itemv_i,\itemv_j$ of $G'(\itemsv',\agentsv')$, we create a new vertex labeled with $\itemv_{i,j}$. Next, we add $\itemv_{i,j}$ to $\itemsv'$ and for every vertex $\agentv_k \in \agentsv'$, we add an edge from $\agentv_k$ to $\itemv_{i,j}$ with weight $w(\agentv_k,\itemv_i) + w(\agentv_k,\itemv_j)$. Finally we remove vertices $\itemv_i$ and $\itemv_j$ from $\itemsv$. See Figure ~\ref{fig:merge}.
\end{definition}

In Lemmas \ref{c1small2} and \ref{pairsmall}, we give upper bounds on the value of the pair of items corresponding to a merged vertex. In Lemma \ref{c1small2}, we show that the value of a merged vertex is less than $2\epsilon_i$ to every agent $\agent_i \in \cone$. This fact is a consequence of Observation \ref{fsmallc1}. Also, in Lemma \ref{pairsmall}, we prove that the value of the items corresponding to a merged vertex is less than $3/4$ to any agent. Lemma \ref{pairsmall} is a direct consequence of $3/4$-irreducibility. In fact, we show that if the condition of Lemma \ref{pairsmall} does not hold, then the problem can be reduced. 

\begin{lemma}
\label{c1small2}
For any agent $\agent_k \in \cone$ and any pair of vertices $\itemv_i, \itemv_j \in \itemsv' \setminus \itemsv'_{1/2}$, $\valu_k(\{\ite_i,\ite_j\}) < 2\epsilon_k$ holds. In particular, total value of the items that belong to a merged vertex is less than $2\epsilon_k$ for $\agent_k$.
\end{lemma}

\begin{lemma}
\label{pairsmall}
 For any pair of vertices $\itemv_i , \itemv_j \in \itemsv' \setminus \itemsv'_{1/2}$ and any vertex $\agentv_k \in \agentsv$, we have $V_k(\{\ite_i,\ite_j\}) < {3/4}$.
\end{lemma}

\begin{corollary} [of Lemma \ref{pairsmall}]
\label{forc2small}
For any agent $\agent_i$ with $\agentv_i \in \agentsv$, there is at most one item $\ite_j$, with $\itemv_j \in \itemsv' \setminus \itemsv'_{1/2}$ and $\valu_i(\{\ite_j\}) \geq {3/8}$.
\end{corollary}

Consider the vertices in $\itemsv' \setminus \itemsv'_{1/2}$. We call a pair $(\itemv_i,\itemv_j)$ of distinct vertices in $\itemsv' \setminus \itemsv'_{1/2}$ \textit{desirable} for $\agentv_k \in \agentsv'$, if $w(\agentv_k,\itemv_i) + w(\agentv_k,\itemv_j) \geq {1/2}$. With this in mind, consider the process described in Algorithm \ref{addvertex}. 

In each step of this process, we find an $\MCMWM$ $M'$ of $G'_{1/2}$. Note that $M'$ changes after each step of the algorithm. Next, we find a pair $(\itemv_i,\itemv_j)$ of the vertices in $\itemsv' \setminus \itemsv'_{1/2}$ that is desirable for at least one agent in $T = \agentsv' \setminus N(F_{G'_{1/2}}(M',\itemsv'_{1/2}))$. If no such pair exists, we terminate the algorithm. Otherwise, we select an arbitrary desirable pair $(\itemv_i,\itemv_j)$ and merge them to obtain a vertex $\itemv_{i,j}$. According to the definition of $T$ in Algorithm \ref{addvertex}, merging a pair  $(\itemv_i,\itemv_j)$ results in an augmenting path in $G'_{1/2}$. Hence, the size of the maximum matching in $G'_{1/2}$ is increased by one. Note that after the termination of Algorithm \ref{addvertex}, either $T = \emptyset$ or  no pair of vertices in $\itemsv' \setminus \itemsv'_{1/2}$ is desirable for any vertex in $T$. 

\begin{lemma} 
\label{sizeeq}
After running Algorithm \ref{addvertex}, we have
$$|F_{G'_{1/2}}(M',\itemsv'_{1/2})| = |N(F_{G'_{1/2}}(M',\itemsv'_{1/2}))|.$$  
\end{lemma}

\begin{algorithm}[t!]
 \KwData{$G'(V(G'),E(G'))$}
 \While{True}{
  $M' = \MCMWM \mbox{  of } G'_{1/2}$\; 
  Find $F_{G'_{1/2}}(M',\itemsv'_{1/2})$\;
  $T = \agentsv' \setminus N(F_{G'_{1/2}}(M',\itemsv'_{1/2}))$\;
   $Q = $ Set of all desirable pairs in $\itemsv' \setminus \itemsv'_{1/2}$ for the agents in $T$\;
  \eIf{$ Q = \emptyset$ }{
   STOP\;
   }{
   Select an arbitrary pair $\itemv_i,\itemv_j$ from $Q$\;
   Merge($\itemv_i,\itemv_j$)\;
  }
 }
 \caption{Merging vertices in $G'$}
 \label{addvertex}
\end{algorithm}

 




We define Cluster $\ctwo$ as the set of agents that correspond to the vertices of $N(F_{G'_{1/2}}(M',\itemsv'_{1/2}))$. Also, denote by $V_{\ctwo}$ the vertices in $N(F_{G'_{1/2}}(M',\itemsv'_{1/2}))$. For each agent $\agent_i \in \ctwo$, we allocate the item corresponding to $M'(\agentv_i)$ (or pair of items in case  $M'(\agentv_i)$ is a merged vertex) to $\agent_i$.


Note that we put the rest of the agents in Cluster $\cthree$. Therefore, Lemma \ref{forc3} holds for all the agents of $\cthree$.

\begin{lemma}[value-lemma]
\label{forc3}
For all $\agent_i \in \cthree$ we have \[ \forall \agent_j \in \ctwo, \valu_i(\firstset_j) < 1/2. \]
\end{lemma}

\subsubsection{Cluster $\ctwo$ Refinement}
The refinement phase of $\ctwo$, is semantically similar to the refinement phase of $\cone$. In the refinement phase of $\ctwo$, we satisfy some of the agents of $\ctwo$ by the items with vertices in $\itemsv' \setminus \itemsv'_{1/2}$. Note that none of the vertices in $\itemsv' \setminus \itemsv'_{1/2}$ is a merged vertex.


The refinement phase of $\ctwo$ is presented in Algorithm \ref{c2ref}. Let $\agent_{i_1}, \agent_{i_2}, \ldots, \agent_{i_k}$ 
\begin{comment}
$\agentv_{i_1}, \agentv_{i_2}, \ldots, \agentv_{i_k}$ 
\end{comment}
be the topological ordering of the agents in $\ctwo$ as described in Section \ref{additive:cef}
\begin{comment}
with respect to their representation graph
\end{comment}
. In Algorithm \ref{c2ref}, We start with $\agentv_{i_1}$ and $W_2 = \emptyset$ and check whether there exists a vertex $\itemv_j \in \itemsv' \setminus (\itemsv'_{1/2} \cup W_2)$ such that $V_{i_1}(\{\ite_j\}) \geq \epsilon_{i_1}$. If so, we add $\itemv_j$ to $W_2$ and satisfy $\agent_{i_1}$ by allocating $\ite_j$ to $\agent_{i_1}$. Next, we repeat the same process for $\agentv_{i_2}$ and continue on to $\agentv_{i_k}$. Note that at the end of the process, $W_2$ refers to the vertices whose corresponding items are allocated to the agents that are satisfied during the refinement step of $\ctwo$. For convenience, let $S_2 = F_{G'_{1/2}}(M',\itemsv'_{1/2})$ and define $\itemsv''$ and $\agentsv''$ as follows:
$$\itemsv'' = \itemsv' \setminus (W_2 \cup S_2),$$
$$\agentsv'' = \agentsv' \setminus V_{\ctwo}.$$

Let $G'' \langle V(G''),E(G'') \rangle$ be the induced subgraph of $G'$ on $V(G'') = \itemsv'' \cup \agentsv''$. We use $G''$ to build Cluster $\cthree$.


\begin{algorithm}[t!]
 \KwData{$G'(V(G'),E(G'))$}
 \KwData{$\agent_{i_1},\agent_{i_2},\ldots,\agent_{i_k}$ = Topological ordering of agents in $\ctwo$}
  \For{$l:1\rightarrow k$}{
	\If{$ \exists \itemv_j \in \itemsv' \setminus (\itemsv'_{1/2} \cup W_2)$ s.t. $V_{i_1}(\{\ite_j\}) \geq \epsilon_{i_l}$)}
	{
		$\secondset_{i_l} = \ite_j$ \;
		$W_2 = W_2 \cup \itemv_j$\;
		$\ctwo = \ctwo \setminus \agent_{i_l}$\;
		${\satagents} = {\satagents} \cup \agent_{i_l}$\;
	}
  }
 \caption{Refinement of $\ctwo$}
 \label{c2ref}
\end{algorithm}


\begin{observation} 
\label{fsmallc2}
After running Algorithm \ref{c2ref}, For every item $\ite_j$ with $\itemv_j \in \itemsv'' \setminus \itemsv''_{1/2} $ and every agent $ \agent_i \in \ctwo$, we have $V_i(\{\ite_j\}) < \epsilon_i$. 
\end{observation}


In the following two lemmas, we give upper bounds on the value of $\secondset_i$ for every agent $\agent_i \in \satagents_2^r$. First, in Lemma \ref{cr2smallc1}, we show that for every agent $\agent_j \in \cone$, $\valu_j(\secondset_i)$ is upper bounded by $\epsilon_j$. Furthermore, by the fact that the agents that are not selected for Clusters $\cone$ and $\ctwo$ belong to Cluster $\cthree$, we show that $\valu_j(\secondset_i)$ is upper bounded by $1/2$ for every agent $\agent_j \in \cthree$. 
\begin{lemma}[value-lemma]
\label{cr2smallc1}
Let $\agent_i \in \satagents_2^r$ be an agent that is satisfied in the refinement phase of Cluster $\ctwo$ and $\agent_j$ be an  agent in $\cone$. Then, $\valu_j(\secondset_i)<\epsilon_j$.
\end{lemma}


\begin{lemma}[value-lemma]
\label{cr2smallc3}
Let $\agent_i \in \satagents_2^r$ be an agent that is satisfied in the refinement phase of Cluster $\ctwo$ and $\agent_j$ be an agent in $\cthree$. Then, $\valu_j(\secondset_i)< 1/2$.
\end{lemma}

\subsubsection{Cluster $\cthree$.} Finally, Cluster $\cthree$ is defined as the set of agents corresponding to the vertices of $\agentsv''$. Let $M''$ be an $\MCMWM$ of $G''_{1/2}$. Note that by Lemma \ref{rem}, all the vertices in $\itemsv''_{1/2}$ are saturated by $M''$. 
For each vertex $\agentv_i$ that is saturated by $M''$, we allocate the item (or pair of items in a case that $M''(\agentv_i)$ is a merged vertex) corresponding to $M''(\agentv_i)$ to $\agent_i$. Unlike the previous clusters, this allocation is temporary. A semi-satisfied agent $\agent_i$ in $\cthree$ may \emph{lend} his $f_i$ to the other agents of $\cthree$. Therefore, we have three type of agents in $\cthree$: 
\begin{enumerate}
    \item \textbf{The semi-satisfied agents}: we denote the set of semi-satisfied agents in $\cthree$ by $\cthree^s$
    \item \textbf{The borrower agents}: the agents that may borrow from a semi-satisfied agent. An agent $\agent_j$ in $\cthree$ is a borrower, if $\agent_j \notin \cthree^s$ and $\max_{\agent_i \in \cthree^S} V_j(f_i) \geq {1/2}$. We denote the set of borrower agents in $\cthree$ by $\cthree^b$.
    \item \textbf{The free agents}: the remaining agents in $\cthree$. We denote the set of free agents by $\cthree^f$.
\end{enumerate}
So far, the agents corresponding to unsaturated vertices in $\agentsv''_{1/2}$ belong to $\cthree^b$ and the agents corresponding to the vertices in $\agentsv'' \setminus \agentsv''_{1/2}$ are in $\cthree^f$. As we see, during the second phase, agents in $\cthree$ may change their type. For example, an agent in $\cthree^s$ may move to $\cthree^f$ or vice versa.  For convenience, for every agent $\agent_i \in \cthree^b$, we define $\epsilon_i$ as follows: 
\begin{equation}
\label{borrowers}
 3/4 - \max_{\agent_j \in \cthree^s}\valu_i(\firstset_j)
\end{equation} 
Note that by the definition, $\epsilon_i \leq 1/4$ holds for every agent of $\cthree^b$.


\begin{figure}[t!]
\centering
\includegraphics[scale=0.5]{figs/overview}
\caption{Overview on the state of the algorithm}
\label{fig:overview}
\end{figure}

In Lemma \ref{lsmall_c3}, we show that the remaining items are not \emph{heavy} for the agents in $\cthree$. The main reason that Lemma \ref{lsmall_c3} holds, is the fact that no pair of vertices is desirable for any agents in $\cthree$ at the end of Algorithm \ref{addvertex}. 
\begin{lemma}
\label{lsmall_c3}
For all $\agent_i \in \cthree$ and $\itemv_j,\itemv_k \in \itemsv'' \setminus \itemsv''_{1/2}$, we have  $V_i(\{\ite_j,\ite_k\}) < {1/2}$.
\end{lemma}

\begin{corollary}[of Lemma \ref{lsmall_c3}]
\label{small_c3}
For any agent $\agent_i \in \cthree$, there is at most one vertex $\itemv_j \in \itemsv'' \setminus \itemsv''_{1/2}$, such that $V_i(\{\ite_j\}) \geq {1/4}$.
\end{corollary}

