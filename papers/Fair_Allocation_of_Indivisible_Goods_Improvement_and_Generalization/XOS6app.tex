\section{XOS Agents}\label{xos}
Class of fractionally subadditive (XOS) set functions is a super class of submodular functions. These functions too, have been subject of many studies in recent years \cite{christodoulou2008bayesian, bhawalkar2011welfare, feige2009maximizing,blumrosen2007welfare, syrgkanis2012bayesian,feldman2013simultaneous,fu2012conditional,feldmancombinatorial,milchtaich1996congestion}. Similar to sub-modular functions, in this section we show a $1/5$-$\MMS$ allocation is possible when all agents have XOS valuations. Furthermore, we complement our proof by providing a polynomial algorithm to find a $1/8$-$\MMS$ allocation in Section \ref{xosalgorithm}.

 %In this section we study the fair  allocation problem with XOS agents. To the best of our knowledge, this is the first work that studies this problem with non-additive valuation functions. We give a proof to the existence of a $1/5$-$\MMS$ allocation for XOS agents. This is followed by an algorithm that finds such an allocation in polynomial time. This is surprising since even finding the $\MMS$ of an XOS function is NP-hard and cannot be implemented in polynomial time unless P=NP. Since every submodular function is also fractionally subadditive, our result carries over to submodular functions as well. For brevity, we defer the proof of some lemmas to Appendix \ref{xosappendix} and just mention the high-level ideas.
 
 
\subsection{Existential Proof}\label{ep}
In this section we show every instance of the fair allocation problem with XOS agents admits a $1/5$-$\MMS$ allocation. 
Without loss of generality, we assume $\MMS_i = 1$ for every agent $\agent_i$. Recall the definition of ceiling functions.
\begin{definition}\label{fxfunction}
Given a set function $f(.)$, we define $\ceil{f}{x}(.)$ as follows:
$$\ceil{f}{x}(S) =
\begin{cases}
	f(S), & \text{if }f(S) \leq x \\
	x, & \text{if }f(S) > x.
\end{cases}$$

\end{definition}

As stated in Lemma \ref{ceilingfunctions}, for every XOS function and every real number $x \geq 0$, $f^x$ is also XOS. The proof of this section is similar to the result of Section \ref{submodularalg}. However, the details are different since XOS functions do not adhere to the nice structure of submodular functions. For every allocation $\mathcal{B}$, we define $\mathsf{ex}^{2/5}(\mathcal{B})$ as follows:
 
$$\mathsf{ex}^{2/5}(\mathcal{B}) = \sum_{\agent_i \in \agents} \ceil{\valu_i}{2/5}(B_i).$$

Now Let $\mathcal{A} = \langle A_1, A_2, \ldots, A_n\rangle$ be an allocation of items to the agents that maximizes $\mathsf{ex}^{2/5}$. Provided that the problem is $1/5$-irreducible, we show $\mathcal{A}$ is a $1/5$-$\MMS$ allocation. Before we proceed to the main proof, we state Lemmas  \ref{xos2lemma}, and \ref{2nsets} as auxiliary observations. 
%Based on Definition \ref{fxfunction}, one can show that given an XOS function $f(.)$, $\ceil{f}{x}(.)$ is also XOS for any $x \geq 0$.

\begin{lemma}\label{xos2lemma}
Let $f(.)$ be an XOS set function and  $f(S) = \beta$ for a set $S \subseteq \domp(f)$. If we divide $S$ into $k$ (possibly empty)  sets $S_1, S_2, \ldots, S_k$ then 
$$\sum_{i=1}^k \Big(f(S) - f(S\setminus S_i)\Big) \leq f(S).$$
\end{lemma}
The complete proof of Lemma \ref{xos2lemma} is included in Appendix \ref{xosappendix}. Roughly speaking, the proof follows from the fact that for at least one of the additive set functions in the representation of $f$, we have $g_j(S) = \beta$. The rest of the proof is trivial by the additive properties of $g_j$.

By Lemma \ref{remove1}, we know that in every $1/5$-irreducible instance of the problem, the value of every item for a person is bounded by $1/5$. 
\begin{comment}
\begin{lemma}\label{xosreducible}
In every $1/5$-irreducible instance of the problem we have $\valu_i(\{\ite_j\}) \leq 1/5$ for every $\agent_i \in \agents$ and $\ite_j \in \items$.
\end{lemma}
\end{comment}
For XOS functions, we again, leverage the reducibility principal to show another important property of the $1/5$-irreducible instances of the problem. 
\begin{lemma}
\label{2nsets}
In a $1/5$-irreducible instance of the problem, for a given agent $\agent_i$ we can divide the items into $2n$ sets $S_1, S_2, \ldots, S_{2n}$ such that
$$\valu_i(S_i) \geq 2/5$$
for every $1 \leq i \leq 2n$.
\end{lemma}
We first apply Lemma \ref{remove1} and show in such instances of the problem the valuation of every agent for every item is bounded by $1/5$. We remark that for every agent $\agent_i$, one can split the items into $n$ partitions such that each partition is worth at least $1$ to $\agent_i$. Combining the two observations, we conclude that such a decomposition is possible for every agent $\agent_i$. The full proof of this lemma is included in Appendix \ref{xosappendix}. Next we prove the main theorem of this section.
\begin{theorem}\label{xosproof}
The fair allocation problem with XOS agents admits a $1/5$-$\MMS$ allocation.
\end{theorem}
\begin{proof}
Similar to what we did in Section \ref{additive}, we only prove this for $1/5$-irreducible instances of the problem. By Observation \ref{reducibility}, we can extend this result to all instances of the problem.

Consider an allocation $\mathcal{A} = \langle A_1, A_2, \ldots, A_n\rangle$ of items to the agents that maximizes $\mathsf{ex}^{2/5}$.
We show that such an allocation is $1/5$-$\MMS$. Suppose for the sake of contradiction that there exists an agent $\agent_i$ who receives a set of items which are together of worth less than $1/5$ to him. More precisely,
$$\valu^{2/5}_i(A_i) = \valu_i(A_i) < 1/5.$$ 
Since the problem is $1/5$-irreducible, by Lemma \ref{2nsets}, we can divide the items into $2n$ sets $S_1, S_2, \ldots, S_{2n}$ such that $\valu_i(S_j) \geq 2/5$ for every $1 \leq j \leq 2n$. Note that in this case, $\valu^{2/5}_i(S_j) = 2/5$ follows from the definition. Moreover by monotonicity, $\valu^{2/5}_i(S_j \cup A_i) = 2/5$ holds for every $j$.

Now consider $2n$ allocations $\mathcal{A}^1, \mathcal{A}^2, \ldots, \mathcal{A}^{2n}$ such that 
$$\mathcal{A}^{j} = \langle A^j_1, A^j_2 \ldots, A^j_n\rangle$$ for every $1 \leq j \leq 2n$ where 

$$A^j_k =
\begin{cases}
A_k \cup S_j, & \text{if }k = i \\
A_k \setminus S_j, & \text{if }k \neq i.
\end{cases}$$
We show at least one of these allocations has a higher for $\mathsf{ex}^{2/5}$ than $\mathcal{A}$.
Since $\valu^{2/5}_i$ is XOS, by Lemma \ref{xos2lemma} we have
\begin{equation*}
\sum_{j=1}^{2n}\Big(\valu^{2/5}_k(A_k) - \valu^{2/5}_k(A_k \setminus S_j)\Big) \leq \valu^{2/5}_k(A_j)
\end{equation*}
for every $\agent_k \neq \agent_i$ and thus
\begin{equation}\label{gulu1}
\begin{split}
\sum_{j=1}^{2n} \valu^{2/5}_k(A^j_k) &= \sum_{j=1}^{2n} \valu^{2/5}_k(A_j \setminus S_j)\\
 &\geq 2n \valu^{2/5}_k(A_k) - \valu^{2/5}_k(A_k)\\
&= (2n-1)\valu^{2/5}_k(A_k)
\end{split}
\end{equation}
Moreover, since $\valu^{2/5}_i(A_i) < 1/5$, we have
\begin{equation}\label{gulu2}
\begin{split}
\sum_{\agent_j \neq \agent_i} \ceil{\valu_j}{2/5}(A_j) & > \sum_{\agent_j \in \agents} \ceil{\valu_j}{2/5}(A_j) - 1/5\\
&=\mathsf{ex}^{2/5}(\mathcal{A})-1/5.
\end{split}
\end{equation}
Furthermore, since $\valu_i^{2/5}(S_j \cup A_i) = 2/5$ for every $1 \leq j \leq 2n$, we have
\begin{equation}\label{gulu3}
\begin{split}
\sum_{\agent_k \neq \agent_i} \valu^{2/5}_k(A^j_k) &= \sum_{\agent_k \in \agents} \valu^{2/5}_k(A^j_k) - 2/5\\
&= \mathsf{ex}^{2/5}(\mathcal{A}^j) - 2/5
\end{split}
\end{equation}
Finally, by combining Inequalities \eqref{gulu1}, \eqref{gulu2}, and \eqref{gulu3} we have
\begin{equation*}
	\begin{split}
	\sum_{j=1}^{2n} \mathsf{ex}^{2/5}(\mathcal{A}^j) & = \sum_{j=1}^{2n}(2/5 +\sum_{\agent_k \neq \agent_i} \valu^{2/5}_k(A^j_k)) \\
	& = 4n/5 + \sum_{j=1}^{2n}\sum_{\agent_k \neq \agent_i} \valu^{2/5}_k(A^j_k)\\  
	&\geq 4n/5 + \sum_{\agent_k \neq \agent_i} (2n-1) \valu^{2/5}_k(A_k)\\
	&\geq 4n/5 + (2n-1)(\mathsf{ex}^{2/5}(\mathcal{A}) - 1/5)\\
	&\geq 2n \cdot \mathsf{ex}^{2/5}(\mathcal{A}) + (4n-2n+1)/5 - \mathsf{ex}^{2/5}(\mathcal{A})\\
	&\geq 2n \cdot \mathsf{ex}^{2/5}(\mathcal{A}) + (2n+1)/5 - \mathsf{ex}^{2/5}(\mathcal{A})
	\end{split} 
\end{equation*}
Now notice that since $\valu^{2/5}_k(A_k) \leq 2/5$, we have
\begin{equation*}
	\begin{split}
	\mathsf{ex}^{2/5}(\mathcal{A}) & = \sum_{k=1}^n \valu^{2/5}_k(A_k)\\
	& \leq \sum_{k=1}^n 2/5\\
	& \leq 2n/5. 
	\end{split}
\end{equation*}
and thus 
\begin{equation*}
	\begin{split}
	\sum_{j=1}^{2n} \mathsf{ex}^{2/5}(\mathcal{A}^j) & \geq 2n \cdot \mathsf{ex}^{2/5}(\mathcal{A}) + (2n+1)/5 - \mathsf{ex}^{2/5}(\mathcal{A})\\
	&\geq 2n \cdot \mathsf{ex}^{2/5}(\mathcal{A}) + (2n+1)/5 - 2n/5\\
	&\geq 2n \cdot \mathsf{ex}^{2/5}(\mathcal{A}) + 1/5.
	\end{split}
\end{equation*}
Therefore, $\mathsf{ex}^{2/5}(\mathcal{A}^j) > \mathsf{ex}^{2/5}(\mathcal{A})+1/10n$ holds for at least one $\mathcal{A}^j$ which contradicts the maximality of $\mathcal{A}$.
\end{proof}



\subsection{Algorithm}\label{xosalgorithm}
In this section we provide a polynomial time algorithm for finding a $1/8$-$\MMS$ allocation for the fair allocation problem with XOS agents. The algorithm is based on a similar idea that we argued for the proof of Theorem \ref{xosproof}. Remark that our algorithm only requires access to demand and XOS oracles. It does \textit{not} have any additional information about the maxmin values. This makes the problem computationally harder since computing the maxmin values is NP-hard~\cite{epstein2014efficient}. We begin by giving a high-level intuition of the algorithm and show the computational obstacles can be overcome by combinatorial tricks. Consider the pseudo-code described in Algorithm \ref{xosalg}.
\begin{algorithm}%[t!]
	\KwData{$\agents, \items, \langle \valu_1, \valu_2, \ldots, \valu_n\rangle$}
	For every $\agent_j$, scale $\valu_j$ to  ensure $\MMS_j = 1$\;
	\While{there exist an agent $\agent_i$ and an item $\ite_j$ such that $\valu_i(\{\ite_j\}) \geq 1/8$}{
		Allocate $\{\ite_j\}$ to $\agent_i$\;
		$\items = \items \setminus \ite_j$\;
		$\agents = \agents \setminus \agent_i$\;
	}
	$\mathcal{A} = $ an arbitrary allocation of the items to the agents\;
	\While{$\min \valu^{1/4}_j(A_j) < 1/8$}{
		$i = $ the agent who receives the lowest value in allocation $\mathcal{A}$\; 
		Find a set $S$ such that:\label{find}
		$\mathsf{ex}^{1/4}(\langle A_1 \setminus S, A_2 \setminus S, \ldots, A_{i-1} \setminus S, A_i \cup S, A_{i+1} \setminus S, \ldots, A_n \setminus S\rangle) \geq \mathsf{ex}^{1/4}(\mathcal{A})+1/12n$\;
		$\mathcal{A} = \langle A_1 \setminus S, A_2 \setminus S, \ldots, A_{i-1} \setminus S, A_i \cup S, A_{i+1} \setminus S, \ldots, A_n \setminus S\rangle$\;
	}
	For every $\agent_i \in \agents$ allocate $A_i$ to $\agent_i$\;
\caption{Algorithm for finding a $1/8$-$\MMS$ allocation}
\label{xosalg}
\end{algorithm}

As we show in Section \ref{former}, Command \ref{find} of the algorithm is always doable. More precisely, there always exists a set $S$ that holds in the condition of Command \ref{find}. Notice that in every step of the algorithm, $\mathsf{ex}^{1/4}(\mathcal{A})$ is increased by at least $1/12n$ and this value is bounded by $1/4\cdot n = n/4$. Therefore the algorithm terminates after at most $3n^2$ steps and the allocation is guaranteed to be $1/8$-$\MMS$.

That said, there are two major computational obstacles in the way of running Algorithm \ref{xosalg}. Firstly, finding a set $S$ that holds in the condition of Command \ref{find} can not be trivially done in polynomial time. Second, scaling the valuation functions to ensure $\MMS_i = 1$ for all agents is NP-hard and cannot be done in polynomial time unless P=NP. To overcome the former, in Section \ref{former} we provide an algorithm for finding such a set $S$ in polynomial time. Next, in Section \ref{latter}, we present a combinatorial trick to run the algorithm in polynomial time without having to deal with NP-hardness of scaling the valuation functions.

\subsubsection{Executing Command \ref{find} in Polynomial Time}\label{former}
In this section we present an algorithm to execute Command \ref{find} of Algorithm \ref{xosalg}. We show that such a procedure can be implemented via demand oracles.

Let for every $\ite_j \notin A_i$, $c_j$ be the amount of contribution that $\ite_j$ makes to $\mathsf{ex}^{1/4}(\mathcal{A})$. We set $p_e = 3(n/(n-1))c_e$ and ask the demand oracle of $\valu_i$ to find a set $S$ that maximizes $\valu_i(S) - \sum_{\ite_j \in S} p_j$. Via a trivial calculation, one can show that $\valu_i(S) - \sum_{\ite_j \in S} p_j \geq 1/4$ holds for at least one set of items. The reason this is correct is that one can divide the items into $n$ partitions where each is worth at least $1$ to $\agent_i$. Moreover, the summation of prices for the items is bounded by $3n/(n-1)\cdot (\sum_{j\neq i} \valu_j^{1/4} (A_j) )\leq 3n/4$. Therefore, for at least one of those partitions  $\valu_i(S) - \sum_{\ite_j \in S} p_j$ is at least $1/4$. Thus, the set that the oracle reports is worth at least $1/4$ to $\agent_i$. 

Now, let $S^*$ be the set that the oracle reports and for every $\ite_j \in S^*$, $c^*_j$ be the contribution of $\ite_j$ to $\valu_i(S^*)$. We sort the items of $S^*$ based on $c^*_j - p_j$ in non-increasing order. Next, we start with an empty bag and add the items in their order to the bag until the total value of the items in the bag to $\agent_i$ reaches $1/4$. Since the value of every item alone is bounded by $1/8$, the total value of the items in the bag to $\agent_i$ is bounded by $3/8$. Thus the contribution of those items to $\mathsf{ex}^{1/4}(\mathcal{A})$ is at most $(3/8) / (3n/(n-1)) \leq 1/8 - 1/(10n)$. Therefore, removing items of the bag from other allocations and adding them to $A_i$, increases $\mathsf{ex}^{1/4}(\mathcal{A})$ by at least $1/10n$.

Remark that one can use the same argument to prove this even if $\MMS_i \geq 1/(1+1/10n)$.

\subsubsection{Running Algorithm \ref{xosalg} in Polynomial Time}\label{latter}
As aforementioned, scaling valuation functions to ensure $\MMS_i = 1$ for every agent $\agent_i$ is an NP-hard problem since determining the maxmin values is hard even for additive agents~\cite{Procaccia:first}. Therefore, unlike Section \ref{former}, in this section we massage the algorithm to make it executable in polynomial time.

Suppose an oracle gives us the maxnmin values of the agents. Provided that we can run Command \ref{find} of Algorithm \ref{xosalg} in polynomial time, we can find a $1/8$-$\MMS$ allocation in polynomial time. Therefore, in case the oracle reports the actual maxmin values, the solution is trivial. However, what if the oracle has an error in its calculations? There are two possibilities: (i) Algorithm \ref{xosalg} terminates and finds an allocation which is $1/8$-$\MMS$ with respect to the reported maxmin values. (ii) The algorithm fails to execute Command \ref{find}, since no such set $S$ holds in the condition of Command \ref{find}. The intellectual merit of this section boils down to investigation of the case when algorithm fails to execute Command \ref{find}. We show, this only happens due to an overly high misrepresentation of the maxmin value for agent $\agent_i$. Note that $\agent_i$ is the agent who receives the lowest value in the last cycle of the execution.

\begin{observation}\label{beautiful}
	Given $\langle d_1, d_2, \ldots, d_n\rangle$ as an estimate for the maxmin values, if Algorithm \ref{xosalg} fails to execute Command \ref{find} for an agent $\agent_i$, then we have
	$$d_i \geq (1+1/10n) \MMS_i.$$
\end{observation} 
Proof of Observation \ref{beautiful} follows from the argument of Section \ref{former}. More precisely, as mentioned in Section \ref{former}, such a set $S$ exists, if $\MMS_i \geq 1/(1+1/10n)$. Thus, given that the procedure explained in Section \ref{former} fails to find such a set, one can conclude the the reported value for $\MMS_i$ is at least $(1/(1+1/10n))$ times its actual value. Based on Observation \ref{beautiful}, we propose Algorithm \ref{oracle} for implementing a maxmin oracle.

\begin{algorithm}%[t!]
	\KwData{$\agents, \items, \langle \valu_1, \valu_2, \ldots, \valu_n\rangle$}
	\For {every $\agent_i \in \agents$}{
		$d_i \leftarrow \valu_i(\items)$\;
	}
	\While{true}{
		Run Algorithm \ref{xosalg} assuming maxmin values are $d_1, d_2, \ldots, d_n$\;
		\If {the Algorithm fails to run Command \ref{find} for an agent $\agent_i$}{
			$d_i \leftarrow d_i / (1+1/10n)$\; 
		}\Else{
			Report the allocation and terminate the algorithm\;
		}
	}
	\caption{Implementing a maxmin oracle}
	\label{oracle}
\end{algorithm}

Note that in the beginning of the algorithm, we set $d_i = \valu_i(\items)$ which is indeed greater than or equal to $\MMS_i$. By Lemma \ref{beautiful}, every time we decrease the value of $d_i$ for an agent $\agent_i$, we preserve the condition $d_i \geq \MMS_i$ for that agent. Therefore, in every step of the algorithm, we have $d_i \geq \MMS_i$ and thus the reported allocation which is $1/8$-$\MMS$ with respect to $d_i$'s is also $1/8$-$\MMS$ with respect to true maxmin values. Thus, the algorithm provides a correct $1/8$-$\MMS$ allocation in the end. All that remains is to show the running time of the algorithm is polynomial.

Notice that every time we decrease $d_i$ for an agent $\agent_i$, we multiply this value by $1/(1+1/10n)$, hence the number of such iterations is polynomial in $n$, unless the valuations are super-exponential in $n$. Since we always assume the input numbers are represented by $\mathsf{poly}(n)$ bits, the number of iterations is bounded by $\mathsf{poly}(n)$ and hence the algorithm terminates after a polynomial number of steps.

\begin{theorem}\label{xa}
	Given access to demand and XOS oracles, there exists a polynomial time algorithm that finds a $1/8$-$\MMS$ allocation for XOS agents.
\end{theorem}

An elegant consequence of Theorem \ref{xa} is a $8$-approximation algorithm for determining the maxmin value of an XOS function with $r$ partitions.
\begin{corollary}
	Given an XOS function $f$, an integer number $r$, and access to demand and XOS oracles of $f$, there exists a $8$-approximation polytime algorithm for determining $\MMS^r_f$.
\end{corollary}
\begin{proof}
	We construct an instance of the fair allocation problem with $r$ agents, all of whom have a valuation function equal to $f$. We find a $1/8$-$\MMS$ allocation of the items to the agents in polynomial time and report the minimum value that an agent receives as output.
	
	The $1/8$ guarantee follows from the fact that every agent receives a subset of values that are worth $1/8$-$\MMS_i$ to him, and since $\MMS_i$ is exactly equal to $\MMS^r_f$, every partition has a value of at least $\MMS^r_f/8$.
\end{proof}
\begin{remark}
	A similar procedure can also be used to overcome the challenge of computing the maxmin values for the algorithm described in Section \ref{submodularalgorithm}.
\end{remark}