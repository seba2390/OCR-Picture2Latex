\section{Omitted Proofs of Section \ref{xos}}\label{xosappendix}

\begin{proof}[of Lemma \ref{xos2lemma}]
According to the definition of XOS function, $f(.)$ is an XOS function with a finite set of additive functions $\{g_1, g_2, \ldots, g_{\alpha}\}$ where $f(S) = \max_{i=1}^{\alpha} g_i(S)$ for any set $S \in \domp(f)$. Let $g_j(.)$ be the additive function which maximizes $S$. Let $g_j(S_1) = \alpha_1, g_j(S_2) = \alpha_2, \ldots, g_j(S_k) = \alpha_k$, which yields $\beta = \sum \alpha_i$. Since $g_j(S_i) = \alpha_i$, $f(S \setminus S_i) \geq \beta - \alpha_i$. Therefore, we have:

\begin{equation}
    \label{hineq15}
    \begin{split}
    \sum f(S) - f(S \setminus S_i) &\leq \sum \beta - (\beta - \alpha_i)\\
    &= \beta\\
    &= f(S)
    \end{split}
\end{equation}


%Since $f(.)$ is an XOS set function, at least for one of the additive functions of $f$ like $f_j$ we have $f_j(S) = \beta$. So according the definition of an additive function it means that $f_j(S_1) + f_j(S_2) + \ldots + f_j(S_k) = \beta$. Therefore, there is a set like $S_i$ such that $f_j(S_i) \ge \frac{\beta}{k}$, which yields that $f(S_i) \ge \frac{\beta}{k}$.
\end{proof}




%Any agent $a_{i}$ has an XOS valuation function $V_{i}(.)$, and we have relaxed this valuation function in a way $\MMS$ of any agent is equal to one. In this subsection we want to describe an existential proof for $1/6$ allocation in XOS valuations. We do not go through the finding of the allocation in this subsection, and we will talk about the algorithm in the next subsection.
\begin{comment}
\begin{proof}[ of Lemma \ref{xosreducible}]
%Suppose for the sake of contradiction that we have at least one item $b_j$ with value equal or more than $1/6$ for $a_k$. Now, suppose that $T = \{a_k\}$, $S = \{b_j\}$, and $A = \langle A_1, A_2, \ldots, A_n\rangle$ is an allocation of $S$ to agents of $T$ where $A_k = (b_j, a_k)$, and $A_i = \emptyset$ for $i \neq k$. Now, $\forall \agent_i \in T$ we have $V_i(A_i) \geq 1/6$. Before allocation $\MMS$ of each agent was equal to one. By the definition of $\MMS$ we know that each agent can divide $\items$ to $n$ sets such that the value of each set be at least one for him. Since $|S| = 1$ the lonely member of $S$ is present in only one of these $n$ sets for each agent. So, we have $n-1$ sets with value at least one for each agent after the allocation, and we have satisfied one of the agents. Therefore, $\forall \agent_i \notin T$ we have $\MMS_{V_i}^{n-|T|} (\items \setminus S)\geq 1$. According to Definition \ref{d1} our instance of problem is $1/6-reducible$ which contradicts our assumption.
Similar to the proof of Lemma \ref{submodularsefr}, suppose that there is an item $b_j$ where $V_k(\{b_j\}) \geq 1/5$ for $a_k$. Let $T = \{a_k\}$, $S = \{b_j\}$, and $\mathcal{A} = \langle A_1, A_2, \ldots, A_n\rangle$ is an allocation of $S$ to agents of $T$ where $A_k = (b_j, a_k)$, and $A_i = \emptyset$ for any $i \neq k$. Similar to the proof of Lemma \ref{submodularsefr} we can prove that using allocation $\mathcal{A}$ our instance of the problem is $1/5$-reducible.
\end{proof}
\end{comment}
%According to Lemma \ref{xosreducible} we can suppose that there is no item with value equal or more than $1/6$ for any agent.
%Now, suppose an allocation which maximizes $\sum_{i=1}^{i=n} V_{i}^{1/3}(S_{i})$, where $S(i)$ is the set of allocated items to $a_{i}$. If after this allocation for all of the agents like $a_{i}$ we have $V_{i}^{1/3}(S_{i}) \ge 1/6$ all of the agents are satisfied by the allocation. Therefore, suppose that there is at least one agent like $a_{i}$ such that $V_{i}^{1/3}(S_{i}) < 1/6$. Now, we want to divide $\items$ to $2n$ sets such that each set has at least $1/3\MMS_{i}$ value for $a_{i}$.


\begin{proof}[of Lemma \ref{2nsets}]
According to the definition of $\MMS$, we know that $a_{i}$ can divide items to $n$ sets ${\cal P} = \langle P_1, P_2, \ldots, P_n \rangle$ such that $V_i(P_j) \geq 1$ for any $P_j$. The catch is that $a_i$ can divide each of these $n$ sets to two disjoint sets such that the value of each of these new sets be at least $2/5$ to him. Let $T = \{b_1, b_2, \ldots, b_\gamma\}$ be one of these $n$ sets, and $g_j(.)$ be an additive function which maximizes $V_i(T)$. Let $T_k = \{b_1, b_2, \ldots, b_k\}$ for any $1 \leq k \leq \gamma$. According to Lemma \ref{remove1}, since the problem is $1/5$-irreducible, the value of any item is less than $1/5$ to $a_i$. Therefore, there is a set $T_k$ among $T_1$ to $T_\gamma$ where $2/5 \leq g_j(T_k) < 3/5$. Since $g_j(.)$ is one of additive functions of XOS function $V_i$, we have $V_i(T_k) \geq 2/5$. Moreover, since $g_j(T_k) < 3/5$, $g_j(T \setminus T_k) \geq 2/5$, which yields $V_i(T \setminus T_k) \geq 2/5$. As a conclusion, we can divide each of $n$ sets to two disjoint sets with at least $2/5$ value to $a_i$.

\end{proof}

 
%\begin{proof}[of Lemma \ref{beautiful}]
%\end{proof}