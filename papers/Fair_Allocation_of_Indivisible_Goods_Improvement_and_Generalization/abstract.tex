We study the problem of fair allocation for indivisible goods. We use the \textit{the maxmin share} paradigm introduced by Budish~\cite{Budish:first} as a measure for fairness. \procacciafirst ~\cite{Procaccia:first} (EC'14) were first to investigate this fundamental problem in the additive setting. In contrast to what real-world experiments suggest, they show that a maxmin guarantee (1-$\MMS$ allocation) is not always possible even when the number of agents is limited to 3. While the existence of an approximation solution (e.g. a $1/2$-$\MMS$ allocation) is quite straightforward, improving the guarantee becomes subtler for larger constants.
\procacciafirst ~\cite{Procaccia:first}\footnote{Recipient of the best student paper award at EC'14.} provide a proof for existence of a $2/3$-$\MMS$ allocation and leave the question open for better guarantees.

Our main contribution is an answer to the above question. We improve the result of \procacciafirst\! to a $3/4$
 factor in the additive setting. The main idea for our $3/4$-$\MMS$ allocation method is clustering the agents. To this end, we introduce three notions and techniques, namely \textit{reducibility}, \textit{matching allocation}, and \textit{cycle-envy-freeness}, and prove the approximation guarantee of our algorithm via non-trivial applications of these techniques. Our analysis involves coloring and double counting arguments that might be of independent interest.

One major shortcoming of the current studies on fair allocation is the additivity assumption on the valuations. We alleviate this by extending our results to the case of submodular, fractionally subadditive, and subadditive settings. More precisely, we give constant approximation guarantees for submodular and XOS agents, and a logarithmic approximation for the case of subadditive agents. Furthermore, we complement our results by providing close upper bounds for each class of valuation functions. Finally, we present algorithms to find such allocations for additive, submodular, and XOS settings in polynomial time. The reader can find a summary of our results in Tables \ref{resultstable} and \ref{resultstable2}. 
