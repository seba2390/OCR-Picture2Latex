\section{Omitted Proofs of Section \ref{additive:observations}}\label{additiveobservationsproof}

\begin{proof}[of Lemma \ref{remove1}]
The key idea is that given $\MMS_i \geq 1$ for an agent $\agent_i$, then for every item $\ite_j \in \items$ we have 
$\MMS_i^{n-1}(\items\setminus\ite_j) \geq 1$. This holds since removing an item from $\items$ will diminish the value of at most one partition in the optimal $n$ partitioning of the items. Therefore, at least $n-1$ partitions have a value of $1$ or more to $\agent_i$ and thus $\MMS_i^{n-1}(\items\setminus\ite_j) \geq 1$.
The rest of the proof follows from the definition of $\alpha$-irreducibility. If the valuation of an item $\ite_j$ to an agent $\agent_i$ is at least $\alpha$, then the problem is $\alpha$-reducible since if we allocate $\ite_j$ to $\agent_i$, we have 
$$\MMS_{\valu_k}^{n-1}(\items \setminus \{\ite_j\}) \geq 1$$
for every agent $\agent_k \neq \agent_i$. This contradicts with the $\alpha$-irreducibility assumption.
\end{proof}

\begin{proof}[of Lemma \ref{remove2}]
Suppose for the sake of contradiction that for every agent $\agent_{i'} \neq \agent_i$ we have $\valu_{i'}(\{\ite_j,\ite_k\}) \leq 1$. By this assumption, we show 
\begin{equation}
 \MMS_{i'}^{n-1}(\items\setminus \{\ite_j,\ite_k\}) \geq 1 \label{saeed1}
\end{equation} holds. This is true since removing two items $\ite_j$ and $\ite_k$ from $\items$ decreases the value of at most two partitions of the optimal partitioning of $\items$ for $\MMS_{i'}$. If $n-1$ partitions remain intact, then Inequality \eqref{saeed1} trivially holds. If not, merging the two partitions that initially contained $\ite_j$ and $\ite_k$ results in a partition with value at least $1$ to $\agent_i$. This partition together with the $n-2$ remaining partitions result in a desirable partitioning of $\items$ into $n-1$ partitions. Therefore, Inequality \eqref{saeed1} holds for any agent $\agent_{i'}$, and this implies that by allocating $S = \{\ite_j, \ite_k\}$ to $\agent_i$, not only does $\valu_i(S) \geq 3/4$ hold, but also for every $\agent_{i'} \neq \agent_i$ we have 
$$\MMS_{i'}^{n-1}(\items\setminus \{\ite_j,\ite_k\}) \geq 1$$
which means the problem is $3/4$-reducible, and it contradicts our assumption.
\end{proof}

\begin{proof}[of Lemma \ref{remove3}]
The proof for this lemma is obtained by applying Lemma \ref{remove2}, $|T|$ times. Consider an agent $\agent_i \notin T$. According to the argument in Lemma \ref{remove2}, if we assign $\ite_{{j_1}}$ and $\ite_{{j_2}}$ to $\agent_{i_1}$, $\agent_i$ can partition the items in $\items \setminus \{ \ite_{{j_1}}$ $\ite_{{j_2}}\}$ into $n-1$ partitions with value at least 1 to $\agent_i$, i.e.
$$\MMS_{i}^{n-1}(\items\setminus \{\ite_{j_1},\ite_{j_2}\}) \geq 1.$$
By the same deduction, after assigning $\ite_{{j_3}}$ and $\ite_{{j_4}}$ to $\agent_{i_2}$, we have
$$\MMS_{i}^{n-2}(\items\setminus \{\ite_{j_1},\ite_{j_2},\ite_{j_3},\ite_{j_4}\}) \geq 1.$$
By repeating above argument $|T|$ times, we have:
$$\MMS_{i}^{n-|T|}(\items\setminus S) \geq 1.$$

On the other hand, by condition $(II)$, every agent $\agent_{i_k}$ satisfies with items $\ite_{j_{2k-1}}$ and $\ite_{j_{2k}}$. This means that we can reduce the instance by satisfying the agents in $T$ by the items in $S$, which is a contradiction by the irreducibility assumption.


\begin{comment}
We show that if all the conditions are met, the instance is $3/4$-reducible. To show this, let $\mathcal{A}$ be an allocation of $S$ to the agents in $T$ such that all of the conditions are met. Suppose that $\mathcal{A} = \langle A_{i_1}, A_{i_2}, \ldots, A_{i_{|T|}}\rangle$ where $A_{i_a} = \{\ite_{j_{2a-1}},\ite_{j_{2a}}\}$ for every $\agent_{i_a} \in T$. According to  condition (ii), after we allocate the items, every $\agent_{i_a} \in T$ has a value of at least $3/4$ for the items assigned to him. Now, if we prove that for every $\agent_i \notin T$ we have $\MMS_{V_i}^{n-|T|} (\items \setminus S)\geq 1$, our problem will be $3/4$-reducible which contradicts our assumption. \\
Suppose there is at least one agent $\agent_i \notin T$ such that $\MMS_{V_i}^{n-|T|} (\items \setminus S) < 1$. 
Before removing $S$, $\agent_i$ can divide $\items$ into $n$ parts ${\cal P} =\langle P_1, P_2, \ldots, P_n \rangle$ such that $V_i(P_k) \geq 1$ for every $k$. Suppose that $\ite_{j_1}$ and $\ite_{j_2}$ are in parts $P_{k_1}$ and $P_{k_2}$ respectively. As a first step, we start a procedure of removing the items of $S$ with $\ite_{j_1}$ and $\ite_{j_2}$ from $P_{k_1}$ and $P_{k_2}$. If $k_1 = k_2$, after this removal, the other $n-1$ parts will have a value of at least $1$ to $\agent_i$. Also, we add the rest of the items of $P_{k_1}$ to another arbitrary part. 

If $k_1 \neq k_2$, we add all of the remaining items in $P_{k_1}$ to $P_{k_2}$. According to the third condition we have $\valu_{i}(\{\ite_{j_{1}},\ite_{j_{2}}\}) \leq 1$. Therefore, after adding the rest of the items of $P_{k_1}$ to $P_{k_2}$, $\valu_{i}(P_{k_2})$ is at least $1$. Hence, after the first step we have $n-1$ parts with value at least $1$ to $\agent_i$. \\
We do the same in all of the $|T|$ steps. Before starting $q$-th step, we have $n-q+1$ remaining parts. In the $q$-th step we remove $\ite_{j_{2q-1}}$ and $\ite_{j_{2q}}$ from their corresponding parts, which are $P_{k_{2q-1}}$ and $P_{k_{2q}}$ respectively. If $k_{2q-1} = k_{2q}$, we add the rest of the items of $P_{k_{2q-1}}$ to another arbitrary part. Otherwise, we add the remaining items of $P_{k_{2q-1}}$ to $P_{k_{2q}}$. Since $\valu_{i}(\{\ite_{j_{2q-1}},\ite_{j_{2q}}\}) \leq 1$, after the $q$-th step, we have $n-q$ parts with a value of at least $1$ to $\agent_i$. Hence, after the $|T|$-th step, we have removed all of the items of $S$, and all of the $n-|T|$ remaining parts have a value of at least $1$ to $\agent_i$. Therefore, $\MMS_{V_i}^{n-|T|} (\items \setminus S) \ge 1$ which is a contradiction.
\end{comment}
\end{proof}


\begin{proof}[of Lemma \ref{rem}]
We define $\parttwo_1$ as the set of vertices in $\parttwo$ that are not saturated by $M$, and $\parttwo_2$ as the set of vertices in $\parttwo$ that are connected to $\parttwo_1$ by an alternating path. Moreover, let $\partone_2 = M(\parttwo_2)$. By definition, $F_{H}(M,\partone) = \partone \setminus \partone_2$ (See Figure \ref{fig:FG}). As discussed before, all the vertices in $\partone_2$ are saturated by $M$. Consequently, all the vertices of $T$ are saturated by $M$ and $|N(T)| \geq |T|$. 

Let $M(T)$ be the set of vertices which are matched to the vertices of $T$ in $M$. We know that every vertex of $T$ is present in at least one of the alternating paths which connect $\parttwo_1$ to $\parttwo_2$. Let $$P = \langle \hat{y}_0, \hat{x}_1, \hat{y}_1, \hat{x}_2, \hat{y}_2, \ldots, \hat{x}_k, \hat{y}_k \rangle$$ be one of these paths that includes at least one of the vertices of $T$. Since $P$ is an alternating path which connects $\parttwo_1$ to $\parttwo_2$, $\hat{y}_0 \in \parttwo_1$ (see Figure \ref{fig:FG4}). In addition, according to the definition of alternating path, every edge $(\hat{x}_j,\hat{y}_j)$ of $P$ belongs to $M$ and every edge $(\hat{x}_j,\hat{y}_{j-1})$ does not belong to $M$. 

Let $\hat{x}_i$ be the first vertex of $T$ that appears in $P$. We know that the edge $(\hat{x}_i,\hat{y}_{i-1})$ does not belong to $M$. On the other hand, since $\hat{x}_i$ is the first vertex of $T$ in $M$, $\hat{x}_{i-1} \notin T$. Note that $\hat{y}_{i-1}$ does not belong to $M(T)$, since every vertex of $M(T)$ is matched with a vertex of $T$ in $M$ and $(\hat{x}_{i-1},\hat{y}_{i-1})$ is in $M$.  The fact that $\hat{y}_{i-1} \notin M(T)$ means $N(T)$ contains at least one vertex that is not in $M(T)$. Since all the vertices in $M(T)$ are in $N(T)$, $|N(T)|>|M(T)|$ and hence, $|N(T)|>|T|$.
\begin{figure}
\centering
\includegraphics[scale=0.6]{figs/matching2}
\caption{Alternating path $P$ connects ${\hat{\cal Y}}_1$ to ${\hat{\cal Y}}_2$ and intersects $T$}
\label{fig:FG4}
\end{figure}
\begin{comment}
Suppose $x_i$ is the first vertex of $T$ in $P$. Since $x_1 \in \parttwo_1$, $x_i \neq x_1$. By the assumption that $x_{i}$ is the first vertex of $T$ in $P$, $x_{i-1}$ does not belong to $T$. In addition, $x_{i-1}$ is connected to $x_i$, since $(x_i,x_{i-1})$ is an edge of $P$. Thus, the vertices in $T$ have at least one neighbour, which is not in $M(T)$. Therefore, $N(T) = M(T)$ cannot hold, and we have $|N(T)| > |M(T)|$ which yields $|N(T)| > |T|$.
\end{comment}
\end{proof}


\begin{proof}[of Lemma \ref{iff}]
If $F_H(M, \partone) = \emptyset$, according to Lemma \ref{rem}, $$\forall T \subseteq \partone \qquad |N(T)| > |T|.$$

On the other hand, suppose that for all $T \subseteq \partone$ we have $|N(T)| > |T|$. We show that $F_H(M, \partone) = \emptyset$. For the sake of contradiction, assume that $F_H(M, \partone) \neq \emptyset$ and let $T = F_H(M, \partone)$. Since there exists a matching from $T$ to $N(T)$ that saturates all the vertices of $N(T)$, we have $|T| \geq |N(T)|$, which is a contradiction. Hence, $F_H(M, \partone) = \emptyset$. 
\end{proof}



\begin{proof}[of Lemma \ref{dag}]
Consider a cycle $L$ in $G_C$. For each vertex $v_j \in L$, there is at least one vertex $v_i \in L$ such that $\agent_i$ envies $\agent_j$. Therefore, Considering $S$ as the set of agents with vertices in $L$, none of the agents of $S$ is a loser. By the same deduction, none of the agents of $S$ is a winner. But this contradicts the fact that the set $C$ is cycle-envy-free.
\end{proof}


\begin{proof}[of Lemma \ref{wm}] We describe the proof for the first condition in more details. The proof for the second condition is almost the same as the first condition. 

\textbf{The first condition}: Suppose that there exists no such vertex. Our goal is to find a new matching of $H$ with the same cardinality, but with more weight. To this end, we construct a directed graph $H'$ from $H$ as follows: for each $\vtwo_j \in T$ we consider a vertex $v_j$ in $V(H')$. Furthermore, there is a directed edge from $v_j$ to $v_i$ in $H'$, if and only if $w(\vone_j,\vtwo_{j}) < w(\vone_i,\vtwo_j)$ in $H$. 

If there exists a vertex $v_j$ with out-degree zero in $H'$, then $\vtwo_j$ is the desired winner in $T$, since
$$ \forall \vtwo_i \in H, w(\vone_j,\vtwo_{j}) \geq w(\vone_i,\vtwo_j).$$
 Otherwise, the out-degree of every vertex in $T$ is non-zero. Therefore, $H'$ has at least one cycle $L = \langle v_{l_1}, v_{l_2}, \ldots, v_{l_{|L|}}\rangle$. Now, if we change matching $M$ by removing the set of edges $$ \{(\vtwo_{l_1},\vone_{l_1}), (\vtwo_{l_2},\vone_{l_2}), \ldots, (\vtwo_{l_{|L|}},\vone_{l_{|L|}})\} $$
from $M$ and adding 
$$\{(\vtwo_{l_1},\vone_{l_2}), (\vtwo_{l_2},\vone_{l_3}),\ldots,(\vtwo_{l_{|L|}},\vone_{l_1})\}$$ 
to $M$, the weight of our matching will be increased. Note that by the definition of an edge in $H'$, we have $$w(\vone_{l_2}, \vtwo_{l_1}) > w(\vone_{l_1}, \vtwo_{l_1}),w(\vone_{l_3}, \vtwo_{l_2}) > w(\vone_{l_2}, \vtwo_{l_2}),\ldots,w(\vone_{l_1}, \vtwo_{l_{|L|}}) > w(\vone_{l_{|L|}}, \vtwo_{l_{|L|}}).$$ But this contradicts the fact that  $M$ was $\MCMWM$ of $H$.


\textbf{The second condition}:
\begin{comment}
The proof of this part is very similar to the proof of the first condition. For the sake of contradiction suppose that for every vertex $\vtwo_j \in T$, there is at least one vertex $\vtwo_i \in T$ where $w(\vone_j,\vtwo_{i}) > w(\vone_i,\vtwo_i)$ in $H$ with $(\vone_j,\vtwo_i) \in E(H)$.
\end{comment}
Similar to the proof of the first condition, we construct a new directed graph $H'$ from $H$ where we have a vertex $v_j$ in $H'$ for each vertex $\vtwo_j$ in $T$. For every pair $\vtwo_i$ and $\vtwo_j$ which are members of $T$ we connect $v_i$ to $v_j$ with a directed edge in $H'$ if 
$$w(\vone_j,\vtwo_{i}) > w(\vone_i,\vtwo_i)$$ in $H$ and $(\vone_j,\vtwo_i) \in E(H)$. Note that if $H'$ contains a vertex $v_i$ with in-degree equal to zero, then $\vtwo_i$ is the desired loser in $T$. Thus, suppose that no vertex in $H'$ has in-degree zero and hence, $H'$ has a directed cycle.  Let $L = \langle \vtwo_{l_1}, \vtwo_{l_2}, \ldots, \vtwo_{l_{|L|}}\rangle$ be a directed cycle in $H'$. Similar to the proof of the previous condition, we leverage $L$ to alter $M$ to a new matching with more weight, which is a contradiction by the maximality of $M$.  

\textbf{The third condition}: If $w(\vone_i,\vtwo_i) < w(\vone_j,\vtwo_i)$, we can replace the edge between $\vone_i$ and $\vtwo_i$ by $(\vone_j, \vtwo_i)$ in $M$ which yields a matching with a greater weight. This contradicts the maximality of $M$.
\end{proof}
