
\begin{proof}[of Lemma \ref{c2null}]
Lemmas \ref{c3null} and \ref{c1null} state that at the end of the algorithm, $\cone = \cthree = \emptyset$. Now, let $\agent_i$ be a winner of $\ctwo$. We consider two cases separately: $\epsilon_i \geq {1/8}$ and $\epsilon_i < {1/8}$.

If $\epsilon_i \geq{1/8}$, the proof follows from a similar argument we used to prove Lemma \ref{c1null}. 

\begin{lemma}
\label{c2rem}
If $\epsilon_i \geq 1/8$, then the following inequality holds:
$$ \sum_{\agent_j \in \satagents} \valu_i( \firstset_j \cup \secondset_j) \leq |\satagents| + 1/8.$$
\end{lemma}
\begin{proof}
We know $\satagents = \satagents_1 \cup \satagents_2 \cup \satagents_3$. For every agent $\agent_j$ in $\satagents_3$, by Lemmas \ref{general} and \ref{c3fsmall}, we know $\valu_i(\firstset_j)<3/4$. Also, according to Lemma \ref{prvalue}, $\valu_i(\secondset_j)< \epsilon_i \leq 1/4$. Therefore,
\begin{equation}
\label{bow}
\sum_{\agent_j \in \satagents_3} \valu_i(\firstset_j \cup \secondset_j) \leq \sum_{\agent_j \in \satagents_3} (3/4 + 1/4) = |\satagents_3|.
\end{equation} 
Now, consider an agent $\agent_j \in \satagents_1$. Note that by Lemma \ref{forc2c3}, $\valu_i(\firstset_j)< 1/2$. Also, remark that either $\agent_j \in \satagents_1^r$  or $\agent_j \in \satagents_1^s$ . If $\agent_j \in \satagents_1^r$ then according to Lemma \ref{gsmallc1r}, $\valu_i(\secondset_j)< 1/2$ holds and hence $\valu_i(\firstset_j \cup \secondset_j)< 1$. Also, If $\agent_j \in \satagents_1^s$, then according to Lemma \ref{prvalue}, $\valu_i(\secondset_j)<2\epsilon_i<1/2$. 
Thus, in both cases, $\valu_i(\firstset_j \cup \secondset_j)< 1$ and hence:
\begin{equation}
\label{hogh}
\sum_{\agent_j \in \satagents_1} \valu_i(\firstset_j \cup \secondset_j) \leq \sum_{\agent_j \in \satagents_1} 1 = |\satagents_1|.
\end{equation} 

Finally consider a satisfied agent $\agent_j \in \satagents_2$. Again, remark that either $\agent_j \in \satagents_2^r$  or $\agent_j \in \satagents_2^s$ holds. 

Consider the case that $\agent_j \in \satagents_2^s$. If $\agent_j \prec_{pr} \agent_i$, then by Observation \ref{epsofcluster}, $\valu_i(\firstset_j) \leq 3/4 - \epsilon_i$ and by Lemma \ref{prvalue}, $\valu_i(\secondset_j) < 2\epsilon_i \leq 1/4 + \epsilon_i$ which means $\valu_i(\firstset_j \cup \secondset_j) < 1$. Moreover, if $\agent_i \prec_{pr} \agent_j$, according to Lemmas \ref{general} and \ref{prvalue}, $\valu_i(\firstset_j \cup \secondset_j) < 3/4 + \epsilon_i \leq 1 $. Thus, we have:

\begin{equation}
\label{f}
\sum_{\agent_j \in \satagents_2^s} \valu_i(\firstset_j \cup \secondset_j) \leq \sum_{\agent_j \in \satagents_2^s} 1 = |\satagents_1|
\end{equation}
It only remains to investigate the case where $\agent_j \in \satagents_2^r$. Note that since $\agent_i$ is not satisfied in the refinement phase of $\ctwo$, if $\agent_i \prec_{pr} \agent_j$, then $\valu_i(\secondset_j)< \epsilon_i \leq 1/4$. Otherwise, we could assign the item in $\secondset_j$ to $\agent_i$ in the refinement phase of $\ctwo$. Also, by Lemma \ref{general}, $\valu_i(\firstset_j)<3/4$ holds which yields $\valu_i(\firstset_j \cup \secondset_j)<1$.  

Finally, if $\agent_j \prec_{pr} \agent_i$, by Observation \ref{epsofcluster} $\valu_i(\firstset_j) \leq 3/4 - \epsilon_i$ holds. Corollary \ref{forc2small} states that there is at most one item $\ite_k$ with $\items_k \in \itemsv' \setminus \itemsv'_{1/2}$ and $\valu_i(\ite_k) \geq 3/8$. Also, note that since $\ite_k$ belongs to  $\itemsv' \setminus \itemsv'_{1/2}$, $\valu_i(\{\ite_k\})<1/2$ holds. For agent $\agent_j$, let $\ite_l$ be the item that is assigned to $\agent_j$ in the refinement of $\ctwo$, i.e., $\secondset_j = \{\ite_l\}$. We have $$\valu_i(\firstset_j \cup \secondset_j) \leq 3/4 - \epsilon_i + \valu_i(\{\ite_l\}).$$  
If $\ite_l \neq \ite_k$, $\valu_i(\firstset_j \cup \secondset_j) \leq 3/4 - \epsilon_i + 3/8$ holds which by the fact that $\epsilon_i \geq{1/8}$, implies $\valu_i(\firstset_j \cup \secondset_j) \leq 3/4 -1/8 + 3/8 \leq 1$. In addition to this, If $\ite_l = \ite_k$, 
$\valu_i(\firstset_j \cup \secondset_j) \leq 3/4 - 1/8 + 4/8 \leq 1+1/8$. But this can happen for at most one agent. Thus, for every agent $\agent_j$ in $\satagents_2^r$, $\valu_i(\firstset_j \cup \secondset_j) \leq 1$ holds and for at most one agent $\agent_j \in \satagents_2^r$, $\valu_i(\firstset_j \cup \secondset_j) \leq 1+1/8$. Thus, we have

\begin{equation}
\label{u}
\sum_{\agent_j \in \satagents_2^r} \valu_i(\firstset_j \cup \secondset_j) \leq |\satagents_2^r| + 1/8.
\end{equation}
Inequality (\ref{u}) together with Inequality (\ref{f}) yields
\begin{equation}
\label{oh}
\sum_{\agent_j \in \satagents_2} \valu_i(\firstset_j \cup \secondset_j) \leq |\satagents_2| + 1/8.
\end{equation}
Furthermore, by Inequalities (\ref{bow}), (\ref{hogh}) and (\ref{oh}) we have
\begin{equation}
\begin{split}
\sum_{\agent_j \in \satagents} \valu_i( \firstset_j \cup \secondset_j)  & = \sum_{\agent_j \in \satagents_1} \valu_i( \firstset_j \cup \secondset_j) + \sum_{\agent_j \in \satagents_2} \valu_i( \firstset_j \cup \secondset_j) + \sum_{\agent_j \in \satagents_3} \valu_i( \firstset_j \cup \secondset_j)\\
& \leq |\satagents_1| + |\satagents_2|+ 1/8 + |\satagents_3| \\
& \leq |\satagents|+1/8. \\
\end{split}
\end{equation}
 
\end{proof}


By Lemma \ref{c2rem}, value of agent $\agent_i$ for the items assigned to the satisfied agents is less than $|\satagents| + 1/8$. Recall that $\ctwo = \cthree = \emptyset$ and hence $|\satagents| = n - |\ctwo|$. Therefore,
\begin{equation}
\sum_{\agent_j \in \satagents} \valu_i( \firstset_j \cup \secondset_j) \leq n - |\ctwo| + 1/8.
\end{equation}
Since $\agent_i$ is a winner of $\ctwo$, for all $\agent_j \in \ctwo$, we have  $\valu_i(\firstset_j)\leq \valu_i(\firstset_i)$. On the other hand, since the total value of all items for $\agent_i$ is equal to $n$ we have

\begin{equation}\label{yy1}
\begin{split}
\valu_i({\fitems}) & = \valu_i(\items) - \sum_{\agent_j \in \ctwo} {\valu_i(\firstset_j)} - \sum_{\agent_j \in \satagents}\valu_i(\firstset_j \cup \secondset_j)\\
& = n - \sum_{\agent_j \in \ctwo} {\valu_i(\firstset_j)} - \sum_{\agent_j \in \satagents}\valu_i(\firstset_j \cup \secondset_j)\\
& \geq n - \sum_{\agent_j \in \ctwo} {\valu_i(\firstset_j)} - \big[ n - |\ctwo| + 1/8 \big] \\
& = |\ctwo| - 1/8 - \sum_{\agent_j \in \ctwo} {\valu_i(\firstset_j)}.
\end{split}
\end{equation}

Also, $\valu_i(\firstset_i) = {3/4} - \epsilon_i$ holds and $\valu_i(\firstset_j) \leq \valu_i(\firstset_i)$ for any $\agent_j \in \ctwo$ follows from the fact that $\agent_i$ is a winner of $\ctwo$. Therefore by Inequality (\ref{yy1}) we have
\begin{equation*}
\begin{split}
\valu_i({\fitems}) & \geq |\ctwo| - 1/8 - \sum_{\agent_j \in \ctwo} {\valu_i(\firstset_j)}\\
& \geq |\ctwo| - 1/8 - \sum_{\agent_j \in \ctwo} {\valu_i(\firstset_i)}\\
& = |\ctwo| - 1/8 - |\ctwo| {\valu_i(\firstset_i)}\\
& = |\ctwo| - 1/8 - |\ctwo| ({3/4} - \epsilon_i)\\
& = |\ctwo| ({1/4} + \epsilon_i) - 1/8.
\end{split}
\end{equation*}
%Furthermore since $\epsilon_i \leq 1/4$, $$\valu_i({\fitems}) > |\ctwo|({1/4}+\epsilon_i) - 1/8.$$ 
Recall that by the assumption $\epsilon_i \geq 1/8$ holds. Moreover, $\epsilon_i\leq 1/4$, and thus
\begin{equation*}
\begin{split} 
\valu_i(\fitems) & \geq |\ctwo| ({1/4} + \epsilon_i) - 1/8\\
& \geq |\ctwo| 2\epsilon_i - 1/8\\
& \geq |\ctwo|  2\epsilon_i - \epsilon_i
\end{split}
\end{equation*}
and since $|\ctwo| \geq 1$, 
\begin{equation*}
\begin{split}
\valu_i(\fitems) &\geq |\ctwo|  2\epsilon_i - \epsilon_i\\
& \geq 2\epsilon_i - \epsilon_i\\
& \geq \epsilon_i
\end{split}
\end{equation*}
 and thus $\fitems$ is feasible for $\agent_i$. This contradicts the termination of the algorithm. 

Next, we investigate the case where $\epsilon < {1/8}$. Our proof for this case is similar to the one for $\cone$. Let $\satagents^r_{1}$ be the agents in $\satagents_1$ that are satisfied in the refinement phase and let $$\items^r_1 = \bigcup_{\agent_j \in \satagents^r_1} \firstset_j \cup \secondset_j.$$ Lemma \ref{forc2} states that the maxmin value of the agents in $\ctwo \cup \cthree$ for the items in $\items' = \items \setminus \items^r_1$ is at least $1$. More precisely for every $\agent_j \in \ctwo$:
\begin{equation}
\label{c1refine}
 \MMS_{j}^{n- |\satagents^r_{1}|} ( {\items} \setminus \items^r_{1}) \geq 1 
 \end{equation}  

We color the items of $\items'$ in one of four colors blue, red, green, or white. Initially, all the items are colored in white. For each agent $\agent_j \in \agents \setminus \satagents^r_1$, if $|\firstset_j|=1$, then we color the item in $\firstset_j$ in blue. Also, if $|\firstset_j|=2$ (which means $\firstset_j$ is corresponding to a merged vertex), color both the elements of $\firstset_j$ in red. In addition to this, if $|\secondset_j|=1$ then color the item in $\secondset_j$ in green. For any set $S \subseteq \items$, we denote the subset of blue, red, green, and white items in $S$ by $\mathcal{B}(S)$,$\mathcal{R}(S)$, $\mathcal{G}(S)$, and $\mathcal{W}(S)$, respectively. Recall that by Lemma \ref{pairsmall}, every pair of items in red or green are worth less that $3/4$ in total to $\agent_i$. In other words,
\begin{equation*}
\valu_i(\{\ite_j, \ite_k\}) \leq 3/4.
\end{equation*}
for any two different items $\ite_j, \ite_k \in \mathcal{B}(\items) \cup \mathcal{G}(\items)$.
  Also, according to Lemmas \ref{prvalue} and \ref{c3fsmall}, every set including white items is worth less than $2\epsilon_i < 1/4$ to $\agent_i$. 

Now, let $n' = n - |\satagents^r_1|$. Let $\cal P$ $= \langle P_1, P_2, \ldots, P_{n'} \rangle$ be the optimal $n'-$partitioning of $\items' $ for $\agent_i$. Recall that by Inequality \eqref{c1refine} the value of every partition in $\cal P$ is at least $1$ for $\agent_i$. Based on the number of blue and red items in every partition, we define three sets of partitions:
\begin{itemize}
    \item $B_{00}:$ Partitions with no red or blue items.
    \item $B_{10}:$ Partitions with blue items, but without any red items.
    \item $B_{01}:$ Partitions that contain at least one red item.
\end{itemize}

Next we prove Lemmas \ref{lowerwhite1} and \ref{lowerwhite2} to be used later in the proof. 
\begin{lemma}
\label{lowerwhite1}
Let $|{\mathcal{G}}(B_{00})|$ be the number of green items in the partitions of $B_{00}$. Then, $$\valu_i(\wcal(B_{00})) \geq (3|B_{00}| - |{\mathcal{G}}(B_{00})|)\cdot 1/4.$$
\end{lemma}
\begin{proof}
Let $B_{00}^j$ be the set of partitions in $B_{00}$ that contain exactly $j$ green items. We have:
\begin{equation}
\label{gbound}
|\mathcal{G}(B_{00})| = \sum_{1 \leq j < \infty} j  |B_{00}^j| \geq |B_{00}^1| + 2|B_{00}^2| + \sum_{3 \leq j < \infty} 3  |B_{00}^j| 
\end{equation}
Also, we have:
\begin{equation}
\label{bbound}
3 |B_{00}| = \sum_{0 \leq j < \infty} 3  |B_{00}^j| = 3  |B_{00}^0| + 3  |B_{00}^1| + 3  |B_{00}^2| + \sum_{3 \leq j < \infty} 3  |B_{00}^j|
\end{equation}
Finally, we argue that the value of white items in $B_{00}$ is at least $|B_{00}^0| + |B_{00}^1|\cdot 1/2 + |B_{00}^3|\cdot 1/4$. This follows from the fact that every green item in $P_k \in B_{00}^1$ has a value less than $1/2$ and by Lemma \ref{pairsmall}, every pair of green items in $P_k \in B_{00}^2$ are worth less than $3/4$ to $\agent_j$. According to the fact that the value of every partition $P_k$ is at least 1, we have:

\begin{equation}
\label{wbound}
\valu_i({\cal{W}}(B_{00})) \geq |B_{00}^0| + |B_{00}^1|\cdot 1/2 + |B_{00}^3|\cdot 1/4 = \big( 4|B_{00}^0| + 2|B_{00}^1| + |B_{00}^2| \big)\cdot 1/4
\end{equation}  

According to Equations \eqref{gbound} and \eqref{bbound}, we have:
\begin{equation}
\label{tbound}
3 |B_{00}| - |\mathcal{G}(B_{00})| \leq 3|B_{00}^0| + 2|B_{00}^1| + |B_{00}^2| \leq 4|B_{00}^0| + 2|B_{00}^1| + |B_{00}^2|
\end{equation}

Next we combine Equations \eqref{wbound} and \eqref{tbound} to obtain: 
\begin{equation}
\valu_i({\cal{W}}(B_{00})) \geq \big( 3 |B_{00}| - |\mathcal{G}(B_{00})| \big)\cdot 1/4
\end{equation}

\end{proof}



\begin{lemma}
\label{lowerwhite2}
$\valu_i(\wcal(B_{10})) \geq (2|B_{10}| - |{\mathcal{B}}(B_{10}) | - |{\mathcal{G}}(B_{10})| )\cdot 1/4$
\end{lemma}
\begin{proof}
First, note that every partition in $B_{10}$ contains at least one blue item. Let $B_{10}^{w}$ be the partitions in $B_{10}$ that contains exactly one blue item and no green item. The other items in each partition of $B_{10}^{w}$, are white. Since the problem is $3/4$-irreducible, the value of every blue item to $\agent_i$ is less than $3/4$ and therefore we have:
$$\valu_i(\wcal(B_{10})) \geq |B_{10}^{w}|\cdot 1/4$$
or 
\begin{equation}
\label{wbound2}
4\valu_i(\wcal(B_{10})) \geq |B_{10}^{w}|.
\end{equation}
Moreover, let $B_{10}^{\bar{w}} = B_{10} \setminus B_{10}^w$. Since every partition in $B_{10}$ contains at least one blue item, every partition in $B_{10}^{\bar{w}}$ contains at least two items with colors blue or green. Thus, we have:
\begin{equation}
\label{gbound2}
 |{\mathcal{G}}(B_{10}^{\bar{w}})| + |{\mathcal{B}}(B_{10}^{\bar{w}})| \geq 2|B_{10}^{\bar{w}}|
\end{equation}
Summing up Equations \eqref{wbound2} and \eqref{gbound2} results in 
$$
4\valu_i(\wcal(B_{10})) +  |{\mathcal{G}}(B_{10}^{\bar{w}})| + |{\mathcal{B}}(B_{10}^{\bar{w}})| \geq 2|B_{10}^{\bar{w}}| + |B_{10}^{w}|
$$
which means:
\begin{equation}
\label{fbound}
4\valu_i(\wcal(B_{10})) \geq 2|B_{10}^{\bar{w}}| -  |{\mathcal{G}}(B_{10}^{\bar{w}})| - |{\mathcal{B}}(B_{10}^{\bar{w}})|  + |B_{10}^{w}|.
\end{equation}

Morover, we have $|{\mathcal{B}}(B_{10})| = |{\mathcal{B}}(B_{10}^{w})|+|{\mathcal{B}}(B_{10}^{\bar{w}})|$. According to the fact that every partition in $B_{10}^w$ contains exactly one blue item, $|{\mathcal{B}}(B_{10}^{w})| = |B_{10}^w|$ and hence, $|{\mathcal{B}}(B_{10})| = |B_{10}^w|+|{\mathcal{B}}(B_{10}^{\bar{w}})|$. By Equation \eqref{fbound}, we have:
$$ 4\valu_i(\wcal(B_{10})) \geq 2|B_{10}^{\bar{w}}| -  |{\mathcal{G}}(B_{10}^{\bar{w}})| -|{\mathcal{B}}(B_{10})|+ |B_{10}^w|  + |B_{10}^{w}|. $$
Finally by the fact that $2|B_{10}^{w}| + 2|B_{10}^{\bar{w}}| = 2|B_{10}|$, we have:
$$ 4\valu_i(\wcal(B_{10})) \geq 2|B_{10}| -  |{\mathcal{G}}(B_{10}^{\bar{w}})| -|{\mathcal{B}}(B_{10})| $$
which is:
$$ \valu_i(\wcal(B_{10})) \geq \big(2|B_{10}| -|{\mathcal{B}}(B_{10})|-  |{\mathcal{G}}(B_{10}^{\bar{w}})| \big) \cdot 1/4$$
\end{proof}


For the partitions in $B_{01}$, we construct a graph $G_{01} \langle V_{01},E_{01} \rangle$, where every vertex $v_j \in V_{01}$ corresponds to a partition $P_j \in B_{01}$. Consider an agent $\agent_j$ such that $\firstset_j$ consists of a pair of red items $\ite_k,\ite_{k'}$ and let $\ite_k \in P_l$ and $\ite_{k'} \in P_{l'}$. We add an edge $(v_l,v_{l'})$ to $E_{01}$. By the definition of $B_{01}$, $P_l,P_{l'} \in B_{01}$ holds. Note that $\ite_k$ and $\ite_{k'}$ might belong to the same partition, i.e., $P_l = P_{l'}$. In this case, we add a loop to $G_{01}$. Furthermore, for every item $\ite_k \in {\mathcal B}(B_{01})$, we add a loop to the vertex $v_l$, where $\ite_k \in P_l$. 

Next, define $R_j$ as the set of partitions in $B_{01}$, such that the degree of their corresponding vertices in $V_{01}$ are equal to $j$. In other words:
$$P_k \in R_j \iff d(v_k)=j$$

Next we prove Lemma \ref{lowerwhite3}.

\begin{lemma}
\label{lowerwhite3}
For $R_1$, we have: $$\valu_i(\wcal(R_1)) \geq  (2|R_1| - |{\mathcal{G}}(R_1)|)\cdot 1/4 $$
\end{lemma}
\begin{proof}
Consider a partition $P_j \in R_1$. Since $d(v_j)=1$, $P_j$ contains exactly one red item and no blue item. Thus, other items in $P_j$ are either green or white. We show that 
\begin{equation}
\label{newway}
|{\mathcal G}(P_j)| + 4.\valu_i(\wcal(P_j)) \geq 2.
\end{equation}
First, argue that if $|{\mathcal G}(P_j)| \geq 2$, then Inequality \eqref{newway} holds. Also, if $|{\mathcal G}(P_j)|=0$, then $\valu_i(\wcal(P_j)) \geq 1/2$, because the value of the red item in $P_j$ is less than $1/2$ (recall that all the red items correspond to the vertices in $\itemsv' \setminus \itemsv'_{1/2}$). This immediately implies the fact that $4.\valu_i(\wcal(P_j)) \geq 2$. Finally, if $|{\mathcal G}(P_j)|=1$, then by Lemma \ref{pairsmall}, the total value of the green and red items in $P_j$ is less than $3/4$ and hence, $\valu_i(\wcal(P_j)) \geq 1/4$ which means $|{\mathcal G}(P_j)| + 4.\valu_i(\wcal(P_j)) \geq 2$.

Since Inequality \eqref{newway} holds for every partition $P_j \in R_1$, we have:

$$\sum_{P_j \in R_1} \big(|{\mathcal G}(P_j)| + 4.\valu_i(\wcal(P_j))\big) \geq 2 |R_1|$$
Therefore,
 $$|{\mathcal G}(R_1)| + 4.\valu_i(\wcal(R_1)) \geq 2|R_1|$$ and hence, $$\valu_i(\wcal(R_1)) \geq  (2|R_1| - |{\mathcal{G}}(R_1)|)\cdot 1/4 $$
\end{proof}

\begin{lemma}
\label{lowerwhite4}
For $R_2$, we have: $$ \valu_i(\wcal(R_2)) \geq (|R_2| - |{\mathcal{G}}(R_2)|)\cdot 1/4 $$
\end{lemma}
\begin{proof}
Let $P_j$ be a partition in $R_2$. First, we show the following inequality holds:
\begin{equation}
\label{gwbound2}
4\valu_i(\wcal(P_j)) + |{\mathcal G}(P_j)| \geq 1
\end{equation}

By the definition of $R_2$, degree of $v_j$ is $2$. Therefore, $P_j$ contains two red items. Note that the degree of the partitions in $B_{01}$ that contain blue items is at least $3$. Thus, $P_j$ contains no blue items. By Lemma \ref{pairsmall}, the total value of the red items in $P_j$ is less than $3/4$. The rest of the items in $P_j$ are either green or white. If $P_j$ contains a green item, then Inequality \eqref{gwbound2} holds. On the other hand, if $P_j$ contains no green items, then $\valu_i(\wcal(P_j)) \geq 1/4$ and hence, $4\valu_i(\wcal(P_j)) \geq 1$. Therefore, Inequality \eqref{gwbound2} holds in both cases. 

Summing up Inequality \eqref{gwbound2} for all the partitions in $R_2$, we have:

$$
\sum_{P_j \in R_2} 4\valu_i(\wcal(P_j)) + |{\mathcal G}(P_j)| \geq |R_2|
$$
which means:
$$
4\valu_i(\wcal(R_2)) + |{\mathcal G}(R_2)| \geq |R_2|
$$
That is:
$$
\valu_i(\wcal(R_2)) \geq \big(|R_2| - |{\mathcal G}(R_2)|\big)\cdot 1/4  
$$
\end{proof}

Putting together Lemmas \ref{lowerwhite1},\ref{lowerwhite2},\ref{lowerwhite3}, and \ref{lowerwhite4} we obtain the following lower bound on the valuation of $\agent_i$ for all white items:
\begin{equation}\label{w1}
\begin{split}
\valu_i(\wcal(\items')) & = \valu_i(\wcal(B_{00})) + \valu_i(\wcal(B_{01})) + \valu_i(\wcal(B_{10}))\\
& \geq \bigg(3|B_{00}| - |{\mathcal{G}}(B_{00})|\bigg)\cdot 1/4 + \bigg(2|B_{10}| - |{\mathcal{B}}(B_{10})| - |{\mathcal{G}}(B_{10})|\bigg)\cdot 1/4 \\ & \hspace{10pt}+ \bigg(2|R_1| - |{\mathcal{G}}(R_1)|\bigg)\cdot 1/4 + \bigg(|R_2| - |{\mathcal{G}}(R_2)|\bigg)\cdot 1/4\\
&= \bigg((3|B_{00}| + 2|B_{10}| + 2|R_1| + |R_2| - |{\mathcal{B}}(B_{10})|) - \big(|{\mathcal{G}}(B_{00})| + |{\mathcal{G}}(B_{10})| + |{\mathcal{G}}(R_1)| + |{\mathcal{G}}(R_2)|\big) \bigg)\cdot 1/4\\
& \geq \bigg((3|B_{00}| + 2|B_{10}| + 2|R_1| + |R_2| )- |{\mathcal{B}}(B_{10})| -  |{\mathcal{G}}(\items')| \bigg)\cdot 1/4
\end{split}
\end{equation}
where $|{\mathcal{G}}(\items')|$ is the total number of green items. 



The items in $\wcal(\items')$ are either allocated to an agent during the second phase, or are still in $\fitems$. Let $\wcal_2$ be the white items that are allocated to an agent during the second phase. We have: 
\begin{equation}
\valu_i(\wcal(\items')) = \valu_i(\wcal_2) + V_i(\fitems)
\end{equation}
Now, we present an upper bound on the value of $\valu_i(\wcal_2)$. First, note that the number of agents in $\satagents \setminus \satagents^r_1$ is $n'$. Each of these $n'$ agents has two sets $\firstset_j$ and $\secondset_j$, that leaves us $2n'$ sets. Since $\secondset_i = \emptyset$ we know that at least one of these sets is empty. Moreover, of all these $|{\mathcal{G}}(\items')|$ sets contain a single green item and $|{\mathcal{B}}(B_{10})| + |E_{01}|$ of the sets contain either a single blue item, or a pair of red items (recall that each edge of $G_{01}$ refers to a blue item or a pair of red items). Therefore, the number of the sets that contain only white items is at most:
$$2n' - 1 - |{\mathcal{G}}(\items')| - |{\mathcal{B}}(B_{10})| - |E_{01}|$$

By Lemmas \ref{prvalue} and \ref{c3fsmall}, the value of every set with white items to $\agent_i$ is less than $2\epsilon_i<1/4$ and hence:
\begin{equation}
\label{w_2}
\valu_i(\wcal_2) \leq (2n' - 1 - |{\mathcal{G}}(\items')| - |{\mathcal{B}}(B_{10})| - |E_{01}|)\cdot 1/4
\end{equation}
Subtracting the lower bound obtained for $\valu_i(\wcal(\items'))$ in \eqref{w1} from the upper bound for $\valu_i(\wcal_2)$ in $\eqref{w_2}$ gives us a lower bound on the value of $\fitems$:
\begin{equation}\label{bachekhoshgel}
\begin{split}
\valu_i(\fitems) &= \valu_i(\wcal(\items')) - \valu_i(\wcal_2)\\
 &\geq \bigg((3|B_{00}| + 2|B_{10}| -|{\mathcal{B}}(B_{10})| + 2|R_1| + |R_2|) -  |{\mathcal{G}}(\items')| \bigg)\cdot 1/4 - \valu_i(\wcal_2)\\
 &\geq \bigg((3|B_{00}| + 2|B_{10}| - |{\mathcal{B}}(B_{10})| + 2|R_1| + |R_2|) -  |{\mathcal{G}}(\items')| \bigg)\cdot 1/4\\
 & \qquad - \bigg(2n' - 1 - |{\mathcal{G}}(\items')| - |{\mathcal{B}}(B_{10})| - |E_{01}|\bigg)\cdot 1/4 \\
&= \bigg(3|B_{00}|+2|B_{10}| + 2|R_1| + |R_2| -2n' + 1 + |E_{01}| \bigg)\cdot 1/4 \\
&= \bigg(2|B_{00}|+2|B_{10}| + |B_{00}| + |E_{01}| + 2|R_1| + |R_2| - 2n' + 1\bigg) \cdot 1/4 
\end{split}
\end{equation} 
Next we provide Lemmas \ref{ebound}, \ref{B00size}, and \ref{Esize} to complete the proof.
\begin{lemma}
\label{ebound}
$|B_{00}| \geq |E_{01}| - |B_{01}|$ 
\end{lemma}
\begin{proof}
First, note that $|B_{00}| + |B_{10}| + |B_{01}|=n'$. Moreover we have $|{\mathcal B}(B_{10})| + |E_{01}| \leq n'$. To show this Lemma, note that each edge in $G_{01}$ corresponds to the first set of an agent in $\satagents \setminus \satagents_1^r$. Also, every blue item in $B_{10}$ corresponds to the first set of an agent in $\satagents \setminus \satagents_1^r$. Therefore, the total number of the agents must be more than this number. By the definition of $B_{10}$, we know that $|{\mathcal B}(B_{10})| \geq |B_{10}|$. Therefore, we have: 

\begin{equation}
\begin{split}
|B_{00}| + |B_{10}| + |B_{01}| &\geq  |{\mathcal B}(B_{10})| + |E_{01}|\\
 &\geq |(B_{10})| + |E_{01}|\\
\end{split}
\end{equation} 
This means:
$$ |B_{00}| \geq |E_{01}| - |B_{01}| $$
\end{proof}

\begin{lemma}
\label{Esize}
$|E_{01}| \geq 3/2 \sum_{j \geq 3}|R_j| + |R_2| + |R_1|/2$ 
\end{lemma}
\begin{proof}
$|E_{01}| = \frac{\sum_{v_j \in V_{01}} d(v_j)}{2} = \frac{\sum_j j|R_j|}{2} \geq 3/2 \sum_{j \geq 3}|R_j| + |R_2| + |R_1|/2.$
\end{proof}

\begin{lemma}
\label{B00size}
$|B_{00}| \geq \frac{\sum_{j \geq 3}|R_j| - |R_1|}{2}$
\end{lemma}
\begin{proof}
By Lemma \ref{ebound}, $|B_{00}| \geq |E_{01}| - |B_{01}| $. Furthermore, by Lemma \ref{Esize}, $$|E_{01}| \geq 3/2 \sum_{j \geq 3}|R_j| + |R_2| + |R_1|/2.$$ By these two inequalities, we have:
\begin{equation}
\label{Ebd}
|B_{00}| \geq 3/2 \sum_{j \geq 3}|R_j| + |R_2| + |R_1|/2 - |B_{01}| 
\end{equation}
Also, since there is a one-to-one correspondence between $B_{01}$ and $V_{01}$, $|B_{01}| = |V_{01}|$ holds. By the definition of $R_j$, we have: 
\begin{equation}
\label{Vbd}
|V_{01}| = \sum_j |R_j| 
\end{equation}

By replacing the value obtained for $B_{01}$ from \eqref{Vbd} into Inequality \eqref{Ebd}, we have:
\begin{equation}
\begin{split}
|B_{00}| &\geq 1/2 \sum_{j \geq 3}|R_j| - |R_1|/2 \\
& = \frac{\sum_{j \geq 3}|R_j| - |R_1|}{2}.
\end{split} 
\end{equation}
  
\end{proof}



By applying Lemmas \ref{B00size} and \ref{Esize} to Inequality \eqref{bachekhoshgel} we have:
\begin{equation*}
\begin{split}
\valu_i(\fitems) & = \bigg(2|B_{00}|+2|B_{10}| + |B_{00}| + |E_{01}| + 2|R_1| + |R_2| - 2n' + 1\bigg) \cdot 1/4\\
&\geq \bigg(2|B_{00}|+2|B_{10}| + \frac{\sum_{j \geq 3}|R_j| - |R_1|}{2} + 3/2 \sum_{j \geq 3}|R_j| + |R_2| + |R_1|/2 + 2|R_1| + |R_2| - 2n' + 1\bigg)\cdot 1/4 \\
&= \bigg(2|B_{00}|+2|B_{10}| + \sum_{j \geq 3} 2|R_j| + 2|R_2| + 2|R_1| - 2n' + 1 \bigg)\cdot 1/4
\end{split}
\end{equation*} 
Finally, note that $\sum_{j \geq 3} 2|R_j| + 2|R_2| + 2|R_1| = 2|V_{01}| = 2|B_{01}|$. This, together with the fact that $|B_{00}| + |B_{01}| + |B_{10}| = n'$, yields $\valu_i(\fitems) \geq (2n' - 2n' + 1)\cdot 1/4$. This means $\valu_i(\fitems) \geq 1/4$ which is a contradiction since $\fitems$ is feasible for $\agent_i$.
\begin{comment}
\begin{lemma}
\label{capability}
Total items in $cal M$ is enough to fill in $2n$ slots. 
\end{lemma}

\begin{proof}

The blue items, can singly fill in a slot. Also, each pair of red items can fill a slot. For simplicity, we assume that each red item can fill a half slot. 

Now, consider the items from the viewpoint of $\agent_i$. Each partition in $B_{10}$ is enough to fill in $2|B_{10}|$ slots. Also, all the partitions $b_i \in B_{01}$ with $b_i \in R_1$ is enough to fill in $ 2.5$ slots. Furthermore every partition $b_i \in B_{01}$ with $b_i \in R_3$ is capable of filling $1.5$ slots. By lemma \ref{B00size}, we know that the number of such partitions is less than $\frac{B_{00}}{2}+ R_1$. On the other hand, each partition in $B_{00}$ is capable of filling $3$ slots. So, total number of slots that can be filled by $\cal M$ is at least:
\begin{align*} 
2|B_{10}| + 3|B_{00}| + 2.5|R_1| + 1.5|R_3| \\ > 2|B_{10}|+ 2|B_{00}| + 2|R_1| + 2 |R_3| - (2*\frac{|R_3|}{2} - \frac{|B_{00}|}{2} + \frac{|R_1|}{2}) \\ > 2|B_{10}|+ 2|B_{00}| + 2|R_1| + 2 |R_3| = 2n
\end{align*}

 So, total items in $\cal M$ is capable of filling $2n$ slots, from the viewpoint of $\agent_i$. 
\end{proof}

So, by lemma \ref{capability}, $\agent_i$ can fill in $2n$ slots, that contradicts the deadlock situation. 
\end{comment}
\end{proof}

