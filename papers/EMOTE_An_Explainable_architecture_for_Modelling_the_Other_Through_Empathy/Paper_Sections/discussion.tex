The hyperparameter $\delta \in [0,1]$ (Equation \ref{eqn:Loss_Function}) balances the trade-off between the two loss terms. As $\delta$ approaches 1, empathetic state $s_{e}$ becomes similar to original state $s_{i}$, resulting in the inferred rewards of the independent agent being the same as the learning agent's rewards. In practice the bottleneck imposed by the second model ($Q_{learn}$) led to the finding that high $\delta$ values 
%\st{(inducing $s_{e}$ closer to $s_{i}$)} 
were beneficial. In particular, it allowed $s_e$ to reproduce common features such as walls and floors, contributing to better performances. Experimental settings for $\delta$ are in the Supplementary.

The multi-objective nature of the loss term in Equation \ref{eqn:Loss_Function}, can make learning stability hard to achieve (as it is not possible for both loss terms to reach 0 concurrently), and thus could be further improved. Additionally, although there are no limits on the number independent agents modelled, a new Imagination Network would be required for each, which could hinder scalability. Future work can look to address this limitation, perhaps by exploring transfer learning based solutions where multiple imagination transformations are learned with the same network. % Lastly, our assumptions of analogy, full trajectory information and fixed policy for the modelled agent could prevent our work from being applied in certain scenarios.} 
Future work could also look to extend EMOTE to situations where all agents are being trained, and where states are only partially observable. 
%\manisha{\st{and explore ways in which the assumption of analogy could be relaxed.}}\thommen{It is possible to do it if the agents are not analogous in any way?}
%, or are following stochastic policies, 
%\st{which would eschew our current assumption of independent agents operating under fixed policies.}

%to settings where there are at least some state features in the environment which elicit a similar response from the learning agent and an independent. This condition does constrain our work.