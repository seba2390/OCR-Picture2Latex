% This must be in the first 5 lines to tell arXiv to use pdfLaTeX, which is strongly recommended.
\pdfoutput=1
% In particular, the hyperref package requires pdfLaTeX in order to break URLs across lines.

\documentclass[11pt]{article}

% Remove the "review" option to generate the final version.
\usepackage[review]{emnlp2021}

% Standard package includes
\usepackage{times}
\usepackage{latexsym}
\usepackage{graphicx}
% \usepackage{cellspace}
\usepackage{tabularx}
\usepackage{graphicx}
\usepackage{amsmath}
\usepackage{siunitx}
\sisetup{output-exponent-marker=\ensuremath{\mathrm{e}}}

\usepackage{xcolor, soul}
\definecolor{beaublue}{rgb}{0.74, 0.83, 0.9}
\sethlcolor{beaublue}


% For proper rendering and hyphenation of words containing Latin characters (including in bib files)
\usepackage[T1]{fontenc}
% For Vietnamese characters
% \usepackage[T5]{fontenc}
% See https://www.latex-project.org/help/documentation/encguide.pdf for other character sets

% This assumes your files are encoded as UTF8
\usepackage[utf8]{inputenc}

% This is not strictly necessary, and may be commented out,
% but it will improve the layout of the manuscript,
% and will typically save some space.
\usepackage{microtype}


\usepackage{colortbl}
\usepackage[T1]{fontenc}
\usepackage{color,soul}
\usepackage{booktabs}
\usepackage{multirow}
\usepackage{multicol}
\usepackage{diagbox}
\usepackage{booktabs}
\usepackage{amssymb}
\usepackage{xspace}
\usepackage{pifont}
\usepackage{enumitem}

\newcommand{\cmark}{\ding{51}\xspace}%
\newcommand{\xmark}{\ding{55}\xspace}%
\newcommand{\numData}{XXX\xspace}
\newcommand{\dataset}{Dataset Name}
\newcommand{\vs}[1]{\textcolor{blue!40}{VS: #1}}
\newcommand{\tc}[1]{\textcolor{red!40}{TC: #1}}
\newcommand{\lt}{\ensuremath <}
\newcommand{\gt}{\ensuremath >}
\newcolumntype{P}[1]{>{\raggedright\arraybackslash}p{#1}}
\newcommand{\specialcell}[2][c]{%
	\begin{tabular}[#1]{@{}c@{}}#2\end{tabular}}
\newcommand{\specialcellleft}[2][l]{%
	\begin{tabular}[#1]{@{}l@{}}#2\end{tabular}}

\definecolor{ForestGreen}{RGB}{34,139,34}



% If the title and author information does not fit in the area allocated, uncomment the following
%
%\setlength\titlebox{<dim>}
%
% and set <dim> to something 5cm or larger.

\title{It's not Rocket Science:\\Interpreting Figurative Language in Narratives}

\author{Tuhin Chakrabarty$^1$\thanks{~~Work done during internship at AI2.}~~~~~Vered Shwartz$^{2,3,4}$~~~~~Yejin Choi$^{2,3}$ \\
 $^1$Columbia University\\
 $^2$Allen Institute for Artificial Intelligence\\
 $^3$Paul G. Allen School of Computer Science \& Engineering, University of Washington\\  
 $^4$University of British Columbia\\
 {\tt\small tuhin.chakr@cs.columbia.edu, vshwartz@cs.ubc.ca, yejinc@allenai.org} \\
}

\begin{document}
\maketitle
\begin{abstract}

\begin{abstract} %  


% Revised version from Noah's version
Representation learning for text via pretraining a language model on a large corpus has become a standard starting point for building NLP systems.  This approach stands in contrast to autoencoders, also trained on raw text, but with the objective of learning to encode each input as a vector that allows full reconstruction.  Autoencoders are attractive because of their latent space structure and generative properties. We therefore explore the construction of a sentence-level autoencoder from a pretrained, frozen transformer language model. We adapt the masked language modeling objective as a generative, denoising one, while only training a sentence bottleneck and a single-layer modified transformer decoder.
We demonstrate that the sentence representations discovered by our model achieve better quality than previous methods that extract representations from pretrained transformers on text similarity tasks, style transfer (an example of controlled generation), and single-sentence classification tasks in the GLUE benchmark, while using fewer parameters than large pretrained models.\footnote{Our code is available at: \url{https://github.com/ivanmontero/autobot}} 



% % Noah's 5/10 version with comments
% Representation learning for text via pretraining a language model on a large corpus has become a standard starting point for building NLP systems.  This approach stands in contrast to autoencoders, also trained on raw text, but with the objective of learning to encode each inputs as a vector that allows full reconstruction.  Autoencoders are attractive because of their generative latent space properties. We therefore explore the construction of a sentence-level autoencoder from a pretrained, frozen transformer language model. We adapt the masked language modeling objective as a generative, denosing one, We demonstrate that the sentence representations discovered by our method are a \nascomment{nail down claims about effiency, it's not clear to me right now} and perform competitively with the state of the art \nascomment{check that claim!} on text similarity tasks, style transfer (an example of controlled generation), and sentence classification tasks in the GLUE benchmark.

% \nascomment{fill this in}. (to keep computational cost low) 

% \nascomment{one thing not clear from the paper and possibly worth raising even here:  what data gets used to learn the autoencoder?  also, we need to nail down terminology.  in some places you say that the autoencoder is ``pretrained'' but I think that's confusing since we start from a pretrained model.  do we want to say that we finetune the autoencoder?} 


% old version commented out 5/10 by Noah
%Methods for obtaining unsupervised language representations have gained much attention recently for their performance on downstream tasks, and have largely relied on partial reconstruction of the input though masked language modeling or next token prediction through language modeling.
%Autoencoders, trained on full reconstruction of the input through a single fixed length, bottlenecked representation, have not been extensively studied for pretraining sentence-level representations, especially with transformers.
%To this end, we propose \textsc{Autobots}, a new class of text autoencoders with a sentence bottleneck derived from pretrained transformers on unlabeled data.
%Our training objective is to reconstruct sentences fully from a learned bottleneck representation while keeping the underlying pretrained model fixed.
%We demonstrate that the resulting sentence representations outperform previous methods on text similarity tasks while being parameter efficient, and can also be used for controlled generation tasks such as style transfer.
%Notably our method maintains the performance of the pretrained language model on other supervised  downstream tasks.







% Pretrained language models based on transformers have been shown to produce sub-optimal sentence representations for tasks requiring text similarity \cite{Reimers2019SentenceBERT}.
% Previous methods that ameliorate this issue have introduced auxiliary sentence similarity objectives that rely on annotated data and compose sentence representations with non-parametric pooling that exhibits limited desirable properties and is specific to text similarity tasks.
% To this end, we propose \textsc{Autobots}, a new class of text autoencoders with a sentence bottleneck derived from pretrained transformers on unlabeled data.
% Our training objective is to to reconstruct sentences fully from a learned bottleneck representation as opposed to partially from variable-length representations in existing masked language modeling objectives.
% We demonstrate that our resulting sentence representations outperform previous methods on text similarity tasks while being parameter efficient, and can also be used for controlled generation tasks such as style transfer.
% Notably our method maintains the performance of the pretrained language model on other supervised  downstream tasks.

% \nikos{new take on the abstract, please check. I tried to emphasize on the sentence BERT method to have a clear message about our goal of the paper. There are some nice points in the previous abstract that should probably be incorporated in the introduction but it would be crucial not to make big claims about pretraining in general since it would be hard to compete with big LM papers. }


% ===== nikos =====
% Pretrained language models based on transformers have been shown to produce sub-optimal sentence representations for tasks requiring text similarity \cite{Reimers2019SentenceBERT}.
% Previous methods that ameliorate this issue have introduced auxiliary sentence similarity objectives that rely on annotated data and compose sentence representations with non-parametric pooling that exhibits limited desirable properties and is specific to text similarity tasks.
% To this end, we propose \textsc{Autobots}, a new class of text autoencoders with a sentence bottleneck derived from pretrained transformers on unlabeled data.
% Our training objective is to to reconstruct sentences fully from a learned bottleneck representation as opposed to partially from variable-length representations in existing masked language modeling objectives.
% We demonstrate that our resulting sentence representations outperform previous methods on text similarity tasks while being parameter efficient, and can also be used for controlled generation tasks such as style transfer.
% Notably our method maintains the performance of the pretrained language model on other supervised  downstream tasks.
% =================





% Methods for obtaining unsupervised language representations have gained much attention recently for their performance on downstream tasks, and have largely relied on partial reconstruction of the input though masked language modeling or next token prediction through language modeling. Autoencoders, trained on full reconstruction of the input through a single fixed length, bottlenecked representation, have not been extensively studied for pretraining sentence-level representations, especially with transformers.
%In this paper, we propose the \textit{Transformer Autoencoder} by converting the architecture of \citet{vaswani17} to an autoencoder that uses attention to bottleneck the input to a single, fixed vector from which the input can be fully reconstructed. Leveraging the encoder advancements of \citet{devlin-etal-2019-bert}, we introduce \textbf{BARNEY}, which stands for \textbf{B}ERT \textbf{A}ggregated \textbf{R}epresentations by Reduci\textbf{n}g and R\textbf{e}constructing Full\textbf{y}, and demonstrate the significant performance of this novel pretraining objective in downstream classification, sentence representation, reconstruction, and controlled generation tasks.

\end{abstract}

% Autoencoders have 

% Unsupervised pretraining has gained much attention recently 
% We propose the transformer

% Methods for obtaining unsupervised language representations have gained much attention recently for their performance on downstream tasks, and have largely relied on partial reconstruction of the input though masked language modeling or next token prediction through language modelling.

% Text autoencoders obtain an unsupervised sentence-level representation by bottlenecking with reconstruction abilities, but largely rely on recurreence in the decoder.

% In this paper, we propose the \textit{Transformer Autoencoder} model by converting the architecture of \citet{vaswani17} to an autoencoder that uses attention to bottleneck the input to a single, fixed vector from which the input can be fully reconstructed. Leveraging the encoder advancements of \citet{devlin-etal-2019-bert}, we introduce textbf{BARNEY}, which stands for \textbf{B}ERT \textbf{A}ggregated \textbf{R}epresentations by Reduci\textbf{n}g and R\textbf{e}constructing Full\textbf{y}, to demonstrate the performance of this novel pretraining objective in downstream classification, sentence representation, reconstruction, and controlled genration tasks.

% by leveraging the encoder advancements of \citet{devlin-etal-2019-bert} in a transformer auto


% effectively enriching the resulting sentence representations with reconstructive properties. Concretely, we introduce a context attention bottleneck after the original encoder, and a modify the decoder to perform full reconstruction of the input conditioned on the bottleneck representation.
 
% Currently langauge representat
% 
 

% VERSION 1:

% % to learn 
% % \textit{BERT isn't an autoencoder, but BARNEY is!}

% Autoencoders have gained much attention for their unsupervised ability to reduce variable length sequences to a single latent space with reconstruction capabilities, yet current architectures rely heavily on recurrence. Similar to autoencoders, current state-of-the-art pretrained transformer models produce variable-length token represenations by learning to partially reconstruct their input, yet rely on a special token to produce sentence representations for most downstream natural language understanding tasks requiring a single vector.


% % Current state-of-the-art pretrained language models produce variable-length token representations by learning to partially reconstruct the input, and rely on a special token to produce sentence representations for most downstream natural language understanding tasks requiring a single vector. 
% %\nikos{we may want to mention that it's not possible to produce the full sentence based on cls alone} 
% %\nikos{the token level representations are used for word-level tasks though, so we have to make sure we don't overclaim here.}

% In this paper, we propose the \textit{Transformer Autoencoder} model that results from  converting the architecture of \citet{vaswani17} to an autoencoder that projects the input to a single, fixed vector from which the input can be fully reconstructed, effectively enriching the resulting sentence representations with reconstructive properties. 
% Concretely, we introduce a context attention bottleneck after the original encoder, and a modify the decoder to perform full reconstruction of the input %(instead of partial) 
% conditioned on the bottleneck representation.
% %to the "encoder-decoder" layer in the decoder stacks.
% %
% %We preserve the parallelism in the original transformer by analyzing dot-product attention's decomposition in the many-to-one and one-to-many settings. \nikos{Not sure I fully get this point. }
% %
% %\textbf{TAE} can be pre-trained on unlabeled corpora to produce a single sequence-level representation. \nikos{this point could be skipped}

% % Lastly, we develop an example framework called \textbf{BARNEY} based on transformer autoencoders for pretraining akin to BERT \citet{devlin-etal-2019-bert}
% Lastly, we introduce \textbf{BARNEY}, which stands for \textbf{B}ERT \textbf{A}ggregated \textbf{R}epresentations by Reduci\textbf{n}g and R\textbf{e}constructing Full\textbf{y}, which both leverages advancements of \citet{devlin-etal-2019-bert} and introduces a novel pretraining objective for transformers. We evaluate its sentence representations on downstream classification, reconstruction and controlled generation tasks.\footnote{Our code will be made available at [link]} 


% %, which stands for \textbf{B}idirectional \textbf{A}ggregated Transformer \textbf{R}epresentations by Reduci\textbf{n}g and R\textbf{e}constructing Full\textbf{y}. \nikos{is it ok if we skip the abbreviations until we decide the proper framing?}
% % and  we evaluate its sentence representations on downstream classification, reconstruction and controlled generation tasks.\footnote{Our code will be made available at [link]} 
\end{abstract}

\section{Introduction}
\label{sec:intro}
% !TEX root = 0-qqQQmain.tex

\section{Introduction}


The success of the collider particle physics program, whose main
player today is the Large Hadron Collider (LHC) at CERN, relies
heavily on our ability to model with high precision and accuracy the
scattering of high energetic protons in Quantum Chromodynamics (QCD).
Thanks to asymptotic freedom and the factorization properties of QCD,
this intrinsically non-perturbative problem can be treated with
perturbative methods, supplemented by non-perturbative information about
the distribution of partons in the proton. Within this picture, an important
role is played by higher order perturbative QCD calculations, which allow
for a reliable and precise description of a wide range of collider processes
and observables.

Thanks to a concerted effort in the high-energy community over the
last few years, it is currently possible to compute predictions for
many interesting reactions to second order in the strong coupling
expansion, $i.e.$ to what is usually referred to as
next-to-next-to-leading order (NNLO). 
This has required, on the one hand, 
major advances in computational techniques
for multi-loop scattering amplitudes~\cite{Tkachov:1981wb,Chetyrkin:1981qh,Hodges:2009hk,Gluza:2010ws,Ita:2015tya,Larsen:2015ped,Bohm:2017qme,Badger:2016uuq,vonManteuffel:2014ixa,Peraro:2016wsq,Peraro:2019svx,Guan:2019bcx,Pak:2011xt,Abreu:2019odu,Heller:2021qkz,Kotikov:1990kg,Bern:1993kr,Remiddi:1997ny,Gehrmann:1999as,Papadopoulos:2014lla,Dixon:1996wi,Henn:2013pwa,Primo:2016ebd,Goncharov,Remiddi:1999ew,Goncharov:2001iea,Goncharov:2010jf,Brown:2008um,Ablinger:2013cf,Panzer:2014caa,Duhr:2011zq,Duhr:2012fh,Duhr:2019tlz},
which, notably, have recently made it possible to compute
various $2 \to 3$ processes up to two loops in QCD~\cite{Badger:2017jhb,Abreu:2017hqn,Abreu:2018aqd,Abreu:2018zmy,Abreu:2018jgq,Abreu:2019rpt,Abreu:2020cwb,Chicherin:2018yne,Chicherin:2019xeg,Chawdhry:2020for,DeLaurentis:2020qle,Chawdhry:2018awn,Abreu:2020xvt,Agarwal:2021grm,Badger:2021nhg,Abreu:2021fuk,Agarwal:2021vdh,Chawdhry:2021mkw,Badger:2021imn,Gehrmann:2015bfy,Papadopoulos:2015jft,Gehrmann:2018yef,Chicherin:2018mue,Chicherin:2020oor}.
On the other hand, the use of these amplitudes to perform phenomenological
studies for the relevant processes at NNLO~\cite{Chawdhry:2019bji,Kallweit:2020gcp,Chawdhry:2021hkp,Czakon:2021mjy} has required the
development of so-called subtraction or slicing frameworks~\cite{GehrmannDeRidder:2005cm,Czakon:2010td,Caola:2017dug,Magnea:2018hab,Herzog:2018ily,DelDuca:2016ily,Cacciari:2015jma,Catani:2007vq,Gaunt:2015pea,Boughezal:2015dva}
to properly deal with the intricate IR divergences that appear in QCD reactions. 

Beyond NNLO, predictions at third order in the
perturbative couplings, i.e.\ at N$^3$LO, are
known only for a handful of important LHC processes~\cite{Anastasiou:2015vya,Duhr:2019kwi,
Dulat:2018bfe,Mistlberger:2018etf,Dreyer:2016oyx,Dreyer:2018qbw,
Billis:2021ecs,Chen:2021isd,Chen:2021vtu}. 
In particular, N$^3$LO results are currently available only for reactions that
require at most three-point three-loop integrals.
Given the remarkable success of the
experimental program at the LHC, it is desirable to extend these
calculations to more complex processes. A particularly interesting
one is di-jet production.  In fact, jets are ubiquitous at
hadron colliders, so understanding their dynamics is of great
interest.
Moreover, di-jet production is the first massless $2\to2$
process that has a non-trivial colour structure. This makes it an
ideal ground for studying the structure of perturbative QCD. For
example, it is by now well known that when four or more coloured
partons interact, starting at the three-loop order, non-trivial
colour correlations can affect the pattern of IR divergences,
generating new structures~\cite{Almelid:2015jia} beyond the standard dipole
formula~\cite{Sterman:2002qn,Aybat:2006wq,Aybat:2006mz,Becher:2009cu,Gardi:2009qi,Becher:2009qa,Dixon:2009gx}. Also, the
non-trivial colour structure may create subtle violations of the
factorization framework that is at the very core of theoretical
predictions at hadronic
colliders~\cite{Catani:2011st,Forshaw:2012bi,Forshaw:2006fk,Becher:2021zkk}.
This makes jet production at hadronic colliders an extremely interesting
process to investigate at higher orders. 

A key ingredient for the study of jet production at N$^3$LO is
provided by the virtual three-loop corrections to the scattering
amplitudes for the production of two jets in massless QCD.  Modulo
crossings, there are three main partonic channels that need to be
computed: four-gluon scattering, the scattering of two quarks and two
gluons, and the scattering of four quarks.  All ingredients necessary
for the calculation of the two-loop QCD corrections to these processes have
been known for a long time~\cite{Smirnov:1999gc,Tausk:1999vh,Glover:2001af,Anastasiou:2002zn,Glover:2003cm}, which 
have made it possible to compute the relevant scattering amplitudes~\cite{Anastasiou:2000kg,Bern:2003ck,Glover:2004si,DeFreitas:2004kmi}. Also, in view of extending these
calculations to three loops, results for the two loop helicity
amplitudes up to order $\epsilon^2$ have been
obtained~\cite{Ahmed:2019qtg}.  For what concerns the three loop
results, instead, the relevant master integrals have been computed in
ref.~\cite{Henn:2020lye}, and have then been used to obtain the first
three loop results for $2 \to 2$ scattering amplitudes in
supersymmetric theories~\cite{Henn:2016jdu,Henn:2019rgj}.  More
recently also the first three loop corrections to the production of
two photons in full QCD have been obtained~\cite{Caola:2020dfu}.

In this paper, we move one step further and consider one of the three classes of partonic processes  listed above, 
namely the scattering of four massless quarks. This particular process is interesting not only
because it allows us, for the first time, to check the full structure of IR divergences at three loops in QCD, 
but also because it involves two external spinor structures. In fact, this property makes the use of the standard form factors
method for the calculation of the helicity amplitudes particularly cumbersome, due to the fact that the  $\gamma$-algebra does not
close in  $d$ space-time dimensions. For the calculation of the helicity amplitudes we then make use of
a different approach, recently described in refs~\cite{Peraro:2019cjj,Peraro:2020sfm}, which allows us to calculate the helicity amplitudes
in a simpler way, corresponding to the 't Hooft-Veltman scheme (tHV)~\cite{tHooft:1972tcz} for the processes considered here.\footnote{See also ref.~\cite{Heller:2020owb} for an application of similar ideas in the case of a chiral theory, and ref.~\cite{Chen:2019wyb} for an alternative approach.} 
In doing this, we also expose some
subtleties in the usual approach to compute helicity amplitudes in 
tHV with the standard form factor method.
The rest of the paper is organised as follows. 
We start in section~\ref{Kinematics} by establishing the notation for the calculation of the
fundamental partonic channel $q\bar{q} \to Q \bar{Q}$, from which all other channels can by obtained by crossing.
We continue in section~\ref{The Scattering Amplitude},
where we describe the colour and tensor decomposition  of the scattering amplitude, and show how to 
efficiently compute the helicity amplitudes without considering evanescent Lorentz structures.
In section~\ref{computation} we provide details on our computational set up,
and in section~\ref{subtraction} we discuss the renormalisation and the infrared structure of the three-loop scattering amplitudes.
In section~\ref{results} we discuss our final results for the main partonic channel.
In section~\ref{extra_results} we then explain how to obtain all other partonic channels from our calculation, both for quarks of equal and different flavour.
Finally we conclude in section~\ref{conclusions}.  In 
appendices~\ref{sec:appB} and~\ref{sec:appA}
we provide some details on the structure of infrared divergences up at three loops, in particular focusing on the explicit derivation of the quadrupole terms, which appear for the first time with the scattering of at least four coloured partons at three loops.
In appendix~\ref{sec:appC}, we review the analytical continuation of the amplitudes to different regions of phase space.


\section{Background}
\label{sec:bg}
\section{Background}
\label{sec:background}

\subsection{Definitions}
\label{sec:background_def}
Diffusion models~\cite{sohl2015deep,ho2020denoising} transform complex data distribution $p_{data}(x)$ into simple noise distribution $\mathcal{N}(0,\mathbf{I})$ and learn to recover data from noise.
The \textit{diffusion process} of diffusion models gradually corrupts data $x_0$ with predefined noise scales $0<\beta _1, \beta _2, ..., \beta_T <1$, indexed by time step $t$.
Corrupted data $x_1,...,x_T$ are sampled from data $x_0\sim p_{data}(x)$, with a diffusion process, which is defined as Gaussian transition:
\begin{equation}\label{eq:forward}
q(x_{t}|x_{t-1})=\mathcal{N}(x_{t};\sqrt{1-\beta _{t}}x_{t-1},\beta _{t}\mathbf{I}).
\end{equation}
Noisy data $x_t$ can be sampled from $x_0$ directly:
\begin{equation}\label{eq:closed}
x_t=\sqrt{\alpha _{t}}x_{0} + \sqrt{1-\alpha _{t}}\epsilon,
\end{equation}
where $\epsilon\sim \mathcal{N}(0,\mathbf{I})$ and $\alpha_{t}:=\prod_{s=1}^t (1-\beta _{s})$. We note that data $x_0$, noisy data $x_1,...,x_T$, and noise $\epsilon$ are of the same dimensionality. 
To ensure $p(x_T)\sim \mathcal{N}(0,\mathbf{I})$ and the reversibility of the diffusion process~\cite{sohl2015deep}, one should set $\beta_t$ to be small and $\alpha_T$ to be near zero. To this end, Ho et al.~\cite{ho2020denoising} and Dhariwal et al.~\cite{dhariwal2021diffusion} use a linear noise schedule where $\beta_t$ increases linearly from $\beta_1$ to $\beta_T$. Nichol et al.~\cite{nichol2021improved} use a cosine schedule where $\alpha_t$ resembles the cosine function.

Diffusion models generate data $x_0$ with the learned \textit{denoising process} $p_\theta(x_{t-1}|x_t)$ which reverses the diffusion process of~\cref{eq:forward}. Starting from noise $x_T\sim \mathcal{N}(0,\mathbf{I})$, we iteratively subtract the noise predicted by noise predictor $\epsilon _\theta$:
\begin{equation}\label{eq:reverse}
x_{t-1}= \frac{1}{\sqrt{1-\beta _t}}(x_t-\frac{\beta_t}{\sqrt{1-\alpha _t}}\epsilon _{\theta}(x_t,t))+\sigma _t z,
\end{equation}
where $\sigma _t^2$ is a variance of the denoising process and $z\sim \mathcal{N}(0,\mathbf{I})$. Ho et al.~\cite{ho2020denoising} used $\beta _t$ as $\sigma _t^2$.


Recent work Kingma et al.~\cite{kingma2021variational} simplified the noise schedules of diffusion models in terms of \textit{signal-to-noise ratio} (SNR). SNR of corrupted data $x_t$ is a ratio of squares of mean and variance from~\cref{eq:closed}, which can be written as:
\begin{equation}\label{eq:snr}
\text{SNR}(t)=\alpha _{t}/(1-\alpha _{t}),
\end{equation}
and thus the variance of noisy data $x_t$ can be written in terms of SNR: $\alpha _t= 1 - 1/(1+\text{SNR}(t))$. We would like to note that SNR($t$) is a monotonically decreasing function.

\begin{figure*}[t!]
  \centering
  \includegraphics[width=1.0\linewidth]{figures/mode.png}
  \caption{\textbf{Information removal of a diffusion process.} (Left) Perceptual distance of corrupted images as a function of signal-to-noise ratio (SNR). Distances are measured between two noisy images either corrupted from the same image (blue) or different images (orange). 
  We averaged distances measured with 200 random triplets from CelebA-HQ. 
  Perceptually recognizable contents are removed when SNR magnitude is between $10^{-2}$ and $10^0$. %, which suggests that diffusion models learn rich visual contents by solving recovery tasks at corresponding noise levels. 
  (Right) Illustration of the diffusion process.}
  \label{fig:mode}
\end{figure*}

\subsection{Training Objectives}
\label{sec:objective}
The diffusion model is a type of variational auto-encoder (VAE); where the encoder is defined as a fixed \textit{diffusion process} rather than a learnable neural network, and the decoder is defined as a learnable \textit{denoising process} that generates data. Similar to VAE, we can train diffusion models by optimizing a variational lower bound (VLB), which is a sum of denoising score matching losses~\cite{vincent2011connection}: $L_{vlb}=\sum_t L_t$, where weights for each loss term are uniform. For each step $t$, denoising score matching loss $L_t$ is a distance between two Gaussian distributions, which can be rewritten in terms of noise predictor $\epsilon_\theta$ as:
\begin{align}\label{eq:vlb}
L_{t}=~&D_{KL}(q(x_{t-1}|x_t,x_0)~||~p_\theta(x_{t-1}|x_t))  \nonumber \\
=~&\mathbb{E}_{x_0,\epsilon}[\frac{\beta _t}{(1-\beta _t)(1
-\alpha _t)}||\epsilon-\epsilon_\theta(x_t, t)||^2].
\end{align}
Intuitively, we train a neural network $\epsilon _\theta$ to predict the noise $\epsilon$ added in noisy image $x_t$ for given time step $t$.

Ho et al.~\cite{ho2020denoising} empirically observed that the following simplified objective is more beneficial to sample quality:
\begin{equation}\label{eq:simple}
L_{simple}=\sum _t \mathbb{E}_{x_0,\epsilon}[||\epsilon-\epsilon_\theta(x_t, t)||^2].
\end{equation}
In terms of VLB, their objective is $L_{simple}=\sum_t\lambda _t L_t$ with weighting scheme $\lambda _t=(1-\beta _t)(1-\alpha _t)/\beta _t$. In a continuous-time setting, this scheme can be expressed in terms of SNR: 
\begin{equation}\label{eq:simple_snr}
\lambda_t = -1/\text{log-SNR}'(t)= -\text{SNR}(t)/\text{SNR}'(t),
\end{equation}
where $\text{SNR}'(t)=\frac{d\text{SNR}(t)}{dt}$. See appendix for derivations.

While Ho et al.~\cite{ho2020denoising} use fixed values for the variance $\sigma _t$, Nichol et al.~\cite{nichol2021improved} propose to learn it with hybrid objective $L_{hybird}=L_{simple}+c L_{vlb}$, where $c = 1e^{-3}$. They observed that learning $\sigma _t$ enables reducing sampling steps while maintaining the generation performance. 
We inherit their hybrid objective for efficient sampling and modify $L_{simple}$ to improve performance.


\subsection{Evaluation Metrics}
We use FID~\cite{heusel2017gans} and KID~\cite{binkowski2018demystifying} for quantitative evaluations.
FID is well-known to be analogous to human perception~\cite{heusel2017gans} and well-used as a default metric~\cite{karras2020training,dhariwal2021diffusion,stylegan,esser2021taming,parmar2021cleanfid} for measuring generation performances. KID is a well-used metric to measure performance on small datasets~\cite{karras2020training,karras2021alias,parmar2021cleanfid}. However, since both metrics are sensitive to the preprocessing~\cite{parmar2021cleanfid}, we use a correctly implemented library~\cite{parmar2021cleanfid}. We compute FID and KID between the generated samples and the entire training set. We measured final scores with 50k samples and conducted ablation studies with 10k samples for efficiency, following~\cite{dhariwal2021diffusion}. We denote them as FID-50k and FID-10k respectively.


\section{Data}
\label{sec:data}
\begin{table}[t]
\setlength{\tabcolsep}{2pt}
\scriptsize
\centering
\begin{tabular}{ll}
\toprule
\textbf{Idioms} & \textbf{Similes} \\ \midrule
% \begin{tabular}
any port in a storm & like a psychic whirlpool\\ 
been there, done that & like a moth-eaten curtain\\  
slap on the wrist & like a first date\\ 
no time like the present & like a train barreling of control\\ 
lay a finger on & like a sodden landscape of melting snow\\  
walk the plank & like a Bunsen burner flame\\  
curry favour & like a moldy old basement\\ 
not to be sneezed at & like a street-bought Rolex\\ 
no peace for the wicked & like an endless string of rosary beads\\

\bottomrule
\end{tabular}
\caption{Examples of idioms and similes present in the narratives in our datasets.}
\label{tab:example_expressions}
\end{table}

We build datasets aimed at testing the understanding of figurative language in narratives, focusing on idioms (Section~\ref{sec:dataidioms}) and similes (Section~\ref{sec:datasimile}). We posit that a model which truly understands the meaning of a figurative expression, like humans do, should be able to infer or decide what happens next in the context of a narrative. Thus, we construct a dataset in the form of the story-cloze test.

\subsection{Idioms}
\label{sec:dataidioms}

We compile a list of idioms, automatically find narratives containing these idioms, and then elicit plausible and implausible continuations from crowdsourcing workers, as follows.

\paragraph{Collecting Idioms.} We compile a list of 554 English idioms along with their definitions from online idiom lexicons.\footnote{\href{http://www.theidioms.com/}{www.theidioms.com},~\href{http://idioms.thefreedictionary.com/}{idioms.thefreedictionary.com}} Table~\ref{tab:example_expressions} presents a sample of the collected idioms.


\paragraph{Collecting Narratives.} We use the Toronto book corpus \cite{zhu2015aligning}, a collection of 11,038 indie ebooks extracted from \url{smashwords.com}. We extract sentences from the corpus containing an idiom from our list, and prepend the 4 preceding sentences to create a narrative. We manually discarded paragraphs that did not form a coherent narrative. We extracted 1,455 narratives with an average length of 80 words, spanning 554 distinct idioms.


\paragraph{Collecting Continuations.} We collected plausible and implausible continuations to the narrative. We used Amazon Mechanical Turk to recruit 117 workers. We provided these workers with the narrative along with the idiom definition, and instructed them to write plausible and implausible continuations that are pertinent to the context, depend on the correct interpretation of the idiom, but which don't explicitly give away the meaning of the idiom. We collected continuations from 3 to 4 workers for each narrative. The average plausible continuation contained 12 words, while the implausible continuations contained 11 words.

To ensure the quality of annotations, we required that workers have an acceptance rate of at least 99\% for 10,000 prior HITs, and pass a qualification test. We then manually inspected the annotations to identify workers that performed poorly in the initial batches, disqualified them from further working on the task, and discarded their annotations. 


Our automatic approach for collecting narratives doesn't account for expressions that may be used figuratively in some contexts but literally in others. For example, the idiom ``run a mile'', i.e. avoiding something in any way possible, may also be used literally to denote running a distance of one mile. To avoid including literal usages, we instructed the workers to flag such examples, which we discard from the dataset. We further manually verified all the collected data. Overall we removed 12 such narratives. 

The final idiom dataset contains 5,101 $\lt$narrative, continuation$\gt$ tuples, exemplified in the top part of Figure~\ref{fig:idiomdata}. We split the examples to train (3,204), validation (355) and test (1,542) sets. To test models' ability to generalize to unseen idioms, we split the data such that there are no overlaps in idioms between train and test.

\subsection{Similes}
\label{sec:datasimile}

A simile is a figure of speech that usually consists of a topic and a vehicle (typically noun phrases) which are compared along a certain property using comparators such as ``like'' or ``as'' \cite{hanks2013lexical,niculae-danescu-niculescu-mizil-2014-brighter}. The property may be mentioned (\emph{explicit} simile) or hidden and left for the reader to infer (\textit{implicit} simile). We focus on implicit similes, that are less trivial to interpret than their explicit counterparts \cite{qadir-etal-2016-automatically}, and test a model's ability to recover the implicit property.

\paragraph{Collecting Similes.} Since there are no reliable methods for automatically detecting implicit similes, we first identify explicit similes based on syntactic cues, and then deterministically convert them to implicit similes. We look for sentences in the Toronto Book corpus containing one of the syntactic structures ``as ADJ/ADV as'' or ``ADJ/ADV like'' as a heuristic for identifying explicit similes. We additionally add the constraint of the vehicle being a noun phrase to avoid examples like ``I worked as hard as him''. We remove the adjectival property to convert the simile to implicit, as demonstrated below:

% \begin{table}[t]
% \centering
% \small
% \begin{tabular}{lP{6cm}}
% \toprule
% \multirow{4}{*}{Explicit} & Now he feels \textbf{calm}, like a high mountain lake without a wind stirring it. \\ \cline{2-2} 
% & Now he feels as \textbf{calm} as a high mountain lake without a wind stirring it. \\ \midrule
% \multirow{2}{*}{Implicit} & Now he feels like a high mountain lake without a wind stirring it. \\ \bottomrule
% \end{tabular}
% \caption{An example of an explicit simile, and the corresponding implicit simile, in which the property ``calm'' was omitted.}
% \label{table:explsim}
% \end{table}

\begingroup
\vspace{2ex}
\renewcommand*{\tabcolsep}{2pt}
{\scriptsize
\hspace{-10pt}
\begin{tabular}{|l|}
\hline
\textbf{Explicit}:\\
He feels \textbf{calm}, like a high mountain lake without a wind stirring it. \\ 
He feels as \textbf{calm} as a high mountain lake without a wind stirring it.\\ \hline
\textbf{Implicit}:\\
He feels like a high mountain lake without a wind stirring it. \\ \hline
\end{tabular}
}
\vspace{2ex}
\endgroup

We collected 520 similes along with their associated property. We asked workers to flag any expression that was not a simile, and manually verified all the collected data. Table~\ref{tab:example_expressions} presents a sample of the collected similes. Many of the similes are original, such as ``like a street-bought Rolex'' which implies that the subject is fake or cheap.

\paragraph{Collecting Narratives.} Once we identified the explicit simile and converted it to its implicit form, we similarly prepend the 4 previous sentences to form narratives. The average length of the narrative was 80 words.

\paragraph{Collecting Continuations.} We repeat the same crowdsourcing setup as for idioms, providing the explicit simile property as the definition. Each narrative was annotated by 10 workers. The average length of continuations was identical to the idiom dataset (12 for plausible and 11 for implausible).

The simile dataset contains 4,996 $\lt$narrative, continuation$\gt$ tuples, exemplified in the bottom part of Figure~\ref{fig:idiomdata}. We split the examples to train (3,100), validation (376) and test (1,520) sets with no simile overlaps between the different sets.

\section{Discriminative Task}
\label{sec:disc}
\begin{figure*}[t]
    \centering
    \includegraphics[width=\textwidth]{figures/knowledge_extraction.pdf} 
    \caption{Extracting inferences from COMET regarding the context (previous sentences in the narrative) and the literal meaning of the content words among the idiom constituents.}
        % \vspace{-2ex}
    \label{fig:knowledge_extraction}
\end{figure*}


The first task we derive from our dataset is of discriminative nature in the setup of the story cloze task. Given a narrative N and two candidate continuations $\{\text{C}_1, \text{C}_2\}$, the goal is to choose which of the continuations is more plausible. 

\subsection{Methods}
\label{sec:disc:methods}

For both idioms and similes, we report the performance of several zero-shot, few-shot and supervised methods as outlined below. Most of our experiments were implemented using the transformers package \cite{wolf-etal-2020-transformers}. 


\paragraph{Zero-shot.} The first type of zero-shot models is based on standard language model score as a proxy for plausibility. We use GPT-2 XL \cite{Radford2019LanguageMA} and GPT-3 \cite{brown2020language} to compute the normalized log-likelihood score of each continuation given the narrative, predicting the continuation with the highest probability: $ \operatorname{argmax}_i P_{LM}(\text{C}_i |\text{N})$. 




We also use UnifiedQA \cite{khashabi-etal-2020-unifiedqa}, a T5-3B model \cite{raffel2020exploring} trained on 20 QA datasets in diverse formats. We don't fine-tune it on our dataset, but instead use it in a zero-shot manner, with the assumption that the model's familiarity with QA format and with the narrative domain through training on the NarrativeQA dataset \cite{kocisky-etal-2018-narrativeqa} would be useful. To cast our task as a QA problem we format the input such that the question is ``Which is more plausible between the two based on the context?''. 


\paragraph{Few-shot.} Language models like GPT-3 have shown impressive performance after being prompted with a small number of labelled examples. A prompting example in which the correct continuation is the first is given in the following format: \texttt{Q: N (1) C$_1$ (2) C$_2$ A: (1)}.


We provided the model with as many prompting examples as possible within the GPT-3 API limit of 2,048 tokens, which is 6 examples. The test examples are provided without the answer and the model is expected to generate \texttt{(1)} or \texttt{(2)}. 

We also use the recently proposed Pattern Exploiting Training model \cite[PET;][]{schick-schutze-2021-just}. PET reformulates the tasks as a cloze question and fine-tunes smaller masked LMs to solve it using a few training examples.\footnote{Specifically, it uses ALBERT XXL V2 \cite{lan2019albert}, which is 784 times smaller than GPT-3.} We use the following input pattern: ``\texttt{N. C$_1$. You are \textunderscore}'' for idioms and ``\texttt{N. C$_1$. That was \textunderscore}'' for similes. PET predicts the masked token and maps it to the label inventory using the verbalizer \{``right'', ``wrong''\} for idioms and \{``expected'', ``unexpected''\} for similes respectively mapping  them to \{\textsc{true}, \textsc{false}\}.\footnote{ We also experimented with the pattern and verbalizer used by \newcite{schick-schutze-2021-just} for MultiRC \cite{khashabi-etal-2018-looking}, with the pattern: ``\texttt{N. Question: Based on the previous passage is C$_1$ a plausible next sentence? \textunderscore}.'' and the verbalizer \{``yes'', ``no''\}, but it performed worse.} We provide each model 100 training examples, train it for 3 epochs, and select the model that yields the best validation accuracy.

\paragraph{Supervised.} We fine-tune RoBERTa-large \cite{liu2019roberta} as a multiple-choice model. For a given instance, we feed each combination of the narrative and a continuation separately to the model in the following format: \texttt{N $\lt\text{s/}\gt\text{C}_i$}. 

We pool the representation of the start token to get a single vector representing each continuation, and feed it into a classifier that predicts the continuation score. The model predicts the continuation with the higher score. We fine-tune the model for 10 epochs with a learning rate of \num{1e-5}  and a batch size of 8, and save the best checkpoint based on validation accuracy.

\begin{figure*}[t]
    \centering
    \includegraphics[width=\textwidth]{figures/knowledge_integration_disc.pdf}
    \vspace{-4ex}
    \caption{Integrating commonsense inferences into a RoBERTa-based discriminative model.}
    % \vspace{-2ex}
    \label{fig:architecture_discriminative}
\end{figure*}


\paragraph{Knowledge-Enhanced.} Inspired by how humans process figurative language, we develop RoBERTa-based models enhanced with commonsense knowledge. We develop two models: the first model obtains additional knowledge to better understand the narrative (\emph{context}), while the second seeks knowledge pertaining to the \emph{literal} meaning of the constituents of the figurative expression (Section~\ref{sec:processing}). In both cases, in addition to the narrative and candidate continuations, the model is also provided with a set of inferences: \{Inf$_1$, ..., Inf$_n$\} that follow from the narrative, as detailed below and demonstrated in Figure~\ref{fig:knowledge_extraction}.

The literal model uses the COMET model \cite{Hwang2021COMETATOMIC2O}, a BART-based language model trained to complete incomplete tuples from ConceptNet. As opposed to extracting knowledge from ConceptNet directly, COMET can generate inferences on demand for any textual input. For an idiom, we retrieve knowledge pertaining to the content words among its constituents, focusing on the following relations: \texttt{UsedFor}, \texttt{Desires}, \texttt{HasProperty}, \texttt{MadeUpOf}, \texttt{AtLocation}, and \texttt{CapableOf}. For each content word, we extract the top 2 inferences for each relation using beam search. For example, given the idiom ``run the gauntlet'', we obtain inferences for ``run'' and ``gauntlet''. We convert the inferences to natural language format based on the templates in \citet{guan2019story}. Given the nature of the simile task, we focused solely on the vehicle's \texttt{HasProperty} relation and obtain the top 12 inferences. For example, given the simile ``like a psychic whirlpool'', we obtain inferences for the phrase ``psychic whirlpool''.

The context model is enhanced with knowledge from ParaCOMET \cite{Gabriel2021ParagraphLevelCT}, trained on ATOMIC. We feed into ParaCOMET all but the last sentence from the narrative, excluding the sentence containing the figurative expression. We generate inferences along ATOMIC dimensions pertaining to the narrator (\texttt{PersonX}), namely: \texttt{xIntent}, \texttt{xNeed}, \texttt{xAttr}, \texttt{xWant}, \texttt{xEffect}, and  \texttt{xReact}. Again, we extract the top 2 inferences for every relation using beam search.

In both models, as demonstrated in Figure~\ref{fig:architecture_discriminative}, the input format $X_{i,j}$ for continuation C$_i$ and inference Inf$_j$ is: \texttt{Inf$_j \lt\text{s/}\gt$ N <s/> C$_i$}.

We compute the score of each of these statements separately, and sum the scores across inferences to get a continuation score:
\begin{equation*}
s_{i} = \sum_{j=1}^{12} s_{i,j} = \sum_{j=1}^{12} \operatorname{scorer}(\operatorname{RoBERTa}(X_{i,j}))
\end{equation*}

\noindent where $\operatorname{scorer}$ is a dropout layer with dropout probability of 0.1 followed by a linear classifier. Finally, the model predicts the continuations with the higher score. We fine-tune the context and literal models for 10 epochs with a learning rate of \num{1e-5} and an effective batch size of 16 for idioms and 64 for similes, and save the best checkpoint based on validation accuracy. 


\subsection{Results}
\label{sec:disc:results}

Table~\ref{tab:discriminative_results} shows the performance of all models on the discriminative tasks. For both similes and idioms, supervised models perform substantially better than few-shot and zero-shot models, but still leave a gap of several points of accuracy behind human performance. Human performance is the average accuracy of two native English speakers on the task. We did not provide them with the idiom definition, and we assume they were familiar with the more common idioms. The models performed somewhat better on idioms than on similes, possibly due to the LMs' familiarity with some common idioms as opposed to the novel similes.

Among the zero-shot models, GPT-2 performed worse than GPT-3 and UnifiedQA, each of which performed best on one of the tasks. In particular, UnifiedQA performed well on idioms, likely thanks to its familiarity with the QA format and with the narrative domain.  

In the idiom task, PET outperformed few-shot GPT-3 by a large margin of 12 points in accuracy for idioms and 3.5 points for simile, which we conjecture is attributed to the different number of training examples: 6 for GPT-3 vs. 100 for PET. The small number of examples used to prompt GPT-3 is a result of the API limit on the number of tokens (2,048) as well as the setup in which all prompting examples are concatenated as a single input. 

Overall, few-shot models performed worse than zero-shot models on both datasets. We conjecture that this is due to two advantages of the zero-shot models. First, the GPT-2 and GPT-3 models performed better than the majority baseline thanks to the similarity between the task (determining which continuation is more plausible) and the language model objective (guessing the next word). Second, the UnifiedQA model performed particularly well thanks to its relevant training. At the same time, both few-shot models had to learn a new task from just a few examples.


\begin{table}[t]
\small
\centering
\begin{tabular}{llll}
\toprule
\textbf{Method} & \textbf{Model} & \textbf{Idiom} & \textbf{Simile} \\ \midrule
\multicolumn{2}{l}{Majority} & 50.0 & 50.8 \\ \midrule
\multirow{3}{*}{Zero-shot} & GPT2-XL & 53.6 & 53.7 \\  
 & GPT3 & 60.2 & 62.4 \\ \
 & UnifiedQA & 67.7 & 60.6 \\ \midrule
\multirow{2}{*}{Few-shot} & GPT3 & 54.1 & 51.7 \\ 
 & PET & 66.1 & 55.2 \\ \midrule
\multirow{2}{*}{Supervised} & RoBERTa & 82.0 & 80.4 \\ 
& -narrative & 65.0 & 67.9 \\ \midrule
\multirow{2}{*}{\begin{tabular}[c]{@{}l@{}}Knowledge \\ Enhanced\end{tabular}} & Context & 82.8 & 79.9 \\ 
 & Literal & \textbf{83.5}* & \textbf{80.6} \\ \midrule
\multicolumn{2}{l}{Human Performance} & \textbf{92.0} & \textbf{95.0} \\ \bottomrule
\end{tabular}
\caption{Model performance (accuracy) on the idiom and simile discriminative tasks. $^*$ Difference is significant ($\alpha <0.07$) between the supervised and knowledge-enhanced models via t-test.}
\label{tab:discriminative_results}
\end{table}


The supervised models leave some room for improvement, and the knowledge-enhanced models narrow the gap for idioms. For similes we see a minor drop in the context model and nearly comparable performance for the literal model. 

\paragraph{Annotation Artifacts.} Human-elicited texts often contain stylistic attributes (e.g. sentiment, lexical choice) that make it easy for models to distinguish correct from incorrect answers without solving the actual task \cite{schwartz-etal-2017-effect,cai-etal-2017-pay,gururangan-etal-2018-annotation,poliak-etal-2018-hypothesis}. Following previous work, we trained a continuation-only baseline, which is a RoBERTa-based supervised model that was trained only on the candidate continuations without the narrative. The results in Table~\ref{tab:discriminative_results} (\texttt{-narrative}) show that the performance is above majority baseline, indicating the existence of \emph{some} bias. However, the performance of this baseline is still substantially worse than the supervised baseline that has access to the full input, with a gap of 17 points for idioms and 12 points for similes, indicating that this bias alone is not enough for solving the task. 


\subsection{Analysis}
\label{sec:disc:analysis}
 
The knowledge-enhanced models provide various types of inferences corresponding to different relations in ConceptNet and ATOMIC. We are interested in understanding the source of improvements from the knowledge-enhanced models over the supervised baseline, by identifying the relations that were more helpful than others. To that end, we analyze the test examples that were incorrectly predicted by the supervised baseline but correctly predicted by each of the knowledge-enhanced models. We split the examples such that every example consists of a single inference, and feed the following input into the model to predict the plausible continuation: \texttt{Inf <s/> N <s/> C}. We focus on the idiom dataset, since for the literal model for similes the only used relation was \texttt{HasProperty} and the context model performed slightly worse than the baseline. 

\begin{table}[t]
\centering
\small
\begin{tabular}{lrlr}
\toprule
\multicolumn{2}{c}{\textbf{Literal}}  &  \multicolumn{2}{c}{\textbf{Context}} \\ \midrule
\texttt{HasProperty} & 82.3 & \texttt{xNeed} & 82.2 \\ 
\texttt{CapableOf}   & \textbf{83.2} & \texttt{xIntent} & 82.6 \\ 
\texttt{Desires}    & 82.5 & \texttt{xWant} & 82.2 \\ 
\texttt{AtLocation}  & 82.7 & \texttt{xReact} &  \textbf{82.8} \\ 
\texttt{UsedFor}     & 82.4 & \texttt{xEffect} & 82.5 \\ 
\texttt{MadeUpOf}    & 82.8 & \texttt{xAttr} & 82.5 \\ \bottomrule
\end{tabular}
\caption{Percents of successful predictions for each relation type for the test set examples.}
\label{tab:disc_analysis}
% \vspace{-3ex}
\end{table}

Table~\ref{tab:disc_analysis} shows the percents of successful test set predictions for each relation type. The relations in the context model perform similarly, with the best relation \texttt{xReact} performing as well as all of the relations (Table~\ref{tab:discriminative_results}). In the literal model, it seems that the combination of all relations is beneficial, whereas the best relation, \texttt{CapableOf}, performs slightly worse than the full model. For a narrative snippet ``Since Dominic isn't \textit{up for grabs} anymore, I figure that I will concentrate on something else, Carmen declares'', the inference ``grabs is capable of hold on to'' was compliant with the meaning of ``up for grabs'' (available or obtainable), and led to the correct prediction of the plausible continuation ``The good news is that there are many other available bachelors out there''. Conversely, the inference corresponding to the \texttt{Desires} relation was ``grab desires making money'' which was irrelevant and led to an incorrect prediction. 





\section{Generative Task}
\label{sec:generative}
\begin{figure*}[t]
    \centering
    \includegraphics[width=1.03\textwidth]{figures/knowledge_integration_gen.pdf}
    \caption{Integrating commonsense inferences into a GPT2-based generative model.}
    \label{fig:architecture_generative}
    % \vspace{-3ex}
\end{figure*}

In the generative task, given a narrative N, the goal is to generate a plausible next sentence that is coherent with the context and consistent with the meaning of the figurative expression. Each instance consists of a reference plausible continuation C.

\subsection{Methods}
\label{sec:generative_methods}

We similarly experiment with zero-shot, few-shot, and supervised models. 

\paragraph{Zero-shot.} We use standard LMs, GPT-2 XL and GPT-3, to generate the next sentence following the narrative. We let the models generate up to 20 tokens, stopping when an end of sentence token was generated. Following preliminary experiments, for GPT-2 XL and the rest of the models we use top-k sampling \cite{fan-etal-2018-hierarchical} as the decoding strategy with $k = 5$ and a softmax temperature of 0.7, while for GPT-3 we use the method provided in the API which is nucleus sampling \cite{holtzman2020curious} with a cumulative probability of $p = 0.9$.

\paragraph{Few-shot.} We prompt GPT-3 with 4 training examples of the form \texttt{Q: N A: C} followed by each individual test example, and decode the answer. 

\paragraph{Supervised.} We fine-tune GPT-2 XL with a language model objective for 3 epochs with a batch size of 2. We also trained T5 large \cite{raffel2020exploring} and BART large \cite{lewis-etal-2020-bart} as encoder-decoder models. Both were trained for 5 epochs for idioms and 20 epochs for similes, with an effective batch size of 64. For each model, we kept the best checkpoint based on the validation set perplexity, and used top-k decoding with $k = 5$ and a temperature of 0.7. 


\paragraph{Knowledge-Enhanced.} We followed the same intuition and inferences we used for the knowledge-enhanced discriminative models (Section~\ref{sec:disc:methods}). We fine-tune the models for one epoch as the effective data size is multiplied by the number of inferences per sample. The overall architecture of the generative knowledge-enhanced model is depicted in Figure~\ref{fig:architecture_generative}. The models are based on GPT-2 XL and trained with a language model objective to predict the next sentence given the narrative and \emph{a single inference}. The input format for inference Inf$_j$ is: \texttt{Inf$_j$ <sep1> N <sep2>}, where \texttt{<sep1>} and \texttt{<sep2>} are special tokens, and the expected output is the plausible continuation C. During inference, we combine the generations from all inferences pertaining to a given narrative. Inspired by \newcite{liu-etal-2021-dexperts}, who ensemble logits from multiple LMs, we ensemble the logits predicted for multiple input prompts using the same model. 

A standard decoding process gets at each time step an input prompt text $x_{\lt t}$ of length $t-1$. The prompt is encoded and the model outputs the logits for the next ($t^{th}$) token, denoted by $z_{t} \in {\rm I\!R}^{|V|}$, where V is the vocabulary. To get a discrete next token, $z_{t}$ is normalized and exponentiated to resemble a probability distribution over the vocabulary: $P(X_t|x_{\lt t}) = \operatorname{softmax}(z_t)$, and the next token $x_{t}$ is sampled from $P(X_{t}|x_{<t})$. This token is then appended to the prompt and the process iteratively continues until a predefined length or until an end of sentence token had been generated.

Our decoding process differs in that at time step t, we compute the logits ${z_t}_{j=1}^{12}$ corresponding to the prompts derived from each of the inferences: \texttt{Inf$_j$ <sep1> N <sep2>} for $j = 1 ... 12$. We sum the logits vectors to obtain $z_t = \sum_{j=1}^{12} {z_t}_{j}$, from which we decode the next token as usual.



\begin{table}[t]
\renewcommand{\arraystretch}{1.3}
\small
\centering
\begin{tabular}{P{1.3cm}lllll}
\toprule
\multirow{2}{*}{\textbf{Method}} & \multirow{2}{*}{\textbf{Model}} & \multicolumn{2}{l}{\textbf{Idiom}} & \multicolumn{2}{l}{\textbf{Simile}} \\ \cline{3-6}
 & & R-L & B-S & R-L & B-S \\ \midrule
\multirow{2}{*}{Zero-shot} & GPT2-XL & 6.2 & 40.2 & 17.0 & 47.7 \\  
 & GPT3 & 8.2 & 33.6 & 13.9 & 40.2 \\ \midrule
Few-shot & GPT3 & 12.8 & 51.2 & 23.1 & 56.1 \\ \midrule
\multirow{3}{*}{Supervised} & GPT2-XL & \textbf{15.9} & \textbf{54.2} & 26.2 & 59.0 \\ 
 & T5-large & 12.9 & 51.0 & 22.9 & 54.9 \\ 
 & BART-large & 12.4 & 48.8 & 26.7 & 58.4\\ \midrule
\multirow{2}{*}{\begin{tabular}[c]{@{}l@{}}Knowledge\\ Enhanced\end{tabular}} & Context & 15.4 & 52.6 & 20.5 & 55.1 \\  
 & Literal & 13.6 & 51.4 & \textbf{28.9} & \textbf{59.1} \\ \bottomrule
\end{tabular}
\caption{Model performance on the generative tasks in terms of automatic metrics. R-L denotes Rouge-L and B-S denotes BERT-Score.}
\label{tab:generative_results}
\end{table}

\subsection{Results}
\label{sec:generative:results}



\begin{figure*}[t]
    \centering
    \includegraphics[width=\textwidth]{figures/system_generated.pdf}
    \caption{Narratives ending in an idiom (top) or a simile (bottom) with the continuations generated by the baseline GPT-2 model and a knowledge-enhanced model, as preferred by human judges.}
    \label{fig:generations}
\end{figure*}


\begin{table}[t]
\small
\centering
\begin{tabular}{lrrrr}
\toprule
\multicolumn{1}{c}{\textbf{Model}} & \multicolumn{2}{c}{\textbf{\underline{Absolute}}} & \multicolumn{2}{c}{\textbf{\underline{Comparative}}} \\
& \multicolumn{1}{c}{\textbf{Idiom}} & \multicolumn{1}{c}{\textbf{Simile}} &  \multicolumn{1}{c}{\textbf{Idiom}} & \multicolumn{1}{c}{\textbf{Simile}} \\ \midrule
GPT2-XL & 56 & 60 & 15 & 18.6 \\ 
+Context & 68 & 68 & \textbf{45} & 16 \\ 
+Literal & 48 & 76 & 13 & \textbf{46.7} \\ 
Human & \textbf{80} & \textbf{88} & - & - \\ \midrule
All & - & - & 8 & 12 \\ 
Neither & - & - & 17 & 6.7 \\ \bottomrule
\end{tabular}
\caption{Percent of times that the generation from each of the models and human-written references was chosen as plausible (absolute) or preferred (comparative) by the majority of workers.}
\label{tab:humaneval}
% \vspace{-3ex}
\end{table}
\begin{table*}[t]
\centering
\small
\renewcommand{\arraystretch}{1.25}
\begin{tabular}{lcP{12cm}}
\toprule
\textbf{Model} & \textbf{Category} & \textbf{Example} \\ \hline
\multirow{5}{*}{\specialcell{Literal \\(Simile)}}  
 & \multirow{2}{*}{\textcircled{1}} & Is that how you got those bruises on your face? They colored her pure white skin \textbf{like a sloppily designed ink blot test} \textit{It was as if her skin was smudged with a brownie.} \\ \cline{2-3} 
 & \multirow{4}{*}{\textcircled{2}} & 
 We stand there munching next to the evening sun-swept grass and whispering trees of Duboce Park, where a bunch of happy-ass dogs are sprinting around in circles while their owners stand there \textbf{like a scarecrow convention}. \textit{The humans were having a great time, jumping and barking and running around.} \\  \cline{2-3}
\multirow{5}{*}{\specialcell{Context \\(Idiom)}} 
 & \multirow{2}{*}{\textcircled{1}} & She could hold some of her thoughts while she was a wolf.Better than the rest of us. Not as well as Sam, but better than me.She and Derrick were \textbf{thick as thieves} \textit{They would always be looking for opportunities to steal} \\ \cline{2-3} 
 & \multirow{3}{*}{\textcircled{3}} & That's why I packed the heat. Just in case I needed some motivation for him to calm down should he be inclined to \textbf{go ballistic} on me. \textit{{\color{red}: because} I was thinking of ways to solve this problem in a peaceful way} \\ \bottomrule
\end{tabular}
\caption{An example for each error category. Each example consists of a narrative, with the figurative expression in bold and the continuation in italic.}
\label{tab:error_analysis_examples}
\end{table*}


\paragraph{Automatic Evaluation.} Table~\ref{tab:generative_results} shows the performance of all the models on the generative tasks in terms of automatic metrics. We report the performance of the recall-oriented n-gram overlap metric Rouge-L \cite{lin-2004-rouge}, typically used for summarization tasks, and the similarity-based BERT-Score \cite{zhang2019bertscore}. We use the latest  implementation to date which replaces BERT with \texttt{deberta-large-mnli}, which is a DeBERTa model \cite{he2020deberta} fine-tuned on MNLI \cite{williams-etal-2018-broad}. In terms of automatic evaluation, the best-performing knowledge-enhanced model (context for idioms and literal for similes) perform similarly to the GPT-2 XL supervised baseline, with slight preference to the baseline for idioms and to the knowledge-enhanced model for similes. Both types of supervised models outperform the zero-shot and few-shot models. 




\begin{table}[t]
\centering
\small
\begin{tabular}{crr}
\toprule
\textbf{Cat.} & \textbf{Literal (Simile)} & \textbf{Context (Idioms)} \\ \hline 
 \textcircled{1} & 50 & 72 \\ 
 \textcircled{2} & 33 & 14 \\
 \textcircled{3} & 17 & 14 \\ \bottomrule
\end{tabular}
\caption{Error categories along with their proportion (in percentage \%) among the implausible continuations.}
\label{tab:error_analysis}
% \vspace{-2ex}
\end{table}





\paragraph{Human Evaluation.}
While automatic metrics provides an estimate of relative model performance, these metrics were often found to have very little correlation with human judgements \cite{novikova-etal-2017-need,krishna2021hurdles}. To account for this we also performed human evaluation of the generated texts for a sample of the test narratives. The human judgements were collected using Amazon Mechanical Turk. Workers were shown a narrative, the meaning of the idiom (or the property of the simile), and a list of 3 generated continuations, one from each of the supervised GPT-2 model, the context model, and the literal model. We performed two types of evaluations. In the absolute evaluation, we randomly sampled 50 narratives for each task, and asked workers to determine for each of the generated continuations along with the human references whether it is plausible or not. In the comparative evaluation, we randomly sampled 100 narratives for idioms and 75 for similes, and presented the workers with a randomly shuffled list of continuations, asking them to choose the most plausible one (or indicate that ``neither of the generations were good'' or ``all are equally good''). In both evaluations, workers were instructed to consider whether the generation is sensical, coherent, follows the narrative, and consistent with the meaning of the figurative expression. Each example was judged by 3 workers and aggregated using majority voting. The inter-annotator agreement was moderate with Krippendorff's $\alpha = 0.68$ and $\alpha = 0.63$ for the absolute and comparative evaluations respectively \cite{Krippendorff2011ComputingKA}. 

In both absolute and comparative performance, Table~\ref{tab:humaneval} shows that for each of the tasks, a knowledge-enhanced model outperformed the baseline GPT-2 model. What makes a more compelling case is that the context model was favored for idioms while the literal model was favored for similes, complying with prior theoretical grounding on these figurative language types. Figure~\ref{fig:generations} shows examples generated by the baseline and the best model for each task. We note that 80\% of the human-written continuations for idioms and 88\% of those in the simile task were judged as plausible. Based on our analysis, the gap from 100\% may be explained by the ambiguity of the narratives that leaves room for subjective interpretation. 

\subsection{Error Analysis}
\label{sec:generative_analysis}


We analyze the continuations labeled as implausible by the annotators, for the best model in each task: context for idioms and literal for similes. We found the following error categories, with percent details in Table~\ref{tab:error_analysis} and exemplified in Table~\ref{tab:error_analysis_examples}: 

\noindent \paragraph{\textcircled{1} Inconsistent with the figurative expression:} The continuation is inconsistent or contradictory to the figurative expression. For instance, the simile in the first row in Table~\ref{tab:error_analysis_examples} is ``like a sloppily designed ink blot test'', for which the property of comparison is ``a pattern of dark blue, purple, and black'', but the generated continuation mentions brownie, which has a \emph{brown} color. Similarly for the idiom ``thick as thieves'' the model generates a literal continuation without understanding its actual meaning ``closest of friends". 


\noindent \paragraph{\textcircled{2} Inconsistent with the narrative:} The continuation is inconsistent or contradictory to the flow of the narrative. For instance, the narrative in the second row in Table~\ref{tab:error_analysis_examples} states that ``the owners who are humans are \emph{standing}'', while the continuation states they are jumping. The model further predicts that the \emph{humans} are barking, instead of the \emph{dogs}. In general, across multiple examples we have found that models tend to confuse the various characters in the narrative.


\noindent \paragraph{\textcircled{3} Spelling or grammar errors:} some generations contained spelling mistakes or introduced grammar errors such as starting with a punctuation or having extra blank spaces. Although we instructed the crowdsourcing workers to ignore such errors, they may have affected their plausibility judgements.


\section{Conclusion}
\label{sec:conclusion}
To understand the figurative language inference capabilities of pre-trained language models  models, we introduce a narrative understanding benchmark focused on idioms and similes.Following the Story Cloze Test we design tasks in both discriminative and generative settings. Through extensive experiments on our benchmark we find that pre-trained language models irrespective of their size struggle to perform well in a zero or few shot setting. Our supervised baseline even though competitive is still behind human performance by a significant margin trained. Finally we show how knowledge-enhanced models that are inspired by the way humans process figurative language outperform all other approaches and is particularly compelling in the generative setting. Finally we find that that while the RoBERTa-large or GPT2-XL model is able to capture some aspects of figurative language, it fails when the interpretation requires word knowledge and pragmatic inferences. We hope this work will spark additional interest in the research community to incorporate and test for figurative language in their NLU systems.
\section{Conclusions and Future Work}\label{section-conclusion}
In this work, we have systematically studied different key notions and results concerning anti-unification of unordered goals, i.e. sets of atoms. We have defined different anti-unification operators and we have studied several desirable characteristics for a common generalization, namely optimal cardinality (lcg), highest $\tau$-value (msg) and variable dataflow optimizations. For each case we have provided detailed worst-case time complexity results and proofs. An interesting case arises when one wants to minimize the number of generalization variables or constrain the generalization relations so as they are built on injective substitutions. In both cases, computing a relevant generalization becomes an NP-complete problem, results that we have formally established.
In addition, we have proven that an interesting abstraction -- namely $k$-swap stability which was introduced in earlier work -- can be computed in polynomially bounded time, a result that was only conjectured in  earlier work. 

Our discussion of dataflow optimization in Section~\ref{section-relation-2} essentially corresponds to a reframing of what authors of related work sometimes call the \textit{merging} operation in rule-based anti-unification approaches as in~\cite{Baumgartner2017}. Indeed, if the "store" manipulated by these approaches contains two anti-unification problems with variables generalizing the same terms, then one can "merge" the two variables to produce their most specific generalization. If the merging is exhaustive, this technique results in a generalization with as few different variables as possible. In this work we isolated dataflow optimization from that specific use case and discussed it as an anti-unification problem in its own right.

While anti-unification of goals in logic programming is not in itself a new subject, to the best of our knowledge our work is the first systematic treatment of the problem in the case where the goals are not sequences but unordered sets. Our work is motivated by the need for a practical (i.e. tractable) generalization algorithm in this context. The current work provides the theoretical basis behind these abstractions, and our concept of $k$-swap stability is a first attempt that is worth exploring in work on clone detection such as~\cite{clones}. 

Other topics for further work include adapting the $k$-swap stable abstraction from the $\preceq^\iota$ relation to dealing with the $\sqsubseteq^\iota$ relation. 
A different yet related topic in need of further research is the question about what anti-unification relation is best suited for what applications. For example, in our own work centered around clone detection in Constraint Logic Programming, anti-unification is seen as a way to measure the distance amongst predicates in order to guide successive syntactic transformations. Which generalization relation is best suited to be applied at a given moment and whether this depends on the underlying constraint context remain open questions that we plan to investigate in the future. 

%The main results of this paper are the polynomial algorithms solving specific anti-unification problems, along with several worst-case time complexity results and proofs. 

% have made efforts to extend the classical anti-unification concepts to the case where the artefacts to generalize are unordered goals. We have done this by considering different levels of atomic abstraction through different generalization relations. W





%Throughout the paper, we have introduced four generalization relations. Figure~\ref{fig-interconnexion} shows how the four relations are linked on a conceptual level. $\sqsubseteq$ is the most general relation as generalization is defined with any substitution. Restricting the definition to injective substitutions or to renamings yields more specific relations, the intersection of which is relation $\preceq^\iota$ where variables are generalized through injective renamings. 

%\begin{figure}[htbp]
%	\begin{center}
%		\begin{tikzpicture}[x=0.75pt,y=0.75pt,yscale=-1,xscale=1]
%		%uncomment if require: \path (0,300); %set diagram left start at 0, and has height of 300
%		
%		%Shape: Ellipse [id:dp330479544492589] 
%		\draw   (150.05,144.64) .. controls (150.05,72.29) and (186.09,13.64) .. (230.55,13.64) .. controls (275,13.64) and (311.05,72.29) .. (311.05,144.64) .. controls (311.05,216.99) and (275,275.64) .. (230.55,275.64) .. controls (186.09,275.64) and (150.05,216.99) .. (150.05,144.64) -- cycle ;
%		%Shape: Ellipse [id:dp3140532351606715] 
%		\draw   (165,87.62) .. controls (165,64.99) and (193.75,46.64) .. (229.22,46.64) .. controls (264.69,46.64) and (293.45,64.99) .. (293.45,87.62) .. controls (293.45,110.26) and (264.69,128.61) .. (229.22,128.61) .. controls (193.75,128.61) and (165,110.26) .. (165,87.62) -- cycle ;
%		%Shape: Ellipse [id:dp9580468020324391] 
%		\draw   (261.25,53.46) .. controls (285.77,54.29) and (304.38,90.37) .. (302.81,134.06) .. controls (301.24,177.74) and (280.08,212.49) .. (255.56,211.67) .. controls (231.03,210.85) and (212.43,174.76) .. (214,131.08) .. controls (215.57,87.39) and (236.72,52.64) .. (261.25,53.46) -- cycle ;
%		
%		% Text Node
%		\draw (172,266) node   {$\sqsubseteq $};
%		% Text Node
%		\draw (235,216) node   {$\preceq $};
%		% Text Node
%		\draw (170,125) node   {$\sqsubseteq^\iota $};
%		% Text Node
%		\draw (254,92) node   {$\preceq^\iota $};
%		\end{tikzpicture}
%	\end{center}
%	\caption{The interconnexions of four generalization relations}
%	\label{fig-interconnexion}
%\end{figure}

%Figure~\ref{fig-interconnexion} shows how the four relations are linked on a conceptual level. When needed in concrete applications, the right generalization operator (or an abstraction) should be used; this of course depends on whether or not the atomic structure should be generalized and the variable dataflow preserved. 

%Future work will focus on the use of such generalization operators in the purpose of applying synctatic transformations on predicates in such a way that the structural distance between them decreases; such a synctatic distance can be evaluated over the most specific generalization of the predicates under scrutiny.

% Uncomment for camera-ready version
% \section*{Acknowledgements}

\bibliography{anthology,custom}
\bibliographystyle{acl_natbib}

\appendix

\section{\vs{Some appendix}}
\label{sec:appendix_whatever}

\vs{relation details comet}
\vs{more examples}
\vs{Change the labels to meaningful names}

\end{document}
