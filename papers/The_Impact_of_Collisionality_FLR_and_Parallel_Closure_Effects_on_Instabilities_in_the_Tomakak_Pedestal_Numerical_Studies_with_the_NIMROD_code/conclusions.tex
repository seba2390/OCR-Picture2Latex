\section{Discussion and Conclusions} 
\label{sec:conclusions}

Our comparisons to the ideal PBM growth rates computed by the ELITE code and
analytic descriptions of the resistive, two-fluid PBM provide another
verification test of the NIMROD code as a means to study the tokamak edge.
Quantitative agreement with ELITE is achieved for these cases, extending prior
work \cite{Burke10} that did not include diverted magnetic topology.

Two-fluid and FLR-drift effects can both produce stabilizing (drift
stabilization through the sheared motion of the ion and electron fluid
responses) and destabilizing (species decoupling and enhanced growth
through mediation by the more mobile electron fluid) effects (see, e.g.,
\cite{King11,King14}).  At intermediate and large mode numbers, the
inclusion of finite-resistivity is destabilizing while the FLR-drift
effects are stabilizing. These effects are largely offsetting, resulting
in a toroidal-mode growth-rate spectrum with the full extended-MHD model
that resembles the ideal calculation. This result is not general but
rather case dependent.  Consistent with ELITE calculations that vary the
ratio of specific heats from the adiabatic through the isothermal limit, we
find the addition of anisotropic thermal conduction produces only a
small modification of the growth rate.

Relative to the NIMROD calculations, analytics \cite{Hastie03} over-predict
both the effects of resistive destabilization and FLR-drift stabilization for
the case studied. Qualitative agreement with the analytic descriptions of the
peeling-ballooning mode is found where high-$n_\phi$ modes are susceptible to
resistive destabilization and drift stabilization in both the analytics and
NIMROD computations.  One of the limits of the analytic derivation is a `local'
approximation which leads to the over prediction of FLR-drift
stabilization~\cite{Hastie00}.  Given the highly complex nature of both the
configuration of the tokamak edge and the two-fluid model equations,
quantitative agreement is not expected as the local approximation neglects the
global configuration and profile effects.

Regarding our investigation into the impact of the size of the ion gyroradius within the
context of NIMROD's full first-order FLR, two-fluid model, we find significant
changes on the growth rates of high-$n_\phi$ modes at varying densities but
constant $\beta$. However, the most unstable mode number is largely unchanged.
Thus we conclude the zeroth-order effect of modifications to the bootstrap
current largely dominates the determination of the most unstable mode during
collisionality scans while lower densities (large $\rho_i$) increase the
magnitude of the drift stabilization on the high-$n_\phi$ modes consistent
with the discussion of Ref.~\cite{Snyder07}. 

% During
% nonlinear simulation, mode coupling causes the details growth-rate spectrum
% to impact the dynamics, and thus the first-order FLR,
% two-fluid effects are significant.

% - Compared NIMROD calculations to ELITE / theory
% -> ad-hoc corrections to \omega_* -- larger than other predictions but case (profile) dependent
% 
% - Need to avoid ITG 
% -> Expected that cases from reconstructions / transport calculation will at most have marginally stable ITG modes
% 
% - Tests of the prediction of the most unstable toroidal mode are important
% -> Mode #s of ELM / EHO?
% -> NL dynamics -> coupling to stable modes can provide a saturation mechanism

% The comparison is possible because NIMROD
% has made significant progress on the extended MHD
% terms~\cite{Sovinec10,King11} Although we have also made
% progress on closures based on the drift kinetic equation~\cite{held15},
% they are not included in this work.  Because the linear mode spectrum
% was insensitive to the parallel diffusivities used, we do not expect
% significant effects from these terms, but as our modeling capability
% progresses, one should always revisit prior verification exercises.
% Although not important in this work because our $k\rho$ remained small,
% some ELM cases can lead to larger $k\rho$ and 
% higher-order terms such as those found in a gyro-Landau-fluid
% models~\cite{snyder97} could be important.
% 
% {\bf SEK: You should always mention limitations in your work}.

% JRK - Ferraro10 -> no xMHD. Will include King11 later. Sovinec10 previously
% JRK - included with this contect. DKE seems out of place here. Why mention GLF
% JRK - without other kinetic approaches? high-order k\rho_i breakdown mentioned in
% JRK - text.

% JRK - move DKE closure discussion to xMHD w/ par closures
% JRK - remove GLF in favor of full ion-orbit kinetic modeling a la Parker and
% JRK - move to rho_i vs. n discussion in xMHD
 

