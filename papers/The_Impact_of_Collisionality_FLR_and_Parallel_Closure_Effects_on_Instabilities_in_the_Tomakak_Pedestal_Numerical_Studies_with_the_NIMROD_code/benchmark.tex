\section{Verification benchmark}
\label{sec:benchmark}

% Relative to the tokamak core, the characteristic time and spatial scales
% are compressed.  However, type-I ELMs still have the instability time
% scale associated with the fast crash is an order of magnitude faster
% than the transport-time scale associated with the processes that govern
% the build up of the pedestal structure.   This separation of time scales
% still allows the standard decomposition of studying the linear
% instabilities about an equilibrium that is used in core modes as well.
% 
% Like the core modes, these long-wavelength instabilities are dominated
% by the stiffness in the ideal MHD terms, even for the cases when they
% may be strictly ideal stable.  Multiple numerical methods have been
% developed to handle this stiffness for both linear and nonlinear codes.
% For the nonlinear codes, one numerical advantage is to 
% separate the fields into steady-state (e.g. the reconstructed fields)
% and time-dependent parts.  The pure steady-state terms are analytically
% eliminated resulting in the largest terms in the system to be removed
% from the numerical computations.
% 
% Although typically only MHD-force balance (a
% Grad-Shafranov solution) is strictly enforced for the steady state, in practice
% all fields associated are time independent. This effectively assumes the
% presence of implicit (in the sense that they are calculable but not calculated)
% sources, fluxes and sinks.  With these assumptions, if the code is run on a
% MHD-stable case, the fields do not change.  Alternatively, the initial fields
% are self-consistently modified by the presence of unstable modes. 
% {\bf SEK: OK -- I think this is a better place to put the discussion, in
%   the end this is confusing unless there is an appendix to explain
%   things in detail.  We need to discuss whether we want to add it.  I
% think not as this is really an EHO discussion.}
% 
% There is no technical reason to make this time-scale decomposition - the NIMROD
% code has the capability to compute the extended-MHD evolution of the
% reconstructed fields. However, it is well-known that physical mechanisms
% outside the scope of the extended-MHD model mediate tokamak transport such as
% neoclassical bootstrap current, toroidal viscosity, and poloidal flow damping;
% neutral beam and RF drive; turbulence; and coupling to the scrape-off layer
% (SOL), neutrals, impurities and the material boundary. Including these effects
% requires explicit calculation of the sources, fluxes and sinks. These
% transport-type calculations are possible and are becoming practical (e.g.
% \cite{held15}), but this sort of integrated modeling remains in the future.

We begin with a study of a high resolution, lower-single-null, JT-60U-like
equilibrium (`Meudas-1'), which was originally employed in a benchmark of the
MARG2D and ELITE codes \cite{Aiba07}, including a
close approach to the X-point \cite{Snyder09}.
This extends previous benchmarks~\cite{Burke10} of ELITE and NIMROD as it
includes diverted magnetic topology and a higher edge safety factor
($q_{95}=6.74$, the safety factor at 95\% of the normalized poloidal flux) that
leads to increased resolution requirements. An ideal-MHD limit is achieved in
NIMROD by using flat density and resistivity profiles inside the last closed
flux surface (LCFS) with small resistivity, $S=10^8$ where $S$ is the Lundquist
number  ($S=\tau_R/\tau_A$), $\tau_A$ is the Alfvén time ($\tau_A=R_o/v_A$),
$v_A$ is the Alfvén velocity ($B/\sqrt{m_i n_i \mu_0}$), $\tau_R$ is the
resistive diffusion time ($\tau_R=R_o^2 \mu_0/\eta$), $R_o=2.936 m$ is the
radius of the magnetic axis, $\eta$ is the electrical resistivity, $\mu_0$ is
the permeability of free space, $m_\alpha$ is a species mass (the $\alpha$
subscript denotes ions or electrons in this work), and $n_\alpha$ is a species
density. The deuteron mass ($m_i = 3.34\times 10^{-27} kg$) is used. In order to
reproduce the vacuum response model outside the LCFS that is used by ELITE, a
low density ($0.01$ of the core density) and high resistivity ($10^7$ times the
core resistivity) is prescribed beyond the LCFS (more details on these
approximations are in Ref.~\cite{Burke10}).  

\begin{figure}
  \centering
  \includegraphics[width=8cm]{ELITEComparison}
  \vspace{-4mm}
  \caption{[Color online]
  Growth rates for the `Meudas-1' benchmark. ELITE with $\Gamma=5/3$ and
  $\Gamma=0$ are compared against results from NIMROD with $\Gamma=5/3$).
  Associated NIMROD data available in Ref.~\cite{king16Z}.}
  \label{fig:ELITEComp}
\end{figure}

\begin{figure}
  \centering
  \includegraphics[width=8cm]{idealConv}
  \vspace{-4mm}
  \caption{[Color online]
  Spectral convergence of the NIMROD code for the ideal-like parameters. 
  The maximum polynomial degree (P) of the basis functions composing the 
  spectral elements in increased in each subsequent line plotted.
  Associated NIMROD data available in Ref.~\cite{king16Z}.}
  \label{idealConv}
\end{figure}

The normalized growth rates ($\gamma \tau_A$ where the linearized mode grows as
$\text{exp}[\gamma t]$) vs.~toroidal mode number ($n_\phi$) from NIMROD and ELITE are
plotted in Fig.~\ref{fig:ELITEComp}.  There is good agreement between the
codes except for $n_\phi$=4 where there is a 27\% relative difference. All
other cases have a relative difference of less than 8\% with typical
differences of 5\%. The NIMROD convergence in terms of the maximum polynomial
order of the spectral elements is shown in Fig.~\ref{idealConv}. Convergence is
most challenging at high wavenumbers where the resolution requirements are most
stringent (the poloidal mesh is composed of $72\times512$ spectral elements).
%SEK: Great point, but total troll bait for reviewers
These cases converge from the unstable side where the growth rate decreases with
enhanced resolution. Thus the excellent agreement between NIMROD and ELITE at
high $n_\phi$ in Fig.~\ref{fig:ELITEComp} may be spurious and indicate that
slightly more resolution is required for $n_\phi$>25, however, the 
shown growth rates are likely within 5\% of their converged values.
Studying nearly ideal cases with extended MHD codes such as NIMROD is challenging 
given the vanishingly small dissipation operators, and convergence is achieved
more quickly with the additional non-ideal terms in the extended-MHD equations,
as in the cases in Sec.~\ref{sec:xMHD}.

% Relative to modeling with extended MHD, ideal-MHD convergence is more challenging 
% given the vanishingly small dissipation operators and convergence is 
% achieved more quickly with all other model equations shown in this work.

\begin{figure}
  \includegraphics[width=8cm]{meudas_n11_BR}
  \caption{[Color online]
  Poloidal cross section of the radial magnetic field component of the
  $n_\phi=11$ peeling-ballooning mode from the `Meudas-1' benchmark case. }
  \vspace{-4mm}
  \label{meudas_n11_BR}
\end{figure}

Figure \ref{meudas_n11_BR} shows a poloidal cross section of the magnetic
($B_R$) eigenmode.  The mode develops an `interference-pattern' structure near
the X-point when inboard and outboard finger-like structures overlap. The
finite-element-mesh nodes are superimposed atop the smallest-scale sub-figure.
As established by Fig.~\ref{idealConv}, this simulation is spatially
and temporally converged. The high resolution required to resolve these
high-$q_{95}$, diverted cases
motivated development of memory-scaling improvements in the NIMROD code.
