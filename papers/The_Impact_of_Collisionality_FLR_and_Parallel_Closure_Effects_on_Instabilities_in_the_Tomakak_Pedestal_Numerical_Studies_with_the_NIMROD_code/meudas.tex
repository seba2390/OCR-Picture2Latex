%% LyX 2.0.6 created this file.  For more info, see http://www.lyx.org/.
%% 
%\documentclass[english,aps,superscriptaddress,showkeys,showpacs,prepri]{revtex4}
\documentclass[english,aps,superscriptaddress,showkeys,showpacs,prepri,twocolumn]{revtex4}
\usepackage[T1]{fontenc}
\pdfoutput=1
%\usepackage[latin9]{inputenc}
\usepackage[letterpaper]{geometry}
\geometry{verbose,tmargin=1in,bmargin=1in,lmargin=1in,rmargin=1in}
\setcounter{secnumdepth}{3}
\usepackage{units}
\usepackage{bbding}
\usepackage{amsmath}
\usepackage{amssymb}
\usepackage{graphicx}
\usepackage{esint}
\usepackage[colorinlistoftodos,prependcaption,textsize=tiny]{todonotes}

\usepackage[utf8]{inputenc}
\usepackage{lmodern} % load a font with all the characters

\makeatletter

%%%%%%%%%%%%%%%%%%%%%%%%%%%%%% LyX specific LaTeX commands.
%% Because html converters don't know tabularnewline
\providecommand{\tabularnewline}{\\}

%%%%%%%%%%%%%%%%%%%%%%%%%%%%%% Textclass specific LaTeX commands.
\@ifundefined{textcolor}{}
{%
 \definecolor{BLACK}{gray}{0}
 \definecolor{WHITE}{gray}{1}
 \definecolor{RED}{rgb}{1,0,0}
 \definecolor{GREEN}{rgb}{0,1,0}
 \definecolor{BLUE}{rgb}{0,0,1}
 \definecolor{CYAN}{cmyk}{1,0,0,0}
 \definecolor{MAGENTA}{cmyk}{0,1,0,0}
 \definecolor{YELLOW}{cmyk}{0,0,1,0}
}

%%%%%%%%%%%%%%%%%%%%%%%%%%%%%% User specified LaTeX commands.
%\usepackage[sort&compress,numbers]{natbib} 
\bibliographystyle{apsrev4-1} 
\usepackage{doi}
\usepackage{hyperref}

\makeatother

\usepackage{babel}
\begin{document}

\title{
The Impact of Collisionality, FLR and Parallel Closure Effects on Instabilities
in the Tomakak Pedestal:   Numerical Studies with the NIMROD code
}
%Edge NIMROD studies and verification:\\ideal-MHD modification by collisionality, \\FLR and parallel-closure effects.}

\author{J. R. King}
\affiliation{Tech-X Corporation, 5621 Arapahoe Ave. Boulder, CO 80303, USA}

\author{A. Y. Pankin}
\affiliation{Tech-X Corporation, 5621 Arapahoe Ave. Boulder, CO 80303, USA}

\author{S. E. Kruger}
\affiliation{Tech-X Corporation, 5621 Arapahoe Ave. Boulder, CO 80303, USA}

\author{P. B. Snyder}
\affiliation{General Atomics, PO Box 85608, San Diego, CA 92186–5608, USA}

\begin{abstract}
The extended-MHD NIMROD code [C.R.~Sovinec and J.R.~King, J.~Comput.~Phys.~{\bf
229}, 5803 (2010)] is verified against the ideal-MHD ELITE code [H.R.~Wilson
\textit{et al.}~Phys.~Plasmas {\bf 9}, 1277 (2002)] on a diverted tokamak
discharge. When the NIMROD model complexity is increased incrementally,
resistive and first-order finite-Larmour radius effects are destabilizing and
stabilizing, respectively. The full result is compared to local analytic
calculations which are found to overpredict both the resistive destabilization
and drift stabilization in comparison to the NIMROD computations.
Published version: Phys. Plasmas 23, 062123 (2016) [\url{http://dx.doi.org/10.1063/1.4954302}]
\end{abstract}

\keywords{edge localized modes, peeling-ballooning modes, extended-MHD,
code verification}

\pacs{52.30.Ex 52.35.Py, 52.55.Fa, 52.55.Tn, 52.65.Kj}
\maketitle

%--------------------------------------------------
%%%%%%%%%%%%%%%%%%%%%%%%%%%
% names
%%%%%%%%%%%%%%%%%%%%%%%%%%%
\newcommand{\modelname}{DCN+\xspace}
\newcommand{\papertitle}{DCN+: Mixed objective and deep residual coattention for question answering}
%RS Not super sure about this title but DCN+ makes not sense and "mixed objective" is unclear and residual coattention is also not (yet) giving anybody a good reason to read the paper...
%\newcommand{\papertitle}{DCN+: Improving Dynamic Coattention Networks for Question Answering}
%\newcommand{\papertitle}{DCN+: Reinforced and Deeper Dynamic Coattention Networks for Question Answering}


\newcommand{\squad}{SQuAD\xspace}

%%%%%%%%%%%%%%%%%%%%%%%%%%%
% shortcuts
%%%%%%%%%%%%%%%%%%%%%%%%%%%
\newcommand{\todo}[1]{\textcolor{orange}{{#1}}\xspace}

%%%%%%%%%%%%%%%%%%%%%%%%%%%
% variables
%%%%%%%%%%%%%%%%%%%%%%%%%%%

\newcommand{\real}{\mathbb{R}}
\newcommand{\loss}{l}
\newcommand{\pstart}{p^{\rm{start}}}
\newcommand{\pend}{p^{\rm{end}}}

\newcommand{\lookup}{L}
\newcommand{\encoded}{E}
\newcommand{\affinity}{A}
\newcommand{\summary}{S}
\newcommand{\context}{C}

\newcommand{\devem}{{\rm EM}_{\rm dev}}
\newcommand{\devf}{{\rm F1}_{\rm dev}}
\newcommand{\testem}{{\rm EM}_{\rm test}}
\newcommand{\testf}{{\rm F1}_{\rm test}}

%%%%%%%%%%%%%%%%%%%%%%%%%%%
% colours
%%%%%%%%%%%%%%%%%%%%%%%%%%%
\definecolor{myblue}{RGB}{67,118,237}
\definecolor{myred}{RGB}{237,39,58}


%%%%%%%%%%%%%%%%%%%%%%%%%%%
% functions
%%%%%%%%%%%%%%%%%%%%%%%%%%%

\newcommand{\softmax}[1]{{\rm softmax}\left({#1}\right)}
\newcommand{\bilstm}{{\rm biLSTM}}
\newcommand{\coattn}{{\rm coattn}}
\newcommand{\concat}[1]{{\rm concat}\left({#1}\right)}
\newcommand{\emb}{{\rm emb}}
\newcommand{\proj}{{\rm proj}}
\newcommand{\dhid}{h}
\newcommand{\demb}{e}
\newcommand{\ddocument}{m}
\newcommand{\dquestion}{n}

\newcommand{\answer}[2]{\rm{ans}\left( {#1}, {#2}\right)}

%%%%%%%%%%%%%%%%%%%%%%%%%%%
% numbers
%%%%%%%%%%%%%%%%%%%%%%%%%%%

% ours
\newcommand{\devemours}{74.5\%\xspace}
\newcommand{\devfours}{83.1\%\xspace}
\newcommand{\emours}{75.1\%\xspace}
\newcommand{\fours}{83.1\%\xspace}
\newcommand{\emoursensemble}{78.9\%\xspace}
\newcommand{\foursensemble}{86.0\%\xspace}

% dcn
\newcommand{\devemdcn}{65.4\%\xspace}
\newcommand{\devfdcn}{75.6\%\xspace}
\newcommand{\emdcn}{66.2\%\xspace}
\newcommand{\fdcn}{75.9\%\xspace}
\newcommand{\emdcnensemble}{71.6\%\xspace}
\newcommand{\fdcnensemble}{80.4\%\xspace}

% bidaf
\newcommand{\devembidaf}{67.7\%\xspace}
\newcommand{\devfbidaf}{77.3\%\xspace}
\newcommand{\embidaf}{68.0\%\xspace}
\newcommand{\fbidaf}{77.3\%\xspace}
\newcommand{\embidafensemble}{73.7\%\xspace}
\newcommand{\fbidafensemble}{81.5\%\xspace}

% sedtbidaf
\newcommand{\devemsedtbidaf}{67.9\%\xspace}
\newcommand{\devfsedtbidaf}{77.4\%\xspace}
\newcommand{\emsedtbidaf}{68.5\%\xspace}
\newcommand{\fsedtbidaf}{78.0\%\xspace}
\newcommand{\emsedtbidafensemble}{73.0\%\xspace}
\newcommand{\fsedtbidafensemble}{80.8\%\xspace}

% mnemonic reader
\newcommand{\devemmr}{70.1\%\xspace}
\newcommand{\devfmr}{79.6\%\xspace}
\newcommand{\emmr}{69.9\%\xspace}
\newcommand{\fmr}{79.2\%\xspace}
\newcommand{\emmrensemble}{73.7\%\xspace}
\newcommand{\fmrensemble}{81.7\%\xspace}

% rnet
\newcommand{\devemrnet}{72.3\%\xspace}
\newcommand{\devfrnet}{80.6\%\xspace}
\newcommand{\emrnet}{72.3\%\xspace}
\newcommand{\frnet}{80.7\%\xspace}
\newcommand{\emrnetensemble}{76.9\%\xspace}
\newcommand{\frnetensemble}{84.0\%\xspace}

% docreader
\newcommand{\devemdocreader}{69.5\%\xspace}
\newcommand{\devfdocreader}{78.8\%\xspace}
\newcommand{\emdocreader}{70.0\%\xspace}
\newcommand{\fdocreader}{79.0\%\xspace}

% fastqa
\newcommand{\devemfastqa}{70.3\%\xspace}
\newcommand{\devffastqa}{78.5\%\xspace}
\newcommand{\emfastqa}{70.8\%\xspace}
\newcommand{\ffastqa}{78.9\%\xspace}

% reasonet
\newcommand{\emreasonet}{69.1\%\xspace}
\newcommand{\freasonet}{78.9\%\xspace}
\newcommand{\emreasonetensemble}{73.4\%\xspace}
\newcommand{\freasonetensemble}{81.8\%\xspace}

%ablation 
%\newcommand{\emdev}{74.3\%\xspace}
%\newcommand{\fdev}{82.5\%\xspace}

\newcommand{\emcove}{71.3\%\xspace}
\newcommand{\fcove}{79.9\%\xspace}
\newcommand{\deltaemcove}{3.2\%\xspace}
\newcommand{\deltafcove}{3.2\%\xspace}

\newcommand{\emnocoattention}{73.1\%\xspace}
\newcommand{\fnocoattention}{81.5\%\xspace}
\newcommand{\deltaemcoattention}{1.4\%\xspace}
\newcommand{\deltafcoattention}{1.6\%\xspace}

\newcommand{\emnomixedobjective}{73.8\%\xspace}
\newcommand{\fnomixedobjective}{82.1\%\xspace}
\newcommand{\deltaemmixedobjective}{0.7\%\xspace}
\newcommand{\deltafmixedobjective}{1.0\%\xspace}

\newcommand{\emnomoe}{74.0\%\xspace}
\newcommand{\fnomoe}{82.4\%\xspace}
\newcommand{\deltaemmoe}{0.5\%\xspace}
\newcommand{\deltafmoe}{0.7\%\xspace}

\newcommand{\emnoworddropout}{73.8\%\xspace}
\newcommand{\fnoworddropout}{82.1\%\xspace}
\newcommand{\deltaemworddropout}{0.5\%\xspace}
\newcommand{\deltafworddropout}{0.4\%\xspace}


%% 2 decimal places
% % ours
% \newcommand{\devemours}{74.46\%\xspace}
% \newcommand{\devfours}{83.12\%\xspace}
% \newcommand{\emours}{75.09\%\xspace}
% \newcommand{\fours}{83.08\%\xspace}
% \newcommand{\emoursensemble}{78.85\%\xspace}
% \newcommand{\foursensemble}{86.00\%\xspace}
% 
% % dcn
% \newcommand{\devemdcn}{65.40\%\xspace}
% \newcommand{\devfdcn}{75.60\%\xspace}
% \newcommand{\emdcn}{66.23\%\xspace}
% \newcommand{\fdcn}{75.90\%\xspace}
% \newcommand{\emdcnensemble}{71.63\%\xspace}
% \newcommand{\fdcnensemble}{80.39\%\xspace}
% 
% % bidaf
% \newcommand{\devembidaf}{67.70\%\xspace}
% \newcommand{\devfbidaf}{77.30\%\xspace}
% \newcommand{\embidaf}{67.97\%\xspace}
% \newcommand{\fbidaf}{77.32\%\xspace}
% \newcommand{\embidafensemble}{73.74\%\xspace}
% \newcommand{\fbidafensemble}{81.53\%\xspace}
% 
% % sedtbidaf
% \newcommand{\devemsedtbidaf}{67.89\%\xspace}
% \newcommand{\devfsedtbidaf}{77.42\%\xspace}
% \newcommand{\emsedtbidaf}{68.48\%\xspace}
% \newcommand{\fsedtbidaf}{77.97\%\xspace}
% \newcommand{\emsedtbidafensemble}{73.02\%\xspace}
% \newcommand{\fsedtbidafensemble}{80.84\%\xspace}
% 
% % mnemonic reader
% \newcommand{\devemmr}{70.10\%\xspace}
% \newcommand{\devfmr}{79.60\%\xspace}
% \newcommand{\emmr}{69.86\%\xspace}
% \newcommand{\fmr}{79.21\%\xspace}
% \newcommand{\emmrensemble}{73.67\%\xspace}
% \newcommand{\fmrensemble}{81.69\%\xspace}
% 
% % rnet
% \newcommand{\devemrnet}{72.30\%\xspace}
% \newcommand{\devfrnet}{80.60\%\xspace}
% \newcommand{\emrnet}{72.30\%\xspace}
% \newcommand{\frnet}{80.70\%\xspace}
% \newcommand{\emrnetensemble}{76.90\%\xspace}
% \newcommand{\frnetensemble}{84.00\%\xspace}
% 
% % docreader
% \newcommand{\devemdocreader}{69.50\%\xspace}
% \newcommand{\devfdocreader}{78.80\%\xspace}
% \newcommand{\emdocreader}{70.00\%\xspace}
% \newcommand{\fdocreader}{79.00\%\xspace}
% 
% % fastqa
% \newcommand{\devemfastqa}{70.30\%\xspace}
% \newcommand{\devffastqa}{78.50\%\xspace}
% \newcommand{\emfastqa}{70.80\%\xspace}
% \newcommand{\ffastqa}{78.90\%\xspace}
% 
% % reasonet
% \newcommand{\emreasonet}{69.10\%\xspace}
% \newcommand{\freasonet}{78.90\%\xspace}
% \newcommand{\emreasonetensemble}{73.40\%\xspace}
% \newcommand{\freasonetensemble}{81.80\%\xspace}
% 
% %ablation 
% \newcommand{\emdev}{74.30\%\xspace}
% \newcommand{\fdev}{82.50\%\xspace}
% 
% \newcommand{\emcove}{71.30\%\xspace}
% \newcommand{\fcove}{79.90\%\xspace}
% \newcommand{\deltaemcove}{3.00\%\xspace}
% \newcommand{\deltafcove}{2.60\%\xspace}
% 
% \newcommand{\emnocoattention}{73.09\%\xspace}
% \newcommand{\fnocoattention}{81.50\%\xspace}
% \newcommand{\deltaemcoattention}{1.21\%\xspace}
% \newcommand{\deltafcoattention}{1.00\%\xspace}
% 
% \newcommand{\emnomixedobjective}{73.81\%\xspace}
% \newcommand{\fnomixedobjective}{82.11\%\xspace}
% \newcommand{\deltaemmixedobjective}{0.49\%\xspace}
% \newcommand{\deltafmixedobjective}{0.39\%\xspace}
% 
% \newcommand{\emnomoe}{73.95\%\xspace}
% \newcommand{\fnomoe}{82.43\%\xspace}
% \newcommand{\deltaemmoe}{0.35\%\xspace}
% \newcommand{\deltafmoe}{0.07\%\xspace}
% 
% \newcommand{\emnoworddropout}{73.83\%\xspace}
% \newcommand{\fnoworddropout}{82.14\%\xspace}
% \newcommand{\deltaemworddropout}{0.47\%\xspace}
% \newcommand{\deltafworddropout}{0.36\%\xspace}   % Useful abbreviations
\section{Introduction}  \label{sec:introduction}

\newcommand\inexpIntro[3]{#1?(#2,#3).}
\newcommand\rinexpIntro[3]{*#1?(#2,#3).}
\newcommand\outexpIntro[3]{#1!(#2,#3).}
\newcommand\outatomIntro[3]{#1!(#2,#3)}

We propose a fully automated method for proving termination of \(\pi\)-calculus processes.
Although there have been a lot of studies on termination analysis for the \(\pi\)-calculus
and related calculi~\cite{Deng06IC,Demangeon07,SangiorgiTermination,KobayashiHybrid,Yoshida04IC,DBLP:journals/jlp/DemangeonHS10,Venet98SAS}, most of them have been rather theoretical,
and there have been surprisingly little efforts in developing  fully automated termination
verification methods and tools based on them. To our knowledge,
Kobayashi's \typical{}~\cite{TyPiCal,KobayashiHybrid} is the only exception that
can prove termination of \(\pi\)-calculus processes (extended with natural numbers)
fully automatically, but its termination analysis is quite limited (see Section~\ref{sec:relatedwork}).

Our method is based on a reduction to termination analysis for sequential programs:
we translate a \(\pi\)-calculus process \(P\) to a sequential program \(S_P\), so that
if \(S_P\) is terminating, so is \(P\). The reduction allows us to use
powerful, mature methods and tools
for termination analysis of sequential programs~\cite{heizmann2016ultimate,freqterm,DBLP:conf/lics/PodelskiR04,Kuwahara2014Termination,DBLP:journals/cacm/CookPR11}.

The idea of the translation is to convert a chain of communications on replicated input
channels to a chain of recursive function calls of the target sequential program.
Let us consider the following Fibonacci process:
\begin{align*}
    & \rinexpIntro{\fib}{n}{r}
        \ifexp{n<2}{ \soutatom{r}{1} \\ &\quad}
                   { \nuexp{s_1} \nuexp{s_2} (\outatomIntro{\fib}{n-1}{s_1} \PAR \outatomIntro{\fib}{n-2}{s_2} \PAR \sinexp{s_1}{x}\sinexp{s_2}{y}\soutatom{r}{x+y}) \\}
    & \PAR \outatomIntro{\fib}{m}{r}
\end{align*}
Here, the process
$\rinexpIntro{\fib}{n}{r} \ldots$ is a function server that computes the \(n\)-th Fibonacci number
in parallel and returns the result to \(r\),
and $\outatom{\fib}{m}{r}$ sends a request for computing the \(m\)-th Fibonacci number;
those who are not familiar with the syntax of the \(\pi\)-calculus may wish to consult
Section~\ref{sec:targetlanguage} first.
To prove that the process above is terminating for any integer \(m\),
it suffices to show that there is no infinite chain of communications on $\fib$:
\[
    \fib(m,r) \to \fib(m_1,r_1) \to \fib(m_2,r_2) \to \cdots.
\]
We convert the process above to the following program:\footnote{The actual translation
  given later is a little more complex.}
\begin{verbatim}
 let rec fib(n) = if n<2 then () else (fib(n-1) [] fib(n-2)) in
 fib(m)
\end{verbatim}
Here, \texttt{[]} represents the non-deterministic choice.
Note that, although the calculation of Fibonacci numbers is not preserved,
for each chain of communications on \texttt{fib}, there is a corresponding
sequence of recursive calls:
\[
\mathtt{fib}(m) \to \mathtt{fib}(m_1) \to \mathtt{fib}(m_2) \to \cdots.
\]
Thus, the termination of the sequential program above implies the termination of
the original process.
As shown in the example above, (i) each communication on a replicated input channel
is converted to a function call, (ii) each communication on a non-replicated input
channel is just removed (or, in the actual translation, replaced by a call of
a trivial function defined by \(f(\seq{x})=(\,)\)), and (iii) parallel composition
is replaced by a non-deterministic choice.
We formalize the translation outlined above and prove its correctness.

The basic translation sketched above sometimes loses too much information.
For example, consider the following process:
\begin{align*}
    & \rinexpIntro{\pre}{n}{r} \soutatom{r}{n-1} \\
    & \PAR \rinexpIntro{f}{n}{r} \ifexp{n<0}{ \soutatom{r}{1} }
                                       { \nuexp{s} (\outatomIntro{\pre}{n}{s} \PAR \sinexp{s}{x}\outatomIntro{f}{x}{r}) } \\
    & \PAR \outatomIntro{f}{m}{r}
\end{align*}
The translation sketched above would yield:
\begin{verbatim}
  let pred(n) = n-1 in
  let rec f(n) = if n<0 then () else (pred(n) [] f(*)) in
  f(m)
\end{verbatim}
Here, \texttt{*} represents a non-deterministic integer: since we have removed
the input $\sinatom{s}{x}$, we do not have information about the value of \( x \).
As a result, the sequential program above is non-terminating, although the original
process is terminating.
To remedy this problem, we also refine the basic translation above by using a refinement
type system for the \(\pi\)-calculus. Using the refinement type system,
we can infer that the value of \(x\) in the original process is less than \(n\),
so that we can refine the definition of \texttt{f} to:
\begin{verbatim}
 let rec f(n) = ... else (pred(n) [] let x=* in assume(x<n);f(x))
\end{verbatim}
The target program is now terminating, from which
we can deduce that the original process is also terminating.
We have implemented an automated tool based on the refined translation above.

The contributions of this paper are summarized as follows.
\begin{itemize}
\item The formalization of the basic translation from the \(\pi\)-calculus
  (extended with integers) to sequential programs, and a proof of its correctness.
\item The formalization of a refined translation based on a refinement type system.
\item An implementation of the refined translation, including automated refinement type
  inference based on CHC solving, and experiments to evaluate the effectiveness of
  our method.
\end{itemize}

The rest of this paper is structured as follows.
Section~\ref{sec:targetlanguage} introduces the source and target languages
of our translation.
Section~\ref{sec:approach} 
formalizes the basic translation, and proves its correctness.
Section~\ref{sec:refinement} refines the basic translation by using a refinement type system.
Section~\ref{sec:implementation} reports an implementation and experiments.
Section~\ref{sec:relatedwork} discusses related work,
and Section~\ref{sec:conclusion} concludes the paper.

\section{Benchmarking of Butterfly Multiply}
\label{sec:appx_benchmark}

We validate that flat butterfly matrices (sum of factors) can speed up multiplication on GPUs
compared to butterfly matrices (products of factors).

Consider the matrix $M \in \mathbb{R}^{n \times n}$ that can be written as products of butterfly factors of
strides of up $k$ (a power of 2), with residual connection:
\begin{equation*}
  M = (I + \lambda \vB_k^{(n)}) (I + \lambda \vB_{k/2}^{(n)}) \dots (I + \lambda \vB_2^{(n)}).
\end{equation*}
The first-order approximation of $M$ has the form of a flat butterfly matrix
with maximum stride $k$ (\cref{sec:flat_butterfly}):
\begin{equation*}
  M_\mathrm{flat} = I + \lambda (\vB_2^{(n)} + \dots + \vB_{k/2}^{(n)} + \vB_k^{(n)}).
\end{equation*}

Notice that $M$ is a product of $\log_2 k$ factors, each has $2n$ nonzeros, so
multiplying $M$ by a input vector $x$ costs $O(n \log k)$ operations (by
sequentially multiplying $x$ by the factors of $M$).
The flat version $M_\mathrm{flat}$ is a sparse matrix with $O(n \log k)$
nonzeros as well, and the cost of multiplying $M_\mathrm{flat} x$ is also
$O(n \log k)$.
However, in practice, multiplying $M_\mathrm{flat} x$ is much more efficient on
GPUs than multiplying $Mx$ because of the ease of parallelization.

We measure the total time of forward and backward passes of multiplying either
$M_\mathrm{flat} x$ and compare to that of multiplying $Mx$ for different
maximum strides, as shown
in~\cref{fig:flat_butterfly_speed}.
We see that ``flattening'' the products brings up to 3$\times$ speedup.
\begin{figure}[ht]
  \centering
  \includegraphics[width=0.6\linewidth]{figs/flat_butterfly_speed.pdf}
  \caption{\label{fig:flat_butterfly_speed}Speedup of multiplying
    $M_\mathrm{flat}x$ compared to multiplying $Mx$. Flattening the products
    yields up 3$\times$ speedup.}
\end{figure}

We use matrix size $1024 \times 1024$ with block size 32.
The input batch size is 2048.
We use the block sparse matrix multiply library from
\url{https://github.com/huggingface/pytorch_block_sparse}.
The speed measurement is done on a V100 GPU.


   
\section{Increasing the model complexity}
\label{sec:xMHD}

% o Don't emphasize M3D-C1, just ref Ferraro
% o 1fl effects:
%  - Describe profile choice for n - avoid ITG like modes w/ 2f parameters
%  - Using nu_perp = k_perp = D_n = elecd(psin=1)
%  - Using Z=3; pe_frac=0.75 => Te = Ti; Gamma=5/3
%  - Effect of density profile - local gamma * tau_a const,
%      -> normalized to core tau_a (unchanged)
%      -> local tau_a ~ sqrt(n) smaller => growth rates uniformly larger
%  - Effect of resistivity profile: high-n destabilization, low-n stabilization
% o parallel effects:
%  - Almost no effect on gamma
%  - expected from ELITE results which show mode \gamma is largely 
%      insensitive to \Gamma (0 in Ferraro, 5/3 and inf here)
% o 2fl FLR effects:
%  - Including full FLR model: Gen. Ohm's law; ion gyrovisc; cross heat fluxes
%  - Drift stabilization at high-n; crosses ideal results @ n~24
%  - Little effect at low n (n<10)
%  - (No destabilization as predicted by HRP)
%  - For this case: computations w/ and w/out x-heat flux give same result
%     -> Modest beta (0.1%); consistent with FLR theory from King and Kruger
% o Reference: Ping, Xu13

Initial-value extended-MHD computations, while more computationally expensive
than ideal-MHD calculations with ELITE, are able to include additional effects
that are more representative of experimental conditions:
% Initial-value extended-MHD computations, while more computationally expensive
% than ideal eigenvalue calculations such as ELITE, are able to include
% additional effects that are more representative of experimental conditions:
\begin{itemize} \cramplist \zapspace
\item 
Instead of a magneto-static vacuum outside the LCFS, NIMROD includes a
cold-plasma region.
\item
Finite-dissipation effects such as resistivity, viscosity and thermal
conduction.
\item 
Density and temperature profiles that prescribe the associated
drifts and resistivity profile.
\item
Parallel-closure effects that represent large diffusivities oriented along the
magnetic field.
\item
First-order FLR-closure and two-fluid effects.
\end{itemize} 
Reference \cite{Ferraro10} investigates the first three of these effects with
the Meudas-1 case. We further this work by including the last two effects.  The
first three effects are also re-investigated and different modifications to the
growth rates are found. These differences are expected as slightly different
profiles for density and temperature are used in order to avoid spurious
electrostatic modes with the two-fluid, first-order FLR model. 
Ref.~\cite{Ferraro10} avoided these modes by employing only the
single-fluid resistive-MHD model.
% {\bf SEK: I think this is review bait:  Why didn't
%   Nate see spurious modes?  What is the source of these spurious modes?
% Etc.  I don't see a clean way of discussing this}

In calculations beyond the ideal model, the density profile is
$n_e(\psi_n)=n_{e0} (p(\psi_n)/p_0)^{0.3}$ where
$n_{e0}=6\times10^{19}\;m^{-3}$ and $\psi_n$ is the normalized poloidal flux.
Here $p$ is the pressure as prescribed by the ideal gas law, $p=n_e T_e+ n_i
T_i$, where $T_\alpha$ is a species' temperature. This choice does not match
Ref.~\cite{Ferraro10}, but effectively avoids spurious electrostatic modes with
the two-fluid, first-order FLR model. These spurious modes are discussed in 
more detail in Sec.~\ref{sec:densityScan}.
% {\bf SEK: Ditto the previous comment} 
The pressure-profile gradient is specified by the Meudas-1 case and setting the
edge temperature to $200 eV$ leads to a core temperature of $5806 eV$.  The
edge temperature is modified independent of the MHD-stability (determined by
$p^\prime$) by the transformation $p(\psi_n) \rightarrow p(\psi_n) + p_c$ where
$p_c$ is a constant. Without scrape-off-layer (SOL) profiles of density and temperature, which are
not included in standard reconstructions, this choice of the edge temperature
also specifies the temperature outside the LCFS.  SOL profiles are required to
get both the correct profiles outside the LCFS, which determines the vacuum
response, and simultaneously set the correct resistivity at the mode resonant
location when Spitzer resistivity is used.  The non-ideal viscous, conductive
and particle diffusivities are set to a small value, one tenth of the
resistivity at the LCFS. 

\begin{figure}
  \centering
  \includegraphics[width=8cm]{modelScan}
  \vspace{-4mm}
  \caption{[Color online]
  Growth rates computed by NIMROD on the `Meudas-1' case with four different
  models: ideal MHD (ideal), resistive MHD with reconstructed density,
  temperature, and Spitzer-resistivity profiles (1f), the 1f model with parallel
  Braginskii closures (1f-Par), the 1f-Par model with ion gyroviscosity, a full
  extended MHD Ohm's law (including the Hall, $\nabla p_e$, and electron inertia
  terms) and separate temperature evolution equations with cross-heat fluxes
  (2f-Par-FLR).
  Associated NIMROD data available in Ref.~\cite{king16Z}.}
  \label{modelScan}
\end{figure}

Figure~\ref{modelScan} plots the growth rates computed by NIMROD on the
`Meudas-1' case with four different models: ideal MHD (ideal), resistive MHD with
reconstructed density, temperature, and Spitzer-resistivity profiles (1f), the
single-fluid model with parallel Braginskii closures (1f-Par) where the 
parallel viscosity is
\begin{multline}
\mathbf{\Pi}_{\parallel i}=
  m_i n_i \nu_{\parallel i}
  \left(\hat{\mathbf{b}}
  \hat{\mathbf{b}}-\frac{1}{3}\mathbf{I}\right) \\
  \times\left(3\hat{\mathbf{b}}\cdot\nabla\mathbf{v}_{i}\cdot
  \hat{\mathbf{b}}-\nabla\cdot\mathbf{v}_{i}\right)\;,
\end{multline}
and the parallel heat flux vector is
\begin{equation}
\mathbf{q}_{\parallel \alpha}=
  -n_i \chi_{\parallel \alpha}
  \hat{\mathbf{b}}\hat{\mathbf{b}}\cdot\nabla T_{\alpha}\;,
\end{equation}
and a two-fluid model with parallel closures that includes ion gyroviscosity
\begin{multline}
\mathbf{\Pi}_{\times i}=
  \frac{m_{i}p_{i}}{4ZeB}[\hat{\mathbf{b}}
    \times\mathbf{W}\cdot
    \left(\mathbf{I}+3\hat{\mathbf{b}}\hat{\mathbf{b}}\right) \\
    -\left(\mathbf{I}+3\hat{\mathbf{b}}\hat{\mathbf{b}}\right)
    \cdot\mathbf{W}\times\hat{\mathbf{b}}]\;,
\end{multline}
where $\mathbf{W}$ is the rate-of-strain tensor
\begin{equation}
\mathbf{W}=\nabla\mathbf{v}+\nabla\mathbf{v}^{T}
           -(2/3)\mathbf{I}\nabla\cdot\mathbf{v}\;,
\end{equation}
a full extended-MHD Ohm's law,
\begin{equation}
\mathbf{E}=
  -\mathbf{v}\times\mathbf{B}+\frac{\mathbf{J}\times\mathbf{B}}{n_ee}
  -\frac{\nabla p_{e}}{n_ee}+\eta\mathbf{J}
  +\frac{m_{e}}{n_e e^2}\frac{\partial\mathbf{J}}{\partial t}\;,
\end{equation}
and separate temperature evolutions with cross-heat fluxes,
\begin{equation}
\mathbf{q}_{\times \alpha}=
  \frac{5p_{\alpha}}{2q_{\alpha}B}
  \hat{\mathbf{b}}\times\nabla T_{\alpha}\;,
\end{equation}
(2f-Par-FLR). Here $v_i$ is the bulk-ion flow, $\nu_{\parallel i}$ is the ion parallel
diffusivity, $\chi_{\parallel \alpha}$ is the species' parallel diffusivity,
$q_\alpha$ ($e$) is the species' (electron) electric charge,  $\mathbf{J}$ is
the current density, $\hat{\mathbf{b}}$ is the magnetic unit vector
($\hat{\mathbf{b}} = \mathbf{B}/B$), and $\mathbf{I}$ is the identity tensor.
An effective ion charge ($Z$) of three is assumed. Each model builds
on the previously listed and includes all prior terms. Electron viscosity is
not included.


% REMOVED FROM INTRO
% For peeling-ballooning modes, the prevailing analytic description of these
% effects is described in Ref.~\cite{Hastie03}. 
% This conventional picture expects two-fluid and FLR effects are destabilizing
% to modes with low toroidal mode number but stabilizing at high toroidal mode
% number. 
% We compare our results with this simple model of drift stabilization where
% $\gamma_{MHD}^2 = \omega (\omega - \omega_*)$.  Here $\gamma_{MHD}$ is the
% ideal-MHD growth rate, $\omega$ is the complex mode frequency and $\omega_*$ is
% the diamagnetic-drift frequency.

% Closure expressions for the viscosity tensor, $\mathbf{\Pi}$, and heat flux,
% $\mathbf{q}$. These expressions are typically decomposed into three parts:
% parallel (large diffusivities oriented along the magnetic field), cross (FLR
% ordered contributions), and perpedicular (small dissipative terms)

When a density profile is included, the normalized growth rate of the mode
remains constant with the single-fluid model when the Alfvén time is computed
at the mode resonant surface. Our calculations are normalized by the Alfvén
time as computed with the values at the magnetic axis. These values remain constant
when a density profile is included such that the normalized mode growth rate is
effectively increased. 

The effect of the resistivity profile is more nuanced.  Finite
resistivity allows for reconnection within the model by relaxing the
frozen-flux constraint. This permits resistive-ballooning modes that
grow faster than their ideal counterparts. Alternatively, the resistive
dissipation can stabilize the modes by acting as a dissipative term and
modifying the response outside the LCFS from that of a vacuum to that of
a plasma. Figure 5 of Ref.~\cite{Ferraro10} shows that the plasma
response is stabilizing relative to a vacuum region. Overall, these
effects stabilize the ballooning modes at low-$n_\phi$ and destabilize
the modes at high-$n_\phi$ as seen in Fig.~\ref{modelScan} when
comparing the single fluid and ideal normalized growth rates. The effect
of including small particle, viscous, and thermal diffusivities on the
resistive-MHD mode is small (not shown).

Including the Braginskii parallel closures with large coefficients
($\chi_{\parallel e} = 10^9\times \eta_0/\mu_0$ and  $\chi_{\parallel i} =
\nu_{\parallel i} = 10^8\times \eta_0/\mu_0$) has little effect on the growth
rates.  When thermal conduction is large, the temperature quickly equilibrates
along field lines to produce an isothermal response.  Thus NIMROD computations
that show little effect from the parallel closures are consistent with the ELITE
results that find the shape of the mode growth-rate spectrum is largely not
sensitive to the value of the ratio of specific heats, $\Gamma$, as seen in
Fig.~\ref{fig:ELITEComp}. When the response to the compressible motion
associated with the sound wave is eliminated ($\Gamma=0$ in the ELITE 
computations), the growth rate is slightly enhanced relative to the adiabatic limit.
If the growth rates are modified beyond this small effect, it would be
an indication that changes in the energy equation can impact the MHD response, 
and thus that closure effects are significant. Although the short-mean-free
path Braginskii-like closure \cite{Braginskii,Catto04} is used in our NIMROD
compuatations, we note that other closures, such as long-mean-free path
\cite{Ramos11} or the general approach of solving drift-kinetic equation
\cite{held15} for a closure can be applied. However, for this case where the
parallel-closure effects are negligible, different parallel closures are
unlikely to significantly modify the results.

With the full two-fluid, FLR model, there is a stabilizing effect on the
intermediate and high-$n_\phi$ modes as expected from analytic
treatments \cite{Hastie03}. A common reduced-MHD approximation for
first-order FLR closures is to assume that the ion-gyroviscous force and
the advection by the diamagnetic drift exactly cancel (this is
colloquially known as the gyroviscous cancellation, see, e.g.
\cite{Coppi64}). However, as discussed by Ramos in
Ref.~\cite{Ramos:2007cn}, ``these cancellations are only partial and not
very useful in practice for general magnetic geometries\ldots''.  This
cancellation is only valid with a large, uniform guide field, without
curvature terms (slab approximation), and in the low-$\beta$ limit (see,
e.g. \cite{King14}).  Use of the full gyroviscous operator is critical
as in addition to effects that appear as diamagnetic drifts, it also
contains terms proportional to first-order FLR magnetic-curvature and
grad-B drifts \cite{King11}. Our modeling contains not just the
first-order FLR drifts from ion gyroviscosity, but also the drifts from
the cross-heat fluxes that enter the equations on the same order.  

Just as the large balance of the ideal-MHD forces can cause
numerical difficulties with the ideal-MHD force operator, the
partial gyroviscous cancellation leads to similar numerical 
pitfalls and verification is important.
The implementation of these terms in NIMROD are verified in cylindrical and
slab magnetic geometries against analytic calculations for
tearing~\cite{Sovinec10,King11,King14} and
ion-temperature-gradient~\cite{Schnack13} (ITG) modes. 
% {\bf SEK: I don't know
%   why you have Schnack13 associated with tearing instead of ITG.  Also,
%   Schnack really didn't use Coppi67 directly but rather had to modify it
%   for the verification purposes, but it's better to just reference his
% paper}.  
% JRK -> previous references Coppi67 for ITG; others for general verfication
To test the effect of these terms against analytic theory
in full tokamak magnetic geometry, we perform a qualitative
comparison with analytic theory in Sec.~\ref{sec:analyticComp}.
Full verification is precluded by the approximations
made in the analytic theories.
%For verification in full tokamak magnetic geometry, we perform a qualitative
%comparison with analytic theory in Sec.~\ref{sec:analyticComp}.

The first-order FLR model is valid for the toroidal modes presented although
this is not necessarily the case for general edge mode modeling.  Assuming
$k\simeq(q/r+1/R)n_\phi$ and evaluating local quantities at the peak of the
mode eigenfunction ($\psi_n = 0.969$) on the outboard midplane, the normalized
ion-gyroradius ($\rho_i=\sqrt{\Gamma m_i T_i}/ZeB$) as a function of toroidal
mode is $k \rho_i \simeq 0.0055n_\phi$.  If the first-order FLR assumption is
violated, more-complex full-ion-orbit kinetic modeling is required to simulate
large-fluctuation ELM dynamics.  For this case, computations produce
approximately the same results with and without the cross-heat fluxes.  This is
consistent with drift analytics \cite{King14} that shows that the cross-heat
flux is only significant at very low values of plasma $\beta$ (here the
$\beta=2\mu_0p/B^2$ is 0.1\%).  Perhaps coincidentally for this case, the
effects of the resistive destabilization and drift stabilization largely
counteract one another. The drift stabilized growth rates are not less than the
ideal calculation until $n_\phi \gtrsim 24$.  

% Old text:
%
% In addition to the ideal MHD benchmark, the M3D-C1 code
% investigated the effects of including a density gradient profile and Spitzer
% resistivity. Our NIMROD studies have furthered this work to
% include two-fluid effects through the generalized Ohm's law (including the
% Hall, $\nabla p_e$, and electron inertia terms), ion gyroviscosity, highly
% anisotropic thermal conduction, and parallel ion
% viscosity.  When the model includes a generalized Ohm's
% law, the result qualitatively agrees with the analytic theory of
% Ref.~\cite{Hastie03} used for ELITE's {\em ad hoc} model; we find a
% destabilization of the intermediate-$n$ mode spectrum and a stabilization of
% the high-$n$ mode spectrum. When ion gyroviscosity is included, there is a
% stabilization of the high-$n$ mode spectrum, consistent with recent results
% from the BOUT++ code \cite{Xu13}. The high-$n$ mode
% spectrum that results from the use of a extended MHD model prevents energy from
% accumulating on the smallest resolvable scales and corrupting the simulations.
% These results are shown in Fig.~\ref{fig:ELITEComp}.

\section{Comparison to drift analytics}
\label{sec:analyticComp}

\begin{figure}
  \centering
  \includegraphics[width=8cm]{FLRidealComp}
  \vspace{-4mm}
  \caption{[Color online]
  Growth rates computed by NIMROD with the ideal, single-fluid (1f) and full
  FLR (2f-Par-FLR) models in comparison to the analytic dispersion relation of
  the FLR-stabilized, ideal ballooning mode with full and reduced $\omega_{*\alpha}$ 
  values. Associated NIMROD data available in Ref.~\cite{king16Z}.}
  \label{FLRidealComp}
\end{figure}

To gain insight into our two-fluid computations, we compare with analytic
descriptions of drift stabilization of the ballooning mode.  The dispersion
relation for drift stabilizations of the ideal, incompressible ballooning mode
is
\begin{equation}
\omega (\omega - \omega_{*i}) = - \gamma_I^2
\label{eq:driftBalloon}
\end{equation}
where $\gamma_I$ is the ideal growth rate in the absence of drift effects and
$\omega$ is the complex frequency ($\omega=i\gamma$) \cite{tang82}.  The
diamagnetic drift velocity is typically defined as the
perpendicular-to-the-magnetic-field flow contribution from the pressure
gradient within the context of a two-fluid, or generalized Ohm's law.  For a
tokamak, neoclassical effects damp the poloidal diamagnetic flow contribution.
Thus in our case the remaining toroidal diamagnetic flow determines the
diamagnetic frequency, $\omega_{*i}= \mathbf{k}\cdot\mathbf{v}_{*i} =
n_\phi/(n_iq_i)\partial p_i/\partial \psi$ where $\mathbf{k}$ is the mode
wavenumber.  Figure \ref{FLRidealComp} shows the comparison to this dispersion
relation with full and reduced diamagnetic drift values. The value of
$\omega_{*i}$ is computed at the radial location where the mode structure
peaks.  The normalized frequency from the diamagnetic drift velocity,
$\omega_{*i}\tau_A$, varies linearly with toroidal mode number as
$0.00483n_\phi$.  This `local' approximation, which neglects the radial
variation of the $\omega_{*i}$ profile, is known to over-estimate the effect of
drift stabilization \cite{Hastie00,Snyder11}.  We find that a reduction of the
value of $\omega_{*i}$ (approximately by a factor of 2-4) gives the best
agreement between Eqn.~\eqref{eq:driftBalloon} and the two-fluid NIMROD
calculation.  A similar comparison of ELITE to the two-fluid BOUT++ model on a
different case finds the drift stabilization is over-predicted \cite{Snyder11}
by a comparable factor.

One of the limitations of Eqn.~\eqref{eq:driftBalloon} is that it does not
include the effect of resistive destabilization on the high-$n_\phi$ modes,
as is included in our two-fluid computations.
The analytic treatment of Ref.~\cite{Hastie03} provides a dispersion relation
(Eqn.~(43) of the reference; referred to as HRP Eqn.~(43) in this discussion) that
includes the effect of resistive destabilization in addition to drift effects.
In order to compare with this theoretical treatment, parameters are computed
on the outboard midplane at the radial location where the mode structure peaks,
consistent with the previous $\omega_{*i}$ calculation. This yields local values
of the safety factor, $q=7.34$, normalized magnetic shear, $s=8.8$ and
normalized pressure gradient, $\alpha=28.8$, where the definitions of
Ref.~\cite{Hastie03} are used.  The $\Delta^\prime_B$ stability parameter is
determined by using the ideal growth rate ($\Gamma=5/3$) and solving Eqn.~35 of
Ref.~\cite{Hastie03}.  $\Delta^\prime_B$ varies with toroidal mode number but
falls in the range of $-17.4$ to $-12.1$. 

Comparison of HRP Eqn.~(43) with $\omega_*=0$ and the NIMROD resistive-MHD results
differ by a factor approximately 2.5 at large toroidal mode numbers where the
analytic calculation results in greater growth rates and there is no
stabilization at low $n_\phi$. As the analytic calculation is a local calculation, it
does not capture the stabilizing influence on the global mode of finite
resistivity, in particular it misses the effect of modifying the response outside the
LCFS from that of a vacuum to that of a resistive plasma.  Our interest is
largely to compare and contrast the predicted drift stabilization and thus we
reduce the resistivity to 35\% of the value used in the NIMROD computations
when calculating the analytic values of HRP Eqn.~(43). This produces growth
rates with $\omega_*=0$ that roughly match the NIMROD resistive-MHD
computations at high $n_\phi$ as shown in Fig.~\ref{HRPcomp}.

\begin{figure}
  \centering
  \includegraphics[width=8cm]{HRPcomp}
  \vspace{-4mm}
  \caption{[Color online]
  Growth rates computed by NIMROD with the ideal, single-fluid (1f) and full
  FLR (2f-Par-FLR) models in comparison to the analytic dispersion relation of
  Ref.~\cite{Hastie03}, Eqn.~(43) with the same parameters as the NIMROD
  computations except reduced $\omega_{*\alpha}$ values and a reduced
  resistivity (0.35 of the NIMROD value).
  Associated NIMROD data available in Ref.~\cite{king16Z}.}
  \label{HRPcomp}
\end{figure}

More significantly, Fig.~\ref{HRPcomp} compares the results of our linear
NIMROD computations with the drift-stabilized growth rates computed from
Eqn.~(43) of Ref.~\cite{Hastie03}. Again, the analytics of HRP Eqn.~(43)
over-estimate the effect of drift stabilization where the reduction of
$\omega_*$ that leads HRP Eqn.~(43) to qualitatively agree with the two-fluid
NIMROD computations is between 2 to 4 (dependent on $n_\phi$). There is little
effect on the low-$n_\phi$ modes and the ion drift-wave resonance predicted in
Ref.~\cite{Hastie03} is not observed.
%This factor is smaller than that required for
%qualitative agreement with Eqn.~\eqref{eq:driftBalloon} as HRP Eqn.~(43)
%includes the effect of resistive destabilization.  


% Compare to analytics
% 1) drift stabilization of the ideal, incompressible ballooning mode
% 2) drift stabilization of the resistive ballooning mode
% Begin with (1):
%   - Disp. relation over predicts FLR stabilization
%     - Does not account for resistive destabilization
%   - Need to reduce w* by 32x to get modest agreement
% Conclusion: profile and poloidal effects are important - Ref HCR
%
% (2) Use HRP Eqn43 to compare resistive ballooning mode disp.
% Use same parameters as NIMROD runs
%    Theory assumes cylindrical s, alpha parameters
%    Values computed on the outboard midplane at 
%    mode resonant surface as reported by ELITE
%      q = 7.34
%      s = 8.8
%      alpha = 28.8
%      w* tau_a = 0.0176 n_phi
% Use HRP Eqn 35 to determine Delta'B from ELITE ideal Gamma=5/3 result
%      Delta'B= -17.4 -- -12.1 (varies by n)
% NIMROD computes visco-resistive growth rate w/ FLR modification
% HRP visco-resistive dispersion neglects Delta'B, not applicable
%      -> HRP shows viscosity is stabilizing
% HRP over-predicts \gamma with \omega_star =0 by factor of ~2.5x
%      -> Need viscosity for quantitative agreement
%      -> reduce resistivity by 0.35 give qualitative agreement
% HRP \omega_star effects over-predict stabilization
%      -> \omega_star; \omega_star/2 => all n modes near marginally stability
%      -> \omega_star/4; \omega_star/8; \omega_star/16 in figure
%      -> \omega_star/8 is most consistent with NIMROD calculations
% Conclusion: profile and poloidal effects are important
%
% o (2) includes resistive destabilization and thus requires a small ad-hoc
%   adjustment of \omega_star.
% o However, Given ad-hoc adjustments must be made to analytics, using the
%   simpler (1) is as good as (2)
% o Full computation best

\section{FLR parameter scan}
\label{sec:densityScan}

In order to ascertain the role of two-fluid effects further, we vary the
density and temperature while scaling parameters to maintain
constant $\beta$ and $S$ (other dissipation parameters are scaled relative to
resistivity). In experimental discharges, the zeroth-order effect on the edge
of modifying the density is modify the plasma collisionality and thus the
bootstrap current. This causes a transition from ballooning-like modes at high
density to peeling-like modes at low density. Our computations use a fixed
equilibrium current and thus are not sensitive to this effect. Instead, we
isolate the effect of the modification of $\rho_i$ (and the associated ion
skin depth, $d_i$) within the context of our two-fluid, first-order FLR model.
This contrasts with Ref.~\cite{xu14} that examines these effects in concert
and Ref.~\cite{Zhu12} that considers only the effect of modifications
to the current profile.

\begin{figure}
  \centering
  \includegraphics[width=8cm]{densityScan}
  \vspace{-4mm}
  \caption{[Color online]
  Growth rates computed by NIMROD on the `Meudas-1' case with the 1f model
  with parallel Braginskii closures (1f-Par; single-fluid limit), and three
  different densities with the two-fluid, FLR model with parallel Braginskii
  closures (2f-Par-FLR): the core densities are $n_e=3\times10^{19}$,
  $6\times10^{19}$ and $1.2\times10^{20}\;m^{-3}$.
  Associated NIMROD data available in Ref.~\cite{king16Z}.}
  \label{densityScan}
\end{figure}

Figure \ref{densityScan} shows the linear growth rate from NIMROD computations
for resistive-MHD and three two-fluid cases with varying densities (the
core density is varied from $3\times10^{19}\;m^{-3}$ to
$1.2\times10^{20}\;m^{-3}$ where $6\times10^{19}\;m^{-3}$ is the value used in
the prior computations discussed in the paper). The effect of
drift-stabilization is stronger at low densities ($\rho_i$ scales as
$1/\sqrt{n_i}$ at constant $\beta$) as expected. The cases with
$n_e=3\times10^{19}\;m^{-3}$ are not plotted for $n_\phi>18$ as the most unstable
mode is no longer the peeling-ballooning mode but rather a dominatly
electro-static mode related to the ITG as studied in the context NIMROD
two-fluid advance in Ref.~\cite{Schnack13}. This region is excluded as
examination of ITG mode dynamics is outside the scope of this work.

% In order to ascertain the role of two-fluid effects further, we vary one of the
% two-fluid FLR parameters that set the ion gyroradius $\rho_i$: the ion skin
% depth, $d_i$, and the plasma $\beta$.  Keeping the Meudas-1 current and pressure
% profiles fixed, we scale the density and temperature profiles in such a way
% that the plasma pressure remains the same in all simulations. This modifies the
% ion skin depth at constant $\beta$. The current (including the bootstrap
% contribution) and the Lundquist and Prandtl numbers are fixed to isolate
% effects from the two-fluid, FLR model only. As shown in
% Fig.~\ref{densityScan}, the low density cases (large $d_i$, $\rho_i$) peak at
% lower $n$ and are stabilized. However, there is a high-$n$ destabilization that
% is related to the ITG drive as captured by the fluid model. This ITG drive has
% been studied further in Ref.~\cite{Schnack13}.  It is our experience that for
% reconstructed cases that include density and temperature measurements, the ITG
% drive is small - likely a result of ITG modification of the profiles to ensure
% at least marginal stability.




\section{Discussion and Conclusions}



Our method based on stabilizing forward and backward pass, resulted in improved accuracy over the baseline and it was able to predict optimal dampening, sharpness and tail-fatness before training. 
Our findings are coherent with the line of research that has established that stabilizing gradients and representations at initialization results in better performance \cite{glorot2010understanding, orthogonal_initialization, he2015delving, roberts2022principles, defazio2022scaling, bengio1994learning, hochreiter1997long, hochreiter2001gradient, arjovsky2016unitary, pascanu2013difficulty}. Moreover it gives an initial reply to the question raised by
\cite{surrogate2019, zenke2021remarkable}, which asked  for a theoretical justification of initialization and SG choice for Spiking Neural Networks. With a similar intention, \cite{rossbroich2022fluctuation} proposed an approach that guarantees sparsity of activity at initialization to pick the weights distribution at initialization, resulting in improved accuracy. Our method differs from theirs in that it starts from a principle of stability to derive constraints, instead of a principle of sparsity. It differs also in that we use it to define the SG shape at initialization, not only the weights distribution, and we can show mathematically how weights initialization is intertwined to the SG shape choice. Our results suggest that a tedious hyper-parameter grid-search can be often avoided by making use of sound and established principles of learning.

One of the conditions was designed to hit the most sensitive part of an SG, its center, which resulted in a low sparsity requirement at initialization. This is very uncommon in the Neuromorphic literature, since sparsity brings large energy gains \cite{henderson2020towards,blouw2019benchmarking, 9395703,taulsnn, rossbroich2022fluctuation}.
However, the energy gains of SNNs also come from their binary activity. A matrix-vector multiplication, with a $\mathbb{R}^{m\times n}$ matrix, has an energy cost of $mnE_{MAC}$ for a real vector, and of $mn\rho E_{AC}$ for a binary vector, where $\rho$ is the Bernouilli probability of the binary vector, and in our case the neuron firing rate, and $E_{AC}, E_{MAC}$ are the energies of an accumulate and a multiply-accumulate operation \cite{yin2021accurate, hunger2005floating}. Since MAC are more costly than AC, 31 times on a $45$nm complementary metal–oxide–semiconductor \cite{yin2021accurate, horowitz20141}, we have energy savings with any $\rho$, e.g., when all neurons fire ($\rho=1$) and when they fire half of the time steps ($\rho=1/2$). This gain does not depend on the simulation speed, since it compares a spiking and an analogue computation, at the same computation speed.
Typically requiring more sparsity through a sparsity encouraging loss term, leads to a measurable decrease in performance \cite{zenke2021remarkable, rossbroich2022fluctuation}. However we observed that it is actually possible to achieve higher performance with higher sparsity, by starting with a strong firing rate at initialization, since their synergy acts as a regularization mechanism. This was possible also because the sparsity encouraging loss term was introduced gradually, and because its contribution was kept comparable to the task loss towards the end of training.

We observed that the more complex the task is and the more complex the network to train is, the more drastic is the difference in performance of different SG shapes. It is known that learning is possible with a wide variety of SG shapes \cite{zenke2021remarkable} and the community has not yet settled for one shape or one method to reliably choose which SG to use in each case \cite{surrogate2019}. We showed how to apply a well known stability principle to the forward and backward pass of the simplest Spiking Neural Network, the LIF, as a starting point, but we think that the principles of good Neuromorphic initialization can be further elaborated, in order to tackle more complex tasks and networks.


 
\appendix
%--------------------------------------------------

\begin{acknowledgments}
We thank Carl Sovinec, Chris Hegna and Nate Ferraro for discussions involving
this paper and N.~Aiba for providing the Meudas-1 equilibrium. This material is
based on work supported by US Department of Energy, Office of Science, Office
of Fusion Energy Sciences under award numbers DE-FC02-06ER54875 and
DE-FC02-08ER54972. This research used resources of the National Energy Research
Scientific Computing Center, a DOE Office of Science User Facility supported by
the Office of Science of the U.S. Department of Energy under Contract
No.~DE-AC02-05CH11231.
\end{acknowledgments}
\bibliographystyle{apsrev4-1}
\bibliography{Biblio}

\end{document}
