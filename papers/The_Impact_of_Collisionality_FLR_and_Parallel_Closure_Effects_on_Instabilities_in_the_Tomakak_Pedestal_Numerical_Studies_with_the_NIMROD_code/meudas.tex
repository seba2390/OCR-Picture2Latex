%% LyX 2.0.6 created this file.  For more info, see http://www.lyx.org/.
%% 
%\documentclass[english,aps,superscriptaddress,showkeys,showpacs,prepri]{revtex4}
\documentclass[english,aps,superscriptaddress,showkeys,showpacs,prepri,twocolumn]{revtex4}
\usepackage[T1]{fontenc}
\pdfoutput=1
%\usepackage[latin9]{inputenc}
\usepackage[letterpaper]{geometry}
\geometry{verbose,tmargin=1in,bmargin=1in,lmargin=1in,rmargin=1in}
\setcounter{secnumdepth}{3}
\usepackage{units}
\usepackage{bbding}
\usepackage{amsmath}
\usepackage{amssymb}
\usepackage{graphicx}
\usepackage{esint}
\usepackage[colorinlistoftodos,prependcaption,textsize=tiny]{todonotes}

\usepackage[utf8]{inputenc}
\usepackage{lmodern} % load a font with all the characters

\makeatletter

%%%%%%%%%%%%%%%%%%%%%%%%%%%%%% LyX specific LaTeX commands.
%% Because html converters don't know tabularnewline
\providecommand{\tabularnewline}{\\}

%%%%%%%%%%%%%%%%%%%%%%%%%%%%%% Textclass specific LaTeX commands.
\@ifundefined{textcolor}{}
{%
 \definecolor{BLACK}{gray}{0}
 \definecolor{WHITE}{gray}{1}
 \definecolor{RED}{rgb}{1,0,0}
 \definecolor{GREEN}{rgb}{0,1,0}
 \definecolor{BLUE}{rgb}{0,0,1}
 \definecolor{CYAN}{cmyk}{1,0,0,0}
 \definecolor{MAGENTA}{cmyk}{0,1,0,0}
 \definecolor{YELLOW}{cmyk}{0,0,1,0}
}

%%%%%%%%%%%%%%%%%%%%%%%%%%%%%% User specified LaTeX commands.
%\usepackage[sort&compress,numbers]{natbib} 
\bibliographystyle{apsrev4-1} 
\usepackage{doi}
\usepackage{hyperref}

\makeatother

\usepackage{babel}
\begin{document}

\title{
The Impact of Collisionality, FLR and Parallel Closure Effects on Instabilities
in the Tomakak Pedestal:   Numerical Studies with the NIMROD code
}
%Edge NIMROD studies and verification:\\ideal-MHD modification by collisionality, \\FLR and parallel-closure effects.}

\author{J. R. King}
\affiliation{Tech-X Corporation, 5621 Arapahoe Ave. Boulder, CO 80303, USA}

\author{A. Y. Pankin}
\affiliation{Tech-X Corporation, 5621 Arapahoe Ave. Boulder, CO 80303, USA}

\author{S. E. Kruger}
\affiliation{Tech-X Corporation, 5621 Arapahoe Ave. Boulder, CO 80303, USA}

\author{P. B. Snyder}
\affiliation{General Atomics, PO Box 85608, San Diego, CA 92186–5608, USA}

\begin{abstract}
The extended-MHD NIMROD code [C.R.~Sovinec and J.R.~King, J.~Comput.~Phys.~{\bf
229}, 5803 (2010)] is verified against the ideal-MHD ELITE code [H.R.~Wilson
\textit{et al.}~Phys.~Plasmas {\bf 9}, 1277 (2002)] on a diverted tokamak
discharge. When the NIMROD model complexity is increased incrementally,
resistive and first-order finite-Larmour radius effects are destabilizing and
stabilizing, respectively. The full result is compared to local analytic
calculations which are found to overpredict both the resistive destabilization
and drift stabilization in comparison to the NIMROD computations.
Published version: Phys. Plasmas 23, 062123 (2016) [\url{http://dx.doi.org/10.1063/1.4954302}]
\end{abstract}

\keywords{edge localized modes, peeling-ballooning modes, extended-MHD,
code verification}

\pacs{52.30.Ex 52.35.Py, 52.55.Fa, 52.55.Tn, 52.65.Kj}
\maketitle

%--------------------------------------------------
\newcommand\UU{\mathcal{U}}
\newcommand\SC{\mathcal{S}}
\newcommand\MM{\mathbb{M}}
\newcommand\MO{\mathbb{M}\mkern-4mu\downarrow}
\newcommand\Idx{\mathsf{Idx}\,}
\newcommand\Cns{\mathsf{Cns}\,}
\newcommand\Typ{\mathsf{Typ}\,}
\newcommand\Pos{\mathsf{Pos}\,}
\newcommand\Decor{\mathsf{Decor}\,}

\newcommand\Set{\mathcal{S}et}

\newcommand\Id{\mathsf{Id}}
\newcommand\Pb{\mathsf{Pb}\,}
\newcommand\Slice{\mathsf{Slice}\,}
\newcommand\Pd{\mathsf{Tree}\,}
\newcommand\Free{\mathsf{Free}\,}
\newcommand\Slc{\mathsf{Slc}\,}

\newcommand{\ooGrp}{\infty\mhyphen\mathsf{Grp}}
\newcommand{\preooCat}{\mathsf{pre}\mhyphen\infty\mhyphen\mathsf{Cat}}
\newcommand{\ovr}{\mkern-8mu\downarrow}
\newcommand{\smovr}{\mkern-4mu\downarrow}

\newcommand{\dsum}[1]{\textstyle{\sum_{(#1)}}\,}
\newcommand{\dprod}[1]{\textstyle{\prod_{(#1)}}\,}
\mathchardef\mhyphen="2D

\newcommand\refl{\mathsf{refl}}
\newcommand\ttt{\mathsf{tt}}
\newcommand\ctr{\mathsf{ctr}\,}
\newcommand\wit{\mhyphen\mathsf{wit}}
\newcommand\coh{\mhyphen\mathsf{coh}}
\newcommand\alg{\mhyphen\mathsf{alg}}

% Antoine's over commands

\newcommand{\da}{{\downarrow}}

\newcommand\MMd{\mathbb{M\da}}
\newcommand\Idxd{\mathsf{Idx}\da\,}
\newcommand\Cnsd{\mathsf{Cns}\da\,}
\newcommand\Typd{\mathsf{Typ}\da\,}
\newcommand\Posd{\mathsf{Pos}\da\,}

\newcommand\upetad{\upeta\da}
\newcommand\upmud{\upmu\da}

\newcommand\lfd{\operatorname{lf\da}}
\newcommand\ndd{\operatorname{nd\da}}

\newcommand\Idd{\mathsf{Id\da}\,}
\newcommand\Pbd{\mathsf{Pb\da}\,}
\newcommand\Sliced{\mathsf{Slice\da}\,}
% end 

\newcommand\Unit{\top}
\newcommand\Empty{\bot}
\newcommand\botelim{\bot\mhyphen\mathsf{elim}}
\newcommand\etapos{\upeta\mhyphen\mathsf{pos}\,}
\newcommand\etaposelim{\upeta\mhyphen\mathsf{pos}\mhyphen\mathsf{elim}\,}
\newcommand\etadec{\upeta\mhyphen\mathsf{dec}\,}
\newcommand\etadecd{\upeta\mhyphen\mathsf{dec}\da\,}
\newcommand\mupos{\upmu\mhyphen\mathsf{pos}\,}
\newcommand\muposfst{\upmu\mhyphen\mathsf{pos}\mhyphen\mathsf{fst}\,}
\newcommand\mupossnd{\upmu\mhyphen\mathsf{pos}\mhyphen\mathsf{snd}\,}
\newcommand\gammaposinl{\upgamma\mhyphen\mathsf{pos}\mhyphen\mathsf{inl}\,}
\newcommand\gammaposinr{\upgamma\mhyphen\mathsf{pos}\mhyphen\mathsf{inr}\,}
\newcommand\gammaposelim{\upgamma\mhyphen\mathsf{pos}\mhyphen\mathsf{elim}\,}

\newcommand\lf{\mathsf{lf}\,}
\newcommand\nd{\mathsf{nd}\,}
\newcommand\fst{\mathsf{fst}\,}
\newcommand\snd{\mathsf{snd}\,}
\newcommand\inl{\mathsf{inl}\,}
\newcommand\inr{\mathsf{inr}\,}

\newcommand\iscontr{\mathsf{is}\mhyphen\mathsf{contr}\,}
\newcommand\ismult{\mathsf{is}\mhyphen\mathsf{mult}\,}
\newcommand\isfibrant{\mathsf{is}\mhyphen\mathsf{fibrant}\,}
\newcommand\carmult{\mathsf{car}\mhyphen\mathsf{is}\mhyphen\mathsf{mult}\,}
\newcommand\relfib{\mathsf{rel}\mhyphen\mathsf{is}\mhyphen\mathsf{fibrant}\,}
\newcommand\isalgebraic{\mathsf{is}\mhyphen\mathsf{algebraic}\,}

\newcommand\OpType{\mathsf{OpetopicType}\,}
\newcommand\OvrOpType{\da\mathsf{OpType}\,}
\newcommand\Car{\mathcal{C}\,}
\newcommand\Rel{\mathcal{R}\,}

\newcommand\lflf{\operatorname{lf-lf}}
\newcommand\ndlf{\operatorname{nd-lf}}

\newcommand{\commentt}[1]{}


%%% Local Variables:
%%% mode: latex
%%% TeX-master: "lics-article"
%%% End:
   % Useful abbreviations
% \leavevmode
% \\
% \\
% \\
% \\
% \\
\section{Introduction}
\label{introduction}

AutoML is the process by which machine learning models are built automatically for a new dataset. Given a dataset, AutoML systems perform a search over valid data transformations and learners, along with hyper-parameter optimization for each learner~\cite{VolcanoML}. Choosing the transformations and learners over which to search is our focus.
A significant number of systems mine from prior runs of pipelines over a set of datasets to choose transformers and learners that are effective with different types of datasets (e.g. \cite{NEURIPS2018_b59a51a3}, \cite{10.14778/3415478.3415542}, \cite{autosklearn}). Thus, they build a database by actually running different pipelines with a diverse set of datasets to estimate the accuracy of potential pipelines. Hence, they can be used to effectively reduce the search space. A new dataset, based on a set of features (meta-features) is then matched to this database to find the most plausible candidates for both learner selection and hyper-parameter tuning. This process of choosing starting points in the search space is called meta-learning for the cold start problem.  

Other meta-learning approaches include mining existing data science code and their associated datasets to learn from human expertise. The AL~\cite{al} system mined existing Kaggle notebooks using dynamic analysis, i.e., actually running the scripts, and showed that such a system has promise.  However, this meta-learning approach does not scale because it is onerous to execute a large number of pipeline scripts on datasets, preprocessing datasets is never trivial, and older scripts cease to run at all as software evolves. It is not surprising that AL therefore performed dynamic analysis on just nine datasets.

Our system, {\sysname}, provides a scalable meta-learning approach to leverage human expertise, using static analysis to mine pipelines from large repositories of scripts. Static analysis has the advantage of scaling to thousands or millions of scripts \cite{graph4code} easily, but lacks the performance data gathered by dynamic analysis. The {\sysname} meta-learning approach guides the learning process by a scalable dataset similarity search, based on dataset embeddings, to find the most similar datasets and the semantics of ML pipelines applied on them.  Many existing systems, such as Auto-Sklearn \cite{autosklearn} and AL \cite{al}, compute a set of meta-features for each dataset. We developed a deep neural network model to generate embeddings at the granularity of a dataset, e.g., a table or CSV file, to capture similarity at the level of an entire dataset rather than relying on a set of meta-features.
 
Because we use static analysis to capture the semantics of the meta-learning process, we have no mechanism to choose the \textbf{best} pipeline from many seen pipelines, unlike the dynamic execution case where one can rely on runtime to choose the best performing pipeline.  Observing that pipelines are basically workflow graphs, we use graph generator neural models to succinctly capture the statically-observed pipelines for a single dataset. In {\sysname}, we formulate learner selection as a graph generation problem to predict optimized pipelines based on pipelines seen in actual notebooks.

%. This formulation enables {\sysname} for effective pruning of the AutoML search space to predict optimized pipelines based on pipelines seen in actual notebooks.}
%We note that increasingly, state-of-the-art performance in AutoML systems is being generated by more complex pipelines such as Directed Acyclic Graphs (DAGs) \cite{piper} rather than the linear pipelines used in earlier systems.  
 
{\sysname} does learner and transformation selection, and hence is a component of an AutoML systems. To evaluate this component, we integrated it into two existing AutoML systems, FLAML \cite{flaml} and Auto-Sklearn \cite{autosklearn}.  
% We evaluate each system with and without {\sysname}.  
We chose FLAML because it does not yet have any meta-learning component for the cold start problem and instead allows user selection of learners and transformers. The authors of FLAML explicitly pointed to the fact that FLAML might benefit from a meta-learning component and pointed to it as a possibility for future work. For FLAML, if mining historical pipelines provides an advantage, we should improve its performance. We also picked Auto-Sklearn as it does have a learner selection component based on meta-features, as described earlier~\cite{autosklearn2}. For Auto-Sklearn, we should at least match performance if our static mining of pipelines can match their extensive database. For context, we also compared {\sysname} with the recent VolcanoML~\cite{VolcanoML}, which provides an efficient decomposition and execution strategy for the AutoML search space. In contrast, {\sysname} prunes the search space using our meta-learning model to perform hyperparameter optimization only for the most promising candidates. 

The contributions of this paper are the following:
\begin{itemize}
    \item Section ~\ref{sec:mining} defines a scalable meta-learning approach based on representation learning of mined ML pipeline semantics and datasets for over 100 datasets and ~11K Python scripts.  
    \newline
    \item Sections~\ref{sec:kgpipGen} formulates AutoML pipeline generation as a graph generation problem. {\sysname} predicts efficiently an optimized ML pipeline for an unseen dataset based on our meta-learning model.  To the best of our knowledge, {\sysname} is the first approach to formulate  AutoML pipeline generation in such a way.
    \newline
    \item Section~\ref{sec:eval} presents a comprehensive evaluation using a large collection of 121 datasets from major AutoML benchmarks and Kaggle. Our experimental results show that {\sysname} outperforms all existing AutoML systems and achieves state-of-the-art results on the majority of these datasets. {\sysname} significantly improves the performance of both FLAML and Auto-Sklearn in classification and regression tasks. We also outperformed AL in 75 out of 77 datasets and VolcanoML in 75  out of 121 datasets, including 44 datasets used only by VolcanoML~\cite{VolcanoML}.  On average, {\sysname} achieves scores that are statistically better than the means of all other systems. 
\end{itemize}


%This approach does not need to apply cleaning or transformation methods to handle different variances among datasets. Moreover, we do not need to deal with complex analysis, such as dynamic code analysis. Thus, our approach proved to be scalable, as discussed in Sections~\ref{sec:mining}.
\section{Verification benchmark}
\label{sec:benchmark}

% Relative to the tokamak core, the characteristic time and spatial scales
% are compressed.  However, type-I ELMs still have the instability time
% scale associated with the fast crash is an order of magnitude faster
% than the transport-time scale associated with the processes that govern
% the build up of the pedestal structure.   This separation of time scales
% still allows the standard decomposition of studying the linear
% instabilities about an equilibrium that is used in core modes as well.
% 
% Like the core modes, these long-wavelength instabilities are dominated
% by the stiffness in the ideal MHD terms, even for the cases when they
% may be strictly ideal stable.  Multiple numerical methods have been
% developed to handle this stiffness for both linear and nonlinear codes.
% For the nonlinear codes, one numerical advantage is to 
% separate the fields into steady-state (e.g. the reconstructed fields)
% and time-dependent parts.  The pure steady-state terms are analytically
% eliminated resulting in the largest terms in the system to be removed
% from the numerical computations.
% 
% Although typically only MHD-force balance (a
% Grad-Shafranov solution) is strictly enforced for the steady state, in practice
% all fields associated are time independent. This effectively assumes the
% presence of implicit (in the sense that they are calculable but not calculated)
% sources, fluxes and sinks.  With these assumptions, if the code is run on a
% MHD-stable case, the fields do not change.  Alternatively, the initial fields
% are self-consistently modified by the presence of unstable modes. 
% {\bf SEK: OK -- I think this is a better place to put the discussion, in
%   the end this is confusing unless there is an appendix to explain
%   things in detail.  We need to discuss whether we want to add it.  I
% think not as this is really an EHO discussion.}
% 
% There is no technical reason to make this time-scale decomposition - the NIMROD
% code has the capability to compute the extended-MHD evolution of the
% reconstructed fields. However, it is well-known that physical mechanisms
% outside the scope of the extended-MHD model mediate tokamak transport such as
% neoclassical bootstrap current, toroidal viscosity, and poloidal flow damping;
% neutral beam and RF drive; turbulence; and coupling to the scrape-off layer
% (SOL), neutrals, impurities and the material boundary. Including these effects
% requires explicit calculation of the sources, fluxes and sinks. These
% transport-type calculations are possible and are becoming practical (e.g.
% \cite{held15}), but this sort of integrated modeling remains in the future.

We begin with a study of a high resolution, lower-single-null, JT-60U-like
equilibrium (`Meudas-1'), which was originally employed in a benchmark of the
MARG2D and ELITE codes \cite{Aiba07}, including a
close approach to the X-point \cite{Snyder09}.
This extends previous benchmarks~\cite{Burke10} of ELITE and NIMROD as it
includes diverted magnetic topology and a higher edge safety factor
($q_{95}=6.74$, the safety factor at 95\% of the normalized poloidal flux) that
leads to increased resolution requirements. An ideal-MHD limit is achieved in
NIMROD by using flat density and resistivity profiles inside the last closed
flux surface (LCFS) with small resistivity, $S=10^8$ where $S$ is the Lundquist
number  ($S=\tau_R/\tau_A$), $\tau_A$ is the Alfvén time ($\tau_A=R_o/v_A$),
$v_A$ is the Alfvén velocity ($B/\sqrt{m_i n_i \mu_0}$), $\tau_R$ is the
resistive diffusion time ($\tau_R=R_o^2 \mu_0/\eta$), $R_o=2.936 m$ is the
radius of the magnetic axis, $\eta$ is the electrical resistivity, $\mu_0$ is
the permeability of free space, $m_\alpha$ is a species mass (the $\alpha$
subscript denotes ions or electrons in this work), and $n_\alpha$ is a species
density. The deuteron mass ($m_i = 3.34\times 10^{-27} kg$) is used. In order to
reproduce the vacuum response model outside the LCFS that is used by ELITE, a
low density ($0.01$ of the core density) and high resistivity ($10^7$ times the
core resistivity) is prescribed beyond the LCFS (more details on these
approximations are in Ref.~\cite{Burke10}).  

\begin{figure}
  \centering
  \includegraphics[width=8cm]{ELITEComparison}
  \vspace{-4mm}
  \caption{[Color online]
  Growth rates for the `Meudas-1' benchmark. ELITE with $\Gamma=5/3$ and
  $\Gamma=0$ are compared against results from NIMROD with $\Gamma=5/3$).
  Associated NIMROD data available in Ref.~\cite{king16Z}.}
  \label{fig:ELITEComp}
\end{figure}

\begin{figure}
  \centering
  \includegraphics[width=8cm]{idealConv}
  \vspace{-4mm}
  \caption{[Color online]
  Spectral convergence of the NIMROD code for the ideal-like parameters. 
  The maximum polynomial degree (P) of the basis functions composing the 
  spectral elements in increased in each subsequent line plotted.
  Associated NIMROD data available in Ref.~\cite{king16Z}.}
  \label{idealConv}
\end{figure}

The normalized growth rates ($\gamma \tau_A$ where the linearized mode grows as
$\text{exp}[\gamma t]$) vs.~toroidal mode number ($n_\phi$) from NIMROD and ELITE are
plotted in Fig.~\ref{fig:ELITEComp}.  There is good agreement between the
codes except for $n_\phi$=4 where there is a 27\% relative difference. All
other cases have a relative difference of less than 8\% with typical
differences of 5\%. The NIMROD convergence in terms of the maximum polynomial
order of the spectral elements is shown in Fig.~\ref{idealConv}. Convergence is
most challenging at high wavenumbers where the resolution requirements are most
stringent (the poloidal mesh is composed of $72\times512$ spectral elements).
%SEK: Great point, but total troll bait for reviewers
These cases converge from the unstable side where the growth rate decreases with
enhanced resolution. Thus the excellent agreement between NIMROD and ELITE at
high $n_\phi$ in Fig.~\ref{fig:ELITEComp} may be spurious and indicate that
slightly more resolution is required for $n_\phi$>25, however, the 
shown growth rates are likely within 5\% of their converged values.
Studying nearly ideal cases with extended MHD codes such as NIMROD is challenging 
given the vanishingly small dissipation operators, and convergence is achieved
more quickly with the additional non-ideal terms in the extended-MHD equations,
as in the cases in Sec.~\ref{sec:xMHD}.

% Relative to modeling with extended MHD, ideal-MHD convergence is more challenging 
% given the vanishingly small dissipation operators and convergence is 
% achieved more quickly with all other model equations shown in this work.

\begin{figure}
  \includegraphics[width=8cm]{meudas_n11_BR}
  \caption{[Color online]
  Poloidal cross section of the radial magnetic field component of the
  $n_\phi=11$ peeling-ballooning mode from the `Meudas-1' benchmark case. }
  \vspace{-4mm}
  \label{meudas_n11_BR}
\end{figure}

Figure \ref{meudas_n11_BR} shows a poloidal cross section of the magnetic
($B_R$) eigenmode.  The mode develops an `interference-pattern' structure near
the X-point when inboard and outboard finger-like structures overlap. The
finite-element-mesh nodes are superimposed atop the smallest-scale sub-figure.
As established by Fig.~\ref{idealConv}, this simulation is spatially
and temporally converged. The high resolution required to resolve these
high-$q_{95}$, diverted cases
motivated development of memory-scaling improvements in the NIMROD code.
   
\section{Increasing the model complexity}
\label{sec:xMHD}

% o Don't emphasize M3D-C1, just ref Ferraro
% o 1fl effects:
%  - Describe profile choice for n - avoid ITG like modes w/ 2f parameters
%  - Using nu_perp = k_perp = D_n = elecd(psin=1)
%  - Using Z=3; pe_frac=0.75 => Te = Ti; Gamma=5/3
%  - Effect of density profile - local gamma * tau_a const,
%      -> normalized to core tau_a (unchanged)
%      -> local tau_a ~ sqrt(n) smaller => growth rates uniformly larger
%  - Effect of resistivity profile: high-n destabilization, low-n stabilization
% o parallel effects:
%  - Almost no effect on gamma
%  - expected from ELITE results which show mode \gamma is largely 
%      insensitive to \Gamma (0 in Ferraro, 5/3 and inf here)
% o 2fl FLR effects:
%  - Including full FLR model: Gen. Ohm's law; ion gyrovisc; cross heat fluxes
%  - Drift stabilization at high-n; crosses ideal results @ n~24
%  - Little effect at low n (n<10)
%  - (No destabilization as predicted by HRP)
%  - For this case: computations w/ and w/out x-heat flux give same result
%     -> Modest beta (0.1%); consistent with FLR theory from King and Kruger
% o Reference: Ping, Xu13

Initial-value extended-MHD computations, while more computationally expensive
than ideal-MHD calculations with ELITE, are able to include additional effects
that are more representative of experimental conditions:
% Initial-value extended-MHD computations, while more computationally expensive
% than ideal eigenvalue calculations such as ELITE, are able to include
% additional effects that are more representative of experimental conditions:
\begin{itemize} \cramplist \zapspace
\item 
Instead of a magneto-static vacuum outside the LCFS, NIMROD includes a
cold-plasma region.
\item
Finite-dissipation effects such as resistivity, viscosity and thermal
conduction.
\item 
Density and temperature profiles that prescribe the associated
drifts and resistivity profile.
\item
Parallel-closure effects that represent large diffusivities oriented along the
magnetic field.
\item
First-order FLR-closure and two-fluid effects.
\end{itemize} 
Reference \cite{Ferraro10} investigates the first three of these effects with
the Meudas-1 case. We further this work by including the last two effects.  The
first three effects are also re-investigated and different modifications to the
growth rates are found. These differences are expected as slightly different
profiles for density and temperature are used in order to avoid spurious
electrostatic modes with the two-fluid, first-order FLR model. 
Ref.~\cite{Ferraro10} avoided these modes by employing only the
single-fluid resistive-MHD model.
% {\bf SEK: I think this is review bait:  Why didn't
%   Nate see spurious modes?  What is the source of these spurious modes?
% Etc.  I don't see a clean way of discussing this}

In calculations beyond the ideal model, the density profile is
$n_e(\psi_n)=n_{e0} (p(\psi_n)/p_0)^{0.3}$ where
$n_{e0}=6\times10^{19}\;m^{-3}$ and $\psi_n$ is the normalized poloidal flux.
Here $p$ is the pressure as prescribed by the ideal gas law, $p=n_e T_e+ n_i
T_i$, where $T_\alpha$ is a species' temperature. This choice does not match
Ref.~\cite{Ferraro10}, but effectively avoids spurious electrostatic modes with
the two-fluid, first-order FLR model. These spurious modes are discussed in 
more detail in Sec.~\ref{sec:densityScan}.
% {\bf SEK: Ditto the previous comment} 
The pressure-profile gradient is specified by the Meudas-1 case and setting the
edge temperature to $200 eV$ leads to a core temperature of $5806 eV$.  The
edge temperature is modified independent of the MHD-stability (determined by
$p^\prime$) by the transformation $p(\psi_n) \rightarrow p(\psi_n) + p_c$ where
$p_c$ is a constant. Without scrape-off-layer (SOL) profiles of density and temperature, which are
not included in standard reconstructions, this choice of the edge temperature
also specifies the temperature outside the LCFS.  SOL profiles are required to
get both the correct profiles outside the LCFS, which determines the vacuum
response, and simultaneously set the correct resistivity at the mode resonant
location when Spitzer resistivity is used.  The non-ideal viscous, conductive
and particle diffusivities are set to a small value, one tenth of the
resistivity at the LCFS. 

\begin{figure}
  \centering
  \includegraphics[width=8cm]{modelScan}
  \vspace{-4mm}
  \caption{[Color online]
  Growth rates computed by NIMROD on the `Meudas-1' case with four different
  models: ideal MHD (ideal), resistive MHD with reconstructed density,
  temperature, and Spitzer-resistivity profiles (1f), the 1f model with parallel
  Braginskii closures (1f-Par), the 1f-Par model with ion gyroviscosity, a full
  extended MHD Ohm's law (including the Hall, $\nabla p_e$, and electron inertia
  terms) and separate temperature evolution equations with cross-heat fluxes
  (2f-Par-FLR).
  Associated NIMROD data available in Ref.~\cite{king16Z}.}
  \label{modelScan}
\end{figure}

Figure~\ref{modelScan} plots the growth rates computed by NIMROD on the
`Meudas-1' case with four different models: ideal MHD (ideal), resistive MHD with
reconstructed density, temperature, and Spitzer-resistivity profiles (1f), the
single-fluid model with parallel Braginskii closures (1f-Par) where the 
parallel viscosity is
\begin{multline}
\mathbf{\Pi}_{\parallel i}=
  m_i n_i \nu_{\parallel i}
  \left(\hat{\mathbf{b}}
  \hat{\mathbf{b}}-\frac{1}{3}\mathbf{I}\right) \\
  \times\left(3\hat{\mathbf{b}}\cdot\nabla\mathbf{v}_{i}\cdot
  \hat{\mathbf{b}}-\nabla\cdot\mathbf{v}_{i}\right)\;,
\end{multline}
and the parallel heat flux vector is
\begin{equation}
\mathbf{q}_{\parallel \alpha}=
  -n_i \chi_{\parallel \alpha}
  \hat{\mathbf{b}}\hat{\mathbf{b}}\cdot\nabla T_{\alpha}\;,
\end{equation}
and a two-fluid model with parallel closures that includes ion gyroviscosity
\begin{multline}
\mathbf{\Pi}_{\times i}=
  \frac{m_{i}p_{i}}{4ZeB}[\hat{\mathbf{b}}
    \times\mathbf{W}\cdot
    \left(\mathbf{I}+3\hat{\mathbf{b}}\hat{\mathbf{b}}\right) \\
    -\left(\mathbf{I}+3\hat{\mathbf{b}}\hat{\mathbf{b}}\right)
    \cdot\mathbf{W}\times\hat{\mathbf{b}}]\;,
\end{multline}
where $\mathbf{W}$ is the rate-of-strain tensor
\begin{equation}
\mathbf{W}=\nabla\mathbf{v}+\nabla\mathbf{v}^{T}
           -(2/3)\mathbf{I}\nabla\cdot\mathbf{v}\;,
\end{equation}
a full extended-MHD Ohm's law,
\begin{equation}
\mathbf{E}=
  -\mathbf{v}\times\mathbf{B}+\frac{\mathbf{J}\times\mathbf{B}}{n_ee}
  -\frac{\nabla p_{e}}{n_ee}+\eta\mathbf{J}
  +\frac{m_{e}}{n_e e^2}\frac{\partial\mathbf{J}}{\partial t}\;,
\end{equation}
and separate temperature evolutions with cross-heat fluxes,
\begin{equation}
\mathbf{q}_{\times \alpha}=
  \frac{5p_{\alpha}}{2q_{\alpha}B}
  \hat{\mathbf{b}}\times\nabla T_{\alpha}\;,
\end{equation}
(2f-Par-FLR). Here $v_i$ is the bulk-ion flow, $\nu_{\parallel i}$ is the ion parallel
diffusivity, $\chi_{\parallel \alpha}$ is the species' parallel diffusivity,
$q_\alpha$ ($e$) is the species' (electron) electric charge,  $\mathbf{J}$ is
the current density, $\hat{\mathbf{b}}$ is the magnetic unit vector
($\hat{\mathbf{b}} = \mathbf{B}/B$), and $\mathbf{I}$ is the identity tensor.
An effective ion charge ($Z$) of three is assumed. Each model builds
on the previously listed and includes all prior terms. Electron viscosity is
not included.


% REMOVED FROM INTRO
% For peeling-ballooning modes, the prevailing analytic description of these
% effects is described in Ref.~\cite{Hastie03}. 
% This conventional picture expects two-fluid and FLR effects are destabilizing
% to modes with low toroidal mode number but stabilizing at high toroidal mode
% number. 
% We compare our results with this simple model of drift stabilization where
% $\gamma_{MHD}^2 = \omega (\omega - \omega_*)$.  Here $\gamma_{MHD}$ is the
% ideal-MHD growth rate, $\omega$ is the complex mode frequency and $\omega_*$ is
% the diamagnetic-drift frequency.

% Closure expressions for the viscosity tensor, $\mathbf{\Pi}$, and heat flux,
% $\mathbf{q}$. These expressions are typically decomposed into three parts:
% parallel (large diffusivities oriented along the magnetic field), cross (FLR
% ordered contributions), and perpedicular (small dissipative terms)

When a density profile is included, the normalized growth rate of the mode
remains constant with the single-fluid model when the Alfvén time is computed
at the mode resonant surface. Our calculations are normalized by the Alfvén
time as computed with the values at the magnetic axis. These values remain constant
when a density profile is included such that the normalized mode growth rate is
effectively increased. 

The effect of the resistivity profile is more nuanced.  Finite
resistivity allows for reconnection within the model by relaxing the
frozen-flux constraint. This permits resistive-ballooning modes that
grow faster than their ideal counterparts. Alternatively, the resistive
dissipation can stabilize the modes by acting as a dissipative term and
modifying the response outside the LCFS from that of a vacuum to that of
a plasma. Figure 5 of Ref.~\cite{Ferraro10} shows that the plasma
response is stabilizing relative to a vacuum region. Overall, these
effects stabilize the ballooning modes at low-$n_\phi$ and destabilize
the modes at high-$n_\phi$ as seen in Fig.~\ref{modelScan} when
comparing the single fluid and ideal normalized growth rates. The effect
of including small particle, viscous, and thermal diffusivities on the
resistive-MHD mode is small (not shown).

Including the Braginskii parallel closures with large coefficients
($\chi_{\parallel e} = 10^9\times \eta_0/\mu_0$ and  $\chi_{\parallel i} =
\nu_{\parallel i} = 10^8\times \eta_0/\mu_0$) has little effect on the growth
rates.  When thermal conduction is large, the temperature quickly equilibrates
along field lines to produce an isothermal response.  Thus NIMROD computations
that show little effect from the parallel closures are consistent with the ELITE
results that find the shape of the mode growth-rate spectrum is largely not
sensitive to the value of the ratio of specific heats, $\Gamma$, as seen in
Fig.~\ref{fig:ELITEComp}. When the response to the compressible motion
associated with the sound wave is eliminated ($\Gamma=0$ in the ELITE 
computations), the growth rate is slightly enhanced relative to the adiabatic limit.
If the growth rates are modified beyond this small effect, it would be
an indication that changes in the energy equation can impact the MHD response, 
and thus that closure effects are significant. Although the short-mean-free
path Braginskii-like closure \cite{Braginskii,Catto04} is used in our NIMROD
compuatations, we note that other closures, such as long-mean-free path
\cite{Ramos11} or the general approach of solving drift-kinetic equation
\cite{held15} for a closure can be applied. However, for this case where the
parallel-closure effects are negligible, different parallel closures are
unlikely to significantly modify the results.

With the full two-fluid, FLR model, there is a stabilizing effect on the
intermediate and high-$n_\phi$ modes as expected from analytic
treatments \cite{Hastie03}. A common reduced-MHD approximation for
first-order FLR closures is to assume that the ion-gyroviscous force and
the advection by the diamagnetic drift exactly cancel (this is
colloquially known as the gyroviscous cancellation, see, e.g.
\cite{Coppi64}). However, as discussed by Ramos in
Ref.~\cite{Ramos:2007cn}, ``these cancellations are only partial and not
very useful in practice for general magnetic geometries\ldots''.  This
cancellation is only valid with a large, uniform guide field, without
curvature terms (slab approximation), and in the low-$\beta$ limit (see,
e.g. \cite{King14}).  Use of the full gyroviscous operator is critical
as in addition to effects that appear as diamagnetic drifts, it also
contains terms proportional to first-order FLR magnetic-curvature and
grad-B drifts \cite{King11}. Our modeling contains not just the
first-order FLR drifts from ion gyroviscosity, but also the drifts from
the cross-heat fluxes that enter the equations on the same order.  

Just as the large balance of the ideal-MHD forces can cause
numerical difficulties with the ideal-MHD force operator, the
partial gyroviscous cancellation leads to similar numerical 
pitfalls and verification is important.
The implementation of these terms in NIMROD are verified in cylindrical and
slab magnetic geometries against analytic calculations for
tearing~\cite{Sovinec10,King11,King14} and
ion-temperature-gradient~\cite{Schnack13} (ITG) modes. 
% {\bf SEK: I don't know
%   why you have Schnack13 associated with tearing instead of ITG.  Also,
%   Schnack really didn't use Coppi67 directly but rather had to modify it
%   for the verification purposes, but it's better to just reference his
% paper}.  
% JRK -> previous references Coppi67 for ITG; others for general verfication
To test the effect of these terms against analytic theory
in full tokamak magnetic geometry, we perform a qualitative
comparison with analytic theory in Sec.~\ref{sec:analyticComp}.
Full verification is precluded by the approximations
made in the analytic theories.
%For verification in full tokamak magnetic geometry, we perform a qualitative
%comparison with analytic theory in Sec.~\ref{sec:analyticComp}.

The first-order FLR model is valid for the toroidal modes presented although
this is not necessarily the case for general edge mode modeling.  Assuming
$k\simeq(q/r+1/R)n_\phi$ and evaluating local quantities at the peak of the
mode eigenfunction ($\psi_n = 0.969$) on the outboard midplane, the normalized
ion-gyroradius ($\rho_i=\sqrt{\Gamma m_i T_i}/ZeB$) as a function of toroidal
mode is $k \rho_i \simeq 0.0055n_\phi$.  If the first-order FLR assumption is
violated, more-complex full-ion-orbit kinetic modeling is required to simulate
large-fluctuation ELM dynamics.  For this case, computations produce
approximately the same results with and without the cross-heat fluxes.  This is
consistent with drift analytics \cite{King14} that shows that the cross-heat
flux is only significant at very low values of plasma $\beta$ (here the
$\beta=2\mu_0p/B^2$ is 0.1\%).  Perhaps coincidentally for this case, the
effects of the resistive destabilization and drift stabilization largely
counteract one another. The drift stabilized growth rates are not less than the
ideal calculation until $n_\phi \gtrsim 24$.  

% Old text:
%
% In addition to the ideal MHD benchmark, the M3D-C1 code
% investigated the effects of including a density gradient profile and Spitzer
% resistivity. Our NIMROD studies have furthered this work to
% include two-fluid effects through the generalized Ohm's law (including the
% Hall, $\nabla p_e$, and electron inertia terms), ion gyroviscosity, highly
% anisotropic thermal conduction, and parallel ion
% viscosity.  When the model includes a generalized Ohm's
% law, the result qualitatively agrees with the analytic theory of
% Ref.~\cite{Hastie03} used for ELITE's {\em ad hoc} model; we find a
% destabilization of the intermediate-$n$ mode spectrum and a stabilization of
% the high-$n$ mode spectrum. When ion gyroviscosity is included, there is a
% stabilization of the high-$n$ mode spectrum, consistent with recent results
% from the BOUT++ code \cite{Xu13}. The high-$n$ mode
% spectrum that results from the use of a extended MHD model prevents energy from
% accumulating on the smallest resolvable scales and corrupting the simulations.
% These results are shown in Fig.~\ref{fig:ELITEComp}.

\section{Comparison to drift analytics}
\label{sec:analyticComp}

\begin{figure}
  \centering
  \includegraphics[width=8cm]{FLRidealComp}
  \vspace{-4mm}
  \caption{[Color online]
  Growth rates computed by NIMROD with the ideal, single-fluid (1f) and full
  FLR (2f-Par-FLR) models in comparison to the analytic dispersion relation of
  the FLR-stabilized, ideal ballooning mode with full and reduced $\omega_{*\alpha}$ 
  values. Associated NIMROD data available in Ref.~\cite{king16Z}.}
  \label{FLRidealComp}
\end{figure}

To gain insight into our two-fluid computations, we compare with analytic
descriptions of drift stabilization of the ballooning mode.  The dispersion
relation for drift stabilizations of the ideal, incompressible ballooning mode
is
\begin{equation}
\omega (\omega - \omega_{*i}) = - \gamma_I^2
\label{eq:driftBalloon}
\end{equation}
where $\gamma_I$ is the ideal growth rate in the absence of drift effects and
$\omega$ is the complex frequency ($\omega=i\gamma$) \cite{tang82}.  The
diamagnetic drift velocity is typically defined as the
perpendicular-to-the-magnetic-field flow contribution from the pressure
gradient within the context of a two-fluid, or generalized Ohm's law.  For a
tokamak, neoclassical effects damp the poloidal diamagnetic flow contribution.
Thus in our case the remaining toroidal diamagnetic flow determines the
diamagnetic frequency, $\omega_{*i}= \mathbf{k}\cdot\mathbf{v}_{*i} =
n_\phi/(n_iq_i)\partial p_i/\partial \psi$ where $\mathbf{k}$ is the mode
wavenumber.  Figure \ref{FLRidealComp} shows the comparison to this dispersion
relation with full and reduced diamagnetic drift values. The value of
$\omega_{*i}$ is computed at the radial location where the mode structure
peaks.  The normalized frequency from the diamagnetic drift velocity,
$\omega_{*i}\tau_A$, varies linearly with toroidal mode number as
$0.00483n_\phi$.  This `local' approximation, which neglects the radial
variation of the $\omega_{*i}$ profile, is known to over-estimate the effect of
drift stabilization \cite{Hastie00,Snyder11}.  We find that a reduction of the
value of $\omega_{*i}$ (approximately by a factor of 2-4) gives the best
agreement between Eqn.~\eqref{eq:driftBalloon} and the two-fluid NIMROD
calculation.  A similar comparison of ELITE to the two-fluid BOUT++ model on a
different case finds the drift stabilization is over-predicted \cite{Snyder11}
by a comparable factor.

One of the limitations of Eqn.~\eqref{eq:driftBalloon} is that it does not
include the effect of resistive destabilization on the high-$n_\phi$ modes,
as is included in our two-fluid computations.
The analytic treatment of Ref.~\cite{Hastie03} provides a dispersion relation
(Eqn.~(43) of the reference; referred to as HRP Eqn.~(43) in this discussion) that
includes the effect of resistive destabilization in addition to drift effects.
In order to compare with this theoretical treatment, parameters are computed
on the outboard midplane at the radial location where the mode structure peaks,
consistent with the previous $\omega_{*i}$ calculation. This yields local values
of the safety factor, $q=7.34$, normalized magnetic shear, $s=8.8$ and
normalized pressure gradient, $\alpha=28.8$, where the definitions of
Ref.~\cite{Hastie03} are used.  The $\Delta^\prime_B$ stability parameter is
determined by using the ideal growth rate ($\Gamma=5/3$) and solving Eqn.~35 of
Ref.~\cite{Hastie03}.  $\Delta^\prime_B$ varies with toroidal mode number but
falls in the range of $-17.4$ to $-12.1$. 

Comparison of HRP Eqn.~(43) with $\omega_*=0$ and the NIMROD resistive-MHD results
differ by a factor approximately 2.5 at large toroidal mode numbers where the
analytic calculation results in greater growth rates and there is no
stabilization at low $n_\phi$. As the analytic calculation is a local calculation, it
does not capture the stabilizing influence on the global mode of finite
resistivity, in particular it misses the effect of modifying the response outside the
LCFS from that of a vacuum to that of a resistive plasma.  Our interest is
largely to compare and contrast the predicted drift stabilization and thus we
reduce the resistivity to 35\% of the value used in the NIMROD computations
when calculating the analytic values of HRP Eqn.~(43). This produces growth
rates with $\omega_*=0$ that roughly match the NIMROD resistive-MHD
computations at high $n_\phi$ as shown in Fig.~\ref{HRPcomp}.

\begin{figure}
  \centering
  \includegraphics[width=8cm]{HRPcomp}
  \vspace{-4mm}
  \caption{[Color online]
  Growth rates computed by NIMROD with the ideal, single-fluid (1f) and full
  FLR (2f-Par-FLR) models in comparison to the analytic dispersion relation of
  Ref.~\cite{Hastie03}, Eqn.~(43) with the same parameters as the NIMROD
  computations except reduced $\omega_{*\alpha}$ values and a reduced
  resistivity (0.35 of the NIMROD value).
  Associated NIMROD data available in Ref.~\cite{king16Z}.}
  \label{HRPcomp}
\end{figure}

More significantly, Fig.~\ref{HRPcomp} compares the results of our linear
NIMROD computations with the drift-stabilized growth rates computed from
Eqn.~(43) of Ref.~\cite{Hastie03}. Again, the analytics of HRP Eqn.~(43)
over-estimate the effect of drift stabilization where the reduction of
$\omega_*$ that leads HRP Eqn.~(43) to qualitatively agree with the two-fluid
NIMROD computations is between 2 to 4 (dependent on $n_\phi$). There is little
effect on the low-$n_\phi$ modes and the ion drift-wave resonance predicted in
Ref.~\cite{Hastie03} is not observed.
%This factor is smaller than that required for
%qualitative agreement with Eqn.~\eqref{eq:driftBalloon} as HRP Eqn.~(43)
%includes the effect of resistive destabilization.  


% Compare to analytics
% 1) drift stabilization of the ideal, incompressible ballooning mode
% 2) drift stabilization of the resistive ballooning mode
% Begin with (1):
%   - Disp. relation over predicts FLR stabilization
%     - Does not account for resistive destabilization
%   - Need to reduce w* by 32x to get modest agreement
% Conclusion: profile and poloidal effects are important - Ref HCR
%
% (2) Use HRP Eqn43 to compare resistive ballooning mode disp.
% Use same parameters as NIMROD runs
%    Theory assumes cylindrical s, alpha parameters
%    Values computed on the outboard midplane at 
%    mode resonant surface as reported by ELITE
%      q = 7.34
%      s = 8.8
%      alpha = 28.8
%      w* tau_a = 0.0176 n_phi
% Use HRP Eqn 35 to determine Delta'B from ELITE ideal Gamma=5/3 result
%      Delta'B= -17.4 -- -12.1 (varies by n)
% NIMROD computes visco-resistive growth rate w/ FLR modification
% HRP visco-resistive dispersion neglects Delta'B, not applicable
%      -> HRP shows viscosity is stabilizing
% HRP over-predicts \gamma with \omega_star =0 by factor of ~2.5x
%      -> Need viscosity for quantitative agreement
%      -> reduce resistivity by 0.35 give qualitative agreement
% HRP \omega_star effects over-predict stabilization
%      -> \omega_star; \omega_star/2 => all n modes near marginally stability
%      -> \omega_star/4; \omega_star/8; \omega_star/16 in figure
%      -> \omega_star/8 is most consistent with NIMROD calculations
% Conclusion: profile and poloidal effects are important
%
% o (2) includes resistive destabilization and thus requires a small ad-hoc
%   adjustment of \omega_star.
% o However, Given ad-hoc adjustments must be made to analytics, using the
%   simpler (1) is as good as (2)
% o Full computation best

\section{FLR parameter scan}
\label{sec:densityScan}

In order to ascertain the role of two-fluid effects further, we vary the
density and temperature while scaling parameters to maintain
constant $\beta$ and $S$ (other dissipation parameters are scaled relative to
resistivity). In experimental discharges, the zeroth-order effect on the edge
of modifying the density is modify the plasma collisionality and thus the
bootstrap current. This causes a transition from ballooning-like modes at high
density to peeling-like modes at low density. Our computations use a fixed
equilibrium current and thus are not sensitive to this effect. Instead, we
isolate the effect of the modification of $\rho_i$ (and the associated ion
skin depth, $d_i$) within the context of our two-fluid, first-order FLR model.
This contrasts with Ref.~\cite{xu14} that examines these effects in concert
and Ref.~\cite{Zhu12} that considers only the effect of modifications
to the current profile.

\begin{figure}
  \centering
  \includegraphics[width=8cm]{densityScan}
  \vspace{-4mm}
  \caption{[Color online]
  Growth rates computed by NIMROD on the `Meudas-1' case with the 1f model
  with parallel Braginskii closures (1f-Par; single-fluid limit), and three
  different densities with the two-fluid, FLR model with parallel Braginskii
  closures (2f-Par-FLR): the core densities are $n_e=3\times10^{19}$,
  $6\times10^{19}$ and $1.2\times10^{20}\;m^{-3}$.
  Associated NIMROD data available in Ref.~\cite{king16Z}.}
  \label{densityScan}
\end{figure}

Figure \ref{densityScan} shows the linear growth rate from NIMROD computations
for resistive-MHD and three two-fluid cases with varying densities (the
core density is varied from $3\times10^{19}\;m^{-3}$ to
$1.2\times10^{20}\;m^{-3}$ where $6\times10^{19}\;m^{-3}$ is the value used in
the prior computations discussed in the paper). The effect of
drift-stabilization is stronger at low densities ($\rho_i$ scales as
$1/\sqrt{n_i}$ at constant $\beta$) as expected. The cases with
$n_e=3\times10^{19}\;m^{-3}$ are not plotted for $n_\phi>18$ as the most unstable
mode is no longer the peeling-ballooning mode but rather a dominatly
electro-static mode related to the ITG as studied in the context NIMROD
two-fluid advance in Ref.~\cite{Schnack13}. This region is excluded as
examination of ITG mode dynamics is outside the scope of this work.

% In order to ascertain the role of two-fluid effects further, we vary one of the
% two-fluid FLR parameters that set the ion gyroradius $\rho_i$: the ion skin
% depth, $d_i$, and the plasma $\beta$.  Keeping the Meudas-1 current and pressure
% profiles fixed, we scale the density and temperature profiles in such a way
% that the plasma pressure remains the same in all simulations. This modifies the
% ion skin depth at constant $\beta$. The current (including the bootstrap
% contribution) and the Lundquist and Prandtl numbers are fixed to isolate
% effects from the two-fluid, FLR model only. As shown in
% Fig.~\ref{densityScan}, the low density cases (large $d_i$, $\rho_i$) peak at
% lower $n$ and are stabilized. However, there is a high-$n$ destabilization that
% is related to the ITG drive as captured by the fluid model. This ITG drive has
% been studied further in Ref.~\cite{Schnack13}.  It is our experience that for
% reconstructed cases that include density and temperature measurements, the ITG
% drive is small - likely a result of ITG modification of the profiles to ensure
% at least marginal stability.



\section{Conclusions}
\label{sec:conclusions}

In this paper, we apply shared-workload techniques at the \sql level for
improving the throughput of \qaasl systems without incurring in additional
query execution costs. Our approach is based on query rewriting for grouping
multiple queries together into a single query to be executed in one go. This
results in a significant reduction of the aggregated data access done by the
shared execution compared to executing queries independently.

%execution times and costs of the shared scan operator when
%varying query selectivity and predicate evaluation. We observed that for
%\athena, although the cost only depends on the amount of data read, it is
%conditioned to its ability to use its statistics about the data. In some cases
%a wrong query execution plan leads to higher query execution costs, which the
%end-user has to pay. 

%\bigquery's minimum query execution cost is determined by
%the input size of a query.  However, the query cost can increase depending not
%just in the amount of computation it requires, but in the mix of resources the
%query requires.  

We presented a cost and runtime evaluation of the shared operator driving data access costs. 
Our experimental study using the TPC-H benchmark confirmed the benefits of our
query rewrite approach. Using a shared execution approach reduces significantly
the execution costs. For \athena, we are able to make it 107x cheaper and for
\bigquery, 16x cheaper taking into account Query 10 which we cannot execute,
but 128x if it is not taken into account. Moreover, when having queries that do
not share their entire execution plan, i.e., using a single global plan, we
demonstrated that it is possible to improve throughput and obtain a 10x cost
reduction in \bigquery.

%We followed the TPC systems pricing guideline for
%computing how expensive is to have a TPC-H workload working on the evaluated
%\qaasl systems. The result is that even though we are able to reduce overall
%costs a TPC-H workload in 15x for \bigquery (128x excluding query 10 which we
%could not optimize) and in 107x for \athena, the overall price is at least 10x
%more expensive than the cheapest system price published by the TPC.

There are multiple ways to extend our work. The first is
to implement a full \sql-to-\sql translation layer to encapsulate the proposed
per-operator rewrites.  Another one is to incorporate the initial work on
building a cost-based optimizer for shared execution
\cite{Giannikis:2014:SWO:2732279.2732280} as an external component for \qaasl
systems.  Moreover, incorporating different lines of work (e.g., adding
provenance computation \cite{GA09} capabilities) also based on query
rewriting is part of our future work to enhance our system.
 
\appendix
%--------------------------------------------------

\begin{acknowledgments}
We thank Carl Sovinec, Chris Hegna and Nate Ferraro for discussions involving
this paper and N.~Aiba for providing the Meudas-1 equilibrium. This material is
based on work supported by US Department of Energy, Office of Science, Office
of Fusion Energy Sciences under award numbers DE-FC02-06ER54875 and
DE-FC02-08ER54972. This research used resources of the National Energy Research
Scientific Computing Center, a DOE Office of Science User Facility supported by
the Office of Science of the U.S. Department of Energy under Contract
No.~DE-AC02-05CH11231.
\end{acknowledgments}
\bibliographystyle{apsrev4-1}
\bibliography{Biblio}

\end{document}
