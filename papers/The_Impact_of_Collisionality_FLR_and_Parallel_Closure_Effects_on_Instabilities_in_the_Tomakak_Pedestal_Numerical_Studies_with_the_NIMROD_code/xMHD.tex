\section{Increasing the model complexity}
\label{sec:xMHD}

% o Don't emphasize M3D-C1, just ref Ferraro
% o 1fl effects:
%  - Describe profile choice for n - avoid ITG like modes w/ 2f parameters
%  - Using nu_perp = k_perp = D_n = elecd(psin=1)
%  - Using Z=3; pe_frac=0.75 => Te = Ti; Gamma=5/3
%  - Effect of density profile - local gamma * tau_a const,
%      -> normalized to core tau_a (unchanged)
%      -> local tau_a ~ sqrt(n) smaller => growth rates uniformly larger
%  - Effect of resistivity profile: high-n destabilization, low-n stabilization
% o parallel effects:
%  - Almost no effect on gamma
%  - expected from ELITE results which show mode \gamma is largely 
%      insensitive to \Gamma (0 in Ferraro, 5/3 and inf here)
% o 2fl FLR effects:
%  - Including full FLR model: Gen. Ohm's law; ion gyrovisc; cross heat fluxes
%  - Drift stabilization at high-n; crosses ideal results @ n~24
%  - Little effect at low n (n<10)
%  - (No destabilization as predicted by HRP)
%  - For this case: computations w/ and w/out x-heat flux give same result
%     -> Modest beta (0.1%); consistent with FLR theory from King and Kruger
% o Reference: Ping, Xu13

Initial-value extended-MHD computations, while more computationally expensive
than ideal-MHD calculations with ELITE, are able to include additional effects
that are more representative of experimental conditions:
% Initial-value extended-MHD computations, while more computationally expensive
% than ideal eigenvalue calculations such as ELITE, are able to include
% additional effects that are more representative of experimental conditions:
\begin{itemize} \cramplist \zapspace
\item 
Instead of a magneto-static vacuum outside the LCFS, NIMROD includes a
cold-plasma region.
\item
Finite-dissipation effects such as resistivity, viscosity and thermal
conduction.
\item 
Density and temperature profiles that prescribe the associated
drifts and resistivity profile.
\item
Parallel-closure effects that represent large diffusivities oriented along the
magnetic field.
\item
First-order FLR-closure and two-fluid effects.
\end{itemize} 
Reference \cite{Ferraro10} investigates the first three of these effects with
the Meudas-1 case. We further this work by including the last two effects.  The
first three effects are also re-investigated and different modifications to the
growth rates are found. These differences are expected as slightly different
profiles for density and temperature are used in order to avoid spurious
electrostatic modes with the two-fluid, first-order FLR model. 
Ref.~\cite{Ferraro10} avoided these modes by employing only the
single-fluid resistive-MHD model.
% {\bf SEK: I think this is review bait:  Why didn't
%   Nate see spurious modes?  What is the source of these spurious modes?
% Etc.  I don't see a clean way of discussing this}

In calculations beyond the ideal model, the density profile is
$n_e(\psi_n)=n_{e0} (p(\psi_n)/p_0)^{0.3}$ where
$n_{e0}=6\times10^{19}\;m^{-3}$ and $\psi_n$ is the normalized poloidal flux.
Here $p$ is the pressure as prescribed by the ideal gas law, $p=n_e T_e+ n_i
T_i$, where $T_\alpha$ is a species' temperature. This choice does not match
Ref.~\cite{Ferraro10}, but effectively avoids spurious electrostatic modes with
the two-fluid, first-order FLR model. These spurious modes are discussed in 
more detail in Sec.~\ref{sec:densityScan}.
% {\bf SEK: Ditto the previous comment} 
The pressure-profile gradient is specified by the Meudas-1 case and setting the
edge temperature to $200 eV$ leads to a core temperature of $5806 eV$.  The
edge temperature is modified independent of the MHD-stability (determined by
$p^\prime$) by the transformation $p(\psi_n) \rightarrow p(\psi_n) + p_c$ where
$p_c$ is a constant. Without scrape-off-layer (SOL) profiles of density and temperature, which are
not included in standard reconstructions, this choice of the edge temperature
also specifies the temperature outside the LCFS.  SOL profiles are required to
get both the correct profiles outside the LCFS, which determines the vacuum
response, and simultaneously set the correct resistivity at the mode resonant
location when Spitzer resistivity is used.  The non-ideal viscous, conductive
and particle diffusivities are set to a small value, one tenth of the
resistivity at the LCFS. 

\begin{figure}
  \centering
  \includegraphics[width=8cm]{modelScan}
  \vspace{-4mm}
  \caption{[Color online]
  Growth rates computed by NIMROD on the `Meudas-1' case with four different
  models: ideal MHD (ideal), resistive MHD with reconstructed density,
  temperature, and Spitzer-resistivity profiles (1f), the 1f model with parallel
  Braginskii closures (1f-Par), the 1f-Par model with ion gyroviscosity, a full
  extended MHD Ohm's law (including the Hall, $\nabla p_e$, and electron inertia
  terms) and separate temperature evolution equations with cross-heat fluxes
  (2f-Par-FLR).
  Associated NIMROD data available in Ref.~\cite{king16Z}.}
  \label{modelScan}
\end{figure}

Figure~\ref{modelScan} plots the growth rates computed by NIMROD on the
`Meudas-1' case with four different models: ideal MHD (ideal), resistive MHD with
reconstructed density, temperature, and Spitzer-resistivity profiles (1f), the
single-fluid model with parallel Braginskii closures (1f-Par) where the 
parallel viscosity is
\begin{multline}
\mathbf{\Pi}_{\parallel i}=
  m_i n_i \nu_{\parallel i}
  \left(\hat{\mathbf{b}}
  \hat{\mathbf{b}}-\frac{1}{3}\mathbf{I}\right) \\
  \times\left(3\hat{\mathbf{b}}\cdot\nabla\mathbf{v}_{i}\cdot
  \hat{\mathbf{b}}-\nabla\cdot\mathbf{v}_{i}\right)\;,
\end{multline}
and the parallel heat flux vector is
\begin{equation}
\mathbf{q}_{\parallel \alpha}=
  -n_i \chi_{\parallel \alpha}
  \hat{\mathbf{b}}\hat{\mathbf{b}}\cdot\nabla T_{\alpha}\;,
\end{equation}
and a two-fluid model with parallel closures that includes ion gyroviscosity
\begin{multline}
\mathbf{\Pi}_{\times i}=
  \frac{m_{i}p_{i}}{4ZeB}[\hat{\mathbf{b}}
    \times\mathbf{W}\cdot
    \left(\mathbf{I}+3\hat{\mathbf{b}}\hat{\mathbf{b}}\right) \\
    -\left(\mathbf{I}+3\hat{\mathbf{b}}\hat{\mathbf{b}}\right)
    \cdot\mathbf{W}\times\hat{\mathbf{b}}]\;,
\end{multline}
where $\mathbf{W}$ is the rate-of-strain tensor
\begin{equation}
\mathbf{W}=\nabla\mathbf{v}+\nabla\mathbf{v}^{T}
           -(2/3)\mathbf{I}\nabla\cdot\mathbf{v}\;,
\end{equation}
a full extended-MHD Ohm's law,
\begin{equation}
\mathbf{E}=
  -\mathbf{v}\times\mathbf{B}+\frac{\mathbf{J}\times\mathbf{B}}{n_ee}
  -\frac{\nabla p_{e}}{n_ee}+\eta\mathbf{J}
  +\frac{m_{e}}{n_e e^2}\frac{\partial\mathbf{J}}{\partial t}\;,
\end{equation}
and separate temperature evolutions with cross-heat fluxes,
\begin{equation}
\mathbf{q}_{\times \alpha}=
  \frac{5p_{\alpha}}{2q_{\alpha}B}
  \hat{\mathbf{b}}\times\nabla T_{\alpha}\;,
\end{equation}
(2f-Par-FLR). Here $v_i$ is the bulk-ion flow, $\nu_{\parallel i}$ is the ion parallel
diffusivity, $\chi_{\parallel \alpha}$ is the species' parallel diffusivity,
$q_\alpha$ ($e$) is the species' (electron) electric charge,  $\mathbf{J}$ is
the current density, $\hat{\mathbf{b}}$ is the magnetic unit vector
($\hat{\mathbf{b}} = \mathbf{B}/B$), and $\mathbf{I}$ is the identity tensor.
An effective ion charge ($Z$) of three is assumed. Each model builds
on the previously listed and includes all prior terms. Electron viscosity is
not included.


% REMOVED FROM INTRO
% For peeling-ballooning modes, the prevailing analytic description of these
% effects is described in Ref.~\cite{Hastie03}. 
% This conventional picture expects two-fluid and FLR effects are destabilizing
% to modes with low toroidal mode number but stabilizing at high toroidal mode
% number. 
% We compare our results with this simple model of drift stabilization where
% $\gamma_{MHD}^2 = \omega (\omega - \omega_*)$.  Here $\gamma_{MHD}$ is the
% ideal-MHD growth rate, $\omega$ is the complex mode frequency and $\omega_*$ is
% the diamagnetic-drift frequency.

% Closure expressions for the viscosity tensor, $\mathbf{\Pi}$, and heat flux,
% $\mathbf{q}$. These expressions are typically decomposed into three parts:
% parallel (large diffusivities oriented along the magnetic field), cross (FLR
% ordered contributions), and perpedicular (small dissipative terms)

When a density profile is included, the normalized growth rate of the mode
remains constant with the single-fluid model when the Alfvén time is computed
at the mode resonant surface. Our calculations are normalized by the Alfvén
time as computed with the values at the magnetic axis. These values remain constant
when a density profile is included such that the normalized mode growth rate is
effectively increased. 

The effect of the resistivity profile is more nuanced.  Finite
resistivity allows for reconnection within the model by relaxing the
frozen-flux constraint. This permits resistive-ballooning modes that
grow faster than their ideal counterparts. Alternatively, the resistive
dissipation can stabilize the modes by acting as a dissipative term and
modifying the response outside the LCFS from that of a vacuum to that of
a plasma. Figure 5 of Ref.~\cite{Ferraro10} shows that the plasma
response is stabilizing relative to a vacuum region. Overall, these
effects stabilize the ballooning modes at low-$n_\phi$ and destabilize
the modes at high-$n_\phi$ as seen in Fig.~\ref{modelScan} when
comparing the single fluid and ideal normalized growth rates. The effect
of including small particle, viscous, and thermal diffusivities on the
resistive-MHD mode is small (not shown).

Including the Braginskii parallel closures with large coefficients
($\chi_{\parallel e} = 10^9\times \eta_0/\mu_0$ and  $\chi_{\parallel i} =
\nu_{\parallel i} = 10^8\times \eta_0/\mu_0$) has little effect on the growth
rates.  When thermal conduction is large, the temperature quickly equilibrates
along field lines to produce an isothermal response.  Thus NIMROD computations
that show little effect from the parallel closures are consistent with the ELITE
results that find the shape of the mode growth-rate spectrum is largely not
sensitive to the value of the ratio of specific heats, $\Gamma$, as seen in
Fig.~\ref{fig:ELITEComp}. When the response to the compressible motion
associated with the sound wave is eliminated ($\Gamma=0$ in the ELITE 
computations), the growth rate is slightly enhanced relative to the adiabatic limit.
If the growth rates are modified beyond this small effect, it would be
an indication that changes in the energy equation can impact the MHD response, 
and thus that closure effects are significant. Although the short-mean-free
path Braginskii-like closure \cite{Braginskii,Catto04} is used in our NIMROD
compuatations, we note that other closures, such as long-mean-free path
\cite{Ramos11} or the general approach of solving drift-kinetic equation
\cite{held15} for a closure can be applied. However, for this case where the
parallel-closure effects are negligible, different parallel closures are
unlikely to significantly modify the results.

With the full two-fluid, FLR model, there is a stabilizing effect on the
intermediate and high-$n_\phi$ modes as expected from analytic
treatments \cite{Hastie03}. A common reduced-MHD approximation for
first-order FLR closures is to assume that the ion-gyroviscous force and
the advection by the diamagnetic drift exactly cancel (this is
colloquially known as the gyroviscous cancellation, see, e.g.
\cite{Coppi64}). However, as discussed by Ramos in
Ref.~\cite{Ramos:2007cn}, ``these cancellations are only partial and not
very useful in practice for general magnetic geometries\ldots''.  This
cancellation is only valid with a large, uniform guide field, without
curvature terms (slab approximation), and in the low-$\beta$ limit (see,
e.g. \cite{King14}).  Use of the full gyroviscous operator is critical
as in addition to effects that appear as diamagnetic drifts, it also
contains terms proportional to first-order FLR magnetic-curvature and
grad-B drifts \cite{King11}. Our modeling contains not just the
first-order FLR drifts from ion gyroviscosity, but also the drifts from
the cross-heat fluxes that enter the equations on the same order.  

Just as the large balance of the ideal-MHD forces can cause
numerical difficulties with the ideal-MHD force operator, the
partial gyroviscous cancellation leads to similar numerical 
pitfalls and verification is important.
The implementation of these terms in NIMROD are verified in cylindrical and
slab magnetic geometries against analytic calculations for
tearing~\cite{Sovinec10,King11,King14} and
ion-temperature-gradient~\cite{Schnack13} (ITG) modes. 
% {\bf SEK: I don't know
%   why you have Schnack13 associated with tearing instead of ITG.  Also,
%   Schnack really didn't use Coppi67 directly but rather had to modify it
%   for the verification purposes, but it's better to just reference his
% paper}.  
% JRK -> previous references Coppi67 for ITG; others for general verfication
To test the effect of these terms against analytic theory
in full tokamak magnetic geometry, we perform a qualitative
comparison with analytic theory in Sec.~\ref{sec:analyticComp}.
Full verification is precluded by the approximations
made in the analytic theories.
%For verification in full tokamak magnetic geometry, we perform a qualitative
%comparison with analytic theory in Sec.~\ref{sec:analyticComp}.

The first-order FLR model is valid for the toroidal modes presented although
this is not necessarily the case for general edge mode modeling.  Assuming
$k\simeq(q/r+1/R)n_\phi$ and evaluating local quantities at the peak of the
mode eigenfunction ($\psi_n = 0.969$) on the outboard midplane, the normalized
ion-gyroradius ($\rho_i=\sqrt{\Gamma m_i T_i}/ZeB$) as a function of toroidal
mode is $k \rho_i \simeq 0.0055n_\phi$.  If the first-order FLR assumption is
violated, more-complex full-ion-orbit kinetic modeling is required to simulate
large-fluctuation ELM dynamics.  For this case, computations produce
approximately the same results with and without the cross-heat fluxes.  This is
consistent with drift analytics \cite{King14} that shows that the cross-heat
flux is only significant at very low values of plasma $\beta$ (here the
$\beta=2\mu_0p/B^2$ is 0.1\%).  Perhaps coincidentally for this case, the
effects of the resistive destabilization and drift stabilization largely
counteract one another. The drift stabilized growth rates are not less than the
ideal calculation until $n_\phi \gtrsim 24$.  

% Old text:
%
% In addition to the ideal MHD benchmark, the M3D-C1 code
% investigated the effects of including a density gradient profile and Spitzer
% resistivity. Our NIMROD studies have furthered this work to
% include two-fluid effects through the generalized Ohm's law (including the
% Hall, $\nabla p_e$, and electron inertia terms), ion gyroviscosity, highly
% anisotropic thermal conduction, and parallel ion
% viscosity.  When the model includes a generalized Ohm's
% law, the result qualitatively agrees with the analytic theory of
% Ref.~\cite{Hastie03} used for ELITE's {\em ad hoc} model; we find a
% destabilization of the intermediate-$n$ mode spectrum and a stabilization of
% the high-$n$ mode spectrum. When ion gyroviscosity is included, there is a
% stabilization of the high-$n$ mode spectrum, consistent with recent results
% from the BOUT++ code \cite{Xu13}. The high-$n$ mode
% spectrum that results from the use of a extended MHD model prevents energy from
% accumulating on the smallest resolvable scales and corrupting the simulations.
% These results are shown in Fig.~\ref{fig:ELITEComp}.
