\section{Introduction}
\label{sec:introduction}

Understanding the complex physics of instabilities in the tokamak pedestal
region is essential for ensuring that tokamaks achieve their highest performance.
Computation plays a critical role in furthering the knowledge of edge dynamics,
including advances in the understanding of type-I edge-localized modes
(ELMs)~\cite{leonard06}.  Nonlinear computation of ELM dynamics remains
challenging, but is becoming tractable with modern computing
resources~\cite{pankin07,sugiyama10,xu10,xu13,xi13,huijsmans13,pamela13}.  ELM
% Note that "linear" is a mathematical concept so applies to modeling
% rather than dynamics; i.e., phenomenologically you cannot measure a
% "linear phase" (you can infer that it is in the linear phase, but it's an inference).
modeling can be roughly broken into three regimes: the early linear regime, the
nonlinear regime, and an energy loss regime.  Two important aspects of the
linear stage are the stability boundary within an operational space;
e.g., pedestal current and pressure gradient; and the linear growth rate for each
toroidal-mode number; i.e.,  the growth-rate or mode spectrum.
The dominant instability drive that sets the linear-stability boundary is from the
peeling-ballooning mode (PBM)~\cite{connor98,Snyder02,snyder04} based on
the ideal-MHD model.
The linear spectrum is critical to the next phase, the nonlinear regime as it
seeds the dynamics.  Non-ideal effects can greatly impact the growth-rate
spectrum and are thus important in determining the nonlinear ELM dynamics.  In
the final regime, high-fidelity simulation of the energy loss requires modeling
of the fast conductive and convective transport, and accurate coupling to
neutrals and the wall.  State-of-the-art computations are beginning to push the
boundary of current capabilities to modeling that captures accurate dynamics
with these additional effects that are relevant during energy
loss~\cite{huijsmans13,pamela13,gui14}.  

As the subsequent nonlinear evolution and energy loss in simulations is greatly
impacted by the amplitudes of initial linear perturbations; a large source of
uncertainty arises from the linear phase of ELM evolution.  Given the
difficulty (or impossibility) of measuring the small perturbations that seed an
ELM, a reasonable model is to seed it from unstable PBMs that grow from
small-wave perturbations.  Although the stability boundary is well described by
ideal MHD, details of the mode spectrum depend upon specific non-ideal effects.
In particular, resistive effects may destabilize and first-order
finite-Larmour-radius (FLR) effects may stabilize the modes. These latter effects,
part of which is drift stabilization, require two-fluid modeling with advanced
FLR closures. It is important to verify that codes accurately capture these
effects within regimes of interest. In this paper, we study these effects on
the linear modes with the extended-MHD NIMROD code \cite{Sovinec04,Sovinec10}.

Type-I ELM stability boundaries are dominated by drives described by ideal MHD
as the ideal-MHD operator is stiff; i.e., near the ideal-MHD stability boundary
small changes in pressure and current-density gradients lead to large changes
in the growth rate due to the ideal-MHD terms becoming large. Thus only small
profile changes are required to cross the boundary, and non-ideal modifications
to the growth rate become secondary when describing the boundary location
within operational phase space. This ideal-MHD stiffness is a general property
that impacts the behavior of a wide class of plasma-instability phenomena.  For
example, in addition to ELM stability, it also describes the dynamics of
disruptions caused by core-mode activity \cite{callen99,kruger05}.  However,
type-I ELMs are more complicated than core-mode disruptions given the broad
toroidal-mode spectrum and associated nonlinear coupling. Similar to studies of
core-mode disruptions, comprehensive understanding of ELM onset requires
evolving the equilibrium on the transport time scale through the instability
boundary with a verified code that includes non-ideal terms to calculate an
accurate initial linear-mode spectrum.


MHD codes that study ELM instabilities can be placed into two broad
categories: 1) extended-MHD codes (e.g. NIMROD and
M3D-C1~\cite{jardin2007high}) and 2) reduced-MHD-based codes (e.g.
BOUT++~\cite{dudson2009bout} and JOREK~\cite{huysmans2009non}).  The
definition of extended MHD used in this work includes the full ideal-MHD
force operator (see Ref.~\cite{Schnack:2006kx} and references therein).
Using the full operator retains the fast compressional-Alfvén waves that
enforce force balance and thus are critical to extended-MHD transport
calculations with sources and sinks (e.g. \cite{Jenkins10}).  Relative to
reduced modeling, the extended-MHD approach also has advantages in
avoiding low-$\beta$ approximations and a straight-forward
implementation of the FLR closures as discussed in
Section~\ref{sec:xMHD}.  However, the inclusion of the full force
operator comes at a cost of increased numerical difficulties when
solving the equations associated with the fast
waves~\cite{gruber2012finite, degtyarev1986methods,
bondeson1991tunable,lutjens1996class, Sovinec15}.   Verification for all
regimes is important, and previous work~\cite{Burke10} verified the ideal-MHD
terms for a non-diverted, peeling-ballooning unstable test case.  As one
expects that transport calculations across the ELM stability boundary coupled
with an instability model that captures FLR drift stabilization significantly
impacts dynamics, emphasis on verifying linear physics of extended-MHD codes
thus naturally follows.

% One of the methods that NIMROD has developed for resolving these
% difficulties is to have the equilibrium terms separated from the
% dynamical terms.  The implications of this separation is discussed in
% the next section. 
% JRK - add this to separate EQ discussion. 

In this work, a verification of the full extended-MHD, initial-value NIMROD
code with the ideal-MHD ELITE code~\cite{wilson02,Snyder02,snyder04} is
performed using the Meudas-1-case comparison~\cite{Snyder09}. This is a
diverted tokamak case that is peeling-ballooning unstable. As we later
discuss, the inclusion of the x-point significantly modifies the mode
structure; making the comparison in full magnetic geometry an important
additional verification.  To assess the impact of the two-fluid, FLR effects,
we compare our results with drift stabilization to a calculation of drift
stabilization on the resistive PBM from Ref.~\cite{Hastie03}.  Although the
approximations made during the analysis of first-order FLR PBMs in
Ref.~\cite{Hastie03} preclude quantitative comparison, we make a qualitative
comparison of our first-order FLR computations to analytics.  

The paper proceeds as follows: In Sec.~\ref{sec:benchmark} we summarize the
benchmark and give an overview of our methodology.  The exhaustive details of
the methods used in this comparison, in particular the mechanism whereby a
vacuum region is effectively included through extended-MHD modeling, are
previously available in Refs.~\cite{Burke10} and~\cite{Ferraro10} and thus our
description of this aspect is brief.  In Sec.~\ref{sec:xMHD} the model
complexity is gradually increased from the ideal-like model through a
resistive-MHD model with experimentally relevant profiles to a full
extended-MHD model which includes both parallel-closure and FLR-drift effects.
The resulting impact on the growth-rate spectrum in analyzed. Comparisons
of the computed drift-stabilized growth rates are made to analytic treatments
of the PBM in Sec.~\ref{sec:analyticComp}.
In Sec.~\ref{sec:densityScan} the density and temperature are varied at fixed
plasma $\beta$ (thus the ion gyroradius, $\rho_i$, is modified) in order to
ascertain how the FLR effects are impacted.  In this study, the bootstrap
current is fixed (the zeroth-order effect from variations of density modify the
bootstrap current) and the impact of FLR drift effects are isolated.
Sec.~\ref{sec:conclusions} summarizes this work and places it within the
broader context of modeling of ELM relaxation events in the tokamak edge with
transport effects.
