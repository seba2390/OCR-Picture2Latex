\section{Comparison to drift analytics}
\label{sec:analyticComp}

\begin{figure}
  \centering
  \includegraphics[width=8cm]{FLRidealComp}
  \vspace{-4mm}
  \caption{[Color online]
  Growth rates computed by NIMROD with the ideal, single-fluid (1f) and full
  FLR (2f-Par-FLR) models in comparison to the analytic dispersion relation of
  the FLR-stabilized, ideal ballooning mode with full and reduced $\omega_{*\alpha}$ 
  values. Associated NIMROD data available in Ref.~\cite{king16Z}.}
  \label{FLRidealComp}
\end{figure}

To gain insight into our two-fluid computations, we compare with analytic
descriptions of drift stabilization of the ballooning mode.  The dispersion
relation for drift stabilizations of the ideal, incompressible ballooning mode
is
\begin{equation}
\omega (\omega - \omega_{*i}) = - \gamma_I^2
\label{eq:driftBalloon}
\end{equation}
where $\gamma_I$ is the ideal growth rate in the absence of drift effects and
$\omega$ is the complex frequency ($\omega=i\gamma$) \cite{tang82}.  The
diamagnetic drift velocity is typically defined as the
perpendicular-to-the-magnetic-field flow contribution from the pressure
gradient within the context of a two-fluid, or generalized Ohm's law.  For a
tokamak, neoclassical effects damp the poloidal diamagnetic flow contribution.
Thus in our case the remaining toroidal diamagnetic flow determines the
diamagnetic frequency, $\omega_{*i}= \mathbf{k}\cdot\mathbf{v}_{*i} =
n_\phi/(n_iq_i)\partial p_i/\partial \psi$ where $\mathbf{k}$ is the mode
wavenumber.  Figure \ref{FLRidealComp} shows the comparison to this dispersion
relation with full and reduced diamagnetic drift values. The value of
$\omega_{*i}$ is computed at the radial location where the mode structure
peaks.  The normalized frequency from the diamagnetic drift velocity,
$\omega_{*i}\tau_A$, varies linearly with toroidal mode number as
$0.00483n_\phi$.  This `local' approximation, which neglects the radial
variation of the $\omega_{*i}$ profile, is known to over-estimate the effect of
drift stabilization \cite{Hastie00,Snyder11}.  We find that a reduction of the
value of $\omega_{*i}$ (approximately by a factor of 2-4) gives the best
agreement between Eqn.~\eqref{eq:driftBalloon} and the two-fluid NIMROD
calculation.  A similar comparison of ELITE to the two-fluid BOUT++ model on a
different case finds the drift stabilization is over-predicted \cite{Snyder11}
by a comparable factor.

One of the limitations of Eqn.~\eqref{eq:driftBalloon} is that it does not
include the effect of resistive destabilization on the high-$n_\phi$ modes,
as is included in our two-fluid computations.
The analytic treatment of Ref.~\cite{Hastie03} provides a dispersion relation
(Eqn.~(43) of the reference; referred to as HRP Eqn.~(43) in this discussion) that
includes the effect of resistive destabilization in addition to drift effects.
In order to compare with this theoretical treatment, parameters are computed
on the outboard midplane at the radial location where the mode structure peaks,
consistent with the previous $\omega_{*i}$ calculation. This yields local values
of the safety factor, $q=7.34$, normalized magnetic shear, $s=8.8$ and
normalized pressure gradient, $\alpha=28.8$, where the definitions of
Ref.~\cite{Hastie03} are used.  The $\Delta^\prime_B$ stability parameter is
determined by using the ideal growth rate ($\Gamma=5/3$) and solving Eqn.~35 of
Ref.~\cite{Hastie03}.  $\Delta^\prime_B$ varies with toroidal mode number but
falls in the range of $-17.4$ to $-12.1$. 

Comparison of HRP Eqn.~(43) with $\omega_*=0$ and the NIMROD resistive-MHD results
differ by a factor approximately 2.5 at large toroidal mode numbers where the
analytic calculation results in greater growth rates and there is no
stabilization at low $n_\phi$. As the analytic calculation is a local calculation, it
does not capture the stabilizing influence on the global mode of finite
resistivity, in particular it misses the effect of modifying the response outside the
LCFS from that of a vacuum to that of a resistive plasma.  Our interest is
largely to compare and contrast the predicted drift stabilization and thus we
reduce the resistivity to 35\% of the value used in the NIMROD computations
when calculating the analytic values of HRP Eqn.~(43). This produces growth
rates with $\omega_*=0$ that roughly match the NIMROD resistive-MHD
computations at high $n_\phi$ as shown in Fig.~\ref{HRPcomp}.

\begin{figure}
  \centering
  \includegraphics[width=8cm]{HRPcomp}
  \vspace{-4mm}
  \caption{[Color online]
  Growth rates computed by NIMROD with the ideal, single-fluid (1f) and full
  FLR (2f-Par-FLR) models in comparison to the analytic dispersion relation of
  Ref.~\cite{Hastie03}, Eqn.~(43) with the same parameters as the NIMROD
  computations except reduced $\omega_{*\alpha}$ values and a reduced
  resistivity (0.35 of the NIMROD value).
  Associated NIMROD data available in Ref.~\cite{king16Z}.}
  \label{HRPcomp}
\end{figure}

More significantly, Fig.~\ref{HRPcomp} compares the results of our linear
NIMROD computations with the drift-stabilized growth rates computed from
Eqn.~(43) of Ref.~\cite{Hastie03}. Again, the analytics of HRP Eqn.~(43)
over-estimate the effect of drift stabilization where the reduction of
$\omega_*$ that leads HRP Eqn.~(43) to qualitatively agree with the two-fluid
NIMROD computations is between 2 to 4 (dependent on $n_\phi$). There is little
effect on the low-$n_\phi$ modes and the ion drift-wave resonance predicted in
Ref.~\cite{Hastie03} is not observed.
%This factor is smaller than that required for
%qualitative agreement with Eqn.~\eqref{eq:driftBalloon} as HRP Eqn.~(43)
%includes the effect of resistive destabilization.  


% Compare to analytics
% 1) drift stabilization of the ideal, incompressible ballooning mode
% 2) drift stabilization of the resistive ballooning mode
% Begin with (1):
%   - Disp. relation over predicts FLR stabilization
%     - Does not account for resistive destabilization
%   - Need to reduce w* by 32x to get modest agreement
% Conclusion: profile and poloidal effects are important - Ref HCR
%
% (2) Use HRP Eqn43 to compare resistive ballooning mode disp.
% Use same parameters as NIMROD runs
%    Theory assumes cylindrical s, alpha parameters
%    Values computed on the outboard midplane at 
%    mode resonant surface as reported by ELITE
%      q = 7.34
%      s = 8.8
%      alpha = 28.8
%      w* tau_a = 0.0176 n_phi
% Use HRP Eqn 35 to determine Delta'B from ELITE ideal Gamma=5/3 result
%      Delta'B= -17.4 -- -12.1 (varies by n)
% NIMROD computes visco-resistive growth rate w/ FLR modification
% HRP visco-resistive dispersion neglects Delta'B, not applicable
%      -> HRP shows viscosity is stabilizing
% HRP over-predicts \gamma with \omega_star =0 by factor of ~2.5x
%      -> Need viscosity for quantitative agreement
%      -> reduce resistivity by 0.35 give qualitative agreement
% HRP \omega_star effects over-predict stabilization
%      -> \omega_star; \omega_star/2 => all n modes near marginally stability
%      -> \omega_star/4; \omega_star/8; \omega_star/16 in figure
%      -> \omega_star/8 is most consistent with NIMROD calculations
% Conclusion: profile and poloidal effects are important
%
% o (2) includes resistive destabilization and thus requires a small ad-hoc
%   adjustment of \omega_star.
% o However, Given ad-hoc adjustments must be made to analytics, using the
%   simpler (1) is as good as (2)
% o Full computation best
