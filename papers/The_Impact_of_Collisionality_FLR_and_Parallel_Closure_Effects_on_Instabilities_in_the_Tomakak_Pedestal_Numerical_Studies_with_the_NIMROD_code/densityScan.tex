\section{FLR parameter scan}
\label{sec:densityScan}

In order to ascertain the role of two-fluid effects further, we vary the
density and temperature while scaling parameters to maintain
constant $\beta$ and $S$ (other dissipation parameters are scaled relative to
resistivity). In experimental discharges, the zeroth-order effect on the edge
of modifying the density is modify the plasma collisionality and thus the
bootstrap current. This causes a transition from ballooning-like modes at high
density to peeling-like modes at low density. Our computations use a fixed
equilibrium current and thus are not sensitive to this effect. Instead, we
isolate the effect of the modification of $\rho_i$ (and the associated ion
skin depth, $d_i$) within the context of our two-fluid, first-order FLR model.
This contrasts with Ref.~\cite{xu14} that examines these effects in concert
and Ref.~\cite{Zhu12} that considers only the effect of modifications
to the current profile.

\begin{figure}
  \centering
  \includegraphics[width=8cm]{densityScan}
  \vspace{-4mm}
  \caption{[Color online]
  Growth rates computed by NIMROD on the `Meudas-1' case with the 1f model
  with parallel Braginskii closures (1f-Par; single-fluid limit), and three
  different densities with the two-fluid, FLR model with parallel Braginskii
  closures (2f-Par-FLR): the core densities are $n_e=3\times10^{19}$,
  $6\times10^{19}$ and $1.2\times10^{20}\;m^{-3}$.
  Associated NIMROD data available in Ref.~\cite{king16Z}.}
  \label{densityScan}
\end{figure}

Figure \ref{densityScan} shows the linear growth rate from NIMROD computations
for resistive-MHD and three two-fluid cases with varying densities (the
core density is varied from $3\times10^{19}\;m^{-3}$ to
$1.2\times10^{20}\;m^{-3}$ where $6\times10^{19}\;m^{-3}$ is the value used in
the prior computations discussed in the paper). The effect of
drift-stabilization is stronger at low densities ($\rho_i$ scales as
$1/\sqrt{n_i}$ at constant $\beta$) as expected. The cases with
$n_e=3\times10^{19}\;m^{-3}$ are not plotted for $n_\phi>18$ as the most unstable
mode is no longer the peeling-ballooning mode but rather a dominatly
electro-static mode related to the ITG as studied in the context NIMROD
two-fluid advance in Ref.~\cite{Schnack13}. This region is excluded as
examination of ITG mode dynamics is outside the scope of this work.

% In order to ascertain the role of two-fluid effects further, we vary one of the
% two-fluid FLR parameters that set the ion gyroradius $\rho_i$: the ion skin
% depth, $d_i$, and the plasma $\beta$.  Keeping the Meudas-1 current and pressure
% profiles fixed, we scale the density and temperature profiles in such a way
% that the plasma pressure remains the same in all simulations. This modifies the
% ion skin depth at constant $\beta$. The current (including the bootstrap
% contribution) and the Lundquist and Prandtl numbers are fixed to isolate
% effects from the two-fluid, FLR model only. As shown in
% Fig.~\ref{densityScan}, the low density cases (large $d_i$, $\rho_i$) peak at
% lower $n$ and are stabilized. However, there is a high-$n$ destabilization that
% is related to the ITG drive as captured by the fluid model. This ITG drive has
% been studied further in Ref.~\cite{Schnack13}.  It is our experience that for
% reconstructed cases that include density and temperature measurements, the ITG
% drive is small - likely a result of ITG modification of the profiles to ensure
% at least marginal stability.


