\section{Source and Target Languages}  \label{sec:targetlanguage}

This section introduces the source and target languages for our reduction.
The source language is the
polyadic \(\pi\)-calculus~\cite{milner1993polyadic} extended with
integers and conditional expressions,
and the target language is a first-order functional language with non-determinism.


\subsection{$\pi$-Calculus}
\subsubsection{Syntax}

Below we assume
a countable set of variables ranged over by \(x, y, z, w,\!\ldots\)
and write \( \mathbb{Z} \) for the set of integers, ranged
over by \( i \).  We write $\tilde{\cdot}$ for (possibly empty) finite
sequences; for example, $\tilde{x}$ abbreviates a sequence
$x_1,\dots,x_n$.  We write \( \len{\tilde{x}} \) for the length of \(\seq{x}\) and
\(\epsilon\) for the empty sequence.

The sets of \emph{processes} and \emph{simple expressions},
ranged over by $P$ and $v$ respectively, are defined inductively
by: %
\begin{align*}
    &P \mbox{ (processes) }::= \ \zeroexp \mid \outexp{x}{\seq{v}}{\seq{w}}P \mid \inexp{x}{\seq{y}}{\seq{z}}P \mid \rinexp{x}{\seq{y}}{\seq{z}}P \mid (P_1 \PAR P_2) \mid \nuexp{x \COL \chty} P \\
    &\hphantom{P \mbox{ (processes) }::=}     \mid \ifexp{v}{P_1}{P_2} \mid \ndlet{x}{P} \\
    &v \mbox{ (simple expressions) }::= \ x \mid i \mid \op(\tilde{v})
\end{align*}
The syntax of processes on the first line is fairly standard, except that
the values sent along each channel consist of two parts: \(\seq{v}\) for integers,
and \(\seq{w}\) for channels; this is for the sake of technical convenience in
presenting the translation to sequential programs. The process \(\zeroexp\)
denotes an inaction, \(\outexp{x}{\tilde{v}}{\tilde{w}}P\) sends
a tuple \((\tilde{v},\tilde{w})\) along the channel \(x\) and behaves like \(P\),
and the process \(\inexp{x}{\tilde{y}}{\tilde{z}}P\) receives
a tuple \((\tilde{v},\tilde{w})\) along the channel \(x\), and behaves like
\([\seq{v}/\seq{y}, \seq{w}/\seq{z}]P\). We often just write \(\seq{v}\) for
\(\seq{v};\epsilon\) or \(\epsilon;\seq{v}\).
The process \(\rinexp{x}{\seq{y}}{\seq{z}}P\) represents infinitely many copies
of \(\inexp{x}{\seq{y}}{\seq{z}}P\) running in parallel.
The process \(P_1\PAR P_2\) runs \(P_1\) and \(P_2\) in parallel,
and \(\nuexp{x\COL\chty}{P}\) creates a fresh channel \shchanged{$x$} of type \(\chty\) (where types
will be introduced shortly) and behaves like \(P\).
The process \(\ifexp{v}{P_1}{P_2}\) executes \(P_1\) if the value of \(v \) is
non-zero, and \(P_2\) otherwise.
The process \(\ndlet{x}{P}\) instantiates the variables \(\seq{x}\) to
some integer values in a non-deterministic manner, and then behaves like \(P\).
The meta-variable \( \op \) ranges over integer operations such as \( + \) or \( \le \).


The free and bound variables are defined as usual.
The only binders are \(\nuexp{x\COL\chty}\)
(which binds \(x\)), \(\ndletsatom{x}\) (which binds \(\seq{x}\)),
\(\inexp{x}{\seq{y}}{\seq{z}}\)
and \(\rinexp{x}{\seq{y}}{\seq{z}}\) (which bind \(\seq{y}\) and \(\seq{z}\)).
Processes are identified up to renaming of bound variables,
and we implicitly apply \( \alpha \)-conversions as necessary.

We write \(P\red Q\) for the standard one-step reduction relation on processes.
The base cases of the communication are given by:
\begin{align*}
    \inexp{x}{\seq{y}}{\seq{z}}P_1 \PAR  \outexp{x}{\seq{v}}{\seq{w}}P_2 &\red [\seq{i}/ \seq{y}, \seq{w} / \seq{z} ]P_1 \PAR P_2 \\
    \rinexp{x}{\seq{y}}{\seq{z}}P_1 \PAR  \outexp{x}{\seq{v}}{\seq{w}}P_2 &\red \rinexp{x}{\seq{y}}{\seq{z}}P_1 \PAR [\seq{i}/ \seq{y}, \seq{w} / \seq{z}]P_1 \PAR P_2
\end{align*}
provided that \( \seq{v} \) evaluates to \( \seq{i} \).
\ifaplas{
  The full definition is given in
  the extended version~\cite{fullversion}}{The full definition is given in Appendix~\ref{sec:operational_semantics}}.
We say that a process \(P\) is \emph{terminating} if there is no infinite
reduction sequence \(P\red P_1\red P_2\red \cdots\).

In the rest of the paper, we consider only well-typed processes.
We write \(\ty\) for the type of integers.
The set of channel types, ranged over by \(\chty\), is given by:
\begin{align*}
  \chty ::= \Chty{\reg}{\tilde{\ty}}{\tilde{\chty}}
\end{align*}
The type $\Chty{\reg}{\tilde{\ty}}{\tilde{\chty}}$
describes channels used for transmitting a tuple \((\seq{v};\seq{w})\)
of integers \(\seq{v}\) and channels \(\seq{w}\) of types \(\seq{\chty}\).
Below we will just write \( \seq{\ty} \) for \( \seq{\ty}; \epsilon \) and \( \seq{\chty} \) for \( \epsilon ;\seq{\chty}\).
The subscript \(\reg\), called a \emph{region}, is a symbol that
abstracts channels; it is used in the translation to sequential programs.
For example, \(\Chty{\reg_1}{\ty}{\sChty{\reg_2}{\ty}}\) is the type of channels
that belong to the region \(\reg_1\) and are
used for transmitting a pair \((i,r)\) where
\(r\) is a channel of region \(\reg_2\) used for transmitting integers.
We use a meta-variable \(\mty\) for an integer or channel type.

Type judgments for processes and simple expressions are of the form \(\env;\chenv\p P\)
and \(\env;\chenv\p v:\mty\), where \(\env\) and \(\chenv\)
are sequences of bindings of the form
\(x\COL\ty\) and \(x\COL\chty\), respectively.
The typing rules are shown in Figure~\ref{fig:simple_type_system}.
Here \( \env; \chenv \vdash \seq{v} \COL \seq{\mty} \) means \( \env; \chenv \p v_i : \mty_i \) holds for each \( i \in \{ 1, \ldots, \len{\seq{v}} \} \).
We omit the explanation of the typing rules as they are standard.
\begin{figure}[tb]
    \centering
    \small
    \begin{minipage}{0.3\linewidth}
        \centering
        \begin{prooftree}
            \AxiomC{}
            \UnaryInfC{$\env; \chenv \vdash \zeroexp$}
        \end{prooftree}
    \end{minipage}
    \begin{minipage}{0.65\linewidth}
        \centering
        \begin{prooftree}
            \AxiomC{$\env; \chenv \vdash v \COL \ty$}
            \AxiomC{$\env; \chenv \vdash P_1$}
            \AxiomC{$\env; \chenv \vdash P_2$}
            \TrinaryInfC{$\env; \chenv \vdash \ifexp{v}{P_1}{P_2}$}
        \end{prooftree}
    \end{minipage}
    \\\vspace*{1ex}
    \begin{minipage}{0.38\linewidth}
        \centering
        \begin{prooftree}
            \AxiomC{$\env; \chenv \vdash P_1$}
            \AxiomC{$\env; \chenv \vdash P_2$}
            \BinaryInfC{$\env; \chenv \vdash P_1 \PAR P_2$}
        \end{prooftree}
    \end{minipage}
    \begin{minipage}{0.25\linewidth}
        \centering
        \begin{prooftree}
            \AxiomC{$\env; \chenv, x\COL\chty \vdash P$}
            \UnaryInfC{$\env; \chenv \vdash \nuexp{x \COL \chty}P$}
        \end{prooftree}
    \end{minipage}
    \begin{minipage}{.3\linewidth}
        \begin{prooftree}
            \AxiomC{$\env, \seq{x}\COL \seq{\ty}; \chenv \vdash P$}
            \UnaryInfC{$\env; \chenv \vdash \ndlet{x}{P}$}
        \end{prooftree}
    \end{minipage}
    \\\vspace*{1ex}
    \begin{minipage}{\linewidth}
        \centering
        \begin{prooftree}
            \AxiomC{$\env; \chenv \vdash x\COL \Chty{\reg}{\seq{\ty}}{\seq{\chty}}$}
            \AxiomC{$\env, \seq{y}\COL \seq{\ty}; \chenv, \tilde{z}\COL \tilde{\chty} \vdash P$}
            \BinaryInfC{$\env; \chenv \vdash \inexp{x}{\seq{y}}{\seq{z}}P $}
        \end{prooftree}
    \end{minipage}
    \\\vspace*{1ex}
    \begin{minipage}{\linewidth}
        \centering
        \begin{prooftree}
            \AxiomC{$\env; \chenv \vdash x\COL \Chty{\reg}{\seq{\ty}}{\seq{\chty}}$}
            \AxiomC{$\env; \chenv \vdash \seq{v}\COL \seq{\ty}$}
            \AxiomC{$\env; \chenv \vdash \seq{w}\COL \seq{\chty}$}
            \AxiomC{$\env; \chenv \vdash P$}
            \QuaternaryInfC{$\env; \chenv \vdash \outexp{x}{\seq{v}}{\seq{w}}P$}
        \end{prooftree}
    \end{minipage}
    \\\vspace*{1ex}
    \begin{minipage}{\linewidth}
        \begin{prooftree}
            \AxiomC{$\env; \chenv \vdash x\COL \Chty{\reg}{\seq{\ty}}{\seq{\chty}}$}
            \AxiomC{$\env, \seq{y}\COL \seq{\ty}; \chenv, \seq{z}\COL \seq{\chty} \vdash P$}
            \BinaryInfC{$\env; \chenv \vdash \rinexp{x}{\seq{y}}{\seq{z}}P$}
        \end{prooftree}
    \end{minipage}
    \\\vspace*{1ex}
    \begin{minipage}{0.2\linewidth}
        \centering
        \begin{prooftree}
            \AxiomC{$x\COL\ty \in \env$}
            \UnaryInfC{$\env; \chenv \vdash x\COL \ty$}
        \end{prooftree}
    \end{minipage}
    \begin{minipage}{0.2\linewidth}
        \centering
        \begin{prooftree}
            \AxiomC{$x\COL\chty \in \chenv$}
            \UnaryInfC{$\env; \chenv \vdash x\COL \chty$}
        \end{prooftree}
    \end{minipage}
    \begin{minipage}{0.18\linewidth}
        \centering
        \begin{prooftree}
            \AxiomC{}
            \UnaryInfC{$\env; \chenv \vdash i\COL \ty$}
        \end{prooftree}
    \end{minipage}
    \begin{minipage}{0.25\linewidth}
        \centering
        \begin{prooftree}
            \AxiomC{$\env; \chenv \vdash \tilde{v}\COL \seq{\ty}$}
            \UnaryInfC{$\env; \chenv \vdash \op(\seq{v})\COL \ty$}
        \end{prooftree}
    \end{minipage}
    \normalsize
    \caption{The typing rules of the simple type system for the $\pi$-calculus}
    \label{fig:simple_type_system}
\end{figure}





\subsection{Sequential Language}
We define the target language of our translation, which
is a first-order functional language with non-determinism.

A \emph{program} is a pair \((\Def, \Exp)\) consisting of (a set of)
\emph{function definitions} \(\Def\) and
an \emph{expression} \(\Exp\), defined by:
\begin{align*}
    \mathcal{D} \ \text{(function definitions)}        \, ::=& \ \{ \fdef{f_1}{\tilde{x}_1}{E_1}, \ldots, \fdef{f_n}{\tilde{x}_n}{E_n} \} \\
    E           \ \text{(expression)}                  \, ::=& \ \skipexp
    \mid \ndlet{x}{E} \mid f(\tilde{v}) \mid E_1 \nondet E_2 \\
                                                             & \ \mid \textbf{if } v \textbf{ then } E_1 \textbf{ else } E_2 \mid \textbf{Assume}(v); E \\
    v           \ \text{(simple expressions)}                       \, ::=& \ x \mid i \mid \op(\tilde{v})
\end{align*}
In a function definition
\( \fdef{f_i}{x_1, \ldots , x_{k_i}}{\Exp_i} \),
the variables \( x_1, \ldots, x_{k_i} \)
are bound in \( \Exp_i \); we identify function definitions up to renaming of bound
variables, and implicitly apply \(\alpha\)-conversions.
The function names \( f_1, \ldots, f_n \) need not be distinct from each other.
  If there are more than one definition for \( f \), then one of the definitions
  will be non-deterministically used when \( f \) is called.
We explain the informal meanings of the nonstandard expressions.
The expression \( \ndlet{x}{E} \)
instantiates \(\seq{x}\) to some integers in a non-deterministic manner.
The expression \(E_1\nondet E_2\) non-deterministically evaluates to
\(E_1\) or \(E_2\).
The expression
\(\textbf{Assume}(v); E \) evaluates to \(E\) if \(v\) is non-zero;
otherwise the whole program is aborted.
The other expressions are standard and their meanings should be clear.

We write \((\Def,\Exp)\sred (\Def,\Exp')\)
for the one-step reduction relation, whose definition is given
in \ifaplas{the extended version~\cite{fullversion}}{Appendix~\ref{sec:operational_semantics}}.
We say that a program is terminating if there is no infinite
reduction sequence.
