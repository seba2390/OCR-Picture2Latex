\section{Proof of the Soundness}  \label{sec:soundness}
Here we prove the soundness of the translation (Theorem~\ref{thm:soundness}) saying that if the sequential program \( (\Def, \Exp) \) obtained by translating \( P \) is terminating, \( P \) is also terminating.
The proof is split into two steps.
First, we show that reductions from \( P \) can be simulated by non-standard reductions from \( (\Def, \Exp ) \) (Lemma~\ref{lem:simulate}).
This implies that if \( (\Def, \Exp) \) is terminating with respect to the non-standard reduction, then \( P \) is terminating.
Then we show that if  \( (\Def, \Exp) \) is terminating with respect to the standard reduction, then \( (\Def, \Exp) \) is terminating with respect to the non-standard reduction (Lemma~\ref{lem:konig}).




We start by preparing some auxiliary lemmas that are used to show the simulation relation.

\begin{lemma}[substitution]
    \label{lem:subst}
    If \( \env; \chenv \vdash \seq{v} : \seq{\ty} \),
       \( \env; \chenv \vdash \seq{w} : \seq{\chty} \)
       and \( \env, \seq{y}:\seq{\ty}; \chenv, \seq{z}:\seq{\chty} \vdash P \Rightarrow
       \prog{\Def}{\Exp} \),
       then \( \env; \chenv \vdash [\seq{v}/\seq{y}, \seq{w} / \seq{z}]P \Rightarrow
       \prog{[\seq{v}/\seq{y}]\Def}{[\seq{v}/\seq{y}] \Exp} \).
\end{lemma}
\begin{proof}
    By induction on the derivation of $\env, \seq{y}:\seq{\ty}; \chenv, \seq{z}:\seq{\chty} \vdash P \Rightarrow \prog{\Def}{\Exp}$.
    \qed
\end{proof}



\begin{lemma} \label{lem:cong}
    If $P \piequiv P'$ and $\env; \chenv \vdash P \Rightarrow \prog{\Def}{\Exp}$,
    then there exists $\Exp'$ such that
    $\Exp \expequiv \Exp'$
    and $\env; \chenv \vdash P' \Rightarrow \prog{\Def}{\Exp'}$.
\end{lemma}
\begin{proof}
    By induction on the construction of $P \piequiv P'$.
    \qed
\end{proof}

Now we prove the simulation relation.

\begin{lemma} \label{lem:simulate}
    If $P \red P'$ and $\env; \chenv \vdash P \Rightarrow \prog{\Def}{\Exp}$,
    then there exist $\Def'$, $\Exp'$ such that \( \Def \subdef \Def' \),
    \( (\Def', \Exp) \nsred^+ (\Def', \Exp') \)
    and $\env; \chenv \vdash P' \Rightarrow \prog{\Def'}{\Exp'}$.
\end{lemma}
\begin{proof}
    By induction on the construction of \( P \red P' \).
    We only give detailed proofs for  interesting cases; the other cases are sketched.

\begin{description}
    \item[Case \rn{R-Comm}:]
    In this case $P \to P'$ must be of the form
    \begin{align*}
    \inexp{x}{\seq{y}}{\seq{z}}P_1 \PAR \outexp{x}{\seq{v}}{\seq{w}}P_2 \red [\seq{i}/ \seq{y} ,\seq{w} / \seq{z}]P_1 \PAR P_2,
    \end{align*}
    where
    $\len{\seq{y}} = \len{\seq{v}}$,
    $\len{\seq{z}} = \len{\seq{w}}$
    and $\seq{v} \Downarrow \seq{i}$.
    Also $\env; \chenv \vdash P \Rightarrow \prog{\Def}{\Exp}$ must be the form of
    \begin{align*}
      \env; \chenv \vdash P \Rightarrow
      \prog{(\ndlet{y}{\Def_1}) \mrg \Def_2}{(\ndlet{y}{\Exp_1}) \nondet (f_\reg(\seq{v}) \oplus \Exp_2)},
    \end{align*}
    where
    \begin{align}
    &\env; \chenv \vdash x : \Chty{\reg}{\seq{\ty}}{\seq{\chty}}
    \qquad \env; \chenv \vdash \seq{v} : \seq{\ty}
    \qquad \env; \chenv \vdash \seq{w} : \seq{\chty} \nonumber \\
    &\env, \seq{y} : \seq{\ty}; \chenv, \seq{z} : \seq{\chty} \vdash P_1 \Rightarrow
    \prog{\Def_1}{\Exp_1} \label{eq:sim:com:trans-p1} \\
    & \env; \chenv \vdash P_2 \Rightarrow \prog{\Def_2}{\Exp_2}. \label{eq:sim:com:trans-p2}
    \end{align}
    By applying Lemma~\ref{lem:subst} to \eqref{eq:sim:com:trans-p1} with
    $\env; \chenv \vdash \seq{i} : \seq{\ty}$,
    $\env; \chenv \vdash \seq{w} : \seq{\chty}$,
    we obtain
    $\env; \chenv \vdash [\seq{i} / \seq{y}, \seq{w})/\seq{z}]P_1 \Rightarrow
    \prog{[\seq{i}/\seq{y}]\Def_1}{[\seq{i}/\seq{y}]\Exp_1}$.
    From this and \eqref{eq:sim:com:trans-p2}, we have
    \begin{align*}
      \env; \chenv \vdash [\seq{i}/ \seq{y}, \seq{w}/ \seq{z}]P_1 \PAR P_2 \Rightarrow
      \prog{[\seq{i}/\seq{y}]\Def_1 \mrg \Def_2}
           {[\seq{i}/\seq{y}]\Exp_1 \nondet \Exp_2}
    \end{align*}
    by applying the rule \rn{SX-Par}.
    Observe that we also have
    \begin{align*}
      \Def = (\ndlet{y}{\Def_1}) \mrg \Def_2 \subdef [\seq{i}/\seq{y}]\Def_1 \mrg \Def_2.
    \end{align*}
    Therefore, for \( (\Def', \Exp')\) we can take \( ([\seq{i}/\seq{y}]\Def_1 \mrg \Def_2, [\seq{i}/\seq{y}]\Exp_1 \nondet \Exp_2 )\) with the following matching reduction sequence:
    \begin{align*}
      \Exp
      &=  (\ndlet{y}{\Exp_1}) \nondet f_\reg(\seq{v}) \nondet \Exp_2 \\
      &\nsred_{\Def'} [\seq{i}/\seq{y}]\Exp_1 \nondet f_\reg(\tilde{v}) \oplus E_2 \tag{\rn{SR-LetND}} \\
        &\nsred_{\Def'}, [\seq{i}/\seq{y}]\Exp_1 \nondet \skipexp \nondet \Exp_2 \tag{by (\rn{SR-App}) and \( \lambda \seq{y}.\skipexp \in \Def'(f_\reg) \) } \\
        &\expequiv [\seq{i}/\seq{y}]\Exp_1 \nondet \Exp_2.
    \end{align*}

    \item[Case \rn{R-RComm}:]
    In this case  $P \to P'$ is of the form
    \begin{align*}
      \rinexp{x}{\seq{y}}{\seq{z}}P_1 \PAR \outexp{x}{\seq{v}}{\seq{w}}P_2 \red \rinexp{x}{\seq{y}}{\seq{z}}P_1 \PAR [(\seq{i},\seq{w})/(\seq{y},\seq{z})]P_1 \PAR P_2,
    \end{align*}
    where
    $\len{\seq{y}} = \len{\seq{v}}$,
    $\len{\seq{z}} = \len{\seq{w}}$
    and $\seq{v} \Downarrow \seq{i}$.
    Moreover, the judgment $\env; \chenv \vdash P \Rightarrow
    \prog{\Def}{\Exp}$ must be of the form
    \begin{align*}
      \env; \chenv \vdash P \Rightarrow
      \prog{\{ \fdef{f_\reg}{\seq{y}}{\Exp_1} \} \mrg (\ndlet{y}{\Def_1}) \mrg \Def_2}
      {\skipexp \nondet f_\reg (\seq{v}) \nondet \Exp_2},
    \end{align*}
    where
    \begin{align}
      &\env; \chenv \vdash x : \Chty{\reg}{\seq{\ty}}{\seq{\chty}}
      \qquad \env; \chenv \vdash \seq{v} : \seq{\ty}
      \qquad \env; \chenv \vdash \seq{w} : \seq{\chty} \nonumber \\
      &\env; \chenv \vdash \rinexp{x}{\seq{y}}{\seq{z}}P_1 \Rightarrow
      \prog{\{ \fdef{f_\reg}{\seq{y}}{\Exp_1} \} \mrg (\ndlet{y}{\Def_1})}{
        \skipexp} \label{eq:sim:rcom:trans-in} \\
      &\env, \seq{y} : \seq{\ty}; \chenv, \seq{z} : \seq{\chty} \vdash P_1 \Rightarrow \Def_1; \Exp_1 \label{eq:sim:rcom:trans-p1}\\
      &\env; \chenv \vdash P_2 \Rightarrow \prog{\Def_2}{\Exp_2}. \label{eq:sim:rcom:trans-p2}
    \end{align}
    Since
    $\env; \chenv \vdash \seq{i} : \tilde{\ty}$ and
    $\env; \chenv \vdash \tilde{w} : \tilde{\chty}$, we can apply the substitution lemma (Lemma~\ref{lem:subst}) to \eqref{eq:sim:rcom:trans-p1} and obtain
    \begin{align*}
      \env; \chenv \vdash [\seq{i} / \seq{y}, \seq{w}/\seq{z}]P_1 \Rightarrow
          \prog{[\seq{i}/\seq{y}]\Def_1}{[\seq{i}/\seq{y}]\Exp_1}.
    \end{align*}
    From this, \eqref{eq:sim:rcom:trans-in} and \eqref{eq:sim:rcom:trans-p2}, we have
    \begin{align*}
      \env; \chenv \vdash P' \Rightarrow
      \begin{aligned}
        &(\{ \fdef{f_\reg}{\seq{y}}{\Exp_1} \} \mrg (\ndlet{y}{\Def_1}) \mrg  [\seq{i}/\seq{y}]\Def_1 \mrg \Def_2, \\
        &\skipexp \nondet [\seq{i}/\seq{y}]\Exp_1 \nondet \Exp_2)
      \end{aligned}
    \end{align*}
    So we can take \( \{ \fdef{f_\reg}{\seq{y}}{\Exp_1} \} \mrg (\ndlet{y}{\Def_1}) \mrg  [\seq{i}/\seq{y}]\Def_1 \mrg \Def_2 \) as  \( \Def' \) and \( \skipexp \nondet [\seq{i}/\seq{y}]\Exp_1 \nondet \Exp_2 \) as \( \Exp' \).
    Now it remains to show that \( \Def \subdef \Def' \) and that there is a reduction sequence from \( (\Def', \Exp) \) to \( (\Def', \Exp') \).
    The relation \( \Def \subdef \Def' \) holds because
    \begin{align*}
      \Def
      &= (\{ \fdef{f_\reg}{\seq{y}}{\Exp_1} \} \mrg (\ndlet{y}{\Def_1}) \mrg \Def_2 \\
      &= \{ \fdef{f_\reg}{\seq{y}}{\Exp_1} \} \mrg (\ndlet{y}{\Def_1}) \mrg (\ndlet{y}{\Def_1}) \mrg \Def_2 \\
      &\subdef \{ \fdef{f_\reg}{\seq{y}}{\Exp_1} \} \mrg (\ndlet{y}{\Def_1}) \mrg [\seq{i} / \seq{y}]{\Def_1} \mrg \Def_2 \tag{\rn{D-ND}} \\
      &=\Def'
    \end{align*}

    Finally, by \rn{SR-App}, we obtain
    \begin{align*}
      \Exp &= \skipexp \nondet f_\reg (\seq{v}) \nondet \Exp_2  \nsred_{\Def'} \skipexp \nondet [\seq{i}/\seq{y}]\Exp_1 \nondet \Exp_2  = \Exp'
    \end{align*}
    as desired.
    \item[Case \rn{R-If-T}:]
      In this case \( P \red P' \) and \( \env; \chenv \vdash P \Rightarrow
      \prog{\Def}{\Exp} \) must be of the form
      \begin{align*}
      &\ifexp{v}{P_1}{P_2} \red P_1 \\
        &\env; \chenv \vdash \ifexp{v} {P_1}{P_2} \Rightarrow
        \prog{\Def_1 \mrg \Def_2}{\ifexp{v}{\Exp_1}{\Exp_2}}
      \end{align*}
      where
      \begin{align*}
        v \Downarrow i \neq 0  \qquad \env; \chenv \vdash v : \ty \\
        \env; \chenv \vdash P_1 \Rightarrow \prog{\Def_1}{\Exp_1} \\
        \env; \chenv \vdash P_2 \Rightarrow \prog{\Def_2}{\Exp_2}.
      \end{align*}
    We can take \( (\Def_1, \Exp_1) \) for \( ( \Def', \Exp' )\) because \( \Def_1 \mrg \Def_2 \subdef \Def_1 \), and \( \Exp \nsred_{\Def_1} \Exp_1 \), which is trivial from \rn{SR-If-T}.

    \item[Case \rn{R-If-F}:]
    Similar to the previous case.

    \item[Case \rn{R-Cong}:]
    In this case \( P \red P' \) must be of the form
    \begin{align*}
      P \piequiv P_1 \red P_1' \piequiv P'.
    \end{align*}
    By Lemma~\ref{lem:cong}, we have
    \begin{align*}
      \env, \chenv \vdash P_1 \Rightarrow \prog{\Def}{\Exp_1} \text{ and } \Exp \expequiv \Exp_1
    \end{align*}
    for some \( \Exp_1 \).
    Thus, by the induction hypothesis, we have
    \begin{align}
      &\env, \chenv \vdash P_1' \Rightarrow \prog{\Def'}{\Exp_1'} \label{eq:sim:cong-P1prime}\\
      & (\Def',  \Exp_1) \nsred^+ (\Def', \Exp_1') \label{eq:sim:cong:red-seq}
    \end{align}
    where \( \Def \subdef \Def' \).
    By applying Lemma~\ref{lem:cong} to \eqref{eq:sim:cong-P1prime}, we obtain
    \begin{align*}
      \env, \chenv \vdash P' \Rightarrow \prog{\Def'}{\Exp'} \text{ and } \Exp_1' \expequiv \Exp'
    \end{align*}
    for some \( \Exp' \).
    It remains to show that \( (\Def', \Exp) \nsred^+  (\Def', \Exp') \), but this is easily shown by repeatedly applying the rule \rn{SR-Cong} along the reduction sequence \eqref{eq:sim:cong:red-seq}.

    \item[Case \rn{R-Par}, \rn{R-Nu} and \rn{R-LetND}:]
    Similar to the previous case, i.e.~follows from the definition of the translation and the induction hypothesis together with Lemma~\ref{lem:subdef}.
\end{description}







\leavevmode\qed
\end{proof}


\begin{lemma}  \label{lem:infinitechain}
    Suppose that \( \emptyset; \emptyset \vdash P \Rightarrow \prog{\Def}{\Exp} \).
    If \( (\mathcal{D}, E) \) is terminating with respect to $\nsred$, then \( P \) is terminating.
\end{lemma}
\begin{proof}
    We show the contraposition.
    Assume that \( P \) is not terminating, i.e.~assume that there exists an infinite reduction sequence $P = P_0 \red P_1 \red \cdots$.
    Let \( \Def_0 = \Def \) and \( \Exp_0 = \Exp \).
    By applying Lemma~\ref{lem:simulate}, for each natural number \( k \ge 1 \), we obtain \( \Def_k \), \( \Exp_k \) such that \( \emptyset; \emptyset \vdash P_k \Rightarrow
    \prog{\Def_k}{\Exp_k} \), \( (\Def_{k}, \Exp_{k-1}) \nsred^+ (\Def_{k}, \Exp_{k}) \) and \( \Def \subdef \Def_k \).
    Hence, by Lemma~\ref{lem:subdef} there exists an infinite reduction sequence
    $(\Def, \Exp) = (\Def, \Exp_0)\allowbreak \nsred^+ (\Def, \Exp_1) \nsred^+ \cdots$.
    \qed
\end{proof}

We now show the relation between standard and non-standard reductions.
\begin{lemma}  \label{lem:konig}
    Assume that $\emptyset; \emptyset \vdash P \Rightarrow \prog{\Def}{\Exp}$.
    If \( (\Def, \Exp) \) is terminating with respect to the standard reduction \( \sred \), then \( (\Def, \Exp)\) is also terminating with respect to the non-standard reduction relation \( \nsred \).
\end{lemma}

To prove the lemma above, we introduce a slight variation of the
non-standard reduction relation:
\((\Def,\Exp) \nsredv{\gamma} (\Def',\Exp')\) where
\(\gamma\in \set{1,2}^*\). (Actually, \(\Def\) does not change during the reduction.)
It is defined by the rules in Figure~\ref{fig:nsredv}.

\begin{figure}[tbp]

  \infrule[NSR-LetND]
          {\len{\seq{x}} = \len{\seq{i}}}
          {(\Def, \ndlet{x}{\Exp} )  \nsredv{\epsilon} (\Def, [\seq{i}/\seq{x}]\Exp)}
\vspace*{1ex}
          \infrule[NSR-App]
{ (\lambda \seq{y}.\Exp) \in  \Def(f)\andalso
 \len{\seq{y}} = \len{\seq{v}}\andalso
 \seq{v} \Downarrow \seq{i}}
{(\Def, f(\seq{v})) \nsredv{\epsilon} (\Def, [\seq{i}/\tilde{y}] \Exp)}

\vspace*{1ex}
\infrule[NSR-If-T]{v \Downarrow i \andalso i \neq 0}
 {(\Def, \ifexp{v}{\Exp_1}{\Exp_2})  \nsredv{\epsilon} (\Def, \Exp_1)}
\vspace*{1ex}
\infrule[NSR-If-F]{v \Downarrow 0}
 {(\Def, \ifexp{v}{\Exp_1}{\Exp_2})  \nsredv{\epsilon} (\Def, \Exp_2)}
\vspace*{1ex}

 \infrule[NSR-ChoBody-L]
 {(\Def, \Exp_1) \nsredv{\gamma} (\Def, \Exp_1')}
 {(\Def, \Exp_1 \nondet \Exp_2) \nsredv{1\cdot \gamma} (\Def, \Exp_1' \nondet \Exp_2)}
\vspace*{1ex}
 \infrule[NSR-ChoBody-R]
 {(\Def, \Exp_2) \nsredv{\gamma} (\Def, \Exp_2')}
 {(\Def, \Exp_1 \nondet \Exp_2) \nsredv{2\cdot \gamma} (\Def, \Exp_1 \nondet \Exp_2')}

\vspace*{1ex}
 \infrule[NSR-Ass-T]
  {v \Downarrow i\andalso i \neq 0}
  {(\Def, \textbf{Assume}(v);E) \nsredv{\epsilon} (\Def, \Exp)}
\vspace*{1ex}
 \infrule[NSR-Ass-F]
  {v \Downarrow 0}
  {(\Def, \textbf{Assume}(v);E) \nsredv{\epsilon} (\Def, \skipexp)}

\caption{A variation of the non-standard reduction relation}
\label{fig:nsredv}
\end{figure}

The only differences of
\((\Def,\Exp) \nsredv{\gamma} (\Def',\Exp')\) from
\((\Def,\Exp) \nsred (\Def',\Exp')\) are that
the reduction is annotated with the position \(\gamma\) that indicates where the reduction occurs,
and that the rule \rn{SR-Cong} for shuffling expressions is forbidden.
Since the rule \rn{SR-Cong} does not affect the reducibility, we can easily
observe the following property. (We omit the proof since it is trivial.)
\begin{lemma}
  \label{lem:nsred-vs-nsredv}
  If \((\Def,\Exp)\) has an infinite reduction sequence with respect to \(\nsred\),
  \((\Def,\Exp)\) has an infinite reduction sequence also with respect to \(\nsredv{\gamma}\).
\end{lemma}

It remains to show that
if \((\Def,\Exp)\) has an infinite reduction sequence
\[(\Def,\Exp)\nsredv{\gamma_1} (\Def,\Exp_1)\nsredv{\gamma_2}
(\Def,\Exp_2)\nsredv{\gamma_3}(\Def,\Exp_3)\nsredv{\gamma_4}\cdots,\]
then
\((\Def,\Exp)\) has an infinite reduction sequence also with respect to \(\sred\).

We write \(\gamma \preceq \gamma'\) if \(\gamma\) is a prefix of \(\gamma'\).
We have the following property.
\begin{lemma}
  \label{lem:inf-nsredv}
  If
\[(\Def,\Exp)\nsredv{\gamma_1} (\Def,\Exp_1)\nsredv{\gamma_2}
(\Def,\Exp_2)\nsredv{\gamma_3}(\Def,\Exp_3)\nsredv{\gamma_4}\cdots,\]
then there exists an infinite sequence
\(i_1 < i_2 < i_3< \cdots\)
such that \(\gamma_{i_j} \preceq \gamma_{i_k}\) for any \(j<k\).
\end{lemma}
\begin{proof}
  The required property obviously holds if   the set \(\set{\gamma_i\mid i\ge 1}\) is finite.
  So, assume that \(\set{\gamma_i\mid i\ge 1}\) is infinite.
  Let \(T\) be the least binary tree that contains, for every \(\gamma_i\),
  the node whose path from the root is \(\gamma_i\).
  By the assumption that \(\set{\gamma_i\mid i\ge 1}\) is infinite,
  \(T\) is an infinite tree. Thus, by K\"onig's lemma,
  \(T\) must have an infinite path, which implies that
  there exists an infinite sequence
  \[\gamma_{i_1} \preceq \gamma_{i_2} \preceq \gamma_{i_3} \preceq \cdots, \]
  as required. \qed
\end{proof}


For an expression \(\Exp\) and a position \(\gamma\in\set{1,2}^*\), we write
\(\Proj{\Exp}{\gamma}\) for the subexpression at \(\gamma\). It is inductively defined by:
\[
\begin{array}{l}
\Proj{\Exp}{\epsilon} = \Exp\\
\Proj{\Exp}{i\cdot \gamma} =
\left\{\begin{array}{ll}
  \Proj{\Exp_i}{\gamma} & \mbox{if $\Exp$ is of the form \(\Exp_1\nondet \Exp_2\)}\\
  \mbox{undefined}\hspace*{2em} & \mbox{otherwise}
\end{array}\right.
\end{array}
\]
The following lemma states the correspondence between \(\nsredv{\gamma}\) and \(\sred\).

\begin{lemma}
\label{lem:nsredv-vs-sred}
  \begin{enumerate}
\item  If \((\Def,\Exp)\nsredv{\gamma}(\Def,\Exp')\),
  then \((\Def, \Proj{\Exp}{\gamma})\sred (\Def,\Proj{\Exp'}{\gamma})\).
\item Suppose \(\Proj{\Exp}{\gamma'}\) is defined and \(\gamma'\not\preceq \gamma\).
  If \((\Def,\Exp)\nsredv{\gamma}(\Def,\Exp')\), then
  \(\Proj{\Exp}{\gamma'}=\Proj{\Exp'}{\gamma'}\).
\item
  If \((\Def,\Exp)\nsredv{\gamma}(\Def,\Exp')\), and \(\gamma'\preceq \gamma\),
  then \((\Def, \Proj{\Exp}{\gamma'} ) \sred^* (\Def, \Proj{\Exp}{\gamma})\).
\end{enumerate}  
\end{lemma}
\begin{proof}
  The properties follow by a straightforward induction on the derivation of
  \((\Def,\Exp)\nsredv{\gamma}(\Def,\Exp')\). \qed
\end{proof}

We are now ready to prove Lemma~\ref{lem:konig}.

\begin{proof}[of Lemma~\ref{lem:konig}]
  We show the contraposition.
  Suppose \((\Def,\Exp)\) has an infinite reduction sequence with respect to \(\nsred\).
  By Lemma~\ref{lem:nsred-vs-nsredv}, there exists an infinite reduction sequence
  \[(\Def,\Exp)\nsredv{\gamma_1} (\Def,\Exp_1)\nsredv{\gamma_2}
(\Def,\Exp_2)\nsredv{\gamma_3}(\Def,\Exp_3)\nsredv{\gamma_4}\cdots.\]
  By Lemma~\ref{lem:inf-nsredv},
 there exists an infinite sequence:
  \[\gamma_{i_1}\preceq \gamma_{i_2}\preceq \gamma_{i_3}\preceq \cdots.\]
  such that \(i_1<i_2<i_3<\cdots\).
  Let us choose a maximal one among such sequences, i.e.,
  a sequence
  \[\gamma_{i_1}\preceq \gamma_{i_2}\preceq \gamma_{i_3}\preceq \cdots.\]
  such that, for any \(i_j\), 
  \(\gamma_{k}\preceq \gamma_{i_{j}}\) implies \(k=i_{j'}\) for some \(j'\le j\).
  Consider the fragment of the infinite reduction sequence:
  \[
  (\Def,\Exp_{i_{\ell-1}})\nsredv{\gamma_{i_{\ell-1}+1}} (\Def,\Exp_{i_{\ell-1}+1})
  \nsredv{\gamma_{i_{\ell-1}+2}}\cdots
  \nsredv{\gamma_{i_{\ell}-1}}
(\Def,\Exp_{i_{\ell}-1})\nsredv{\gamma_{i_\ell}}(\Def,\Exp_{i_\ell})
  \]
  for each \(\ell>0\). 
  (Here, we define \(\gamma_0 = \epsilon\), \(i_0=0\) and \(E_0 = E\).)
  By Lemma~\ref{lem:nsredv-vs-sred} (1)
  and \((\Def,\Exp_{i_{\ell}-1})\nsredv{\gamma_{i_\ell}}(\Def,\Exp_{i_\ell})\),
  we have
  \[(\Def,\Proj{\Exp_{i_{\ell}-1}}{\gamma_{i_\ell}})\sred
  (\Def,\Proj{\Exp_{i_{\ell}}}{\gamma_{i_\ell}}).\]
  By Lemma~\ref{lem:nsredv-vs-sred} (2) (note that
  since none of \(\gamma_{i_{\ell-1}+1},\ldots,\gamma_{i_{\ell}-1}\) is a
  prefix of \(\gamma_{i_{\ell}}\) by the assumption on maximality,
  \(\Proj{\Exp_{i_{\ell-1}}}{\gamma_{i_\ell}}\) is defined),
  we have
  \[\Proj{\Exp_{i_{\ell-1}}}{\gamma_{i_\ell}} =
  \Proj{\Exp_{i_{\ell-1}+1}}{\gamma_{i_\ell}} = \cdots =
  \Proj{\Exp_{i_{\ell}-1}}{\gamma_{i_\ell}}.\]
  Thus, together with Lemma~\ref{lem:nsredv-vs-sred} (3), we obtain:
  \[(\Def,\Proj{\Exp_{i_{\ell-1}}}{\gamma_{i_{\ell-1}}})\sred^*
  (\Def,\Proj{\Exp_{i_{\ell-1}}}{\gamma_{i_{\ell}}})\sred
  (\Def,\Proj{\Exp_{i_{\ell}}}{\gamma_{i_\ell}}).\]
  Therefore,
  we have an infinite reduction sequence
  \[(\Def,\Exp)=(\Def,\Proj{\Exp_{i_0}}{\gamma_{i_0}})\sred^+ (\Def,\Proj{\Exp_{i_1}}{\gamma_{i_1}})
  \sred^+ (\Def,\Proj{\Exp_{i_2}}{\gamma_{i_2}})
  \sred^+ (\Def,\Proj{\Exp_{i_3}}{\gamma_{i_3}})
  \sred^+ \cdots,\]
  as required. \qed
\end{proof}




    
    



Finally, the soundness (Theorem~\ref{thm:soundness}) follows from Lemmas~\ref{lem:infinitechain} and \ref{lem:konig}.
