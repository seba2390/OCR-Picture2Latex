
\section{Experiments}\label{sec:experiments}
To assess the quality of the sentence representations learned by our model we evaluate on sentence similarity (Section \ref{sec:sim}), classification (Section \ref{sec:clf}), and generation tasks (Section \ref{sec:gen}).

%\nikos{we need to be more specific here. what are the questions we would like to answer specifically or our hypothesis. make sure the points we make correspond to the claims that we make in the intro/abstract and are clearly stated.}
% For fair comparison, we pretrain \textsc{Autobots} on the same Wikipedia + BooksCorpus dataset from BERT. To avoid training from scratch we use the pretrained BERT model as a starting point for our encoder function. We consider several different methods of pretraining. \nikos{these sound like experimental details, I would mention the basic ones in a subsection here and defer the rest for the supplementary (make sure we include everything there).}

\subsection{Settings} 
%\nikos{Add some basic details here. What data do we use + citation, where do we pretrain RoBERTa and for how long? fixed vs finetuning upper layers enc etc}
\paragraph{Datasets}
Since the RoBERTa dataset is not publicly available, we use for pretraining the exact same dataset as BERT \citep{devlin-etal-2019-bert}, which is composed of BooksCorpus \citep{zhu2015aligning} and English Wikipedia. For sentence similarity, we use the Natural Language Inference (NLI) dataset \citep{bowman2015snli} for finetuning and evaluate on the Semantic Textual Similarity (STS) dataset \citep{cer2017semeval}, following \citet{conneau2017supervised}.  For classification, we use mainly single-sentence datasets from the GLUE benchmark \citep{wang2018glue}, namely Stanford Sentiment Treebank (SST) and Corpus of Linguistic Acceptability (CoLA) datasets, but we also report the average performance on the remaining datasets. For generation, we use the Yelp reviews dataset \citep{shen2017style}.

% \nikos{we need to clarify if we use the exact same dataset and why. is it a subset of it or not?} \nikos{let's also mention here the details about datasets we use to evaluate each of the tasks for similarity, classsification and generation.}



% BooksCorpus (800M words) (Zhu et al.,
% 2015) and English Wikipedia (2,500M words). 
\paragraph{Baselines}
%\nikos{Mention which sentence representation methods we compare to (basic pooled, SBERT, etc) for each task.} 
For sentence similarity, we compare to SBERT which is a competitive method for deriving informative sentence representations from pretrained language models \cite{Reimers2019SentenceBERT}.
% \footnote{Note that, in contrast, our model does not require additional data during the pretraining phase. }
They obtain sentence representations by using simple pooling methods over BERT representations such as mean and max (instead of the CLS token representation) then finetuning the whole pretrained model using Siamese networks on a combination of natural language inference data.  To compare with them on sentence similarity, we incorporate our model within their framework and follow their settings and training/evaluation protocol (details in Appendix~\ref{adx:sentrep}). 

For sentence classification, we compare our model to RoBERTa-base and RoBERTa-large models \cite{liu2019RoBERTa}. Note that BART \cite{lewis2019bart} achieves similar results to RoBERTa, so a similar comparison can be made.  % They achieve sentence representations by encoding the premise $x$ and hypothesis $y$ sentences separately to get $\mathbf{z_x}$ and $\mathbf{z_y}$ respectively, then finetune a linear classification layer with softmax loss over the concatenation of $[\mathbf{z_x};\mathbf{z_y};|\mathbf{z_x}−\mathbf{z_y}|]$. \nikos{I feel this part can be skipped, is it that important to mention it in detail? Mentioned a high-level view of this.}

For sentence generation tasks, we compare to a strong and efficient style transfer method by \citet{shen2019educating}, which is a recurrent network-based denoising text autoencoder on in domain data. The style transfer is achieved  through vector arithmetic, namely computing a “sentiment vector” $\mathbf{v}$ by taking the vector difference between 100 negative and positive sentences, then evaluating by taking an input sentence, encoding it, adding a multiple of the sentiment vector to the sentence representation, then decoding the resulting representation. In addition to the denoising auto encoder (DAE) of \citet{shen2019educating}, we include more sophisticated methods for style transfer that are more computationally expensive such as fast gradient iterative modification (FGIM) of \citet{wang2019controllable} and Emb2Emb of \citet{mai2020plug} for reference.
 

% \nikos{Make clear on which tasks we finetune the whole model and which ones RoBERTa model is kept fixed. This is not clear at all up to this point.} \ivan{they were only used in generation (figure 2), but it's a question if we want to include them or not (and make everything strictly using a fixed encoder). The current setup of the paper doesn't not include them} \nikos{I meant when finetuning on a downstream task. regarding finetuning with the autoencoding objective we still need to mention what options we consider; hmm, yes let's describe everything and we decide after. }

% \subsection{BARNEY Pretraining for Classification}
% We outline several methods to pretrain BARNEY, and perform experiments on the sentence similarity task (SST2) with RoBERTa-base BARNEY, to observe the relative performance on single-sentence classification.

% \input{tables/barney_training_sst}

% \noindent \textbf{Datasets} \nikos{Mention some basic details about the datasets used here.}
% In our experiments
% NLI, GLUE, STS
 
% \noindent \textbf{Model configuration} \nikos{Only the basic ones and defer to the supplementary for the details such as number of epochs, learning rate etc. (e.g. we follow configuration from ...)}

% \noindent \textbf{Baselines} \nikos{Describe our baselines and the versions of our model that we examined.}



% \noindent \textbf{Datasets} \nikos{Mention some basic details about the datasets used here.}
% In our experiments
% NLI, GLUE, STS
 
% \noindent \textbf{Model configuration} \nikos{Only the basic ones and defer to the supplementary for the details such as number of epochs, learning rate etc. (e.g. we follow configuration from ...)}

% \noindent \textbf{Baselines} \nikos{Describe our baselines and the versions of our model that we examined.}


% \ivan{For each experiment, I will follow your Datasets, Model Configuration, Baselines format in the description}

\subsection{Sentence Similarity} \label{sec:sim}
%\nikos{you could mention the more general idea behind experiments in the first paragraph of this section and point to subsections (e.g. that we evaluate the representation and sentence generation quality of sentence representations extracted from pretrained models.}
% {
%\citet{conneau2017supervised} show that universal sentence representations can be obtained from finetuning on the NLI training data and evaluating the model's efficacy using the STS benchmark data. \nikos{Mentioned in the dataset section?} 

% The spearmen
 The results on the sentence similarity task are displayed in Table~\ref{tab:nli_sts}.
%  \nascomment{this reference is wrong, it refers to a section not a table; probably the label command isn't inside the table's caption?}. 
 Due to resource constraints and unreported results by prior work, we report our model only with RoBERTa-base.
 We can observe that \textsc{Autobot} applied to RoBERTa-base significantly outperforms other supervised base transformer methods. Additionally,  \textsc{Autobot} approaches the performance of large  transformers while having a minimal   parameter   overhead of 1.6\%.    % \nascomment{make this quantitative}. \nikos{done}
%  \ivan{finish}
% \nascomment{maybe footnote this:}

We also find that \textsc{Autobot} without any supervision (\textsc{Autobot}-base unsup.) outperforms all of the unsupervised methods, and most notably improves upon average BERT embeddings by 26.1\%. This demonstrates that our approach is effective in both supervised and unsupervised settings.
%  \nikos{here, we should describe the findings for two settings supervised and unsupervised. currently, there is no discussion of the result in the unsupervised setting. }
% \nascomment{this is strange wording; do you mean that previous work didn't report it?}, we report our  model only applied to RoBERTa-base and leave the hyperparameter search required for large models to future work.
 
%  \nascomment{tables are perhaps ordered wrongly?  in the table currently numbered 2, mark which system(s) are ours.}
 


\begin{table}[t]
	\centering 
	\footnotesize
	\renewcommand{\arraystretch}{1.3}
	\begin{tabular}{l | c | c}
		\toprule
		\textbf{Model} & \textbf{Spearman} & \textbf{Parameters} \\ \midrule
		\multicolumn{3}{l}{\textit{Unsupervised}} \\\midrule
% 		\multicolumn{2}{l}{\textit{Trained on NLI (not STS Benchmark)}} \\ \hline
		Avg.\ GloVe embeddings & 58.02 & - \\
		Avg.\ BERT embeddings &  46.35 & - \\
		\textsc{Autobot}-base unsup. & \textbf{58.49} & - \\\midrule
		\multicolumn{3}{l}{\textit{Supervised}} \\\midrule
		InferSent - GloVe &  68.03 & - \\
		Universal Sentence Encoder &  74.92 & - \\
% 		SBERT-base & 77.03 & \\
% 		SBERT-large & 79.23 & \\\hline

        % BERT-base & 74.81 & 110M \\
        RoBERTa-base & 75.37 & 125M\\
% 		SBERT-base & 76.81 & 110M \\
		SRoBERTa-base & 76.89 & 125M \\
% 		AUTOBOT BERT-base & 77.03 & 111M \\
		\textsc{Autobot}-base (ours) & \textbf{78.59} & 127M \\\hline
        % BERT-large & 78.67 & 336M \\
        RoBERTa-large & 80.16 & 355M \\
% 		SBERT-large & 79.23 & 336M \\
% 		SRoBERTa-large &  \textbf{80.32}  & 355M \\
% 		AUTOBOT BERT-large & 77.01 & 338M \\
% 		AUTOBOT RoBERTa-large & 79.93 & 360M \\
		 
% 		AUTOBOT RoBERTa-base ft1 & 77.24 & \\
% 		ft 2 & 76.17 & \\
% 		ft 3 & 76.20 & \\
% 		ft 1 10k & 78.26 & \\
% 		ft 2 10k & 77.03 & \\
% 		ft 3 10k & 77.37 & \\
% 		SBERT-base  &  85.35 &  \\
% 		SRoBERTa-base  & 84.79  & \\
% 		SBERT-large & 86.10 &  \\
% 		SRoBERTa-large & 86.15  &\\\hline 
% 		BARNEY BERT-base &  84.25 &  \\  % 84.31
% 		BARNEY RoBERTa-base &  & \\
% 		BARNEY BERT-large &  &\\
% 		BARNEY RoBERTa-large &  &\\
% 		\multicolumn{2}{l}{\textit{Trained on STS Benchmark}} \\ \hline
% 		BERT-base & 84.30 $\pm$ 0.76  \\
% 		SBERT-base & 84.67 $\pm$ 0.19 \\ 
% 		SRoBERTa-base & 84.92 $\pm$ 0.34 \\
% 		BARNEY RoBERTa-base &  $\pm$  \\
% 		BARNEY BERT-base &  $\pm$  \\ \hline 
		
% 		BERT-large  & 85.64 $\pm$ 0.81 \\ 
% 		SBERT-large & 84.45 $\pm$ 0.43 \\ 
% 		SRoBERTa-large & 85.02 $\pm$ 0.76 \\ 
% 		BARNEY RoBERTa-base &  $\pm$  \\
% 		BARNEY BERT-base &  $\pm$  \\ \midrule
		
% 		\multicolumn{2}{l}{\textit{Trained on NLI + STS benchmark}} \\ \hline
		
% 		BERT-base & 88.33 $\pm$ 0.19 \\ 
% 		SBERT-base & 85.35 $\pm$ 0.17 \\ 
% 		SRoBERTa-base & 84.79 $\pm$ 0.38 \\ 
% 		BARNEY RoBERTa-base &  $\pm$  \\
% 		BARNEY BERT-base &  $\pm$  \\ \hline 
		
% 		BERT-large & 88.77 $\pm$ 0.46 \\ 
% 		BARNEY RoBERTa-base &  $\pm$  \\
% 		BARNEY BERT-base &  $\pm$  \\
    \bottomrule
	\end{tabular}
	\caption{ \label{tab:nli_sts}On semantic textual similarity (STS), \textsc{Autobot} outperforms previous sentence representation methods and reaches a score similar to RoBERTa-large while having fewer parameters.   %The transformer models were finetuned on the natural language inference training set, and 
	We report Spearman's rank correlation on the test set and the model sizes are reported in terms of trained parameter size.}
% 	The test performance of different models finetuned on the NLI training set then evacuated on the STS test set. The model sizes are reported in parameter size for comparison. 

% 	\ivan{SBERT, whose framework we evaluate in using their hyperparameters, doesn't even have a significant improvement in the large model. I suspect this is due to not enough hyperparameter search. Should we keep just RoBERTa-large for the large models to keep our claim?} \ivan{They also actually don't report RoBERTa-large results}
% 	Trained only on NLI, eval on STS 
	
% 	\ivan{Only show this, and rerun these experiments. Might just show RoBERTa results for simplicity} \nikos{fix acronyms here and in other places in the text. btw are these results up-to-date?}
	 % Evaluation on the STS benchmark test set. BERT systems were trained with 10 random seeds and 4 epochs. SBERT was fine-tuned on the STSb dataset, SBERT-NLI was pretrained on the NLI datasets, then fine-tuned on the STSb dataset.
	
\end{table}



% 	\begin{tabular}{l|c}
% 		\toprule
% 		\textbf{Model} & \textbf{Spearman} \\ \midrule
% 		\multicolumn{2}{l}{\textit{Trained on NLI (not STS Benchmark)}} \\ \hline
% 		Avg.\ GloVe embeddings & 58.02\\
% 		Avg.\ BERT embeddings &  46.35\\
% 		InferSent - GloVe &  68.03 \\
% 		Universal Sentence Encoder &  74.92\\
% 		SBERT-base  &  77.03\\
% 		SBERT-large & 79.23 \\
% 		BARNEY BERT-base & \\
% 		BARNEY BERT-large & \\ \midrule
% 		\multicolumn{2}{l}{\textit{Trained on STS Benchmark}} \\ \hline
% 		BERT-base & 84.30 $\pm$ 0.76  \\
% 		SBERT-base & 84.67 $\pm$ 0.19 \\ 
% 		SRoBERTa-base & \textbf{84.92} $\pm$ 0.34 \\
% 		BARNEY RoBERTa-base &  $\pm$  \\
% 		BARNEY BERT-base &  $\pm$  \\ \hline 
		
% 		BERT-large  & \textbf{85.64} $\pm$ 0.81 \\ 
% 		SBERT-large & 84.45 $\pm$ 0.43 \\ 
% 		SRoBERTa-large & 85.02 $\pm$ 0.76 \\ 
% 		BARNEY RoBERTa-base &  $\pm$  \\
% 		BARNEY BERT-base &  $\pm$  \\ \midrule
		
% 		\multicolumn{2}{l}{\textit{Trained on NLI + STS benchmark}} \\ \hline
		
% 		BERT-base & \textbf{88.33} $\pm$ 0.19 \\ 
% 		SBERT-base & 85.35 $\pm$ 0.17 \\ 
% 		SRoBERTa-base & 84.79 $\pm$ 0.38 \\ 
% 		BARNEY RoBERTa-base &  $\pm$  \\
% 		BARNEY BERT-base &  $\pm$  \\ \hline 
		
% 		BERT-large & \textbf{88.77} $\pm$ 0.46 \\ 
% 		BARNEY RoBERTa-base &  $\pm$  \\
% 		BARNEY BERT-base &  $\pm$  \\ \bottomrule
% 	\end{tabular}






% % 		\multicolumn{2}{l}{\textit{Trained on NLI (not STS Benchmark)}} \\ \hline
% 		Avg.\ GloVe embeddings & 58.02 & - \\
% 		Avg.\ BERT embeddings &  46.35 & - \\
% 		InferSent - GloVe &  68.03 & - \\
% 		Universal Sentence Encoder &  74.92 & - \\\hline
% % 		SBERT-base & 77.03 & \\
% % 		SBERT-large & 79.23 & \\\hline

%         BERT-base & 74.81 & 110M \\
%         RoBERTa-base & 75.37 & 125M\\
%         BERT-large & & 336M \\
%         RoBERTa-large & & 355M \\\hline
		
		
% 		SBERT-base & 76.81 & 110M \\
% 		SRoBERTa-base & 76.89 & 125M \\
% 		SBERT-large & 79.23 & 336M \\
% 		SRoBERTa-large &   & 355M \\\hline
		
% 		AUTOBOT BERT-base & 77.03 & \\
% 		AUTOBOT RoBERTa-base & 78.59 & \\
% % 		AUTOBOT RoBERTa-base ft1 & 77.24 & \\
% % 		ft 2 & 76.17 & \\
% % 		ft 3 & 76.20 & \\
% % 		ft 1 10k & 78.26 & \\
% % 		ft 2 10k & 77.03 & \\
% % 		ft 3 10k & 77.37 & \\
% 		AUTOBOT BERT-large & \\
% 		AUTOBOT RoBERTa-large & \\
% % 		SBERT-base  &  85.35 &  \\
% % 		SRoBERTa-base  & 84.79  & \\
% % 		SBERT-large & 86.10 &  \\
% % 		SRoBERTa-large & 86.15  &\\\hline 
% % 		BARNEY BERT-base &  84.25 &  \\  % 84.31
% % 		BARNEY RoBERTa-base &  & \\
% % 		BARNEY BERT-large &  &\\
% % 		BARNEY RoBERTa-large &  &\\
% % 		\multicolumn{2}{l}{\textit{Trained on STS Benchmark}} \\ \hline
% % 		BERT-base & 84.30 $\pm$ 0.76  \\
% % 		SBERT-base & 84.67 $\pm$ 0.19 \\ 
% % 		SRoBERTa-base & 84.92 $\pm$ 0.34 \\
% % 		BARNEY RoBERTa-base &  $\pm$  \\
% % 		BARNEY BERT-base &  $\pm$  \\ \hline 
		
% % 		BERT-large  & 85.64 $\pm$ 0.81 \\ 
% % 		SBERT-large & 84.45 $\pm$ 0.43 \\ 
% % 		SRoBERTa-large & 85.02 $\pm$ 0.76 \\ 
% % 		BARNEY RoBERTa-base &  $\pm$  \\
% % 		BARNEY BERT-base &  $\pm$  \\ \midrule
		
% % 		\multicolumn{2}{l}{\textit{Trained on NLI + STS benchmark}} \\ \hline
		
% % 		BERT-base & 88.33 $\pm$ 0.19 \\ 
% % 		SBERT-base & 85.35 $\pm$ 0.17 \\ 
% % 		SRoBERTa-base & 84.79 $\pm$ 0.38 \\ 
% % 		BARNEY RoBERTa-base &  $\pm$  \\
% % 		BARNEY BERT-base &  $\pm$  \\ \hline 
		
% % 		BERT-large & 88.77 $\pm$ 0.46 \\ 
% % 		BARNEY RoBERTa-base &  $\pm$  \\
% % 		BARNEY BERT-base &  $\pm$  \\   \begin{table}[t]
	\centering 
	\footnotesize
	\begin{tabular}{l|c}
		\toprule
		\textbf{Pooling} & \textbf{Spearman} \\ \midrule
% 		\multicolumn{3}{|l|}{\textit{Pooling Strategy}} \\ \hline
		\texttt{MEAN} & 80.78 \\
		\texttt{MAX} & 78.76 \\
		\texttt{CLS} &  79.67 \\
		$\beta$ \text{ (ours)} & \textbf{81.88} \\
		\bottomrule
% 	    \textbf{Pooling} & \textbf{NLI} & \textbf{STSb} \\ \midrule
% % 		\multicolumn{3}{|l|}{\textit{Pooling Strategy}} \\ \hline
% 		\texttt{MEAN} & 80.78   & 87.44 \\
% 		\texttt{MAX} & 79.07 & 69.92 \\
% 		\texttt{CLS} & 79.80 & 86.62  \\
% 		\texttt{CAB}, Fixed (ours) & \textbf{81.88} & \\
% 		\bottomrule
% 		\hline
% 		\multicolumn{3}{|l|}{\textit{Concatenation}} \\ \hline
% 		$(u, v)$ & 66.04 & -\\
% 		$(|u-v|)$ & 69.78 & - \\
% 		$(u*v)$ & 70.54 & -\\
% 		$(|u-v|, u*v)$ & 78.37  & -\\
% 		$(u, v, u*v)$ & 77.44 & -  \\
% 		$(u, v, |u-v|)$ &  \textbf{80.78} & - \\
% 		$(u, v, |u-v|, u*v)$ & 80.44 & - \\ 
% 		\hline	
	\end{tabular}
	\caption{Performance of sentence representations from RoBERTa trained with different pooling methods on NLI data and then evaluated on STS benchmark's development set %(STSb)
	in terms of Spearman's rank correlation.}
	\label{tab:pooling}
\end{table}

% DEV RESULTS TO ADD
% MEAN  0.8092
% MAX   0.7876
% CLS   0.7967  0.7999
% SB    
% 




% 		\texttt{MEAN} & 80.78 \\
% 		\texttt{MAX} & 79.07 \\
% 		\texttt{CLS} & 79.80 \\
% 		\texttt{SB} (ours) & \textbf{81.88} \\
 
% We directly compare to the results of \citet{Reimers2019SentenceBERT} as well as 
% }\ivan{Finish
% We directly compare against finetuning the original pretrained versions of BERT and RoBERTa, as well as \citet{Reimers2019SentenceBERT}'s 
% \citet{Reimers2019SentenceBERT} perform the methodology of  by using simple pooling methods of BERT representations, such as mean, max, and cls, and evaluate their performance on sentence similarity task. We train and evaluate \textsc{Autobots} with a similar setup and its unique context attention bottleneck to compare against their results.
% \citet{conneau2017supervised} show that one can obtain universal sentence representations by finetuning on natural langauge inference data. \citet{Reimers2019SentenceBERT} show how this methodology can be applied to pooled BERT representations to obtain such sentence representations. We compaire
We find in Table~\ref{tab:pooling} that using the proposed sentence bottleneck based on learned context  provides 
% \nascomment{do not use this word if you're not doing hypothesis tests} significant 
noticeable gains over using simpler pooling methods from prior work. We suspect this is due to the additional flexibility provided by our bottleneck acting as ``weighted pooling'' by attending over all tokens to compute the final representation, as opposed to equal contribution of all tokens regardless of the input. 




% We also find it maintains same relative performance in the large transformer models. Due to resource constraints, we use SBERT's defaults leave further hyperparameter optimization of the large transformer models to future work.

% we compare directly the performance of \textsc{Autobots} to other models on the sentence similarity task, ones which have not been trained on STS data. We find that \textsc{Autobots} ends up performing singificantly better, and ends up achieving SBERT-large level performance with significantly less parameters.


\subsection{Sentence Classification} \label{sec:clf}

The results on single-sentence classification tasks and other tasks from the GLUE benchmark are displayed in Table \ref{tab:glue}. We find that \textsc{Autobot} provides a 
% \nascomment{do not use this word if there's no test} significant
noticable performance increase on single-sentence tasks, specifically on the CoLA datasets when using both the RoBERTa-base and RoBERTa-large models.
% This demonstrates that our model is effective regardless \nascomment{too strong; we only tested two model sizes!} of the pretrained language model's size used.
Additionally, we also find that \textsc{Autobot}, when fed both sentences concatenated for dual sentence GLUE tasks, maintains the original performance of the underlying pretrained encoder. 
\begin{table}[t]
\centering
\footnotesize
\renewcommand{\arraystretch}{1.3}
\begin{tabular}{l | c | c | c}
\toprule
\textbf{Model} & \bf SST & \bf CoLA & \bf Others (avg) \\
\midrule 
% \multicolumn{10}{l}{\textit{Transformer Models}}\\
% Base &  &  &  &  &  &  &  &  &  \\
% \quad + Masked &  &  &  &  &  &  &  &  &  \\
% \quad + MLM &  &  &  &  &  &  &  &  &  \\
% Large &  &  &  &  &  &  &  &  &  \\
% \quad + Masked &  &  &  &  &  &  &  &  &  \\
% \quad + MLM &  &  &  &  &  &  &  &  &  \\
% \midrule 
% 84.4 88.4 86.7 92.7
% $\text{BERT}$  & 84.3 & 88.4 &  &  & 92.7 & 86.7 &  &  &  \\
% $\text{BARNEY}_\text{BERT}$ &  &  &  &  &  &  &  &  &  \\
RoBERTa-base & 94.8 & 63.6 & 88.7 \\
\textsc{Autobot}-base & \textbf{95.0} & \textbf{66.0} & 88.7 \\
\midrule 
% $\text{BERT}$ & 86.6 & 92.3 & 91.3 & 70.4 & 93.2 & 88.0 & 60.6 & 90.0 &  -\\
% $\text{BARNEY}_\text{BERT}$ &  &  &  &  &  &  &  &  &  \\
RoBERTa-large & 96.4 & 68.0 & 91.1 \\
\textsc{Autobot}-large & \textbf{96.9} & \textbf{70.2} & 91.1 \\
\bottomrule
\end{tabular} % 87.6 92.8 91.9 78.7 94.8 90.2 63.6 91.2
\caption{%\nikos{can we fit the other tasks here too? if available (otherwise in appendix)} \ivan{I could add a col. for avg. of other tasks} \nikos{great idea! we should definitely do that} 
%\nikos{here we are not comparing with SRoberta?} \ivan{Hmmm, SRoBERTa was only intended for sentence representations and never evaluated for classification, so I didn't do that experiment}
Single-sentence GLUE classification dev.~results. Median accuracy is reported over over three random seeds. Our model improves performance on single-sentence classification tasks over both base and large RoBERTa models while maintaining their performance on the remaining multi-sentence tasks. % \ivan{Explain why we exclude other tasks}
% We exclude other GLUE task metrics as AUTOBOT achieves an identical performance as the underlying transformer in the encoder on dual sentence tasks.
% \ivan{Perhaps reduce this table to only show the single-sentence results? (MNLI, SST, CoLA) And note results unchanged for others} \nikos{good idea, let's keep the results for other non sentence similarity tasks to support the point that performance is largely maintained on the other tasks (unclear if this is the case with sentence bert).}
}


% Averages are obtained from the GLUE leaderboard.
\label{tab:glue}
\end{table}

% take the median, cite jesse paper



% \begin{table*}[ht]
% \centering
% \begin{tabular}{lcccccccccc}
% \toprule
% & \bf MNLI & \bf QNLI & \bf QQP & \bf RTE & \bf SST & \bf MRPC & \bf CoLA & \bf STS & \bf Avg \\
% \midrule 
% % \multicolumn{10}{l}{\textit{Transformer Models}}\\
% % Base &  &  &  &  &  &  &  &  &  \\
% % \quad + Masked &  &  &  &  &  &  &  &  &  \\
% % \quad + MLM &  &  &  &  &  &  &  &  &  \\
% % Large &  &  &  &  &  &  &  &  &  \\
% % \quad + Masked &  &  &  &  &  &  &  &  &  \\
% % \quad + MLM &  &  &  &  &  &  &  &  &  \\
% % \midrule 
% \multicolumn{10}{l}{\textit{Base Models}}\\
% % 84.4 88.4 86.7 92.7
% % $\text{BERT}$  & 84.3 & 88.4 &  &  & 92.7 & 86.7 &  &  &  \\
% % $\text{BARNEY}_\text{BERT}$ &  &  &  &  &  &  &  &  &  \\
% $\text{RoBERTa-base}$ & 87.6 & 92.8 & 91.9 & 78.7 & 94.8 & 90.2 & 63.6 & 91.2 & \\
% $\text{BARNEY}_{\text{RoBERTA-base}}$ & \textbf{88.0} &  & 91.9 &  & \textbf{95.0} & & \textbf{66.0} & &   \\
% \midrule 
% \multicolumn{10}{l}{\textit{Large Models}}\\
% % $\text{BERT}$ & 86.6 & 92.3 & 91.3 & 70.4 & 93.2 & 88.0 & 60.6 & 90.0 &  -\\
% % $\text{BARNEY}_\text{BERT}$ &  &  &  &  &  &  &  &  &  \\
% $\text{RoBERTa-large}$ & 90.2 & 94.7 & 92.2 & 86.6 & 96.4 & 90.9 & 68.0 & 92.4 \\
% $\text{BARNEY}_\text{RoBERTA-large}$ & 90.5 &  &  &  & 96.9 & & \textbf{70.2} &  &  \\
% \bottomrule
% \end{tabular} % 87.6 92.8 91.9 78.7 94.8 90.2 63.6 91.2
% \caption{
% Dev results on GLUE.
% % Results on GLUE. All results are based on a 24-layer architecture.
% % \bertlarge{} and \xlnetlarge{} results are from \newcite{devlin2018bert} and \newcite{yang2019xlnet}, respectively.
% % \ourmodel{} results on the development set are a median over five runs.
% % \ourmodel{} results on the test set are ensembles of \emph{single-task} models.
% \ivan{Perhaps reduce this table to only show the single-sentence results? (MNLI, SST, CoLA) And note results unchanged for others} \nikos{good idea, let's keep the results for other non sentence similarity tasks to support the point that performance is largely maintained on the other tasks (unclear if this is the case with sentence bert).}
% For RTE, STS and MRPC we finetune starting from the MNLI model instead of the baseline pretrained model.}


% % Averages are obtained from the GLUE leaderboard.
% \label{tab:roberta_glue}
% \end{table*} Hence, our model improves the quality of the sentence representations from pretrained transformer models without hurting their performance.  

%The GLUE benchmark is a collection of diverse natural language understanding text classification, and involves both single and dual sentence tasks. 
%We fully finetune \textsc{autobots} single sentence tasks and compare performance with \citet{liu2019RoBERTa} in Table~\ref{tab:glue}. \nikos{moved them above}

% We evaluate \textsc{Autobots} pretrained with the Denoising, Fixed methodology report its performance on the dev set of each task in Table 1. \\
% We find that on multiple tasks \textsc{Autobots} on the base models end up performing on par with the large models with a mere fraction of additional parameters and compute. 



% \ivan{describe finetuning: fully finetuned autobots}
% \ivan{Should we include results where the final layers of the pretrained model are trainable? It would make figure 2 a bit more crowded}\nikos{let's add them in the supplementary section.}

% \ivan{Since these are classification tasks, I feel like BARNEY would have a good chance at beating BERT as a baseline. These classification task require backpropagation through BARNEY, in addition to the linear classification layer, in the finetuning process.}
% \nikos{We should definitely try this. I am curious to see how the autoencoder pretraining objective impacts the results. } 



% \subsection{SentEval}
% SentEval [cite] is an evaluation framework for fixed sentence embeddings on 17 downstream tasks. We follow a setup similar to \cite{Reimers2019SentenceBERT} by finetuning on a combination of Natural Language Inference and Semantic Textual Similarity training sets, then finetuning a linear regression head on the fixed representations for each task. We examine the perfomrance of both BARNEY and BERT + CAB.
% % \ivan{These are 17 downstream tasks that \textit{don't} backpropagate through the sentence encoder, but rather, evaluate the fixed-length sentence embeddings themselves. \url{https://github.com/facebookresearch/SentEval}. If we include BERT's [CLS] token in this framework, we definitely have a good shot here}
% \nikos{I don't believe that this evaluation will measure the full potential of the method since it's best on fixed embeddings. By the way some of these tasks overlap with GLUE where we plan to test anyway with finetuning (e.g.SST/MRPC). So, I'd rather suggest to look for some controlled generation task such as sentiment style transfer (e.g. like the one used by Shen et al 2020) to demonstrate the benefits for a real  generation task (other than the reconstruction tests below).}
% % \ivan{Gotcha. In SentenceBERT they test the fixed embeddings after finetuning to the NLI dataset, which I feel like a good table to include would be an extra row for our method on table 5: \url{https://arxiv.org/pdf/1908.10084.pdf}}


% \subsection{BARNEY Pretraining for Generation}
% We outline several methods to pretrain BARNEY, and perform experiments on the sentence similarity task (SST2) with RoBERTa-base BARNEY, to observe the relative performance on single-sentence classification.





% \subsection{Multilingual}

% \subsection{Pretraining}
% \begin{itemize}
%     \item Train two models on the exact same data for the exact same amount of training steps. To simulate the same amount of parameters, use one extra layer for the MLM approach
%     \item Model 1: 7 layer, 512 hidden size transformer encoder trained on just the MLM objective
%     \item Model 2: 6 layer, 512 hidden size BARNEY trained on MLM objective in conjunction with reconstruction objective.
%     \item Show the down-stream MNLI performance difference after certain amounts of steps (could be a plot)
%     \item Show the downstream SQuAD (MNLI?) performance difference at the end of BARNEY wihtout the conetext attention bottleneck (see if the reconstruction objective helps with better token-level representations, since all tokens are updated each step, rather than only 15\% of them in MLM)
% \end{itemize}

\subsection{Sentence Generation} \label{sec:gen}
% sentiment style transfer (e.g. like the one used by Shen et al 2020)
%\citet{mikolov2013linguistic} previously discovered that embeddings from unsupervised learning can capture linguistic relationships via simple arithmetic. \nikos{let's keep this for the longer version?}

% A canonical example is the embedding arithmetic “King” - “Man” + “Woman” $≈$ “Queen”.

%To evaluate latent space properties of autoencoders, \citet{shen2019educating} propose evaluating unsupervised style transfer through vector arithmetic computing a “sentiment vector” $v$ by taking the vector difference between 100 negative and positive sentences, then evaluating by taking an input sentence, encoding it, adding a multiple of the sentiment vector to the sentence representation, then decoding the resulting representation. \nikos{Moving it up}

% \nascomment{missing number in lower right of table 3 is not explained}

For sentence generation, we focus on the sentiment transfer task proposed by \citet{shen2019educating} both with and without further training on in-domain data from Yelp. When finetuning, we perform an additional 10K optimization steps using the Yelp dataset.  Note that all the baselines require training on in-domain data, while this is optional for our model.  In Figure~\ref{fig:generation}
% \nascomment{reference goes to a section not a figure}
we find that the \textsc{Autobot} model not exposed to the Yelp dataset during finetuning performed on par with the 
% \nascomment{DAE is not explained, either in the text or the figure -- reader doesn't know what it is}
DAE that was trained specifically on Yelp. Additionally, \textsc{Autobot} outperforms the DAE in the above-40 percent accuracy range when finetuned on in-domain data.
% \nascomment{what follows is a bit too vague.  either explain more fully or cut} Lastly, we additionally find that with partial finetuning of the pretrained transformer encoder \textsc{Autobot} achieves higher BLEU for a given accuracy, which we defer to the appendix.
We include \textsc{autobot} results with partial finetuning of the encoder in the appendix, which we find considerably improves the Self-BLEU metric.
% \nascomment{there's no supplementary ... I think we referred to it as appendix before?}.
Since \textsc{AUTOBOT} uses vector arithmetic, inference is as fast as the DAE and over twice that of other methods.
% \nikos{say something about speed compared to previous methods}
% and compare directly with We compare directly with \citet{shen2019educating} and their denoising autoencoder (DAE) trained on Yelp, and include more sophisiticated methods for style transfer such as fast gradient iterative modification (FGIM) of \citet{wang2019controllable} and Emb2Emb of \citet{mai2020plug} for reference \citet{shen2019educating} and their denoising autoencoder (DAE) trained on Yelp. We evaluate \textsc{autobots}
% wang2019controllable

% , and explore the effect of making several last layers of the pretrained transformer trainable during autoencoding. 

% To evaluate properties of the latent space of \textsc{Autobots}, we perform the experiment of \citet{shen2019educating} were we compute a “sentiment vector” $v$ from 100 negative and positive sentences, and change the sentiment of a sentence by encoding it, adding a multiple of the sentiment vector to the sentence representation, then decoding the resulting representation.

% We plot the sentiment classification accuracy versus self-BLEU curve as we vary the multiple of the sentiment vector added used in .


% FGIM \citep{wang2019controllable} fast gradient iterative modification


% We also observe that when more of the final layers of the pretrained encoder are allowed to be finetuned with the autoencoding objective, \textsc{autobots} obtain higher Self-BLEU in the sub-90 percent accuracy range.

% \ivan{describe finetuning: fixed autobots evaluated}
% and use it to change the sentiment of the test sentences.

% %% !TEX root=econ_dispatch.tex
In this section, we first propose a reliable static renewable power scenario generation method in each time interval $1,\dots,T$. Then we present an efficient dynamic renewable power scenario generation method for the entire time horizon.

\subsection {Static Scenario Generation}

By the joint distribution of multiple RPPs in \eqref{cjdistribution}, scenarios can be generated to represent the uncertainties and spatial correlation of all RPPs in the system. However, with the increase of the number of RPPs, classical random sampling methods such as inverse transform sampling and Latin hypercube sampling \cite{L_sampling} become hard to be employed due to matrix size and computational limitations. Other classical sampling methods such as rejection sampling tend to have very large rejection rate for a high number of dimensions.

To this end, a reliable static renewable power scenario generation method based on Gibbs sampling \cite{Gibbs} is proposed to sample for the conditional joint distribution function of actual available power of RPPs in \eqref{cjdistribution}. Compared with directly sampling by the conditional joint distribution \cite{copula_Zhang}, Gibbs sampling converts the sampling process of joint distribution in \eqref{cjdistribution} to $J+K$ sampling processes of conditional distribution in \eqref{ccdistribution}. Namely, let $U$ be a random variable generated uniformly within $[0,1]$, then each RPP can be sampled via the inverse transform:
\begin{equation} \label{inversesampling}
w_{a,j}=F_{a,j}^{-1}(U),\quad s_{a,k}=F_{a,k}^{-1}(U)
\end{equation}
where $F_{a,j}^{-1}$ and $F_{a,k}^{-1}$ is the inverse function of $F_{a,j}$ and $F_{a,k}$, respectively.

Gibbs sampling needs a burn-in process \cite{burn_in} before it converges to the true distribution in \eqref{cjdistribution}. So we throw out $N_{b}$ (e.g. 1000) samples in the beginning the process. The detailed procedure of static scenarios generation is:
\begin{enumerate}%[noitemsep,nolistsep]
	\item Setting the number of renewable power scenarios: $N_{sc}$ (e.g. 5000), the total number of samples is $N_{sc}+N_{b}$.
	\item Setting the initial sampling values to be the forecasted power for each RPP.
	% $w_{a,{1}}^{i}$,...,$w_{a,j}^{i}$,..., $w_{a,J}^{i}$, $s_{a,{\it 1}}^{i}$,...,$s_{a,k}^{i}$,...,$s_{a,K}^{i}$, {\it i}=0...$N_{sc}+N_{b}$, ({\it i}=0 at this step). To  speed up the burn-in process, the forecast power of each RPP (i.e. $F_{re}$) are regarded as the initial sampling value.
	\item Employing inverse transform sampling in \eqref{inversesampling} in a round robin fashion for each scenario generation step (indexed by $i$):

\begin{itemize}
	\item $f(w_{a,{1}}^{i}|w_{a,2}^{i}...w_{a,J}^{i},s_{a,{1}}^{i}...s_{a,K}^{i},\mathbf{f})$
	\item $f(w_{a,{\it j}}^{i}|w_{a,{1}}^{i+1}...w_{a,{{\it j}-1}}^{i+1},w_{a,{{\it j}+1}}^{i}...w_{a,J}^{i},s_{a,{1}}^{i}...s_{a,K}^{i},\mathbf{f})$
	\item $...$
	\item $f(s_{a,{\it k}}^{i}|w_{a,{1}}^{i+1}...w_{a,J}^{i+1},s_{a,{1}}^{i+1}...s_{a,{{\it k}-1}}^{i+1},s_{a,{{\it k}+1}}^{i}...s_{a,K}^{i},\mathbf{f})$
	\item $f(s_{a,{\it K}}^{i}|w_{a,{1}}^{i+1}...w_{a,J}^{i+1},s_{a,{1}}^{i+1}...s_{a,{{\it K}-1}}^{i+1},\mathbf{f})$
\end{itemize}

	\item Repeating 3 from {\it i}=1...$N_{sc}+N_{b}$. Disregard the first $N_{b}$ scenarios and we get $N_{sc}$ renewable power scenarios.

\end{enumerate}

{An important feature of the proposed static scenario generation method is that with the increase of the number of RPPs, the computational space complexity remains same and the computational time complexity increases linearly, effectively mitigating the curse of dimensionality.}

\subsection {Dynamic Scenario Generation}
%\todo{Why is this dynamic? Also, does variability just mean correlation?}
{A dynamic scenario is a scenario that considers the variability (i.e., temporal correlation) of the output of a RPP.} The method presented in the last section can generate renewable power scenarios of conditional joint distribution (c.f. \eqref{cjdistribution}) which captures the marginal uncertainties and spatial correlation. In this section we extend it to capture the temporal correlation among the time points in a scenario, which is also of vital importance in power system operations~\cite{sce_generation_Ma,PCA,sce_generation_Pinson}.
 % which represent the uncertainties and correlations in each time interval \todo{(i.e., spatial correlation)}. However, for renewable power scenarios, variability is as same importance as uncertainties \cite{sce_generation_Ma}\cite{PCA}\cite{sce_generation_Pinson}.

To capture the variability, some new variables are introduced. Take a WPP for instance, a new random variable $Z_{a,j}^{t}$ is introduced which follows
the standard Gaussian distribution with zero mean and unit standard deviation. Since the value of CDF of $Z_{a,j}^{t}$ is uniformly distributed over [0,1], the uniform distribution $U$ in \eqref{inversesampling} can be replaced by a CDF $\Phi(Z_{a,j}^{t})$.  Given the realization of random variable $Z_{a,j}^{t}$, $w_{a,j}^{t}$ can be sampled as follows:



\begin{equation} \label{transform}
\begin{aligned}
w_{a,j}^t=F_{a,j}^{-1}(\Phi(Z_{a,j}^{t}))
\end{aligned}
\end{equation}

To consider the variability of each RPP, it is assumed that the joint distribution of $Z_{a,j}^{t}$ follows a multivariate Gaussian distribution $Z_{a,j}^{t} \sim N(\mu_{j},\Sigma_{j})$. The expectation of $\mu_{j}$ is a vector of zeros and the covariance matrix $\Sigma_{j}$ satisfies


\begin{equation} \label{matrix}
\Sigma_j=\left[
\begin{matrix}
\sigma_{1,1}^{j}&\sigma_{1,2}^{j}&\dots&\sigma_{1,{\it T}}^{j}&\\
\sigma_{2,1}^{j}&\sigma_{2,2}^{j}&\dots&\sigma_{2,{\it T}}^{j}&\\
\vdots&\vdots&\ddots&\vdots&\\
\sigma_{{\it T},1}^{j}&\sigma_{{\it T},2}^{j}&\dots&\sigma_{{\it T},{\it T}}^{j}&\\
\end{matrix}
\right]
\end{equation}

\noindent where $\sigma_{m,n}^{j}=cov(Z_{a,j}^{m},Z_{a,j}^{n})$, {\it m}, {\it n}=1,2...{\it T}, $\sigma_{{\it m}, {\it n}}^{j}$ is the covariance of $Z_{a,j}^{m}$ and $Z_{a,j}^{n}$.

The covariance structure of $\Sigma_j$ can be identified by covariance $\sigma_{m,n}^{j}$. As is done in \cite{sce_generation_Ma}\cite{sce_generation_Pinson}, an exponential covariance function is employed to model $\sigma_{m,n}^{j}$ in \eqref{matrix},

\begin{equation} \label{exponential}
\begin{aligned}
\sigma_{m,n}^{j}=\rm exp(-\frac{|{\it m}-{\it n}|}{\epsilon_{\it j}}) \quad 0 \le {\it m},  {\it n} \le {\it T}
\end{aligned}
\end{equation}

\noindent where $\epsilon_{\it j}$ is the range parameter controlling the strength of the
correlation of random variables $Z_{a,j}^{t}$ among the set of lead-time. Similar to \cite{sce_generation_Ma}, $\epsilon_{\it j}$ can be determined by comparing the distribution of renewable power variability of the generated scenarios by the indicator in \cite{sce_generation_Ma}. Here, assuming that the  range parameter $\epsilon_{\it j}$ of each RPP have been obtained, the flowchart of dynamic renewable power scenario generation method is as shown in Fig.~\ref{flowchart}.

\begin{figure}[!htb]
	\begin{center}
		\includegraphics[trim = 10 250 60 200, clip, width=1.0\columnwidth]{flowchart.eps}\\
		\caption{Flowchart of dynamic renewable power scenario generation method}\label{flowchart}
	\end{center}
\end{figure}

Before generating $N_{sc}$ scenarios, small amount of scenarios are generated to obtain the range parameter of each RPP. After all the range parameters in \eqref{matrix} are obtained, we can start the dynamic wind power scenarios generation in Fig.~\ref{flowchart}. At each time interval, they follow the conditional joint distribution in \eqref{cjdistribution} and among the time horizon, the variability is considered.

One thing that need to be noticed is that each static scenario generation process in Fig. 1 does not affect each other after the random data set is determined. Parallel computing can be employed to increase the computation efficiency to meet the real-time requirement.

In scenario-based method, the above generated scenarios should be reduced to certain number of scenarios that deemed as the most probability occur. A scenario reduction method in \cite{YishenWang} is employed in this paper for the reason that it has great efficiency compared with other methods to meet the real-time requirement.


% \subsection{Reconstruction Quality}
% \ivan{Perhaps we should evaluate the reconstruction ability compared to other autoencoders? We could focus on just BooksCorpus, create our own test set, and test reconstruction quality. If TAE/BARNEY is really good, we could introduce EM as a metric}
% \nikos{ What do you mean by EM? I worry that we may not have the space for introducing a new evaluation too (we have new architecture + pretraining framework already). How about using the Yelp dataset to compare directly with Shen et al 2020 in their own setup? }


% \subsection{Latent Properties}
% \ivan{We could also show an example of encoding two sentences, then showing the decoding of the linear interpolation of between the two? Perhaps a 2D dim-red of the embeddings?} \nikos{Sure, that'd be great. E.g. like the ones here \url{https://arxiv.org/pdf/1511.06349.pdf}}


\vspace{-3mm}
\begin{figure}[ht]
\centering
\hspace{-2mm}\includegraphics[width=0.49\textwidth]{figures/generation_more_points.png}
\caption{ 
% \nikos{it'd be nice also to add some points from prior work (pnp paper models)?} \ivan{I'll try finding and adding some} 
% \nikos{great!} \nikos{the fontsize on legends, xticks, yticks, xlabel, ylabel still look quite small. }
% \nikos{could we remove two of the autobot models to make the diagram more clear? it'd be nice also to add some points from prior work (pnp paper models)? no need to have a line for these others. could you me the markers equal size and a bit bigger? I'd use a solid or dotted line without markers for the baseline.}
% \nikos{please also make font for the axes and their titles bigger. I'd also remove the top title from the figure, caption is sufficient and you save up space.}
Automatic evaluations of vector arithmetic for sentiment transfer, plotted as accuracy vs. self-BLEU. Accuracy (ACC) is measured by a sentiment classifier, and values for varying multiples of the sentiment vector are plotted. Upper right is better.
% \ivan{We also have examples where the last 3 layers of the pretrained transformer were allowed to be finetuned} \nikos{these results are without finetuning? yeah if it is very crowded we should include the ones with finetuning in the supplementary.  }\nikos{can we increase the font sizes a bit more and the thickness of the lines? I'd use dashes instead of dots for the baseline with thicker lines because it looks more distinctive.}
% \nikos{add efficiency column that compares speed, add more tables from pnp autoencoder paper. remove ppl if the results are not calculated with the same vocab}
\label{fig:generation}
}
\end{figure}