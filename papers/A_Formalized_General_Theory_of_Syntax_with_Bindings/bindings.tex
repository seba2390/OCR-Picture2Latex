\documentclass{llncs}
\usepackage{amsmath,amssymb}
\usepackage{stmaryrd}
%\usepackage{mathpartir}
\usepackage{tikz}
\usetikzlibrary{shapes,calc}
\usepackage{url}
\usepackage{multirow}
\usepackage{multicol}
\usepackage{rotating}
%\usepackage{mathrel}
\usepackage{listings}
\usepackage{lmodern} % Set all fonts to Latin Modern
\usepackage{mathptmx} % Reset text and math fonts to Times (load AFTER lmodern)
\usepackage[utf8]{inputenc}
\usepackage[T1]{fontenc}
\DeclareSymbolFont{letters}{OML}{txmi}{m}{it}
\DeclareMathAlphabet{\mathcal}{OMS}{cmsy}{m}{n}
\usepackage{bussproofs}
\usepackage{hyphenat}
\usepackage{enumitem} 
\usepackage{fancyvrb}
\usepackage{listings}

%\addtolength{\textheight}{1cm}
  


%Display style with less space:

\makeatletter
\g@addto@macro \normalsize {%
 \setlength\abovedisplayskip{2.2pt}%
 \setlength\belowdisplayskip{2.2pt}%
}


% Itemization: 

\newenvironment{myitem}[1][]
  {\itemize[leftmargin=*,topsep=0.3ex,itemsep=1pt, #1]}
  {\enditemize}

\newenvironment{myitemd}[1][]
  {\itemize[leftmargin=*,topsep=0ex,itemsep=-5pt, #1]}
  {\enditemize}
\makeatother

%Space before and after sections
%% Save the class definition of \subparagraph
\let\llncssubparagraph\subparagraph
%% Provide a definition to \subparagraph to keep titlesec happy
\let\subparagraph\paragraph
%% Load titlesec
\usepackage[compact]{titlesec}
%% Revert \subparagraph to the llncs definition
\let\subparagraph\llncssubparagraph
\usepackage{titlesec}
\titlespacing*{\section}{0.5pt}{1.1\baselineskip}{0.3\baselineskip}
\titlespacing*{\subsection}{0.5pt}{0.65\baselineskip}{0.25\baselineskip}

\usepackage{color}

\definecolor{light-gray}{gray}{0.85}
%\newcommand\coll[1]{{\colorbox{light-gray}{$#1$}}}
\newcommand\coll[1]{\mbox{\colorbox{light-gray}{$\!#1\!$}}}

\let\proof\relax 
\let\endproof\relax

\usepackage{amsthm} 
%\usepackage{ntheorem}

\pagestyle{plain}

\usepackage{varioref}

%\usepackage{hyperref}

\usepackage[
   a4paper,
   pdftex,  
   pdfkeywords={},
   pdfborder={0 0 0},
   draft=false,
   bookmarksnumbered,
   bookmarks,
   bookmarksdepth=2,
   bookmarksopenlevel=2,
   bookmarksopen]{hyperref}

\usepackage{cleveref}
\usepackage{cite}
\usepackage{calc}




%\newcommand{\comment}[1]{}



%Theorems and such
\newtheorem{thm}[lemma]{Theorem}
\newtheorem{prop}[lemma]{Prop}
\newtheorem{conv}[lemma]{Convention}
%[chapter]
%\newtheorem{remark}[theorem]{Remark}
%\newtheorem{pproposition}{Prop$^*$}
%\newtheorem{proposition}[theorem]{Prop}
%\newtheorem{corollary}[theorem]{Corollary}
%\newtheorem{definition}[theorem]{Definition}
\newtheorem{ass}[lemma]{Assumption}
%\newtheorem{example}[theorem]{Example}
%\newenvironment{proof}{{\noindent\it Proof:\ }}

\newcommand\leftOut[1]{}

%General mathematics

\newcommand{\pfun}{\nrightarrow}

\newcommand{\Hand}{\mbox{\large $\&$}}  %Horn ``and"

\newcommand{\bT}[1]{\framebox[1.1\width]{#1}}  %bT means ``box this"
%\newcommand{\qed}{$\;\;\Box$}
\newcommand{\<}{\langle}
\renewcommand{\>}{\rangle}
\newcommand{\sm}{\setminus}
\newcommand{\btw}{\overline}
\newcommand{\ov}{\overline}
\newcommand{\tup}{\overline}
\newcommand{\eps}{\epsilon}
\newcommand{\con}{\mbox{$\,\copyright\;$}}
\newcommand{\ds}{\displaystyle}
\renewcommand{\vec}[1]{\overline{#1}}
\renewcommand{\phi}{\varphi}
\newcommand{\s}{\models}
\newcommand{\bw}{\bigwede}
\newcommand{\bv}{\bigvee}
\newcommand{\w}{\wedge}
\renewcommand{\v}{\vee}
\newcommand{\sta}{\stackrel}
\newcommand{\su}{\subseteq}
\renewcommand{\iff}{\Longleftrightarrow}
\newcommand{\orc}{\forall}
\newcommand{\ex}{\exists}
\renewcommand{\o}{\circ}
\renewcommand{\Im}{\mbox{\textit{Im}}}
\newcommand{\incl}{\hookrightarrow}
\renewcommand{\partial}{\rightharpoondown}
\newcommand{\la}{\leftarrow}
\newcommand{\defRa}{\looparrowright}
\newcommand{\ra}{\rightarrow}
\renewcommand{\implies}{\Rightarrow}
\newcommand{\Ra}{\Rightarrow}
\newcommand{\lra}{\rightarrow}
\newcommand{\lla}{\leftarrow}
\newcommand{\Lra}{\Longrightarrow}
\newcommand{\LRA}{\Longrightarrow}

\newcommand{\lamterm}{{\mbox{\rm\textsf{\small lam}}}}
\newcommand{\lam}{{\mbox{\rm\textsf{\small lam}}}}
\newcommand{\llam}{{\mbox{\textsf{\small \rm\scriptsize{lam}}}}}



%Inference rules
\newcommand{\rsa}{\rightsquigarrow}
\newcommand{\red}{\rightsquigarrow}
\newcommand{\Rred}{\mbox{ $\,\vspace{-3ex}\red$\hspace{-2.6ex}$\red\,$ }}
\newcommand{\RRred}{\mbox{ $\vspace{-3ex}\red$\hspace{-2.6ex}$\red_{\textit{\scriptsize aux}}$ }}
\newcommand{\RredT}{\mbox{$\,{\Rred\!\!^*}\;$}}
\newcommand{\vvdash}{\mbox{ $\rm{I}\!\!\!\vdash\,$}}
\newcommand{\vvvdash}{\mbox{ $\rm{I}\!\!\!\vdash_{\textit{\scriptsize aux}}\,$}}
\newcommand{\vdashEq}{\vdash_{\textrm{\scriptsize eq}}}
\newcommand{\vdashRaw}{\mbox{ $\rm{I}\!\!\!\vdash\;$}}
\newcommand{\trans}[3]{\mbox{$#1\stackrel{#2}{\rsa}\,#3\;$}}
\newcommand{\transB}[2]{\mbox{$#1\,\rsa\;#2$}} %read: Blank (i.e., unlabeled) transition
\newcommand{\infer}[2]{\mbox{$\displaystyle \frac{#1}{#2}$}}  %read: INFERence rule
\newcommand{\inferN}[3]{\mbox{$\displaystyle \frac{#2}{#3}$}\mbox{\rm\textsf{(#1)}}} %read: ... with Name
\newcommand{\inferS}[3]{\mbox{$\displaystyle \frac{#1}{#2}$}\;\mbox{$#3$}} %read: ... with Side condition
\newcommand{\inferNS}[4]{\mbox{$\displaystyle\frac{#2}{#3}\!\!$}  %read: with Name and Side condition
                          \mbox{$\begin{array}{l}   \mbox{\rm\textsf{(#1)}} \\
                                                     \mbox{$#4$}
                                  \end{array}$}}
%\newcommand{\RLS}[4]{\mbox{$\displaystyle \frac{#2}{#3}$}
%                         \mbox{\hspace{-1.5ex} $\begin{array}{l}   \mbox{\bf(#1)} \\
%                                                    \mbox{$#4$}
%                                 \end{array}$}}
%\newcommand{\RL}[3]{\mbox{$\displaystyle \frac{#2}{#3}${\mbox{ \bf(#1)}}}}

\newcommand{\axmS}[3]{\mbox{$\displaystyle #2\;\;${\mbox{ \bf(#1)}}}\mbox{$#3$}}
\newcommand{\axmSD}[3]{\mbox{$\displaystyle #2$}
                         \mbox{\hspace{-1.5ex} $\begin{array}{l}   \mbox{\bf(#1)} \\
                                                    \mbox{$#3$}
                                 \end{array}$}}
\newcommand{\axmSpecial}[3]{\mbox{$\displaystyle #2$}
                         \mbox{\hspace{-8.5ex} $\begin{array}{c}   \mbox{\bf(#1)} \\
                                                    \mbox{$#3$}
                                 \end{array}$}}
\newcommand{\axm}[2]{\mbox{$\displaystyle #2\;\;${\mbox{ \bf(#1)}}}}



%Caligraphic items
\newcommand{\SN}{\mbox{${\cal SN}$}}
%\newcommand{\A}{\mbox{${\cal A}$}}
\newcommand{\B}{\mbox{${\cal B}$}}
\newcommand{\C}{\mbox{${\cal C}$}}

\newcommand\abstraction{quasiabstraction}
\newcommand\abstractions{quasiabstractions}


% constants and operators
%\newcommand{\lift}{\mbox{\rm\textsf{\small lift}}}
\newcommand{\abss}{\mbox{\rm\textsf{\small abs}}}
\newcommand{\Suc}{\mbox{\rm\textsf{\small Suc}}}
\newcommand{\pred}{\mbox{\rm\textsf{\small pred}}}
\newcommand{\Branch}{\mbox{\rm\textsf{\small Branch}}}
\newcommand{\qSkel}{\mbox{\rm\textsf{\small qSkel}}}
\newcommand{\qSkelAbs}{\mbox{\rm\textsf{\small qSkelAbs}}}
\newcommand{\skel}{\mbox{\rm\textsf{\small skel}}}
\newcommand{\skelAbs}{\mbox{\rm\textsf{\small skelAbs}}}
\newcommand{\sem}{\mbox{\rm\textsf{\small sem}}}
\newcommand{\semAbs}{\mbox{\rm\textsf{\small semAbs}}}
\newcommand{\proc}{\mbox{\rm\textsf{\small proc}}}
%\newcommand{\expp}{\mbox{\rm\textsf{\small exp}}}
\newcommand{\varexp}{\mbox{\rm\textsf{\small varexp}}}
\renewcommand{\exp}{\mbox{\rm\textsf{\small exp}}}

\newcommand{\liftAllt}{\mbox{\rm\textsf{\small liftAll$_2$}}}
\newcommand{\proj}{\mbox{\rm\textsf{\small proj}}}
\newcommand{\projAbs}{\mbox{\rm\textsf{\small projAbs}}}
\newcommand{\good}{\mbox{\rm\textsf{\small good}}}
\newcommand{\qGood}{\mbox{\rm\textsf{\small qGood}}}
\newcommand{\dom}{\mbox{\rm\textsf{\small dom}}}
\newcommand{\goodAbs}{\mbox{\rm\textsf{\small goodAbs}}}
\newcommand{\qGoodAbs}{\mbox{\rm\textsf{\small qGoodAbs}}}
\newcommand{\al}{\mbox{\rm\textsf{\small alpha}}}
\newcommand{\alAbs}{\mbox{\rm\textsf{\small alphaAbs}}}
\newcommand{\It}{\mbox{\rm\textsf{\small Iter}}}
\newcommand{\tbody}{\mbox{\rm\textsf{\small tbody}}}
\newcommand{\funt}{\mbox{\rm\textsf{\small funt}}}
\newcommand{\Fun}{\mbox{\rm\textsf{\small Fun}}}
\newcommand{\whVar}{\mbox{\rm\textsf{\small whVar}}}
\newcommand{\whApp}{\mbox{\rm\textsf{\small whApp}}}
\newcommand{\whLam}{\mbox{\rm\textsf{\small whLam}}}
\newcommand{\CON}{\mbox{\rm\textsf{\small CON}}}
\newcommand{\lambdaa}{\mbox{\rm\textsf{\small lambda}}}
\newcommand{\proper}{\mbox{\rm\textsf{\small proper}}}
\newcommand{\abstr}{\mbox{\rm\textsf{\small abstr}}}
\newcommand{\toNA}{\mbox{\rm\textsf{\small toNA}}}
\newcommand{\toN}{\mbox{\rm\textsf{\small toN}}}
\newcommand{\toH}{\mbox{\rm\textsf{\small toH}}}
\newcommand{\toP}{\mbox{\rm\textsf{\small toP}}}
\newcommand{\toCLN}{\mbox{\rm\textsf{\small toCLN}}}
\newcommand{\combIC}{\mbox{\rm\textsf{\small combIC}}}
\newcommand{\combAC}{\mbox{\rm\textsf{\small combAC}}}
\newcommand{\MetaCt}{\mbox{\rm\textsf{\small Meta.Ct}}}
\newcommand{\Metanf}{\mbox{\rm\textsf{\small Meta.nf}}}
\newcommand{\Eval}{\mbox{\rm\textsf{\small Eval}}}
\newcommand{\ObjAbs}{\mbox{\rm\textsf{\small Obj.Abs}}}
\newcommand{\ObjApp}{\mbox{\rm\textsf{\small Obj.App}}}
\newcommand{\ObjLm}{\mbox{\rm\textsf{\small Obj.Lm}}}
\newcommand{\MetaAbs}{\mbox{\rm\textsf{\small Meta.Abs}}}
\newcommand{\MetaApp}{\mbox{\rm\textsf{\small Meta.App}}}
\newcommand{\rep}{\mbox{\rm\textsf{\small rep}}}
\newcommand{\repAbs}{\mbox{\rm\textsf{\small repAbs}}}
\newcommand{\Const}{\mbox{\rm\textsf{\small Const}}}
\newcommand{\undefinedd}{\mbox{\rm\textsf{\small undefined}}}
\newcommand{\checkk}{\mbox{\rm\textsf{\small check}}}
\newcommand{\checkAbss}{\mbox{\rm\textsf{\small checkAbs}}}
\newcommand{\qInit}{\mbox{\rm\textsf{\small qInit}}}
\newcommand{\qInitAbs}{\mbox{\rm\textsf{\small qInitAbs}}}
\newcommand{\iter}{\mbox{\rm\textsf{\small iter}}}
\newcommand{\iterAbs}{\mbox{\rm\textsf{\small iterAbs}}}
\newcommand{\errMOD}{\mbox{\rm\textsf{\small errMOD}}}
\newcommand{\ERR}{\mbox{\rm\textsf{\small ERR}}}
\newcommand{\iterSTR}{\mbox{\rm\textsf{\small iterSTR}}}
\newcommand{\liftAll}{\mbox{\rm\textsf{\small liftAll}}}
%\newcommand{\lift}{\mbox{\rm\textsf{\small lift}}}
\newcommand{\lift}{\mbox{$\uparrow$}}
\newcommand{\gWls}{\mbox{\rm\textsf{\small gWls}}}
\newcommand{\ggWls}{\mbox{\rm\textsf{\small ggWls}}}
\newcommand{\gWlsAbs}{\mbox{\rm\textsf{\small gWlsAbs}}}
\newcommand{\gFreshAbs}{\mbox{\rm\textsf{\small gFreshAbs}}}
\newcommand{\gSwap}{\mbox{\rm\textsf{\small gSwap}}}
\newcommand{\gSwapAbs}{\mbox{\rm\textsf{\small gSwapAbs}}}
\newcommand{\gSubst}{\mbox{\rm\textsf{\small gSubst}}}
\newcommand{\gSubstAbs}{\mbox{\rm\textsf{\small gSubstAbs}}}
\newcommand{\wlsFSb}{\mbox{\rm\textsf{\small wlsFSb}}}
\newcommand{\wlsFSw}{\mbox{\rm\textsf{\small wlsFSw}}}
\newcommand{\wlsFSbSw}{\mbox{\rm\textsf{\small wlsFSbSw}}}
\newcommand{\wlsFSwSb}{\mbox{\rm\textsf{\small wlsFSwSb}}}
\newcommand{\MOD}{\mbox{\rm\textsf{\small MOD}}}
\newcommand{\gWlsAllDisj}{\mbox{\rm\textsf{\small gWlsAllDisj}}}
\newcommand{\gWlsAbsIsInBar}{\mbox{\rm\textsf{\small gWlsAbsIsInBar}}}
\newcommand{\gConsPresGWls}{\mbox{\rm\textsf{\small gConsPresGWls}}}
\newcommand{\gSubstAllPresGWlsAll}{\mbox{\rm\textsf{\small gSubstPresGWlsAll}}}
\newcommand{\gFreshCls}{\mbox{\rm\textsf{\small gFreshCls}}}
\newcommand{\gSubstCls}{\mbox{\rm\textsf{\small gSubstCls}}}
\newcommand{\gAbsRen}{\mbox{\rm\textsf{\small gAbsRen}}}
\newcommand{\termFSbMorph}{\mbox{\rm\textsf{\small termFSbMorph}}}
\newcommand{\termFSwMorph}{\mbox{\rm\textsf{\small termFSwMorph}}}
\newcommand{\termFSwSbMorph}{\mbox{\rm\textsf{\small termFSwSbMorph}}}
\newcommand{\rec}{\mbox{\rm\textsf{\small rec}}}
\newcommand{\recAbs}{\mbox{\rm\textsf{\small recAbs}}}
\newcommand{\wlsSEM}{\mbox{\rm\textsf{\small wlsSEM}}}
\newcommand{\SEMM}{\mbox{\rm\textsf{\small SEM}}}
\newcommand{\compInt}{\mbox{\rm\textsf{\small compInt}}}
\newcommand{\semInt}{\mbox{\rm\textsf{\small semInt}}}
\newcommand{\semIntAbs}{\mbox{\rm\textsf{\small semIntAbs}}}
\newcommand{\supp}{\mbox{\rm\textsf{\small supp}}}
\newcommand{\FV}{\mbox{\rm\textsf{\small FV}}}
\newcommand{\ABS}{\mbox{\rm\textsf{\small ABS}}}
\newcommand{\finite}{\mbox{\rm\textsf{\small finite}}}
\newcommand{\rev}{\mbox{\rm\textsf{\small rev}}}
\newcommand{\rrev}{{\mbox{\rm\textsf{\scriptsize rev}}}}
\renewcommand{\max}{\mbox{\rm\textsf{\small max}}}
\newcommand{\qDepth}{\mbox{\rm\textsf{\small qDepth}}}
\newcommand{\depthh}{\mbox{\rm\textsf{\small depth}}}
\newcommand{\swapped}{\mbox{\rm\textsf{\small swapped}}}
\newcommand{\swapping}{\mbox{\rm\textsf{\small swapping}}}
\newcommand{\varsOf}{\mbox{\rm \textsf{\small vars\hspace{-0.1ex}Of}}}
\newcommand{\None}{\mbox{\rm\textsf{\small None}}}
\newcommand{\Some}{\mbox{\rm\textsf{\small Some}}}
\newcommand{\wls}{\mbox{\rm\textsf{\small wls}}}
\newcommand{\wlsAbs}{\mbox{\rm\textsf{\small wlsAbs}}}
\newcommand{\wlsInp}{\mbox{\rm\textsf{\small wlsInp}}}
\newcommand{\wlsBinp}{\mbox{\rm\textsf{\small wlsBinp}}}
\newcommand{\ost}{\mbox{\rm\textsf{\small o}}}
\newcommand{\tst}{\mbox{\rm\textsf{\small t}}}
\newcommand{\kst}{\mbox{\rm\textsf{\small k}}}
\newcommand{\oost}{{\mbox{\rm\textsf{\scriptsize o}}}}
\newcommand{\ttst}{{\mbox{\rm\textsf{\scriptsize t}}}}
\newcommand{\kkst}{{\mbox{\rm\textsf{\scriptsize k}}}}
\newcommand{\vo}{\mbox{\rm\textsf{\small vo}}}
\newcommand{\vt}{\mbox{\rm\textsf{\small vt}}}
\newcommand{\tapp}{\mbox{\rm\textsf{\small tapp}}}
\newcommand{\Itapp}{\mbox{\rm\textsf{\small Itapp}}}
\newcommand{\tlam}{\mbox{\rm\textsf{\small tlam}}}
\newcommand{\Itlam}{\mbox{\rm\textsf{\small Itlam}}}
\newcommand{\tprod}{\mbox{\rm\textsf{\small tprod}}}
\newcommand{\Itprod}{\mbox{\rm\textsf{\small Itprod}}}
\newcommand{\type}{\mbox{\rm\textsf{\small type}}}
\newcommand{\kprod}{\mbox{\rm\textsf{\small kprod}}}
\newcommand{\Ikprod}{\mbox{\rm\textsf{\small Ikprod}}}
\newcommand{\vlm}{\mbox{\rm\textsf{\small vlm}}}
\newcommand{\Iapp}{\mbox{\rm\textsf{\small Iapp}}}
\newcommand{\Ilam}{\mbox{\rm\textsf{\small Ilam}}}
\newcommand{\Inp}{\mbox{\rm\textsf{\small Inp}}}
\newcommand{\Out}{\mbox{\rm\textsf{\small Out}}}
\newcommand{\sameDom}{\mbox{\rm\textsf{\small sameDom}}}
\newcommand{\isInBar}{\mbox{\rm\textsf{\small isInBar}}}
\newcommand{\Type}{\mbox{\rm\textsf{\small Type}}}
\newcommand{\qOp}{\mbox{\rm\textsf{\small q\hspace*{-0.1ex}Op}}}
\newcommand{\gOp}{\mbox{\rm\textsf{\small g\hspace*{-0.1ex}Op}}}
\newcommand{\qAbs}{\mbox{\rm\textsf{\small q\hspace*{-0.1ex}Abs}}}
%\newcommand{\gAbs}{\mbox{\rm\textsf{\small g\hspace*{-0.1ex}Abs}}}
\newcommand{\asSort}{\mbox{\rm\textsf{\small asSort}}}
\newcommand{\aasSort}{{\mbox{\textsf{\small \rm\scriptsize{asSort}}}}}
\newcommand{\stOf}{\mbox{\rm\textsf{\small stOf}}}
\newcommand{\sstOf}{{\mbox{\textsf{\small \rm\scriptsize{stOf}}}}}
\newcommand{\arOf}{\mbox{\rm\textsf{\small arOf}}}
\newcommand{\aarOf}{{\mbox{\textsf{\small \rm\scriptsize{arOf}}}}}
\newcommand{\barOf}{\mbox{\rm\textsf{\small barOf}}}
\newcommand{\bbarOf}{{\mbox{\textsf{\small \rm\scriptsize{barOf}}}}}
\newcommand{\cut}{\mbox{\rm\textsf{\small cut}}}
\newcommand{\vmapDB}{\mbox{\rm\textsf{\small vmapDB}}}
\newcommand{\inj}{\mbox{\rm\textsf{\small inj}}}
\newcommand{\insertt}{\mbox{\rm\textsf{\small insert}}}
\newcommand{\setsum}{\mbox{\rm\textsf{\small setsum}}}
\newcommand{\Ct}{\mbox{\rm\textsf{\small Ct}}}
\newcommand{\InVl}{\mbox{\rm\textsf{\small In\hspace*{-0.1ex}Vl}}}
\newcommand{\cps}{\mbox{\rm\textsf{\small cps}}}
\newcommand{\cpsvl}{\mbox{\rm\textsf{\small cps$_{\textit{\scriptsize vl}}$}}}
\newcommand{\cdev}{\mbox{\rm\textsf{\small cdev}}}
\newcommand{\no}{\mbox{\rm\textsf{\small no}}}
\newcommand{\pr}{\mbox{\rm\textsf{\small pr}}}
\newcommand{\app}{\mbox{{\rm\textsf{\small app}}}}
\newcommand{\aapp}{{\mbox{\textsf{\small \rm\scriptsize{app}}}}}
\newcommand{\lm}{\mbox{{\rm\textsf{\small lm}}}}
\newcommand{\llm}{{\mbox{\textsf{\rm\scriptsize{llm}}}}}

\newcommand{\qAFresh}{{\mbox{\textsf{qAFresh}}}}
\newcommand{\typeOf}{{\mbox{\textsf{typeOf}}}}
\newcommand{\card}{{\mbox{\rm\textsf{\small card}}}}
\newcommand{\cWls}{\mbox{\rm\textsf{\small cWls}}}
\newcommand{\Ttrue}{\mbox{\rm\textsf{\small True}}}
\newcommand{\Ffalse}{\mbox{\rm\textsf{\small False}}}
\newcommand{\Var}{\mbox{\rm\textsf{\small Var}}}
\newcommand{\qVar}{\mbox{\rm\textsf{q\hspace*{-0.2ex}Var}}}
\newcommand{\dVar}{\mbox{\rm\textsf{\small d\hspace*{-0.2ex}Var}}}
\newcommand{\tVar}{\mbox{\rm\textsf{\small t\hspace*{-0.2ex}Var}}}
\newcommand{\gVar}{\mbox{\rm\textsf{\small g\hspace*{-0.2ex}Var}}}
\newcommand{\ggVar}{\mbox{\rm\textsf{\small gg\hspace*{-0.2ex}Var}}}
\newcommand{\asHctxt}{\mbox{\rm\textsf{\small asHctxt}}}
\newcommand{\isCtxt}{\mbox{\rm\textsf{\small isCtxt}}}
\newcommand{\Arr}{\mbox{{\rm \textsf{\small Arr}}}}
\newcommand{\fArr}{\mbox{\rm\textsf{\footnotesize Arr}}}
\newcommand{\Al}{\mbox{{\rm \textsf{\small Al}}}}
\newcommand{\App}{\mbox{\rm\textsf{\small App}}}
\newcommand{\Tapp}{\mbox{\rm\textsf{\small T\hspace*{-0.1ex}app}}}
\newcommand{\Tprod}{\mbox{\rm\textsf{\small T\hspace*{-0.2ex}prod}}}
\newcommand{\Kprod}{\mbox{\rm\textsf{\small K\hspace*{-0.1ex}prod}}}
\newcommand{\qApp}{\mbox{\rm\textsf{\small q\hspace*{-0.1ex}App}}}
\newcommand{\gApp}{\mbox{\rm\textsf{\small g\hspace*{-0.1ex}App}}}
\newcommand{\Lam}{\mbox{\rm\textsf{\small Lam}}}
\newcommand{\Tlam}{\mbox{\rm\textsf{\small T\hspace*{-0.1ex}lam}}}
\newcommand{\qLam}{\mbox{\rm\textsf{\small q\hspace*{-0.1ex}Lam}}}
\newcommand{\Lm}{\mbox{\rm\textsf{\small Lm}}}
\newcommand{\qLm}{\mbox{\rm\textsf{\small qLm}}}
\newcommand{\gLm}{\mbox{\rm\textsf{\small g\hspace*{-0.1ex}Lm}}}
\newcommand{\Abs}{\mbox{\rm\textsf{\small Abs}}}
\newcommand{\Dabs}{\mbox{\rm\textsf{\small Dabs}}}
\newcommand{\Tabs}{\mbox{\rm\textsf{\small Tabs}}}
\newcommand{\cInV}{\mbox{\rm\textsf{\small cInV}}}
\newcommand{\cApp}{\mbox{\rm\textsf{\small cApp}}}
\newcommand{\cLam}{\mbox{\rm\textsf{\small cLam}}}
\newcommand{\cArr}{\mbox{\rm\textsf{\small cArr}}}
\newcommand{\cAl}{\mbox{\rm\textsf{\small cAl}}}
\newcommand{\fresh}{\mbox{\rm\textsf{\small fresh}}}
\newcommand{\FRESH}{\mbox{\rm\textsf{\small FRESH}}}
\newcommand{\ctFresh}{\mbox{\rm\textsf{\small ct\hspace*{-0.1ex}Fresh}}}
\newcommand{\qFresh}{\mbox{\rm\textsf{\small q\hspace*{-0.1ex}Fresh}}}
\newcommand{\qFreshAbs}{\mbox{\rm\textsf{\small qFreshAbs}}}
\newcommand{\gFresh}{\mbox{\rm\textsf{\small g\hspace*{-0.1ex}Fresh}}}
\newcommand{\ffresh}{\mbox{\rm\textsf{\footnotesize fresh}}}
\newcommand{\freshAbs}{\mbox{\rm\textsf{\small freshAbs}}}
\newcommand{\FRESHABS}{\mbox{\rm\textsf{\small FRESHABS}}}
\newcommand{\freshEnv}{\mbox{\rm\textsf{\small freshEnv}}}
\newcommand{\pickFresh}{\mbox{\rm\textsf{pickFresh}}}
\newcommand{\pickDistinct}{\mbox{\rm\textsf{pickDistinct}}}
\newcommand{\id}{\mbox{\rm\textsf{\small id}}}
\newcommand{\AppL}{\mbox{\rm\textsf{\small AppL}}}
\newcommand{\varOfAbs}{\mbox{\rm\textsf{\small varOfAbs}}}
\newcommand{\termOfAbs}{\mbox{\rm\textsf{\small termOfAbs}}}

\newcommand{\toDB}{\mbox{\rm\textsf{\small to$\vspace{-1ex}$D$\vspace{-1ex}$B}}}
\newcommand{\mkFst}{\mbox{\rm\textsf{\small mk$\vspace{-1ex}$Fst}}}

\newcommand{\shift}{\mbox{\rm\textsf{\small shift}}}
\newcommand{\mdr}{\mbox{\rm\textsf{\small mdr}}}
\newcommand{\mmdr}{\mbox{\scriptsize \rm\textsf{mdr}}}
\newcommand{\newGoal}{\mbox{\rm\textsf{\small newGoal}}}
\newcommand{\sync}{\mbox{\rm\textsf{\small sync}}}
\newcommand{\PREF}{\mbox{\rm\textsf{\small PREF}}}
\newcommand{\PARL}{\mbox{\rm\textsf{\small PARL}}}
\newcommand{\PARR}{\mbox{\rm\textsf{\small PARR}}}
\newcommand{\PARS}{\mbox{\rm\textsf{\small PARS}}}
\newcommand{\REPL}{\mbox{\rm\textsf{\small REPL}}}
\newcommand{\REPLS}{\mbox{\rm\textsf{\small REPLS}}}
\newcommand{\DRL}[1]{\mbox{\rm\textsf{\small DRL$_{#1}$}}}
\newcommand{\PLUS}{\mbox{\rm\textsf{\small PLUS}}}
\newcommand{\TIMES}{\mbox{\rm\textsf{\small TIMES}}}

\newcommand{\Par}{\mbox{\rm\textsf{\small Par}}}
\newcommand{\Repl}{\mbox{\rm\textsf{\small Repl}}}
\newcommand{\Pref}{\mbox{\rm\textsf{\small Pref}}}
\newcommand{\Left}{\mbox{\rm\textsf{\small Left}}}
\newcommand{\Right}{\mbox{\rm\textsf{\small Right}}}
\newcommand{\fst}{\mbox{\rm\textsf{\small fst}}}
\newcommand{\snd}{\mbox{\rm\textsf{\small snd}}}
\newcommand{\llabels}{\mbox{\rm\textsf{\small labels}}}
\newcommand{\proves}{\mbox{\rm\textsf{\small proves}}}
\newcommand{\correct}{\mbox{\rm\textsf{\small correct}}}
\newcommand{\Ax}{\mbox{\rm\textsf{\small Ax}}}
\newcommand{\Split}{\mbox{\rm\textsf{\small Split}}}
\newcommand{\Coind}{\mbox{\rm\textsf{\small Coind}}}
\newcommand{\bis}{\mbox{\rm\textsf{\small bis}}}
\newcommand{\ubis}{\mbox{\rm\textsf{\small ubis}}}
\newcommand{\simul}{\mbox{\rm\textsf{\small simul}}}
\newcommand{\Retr}{\mbox{\rm\textsf{\small Retr}}}
\newcommand{\step}{\mbox{\rm\textsf{\small step}}}
\newcommand{\disjoint}{\mbox{\rm\textsf{\small disjoint}}}
\newcommand{\nonrep}{\mbox{\rm\textsf{\small nonrep}}}
\newcommand{\hyps}{\mbox{\rm\textsf{\small hyps}}}
\newcommand{\cnc}{\mbox{\rm\textsf{\small cnc}}}
\newcommand{\side}{\mbox{\rm\textsf{\small side}}}
\newcommand{\sside}{\mbox{\scriptsize{\rm\textsf{side}}}}
\newcommand{\pside}{\mbox{\rm\textsf{\small pside}}}
\newcommand{\op}{\mbox{\rm\textsf{\small op}}}
\newcommand{\Op}{\mbox{\rm\textsf{\small Op}}}
\newcommand{\OP}{\mbox{\rm\textsf{\small OP}}}
\newcommand{\set}{\mbox{\rm\textsf{\small set}}}
\newcommand{\subst}{\mbox{\rm\textsf{\small subst}}}
\newcommand{\qSubstRel}{\mbox{\rm\textsf{\small qSubstRel}}}
\newcommand{\vars}{\mbox{\rm\textsf{\small vars}}}
\newcommand{\llength}{\mbox{\rm\textsf{\small length}}}
\newcommand{\Cons}{\mbox{\rm\textsf{\small Cons}}}
\newcommand{\Nil}{\mbox{\rm\textsf{\small Nil}}}
\newcommand{\CONS}{\mbox{\rm\textsf{\small CONS}}}
\newcommand{\NIL}{\mbox{\rm\textsf{\small NIL}}}
\newcommand{\theXs}{\mbox{\rm\textsf{\small theXs}}}
\newcommand{\theXXs}{\mbox{\rm\textsf{\small theX$\!$Xs}}}
\newcommand{\theYs}{\mbox{\rm\textsf{\small theYs}}}
\newcommand{\theS}{\mbox{\rm\textsf{\small theS}}}
\newcommand{\theT}{\mbox{\rm\textsf{\small theT}}}
\newcommand{\theps}{\mbox{\rm\textsf{\small theps}}}
\newcommand{\theI}{\mbox{\rm\textsf{\small theI}}}
\newcommand{\thef}{\mbox{\rm\textsf{\small thef}}}
\newcommand{\ttheI}{\mbox{\scriptsize{\rm\textsf{theI}}}}
\newcommand{\src}{\mbox{\rm\textsf{\small src}}}
\newcommand{\tgt}{\mbox{\rm\textsf{\small tgt}}}
\newcommand{\congCl}{\mbox{\rm\textsf{\small congCl}}}
\newcommand{\sstvsmCl}{\mbox{\rm\textsf{\small sstvsmCl}}}

\newcommand{\ded}{\mbox{\rm\textsf{\small ded}}}
\newcommand{\rded}{\mbox{\rm\textsf{\small rded}}}

\newcommand{\Zero}{\mbox{\rm\textsf{\small Zero}}}
\newcommand{\Plus}{\mbox{\rm\textsf{\small Plus}}}
\newcommand{\Times}{\mbox{\rm\textsf{\small Times}}}
\newcommand{\Odd}{\mbox{\rm\textsf{\small Odd}}}
\newcommand{\Even}{\mbox{\rm\textsf{\small Even}}}
\newcommand{\Zip}{\mbox{\rm\textsf{\small Zip}}}
\newcommand{\odd}{\mbox{\rm\textsf{\small odd}}}
\newcommand{\even}{\mbox{\rm\textsf{\small even}}}
\newcommand{\zip}{\mbox{\rm\textsf{\small zip}}}
\newcommand{\map}{\mbox{\rm\textsf{\small map}}}
\newcommand{\VAR}{\mbox{{\rm \textsf{\small V$\hspace*{-0.1ex}$A$\hspace*{-0.1ex}$R}}}}
\newcommand{\APP}{\mbox{{\rm \textsf{\small A$\hspace*{-0.1ex}$P$\hspace*{-0.1ex}$P}}}}
\newcommand{\LM}{\mbox{{\rm \textsf{\small L$\hspace*{-0.1ex}$M}}}}
\newcommand{\All}{{\mbox{\rm\textsf{\small All}}}}  % the first-order forall

%placeholder (mixfix) operators
\newcommand{\abis}{\mbox{$\;\preceq\;$}}
\newcommand{\oabis}{\mbox{$\;\preceq^o\;$}}
\newcommand{\babis}{\mbox{$\;\preceq^\bullet\;$}}
\newcommand{\Himp}{\mbox{\large $\;=\!>\;$}}   %Horn implication
\newcommand{\Bredn}{\mbox{\large $\;=\hspace*{-0.2ex}=\!>_n\;$}} %big-step reduction
\newcommand{\MBredn}{\mbox{\large $\;=\hspace*{-0.2ex}=\!>_{nM}\;$}} %meta big-step reduction
\renewcommand{\ae}{\simeq_\alpha}
\newcommand{\SUBSTArg}[1]{\mbox{$\_\,[\_\, / #1]$}}
\newcommand{\SUBST}{\mbox{$\_\,[\_\, / \!\_\,]$}}
\newcommand{\bSUBST}{\mbox{$\_\,[\_\, / \!\_\,]^{\!b}$}}
\newcommand{\gSUBST}{\mbox{$\_\,[\_\, / \!\_\,]^{\!g}$}}
\newcommand{\oSUBST}{\mbox{$\_\,[\_\, /_{\!\!o}\, \_\,]$}}
\newcommand{\mSUBST}{\mbox{$\_\,[\_\, /_{\!\!m}\, \_\,]$}}
\newcommand{\gSSUBST}{\mbox{$(\_\;,\_\;)[(\_\;,\_\;) / \_\;]^{\!g}$}}
\newcommand{\qSUBST}{\mbox{$\_\,[\_\, / \!\_\,]^{\!q}$}}
\newcommand{\SWAPArg}[2]{\mbox{$\_\,[#1 \wedge #2]$}}
\newcommand{\SWAP}{\mbox{$\_\,[\_ \wedge \!\_\,]$}}
\newcommand{\gSWAP}{\mbox{$\_\,[\_ \wedge \!\_\,]^{\!g}$}}
\newcommand{\qSWAP}{\mbox{$\_\,[\_ \wedge \!\_\,]^{\!q}$}}
\newcommand{\vSWAP}{\mbox{$\_\,[\_ \wedge \!\_\,]^{\!v}$}}
\newcommand{\SEM}{\mbox{$[\_\,]$}}

%metavariables
%metavariables
\newcommand{\phiAbs}{\mbox{$\phi$\textit{Abs}}}
\newcommand{\f}{\mbox{\textit{f}}}
\newcommand{\fAbs}{\mbox{\textit{fAbs}}}
\newcommand{\g}{\mbox{\textit{g}}}
\newcommand{\gAbs}{\mbox{\textit{gAbs}}}
\newcommand{\Pff}{\mbox{\textit{Pf}}}
\newcommand{\Ps}{\mbox{\textit{Ps}}}
\newcommand{\inp}{\mbox{\textit{inp}}}
\newcommand{\binp}{\mbox{\textit{binp}}}
\newcommand{\qinp}{\mbox{\textit{qinp}}}
\newcommand{\qbinp}{\mbox{\textit{qbinp}}}
\newcommand{\tx}{\mbox{\textit{tx}}}
\newcommand{\ty}{\mbox{\textit{ty}}}
\newcommand{\tz}{\mbox{\textit{tz}}}
\newcommand{\tX}{\mbox{\textit{t\hspace*{-0.1ex}X}}}
\newcommand{\gX}{\mbox{\textit{g\hspace*{-0.1ex}X}}}
\newcommand{\kX}{\mbox{\textit{k\hspace*{-0.1ex}X}}}
\newcommand{\qX}{\mbox{\textit{q\hspace*{-0.1ex}X}}}
\newcommand{\mX}{\mbox{\textit{m\hspace*{-0.1ex}X}}}
\newcommand{\mU}{\mbox{\textit{m\hspace*{-0.3ex}U}}}
\newcommand{\mV}{\mbox{\textit{m\hspace*{-0.3ex}V}}}
\newcommand{\ttX}{\mbox{\textit{\scriptsize tX}}}
\newcommand{\tY}{\mbox{\textit{t\hspace*{-0.2ex}Y}}}
\newcommand{\kY}{\mbox{\textit{k\hspace*{-0.2ex}Y}}}
\newcommand{\gY}{\mbox{\textit{g\hspace*{-0.2ex}Y}}}
\newcommand{\qY}{\mbox{\textit{q\hspace*{-0.2ex}Y}}}
\newcommand{\mY}{\mbox{\textit{m\hspace*{-0.1ex}Y}}}
\newcommand{\tZ}{\mbox{\textit{t\hspace*{-0.1ex}Z}}}
\newcommand{\gZ}{\mbox{\textit{g\hspace*{-0.1ex}Z}}}
\newcommand{\kZ}{\mbox{\textit{k\hspace*{-0.1ex}Z}}}
\newcommand{\qZ}{\mbox{\textit{q\hspace*{-0.1ex}Z}}}
\newcommand{\mZ}{\mbox{\textit{m\hspace*{-0.1ex}Z}}}
\newcommand{\tA}{\mbox{\textit{t\hspace*{-0.1ex}A}}}
\newcommand{\kA}{\mbox{\textit{k\hspace*{-0.1ex}A}}}
\newcommand{\qA}{\mbox{\textit{q\hspace*{-0.1ex}A}}}
\newcommand{\mA}{\mbox{\textit{m\hspace*{-0.1ex}A}}}
\newcommand{\tB}{\mbox{\textit{t\hspace*{-0.1ex}B}}}
\newcommand{\kB}{\mbox{\textit{k\hspace*{-0.1ex}B}}}
\newcommand{\qB}{\mbox{\textit{q\hspace*{-0.1ex}B}}}
\newcommand{\mB}{\mbox{\textit{m\hspace*{-0.1ex}B}}}
\newcommand{\tC}{\mbox{\textit{t\hspace*{-0.1ex}C}}}
\newcommand{\kC}{\mbox{\textit{k\hspace*{-0.1ex}C}}}
\newcommand{\qC}{\mbox{\textit{q\hspace*{-0.1ex}C}}}
\newcommand{\mC}{\mbox{\textit{m\hspace*{-0.1ex}C}}}
\newcommand{\Zs}{\mbox{\textit{Zs}}}

\newcommand{\bs}{\mbox{\textit{bs}}}
\newcommand{\as}{\mbox{\textit{as}}}
\newcommand{\aas}{\mbox{\scriptsize{\textit{as}}}}
\newcommand{\xs}{\mbox{\textit{xs}}}
\newcommand{\ys}{\mbox{\textit{ys}}}
\newcommand{\zs}{\mbox{\textit{zs}}}
\newcommand{\ps}{\mbox{\textit{ps}}}
\newcommand{\Ts}{\mbox{\textit{Ts}}}
\newcommand{\rl}{\mbox{\textit{rl}}}
\newcommand{\rrl}{\mbox{\scriptsize{\textit{rl}}}}
\newcommand{\drl}{\mbox{\textit{drl}}}
\newcommand{\ddrl}{\mbox{\scriptsize \textit{drl}}}
\newcommand{\Rls}{\mbox{\textit{Rls}}}
\newcommand{\Xs}{\mbox{\textit{Xs}}}
\newcommand{\XX}{\mbox{\textit{X$\!$X}}}
\newcommand{\XXs}{\mbox{\textit{X$\!$Xs}}}
\newcommand{\Ys}{\mbox{\textit{Ys}}}

\newcommand{\Xvl}{{\mbox{\textit{X$\hspace*{-0.1ex}$vl}}}}
\newcommand{\Yvl}{{\mbox{\textit{Y$\hspace*{-0.1ex}$vl}}}}
\newcommand{\Zvl}{{\mbox{\textit{Z$\hspace*{-0.1ex}$vl}}}}


%signatures and theories
\renewcommand{\L}{\mbox{$\cal L$}}
\newcommand{\Horn}{\mbox{$\cal H$}}
\newcommand{\hHorn}{{\mbox{\scriptsize $\cal H$}}}
\newcommand{\SigSubst}{\mbox{$\Sigma_{\textsf{\scriptsize sb}}$}}
\newcommand{\HornSubst}{\mbox{$\Horn_{\textsf{\scriptsize sb}}$}}
\newcommand{\SigFresh}{\mbox{$\Sigma_T^{\textsf{\scriptsize fr}}$}}
\newcommand{\HornFresh}{\mbox{$\Horn_{\textsf{\scriptsize fr}}$}}
\newcommand{\SigSwap}{\mbox{$\Sigma_{\textsf{\scriptsize sw}}$}}
\newcommand{\HornSwap}{\mbox{$\Horn_{\textsf{\scriptsize sw}}$}}
\newcommand{\SigFSw}{\mbox{$\Sigma_{\textsf{\scriptsize fr,sw}}$}}
\newcommand{\HornFSw}{\mbox{$\Horn_{\textsf{\scriptsize fr,sw}}$}}
\newcommand{\SigFS}{\mbox{$\Sigma_{\textsf{\scriptsize fr,sb}}$}}
\newcommand{\HornFS}{\mbox{$\Horn_{\textsf{\scriptsize fr,sb}}$}}
\newcommand{\SigFSS}{\mbox{$\Sigma_{\textsf{\scriptsize fr,sb,sw}}$}}
\newcommand{\HornFSS}{\mbox{$\Horn_{\textsf{\scriptsize fr,sb,sw}}$}}
\newcommand{\HornFSSP}{\mbox{$\Horn'_{\textsf{\scriptsize fr,sb,sw}}$}}
\newcommand{\SigSet}{\mbox{$\Sigma_S$}}
\newcommand{\HornSet}{\mbox{$\Horn_S$}}


%types
\newcommand{\val}{\bf val}
\newcommand{\tre}{\bf tree}
\newcommand{\A}{\bf A}
\newcommand{\T}{\bf T}
\newcommand{\D}{\bf D}
\newcommand{\G}{\bf G}
\newcommand{\Dt}{\bf Dt}
\newcommand{\Da}{\bf Da}
\newcommand{\chan}{\mbox{\bf chan}}
\newcommand{\nat}{\mbox{\bf nat}}
\newcommand{\labell}{\mbox{\bf label}}
\newcommand{\state}{\mbox{\bf state}}
\newcommand{\whterm}{\mbox{\bf whTerm}}
\newcommand{\atm}{\mbox{\bf atm}}
\newcommand{\oo}{\mbox{\bf oo}}
\newcommand{\expp}{\mbox{\bf exp}}
\newcommand{\procc}{\mbox{\bf proc}}
\newcommand{\termCFS}{\mbox{$\term_{C,F,S}$}}
\newcommand{\Objterm}{\mbox{\bf Obj.term}}
\newcommand{\Metaterm}{\mbox{\bf Meta.term}}
\newcommand{\Sh}{\mbox{\bf Sh}}
\newcommand{\Term}{\mbox{\bf Term}}
\newcommand{\VarT}{\mbox{\bf Var}}
\newcommand{\Set}{\mbox{\bf Set}}
\newcommand{\eic}{\mbox{\bf eic}}
\newcommand{\perm}{\mbox{\bf perm}}
\newcommand{\qAbsT}{\mbox{\bf q$\hspace{-0.1ex}$Abs}}
\newcommand{\gAbsT}{\mbox{\bf g$\hspace{-0.1ex}$Abs}}
\newcommand{\inpT}{\mbox{\bf inp}}
\newcommand{\qinpT}{\mbox{\bf qinp}}
\newcommand{\qbinpT}{\mbox{\bf qbinp}}
\newcommand{\binpT}{\mbox{\bf binp}}
\newcommand{\varsort}{\mbox{\bf varsort}}
\newcommand{\sort}{\mbox{\bf sort}}
%\newcommand{\opsym}{\mbox{\bf op$\hspace{-0.1ex}$Sym}}
\newcommand{\dB}{\mbox{\bf dB}}
\newcommand{\termfl}{\mbox{\bf term$_{\textit{\scriptsize fl}}$}}
\newcommand{\termvl}{\mbox{\bf term$_{\textit{\scriptsize vl}}$}}
\newcommand{\absvlfl}{\mbox{\bf abs$_{\textit{\scriptsize (vl,fl)}}$}}
\newcommand{\fset}{\mbox{\bf fset}}
\newcommand{\FOvar}{\mbox{\bf FOvar}}
\newcommand{\FOterm}{\mbox{\bf FOterm}}
\newcommand{\itemm}{\mbox{\bf item}}
\newcommand{\iitemm}{\mbox{\bf \scriptsize item}}
\newcommand{\nnat}{{\mbox{\scriptsize{$I\!\!N$}}}}
\newcommand{\Nat}{\mbox{$I\!\!N$}}
\newcommand{\nNat}{\mbox{\scriptsize{$I\!\!N$}}}
\newcommand{\bool}{\mbox{\bf bool}}
\newcommand{\const}{\mbox{\bf const}}
\newcommand{\metaConst}{\mbox{\bf meta\hspace*{-0.1ex}Const}}
\newcommand{\var}{\mbox{\bf var}}
\newcommand{\vvar}{\mbox{\bf \scriptsize var}}
\newcommand{\qTerm}{\mbox{\bf q\hspace*{-0.1ex}Term}}
\newcommand{\gTerm}{\mbox{\bf g\hspace*{-0.1ex}Term}}
\newcommand{\sTerm}{\mbox{\bf s\hspace*{-0.1ex}Term}}
\newcommand{\term}{\mbox{\bf term}}
\newcommand{\clterm}{\mbox{\bf clterm}}
\newcommand{\abs}{\mbox{\bf abs}}
\newcommand{\absT}{\mbox{\bf abs2}}
\newcommand{\env}{\mbox{\bf env}}
\newcommand{\valT}{\mbox{\bf val}}

\newcommand{\dvar}{\mbox{\bf dvar}}
\newcommand{\dterm}{\mbox{\bf dterm}}
\newcommand{\dabs}{\mbox{\bf dabs}}
\newcommand{\dabsT}{\mbox{\bf dabs2}}
\newcommand{\denv}{\mbox{\bf denv}}

\newcommand{\tvar}{\mbox{\bf tvar}}
\newcommand{\tterm}{\mbox{\bf tterm}}
\newcommand{\qterm}{\mbox{\bf qterm}}
\newcommand{\ttterm}{\mbox{{\scriptsize \bf tterm}}}
\newcommand{\smallTerm}{\mbox{{\scriptsize \bf term}}}
\newcommand{\tabs}{\mbox{\bf tabs}}
\newcommand{\qabs}{\mbox{\bf qabs}}
\newcommand{\tabsT}{\mbox{\bf tabs2}}
\newcommand{\tenv}{\mbox{\bf tenv}}
\newcommand{\ctxt}{\mbox{\bf ctxt}}
\newcommand{\Hctxt}{\mbox{\bf Hctxt}}

\renewcommand{\iint}{\mbox{\bf int}}
\newcommand{\indexx}{\mbox{\bf index}}
\newcommand{\iindexx}{{\mbox{\textsf{\small \rm\scriptsize{index}}}}}
\newcommand{\bindexx}{\mbox{\bf bindex}}
\newcommand{\bbindexx}{{\mbox{\textsf{\small \rm\scriptsize{bindex}}}}}
\newcommand{\prf}{\mbox{\bf prf}}
\newcommand{\rel}{\mbox{\bf rel}}
\newcommand{\sstring}{\mbox{\bf string}}

\newcommand{\act}{\mbox{\bf act}}
\newcommand{\param}{\mbox{\bf param}}
\newcommand{\rrule}{\mbox{\bf rule}}  % cannot use \rule, as it creates problems
\newcommand{\opsym}{\mbox{\bf opsym}}
\newcommand{\eqn}{\mbox{\bf equation}}

%type constructors
\newcommand{\option}{\mbox{\bf option}}
\newcommand{\Option}{\mbox{\bf Option}}
\newcommand{\Inputt}{\mbox{\bf Input}}
\newcommand{\inputt}{\mbox{\bf input}}
\newcommand{\Cur}{\mbox{\bf Cur}}  %currying
\renewcommand{\P}{\mbox{\bf \textsc{P}}} %powerset
\newcommand{\Pf}{\mbox{$\P_{\!f}$}}  %finite powerset
\newcommand{\Pfs}{\mbox{\textit{Pfs}}}
\newcommand{\Pne}{\mbox{{\bf \textsc{P}}$_{\!\!\not=\emptyset}$}}  % set of nonempty sets
\newcommand{\List}{\mbox{\bf List}}
\newcommand{\Ftrans}{\mbox{\bf Ftrans}}  %formal transitions


%Signature operation and relation symbols
\newcommand{\freshS}[1]{\bT{\textsf{fresh$_#1$}}}
\newcommand{\ffreshS}[1]{\bT{\scriptsize \textsf{fresh$_#1$}}}
\newcommand{\EmpS}{\bT{\textsf{Emp}}}
\newcommand{\SinglS}[1]{\bT{\textsf{Singl$_#1$}}}
\newcommand{\InsertS}[1]{\bT{\textsf{Insert$_#1$}}}
\newcommand{\substS}[1]{\bT{\textsf{subst$_#1$}}}
\newcommand{\ssubstS}[1]{\bT{\scriptsize \textsf{subst$_#1$}}}
\newcommand{\swapS}[2]{\bT{\textsf{swap$_{#1,#2}$}}}
\newcommand{\VarS}[1]{\bT{\textsf{Var$_#1$}}}
\newcommand{\vVarS}[1]{\bT{\scriptsize \textsf{Var$_#1$}}}
\newcommand{\AppS}{\bT{\textsf{App}}}
\newcommand{\aAppS}{\bT{\scriptsize \textsf{App}}}
\newcommand{\LmS}[1]{\bT{\textsf{Lm$_#1$}}}
\newcommand{\lLmS}[1]{\bT{\scriptsize \textsf{Lm$_#1$}}}
\newcommand{\NilS}{\bT{\textsf{Nil}}}
\newcommand{\ConsS}[1]{\bT{\textsf{Cons$_#1$}}}
\newcommand{\UnS}{\bT{\textsf{Un}}}


%Isabelle keywords
\newcommand{\datatype}{\mbox{\textsc{Datatype}}}
\newcommand{\record}{\mbox{\textsc{Record}}}


%Isabelle theory names
\newcommand{\MyNats}{\mbox{\rm\textsf{My$\_$Nats}}}
\newcommand{\MyLists}{\mbox{\rm\textsf{My$\_$Lists}}}
\newcommand{\Terms}{\mbox{\rm\textsf{Terms}}}
\newcommand{\Closures}{\mbox{\rm\textsf{Closures}}}
\newcommand{\RulesSyntax}{\mbox{\rm\textsf{Rules$\_$Syntax}}}
\newcommand{\RulesSemantics}{\mbox{\rm\textsf{Rules$\_$Semantics}}}
\newcommand{\Bisimilarity}{\mbox{\rm\textsf{Bisimilarity}}}
\newcommand{\DerivedRules}{\mbox{\rm\textsf{Derived$\_$Rules}}}
\newcommand{\RawDeduction}{\mbox{\rm\textsf{Raw$\_$Deduction}}}
\newcommand{\Deduction}{\mbox{\rm\textsf{Deduction}}}

\newcommand{\HOASD}{\mbox{\rm\textsf{HOAS\_View\_D}}}
\newcommand{\HOAST}{\mbox{\rm\textsf{HOAS\_View\_T}}}
\newcommand{\Inf}{\mbox{\rm\textsf{Inference}}}
\newcommand{\HOASInf}{\mbox{\rm\textsf{HOAS\_Rep\_Inference}}}
\newcommand{\HOASWork}{\mbox{\rm\textsf{HOAS\_at\_Work}}}


%CCS stuff
\newcommand{\tTS}{\mbox{{\rm{\scriptsize TS$\,$}}}}
\newcommand{\cCCS}{\mbox{{\rm{\scriptsize CCS$\,$}}}}
\newcommand{\cCCST}{\mbox{{\rm{\scriptsize CCST$\,$}}}}
\newcommand{\del}{\mbox{\textit{del}}}
\newcommand{\ddel}{\mbox{\textit{\scriptsize del}}}
\newcommand{\sbis}{\sim}
\newcommand{\concat}{\mbox{ $\#$ }}


%lorenzos:

\newcommand\deq{\mathrel{\overset{\makebox[0pt]{\mbox{\normalfont\tiny\sffamily def}}}{=}}}
\newcommand{\cL}{{\cal L}}
\newcommand{\wlsPar}{\mbox{\rm\textsf{\small wlsPar}}}
\newcommand{\Iff}{\textsf{Iff}}
\newcommand{\llist}{\mbox{\bf list}}
\newcommand\thmcontinues[1]{Continued}
\newcommand{\wlsEnv}{\mbox{\rm\textsf{\small wlsEnv}}}
\newcommand{\sett}{\mbox{\bf set}}
\newcommand{\AND}{\mbox{\rm\textsf{\small AND}}}
\newcommand{\ALL}{\mbox{\rm\textsf{\small ALL}}}
\newcommand{\sAbs}{\mbox{\rm\textsf{\small sAbs}}}
\newcommand{\pAbs}{\mbox{\rm\textsf{\small pAbs}}}
\newcommand{\sOp}{\mbox{\rm\textsf{\small sOp}}}
\newcommand{\sX}{\mbox{\textit{sX}}}
\newcommand{\sA}{\mbox{\textit{sA}}}
\newcommand{\sWls}{\mbox{\rm\textsf{\small sWls}}}
\newcommand{\sAAbs}{\mbox{\bf sAbs}}
%\newcommand{\SEMM}{\mbox{\rm\textsf{\small SEM}}}
\newcommand{\semDom}{\mbox{\bf semDom}}
\newcommand{\recordd}{\mbox{\rm\textsf{record}}}
%\newcommand{\wlsSEM}{\mbox{\rm\textsf{\small wlsSEM}}}
\newcommand{\sWlsNE}{\mbox{\rm\textsf{\small sWlsNE}}}
\newcommand{\sWlsDisj}{\mbox{\rm\textsf{\small sWlsDisj}}}
\newcommand{\sOpPrSWls}{\mbox{\rm\textsf{\small sOpPrSWls}}}
\newcommand{\ga}{g_a}
\newcommand{\prWls}{\mbox{\rm\textsf{\small prWls}}}
\newcommand{\prWlsAbs}{\mbox{\rm\textsf{\small prWlsAbs}}}
\newcommand{\prVar}{\mbox{\rm\textsf{\small prVar}}}
\newcommand{\prAbs}{\mbox{\rm\textsf{\small prAbs}}}
\newcommand{\prOp}{\mbox{\rm\textsf{\small prOp}}}
\newcommand{\prFresh}{\mbox{\rm\textsf{\small prFresh}}}
\newcommand{\prSubst}{\mbox{\rm\textsf{\small prSubst}}}
\newcommand{\asIMOD}{\mbox{\rm\textsf{\small asIMOD}}}
\newcommand{\igWls}{\mbox{\rm\textsf{\small igWls}}}
%\newcommand{\ggWls}{\mbox{\rm\textsf{\small ggWls}}}
\newcommand{\igWlsAbs}{\mbox{\rm\textsf{\small igWlsAbs}}}
\newcommand{\igFreshAbs}{\mbox{\rm\textsf{\small igFreshAbs}}}
\newcommand{\igSwap}{\mbox{\rm\textsf{\small igSwap}}}
\newcommand{\igSwapAbs}{\mbox{\rm\textsf{\small igSwapAbs}}}
\newcommand{\igSubst}{\mbox{\rm\textsf{\small igSubst}}}
\newcommand{\igSubstAbs}{\mbox{\rm\textsf{\small igSubstAbs}}}
\newcommand{\igWlsAllDisj}{\mbox{\rm\textsf{\small igWlsAllDisj}}}
\newcommand{\igWlsAbsIsInBar}{\mbox{\rm\textsf{\small igWlsAbsIsInBar}}}
\newcommand{\igConsPresGWls}{\mbox{\rm\textsf{\small igConsPresGWls}}}
\newcommand{\igSubstAllPresGWlsAll}{\mbox{\rm\textsf{\small igSubstPresGWlsAll}}}
\newcommand{\igSubstAllPresGWls}{\mbox{\rm\textsf{\small igSubstPresGWls}}}
\newcommand{\igFreshCls}{\mbox{\rm\textsf{\small igFreshCls}}}
\newcommand{\igSubstCls}{\mbox{\rm\textsf{\small igSubstCls}}}
\newcommand{\igAbsRen}{\mbox{\rm\textsf{\small igAbsRen}}}
\newcommand{\igAbsT}{\mbox{\bf igAbs}}
\newcommand{\igTerm}{\mbox{\bf igTerm}}
\newcommand{\igOp}{\mbox{\rm\textsf{\small igOp}}}
\newcommand{\igAbs}{\mbox{\rm\textsf{\small igAbs}}}
\newcommand{\igVar}{\mbox{\rm\textsf{\small igVar}}}
\newcommand{\igFresh}{\mbox{\rm\textsf{\small igFresh}}}
\newcommand{\iwlsFSb}{\mbox{\rm\textsf{\small iwlsFSb}}}
\newcommand{\iwlsFSw}{\mbox{\rm\textsf{\small iwlsFSw}}}
\newcommand{\iwlsFSbSw}{\mbox{\rm\textsf{\small iwlsFSbSw}}}
\newcommand{\iwlsFSwSb}{\mbox{\rm\textsf{\small iwlsFSwSb}}}
\newcommand{\igWlsInp}{\mbox{\rm\textsf{\small igWlsInp}}}
\newcommand{\igWlsBinp}{\mbox{\rm\textsf{\small igWlsBinp}}}
\newcommand{\igFreshInp}{\mbox{\rm\textsf{\small igFreshInp}}}
\newcommand{\igFreshBinp}{\mbox{\rm\textsf{\small igFreshBinp}}}
\newcommand{\igSubstInp}{\mbox{\rm\textsf{\small igSubstInp}}}
\newcommand{\igSubstBinp}{\mbox{\rm\textsf{\small igSubstBinp}}}
\newcommand{\igAbsCongU}{\mbox{\rm\textsf{\small igAbsCongU}}}
\newcommand{\FSbImorph}{\mbox{\rm\textsf{\small FSbImorph}}}
\newcommand{\hA}{\mathit{hA}}
\newcommand{\ipresIGWlsAll}{\mbox{\rm\textsf{\small ipresIGWlsAll}}}
\newcommand{\ipresIGCons}{\mbox{\rm\textsf{\small ipresIGCons}}}
\newcommand{\ipresIGFreshAll}{\mbox{\rm\textsf{\small ipresIGFreshAll}}}
\newcommand{\ipresIGSubstAll}{\mbox{\rm\textsf{\small ipresIGSubstAll}}}
\newcommand{\termMOD}{\mbox{\rm\textsf{\small termMOD}}}
\newcommand{\substAbs}{\mbox{\rm\textsf{\small substAbs}}}
\newcommand{\gLam}{\mbox{\rm\textsf{\small g\hspace*{-0.1ex}Lam}}}
\newcommand{\igApp}{\mbox{\rm\textsf{\small ig\hspace*{-0.1ex}App}}}
\newcommand{\igLam}{\mbox{\rm\textsf{\small ig\hspace*{-0.1ex}Lam}}}
%\newcommand{\igAbs}{\mbox{\rm\textsf{\small ig\hspace*{-0.1ex}Abs}}}
%\newcommand{\igFresh}{\mbox{\rm\textsf{\small ig\hspace*{-0.1ex}Fresh}}}
\newcommand{\swap}{\mbox{\rm\textsf{\small swap}}}












\newcommand{\overbar}[1]{\mkern 1.5mu\overline{\mkern-1.5mu#1\mkern-1.5mu}\mkern 1.5mu}

%\newtheorem{theorema}[lemma]{Theorem}
%\newtheorem{corollarya}[lemma]{Corollary}

\allowdisplaybreaks


%\crefname{theorema}{theorem}{theorems}
%\Crefname{theorema}{Theorem}{Theorems}

%\crefname{corollarya}{corollary}{corllaries}
%\Crefname{corollarya}{Corollary}{Corollaries}

\allowdisplaybreaks
\setlength{\jot}{1pt}

\usepackage{thmtools}
\usepackage{thm-restate}
\usepackage{enumitem}

\declaretheorem[style=definition]{Example}

%Control spacing before and after theorems:

\makeatletter
\def\thm@space@setup{%
  \thm@preskip=1.22ex
  \thm@postskip=\thm@preskip % or whatever, if you don't want them to be equal
}
\makeatother

\begin{document}

%\title{Polymorphic HOL with Ad Hoc Overloading, Consistently} 
\title{A Formalized General Theory of Syntax with Bindings\thanks{This technical report is an extended version of the conference paper \cite{our-own-paper}. }
}
%
\author{Lorenzo Gheri\inst{1} \and Andrei Popescu\inst{1,2}}
\institute{
Department of Computer Science, 
%School of Science and Technology, 
Middlesex University London, UK
\and 
Institute of Mathematics Simion Stoilow of the Romanian Academy, Bucharest, Romania
}




\maketitle


\begin{abstract}
\vspace*{-5ex}
We present the formalization of a theory of syntax with bindings that has 
been developed and refined over the last decade to support several large formalization 
efforts. Terms are defined for an arbitrary number of constructors of varying 
numbers of inputs, quotiented to alpha-equivalence and sorted 
according to a binding signature. 
The theory includes a rich collection of properties of the standard operators  
on terms, such as substitution and freshness. %the freshness predicate. 
It also includes induction and recursion principles and support for semantic interpretation, 
all tailored for smooth interaction with the bindings and the standard operators.  %and the standard operators. 
%Finally, the theory offers support for interpreting terms in semantic domains 
%in an operator-compatible way. 
\end{abstract}






\vspace*{-5ex}
\section{Introduction}

Syntax with bindings is an essential ingredient in the formal specification and 
implementation of logics and programming languages. However, correctly and formally specifying, 
assigning semantics to,
and reasoning about bindings is notoriously difficult and error-prone. This fact is 
widely recognized in the formal verification community 
and is reflected in %the presence of 
manifestos and benchmarks 
such as the influential POPLmark challenge \cite{POPLmark}. 
 
In the past decade, in a framework developed intermittently 
starting with the second author's PhD \cite{pop-thesis} 
and moving into the first author's ongoing PhD, 
a series of results in logic and $\lambda$-calculus have been formalized in 
Isabelle/HOL \cite{nipkow-et-al-2002,nipkow-klein-2014}.  
These include classic results (e.g., FOL completeness and soundness of 
Skolemization \cite{blanchette-et-al-2014-ijcar,soundCompl-jou,blanchette-frocos2013}, 
$\lambda$-calculus standardization and Church-Rosser theorems \cite{pop-recPrin,pop-thesis}, System F 
strong normalization \cite{pop-HOASOnFOAS}), as well as 
%applications to 
the meta-theory of Isabelle's Sledgehammer 
tool \cite{blanchette-frocos2013,blanchette-et-al-2013-types}.  
 
In this paper, we present the Isabelle/HOL formalization of the framework itself (made available from
the paper's website~\cite{binding-scripts}). 
While concrete system syntaxes %such as the $\lambda$-calculus and FOL 
differ in their %concrete 
details, there are some fundamental phenomena 
concerning bindings %, substitution and semantic interpretation 
that follow the same generic  
principles. It is these fundamental phenomena that our framework aims to capture, 
by mechanizing a form of universal algebra for bindings. The framework has evolved 
over the years %growing %and improving 
through feedback from %the 
concrete 
application challenges: Each time a tedious, seemingly routine construction was encountered, 
a question arose as to whether this could be performed once and for 
all in a syntax-agnostic fashion. %, which often led to an addition to the general theory. 

The paper is structured as follows. We start with an example-driven overview of our design decisions %behind the framework 
(Section~\ref{sec-exa}). Then we present the general theory: %the construction of 
terms as alpha-equivalence classes of ``quasiterms,'' standard operators on terms 
and their basic properties (Section~\ref{GenSet}), 
custom induction and recursion schemes (Section~\ref{sec-reas}), 
including support for the semantic interpretation of syntax, and the sorting 
of terms according to a signature (Section~\ref{sec-sorting}). 
%%%
%Finally, we sketch the various applications 
%of the framework (Section~\ref{sec-app}). %, 
%%
%pointing out the usage of its various features. % in the vaarious case studies.
%A discussion of related work (Section 1234) concludes the paper.
%
Within the large body of formalizations in the area (Section~\ref{sec-RelWork}), 
distinguishing features of our work are the general setting (many-sorted signature, possibly infinitary syntax), 
a rich theory of the standard operators, and operator-aware recursion. 


     




%\section{Informal Description}
\section%{Motivating Examples and Discussion} 
{Design Decisions}
\label{sec-exa}


In this section, we use some examples to motivate our design choices for the theory. 
We also introduce conventions and notations that will be relevant throughout the paper. 


%\begin{Example}[name=Untyped $\lambda$-Calculus, label=exa:one] \label{lamEx} \rm 
The paradigmatic example of syntax with bindings is %of course 
that of the $\lambda$-calculus \cite{bar-lam}. 
We assume an infinite supply of variables, $x \in \var$. The $\lambda$-terms, 
$X,Y \in \term_\lambda$, are defined by the following BNF grammar:
%
$$
\begin{array}{lll}
X    &::=& \Var\;x \mid \App\;X\;Y  \mid \Lm\;x\;X
\end{array}
$$
%
Thus, a $\lambda$-term is either %(a copy of) 
a variable, or an application, or a $\lambda$-abstraction. 
%However, 
This grammar specification, while sufficient for first-order abstract syntax, 
%no longer tells the whole story 
is incomplete when it comes to syntax with bindings---% 
%To complete it, 
we also 
need to indicate which %of the 
operators introduce bindings and in which of their arguments. 
Here, $\Lm$ is the only binding operator: When applied to the variable 
$x$ and the term $X$, it binds $x$ in $X$. 
After knowing the binders, the usual convention is to {\em identify terms modulo alpha-equivalence}, 
i.e., to treat as equal terms that only differ in the names of %the 
bound variables, %. For example, 
such as, e.g., 
$\Lm\,\coll{x}\,(\App\;(\Var\,\coll{x})\;(\Var\;y))$ and 
$\Lm\,\coll{z}\,(\App\;(\Var\,\coll{z})\;(\Var\;y))$. % are being identified. 
%
The end results of our theory will involve terms modulo alpha.
%, 
%in that $\term_\lambda$ will be not the collection of 
%raw terms, but of alpha-equivalence classes---
We will call the raw terms 
``quasiterms,'' reserving the word ``term'' for alpha-equivalence classes.     
%The %the 
%high-level syntactic constructors and operators will be functions on terms, i.e., on alpha-classes of quasiterms. 


\subsection{Standalone Abstractions}
\label{prel-abs}

To make the binding structure manifest, we will ``quarantine'' the bindings and their associated 
intricacies into %a single out the 
the 
notion of {\em abstraction}, which 
is a pairing of a variable and a term, again modulo alpha. For example, for the $\lambda$-calculus we will have
$$
\begin{array}{cc}
X    \;::=\; \Var\;x \mid \App\;X\;Y  \mid \Lam\;A
\hspace*{6ex}&\hspace*{6ex}
A    \;::=\; \Abs\;x\;X
\end{array}
$$
%
where $X$ are terms and $A$ %are 
abstractions. Within $\Abs\;x\;X$, we assume that 
$x$ is bound in $X$. 
%Again, this induces a standard notion of alpha-equivalence, which by quotienting becomes 
%identity (on both syntactic categories). 
%For example, the two abstractions,  
%$\Abs\;x\;(\App\;(\Var\;x)\;(\Var\;y))$ and $\Abs\;z\;(\App\;(\Var\;z)\;(\Var\;y))$,  
%are identified. 
%Just like terms, abstractions are identified modulo alpha. 
%
%\par
The $\lambda$-abstractions $\Lm\;x\;X$ of the the original syntax     
%are captured by passing abstractions explicitly to $\Lam$, 
%e.g., 
are now written $\Lam\;(\Abs\;x\;X)$.   
%This abstraction-based alternative description of 
%bindings syntax is equivalent, but 
%has the advantage of ``quarantining'' %confining 
%the intricacies of alpha-equivalence 
%into a specialized syntactic category.  
%


\subsection{Freshness and Substitution}
\label{prel-freshSubst}

The two most fundamental and most standard operators on $\lambda$-terms are: %include: 
\begin{myitem}
\item the freshness predicate, $\fresh : \var \ra \term_\lambda \ra \bool$, 
where $\fresh\;x\;X$ states that $x$ is fresh for (i.e., does not occur free in) $X$; 
for example, it holds that $\fresh\;x\;(\Lam\;(\Abs \allowbreak\;x\;(\Var\;x)))$ and $\fresh\;x\;(\Var\;y)$ (when $x\not=y$), 
but not that $\fresh\;x\;(\Var\;x)$. 
\item the substitution operator, $\_[\_/\_] : \term_\lambda \ra \term_\lambda \ra \var \ra \term_\lambda$, 
where $Y\,[X/x]$ denotes the (capture-free) substitution of %the 
term $X$ for (all free occurrences of) 
%the 
variable $x$ in %the 
term $Y$; 
%for example, 
e.g., 
if $Y$ is $\Lam\;(\Abs\;x\;(\App\;(\Var\;x)\;(\Var\;y)))$ and $x \not\in \{y,z\}$, then: 
\begin{myitem}
\item $Y\,[(\Var\;z)/y] = \Lam\;(\Abs\;x\;(\App\;(\Var\;x)\;(\Var\;z)))$
\item $Y\,[(\Var\;z)/x] = Y$ (since bound occurrences like those of $x$ in $Y$ are not affected)
\end{myitem}
\end{myitem}
%
And there are corresponding operators for abstractions---e.g., $\freshAbs\;x\;(\Abs\;x\;(\Var\;x))$ holds. 
Freshness and substitution are pervasive in the meta-theory of $\lambda$-calculus, as well as in most logical 
systems and formal semantics of programming languages. The basic properties of these operators 
lay at the core of %all 
important meta-theoretic results in these fields---our formalized theory aims at the exhaustive 
coverage of these basic properties.  
%

  
\subsection{Advantages and Obligations from Working with Terms Modulo Alpha}

In our theory, we start with defining quasiterms and \abstractions{} and their alpha-equivalence. 
Then, after proving all the syntactic constructors and standard operators to be compatible with alpha, we quotient 
to alpha, obtaining what we call terms and abstractions, and define the versions of these operators on 
quotiented items. 
%
%For the $\lambda$-calculus example, 
For example, 
let $\qterm_\lambda$ and $\qabs_\lambda$ be the types  
of quasiterms and \abstractions{} in $\lambda$-calculus. Here, the \abstraction{} constructor, 
$\qAbs : \var \ra \qterm_\lambda \ra \qabs_\lambda$, 
is a free %(injective and non-overlapping) 
constructor, of the kind produced by standard datatype 
specifications \cite{berghofer-wenzel-1999,blanchette-et-al-2014-tru}. 
%
%Let $\equiv_\alpha$ denote alpha-equivalence for both types, 
%so that 
The types $\term_\lambda$ and $\abs_\lambda$ are $\qterm_\lambda$ and $\qabs_\lambda$ quotiented to alpha.  
We prove compatibility of $\qAbs$ with alpha %, namely, $X \equiv_\alpha Y \Lra \qAbs\;z\;X \equiv \qAbs\;z\;Y$; 
and then define $\Abs : \var \ra \term_\lambda \ra \abs_\lambda$ by lifting $\qAbs$ to quotients. 

The decisive advantages of working 
with quasiterms and \abstractions{} modulo alpha, i.e., with terms and abstractions, 
are that 
(1) substitution behaves well (e.g., is compositional) 
and 
(2) Barendregt's variable convention \cite{bar-lam} (of assuming, w.l.o.g.,   
the bound variables fresh for the parameters) can be invoked in proofs. 
%For example, if $x$ and $z$ are fresh for $X$ and $Z$,   
%then the equality $Y\,[X/x]\,[Z/z] = Y\,[Z/z]\,[X/x]$ holds on terms, but not on quasiterms. 

%
However, this choice brings the obligation to prove that all concepts on terms 
are compatible with alpha. Without employing suitable abstractions, 
this can become quite difficult 
even in the most ``banal'' contexts. %For example, 
Due to nonfreeness, primitive recursion 
%over quasiterms works 
%nicely, 
on terms requires a proof that the definition is well formed, i.e., that the overlapping 
cases lead to the same result. %\cite{pitts-AlphaStructural} 
%\cite[\S2]{pop-recPrin}. 
%
As for Barendregt's convention, its rigorous usage in %inductive 
proofs needs a principle that goes 
beyond the usual structural induction for free datatypes. 

A framework that deals gracefully with these obligations can make an important 
difference in applications---enabling the formalizer to quickly leave behind low-level ``bootstrapping'' issues 
%such as compatibility with alpha 
and move to the interesting core of the results. 
%Our theory will pay special attention to these delicate matters,
To address these obligations, we formalize %our theory incorporates 
state-of-the-art techniques from the literature \cite{pitts-AlphaStructural,UrbanTasson,pop-recPrin}. 



%bootstrap the definitions to move to the interesting 
%core of the result, and also to take the Barendregt convention 
%instead of engaging in tedious variable renamings. To support these, 
%our theory formalizes state-of-the-art techniques from the literature \cite{pitts-AlphaStructural,UrbanTasson,pop-recPrin}. 
%
%\end{Example}


\subsection{Many-Sortedness} \label{prel-manySorted}

While $\lambda$-calculus has only one syntactic category of terms (to which we 
added that of abstractions for convenience), this is often not the case. %For example, 
FOL has two: % categories: 
terms and formulas. The Edinburgh Logical Framework (LF) \cite{har-fra} has three: 
object families, type families and kinds. More complex calculi %and programming languages 
can have many syntactic categories. 

Our framework will %of course wish to 
capture these phenomena. 
We will call the syntactic categories {\em sorts}. 
We will distinguish syntactic categories for terms (the sorts) from those for variables (the {\em varsorts}). 
%We make this distinction because in many cases
%This captures cases when  
%not all term categories correspond to variable categories: 
Indeed, e.g., 
in FOL we do not have variables ranging over formulas,  
%(although they can be added in a second-order extension of FOL), 
in the $\pi$-calculus \cite{MilPiBook} we have channel names %/variables, 
but no process variables, etc.  
%The varsorts will be assumed to be included in the sorts, but not vice versa.

Sortedness is important, but formally quite heavy. 
In our formalization, we postpone dealing with 
%the bureaucracy of sortedness 
it for as long as possible. 
We introduce an 
intermediate notion of {\em good} term, for which we are able to 
build the bulk of the theory---only as the very last step we introduce many-sorted signatures
and transit from ``good'' to ``sorted.''
%, 
%prove that well-sorted terms in these signatures are 
%good, and transport all results to well-sorted terms. 




\subsection{Possibly Infinite Branching} \label{prel-infBranch}

Nominal Logic's \cite{pitts01nominal,UrbanTasson} 
notion of finite support %, popularized by Nominal Logic \cite{pitts01nominal,UrbanTasson}, 
has become central in state-of-the-art techniques for reasoning about bindings. 
Occasionally, however, important developments 
%take the liberty to 
%go beyond 
step outside finite support. For example, 
%here is a simplified version of the syntactic categories 
(a simplified) % version of 
CCS \cite{MilCCSBook} has the following syntactic categories 
of data expressions $E \in \expp$ and processes $P \in \procc$: 
%
$$
\begin{array}{cc}
E    \;::=\; \Var\;x \mid 0 \mid E + E
\hspace*{6ex}
&
\hspace*{6ex}
P    \;::=\; \Inp\;c\;x\;P \mid \Out\;c\;e\;P \mid %P | F \mid 
\sum_{i\in I}P_i 
\end{array}   
$$
%
Above, $\Inp\;c\;x\;P$, usually written $c(x).\,P$, is an input prefix $c(x)$ followed by a 
continuation process $P$, with $c$ being a channel %(from a fixed set) 
and $x$ a variable which is bound in $P$. Dually, 
$\Out\;c\;E\;P$, usually written $c\,\ov{E}.\,P$, is an output-prefixed process
% followed by a continuation process $P$, with $c$ a channel 
%as before 
with $E$ an expression. % (that can contain variables). 
%$|$ is parallel composition.  
The exotic constructor here is the sum $\sum$, which models nondeterministic choice from a collection $(P_i)_{i \in I}$ of 
alternatives indexed by a set $I$. It is important that $I$ is allowed to be infinite, 
%since this way one can model 
%in the calculus 
for modeling 
different decisions based on different received inputs. 
%However, this way 
But then process terms may use infinitely many variables, i.e., may not be finitely supported.  
Similar issues %situations 
arise %e.g., 
in infinitary FOL \cite{kei-mod} and Hennessey-Milner logic \cite{henessy-milner-logic}. 
%
In our theory, we cover such %possibly 
infinitely branching syntaxes.  
%---partly for capturing 
%examples like the above, partly from the gratuitiously scientific curiosity to see how the theory can be made 
%to work without finite support. 
%It turns out that the answer is provided by taking supplies of 
%variables in quantities prescribed by regular cardinals. %sufficiently large 
%when replacing this assumption with a more general assumption on cardinalities. 








\section%Many-Sorted Terms and Their Basic Operators} 
{General Terms with Bindings}
\label{GenSet}


We start the presentation of our formalized theory, in its journey 
from quasiterms (\ref{subsec-qterms}) to terms via alpha-equivalence (\ref{subsec-alpha}). 
The journey is fueled by the availability of fresh variables, 
ensured by cardinality assumptions on constructor branching 
 and variables (\ref{subsec-good}).   
%
%Having bootstrapped the term definition, we move to a 
%The journey 
It culminates with a systematic   
study of the standard term operators % and their properties 
(\ref{subsec-termsTh}). 


\subsection{Quasiterms}
\label{subsec-qterms}

The types $\qterm$ and $\qabs$, of quasiterms and \abstractions{}, are defined as mutually recursive datatypes 
polymorphic in the following type variables: $\indexx$ and $\bindexx$, of indexes for 
free and bound arguments, $\varsort$, of varsorts, i.e., sorts of variables, 
%, $\sort$, of sorts 
%(the last two distinguishing between syntactic categories of variables and terms, 
%as explained in Section~\ref{prel-manySorted}), 
and $\opsym$, of (constructor) operation symbols. 
For readability, below we omit the occurrences of these type variables as parameters to $\qterm$ and $\qabs$: 
%
$$
\begin{array}{rrcl}
\textsf{datatype } &
  \qterm &\;=\;& \qVar\;\varsort\;\var  \mid 
            \\
         &&&
   \qOp\;\opsym\;((\indexx,\qterm)\,\inputt)\;((\bindexx,\qabs)\,\inputt)
\\
\textsf{and } &
  \qabs &\;=\;& \qAbs\;\varsort\;\var\;\qterm
\end{array}
$$
%
\par
%
Thus, any quasiabstraction has the form $\qAbs\;\xs\;x\;X$, putting together the variable $x$ %assumed to be 
of varsort $\xs$ 
with the quasiterm $X$, indicating the binding of $x$ in $X$. 
On the other hand, 
a quasiterm is either an injection $\qVar\;\xs\;x$, of a variable $x$ of varsort $\xs$, 
or has the form $\qOp\;\delta\;\inp\;\binp$ , i.e., consists of 
an operation symbol %(freely) 
applied to %zero or more 
some inputs that can be either free, $\inp$, or bound, $\binp$. 
%Note that variables are injected into quasiterms together with a sort indication: $\qVar\;\xs\;x$ 
%is the variable $x$ of sort $\xs$  regarded as a quasiterm. 
%

We use $(\alpha,\beta)\,\inputt$ as a type synonym for $\alpha \ra \beta\;\option$, 
the type of partial functions from $\alpha$ to $\beta$; such a function returns either 
$\None$ (representing ``undefined'') or $\Some\;b$ for $b : \beta$. This type models 
inputs to the quasiterm constructors of varying number of arguments. An operation symbol 
$\delta : \opsym$ can be applied, via $\qOp$, to:
%
(1) a varying number of free inputs, i.e., 
families of quasiterms modeled as members of $(\indexx,\qterm)\,\inputt$
and 
(2) a varying number of bound inputs, i.e., 
families of \abstractions{} modeled as members of $(\indexx,\qabs)\,\inputt$.
%
%
%Above, ``any number'' means ``any number smaller than the cardinality of $\indexx$ or $\bindexx$.''
For example, taking $\indexx$ to be $\nat$ we capture $n$-ary operations 
for any %number 
$n$ (passing to $\qOp\;\delta$ inputs   
%that are 
defined only on $\{0,\ldots,n-1\}$), as well as as countably-infinitary operations 
(passing to $\qOp\;\delta$ inputs defined on the whole $\nat$). 

Note that, so far, we consider sorts of variables but not sorts of terms. 
The latter will come much later, in Section~\ref{sec-sorting}, when we introduce signatures. 
Then, we will gain 
control %, in the specification, 
(1) on which varsorts 
should be embedded in which term sorts and 
(2) on which operation symbols are allowed to be applied to which sorts 
of terms. % and to which numbers of arguments. 
But, until then, we will develop the interesting part of the theory of bindings without sorting the terms.
%---the varsorts being the only compromise we make to sorting. 
%This will be the task of signatures, introduced in Section~\ref{sec-sorting}.
%For this, we need information on the embedding of varsorts into sorts and on 
%the arities of the operation symbols. 
%To this end, we will introduce binding signatures, which will allow us to rule out non-well-formed entities. 






%However, since working untyped/unsorted is less bureaucratic, 
%we would like to introduce signatures and term sorting as late as possible in the development. 
On quasiterms, we define 
%It turns out that 
freshness, $\qFresh : \varsort \ra \var \ra \qterm \ra \bool$, 
%unary 
substitution, $\_[\_ / \_]_{\_} : \qterm \ra \qterm \ra \var \ra \varsort \ra \qterm$, 
parallel substitution, $\_[\_] : \qterm \ra (\varsort \ra \var \ra \qterm\;\option) \ra \qterm$, 
swapping, $\_[\_ \wedge \_]_{\_} : \qterm \allowbreak\ra \var \ra \var \ra \varsort \ra \qterm$, and alpha-equivalence, 
$\al : \qterm \ra \qterm \ra \bool$---and corresponding operators on \abstractions{}: $\qFreshAbs$, 
$\alAbs$, etc. 

The definitions proceed as expected, with picking suitable fresh 
variables in the case of substitutions and alpha. For parallel substitution, 
given a 
(partial) variable-to-quasiterm assignment $\rho: \varsort \ra \var \ra \qterm\;\option$, 
the quasiterm $X[\rho]$ is obtained by substituting, 
for each free variable $x$ of sort $\xs$ in $X$ for which $\rho$ is defined, the quasiterm $Y$ where  
$\rho\;\xs\;x = \Some\;Y$. 
%
%Below, 
We only show the formal definition of alpha. 



\subsection{Alpha-Equivalence}
\label{subsec-alpha}

\begin{figure}[t]
\hspace*{-1ex}
{\footnotesize
$
\begin{array}{rcl}
\al\;(\qVar\;\xs\;x)\;(\qVar\;\xs'\;x') &\iff&
   \xs = \xs' \;\wedge\; x = x'
\\
\al\,(\qOp\;\delta\;\inp\;\binp)\,(\qOp\;\delta'\;\inp'\;\binp') &\;\iff\;&
   \delta = \delta' \,\wedge\, \lift\,\al\;\inp\;\inp' \,\wedge\, \lift\,\alAbs\;\binp\;\binp' 
\\
\al\;(\qVar\;\xs\;x)\;(\qOp\;\delta'\;\inp'\;\binp') &\iff&
   \Ffalse
\\
\al\;(\qOp\;\delta\;\inp\;\binp)\;(\qVar\;\xs'\;x') &\iff&
   \Ffalse
\\
\alAbs\;(\qAbs\;\xs\;x\;X)\;(\qAbs\;\xs'\;x'\;X')  &\iff&
    \xs = \xs' \;\wedge\;(\exists y \notin \{x,x'\}.\;\qFresh\;\xs\;y\;X \;\wedge\;
\\
&&
 \qFresh\;\xs\;y\;X' \;\wedge\; \al\;(X [ y \wedge x]_{xs})\;(X' [ y \wedge x']_{xs}))
\end{array}
$
}
\vspace*{-2ex} %% TYPESETTING
\caption{Alpha-Equivalence} % (b) Wrong definition of $\l$}
\label{fig-alpha}
\vspace*{-4ex}
\end{figure}


We define the %alpha-equivalence 
predicates $\al$ (on quasiterms) and $\alAbs$ (on \abstractions{}) 
mutually recursively, as shown in Fig.~\ref{fig-alpha}.  For variable quasiterms, we require %plain 
equality on both the variables and their sorts. For $\qOp$ quasiterms, we %simply 
recurse through the components, $\inp$ and $\binp$. Given any predicate $P : \beta^2 \ra \bool$, we write 
$\lift\,P$ for its lifting to $(\alpha,\beta)\,\inputt^2 \ra \bool$, 
defined as $\lift\,P\;\inp\;\inp' \iff \forall i.\;\mbox{ case }(\inp\;i,\inp'\;i) \mbox{ of } 
(\None,\None) \allowbreak \Ra \Ttrue \mid(\Some\;b,\,\Some\;b') \Ra P\;b\;b' \mid\_ \Ra \Ffalse$.
Thus, $\lift\,P$ relates two inputs  
just in case they have the same domain and their %returned 
results are componentwise 
related.

\begin{conv}\rm
%To lighten notation, 
Throughout this paper, we write $\lift$ for the 
natural 
lifting of the various operators from terms and abstractions to free or bound inputs. 
%These include alpha-equivalence, as well as freshness and substitution (on either 
%terms or quasiterms). 
\end{conv}

%The most specific
In Fig.~\ref{fig-alpha}'s clause for \abstractions{}, we require that the bound variables 
are of the same sort and %that 
there exists some fresh $y$ such that $\al$ holds 
for the terms where $y$ is swapped with the bound variable. Following %insight from 
Nominal Logic, 
we prefer to use swapping instead of substitution in alpha-equivalence, % in the definition of alpha, 
since this 
leads to simpler proofs. 




\subsection{Good Quasiterms and Regularity of Variables}
\label{subsec-good}

%Although definable for unsorted quasiterms, 
In general, % (without adding any restriction), 
%Without any further assumptions 
$\al$ will not be an equivalence, namely, % on quasiterms, more precisely, 
will not be transitive: Due to the arbitrarily wide branching of the constructors, 
we may not always have fresh variables $y$ 
available in an attempt to prove transitivity by induction. 
%
To remedy this, we restrict ourselves to ``good'' quasiterms, whose constructors do not branch  
beyond the cardinality of $\var$. Goodness is defined as the 
mutually recursive predicates 
$\qGood$ and $\qGoodAbs$: % by the following equations:
%
$$
\begin{array}{rcl}
\qGood\;(\qVar\;\xs\;x) &\iff&
   \Ttrue
\\
\qGood\;(\qOp\;\delta\;\inp\;\binp)  &\;\iff\;&
   \lift\,\qGood\;\inp \;\wedge\; \lift\,\qGoodAbs\;\binp \;\wedge\;
\\ &&
|\dom\;\inp| < |\var| \;\wedge\; |\dom\;\binp| < |\var|
\\
\qGoodAbs\;(\qAbs\;\xs\;x\;X) &\iff&
    \qGood\;X
\end{array}
$$
%
where, given a partial function $f$, we write $\dom\;f$ for its domain. 
%


%\subsection{Regular Cardinality for Variables}
%\label{subsec-varReg}

Thus, for good items, we hope to always have a supply of fresh variables. Namely, we
hope to prove %this:
$
\qGood\;X \Lra \forall \xs.\;\exists x.\;\qFresh\;\xs\;x\;X
$. 
But goodness is not enough. 
We also need a special property for the type $\var$ of variables. 
In the case of finitary syntax, it suffices to take $\var$ to be 
countably infinite, since a finitely branching term will contain fewer than $|\var|$ variables 
(here, meaning a finite number of them)---this can be proved by induction on terms, 
%taking advantage of 
using 
the fact that a finite union of finite sets is finite. 

So let us attempt to prove the same in our general case. 
%prove the same property for good terms, 
In the inductive $\qOp$ case, we know from goodness 
that the branching is smaller than $|\var|$, so to conclude we would need the following: 
{\em A union of sets smaller than $|\var|$ indexed by a set smaller than $|\var|$ 
stays smaller than $|\var|$.} It turns out that this is a well-studied property 
of cardinals, called {\em regularity}---with $|\nat|$ being the smallest regular cardinal. 
%
Thus, the desirable generalization of countability is regularity 
(which is available from Isabelle's cardinal library \cite{cardHOL}). 
%For the rest of the paper, all theorems will implicitly assume:
Henceforth, we will assume: 

\begin{ass}\rm \label{ass-reg}
$|\var|$ is a regular cardinal. 
\end{ass}

%We can now prove a strengthening of the needed property---
We will thus have not 
only one, but a $|\var|$ number of fresh variables: 

\begin{prop}\rm
%Assume $|\var|$ is a regular cardinal. 
$\qGood\;X \Lra \forall \xs.\;|\{x.\;\qFresh\;\xs\;x\;X\}| = |\var|$
\end{prop}


Now we can prove, for good items, the properties of alpha familiar from 
the $\lambda$-calculus, including it being an equivalence and an alternative formulation 
of the abstraction case, where ``there exists a fresh $y$'' is replaced with 
``for all fresh $y$.'' While the ``exists'' variant is useful when proving 
that two terms are alpha-equivalent, the ``forall'' variant gives stronger inversion and induction 
rules for proving implications from $\al$. 
(Such fruitful ``exsist-fresh/forall-fresh,'' or ``some-any'' dychotomies have been previously discussed in the context of bindings, e.g, in \cite{DBLP:conf/tphol/NorrishV07,aydemirPOPL08,MillTiu-proofThGenJudg}.) 


\begin{prop}\rm \label{lem-alpha}
The following hold: 
\\(1) $\al$ and $\alAbs$ are equivalences on good quasiterms and \abstractions{}
%the sets $\{X : \qterm \mid \qGood\;X\}$ and $\{X : \qabs \mid \qGoodAbs\;A\}$ 
\\(2) The predicates defined by replacing, in Fig.~\ref{fig-alpha}'s definition, the abstraction 
case with  
$$
\begin{array}{c}
\alAbs\;(\qAbs\;\xs\;x\;X)\;(\qAbs\;\xs'\;x'\;X')  \;\iff
\\
    \xs = \xs' \wedge (\coll{\forall}\! y \notin \{x,x'\}.\, \qFresh\,\xs\,y\,X \wedge
 \qFresh\,\xs\,y\,X' \!\coll{\Lra}\! \al (X [ y \wedge x]_{xs}) (X' [ y \wedge x']_{xs})) 
\end{array}
$$
%
coincide with $\al$ and $\alAbs$. 
%
\end{prop}




\subsection{Terms and Their Properties}\label{subsec-termsTh} 


We define $\term$ and $\abs$ as collections %sets 
of $\al$- and $\alAbs$- equivalence classes of $\qterm$ and $\qabs$. 
Since $\qGood$ and $\qGoodAbs$ are compatible with $\al$ and $\alAbs$, we %can 
lift them %these predicates 
to corresponding predicates on terms and abstractions, $\good$ and $\goodAbs$. 
%For them, we prove that they satisfy 
%the same clauses as those defining the quasiterm and \abstraction{} version (from Section ).
%We also define the predicates $\good$ and $\goodAbs$ for terms and abstractions, perfectly similarly to how we defined 
%$\qGood$ and $\qGoodAbs$ for quasiterms and \abstractions{} in Section1 234. 
%Thus, good terms are alpha-equivalence classes of quasiterms that do not branch beyond the cardinality of $\var$.  
%
\leftOut{
It is routine to show that, through the canonical projections 
$\proj : \qterm \ra \term$ and $\projAbs : \qabs \ra \abs$, good items correspond to good quasiitems:

\begin{prop}\rm \label{lem-good}
For all quasiterms $X$ and \abstractions{} $A$, it holds that 
$\qGood\;X \iff \good (\proj\;X)$ and $\qGoodAbs\;A \iff \goodAbs\;(\projAbs\;A)$
\end{prop}
}

%Point (1) of the above lemma allows us to define the types $\term$ and $\abs$ 
%as quotients of $\{X : \qterm \mid \good\;X\}$ 
%and $\{X : \qabs \mid \goodAbs\;A\}$ under $\al$ and $\alAbs$. 

We also prove that all constructors and operators 
are alpha-compatible,   
%and preserve goodness, 
which allows lifting them to terms: 
%For example, we have 
$\Var : \varsort \ra \var \ra \term$, 
$\Op : \opsym \ra (\indexx,\term)\allowbreak\inputt \ra (\bindexx,\abs)\,\inputt \ra \term$, % and 
$\Abs : \varsort \ra \var \ra \term \ra \abs$, %as well as 
$\fresh : \varsort \ra \term \ra \bool$, $\_[\_ / \_]_{\_} : \term \ra \term \ra \var \ra \varsort \ra \term$, etc. 

\leftOut{
\begin{myitem}
\item The syntactic constructors:
\begin{myitem} 
\item embedding of variables, $\Var : \varsort \ra \var \ra \term$
\item application of operation symbol, $\Op : \opsym \ra (\indexx,\term)\inputt \ra (\bindexx,\abs)\inputt \ra \term$
\item constructor of abstractions, $\Abs : \varsort \ra \var \ra \term \ra \abs$
\end{myitem}
\item The standard operators: 
\begin{myitem}
\item freshness, $\fresh : \varsort \ra \term \ra \bool$, 
\item substitution, $\_[\_ / \_]_{\_} : \term \ra \term \ra \var \ra \varsort \ra \term$, 
\item swapping, $\_[\_ \wedge \_]_{\_} : \term \ra \var \ra \var \ra \varsort \ra \term$
\end{myitem}
\end{myitem}

Note that, even though the terms are not sorted, the operators contain varsort 
information. For example, $Y[X/x]_{xs}$ will substitute, in $Y$, $X$ for all occurrences of 
$x$ with sort $x$, i.e., as $\Var\;\xs\;x$. We need this information in order to 
the operators to behave as expected later, when we sort the terms. 
} % end leftOut

To establish an abstraction barrier that sets terms free from their quasiterm origin, 
we prove that the syntactic constructors 
mostly behave %as if they were 
like %ordinary datatype 
free constructors, in that $\Var$, $\Op$ and $\Abs$ are exhaustive and 
$\Var$ and $\Op$ are injective and nonoverlapping. %, in that a $\Var$-term is never equal to an $\Op$-term 
%
True to the quarantine principle expressed in Section~\ref{prel-abs},  
the only nonfreeness incident occurs for %the abstraction constructor, 
$\Abs$. 
%``concession'' we make to the binding structure is that $\Abs$ is noninjective.
Its equality behavior 
is regulated by the ``exists fresh'' and ``forall fresh'' 
properties inferred from the definition of $\alAbs$ and Prop.~\ref{lem-alpha}(2), respectively: 

\begin{prop}\rm \label{lem-Abs}
Assume $\good\;X$ and $\good\;X'$. Then the following are equivalent:
\\(1) $\Abs\;\xs\;x\;X = \Abs\;\xs'\;x'\;X'$ 
\\(2) $\xs = \xs' \,\wedge\,(\exists y \notin \{x,x'\}.\;\fresh\;\xs\;y\;X \,\wedge\,
 \fresh\;\xs\;y\;X' \,\wedge\, X [ y \wedge x]_{xs} = X' [ y \wedge x']_{xs})$
\\(3) $\xs = \xs' \,\wedge\,(\forall y \notin \{x,x'\}.\;\fresh\;\xs\;y\;X \,\wedge\,
 \fresh\;\xs\;y\;X' \Lra X [ y \wedge x]_{xs} = X' [ y \wedge x']_{xs})$
%
\end{prop}

%For characterizing abstraction equality, substitution can be used 
%instead of swapping. In particular, %the following 
Useful rules for abstraction equality also hold with %variable-for-variable 
substitution:
%equality introduction and bound-variable renaming rules proceed by means of substituting with 
%fresh variables:
%--- in Section~\ref{sec-RecDef}, these rules are shown to be essential  
%in characterizing terms among all ``models'' of freshness and substituion.

\begin{prop}\rm \label{lem-Abs-subst}
Assume $\good\;X$ and $\good\;X'$. Then the following hold: 
\\(1) $y \notin \{x,x'\} \,\wedge\, \fresh\;\xs\;y\;X \,\wedge\,
 \fresh\;\xs\;y\;X' \,\wedge\, X\,[(\Var\;\xs\;y)\,/\,x]_{xs} = X'\,[(\Var\;\xs\;y)\,/\,x']_{xs} \Lra \Abs\;\xs\;x\;X = \Abs\;\xs\;x'\;X'$ 
%
\\(2) $\fresh\;\xs\;y\;X \;\Lra\; \Abs\;\xs\;x\;X = \Abs\;\xs\;y\;(X\,[(\Var\;\xs\;y)\,/\,x]_{xs})$
\end{prop}
 

%Moreover, 
To completely seal the abstraction barrier, for all the standard operators we prove 
simplification rules regarding their interaction with the constructors, 
which makes the former behave as if they had been defined in terms of the latter. 
For example, the following facts resemble an inductive definition of freshness (as a predicate): 

\begin{prop}\label{lem-imp-fresh}\rm
Assume $\good\;X$, %$\good\;Y$, 
$\lift\,\good\;\inp$, $\lift\,\good\;\binp$, 
$|\dom\;\inp| < |\var|$ and $|\dom\;\binp| < |\var|$. 
The following hold:
%
\\(1) $(\ys,y) \not= (\xs,x) \,\Lra\, \fresh\;\ys\;y\;(\Var\;\xs\;x)$
%
\\(2) $\lift\,(\fresh\;\ys\;y)\;\inp \,\wedge\, \lift\,(\freshAbs\;\ys\;y)\;\binp \;\Lra\; 
\fresh\;\ys\;y\;(\Op\;\delta\;\inp\;\binp)$
%
\\(3) $(\ys,y) = (\xs,x)\,\vee\, \fresh\;\ys\;y\;X \;\Lra\; \freshAbs\;\ys\;y\;(\Abs\;\xs\;x\;X)$
\end{prop}

Here and elsewhere, when dealing with $\Op$, we make cardinality 
assumptions on the domains of the inputs to make sure the  
terms $\Op\;\delta\;\inp\;\binp$ are good. 

We can further improve on Prop.~\ref{lem-imp-fresh}, obtaining ``iff'' facts that resemble a primitively recursive 
definition of freshness (as a function): 

\begin{prop}\label{lem-simp-fresh}\rm
Prop.~\ref{lem-imp-fresh} stays true if the implications %($\!\Lra\!$) 
are replaced by equivalences ($\!\iff\!$). 
\end{prop}

For substitution, we prove facts with a similarly primitive recursion flavor:

\begin{prop}\label{lem-simp-subst}\rm
Assume $\good\;X$, $\good\;Y$, $\lift\,\good\;\inp$, $\lift\,\good\;\binp$, 
$|\dom\;\inp| < |\var|$ and $|\dom\;\binp| \allowbreak< |\var|$. 
The following hold:
%
\\(1) $(\Var\;\xs\;x)\,[Y/y]_{ys} \;=\; (\mbox{if $(\xs,x) = (\ys,y)$ then $Y$ else $\Var\;\xs\;x$})$ 
%
\\(2) $(\Op\;\delta\;\inp\;\binp)\,[Y/y]_{ys} = 
\Op\;\delta\;(\lift\,(\_[Y/y]_{ys})\,\inp)\;(\lift\,(\_[Y/y]_{ys})\,\binp)$ 
%
%where $\inp\,[Y/y]_{ys}$ denotes $\lambda i.\;
%\mbox{case $\inp\;i$ of $\None$ $\Ra$ $\None$ $\mid$ $\Some\;X$ $\Ra$ $X[Y/y]_{ys}$}$, and similarly for $\binp$
%
\\(3) $(\xs,x) \not= (ys,y) \;\wedge\; \fresh\;\xs\;x\;Y
\;\Lra\; 
(\Abs\;\xs\;x\;X)\,[Y/y]_{ys} \,=\, \Abs\;\xs\;x\;(X\,[Y/y]_{ys})
$
\end{prop}
 
We also prove generalizations of Prop.\ \ref{lem-simp-subst}'s facts for parallel substitution, for example, 
%
$
\lift\,(\fresh\;\xs\;x)\;\rho \allowbreak \Lra 
(\Abs\;\xs\;x\;X)\,[\rho] \,=\, \Abs\;\xs\;x\;(X\,[\rho])
$. 
%
%where $\fresh\;\xs\;x\;\rho$ means $\forall \ys,y,Y.\;\rho\;\ys\;y = \Some\;Y \,\Lra \,
%(\xs,x) \not= (ys,y) \,\wedge\, \fresh\;\xs\;x\;Y$. 


Note that, for properties involving $\Abs$, the simplification rules %typically 
require 
%a freshness assumption for 
freshness of the bound variable: 
$\freshAbs\;\ys\;y\;(\Abs\;\xs\;x\;X)$ is reducible to $\fresh\;\ys\;y\;X$ 
only if $(\xs,x)$ is distinct from $(\ys,y)$, 
$(\Abs\;\xs\;x\;X)\,[Y/y]_{ys}$ is expressible in terms of $X\,[Y/y]_{ys}$ 
only if $(\xs,x)$ is distinct from $(\ys,y)$ and fresh for $Y$, etc. 
%Therefore, to achieve smooth (quasi-automatic) proofs of facts involving these operators, we would like 
%to be able to assume freshness---we discuss this is Section~\ref{Ind}.  

Finally, we %also 
prove lemmas that regulate the interaction between the standard operators, in all possible combinations: 
freshness versus swapping, freshness versus substitution, substitution versus substitution, etc. 
%For example:
Here are a few samples: 
%

\begin{prop}\label{lem-long}\rm
If the terms $X,Y,Y_1,Y_2,Z$ are $\good$ and the assignments $\rho,\rho'$ are $\lift\,\good$,   
%
%The following hold:
then: 
\\(1) Swapping distributes over all operators, including, e.g., substitution:
%
$$Y\,[X/x]_{xs}\,[z_1 \wedge z_2]_{zs} \;=\; (Y\,[z_1 \wedge z_2]_{zs})\,[(X[z_1 \wedge z_2]_{zs})\,/\,(x[z_1 \wedge z_2]_{xs,zs}) ]_{xs}$$
%
where $x[z_1 \wedge z_2]_{xs,zs} \;=\; (\mbox{if $\xs = \zs$ then $x[z_1 \wedge z_2]$ else $x$})$
%
\\(2) Substitution of the same variable (and of the same varsort) distributes over itself: 
%
$$
X\ [Y_1 / y]_{ys}\, [Y_2 / y]_{ys} \;=\; X\,[(Y_1\,[Y_2 / y]_{ys}) / y]_{ys}
$$
%
\noindent
(3) Substitution of different variables distributes over itself, assuming %suitable 
freshness: % conditions:
%
$$(\ys \neq \zs\ \lor\ y \neq z) \,\wedge\, \fresh\;\ys\;y\;Z
\;\Lra\; 
X\,[Y / y]_{ys}\, [Z / z]_{zs} \,=\, (X\,[Z / z]_{zs})\ [(Y\,[Z / z]_{zs}) / y]_{ys}
$$
%
\noindent
(4) Freshness for a substitution decomposes into freshness for its participants: 
%
$$\fresh\;\zs\;z\;(X[Y / y]_{ys}) 
\iff
((\zs,z)=(\ys,y) \lor \fresh\;\zs\;z\;X) \,\land\,  (\fresh\;\ys\;y\;X \lor \fresh\;\zs\;z\;Y)$$
%
\noindent
\leftOut{
(5) Parallel substitution distributes over unary substitution:
$$
X\,[Y / y]_{ys}\,[\rho] \,=\, X\,[\rho\,[y \la Y[\rho]]_{ys}]
$$
where $\rho[y \la Y[\rho]]$ is the assignment $\rho$ updated with value $\Some\,(Y[\rho])$ for $y$. 
%
\\
} %end leftOut
(5) Parallel substitution is compositional: %is compositional: % with a single one over the composed assignment: 
$$X\,[\rho]\,[\rho'] \;=\; X\,[\rho \bullet \rho']$$
where $\rho \bullet \rho'$ is the monadic composition of $\rho$ and $\rho'$, defined as 
%
$$
(\rho \bullet \rho')\,\xs\;x \,=\; \mbox{case $\rho\;\xs\;x$ of $\None$ $\Ra$ $\rho'\;\xs\;x$ $\mid$ $\Some\;X$ $\Ra$ $X[\rho']$}
$$
\end{prop}


In summary, we have formalized quite exhaustively the general-purpose properties of all syntactic constructors 
and standard operators. %\footnote{We challenge 
%the reader to contact us with any property we may have failed to cover \cite{binding-scripts}.}
%of these standard operators that is not covered by our 
%formalization \cite{binding-scripts}.} 
%
Some of these properties are %quite 
subtle. In formalization of concrete results for particular syntaxes, 
they are likely to require a lot of time to even formulate them correctly, let alone prove them---which would be 
wasteful, 
since they are independent on the particular syntax.
%, 
%it is wasteful to restate and reprove them in each particular development. 





\section{ Reasoning and Definition 
%Induction and Recursion 
Principles} \label{sec-reas} 

%Terms introduce a ``modulo alpha'' level of abstraction level on top of quasiterms. 
%Next,  
We formalize schemes for induction (\ref{Ind}), recursion and semantic interpretation (\ref{sec-RecDef}) that 
%live up to this abstraction, %by 
%accommodae 
realize the Barendregt %freshness 
convention and are compatible 
with the standard operators. % on alpha-classes.  

\subsection{Fresh Induction} \label{Ind}

We introduce fresh induction by an example. 
To prove Prop.~\ref{lem-long}(4), we use (mutual) structural induction over terms and abstractions, 
proving the statement together with the corresponding statement for abstractions,  
%
$\freshAbs\;\zs\;z\;(A[Y / y]_{ys}) 
\iff
((\zs,z) = (\ys,y) \lor \freshAbs\;\zs\;z\;A) \,\land\,  (\freshAbs\;\ys\;y\;A \lor \fresh\;\zs\;z\;Y)
$. 
%
The proof's only interesting case 
is the $\Abs$ case, say, for abstractions of the form 
$\Abs\;\xs\;x\;X$. 
However, 
if we were able to assume freshness of $(\xs,x)$ for all the statement's parameters, 
namely $Y$, $(\ys,y)$ and $(\zs,z)$,  
this case would also become ``uninteresting,'' following automatically from the induction hypothesis 
by mere 
simplification, as shown below (with the freshness assumptions highlighted):  % (via our simplification rules for the standard operators). %, illustrated in Prop.~\ref{lem-simp}). 
%Here are the simplification steps, where we highlight the freshness assumptions 
%without which simplification would get stuck:
%
$$
\begin{array}{l}
\freshAbs\;\zs\;z\;((\Abs\;\xs\;x\;X)\,[Y/y]_{ys}) \\
  \Updownarrow \mbox{\small (by Prop.~\ref{lem-simp-subst}(3), since 
\coll{\mbox{$(\xs,x) \not= (\ys,y)$ and $\fresh\;\xs\;x\;Y$)}}
} \\
\freshAbs\;\zs\;z\;(\Abs\;\xs\;x\;(X\,[Y/y]_{ys})) \\
  \Updownarrow \mbox{\small (by Prop.~\ref{lem-simp-fresh}(3), since \coll{\mbox{$(\xs,x) \not= (\zs,z)$)}}} \\
\fresh\;\zs\;z\;(X\,[Y/y]_{ys}) \\
  \Updownarrow \mbox{\small (by Induction Hypothesis)} \\
((\zs,z) = (\ys,y) \lor \fresh\;\zs\;z\;X) \,\land\,  (\fresh\;\ys\;y\;X \lor \fresh\;\zs\;z\;Y) \\
  \Updownarrow \mbox{\small (by Prop.~\ref{lem-simp-fresh}(3) applied twice, 
since \coll{\mbox{$(\xs,x) \not= (\zs,z)$}} and \coll{\mbox{$(\xs,x) \not= (\ys,y)$)}} } \\
((\zs,z) = (\ys,y) \lor \freshAbs\;\zs\;z\;(\Abs\;\xs\;x\;X)) \,\land\,  (\freshAbs\;\ys\;y\;(\Abs\;\xs\;x\;X) \lor \fresh\;\zs\;z\;Y)
\end{array} 
$$

The practice of assuming freshness, known in the literature as the Barendregt convention, 
is a hallmark in informal reasoning about bindings. 
Thanks to insight from Nominal Logic 
\cite{pitts-AlphaStructural,UrbanTasson,urban-Barendregt}, 
we also know how to apply this morally correct convention fully rigorously. 
To capture it in our formalization, we %first 
model parameters 
$p : \param$ as anything that allows for a notion of freshness, or, alternatively, 
provides a set of (free) variables for each varsort, 
$\varsOf : \param \ra \varsort \ra \var\;\sett$. 
With this, a ``fresh induction'' principle can be formulated,  
%An induction principle 
%with the assumption of freshness w.r.t.\ parameters  
%holds provided 
if all parameters have fewer variables than $|\var|$ (in particular, if they have only finitely many). 
%$\varsOf\;p$ has cardinality smaller than $\var$ (in particular, if $\varsOf\;p$ is finite). 

  
\begin{thm}\label{th-fresh-induct} \rm
Let $\phi : \term \ra \param \ra \bool$ and $\phiAbs : \abs \ra \param \ra \bool$. % and 
Assume: % the following: % hold:
\\(1) $\forall \xs,p.\;|\varsOf\;\xs\;p| < |\var|$
\\(2) $\forall \xs,x,p.\;\phi\;(\Var\;\xs\;x)\;p$
\\(3) $\forall \delta,\inp,\binp,p.\;  
  |\dom\;\inp| < |\var| \,\wedge\, |\dom\;\binp| < |\var|  \,\wedge\,
\lift\,(\lambda X.\;\good\;X \,\wedge\,  (\forall q.\,\phi\;X\;q))\;\allowbreak\inp  \,\wedge\, 
\lift\,(\lambda A.\;\goodAbs\;A \,\wedge\, 
(\forall q.\,\phiAbs\;A\;q))\,\binp \Lra \phi\,(\Op\;\delta\;\inp\;\binp)\;p$
\\(4) $\forall \xs,x,X,p.\;\good\;X \,\wedge\, \phi\;X\;p \,\wedge\, 
\coll{x \not\in \varsOf\;\xs\;p} \,\Lra\, \phiAbs\,(\Abs\;\xs\;x\;X)\;p$

Then $\forall X,p.\;\good\;X \Lra \phi\;X\;p$ and 
$\forall A,p.\;\goodAbs\;A  \Lra \phiAbs\;A\;p$. % hold.  
\end{thm} 

Highlighted is the essential difference from the usual structural induction: The bound 
variable $x$ can be assumed fresh for the parameter $p$ (on its varsort, $\xs$). 
Note also that, in the $\Op$ case, we lift to inputs the predicate 
%Finally, what is carried over inductively in the $\Op$ case is the predicate 
as quantified universally over all parameters.  

\leftOut{ %maybe for the end product
To make it easier to instantiate to commonly used parameters, we prove a corollary 
for custom parameters that may contain variables, terms, abstractions and assignments 
of variables to terms:
%
$$
\textsf{datatype }\param \;=\; \Par\;(\varsort \ra \var\;\llist)\;(\term\;\llist)
\;(\abs\;\llist)\;((\var \ra \term\;\option)\;\llist)
$$
%
and define $\varsOf\;p$ to take 
}

Back to Prop.~\ref{lem-long}(4), this follows automatically by fresh induction 
(plus the shown simplifications), after recognizing as parameters the variables $(\ys,y)$ and 
$(\zs,z)$ and the term $Y$---formally, 
taking $\param = (\varsort \times \var)^2 \times \term$ and 
$\varsOf\;\xs\;((\ys,y),(\zs,z),\allowbreak Y) = \{y \mid \xs = \ys\} \,\cup\, 
\{z \mid \xs = \zs\}  \,\cup\, 
\{x \mid \neg\,\fresh\;\xs\;x\;Y\}$. 
%for all $(L,X):\param$.
%, 
%and $P$ to be the singleton 
%$\{p\}$, where $p = (\{(\ys,y), (\zs,z)\},Y)$. 


 
\subsection{Freshness- and Substitution- Sensitive Recursion}\label{sec-RecDef}
%motivation but not too much: semantic domain interpretation
%theorem1
%theorem2

A {\em freshness-substitution (FS) model} consists of two collections of elements endowed 
with term- 
and abstraction- like operators satisfying some 
characteristic properties of terms. 
%
More precisely, it consists of: 
\begin{myitem}
\item two types, $\T$ and $\A$ %and a set $\Mod : \model\;\sett$
\item operations corresponding to the %term 
constructors: $\VAR : \varsort \ra \var \ra \T$, 
$\OP : \opsym \allowbreak\ra (\indexx,\T)\;\inputt \ra (\bindexx,\A)\;\inputt \ra \T$, 
$\ABS : \varsort \ra \var \ra \T \ra \A$
\item operations corresponding to freshness and substitution:
$\FRESH : \varsort \ra \var \ra \T \ra \bool$, $\FRESHABS : \varsort \ra \var \ra \A \ra \bool$, 
$\_[\_/\_]_\_ : \T \ra \T \ra \var \ra \varsort \ra \T$ and 
$\_[\_/\_]_\_ : \A \ra \T \ra \var \ra \varsort \ra \A$
\end{myitem}
%
and it is required to satisfy the analogues of:
\begin{myitem}
\item the implicational simplification rules for $\fresh$ from Prop.~\ref{lem-imp-fresh} 
\\(for example, $(\ys,y) \not= (\xs,x) \,\Lra\, \FRESH\;\ys\;y\;(\VAR\;\xs\;x)$)
\item the simplification rules for substitution from Prop.~\ref{lem-simp-subst}
\item the substitution-based abstraction equality rules from Prop.~\ref{lem-Abs-subst}
\end{myitem}


\begin{thm}\label{th-rec} \rm
The good terms and abstractions form the initial FS model. Namely, 
for any FS model as above, there exist the functions $\f:\term \ra \T$ 
and $\fAbs : \abs \ra \A$ that commute, on good terms, 
with the constructors and with substitution and preserve freshness: 
%
$$
\begin{array}{ll}
\f\,(\Var\;\xs\;x) = \VAR\;\xs\;x
&\hspace*{2ex}
\f\,(\Op\;\delta\;\inp\;\binp) = \OP\;\delta\;(\lift\,\f\;\inp)\;(\lift\,\fAbs\;\binp)
\\
\fAbs\,(\Abs\;\xs\;x\;X) = \ABS\;\xs\;x\;(f\;X)
&\hspace*{2ex}
\\
\f\,(X\,[Y/y]_{ys}) = (\f\;X)\,[(\f\;Y)/y]_{ys} 
&\hspace*{2ex}
\fAbs\,(A\,[Y/y]_{ys}) = (\fAbs\;A)\,[(\f\;Y)/y]_{ys} 
\\
\fresh\;\xs\;x\;X \,\Lra\, \FRESH\;\xs\;x\;(\f\;X) 
&\hspace*{2ex}
\freshAbs\;\xs\;x\;A \,\Lra\, \FRESHABS\;\xs\;x\;(\fAbs\;A)
\end{array}
$$
%
\par
In addition, the two functions are uniquely determined on good terms and abstractions, 
in that, for all other functions $\g:\term \ra \T$ 
and $\gAbs :\abs \ra \A$ satisfying the same commutation and preservation properties, 
it holds that 
$\f$ and $\g$ are equal on good terms and $\fAbs$ and $\gAbs$ are equal on good abstractions.
%that commute, on good terms, 
%with the constructors and with substitution and preserve freshness
\end{thm}  

Like any initiality property, this theorem represents a primitive recursion principle. 
%To see this, 
Consider first the simpler case of lists over a type $\G$, with constructors $\Nil : \G\;\llist$ 
and $\Cons : \G \ra \G\;\llist \ra \G\;\llist$. To define, by primitive recursion, 
a function from lists, say, $\llength : \G\;\llist \ra \nat$, we need to indicate 
what is $\Nil$ mapped to, here $\llength\;\Nil = 0$, and, recursively, what 
is $\Cons$ mapped to, here $\llength\;(\Cons\;a\;\as) = 1 + \llength\;\as$. 
We can rephrase this by saying: If we define ``list-like'' operators on the target domain---
here, taking $\NIL : \nat$ to be $0$ and $\CONS : \G \ra \nat \ra \nat$ to be $\lambda g,n.\;1+n$---then 
%in exchange 
%In exchange for the user providing $\NIL$ and $\CONS$, 
the recursion principle offers us a function $\llength$ that commutes with the constructors: 
$\llength\;\Nil = \NIL = 0$ and $\llength\,(\Cons\;a\;\as) = \CONS\;a\;(\llength\;\as) = 1 + \llength\;\as$. 
%
For terms, we have a similar situation, except that (1) substitution and freshness 
are considered in addition to the constructors and 
(2) paying the price for lack of freeness, some conditions need to be verified  
to %qualify 
deem the operations ``term-like.''  

This recursion principle was discussed in \cite{pop-recPrin} for the %particular 
syntax of $\lambda$-calculus 
and shown to have many useful applications.   
Perhaps the most useful one is the seamless interpretation 
of syntax in semantic domains, in a manner that is guranteed to be 
compatible with alpha, substitution and freshness. We formalize this in our general setting:

A {\em semantic domain} consists of two collections of elements endowed with interpretations 
of the $\Op$ and $\Abs$ constructors, the latter 
in a higher-order fashion---interpreting variable binding as (meta-level) functional binding. 
Namely, it consists of:
\begin{myitem}
\item two types, $\Dt$ and $\Da$ 
\item %an operation interpretation 
a function $\op : \opsym \ra (\indexx,\Dt)\;\inputt \ra (\bindexx,\Da)\;\inputt \ra \Dt$ 
\item %a higher-order abstraction interpretation
 a function $\abss : \varsort \ra (\Dt \ra \Dt) \ra \Da$
\end{myitem} 


\begin{thm}\label{th-sem} \rm
The terms and abstractions are %naturally 
interpretable in any semantic domain. 
Namely, if $\val$ is the type of valuations of %the 
variables in the domain, 
$\varsort \ra \var \ra \Dt$, 
there exist the functions 
$\sem : \term \ra \val \ra \Dt$  
and $\semAbs : \abs \ra \val \ra \Da$ such that:
\begin{myitem}
\item $\sem\,(\Var\;\xs\;x)\,\rho = \rho\;\xs\;x$
\item $\sem\,(\Op\;\delta\;\inp\;\binp)\,\rho = \op\;\delta\;
  (\lift\,(\lambda X.\,\sem\;X\;\rho)\,\inp)\;(\lift\,(\lambda A.\,\semAbs\;A\;\rho)\,\binp)$
\item $\semAbs\,(\Abs\;\xs\;x\;X)\,\rho = \abss\;\xs\,(\lambda d.\;\sem\;X\;(\rho[(\xs,x) \la d]))$
\end{myitem}
%
\par
In addition, the interpretation functions map syntactic substitution and freshness to semantic versions of the 
concepts: 
\begin{myitem}
\item $\sem\,(X[Y/y]_{ys})\,\rho = \sem\;X\;(\rho[(\ys,y) \la \sem\;Y\;\rho])$
\item $\fresh\;\xs\;x\;X \;\Lra\;(\forall \rho,\rho'.\;\rho=_{(xs,x)}\rho' \,\Lra\, \sem\;X\;\rho = \sem\;X\;\rho')$,  
\\where ``$=_{(xs,x)}$'' means equal everywhere but on $(\xs,x)$ 
\end{myitem}
\end{thm}

Theorem~\ref{th-sem} is the foundation for many particular semantic interpretations, including 
that of $\lambda$-terms in Henkin models and that of FOL terms and formulas in FOL models. 
It guarantees compatibility with alpha and proves, as bonuses, a freshness and a substitution 
property.   
The freshness property is nothing but the notion that the interpretation 
only depends on the free variables, whereas the substitution property generalizes 
what is usually called {\em the substitution lemma}, stating that interpreting a substituted term 
is the same as interpreting the original term in a ``substituted'' environment. 

This theorem follows by an instantiation of the recursion Theorem \ref{th-rec}: 
taking $\T$ and $\A$ to be $\val \ra \Dt$ and $\val \ra \Da$ and taking the 
term/abstraction-like operations as prescribed by the desired clauses for $\sem$ and $\semAbs$---e.g., 
$\VAR\;\xs\;x$ is $\lambda \rho.\;\rho\;\xs\;x$. 
%The characteristic properties 
%required by Theorem \ref{th-rec} amount to straightforward properties of 
%valuation functions. %, such as TODO







\section{Sorting the Terms}  \label{sec-sorting}  


So far, we have %built 
a framework where the operations take as 
free and bound inputs partial families of terms and abstractions.  
%indexed by two fixed types  ($\indexx$ and $\bindexx$). 
All theorems refer to good (i.e., sufficiently 
low-branching) terms and abstractions. 
%
%
However, we promised a theory that is applicable to terms over many-sorted binding signatures. 
%
Thanks to the choice of a flexible notion of input, it is not hard 
to cast our results into such a many-sorted setting. 
Given a suitable 
notion of signature (\ref{subsec-sign}), 
we classify terms according to sorts (\ref{subsec-termsSig}) 
and prove that well-sorted terms are good (\ref{subsec-goodToSort})---this 
gives us sorted versions of all theorems (\ref{subsec-endProd}). 


\subsection{Binding Signatures} \label{subsec-sign}

%We will employ binding signatures to specify the arities and result sorts of operation symbols 
%and the embeddings of varsorts into sorts. 
%

A {\em (binding) signature} is a tuple
%
$(\indexx,\bindexx,\varsort,\sort,\opsym,\asSort,\stOf,\arOf,\allowbreak\barOf)$, 
%
where $\indexx$, $\bindexx$, $\varsort$ and $\opsym$ are types (with the previously discussed intuitions) and 
$\sort$ is a new type, of sorts for terms. Moreover: 
\begin{myitem}
\item $\asSort : \varsort \ra \sort$ is an injective map, embedding varsorts into sorts
%(this is the inclusion/injection of varsorts into sorts discussed in Section~\ref{prel-manySorted})
\item $\stOf : \opsym \ra \sort$, read ``the (result) sort of''
\item $\arOf : \opsym \ra (\indexx,\sort)\,\inputt$, read ``the (free) arity of"
\item $\barOf : \opsym \ra (\bindexx,\varsort \times \sort)\,\inputt$, read ``the bound arity of" % (``barity of", for short)
\end{myitem}

Thus, %in addition to fixing the types of varsorts, sorts and operation symbols, 
a signature prescribes which varsorts correspond to which sorts (as discussed in Section~\ref{prel-manySorted}) 
and, for each operation symbol, 
which are the sorts of its free inputs (the arity), of its bound (abstraction) inputs (the bound arity),  
and of its result.
%The index types $\indexx$ and $\bindexx$ are auxiliary types used for 
%creating families and inputs, and in concrete cases they need to be large enough for the desired 
%(b)arities---for finitary syntaxes, taking them to be $\nat$ is sufficient. 

When we give examples for our concrete syntaxes in Section~\ref{sec-exa}, 
we will write $(i_1\mapsto a_1,\ldots,i_n\mapsto a_n)$ for the partial 
function that sends each $i_k$ to $\Some\;a_k$ and everything else to $\None$. In particular, $()$ 
denotes the totally undefined function. 

For the $\lambda$-calculus syntax, we take $\indexx = \bindexx = \nat$, $\varsort = \sort = \{\lamterm\}$ (a singleton datatype), 
$\opsym = \{\App,\Lam\}$, % (a two-element datatype), 
$\asSort$ to be the identity and $\stOf$ to be the unique function to $\{\lamterm\}$. 
Since $\App$ has two free inputs and no bound input, we use the first two elements of $\nat$ as free arity 
and nothing for the 
bound arity:
$\arOf\;\App =(0\mapsto \lamterm,\,1\mapsto\lamterm)$, 
$\barOf\;\App = ()$. 
%
By contrast, since $\Lam$ has no free input and one bound input, we use nothing for the 
free arity, and the first element of $\nat$ for the bound arity: 
$\arOf\;\Lam = ()$, 
$\barOf\;\Lam = (0\mapsto (\lamterm,\lamterm))$. 

For the CCS example in Section~\ref{prel-infBranch}, %the same section, 
we fix a type $\chan$ of channels. 
%
We choose a cardinal upper bound $\kappa$ for the branching of sum ($\sum$), and choose a type 
$\indexx$ of cardinality $\kappa$. %---since we have a well-order on $\indexx$, we can speak of the elements $0,1$, etc. $\in \indexx$. 
%
For $\bindexx$, we do not need anything special, so we take it to be $\nat$. 
We have two sorts, of expressions and processes, 
so we take $\sort = \{\exp,\proc\}$.  
Since we have expression variables but no process variables, we take $\varsort = \{\varexp\}$ and $\asSort$ to send 
$\varexp$ to $\exp$. 
We define $\opsym$ as the following datatype:
%
%$$\textsf{datatype } 
$\opsym \; =\; \Zero \mid \Plus \mid \Inp\;\chan \mid \Out\;\chan \mid {\textstyle \sum}\,(\indexx\;\sett)$. 
%
%Note that the $\Inp$ and $\Out$ constructors are parameterized by channels as an ``external'' parameter. 
%Similarly, $\sum$ is parameterized by an index set. 
%All these effectively make 
%for an infinite number of operation symbols. 
%
The free and bound arities and sorts of the operation symbols are as expected. For example, 
$\Inp\;c$ acts 
similarly to $\lambda$-abstraction, but binds, in $\proc$ terms, variables of a different sort, $\varexp$: 
$\arOf\,(\Inp\;c) = ()$, 
$\barOf\,(\Inp\;c) = (0\mapsto (\varexp,\proc))$. 
%
For $\sum I$ with $I:\indexx\;\sett$, the arity is only defined for elements of $I$, namely  
%
$\arOf\;(\sum I) = ((i\in I)\mapsto \proc)$. 
%$\barOf\;(\sum I) = ()$. 


\subsection{Well-Sorted Terms Over a Signature}
\label{subsec-termsSig}

Based on the information from a signature, we can distinguish our terms of interest, 
namely those that are well-sorted in the sense that:
\begin{myitem}
\item all %their 
variables are embedded into terms of sorts compatible with their varsorts 
\item all %their 
operation symbols are applied according their free and bound arities
\end{myitem}

This is modeled by well-sortedness predicates $\wls: \sort \ra \term \ra \bool$ and 
$\wlsAbs: \varsort \ra \sort \ra \abs \ra \bool$, where $\wls\;s\;X$ 
states that $X$ is a well-sorted term of sort $s$ and $\wlsAbs\;(\xs,s)\;A$ 
states that $A$ is a well-sorted abstraction binding an $\xs$-variable in an $s$-term. 
%
They are defined mutually inductively by the 
following clauses: 
$$\begin{array}{rcl}
&&\wls\;(\asSort\;\xs)\;(\Var\;\xs\;x)
\\
\lift\,\wls\;(\arOf\;\delta)\;\inp \,\wedge\, \lift\,\wlsAbs\;(\barOf\;\delta)\;\binp &\;\LRA\;&
 \wls\;(\stOf\;\delta)\;(\Op\;\delta\;\inp\;\binp)
\\
\isInBar\,(\xs,s) \,\wedge\, \wls\;s\;X &\;\LRA\;& \wlsAbs\;(\xs,s)\;(\Abs\;\xs\;x\;X)
\end{array}
$$
%
where $\isInBar\,(\xs,s)$ states that the pair $(\xs,s)$ is in the bound arity of at 
least one operation symbol $\delta$, i.e., $\barOf\;\delta\;i = (\xs,s)$ for some $i$---
this %restriction 
rules out unneeded abstractions. 

%recognizes %the fact 
%that abstractions are not useful in themselves, 
%but only as bound inputs for terms---so it rules out unused abstractions. %the unused pairs varsort-sort. 
%again, we use $\lift$ to denote the lifting of predicates to inputs and bound inputs. 
%For example, $\lift\,\wls\;(\arOf\;\delta)\;\inp$ means that $\dom\;(\arOf\;\delta) = \dom\;\inp$ 
%and $\forall\,i\;X\;s.\;\arOf\;\delta\;i = \Some\;s \;\wedge\; \inp\;i = \Some\;X \LRA \wls\;s\;X$. 


%
Let us illustrate sorting for our running examples. 
In the $\lambda$-calculus syntax, 
let $X = \Var\;\lam\;x$, 
$A = \Abs\;\lam\;x\;X$, and 
$Y = \Op\;\Lam\;()\;(0 \mapsto A)$. 
These correspond to what, in the unsorted BNF notation from Section~\ref{prel-abs}, 
we would write $\Var\;x$, $\Abs\;x\;X$ and $\Lam\;(\Abs\;x\;X)$. 
In our sorting system, $X$ and $Y$ are both well-sorted terms at sort $\lam$ 
(written $\wls\;\lam\;X$ and $\wls\;\lam\;Y$) and 
$A$ is a well-sorted abstraction at sort $(\lam,\lam)$ (written $\wlsAbs\,(\lam,\lam)\,A$). 
%By contrast, e.g., $\Op\;\Lam\;(0\mapsto X)\;()$ is not well-sorted, 
%because it contains an attempt to apply $\Lam$ to a free argument, violating $\Lam$'s arity. 


For CCS, we have that 
$E = \Op\;\Zero\,()\,()$ and 
$F = \Op\;\Plus\;(0\mapsto E,\;1\mapsto E)\,()$ are well-sorted terms of sort $\exp$. 
Moreover, $P = \Op\,(\sum \emptyset)\,()\,()$ and 
$Q = \Op\,(\Out\;c)\,\allowbreak(0\mapsto F,1 \mapsto P)\,()$ are well-sorted terms of sort $\proc$. 
(Note that $P$ is a sum over the empty set of choices, i.e., the null process, whereas 
$Q$ represents a process that outputs the value of $0+0$ on channel $c$ and then stops.)
%
If, e.g., 
we swap the arguments of $\Out\;c$ in $Q$, we obtain 
$\Op\,(\Out\;c)\,(0 \mapsto P,1\mapsto F)\,()$, which is not well-sorted: 
In the inductive clause for $\wls$, the input $(0 \mapsto P,1\mapsto F)$ fails 
to match the arity of $\Out\;c$, %which is 
$(0 \mapsto \exp,1\mapsto \proc)$.  


\subsection{From Good to Well-Sorted}
\label{subsec-goodToSort}
  
Recall that goodness means ``does not branch beyond $|\var|$.'' On the other hand, 
well-sortedness imposes that, for each applied operation symbol $\delta$, 
its inputs have same domains, 
i.e., {\em  only branch as much}, as the arities of $\delta$. Thus, it suffices 
to assume the arity domains smaller than $|\var|$. 
%In fact, 
We will more strongly assume that the types of sorts and indexes 
(the latter subsuming the arity domains) 
are 
all smaller than $|\var|$: 

\begin{ass}\rm \label{ass-varLarge}
$|\sort| < |\var| \;\wedge\; |\indexx| < |\var| \;\wedge\; |\bindexx| < |\var|$ 
\end{ass}

Now we can prove:
%
\begin{prop} \label{lem-wls-good}
$(\wls\;s\;X \Lra \good\;X) \;\wedge\; (\wls\;(\xs,s)\;A \Lra \goodAbs\;A)$
\end{prop}

In addition, we prove that all the standard operators preserve well-sortedness. 
For example, we prove that if we substitute, in the well-sorted term $X$ of sort $s$, 
for the variable $y$ of varsort $\ys$, the well-sorted term $Y$ of sort 
corresponding to $\ys$, then we obtain a well-sorted term of sort $s$: 
%
$\wls\;s\;X \;\wedge\; \wls\;(\asSort\;\ys)\;Y \;\Lra\; \wls\;s\;(X\,[Y/y]_{ys})$. 

Using the preservation properties and 
Prop.~\ref{lem-wls-good}, we transfer the entire  
theory %described in 
of Sections \ref{subsec-termsTh} and \ref{sec-reas} from good terms to well-sorted terms---e.g., 
Prop.~\ref{lem-long}(2) becomes: 
%
$$
\begin{array}{c}
\coll{\wls\;s\;X \,\wedge\, \wls\;(\asSort\;\ys)\;Y_1 \,\wedge\, \wls\;(\asSort\;\ys)\;Y_2} 
\Lra
X\,[Y_1 / y]_{ys}\,[Y_2 / y]_{ys} = \ldots %X\,[(Y_1\,[Y_2 / y]_{ys}) / y]_{ys}
\end{array}
$$

The transfer is mostly straightforward for all facts, including the induction theorem. 
(For stating the well-sorted version of the recursion and semantic interpretation theorems, 
there is some additional bureaucracy since we also need sorting predicates on the target domain%as well
---Appendix~\ref{app-RecDef} gives details.) 


%
There is an important remaining question: Are our two Assumptions 
(\ref{ass-reg} and \ref{ass-varLarge}) satisfiable? That is, 
can we find, for any types $\sort$, $\indexx$ and $\bindexx$, 
a type $\var$ larger than these such that $|\var|$ is regular? 
%
Fortunately, the theory of cardinals 
again provides us with a positive answer: 
Let $\G = \nat + \sort + \indexx + \bindexx$. 
Since %we know that 
any successor of an infinite cardinal 
is regular, we can 
take $\var$ to have the same cardinality as the successor of $|\G|$, by defining 
$\var$ as a suitable subtype of $\G\;\sett$. % (known to be large enough for the purpose). 
In the case of all operation symbols being finitary, i.e., with 
their arities having finite domains, we do not need the above fancy construction,  
but can simply take $\var$ to be a copy of $\nat$.  

%to be a type of cardinality the successor
%a type of such cardinality. More 
%precisely, we carve $\var$ out, via typedef, from $(\nat + \sort + \indexx + \bindexx)\;\sett$, with $\nat$ 
%ensuring infiniteness---and this is precisely what we do in our formalization, providing a desirable 
%type $\var$. 


\subsection{End Product}
\label{subsec-endProd}

All in all, %the end product of 
our formalization provides a theory of syntax with bindings over an arbitrary many-sorted signature.
%, 
%featuring a rich set of properties of the standard operators (freshness, unary and parallel substitution and swapping) 
%and induction and recursion principles. 
The signature is formalized as an Isabelle locale \cite{Locales} 
that fixes 
the types $\var$, $\sort$, $\varsort$, $\indexx$, $\bindexx$ and $\opsym$ 
and the constants $\asSort$, $\arOf$ and $\barOf$ and 
assumes the injectivity of $\asSort$ and the $\var$ 
properties (Assumptions \ref{ass-reg} and \ref{ass-varLarge}).     
%(A suitable type $\var$ is constructed from the existing types, as described in \ref{subsec-goodToSort}.)
All end-product theorems are placed %contained %proved 
in this locale. 


The whole formalization 
%is very large, 
%reflecting the fact that our theory was not built for achieving a fixed target result, 
%but for providing a rich theory for later use. Oftentimes, we were happy to provide 
%small variations of a given result, e.g., induction schema, knowing that this is 
%slightly more convenient to use. 
%It 
consists of 22700 lines of code (LOC). Of these, 3300 LOC are dedicated 
to quasiterms, their standard operators and alpha-equivalence. 
3700 LOC are dedicated to the definition of terms and the lifting of results from quasiterms.
Of the latter, the properties of substitution were the most extensive---2500 LOC out of 
the whole 3700---since substitution, unlike freshness and swapping, 
requires heavy variable renaming, which complicates the proofs. 

The induction and recursion schemes presented in Section~\ref{sec-reas} are  
not the only schemes we formalized (but are the most useful ones). 
We also proved a variety of lower-level induction schemes 
based on the skeleton of the terms (a generalization of depth for possibly infinitely branching terms) 
and schemes that are easier to instantiate---e.g., by pre-instantiating Theorem~\ref{th-fresh-induct}   
with commonly used parameters such as variables, terms and environments. 
As for the recursion Theorem~\ref{th-rec}, we additionally proved a more flexible scheme that allows 
the recursive argument, and not only the recursive result, to be referred---this is 
%essentially 
{\em full-fledged primitive recursion}, whereas Theorem \ref{th-rec} only implements {\em iteration}. 
%
Also, we proved schemes for recursion that factor swapping \cite{norrish-MechanisingLambdaInFirstOrder} 
instead of and in addition to substitution. %, 
%but also swapping %(a la Michael Norrish  %).   
All together, these constitute 8000 LOC. %(The appendix gives more details.)

The remaining 7700 LOC of the formalization are dedicated to transiting from good terms 
to sorted terms. Of these, 3500 LOC are taken by the sheer statement 
of our many end-product theorems. Another fairly large part, 2000 LOC, is dedicated to 
transferring all the variants of the recursion 
Theorem~\ref{th-rec} and the interpretation Theorem~\ref{th-sem}, which require conceptually straightforward
but technically tedious moves back and forth 
between sorted terms and sorted elements of the target domain.   

%As expected, throughout the development automation worked well only after enabling 
%the high-level recursion and induction principles. 











\section{Discussion,  
Related Work and Future Work} \label{sec-RelWork} 


There is a large amount of literature on formal approaches to syntax with bindings. 
(See \cite[\S2]{POPLmark}, \cite[\S6]{momFelty-Hybrid4}  and \cite[\S2.10,\S3.7]{pop-thesis} for overviews.)  
%Here, we only discuss work 
%that is most related to ours, splitting the discussion according to 
%our main contributions. 
%
Our work, nevertheless, fills a gap in the literature: 
It is the first theory of binding syntax mechanized in 
a universal algebra fashion, i.e., with sorts and many-sorted term constructors specified 
by a binding signature,  
%, of a kind considered in several non-mechanized general   settings 
as employed in several theoretical developments, e.g., \cite{fio-abs,pitts-AlphaStructural,sun-alg,DBLP:journals/tcs/0001R15}.  
The universal algebra aspects 
of our approach are the consideration of an {\em arbitrary signature} and the singling out of 
the collection of terms and the operations on them 
as an {\em initial object in a category of models/algebras} (which yields a recursion principle). 
We do not consider arbitrary equational varieties (like in \cite{sun-alg}), 
but only focus on selected equations and Horn clauses that characterize the term 
models (like in \cite{pitts-AlphaStructural}). 

{\bf Alternatives to Universal Algebra}  
A popular alternative to our universal algebra approach is higher-order abstract syntax (HOAS) 
\cite{har-fra,weakHOAS,momFelty-Hybrid4,chlipala-Parametric,gun-proper}: 
the reduction of all 
bindings to a single binding---that of a fixed $\lambda$-calculus. 
Compared to universal algebra, HOAS's %obvious 
advantage is lighter formalizations, 
whereas the disadvantage is the need to prove the representation's adequacy (which 
involves reasoning about substitution) and, in some frameworks, 
the need to rule out the resulted junk (also known as exotic terms).  

Another alternative, very successfully used in HOL-based provers such as HOL4 \cite{slind-norrish-2008} and Isabelle/HOL, 
is the ``package'' approach: Instead of deeply embedding sorts and 
operation symbols like we do, packages take a user a specification 
of the desired types and operations and prove all the theorems for that instance (on a dynamic basis).  
%(but following a universal algebra ``blueprint''). 
Nominal Isabelle \cite{urban-2008,urbanGeneralBinders} is 
a popular such package, which implements terms with bindings for Isabelle/HOL. 
From a theoretical perspective, a universal algebra theory has a wider appeal, as it models 
``statically'' the meta-theory in its whole generality. However, a package 
is more practical, since most proof assistant users only care about the particular instance 
syntax used in their development. 
In this respect, simply instantiating our signature with the particular syntax is 
not entirely satisfactory, since it is not sufficiently ``shallow''---e.g., one would like to have actual operations such as $\Lam$     
instead of applications of $\Op$ to a $\Lam$ operation symbol, 
and would like to have actual types, such as $\expp$ and $\procc$, instead 
of the well-sortedness predicate applied to sorts, $\wls\;\exp$ and $\wls\;\proc$. 
%
For our applications, so far we have 
manually transited from our ``deep'' signature instances 
to the more usable shallow version sketched above.  
In the future, we plan to have this transit process automated, obtaining the best of both worlds, 
namely a universal algebra theory that also acts as a {\em statically certified} package. 
(This approach has been prototyped for a %much 
smaller theory: that of 
nonfree equational datatypes \cite{schropp-nonfree}.)

 



{\bf Theory of Substitution and Semantic Interpretation}  
%
%A major motivation for
The main goal of our work was the development of 
as much as possible from the theory of syntax 
for an arbitrary syntax. 
%in a syntax-independent fashion. 
To our knowledge, 
none of the existing frameworks provides %syntax-independent 
support for substitution 
and the interpretation of terms in semantic domains at this level of generality. 
Consequently, formalizations for concrete syntaxes, even those based on 
sophisticated packages such as Nominal Isabelle or the similar 
tools and formalizations in Coq \cite{nominalCoq,aydemirPOPL08,Hirschowitz:2012}, have to redefine these standard concepts 
and prove their properties over and over again---an unnecessary consumption 
of time and brain power. 



{\bf Induction and Recursion Principles}   
% 
There is a rich literature on these topics, which are connected to the quest, 
pioneered by 
Gordon and Melham \cite{gor-5axAlpha}, 
of understanding terms with bindings modulo alpha as an abstract datatype. 
We formalized the Nominal structural induction principle 
from \cite{pitts-AlphaStructural}, which is also implemented in Nominal Isabelle. 
By contrast, we did not go after 
the Nominal recursion principle. Instead, we chose to stay more faithful to the 
abstract datatype desideratum,  
generalizing to an arbitrary syntax 
our own schema for substitution-aware recursion \cite{pop-recPrin} and  
Michael Norrish's schema for swapping-aware recursion \cite{norrish-MechanisingLambdaInFirstOrder}---both 
of which can be read as stating that terms with bindings are Horn-abstract datatypes, i.e., 
are initial models of certain Horn theories 
\cite[\S3,\S8]{pop-recPrin}. 
%What is currently missing from our theory 
%is support for fresh rule induction in the style of Urban et al.~\cite{urban-Barendregt}.
%---although arguably this notion goes be
 

{\bf Generality of the Framework} 
%
Our constructors are restricted to binding at most one variable 
in each input---a limitation that makes our framework far from ideal for representing 
complex binders such as the let patterns of POPLmark's Challenge 2B.     
In contrast, the specification language Ott \cite{ott-tool} 
and 
Isabelle's Nominal2 package \cite{urbanGeneralBinders} were specifically designed 
to address such complex, possibly recursive binders.  
Incidentally, the Nominal2 package also separates abstractions from terms, like we do, but their abstractions 
are significantly more expressive; their terms are also quotiented to  
alpha-equivalence, which is defined via flattening the binders into 
finite sets or lists of variables (atoms).   
      
On the other hand, to the best of our knowledge, our formalization is the first to capture infinitely branching terms 
and our foundation of alpha equivalence on the regularity of $|\var|$ 
is also a theoretical novelty---constituting a less exotic alternative 
to Murdoch Gabbay's work on infinitely supported objects in nonstandard set theory \cite{Gabbay:2007}.
%formalizing a connection between 
%which have not been 
%formally studied before. 
This flexibility would be needed to formalize 
calculi such as  
infinite-choice process algebra, for which 
infinitary structures 
%such as cylindric algebras \cite{cylindric} 
have been previously employed to give semantics \cite{Lut02}.



{\bf Future Generalizations and Integrations}  
%
Our theory currently addresses mostly {\em structural} aspects of terms. A next step would be 
to cover {\em behavioral} aspects, such as formats for SOS rules and their interplay with binders, 
perhaps building on existing Isabelle formalizations of process algebras and programming languages 
(e.g., \cite{pop-coind,prob-nonint,conc-nonint,DBLP:conf/popl/NipkowO98,lochbihler-2012,psiInIsa}). 
  
Another exciting prospect is the integration of our framework with Isabelle's recent  
package for inductive and coinductive datatypes \cite{blanchette-et-al-2014-tru} 
based on bounded natural functors (BNFs),  
which follows a compositional design \cite{traytel-et-al-2012}
and provides flexible ways to nest types \cite{nonuniform-lics2017} and mix recursion with 
corecursion \cite{fouco,amico}, but does not yet cover terms with bindings. 
Achieving compositionality in the presence of bindings will require a substantial 
refinement of the notion of BNF (since terms with bindings form only partial functors w.r.t.\ their sets of free variables). 






  


\leftOut{
%\section{Conclusion}
\ \\
{\bf Conclusion} 
We presented %what constitutes, to our best knowledge, 
the first mechanization  
of a theory of bindings over a many-sorted signature. %universal algebra of 
%bindings in a proof assistant. 
This formed the basis of several applications of 
formalized meta-theory for concrete syntaxes,    
which 
\href{http://curiosamathematica.tumblr.com/post/151061725677/q-what-do-you-call-someone-reading-a-category}
{so far have been developed by the second author 
and his coauthors. %collaborators  
We hope that the presentation of our framework 
and its underlying principles and design decisions 
will facilitate its use by non-coauthors too.}
}

\smallskip
\paragraph*{\small Acknowledgment}  

\small
We thank the
anonymous reviewers for suggesting textual improvements.
%
%
Popescu has received funding from UK's Engineering and Physical Sciences Research Council 
(EPSRC) via the grant EP/N019547/1, Verification of Web-based  Systems (VOWS).     


\bibliographystyle{splncs03}
\bibliography{bib}{}

%\end{document}

\newpage
\section{Dataset Visualizations}
\label{sec:app_dataset_visuals}

%%%%%%
%%
%%
\subsection{Examples of each view class}
\newcommand{\BC}{0.33}
\setlength{\tabcolsep}{0.1cm}
\begin{figure}[!h]
\begin{tabular}{c c c c}
    PLAX  & PSAX & OTHER 
    \\
    \includegraphics[width=\BC\textwidth]{figures/small_appendix/Appendix_PLAX1.jpg}
    &
    \includegraphics[width=\BC\textwidth]{figures/small_appendix/Appendix_PSAX1.jpg}
    &
    \includegraphics[width=\BC\textwidth]{figures/small_appendix/Appendix_Other1.jpg}
    &
   
    \\
    
    \includegraphics[width=\BC\textwidth]{figures/small_appendix/Appendix_PLAX2.jpg}
    &
    \includegraphics[width=\BC\textwidth]{figures/small_appendix/Appendix_PSAX2.jpg}
    &
    \includegraphics[width=\BC\textwidth]{figures/small_appendix/Appendix_Other2.jpg}
    &
   
     \\
     
     \includegraphics[width=\BC\textwidth]{figures/small_appendix/Appendix_PLAX3.jpg}
    &
    \includegraphics[width=\BC\textwidth]{figures/small_appendix/Appendix_PSAX3.jpg}
    &
    \includegraphics[width=\BC\textwidth]{figures/small_appendix/Appendix_Other3.jpg}
    &
   
     \\
     
     \includegraphics[width=\BC\textwidth]{figures/small_appendix/Appendix_PLAX4.jpg}
    &
    \includegraphics[width=\BC\textwidth]{figures/small_appendix/Appendix_PSAX4.jpg}
    &
    \includegraphics[width=\BC\textwidth]{figures/small_appendix/Appendix_Other4.jpg}
    &
   
    \end{tabular}	
    \caption{Examples of images for each possible view label in our dataset. \emph{From left to right:} Four examples of peristernal long axis (PLAX) view, four examples of peristernal short axis (PSAX) view, and four examples of other kinds of view in our ``Other'' class. }
    \label{fig:VIEW_SAMPLES_APPENDIX}
\end{figure}

%%%%%%
%%
%%
\newpage
\subsection{Examples of each view for a Severe AS patient}
\newcommand{\BA}{0.33}
\setlength{\tabcolsep}{0.1cm}
\begin{figure}[!h]
\begin{tabular}{c c c c}
    PLAX  & PSAX & OTHER 
    \\
    \includegraphics[width=\BA\textwidth]{figures/small_appendix/SevereAS_11112007_PLAX1.jpg}
    &
    \includegraphics[width=\BA\textwidth]{figures/small_appendix/SevereAS_11112007_PSAX1.jpg}
    &
    \includegraphics[width=\BA\textwidth]{figures/small_appendix/SevereAS_11112007_Other1.jpg}
    &
    
    \\
    
    \includegraphics[width=\BA\textwidth]{figures/small_appendix/SevereAS_11112007_PLAX2.jpg}
    &
    \includegraphics[width=\BA\textwidth]{figures/small_appendix/SevereAS_11112007_PSAX2.jpg}
    &
    \includegraphics[width=\BA\textwidth]{figures/small_appendix/SevereAS_11112007_Other2.jpg}
    &
   
     \\
     
     \includegraphics[width=\BA\textwidth]{figures/small_appendix/SevereAS_11112007_PLAX3.jpg}
    &
    \includegraphics[width=\BA\textwidth]{figures/small_appendix/SevereAS_11112007_PSAX3.jpg}
    &
    \includegraphics[width=\BA\textwidth]{figures/small_appendix/SevereAS_11112007_Other3.jpg}
    &
  
    \end{tabular}	
    \caption{Examples of images from a patient with Severe AS in our dataset. \emph{From left to right:} Three examples of parasternal long axis (PLAX) view, three examples of parasternal short axis (PSAX) view, and three examples of other kinds of view in our ``Other'' class. }
    \label{fig:PatientSevereAS}
\end{figure}


%%%%%%
%%
%%
\newpage
\subsection{Examples of each view for a No AS patient}
\newcommand{\BB}{0.33}
\setlength{\tabcolsep}{0.1cm}
\begin{figure}[!h]
\begin{tabular}{c c c c}
    PLAX  & PSAX & OTHER 
    \\
    \includegraphics[width=\BB\textwidth]{figures/small_appendix/NoAS_1996889_PLAX1.jpg}
    &
    \includegraphics[width=\BB\textwidth]{figures/small_appendix/NoAS_1996889_PSAX1.jpg}
    &
    \includegraphics[width=\BB\textwidth]{figures/small_appendix/NoAS_1996889_Other1.jpg}
    &
    
    \\
    
    \includegraphics[width=\BB\textwidth]{figures/small_appendix/NoAS_1996889_PLAX2.jpg}
    &
    \includegraphics[width=\BB\textwidth]{figures/small_appendix/NoAS_1996889_PSAX2.jpg}
    &
    \includegraphics[width=\BB\textwidth]{figures/small_appendix/NoAS_1996889_Other2.jpg}
    &
   
     \\
     
     \includegraphics[width=\BB\textwidth]{figures/small_appendix/NoAS_1996889_PLAX3.jpg}
    &
    \includegraphics[width=\BB\textwidth]{figures/small_appendix/NoAS_1996889_PSAX3.jpg}
    &
    \includegraphics[width=\BB\textwidth]{figures/small_appendix/NoAS_1996889_Other3.jpg}
    &
  
    \end{tabular}	
    \caption{Examples of images from a patient with No AS in our dataset. \emph{From left to right:} Three examples of parasternal long axis (PLAX) view, three examples of parasternal short axis (PSAX) view, and three examples of other kinds of view in our ``Other'' class. }
    \label{fig:PatientNoAS}
\end{figure}



\newpage 
\section{Further Results}

\subsection{Assessment of ensembling}

Table~\ref{tab:best_single_checkpoint_VS_ensemble_FS_echo260} compares using a single checkpoint (one point estimate of neural network weight vector $\theta$) to using an ensemble of parameters aggregated from the last 25 checkpoints (one per epoch).

\begin{table}[!h]
    \centering
    \begin{tabular}{c|cccc|c}
    \textit{Diagnosis classification} & Split 1  & Split 2 & Split 3 & Split 4 & Average\\
    \hline
    Best single checkpoint  & 61.81 & 59.79 & 56.05 & 64.21 & 60.46\\
    Ensemble  & 62.95 & 61.03 & 56.58 & 63.84 & \textbf{61.13}
	\\ \hline
    \textit{View classification}  &   &  &  &  & 
    \\ \hline
    Best single checkpoint  & 93.03 & 93.24 & 92.39 & 93.79 & 93.11\\
    Ensemble  & 92.37 & 93.24 & 93.72 & 93.87 & \textbf{93.30}\\
    \end{tabular}
    \caption{Comparing best single checkpoint performance with ensemble performance on \textbf{Full-size \datasetName-156-52}}
    \label{tab:best_single_checkpoint_VS_ensemble_FS_echo260}
\end{table}


%%%%%%
%%
%%
\subsection{Patient-level diagnosis performance on bonus heldout set}

Table~\ref{tab:diagnosis classification patient unlabeled_heldout_174} examines the performance of the best labeled-set-only methods and MixMatch methods on the 174 patient studies that have diagnosis but no view labels.
 While the images used here were originally included in the unlabeled training set (which was used to train SSL methods like MixMatch), the diagnosis labels were not provided at all during training time. 
 We thus still believe this is an authentic test of generalization given the scarcity of labeled data available for our task.
 Of course, additional independent evaluation (especially from another institution) is needed.

\begin{table}[!h]
    \centering
    \begin{tabular}{l l l|rrrr|c}
    Pretrain & Method & Voting
    & Split 1  & Split 2 & Split 3 & Split 4 & average\\
    \hline
    & Basic WRN & Simple average & 76.73 & 75.25 & 76.87 & 81.88 & 77.68\\
    & Basic WRN & View-prioritized & 73.63 & 83.21 & 79.70 & 80.08 & 79.18\\
    %SSL & FS & MixMatch & Priority view + confidence & 94.58 & 84.17 & 77.50 & 92.5 & 87.19\\
    \hline
    & MixMatch & Simple average & 85.32 & 76.29 & 74.14 & 79.95 & 78.93\\
    view & MixMatch & Simple average & 83.36 & 77.96 & 75.61 & 81.37 & 79.58\\
    & MixMatch & View-prioritized & 83.27 & 83.76 & 82.34 & 82.83 & \textbf{83.05}\\
    view & MixMatch & View-prioritized & 82.53 & 86.15 & 79.62 & 83.27 & 82.89\\
    %view & MixMatch & LR with view-priority & 80.42 & 84.24 & 76.58 & 80.67 & 80.48\\
    %(MixMatch transfered) + MysteryMethod & NA & NA & NA\\ 
    \end{tabular}
    \caption{Patient-level AS Severity Diagnosis Classification on the \textbf{bonus heldout set} of 174 patients for whom we have diagnosis labels only (no view labels). We show balanced accuracy on models trained on each of the four folds on four \textbf{full-size \datasetName-156-52} dataset.
    }%endcaption
    \label{tab:diagnosis classification patient unlabeled_heldout_174}
\end{table}


%%%%%%
%%
%%
\subsection{Assessment of MixMatch hyperparameter sensitivity}

In Table~\ref{tab:MixMatch hyperparameters ablation study}, we consider four possible strategies for setting the hyperparameters of MixMatch, varying two  key settings for the weight on unlabeled loss $\lambda$. First, we vary whether the final value of $\lambda$ is set to its \emph{best} value among a grid of candidates (based on validation set performance), or \emph{fixed} to a constant.
Second, we vary whether $\lambda$ remains fixed over iterations throughout a training run, or is updated over iterations on a linear ramp schedule from 0 to its final target value. 

From this comparison, we see we consistent gains across splits (average gain across splits of over 1.6\% balanced accuracy) for using a delayed ramp up schedule with target value selected via grid search.

\begin{table}[!h]
    \centering
    \begin{tabular}{l l| rrrr | r}
    Final $\lambda$ value & $\lambda$ update schedule & Split 1  & Split 2 & Split 3 & Split 4 & Average\\
    \hline
    best on val & Delayed ramp-up  & 65.57 & 62.69 & 60.87 & 66.29 & 63.86\\
    best on val & Immediate ramp-up & 65.07 & 61.87 & 60.82 & 65.37 & 63.28\\
    best on val & Constant  & 65.03 & 61.52 & 58.87 & 65.22 & 62.66\\
    100 (fixed) & Constant & 63.94 & 61.79 & 58.87 & 64.35 & 62.24\\
    \end{tabular}
    \caption{Ablation study of different settings of the unlabeled loss weight $\lambda$ for MixMatch. AS severity diagnosis classification for individual images on the \textbf{full-size \datasetName-156-52} dataset. showing balanced accuracy averaged over the test sets from multiple folds (each fold’s test set contains all images from 52 patients). }%endcaption
    \label{tab:MixMatch hyperparameters ablation study}
\end{table}



%%%%%%
%%
%%
\subsection{Assessment of alternative view prioritization strategy using thresholding}


An anonymous reviewer suggested an alternative strategy for prioritizing images of relevant view.
The alternative strategy works as follows: for each image, we compute the predicted probability that the image is a ``relevant view'' (either PLAX and PSAX) by summing the probabilities of each view type.
However, instead of using this raw probability as a weight (as our chosen method does), we use a \emph{cutoff threshold} and simply average the diagnosis predictions of images whose relevant view probability is above the cutoff.
For each patient, we use the majority vote prediction of the diagnosis from the images of relevant views.
The value of the cutoff threshold is selected using the validation set to maximize balanced accuracy.

Table~\ref{tab:Suggested_Aggregation_Ablation} shows the performance of this strategy (``threshold-then-average'') on the full-size dataset.
Using this alternative prioritization strategy together with our suggested methodology for patient-level diagnosis (using MixMatch, pretraining on view), we find the average test set balanced accuracy is around 85.8\%, while the weighted average strategy in the main paper achieves over 90\% balanced accuracy. We take this as reasonably decisive evidence that a weighted average (rather than a simple cutoff) should be preferred.

\begin{table}[!h]
    \centering
    \begin{tabular}{l l l|rrrr|c}
    Pretrain & Method & Aggregation across images
    & Split 1  & Split 2 & Split 3 & Split 4 & average\\
    \hline
    & Basic WRN & Threshold-then-Average & 85.42 & 86.25 & 79.17 & 92.50 & 85.84 \\
    %SSL & FS & MixMatch & Priority view + confidence & 94.58 & 84.17 & 77.50 & 92.5 & 87.19\\
    & MixMatch & Threshold-then-Average & 83.33 & 84.17 & 77.50 & 94.58 & 84.90 \\
    view & MixMatch & Threshold-then-Averagen & 86.67 & 80.00 & 82.50 & 94.17 & 85.84\\
    %view & MixMatch & LR with view-priority & 87.08 & 82.08 & 85.00 & 88.75 & 85.73\\
    %(MixMatch transfered) + MysteryMethod & NA & NA & NA\\ 
    \end{tabular}
    \caption{Alternative view-prioritizing strategy for patient-level AS severity diagnosis classification on the \textbf{full-size \datasetName-156-52} dataset, showing balanced accuracy on the test set across multiple folds (each fold’s test set contains 52 patients).}
    %endcaption
    \label{tab:Suggested_Aggregation_Ablation}
\end{table}



%%%%%%
%%
%%
\subsection{ROC Curve of patient-level diagnosis: no AS vs. mild/moderate/severe AS}

Fig.~\ref{fig: No AS vs Some AS} shows receiver operating curves for several methods for the task of distinguishing no AS vs Some AS (which aggregates both the mild/moderate and severe levels in the 3-level diagnosis task of the main paper).

\begin{figure}[!h]
\begin{tabular}{c c}
	\includegraphics[width=0.43\textwidth]{figures/fold0_multitask_PatientLevel_NoVSSome_NormalizedPriorityStrategyClassProbabilityScore.pdf}
	&
    \includegraphics[width=0.43\textwidth]{figures/fold1_multitask_PatientLevel_NoVSSome_NormalizedPriorityStrategyClassProbabilityScore.pdf}
	\\
	(a) Split 1 & (b) Split 2
	\\
	\includegraphics[width=0.43\textwidth]{figures/fold2_multitask_PatientLevel_NoVSSome_NormalizedPriorityStrategyClassProbabilityScore.pdf}
	&
    \includegraphics[width=0.43\textwidth]{figures/fold3_multitask_PatientLevel_NoVSSome_NormalizedPriorityStrategyClassProbabilityScore.pdf}
	\\
	(c) Split 3 & (d) Split 4
\end{tabular}
    
\caption{ROC curves for binary diagnosis task (no AS vs ``mild/moderate/severe AS'') on \textbf{full-size \datasetName-156-52}.
    }%endcaption
    \label{fig: No AS vs Some AS}
\end{figure}

\section{Methodological Details}

\subsection{Image processing details}
\label{sec:removing_doppler}

\paragraph{Removing doppler images.}
In the raw data of all imagery available for an echocardiogram study, 
we obtained TIFF files that represent both cineloops and Doppler images.

We verified in our labeled set that all Doppler images have one of the following landscape aspect ratio $(831, 323)$, $(901, 384)$, $(901, 390)$, $(704, 305)$, $(831, 421)$, $(901, 469)$ or $(563, 294)$. Only the Dopplers have these aspect ratios. We thus filtered out Doppler completely via these aspect ratios. 

\paragraph{Downsizing}
The original images are provided as high-resolution TIFF format images (hundreds of pixels per side) of varying aspect ratios. Generally, we can expect that both view and diagnosis classifiers would perform better given higher-resolution input (and holding other factors the same). The main trade-off of processing higher-resolution images is increased runtime and memory requirements. In our preliminary experiments, we compared downsizing all images to a standard square aspect ratio at 3 possible sizes: 32x32, 64x64 and 128x128. We found that 64x64 achieves a good balance between model performance and computation cost. 
A prior study by \citet{madaniDeepEchocardiographyDataefficient2018} provides a more extensive study of optimal resolution size. The interested reader can refer to their work for more details. 


\subsection{Architecture Settings and Hyperparameters}
\label{sec:arch_and_hyperparameters}

\paragraph{Weighted cross-entropy for labeled loss}
To counteract the effect of class imbalance in the dataset, we use weighted cross-entropy for the labeled loss. For an input image $x$ whose true label $y$ indicates it belongs to class $c$, the weighted cross-entropy assumes the following form:
\begin{align}
\mathcal{L}^L(\theta, x) = - w_{c} \log \hat{p}_{c}(\theta, x),
\end{align}
where $\hat{p}_{c}$ is the predicted probability of class $c$. The weight $w_{c}$ is calculated using the training set statistics as follow:
\begin{align}
w_{c} = \frac{\prod_{k\neq c}{N_{k}}}{\sum_{j}\prod_{k \neq j}{N_{k}}}
\end{align}
where $N_{k}$ is the number of images of class $k$ in the training set.

\paragraph{Common architecture.}
Following~\citet{oliverRealisticEvaluationDeep2018}, for all considered methods, we use the \emph{same} backbone neural network architecture: a wide residual network~\citep{zagoruykoWideResidualNetworks2017} with 28 layers (WRN-28), which has total of 5,931,683 parameters.
This same network architecture is used in the original MixMatch evaluation~\citep{berthelotMixmatchHolisticApproach2019} with promising results.

\paragraph{Common training protocol.}
All SSL methods we consider follow the loss minimization framework with two primary losses (one for ``labeled'' data and one for ``unlabeled'' data) in Eq.~\eqref{eq:standard-SSL-loss-template}.
We allow every method to train for 32 epochs (where each epoch processes $2^{16}$ images, as in \citet{berthelotMixmatchHolisticApproach2019}).
Our preliminary experiments suggest that after 30 epochs all methods effectively converge in terms of validation balanced accuracy. 

\paragraph{Common regularization.}
For all methods, we expect performance will be vulnerable to overfitting, so we impose an L2-norm penalty on the weights $\theta$, also known as weight decay. Each method selects its preferred value of this penalty strength hyperparameter. We searched values in [0.0002, 0.002, 0.02].

\paragraph{Common optimization.}
We use ADAM \citep{kingma2014adam} to optimize each model.
Each method selects the value of the step size (learning rate) as a hyperparameter. We experimented with 0.002 and 0.0007
%HZ: 'performance being sensitive to learning rate' is very reasonable. But we don't have an ablation to back it. 
%We find performance is sensitive to the step size (learning rate) hyperparameter, so we perform a grid search and select the value that maximizes balanced accuracy on the validation set.

\paragraph{Hyperparameters for Pseudo-Label.}
Beyond the usual hyperparameters for our loss-minimization SSL framework, another important hyperparameter for pseudo-label is the threshold $\tau$. We find that performance is not very sensitive to the chosen $\tau$ value as long as it is within a certain range. We set $\tau$ to 0.95, as done in past literature that evaluates Pseudo-Label as an SSL method ~\citep{oliverRealisticEvaluationDeep2018,berthelotMixmatchHolisticApproach2019, berthelotRemixmatchSemisupervisedLearning2019, sohnFixmatchSimplifyingSemisupervised2020}.


\paragraph{Hyperparameters for VAT.}
Beyond the usual hyperparameters for our SSL framework, for VAT we need to select a value for $\epsilon$.
In \citet{miyatoVirtualAdversarialTraining2019}, the authors claimed that they can achieve superior performance by tuning only $\epsilon$ and fixing $\lambda$ to 1. In our experiment, we used the default $\lambda$ as in \cite{berthelotMixmatchHolisticApproach2019} and searched the value of $\epsilon$ in [2, 6, 18], together with learning rate and weight decay. We select the best hyperparameters using validation set performance. 


\paragraph{Hyperparameters for MixMatch.}
Beyond the usual hyperparameters for our SSL framework, the key hyperparameters for MixMatch include the number of augmentations $K$, the temperature $T>0$ used for sharpening, interpolation hyperparameter $\alpha$ and unlabeled loss coefficient $\lambda$. We set $K=2$, $T=0.5$, and $\alpha=0.75$ as done in \citet{berthelotMixmatchHolisticApproach2019}, and search for $\lambda$ in the range [10, 30, 75, 100, 130] using validation set. 

\paragraph{Hyperparameters for Multitask training.}
We searched $\gamma$, the hyperparameter that control the strength of the auxilliary view loss in Eq.~\eqref{eq:multitask}, in the range [10, 3, 1, 0.3, 0.1]. The best $\alpha$ is selected together with other hyperparameters on validation set. 




\end{document}
\grid
