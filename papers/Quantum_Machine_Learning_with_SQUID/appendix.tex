\section{Gradient evaluation for variational quantum models}
\label{app:gradients}

In this appendix we provide a derivation of the scheme employed in the main text to evaluate gradients with respect to the parameters of the quantum models. Following the discussion in Sec.~\ref{sec:qmodels} of the main text, a generic variational quantum model is defined by a unitary transformation $W(\vec{w})$ dependent on $M$ real parameters $\vec{w}$ taking values in $[0,\pi)$. The general construction employed in this work consists in decomposing the full unitary operation $W(\vec{w})$ into a combination of one and two qubit unitaries denoted by $U^1_k(\alpha,\beta,\gamma)$ and $U^2_{jk}(\theta,\phi,\eta)$ respectively. The subscript indices indicate the qubit (or pairs of qubits) the unitary acts upon. Given the simple structure of a generic circuit as the one depicted in Eq.~\eqref{circ:four_qubits}, one can obtain a closed form expression for every component of the gradient by looking at the individual gradients of the basic unitaries $U^1$ and $U^2$ directly. Here we will use as a concrete example the generic $SU(4)$ unitary transformation from Eq.~\eqref{circ:two_qubits} which we reproduce here for convenience
\begin{equation*}
\Qcircuit @C=0.5em @R=.7em {
& \gate{U^1_0(\alpha_0,\beta_0,\gamma_0)} & \multigate{1}{U^2_{01}(\theta,\phi,\eta)}&\gate{U^1_0(\alpha_2,\beta_2,\gamma_2)} & \qw\\
& \gate{U^1_1(\alpha_1,\beta_1,\gamma_1)} & \ghost{U^2_{01}(\theta,\phi,\eta)}&\gate{U^1_1(\alpha_3,\beta_3,\gamma_3)} & \qw\\
}\;.
\end{equation*}
Note that in this case $M=15$. 
In the following we will assume, without loss of generality, that this unitary operation is applied to the initial state $\ket{00}$ of the two qubit register and denote the resulting state by
\begin{equation}
\rvert \Phi\left(\vec{w}\right)\rangle = W\left(\vec{w}\right)\ket{0}\;.
\end{equation}
The classical output generated by a measurement on the qubit register can be completely characterized by the $4$ probabilities to find the system in each one of the possible basis states:
\begin{equation}
\label{eq:out_prob}
p_k = Tr\left[\Pi_k\rvert  \Phi\left(\vec{w}\right)\rangle\langle\Phi\left(\vec{w}\right)\lvert \right]
\end{equation}
where we have introduced explicitly the projectors
\begin{equation}
\begin{split}
\Pi_0 &= \rvert00\rangle\langle00\lvert\quad\Pi_1 = \rvert01\rangle\langle01\lvert\\
\Pi_2 &= \rvert10\rangle\langle10\lvert\quad\Pi_3 = \rvert11\rangle\langle11\lvert\;.
\end{split}
\end{equation}

Computing the derivatives with respect to the 12 angles corresponding to the 4 one qubit $SU(2)$ operations is straightforward by recalling the definition Eq.~\eqref{eq:u1_unitary} of $U^1$ in terms of exponentials of Pauli operators. As an example, the derivative with respect to $\gamma_0$ of any of the probabilities in Eq.~\eqref{eq:out_prob} can be written explicitly as
\begin{equation}
\begin{split}
\label{eq:gradient}
\frac{\partial}{\partial \gamma_0} p_k =& \frac{\partial}{\partial \gamma_0} \langle00\lvert W^\dagger(\vec{w})\Pi_kW(\vec{w})\rvert00\rangle\\
=&i\langle00\lvert W^\dagger(\vec{w})\Pi_kW(\vec{w})Y_0\rvert00\rangle\\
&-i\langle00\lvert Y_0W^\dagger(\vec{w})\Pi_kW(\vec{w})\rvert00\rangle\\
=&-2\mathcal{I}\left[\langle00\lvert W^\dagger(\vec{w})\Pi_kW(\vec{w})Y_0\rvert00\rangle\right]\\
=&2\mathcal{R}\left[\langle00\lvert W^\dagger(\vec{w})\Pi_k\frac{\partial W(\vec{w})}{\partial \gamma_0}\rvert00\rangle\right]\;,
\end{split}
\end{equation}
where $\mathcal{I}$ ($\mathcal{R}$) indicating the imaginary (real) part. Note that, for all of the 12 parameters characterizing the single qubit transformations, the derivative of the full variational circuit unitary $W$ can be expressed in terms of the same parametrized unitary with the appropriate angle angle shifted by $\pi/2$. For the case of $\gamma_0$ considered above we have for instance:
\begin{equation}
\frac{\partial W(\vec{w})}{\partial \gamma_0} = i W(\vec{w})Y_0 = W\left(\vec{w}'(\gamma_0)\right)
\end{equation}
with a new set of parameters given by
\begin{equation}
\begin{split}
\vec{w}'(\gamma_0) = \bigg(&\alpha_0,\beta_0,\gamma_0+\frac{\pi}{2},\alpha_1,\beta_1,\gamma_1,\\
&\theta,\phi,\eta,\\
&\alpha_2,\beta_2,\gamma_2,\alpha_3,\beta_3,\gamma_3\bigg)\;.
\end{split}
\end{equation}
Using the optimal implementation for the more general $SU(4)$ transformation derived in Ref.~\cite{vatan2004} (see Fig.6 there) one can show that we have the same property for the 2 qubit unitary $U^2_{jk}$. This property is usually referred to as the parameter shift rule~\cite{Mitarai_2018,Schuld_2019}.

In order to estimate the expectation values in the last line of Eq.~\eqref{eq:gradient} we can employ two strategies: if the required number of output probabilities $K$ is the maximum possible one with $n$ qubits (ie. $K=2^n$), it is convenient to first decompose the projectors in the computational basis states into a linear combination of $K=2^n$ diagonal operators obtained by considering all the possible tensor products of identities and Pauli Z, and then to evaluate each one of the resulting expectation values using a single Hadamard test each. The total number of separate circuits required for this approach is then $KM$, with $M$ the total number of parameters.

In the more realistic situation where $K\ll2^n$ instead, the strategy just described will still require an exponential number of measurement in the size of the qubit register. A more efficient alternative can be obtained by evaluating explicitly the $K$ pairs of expectation values
\begin{equation}
r_k = \mathcal{R} \left[\langle 00 \lvert W^\dagger(\vec{w}) \rvert k\rangle\right]\quad i_k = \mathcal{I} \left[\langle 00 \lvert W^\dagger(\vec{w}) \rvert k\rangle\right]\;,
\end{equation}
with $\rvert k\rangle$ the computational basis state associated to the projector $\Pi_k$. These expectation values can be estimated using an Hadamard test with one additional ancilla qubit and require the execution of $2K$ independent circuits (one each for real and imaginary part).

For each one of the $M$ parameters, we then use additional $2K$ Hadamard tests to estimate the expectation values associated with the shifted unitaries
\begin{equation}
\widetilde{r}_{km} = \mathcal{R} \left[\langle k \lvert W(\vec{w})_m' \rvert 00\rangle\right]\quad \widetilde{i}_{km} = \mathcal{I} \left[\langle k \lvert W(\vec{w})_m' \rvert 00\rangle\right]\;,
\end{equation}
where we used the compact notation $W(\vec{w})_m'$ to indicate the derivative with respect to the $m$-th parameter.
This requires a total of $2KM$ independent circuit executions for a total of $2K(M+1)$ observables. The gradient can then be computed as
\begin{equation}
\begin{split}
\label{eq:gradient_Htest}
\frac{\partial}{\partial w_m} p_k =&2\mathcal{R}\left[\langle00\lvert W^\dagger(\vec{w})\Pi_kW(\vec{w})_m'\rvert00\rangle\right]\\
=&2\left(r_k\widetilde{r}_{km}- i_k\widetilde{i}_{km}\right)\;.
% =&2\mathcal{R}\left[\left(r_k+i i_k\right)\left(\widetilde{r}_{km}+i \widetilde{i}_{km}\right)\right]\\
\end{split}
\end{equation}

An alternative approach to reduce the number of independent circuits needed for gradient evaluation is to use expectation values of unitary operators instead of projectors. This extension can be easily implemented within the SQUID framework.

\section{Additional information on the MNIST benchmark}
\label{app:mnist}

\begin{table}[]
\centering
\begin{tabular}{l|l|l|l|l|l}
model & $M_0$ & $K_{tot}$ & Accuracy & TR$_{90}$ & VR$_{90}$\\ \hline
$cA$ & 3 & 2366 & $0.9915^{+4}_{-28}$ & 0.9966 & 0.9951\\
$cB$ & 6 & 4727 & $0.9903^{+20}_{-18}$ & 0.9973 & 0.9947 \\
$cC$ & 12 & 9449 & $0.9881^{+44}_{-2}$ & 0.9980 & 0.9967 \\
$cD$ & 24 & 18893 & $0.9895^{+8}_{-4}$ & 0.9994 & 0.9952 \\
$cE$ & 40 & 31485 & $0.9911^{+12}_{20}$ & 1 & 0.9950 \\
$cF$ & 60 & 47225 & $0.9891^{+16}_{20}$ & 1 & 0.9964
\end{tabular}
\caption{Results for the classical feed-forward models described in the main text. \label{tab:cl_res}}
\end{table}

We report in Tab.~\ref{tab:cl_res} the parameters and results for the classical models used in the MNIST classification discussed in Sec.~\ref{sec:results} and corresponding to the results presented in Fig.~\ref{fig:cl_accuracy} on the main text.
The last two columns in Tab.~\ref{tab:cl_res} denoted by TR$_{90}$ and VR$_{90}$ show the boundary value for the $90\%$ accuracy percentile, the latter refers to the validation data while the former to the training set. The estimated errors correspond to a $68\%$ confidence interval.

The parameters of the separable quantum models considered in the main text, together with the results obtained from training on the MNIST classification problem, are presented in Tab.~\ref{tab:qm_res_A}. The models with a latent space corresponding to the restricted output for the quantum layer are indicated with a subscript $1$ in the table. 

\begin{table}[]
\centering
\resizebox{\columnwidth}{!}{
\begin{tabular}{l|l|l|l|l|l|l}
model & $M_0$ & $M_1$ & $K_{tot}$ & Accuracy& TR$_{90}$ & VR$_{90}$ \\ \hline
 $qA(1)$ & 3 & 2 & 2358 & $0.9837^{+30}_{-36}$ & 0.9898& 0.9872  \\  \hline
$qB(2)$  & 6 & 4 & 4715 & $0.9847^{+32}_{-32}$  & 0.9920& 0.9889 \\
$qB_1(2)$  & 6 & 1 & 4713 & $0.9854^{+16}_{-20}$ & 0.9921 & 0.9887 \\ \hline
$qC(4)$ & 12 & 16 & 9437 & $0.9859^{+26}_{-18}$  & 0.9948 & 0.9891\\
$qC_1(4)$ & 12 & 1 & 9423 & $0.9839^{+26}_{-16}$ & 0.9905 & 0.9868 \\ \hline
$qD(6)$ & 18 & 64 & 14195 & $0.9869^{+24}_{-38}$ & 0.9948 & 0.9904 \\
$qD_1(6)$  & 18 & 1 & 14133 & $0.98125^{+38}_{-34}$ & 0.9881& 0.9858
\end{tabular}
}
\caption{Results for the first set of separable quantum models described in the text. The parenthesis in the label for the quantum models indicates the number of qubits $N$ employed.\label{tab:qm_res_A}}
\end{table}


The same convention is used in Tab.~\ref{tab:qm_res_B} where we present the parameters and results for the quantum models with entanglement described in Sec.~\ref{sec:qm_res_withent}. We also show in Fig.~\ref{fig:qm_train} the evolution of the accuracy for both the training set (left panels) and validation set (right panels). The top two panels are obtained with models with maximal output size ($M_1=16$ in this case) while the bottom panels show the results using a restricted output model with $M_1=1$.

\begin{table}[]
\centering
\resizebox{\columnwidth}{!}{
\begin{tabular}{l|l|l|l|l|l|l}
model  & $M_0$ & $M_1$ & $K_{tot}$ & Accuracy& TR$_{90}$ & VR$_{90}$ \\ \hline
$qE(2)$  & 9 & 4 & 7070 & $0.9879^{+12}_{-24}$ & 0.9940 & 0.9899 \\
$qE_1(2)$ & 9 & 1 & 7068 & $0.9865^{+22}_{-18}$ & 0.9933 & 0.9892 \\ \hline
$qF(2)$  & 15 & 4 & 11780 & $0.9875^{+18}_{-24}$ & 0.9960 & 0.99 \\
$qF_1(2)$  & 15 & 1 & 11778 & $0.9859^{+24}_{-16}$ & 0.9926 & 0.9893 \\ \hline
$qG(4)$ & 30 & 16 & 23567 & $0.9889^{+28}_{-22}$ & 0.9987 & 0.9924 \\
$qG_1(4)$ & 30 & 1 & 23553 & $0.9847^{+24}_{-34}$ & 0.9897 & 0.9879 \\ \hline
$qH(4)$ & 39 & 16 & 30632 & $0.9885^{+24}_{-22}$ & 0.9984 & 0.9915 \\
$qH_1(4)$& 39 & 1 & 30618 & $0.9833^{+18}_{-30}$ & 0.9867 & 0.9854 \\ \hline
$qI(4)$ &  57 & 16 & 44762 & $0.9883^{+22}_{-24}$ & 0.9993 & 0.9911 \\
$qI_1(4)$  & 57 & 1 & 44748 & $0.9810^{+24}_{-16}$ & 0.9863 & 0.9840 \\  \hline
$qL(6)$  & 45 & 64 & 35390 & $0.9895^{+18}_{-16}$ & 0.9995 & 0.9920 \\
$qL_1(6)$ & 45 & 1 & 35328 & $0.9823^{+32}_{-34}$ & 0.9871 & 0.9867 \\ \hline
\end{tabular}}
\caption{Results for the  set of quantum models described in the text. The parenthesis in the label for the quantum models indicates the number of qubits $N$ employed.\label{tab:qm_res_B}}
\end{table}

\begin{figure}[]
	\centering
	 \includegraphics[width=0.45\textwidth]{plots/quantum_training_band}
	\caption{ Example of training of the largest quantum model described in the main text. The left panels show the increase in classification accuracy for the training set as a function of the number of epochs. The right panels shows the same for the validation set. The top panels are for the full model with the maximum number of outputs $M_1=16$ while the bottom panels are for the minimal model with a single output. The dashed black line indicates $99\%$ accuracy. All bands are $90\%$ confidence intervals with means indicated by the solid lines.}
	\label{fig:qm_train}
\end{figure}

Finally, we present in Fig.~\ref{fig:aqm_accuracy} an extension of the results presented in Fig.~\ref{fig:qm_accuracy_bounds} where in the left panels we show the accuracy histograms for the largest model considered in this work for both the full output model (panel(c)) and the restricted output model (panel(d)).

\begin{figure}[]
	\centering
	 \includegraphics[width=0.45\textwidth]{plots/quantum_scaling}
	\caption{ Results for the accuracy achieved on MNIST with the classical model (green points) and the quantum models with entanglement described in the text (red and blue points). Panel (a) shows the accuracy as a function of the number of parameters for the classical model (green solid circles), the full quantum models (blue solid circles) and the quantum models with a single output variable $M_1=1$ (red solid circles) indicated by ``min''. The inset panel (b) shows the location of the $90\%$ accuracy percentile for the classical (green squares), full quantum (blue squares) and minimal quantum model (red squares) with a single output. Panels (c) and (d) show the histogram of achieved accuracies for the largest quantum model considered using $N=4$ qubits with either the full number of possible output variables ($M_1=16$, top panel) and the single output (bottom panel). }
	\label{fig:aqm_accuracy}
\end{figure}
