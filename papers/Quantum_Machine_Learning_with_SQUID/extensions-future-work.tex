\section{SQUID Extensions and Future Work}
\label{sec:ext}
As mentioned is Section~\ref{sec:squid} SQUID is designed with extensibility in mind.
This means that it should be easy to write additional packages on top of it, as well that additional features should involve changing at most few modules. In the following two sections we describe the SQUID Helpers package, designed to allow the use of SQUID using user-provided configuration files, and comment on possible extensions of the framework.

\subsection{SQUID Helpers Package}
\label{sec:squid_hlp}
To show how such extension could function, but also to stream-line the workload for use cases,
we provide SQUID Helpers extension.
It allows user to use yaml configuration files to run SQUID code.
As a result bootstrapping multiple runs of multiple test cases and aggregation of the results is much easier.

The conversion from configuration files to SQUID is done when a file is first read, and if a bootstrap option is provided, then random seed is changed during every iteration.
The change of the seed is deterministic, and hence all of the results are exactly reproducible.
At the initialization step, there is a single batch forward pass of data to ensure that dimensions of models inputs and outputs match.
This is done due to fail-first and fail-fast principle.
A single forward pass of data is faster than a pass of a single batch, and in perspective of training it is very cheap.
The code then runs a typical training for-loop, with an additional call to \texttt{backward} function of Main Model, as explained in Section~\ref{subsec:squid-main-model}.
In the end, all of the results, as well as configuration is saved to a single location.
If bootstrap was used, along with the results for each run, there is a folder with aggregated results created for simpler analysis.

Helpers extension also provides very basic plotting utilities for the results.
However, those are meant as a example of processing the output folders, since plots are highly dependent on studies performed.

\subsection{Future Work}

There is a vast amount of possible extensions to SQUID, some of which can be included directly in a main project, while others can be used as standalone packages.
The main advantage of the SQUID is that it allows for abstracting communication with a specific backend.
To do so however, custom \texttt{QuantumModel} subclasses interfacing with the backend API will be needed.

Additionally, as of right now SQUID allows only for a classification and regression tasks.
More advanced scenarios would require changes both in SQUID as well as, to a larger extent, SQUID helpers.

Another, and much sooner addition to SQUID will be to start supporting various quantum computing frameworks.
For example, Qiskit has created a great package for optimization of quantum circuits~\cite{Qiskit}.
This would fit perfectly into SQUID ecosystem with a translation layer.
Such addition would allow the user to implement hybrid models without the explicit definition of circuits for training, provided the circuit gradient are correctly passed by the backward function.

There are also others frameworks that have either already similar behavior or plans for optimization packages.
Modularity of SQUID would allow code to be backend agnostic, and work uniformly across multiple types of quantum devices.
