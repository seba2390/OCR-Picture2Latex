%\documentclass[a4paper,10pt]{article}
\documentclass[a4paper,aps,prl,floatfix,showpacs,superscriptaddress,notitlepage]{revtex4-1}

\usepackage[utf8]{inputenc}
\usepackage{graphics}
\usepackage{epsfig}
\usepackage{times}
\usepackage{xcolor}   
\usepackage{amsfonts}
\usepackage{amssymb}
\usepackage{amsthm}
\usepackage{bm} 
\usepackage{amsmath}
\usepackage{xr-hyper} 
\usepackage{hyperref}
\usepackage{color}
\usepackage{graphicx}
\usepackage{subfigure}
\usepackage{textcomp}
\usepackage{wasysym}
\usepackage{soul}
%\usepackage{authblk}
\usepackage{verbatim}
\usepackage[normalem]{ulem}
%\usepackage[symbol]{footmisc}

\externaldocument[txt-]{PRL_draft}[PRL_draft.pdf]

\def\ndof{N_{\textrm{d.o.f.}}} % The number of degrees of freedom
\def\ncons{N_c} % The number of scalar constraints
\def\nzero{N_z} % The number of zero modes
\def\nss{N_{\textrm{ss}}} % The number of self stresses
\def\coup{J} % a coupling constant
\def\spin{\mathbf{S}} % a unit spin vector
\def\spincomp{S} % a component of a spin vector
\def\len{l} % a ``length'' associated with either a side or a spin vector
\def\rad{\mathbf{r}} % The position of an origami vertex
\def\spinmat{\mathbf{A}} % The spin matrix
\def\wavevec{\mathbf{k}} % The wavevector
\def\toppol{\mathbf{P}^T} % The topological polarization
\def\doubleunderline#1{\underline{\underline{#1}}}
\def\bvec{\mathbf{b}} % a basis vector
\def\lvec{\matbhf{L}} % a lattice vector 

\begin{document}

%opening
\title{Topology and geometry of spin origami: supplementary material}

\author{Krishanu Roychowdhury}
\affiliation{Laboratory of Atomic And Solid State Physics, Cornell University, Ithaca, NY 14853.}
\affiliation{Kavli Institute for Theoretical Physics, University of California, Santa Barbara, CA 93106-4030.}
\author{D. Zeb Rocklin}
\affiliation{Laboratory of Atomic And Solid State Physics, Cornell University, Ithaca, NY 14853.}
\affiliation{School of Physics, Georgia Institute of Technology, Atlanta, GA 30332.}
\author{Michael J. Lawler\footnote{Corresponding author.}}
\affiliation{Laboratory of Atomic And Solid State Physics, Cornell University, Ithaca, NY 14853.}
\affiliation{Kavli Institute for Theoretical Physics, University of California, Santa Barbara, CA 93106-4030.}
\affiliation{Department of Physics, Binghamton University, Binghamton, NY, 13902.}

\maketitle

\section{Derivation of the star condition}\label{secone} 

\begin{figure}
\centering
 \includegraphics[width=10cm]{supp_fig_1.png}
 \caption{(a) The spin assignment with arrows denoting the orientation associated with the definition of the spin vector -- positive along the direction of the arrow. (b) The corresponding kagome Star of David with exchange interactions marked is inscribed inside the origami unit cell.}
\label{fig_latt1}
\end{figure}
%

The following table is provided to make connections between the set of spins and the set of triangles in the origami sheet [Supplementary Fig.~\ref{fig_latt1} (a)] with the exchange interactions marked on the corresponding kagome Star of David [Supplementary Fig.~\ref{fig_latt1} (b)].
\begin{center}
 \begin{tabular}{|c|c|c|} 
 \hline
 Triangle label & Spins in the triangle & exchange interactions in kagome \\  
 \hline
 1 & $\spin_0$, $\spin_1$, $\spin_7$ & $J_7$, $J_{8}$, $J_{14}$ \\
 \hline
 2 & $\spin_1$, $\spin_2$, $\spin_8$ & $J_5$, $J_{6}$, $J_{15}$  \\
 \hline
 3 & $\spin_2$, $\spin_3$, $\spin_9$ & $J_3$, $J_{4}$, $J_{16}$  \\
 \hline
 4 & $\spin_0$, $\spin_5$, $\spin_{6}$ & $J_9$, $J_{10}$, $J_{13}$  \\
 \hline
 5 & $\spin_4$, $\spin_5$, $\spin_{11}$ & $J_{11}$, $J_{12}$, $J_{18}$  \\  
 \hline
 6 & $\spin_3$, $\spin_4$, $\spin_{10}$ & $J_{1}$, $J_{2}$, $J_{17}$  \\  
 \hline
\end{tabular}
\end{center}
Now consider the two adjacent triangles with label 4 and 1 in Supplementary Fig.~\ref{fig_latt1} (a) which are associated with the constraints:
\begin{equation}
\spin_{\triangle_4} = \ell^{\triangle_4}_0 \spin_0 + \ell^{\triangle_4}_5 \spin_5 + \ell^{\triangle_4}_6 \spin_6 = 0~~;~~\spin_{\triangle_1} = \ell^{\triangle_1}_0 \spin_0 + \ell^{\triangle_1}_1 \spin_1 + \ell^{\triangle_1}_{7} \spin_{7} = 0. 
\end{equation}
Following $J_{ij}=J_{\triangle_m} \ell^{\triangle_m}_i\ell^{\triangle_m}_j$ from Eq.~\ref{txt-eq:ham1} of the main text, we find that for the common spin $\spin_0$,
\begin{equation}
 \ell^{\triangle_4}_0=\sqrt{J_{9}J_{13}/J_{\triangle_4} J_{10}}~~;~~\ell^{\triangle_1}_0=\sqrt{J_{8}J_{14}/J_{\triangle_1} J_7}.
\end{equation}
Construction of an origami sheet follows from equating these two lengths which we refer to as a {\it length consistency condition}. This leads to 
\begin{equation}
 J_{\triangle_1} = \bigg(\frac{J_8J_{10}J_{14}}{J_7J_9J_{13}}\bigg)J_{\triangle_4}.
 \label{eq:cond1}
\end{equation}
%
Continuing this way, for triangle 1 and 2 we could apply the length consistency condition to the edge that corresponds to the common spin $\spin_1$ obtaining
\begin{equation}
 J_{\triangle_2} = \bigg(\frac{J_6J_{8}J_{15}}{J_5J_7J_{14}}\bigg)J_{\triangle_1}.
 \label{eq:cond2}
\end{equation}
Considering the consecutive adjacent pairs of triangles in a similar way, we get
\begin{equation}
 \begin{split}
  J_{\triangle_3} = \bigg(\frac{J_4J_{6}J_{16}}{J_3J_5J_{15}}\bigg)J_{\triangle_2}, \\
  J_{\triangle_6} = \bigg(\frac{J_2J_{4}J_{17}}{J_1J_3J_{16}}\bigg)J_{\triangle_3}, \\
  J_{\triangle_5} = \bigg(\frac{J_2J_{12}J_{18}}{J_1J_{11}J_{17}}\bigg)J_{\triangle_6}.
 \end{split}
 \label{eq:cond3}
\end{equation}
In other words, we get a ratio for the adjacent triangle 4 and 5 by demanding length consistency for all the edges representing $\spin_j$ for $j=0$ to $4$
\begin{equation}
 J_{\triangle_5} = \bigg(\frac{J_2J_{12}J_{18}}{J_1J_{11}J_{17}}\bigg)\bigg(\frac{J_2J_{4}J_{17}}{J_1J_3J_{16}}\bigg)\bigg(\frac{J_4J_{6}J_{16}}{J_3J_5J_{15}}\bigg)\bigg(\frac{J_6J_{8}J_{15}}{J_5J_7J_{14}}\bigg)\bigg(\frac{J_8J_{10}J_{14}}{J_7J_9J_{13}}\bigg) J_{\triangle_4}.
 \label{eq:cond4}
\end{equation}
These two triangles share the common spin $\spin_5$, and applying length consistency for that edge implies
\begin{equation}
 J_{\triangle_5} = \bigg(\frac{J_9J_{11}J_{18}}{J_{10}J_{12}J_{13}}\bigg)J_{\triangle_4}.
 \label{eq:cond5}
\end{equation}
Comparing Eq.~\ref{eq:cond5} with Eq.~\ref{eq:cond4} we get
\begin{equation}
  (J_{1}J_{3}J_{5}J_{7}J_{9}J_{11})^2=(J_{2}J_{4}J_{6}J_{8}J_{10}J_{12})^2,
 \label{eq:cond6}
\end{equation}
from which the star condition (presented in Eq.~\ref{txt-eq:star} of the main text) follows as all $J$'s are positive. The condition ensures closure of the origami hexagon comprising six adjacent triangles joined at a common vertex. As such, when it is satisfied for each Star of David of the spin system, the ground state configurations may be mapped onto the ground state configurations of a particular nonflattenable triangulated origami sheet. \\

%%%%%%%%%%%%%%%%%%%%%%%%%%%%% The rigidity matrix %%%%%%%%%%%%%%%%%%%%%%%%%%%%%%%%%%%%%%%%%%%%%%%%%%%%%%%%%%%%%%%%%%%%%%%%%%%

\section{The rigidity matrix and spin wave dispersions in ${\rm Cs}_2{\rm CeCu}_3{\rm F}_{12}$}\label{sectwo}

The quantities $J_{\triangle_m}$ mentioned above are merely scale factors that ensures the length consistency on each edge of the origami sheet. They do not alter the zero modes but modify the finite frequencies of the spin waves; in that sense, $\tilde{{\bf S}}_{\triangle_m}\equiv\sqrt{J_{\triangle_m}}{\bf S}_{\triangle_m}$ is equally good a constraint function to consider. The spin Hamiltonian in this language is 
\begin{equation}
 H = \frac{1}{2}\sum_{\triangle_m} \sum_{j\in\triangle_m} (a^{\triangle_m}_j{\bf S}_j)^2~~;~~ a^{\triangle_m}_j=\sqrt{J_{ij}J_{ik}/J_{jk}}~~ \text{for triangle}~~\triangle_m =\langle ijk\rangle. 
\end{equation}
Below we list down the constraints using which the rigidity matrix for the KHAF in ${\rm Cs}_2{\rm CeCu}_3{\rm F}_{12}$ is calculated [see Supplementary Fig.~\ref{fig_latt2} (a) and (b)].
\begin{equation}
 \begin{split}
  a^{\triangle_0}_7{\bf S}_7 + a^{\triangle_0}_9{\bf S}_9 + a^{\triangle_0}_{11}{\bf S}_{11} &= 0~~;~~ a^{\triangle_0}_7=\sqrt{J_2}, a^{\triangle_0}_9=\sqrt{J_2}, a^{\triangle_0}_{11}=\sqrt{J_3^2/J_2} \\
  a^{\triangle_1}_7{\bf S}_7 + a^{\triangle_1}_0{\bf S}_0 + a^{\triangle_1}_{1}{\bf S}_{1} &= 0~~;~~ a^{\triangle_1}_7=\sqrt{J_1}, a^{\triangle_1}_0=\sqrt{J_4^2/J_1}, a^{\triangle_1}_{1}=\sqrt{J_1} \\
  a^{\triangle_2}_2{\bf S}_2 + a^{\triangle_2}_1{\bf S}_1 + a^{\triangle_2}_{8}{\bf S}_{8} &= 0~~;~~ a^{\triangle_2}_2=\sqrt{J_2}, a^{\triangle_2}_1=\sqrt{J_2}, a^{\triangle_2}_{8}=\sqrt{J_3^2/J_2} \\
  a^{\triangle_3}_2{\bf S}_2 + a^{\triangle_3}_9{\bf S}_9 + a^{\triangle_3}_{3}{\bf S}_{3} &= 0~~;~~ a^{\triangle_3}_2=\sqrt{J_1}, a^{\triangle_3}_9=\sqrt{J_1}, a^{\triangle_3}_{3}=\sqrt{J_4^2/J_1} \\
  a^{\triangle_4}_0{\bf S}_0 + a^{\triangle_4}_6{\bf S}_6 + a^{\triangle_4}_{5}{\bf S}_{5} &= 0~~;~~ a^{\triangle_4}_0=\sqrt{J_4^2/J_1}, a^{\triangle_4}_6=\sqrt{J_1}, a^{\triangle_4}_{5}=\sqrt{J_1} \\
  a^{\triangle_5}_4{\bf S}_4 + a^{\triangle_5}_5{\bf S}_5 + a^{\triangle_5}_{11}{\bf S}_{11} &= 0~~;~~ a^{\triangle_5}_4=\sqrt{J_2}, a^{\triangle_5}_5=\sqrt{J_2}, a^{\triangle_5}_{11}=\sqrt{J_3^2/J_2} \\
  a^{\triangle_6}_3{\bf S}_3 + a^{\triangle_6}_4{\bf S}_4 + a^{\triangle_6}_{10}{\bf S}_{10} &= 0~~;~~ a^{\triangle_6}_3=\sqrt{J_4^2/J_1}, a^{\triangle_6}_4=\sqrt{J_1}, a^{\triangle_6}_{10}=\sqrt{J_1} \\
  a^{\triangle_7}_6{\bf S}_6 + a^{\triangle_7}_8{\bf S}_8 + a^{\triangle_7}_{10}{\bf S}_{10} &= 0~~;~~ a^{\triangle_7}_6=\sqrt{J_2}, a^{\triangle_7}_8=\sqrt{J_3^2/J_2}, a^{\triangle_7}_{10}=\sqrt{J_2}. \\
 \end{split}
\end{equation}
Let us fix one of the scale factors, say $J_{\triangle_1}=J_{\triangle}$. All other scale factors can be derived from $J_{\triangle_1}$ by demanding length consistency on the edges of the origami which yields
\begin{equation}
 J_{\triangle_0}=J_{\triangle}J_2/J_1, J_{\triangle_2}=J_{\triangle}J_2/J_1, J_{\triangle_3}=J_{\triangle}, J_{\triangle_4}=J_{\triangle}, J_{\triangle_5}=J_{\triangle}J_2/J_1, J_{\triangle_6}=J_{\triangle}, J_{\triangle_7}=J_{\triangle}J_2/J_1. 
\end{equation}
The nonflattenable origami sheet to which the ground state of KHAF in ${\rm Cs}_2{\rm CeCu}_3{\rm F}_{12}$ corresponds has three distinct edge lengths $l_a$, $l_b$ and $l_c$ [see Supplementary Fig.~\ref{fig_latt2} (c)] that are related to the interactions $J_1$, $J_2$, $J_3$ and $J_4$ as
\begin{equation}
 l_a=\frac{J_3}{J_2}\sqrt{\frac{J_1}{J_\triangle}}~~;~~l_b=\sqrt{\frac{J_1}{J_\triangle}}~~;~~l_c=\frac{J_4}{J_1}\sqrt{\frac{J_1}{J_\triangle}}, 
\end{equation}
such that $l_a/l_b=J_3/J_2$ and $l_c/l_b=J_4/J_1$. In our calculations, WLOG we set $J_\triangle$ to be the largest energy scale of the problem which is $J_1$. This sets $l_b=1$, $l_a=J_3/J_2$ and $l_c=J_4/J_1$ in the origami (the values of $J_{1,2,3,4}$ are provided in the main text).

\begin{figure}
\centering
 \includegraphics[width=16cm]{supp_fig_4.png}
 \caption{(a) The origami unit cell for ${\rm Cs}_2{\rm CeCu}_3{\rm F}_{12}$ with the spins and faces labelled. (b) The corresponding kagome Star of David for ${\rm Cs}_2{\rm CeCu}_3{\rm F}_{12}$ with exchange interactions marked is inscribed inside the origami unit cell. (c) The origami unit cell with edge lengths and vertices specified.}
\label{fig_latt2}
\end{figure}

The origami unit cell contains four distinct vertices characterized by four basis vectors ${\bf b}_1$, ${\bf b}_2$, ${\bf b}_3$ and ${\bf b}_4$ and two lattice vectors ${\bf L}_1$ and ${\bf L}_2$ [see Supplementary Fig.~\ref{fig_latt2} (c)]. The unit vectors along the edges are the spins given in Supplementary Fig.~\ref{fig_latt2} (a) whose orientations are obtained through the following procedure. The interaction pattern in ${\rm Cs}_2{\rm CeCu}_3{\rm F}_{12}$ [Supplementary Fig.~\ref{fig_latt2} (b)] demands, for a given basis, the edge from $\bvec_1$ to $\bvec_4$ and the one from $\bvec_2$ to $\bvec_3$ is equal. The vertices at ${\bf b}_1$, ${\bf b}_2$, and ${\bf b}_4$ form an isosceles triangle [triangle 0 in Supplementary Fig.~\ref{fig_latt2} (a)] which can be defined entirely on a plane with ${\bf b}_1=(0,0,0)$, ${\bf b}_2=(l_a,0,0)$ and ${\bf b}_4=(l_a/2,\sqrt{l_b^2-l_a^2/4},0)$. Now the origami configuration demands the following set of equations
\begin{equation}
\begin{split}
 || {\bf b}_3 - {\bf b}_4 || &= l_c \\ 
 || {\bf b}_3 - {\bf b}_2 || &= l_b \\ 
 ||{\bf b}_1 + {\bf L}_1 - {\bf b}_2 || &= l_a \\ 
 ||{\bf b}_1 + {\bf L}_1 - {\bf b}_3 || &= l_b \\ 
 ||{\bf b}_4 + {\bf L}_1 - {\bf b}_3 || &= l_c \\ 
\end{split}
\end{equation}
to solve for ${\bf b}_3$ and ${\bf L}_1$, which together have six variables ($x$, $y$ and $z$ component of each). This leads us to define a tuning parameter ${\bf b}_3^z=b_0$ (in Fig.~\ref{txt-figg3} of the main text) in a range $[-0.12,0.12]$ to allow for real solutions, thus, defining a one-dimensional family of configurations for the nonflattenable sheet. We further notice that setting ${\bf L}_2={\bf b}_3+{\bf b}_4-{\bf b}_1-{\bf b}_2$ satisfies all other constraints on the sheet and lead to the symmetries associated with the spin configuration resulting in diamond-shaped faces of the origami noted in the main text. For a given $b_0$, the spin configuration of the ground state is then uniquely specified through the vectors ${\bf b}_1$, ${\bf b}_2$, ${\bf b}_3$, ${\bf b}_4$, ${\bf L}_1$, and ${\bf L}_2$. Distortions can be made to the sheet by introducing a parameter $t_0$ to modify the definition of ${\bf L}_2$ as ${\bf L}_2={\bf b}_3+{\bf b}_4-{\bf b}_1-t_0{\bf b}_2$. When $t_0$ is set away from 1, it produces a combined effects of shearing and stretching the sheet as noted in Fig.~\ref{txt-figg4} of the main text. \\

The directions of the 12 spins in the unit cell of the origami are the following [see Supplementary Fig.~\ref{fig_latt2} (a) and (c)]
\begin{equation}
\begin{split}
 {\bf S}_0    &= ({\bf b}_3 - {\bf b}_4)/|| {\bf b}_3 - {\bf b}_4 || \\ 
 {\bf S}_1    &= ({\bf b}_2 - {\bf b}_3)/|| {\bf b}_2 - {\bf b}_3 || \\
 {\bf S}_2    &= ({\bf b}_3 - {\bf b}_1 - {\bf L}_1)/|| {\bf b}_3 - {\bf b}_1 - {\bf L}_1 || \\
 {\bf S}_3    &= ({\bf b}_4 + {\bf L}_1 - {\bf b}_3)/|| {\bf b}_4 + {\bf L}_1 - {\bf b}_3 || \\
 {\bf S}_4    &= ({\bf b}_3 - {\bf b}_2 - {\bf L}_2)/|| {\bf b}_3 - {\bf b}_2 - {\bf L}_2 || \\
 {\bf S}_5    &= ({\bf b}_1 + {\bf L}_2 - {\bf b}_3)/|| {\bf b}_1 + {\bf L}_2 - {\bf b}_3 || \\
 {\bf S}_6    &= ({\bf b}_4 - {\bf b}_1 - {\bf L}_2)/|| {\bf b}_4 - {\bf b}_1 - {\bf L}_2 || \\
 {\bf S}_7    &= ({\bf b}_4 - {\bf b}_2)/|| {\bf b}_4 - {\bf b}_2 || \\
 {\bf S}_8    &= ({\bf b}_1 + {\bf L}_1 - {\bf b}_2)/|| {\bf b}_1 + {\bf L}_1 - {\bf b}_2 || \\
 {\bf S}_9    &= ({\bf b}_1 - {\bf b}_4)/|| {\bf b}_1 - {\bf b}_4 || \\
 {\bf S}_{10} &= ({\bf b}_2 - {\bf b}_4 + {\bf L}_2 - {\bf L}_1)/|| {\bf b}_2 - {\bf b}_4 + {\bf L}_2 - {\bf L}_1 || \\
 {\bf S}_{11} &= ({\bf b}_2 - {\bf b}_1)/|| {\bf b}_2 - {\bf b}_1 || \\ 
\end{split}
\end{equation}
As these spins are unit vectors, each of them can be parametrized by two canonical variables, $x^j$ and $p_j$ as
\begin{equation}
  {\bf{S}}_j=\big\{ \cos(q^j)\sqrt{1-p^2_j},~\sin(q^j)\sqrt{1-p^2_j},~p_j \big\}, \nonumber
\end{equation}
with the Poisson bracket $\{q^i,p_j\}=\delta^i_j$. The rigidity matrix ${\mathcal R}$ is defined by linearizing the constraints about the ground state as $S_{\triangle\alpha} = {\mathcal R}_{\triangle\alpha,j\mu}x^{j\mu}$ (see Eq.~\ref{txt-eq:rig} of the main text), where $j\in\triangle$ and $x^{j\mu}\in(q^j,p_j)$. \\

An explicit construction of ${\mathcal R}$ follows by considering the basis 
\begin{equation}
 \tau_1= [q_0,q_1,\cdots,\\ q_{11},p_0,p_1,\cdots,p_{11}]^T \nonumber \\
\end{equation}
corresponding to the twelve spins $\spin_0,\spin_1,\cdots,\spin_{11}$ in the unit cell and 
\begin{equation}
 \tau_2=[\triangle_0^x,\cdots,\triangle_7^x,\triangle_0^y,\cdots,\triangle_7^y,\triangle_0^z,\cdots,\triangle_7^z]^T \nonumber \\
\end{equation}
corresponding to the eight faces in the unit cell shown in Supplementary Fig.~\ref{fig_latt2} (a) such that $\tau_2=\mathcal{R}\cdot \tau_1$. \\


To translate to the momentum space, one needs to consider a Fourier transformation of $\mathcal{R}$. We chose the convention
\begin{equation}
  {\mathcal R}_{\triangle\alpha,j\mu} = \frac{1}{\sqrt{N}} \sum_{\bf{k}} {\mathcal R}_{{\bf{D}}\alpha,{\bf{d}}\mu} ({\bf{k}}) ~ e^{\iota {\bf{k}}\cdot({\bf{R}}_i-{\bf{R}}'_i)},
  \label{fourier_A}
\end{equation}
where $N$ is the number of sites in a unit cell; ${\bf{R}}_i$ and ${\bf{R}}'_i$ denote the position vectors of the unit cells which contain the triangle $\triangle$ at ${\bf{R}}_i+{\bf{D}}$ and the site $j$ at ${\bf{R}}'_i+{\bf{d}}$ respectively. Our particular choice of unit cells is shown in Fig. \ref{fig_latt2} (a). The sites associated with phases $z_1=e^{\iota k_1a}$, $z_2=e^{\iota k_2a}$, $z_1^{-1}$, $z_2^{-1}$ lie in neighboring unit cells. Here $k_1a = {\bf k}\cdot{\bf T}_1$, $k_2a = {\bf k}\cdot{\bf T}_2$, ${\bf T}_1$ and ${\bf T}_2$ are Bravais lattice vectors for the ordering pattern. With this choice of Fourier transform convention and unit cells, we find that ${\rm Det}[\mathcal{R}({\bf k})]$ is real everywhere in the Brillouin zone.\\


Dispersions of the spin waves (the magnon frequencies) are associated with the equation of motion of $x^{i\mu}$. Considering $J^{\triangle\alpha,\triangle'\beta}=\delta^{\triangle\alpha,\triangle'\beta}$ for models with only nn interactions these equations read
\begin{equation}
 \dot{x}^{i\mu} = \sigma^{i\mu,j\nu} {\mathcal R}^{T}_{j\nu,\triangle\alpha}
\delta^{\triangle\alpha,\triangle'\beta}{\mathcal R}_{\triangle'\beta,k\lambda} x^{k\lambda}\to \dot{\bf x} = {\bf \sigma}{\bf \mathcal R}^T{\bf \mathcal R}{\bf x},
 \label{eigen1}
\end{equation}
where $\sigma^{i\mu,j\nu}=\{x^{i\mu},x^{j\nu}\}=\delta^{ij}\epsilon^{\mu\nu}$ is the Poisson bracket tensor with $\epsilon^{\mu\nu}$ the two-dimensional Levi-Civita tensor. Using the momentum space representation following Eq.~\ref{fourier_A}, the magnon frequencies $\omega({\bf{k}})$ are calculated [and plotted in Fig.~\ref{txt-figg4} (d) of the main text] by diagonalizing the matrix ${\bf{\sigma}}{\mathcal R}^T(-{\bf{k}}) {\mathcal R}({\bf{k}})$, cf. Eq.~\ref{eigen1}.\\

%%%%%%%%%%%%%%%%%%%%%%%%%%%%% Weyl line nodes %%%%%%%%%%%%%%%%%%%%%%%%%%%%%%%%%%%%%%%%%%%%%%%%%%%%%%%%%%%%%%%%%%%%%%%%%%%

\section{Influence of Gaussian curvature of the origami sheet on magnetic structure}\label{secthree}

\begin{figure}
\centering
 \includegraphics[width=8cm]{supp_fig_5.png}
 \caption{The plot of $\mathcal{G}$ in the space of interaction ratios $J_3/J_2$ and $J_4/J_1$ (see Eq.~\ref{eq:nonflat}) with contours of specific $\mathcal{G}$ values [the solid(dashed) contours represent positive(negative) values] shown. The green point represents the ${\rm Cs}_2{\rm CeCu}_3{\rm F}_{12}$ compound which lies close to the $\mathcal{G}=0$ line. Tuning the interactions in ${\rm Cs}_2{\rm CeCu}_3{\rm F}_{12}$ would drive a topological phase transition (change in the topology of the origami sheet) characterized by the change of the sign of $\mathcal{G}$.}
\label{fig_latt3}
\end{figure}


Any smooth curved surface embedded in three-dimensional space has Gaussian curvature, the product of the two principle curvatures (inverse radii of curvature) at that point~\cite{kamien2002geometry}. The origami surfaces considered in the present work are not smooth, but exist as the limit of smooth surfaces with tubular curves along edges and spherical caps at vertices. These caps are then the only sources of Gaussian curvature on the sheet, and while their curvature diverges, the integral of curvature over the cap has a well-defined limit equal to the \emph{angle deficit} 

\begin{align}
\label{eq:gauss}
\mathcal{G}_{\hexagon} = 2\pi - \sum_{\langle ij \rangle \in \hexagon} \theta_{ij},
\end{align}

\noindent where the sum is over interior angles of the triangles adjoining the vertex~\cite{calladine1988theory}. Only when this curvature vanishes can all the edges adjoining the vertex become coplanar, a necessary condition for a zero mode to become localized at the vertex. This mode consists of the vertex displacing transverse to the plane, or to the six spins rotating transverse to their own plane in the magnetic system. For the nonflattenable origami sheet corresponding to the ground state of the KHAF model in ${\rm Cs}_2{\rm CeCu}_3{\rm F}_{12}$, vertices have a finite Gaussian curvature $\pm \mathcal{G}$ with

\begin{equation}
 \mathcal{G} = 4\cos^{-1}\frac{J_3}{2J_2} - 4\cos^{-1}\frac{J_4}{2J_1},
 \label{eq:nonflat}
\end{equation}

\noindent which arises because of the two distinct types of kagome triangles, consequently two distinct types of origami triangles whose shapes are determined by the ratios $J_4/J_1$ (the kagome triangle consisting of two $J_4$ and one $J_1$) and $J_3/J_2$ (the kagome triangle consisting of two $J_3$ and one $J_2$) as shown in Supplementary Fig.~\ref{fig_latt2} (b). When the ratios are the same, it leads to a flat sheet (locally flat at each vertex) with $\mathcal{G}=0$. 

One can make distortions to the compound to track the variation of $\mathcal{G}$ as the interactions change (there exist a number of experiments on kagome systems where tuning the exchange interactions have been made possible by applying uniaxial stress or pressure, see the main text for references). A plot of $\mathcal{G}$ in the space of $J_3/J_2$ and $J_4/J_1$ reveals interesting features as shown in Supplementary Fig.~\ref{fig_latt3}. The plot suggests that tuning those parameters appropriately could drive a transition in which the quantity $\mu\equiv{\rm sgn}(\mathcal{G})$ flips. Consequently, the origami sheet passes from one nonflattenable state to another through a locally flat sheet (locally flat at each vertex). Emergence of zero modes corresponding to that flat sheet at the transition point (when the ratios $J_3/J_2$ and $J_4/J_1$ are the same) indicates a topological phase transition characterized by $\mu$ signifying it a topological invariant for the KHAF model. 

While the curvatures of individual vertices are local, geometric properties, the curvature of the whole unit cell is fixed by the fundamental topology of the two-dimensional surface. The celebrated Gauss-Bonnet theorem~\cite{kamien2002geometry} holds that for any compact two-dimensional manifold $M$, its Euler characteristic $\chi$ can be related to the integrals of its Gaussian curvature $K$ over its bulk and its geodesic curvature $k_g$ around its boundary:

\begin{align}
2\pi \chi = \iint K d A + \oint k_g d \ell.
\end{align}

\noindent We choose as our boundary a counter-clockwise path around one or more unit cells of the origami sheet. This disk-type patch has Euler characteristic one and because the boundary is along the periodic cell in opposite directions, most of its curvature terms cancel out. What remains are the rotations that account for the full revolution of the orientation vector by $2\pi$, canceling out with the contribution from the nonzero Euler characteristic. As a result, we have that the Gaussian curvature integrated over the unit cell must vanish:

\begin{align}
\iint_\textrm{cell} K d A = \sum_{\hexagon\in\textrm{cell}} \mathcal{G}_{\hexagon} = 0.
\end{align}

\noindent Thus, topology restricts the geometry of the origami such that the total angle deficit around the unit cell must vanish. This is particularly striking for ${\rm Cs}_2{\rm CeCu}_3{\rm F}_{12}$, which has only two types of vertex, which then must necessarily have opposite angle deficits.

\section{Topological Weyl line nodes in periodic spin origami}\label{secfour}
We generate generic periodic origami, origami with no special point group symmetry just a translational symmetry and study their Weyl line nodes. We do so by solving for origami sheets of a corresponding kagome antiferromagnet with no special conditions on their exchange interactions other than the star condition discussed in the main text. As shown in Supplementary Fig.~\ref{fig_weyl}, these line nodes move around as we tune parameters and can even annihilate and are characterized by the topological invariant $\eta({\bf k})$ as expected from the theory presented in the main text. 

\begin{figure*}
\centering
 \includegraphics[width=17.6cm]{supp_fig_2.png}
 \caption{(a) Weyl line nodes (thick red lines) appearing in the plot of the topological number $\eta$ (Eq.~\ref{txt-eq:topo} of the main text) over the Brillouin zone of a generic KHAF corresponding to a nonflattenable origami surface. The lines separate the two regions of positive (yellow) and negative (blue) values of $\eta$ denoted by `+' and `-' respectively. The quantity $t_0$ produces distortions involving the coupling constants of the spin model in Eq.~\ref{txt-eq:ham1} of the main text. A continuous deformation of the sheet by tuning a parameter (we call $t_0$) sequentially leads to (a)$\rightarrow$(b)$\rightarrow$(c)$\rightarrow$(d) in which first a pair of line nodes vanishes in (b) (at the Brillouin zone boundary the vanishing takes place at the $M$ points), then the `-' region gradually shrinks down as in (c) till in (d), the nodes disappear altogether except the $\Gamma$ point which remains always gapless owing to the global spin rotation symmetry.}
 \label{fig_weyl} 
\end{figure*}


%%%%%%%%%%%%%%%%%%%%%%%%%%%%% Symmetry analysis %%%%%%%%%%%%%%%%%%%%%%%%%%%%%%%%%%%%%%%%%%%%%%%%%%%%%%%%%%%%%%%%%%%%%%%%%%%

\section{Symmetry analysis of ${\rm Cs}_2{\rm CeCu}_3{\rm F}_{12}$ ground state}\label{secfive}

The unexpected Dirac line nodes (doubly degenerate line nodes) appear together with a point group symmetry in our analysis of ${\rm Cs}_2{\rm CeCu}_3{\rm F}_{12}$. The generic nonflattenable origami presented in Fig.~\ref{txt-figg2} (d) of the main text has singly degenerate line nodes and no point group symmetry. So it is natural to expect the double degeneracy follows from the added symmetry. Here we prove that this is the case.\\


To reveal the symmetry of ${\rm Cs}_2{\rm CeCu}_3{\rm F}_{12}$'s periodic nonflattenable origami sheet, we have plotted in Supplementary Fig.~\ref{figM1}, one component of its spin structure in real space as well constructed a ``sphere plot" of the 12 spins in its unit cell. The sphere plot places the tail of each spin at the origin of three dimensional Euclidean space and illuminates the possibility of a symmetry in spin space that might be combined with a space group transformation.\\

\begin{figure}
\centering
 \includegraphics[width=6cm]{supp_fig_3a.png}
 \includegraphics[width=6cm]{supp_fig_3b.png}
 \caption{Plotting spin ground state pattern to reveal symmetries. {\bf Left}: the $x$-component of the spin vectors plotted on the kagome lattice. This reveals the $C_2^{site}$ symmetry about sites 0, 3, 8 and 11. {\bf Right}: a ``sphere plot'' of the spins within the unit cell consisting of placing the tail of each spin at the origin of a 3 dimensional Euclidean space. This reveals a mirror symmetry $M$ in spin space.}
\label{figM1}
\end{figure}


The plots show the following symmetries exist:

\begin{itemize}
\setlength{\itemindent}{0.63cm}
\item A 180$^o$ $C_2^{site}$ spatial rotation symmetry about four of the sites within the unit cell that sends ${\bf r} \equiv {\bf R} + {\bf d} \to -{\bf r}$. These are all the same transformation up to translations (i.e. the same transformation within the point group).
\item A mirror symmetry $M$ in spin space followed by either a second $C_2^{hex}$ rotation about the center of a hexagon or a translation $T_1^{1/2}$ by half the unit cell of the ordering pattern.
\end{itemize}


In total, the three transformations $C_2^{site}$, $MC_{2}^{hex}$ and $MT_1^{1/2}$ together with the identity make the group $Z_2\times Z_2$. Because all variables are real in real space, there is also a complex-conjugation symmetry $K$ as pointed out in Ref.~\cite{kane2014topological} of the main text. This additional symmetry plays a similar role as time reversal symmetry in magnetic space groups. It can be combined with any of the above four unitary symmetries. Hence there are eight elements to the point group. \\


In ${\bf k}$-space, the symmetry becomes more interesting. Remarkably, four of the eight elements of the preceding symmetry group transform a given point in ${\bf k}$ space into itself. This is because $C_2$ sends ${\bf k} \to -{\bf k}$ and so does $K$. The symmetry group at a ${\bf k}$ point is then
\begin{equation}
G(k) = \{e,KC_2^{site}({\bf k}), KMC_2^{hex}, MT_1^{1/2}\}, \nonumber
\end{equation}
which is again $Z_2\times Z_2$. \\


The consequences of this symmetry are the following:

\begin{enumerate}
\item The physical frequencies have doubly degenerate bands and a line node must have double degeneracy.
\item The rigidity matrix ${\mathcal R}({\bf k})$ can be made real and placed in a block diagonal form with two blocks ${\mathcal R}_+({\bf k})$ and ${\mathcal R}_-({\bf k})$, one for $MT_1^{1/2} = +1$ and one for $MT_1^{1/2}=-1$.
\item Either $\eta_+ \equiv \text{sign}\ \text{Det} [{\mathcal R}_+({\bf k})]$ or $\eta_- \equiv \text{sign}\ \text{Det} [{\mathcal R}_-({\bf k})]$ serves as a topological invariant that demands the gap closes if it changes. They must change sign at the Dirac line nodes.
\item The previous topology characterized by the topological invariant $\eta$ is rendered trivial for $\eta=\eta_+\eta_-$ never changes sign in the Brillouin zone and so never changes so long as this symmetry is preserved.
\end{enumerate}


The remaining parts of this section prove these consequences.\\

To prove that the physical frequencies are doubly degenerate we start with the following observation. If the mirror transformation $M$ in the spin space is chosen to be the $xy$-plane, then when acting on the spin deviations ${\bf x}({\bf k})$ it takes the form of a $\tau_z$ Pauli matrix. Since the Poisson bracket tensor in this space is $\sigma = \iota\tau_y$ we discover
\begin{equation}
  \{KMC_2^{hex}, {\bf{\sigma}} {\mathcal R}^T(-{\bf{k}}) {\mathcal R}({\bf{k}})\} = 0~;~
  \{MT_1^{1/2}, {\bf{\sigma}} {\mathcal R}^T(-{\bf{k}}) {\mathcal R}({\bf{k}})\} = 0. \nonumber
\end{equation}
Namely, transformations involving the mirror symmetry $M$ in the spin space anti-commute with the matrix entering the eigenvalue problem for the physical frequencies. This in turn implies the eigenvalues of this matrix come in pairs with opposite signs since if $v$ is an eigenvector of ${\bf{\sigma}} {\mathcal R}^T(-{\bf{k}}) {\mathcal R}({\bf{k}})$ with eigenvalue $\iota\omega$ then $MT_1^{1/2}\cdot v$ is an eigenvector with eigenvalue $-\iota\omega$. But the physical frequency this eigenmode corresponds to is still the same positive number $\omega$. Hence the bands are doubly degenerate.\\


To prove that the rigidity matrix takes a block diagonal form, we need to transform to the basis where the representation of the group that acts on ${\mathcal R}({\bf{k}})$ in k-space is irreducible.
%
We carry this out in two steps. First we diagonalize matrices corresponding to the $MT_1^{1/2}$ transformation. Since the square of this symmetry is the identity, it has eigenvalues $\pm 1$ as noted above. Thus in the basis that diagonalizes this matrix, ${\mathcal R}({\bf{k}})$ forms two blocks. Next we transform from this basis to a new basis which diagonalizes the antiunitary transformation $KC_2^{site}({\bf k})$. Since the square of this symmetry is also the identity, we can transform to a basis where ${\mathcal R}({\bf{k}})$ is real. Because the symmetry is antiunitary, however, we can find a complete basis where all eigenvectors have eigenvalue $KC_2^{site}({\bf k})=+1$.
%
We do so as follows. Starting with one eigenvector $v_1$ of $MT_1^{1/2}$ we create $w_1 = v_1 + KC_2^{site}({\bf k})\cdot v_1$ if this is non-zero otherwise we set $w_1 = \iota v_1$. We then take another eigenvector $v_2$ and create $w_2$ in a similar way. But we further apply Gram-Schmidt to $w_2$ following Wigner~\cite{wigner1960normal} to ensure it is orthogonal to $w_1$. We then continue with each eigenvector $v$ in $MT_1^{1/2}$. In the end, the procedure produces a basis $\{w_i\}$ which has eigenvalues $MT_1^{1/2}=\pm 1$ and $KC_2^{site}({\bf k})=1$. Since there are no further symmetries to use because $KMC_2^{hex}$ is just the product of these two symmetries, this is the eigenbasis of the symmetry group at a ${\bf k}$ point. \\


To define a continuous basis throughout the Brillouin zone, one needs to adopt a smooth gauge of parallel transport by comparing this basis to the one similarly produced at a nearby k-point ${\bf k}'$. The procedure involves the non-abelian form of the Berry connection, specifically the overlap matrix between the two bases at adjacent points ${\bf k}$ and ${\bf k}'$ as $\mathcal{B}_{ij} = v_i^\dagger({\bf k})\cdot v_j({\bf k}')$   ~\cite{soluyanov2012smooth}. One then computes the singular value decomposition $\mathcal{B}= U\cdot\Sigma\cdot V^\dagger$ and carries out a unitary transformation of the basis at ${\bf k}'$ as
\begin{equation}
 v_i ({\bf k}') = \sum_j (V\cdot U^\dagger)_{ji}  ~ v_j ({\bf k}') \nonumber
\end{equation}
This is done separately for the eigenvalue $MT_1^{1/2}= 1$ and $MT_1^{1/2}=-1$ bases. The resulting matrix $\mathcal{B}' = U\cdot\Sigma\cdot U^\dagger$ is Hermitian and thus defines a parallel transport of the vectors and a continuous basis in the Brillouin zone as required. \\


Finally, the antisymmetry and the block-diagonal form of ${\mathcal R}$ additionally explains the double degeneracy of the Dirac line nodes, the behavior of the new topological invariants $\eta_+$, $\eta_-$ and the behavior of the old topological invariant $\eta$. Consider an eigenvector $v$ of ${\bf{\sigma}} {\mathcal R}^T(-{\bf{k}}) {\mathcal R}({\bf{k}})$ with eigenvalue $\iota\omega$. The eigenvector $MT_1^{1/2}\cdot v$ has eigenvalue $-\iota\omega$ by the antisymmetry. This further implies in this subspace of two vectors that the transformation $MT_1^{1/2} = \sigma_x$, where $\sigma_x$ is the usual Pauli matrix. But this matrix has two eigenvalues $\pm1$ and so in the basis that places ${\mathcal R}({\bf{k}})$ in block diagonal form and makes it real, one zero mode belongs to the $MT_1^{1/2}= 1$ sector and one belongs to the $MT_1^{1/2}=- 1$ sector. Hence, as $\omega$ passes through zero at a line node, both $\eta_+$ and $\eta_-$ change sign and their product $\eta$ always remains the same throughout the Brillouin zone.


\bibliographystyle{unsrt}
\bibliography{reference_ordered}

\end{document}
