\documentclass[superscriptaddress,amsmath,amssymb,aps,onecolumn]{revtex4}

%\documentclass[aps,prb,superscriptaddress,floatfix,twocolumn]{revtex4}

%\usepackage{babel}
%\addto{\captionsenglish}{\renewcommand*{\appendixname}{}}
\usepackage{wrapfig}
\usepackage{lipsum}

\usepackage{tikz}
\usepackage{graphicx}% Include figure files
\graphicspath{{figures/}}
\usepackage{dcolumn}% Align table columns on decimal point
%\usepackage{multicol}
\usepackage{bm}% bold math
\usepackage{hyperref}
\usepackage{multirow}
\usepackage{array}
\usepackage{booktabs}
\usepackage{ctable}
\usepackage{upgreek}
\usepackage{epsfig,psfrag,subfigure,amsopn}
\usepackage{mathrsfs}
\usepackage{amssymb}
\usepackage{amsbsy}
\usepackage{color}
\usepackage{cancel}
\usepackage{}
\usepackage{xr}
\externaldocument{manuscript-resubmit-clean}
\def\greenw#1{{\color{black} #1}}

\usepackage{tcolorbox}

%\usepackage{epsf}
\usepackage{pifont}
\usepackage{marginnote}
\usepackage{float}
%\usepackage{eufrak}
%%----------------------------------------------------------------------
%%----------------------------------------------------------------------
%%----------------------------------------------------------------------
%%----------------------------------------------------------------------
%%my definitions
\newcommand{\bcen}{\begin{center}}
	\newcommand{\ecen}{\end{center}}
\newcommand{\btab}{\begin{tabular}}
	\newcommand{\etab}{\end{tabular}}
\newcommand{\bdes}{\begin{description}}
	\newcommand{\edes}{\end{description}}
\newcommand{\mc}{\multicolumn}
\newcommand{\ul}{\underline}
\newcommand{\beq}{\begin{equation}}
\newcommand{\eeq}{\end{equation}}
\newcommand{\bea}{\begin{eqnarray}}
\newcommand{\eea}{\end{eqnarray}}
\newcommand{\non}{\nonumber}
\newcommand{\etal}{et.~al.\ }
\newcommand{\half}{\frac{1}{2}}
\newcommand{\bary}{\begin{array}}
	\newcommand{\eary}{\end{array}}
\newcommand{\benum}{\begin{enumerate}}
	\newcommand{\eenum}{\end{enumerate}}
\newcommand{\bitem}{\begin{itemize}}
	\newcommand{\eitem}{\end{itemize}}
\newcommand{\cuup}[1]{c_{#1 \uparrow}}
\newcommand{\cdown}[1]{c_{#1 \downarrow}}
\newcommand{\cdup}[1]{c^\dagger_{#1 \uparrow}}
\newcommand{\cddown}[1]{c^\dagger_{#1 \downarrow}}
%
%bold greek characters
%
\newcommand{\al}{\alpha}
\newcommand{\de}{\delta}
\newcommand{\ep}{\epsilon}
\newcommand{\ga}{\gamma}
\newcommand{\lam}{\lambda}
\newcommand{\om}{\omega}
\newcommand{\si}{\sigma}
\newcommand{\dg}{\dagger}
\newcommand{\beps}{\mbox{\boldmath $ \epsilon $}}
\newcommand{\bsig}{\mbox{\boldmath $ \sigma $}}
\newcommand{\bpi}{\mbox{\boldmath $ \pi $}}
\newcommand{\bkap}{\mbox{\boldmath $ \kappa $}}
\newcommand{\bgam}{\mbox{\boldmath $ \gamma $}}
\newcommand{\bphi}{\mbox{\boldmath $ \phi $}}
\newcommand{\balp}{\mbox{\boldmath $ \alpha $}}
\newcommand{\beot}{\mbox{\boldmath $ \eta $}}
\newcommand{\btau}{\mbox{\boldmath $ \tau $}}
\newcommand{\blam}{{\boldsymbol{\lambda}}}
\newcommand{\bomg}{\mbox{\boldmath $ \omega $}}
\newcommand{\bOmg}{\mbox{\boldmath $ \Omega $}}
\newcommand{\bxhi}{\mbox{\boldmath $ \xi $}}
\newcommand{\bmu} {\mbox{\boldmath $ \mu $}}
\newcommand{\bnu} {\mbox{\boldmath $ \nu $}}
\newcommand{\bdelta}{{\boldsymbol{\delta}}}
\newcommand{\bDelta}{{\boldsymbol{\Delta}}}
\newcommand{\bPi}{{\boldsymbol{\Pi}}}
\newcommand{\bpsi}{\mbox{\boldmath $ \psi $}}
\newcommand{\brho}{\mbox{\boldmath $ \rho $}}
\newcommand{\bGam}{{\boldsymbol{\Gamma}}}
\newcommand{\bLam}{\mbox{\boldmath $ \Lambda $}}
\newcommand{\bPhi}{\mbox{\boldmath $ \Phi $}}
\newcommand{\bOne}{{\boldsymbol 1}}
%
%bold latin
%
%\newcommand{\ba} { \mbox{\boldmath $a$}}
\newcommand{\ba} { \bm{a} }
\newcommand{\bb} { \mbox{\boldmath $b$}}
%\newcommand{\bc} { \mbox{\boldmath $c$}}
\newcommand{\bc} { {\mathbf c} }
\newcommand{\bd} { \mbox{\boldmath $d$}}
\newcommand{\be} { \mbox{\boldmath $e$}}
\newcommand{\bff}{ \mbox{\boldmath $f$}}
\newcommand{\bg} { \mbox{\boldmath $g$}}
\newcommand{\bh} { \mbox{\boldmath $h$}}
%\newcommand{\bi} { \mbox{\boldmath $i$}}
\newcommand{\bi}{\bibitem}
\newcommand{\bj} { \mbox{\boldmath $j$}}
\newcommand{\bk} { \bm{k} }
%\newcommand{\bk} { \mbox{\boldmath $k$}}
\newcommand{\bl} { \mbox{\boldmath $l$}} 
\newcommand{\bmm} { \mbox{\boldmath $m$}}
\newcommand{\bn} { \mbox{\boldmath $n$}}
\newcommand{\bo} { \mbox{\boldmath $o$}}
\newcommand{\bp} { \bm{p} }
\newcommand{\bq} { \bm{q} }
%\newcommand{\bq} { \mbox{\boldmath $q$}}
\newcommand{\br} { \boldsymbol{r}}
\newcommand{\bs} { \mbox{\boldmath $s$}}
\newcommand{\bt} {\boldsymbol{t}} 
\newcommand{\bu} { \mbox{\boldmath $u$}}
\newcommand{\bv} { \mbox{\boldmath $v$}}
\newcommand{\bw} { \mbox{\boldmath $w$}}
\newcommand{\bx} { \mbox{\boldmath $x$}}
\newcommand{\by} { \mbox{\boldmath $y$}}
\newcommand{\bz} { \mbox{\boldmath $z$}}

\newcommand{\bA} { \mbox{\boldmath $A$}}
\newcommand{\bB} { \mbox{\boldmath $B$}}
\newcommand{\bC} { \mbox{\boldmath $C$}}
\newcommand{\bD} { \mbox{\boldmath $D$}}
\newcommand{\bE} { \mbox{\boldmath $E$}}
\newcommand{\bF} { \mbox{\boldmath $F$}}
\newcommand{\bG} { \mbox{\boldmath $G$}}
\newcommand{\bH} { \mbox{\boldmath $H$}}
\newcommand{\bI} { \mbox{\boldmath $I$}}
\newcommand{\bJ} { \mbox{\boldmath $J$}}
\newcommand{\bK} { \mbox{\boldmath $K$}}
\newcommand{\bL} { \mbox{\boldmath $L$}}
\newcommand{\bM} { \mbox{\boldmath $M$}}
\newcommand{\bN} { \mbox{\boldmath $N$}}
\newcommand{\bO} { \mbox{\boldmath $O$}}
\newcommand{\bP} { \mbox{\boldmath $P$}}
\newcommand{\bQ} { \boldsymbol{Q} }
%\newcommand{\bQ} { \mbox{\boldmath $Q$}}
\newcommand{\bR} { {\mathbf R} }
%\newcommand{\bR} { \mbox{\boldmath $R$}}
\newcommand{\bS} { \mbox{\boldmath $S$}}
\newcommand{\bT} { \mbox{\boldmath $T$}}
\newcommand{\bU} { \mbox{\boldmath $U$}}
\newcommand{\bV} { \mbox{\boldmath $V$}}
\newcommand{\bW} { \mbox{\boldmath $W$}}
\newcommand{\bX} { \mbox{\boldmath $X$}}
\newcommand{\bY} { \mbox{\boldmath $Y$}}
\newcommand{\bZ} { \mbox{\boldmath $Z$}}
\newcommand{\bzero} { {\boldsymbol{0}}}
\newcommand{\bfell} {\mbox{\boldmath $ \ell $}}

%
%special math symbols
%
\newcommand{\dou}{\partial}
\newcommand{\leftjb} {[\![}
\newcommand{\rightjb} {]\!]}
\newcommand{\ju}[1]{ \leftjb #1 \rightjb }
\newcommand{\D}[1]{\mbox{d}{#1}} 
\newcommand{\grad}{\mbox{\boldmath $\nabla$}}
\newcommand{\modulus}[1]{|#1|}
\renewcommand{\div}[1]{\grad \cdot #1}
\newcommand{\curl}[1]{\grad \times #1}
\newcommand{\mean}[1]{\langle #1 \rangle}
\newcommand{\bra}[1]{{\langle #1 |}}
\newcommand{\ket}[1]{| #1 \rangle}
\newcommand{\braket}[2]{\langle #1 | #2 \rangle}
\newcommand{\dbdou}[2]{\frac{\dou #1}{\dou #2}}
\newcommand{\dbdsq}[2]{\frac{\dou^2 #1}{\dou #2^2}}
\newcommand{\Pint}[2]{ P \!\!\!\!\!\!\!\int_{#1}^{#2}}
\newcommand{\PARA}[1]{{\noindent {$\bigstar$} {\sc #1}:~}}

%
%abbreviations for equations etc
% 
\newcommand{\eqn}[1] {eqn.~(\ref{#1})}
\newcommand{\prn}[1] {(\ref{#1})}
\newcommand{\sect}[1] {Section~\ref{#1}}
\newcommand{\Sect}[1] {Section~\ref{#1}}
\newcommand{\fig}[1]{fig.~\ref{#1}}
\newcommand{\Fig}[1]{Fig.~\ref{#1}}

%
% Roman Numerals
%
\makeatletter
\newcommand{\rmnum}[1]{\romannumeral #1}
\newcommand{\Rmnum}[1]{\expandafter\@slowromancap\romannumeral #1@}
\makeatother

%
%Other utilities
%
\newcommand{\uncon}[1]{\centerline{\epsfysize=#1 \epsfbox{/usr2/shenoy/styles/construction.eps}}}
\newcommand{\checkup}[1]{{(\tt #1)}\typeout{#1}}
\newcommand{\ttd}[1]{{\color[rgb]{1,0,0}{\bf #1}}}
\newcommand{\ttds}[1]{{\color[rgb]{0,0,1}{\bf #1}}}
\newcommand{\red}[1]{{\color[rgb]{1,0,0}{\protect{#1}}}}
\newcommand{\blue}[1]{{\color[rgb]{0,0,1}{#1}}}
\newcommand{\green}[1]{{\color[rgb]{0.0,0.5,0.0}{#1}}}
\newcommand{\citebyname}[1]{\citeauthor{#1}\cite{#1}}
\newcommand{\signum}[0]{\mathop{\mathrm{sign}}}
\newcommand{\skup}{\ket{s \uparrow}}
\newcommand{\skdn}{\ket{s \downarrow}}
\newcommand{\pkup}{\ket{p \uparrow}}
\newcommand{\pkdn}{\ket{p \downarrow}}
\newcommand{\sbup}{\bra{s \uparrow}}
\newcommand{\sbdn}{\bra{s \downarrow}}
\newcommand{\pbup}{\bra{p \uparrow}}
\newcommand{\pbdn}{\bra{p \downarrow}}

%width of figures
\newcommand{\myfigwidth}{0.4\paperwidth}
\newcommand{\myhalffigwidth}{0.2\paperwidth}

\newcommand{\asc}{a_{sc}}
\newcommand{\as}{a_s}
\newcommand{\Eb}{E_b}
\newcommand{\Ef}{E_F}
\newcommand{\kf}{k_F}
\newcommand{\lambdaT}{{\lambda_T}}
\newcommand{\cF}{{\cal F} }
\newcommand{\ie}{{i.e., } }
\newcommand{\etaT}{{\eta_t}}
\newcommand{\muNI}{{\mu_{NI}}}

\setcounter{equation}{0}

\setcounter{figure}{0}

\renewcommand{\theequation}{S\arabic{equation}}

\renewcommand{\thefigure}{S\arabic{figure}}


%set this to see the name of the labels in the margins
%\newcommand{\mylabel}[1]{\label{#1}{\marginnote{\tiny{\tt #1}}}}
%
%or this for this for doing nothing
\newcommand{\mylabel}[1]{\label{#1}} 

%%
\newcommand{\myonlinecite}[1]{[\onlinecite{#1}]}
\newcommand{\mycite}[1]{\cite{#1}}
%%%%%\newcommand{\mycite}[1]{{\tt[#1]}\cite{#1}}
%\newcommand{\mycite}[1]{\cite{#1}}
\newcommand{\sectionprl}[1]{{\em #1}\/.---}




\begin{document}
	
	%\red{\textbf{Very Preliminary}}	
	
	%\author{}
	%	\email{saikat.santra@icts.res.in}
	%	\affiliation{International Centre for Theoretical Sciences, Tata Institute of Fundamental Research, Bengaluru 560089, India}
	
	\date{\today}
	\newcommand{\titlename}{Supplementary material to ``Extreme value statistics and arcsine laws for heterogeneous diffusion processes"}
	
	
	\title{\titlename}
	
\author{Prashant Singh}
\email{prashant.singh@icts.res.in}
\affiliation{International Centre for Theoretical Sciences, Tata Institute of Fundamental Research, Bengaluru 560089, India}
\maketitle
	
	%\section*{\underline{Supplementary Material}}
	%\tableofcontents 
	
	%\vspace{0.8cm}
%\begin{multicols}{2}	
This supplementary material (SM) provides a detailed derivation of some results which were used in the main text. For self-containedness, let us recall our model of heterogeneous diffusion processes (HDP) where the position of the particle evolves as
\begin{align}
\frac{dx}{dt} = \sqrt{2 D(x)} ~\eta(t),
\label{new-appen-model-eq-1}
\end{align}
where $\eta(t)$ is the Gaussian white noise with zero mean and correlation $\langle \eta(t) \eta(t') \rangle = \delta(t-t')$. We focus on the power-law form of the diffusion coefficient:
\begin{align}
D(x) =  \frac{D_0~l^{\alpha}}{|x|^{\alpha}},
%\label{new-appen-model-eq-2}
\label{new-appen-extreme-eq-3}
\end{align}
where $D_0$ is a positive constant that sets the strength of the noise and $l$ is the length scale over which $D(x)$ changes. The exponent $\alpha$ quantifies the strength of the gradient of $D(x)$. Throughout this SM, we consider $\alpha >-1$ and interpret Eq. \eqref{new-appen-model-eq-1} in Ito-setup.
	
\section{Derivation of the survival probability $S_M(t|x_0)$}
\label{appen-surv}
In this section, we derive the expression of the survival probability $S_M(t|x_0)$ which is significant in computing the extremal statistics of the model. To this aim, we begin with the backward Fokker Planck equation for $S_M(t|x_0)$ in Ito-set up \cite{newRedner}
\begin{align}
 \partial _t S_M(t|x_0) & = D(x_0) \partial _{x_0 } ^2 S_M(t|x_0). 
\end{align} 
Taking Laplace transformation of this equation with respect to $t ~(\to s)$ yields
\begin{align}
s \bar{S}_M(s|x_0)-1 = D(x_0) \partial _{x_0 }^2 \bar{S}_M(s|x_0).
\label{extreme-eq-6}
\end{align}
Next, we perform the transformation
\begin{align}
\bar{S}_M(s|x_0)=\frac{1}{s}+U(x_0),
\label{appen-surv-eq-1}
\end{align}
in Eq. \eqref{extreme-eq-6} which results in a homogeneous differential equation of the form
\begin{align}
s U(x_0) = D(x_0) \partial _{x_0 }^2 U(x_0),
\label{appen-surv-eq-2}
\end{align}
with $D(x_0)$ defined in Eq. \eqref{new-appen-extreme-eq-3}. Recall that $M \geq 0$ and $x_0 \leq M$. We proceed to solve Eq. \eqref{appen-surv-eq-2} separately for $x_0 >0$ and $x_0 <0$ regions. For $x_0 >0$, we perform the transformation $y = \frac{x_0 ^{2+\alpha}}{(2+\alpha)^2 l^{\alpha} D_0}$ and rewrite Eq. \eqref{appen-surv-eq-2} as
\begin{align}
y \frac{\partial ^2 U}{\partial y^2} + \left(\frac{1+\alpha}{2+\alpha} \right) \frac{\partial  U}{\partial y} = s U.
\label{appen-surv-eq-3}
\end{align}
The solution of this equation is given in terms of the modified Bessel functions as
\begin{align}
U(y) = \mathbb{C}_1 y^{\frac{1}{2(2+\alpha)}} K_{\frac{1}{2+\alpha}} \left( 2 \sqrt{s y}\right)+\mathbb{C}_2 y^{\frac{1}{2(2+\alpha)}} I_{\frac{1}{2+\alpha}} \left( 2 \sqrt{s y}\right).
\label{appen-surv-eq-4}
\end{align}
Here $\mathbb{C}_1$ and $\mathbb{C}_2$ are constants independent of $y$ but may depend on $s$. Writing the solution in terms of $x_0$ and using Eq. \eqref{appen-surv-eq-1} to write $\bar{S}_M(s|x_0)$ in terms of $U(x_0)$, we get
\begin{align}
\bar{S}_M(s|x_0) &= \frac{1}{s} +\mathbb{C}_1 \sqrt{x_0}K_{\frac{1}{2+\alpha}} \left(  (a_s x_0)^{\frac{2+\alpha}{2}}\right)+\mathbb{C}_2 \sqrt{x_0}I_{\frac{1}{2+\alpha}} \left(  (a_s x_0)^{\frac{2+\alpha}{2}}\right).
\label{appen-surv-eq-5}
\end{align} 
where the function $a_s$ is given by
\begin{align}
a_s  = \left(\frac{s}{\mathcal{D}_{\alpha}} \right)^{\frac{1}{2+\alpha}},~~~\text{with }\mathcal{D}_{\alpha} = \frac{D_0 l^{\alpha} (2+\alpha)^2}{4}.\label{new-appen-extreme-eq-9}
\end{align}
Recall that the solution in Eq. \eqref{appen-surv-eq-5} is true only for $x_0 >0$. For $x_0 <0$, we proceed similarly to get
\begin{align}
\bar{S}_M(s|x_0) = &\frac{1}{s} +\mathbb{C}_3 \sqrt{|x_0|}K_{\frac{1}{2+\alpha}} \left(  (a_s |x_0|)^{\frac{2+\alpha}{2}}\right) +\mathbb{C}_4 \sqrt{|x_0|}I_{\frac{1}{2+\alpha}} \left(  (a_s |x_0|)^{\frac{2+\alpha}{2}}\right).
\label{appen-surv-eq-6}
\end{align}
The task now is to evaluate the constants $\mathbb{C}_1,~\mathbb{C}_2,~\mathbb{C}_3$ and $\mathbb{C}_4$ which are, in principle, functions of $s$. To compute them, we first note that the survival probability $S_M(t|x_0)$ and its derivative $\partial _{x_0} S_M(t|x_0)$ are continuous across $x_0 = 0$ which implies that $\bar{S}_M(s|x_0)$ and $\partial _{x_0} \bar{S}_M(s|x_0)$ are also continuous. This implies
\begin{align}
& \bar{S}_M(s|x_0 \to 0^+) = \bar{S}_M(s|x_0 \to 0^-), \label{appen-surv-eq-7} \\
& \left[\frac{\partial \bar{S}_M(s|x_0 )}{\partial x_0}\right] _{x_0 \to 0^+} = \left[\frac{\partial \bar{S}_M(s|x_0 )}{\partial x_0}\right] _{x_0 \to 0^-}. \label{appen-surv-eq-8}
\end{align}
\greenw{Note that continuity of the derivative in the second equation is well-defined only for $\alpha >-1$ as can be verified explicitly by plugging the solutions of $\bar{S}_M(s|x_0)$ from Eqs. \eqref{appen-surv-eq-5} and \eqref{appen-surv-eq-6}.} Next, we recall the boundary conditions of $S_M(t|x_0)$ in Eqs. (20) and (21) in the main text. Translating these conditions in terms of $\bar{S}_M(s|x_0 )$ yields
\begin{align}
& \bar{S}_M(s|x_0 \to M) = 0,  \label{appen-surv-eq-9} \\
& \bar{S}_M(s|x_0 \to -\infty) = \frac{1}{s} \label{appen-surv-eq-10}.
\end{align}
Using the four conditions [Eqs. \eqref{appen-surv-eq-7}-\eqref{appen-surv-eq-10}], we obtain the constants $\mathbb{C}_1,~\mathbb{C}_2,~\mathbb{C}_3$ and $\mathbb{C}_4$. This, then, completely specifies the Laplace transform $\bar{S}_M(s|x_0)$ in Eqs. \eqref{appen-surv-eq-5} and \eqref{appen-surv-eq-6}. However, since we are eventually interested in computing the joint distribution of the maximum $M(t)$ and arg-max $t_m(t)$ [Eq. (17) in the main text], we provide the solution only for $x_0 \geq 0$. As evident from Eq. \eqref{appen-surv-eq-5}, we then need only the expressions of $\mathbb{C}_1$ and $\mathbb{C}_2$ which read
\begin{align}
& \mathbb{C}_1 = \frac{\mathbb{C}_2}{\Gamma \left(\frac{1}{2+\alpha} \right) \Gamma \left(\frac{1+\alpha}{2+\alpha} \right)}, \label{appen-surv-eq-11} \\
& \mathbb{C}_2 = -\frac{1}{s f _{\alpha}(M)},~~~~~\text{with },\label{appen-surv-eq-12} \\
&f _{\alpha}(M) = \frac{1}{2\sqrt{a_s}} \mathcal{H}_{\frac{1}{2+\alpha}} \left( (a_s M)^{\frac{2+\alpha}{2}} \right),
\end{align}
where $\mathcal{H}_{\beta}(x_0)$ in the last equation is defined as
\begin{align}
\mathcal{H}_{\beta} (x_0) = x_0 ^{\beta} \left[ I _{\beta}(x_0)+I_{-\beta} (x_0)\right]. \label{new-appen-extreme-eq-8} 
\end{align}
Finally, inserting these expressions of $ \mathbb{C}_1$ and $ \mathbb{C}_2$ in Eq. \eqref{appen-surv-eq-5}, we find 
\begin{align}
\bar{S}_M(s|x_0) = \frac{1}{s} \left[1- \frac{\mathcal{H}_{\frac{1}{2+\alpha}} \left( (a_s x_0)^{\frac{2+\alpha}{2}} \right)}{\mathcal{H}_{\frac{1}{2+\alpha}} \left( (a_s M)^{\frac{2+\alpha}{2}} \right)} \right],
\label{appen-surv-eq-13} 
\end{align}
which has been written in Eq. (22) in the main text.


\section{Marginal distribution $P_m(M|t)$}
\label{new-appen-ILT-J}
Let us now derive the exact form of the Marginal distribution $P_m(M|t)$ for the maximum $M$. In Eq. (28) of the main text, we obtained the Laplace transform $\bar{P}_m(M|s)$ of the marginal distribution $M$ to be
\begin{align}
\bar{P}_m(M|s)&  =-\frac{d \bar{J}(M,s)}{dM},~~~~~~~~~~~~\text{with} \label{new-appen-extreme-eq-15} \\
\bar{J}(M,s)&= \frac{\mathcal{H}_{\frac{1}{2+\alpha}} \left(0 \right)}{s ~\mathcal{H}_{\frac{1}{2+\alpha}} \left( (a_s M)^{\frac{2+\alpha}{2}} \right)}, \label{new-appen-extreme-eq-16} 
\end{align}
Here, we perform the inverse Laplace transformation of $\bar{P}_m(M|s)$ to obtain the distribution $P_m(M|t)$ as quoted in Eq. (8) of the main text. From Eq. \eqref{new-appen-extreme-eq-15}, we observe that $\bar{P}_m(M|s)$ is simply the derivative of $\bar{J}(M,s)$. In the time domain, this will correspond to
\begin{align}
P_m(M|t) = - \frac{dJ(M,t)}{dM},
\label{appen-ILT-J-Eq-0}
\end{align}
where $J(M,t)$ is the inverse Lapalace transform of $\bar{J}(M,s)$. We therefore proceed to compute the inverse Laplace transformation of  $\bar{J}(M,s)$ in Eq. \eqref{new-appen-extreme-eq-16}. Formally, $J(M,t)$ can be written as
\begin{align}
J(M,t) &= \frac{1}{2 \pi i} \int _{-i \infty}^{i \infty} ds~ e^{st}~\bar{J}(M,s), \\
& = \frac{\mathcal{H}_{\frac{1}{2+\alpha}} \left(0 \right)}{2 \pi i} \int _{-i \infty}^{i \infty} ds \frac{e^{st}}{s ~\mathcal{H}_{\frac{1}{2+\alpha}} \left( \sqrt{\frac{s M^{2+\alpha}}{\mathcal{D}_{\alpha}}}  \right)}.
\label{appen-ILT-J-Eq-1}
\end{align}
Notice that $s=0$ is a branch point. To perform this Bromwich integration, we consider the contour of form shown in Figure \ref{contour-fig}. Since the integrand is analytic inside this contour, the Cauchy theorem gives
\begin{align}
\int _{\Gamma _1}+\int _{\Gamma _2}+\int _{\Gamma _3}+\int _{\Gamma _4}+\int _{\Gamma _5}+\int _{\Gamma _6} = 0.\label{appen-ILT-J-Eq-2-new}
\end{align}
$\int _{\Gamma _1} = J(M,t)$ is the integral that we need. Let us now perform these integrals along different paths separately. Recall that the real part of $s$ along $\Gamma _2$ and $\Gamma _6$ is negative and in the limit $|s| \to \infty$, the contribution becomes exactly zero.
\begin{figure}[t]
\includegraphics[scale=0.4]{contour-Bromwich.pdf}
%\includegraphics[scale=0.21]{scaling-M-alp0p5.pdf}
%\includegraphics[scale=0.25]{new_prob_dist.pdf}
\centering
\caption{Contour for the Bromwich integration of $\bar{J}(M,s)$ in Eq. \eqref{appen-ILT-J-Eq-1}}
\label{contour-fig}
\end{figure}

To evaluate integral along $\Gamma _3$, we substitute $s = R e^{i \pi}$, where $R$ varies from $\infty$ to $0$. Under this transformation, the integral $\int _{\Gamma _3}$ becomes
\begin{align}
\int _{\Gamma _3} = \frac{\mathcal{H}_{\frac{1}{2+\alpha}} \left(0 \right)}{2 \pi i} \int _{\infty}^{0} dR \frac{e^{-Rt}}{R ~\mathcal{H}_{\frac{1}{2+\alpha}} \left( i\sqrt{\frac{R M^{2+\alpha}}{\mathcal{D}_{\alpha}}}  \right)}.
\label{appen-ILT-J-Eq-2}
\end{align}
To simplify further, we substitute $w = \frac{R M^{2+\alpha}}{\mathcal{D}_{\alpha}}$ in this equation and get
\begin{align}
\int _{\Gamma _3} = -\frac{\mathcal{H}_{\frac{1}{2+\alpha}} \left(0 \right)}{2 \pi i} \int _{0}^{\infty} dw \frac{e^{-\frac{\mathcal{D}_{\alpha} t}{M^{2+\alpha}}w}}{w ~\mathcal{H}_{\frac{1}{2+\alpha}} \left( i \sqrt{w} \right)}.
\label{appen-ILT-J-Eq-3}
\end{align}
Next, we evaluate the integral along $\Gamma _5$. For this, we substitute $s = e^{-i \pi} R$ in the integrand and perform similar simplications as done for the integral $\int _{\Gamma _3}$. We then obtain
\begin{align}
\int _{\Gamma _5} = \frac{\mathcal{H}_{\frac{1}{2+\alpha}} \left(0 \right)}{2 \pi i} \int _{0}^{\infty} dw \frac{e^{-\frac{\mathcal{D}_{\alpha} t}{M^{2+\alpha}}w}}{w ~\mathcal{H}_{\frac{1}{2+\alpha}} \left( -i \sqrt{w} \right)}.
\label{appen-ILT-J-Eq-4}
\end{align}
Finally, we perform the integration along $\Gamma _4$ for which we replace $s=\delta e^{i \theta}$ and take $\delta \to 0^+$ limit. It is easy to show that the resultant integration is $\int _{\Gamma _4} = -1$.

Putting all the intergations together in Eq. \eqref{appen-ILT-J-Eq-2-new} and noting that $\int _{\Gamma _1} = J(M,t)$, we get
\begin{align}
J(M,t) &= -\int _{\Gamma _3}-\int _{\Gamma _4}-\int _{\Gamma _5}, \\
& = 1-\frac{\mathcal{H}_{\frac{1}{2+\alpha}} \left(0 \right)}{(2+\alpha)} \int _{0}^{\infty} \frac{dw}{w} e^{-\frac{\mathcal{D}_{\alpha} t}{M^{2+\alpha}}w}~\mathbb{H}_{\frac{1}{2+\alpha}}(\sqrt{w}),
\label{appen-ILT-J-Eq-5}
\end{align}
where the function $\mathbb{H}_{\beta}(w)$ is given by
\begin{align}
\mathbb{H}_{\beta}(w)= \frac{1}{2 \pi \beta i} \left[\frac{1}{\mathcal{H}_{\beta} \left( -i w \right)}-\frac{1}{\mathcal{H}_{\beta} \left( i w \right)} \right].
\label{appen-ILT-J-Eq-6}
\end{align}
Coming back to our goal of computing the distribution $P_m(M|t)$, we plug $J(M,t)$ from Eq. \eqref{appen-ILT-J-Eq-5} in Eq. \eqref{appen-ILT-J-Eq-0} to get
\begin{align}
P_m(M|t) = \frac{1}{\left( \mathcal{D}_{\alpha} t\right)^{\frac{1}{2+\alpha}}} \mathcal{F}_{\alpha} \left( \frac{M}{\left( \mathcal{D}_{\alpha} t\right)^{\frac{1}{2+\alpha}}} \right),
\label{new-appen-ILT-J-Eq-8} 
\end{align}
with $\mathcal{D}_{\alpha}$ given in Eq. \eqref{new-appen-extreme-eq-9} and the function $\mathcal{F}_{\alpha}(z)$ defined asw
\begin{align}
 \mathcal{F}_{\alpha}(z) = \frac{\mathcal{H}_{\frac{1}{2+\alpha}} \left(0 \right)}{z^{3+\alpha}} \int _{0}^{\infty} dw ~e^{-\frac{w}{z^{2+\alpha}}} ~\mathbb{H}_{\frac{1}{2+\alpha}}(\sqrt{w}), \label{new-appen-extreme-eq-18} 
\end{align}
where $\mathcal{H}_{\beta}(x_0)$ is in Eq. \eqref{new-appen-extreme-eq-8}. 

\section{Asymptotic behaviour of the scaling function $\mathcal{F}_{\alpha}(z)$}
\label{appen-asy-PM}
We now proceed to analyse the asymptotic behaviour of the scaling function  $\mathcal{F}_{\alpha}(z)$ in Eq. \eqref{new-appen-extreme-eq-18}. For simplicity, we look at the small and large $z$ behaviours separately below.
\subsection{Small $z$ behaviour of $\mathcal{F}_{\alpha}(z)$}
To obtain $\mathcal{F}_{\alpha} \left(z \to 0 \right)$, we consider the form of $\bar{P}(M|s)$ in terms of $\bar{J}(M,s)$ as shown in Eq. \eqref{new-appen-extreme-eq-15} and look at the small $M$ behaviour of $\bar{J}(M,s)$. As evident from Eq. \eqref{new-appen-extreme-eq-16}, one then needs to specify the  small-$x$ form of $\mathcal{H}_{\beta}(x)$. For $x \to 0$, the modified Bessel function is
\begin{align}
I _{\beta}(x) \simeq \frac{x^{\beta}}{2^{\beta} \Gamma(1+\beta)} + \frac{x^{2+\beta}}{2^{2+\beta}\Gamma(2+\beta)},
\label{appen-asy-PM-eq-1}
\end{align}
which we use in Eq. \eqref{new-appen-extreme-eq-8} to get
\begin{align}
\mathcal{H}_{\beta}(x) \simeq \mathcal{H}_{\beta} (0) +\frac{x^{2 \beta}}{2^{\beta} \Gamma(1+\beta)},~~~\text{as } x \to 0.
\label{appen-asy-PM-eq-2}
\end{align}
Substituting this form of $\mathcal{H}_{\beta}(x)$ in $\bar{J}(M,s)$ in Eq. \eqref{new-appen-extreme-eq-16}, we get 
\begin{align}
\bar{J}(M,s) \simeq \frac{C_{\alpha} \mathcal{D}_{\alpha} ^{\frac{1}{2+\alpha}}}{s \left( C_{\alpha} \mathcal{D}_{\alpha} ^{\frac{1}{2+\alpha}} +M s^{\frac{1}{2+\alpha}}\right)},~~~\text{as }M \to 0,
\label{appen-asy-PM-eq-3}
\end{align}
with $C_{\alpha} = \frac{2^{2/2+\alpha}}{(2+\alpha)} \frac{\Gamma \left( \frac{1}{2+\alpha}\right)}{\Gamma \left( \frac{1+\alpha}{2+\alpha}\right)}$. Next, we insert Eq. \eqref{appen-asy-PM-eq-3} in Eq. \eqref{new-appen-extreme-eq-15} to yield the small-$M$ behaviour of $\bar{P}_m(M|s)$ for $\alpha \neq 0$ as
\begin{align}
\bar{P}_m(M|s) \simeq \frac{1}{C_{\alpha}  \mathcal{D}_{\alpha} ^{\frac{1}{2+\alpha}} s^{\frac{1+\alpha}{2+\alpha}}}-\frac{2M}{C_{\alpha}^2  \mathcal{D}_{\alpha} ^{\frac{2}{2+\alpha}} s^{\frac{\alpha}{2+\alpha}}}.
\label{appen-asy-PM-eq-5}
\end{align}
Finally, we perform the inverse Laplace tramsformation and obtain
\begin{align}
P_m(M|t) = \frac{1}{\left( \mathcal{D}_{\alpha} t\right)^{\frac{1}{2+\alpha}}} \mathcal{F}_{\alpha} \left( \frac{M}{\left( \mathcal{D}_{\alpha} t\right)^{\frac{1}{2+\alpha}}} \right),
\label{appen-asy-PM-eq-6}
\end{align}
where the scaling function $\mathcal{F}_{\alpha} (z )$ has the form
\begin{align}
\mathcal{F}_{\alpha} (z \to 0) & \simeq \frac{1}{C_{\alpha} \Gamma \left( \frac{1+\alpha}{2+\alpha}\right)}-\frac{2z}{C_{\alpha} ^2 \Gamma\left( \frac{\alpha}{2+\alpha}\right)},~~~\text{for }\alpha \neq 0.
\label{appen-asy-PM-eq-7}
\end{align}
On the other hand, for $\alpha = 0 $, we saw that $\mathcal{F}_{\alpha}(z) = \frac{e^{-z^2/4}}{\sqrt{\pi}}$ [see Eq. (31) in the main text] which for small $z$ becomes
\begin{align}
\mathcal{F}_{\alpha} (z \to 0) \simeq \frac{1}{\sqrt{\pi}} \left( 1-\frac{z^2}{4}\right),~~~~\text{for }\alpha = 0.
\label{appen-asy-PM-eq-8}
\end{align}
Eqs. \eqref{appen-asy-PM-eq-7} and \eqref{appen-asy-PM-eq-8} completely specifies the small-$z$ behaviour of the scaling function $\mathcal{F}_{\alpha}(z) $ for all values of $\alpha$.

\subsection{Large $z$ behaviour of $\mathcal{F}_{\alpha}(z)$}
We next look at the large-$z$ behaviour of $\mathcal{F}_{\alpha}(z)$. Once again, this comes down to analysing the large-$M$ behaviour of $\bar{J}(M,s)$ via Eq. \eqref{new-appen-extreme-eq-15}. Using the asmptotic expression of modified Bessel function for large $x$ as
\begin{align}
I_{\beta}(x) \simeq \frac{e^{x}}{\sqrt{2 \pi x}},
\label{appen-asy-PM-eq-9}
\end{align}
in Eq. \eqref{new-appen-extreme-eq-8}, we get $\mathcal{H}_{\beta}(x)$ as 
\begin{align}
\mathcal{H}_{\beta}(x) \simeq \sqrt{\frac{2}{\pi}} \frac{e^x}{x^{\frac{1}{2}-\beta}},~~~~~ \text{as }x \to \infty.
\label{appen-asy-PM-eq-10}
\end{align}
Substituting this in the expression of $\bar{J}(M,s)$ in Eq. \eqref{new-appen-extreme-eq-16}, we find 
\begin{align}
\bar{J}(M,s) \simeq \frac{\mathcal{H}_{\frac{1}{2+\alpha}}(0)}{s} \sqrt{\frac{\pi}{2}} \left( a_s M\right)^{\frac{\alpha}{4}} e^{-\left( a_s M\right)^{\frac{2+\alpha}{2}}}.
\label{appen-asy-PM-eq-11}
\end{align}
as $M \to \infty$. Inserting this in the expression of $\bar{P}_m(M|s)$ in Eq. \eqref{new-appen-extreme-eq-15} and performing some algebraic simplications, we get
\begin{align}
\bar{P}_m(M|s) \simeq \frac{Z_{\alpha}~\text{exp} \left( -2b_M \sqrt{s} \right)}{s^{\frac{4+\alpha}{4(2+\alpha)}}} ,~~\text{as }M \to \infty,
\label{appen-asy-PM-eq-12}
\end{align}
where $b_M$ and $Z_{\alpha}$ are defined as
\begin{align}
& b_M = \frac{1}{2}\sqrt{\frac{M^{2+\alpha}}{\mathcal{D}_{\alpha}}}, \label{appen-asy-PM-eq-13} \\
& Z_{\alpha} = \frac{(2+\alpha)\sqrt{\pi}}{2\sqrt{2}} \left(  \frac{M^{\frac{3 \alpha}{4}}\mathcal{H}_{\frac{1}{2+\alpha}}(0)}{\mathcal{D}_{\alpha}^{\frac{4+3\alpha}{4(2+\alpha)}}}\right).\label{appen-asy-PM-eq-14}
\end{align}
We now proceed to perform the inverse Laplace transformation of $\bar{P}_m(M|s)$ in Eq. \eqref{appen-asy-PM-eq-12}. To this end, we exploit the convolution property of Laplace transformation to write
\begin{align}
P_m(M|t) \simeq &  ~Z_{\alpha} \int _{0}^{t} dT ~\mathcal{L}_{s \to t- T} \left[\frac{1}{s^{\frac{4+\alpha}{4(2+\alpha)}}} \right] ~~~~~~~~~~~~~~~~~~~~~~~~\nonumber \\
& ~~~~~~~~~ \times \mathcal{L}_{s \to T} \left[ \text{exp} \left( -2b_M \sqrt{s} \right) \right],
\label{appen-asy-PM-eq-15}
\end{align} 
where the notation $\mathcal{L}_{s \to t}$ denotes the inverse Laplace transformation. Using the inverse Laplace transformations
\begin{align}
&\mathcal{L}_{s \to t}\left[ \text{exp} \left( -2b_M \sqrt{s} \right) \right] = \frac{b_M~e^{-\frac{b_m^2}{t}}}{\sqrt{\pi t^3}},
\label{appen-asy-PM-eq-16} \\
&\mathcal{L}_{s \to t} \left[ s^{-\frac{4+\alpha}{4(2+\alpha)}}\right] = \frac{t^{\frac{4+\alpha}{4(2+\alpha)}-1}}{\Gamma\left( \frac{4+\alpha}{4(2+\alpha)} \right) }, \label{appen-asy-PM-eq-17}
\end{align}
in Eq. \eqref{appen-asy-PM-eq-15}, we find
\begin{align}
P_m(M|t) \simeq \frac{Z_{\alpha} M^{\frac{2+\alpha}{2}}~\mathbb{I} \left( \frac{M^{2+\alpha}}{4 \mathcal{D}_{\alpha} t} \right)}{\sqrt{4 \pi \mathcal{D}_{\alpha}} ~\Gamma \left(\frac{4+\alpha}{8+4 \alpha} \right) t^{\frac{8+5\alpha}{8+4 \alpha}}}, 
\label{appen-asy-PM-eq-18}
\end{align} 
where the function $\mathbb{I}(y)$ is defined as
\begin{align}
\mathbb{I}(y) = \int _{0}^{1} dw \frac{e^{-y/w}}{w^{3/2} (1-w)^{\frac{4+3 \alpha}{8+4 \alpha}}},~~~~y>0.
\label{appen-asy-PM-eq-19}
\end{align}
As seen in Eq. \eqref{appen-asy-PM-eq-18}, we now need to evaluate $\mathbb{I} \left( \frac{M^{2+\alpha}}{4 \mathcal{D}_{\alpha} t} \right)$ for large $M$ in order to compute the distribution $P_m(M|t)$. In Sec. \ref{appen-IIa}, we have shown that the asymptotic expression of the function $\mathbb{I}(y)$ is
\begin{align}
\mathbb{I}(y) \simeq \frac{e^{-y}}{y^{\frac{4+\alpha}{8+4 \alpha}}}~\Gamma \left(\frac{4+\alpha}{8+4 \alpha} \right),~~~\text{as }y \to \infty.
\label{appen-asy-PM-eq-20}
\end{align}
Finally, plugging this in Eq. \eqref{appen-asy-PM-eq-18}, we find
\begin{align}
P_m(M|t) = \frac{1}{\left( \mathcal{D}_{\alpha} t\right)^{\frac{1}{2+\alpha}}} \mathcal{F}_{\alpha} \left( \frac{M}{\left( \mathcal{D}_{\alpha} t\right)^{\frac{1}{2+\alpha}}} \right),
\label{appen-asy-PM-eq-21}
\end{align}
with the scaling function given by
\begin{align}
\mathcal{F}_{\alpha} \left( z \to \infty \right)\simeq \frac{(2+\alpha)\mathcal{H}_{\frac{1}{2+\alpha}}(0) }{2^{\frac{3+2\alpha}{2+\alpha}}}  z^{\alpha} ~e^{-\frac{z^{2+\alpha}}{4} },
\label{appen-asy-PM-eq-22}
\end{align}
as $z \to \infty$. This expression has been quoted in Eq. (33) in the main text.

\subsection{$\mathbb{I}(y)$ in Eq. \eqref{appen-asy-PM-eq-19} as $y \to \infty$}
\label{appen-IIa}
Here, we evaluate the asymptotic form of the integral $\mathbb{I}(y)$ in Eq. \eqref{appen-asy-PM-eq-20} as $y \to \infty$ which was instrumental in computing $P_m(M|t)$ for large $M$. For this, we  first replace $w=\bar{w}^{-1}$ in Eq. \eqref{appen-asy-PM-eq-19} and rewrite
\begin{align}
\mathbb{I}(y) = \int _{1}^{\infty} \frac{d \bar{w}}{\bar{w}^{\frac{\alpha}{8+4 \alpha}}}~\frac{e^{-y \bar{w}}}{\left( \bar{w}-1 \right)^{\frac{4+3 \alpha}{8+4 \alpha}}}.
\label{appen-asy-PM-eq-22}
\end{align}
For $y \to \infty$, the integral will be dominated by small values of $\bar{w}$ which in the given domain of integration is equal to $1$. Consequently, we get
\begin{align}
\mathbb{I} \left( y \to \infty \right) \simeq \int _{1}^{\infty} d \bar{w}~\frac{e^{-y \bar{w}}}{\left( \bar{w}-1 \right)^{\frac{4+3 \alpha}{8+4 \alpha}}}.
\label{appen-asy-PM-eq-23}
\end{align}
Finally, changing the variable $\bar{w}=w+1 $, we get
\begin{align}
\mathbb{I} \left( y \to \infty \right) &\simeq e^{-y}~\int _{0}^{\infty} d w~\frac{e^{-y w}}{w^{\frac{4+3 \alpha}{8+4 \alpha}}}, 
\\
& \simeq \frac{e^{-y}}{y^{\frac{4+\alpha}{8+4 \alpha}}}~\Gamma \left(\frac{4+\alpha}{8+4 \alpha} \right).
\label{appen-asy-PM-eq-23}
\end{align}
This result has been quoted in Eq. \eqref{appen-asy-PM-eq-20}.

\begin{figure}[t]
\includegraphics[scale=0.4]{contour-2.pdf}
%\includegraphics[scale=0.21]{scaling-M-alp0p5.pdf}
%\includegraphics[scale=0.25]{new_prob_dist.pdf}
\centering
\caption{Contour for the Bromwhich integration of $\mathbb{E}(s,t_m,w) $ in Eq. \eqref{appen-ILTYY-eq-4}}
\label{contour-fig-2}
\end{figure}
\section{Marginal distribution $\mathcal{P}_m(t_m|t)$}
\label{ILTYY}
Let us now look at the distribution $\mathcal{P}_m(t_m|t)$ of the arg-maximum $t_m$. In Eq. (36) of the main text, we had obtained the double Laplace transform $\bar{\mathcal{P}}_m(p|s)$ as
\begin{align}
\bar{\mathcal{P}}_m(p|s) &= \int _{0}^{\infty} dM \frac{  \mathcal{H}_{\frac{1}{2+\alpha}} \left(0 \right) }{s~\mathcal{H}_{\frac{1}{2+\alpha}} \left( (a_{s+p} M)^{\frac{2+\alpha}{2}} \right)}
 \frac{\partial _M \left[ \mathcal{H}_{\frac{1}{2+\alpha}} \left( (a_s M)^{\frac{2+\alpha}{2}} \right) \right]}{ \mathcal{H}_{\frac{1}{2+\alpha}} \left( (a_s M)^{\frac{2+\alpha}{2}} \right)}.
\label{new-appen-extreme-eq-30}
\end{align}
We further simplify this expression by using the identity
$$\frac{d}{dy}  \mathcal{H}_{\frac{1}{2+\alpha}} \left( y^{\frac{2+\alpha}{2}} \right) = \left(\frac{2+\alpha}{2}\right) \mathcal{H}_{\frac{1+\alpha}{2+\alpha}} \left( y^{\frac{2+\alpha}{2}} \right), $$
from Eq. \eqref{new-appen-extreme-eq-8} and change the variable $a_{s+p} M=w$. Eq. \eqref{new-appen-extreme-eq-30} can then be rewritten as
\begin{align}
&\bar{\mathcal{P}}_m(p|s) = \frac{ \mathcal{H}_{\frac{1}{2+\alpha}} \left(0 \right) }{2 (2+\alpha)^{-1}} \int _{0}^{\infty} dw \frac{\bar{\mathbb{Y}}_{\alpha} \left(s,p,w \right)}{\mathcal{H}_{\frac{1}{2+\alpha}} \left(w^{\frac{2+\alpha}{2}} \right)}, ~~\text{with}\label{extreme-eq-31}\\
& \bar{\mathbb{Y}}_{\alpha} \left(s,p,w \right) = \frac{\mathcal{H}_{\frac{1+\alpha}{2+\alpha}} \left(\sqrt{\frac{s}{s+p}}w^{\frac{2+\alpha}{2}} \right)}{s^{\frac{1+\alpha}{2+\alpha}}(s+p)^{\frac{1}{2+\alpha}}~\mathcal{H}_{\frac{1}{2+\alpha}} \left(\sqrt{\frac{s}{s+p}}w^{\frac{2+\alpha}{2}} \right)}.\label{extreme-eq-32}
\end{align} 
Notice that all $s$ and $p$ dependences in Eq. \eqref{extreme-eq-31} now appear in the function $\bar{\mathbb{Y}}_{\alpha} \left(s,p,w \right) $. To get the distribution in time domain, one then needs to perform the double inverse Laplace transformation of $\bar{\mathbb{Y}}_{\alpha} \left(s,p,w \right)$. Denoting this double inverse Laplace transformation by $\mathbb{Y}_{\alpha} \left(t,t_m,w \right)$, \textit{i.e.} 
\begin{align}
\mathbb{Y}_{\alpha} \left(t,t_m,w \right) = \mathcal{L}_{s \to t}\mathcal{L}_{p \to t_m} \left[ \bar{\mathbb{Y}}_{\alpha} \left(s,p,w \right) \right].
\label{appen-ILTYY-eq-2}
\end{align}
Notation $\mathcal{L}_{s \to t}$ denotes the inverse Laplace transformation from $s \to t$. Using Eq. \eqref{appen-ILTYY-eq-2}, we can now write the distribution $\mathcal{P}_m(t_m|t)$ as
\begin{align}
\mathcal{P}_m(t_m|t) = \frac{ \mathcal{H}_{\frac{1}{2+\alpha}} \left(0 \right) }{2 (2+\alpha)^{-1}} \int _{0}^{\infty} dw \frac{\mathbb{Y}_{\alpha} \left(t,t_m,w \right)}{\mathcal{H}_{\frac{1}{2+\alpha}} \left(w^{\frac{2+\alpha}{2}} \right)}.
\label{appen-ILTYY-eq-1}
\end{align}  
Let us now proceed to perform the inversion in Eq. \eqref{appen-ILTYY-eq-2}. For the inversion from $p \to t_m$, we write the Bromwich integral
\begin{align}
\mathbb{E}(s,t_m,w) &= \mathcal{L}_{p \to t_m} \left[ \bar{\mathbb{Y}}_{\alpha} \left(s,p,w \right) \right], \label{appen-ILTYY-eq-3}\\
& =\frac{1}{2 \pi i} \int _{-i \infty}^{i \infty} dp~ e^{p t_m}~ \bar{\mathbb{Y}}_{\alpha} \left(s,p,w \right) \label{appen-ILTYY-eq-4}.
\end{align}
Note that the integrand $\bar{\mathbb{Y}}_{\alpha} \left(s,p,w \right)$ has a branch point at $p=-s$ [see Eq. \eqref{extreme-eq-32}]. To perform this complex integration, we consider contour shown in Figure \ref{contour-fig-2}. Since the integrand $\bar{\mathbb{Y}}_{\alpha} \left(s,p,w \right)$ is analytic inside this contour, the Cauchy theorem gives
\begin{align}
\int _{\Gamma _1}+\int _{\Gamma _2}+\int _{\Gamma _3}+\int _{\Gamma _4}+\int _{\Gamma _5}+\int _{\Gamma _6} =0. 
\label{appen-ILTYY-eq-5}
\end{align}
Here $\int _{\Gamma _1}$ is the integration that we need. Let us now calculate these integrations along different paths. Note that the real part of $p$ is negative along paths $\Gamma_2$ and $\Gamma _6$ and in the limit $|p| \to \infty$, the integrand becomes exactly zero. Hence, the contribution of these two paths is equal to zero.

Along $\Gamma _4$, we replace $p=-s + \delta e^{i \theta}$ and take limit $\delta \to 0^+$. For small $\delta$, we find that the integrand scales as $\sim \sqrt{\delta}$ which vanishes as $\delta \to 0^+$. Hence, the contribution of this path is also zero. 

Next, we look at the contribution of $\Gamma _3$. For this path, we replace $p=-s+u s e^{i \pi} w^{2+\alpha} $ and perform some algebraic simplifications to get
\begin{align}
\int _{\Gamma _3} = \frac{w^{1+\alpha}~e^{-\frac{i \pi }{2+\alpha}}}{2 \pi i} \int _{0}^{\infty} du \frac{e^{-(1+u w^{2+\alpha}) s t_m}}{u^{\frac{1}{2+\alpha}}} \frac{\mathcal{H}_{\frac{1+\alpha}{2+\alpha}} \left( -\frac{i}{\sqrt{u}}\right)}{\mathcal{H}_{\frac{1}{2+\alpha}} \left( -\frac{i}{\sqrt{u}}\right)}. \nonumber
\end{align}  
Similarly, for $\Gamma _5$, we replace $p=-s+u s e^{-i \pi} w^{2+\alpha} $ and get 
 \begin{align}
\int _{\Gamma _5} = -\frac{w^{1+\alpha}~e^{\frac{i \pi }{2+\alpha}}}{2 \pi i} \int _{0}^{\infty} du \frac{e^{-(1+u w^{2+\alpha}) s t_m}}{u^{\frac{1}{2+\alpha}}} \frac{\mathcal{H}_{\frac{1+\alpha}{2+\alpha}} \left( \frac{i}{\sqrt{u}}\right)}{\mathcal{H}_{\frac{1}{2+\alpha}} \left( \frac{i}{\sqrt{u}}\right)}. \nonumber
\end{align}  
Plugging all these contribution in Eq. \eqref{appen-ILTYY-eq-5} and noting $\int _{\Gamma _1} = \mathbb{E}(s,t_m,w)$ , we find
\begin{align}
\mathbb{E}(s,t_m,w) = \int _{0}^{\infty} du~ \frac{w^{1+\alpha}~e^{-(1+u w^{2+\alpha}) s t_m}}{u^{\frac{1}{2+\alpha}}}~\mathbb{X}_{\alpha} \left( \sqrt{u}\right), 
\label{appen-ILTYY-eq-6}
\end{align}
where the function $\mathbb{X}_{\alpha} \left( u\right)$ is defined as
\begin{align}
\mathbb{X}_{\alpha} \left( u\right) =\frac{e^{\frac{i \pi }{2+\alpha}}}{2 \pi i} \left[\frac{\mathcal{H}_{\frac{1+\alpha}{2+\alpha}} \left( \frac{i}{{u}}\right)}{\mathcal{H}_{\frac{1}{2+\alpha}} \left( \frac{i}{{u}}\right)} -\frac{e^{-\frac{2i \pi }{2+\alpha}}~\mathcal{H}_{\frac{1+\alpha}{2+\alpha}} \left( -\frac{i}{{u}}\right)}{\mathcal{H}_{\frac{1}{2+\alpha}} \left(- \frac{i}{{u}}\right)} \right].
\label{new-appen-ILTYY-eq-7}
\end{align}
Inserting $\mathbb{E}(s,t_m,w)$ from Eq. \eqref{appen-ILTYY-eq-6} in $\mathbb{Y}_{\alpha} \left(t,t_m,w \right)$ in Eq. \eqref{appen-ILTYY-eq-2} and performing the inversion with $s$, we get
\begin{align}
\mathbb{Y}_{\alpha} \left(t,t_m,w \right) = \frac{\mathbb{X}_{\alpha} \left( \sqrt{\frac{t-t_m}{t_m w^{2+\alpha}}}  \right)}{t_m^{\frac{1+\alpha}{2+\alpha}} (t-t_m)^{\frac{1}{2+\alpha}}}.
\label{appen-ILTYY-eq-71}
\end{align}
Finally, substituting this form of $\mathbb{Y}_{\alpha} \left(t,t_m,w \right)$ in Eq. \eqref{appen-ILTYY-eq-1} results in the expression of $\mathcal{P}_m(t_m|t)$ as
\begin{align}
\mathcal{P}_m(t_m|t) = \frac{1}{t} ~\mathcal{G} _{m}^{\alpha} \left( \frac{t_m}{t}\right),
\label{new-appen-extreme-res-eq-1}
\end{align}
with the scaling function $\mathcal{G} _{m}^{\alpha} \left( z \right)$ defined as
\begin{align}
~~~~~~~~\mathcal{G} _{m}^{\alpha} \left( z \right) = \frac{(2+\alpha)\mathcal{H}_{\frac{1}{2+\alpha}} \left(0 \right)}{2z^{\frac{1+\alpha}{2+\alpha}} (1-z)^{\frac{1}{2+\alpha}}}~\int _{0}^{\infty} dw\frac{\mathbb{X}_{\alpha} \left( \sqrt{\frac{1-z}{z w^{2+\alpha}}} \right)}{\mathcal{H}_{\frac{1}{2+\alpha}}\left( w^{\frac{2+\alpha}{2}}\right)}.
\label{new-appen-extreme-res-eq-2}
\end{align}


\section{Asymptotic form of $\mathcal{G}_m^{\alpha}(z)$}
\label{apen-Gm}
We next analyse the asymptotic behaviour of the scaling function $\mathcal{G}_m^{\alpha}(z)$ in Eq. \eqref{new-appen-extreme-res-eq-2} as $z \to 0$ and $z \to 1$. For clarity, we present this analysis separately for $z \to 0$ and $z \to 1$.
\subsection{$\mathcal{G}_m^{\alpha}(z)$ as $z \to 0$}
To get the asymptotic form of $\mathcal{G}_m^{\alpha}(z)$ as $z \to 0$, we see from Eq. \eqref{new-appen-extreme-res-eq-2} that one needs to specify the behaviour of $\mathbb{X}_{\alpha} \left( \sqrt{\frac{1-z}{z w^{2+\alpha}}} \right)$ as $z \to 0$. To this end, we use the expression of $\mathbb{X}_{\alpha}(x)$ in Eq. \eqref{new-appen-ILTYY-eq-7} as $x \to \infty$ which reads
\begin{align}
\mathbb{X}_{\alpha}\left(x \to \infty \right) \simeq \frac{\sin \left( \frac{\pi}{2+\alpha}\right)}{ \pi} \frac{\mathcal{H}_{\frac{1+\alpha}{2+\alpha}}(0)}{\mathcal{H}_{\frac{1}{2+\alpha}}(0)}.
\end{align}  
Plugging this in Eq. \eqref{new-appen-extreme-res-eq-2}, we see that
\begin{align}
\mathcal{G}_m^{\alpha}(z) \sim z^{-\frac{1+\alpha}{2+\alpha}}, ~~~~\text{as } z \to 0.
\end{align}
\subsection{$\mathcal{G}_m^{\alpha}(z)$ as $z \to 1$}
Once again, we use the expression $\mathcal{G}_m^{\alpha}(z)$ in Eq. \eqref{new-appen-extreme-res-eq-2}. As evident from this equation, one then needs to specify the small-$x$ behaviour of $\mathbb{X}_{\alpha}(x)$. Using $\mathcal{H}_{\beta}\left(u \to \infty\right) \simeq \sqrt{\frac{1}{\pi}} e^{u} u^{\beta -\frac{1}{2}}$ in Eq. \eqref{new-appen-ILTYY-eq-7}, it is easy to show that
\begin{align}
\mathbb{X}_{\alpha}\left(x \to 0 \right) \simeq \frac{1}{\pi x^{\frac{\alpha}{2+\alpha}}}.
\end{align} 
Inserting this in Eq. \eqref{new-appen-extreme-res-eq-2} then yields
\begin{align}
\mathcal{G}_m^{\alpha}(z) \sim (1-z)^{-\frac{1}{2}},~~~\text{as } z \to 1.
\end{align}


\section{Residence time distribution $\mathcal{P}_r \left( t_r|t \right)$}
\label{appen-sol-resi}
We now look at the second arcsine law which concerns the statistics of the residence time $t_r$. We saw that the Laplace transform  $\mathcal{Q}(p,x_0|t)$ for the residence time obeys the backward master equation
\begin{align}
\partial _t \mathcal{Q}(p,x_0|t) = \left[D(x_0) \partial _{x_0}^2-p~ \Theta (x_0) \right]\mathcal{Q}(p,x_0|t),
\label{new-appen-resi-eq-3}
\end{align}
where $D(x_0)$ is given in Eq. \eqref{new-appen-extreme-eq-3}. The corresponding initial and boundary conditions were derived in Sec.  IV in the main text. They read
\begin{align}
& \mathcal{Q}(p,x_0|t \to 0) = 1, \label{new-appen-resi-eq-4}\\
& \mathcal{Q}(p,x_0 \to -\infty |t) = 1, \label{new-appen-resi-eq-5} \\
& \mathcal{Q}(p,x_0 \to \infty |t) = e^{-pt}. \label{new-appen-resi-eq-6}
\end{align}

Here we will solve the backward master equation \eqref{new-appen-resi-eq-3} along with these conditions to get the probability distribution of the residence time $t_r(t)$. For this, we first take the Laplace transformation of Eq. \eqref{new-appen-resi-eq-3} with respect to $t$ and rewrite it as
\begin{align}
\left[ s+p ~\Theta (x_0) \right] \bar{\mathcal{Q}}(p,x_0|s) - 1= D(x_0) \partial _{x_0}^2\bar{\mathcal{Q}}(p,x_0|s),
\label{appen-sol-resi-eq-1}
\end{align}
where $\bar{\mathcal{Q}}(p,x_0|s)$ is the Laplace transformation of $\mathcal{Q}(p,x_0|t)$. For $x_0>0$, this equation becomes
\begin{align}
 (s+p) \bar{\mathcal{Q}}(p,x_0|s) - 1= D(x_0) \partial _{x_0}^2\bar{\mathcal{Q}}(p,x_0|s). 
\label{appen-sol-resi-eq-2}
\end{align}
To simplify this equation further, we make the following transformations: 
\begin{align}
&y = \frac{x_0 ^{2+\alpha}}{(2+\alpha)^2 l^{\alpha} D_0}, \\
&\bar{\mathcal{Q}}(p,x_0|s) = \frac{1}{s+p}+\mathbb{Q}(p,x_0|s),
\end{align}
and rewrite Eq. \eqref{appen-sol-resi-eq-2} in terms of $\mathbb{Q}(p,x_0|s)$ and $y$ as
%\begin{align}
% (s+p) \mathbb{Q}(p,x_0|s) = D(x_0) \partial _{x_0}^2 \mathbb{Q}(p,x_0|s). 
%\label{appen-sol-resi-eq-3}
%\end{align}
%We now make the transformation $y = \frac{x_0 ^{2+\alpha}}{(2+\alpha)^2 l^{\alpha} D_0}$.
\begin{align}
y \frac{\partial ^2 \mathbb{Q}}{\partial y^2} + \left(\frac{1+\alpha}{2+\alpha} \right) \frac{\partial  \mathbb{Q}}{\partial y} = (s+p) \mathbb{Q}.
\label{appen-sol-resi-eq-4}
\end{align}
This equation can now be solved and its solutions are given in terms of the modified bessel functions as $y^{\frac{1}{2(2+\alpha)}} I_{\frac{1}{2+\alpha}} \left( 2 \sqrt{(s+p) y}\right)$ and $y^{\frac{1}{2(2+\alpha)}} K_{\frac{1}{2+\alpha}} \left( 2 \sqrt{(s+p) y}\right)$. However, the former solution diverges in the limit $y \to \infty$. Hence, we consider only the later solution and write finally for $\bar{\mathcal{Q}}(p,x_0|s)$ as
\begin{align}
\bar{\mathcal{Q}}(p,x_0|s) & = \frac{1}{s+p} +\mathbb{C}_5 \sqrt{x_0}~K_{\frac{1}{2+\alpha}} \left( (a_{s+p}~ x_0)^{\frac{2+\alpha}{2}}\right),
\label{appen-sol-resi-eq-5}
\end{align}
where $a_{s+p}$ is given in Eq. \eqref{new-appen-extreme-eq-9} and $\mathbb{C}_5$ is a function independent of $y$, but can, in principle, depend on $s$ and $p$. Recall that the solution in Eq. \eqref{appen-sol-resi-eq-5} holds only for $x_0 >0$. Proceeding similarly for $x_0 <0$, we get
\begin{align}
\bar{\mathcal{Q}}(p,x_0|s) & = \frac{1}{s} +\mathbb{C}_6 \sqrt{|x_0|}~K_{\frac{1}{2+\alpha}} \left( (a_{s}~| x_0|)^{\frac{2+\alpha}{2}}\right).
\label{appen-sol-resi-eq-61}
\end{align}
Once again $\mathbb{C}_6$ here is a function independent of $y$, but can, in principle, depend on $s$ and $p$. Now the task is to compute these functions $\mathbb{C}_5$ and $\mathbb{C}_6$ in Eqs. \eqref{appen-sol-resi-eq-5} and \eqref{appen-sol-resi-eq-61}. For this computation, we use the continuity of $\bar{\mathcal{Q}}(p,x_0|s) $ and $\partial _{x_0} \bar{\mathcal{Q}}(p,x_0|s) $ across $x_0 = 0$:
\begin{align}
\bar{\mathcal{Q}} \left(p,x_0 \to 0^+|s \right)& =\bar{\mathcal{Q}} \left(p,x_0 \to 0^-|s \right), \\
\left( \frac{\partial \bar{\mathcal{Q}} \left(p,x_0|s \right)}{\partial x_0}\right) _{x_0 \to 0^+}& = \left( \frac{\partial \bar{\mathcal{Q}} \left(p,x_0|s \right)}{\partial x_0}\right) _{x_0 \to 0^-}. \label{pras-neww} 
\end{align}
Then using these forms of $\mathbb{C}_5$ and $\mathbb{C}_6$ in Eqs. \eqref{appen-sol-resi-eq-5} and \eqref{appen-sol-resi-eq-61}, we obtain the expression of $\bar{\mathcal{Q}}(p,x_0|s)$ for all $x_0$. However, we are interested only in the situation where the particle starts initially from the origin. For this case, we have
\begin{align}
\bar{\mathcal{Q}}(p|s) = \frac{1}{s}-\frac{p}{s(s+p)} \left[1+ \left( \frac{s}{s+p} \right) ^{\frac{1}{2+\alpha}} \right]^{-1},
\label{appen-sol-resi-eq-6}
\end{align}
where we have used the short hand notation $\bar{\mathcal{Q}}(p|s)$ for $\bar{\mathcal{Q}}(p,x_0=0|s)$. 
%\greenw{We note that the solution in Eq. \eqref{appen-sol-resi-eq-6} is valid only for $\alpha >-1$. For $\alpha \leq -1$, we find that the derivative $\frac{\partial \bar{\mathcal{Q}} \left(p,x_0|s \right)}{\partial x_0}$ diverges as $x_0 \to 0^{\pm}$ which makes Eq. \eqref{pras-neww} ill-defined.}
Performing inversion of the Laplace transform in Eq. \eqref{appen-sol-resi-eq-6} for $\alpha >-1$, we get \cite{newLamperti58, newCarmi2010}
\begin{align}
\mathcal{P}_r(t_r|t) = \frac{1}{t} ~\mathcal{G} _{r}^{\alpha} \left( \frac{t_r}{t}\right),
\label{new-appen-resi-eq-8}
\end{align}
where the scaling function $\mathcal{G} _{r}^{\alpha} \left( z \right)$ is given by
\begin{align}
& ~~~~~~~~~~\mathcal{G} _{r}^{\alpha} \left( z \right) = \frac{ \sin \left( \frac{\pi}{2+\alpha}\right)  \pi^{-1}~ \left[ z(1-z) \right]^{-\frac{1+\alpha}{2+\alpha}}}{z^{\frac{2}{2+\alpha}}+(1-z)^{\frac{2}{2+\alpha}}+2  \cos \left( \frac{\pi}{2+\alpha}\right) \left[z(1-z) \right] ^{\frac{1}{2+\alpha}}}.
\label{new-appen-resi-eq-9}
\end{align}

\section{Distribution $\mathcal{P}_{ \ell } \left( t_{\ell}|t \right)$ of the last-passage time}
\label{appen-dist-Pl}
This section deals with the arcsine law for the last-passage time $t_{\ell}$. As discussed in Sec. IV in the main text, the Laplace transform $\bar{\mathcal{P}}_{ \ell } \left( p|s \right)$ of this distribution reads
\begin{align}
\bar{\mathcal{P}}_{ \ell } \left( p|s \right) = \frac{\bar{\mathbb{P}}(\epsilon, s+p|0)~\bar{S}_0(s|\epsilon)}{\mathcal{N}_L(\epsilon)}. 
\label{new-appen-last-passage-time-eq-2}
\end{align}
where $\bar{\mathbb{P}}(x, s|0)$ stands for the Laplace transformation of the distribution $\mathbb{P}(x,t|0)$ in the free space. Also, $\bar{S}_0(s|x)$ is the Laplace transform of the survival probability as discussed before. In what follows, we first compute these two Laplace transforms and then use them in Eq. \eqref{new-appen-last-passage-time-eq-2} to calculate $\bar{\mathcal{P}}_{ \ell } \left( p|s \right)$.
\subsection{Computation of $\bar{\mathbb{P}}(\epsilon, s|0)$}
\label{appen-dist-Pl-prob}
Denoting the probability distribution to be at $x$ at time $t$ starting from the origin by $\mathbb{P}(x,t|0)$, we write the Fokker-Planck equation (Ito sense) for $\mathbb{P}(x,t|0)$ as 
\begin{align}
\partial _t \mathbb{P}(x,t|0) = \partial _{x}^2 \left[ D(x) \mathbb{P}(x,t|0) \right],
\label{appen-dist-Pl-eq-1}
\end{align} 
where $D(x)$ is given in Eq. \eqref{new-appen-extreme-eq-3}. Taking Laplace tramsformation of this equation gives  
\begin{align}
s \bar{\mathbb{P}}(x,s|0)-\delta(x) = \partial _{x}^2 \left[ D(x) \bar{\mathbb{P}}(x,s|0) \right].
\label{appen-dist-Pl-eq-2}
\end{align}
For $x \neq 0$, we get rid of the delta function and obtain
\begin{align}
s \bar{\mathbb{P}}(x,s|0) = \partial _{x}^2 \left[ D(x) \bar{\mathbb{P}}(x,s|0) \right].
\label{appen-dist-Pl-eq-3}
\end{align}
Let us first analyse this differential equation for $x>0$. To simplify Eq. \eqref{appen-dist-Pl-eq-3}, we perform the transformations
\begin{align}
 & y = \frac{x^{2+\alpha}}{(2+\alpha)^2 l^{\alpha} D_0}, \label{appen-dist-Pl-eq-4} \\
 & \mathbb{Z}_L \left(x,s \right)= D(x) \bar{\mathbb{P}}(x,s|0), \label{appen-dist-Pl-eq-5} 
\end{align}
and rewrite it as
\begin{align}
y \frac{\partial ^2 \mathbb{Z}_L}{\partial y^2} + \left(\frac{1+\alpha}{2+\alpha} \right) \frac{\partial  \mathbb{Z}_L}{\partial y} = s \mathbb{Z}_L.
\label{appen-dist-Pl-eq-6} 
\end{align}
One can now solve this equation exactly and write solutions in terms of the modified bessel functions as $y^{\frac{1}{2(2+\alpha)}} I_{\frac{1}{2+\alpha}} \left( 2 \sqrt{s y}\right)$ and $y^{\frac{1}{2(2+\alpha)}} K_{\frac{1}{2+\alpha}} \left( 2 \sqrt{s y}\right)$. Note that the former solution diverges in the limit $y \to \infty$. Hence, we consider only the later solution and write finally $\bar{\mathbb{P}}(x,s|0)$ as
\begin{align}
\bar{\mathbb{P}}(x,s|0) & = \frac{\mathbb{C}_7 \sqrt{x}}{D(x)}~K_{\frac{1}{2+\alpha}} \left( (a_{s}~ x)^{\frac{2+\alpha}{2}}\right),,
\label{appen-dist-Pl-eq-7} 
\end{align}
where $a_{s}$ is given in Eq. \eqref{new-appen-extreme-eq-9} and $\mathbb{C}_7$ is a constant which, in principle, depends of $s$ and $\alpha$. Recall that the expression of $\bar{\mathbb{P}}(x,s|0)$ in Eq. \eqref{appen-dist-Pl-eq-7} is valid only for $x>0$. To compute this for $x<0$, we recall that the problem possesses $x \to -x$ symmetry in the infinite line. Consequently, the same solution also applies for $x<0$ and we have
\begin{align}
\bar{\mathbb{P}}(x,s|0) & = \frac{\mathbb{C}_7 \sqrt{x}}{D(x)}~K_{\frac{1}{2+\alpha}} \left( (a_{s}~ |x|)^{\frac{2+\alpha}{2}}\right),
\label{appen-dist-Pl-eq-81} 
\end{align}
for all $x$. The task now is to compute the constant $\mathbb{C}_7$. To compute it, we integrate Eq. \eqref{appen-dist-Pl-eq-2} with respect to $x$ from $\zeta$ to $-\zeta$ and take $\zeta \to 0^+$. This gives rise to the following condition:
\begin{align}
\left( \frac{ \partial\left[ D(x) \bar{\mathbb{P}}(x,s|0) \right] }{\partial x} \right)_{ 0^+}-\left( \frac{ \partial\left[ D(x) \bar{\mathbb{P}}(x,s|0) \right] }{\partial x} \right)_{ 0^-}=-1.
\label{appen-dist-Pl-eq-9} 
\end{align}
We next insert $\bar{\mathbb{P}}(x,s|0)$ from Eq. \eqref{appen-dist-Pl-eq-81} in Eq. \eqref{appen-dist-Pl-eq-9} which results in $\mathbb{C}_7$ as
\begin{align}
\mathbb{C}_7(s) = \frac{2^{\frac{1}{2+\alpha}}}{(2+\alpha) \Gamma \left( \frac{1+\alpha}{2+\alpha}\right)} a_s^{-\frac{1}{2}}.
\label{appen-dist-Pl-eq-7}
\end{align}
Inserting this in Eq. \eqref{appen-dist-Pl-eq-81}, we obtain the exact form of $\bar{\mathbb{P}}(x,s|0)$. Since, we are interested in $\bar{\mathbb{P}}(\epsilon, s|0)$ with $\epsilon \to 0$ for last passage time [see Eq. \eqref{new-appen-last-passage-time-eq-2}], we provide below only the expression of $\bar{\mathbb{P}}(\epsilon, s|0)$:
\begin{align}
\bar{\mathbb{P}}(\epsilon, s|0) \simeq \frac{\mathcal{A}_L(\epsilon)}{s^{\frac{1}{2+\alpha}}},
\label{appen-dist-Pl-eq-8}
\end{align}
where the function $\mathcal{A}_L(\epsilon)$ is defined as
\begin{align}
\mathcal{A}_L(\epsilon) =  \frac{2^{-\frac{\alpha}{2+\alpha}}~\mathcal{D}_{\alpha}~ \Gamma \left( \frac{1}{2+\alpha} \right)}{D(\epsilon)(2+\alpha)  \Gamma \left( \frac{1+\alpha}{2+\alpha} \right)}.
\label{appen-dist-Pl-eq-9}
\end{align}
\subsection{Computation of $\bar{S}_0(s|\epsilon) $}
We now calcuate $\bar{S}_0(s|\epsilon) $ which is essential for computing $\bar{\mathcal{P}}_{ \ell } \left( p|s \right)$ in Eq. \eqref{new-appen-last-passage-time-eq-2}. To compute $\bar{S}_0(s|\epsilon) $, we proceed exactly as in Sec. \ref{appen-surv}. In order to avoid the repetition, we present only the final result here.
\begin{align}
\bar{S}_0(s|\epsilon) \simeq \frac{\mathcal{B}_L(\epsilon)}{s^{\frac{1+\alpha}{2+\alpha}}},
\label{appen-dist-Pl-eq-10}
\end{align}
where the function $\mathcal{B}_L(\epsilon)$ is defined as
\begin{align}
\mathcal{B}_L(\epsilon) = \frac{\epsilon (2+\alpha)~ \Gamma \left( \frac{1+\alpha}{2+\alpha}\right)}{\left( 4 \mathcal{D}_{\alpha} \right)^{\frac{1}{2+\alpha}}  \Gamma \left( \frac{1}{2+\alpha}\right)}.
\label{appen-dist-Pl-eq-11}
\end{align}
Plugging the forms of $\bar{\mathbb{P}}(\epsilon, s|0)$ and $\bar{S}_0(s|\epsilon)$ from Eqs. \eqref{appen-dist-Pl-eq-8} and \eqref{appen-dist-Pl-eq-10} in Eq. \eqref{new-appen-last-passage-time-eq-2} gives
\begin{align}
\bar{\mathcal{P}}_{ \ell } \left( p|s \right) \simeq \frac{\mathcal{A}_L (\epsilon)~\mathcal{B}_L (\epsilon)}{\mathcal{N}_L(\epsilon)}~\frac{1}{s^{\frac{1+\alpha}{2+\alpha}} \left(s+p \right)^{\frac{1}{2+\alpha}}}.
\label{new-appen-last-passage-time-eq-5}
\end{align}
We now have to specify the normalisation factor $\mathcal{N}_L(\epsilon)$. To evaluate this factor, we use the normalisation condition $\bar{\mathcal{P}}_{ \ell } \left( 0|s \right) = 1/s $ from which it is easy to show that $\mathcal{N}_L(\epsilon) =\mathcal{A}_L (\epsilon)~\mathcal{B}_L (\epsilon)$. This leads us to write $\bar{\mathcal{P}}_{ \ell } \left( p|s \right)$ as
\begin{align}
\bar{\mathcal{P}}_{ \ell } \left( p|s \right)= \frac{1}{s^{\frac{1+\alpha}{2+\alpha}} \left(s+p \right)^{\frac{1}{2+\alpha}}}.
\label{new-appen-last-passage-time-eq-6}
\end{align}
\greenw{Finally performing the double inverse Laplace transformation for $\alpha >-1$ gives that the distribution $\bar{\mathcal{P}}_{ \ell } \left( p|s \right)$ obeys the scaling relation} 
\begin{align}
\mathcal{P}_{ \ell } \left( t_{\ell}|t \right) = \frac{1}{t} \mathcal{G}_{\ell}^{\alpha} \left( \frac{t_{\ell}}{t}\right),
\label{new-appen-last-passage-time-eq-7}
\end{align}
with the scaling function $\mathcal{G}_{\ell}^{\alpha} \left( z \right)$ given by
\begin{align}
\mathcal{G}_{\ell}^{\alpha} \left( z \right) = \frac{z^{-\frac{1+\alpha}{2+\alpha}} \left( 1-z \right) ^{-\frac{1}{2+\alpha}}}{\Gamma \left( \frac{1+\alpha}{2+\alpha}\right)~\Gamma \left( \frac{1}{2+\alpha}\right)}.
\label{new-appen-last-passage-time-eq-9}
\end{align}


\begin{thebibliography}{}


\bibitem{newRedner}
S. Redner, \textit{A Guide to First-Passage Processes} (Cambridge
University Press, Cambridge, England, 2001).

\bibitem{newLamperti58}
J. Lamperti, An Occupation Time Theorem for A Class of Stochastic Processes, {Trans. Amer. Math. Soc.}, \textbf{88} 380 (1958).

\bibitem{newCarmi2010}
S. Carmi, L. Turgeman  and E. Barkai, On Distributions of Functionals of Anomalous Diffusion Paths, J. Stat. Phys. \textbf{141}, 1071 (2010).

\end{thebibliography}

\end{document}

