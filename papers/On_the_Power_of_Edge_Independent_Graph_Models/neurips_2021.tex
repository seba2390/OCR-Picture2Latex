\documentclass{article}

% if you need to pass options to natbib, use, e.g.:
%     \PassOptionsToPackage{numbers, compress}{natbib}
% before loading neurips_2021

% ready for submission
%\usepackage{neurips_2021}

% to compile a preprint version, e.g., for submission to arXiv, add add the
% [preprint] option:
     \usepackage[nonatbib,preprint]{neurips_2021}

% to compile a camera-ready version, add the [final] option, e.g.:
%     \usepackage[nonatbib,final]{neurips_2021}

% to avoid loading the natbib package, add option nonatbib:
%\usepackage[nonatbib]{neurips_2021}

\usepackage[utf8]{inputenc} % allow utf-8 input
\usepackage[T1]{fontenc}    % use 8-bit T1 fonts
\usepackage{hyperref}       % hyperlinks
\usepackage{url}            % simple URL typesetting
\usepackage{booktabs}       % professional-quality tables
\usepackage{amsfonts}       % blackboard math symbols
\usepackage{nicefrac}       % compact symbols for 1/2, etc.
\usepackage{microtype}      % microtypography
\usepackage{xcolor}         % colors

% Things we're adding
\usepackage{amssymb,amsthm,amsmath}
\usepackage{bm}
\usepackage{float}
\usepackage{graphicx}
\usepackage{algorithm}
%\usepackage{algorithmic}
\usepackage{algpseudocode}
\graphicspath{ {./new_images/} }
\DeclareGraphicsExtensions{.pdf,.jpeg,.png}
\usepackage{multibib}

\newcommand{\algoname}[1]{\textnormal{\textsc{#1}}}
\newcommand{\comment}[1]{\text{\phantom{(#1)}} \tag{#1}}
\newcommand{\spara}[1]{\smallskip\noindent{\bf #1}}
\newif\ifdraft

\newcommand{\todo}[1]{\textcolor{blue}{TODO: #1}}
\newcommand{\Cam}[1]{\textcolor{blue}{Cam: #1}}
\newcommand{\Dan}[1]{\textcolor{cyan}{Dan: #1}}
\newcommand{\Kon}[1]{\textcolor{orange}{Kon: #1}}
\newcommand{\eqdef}{\mathbin{\stackrel{\rm def}{=}}}
\makeatletter
\def\hlinewd#1{%
	\noalign{\ifnum0=`}\fi\hrule \@height #1 \futurelet
	\reserved@a\@xhline}
\makeatother

\newtheorem{theorem}{Theorem}
\newtheorem{conjecture}[theorem]{Conjecture}
\newtheorem{corollary}[theorem]{Corollary}
\newtheorem{lemma}[theorem]{Lemma}
\newtheorem{fact}[theorem]{Fact}
\newtheorem{claim}[theorem]{Claim}
\newtheorem{assumption}[theorem]{Assumption}
\newtheorem{definition}{Definition}
\newtheorem{problem}[definition]{Problem}
\newtheorem{connection}{Connection}
\newtheorem{example}[theorem]{Example}


\newtheorem*{rep@theorem}{\rep@title}
\newcommand{\newreptheorem}[2]{%
	\newenvironment{rep#1}[1]{%
		\def\rep@title{#2 \ref{##1}}%
		\begin{rep@theorem}}%
		{\end{rep@theorem}}}
\makeatother
\newreptheorem{theorem}{Theorem}
\newreptheorem{claim}{Claim}

\newcommand{\R}{\mathbb{R}}
\newcommand{\C}{\mathbb{C}}
\newcommand{\N}{\mathbb{N}}
\newcommand{\HH}{\mathcal{H}}
\newcommand{\bs}[1]{\boldsymbol{#1}}
\newcommand{\bv}[1]{\mathbf{#1}}
\newcommand{\wh}{\widehat}
\newcommand{\norm}[1]{\|#1\|}
\newcommand{\opnorm}[1]{\|#1\|_\mathrm{op}}
\DeclareMathOperator{\supp}{\mathrm{supp}}
\DeclareMathOperator{\poly}{poly}
\DeclareMathOperator{\cp}{cap}
\DeclareMathOperator{\cheb}{cheb}
\DeclareMathOperator{\argmax}{argmax}
\DeclareMathOperator{\sinc}{sinc}
\DeclareMathOperator*{\argmin}{arg\,min}
\newcommand{\E}{\mathbb{E}}

\DeclareMathOperator{\tr}{tr}
\DeclareMathOperator{\rank}{rank}
\DeclareMathOperator{\err}{err}
\DeclareMathOperator{\erf}{erf}
\DeclareMathOperator{\range}{range}
\DeclareMathOperator{\Null}{null}
% / Things we're adding

\title{On the Power of Edge Independent Graph Models}

% The \author macro works with any number of authors. There are two commands
% used to separate the names and addresses of multiple authors: \And and \AND.
%
% Using \And between authors leaves it to LaTeX to determine where to break the
% lines. Using \AND forces a line break at that point. So, if LaTeX puts 3 of 4
% authors names on the first line, and the last on the second line, try using
% \AND instead of \And before the third author name.

\author{
	Sudhanshu Chanpuriya \\ University of Massachusetts Amherst \\ \texttt{schanpuriya@umass.edu }
	\And 
	Cameron Musco\\ University of Massachusetts Amherst\\ \texttt{cmusco@cs.umass.edu}
	\AND
	Konstantinos Sotiropoulos\\ Boston University\\ \texttt{ksotirop@bu.edu} 
	\And
	Charalampos E. Tsourakakis\\ Boston University \& ISI Foundation \\ \texttt{tsourolampis@gmail.com} 
}

  % branched from Michael Forbes' macro document
  % September 2014
  \usepackage{nth}
  \usepackage{intcalc}

  \newcommand{\cSTOC}[1]{\nth{\intcalcSub{#1}{1968}}\ Annual\ ACM\ Symposium\ on\ Theory\ of\ Computing\ (STOC)}
  \newcommand{\cFSTTCS}[1]{\nth{\intcalcSub{#1}{1980}}\ International\ Conference\ on\ Foundations\ of\ Software\ Technology\ and\ Theoretical\ Computer\ Science\ (FSTTCS)}
  \newcommand{\cCCC}[1]{\nth{\intcalcSub{#1}{1985}}\ Annual\ IEEE\ Conference\ on\ Computational\ Complexity\ (CCC)}
  \newcommand{\cFOCS}[1]{\nth{\intcalcSub{#1}{1959}}\ Annual\ IEEE\ Symposium\ on\ Foundations\ of\ Computer\ Science\ (FOCS)}
  \newcommand{\cRANDOM}[1]{\nth{\intcalcSub{#1}{1996}}\ International\ Workshop\ on\ Randomization\ and\ Computation\ (RANDOM)}
  \newcommand{\cISSAC}[1]{#1\ International\ Symposium\ on\ Symbolic\ and\ Algebraic\ Computation\ (ISSAC)}
  \newcommand{\cICALP}[1]{\nth{\intcalcSub{#1}{1973}}\ International\ Colloquium\ on\ Automata,\ Languages and\ Programming\ (ICALP)}
  \newcommand{\cCOLT}[1]{\nth{\intcalcSub{#1}{1987}}\ Annual\ Conference\ on\ Computational\ Learning\ Theory\ (COLT)}
  \newcommand{\cCSR}[1]{\nth{\intcalcSub{#1}{2005}}\ International\ Computer\ Science\ Symposium\ in\ Russia\ (CSR)}
  \newcommand{\cMFCS}[1]{\nth{\intcalcSub{#1}{1975}}\ International\ Symposium\ on\ the\ Mathematical\ Foundations\ of\ Computer\ Science\ (MFCS)}
  \newcommand{\cPODS}[1]{\nth{\intcalcSub{#1}{1981}}\ Symposium\ on\ Principles\ of\ Database\ Systems\ (PODS)}
  \newcommand{\cSODA}[1]{\nth{\intcalcSub{#1}{1989}}\ Annual\ ACM-SIAM\ Symposium\ on\ Discrete\ Algorithms\ (SODA)}
  \newcommand{\cNIPS}[1]{Advances\ in\ Neural\ Information\ Processing\ Systems\ \intcalcSub{#1}{1987} (NeurIPS)}
  \newcommand{\cWALCOM}[1]{\nth{\intcalcSub{#1}{2006}}\ International\ Workshop\ on\ Algorithms\ and\ Computation\ (WALCOM)}
  \newcommand{\cSoCG}[1]{\nth{\intcalcSub{#1}{1984}}\ Annual\ Symposium\ on\ Computational\ Geometry\ (SCG)}
  \newcommand{\cKDD}[1]{\nth{\intcalcSub{#1}{1994}}\ ACM\ SIGKDD\ International\ Conference\ on\ Knowledge\ Discovery\ and\ Data\ Mining\ (KDD)}
  \newcommand{\cICML}[1]{\nth{\intcalcSub{#1}{1983}}\ International\ Conference\ on\ Machine\ Learning\ (ICML)}
  \newcommand{\cAISTATS}[1]{\nth{\intcalcSub{#1}{1997}}\ International\ Conference\ on\ Artificial\ Intelligence\ and\ Statistics\ (AISTATS)}
  \newcommand{\cITCS}[1]{\nth{\intcalcSub{#1}{2009}}\ Conference\ on\ Innovations\ in\ Theoretical\ Computer\ Science\ (ITCS)}
  \newcommand{\cPODC}[1]{{#1}\ ACM\ Symposium\ on\ Principles\ of\ Distributed\ Computing\ (PODC)}
  \newcommand{\cAPPROX}[1]{\nth{\intcalcSub{#1}{1997}}\ International\ Workshop\ on\ Approximation\ Algorithms\ for\  Combinatorial\ Optimization\ Problems\ (APPROX)}
  \newcommand{\cSTACS}[1]{\nth{\intcalcSub{#1}{1983}}\ International\ Symposium\ on\ Theoretical\ Aspects\ of\  Computer\ Science\ (STACS)}
  \newcommand{\cMTNS}[1]{\nth{\intcalcSub{#1}{1991}}\ International\ Symposium\ on\ Mathematical\ Theory\ of\  Networks\ and\ Systems\ (MTNS)}
  \newcommand{\cICM}[1]{International\ Congress\ of\ Mathematicians\ {#1} (ICM)}
  \newcommand{\cWWW}[1]{\nth{\intcalcSub{#1}{1991}}\ International\ World\ Wide\ Web\ Conference\ (WWW)}
  \newcommand{\cICLR}[1]{\nth{\intcalcSub{#1}{2012}}\ International\ Conference\ on\ Learning\ Representations\ (ICLR)}
  \newcommand{\cICCV}[1]{\nth{\intcalcSub{#1}{1994}}\ IEEE\ International\ Conference\ on\ Computer\ Vision\ (ICCV)}
  \newcommand{\cICASSP}[1]{#1\ International\ Conference\ on\ Acoustics,\ Speech,\ and\ Signal\ Processing\ (ICASSP)}
  \newcommand{\cUAI}[1]{\nth{\intcalcSub{#1}{1984}}\ Annual\ Conference\ on\ Uncertainty\ in\ Artificial\ Intelligence\ (UAI)}
    \newcommand{\cAAAI}[1]{\nth{\intcalcSub{#1}{1986}}\ AAAI\ Conference\ on\ Artificial\ Intelligence\ (AAAI)}

  \newcommand{\pSTOC}[1]{Preliminary\ version\ in\ the\ \cSTOC{#1}}
  \newcommand{\pFSTTCS}[1]{Preliminary\ version\ in\ the\ \cFSTTCS{#1}}
  \newcommand{\pCCC}[1]{Preliminary\ version\ in\ the\ \cCCC{#1}}
  \newcommand{\pFOCS}[1]{Preliminary\ version\ in\ the\ \cFOCS{#1}}
  \newcommand{\pRANDOM}[1]{Preliminary\ version\ in\ the\ \cRANDOM{#1}}
  \newcommand{\pISSAC}[1]{Preliminary\ version\ in\ the\ \cISSAC{#1}}
  \newcommand{\pICALP}[1]{Preliminary\ version\ in\ the\ \cICALP{#1}}
  \newcommand{\pCOLT}[1]{Preliminary\ version\ in\ the\ \cCOLT{#1}}
  \newcommand{\pCSR}[1]{Preliminary\ version\ in\ the\ \cCSR{#1}}
  \newcommand{\pMFCS}[1]{Preliminary\ version\ in\ the\ \cMFCS{#1}}
  \newcommand{\pPODS}[1]{Preliminary\ version\ in\ the\ \cPODS{#1}}
  \newcommand{\pSODA}[1]{Preliminary\ version\ in\ the\ \cSODA{#1}}
  \newcommand{\pNIPS}[1]{Preliminary\ version\ in\ \cNIPS{#1}}
  \newcommand{\pWALCOM}[1]{Preliminary\ version\ in\ the\ \cWALCOM{#1}}
  \newcommand{\pSoCG}[1]{Preliminary\ version\ in\ the\ \cSoCG{#1}}
  \newcommand{\pKDD}[1]{Preliminary\ version\ in\ the\ \cKDD{#1}}
  \newcommand{\pICML}[1]{Preliminary\ version\ in\ the\ \cICML{#1}}
  \newcommand{\pAISTATS}[1]{Preliminary\ version\ in\ the\ \cAISTATS{#1}}
  \newcommand{\pITCS}[1]{Preliminary\ version\ in\ the\ \cITCS{#1}}
  \newcommand{\pPODC}[1]{Preliminary\ version\ in\ the\ \cPODC{#1}}
  \newcommand{\pAPPROX}[1]{Preliminary\ version\ in\ the\ \cAPPROX{#1}}
  \newcommand{\pSTACS}[1]{Preliminary\ version\ in\ the\ \cSTACS{#1}}
  \newcommand{\pMTNS}[1]{Preliminary\ version\ in\ the\ \cMTNS{#1}}
  \newcommand{\pICM}[1]{Preliminary\ version\ in\ the\ \cICM{#1}}
  \newcommand{\pWWW}[1]{Preliminary\ version\ in\ the\ \cWWW{#1}}
  \newcommand{\pICLR}[1]{Preliminary\ version\ in\ the\ \cICLR{#1}}
  \newcommand{\pICCV}[1]{Preliminary\ version\ in\ the\ \cICCV{#1}}
  \newcommand{\pICASSP}[1]{Preliminary\ version\ in\ the\ \cICASSP{#1}}
  \newcommand{\pUAI}[1]{Preliminary\ version\ in\ the\ \cUAI{#1}, #1}
    \newcommand{\pAAAI}[1]{Preliminary\ version\ in\ the\ \cAAAI{#1}, #1}



  \newcommand{\STOC}[1]{Proceedings\ of\ the\ \cSTOC{#1}}
  \newcommand{\FSTTCS}[1]{Proceedings\ of\ the\ \cFSTTCS{#1}}
  \newcommand{\CCC}[1]{Proceedings\ of\ the\ \cCCC{#1}}
  \newcommand{\FOCS}[1]{Proceedings\ of\ the\ \cFOCS{#1}}
  \newcommand{\RANDOM}[1]{Proceedings\ of\ the\ \cRANDOM{#1}}
  \newcommand{\ISSAC}[1]{Proceedings\ of\ the\ \cISSAC{#1}}
  \newcommand{\ICALP}[1]{Proceedings\ of\ the\ \cICALP{#1}}
  \newcommand{\COLT}[1]{Proceedings\ of\ the\ \cCOLT{#1}}
  \newcommand{\CSR}[1]{Proceedings\ of\ the\ \cCSR{#1}}
  \newcommand{\MFCS}[1]{Proceedings\ of\ the\ \cMFCS{#1}}
  \newcommand{\PODS}[1]{Proceedings\ of\ the\ \cPODS{#1}}
  \newcommand{\SODA}[1]{Proceedings\ of\ the\ \cSODA{#1}}
  \newcommand{\NIPS}[1]{\cNIPS{#1}}
  \newcommand{\WALCOM}[1]{Proceedings\ of\ the\ \cWALCOM{#1}}
  \newcommand{\SoCG}[1]{Proceedings\ of\ the\ \cSoCG{#1}}
  \newcommand{\KDD}[1]{Proceedings\ of\ the\ \cKDD{#1}}
  \newcommand{\ICML}[1]{Proceedings\ of\ the\ \cICML{#1}}
  \newcommand{\AISTATS}[1]{Proceedings\ of\ the\ \cAISTATS{#1}}
  \newcommand{\ITCS}[1]{Proceedings\ of\ the\ \cITCS{#1}}
  \newcommand{\PODC}[1]{Proceedings\ of\ the\ \cPODC{#1}}
  \newcommand{\APPROX}[1]{Proceedings\ of\ the\ \cAPPROX{#1}}
  \newcommand{\STACS}[1]{Proceedings\ of\ the\ \cSTACS{#1}}
  \newcommand{\MTNS}[1]{Proceedings\ of\ the\ \cMTNS{#1}}
  \newcommand{\ICM}[1]{Proceedings\ of\ the\ \cICM{#1}}
  \newcommand{\WWW}[1]{Proceedings\ of\ the\ \cWWW{#1}}
  \newcommand{\ICLR}[1]{Proceedings\ of\ the\ \cICLR{#1}}
  \newcommand{\ICCV}[1]{Proceedings\ of\ the\ \cICCV{#1}}
  \newcommand{\ICASSP}[1]{Proceedings\ of\ the\ \cICASSP{#1}}
  \newcommand{\UAI}[1]{Proceedings\ of\ the\ \cUAI{#1}}
   \newcommand{\AAAI}[1]{Proceedings\ of\ the\ \cAAAI{#1}}


  \newcommand{\arXiv}[1]{\href{http://arxiv.org/abs/#1}{arXiv:#1}}
  \newcommand{\farXiv}[1]{Full\ version\ at\ \arXiv{#1}}
  \newcommand{\parXiv}[1]{Preliminary\ version\ at\ \arXiv{#1}}
  \newcommand{\CoRR}{Computing\ Research\ Repository\ (CoRR)}

  \newcommand{\cECCC}[2]{\href{http://eccc.hpi-web.de/report/20#1/#2/}{Electronic\ Colloquium\ on\ Computational\ Complexity\ (ECCC),\ Technical\ Report\ TR#1-#2}}
  \newcommand{\ECCC}{Electronic\ Colloquium\ on\ Computational\ Complexity\ (ECCC)}
  \newcommand{\fECCC}[2]{Full\ version\ in\ the\ \cECCC{#1}{#2}}
  \newcommand{\pECCC}[2]{Preliminary\ version\ in\ the\ \cECCC{#1}{#2}}
\begin{document}

\maketitle

\begin{abstract}
\label{sec:abstract}

%% 1. what is the problem 
Scientific applications that run on leadership computing facilities often face the challenge 
of being unable to fit leading science cases onto accelerator devices due to memory constraints 
(memory-bound applications).
%
% 2. what is your solution 
In this work, the authors studied one such US Department of Energy mission-critical condensed matter 
physics application, Dynamical Cluster Approximation (DCA++), and this paper discusses how device memory-bound challenges were successfully reduced  by proposing an effective 
``all-to-all'' communication method---a ring communication algorithm. 
%
This implementation takes advantage of acceleration on GPUs and remote direct memory access (RDMA) for fast data exchange between GPUs. 
%
\\Additionally, the ring algorithm was optimized with sub-ring communicators
and multi-threaded support to further reduce communication overhead and 
expose more concurrency, respectively.
%
% 3. What's the cherry-picked evaluation result you want to mention
The computation and communication were also analyzed 
by using the Autonomic Performance Environment for Exascale 
(APEX) profiling tool,  and this paper further discusses the 
performance trade-off for the ring algorithm implementation. 
%
The memory analysis on the ring algorithm shows that the allocation size for the authors' most 
memory-intensive data structure per GPU is now reduced to $1/p$ of the original size, where $p$ is the number of GPUs in the ring communicator.
%
The communication analysis suggests that 
the distributed Quantum Monte Carlo execution time grows linearly as sub-ring size increases, and the cost of messages passing through the network interface connector could be a limiting factor.


%
% \todoRed{Ronnie: Next sentence needs rewrite, too much information about Green's function that no one knows in the abstract; recommend generalizing.} \emph {However, DCA++ is currently facing memory-bound challenge as 
% a larger device array $G_t$ is limited by device memory size, where
% $G_t$ is a two-particle Green's function that allows condensed matter
% scientists to explore larger and more complex (higher fidelity)
% physics cases.}

\end{abstract}

\keywords{DCA++, Quantum Monte Carlo, GPU Remote Direct Memory Access, memory-bound issue, exascale machines}


%\begin{abstract}
%We study the limits of \emph{edge independent random graph models}, in which  each edge is added to the graph independently with some probability. Such models include the Erd\"{o}s-R\'{e}nyi and stochastic block models, as well as many  modern neural-network-based generative models, such as NetGAN, variational graph autoencoders, and CELL. %We explore the limits of edge independent models. 
%We show that subject to a \emph{bounded overlap} condition, which ensures that the model does not simply memorize a single graph, edge independent models are inherently limited in their ability to generate graphs with high triangle and other subgraph densities. Notably, such high densities are known to appear in real-world social networks and other connection graphs. We complement our negative results with a simple baseline that balances overlap and accuracy, performing comparably to more complex generative models in reconstructing many graph statistics. 
%\end{abstract}

\section{Introduction}
\label{sec:intro}

Our work centers on \emph{edge independent graph models}, in which each edge $(i,j)$ is added to the graph independently with some probability $P_{ij} \in [0,1]$. Formally,
\begin{definition}[Edge Independent Graph Model]\label{def:ei}
For any symmetric matrix $P \in [0,1]^{n \times n}$ let $\mathcal{G}(P)$ be the distribution over undirected unweighted graphs where $G \sim  \mathcal{G}(P)$ contains edge $(i,j)$ independently, with probability $P_{ij}$. I.e., $p(G) = \prod_{(i,j) \in E(G)} P_{ij} \cdot \prod_{(i,j) \notin E(G)} (1-P_{ij})$.
\end{definition}

Edge independent models encompass many  classic random graph models. This includes the Erd\"{o}s-R\'{e}nyi  model, where for all $i \neq j$, $P_{ij} = p$ for  some fixed $p \in [0,1]$ \cite{ErdosRenyi:1960}. It also includes the stochastic block model where $P_{ij} = p$ if two nodes are in the same community and $P_{ij} = q$ if two nodes are in different communities for some fixed $p,q \in [0,1]$ with $q < p$ \cite{SnijdersNowicki:1997}. Other examples include e.g., the Chung-Lu configuration model \cite{ChungLu:2002}, stochastic Kronecker graphs \cite{LeskovecChakrabartiKleinberg:2010}.

Recently, significant attention has focused on \emph{graph generative models}, which seek to learn a distribution over graphs that share similar properties to a given training graph, or set of graphs. Many algorithms parameterize this distribution as an edge independent model or closely related distribution. % based on a probability matrix $P \in [0,1]^{n \times n}$ learned from the input. 
E.g., NetGAN and the closely related CELL model both produce $P \in [0,1]^{n \times n}$ and then sample edges independently without replacement with probabilities proportional to its entries, ensuring that at least one edge is sampled adjacent to each node \cite{bojchevski2018netgan,rendsburgnetgan}.
%\begin{itemize}
%\item NetGAN Without GAN (CELL) \cite{rendsburgnetgan} -- explicitly uses an edge independent model except avoids self loops and isolated nodes.
%\item NetGAN \cite{bojchevski2018netgan} -- create a symmetric score matrix, and sample without replacement with probabilities proportional to the entries. Sample at least one edge from each node, using the row distribution for that node.
Variational Graph Autoencoders (VGAE), GraphVAE, Graphite, and MolGAN are also all based on edge independent models \cite{KipfWelling:2016,SimonovskyKomodakis:2018,De-CaoKipf:2018,GroverZweigErmon:2019}. 
%\item 
%MolGAN \cite{De-CaoKipf:2018}, which focuses on modeling molecular graphs, is based on a closely related model where there are multiple node and edge types. The generated graph can be thought of as a union of edge independent graphs, one for each edge type. 
% -- For expected adjacency tensor $n \times n \times p$ where $p$ is the number of bond types and molecule matrix $n \times k$ where $k$ is the number of molecule types, and then sample from these using 'categorical sampling'. I'm not sure exactly what this means... but might work for us. 
%\todo{Look in code to see exactly what they are doing.}
%\item Graphite \cite{GroverZweigErmon:2019} -- directly uses an edge independent model.
%\item Variational Graph Autoencoders \cite{KipfWelling:2016} -- seem to use edge independent model
%\item GraphVAE \cite{SimonovskyKomodakis:2018} -- also using the variational autoencoder approach, and I think edge independent although less clear? It might be that they just take max entries in a continous adjacency matrix, which is randomized due to a random input to the network.
%\end{itemize}

%Standard edge independent models:
%\begin{itemize}
%\item Erdos R\'{e}nyi graphs
%\item Stochastic block model
%\item Chung-Lu configuration model
%\item \url{https://services.math.duke.edu/~rtd/math777/CL_PNAS.pdf}
%\item Kronecker graphs?
%\end{itemize}

Given their popularity in both classical and modern graph generative models, it is natural to ask:
\begin{quote} \emph{How suited are edge independent models to modeling real-world networks. Are they able to capture features such as power-law degree distributions, small-world properties, and high clustering coefficients (triangle densities)? }
\end{quote}

\subsection{Impossibility Results for Edge Independent Models}

In this work we focus on the ability of edge independent models to generate graphs with high triangle, or other small subgraph densities. High triangle density (equivalently, a high clustering coefficient) is a well-known hallmark of real-work networks \cite{WattsStrogatz:1998,SalaCaoWilson:2010,DurakPinarKolda:2012} and has been the focus of recent work exploring the power and limitations of edge-independent graph models \cite{SeshadhriSharmaStolman:2020,ChanpuriyaMuscoSotiropoulos:2020}.

It is clear that edge independent models can generate triangle dense graphs. In particular, $P \in [0,1]^{n \times n}$ in Def. \ref{def:ei} can be set to the binary adjacency matrix of any undirected graph, and $\mathcal{G}(P)$ will generate that graph with probability $1$, no matter how triangle dense it is. However, this would not be a particularly interesting generative model -- ideally $\mathcal{G}(P)$ should generate a wide range of graphs. To capture this intuitive notion, we define the \emph{overlap} of an edge-independent model, which is closely related to the overlap stopping criterion for training used in training graph generative models \cite{bojchevski2018netgan,rendsburgnetgan}.
\begin{definition}[Expected Overlap]\label{def:ov} For symmetric $P \in [0,1]^{n \times n}$ let $V(P) \eqdef \E_{G \sim \mathcal{G}(P)} |E(G)|$ and
% the expected overlap of two graphs drawn independently from $\mathcal{G}(P)$ is:
\begin{align*}
Ov(P) \eqdef \frac{\E_{G_1,G_2 \sim \mathcal{G}(P)} |E(G_1) \cap E(G_2)|}{V(P)}.
\end{align*}
\end{definition}
That is, for any $P \in [0,1]^{n \times n}$, $Ov(P) \in [0,1]$ is the ratio of the expected number of edges shared by two graphs drawn independently from $\mathcal{G}(P)$ to the expected number of edges in a graph drawn from $\mathcal{G}(P)$. In one extreme, when $P$ is a binary adjacency matrix, $Ov(P) = 1$, and our generative model has simply memorized a single graph. In the other, if $P_{ij} = p$ for all $i \neq j$ (i.e., $\mathcal{G}(P)$ is Erd\"{o}s-R\'{e}nyi), $Ov(P) = p$. This is the minimum possible overlap when $V(P) = p \cdot {n \choose 2}$.

Our main result is that for any edge independent model with bounded overlap, $G \sim \mathcal{G}(P)$ cannot have too many triangles in expectation. In particular:
\begin{theorem}[Main Result -- Expected Triangles]\label{thm:tri} For a graph $G$, let $\Delta(G)$ denote the number of triangles in $G$. Consider symmetric $P \in [0,1]^{n \times n}$.
\begin{align*}
\E_{G \sim \mathcal{G}(P)} \left [\Delta(G) \right ]\le \frac{\sqrt{2}}{3} \cdot Ov(P)^{3/2} \cdot V(P)^{3/2}.
\end{align*}
\end{theorem}
As an example, consider the setting where the distribution generates sparse graphs, with $V(P) = \Theta(n)$. Theorem \ref{thm:tri} shows that whenever  $Ov(P) = o(1/n^{1/3})$, $\E_{G \sim \mathcal{G}(P)} \Delta(G) = o(n)$ -- i.e. the graph is very triangle sparse with the number of triangles sublinear in the number of nodes. %A distribution with higher triangle density thus requires $Ov(P) = \Omega(1/n^{1/3})$. 
This verifies that  an Erd\"{o}s-R\'{e}nyi graph cannot achieve simultaneously linear number of edges  (i.e., $Ov(P) = O(1/n)$ ) and super-linear number of triangles (i.e., $Ov(P) = \Omega(1/n^{1/3})$) under our proposed lens of viewing generative models. 

%On the otherhand, if the distribution simply memorizes a single sparse graph, and has $Ov(P) =1$, then the theorem allows $\E_{G \sim \mathcal{G}(P)} \left [\Delta(G) \right ]$ to be as large as $\Theta(n^{3/2})$. This is indeed matched, when the model simply memorizes the clique on $n^{1/2}$ nodes.

We extend Theorem \ref{thm:tri} to give similar bounds for the density of squares and other $k$-cycles (Thm. \ref{thm:k}), as well as for the global clustering coefficient (Thm. \ref{thm:cc}). In all cases we show that our bounds are tight -- e.g., in the triangle case, %for any $\gamma \in (0,1]$,
 there is indeed an edge independent model with %with $Ov(P) = \gamma$ and 
 $\E_{G \sim \mathcal{G}(P)} \left [\Delta(G) \right ] = \Theta \left (Ov(P)^{3/2} \cdot V(P)^{3/2} \right )$, matching the lower bound in Theorem \ref{thm:tri}. 
 
 %Our proofs are extremely simple -- in short, $Ov(P)$ is closely related to the Frobenius norm $\norm{P}_F^2$. 

%A number of recent papers have focused on this question \cite{SeshadhriSharmaStolman:2020,ChanpuriyaMuscoSotiropoulos:2020}

\subsection{Empirical Findings}

Our theoretical results help explain why, despite performing well in a variety of other metrics, edge independent graph generative models have been reported to generate graphs with many fewer triangles and squares on average than the real-world graphs that they are trained on. Rendsburg et al. \cite{rendsburgnetgan} test a suite of these models, including their own CELL model and the related NetGAN model \cite{bojchevski2018netgan}. Of all these models, when trained on the \textsc{Cora-ML} graph with 2,802 triangles and 14,268 squares, none is able to generate graphs with more than 1,461 triangles and 6,880 squares on average. Similar gaps are observed for a number of other graphs.
 Rendsburg et al. also report that the triangle count increases as their notion of overlap (closely related to Def. \ref{def:ov}) increases. Theorem \ref{thm:tri} demonstrates that this underestimation of triangle count, and its connection to overlap is \emph{inherent to all edge independent models, no matter how refined a method used to learn the underlying probability matrix $P$}. 
 
 While our theoretical results bound the performance of any  edge independent model, there may still be variation in how specific models trade-off overlap and realistic graph generation. 
 To better understand this trade-off, we introduce two simple models with easily tunable overlap as baselines. One is based on reproducing the degree sequence of the original graph; the other, which is even simpler, is based on reproducing the volume.
 In both  models, $P$ is a  weighted average of the input graph adjacency matrix and a probability matrix of minimal complexity which matches either the input degrees or the volume. In the latter case, to match just the volume, we simply use an Erd\"{o}s-R\'{e}nyi graph. In the former case, to match the degree sequence, we introduce our own model, the \emph{odds product model}; this model is similar to the Chung-Lu configuration model \cite{ChungLu:2002}, but, unlike Chung-Lu, is able to match degree sequences of real-world graphs with high maximum degree.
We find that these simple baselines are often competitive with more complex models like CELL in terms of matching key graph statistics, like triangle count and clustering coefficient, at similar levels of overlap. %\Cam{What should the takeway from this finding be? Can we add a sentence or two?}

%\todo{Collect results here showing which features they tend to  capture well and which they tend not too. }

\subsection{Related Work}\label{sec:rel}

\noindent\textbf{Existing impossibility results.} Our work is inspired by that of Seshadhri et al. \cite{SeshadhriSharmaStolman:2020}, which also proves limitations on the ability of edge independent models to represent triangle dense graphs. They show that if $P = \max(0,\min(1,XX^T))$ where $X \in \R^{n \times k}$ for $k \ll n$ and the max and min are applied entrywise, then $G \sim \mathcal{G}(P)$ cannot have many triangles adjacent to low-degree nodes in expectation. This setting arises commonly when $P$ is generated using low-dimensional node embeddings -- represented by the rows of $X$. Chanpuriya et al. \cite{ChanpuriyaMuscoSotiropoulos:2020}, show that in a slightly more general model, where $P = \max(0,\min(1,XY^T))$, this lower bound no longer holds -- $X,Y \in \R^{n \times k}$ can be chosen so that $P$ is the binary adjacency matrix of any graph with maximum degree upper bounded by $O(k)$ -- no matter how triangle dense that graph is. Thus, even such low-rank edge independent models can represent triangle dense graphs -- by memorizing a single one. In the appendix, we prove a similar result when $P$ is generated from the CELL model of \cite{rendsburgnetgan}, which simplifies NetGAN \cite{bojchevski2018netgan}.

%However, this comes at the cost of high overlap -- if $P$ is a binary adjacency matrix, $Ov(P) = 1$. 
Our results show that this trade-off between the ability to capture triangle density and memorization is inherent -- even without any low-rank constraint, edge independent models with low overlap simply cannot represent graphs with high triangle or other small subgraph density.

It is well understood that specific edge independent models, e.g., Erd\"{o}s-R\'{e}nyi graphs, the Chung-Lu model, and stochastic Kronecker graphs, do not capture many properties of real-world networks, including high triangle density \cite{WattsStrogatz:1998,PinarSeshadhriKolda:2012}. Our results can be viewed as a generalization of these observations, to all edge independent models with low overlap. Despite the limitations of classic models, edge independent models are still very prevalent in today's literature on graph generative models. Our more general results make clear the limitations of this approach.

\noindent\textbf{Non-independent models.} While edge independent models are very prevalent in the literature, many important models do not fit into this framework. Classic models include the Barab\'{a}si–Albert and other preferential attachment models \cite{BarabasiAlbert:1999}, Watts–Strogatz  small-world graphs \cite{WattsStrogatz:1998}, and random geometric graphs \cite{DallChristensen:2002}. Many of these models were introduced directly in response to shortcomings of classic edge independent models, including their  inability to produce high triangle densities

More recent graph generative models include 
GraphRNN \cite{YouYingRen:2018} and a number of other works \cite{LiVinyalsDyer:2018,LiaoLiSong:2019}.
%-- generates one node at a time, along with all its connections. In simplified models, edges from node are added independently. But probabilities may depend on previous connections. thus this doesn't seem to fall under the edge independent model. You could easily generate triangle dense graphs with this set up.
%\item \cite{LiVinyalsDyer:2018} -- also generates things as a non-independent sequence.
%\end{itemize} 
Our impossibility results do not apply to such models, and in fact suggest that perhaps they may be preferable to edge independent models, if a distribution over graphs with high triangle density is desired. A very interesting direction for future work would be to prove limitations on broad classes of non-independent models, and perhaps to understand exactly what type of correlation amongst edges is needed to generate graphs with both low overlap \footnote{We note that for non-edge independent models, the measure of overlap as defined earlier should be adapted to take into account the order (permutation) of the vertices in the final graph. In particular, the overlap in this case should be the maximum value of it over any permutation of the vertex set.}and hallmark features of real-world networks.


\section{Impossibility Results for Edge Independent Models}
\label{sec:impossibility}

We now prove our main results on the limitations of edge independent models with bounded overlap. % in generating graphs with high triangle and other $k$-cycle densities. 
We start with a simple lemma that will be central in all our proofs.

%\begin{definition}[Expected Volume] For $P \in [0,1]^{n \times n}$ the expected volume of a graph drawn from $\mathcal{G}(P)$ is:
%\begin{align*}
%V(P) = \E_{G \sim \mathcal{G}(P)}[2\cdot |E(G)|]
%\end{align*}
%\end{definition}

\begin{lemma}\label{lem:mainSimple} For any symmetric $P \in [0,1]^{n \times n}$, $\frac{\norm{P}_F^2}{2} \le Ov(P) \cdot V(P) \le \norm{P}_F^2.$
\end{lemma}
\begin{proof}
Let $I[(i,j) \in G]$ be the $0,1$ indicator random variable that an edge $(i,j)$ appears in the graph $G$.  $Ov(P) \cdot V(P) = \E_{G_1,G_2 \sim \mathcal{G}(P)} |E(G_1) \cap E(G_2)|$. 
By linearity of expectation and the independence of $G_1$ and $G_2$ we have,
$$Ov(P) \cdot V(P) = \E_{G_1,G_2 \sim \mathcal{G}(P)} \sum_{i \le j} I[(i,j) \in G_1] \cdot I[(i,j) \in G_2] = \sum_{i \le j} P_{ij}^2.$$
The bound follows since $P$ is symmetric. Note that the lower bound $\frac{\norm{P}_F^2}{2} \le Ov(P) \cdot V(P)$ is an equality if $P$ is $0$ on the diagonal -- i.e., there is no probability of self loops.
\end{proof}

%\begin{definition}[Expected Degree and Triangle Density] For $P \in [0,1]^{n \times n}$ the expected degree 
% density of a graph drawn from $\mathcal{G}(P)$ is:
%\begin{align*}
%T(P) = \E_{G \sim \mathcal{G}(P)} \sum_{.
%\end{align*}
%\end{definition}

\subsection{Triangles}

Lemma \ref{lem:mainSimple} connects $Ov(P) \cdot V(P)$ to $\norm{P}_F^2$ and in turn the eigenvalue spectrum of $P$ since $\norm{P}_F^2 = \sum_{i=1}^n \lambda_i(P)^2$, where $\lambda_1(P),\ldots, \lambda_n(P) \in \R$ are the eigenvalues of $P$. The expected number of triangles in $G \sim \mathcal{G}(P)$ can be written in terms of this spectrum as well, allowing us to relate overlap to this expected triangle count, and prove our main theorem (Theorem \ref{thm:tri}), restated below.

\begin{reptheorem}{thm:tri} For a graph $G$, let $\Delta(G)$ denote the number of triangles in $G$. Consider symmetric $P \in [0,1]^{n \times n}$. 
\begin{align*}
\E_{G \sim \mathcal{G}(P)} \left [\Delta(G) \right ]\le \frac{\sqrt{2}}{3} \cdot Ov(P)^{3/2} \cdot V(P)^{3/2}.
\end{align*}
\end{reptheorem}
\begin{proof}
By linearity of expectation,
\begin{align}
\E_{G \sim \mathcal{G}(P)} \left [\Delta(G) \right ] &= \frac{1}{6} \sum_{i=1}^n \sum_{j=1}^n \sum_{k=1}^n \Pr \left [(i,j) \in E(G) \cap (j,k) \in E(G) \cap (k,i) \in E(G) \right]\nonumber \\
&= \frac{1}{6} \sum_{i=1}^n \sum_{j=1}^n \sum_{k=1}^n P_{ij} P_{jk} P_{ki} = \frac{1}{6} \tr(P^3) = \frac{1}{6} \sum_{i=1}^n \lambda_i(P)^3.\label{eq:sixth}
\end{align}
Letting $\lambda_1(P)$ denote the largest magnitude eigenvalue of $P$, we can in turn bound
$$\tr(P^3) \le |\lambda_1(P)| \cdot \sum_{i=1}^n \lambda_i(P)^2 = |\lambda_1(P)| \cdot \norm{P}_F^2.$$ Since $|\lambda_1(P)| \le \norm{P}_F$, this gives via Lemma \ref{lem:mainSimple}
$$\tr(P^3) \le \norm{P}_F^3 \le 2 \sqrt{2} \cdot Ov(P)^{3/2} \cdot V(P)^{3/2}.$$
%where the last inequality follows from Lemma \ref{lem:mainSimple}. 
Combining this bound with \eqref{eq:sixth} completes the theorem.
\end{proof}

%\Cam{Switch to ER example but still mention clique for intuition.}
The bound of Theorem \ref{thm:tri} is tight up to constants, for any possible value of $Ov(P)$. The tight example is when $P$ is simply an Erd\"{o}s-R\'{e}nyi graph.

% is when $P$ consists of a clique on $\Theta(Ov(P) \cdot \sqrt{n})$ nodes, unioned with a sparse Erd\"{o}s-R\'{e}nyi graph.
\begin{theorem}[Tightness of Expected Triangle Bound]\label{thm:tight}
For any $\gamma \in (0,1]$, there exists a symmetric $P \in [0,1]^{n \times n}$ with $Ov(P) = \gamma$ and $\E_{G \sim \mathcal{G}(P)} [\Delta(G)] = \Theta( \gamma^{3/2} \cdot V(P)^{3/2})$.
 \end{theorem}
 \begin{proof}
 Let $P_{ij} = \gamma$ for all $i \neq j $. We have $V(P) = \gamma \cdot {n \choose 2}$ and $Ov(P) \cdot V(P) = \gamma^2 \cdot {n \choose 2}$ Thus, $Ov(P)= \gamma$. Further,  by linearity of expectation, 
 $$\E_{G \sim \mathcal{G}(P)} [\Delta(G)] = \gamma^3 \cdot {n \choose 3} = \Theta(\gamma^3 \cdot n^3) = \Theta(\gamma^{3/2} \cdot V(P)^{3/2}).$$
% 
% Let $S$ be a subset of $\sqrt{ \gamma  n/2}$ nodes. %, where $c$ is a fixed constant.  
%Let  $P_{ij} = 1$ for all $i \neq j$, $i,j \in S$ and $P_{ij} = 1/n$ for all other pairs $i,j$ with $i \neq j$.    
% We have $V(P) \ge \frac{1}{n} \cdot {n \choose 2} = \frac{n-1}{2} $. %On the rest of the graph we will have $ \Theta(n)$ edges in expectation. 
% Further, by Lemma \ref{lem:mainSimple},
% $$Ov(P) \cdot V(P) \le \norm{P}_F^2 \le {\sqrt{\gamma n/2} \choose 2 } + \frac{1}{n^2} \cdot n^2 \le \frac{\gamma n}{4} +1.$$
% Combined with our bound on $V(P)$, as long as $n$ is large enough so that  $\gamma n \ge 6$,
% $$Ov(P) \le \frac{\frac{\gamma n}{4} +1}{\frac{n-1}{2}} = \frac{\frac{\gamma n}{2} +2}{n-1} \le \frac{\frac{\gamma n}{2}(1+2/3)}{5/6 \cdot n}  \le \gamma.$$
% 
% Finally, for $G \sim \mathcal{G}(P)$, $\Delta(G)$ is lower bounded by the number of triangles in the clique on $S$ which $G$ contains with probability $1$ -- i.e., 
% $$\E_{G \sim \mathcal{G}(P)}[\Delta(G)] \ge {\sqrt{\gamma n/2} \choose 3}= \Theta(\gamma^{3/2} \cdot n^{3/2}) =  \Theta(\gamma^{3/2} \cdot V(P)^{3/2}).$$
  \end{proof}
  We note that another example when Theorem \ref{thm:tri} is tight is when $P$ is a union of a fixed clique on $\Theta(\gamma \cdot n)$ nodes and an Erd\"{o}s-R\'{e}nyi graph with connection probability $1/n$ on the rest of the nodes.
  
\subsection{Squares and Other $k$-cycles}
  
We can extend Thm. \ref{thm:tri} to bound the expected number of $k$-cycles in $G \sim \mathcal{G}(P)$ in terms of $Ov(P)$. %, we prove in the appendix,
%
%For example,
% \begin{theorem}[Bound on Expected Squares]\label{thm:square} For a graph $G$, let $\square(G)$ denote the number of squares (4-cycles) in $G$. Consider $P \in [0,1]^{n \times n}$ with $Ov(P) \le \gamma \cdot V(P)$ for some $\gamma <1$. Then
%\begin{align*}
%\E_{G \sim \mathcal{G}(P)} \left [\square(G) \right ]\le \frac{1}{2} \cdot \gamma^2 \cdot V(P)^2.
%\end{align*}
%\end{theorem}
%\begin{proof}
%The number of squares is the number of non-backtracking 4-cycles in $G$ (i.e. squares), which can be written as:
%\begin{align*}
%\E_{G \sim \mathcal{G}(P)} \left [\square(G) \right ] = \frac{1}{8} \cdot \sum_{i =1}^n \sum_{j \in [n] \setminus i} \sum_{k \in [n]\setminus\{i,j\}} \sum_{\ell \in [n]\setminus \{i,j,k\}} P_{ij} P_{jk} P_{k \ell} P_{\ell i}.
%\end{align*}
% The $1/8$ factor accounts for the fact that in the sum, each square is counted $8$ times -- once for each potential starting vector $i$ and once of each direction it may be traversed. We then can bound
%\begin{align*}
%\E_{G \sim \mathcal{G}(P)} \left [\square(G) \right ]  \le \frac{1}{8} \cdot\sum_{i \in [n]} \sum_{j \in [n]} \sum_{k \in [n]} \sum_{\ell \in [n]} P_{ij} P_{jk} P_{k \ell} P_{\ell i} = \frac{1}{8} \cdot \tr(P^4).
%\end{align*}
%This in turn gives
%\begin{align*}
%\E_{G \sim \mathcal{G}(P)} \left [\square(G) \right ] \le \frac{1}{8} \cdot |\lambda_1(P)|^2 \cdot \norm{P}_F^2 \le \frac{1}{8} \norm{P}_F^4 \le \frac{1}{2} Ov(P)^2.
%\end{align*}
%This completes the theorem after plugging in  $Ov(P) \le \gamma \cdot V(P)$.
%\end{proof}
%It is not hard to see that Theorem \ref{thm:square} is also tight in the same example as Theorem \ref{thm:tight}, since a clique on $\sqrt{c \gamma n}$ vertices will have ${\sqrt{c \gamma n} \choose 4} = \Theta(\gamma^2 n^2) = \Theta(\gamma^2  \cdot V(P)^2)$ squares. We can also just directly extend the proof of Theorem \ref{thm:square} to  bound the expected number of $k$-cycles in $G$, giving a bound which is also tight up to constants by the example of Theorem \ref{thm:tight} for $k = O(1)$.
\begin{theorem}[Bound on Expected $k$-cycles]\label{thm:k}
For a graph $G$, let $C_k(G)$ denote the number of $k$-cycles in $G$. Consider symmetric $P \in [0,1]^{n \times n}$. 
% with $Ov(P) \le \gamma \cdot V(P)$ for some $\gamma <1$. Then
\begin{align*}
\E_{G \sim \mathcal{G}(P)} \left [C_k(G) \right ] \le \frac{2^{k/2}}{2k} \cdot Ov(P)^{k/2} \cdot V(P)^{k/2}.
\end{align*}
\end{theorem}
\begin{proof}
For notational simplicity, we focus on $k = 4$. The proof directly extends to general $k$. $C_4(G)$ is the number of non-backtracking 4-cycles in $G$ (i.e. squares), which can be written as
\begin{align*}
\E_{G \sim \mathcal{G}(P)} \left [C_4(G) \right ] = \frac{1}{8} \cdot \sum_{i =1}^n \sum_{j \in [n] \setminus i} \sum_{k \in [n]\setminus\{i,j\}} \sum_{\ell \in [n]\setminus \{i,j,k\}} P_{ij} P_{jk} P_{k \ell} P_{\ell i}.
\end{align*}
 The $1/8$ factor accounts for the fact that in the sum, each square is counted $8$ times -- once for each potential starting vector $i$ and once of each direction it may be traversed. For general $k$-cycles this factor would be $\frac{1}{2k}$. We then can bound
\begin{align*}
\E_{G \sim \mathcal{G}(P)} \left [C_4(G) \right ]  \le \frac{1}{8} \cdot\sum_{i \in [n]} \sum_{j \in [n]} \sum_{k \in [n]} \sum_{\ell \in [n]} P_{ij} P_{jk} P_{k \ell} P_{\ell i} = \frac{1}{8} \cdot \tr(P^4).
\end{align*}
For general $k$-cycles this bound would be $\E_{G \sim \mathcal{G}(P)} \left [C_k(G) \right ] \le \frac{1}{2k} \tr(P^k).$
This in turn gives
\begin{align*}
\E_{G \sim \mathcal{G}(P)} \left [C_k(G) \right ] \le \frac{1}{2k} \cdot |\lambda_1(P)|^{k-2} \cdot \norm{P}_F^{2} \le \frac{1}{2k} \norm{P}_F^k \le \frac{2^{k/2}}{2k} Ov(P)^{k/2} \cdot V(P)^{k/2},
\end{align*}
where the last bound follows from Lemma \ref{lem:mainSimple}.
This completes the theorem..
\end{proof}
It is not hard to see that Theorem \ref{thm:k} is also tight up to a constant depending on $k$ for any overlap $\gamma \in (0,1]$, also for an Erd\"{o}s-R\'{e}nyi graph with connection probability $\gamma$. 
\begin{theorem}[Tightness of Expected $k$-cycle Bound]\label{thm:tightK}
For any $\gamma \in (0,1]$, there exists $P \in [0,1]^{n \times n}$ with $Ov(P) = \gamma$ and $\E_{G \sim \mathcal{G}(P)} [C_k(G)] = \Theta \left ( \frac{\gamma^{k/2} \cdot V(P)^{k/2}}{k!} \right)$.
 \end{theorem}
%\begin{proof}
%Consider the same $P$ as Thm. \ref{thm:tight}. The clique on $\sqrt{\gamma n/2}$ vertices has $ { \sqrt{\gamma n/2} \choose k} = \Theta \left (\frac{\gamma^{k/2} n^{k/2}}{k!} \right)$ $k$-cycles. So $Ov(P) \le \gamma$ and $\E_{G \sim \mathcal{G}(P)} \left [C_k(G) \right ] = \Theta \left (\frac{\gamma^{k/2}  \cdot V(P)^{k/2}}{k!} \right)$, giving the result. % Theorem \ref{thm:k} up to a $\Theta \left (\frac{2^{k/2} \cdot k!}{2k} \right )$ factor.
%\end{proof}

%We can also just directly extend the proof of Theorem \ref{thm:square} to  bound the expected number of $k$-cycles in $G$, giving a bound which is also tight up to constants by the example of Theorem \ref{thm:tight} for $k = O(1)$.
%Theorem \ref{thm:kclique} is also tight up to constants for constant $k$, by the same example as Theorem \ref{thm:tight}.
  
%  We can easily extend Theorem \ref{thm:tri} to $k$-cliques. We have
%  
%  \begin{theorem}[Bound on Expected $k$-Cliques]\label{thm:kclique} For a graph $G$, let $kC(G)$ denote the number of k-Cliques in $G$. Consider $P \in [0,1]^{n \times n}$ with $Ov(P) \le \gamma \cdot V(P)$ for some $\gamma <1$. Then
%\begin{align*}
%\E_{G \sim \mathcal{G}(P)} \left [kC(G) \right ]\le \frac{2^{k/2}}{k!} \cdot \gamma^{k/2} \cdot V(P)^{k/2}
%\end{align*}
%\end{theorem}
%\begin{proof}
%We can write $\E_{G \sim \mathcal{G}(P)} \left [kC(G) \right ] = \frac{1}{k!} \cdot \tr(P^k).$ Additionally, $\tr(P^{k}) \le |\lambda_1(P)|^{k-2} \cdot \tr(P^2) = |\lambda_1(P)|^{k-2} \cdot \norm{P}_F^2$, where $\lambda_1(P)$ is the largest magnitude eigenvalue of $P$. Since $\lambda_1(P)^2 \le \norm{P}_F^2 = \sum_{i=1}^n \lambda_i(P)^2$, this gives
%$$\tr(P^k) \le \norm{P}_F^k \le 2^{k/2} \cdot Ov(P)^{k/2},$$
%where the last inequality follows from Lemma \ref{lem:mainSimple}. This completes the lemma after using the assumption that $Ov(P) \le \gamma \cdot V(P)$.
%\end{proof}
%Theorem \ref{thm:kclique} is also tight up to constants for constant $k$, by the same example as Theorem \ref{thm:tight}.
    
\subsection{Clustering Coefficient}
Theorem \ref{thm:tri} shows that the expected number of triangles generated by an edge independent model is bounded in terms of the model's overlap. Intuitively, we thus expect that graphs generated by the edge independent model will have low global clustering coefficient, which is the fraction of wedges in the graph that are closed into triangles \cite{WattsStrogatz:1998}.

\begin{definition}[Global Clustering Coefficient]\label{def:cc}
For a graph $G$ with $\Delta(G)$ triangles, no self-loops, and node degrees $d_1,d_2,\ldots, d_n$, the global clustering coefficient is given by
\begin{align*}
C(G) = \frac{3 \Delta(G)}{\sum_{i=1}^n d_i(d_i-1)}.
\end{align*}
\end{definition}
%
%
We extend Theorem \ref{thm:tri} to give a bound on $E_{G \sim \mathcal{G}(P)} \left [C(G) \right ]$ in terms of $Ov(P)$. The proof is related, but  more complex due to the $\sum_{i=1}^n d_i(d_i-1)$ in the denominator of $C(G)$.
\begin{theorem}[Bound on Expected Clustering Coefficient]\label{thm:cc}
Consider symmetric $P \in [0,1]^{n \times n}$ with zeros on the diagonal and with $V(P) \ge 2 n$. %  and  $O(P) < \gamma \cdot V(P)$ for some $\gamma < 1$. Then % for $G \sim \mathcal{G}(P)$ with probability at least $1-\exp(-\Theta(n))$, the clustering coefficient is bounded by
\begin{align*}
E_{G \sim \mathcal{G}(P)} \left [C(G) \right ] = O \left (\frac{Ov(P)^{3/2} \cdot n}{V(P)^{1/2}} \right ).
\end{align*}
\end{theorem}
%
\begin{proof}
By Theorem \ref{thm:tri} we have $\E_{G \sim \mathcal{G}(P)} \left [3 \Delta(G) \right ]\le \sqrt{2} \cdot  Ov(P)^{3/2} \cdot V(P)^{3/2}$. We will show that with high probability, $\sum_{i=1}^n d_i (d_i - 1) = \Omega (V(P)^2/n)$, which will give the theorem.
Note that $\E_{G \sim \mathcal{G}(P)} \left [\sum_{i =1}^n d_i \right ] = \E_{G \sim \mathcal{G}(P)}[2|E(G)|] = 2 V(P)$. Thus,
by a Bernstein bound, for large enough $n$ since $V(P) \ge 2n$.
\begin{align*}
\Pr \left [ \left |\sum_{i =1}^n d_i - 2V(P) \right | \ge V(P)/5 \right ] \le 2 \exp\left (- \frac{V(P)^2/50}{V(P)+V(P)/15} \right ) \ll \frac{1}{n^2},
\end{align*}
We can bound 
%
%
%
 %$1-\exp(-\Theta(n))$, $$9/10 \cdot V(P) \le \sum_{i=1}^n d_i \le 11/10 \cdot V(P).$$ We can bound  
 $\sum_{i=1}^{n} d_i^2 \ge \frac{\left (\sum_{i=1}^n d_i\right)^2}{n}$. Thus, with probability $\ge 1-1/n^2$,
\begin{align*}
\sum_{i=1}^n d_i(d_i - 1) \ge \frac{(8/5)^2 \cdot V(P)^2}{n} - \frac{12}{5} V(P) \ge \frac{V(P)^2}{n},
\end{align*}
where in the last step we use that $V(P) \ge 2n$ and so $\frac{12}{5} \cdot V(P) \le \frac{6}{5} \cdot \frac{V(P)^2}{n}$. Combined with our bound on $\E_{G \sim \mathcal{G}(P)} \left [3 \Delta(G) \right ]$, and the fact that $C(G) \le 1$ always, we have
\begin{align*}
E_{G \sim \mathcal{G}(P)} \left [C(G) \right ] = O \left ( \frac{ Ov(P)^{3/2} V(P)^{3/2}}{\frac{V(P)^2}{n}} + \frac{1}{n^2} \right ) = O \left (\frac{Ov(P)^{3/2} \cdot n}{V(P)^{1/2}} \right ).
\end{align*}
\end{proof}

Thus, to have a constant clustering coefficient for a graph with $O(n)$ edges in expectation, we need $Ov(P) = \Omega(1/n^{1/3})$. Note that the requirement of $V(P) \ge 2 n$ is very mild -- it means that the expected average degree is at least $1$.

As with our triangle bound, Theorem \ref{thm:cc} is tight when $\mathcal{G}(P)$ is just an Erd\"{o}s-R\'{e}nyi distribution.
\begin{theorem}[Tightness of Expected Clustering Coefficient Bound]\label{thm:tightCC}
For any $\gamma \in (0,1]$, there exists $P \in [0,1]^{n \times n}$ with zeros on the diagonal, $Ov(P) \le \gamma$ and $\E_{G \sim \mathcal{G}(P)} [C(G)] = \Theta \left (\frac{\gamma^{3/2} \cdot n}{V(P)^{1/2}} \right )$.
 \end{theorem}
 \begin{proof}
Let $P_{ij} = \gamma$ for all $i \neq j$. We have $V(P) = \gamma \cdot {n \choose 2} = \Theta(\gamma n^2)$ and $Ov(P) =  \gamma$. Additionally,  %and $\E_{G \sim \mathcal{G}(P)} [C(G)] = \Theta \left (\frac{\gamma^{3/2} \cdot n}{V(P)^{1/2}} \right )$.
% We have $V(P) = \gamma \cdot n^2$ and $Ov(P) = \gamma^2 \cdot n^2$ which confirms that $Ov(P) \le \gamma \cdot V(P)$. We have 
 $\E[\Delta(G)] = \Theta(\gamma^3 \cdot n^3)$, and, if $n$ is large enough with respect to $\gamma$, with very high probability, $\sum_{i=1}^n d_i(d_i-1) \le \sum_{i=1}^n d_i^2 = O(\gamma^2 n^3)$. This gives:
 \begin{align*}
\E_{G \sim \mathcal{G}(P)} [C(G)] = \Theta (\gamma) = \Theta \left (\frac{\gamma^{3/2} \cdot  n}{\gamma^{1/2} \cdot n} \right ) = \Theta \left ( \frac{\gamma^{3/2} \cdot n}{V(P)^{1/2}} \right ).
 \end{align*}
 \end{proof}
 
% \subsection{Degree Sequence}
% 
% We conclude
% 
%\begin{theorem}[Degree Sequence Overlap Bound]
% Let $z \in \R^{n}$ be a degree sequence, with $Z = \max_i z_i$ and $vol = \norm{z}_1$. If $Z = \sqrt{vol \cdot \gamma}$ for some $\gamma \le 1$, then for any $P$ with $\E_{G \sim \mathcal{G}(P)} d_i = z_i$ for all $i$, we have $OV(P) \ge \gamma \cdot V(P)$.
% \end{theorem}
%\begin{proof}
%Consider the degree sequence $z$ corresponding to a graph which is a union of $1/\gamma$ cliques (including self-loops) each containing m nodes. We have $Z = m$ and $vol = m^2/\gamma$. Thus $Z = \sqrt{vol \cdot \gamma}$ as required.
%
%Now, for $P$ to satisfy  $\E_{G \sim \mathcal{G}(P)} d_i = z_i$ for all $i$, P must have rows/column summing to $m$ for each node in a clique and $0$'s in all other rows/columns. Since there are $m/\gamma$ nodes total in the cliques, we can maximally spread out the mass in a row of $P$ corresponding to a clique node by having value $\gamma$ in each of the $m/\gamma$ locations where we can place non-zero value. This $P$ contains an $m/\gamma \times m/\gamma$ block of all $\gamma$s, with all other entries set to $0$. We thus have $Ov(P) \le  ||P||_F^2 = m^2$. Further, $V(P) = m^2/\gamma$. So $Ov(P) = \gamma \cdot V(P)$. This is the the minimum value achievable sine we have minimized $\norm{p_i}_2^2/\norm{p_i}_1$ for every row $p_i$ of $P$.
%
%\end{proof}
%
% \begin{theorem}[Bounded Overlap for the Chung-Lu Model]\label{thm:chung}
% Let $z \in \R^{n}$ be a degree sequence, with $Z = \max_i z_i$ and $vol = \norm{z}_1$. If $Z \le \sqrt{vol \cdot \gamma}$ for some $\gamma \le 1$ then letting $P =  \frac{1}{vol} \cdot z z^T$, for all $i$, $\E_{G \sim \mathcal{G}(P)} d_i = z_i$. Further, $Ov(P) \le \gamma \cdot V(P).$
% \end{theorem}
% \begin{proof}
% This is immediate since all entries of $P$ are bounded by $Z^2/vol \le \gamma$. Thus, $Ov(P) \le \norm{P}_F^2 \le \gamma \cdot \norm{P}_1 = \gamma \cdot V(P)$, where $\norm{P}_1 = V(P)$ is the entrywise $\ell_1$ norm.
% \end{proof}


\section{Baseline Edge Independent Models}
\label{sec:proposed}

\section{Flow-Packet Hybrid Traffic Classification}
\label{sec:proposed}

We propose FPHTC for a router that needs to conduct class-aware traffic processing. In this section, we provide a detailed description of our scheme. A diagram illustrating the overall framework of FPHTC is given in Fig.~\ref{fig:scheme}.


\subsection{Core Components of FPHTC}
\subsubsection{Router}
The router accepts an incoming stream of packets and processes them according to their service classes using the routing policy. The basic structure and function of such a routing policy are well-defined in prior works on packet classification \cite{Gupta99, Gupta01}. Throughout our work, we focus on how to generate routing
policy rules by training a machine learning model for packet-based traffic classification, where the chosen header fields of each packet are its features, i.e., the inputs into the learning model, and the packet is classified by the learning model to determine its CoS. For example, the chosen header fields may be the source IP address, destination IP address, source port number, and destination port number, among others, and the possible actions may be to route a packet as delay sensitive, delay moderate, or delay tolerant.

\subsubsection{Flow-based Traffic Classifier}
The flow-based traffic classifier resides outside the router, in some powerful equipment that can handle the heavy computation required by sophisticated machine learning techniques. It is a complex and highly accurate machine learning model that can classify a traffic flow in terms of CoS for all of its packets. It is trained using a number of bidirectional TCP flows with a set of flow-level statistical features extracted from the raw dataset.

Various methods are possible to generate the training dataset for the flow-based traffic classifier. In this work, since we are ultimately interested in online classification to handle changing traffic pattern over time, we propose to use a continuously updated recording of the past traffic. Specifically, we use a traffic mirror and a traffic selector, as shown in Fig.~\ref{fig:scheme}, to separate a selected small portion of the incoming traffic flows. The selected flows are then labeled using a Deep Packet Inspection (DPI) module according to their CoS. The true CoS labels obtained by DPI are used to train the flow-based classifier. We note that DPI cannot be used to replace the role of the flow-based classifier for all flows, due to its prohibitive cost and delay for common encrypted traffic. 

The role of the flow-based traffic classifier designer includes data preprocessing, hyperparameter selection, and finally, training the flow-based classifier. Once the flow-based classifier is trained, we use it to infer the CoS labels of all incoming flows captured by the traffic mirror. Then all packets belonging to a flow can be tagged by CoS label of the flow. We note that the CoS labels generated in this way, by a flow-based classifier, are too late to be used in the \textit{routing} of the labeled packets. However, what this achieves is to create a packet-level dataset for \textit{training} the packet-based routing policy as explained below.

\subsubsection{Packet-based Routing Policy Designer}

The packet-based routing policy designer takes labeled packets from the flow-based classifier as input, and it outputs a routing policy for the router. Specifically,  the routing policy designer trains a packet-based classifier using the labeled packets as the training dataset. 

In this work, we use the binary decision tree learning model for the packet-based classifier. In the decision tree, each path from the root to a node is a routing policy rule. Thus, to obtain routing policy rules that can be used in the router, the routing policy designer only needs to train a decision tree on the packet-level dataset. Furthermore, we note that the number of routing policy rules equals the number of leaf nodes in the decision tree. This provides an easy way to control the size of the routing policy, i.e., the routing policy designer can limit the maximum number of leaf nodes while training the decision tree.

\subsection{Construction of Routing Policy}

The construction of the routing policy in FPHTC involves transferring learned knowledge from the flow-based classifier to the routing policy designer. In the machine learning literature, knowledge distillation \cite{Hinton15, Vapnik16} is a technique where a simple student model is trained on the predictions supplied by a highly accurate and complex teacher model. In FPHTC, we train a decision tree at the routing policy designer using the predictions from the flow-based classifier as training targets. In essence, the routing policy designer tries to approximate the performance of the flow-based classifier. 

The flow-based classifier is trained with flow-level statistical features whereas the routing policy designer uses only some features that can be read directly from the packet header. Therefore, it is clear that the learned routing policy will perform worse than the flow-based classifier given the same traffic data for training. However, since there are unlabeled training data available, i.e., those that have not been labeled by DPI, we can label those data samples using our flow-based classifier to substantially enlarge the training dataset for the routing policy designer. Since the decision tree at the routing policy designer is trained on a much larger dataset than that of the flow-based classifier, the performance of the routing policy can be close to that of the flow-based classifier. More importantly, since the routing policy created by FPHTC utilizes information learned from a more powerful flow-based classifier, it can substantially outperform a regular packet-based classifier trained using only the small amount of labels generated by DPI.


\subsection{Routing Policy Update Procedure in Online Setting}

In a practical system, the data pattern of the incoming traffic changes over time, e.g., due to new applications appearing in the network, or changing user behavior. Therefore, we design FPHTC to dynamically update the routing policy over time.

In Fig.~\ref{fig:online}, we illustrate how the modules sequentially function over a continuous stream of traffic. At any given time slot, we collect and label a small portion of the incoming traffic flows using DPI to train the flow-based classifier. Meanwhile, we continue to collect flows to be used in the training of the routing policy. Once the flow-based classifier is trained, we use it to label those collected flows not labeled by DPI. Then, the routing policy designer trains a decision tree to generate the routing policy, which is then updated to the router. 

One important question is whether we should repeat these steps and update the routing policy at each time slot. If the traffic data pattern does not change too frequently, routing policy update at every time slot would be a waste of resources. To re-train the flow-based classifier, the labeling cost using DPI would also be expensive. A cost-effective solution is to update the routing policy only when the traffic pattern has altered significantly. This can be inferred by measuring the performance deterioration at the router. A feedback signal can be generated, for example, based on the increase in packet drop or congestion, to indicate that a routing policy update is necessary. We demonstrate the adaptiveness of FPHTC in the online setting in Section \ref{sec:results}.

\begin{figure}[t]
	\centering
	\includegraphics[width=9cm]{"figures/online".pdf}
	\caption{FPHTC in online setting.}
	\label{fig:online}
\end{figure}

\section{Experimental Results}
\label{sec:exp}
\subsection{Unsupervised Grammar Induction}

\subsubsection{Setup}\label{sec:LM_setup}
\paragraph{Baselines and Evaluation.} 
For comparison, we include six recent strong models for unsupervised parsing with available open source implementations: StructFormer \cite{DBLP:conf/acl/ShenTZBMC20}, Ordered Neurons~\cite{DBLP:conf/iclr/ShenTSC19}, URNNG~\cite{dblp:conf/naacl/kimrykdm19}, DIORA~\cite{dblp:conf/naacl/drozdovvyim19}, C-PCFG~\cite{kim-etal-2019-compound}, and R2D2~\cite{hu-etal-2021-r2d2}. 
To observe the marginal gain from pretraining, we also include Fast-R2D2 without pretraining denoted as Fast-R2D2$_{\rm w/o}$.
Following~\newcite{htut-etal-2018-grammar}, we train all systems on a training set consisting only of raw text, and evaluate and report the results on an annotated test set. 
As an evaluation metric, we adopt sentence-level unlabeled $F_1$ computed using the script from \newcite{kim-etal-2019-compound}.
We compare against the non-binarized gold trees per convention.
The results of Fast-R2D2 are obtained from 3 runs of each model with different random seeds in pre-training.
The best checkpoint for each system is picked based on scores on the validation set. 
Fast-R2D2 is pretrained with span constraints for the word level but without span constraints for the word-piece level.
To support word-piece level evaluation, 
we convert gold trees to word-piece level trees 
by simply breaking each terminal node into a non-terminal node with its word-pieces as terminals, e.g., (NN discrepancy) into (NN (WP disc) (WP \#\#re) (WP \#\#pan) (WP \#\#cy)).

\paragraph{Environment.} EFLOPS~\cite{DBLP:conf/hpca/DongCZYWFZLSPGJ20} is a highly scalable distributed training system designed by Alibaba. With its optimized hardware architecture and co-designed supporting software tools, including ACCL~\cite{DBLP:journals/micro/DongWFCPTLLRGGL21} and KSpeed (the high-speed data-loading service), it could easily be extended to 10K nodes (GPUs) with linear scalability.

\paragraph{Hyperparameters.} The tree encoder of our model uses 4-layer Transformers with 768-dimensional embeddings, 
3,072-dimensional hidden layer representations, and 12 attention heads. 
The top-down parser of our model uses a 4-layer bidirectional LSTM with 128-dimensional embeddings and 256-dimensional hidden layer. The sampling number $K$ is set to be 256.
Training is conducted using Adam optimization with weight decay using a learning rate of $5 \times 10^{-5}$ for the tree encoder and $1 \times 10^{-2}$ for the top-down parser.
The batch size is set to 64 per GPU for $m$=$4$, though we also limit the maximum total length for each batch, such that excess sentences are moved to the next batch. The limit is set to 1,536. It takes about 120 hours for 60 epochs of training with $m$=$4$ on 8 A100 GPUs.

\paragraph{Data.}  For English, to fully leverage the scalability of Fast-R2D2, we pretrain Fast-R2D2 on WikiText103~\cite{DBLP:conf/iclr/MerityX0S17}
and then fine-tune the model on the Penn Treebank (PTB)~\cite{marcus-etal-1993-building}
for 10 epochs with the same objective.
WikiText103 is split at the sentence level, and sentences longer than 200 after tokenization are discarded (about 0.04‰ of the original data). 
The total number of sentences is 4,089,500, and the average sentence length is 26.97.
For Chinese, we use a subset of Chinese Wikipedia (Simplified Characters) for pretraining, specifically the first 10,000,000 sentences shorter than 150 characters and then fine-tune on Chinese Penn Treebank (CTB) 8.0~\cite{ctb8}.
We test our approach on PTB WSJ data with the standard splits (2--21 for training, 22 for validation, 23 for test) and the same preprocessing as in recent work \cite{kim-etal-2019-compound}, where we discard punctuation and lower-case all tokens. 
To explore the universality of the model across languages, we further evaluate using the CTB,
on which we also remove punctuation.
Note that in all settings, the training and fine-tuning is conducted entirely on raw unannotated text.

\subsubsection{Results and Discussion}

\begin{table}
\newcommand{\invzero}{\hphantom{0}}
\begin{center}
\setlength{\tabcolsep}{3.pt}
\resizebox{0.45\textwidth}{!}{
\begin{tabular}{@{}l|l|l|l|l@{}}
                    &  eval & mem. & \multicolumn{1}{c|}{WSJ}  & \multicolumn{1}{c}{CTB}  \\
Model               & gran. & cplx  &  $F_1(\mu)$ & $F_1(\mu)$\\ \hline \hline
Left Branching (W)  & WD & $O(n)$& \invzero 8.15  & 11.28 \\
Right Branching (W) & WD & $O(n)$& 39.62 & 27.53 \\
Random Trees (W)    & WD & $O(n)$ & 17.76 & 20.17 \\
\hline
URNNG (W)           & WD & $O(n^3)$& 45.4$^\dag$ & ~~--- \\
ON-LSTM (W)         & WD & $O(n)$  & 47.7$^\dag$ & 24.73 \\
DIORA (W)           & WD & $O(n^3)$& 51.4 & ~~---  \\
StructFormer (W)    & WD & $O(n^2)$& 54.0$^\ddagger$ & ~~--- \\
C-PCFG (W)          & WD & $O(n^3)$& 55.2$^\dag$ & 49.95 \\ \hline
R2D2 (WP)           & WD & $O(n)$ & 48.11 & 44.85  \\
Fast-R2D2$^*$(W)$_{\rm w/o}$ & WD & $O(n)$ & 48.24 & 45.24 \\
Fast-R2D2$^*$(WP)$_{\rm w/o}$ & WD & $O(n)$ & 48.89 & 45.26 \\
Fast-R2D2$^*$(WP)  & WD & $O(n)$ & \textbf{57.22} & \textbf{53.13} \\
\hline \hline
R2D2 (WP)           & WP & $O(n)$  & 52.28 & 63.94 \\ 
Fast-R2D2(WP)      & WP & $O(n)$ & 50.20 & \textbf{67.79} \\
Fast-R2D2$^*$(WP)  & WP & $O(n)$& \textbf{53.88} & 67.74 \\ \hline
\end{tabular}
}
\end{center}
\caption{Unsupervised parsing results with words (W) or word-pieces (WP) as input. ``eval gran." is short for evaluation granularity.
        Values marked with $^{\dag}$ are taken from \newcite{kim-etal-2019-compound}, while $^{\ddagger}$ denotes values taken from \newcite{DBLP:conf/acl/ShenTZBMC20}.
        The bottom three systems are all pre-trained or trained 
        at the word-piece level \textbf{without} span constraints and are measured against word-piece level golden trees. ${\rm w/o}$ means without pretraining.}
\label{tbl:constituency_parsing}
\end{table}


Table~\ref{tbl:constituency_parsing} shows the results of all systems with words (W) and word-pieces (WP) as input on the WSJ and CTB test sets. 
When we evaluate all systems on word-level golden trees, 
our Fast-R2D2 performs substantially better than R2D2 across both datasets.
We denote as Fast-R2D2 the method of using the parser to guide the pruning and selecting the best tree using the chart table and as Fast-R2D2$^*$ the system that uses the top-down parser for tree induction with subsequent R2D2 encoding.
Interestingly, the results suggest that Fast-R2D2$^*$ outperforms Fast-R2D2, especially on the WSJ test set.
Additionally, pretrained Fast-R2D2$^*$
outperforms the models specifically designed for grammar induction.

\begin{table}[!htb]
\small
\begin{center}
\setlength{\tabcolsep}{3.5pt}
\resizebox{0.48\textwidth}{!}{ %
\begin{tabular}{@{}ll| l l l l l l@{}}
 & Model  & WD & NNP & VP & SBAR\\\hline \hline
\multirow{5}{*}{\rotatebox[origin=c]{90}{WSJ}} & DIORA (WP)  & 94.63 & 77.83 & 17.30 & 22.16\\
& C-PCFG (W)                  & ~~--- & ~~--- & 41.7$^\dag$ & 56.1$^\dag$ \\
& C-PCFG (WP)                  & 87.35 & 66.44 & 23.63 & 40.40 \\
& R2D2 (WP)    & \textbf{99.76} & \textbf{86.76} & 24.74 & 39.81\\
& Fast-R2D2$^*$ (WP) & 97.67 & 83.44 & \textbf{63.80} & \textbf{65.68} \\ \hline \hline
\multirow{3}{*}{\rotatebox[origin=c]{90}{CTB}} & C-PCFG(WP) &89.34 & 46.74 & 39.53 & ~~---\\
 & R2D2 (WP) & 97.16 & 67.19 & 37.90 & ~~---\\
 & Fast-R2D2$^*$ (WP) & \textbf{97.80} & \textbf{68.57} & \textbf{46.59} & ~~---
 \\ \hline \hline
\end{tabular}
}
\end{center}
\caption{Recall of constituents and words. WD means word.  Values with $^{\dag}$ are taken from \newcite{kim-etal-2019-compound}.}
\label{tbl:unsupervised_chunking}
\end{table}

Following \newcite{dblp:conf/naacl/kimrykdm19} and \newcite{drozdov-etal-2020-unsupervised},
we also compute the recall of constituents when evaluating on word-piece level golden trees.
Besides standard constituents, we also compare the recall of word-piece chunks and proper noun chunks. 
Proper noun chunks are extracted by finding adjacent unary nodes with the same parent and tag NNP. 
Table~\ref{tbl:unsupervised_chunking} reports the recall scores for constituents and words on the WSJ and CTB test sets. 
Compared with the R2D2 baseline, 
our Fast-R2D2 performs slightly worse for small semantic units, 
but significantly better over larger semantic units (such as VP and SBAR) on the WSJ test set.
On the CTB test set, our Fast-R2D2 outperforms R2D2 on all constituents. 

From Tables~\ref{tbl:constituency_parsing}~and~\ref{tbl:unsupervised_chunking}, 
we conclude that Fast-R2D2 overall obtains better results than R2D2 on CTB, while faring slightly worse than R2D2 only for small semantic units on WSJ. We conjecture that this difference stems from differences in  tokenization between Chinese and English. 
Chinese is a character-based language without complex morphology, where collocations of characters are consistent with the language, making it easier for the top-down parser to learn them well. 
In contrast, word-pieces for English are built based on statistics, and individual word-pieces are not necessarily natural semantic units. Thus, there may not be sufficient semantic self-consistency, such that it is harder for a top-down parser with a small number of parameters to fit it well.

\subsection{Downstream Tasks}
We next consider the effectiveness of Fast-R2D2 in downstream tasks. This experiment is not intended to advance the state-of-the-art on the GLUE benchmark but rather to assess to what extent our approach performs respectably against the dominant inductive bias as in conventional sequential Transformers.

\subsubsection{Setup}
\paragraph{Data and Baseline.}
We fine-tune pretrained models on several datasets,
including SST-2, CoLA, QQP, and MNLI from the GLUE benchmark~\cite{wang2018glue}.
As sequential Transformers with their dominant inductive bias remain the norm for numerous NLP tasks, 
we mainly compare Fast-R2D2 with \bert~\cite{devlin2018} as a representative pretrained model based on a sequential Transformer. 
We did not include recursive models such as Gumbel-Tree-LSTMs~\cite{DBLP:conf/aaai/ChoiYL18} and CRvNN~\cite{DBLP:conf/icml/ChowdhuryC21} among our baselines, as they are not pretrained models.
In order to compare the two forms of inductive bias fairly and efficiently,
we pretrain \bert models with 4 layers and 12 layers as well as our Fast-R2D2 from scratch on the WikiText103 corpus following Section~\ref{sec:LM_setup}. 
Considering that longer inputs in the pre-training stage are helpful for BERT’s downstream task performance, we use the original corpus that is not split into sentences as inputs.
For simplicity, Fast-R2D2 is fine-tuned without span constraints.
Following the common settings, we add an MLP layer over the root representation of the R2D2 encoder for single-sentence classification. 
For cross-sentence tasks such as QQP and MNLI, we feed the root representations of the two sentences into the pretrained tree encoder of R2D2 as left and right inputs, 
and also add a new task ID as another input term to the R2D2 encoder. 
Then we feed the hidden output of the new task ID into another MLP layer to predict the final label.
We train all systems across the four datasets for 10 epochs 
with a learning rate of $5\times 10^{-5}$, batch size $64$, and maximum input length $200$.
We validate each model in each epoch and report the best results on development sets.

\begin{table}
\begin{center}
\setlength{\tabcolsep}{1.5pt}
\resizebox{0.48\textwidth}{!}{
\begin{tabular}{l|c|r r|r r}
\multirow{4}{*}{Model} & \multirow{4}{*}{Para.} & \multicolumn{2}{c|}{Single sent.} & \multicolumn{2}{c}{Cross sent.} \\
 &  & \begin{tabular}[c]{@{}l@{}}SST-2\\ (Acc.)\end{tabular} & \begin{tabular}[c]{@{}l@{}}CoLA\\ (Mcc.)\end{tabular} & \begin{tabular}[c]{@{}l@{}}QQP\\ (F1)\end{tabular} & \begin{tabular}[c]{@{}l@{}}MNLI\\m/mm\\ (Acc.)\end{tabular}            \\ \hline \hline
\bert (4L)  & 52M & 84.98 & 17.07 & 84.01 & 73.73/74.63 \\
\bert (12L) & 116M & 90.25 & 40.72 & 87.13 & 80.00/80.41 \\ \hline
R2D2        & 52M & 89.33 & 34.79 & 84.27 &  69.35/68.72 \\ \hline
Fast-R2D2$^\dag$& {\multirow{2}{*}{\begin{tabular}[c]{@{}c@{}}\\52M/\\ 10M\end{tabular}}} & 87.50 & 8.67 & 83.97 & 69.53/69.50 \\
Fast-R2D2$^*\dag$& {} & 88.30 & 10.14 & 84.07 & 69.36/69.11 \\
Fast-R2D2  & {} & 90.25 & 38.45 & 84.35 & 69.36/68.80 \\ 
Fast-R2D2$^*$& {} & 90.71 & 40.11 & 84.32 & 69.64/69.57\\
\hline \hline
\end{tabular}
}
\end{center}
\caption{Downstream results. All systems are pretrained from scratch on WikiText103.
        Para.\ describes the number of parameters for each model. Fast-R2D2 contains the R2D2 encoder and top-down parser, two components with 52M and 10M parameters, respectively.
        Mcc.\ stands for Matthew's correlation coefficient.
        Fast-R2D2 with $\dag$ are models fine-tuned without $\mathcal{L}_\mathrm{bilm}$ for an ablation study.
    }\vspace{-10pt}
\label{tbl:classification}
\end{table}
\subsubsection{Results and Discussion}
Table~\ref{tbl:classification} shows the corresponding scores on SST-2, CoLA, QQPl, and MNLI. 
In terms of the parameter size, our Fast-R2D2 model has 52M and 10M parameters for the R2D2 encoder and top-down parser, respectively.
It is clear that 12-layer \bert is significantly better than 4-layer \bert.
As mentioned in Section~\ref{sec:downstream}, Fast-R2D2 has two options to construct the final tree and representation for a given input sentence:
Fast-R2D2$^*$ uses the output tree from the top-down parser, while Fast-R2D2 uses the best tree inferred by the R2D2 encoder.
Similar to the results for unsupervised parsing, Fast-R2D2$^*$ in classification tasks again outperforms Fast-R2D2.
We hypothesize that trees generated by the top-down parser without Gumbel noise are more stable and reasonable.
Fast-R2D2 significantly outperforms 4-layer \bert and achieves competitive results compared to 12-layer \bert in single sentence classification tasks such as SST-2 and CoLA, but still performs significantly worse in the cross-sentence tasks. 
We believe this is an expected result, as there is no cross-attention mechanism in the inductive bias of Fast-R2D2. 
However, the performance of Fast-R2D2 on classification tasks shows that the inductive bias of R2D2 has higher parameter utilization than sequentially applied Transformers.
Importantly, we demonstrate that a Recursive Neural Network variant with an unsupervised parser can achieve comparable results to pretrained sequential Transformers even with fewer parameters and interpretable intermediate results, 
Hence, our Fast-R2D2 framework provides an alternative for NLP tasks.

\subsection{Speed Evaluation}
To assess the time cost, we mainly compare sequential Transformers and Fast-R2D2 in forced encoding on various sequence length ranges. We randomly select 1,000 sentences for each range from WikiText103 and report the average time consumption on a single A100 GPU. \bert is based on the open source Transformers library\footnote{\url{https://github.com/huggingface/transformers}} and R2D2 is based on the official code in \newcite{hu-etal-2021-r2d2}.\footnote{\url{https://github.com/alipay/StructuredLM_RTDT/tree/r2d2}}

\begin{table}% [htb!]
\small
\begin{center}
\setlength{\tabcolsep}{3.pt}
\resizebox{0.45\textwidth}{!}{
\begin{tabular}{l|rrrr}
\multirow{2}{*}{Model} & \multicolumn{4}{c}{Sequence Length Ranges} \\\cline{2-5}
      & \multicolumn{1}{c|}{0--50} & \multicolumn{1}{l|}{50--100} & \multicolumn{1}{l|}{100--200} & 200--500 \\ 
\hline
\bert (12L) & \multicolumn{1}{r|}{1.36}     & \multicolumn{1}{r|}{1.46}       & \multicolumn{1}{r|}{1.62}        & 2.38 \\ \hline
R2D2  & \multicolumn{1}{r|}{38.06}     & \multicolumn{1}{r|}{173.74}       & \multicolumn{1}{r|}{555.95}        &    ---     \\
Fast-R2D2  & \multicolumn{1}{r|}{4.67} & \multicolumn{1}{r|}{14.91} & \multicolumn{1}{r|}{39.73} & 150.26 \\
Fast-R2D2* & \multicolumn{1}{r|}{1.28} & \multicolumn{1}{r|}{2.96}  & \multicolumn{1}{r|}{5.56}  & 10.70 \\ 
\hline \hline
\end{tabular}
}
\end{center}
\caption{Inference time in seconds for various systems to process 1,000 sentences with a batch size of 50.}
\label{tbl:speed_test}
\end{table}

Table~\ref{tbl:speed_test} shows the inference time in seconds for different systems to process 1,000 sentences with a batch size of 50.
Running R2D2 is time-consuming, since the heuristic pruning method involves substantial memory exchanges between GPU and CPU. 
In Fast-R2D2, we alleviate this problem by using model-guided pruning to accelerate the chart table processing,
in conjunction with a code implementation in CUDA, Fast-R2D2 reduces the inference time significantly. 
Fast-R2D2$^{*}$ further improves the inference speed by running forced encoding in parallel over the binary tree generated by the parser, which is about 30--50 times faster than R2D2 in various ranges. 
Although there is still a gap in speed compared to sequential Transformers, Fast-R2D2$^{*}$ is sufficiently fast for most NLP tasks while producing interpretable intermediate representations.


\section{Conclusion}
Our theoretical results prove limitations on the ability of any edge independent graph generative model to produce networks that match the high triangle densities of real-world graphs, while still generating a diverse set of networks, with low model overlap. These results match empirical findings that popular edge independent models indeed systematically underestimate triangle density, clustering coefficient, and related measures. Despite the popularity of edge independent models, many non-independent models, such as graph RNNs \cite{YouYingRen:2018} have been proposed. An interesting future direction would be to study the representative power and limitations of such models, giving general theoretical results that provide a foundation for the study of graph generative models.

\clearpage
\bibliographystyle{plain}
\bibliography{neurips_2021}

\clearpage
\appendix

\section{Exact Embeddings in the CELL Model}

Recently, Rendsburg et al \cite{rendsburgnetgan} propose the CELL graph generator: a major simplification of the NetGAN algorithm for \cite{bojchevski2018netgan}, which gives comparable performance, much faster runtimes, and helps clarify the key components of  the generator. CELL uses a simple low-rank factorization model. Here we prove that, when its rank parameter is $k$, the CELL model can `memorize' any graph with degree bounded by $O(k)$. This allows the model to trivially produce distributions with very high expected triangle densities. However, as our main results show, this inherently requires memorization and high overlap. 

Our result can be viewed as an extension of the results of \cite{ChanpuriyaMuscoSotiropoulos:2020}, which considers a different edge independent model. The proof techniques are very similar.
Interestingly, our result seem to indicate that the good generalization of CELL in link prediction tests may mostly be due to the fact that this model is not fully optimized, to the point of memorizing the input. %It is not due to regularization or bounded expressiveness of the model itself.
%
% which seems to generate random networks with comparable properties to an seed input graph and exhibits good generalization in link prediction tests -- i.e., given a subset of training edges, the algorithm generates a graph which contains the remaining edges of the graph with good probability.

%Here we argue that any  generalization performance  of CELL is based on incomplete optimization. If the model is fully  optimized, it can produce a very low-dimension embedding that generates the true input graph with probability $1$, obviating its usefulness as a graph generator and any generalization properties.

%\todo{Insert more complete description of CELL.}

\noindent \textbf{The CELL Model.} We first describe the CELL model introduced in \cite{rendsburgnetgan}.
\begin{enumerate}
\item Given a graph adjacency matrix $A \in \{0,1\}^{n \times n}$, let
\begin{align}\label{eq:cell}
W^\star = \min_{\substack{W \in \R^{n \times n} \\\rank(W) \le k}}\quad \sum_{i,j=1}^n A_{ij} \log \sigma_{rows}(W)_{ij},
\end{align}
where $\sigma_{rows}(W)$ applies a softmax rowwise to $W$ -- ensuring that each row of $\sigma_{rows}(W)$ sums to $1$. 
\item Let $P^\star = \sigma_{rows}(W^\star)$ and let $\pi \in \R^n$ be the eigenvector satisfying $\pi^T P^\star = \pi^T$.
\item Let $P = \max(diag(\pi) P^*, (diag(\pi) P^\star)^T)$.
\item Generate $G \sim G(P)$.
\end{enumerate}
Note that the last step described above is slightly different than the approach taken in CELL. Rather than use an edge-independent model as in Def. \ref{def:ei}, they form $G$ by sampling edges without replacement, with probability proportional to the entries in $P$. They also insure that at least one edge is sampled adjacent to every node. However, this distinction is minor.

\noindent \textbf{Unconstrained Optimum.}
We first show that, if the rank constraint in \eqref{eq:cell} is removed, then the optimal $W^\star$ has $\sigma_{rows}(W^\star) = P^\star =  D^{-1} A$, where $D$ is the diagonal degree matrix. At this minimum, we can check that $\pi_i = d_i$, the degree of the $i^{th}$ node, and thus $diag(\pi)  = D$ and $P = A$. That is, the model simply outputs the input graph with probability $1$.  
\begin{theorem}[CELL Optimum]\label{thm:piecewise}
The unconstrained CELL objective function \eqref{eq:cell} is minimized when $\sigma_{rows}(W) = D^{-1} A$. At this minimum, the edge independent model $P$ is simply $A$. That is, the model just returns the input graph with probability $1$.
\end{theorem}
\begin{proof}
%\todo{Prove this. I am confident it should hold. Proof will first I think use the fact that $\log$ is convex? This should give that $\sigma_{rows}(W) = P$ is the unconstrained minimizer. Since basically in row $i$ you should want to balance the large values of the softmax to all be $1/d_i$. Proving that at the minimum $A^\dagger = A$ is just working through the steps of generating $A^\dagger$ when $\sigma_{rows}(W) = P$.}
It suffices to consider the $i^{th}$ row of $W$ for each $i \in [n]$, since the objective function of \eqref{eq:cell} breaks down rowwise. Let $w_i,a_i \in \R^n$ be the $i^{th}$ rows of $\sigma_{rows}(W)$ and $A$ respectively. Note that $w_i$ is a probability vector, with $w_i(j) \ge 0$ for all $j$ and $\sum_{j=1}^n w_i(j) = 1$.

We seek to minimize $\sum_{j=1}^n A_{ij} \log [w_i(j)].$ We need to show that this objective is minimized when $w_i = 1/d_i \cdot a_i$ -- i.e., when $w_i$ places mass $1/d_i$ at each nonzero entry in $a_i$ $1/d_i \cdot a_i$ is the $i^{th}$ row of $D^{-1}A$, so applying this argument to all $i$ gives that $\sigma_{rows}(W) = D^{-1} A$ is the overall minimizer. Assume for the sake of contradiction that there is some other minimizer $w^\star \neq 1/d_i \cdot a_i$. Since $\sum_{j=1}^n w^\star(j) = 1$, we must have $w^\star(j) \le 1/d_i$ for some $j$ where $a_i = 1$. In turn, there must be some $j'$ with either (1) $w^\star(j') \ge 0$ and $a_i(j') = 0$ or (2) $w^\star(j') \ge 1/d_i$ and $a_i(j') = 1$. In case (1), clearly moving $w^\star(j')$ mass from $j'$ to $j$ will decrease the objective function. In case (2), due to the concavity of the log function, moving $w^\star(j') - 1/d_i$ mass from $j'$ to $j$ will also decrease the objective function. Thus, $w^\star$ cannot be a minimizer, completing the proof.
%
%We know that $\sigma_{rows}(W)_{ij} \le 1$ for all $j$ and further that $\sum_{j=1}^n \sigma_{rows}(W)_{ij} = 1$. 
\end{proof}

\noindent \textbf{Rank-Constrained Optimum.}
We next show that the unconstrained optimum of $\sigma_{rows}(W) = D^{-1} A$, which leads to CELL memorizing the input graph (Thm. \ref{thm:piecewise}) can be achieved even with the rank constraint of \eqref{eq:cell}, as long as $k \ge 2\Delta+1$, where $\Delta$ is the maximum degree of the input graph. 
%We next show that we can achieve an essentially exact factorization.
\begin{theorem}[CELL Exact Factorization]\label{thm:exact}
If $A$ is an adjacency matrix with maximum degree $\Delta$, there is a rank $2\Delta+1$ matrix $W$ with 
$$\sigma_{rows}(W) = D^{-1}A + E$$
where $\norm{E}_2 \le \epsilon$. Note that  the rank of $W$ does not depend on $\epsilon$, and so we can drive $\epsilon \rightarrow 0$ and find a rank-$2\Delta+1$ $W$ which is arbitrarily close to minimizing \eqref{eq:cell} and thus produces $P$ which is arbitrarily close to $A$.
\end{theorem}
%
%Note that in the above conjecture, the rank of $W$ does not depend on $\epsilon$, and so we can drive $\epsilon$ arbitrarily small and find rank-$2\Delta$ $W$ with is arbitrarily close to minimizing \eqref{eq:cell} and thus produces $A^\dagger$ which is arbitrarily close to $A$. \todo{Maybe include a corollary on this?}
%
%Note that in the NetGAN Without GAN paper, the rank parameter generally tends to be chosen way higher than $2\Delta$. See Section C.8 here: \url{https://proceedings.icml.cc/static/paper_files/icml/2020/4540-Supplemental.pdf}. Thus assuming the Theorem is correct, it would imply that any generalization here is just due to not fully optimizing the objective.
%
\begin{proof}

Let $V \in \R^{n \times 2\Delta+1}$ be the Vandermonde matrix with $V_{t,j} = t^{j-1}$. For any $x \in \R^{2\Delta+1}$, $[Vx](t) = \sum_{j = 1}^{2\Delta+1} x({j}) \cdot t^{j-1}$. That is: $Vx$ is a degree $2\Delta$  polynomial evaluated at the integers $t = 1,\ldots,n$.

Let  $a_i$ be the $i^{th}$ row of $A$. Note that $a_i$ has at most $\Delta$ nonzeros whose positions we denote by $t_1,t_2,\ldots,t_{d_i}$.
To prove the theorem, for each row $a_i$, we will construct a polynomial $Vx_i$ which has the \emph{same positive value} at each $t_1,t_2,\ldots,t_{d_i}$ and is negative all all other integers $t$. Then, we will let $X \in \R^{2\Delta +1 \times n}$ be the matrix with columns $x_{i}$ and $W = (VX)^T $. Note that $\rank(W) \le 2\Delta+1$, and $W$ is equal to a fixed positive value whenever A is one and negative whenever it is zero. If we scale $W$ by a very large number (which does not affect its rank), we will have $\sigma_{rows}(W)$ arbitrarily close to $D^{-1} A$, since the rowwise softmax will place equal probability on each positive entry in row $i$ of $W$ and arbitrarily close to $0$ probability on each negative. So the row will exactly have $d_i$ at the nonzero entries of $a_i$, entries each equal to $1/d_i$.

It remains to exhibit the polynomial need to construct $W$. We start by constructing a polynomial of degree $2\Delta$ that is positive on each nonzero position $t_1,t_2,\ldots,t_{d_i}$ of $a_i$ and negative at all other indices. Later we will modify this polynomial to have the same positive value at each nonzero position of $a_i$. %. %As a warm-up, We first show that we can choose $x_i$ so that $sign(V x_i) = a_i$ where the $sign$ function is applied entrywise, and outputs $0$ for negative numbers, $1$ for non-negative numbers. 
%To do this we equivalently must find a polynomial which is positive at all integers $t$ with $a_i(t) = 1$ and negative at all $t$ with $a_i(t) = 0$.  Say $a_i$ is nonzero at positions $t_1,t_2,\ldots,t_{d_i}$.
 Let $r_{j,L}$ and $r_{j,U}$ be any values with $t_{j} -1 < r_{j,L} < t_j$ and $t_{j} < r_{j,U} < t_{j} +1$. Consider the polynomial with roots at each $r_{j,L}$ and $r_{j,U}$ -- this polynomial has $2 d_i \le 2\Delta$ roots and so degree at most $2\Delta$. It will flip signs just at each  $r_{j,L}$ and $r_{j,U}$, and will in fact have the same sign at  $t_1,t_2,\ldots,t_{d_i}$ (either positive or negative). Simply negativing the coefficients we can ensure that this sign is positive, while it is negative at all other indices, giving the result. % the sign to be positive, we have the result. %This argument gives Theorem 6 of \cite{chanpuriya2020node}.


The polynomial above can be written as $p(t) = \prod_{j=1}^{d_i} (t-r_{j,U}) (t-r_{j,L})$. Choose $r_{j,U} = t_j + \epsilon w_j$ and $r_{j,L} = t_j - \epsilon w_j$, where $\epsilon$ is arbitrarily small and $w_j$ is a weight chosen specifically for $t_j$ which we'll set later. We have for any $k = 1,\cdots, d_i$,
\begin{align*}
\lim_{\epsilon \rightarrow 0} \frac{p(t_k)}{\epsilon^2} &= \lim_{\epsilon \rightarrow 0} \frac{\prod_{j=1}^\Delta (t_k-t_j+\epsilon w_j) (t_k-t_j - \epsilon w_j)}{\epsilon^2}\\
& =  \lim_{\epsilon \rightarrow 0} \frac{- \epsilon^2 w_k^2 \cdot \prod_{j\neq k} (t_k - t_j)^2}{\epsilon^2}\\
& = -w_k^2 \cdot \prod_{j\neq k} (t_k - t_j)^2.
\end{align*}
This, if we set $w_k = \frac{1}{\prod_{j\neq k} (t_k - t_j)}$, in the limit as $\epsilon \rightarrow 0$ we will have $p(t_k)/\epsilon^2 = -1$. If we negate and scale the polynomially appropriately (which doesn't change its degree) we will have $p(t_k)$ arbitrarily close to one for each nonzero index $t_k$, and negative for each zero index. This gives the theorem.
\end{proof}

\end{document}
