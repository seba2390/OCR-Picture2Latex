\documentclass{article}

% if you need to pass options to natbib, use, e.g.:
%     \PassOptionsToPackage{numbers, compress}{natbib}
% before loading neurips_2021

% ready for submission
%\usepackage{neurips_2021}

% to compile a preprint version, e.g., for submission to arXiv, add add the
% [preprint] option:
     \usepackage[nonatbib,preprint]{neurips_2021}

% to compile a camera-ready version, add the [final] option, e.g.:
%     \usepackage[nonatbib,final]{neurips_2021}

% to avoid loading the natbib package, add option nonatbib:
%\usepackage[nonatbib]{neurips_2021}

\usepackage[utf8]{inputenc} % allow utf-8 input
\usepackage[T1]{fontenc}    % use 8-bit T1 fonts
\usepackage{hyperref}       % hyperlinks
\usepackage{url}            % simple URL typesetting
\usepackage{booktabs}       % professional-quality tables
\usepackage{amsfonts}       % blackboard math symbols
\usepackage{nicefrac}       % compact symbols for 1/2, etc.
\usepackage{microtype}      % microtypography
\usepackage{xcolor}         % colors

% Things we're adding
\usepackage{amssymb,amsthm,amsmath}
\usepackage{bm}
\usepackage{float}
\usepackage{graphicx}
\usepackage{algorithm}
%\usepackage{algorithmic}
\usepackage{algpseudocode}
\graphicspath{ {./new_images/} }
\DeclareGraphicsExtensions{.pdf,.jpeg,.png}
\usepackage{multibib}

\newcommand{\algoname}[1]{\textnormal{\textsc{#1}}}
\newcommand{\comment}[1]{\text{\phantom{(#1)}} \tag{#1}}
\newcommand{\spara}[1]{\smallskip\noindent{\bf #1}}
\newif\ifdraft

\newcommand{\todo}[1]{\textcolor{blue}{TODO: #1}}
\newcommand{\Cam}[1]{\textcolor{blue}{Cam: #1}}
\newcommand{\Dan}[1]{\textcolor{cyan}{Dan: #1}}
\newcommand{\Kon}[1]{\textcolor{orange}{Kon: #1}}
\newcommand{\eqdef}{\mathbin{\stackrel{\rm def}{=}}}
\makeatletter
\def\hlinewd#1{%
	\noalign{\ifnum0=`}\fi\hrule \@height #1 \futurelet
	\reserved@a\@xhline}
\makeatother

\newtheorem{theorem}{Theorem}
\newtheorem{conjecture}[theorem]{Conjecture}
\newtheorem{corollary}[theorem]{Corollary}
\newtheorem{lemma}[theorem]{Lemma}
\newtheorem{fact}[theorem]{Fact}
\newtheorem{claim}[theorem]{Claim}
\newtheorem{assumption}[theorem]{Assumption}
\newtheorem{definition}{Definition}
\newtheorem{problem}[definition]{Problem}
\newtheorem{connection}{Connection}
\newtheorem{example}[theorem]{Example}


\newtheorem*{rep@theorem}{\rep@title}
\newcommand{\newreptheorem}[2]{%
	\newenvironment{rep#1}[1]{%
		\def\rep@title{#2 \ref{##1}}%
		\begin{rep@theorem}}%
		{\end{rep@theorem}}}
\makeatother
\newreptheorem{theorem}{Theorem}
\newreptheorem{claim}{Claim}

\newcommand{\R}{\mathbb{R}}
\newcommand{\C}{\mathbb{C}}
\newcommand{\N}{\mathbb{N}}
\newcommand{\HH}{\mathcal{H}}
\newcommand{\bs}[1]{\boldsymbol{#1}}
\newcommand{\bv}[1]{\mathbf{#1}}
\newcommand{\wh}{\widehat}
\newcommand{\norm}[1]{\|#1\|}
\newcommand{\opnorm}[1]{\|#1\|_\mathrm{op}}
\DeclareMathOperator{\supp}{\mathrm{supp}}
\DeclareMathOperator{\poly}{poly}
\DeclareMathOperator{\cp}{cap}
\DeclareMathOperator{\cheb}{cheb}
\DeclareMathOperator{\argmax}{argmax}
\DeclareMathOperator{\sinc}{sinc}
\DeclareMathOperator*{\argmin}{arg\,min}
\newcommand{\E}{\mathbb{E}}

\DeclareMathOperator{\tr}{tr}
\DeclareMathOperator{\rank}{rank}
\DeclareMathOperator{\err}{err}
\DeclareMathOperator{\erf}{erf}
\DeclareMathOperator{\range}{range}
\DeclareMathOperator{\Null}{null}
% / Things we're adding

\title{On the Power of Edge Independent Graph Models}

% The \author macro works with any number of authors. There are two commands
% used to separate the names and addresses of multiple authors: \And and \AND.
%
% Using \And between authors leaves it to LaTeX to determine where to break the
% lines. Using \AND forces a line break at that point. So, if LaTeX puts 3 of 4
% authors names on the first line, and the last on the second line, try using
% \AND instead of \And before the third author name.

\author{
	Sudhanshu Chanpuriya \\ University of Massachusetts Amherst \\ \texttt{schanpuriya@umass.edu }
	\And 
	Cameron Musco\\ University of Massachusetts Amherst\\ \texttt{cmusco@cs.umass.edu}
	\AND
	Konstantinos Sotiropoulos\\ Boston University\\ \texttt{ksotirop@bu.edu} 
	\And
	Charalampos E. Tsourakakis\\ Boston University \& ISI Foundation \\ \texttt{tsourolampis@gmail.com} 
}

  % branched from Michael Forbes' macro document
  % September 2014
  \usepackage{nth}
  \usepackage{intcalc}

  \newcommand{\cSTOC}[1]{\nth{\intcalcSub{#1}{1968}}\ Annual\ ACM\ Symposium\ on\ Theory\ of\ Computing\ (STOC)}
  \newcommand{\cFSTTCS}[1]{\nth{\intcalcSub{#1}{1980}}\ International\ Conference\ on\ Foundations\ of\ Software\ Technology\ and\ Theoretical\ Computer\ Science\ (FSTTCS)}
  \newcommand{\cCCC}[1]{\nth{\intcalcSub{#1}{1985}}\ Annual\ IEEE\ Conference\ on\ Computational\ Complexity\ (CCC)}
  \newcommand{\cFOCS}[1]{\nth{\intcalcSub{#1}{1959}}\ Annual\ IEEE\ Symposium\ on\ Foundations\ of\ Computer\ Science\ (FOCS)}
  \newcommand{\cRANDOM}[1]{\nth{\intcalcSub{#1}{1996}}\ International\ Workshop\ on\ Randomization\ and\ Computation\ (RANDOM)}
  \newcommand{\cISSAC}[1]{#1\ International\ Symposium\ on\ Symbolic\ and\ Algebraic\ Computation\ (ISSAC)}
  \newcommand{\cICALP}[1]{\nth{\intcalcSub{#1}{1973}}\ International\ Colloquium\ on\ Automata,\ Languages and\ Programming\ (ICALP)}
  \newcommand{\cCOLT}[1]{\nth{\intcalcSub{#1}{1987}}\ Annual\ Conference\ on\ Computational\ Learning\ Theory\ (COLT)}
  \newcommand{\cCSR}[1]{\nth{\intcalcSub{#1}{2005}}\ International\ Computer\ Science\ Symposium\ in\ Russia\ (CSR)}
  \newcommand{\cMFCS}[1]{\nth{\intcalcSub{#1}{1975}}\ International\ Symposium\ on\ the\ Mathematical\ Foundations\ of\ Computer\ Science\ (MFCS)}
  \newcommand{\cPODS}[1]{\nth{\intcalcSub{#1}{1981}}\ Symposium\ on\ Principles\ of\ Database\ Systems\ (PODS)}
  \newcommand{\cSODA}[1]{\nth{\intcalcSub{#1}{1989}}\ Annual\ ACM-SIAM\ Symposium\ on\ Discrete\ Algorithms\ (SODA)}
  \newcommand{\cNIPS}[1]{Advances\ in\ Neural\ Information\ Processing\ Systems\ \intcalcSub{#1}{1987} (NeurIPS)}
  \newcommand{\cWALCOM}[1]{\nth{\intcalcSub{#1}{2006}}\ International\ Workshop\ on\ Algorithms\ and\ Computation\ (WALCOM)}
  \newcommand{\cSoCG}[1]{\nth{\intcalcSub{#1}{1984}}\ Annual\ Symposium\ on\ Computational\ Geometry\ (SCG)}
  \newcommand{\cKDD}[1]{\nth{\intcalcSub{#1}{1994}}\ ACM\ SIGKDD\ International\ Conference\ on\ Knowledge\ Discovery\ and\ Data\ Mining\ (KDD)}
  \newcommand{\cICML}[1]{\nth{\intcalcSub{#1}{1983}}\ International\ Conference\ on\ Machine\ Learning\ (ICML)}
  \newcommand{\cAISTATS}[1]{\nth{\intcalcSub{#1}{1997}}\ International\ Conference\ on\ Artificial\ Intelligence\ and\ Statistics\ (AISTATS)}
  \newcommand{\cITCS}[1]{\nth{\intcalcSub{#1}{2009}}\ Conference\ on\ Innovations\ in\ Theoretical\ Computer\ Science\ (ITCS)}
  \newcommand{\cPODC}[1]{{#1}\ ACM\ Symposium\ on\ Principles\ of\ Distributed\ Computing\ (PODC)}
  \newcommand{\cAPPROX}[1]{\nth{\intcalcSub{#1}{1997}}\ International\ Workshop\ on\ Approximation\ Algorithms\ for\  Combinatorial\ Optimization\ Problems\ (APPROX)}
  \newcommand{\cSTACS}[1]{\nth{\intcalcSub{#1}{1983}}\ International\ Symposium\ on\ Theoretical\ Aspects\ of\  Computer\ Science\ (STACS)}
  \newcommand{\cMTNS}[1]{\nth{\intcalcSub{#1}{1991}}\ International\ Symposium\ on\ Mathematical\ Theory\ of\  Networks\ and\ Systems\ (MTNS)}
  \newcommand{\cICM}[1]{International\ Congress\ of\ Mathematicians\ {#1} (ICM)}
  \newcommand{\cWWW}[1]{\nth{\intcalcSub{#1}{1991}}\ International\ World\ Wide\ Web\ Conference\ (WWW)}
  \newcommand{\cICLR}[1]{\nth{\intcalcSub{#1}{2012}}\ International\ Conference\ on\ Learning\ Representations\ (ICLR)}
  \newcommand{\cICCV}[1]{\nth{\intcalcSub{#1}{1994}}\ IEEE\ International\ Conference\ on\ Computer\ Vision\ (ICCV)}
  \newcommand{\cICASSP}[1]{#1\ International\ Conference\ on\ Acoustics,\ Speech,\ and\ Signal\ Processing\ (ICASSP)}
  \newcommand{\cUAI}[1]{\nth{\intcalcSub{#1}{1984}}\ Annual\ Conference\ on\ Uncertainty\ in\ Artificial\ Intelligence\ (UAI)}
    \newcommand{\cAAAI}[1]{\nth{\intcalcSub{#1}{1986}}\ AAAI\ Conference\ on\ Artificial\ Intelligence\ (AAAI)}

  \newcommand{\pSTOC}[1]{Preliminary\ version\ in\ the\ \cSTOC{#1}}
  \newcommand{\pFSTTCS}[1]{Preliminary\ version\ in\ the\ \cFSTTCS{#1}}
  \newcommand{\pCCC}[1]{Preliminary\ version\ in\ the\ \cCCC{#1}}
  \newcommand{\pFOCS}[1]{Preliminary\ version\ in\ the\ \cFOCS{#1}}
  \newcommand{\pRANDOM}[1]{Preliminary\ version\ in\ the\ \cRANDOM{#1}}
  \newcommand{\pISSAC}[1]{Preliminary\ version\ in\ the\ \cISSAC{#1}}
  \newcommand{\pICALP}[1]{Preliminary\ version\ in\ the\ \cICALP{#1}}
  \newcommand{\pCOLT}[1]{Preliminary\ version\ in\ the\ \cCOLT{#1}}
  \newcommand{\pCSR}[1]{Preliminary\ version\ in\ the\ \cCSR{#1}}
  \newcommand{\pMFCS}[1]{Preliminary\ version\ in\ the\ \cMFCS{#1}}
  \newcommand{\pPODS}[1]{Preliminary\ version\ in\ the\ \cPODS{#1}}
  \newcommand{\pSODA}[1]{Preliminary\ version\ in\ the\ \cSODA{#1}}
  \newcommand{\pNIPS}[1]{Preliminary\ version\ in\ \cNIPS{#1}}
  \newcommand{\pWALCOM}[1]{Preliminary\ version\ in\ the\ \cWALCOM{#1}}
  \newcommand{\pSoCG}[1]{Preliminary\ version\ in\ the\ \cSoCG{#1}}
  \newcommand{\pKDD}[1]{Preliminary\ version\ in\ the\ \cKDD{#1}}
  \newcommand{\pICML}[1]{Preliminary\ version\ in\ the\ \cICML{#1}}
  \newcommand{\pAISTATS}[1]{Preliminary\ version\ in\ the\ \cAISTATS{#1}}
  \newcommand{\pITCS}[1]{Preliminary\ version\ in\ the\ \cITCS{#1}}
  \newcommand{\pPODC}[1]{Preliminary\ version\ in\ the\ \cPODC{#1}}
  \newcommand{\pAPPROX}[1]{Preliminary\ version\ in\ the\ \cAPPROX{#1}}
  \newcommand{\pSTACS}[1]{Preliminary\ version\ in\ the\ \cSTACS{#1}}
  \newcommand{\pMTNS}[1]{Preliminary\ version\ in\ the\ \cMTNS{#1}}
  \newcommand{\pICM}[1]{Preliminary\ version\ in\ the\ \cICM{#1}}
  \newcommand{\pWWW}[1]{Preliminary\ version\ in\ the\ \cWWW{#1}}
  \newcommand{\pICLR}[1]{Preliminary\ version\ in\ the\ \cICLR{#1}}
  \newcommand{\pICCV}[1]{Preliminary\ version\ in\ the\ \cICCV{#1}}
  \newcommand{\pICASSP}[1]{Preliminary\ version\ in\ the\ \cICASSP{#1}}
  \newcommand{\pUAI}[1]{Preliminary\ version\ in\ the\ \cUAI{#1}, #1}
    \newcommand{\pAAAI}[1]{Preliminary\ version\ in\ the\ \cAAAI{#1}, #1}



  \newcommand{\STOC}[1]{Proceedings\ of\ the\ \cSTOC{#1}}
  \newcommand{\FSTTCS}[1]{Proceedings\ of\ the\ \cFSTTCS{#1}}
  \newcommand{\CCC}[1]{Proceedings\ of\ the\ \cCCC{#1}}
  \newcommand{\FOCS}[1]{Proceedings\ of\ the\ \cFOCS{#1}}
  \newcommand{\RANDOM}[1]{Proceedings\ of\ the\ \cRANDOM{#1}}
  \newcommand{\ISSAC}[1]{Proceedings\ of\ the\ \cISSAC{#1}}
  \newcommand{\ICALP}[1]{Proceedings\ of\ the\ \cICALP{#1}}
  \newcommand{\COLT}[1]{Proceedings\ of\ the\ \cCOLT{#1}}
  \newcommand{\CSR}[1]{Proceedings\ of\ the\ \cCSR{#1}}
  \newcommand{\MFCS}[1]{Proceedings\ of\ the\ \cMFCS{#1}}
  \newcommand{\PODS}[1]{Proceedings\ of\ the\ \cPODS{#1}}
  \newcommand{\SODA}[1]{Proceedings\ of\ the\ \cSODA{#1}}
  \newcommand{\NIPS}[1]{\cNIPS{#1}}
  \newcommand{\WALCOM}[1]{Proceedings\ of\ the\ \cWALCOM{#1}}
  \newcommand{\SoCG}[1]{Proceedings\ of\ the\ \cSoCG{#1}}
  \newcommand{\KDD}[1]{Proceedings\ of\ the\ \cKDD{#1}}
  \newcommand{\ICML}[1]{Proceedings\ of\ the\ \cICML{#1}}
  \newcommand{\AISTATS}[1]{Proceedings\ of\ the\ \cAISTATS{#1}}
  \newcommand{\ITCS}[1]{Proceedings\ of\ the\ \cITCS{#1}}
  \newcommand{\PODC}[1]{Proceedings\ of\ the\ \cPODC{#1}}
  \newcommand{\APPROX}[1]{Proceedings\ of\ the\ \cAPPROX{#1}}
  \newcommand{\STACS}[1]{Proceedings\ of\ the\ \cSTACS{#1}}
  \newcommand{\MTNS}[1]{Proceedings\ of\ the\ \cMTNS{#1}}
  \newcommand{\ICM}[1]{Proceedings\ of\ the\ \cICM{#1}}
  \newcommand{\WWW}[1]{Proceedings\ of\ the\ \cWWW{#1}}
  \newcommand{\ICLR}[1]{Proceedings\ of\ the\ \cICLR{#1}}
  \newcommand{\ICCV}[1]{Proceedings\ of\ the\ \cICCV{#1}}
  \newcommand{\ICASSP}[1]{Proceedings\ of\ the\ \cICASSP{#1}}
  \newcommand{\UAI}[1]{Proceedings\ of\ the\ \cUAI{#1}}
   \newcommand{\AAAI}[1]{Proceedings\ of\ the\ \cAAAI{#1}}


  \newcommand{\arXiv}[1]{\href{http://arxiv.org/abs/#1}{arXiv:#1}}
  \newcommand{\farXiv}[1]{Full\ version\ at\ \arXiv{#1}}
  \newcommand{\parXiv}[1]{Preliminary\ version\ at\ \arXiv{#1}}
  \newcommand{\CoRR}{Computing\ Research\ Repository\ (CoRR)}

  \newcommand{\cECCC}[2]{\href{http://eccc.hpi-web.de/report/20#1/#2/}{Electronic\ Colloquium\ on\ Computational\ Complexity\ (ECCC),\ Technical\ Report\ TR#1-#2}}
  \newcommand{\ECCC}{Electronic\ Colloquium\ on\ Computational\ Complexity\ (ECCC)}
  \newcommand{\fECCC}[2]{Full\ version\ in\ the\ \cECCC{#1}{#2}}
  \newcommand{\pECCC}[2]{Preliminary\ version\ in\ the\ \cECCC{#1}{#2}}
\begin{document}

\maketitle

  In this paper, we explore the connection between secret key agreement and secure omniscience within the setting of the multiterminal source model with a wiretapper who has side information. While the secret key agreement problem considers the generation of a maximum-rate secret key through public discussion, the secure omniscience problem is concerned with communication protocols for omniscience that minimize the rate of information leakage to the wiretapper. The starting point of our work is a lower bound on the minimum leakage rate for omniscience, $\rl$, in terms of the wiretap secret key capacity, $\wskc$. Our interest is in identifying broad classes of sources for which this lower bound is met with equality, in which case we say that there is a duality between secure omniscience and secret key agreement. We show that this duality holds in the case of certain finite linear source (FLS) models, such as two-terminal FLS models and pairwise independent network models on trees with a linear wiretapper. Duality also holds for any FLS model in which $\wskc$ is achieved by a perfect linear secret key agreement scheme. We conjecture that the duality in fact holds unconditionally for any FLS model. On the negative side, we give an example of a (non-FLS) source model for which duality does not hold if we limit ourselves to communication-for-omniscience protocols with at most two (interactive) communications.  We also address the secure function computation problem and explore the connection between the minimum leakage rate for computing a function and the wiretap secret key capacity.
  
%   Finally, we demonstrate the usefulness of our lower bound on $\rl$ by using it to derive equivalent conditions for the positivity of $\wskc$ in the multiterminal model. This extends a recent result of Gohari, G\"{u}nl\"{u} and Kramer (2020) obtained for the two-user setting.
  
   
%   In this paper, we study the problem of secret key generation through an omniscience achieving communication that minimizes the 
%   leakage rate to a wiretapper who has side information in the setting of multiterminal source model.  We explore this problem by deriving a lower bound on the wiretap secret key capacity $\wskc$ in terms of the minimum leakage rate for omniscience, $\rl$. 
%   %The former quantity is defined to be the maximum secret key rate achievable, and the latter one is defined as the minimum possible leakage rate about the source through an omniscience scheme to a wiretapper. 
%   The main focus of our work is the characterization of the sources for which the lower bound holds with equality \textemdash it is referred to as a duality between secure omniscience and wiretap secret key agreement. For general source models, we show that duality need not hold if we limit to the communication protocols with at most two (interactive) communications. In the case when there is no restriction on the number of communications, whether the duality holds or not is still unknown. However, we resolve this question affirmatively for two-user finite linear sources (FLS) and pairwise independent networks (PIN) defined on trees, a subclass of FLS. Moreover, for these sources, we give a single-letter expression for $\wskc$. Furthermore, in the direction of proving the conjecture that duality holds for all FLS, we show that if $\wskc$ is achieved by a \emph{perfect} secret key agreement scheme for FLS then the duality must hold. All these results mount up the evidence in favor of the conjecture on FLS. Moreover, we demonstrate the usefulness of our lower bound on $\wskc$ in terms of $\rl$ by deriving some equivalent conditions on the positivity of secret key capacity for multiterminal source model. Our result indeed extends the work of Gohari, G\"{u}nl\"{u} and Kramer in two-user case.

%\begin{abstract}
%We study the limits of \emph{edge independent random graph models}, in which  each edge is added to the graph independently with some probability. Such models include the Erd\"{o}s-R\'{e}nyi and stochastic block models, as well as many  modern neural-network-based generative models, such as NetGAN, variational graph autoencoders, and CELL. %We explore the limits of edge independent models. 
%We show that subject to a \emph{bounded overlap} condition, which ensures that the model does not simply memorize a single graph, edge independent models are inherently limited in their ability to generate graphs with high triangle and other subgraph densities. Notably, such high densities are known to appear in real-world social networks and other connection graphs. We complement our negative results with a simple baseline that balances overlap and accuracy, performing comparably to more complex generative models in reconstructing many graph statistics. 
%\end{abstract}

\section{Introduction}
\label{sec:intro}

Our work centers on \emph{edge independent graph models}, in which each edge $(i,j)$ is added to the graph independently with some probability $P_{ij} \in [0,1]$. Formally,
\begin{definition}[Edge Independent Graph Model]\label{def:ei}
For any symmetric matrix $P \in [0,1]^{n \times n}$ let $\mathcal{G}(P)$ be the distribution over undirected unweighted graphs where $G \sim  \mathcal{G}(P)$ contains edge $(i,j)$ independently, with probability $P_{ij}$. I.e., $p(G) = \prod_{(i,j) \in E(G)} P_{ij} \cdot \prod_{(i,j) \notin E(G)} (1-P_{ij})$.
\end{definition}

Edge independent models encompass many  classic random graph models. This includes the Erd\"{o}s-R\'{e}nyi  model, where for all $i \neq j$, $P_{ij} = p$ for  some fixed $p \in [0,1]$ \cite{ErdosRenyi:1960}. It also includes the stochastic block model where $P_{ij} = p$ if two nodes are in the same community and $P_{ij} = q$ if two nodes are in different communities for some fixed $p,q \in [0,1]$ with $q < p$ \cite{SnijdersNowicki:1997}. Other examples include e.g., the Chung-Lu configuration model \cite{ChungLu:2002}, stochastic Kronecker graphs \cite{LeskovecChakrabartiKleinberg:2010}.

Recently, significant attention has focused on \emph{graph generative models}, which seek to learn a distribution over graphs that share similar properties to a given training graph, or set of graphs. Many algorithms parameterize this distribution as an edge independent model or closely related distribution. % based on a probability matrix $P \in [0,1]^{n \times n}$ learned from the input. 
E.g., NetGAN and the closely related CELL model both produce $P \in [0,1]^{n \times n}$ and then sample edges independently without replacement with probabilities proportional to its entries, ensuring that at least one edge is sampled adjacent to each node \cite{bojchevski2018netgan,rendsburgnetgan}.
%\begin{itemize}
%\item NetGAN Without GAN (CELL) \cite{rendsburgnetgan} -- explicitly uses an edge independent model except avoids self loops and isolated nodes.
%\item NetGAN \cite{bojchevski2018netgan} -- create a symmetric score matrix, and sample without replacement with probabilities proportional to the entries. Sample at least one edge from each node, using the row distribution for that node.
Variational Graph Autoencoders (VGAE), GraphVAE, Graphite, and MolGAN are also all based on edge independent models \cite{KipfWelling:2016,SimonovskyKomodakis:2018,De-CaoKipf:2018,GroverZweigErmon:2019}. 
%\item 
%MolGAN \cite{De-CaoKipf:2018}, which focuses on modeling molecular graphs, is based on a closely related model where there are multiple node and edge types. The generated graph can be thought of as a union of edge independent graphs, one for each edge type. 
% -- For expected adjacency tensor $n \times n \times p$ where $p$ is the number of bond types and molecule matrix $n \times k$ where $k$ is the number of molecule types, and then sample from these using 'categorical sampling'. I'm not sure exactly what this means... but might work for us. 
%\todo{Look in code to see exactly what they are doing.}
%\item Graphite \cite{GroverZweigErmon:2019} -- directly uses an edge independent model.
%\item Variational Graph Autoencoders \cite{KipfWelling:2016} -- seem to use edge independent model
%\item GraphVAE \cite{SimonovskyKomodakis:2018} -- also using the variational autoencoder approach, and I think edge independent although less clear? It might be that they just take max entries in a continous adjacency matrix, which is randomized due to a random input to the network.
%\end{itemize}

%Standard edge independent models:
%\begin{itemize}
%\item Erdos R\'{e}nyi graphs
%\item Stochastic block model
%\item Chung-Lu configuration model
%\item \url{https://services.math.duke.edu/~rtd/math777/CL_PNAS.pdf}
%\item Kronecker graphs?
%\end{itemize}

Given their popularity in both classical and modern graph generative models, it is natural to ask:
\begin{quote} \emph{How suited are edge independent models to modeling real-world networks. Are they able to capture features such as power-law degree distributions, small-world properties, and high clustering coefficients (triangle densities)? }
\end{quote}

\subsection{Impossibility Results for Edge Independent Models}

In this work we focus on the ability of edge independent models to generate graphs with high triangle, or other small subgraph densities. High triangle density (equivalently, a high clustering coefficient) is a well-known hallmark of real-work networks \cite{WattsStrogatz:1998,SalaCaoWilson:2010,DurakPinarKolda:2012} and has been the focus of recent work exploring the power and limitations of edge-independent graph models \cite{SeshadhriSharmaStolman:2020,ChanpuriyaMuscoSotiropoulos:2020}.

It is clear that edge independent models can generate triangle dense graphs. In particular, $P \in [0,1]^{n \times n}$ in Def. \ref{def:ei} can be set to the binary adjacency matrix of any undirected graph, and $\mathcal{G}(P)$ will generate that graph with probability $1$, no matter how triangle dense it is. However, this would not be a particularly interesting generative model -- ideally $\mathcal{G}(P)$ should generate a wide range of graphs. To capture this intuitive notion, we define the \emph{overlap} of an edge-independent model, which is closely related to the overlap stopping criterion for training used in training graph generative models \cite{bojchevski2018netgan,rendsburgnetgan}.
\begin{definition}[Expected Overlap]\label{def:ov} For symmetric $P \in [0,1]^{n \times n}$ let $V(P) \eqdef \E_{G \sim \mathcal{G}(P)} |E(G)|$ and
% the expected overlap of two graphs drawn independently from $\mathcal{G}(P)$ is:
\begin{align*}
Ov(P) \eqdef \frac{\E_{G_1,G_2 \sim \mathcal{G}(P)} |E(G_1) \cap E(G_2)|}{V(P)}.
\end{align*}
\end{definition}
That is, for any $P \in [0,1]^{n \times n}$, $Ov(P) \in [0,1]$ is the ratio of the expected number of edges shared by two graphs drawn independently from $\mathcal{G}(P)$ to the expected number of edges in a graph drawn from $\mathcal{G}(P)$. In one extreme, when $P$ is a binary adjacency matrix, $Ov(P) = 1$, and our generative model has simply memorized a single graph. In the other, if $P_{ij} = p$ for all $i \neq j$ (i.e., $\mathcal{G}(P)$ is Erd\"{o}s-R\'{e}nyi), $Ov(P) = p$. This is the minimum possible overlap when $V(P) = p \cdot {n \choose 2}$.

Our main result is that for any edge independent model with bounded overlap, $G \sim \mathcal{G}(P)$ cannot have too many triangles in expectation. In particular:
\begin{theorem}[Main Result -- Expected Triangles]\label{thm:tri} For a graph $G$, let $\Delta(G)$ denote the number of triangles in $G$. Consider symmetric $P \in [0,1]^{n \times n}$.
\begin{align*}
\E_{G \sim \mathcal{G}(P)} \left [\Delta(G) \right ]\le \frac{\sqrt{2}}{3} \cdot Ov(P)^{3/2} \cdot V(P)^{3/2}.
\end{align*}
\end{theorem}
As an example, consider the setting where the distribution generates sparse graphs, with $V(P) = \Theta(n)$. Theorem \ref{thm:tri} shows that whenever  $Ov(P) = o(1/n^{1/3})$, $\E_{G \sim \mathcal{G}(P)} \Delta(G) = o(n)$ -- i.e. the graph is very triangle sparse with the number of triangles sublinear in the number of nodes. %A distribution with higher triangle density thus requires $Ov(P) = \Omega(1/n^{1/3})$. 
This verifies that  an Erd\"{o}s-R\'{e}nyi graph cannot achieve simultaneously linear number of edges  (i.e., $Ov(P) = O(1/n)$ ) and super-linear number of triangles (i.e., $Ov(P) = \Omega(1/n^{1/3})$) under our proposed lens of viewing generative models. 

%On the otherhand, if the distribution simply memorizes a single sparse graph, and has $Ov(P) =1$, then the theorem allows $\E_{G \sim \mathcal{G}(P)} \left [\Delta(G) \right ]$ to be as large as $\Theta(n^{3/2})$. This is indeed matched, when the model simply memorizes the clique on $n^{1/2}$ nodes.

We extend Theorem \ref{thm:tri} to give similar bounds for the density of squares and other $k$-cycles (Thm. \ref{thm:k}), as well as for the global clustering coefficient (Thm. \ref{thm:cc}). In all cases we show that our bounds are tight -- e.g., in the triangle case, %for any $\gamma \in (0,1]$,
 there is indeed an edge independent model with %with $Ov(P) = \gamma$ and 
 $\E_{G \sim \mathcal{G}(P)} \left [\Delta(G) \right ] = \Theta \left (Ov(P)^{3/2} \cdot V(P)^{3/2} \right )$, matching the lower bound in Theorem \ref{thm:tri}. 
 
 %Our proofs are extremely simple -- in short, $Ov(P)$ is closely related to the Frobenius norm $\norm{P}_F^2$. 

%A number of recent papers have focused on this question \cite{SeshadhriSharmaStolman:2020,ChanpuriyaMuscoSotiropoulos:2020}

\subsection{Empirical Findings}

Our theoretical results help explain why, despite performing well in a variety of other metrics, edge independent graph generative models have been reported to generate graphs with many fewer triangles and squares on average than the real-world graphs that they are trained on. Rendsburg et al. \cite{rendsburgnetgan} test a suite of these models, including their own CELL model and the related NetGAN model \cite{bojchevski2018netgan}. Of all these models, when trained on the \textsc{Cora-ML} graph with 2,802 triangles and 14,268 squares, none is able to generate graphs with more than 1,461 triangles and 6,880 squares on average. Similar gaps are observed for a number of other graphs.
 Rendsburg et al. also report that the triangle count increases as their notion of overlap (closely related to Def. \ref{def:ov}) increases. Theorem \ref{thm:tri} demonstrates that this underestimation of triangle count, and its connection to overlap is \emph{inherent to all edge independent models, no matter how refined a method used to learn the underlying probability matrix $P$}. 
 
 While our theoretical results bound the performance of any  edge independent model, there may still be variation in how specific models trade-off overlap and realistic graph generation. 
 To better understand this trade-off, we introduce two simple models with easily tunable overlap as baselines. One is based on reproducing the degree sequence of the original graph; the other, which is even simpler, is based on reproducing the volume.
 In both  models, $P$ is a  weighted average of the input graph adjacency matrix and a probability matrix of minimal complexity which matches either the input degrees or the volume. In the latter case, to match just the volume, we simply use an Erd\"{o}s-R\'{e}nyi graph. In the former case, to match the degree sequence, we introduce our own model, the \emph{odds product model}; this model is similar to the Chung-Lu configuration model \cite{ChungLu:2002}, but, unlike Chung-Lu, is able to match degree sequences of real-world graphs with high maximum degree.
We find that these simple baselines are often competitive with more complex models like CELL in terms of matching key graph statistics, like triangle count and clustering coefficient, at similar levels of overlap. %\Cam{What should the takeway from this finding be? Can we add a sentence or two?}

%\todo{Collect results here showing which features they tend to  capture well and which they tend not too. }

\subsection{Related Work}\label{sec:rel}

\noindent\textbf{Existing impossibility results.} Our work is inspired by that of Seshadhri et al. \cite{SeshadhriSharmaStolman:2020}, which also proves limitations on the ability of edge independent models to represent triangle dense graphs. They show that if $P = \max(0,\min(1,XX^T))$ where $X \in \R^{n \times k}$ for $k \ll n$ and the max and min are applied entrywise, then $G \sim \mathcal{G}(P)$ cannot have many triangles adjacent to low-degree nodes in expectation. This setting arises commonly when $P$ is generated using low-dimensional node embeddings -- represented by the rows of $X$. Chanpuriya et al. \cite{ChanpuriyaMuscoSotiropoulos:2020}, show that in a slightly more general model, where $P = \max(0,\min(1,XY^T))$, this lower bound no longer holds -- $X,Y \in \R^{n \times k}$ can be chosen so that $P$ is the binary adjacency matrix of any graph with maximum degree upper bounded by $O(k)$ -- no matter how triangle dense that graph is. Thus, even such low-rank edge independent models can represent triangle dense graphs -- by memorizing a single one. In the appendix, we prove a similar result when $P$ is generated from the CELL model of \cite{rendsburgnetgan}, which simplifies NetGAN \cite{bojchevski2018netgan}.

%However, this comes at the cost of high overlap -- if $P$ is a binary adjacency matrix, $Ov(P) = 1$. 
Our results show that this trade-off between the ability to capture triangle density and memorization is inherent -- even without any low-rank constraint, edge independent models with low overlap simply cannot represent graphs with high triangle or other small subgraph density.

It is well understood that specific edge independent models, e.g., Erd\"{o}s-R\'{e}nyi graphs, the Chung-Lu model, and stochastic Kronecker graphs, do not capture many properties of real-world networks, including high triangle density \cite{WattsStrogatz:1998,PinarSeshadhriKolda:2012}. Our results can be viewed as a generalization of these observations, to all edge independent models with low overlap. Despite the limitations of classic models, edge independent models are still very prevalent in today's literature on graph generative models. Our more general results make clear the limitations of this approach.

\noindent\textbf{Non-independent models.} While edge independent models are very prevalent in the literature, many important models do not fit into this framework. Classic models include the Barab\'{a}si–Albert and other preferential attachment models \cite{BarabasiAlbert:1999}, Watts–Strogatz  small-world graphs \cite{WattsStrogatz:1998}, and random geometric graphs \cite{DallChristensen:2002}. Many of these models were introduced directly in response to shortcomings of classic edge independent models, including their  inability to produce high triangle densities

More recent graph generative models include 
GraphRNN \cite{YouYingRen:2018} and a number of other works \cite{LiVinyalsDyer:2018,LiaoLiSong:2019}.
%-- generates one node at a time, along with all its connections. In simplified models, edges from node are added independently. But probabilities may depend on previous connections. thus this doesn't seem to fall under the edge independent model. You could easily generate triangle dense graphs with this set up.
%\item \cite{LiVinyalsDyer:2018} -- also generates things as a non-independent sequence.
%\end{itemize} 
Our impossibility results do not apply to such models, and in fact suggest that perhaps they may be preferable to edge independent models, if a distribution over graphs with high triangle density is desired. A very interesting direction for future work would be to prove limitations on broad classes of non-independent models, and perhaps to understand exactly what type of correlation amongst edges is needed to generate graphs with both low overlap \footnote{We note that for non-edge independent models, the measure of overlap as defined earlier should be adapted to take into account the order (permutation) of the vertices in the final graph. In particular, the overlap in this case should be the maximum value of it over any permutation of the vertex set.}and hallmark features of real-world networks.


\section{Impossibility Results for Edge Independent Models}
\label{sec:impossibility}

We now prove our main results on the limitations of edge independent models with bounded overlap. % in generating graphs with high triangle and other $k$-cycle densities. 
We start with a simple lemma that will be central in all our proofs.

%\begin{definition}[Expected Volume] For $P \in [0,1]^{n \times n}$ the expected volume of a graph drawn from $\mathcal{G}(P)$ is:
%\begin{align*}
%V(P) = \E_{G \sim \mathcal{G}(P)}[2\cdot |E(G)|]
%\end{align*}
%\end{definition}

\begin{lemma}\label{lem:mainSimple} For any symmetric $P \in [0,1]^{n \times n}$, $\frac{\norm{P}_F^2}{2} \le Ov(P) \cdot V(P) \le \norm{P}_F^2.$
\end{lemma}
\begin{proof}
Let $I[(i,j) \in G]$ be the $0,1$ indicator random variable that an edge $(i,j)$ appears in the graph $G$.  $Ov(P) \cdot V(P) = \E_{G_1,G_2 \sim \mathcal{G}(P)} |E(G_1) \cap E(G_2)|$. 
By linearity of expectation and the independence of $G_1$ and $G_2$ we have,
$$Ov(P) \cdot V(P) = \E_{G_1,G_2 \sim \mathcal{G}(P)} \sum_{i \le j} I[(i,j) \in G_1] \cdot I[(i,j) \in G_2] = \sum_{i \le j} P_{ij}^2.$$
The bound follows since $P$ is symmetric. Note that the lower bound $\frac{\norm{P}_F^2}{2} \le Ov(P) \cdot V(P)$ is an equality if $P$ is $0$ on the diagonal -- i.e., there is no probability of self loops.
\end{proof}

%\begin{definition}[Expected Degree and Triangle Density] For $P \in [0,1]^{n \times n}$ the expected degree 
% density of a graph drawn from $\mathcal{G}(P)$ is:
%\begin{align*}
%T(P) = \E_{G \sim \mathcal{G}(P)} \sum_{.
%\end{align*}
%\end{definition}

\subsection{Triangles}

Lemma \ref{lem:mainSimple} connects $Ov(P) \cdot V(P)$ to $\norm{P}_F^2$ and in turn the eigenvalue spectrum of $P$ since $\norm{P}_F^2 = \sum_{i=1}^n \lambda_i(P)^2$, where $\lambda_1(P),\ldots, \lambda_n(P) \in \R$ are the eigenvalues of $P$. The expected number of triangles in $G \sim \mathcal{G}(P)$ can be written in terms of this spectrum as well, allowing us to relate overlap to this expected triangle count, and prove our main theorem (Theorem \ref{thm:tri}), restated below.

\begin{reptheorem}{thm:tri} For a graph $G$, let $\Delta(G)$ denote the number of triangles in $G$. Consider symmetric $P \in [0,1]^{n \times n}$. 
\begin{align*}
\E_{G \sim \mathcal{G}(P)} \left [\Delta(G) \right ]\le \frac{\sqrt{2}}{3} \cdot Ov(P)^{3/2} \cdot V(P)^{3/2}.
\end{align*}
\end{reptheorem}
\begin{proof}
By linearity of expectation,
\begin{align}
\E_{G \sim \mathcal{G}(P)} \left [\Delta(G) \right ] &= \frac{1}{6} \sum_{i=1}^n \sum_{j=1}^n \sum_{k=1}^n \Pr \left [(i,j) \in E(G) \cap (j,k) \in E(G) \cap (k,i) \in E(G) \right]\nonumber \\
&= \frac{1}{6} \sum_{i=1}^n \sum_{j=1}^n \sum_{k=1}^n P_{ij} P_{jk} P_{ki} = \frac{1}{6} \tr(P^3) = \frac{1}{6} \sum_{i=1}^n \lambda_i(P)^3.\label{eq:sixth}
\end{align}
Letting $\lambda_1(P)$ denote the largest magnitude eigenvalue of $P$, we can in turn bound
$$\tr(P^3) \le |\lambda_1(P)| \cdot \sum_{i=1}^n \lambda_i(P)^2 = |\lambda_1(P)| \cdot \norm{P}_F^2.$$ Since $|\lambda_1(P)| \le \norm{P}_F$, this gives via Lemma \ref{lem:mainSimple}
$$\tr(P^3) \le \norm{P}_F^3 \le 2 \sqrt{2} \cdot Ov(P)^{3/2} \cdot V(P)^{3/2}.$$
%where the last inequality follows from Lemma \ref{lem:mainSimple}. 
Combining this bound with \eqref{eq:sixth} completes the theorem.
\end{proof}

%\Cam{Switch to ER example but still mention clique for intuition.}
The bound of Theorem \ref{thm:tri} is tight up to constants, for any possible value of $Ov(P)$. The tight example is when $P$ is simply an Erd\"{o}s-R\'{e}nyi graph.

% is when $P$ consists of a clique on $\Theta(Ov(P) \cdot \sqrt{n})$ nodes, unioned with a sparse Erd\"{o}s-R\'{e}nyi graph.
\begin{theorem}[Tightness of Expected Triangle Bound]\label{thm:tight}
For any $\gamma \in (0,1]$, there exists a symmetric $P \in [0,1]^{n \times n}$ with $Ov(P) = \gamma$ and $\E_{G \sim \mathcal{G}(P)} [\Delta(G)] = \Theta( \gamma^{3/2} \cdot V(P)^{3/2})$.
 \end{theorem}
 \begin{proof}
 Let $P_{ij} = \gamma$ for all $i \neq j $. We have $V(P) = \gamma \cdot {n \choose 2}$ and $Ov(P) \cdot V(P) = \gamma^2 \cdot {n \choose 2}$ Thus, $Ov(P)= \gamma$. Further,  by linearity of expectation, 
 $$\E_{G \sim \mathcal{G}(P)} [\Delta(G)] = \gamma^3 \cdot {n \choose 3} = \Theta(\gamma^3 \cdot n^3) = \Theta(\gamma^{3/2} \cdot V(P)^{3/2}).$$
% 
% Let $S$ be a subset of $\sqrt{ \gamma  n/2}$ nodes. %, where $c$ is a fixed constant.  
%Let  $P_{ij} = 1$ for all $i \neq j$, $i,j \in S$ and $P_{ij} = 1/n$ for all other pairs $i,j$ with $i \neq j$.    
% We have $V(P) \ge \frac{1}{n} \cdot {n \choose 2} = \frac{n-1}{2} $. %On the rest of the graph we will have $ \Theta(n)$ edges in expectation. 
% Further, by Lemma \ref{lem:mainSimple},
% $$Ov(P) \cdot V(P) \le \norm{P}_F^2 \le {\sqrt{\gamma n/2} \choose 2 } + \frac{1}{n^2} \cdot n^2 \le \frac{\gamma n}{4} +1.$$
% Combined with our bound on $V(P)$, as long as $n$ is large enough so that  $\gamma n \ge 6$,
% $$Ov(P) \le \frac{\frac{\gamma n}{4} +1}{\frac{n-1}{2}} = \frac{\frac{\gamma n}{2} +2}{n-1} \le \frac{\frac{\gamma n}{2}(1+2/3)}{5/6 \cdot n}  \le \gamma.$$
% 
% Finally, for $G \sim \mathcal{G}(P)$, $\Delta(G)$ is lower bounded by the number of triangles in the clique on $S$ which $G$ contains with probability $1$ -- i.e., 
% $$\E_{G \sim \mathcal{G}(P)}[\Delta(G)] \ge {\sqrt{\gamma n/2} \choose 3}= \Theta(\gamma^{3/2} \cdot n^{3/2}) =  \Theta(\gamma^{3/2} \cdot V(P)^{3/2}).$$
  \end{proof}
  We note that another example when Theorem \ref{thm:tri} is tight is when $P$ is a union of a fixed clique on $\Theta(\gamma \cdot n)$ nodes and an Erd\"{o}s-R\'{e}nyi graph with connection probability $1/n$ on the rest of the nodes.
  
\subsection{Squares and Other $k$-cycles}
  
We can extend Thm. \ref{thm:tri} to bound the expected number of $k$-cycles in $G \sim \mathcal{G}(P)$ in terms of $Ov(P)$. %, we prove in the appendix,
%
%For example,
% \begin{theorem}[Bound on Expected Squares]\label{thm:square} For a graph $G$, let $\square(G)$ denote the number of squares (4-cycles) in $G$. Consider $P \in [0,1]^{n \times n}$ with $Ov(P) \le \gamma \cdot V(P)$ for some $\gamma <1$. Then
%\begin{align*}
%\E_{G \sim \mathcal{G}(P)} \left [\square(G) \right ]\le \frac{1}{2} \cdot \gamma^2 \cdot V(P)^2.
%\end{align*}
%\end{theorem}
%\begin{proof}
%The number of squares is the number of non-backtracking 4-cycles in $G$ (i.e. squares), which can be written as:
%\begin{align*}
%\E_{G \sim \mathcal{G}(P)} \left [\square(G) \right ] = \frac{1}{8} \cdot \sum_{i =1}^n \sum_{j \in [n] \setminus i} \sum_{k \in [n]\setminus\{i,j\}} \sum_{\ell \in [n]\setminus \{i,j,k\}} P_{ij} P_{jk} P_{k \ell} P_{\ell i}.
%\end{align*}
% The $1/8$ factor accounts for the fact that in the sum, each square is counted $8$ times -- once for each potential starting vector $i$ and once of each direction it may be traversed. We then can bound
%\begin{align*}
%\E_{G \sim \mathcal{G}(P)} \left [\square(G) \right ]  \le \frac{1}{8} \cdot\sum_{i \in [n]} \sum_{j \in [n]} \sum_{k \in [n]} \sum_{\ell \in [n]} P_{ij} P_{jk} P_{k \ell} P_{\ell i} = \frac{1}{8} \cdot \tr(P^4).
%\end{align*}
%This in turn gives
%\begin{align*}
%\E_{G \sim \mathcal{G}(P)} \left [\square(G) \right ] \le \frac{1}{8} \cdot |\lambda_1(P)|^2 \cdot \norm{P}_F^2 \le \frac{1}{8} \norm{P}_F^4 \le \frac{1}{2} Ov(P)^2.
%\end{align*}
%This completes the theorem after plugging in  $Ov(P) \le \gamma \cdot V(P)$.
%\end{proof}
%It is not hard to see that Theorem \ref{thm:square} is also tight in the same example as Theorem \ref{thm:tight}, since a clique on $\sqrt{c \gamma n}$ vertices will have ${\sqrt{c \gamma n} \choose 4} = \Theta(\gamma^2 n^2) = \Theta(\gamma^2  \cdot V(P)^2)$ squares. We can also just directly extend the proof of Theorem \ref{thm:square} to  bound the expected number of $k$-cycles in $G$, giving a bound which is also tight up to constants by the example of Theorem \ref{thm:tight} for $k = O(1)$.
\begin{theorem}[Bound on Expected $k$-cycles]\label{thm:k}
For a graph $G$, let $C_k(G)$ denote the number of $k$-cycles in $G$. Consider symmetric $P \in [0,1]^{n \times n}$. 
% with $Ov(P) \le \gamma \cdot V(P)$ for some $\gamma <1$. Then
\begin{align*}
\E_{G \sim \mathcal{G}(P)} \left [C_k(G) \right ] \le \frac{2^{k/2}}{2k} \cdot Ov(P)^{k/2} \cdot V(P)^{k/2}.
\end{align*}
\end{theorem}
\begin{proof}
For notational simplicity, we focus on $k = 4$. The proof directly extends to general $k$. $C_4(G)$ is the number of non-backtracking 4-cycles in $G$ (i.e. squares), which can be written as
\begin{align*}
\E_{G \sim \mathcal{G}(P)} \left [C_4(G) \right ] = \frac{1}{8} \cdot \sum_{i =1}^n \sum_{j \in [n] \setminus i} \sum_{k \in [n]\setminus\{i,j\}} \sum_{\ell \in [n]\setminus \{i,j,k\}} P_{ij} P_{jk} P_{k \ell} P_{\ell i}.
\end{align*}
 The $1/8$ factor accounts for the fact that in the sum, each square is counted $8$ times -- once for each potential starting vector $i$ and once of each direction it may be traversed. For general $k$-cycles this factor would be $\frac{1}{2k}$. We then can bound
\begin{align*}
\E_{G \sim \mathcal{G}(P)} \left [C_4(G) \right ]  \le \frac{1}{8} \cdot\sum_{i \in [n]} \sum_{j \in [n]} \sum_{k \in [n]} \sum_{\ell \in [n]} P_{ij} P_{jk} P_{k \ell} P_{\ell i} = \frac{1}{8} \cdot \tr(P^4).
\end{align*}
For general $k$-cycles this bound would be $\E_{G \sim \mathcal{G}(P)} \left [C_k(G) \right ] \le \frac{1}{2k} \tr(P^k).$
This in turn gives
\begin{align*}
\E_{G \sim \mathcal{G}(P)} \left [C_k(G) \right ] \le \frac{1}{2k} \cdot |\lambda_1(P)|^{k-2} \cdot \norm{P}_F^{2} \le \frac{1}{2k} \norm{P}_F^k \le \frac{2^{k/2}}{2k} Ov(P)^{k/2} \cdot V(P)^{k/2},
\end{align*}
where the last bound follows from Lemma \ref{lem:mainSimple}.
This completes the theorem..
\end{proof}
It is not hard to see that Theorem \ref{thm:k} is also tight up to a constant depending on $k$ for any overlap $\gamma \in (0,1]$, also for an Erd\"{o}s-R\'{e}nyi graph with connection probability $\gamma$. 
\begin{theorem}[Tightness of Expected $k$-cycle Bound]\label{thm:tightK}
For any $\gamma \in (0,1]$, there exists $P \in [0,1]^{n \times n}$ with $Ov(P) = \gamma$ and $\E_{G \sim \mathcal{G}(P)} [C_k(G)] = \Theta \left ( \frac{\gamma^{k/2} \cdot V(P)^{k/2}}{k!} \right)$.
 \end{theorem}
%\begin{proof}
%Consider the same $P$ as Thm. \ref{thm:tight}. The clique on $\sqrt{\gamma n/2}$ vertices has $ { \sqrt{\gamma n/2} \choose k} = \Theta \left (\frac{\gamma^{k/2} n^{k/2}}{k!} \right)$ $k$-cycles. So $Ov(P) \le \gamma$ and $\E_{G \sim \mathcal{G}(P)} \left [C_k(G) \right ] = \Theta \left (\frac{\gamma^{k/2}  \cdot V(P)^{k/2}}{k!} \right)$, giving the result. % Theorem \ref{thm:k} up to a $\Theta \left (\frac{2^{k/2} \cdot k!}{2k} \right )$ factor.
%\end{proof}

%We can also just directly extend the proof of Theorem \ref{thm:square} to  bound the expected number of $k$-cycles in $G$, giving a bound which is also tight up to constants by the example of Theorem \ref{thm:tight} for $k = O(1)$.
%Theorem \ref{thm:kclique} is also tight up to constants for constant $k$, by the same example as Theorem \ref{thm:tight}.
  
%  We can easily extend Theorem \ref{thm:tri} to $k$-cliques. We have
%  
%  \begin{theorem}[Bound on Expected $k$-Cliques]\label{thm:kclique} For a graph $G$, let $kC(G)$ denote the number of k-Cliques in $G$. Consider $P \in [0,1]^{n \times n}$ with $Ov(P) \le \gamma \cdot V(P)$ for some $\gamma <1$. Then
%\begin{align*}
%\E_{G \sim \mathcal{G}(P)} \left [kC(G) \right ]\le \frac{2^{k/2}}{k!} \cdot \gamma^{k/2} \cdot V(P)^{k/2}
%\end{align*}
%\end{theorem}
%\begin{proof}
%We can write $\E_{G \sim \mathcal{G}(P)} \left [kC(G) \right ] = \frac{1}{k!} \cdot \tr(P^k).$ Additionally, $\tr(P^{k}) \le |\lambda_1(P)|^{k-2} \cdot \tr(P^2) = |\lambda_1(P)|^{k-2} \cdot \norm{P}_F^2$, where $\lambda_1(P)$ is the largest magnitude eigenvalue of $P$. Since $\lambda_1(P)^2 \le \norm{P}_F^2 = \sum_{i=1}^n \lambda_i(P)^2$, this gives
%$$\tr(P^k) \le \norm{P}_F^k \le 2^{k/2} \cdot Ov(P)^{k/2},$$
%where the last inequality follows from Lemma \ref{lem:mainSimple}. This completes the lemma after using the assumption that $Ov(P) \le \gamma \cdot V(P)$.
%\end{proof}
%Theorem \ref{thm:kclique} is also tight up to constants for constant $k$, by the same example as Theorem \ref{thm:tight}.
    
\subsection{Clustering Coefficient}
Theorem \ref{thm:tri} shows that the expected number of triangles generated by an edge independent model is bounded in terms of the model's overlap. Intuitively, we thus expect that graphs generated by the edge independent model will have low global clustering coefficient, which is the fraction of wedges in the graph that are closed into triangles \cite{WattsStrogatz:1998}.

\begin{definition}[Global Clustering Coefficient]\label{def:cc}
For a graph $G$ with $\Delta(G)$ triangles, no self-loops, and node degrees $d_1,d_2,\ldots, d_n$, the global clustering coefficient is given by
\begin{align*}
C(G) = \frac{3 \Delta(G)}{\sum_{i=1}^n d_i(d_i-1)}.
\end{align*}
\end{definition}
%
%
We extend Theorem \ref{thm:tri} to give a bound on $E_{G \sim \mathcal{G}(P)} \left [C(G) \right ]$ in terms of $Ov(P)$. The proof is related, but  more complex due to the $\sum_{i=1}^n d_i(d_i-1)$ in the denominator of $C(G)$.
\begin{theorem}[Bound on Expected Clustering Coefficient]\label{thm:cc}
Consider symmetric $P \in [0,1]^{n \times n}$ with zeros on the diagonal and with $V(P) \ge 2 n$. %  and  $O(P) < \gamma \cdot V(P)$ for some $\gamma < 1$. Then % for $G \sim \mathcal{G}(P)$ with probability at least $1-\exp(-\Theta(n))$, the clustering coefficient is bounded by
\begin{align*}
E_{G \sim \mathcal{G}(P)} \left [C(G) \right ] = O \left (\frac{Ov(P)^{3/2} \cdot n}{V(P)^{1/2}} \right ).
\end{align*}
\end{theorem}
%
\begin{proof}
By Theorem \ref{thm:tri} we have $\E_{G \sim \mathcal{G}(P)} \left [3 \Delta(G) \right ]\le \sqrt{2} \cdot  Ov(P)^{3/2} \cdot V(P)^{3/2}$. We will show that with high probability, $\sum_{i=1}^n d_i (d_i - 1) = \Omega (V(P)^2/n)$, which will give the theorem.
Note that $\E_{G \sim \mathcal{G}(P)} \left [\sum_{i =1}^n d_i \right ] = \E_{G \sim \mathcal{G}(P)}[2|E(G)|] = 2 V(P)$. Thus,
by a Bernstein bound, for large enough $n$ since $V(P) \ge 2n$.
\begin{align*}
\Pr \left [ \left |\sum_{i =1}^n d_i - 2V(P) \right | \ge V(P)/5 \right ] \le 2 \exp\left (- \frac{V(P)^2/50}{V(P)+V(P)/15} \right ) \ll \frac{1}{n^2},
\end{align*}
We can bound 
%
%
%
 %$1-\exp(-\Theta(n))$, $$9/10 \cdot V(P) \le \sum_{i=1}^n d_i \le 11/10 \cdot V(P).$$ We can bound  
 $\sum_{i=1}^{n} d_i^2 \ge \frac{\left (\sum_{i=1}^n d_i\right)^2}{n}$. Thus, with probability $\ge 1-1/n^2$,
\begin{align*}
\sum_{i=1}^n d_i(d_i - 1) \ge \frac{(8/5)^2 \cdot V(P)^2}{n} - \frac{12}{5} V(P) \ge \frac{V(P)^2}{n},
\end{align*}
where in the last step we use that $V(P) \ge 2n$ and so $\frac{12}{5} \cdot V(P) \le \frac{6}{5} \cdot \frac{V(P)^2}{n}$. Combined with our bound on $\E_{G \sim \mathcal{G}(P)} \left [3 \Delta(G) \right ]$, and the fact that $C(G) \le 1$ always, we have
\begin{align*}
E_{G \sim \mathcal{G}(P)} \left [C(G) \right ] = O \left ( \frac{ Ov(P)^{3/2} V(P)^{3/2}}{\frac{V(P)^2}{n}} + \frac{1}{n^2} \right ) = O \left (\frac{Ov(P)^{3/2} \cdot n}{V(P)^{1/2}} \right ).
\end{align*}
\end{proof}

Thus, to have a constant clustering coefficient for a graph with $O(n)$ edges in expectation, we need $Ov(P) = \Omega(1/n^{1/3})$. Note that the requirement of $V(P) \ge 2 n$ is very mild -- it means that the expected average degree is at least $1$.

As with our triangle bound, Theorem \ref{thm:cc} is tight when $\mathcal{G}(P)$ is just an Erd\"{o}s-R\'{e}nyi distribution.
\begin{theorem}[Tightness of Expected Clustering Coefficient Bound]\label{thm:tightCC}
For any $\gamma \in (0,1]$, there exists $P \in [0,1]^{n \times n}$ with zeros on the diagonal, $Ov(P) \le \gamma$ and $\E_{G \sim \mathcal{G}(P)} [C(G)] = \Theta \left (\frac{\gamma^{3/2} \cdot n}{V(P)^{1/2}} \right )$.
 \end{theorem}
 \begin{proof}
Let $P_{ij} = \gamma$ for all $i \neq j$. We have $V(P) = \gamma \cdot {n \choose 2} = \Theta(\gamma n^2)$ and $Ov(P) =  \gamma$. Additionally,  %and $\E_{G \sim \mathcal{G}(P)} [C(G)] = \Theta \left (\frac{\gamma^{3/2} \cdot n}{V(P)^{1/2}} \right )$.
% We have $V(P) = \gamma \cdot n^2$ and $Ov(P) = \gamma^2 \cdot n^2$ which confirms that $Ov(P) \le \gamma \cdot V(P)$. We have 
 $\E[\Delta(G)] = \Theta(\gamma^3 \cdot n^3)$, and, if $n$ is large enough with respect to $\gamma$, with very high probability, $\sum_{i=1}^n d_i(d_i-1) \le \sum_{i=1}^n d_i^2 = O(\gamma^2 n^3)$. This gives:
 \begin{align*}
\E_{G \sim \mathcal{G}(P)} [C(G)] = \Theta (\gamma) = \Theta \left (\frac{\gamma^{3/2} \cdot  n}{\gamma^{1/2} \cdot n} \right ) = \Theta \left ( \frac{\gamma^{3/2} \cdot n}{V(P)^{1/2}} \right ).
 \end{align*}
 \end{proof}
 
% \subsection{Degree Sequence}
% 
% We conclude
% 
%\begin{theorem}[Degree Sequence Overlap Bound]
% Let $z \in \R^{n}$ be a degree sequence, with $Z = \max_i z_i$ and $vol = \norm{z}_1$. If $Z = \sqrt{vol \cdot \gamma}$ for some $\gamma \le 1$, then for any $P$ with $\E_{G \sim \mathcal{G}(P)} d_i = z_i$ for all $i$, we have $OV(P) \ge \gamma \cdot V(P)$.
% \end{theorem}
%\begin{proof}
%Consider the degree sequence $z$ corresponding to a graph which is a union of $1/\gamma$ cliques (including self-loops) each containing m nodes. We have $Z = m$ and $vol = m^2/\gamma$. Thus $Z = \sqrt{vol \cdot \gamma}$ as required.
%
%Now, for $P$ to satisfy  $\E_{G \sim \mathcal{G}(P)} d_i = z_i$ for all $i$, P must have rows/column summing to $m$ for each node in a clique and $0$'s in all other rows/columns. Since there are $m/\gamma$ nodes total in the cliques, we can maximally spread out the mass in a row of $P$ corresponding to a clique node by having value $\gamma$ in each of the $m/\gamma$ locations where we can place non-zero value. This $P$ contains an $m/\gamma \times m/\gamma$ block of all $\gamma$s, with all other entries set to $0$. We thus have $Ov(P) \le  ||P||_F^2 = m^2$. Further, $V(P) = m^2/\gamma$. So $Ov(P) = \gamma \cdot V(P)$. This is the the minimum value achievable sine we have minimized $\norm{p_i}_2^2/\norm{p_i}_1$ for every row $p_i$ of $P$.
%
%\end{proof}
%
% \begin{theorem}[Bounded Overlap for the Chung-Lu Model]\label{thm:chung}
% Let $z \in \R^{n}$ be a degree sequence, with $Z = \max_i z_i$ and $vol = \norm{z}_1$. If $Z \le \sqrt{vol \cdot \gamma}$ for some $\gamma \le 1$ then letting $P =  \frac{1}{vol} \cdot z z^T$, for all $i$, $\E_{G \sim \mathcal{G}(P)} d_i = z_i$. Further, $Ov(P) \le \gamma \cdot V(P).$
% \end{theorem}
% \begin{proof}
% This is immediate since all entries of $P$ are bounded by $Z^2/vol \le \gamma$. Thus, $Ov(P) \le \norm{P}_F^2 \le \gamma \cdot \norm{P}_1 = \gamma \cdot V(P)$, where $\norm{P}_1 = V(P)$ is the entrywise $\ell_1$ norm.
% \end{proof}


\section{Baseline Edge Independent Models}
\label{sec:proposed}
\label{sec:propose}

\paragraph{Correlation with listwise ground-truth}
Before describing our new QPP evaluation framework \proposed, we begin by introducing the required notation. Formally, a QPP estimate is a function of the form $\phi(Q, M_k(Q)) \mapsto \mathbb{R}$, where $M_k(Q)$ is the set of top-$k$ ranked documents retrieved by an IR model $M$ for a query $Q \in \mathcal{Q}$, a benchmark set of queries.

For the purpose of listwise evaluation, for each $Q\in \mathcal{Q}$, we first compute the value of a target IR evaluation metric, $\mu(Q)$ that reflects the quality of the retrieved list $M_k(Q)$. The next step uses these $\mu(Q)$ scores to induce a \textit{ground-truth ranking} of the set $\mathcal{Q}$, or in other words, arrange the queries by their decreasing (or increasing) $\mu(Q)$ values, i.e., 
\begin{equation}
\mathcal{Q}_\mu = \{Q_i \in \mathcal{Q}: \mu(Q_i) > \mu(Q_{i+1}),
\, \forall i=1,\ldots,|\mathcal{Q}|-1\}  \}
\end{equation}
Similarly, the evaluation framework also yields a \emph{predicted ranking} of the queries, where this time the queries are sorted by the QPP estimated scores, i.e.,
\begin{equation}
\mathcal{Q}_\phi = \{Q_i \in \mathcal{Q}: \phi(Q_i) > \phi(Q_{i+1}),
\, \forall i=1,\ldots,|\mathcal{Q}|-1 \} 
\label{qpp_listwise_pred}
\end{equation}
A listwise evaluation framework then computes the rank correlation between these two ordered sets
$\gamma(\mathcal{Q}_\mu, \mathcal{Q}_\phi),\,\,\text{where}\,\, \gamma: \mathbb{R}^{|\mathcal{Q}|}\times\mathbb{R}^{|\mathcal{Q}|} \mapsto [0,1]$ is a correlation measure, such as Kendall's $\tau$.

\paragraph{Individual ground-truth}
In contrast to listwise evaluations, where the ground-truth takes the form of an ordered set of queries, pointwise QPP evaluation involves making $|\mathcal{Q}|$ \textit{independent comparisons}. Each comparison is made between a query $Q$'s predicted QPP score $\phi(Q)$ and its retrieval effectiveness measure $\mu(Q)$, i.e.,
\begin{equation}
\eta(\mathcal{Q}, \mu, \phi) \defas \frac{1}{|\mathcal{Q}|}\sum_{Q \in \mathcal{Q}}\eta(\mu(Q), \phi(Q))
\label{eq:pwcorr}  
\end{equation}
Unlike the rank correlation $\gamma$, 
here $\eta$ is a pointwise correlation function of the form $\eta:\mathbb{R}\times \mathbb{R}\mapsto\mathbb{R}$.
It is often convenient to think of $\eta$ as the inverse of a \emph{distance} function that measures the extent to which a predicted value deviates from the corresponding true value.
In contrast to ground-truth evaluation metrics, most QPP estimates (e.g., NQC, WIG etc.) are not bounded within $[0, 1]$. Therefore, to employ a distance measure, each QPP estimate $\phi(Q)$ must be normalized to the unit interval. Subsequently, $\eta$ can be defined as
$\eta(\mu(Q), \phi(Q)) \defas 1-|\mu(Q) - \phi(Q)/\aleph|$,
where $\aleph$ is a normalization constant which is sufficiently large to ensure that the denominator is positive.

\paragraph{Selecting an IR metric for pointwise QPP evaluation}

In general, an unsupervised QPP estimator will be agnostic with respect to the target IR metric $\mu$. For instance, NQC scores can be seen as being approximations of AP@100 values, but can also be interpreted as approximating any other metric, such as nDCG@20 or P@10. Therefore, a question arises around which metric should be used to compute the individual correlations in Equation \ref{eq:pwcorr}. Of course, the results can differ substantially for different choices of $\mu$, e.g., AP or nDCG. This is also the case for listwise QPP evaluation, as reported in \cite{dg22ecir}. To reduce the effect of such variations, we now propose a simple yet effective solution.

\paragraph{Metric-agnostic pointwise QPP evaluation}
For a set of evaluation functions
$\mu \in \mathcal{M}$ (e.g., $\mathcal{M} = \{\text{AP@100}, \text{nDCG@20},\ldots\}$), we employ an aggregation function to compute the overall pointwise correlation (Equation \ref{eq:pwcorr}) of a QPP estimate with respect to each metric.
Formally,
\begin{equation}
\eta(Q,\mathcal{M},\phi) = \Sigma_{\mu \in \mathcal{M}} 
(1-|\mu(Q) - \phi(Q)/\aleph|), \label{eq:avgpwcorr}
\end{equation}
where $\Sigma$ denotes an aggregation function (it does not indicate summation). In particular, we use the most commonly-used such functions as choices for $\Sigma$: `minimum', `maximum', and `average' -- i.e., $\Sigma \in \{\text{avg}, \text{min}, \text{max}\}$.
Next, we find the average over these values computed for a given set of queries $\mathcal{Q}$, i.e., we substitute $\eta(Q,\mathcal{M},\phi)$ from Equation \ref{eq:avgpwcorr} into the summation of Equation \ref{eq:pwcorr}.

\section{Experimental Results}
\label{sec:exp}
\section{Experiment \& Analysis}
\label{sec:exp}
In this section, we first introduce the experimental set-up. Then, we report the performances of baselines and the proposed steep slope loss on ImageNet, followed by comprehensive analyses. 
% At last, we present comprehensive analyses to help better understand the efficacy of the proposed loss.

\noindent\textbf{Experimental Set-Up}.
We use ViT B/16 \cite{Dosovitskiy_ICLR_2021} and ResNet-50 \cite{He_CVPR_2016} as the classifiers, and the respective backbones are used as the oracles' backbones. We denote the combination of oracles and classifiers as \textlangle O, C\textrangle. There are four combinations in total, \ie \textlangle ViT, ViT\textrangle, \textlangle ViT, RSN\textrangle, \textlangle RSN, ViT\textrangle, and \textlangle RSN, RSN\textrangle, where RSN stands for ResNet.
In this work, we adopt three baselines, \ie the cross entropy loss \cite{Cox_JRSS_1972}, focal loss \cite{Lin_ICCV_2017}, and TCP confidence loss \cite{Corbiere_NIPS_2019}, for comparison purposes.

The experiment is conducted on ImageNet \cite{Deng_CVPR_2009}, which consists of 1.2 million labeled training images and 50000 labeled validation images. It has 1000 visual concepts. Similar to the learning scheme in \cite{Corbiere_NIPS_2019}, the oracle is trained with training samples and evaluated on the validation set. During the training process of the oracle, the classifier works in the evaluation mode so training the oracle would not affect the parameters of the classifier. Moreover, we conduct the analyses about how well the learned oracle generalizes to the images in the unseen domains. To this end, we apply the widely-used style transfer method \cite{Geirhos_ICLR_2019} and the functional adversarial attack method \cite{Laidlaw_NeurIPS_2019} to generate two variants of the validation set, \ie stylized validation set and adversarial validation set. \REVISION{Also, we adopt ImageNet-C \cite{Hendrycks_ICLR_2018} for evaluation, which is used for evaluating robustness to common corruptions.}
% Then, the oracle trained with regular training samples would be evaluated with the samples that are in the two unseen domains.

% To understand how the learned oracle work on unseen domains, the oracle is learned with training samples and is evaluated with three types of samples, the samples on the same domain as training samples and the samples on two unseen domains. We base our experiments on ImageNet \cite{Deng_CVPR_2009}, a widely-used large-scale dataset. Except for the training set and the validation set, we use the stylized ImageNet validation set \cite{Geirhos_ICLR_2019} and an ImageNet validation set that are perturbed by the functional adversarial attack technique \cite{Laidlaw_NeurIPS_2019}.
% (Introduce models here.)
% (Introduce hyperpaprameters here.)

The oracle's backbone is initialized by the pre-trained classifier's backbone and trained by fine-tuning using training samples and the trained classifier.
% As the oracle's backbone is initialized by the pre-trained classifier's backbone, the training process of the oracles is equivalent to the process of fine-tuning the initialized oracles.
Training the oracles with all the loss functions uses the same hyperparameters, such as learning rate, weight decay, momentum, batch size, etc.
The details for the training process and the implementation are provided in \appref{sec:implementation}.

For the focal loss, we follow \cite{Lin_ICCV_2017} to use $\gamma=2$,  which leads to the best performance for object detection.
For the proposed loss, we use $\alpha^{+}=1$ and $\alpha^{-}=3$ for the oracle that is based on ViT's backbone, while we use $\alpha^{+}=2$ and $\alpha^{-}=5$ for the oracle that is based on ResNet's backbone.

Following \cite{Corbiere_NIPS_2019}, we use FPR-95\%-TPR, AUPR-Error, AUPR-Success, and AUC as the metrics.
FPR-95\%-TPR is the false positive rate (FPR) when true positive rate (TPR) is equal to 95\%. 
AUPR is the area under the precision-recall curve. 
Specifically, AUPR-Success considers the correct prediction as the positive class, whereas AUPR-Error considers the incorrect prediction as the positive class.
AUC is the area under the receiver operating characteristic curve, which is the plot of TPR versus FPR.
Moreover, we use TPR and true negative rate (TNR) as additional metrics because they assess overfitting issue, \eg TPR=100\% and TNR=0\% imply that the trustworthiness predictor is prone to view all the incorrect predictions to be trustworthy. %due to overfitting.

% \noindent\textbf{Baselines \& Metrics}.
% We adopt widely-used loss functions, \ie cross entropy and focal loss, as the baselines. To comprehensively understand and measure oracles' performance, we use KL divergence and Bhattacharya coefficient to measure the correlation between two feature distributions, use true positive rate (TPR), true negative rate (TNR), accuracy (Acc=$(TP+TN)/Total$), F1 score, precision (P), and recall (R) to measure the efficacy of predicting trustworthiness. Specifically, we add Acc\textsubscript{P} and Acc\textsubscript{N} to understand how much TP and TN contribute to Acc. This is useful when the model overfits the data, \ie classifying all the images as either positives or negatives. Moreover, to differentiate the accuracy of classification from the accuracy of predicting trustworthiness, we denote the classifier's accuracy as C-Acc, and the oracle's accuracy as O-Acc.

% \begin{table}[!t]
	\centering
	\caption{\label{tbl:avg_perf}
	    Averaged performance over the regular ImageNet validation set, the stylized ImageNet validation set, and the adversarial ImageNet validation set. The oracle is trained with the cross entropy (CE) loss, the focal loss, and the proposed steep slope (SS) loss on the ImageNet training set. The resulting oracles w.r.t. each loss are evaluated on the three validation sets. The classifier is used in the evaluation mode in the experiment. $d_{KL}$ represents KL divergence, while $c_{B}$ represents Bhattacharyya coefficient.
	}
	\adjustbox{width=1\columnwidth}{
	\begin{tabular}{L{7ex} C{8ex} C{8ex} C{8ex} C{10ex} C{8ex} C{8ex} C{8ex} C{8ex} C{8ex} C{10ex} C{8ex}}
		\toprule
		Loss & C-Acc & $d_{KL}\uparrow$ & $c_{B}\downarrow$ & TPR & TNR & O-Acc & O-Acc\textsubscript{P} & O-Acc\textsubscript{N} & F1 & P & R  \\
		\cmidrule(lr){1-1} \cmidrule(lr){2-2} \cmidrule(lr){3-3} \cmidrule(lr){4-4} \cmidrule(lr){5-5} \cmidrule(lr){6-6} \cmidrule(lr){7-7} \cmidrule(lr){8-8} \cmidrule(lr){9-9} \cmidrule(lr){10-10} \cmidrule(lr){11-11} \cmidrule(lr){12-12}
		& \multicolumn{11}{c}{Oracle: ViT, classifier: ViT} \\
		\cmidrule(lr){2-12}
		CE & 35.74 & 0.5138 & 0.9983 & 99.98 & 0.04 & 35.78 & 35.74 & 0.04 & 0.4382 & 0.3575 & 0.8444 \\
        Focal & 35.74 & 0.5224 & 0.9972 & 99.23 & 1.30 & 36.22 & 35.43 & 0.78 & 0.4374 & 0.3579 & 0.8403 \\
        SS & 35.74 & 1.0875 & 0.9302 & 73.62 & 47.23 & 63.94 & 29.84 & 34.10 & 0.4727 & 0.4430 & 0.5964 \\ \midrule
		& \multicolumn{11}{c}{Oracle: ResNet, classifier: ViT} \\
		\cmidrule(lr){2-12}
		CE & & & & & & & & & & & \\
		Focal & & & & & & & & & & & \\
		SS & & & & & & & & & & & \\
		\bottomrule	
	\end{tabular}}
\end{table}

% \begin{figure}[!t]
% 	\centering
% 	\subfloat{\includegraphics[width=0.32\textwidth]{fig/sigmoid_imagenet_trfeat}    } \hfill
% 	\subfloat{\includegraphics[width=0.32\textwidth]{fig/focal_imagenet_trfeat}    } \hfill
% 	\subfloat{\includegraphics[width=0.32\textwidth]{fig/steep_imagenet_trfeat}    } \\
% 	\subfloat{\includegraphics[width=0.32\textwidth]{fig/sigmoid_imagenet_valfeat}    } \hfill
% 	\subfloat{\includegraphics[width=0.32\textwidth]{fig/focal_imagenet_valfeat}    } \hfill
% 	\subfloat{\includegraphics[width=0.32\textwidth]{fig/steep_imagenet_valfeat}    } \\
% 	\subfloat{\includegraphics[width=0.32\textwidth]{fig/sigmoid_imagenet_valfeat_sty}    } \hfill
% 	\subfloat{\includegraphics[width=0.32\textwidth]{fig/focal_imagenet_valfeat_sty}    } \hfill
% 	\subfloat{\includegraphics[width=0.32\textwidth]{fig/steep_imagenet_valfeat_sty}    } \\
% 	\subfloat{\includegraphics[width=0.32\textwidth]{fig/sigmoid_imagenet_valfeat_adv}    } \hfill
% 	\subfloat{\includegraphics[width=0.32\textwidth]{fig/focal_imagenet_valfeat_adv}    } \hfill
% 	\subfloat{\includegraphics[width=0.32\textwidth]{fig/steep_imagenet_valfeat_adv}    }
% 	\caption{\label{fig:distribution}
%     	Feature distributions w.r.t. the cross entropy (first column), focal (second column), and the proposed steep slope (third column) losses on the ImageNet training set (second row), ImageNet validation set (first row), stylized ImageNet validation set (third row), and adversarial ImageNet validation set (fourth row). CE stands for cross entropy, while SS stands for steep slope.
%     % 	\REVISION{\textit{Baseline} indicates ResNet GEM.}
%     	}
% \end{figure}

% \begin{table}[!t]
	\centering
	\caption{\label{tbl:perf_rsn_vit}
	    Performances on the regular ImageNet validation set, the stylized ImageNet validation set, and the adversarial ImageNet validation set. In this experiment, ResNet-50 is used for the oracle backbone while ViT is used for the classifier. The classifier is used in the evaluation mode in the experiment.
	}
	\adjustbox{width=1\columnwidth}{
	\begin{tabular}{L{8ex} C{8ex} C{8ex} C{8ex} C{10ex} C{8ex} C{8ex} C{8ex}}
		\toprule
		Loss & Acc$\uparrow$ & FPR-95\%-TPR$\downarrow$ & AURP-Error$\uparrow$ & AURP-Success$\uparrow$ & AUC$\uparrow$ & TPR$\uparrow$ & TNR$\uparrow$ \\
		\cmidrule(lr){1-1} \cmidrule(lr){2-2} \cmidrule(lr){3-3} \cmidrule(lr){4-4} \cmidrule(lr){5-5} \cmidrule(lr){6-6} \cmidrule(lr){7-7} \cmidrule(lr){8-8}
		& \multicolumn{7}{c}{Regular validation set} \\
		\cmidrule(lr){1-1} \cmidrule(lr){2-8}
		CE & 83.90 & 92.58 & 14.59 & 85.57 & 53.78 & 100.00 & 0.00 \\
		Focal & 83.90 & 94.92 & 14.87 & 85.26 & 52.49 & 100.00 & 0.00 \\
		TCP & 83.90 & 91.63 & 14.17 & 86.06 & 55.37 & 100.00 & 0.00 \\
% 		SS & 83.90 & 89.86 & 11.99 & 89.49 & 62.75 & 67.74 & 48.98 \\
		SS & 83.90 & 88.63 & 11.75 & 89.87 & 64.11 & 95.41 & 10.48 \\
% 		& 83.90 & 88.63 & 11.75 & 89.87 & 64.11 & 95.41 & 10.48 \\
%         & 83.90 & 88.72 & 11.76 & 89.85 & 64.01 & 91.10 & 18.51 \\
% 		& 83.90 & 88.25 & 11.54 & 90.24 & 65.23 & 88.23 & 24.25 \\ rsn152
		\midrule
		& \multicolumn{7}{c}{Stylized validation set} \\
		\cmidrule(lr){1-1} \cmidrule(lr){2-8}
		CE & 15.94 & 86.54 & 79.32 & 22.00 & 61.74 & 100.00 & 0.00 \\
		Focal & 15.94 & 95.04 & 85.18 & 14.94 & 48.20 & 100.00 & 0.00 \\
		TCP & 15.94 & 90.82 & 80.69 & 19.96 & 58.34 & 100.00 & 0.00 \\
		SS & 15.94 & 93.80 & 82.10 & 18.19 & 54.18 & 56.94 & 48.88 \\
% 		& 15.94 & 94.27 & 83.09 & 16.97 & 52.35 & 92.13 & 9.03 \\ 52
% 		& 15.94 & 95.76 & 84.28 & 15.74 & 49.02 & 93.83 & 5.52 \\ a62
        \midrule
		& \multicolumn{7}{c}{Adversarial validation set} \\
		\cmidrule(lr){1-1} \cmidrule(lr){2-8}
        CE & & & & & & &  \\
        Focal & & & & & & &  \\
        TCP & & & & & & & \\
        SS & & & & & & &  \\
		\bottomrule	
	\end{tabular}}
\end{table}



% \begin{figure}[!t]
	\centering
	\subfloat[\textlangle ViT, ViT\textrangle]{\includegraphics[width=0.24\textwidth]{fig/risk/risk_vit_vit}} \hfill
	\subfloat[\textlangle ViT, RSN\textrangle]{\includegraphics[width=0.24\textwidth]{fig/risk/risk_vit_rsn}} \hfill
	\subfloat[\textlangle RSN, ViT\textrangle]{\includegraphics[width=0.24\textwidth]{fig/risk/risk_rsn_vit}}
	\hfill
	\subfloat[\textlangle RSN, RSN\textrangle]{\includegraphics[width=0.24\textwidth]{fig/risk/risk_rsn_rsn}}
% 	\subfloat[O: ViT, C: ViT, Loss: TCP]{\includegraphics[width=0.24\textwidth]{fig/tsne/tsne_tcp}    } \hfill
% 	\subfloat[O: ViT, C: ViT, Loss: SS]{\includegraphics[width=0.24\textwidth]{fig/tsne/tsne_steep}    } 
    \\
	\caption{\label{fig:anal_risk}
    	Curves of risk vs. coverage. Selective risk represents the percentage of errors in the remaining validation set for a given coverage.
    	The curves correspond to the oracles used in \tabref{tbl:all_perf_w_std}.
    % 	\REVISION{\textit{Baseline} indicates ResNet GEM.}
    	}
\end{figure}

% \begin{figure}[!t]
	\centering
	\subfloat[O: ViT, C: ViT, Loss: CE]{\includegraphics[width=0.24\textwidth]{fig/tsne/tsne_ce}    } \hfill
	\subfloat[O: ViT, C: ViT, Loss: Focal]{\includegraphics[width=0.24\textwidth]{fig/tsne/tsne_focal}    } \hfill
	\subfloat[O: ViT, C: ViT, Loss: TCP]{\includegraphics[width=0.24\textwidth]{fig/tsne/tsne_tcp}    } \hfill
	\subfloat[O: ViT, C: ViT, Loss: SS]{\includegraphics[width=0.24\textwidth]{fig/tsne/tsne_steep}    } \\
	\caption{\label{fig:anal_tsne}
    	Analysis of t-SNE.
    % 	\REVISION{\textit{Baseline} indicates ResNet GEM.}
    	}
\end{figure}

% \begin{table}[!t]
% 	\centering
% 	\caption{\label{tbl:noise}
% 	    Correctness of oracle on the ImageNet validation set. The oracles are trained with the ImageNet training set. The underlined architecture indicates the architecture of Bayesian network. Leave-out rate indicates the proportion of samples that are ruled out by the oracle. Ideally, it should be equivelant to 1-Acc.
% 	}
% 	\adjustbox{width=1\columnwidth}{
% 	\begin{tabular}{L{10ex} C{12ex} C{12ex} C{9ex} C{9ex} C{9ex} C{9ex} C{9ex} C{9ex} C{9ex}}
% 		\toprule
% 		Dataset & Oracle & Classifier & Acc & O-Acc & O-TP & O-FP & F1 & Precision & Recall \\
% 		\cmidrule(lr){1-1} \cmidrule(lr){2-2} \cmidrule(lr){3-3} \cmidrule(lr){4-4} \cmidrule(lr){5-5} \cmidrule(lr){6-6} \cmidrule(lr){7-7} \cmidrule(lr){8-8} \cmidrule(lr){9-9} \cmidrule(lr){10-10}
% 		Regular & ViT-sigm            & ViT & 83.90 & 83.93 & 83.41 & 15.57 & 0.9121 & 0.8426 & 0.9941    \\
% 		Regular & ViT-Gauss            & ViT & 83.90 & 83.95 & 83.26 & 15.41 & 0.9121 & 0.8438 & 0.9924    \\
% 		Regular & ViT-exp            & ViT & 83.90 & 82.11 &  &  &  &  &     \\  \midrule
% 		Stylized & ViT-sigm            & ViT & 15.93 & 20.62 & 15.36 & 78.79 & 0.2790 & 0.1631 & 0.9639    \\
% 		Stylized & ViT-Gauss            & ViT & 15.93 & 46.28 & 13.01 & 50.79 & 0.3263 & 0.2039 & 0.8163    \\
% 		Stylized & ViT-exp            & ViT & 15.93 & 72.23 &  &  &  &  &     \\ \midrule
% 		Adv & ViT-sigm            & ViT & 7.41 & 11.23 & & & 0.1307 & 0.0762 & 0.5336    \\
% 		Adv & ViT-Gauss            & ViT & 7.41 & 11.15 & 7.14 & 88.79 & 0.1270 & 0.0744 & 0.5088 \\
% 		Adv & ViT-exp            & ViT & 7.41 & 32.57 &  &  &  &  &     \\ 
% 		\bottomrule	
% 	\end{tabular}}
% \end{table}

% \begin{table}[!t]
	\centering
	\caption{\label{tbl:all_perf}
	    Performance on the ImageNet validation set. The averaged scores are computed over three runs. The oracles are trained with the ImageNet training samples. The classifier is used in the evaluation mode in the experiment. Acc is the classification accuracy (\%) and is helpful to understand the proportion of correct predictions. \textit{SS} stands for the proposed steep slope loss.
	   % For example, Acc=83.90\% implies that 83.90\% of predictions is trustworthy and 16.10\% of predictions is untrustworthy.
	}
	\adjustbox{width=1\columnwidth}{
	\begin{tabular}{C{10ex} L{9ex} C{8ex} C{10ex} C{8ex} C{8ex} C{8ex} C{8ex} C{8ex}}
		\toprule
		\textbf{\textlangle O, C\textrangle} & \textbf{Loss} & \textbf{Acc$\uparrow$} & \textbf{FPR-95\%-TPR$\downarrow$} & \textbf{AUPR-Error$\uparrow$} & \textbf{AUPR-Success$\uparrow$} & \textbf{AUC$\uparrow$} & \textbf{TPR$\uparrow$} & \textbf{TNR$\uparrow$} \\
		\cmidrule(lr){1-1} \cmidrule(lr){2-2} \cmidrule(lr){3-3} \cmidrule(lr){4-4} \cmidrule(lr){5-5} \cmidrule(lr){6-6} \cmidrule(lr){7-7} \cmidrule(lr){8-8} \cmidrule(lr){9-9} 
		\multirow{4}{*}{\textlangle ViT, ViT \textrangle} & CE & 83.90 & 93.01 & \textbf{15.80} & 84.25 & 51.62 & \textbf{99.99} & 0.02 \\
		 & Focal \cite{Lin_ICCV_2017} & 83.90 & 93.37 & 15.31 & 84.76 & 52.38 & 99.15 & 1.35 \\
		 & TCP \cite{Corbiere_NIPS_2019} & 83.90 & 88.38 & 12.96 & 87.63 & 60.14 & 99.73 & 0.00 \\
		 & SS & 83.90 & \textbf{80.48} & 10.26 & \textbf{93.01} & \textbf{73.68} & 87.52 & \textbf{38.27} \\
		\midrule
		\multirow{4}{*}{\textlangle ViT, RSN\textrangle} & CE & 68.72 & 93.43 & 30.90 & 69.13 & 51.24 & \textbf{99.90} & 0.20 \\
		 & Focal \cite{Lin_ICCV_2017} & 68.72 & 93.94 & \textbf{30.97} & 69.07 & 51.26 & 93.66 & 7.71 \\
		 & TCP \cite{Corbiere_NIPS_2019} & 68.72 & 83.55 & 23.56 & 79.04 & 66.23 & 94.25 & 0.00 \\
		 & SS & 68.72 & \textbf{77.89} & 20.91 & \textbf{85.39} & \textbf{74.31} & 68.32 & \textbf{67.53} \\
        \midrule
		\multirow{4}{*}{\textlangle RSN, ViT\textrangle} & CE & 83.90 & 93.29 & 14.74 & 85.40 & 53.43 & \textbf{100.00} & 0.00 \\
		 & Focal \cite{Lin_ICCV_2017} & 83.90 & 94.60 & \textbf{14.98} & 85.13 & 52.37 & \textbf{100.00} & 0.00 \\
		 & TCP \cite{Corbiere_NIPS_2019} & 83.90 & 91.93 & 14.12 & 86.12 & 55.55 & \textbf{100.00} & 0.00 \\
         & SS & 83.90 & \textbf{88.70} & 11.69 & \textbf{90.01} & \textbf{64.34} & 96.20 & \textbf{9.00} \\
% 		RSN & ViT & SS & 83.90 & 89.86 & 11.99 & 89.49 & 62.75 & 67.74 & 48.98 \\
        \midrule
        \multirow{4}{*}{\textlangle RSN, RSN\textrangle} & CE & 68.72 & 94.84 & 29.41 & 70.79 & 52.36 & \textbf{100.00} & 0.00 \\
		 & Focal \cite{Lin_ICCV_2017} & 68.72 & 95.16 & \textbf{29.92} & 70.23 & 51.43 & 99.86 & 0.08 \\
		 & TCP \cite{Corbiere_NIPS_2019} & 68.72 & 88.81 & 24.46 & 77.79 & 62.73 & 99.23 & 0.00 \\
         & SS & 68.72 & \textbf{86.21} & 22.53 & \textbf{81.88} & \textbf{67.92} & 79.20 & \textbf{42.09} \\
		\bottomrule	
	\end{tabular}}
\end{table}
\begin{table}[!t]
	\centering
	\caption{\label{tbl:all_perf_w_std}
	    Performance on the ImageNet validation set. The mean and the standard deviation of scores are computed over three runs. The oracles are trained with the ImageNet training samples. The classifier is used in the evaluation mode. Acc is the classification accuracy and is helpful to understand the proportion of correct predictions. \textit{SS} stands for the proposed steep slope loss.
	   % For example, Acc=83.90\% implies that 83.90\% of predictions is trustworthy and 16.10\% of predictions is untrustworthy.
	}
	\adjustbox{width=1\columnwidth}{
	\begin{tabular}{C{10ex} L{10ex} C{8ex} C{10ex} C{10ex} C{10ex} C{10ex} C{10ex} C{10ex}}
		\toprule
		\textbf{\textlangle O, C\textrangle} & \textbf{Loss} & \textbf{Acc$\uparrow$} & \textbf{FPR-95\%-TPR$\downarrow$} & \textbf{AUPR-Error$\uparrow$} & \textbf{AUPR-Success$\uparrow$} & \textbf{AUC$\uparrow$} & \textbf{TPR$\uparrow$} & \textbf{TNR$\uparrow$} \\
		\cmidrule(lr){1-1} \cmidrule(lr){2-2} \cmidrule(lr){3-3} \cmidrule(lr){4-4} \cmidrule(lr){5-5} \cmidrule(lr){6-6} \cmidrule(lr){7-7} \cmidrule(lr){8-8} \cmidrule(lr){9-9} 
		\multirow{4}{*}{\textlangle ViT, ViT \textrangle} & CE & 83.90 & 93.01$\pm$0.17 & \textbf{15.80}$\pm$0.56 & 84.25$\pm$0.57 & 51.62$\pm$0.86 & \textbf{99.99}$\pm$0.01 & 0.02$\pm$0.02 \\
		 & Focal \cite{Lin_ICCV_2017} & 83.90 & 93.37$\pm$0.52 & 15.31$\pm$0.44 & 84.76$\pm$0.50 & 52.38$\pm$0.77 & 99.15$\pm$0.14 & 1.35$\pm$0.22 \\
		 & TCP \cite{Corbiere_NIPS_2019} & 83.90 & 88.38$\pm$0.23 & 12.96$\pm$0.10 & 87.63$\pm$0.15 & 60.14$\pm$0.47 & 99.73$\pm$0.02 & 0.00$\pm$0.00 \\
		 & SS & 83.90 & \textbf{80.48}$\pm$0.66 & 10.26$\pm$0.03 & \textbf{93.01}$\pm$0.10 & \textbf{73.68}$\pm$0.27 & 87.52$\pm$0.95 & \textbf{38.27}$\pm$2.48 \\
		\midrule
		\multirow{4}{*}{\textlangle ViT, RSN\textrangle} & CE & 68.72 & 93.43$\pm$0.28 & 30.90$\pm$0.35 & 69.13$\pm$0.36 & 51.24$\pm$0.63 & \textbf{99.90}$\pm$0.04 & 0.20$\pm$0.00 \\
		 & Focal \cite{Lin_ICCV_2017} & 68.72 & 93.94$\pm$0.51 & \textbf{30.97}$\pm$0.36 & 69.07$\pm$0.35 & 51.26$\pm$0.62 & 93.66$\pm$0.29 & 7.71$\pm$0.53 \\
		 & TCP \cite{Corbiere_NIPS_2019} & 68.72 & 83.55$\pm$0.70 & 23.56$\pm$0.47 & 79.04$\pm$0.91 & 66.23$\pm$1.02 & 94.25$\pm$0.96 & 0.00$\pm$0.00 \\
		 & SS & 68.72 & \textbf{77.89}$\pm$0.39 & 20.91$\pm$0.05 & \textbf{85.39}$\pm$0.16 & \textbf{74.31}$\pm$0.21 & 68.32$\pm$0.41 & \textbf{67.53}$\pm$0.62 \\
        \midrule
		\multirow{4}{*}{\textlangle RSN, ViT\textrangle} & CE & 83.90 & 93.29$\pm$0.53 & 14.74$\pm$0.17 & 85.40$\pm$0.20 & 53.43$\pm$0.28 & \textbf{100.00}$\pm$0.00 & 0.00$\pm$0.00 \\
		 & Focal \cite{Lin_ICCV_2017} & 83.90 & 94.60$\pm$0.53 & \textbf{14.98}$\pm$0.21 & 85.13$\pm$0.24 & 52.37$\pm$0.51 & \textbf{100.00}$\pm$0.00 & 0.00$\pm$0.00 \\
		 & TCP \cite{Corbiere_NIPS_2019} & 83.90 &91.93$\pm$0.49 & 14.12$\pm$0.12 & 86.12$\pm$0.15 & 55.55$\pm$0.46 & \textbf{100.00}$\pm$0.00 & 0.00$\pm$0.00 \\
         & SS & 83.90 & \textbf{88.70}$\pm$0.08 & 11.69$\pm$0.04 & \textbf{90.01}$\pm$0.10 & \textbf{64.34}$\pm$0.16 & 96.20$\pm$0.73 & \textbf{9.00}$\pm$1.32 \\
% 		RSN & ViT & SS & 83.90 & 89.86 & 11.99 & 89.49 & 62.75 & 67.74 & 48.98 \\
        \midrule
        \multirow{4}{*}{\textlangle RSN, RSN\textrangle} & CE & 68.72 & 94.84$\pm$0.27 & 29.41$\pm$0.18 & 70.79$\pm$0.19 & 52.36$\pm$0.41 & \textbf{100.00}$\pm$0.00 & 0.00$\pm$0.00 \\
		 & Focal \cite{Lin_ICCV_2017} & 68.72 & 95.16$\pm$0.19 & \textbf{29.92}$\pm$0.38 & 70.23$\pm$0.44 & 51.43$\pm$0.50 & 99.86$\pm$0.05 & 0.08$\pm$0.03 \\
		 & TCP \cite{Corbiere_NIPS_2019} & 68.72 & 88.81$\pm$0.24 & 24.46$\pm$0.12 & 77.79$\pm$0.29 & 62.73$\pm$0.14 & 99.23$\pm$0.14 & 0.00$\pm$0.00 \\
         & SS & 68.72 & \textbf{86.21}$\pm$0.44 & 22.53$\pm$0.03 & \textbf{81.88}$\pm$0.10 & \textbf{67.92}$\pm$0.11 & 79.20$\pm$2.50 & \textbf{42.09}$\pm$3.77 \\
		\bottomrule	
	\end{tabular}}
\end{table}

\noindent\textbf{Performance on Large-Scale Dataset}. 
The result on ImageNet are reported in \tabref{tbl:all_perf_w_std}. We have two key observations. Firstly, training with the cross entropy loss, focal loss, and TCP confidence loss lead to overfitting the imbalanced training samples, \ie the dominance of trustworthy predictions. Specifically, TPR is higher than 99\% while TNR is less than 1\% in all cases. Secondly, the performance of predicting trustworthiness is contingent on both the oracle and the classifier. When a high-performance model (\ie ViT) is used as the oracle and a relatively low-performance model (\ie ResNet) is used as the classifier, cross entropy loss and focal loss achieve higher TNRs than the loss functions with the other combinations. In contrast, the two losses with \textlangle ResNet, ViT\textrangle~ lead to the lowest TNRs (\ie 0\%). %, compared to the cases with the other combinations.

Despite the combinations of oracles and classifiers, the proposed steep slope loss can achieve significantly higher TNRs than using the other loss functions, while it achieves desirable performance on FPR-95\%-TPR, AUPR-Success, and AUC. This verifies that the proposed loss is effective to improve the generalizability for predicting trustworthiness. Note that the scores of AUPR-Error and TPR yielded by the proposed loss are lower than that of the other loss functions. Recall that AUPR-Error aims to inspect how easy to detect failures and depends on the negated trustworthiness confidences w.r.t. incorrect predictions \cite{Corbiere_NIPS_2019}. The AUPR-Error correlates to TPR and TNR. When TPR is close to 100\% and TNR is close to 0\%, it indicates the oracle is prone to view all the predictions to be trustworthy. In other words, almost all the trustworthiness confidences are on the right-hand side of $p(o=1|\theta,\bm{x})=0.5$. Consequently, when taking the incorrect prediction as the positive class, the negated confidences are smaller than -0.5. On the other hand, the oracle trained with the proposed loss intends to yield the ones w.r.t. incorrect predictions that are smaller than 0.5. In general, the negated confidences w.r.t. incorrect predictions are greater than the ones yielded by the other loss functions. In summary, a high TPR score and a low TNR score leads to a high AUPR-Error.

To intuitively understand the effects of all the loss functions, we plot the histograms of trustworthiness confidences w.r.t. true positive (TP), false positive (FP), true negative (TN), and false negative (FN) in \figref{fig:histogram_part}. The result confirms that the oracles trained with the baseline loss functions are prone to predict overconfident trustworthiness for incorrect predictions, while the oracles trained with the proposed loss can properly predict trustworthiness for incorrect predictions.

% On the other hand, the proposed steep slope loss show better generalizability over the three domains, where TPR is 73.62\% and TNR is 47.23\%. Secondly, the learned oracles exhibit consistent separability over the three domains through the lens of KL divergence and Bhttacharya coefficient. This is aligned with the intuition that a model that work well on a domain is likely to work well on other domains. 

\begin{figure}[!t]
	\centering
	\subfloat[\textlangle ViT, ViT\textrangle + CE]{\includegraphics[width=0.24\textwidth]{fig/hist/ce_vit_vit_val}    } \hfill
	\subfloat[\textlangle ViT, ViT\textrangle + Focal]{\includegraphics[width=0.24\textwidth]{fig/hist/focal_vit_vit_val}    } \hfill
	\subfloat[\textlangle ViT, ViT\textrangle + TCP]{\includegraphics[width=0.24\textwidth]{fig/hist/tcp_vit_vit_val}    } \hfill
	\subfloat[\textlangle ViT, ViT\textrangle +  SS]{\includegraphics[width=0.24\textwidth]{fig/hist/ss_vit_vit_val}    } \\
	\subfloat[\textlangle ViT, RSN\textrangle + CE]{\includegraphics[width=0.24\textwidth]{fig/hist/ce_vit_rsn_val}    } \hfill
	\subfloat[\textlangle ViT, RSN\textrangle + Focal]{\includegraphics[width=0.24\textwidth]{fig/hist/focal_vit_rsn_val}    } \hfill
	\subfloat[\textlangle ViT, RSN\textrangle + TCP]{\includegraphics[width=0.24\textwidth]{fig/hist/tcp_vit_rsn_val}    } \hfill
	\subfloat[\textlangle ViT, RSN\textrangle + SS]{\includegraphics[width=0.24\textwidth]{fig/hist/ss_vit_rsn_val}    } \\
% 	\subfloat[\textlangle RSN, ViT\textrangle + CE]{\includegraphics[width=0.24\textwidth]{fig/hist/ce_rsn_vit_val}    } \hfill
% 	\subfloat[\textlangle RSN, ViT\textrangle + Focal]{\includegraphics[width=0.24\textwidth]{fig/hist/focal_rsn_vit_val}    } \hfill
% 	\subfloat[\textlangle RSN, ViT\textrangle + TCP]{\includegraphics[width=0.24\textwidth]{fig/hist/tcp_rsn_vit_val}    } \hfill
% 	\subfloat[\textlangle RSN, ViT\textrangle + SS]{\includegraphics[width=0.24\textwidth]{fig/hist/ss_rsn_vit_val}    } \\
% 	\subfloat[\textlangle RSN, RSN\textrangle + CE]{\includegraphics[width=0.24\textwidth]{fig/hist/ce_rsn_rsn_val}    } \hfill
% 	\subfloat[\textlangle RSN, RSN\textrangle + Focal]{\includegraphics[width=0.24\textwidth]{fig/hist/focal_rsn_rsn_val}    } \hfill
% 	\subfloat[\textlangle RSN, RSN\textrangle + TCP]{\includegraphics[width=0.24\textwidth]{fig/hist/tcp_rsn_rsn_val}    } \hfill
% 	\subfloat[\textlangle RSN, RSN\textrangle + SS]{\includegraphics[width=0.24\textwidth]{fig/hist/ss_rsn_rsn_val}    } \\
	\caption{\label{fig:histogram_part}
    	Histograms of trustworthiness confidences w.r.t. all the loss functions on the ImageNet validation set.
    	The oracles that are used to generate the confidences are the ones used in \tabref{tbl:all_perf_w_std}. The histograms generated with \textlangle RSN, ViT\textrangle and \textlangle RSN, RSN\textrangle are provided in \appref{sec:histogram}.
    % 	appendix \ref{sec:histogram}.
    % 	the cross entropy (first column), focal loss (second column), TCP confidence loss (third column), and the proposed steep slope loss (fourth column) on the ImageNet validation set.
    % 	\REVISION{\textit{Baseline} indicates ResNet GEM.}
    	}
    \vspace{-1ex}
\end{figure}

% \begin{wrapfigure}{r}{0.5\textwidth}
\begin{table}[!t]
	\centering
	\caption{\label{tbl:perf_mnist}
	    Performance on MNIST and CIFAR-10.
	   % We use the official TCP code, but find out that there are several bugs and we couldn't reproduce the performance reported in their paper, not even close. Below are the best results by fixing a few bugs, according to the technical details in the paper.
	}
	\adjustbox{width=1\columnwidth}{
	\begin{tabular}{C{12ex} L{15ex} C{8ex} C{10ex} C{8ex} C{8ex} C{8ex} C{8ex} C{8ex}}
		\toprule
		\textbf{Dataset} & \textbf{Loss} & \textbf{Acc$\uparrow$} & \textbf{FPR-95\%-TPR$\downarrow$} & \textbf{AUPR-Error$\uparrow$} & \textbf{AUPR-Success$\uparrow$} & \textbf{AUC$\uparrow$} & \textbf{TPR$\uparrow$} & \textbf{TNR$\uparrow$} \\
		\cmidrule(lr){1-1} \cmidrule(lr){2-2} \cmidrule(lr){3-3} \cmidrule(lr){4-4} \cmidrule(lr){5-5} \cmidrule(lr){6-6} \cmidrule(lr){7-7} \cmidrule(lr){8-8} \cmidrule(lr){9-9}
		\multirow{6}{*}{MNIST} & MCP \cite{Hendrycks_ICLR_2017} & 99.10 & 5.56 & 35.05 & \textbf{99.99} & 98.63 & 99.89 & \textbf{8.89} \\
		& MCDropout \cite{Gal_ICML_2016} & 99.10 & 5.26 & 38.50 & \textbf{99.99} & 98.65 & - & - \\
		& TrustScore \cite{Jiang_NIPS_2018} & 99.10 & 10.00 & 35.88 & 99.98 & 98.20 & - & - \\
		& TCP \cite{Corbiere_NIPS_2019} & 99.10 & 3.33 & \textbf{45.89} & \textbf{99.99} & 98.82 & 99.71 & 0.00 \\
		& TCP$\dagger$ & 99.10 & 3.33 & 45.88 & \textbf{99.99} & 98.82 & 99.72 & 0.00 \\
		& SS & 99.10 & \textbf{2.22} & 40.86 & \textbf{99.99} & \textbf{98.83} & \textbf{100.00} & 0.00 \\
		\midrule
		\multirow{6}{*}{CIFAR-10} & MCP \cite{Hendrycks_ICLR_2017} & 92.19 & 47.50 & 45.36 & 99.19 & 91.53 & 99.64 & 6.66 \\
		& MCDropout \cite{Gal_ICML_2016} & 92.19 & 49.02 & 46.40 & \textbf{99.27} & 92.08 & - & - \\
		& TrustScore \cite{Jiang_NIPS_2018} & 92.19 & 55.70 & 38.10 & 98.76 & 88.47 & - & - \\
		& TCP \cite{Corbiere_NIPS_2019} & 92.19 & 44.94 & 49.94 & 99.24 & 92.12 & \textbf{99.77} & 0.00 \\
		& TCP$\dagger$ & 92.19 & 45.07 & 49.89 & 99.24 & 92.12 & 97.88 & 0.00 \\
		& SS & 92.19 & \textbf{44.69 }& \textbf{50.28}  & 99.26 & \textbf{92.22} & 98.46 & \textbf{28.04} \\
		\bottomrule	
	\end{tabular}}
\end{table}
% \end{wrapfigure}

% \begin{figure}[!t]
% 	\centering
% 	\subfloat[Official TCP  plot]{\includegraphics[width=0.45\textwidth]{fig/hist/tcphp_mnist_tefeat}    } \hfill
% 	\subfloat[Proposed with pretrained baseline ]{\includegraphics[width=0.45\textwidth]{fig/hist/steephp_mnist_tefeat}    } \\
% 	\subfloat[TCP with trained baseline]{\includegraphics[width=0.45\textwidth]{fig/hist/tcplp_mnist_tefeat}    } \hfill
% 	\subfloat[Proposed with trained baseline ]{\includegraphics[width=0.45\textwidth]{fig/hist/steeplp_mnist_tefeat}    }
% 	\caption{
%     	Reproduction and comparison.
%     	}
% \end{figure}

\noindent\textbf{Separability between Distributions of Correct Predictions and Incorrect Predictions}.
As observed in \figref{fig:histogram_part}, the confidences w.r.t. correct and incorrect predictions follow Gaussian-like distributions.
Hence, we can compute the separability between the distributions of correct and incorrect predictions from a probabilistic perspective.
% There are two common tools to achieve the goal, \ie Kullback–Leibler (KL) divergence \cite{Kullback_AMS_1951} and Bhattacharyya distance \cite{Bhattacharyya_JSTOR_1946}.
Given the distribution of correct predictions {\small $\mathcal{N}_{1}(\mu_{1}, \sigma^{2}_{1})$} and the distribution of correct predictions {\small $\mathcal{N}_{2}(\mu_{2}, \sigma^{2}_{2})$}, we use the average Kullback–Leibler (KL) divergence {\small $\bar{d}_{KL}(\mathcal{N}_{1}, \mathcal{N}_{2})$} \cite{Kullback_AMS_1951} and Bhattacharyya distance {\small $d_{B}(\mathcal{N}_{1}, \mathcal{N}_{2})$} \cite{Bhattacharyya_JSTOR_1946} to measure the separability. 
More details and the quantitative results are reported in \appref{sec:separability}. 
In short, the proposed loss leads to larger separability than the baseline loss functions. 
This implies that the proposed loss is more effective to differentiate incorrect predictions from correct predictions.

\noindent\textbf{Performance on Small-Scale Datasets}.
We also provide comparative experimental results on small-scale datasets, \ie MNIST \cite{Lecun_IEEE_1998} and CIFAR-10 \cite{Krizhevsky_TR_2009}.
\REVISION{The results are reported in \tabref{tbl:perf_mnist}.}
% The experiment details and results are reported in \appref{sec:mnist}.
The proposed loss outperforms TCP$\dagger$ on metric FPR-95\%-TPR on both MNIST and CIFAR-10, and additionally achieved good performance on metrics AUPR-Error and TNR on CIFAR-10.
This shows the proposed loss is able to adapt to relatively simple data.
\REVISION{More details can be found in \appref{sec:mnist}.}

\noindent\textbf{Generalization to Unseen Domains}.
In practice, the oracle may run into the data in the domains that are different from the ones of training samples.
Thus, it is interesting to find out how well the learned oracles generalize to the unseen domain data.
% To this end, we apply a style transfer method \cite{Geirhos_ICLR_2019} and the functional adversarial attack method \cite{Laidlaw_NeurIPS_2019} to generate the stylized ImageNet validation set and the adversarial ImageNet validation set.
Using the oracles trained with the ImageNet training set (\ie the ones used in \tabref{tbl:all_perf_w_std}), we evaluate it on the stylized ImageNet validation set \cite{Geirhos_ICLR_2019}, adversarial ImageNet validation set \cite{Laidlaw_NeurIPS_2019}, and corrupted ImageNet validation set \cite{Hendrycks_ICLR_2018}.
% and evaluated on the two variants of the validation set.
\textlangle ViT, ViT\textrangle~ is used in the experiment.

The results on the stylized ImageNet, adversarial ImageNet, and ImageNet-C are reported in \tabref{tbl:perf_vit_vit}, \REVISION{More results on ImageNet-C are reported in \tabref{tbl:perf_imagenetc}}.
As all unseen domains are different from the one of the training set, the classification accuracies are much lower than the ones in \tabref{tbl:all_perf_w_std}. 
The adversarial validation set is also more challenging than the stylized validation set \REVISION{and the corrupted validation set}.
As a result, the difficulty affects the scores across all metrics.
The oracles trained with the baseline loss functions are still prone to recognize the incorrect prediction to be trustworthy.
The proposed loss consistently improves the performance on FPR-95\%-TPR, AUPR-Sucess, AUC, and TNR.
Note that the adversarial perturbations are computed on the fly \cite{Laidlaw_NeurIPS_2019}. Instead of truncating the sensitive pixel values and saving into the images files, we follow the experimental settings in \cite{Laidlaw_NeurIPS_2019} to evaluate the oracles on the fly.
Hence, the classification accuracies w.r.t. various loss function are slightly different but are stably around 6.14\%.

% Also, we report the performances on each domain in \tabref{tbl:perf_vit_vit} and \tabref{tbl:perf_rsn_vit}.
% They shows that the cross entropy and focal loss work well on the regular validation set, but work poorly on the stylized and adversarial validation sets. This confirms the overfitting resulted from the learning with the cross entropy and focal loss.

\begin{table}[!t]
	\centering
	\vspace{-1ex}
	\caption{\label{tbl:perf_vit_vit}
	   % Histograms of trustworthiness confidences w.r.t. all the loss functions on the stylized ImageNet validation set (stylized val) and the adversarial ImageNet validation set (adversarial val). \textlangle ViT, ViT\textrangle is used in the experiment and the domains of the two validation sets are different from the one of the training set that is used for training the oracle.
	    Performance on the stylized ImageNet validation set, the adversarial ImageNet validation set, and one (Defocus blur) of validation sets in ImageNet-C. Defocus blus is at at the highest level of severity.
	    \textlangle ViT, ViT\textrangle~ is used in the experiment and the domains of the two validation sets are different from the one of the training set that is used for training the oracle. The corresponding histograms are available in \appref{sec:histogram}. More results on ImageNet-C can be found in \tabref{tbl:perf_imagenetc}.
	   % In this experiment, ViT is used for both the oracle backbone and the classifier. The oracle is trained with the CE loss, the focal loss, and the proposed steep slope loss on the ImageNet training set. The resulting oracles w.r.t. each loss are evaluated on the three validation sets. The classifier is used in the evaluation mode in the experiment.
	}
	\adjustbox{width=1\columnwidth}{
	\begin{tabular}{C{15ex} L{10ex} C{8ex} C{10ex} C{8ex} C{8ex} C{8ex} C{8ex} C{8ex}}
		\toprule
		\textbf{Dataset} & \textbf{Loss} & \textbf{Acc$\uparrow$} & \textbf{FPR-95\%-TPR$\downarrow$} & \textbf{AUPR-Error$\uparrow$} & \textbf{AUPR-Success$\uparrow$} & \textbf{AUC$\uparrow$} & \textbf{TPR$\uparrow$} & \textbf{TNR$\uparrow$} \\
		\cmidrule(lr){1-1} \cmidrule(lr){2-2} \cmidrule(lr){3-3} \cmidrule(lr){4-4} \cmidrule(lr){5-5} \cmidrule(lr){6-6} \cmidrule(lr){7-7} \cmidrule(lr){8-8} \cmidrule(lr){9-9}
% 		& \multicolumn{7}{c}{Regular validation set} \\
% 		\cmidrule(lr){1-1} \cmidrule(lr){2-8}
% 		CE & 83.90 & 92.83 & 15.08 & 84.99 & 52.78 & 100.00 & 0.01 \\
% 		Focal & 83.90 & 92.68 & 14.69 & 85.46 & 53.47 & 99.06 & 1.61 \\
% 		TCP & 83.90 & 88.07 & 12.86 & 87.80 & 60.45 & 99.72 & 1.02 \\
% % 		TCP & 83.90 & 86.45 & 12.12 & 88.95 & 63.39 & 99.07 & 3.06 \\
% 		SS & 83.90 & 80.89 & 10.31 & 92.90 & 73.31 & 88.44 & 35.64 \\
% 		\midrule
% 		& \multicolumn{7}{c}{Stylized validation set} \\
% 		\cmidrule(lr){1-1} \cmidrule(lr){2-8}
		\multirow{4}{*}{Stylized \cite{Geirhos_ICLR_2019}} & CE & 15.94 & 95.52 & 84.18 & 15.86 & 49.07 & \textbf{99.99} & 0.02 \\
		& Focal \cite{Lin_ICCV_2017} & 15.94 & 95.96 & \textbf{85.90} & 14.30 & 46.01 & 99.71 & 0.25 \\
		& TCP \cite{Corbiere_NIPS_2019} & 15.94 & 93.42 & 80.17 & 21.25 & 57.29 & 99.27 & 0.00 \\
% 		& TCP & 15.94 & 93.19 & 78.53 & 24.52 & 60.31 & 95.41 & 6.24 \\
		& SS & 15.94 & \textbf{89.38} & 75.08 & \textbf{34.39} & \textbf{67.68} & 44.42 & \textbf{81.22} \\
        \midrule
% 		& \multicolumn{7}{c}{Adversarial validation set} \\
% 		\cmidrule(lr){1-1} \cmidrule(lr){2-8}
        \multirow{4}{*}{Adversarial \cite{Laidlaw_NeurIPS_2019}} & CE & 6.14 & 94.35 & \textbf{93.70} & 6.32 & 51.28 & \textbf{99.97} & 0.06 \\
        & Focal \cite{Lin_ICCV_2017} & 6.15 & 93.67 & 93.48 & 6.56 & 52.39 & 99.06 & 1.43 \\
        & TCP \cite{Corbiere_NIPS_2019} & 6.11 & 93.94 & 92.77 & 7.55 & 55.81 & 99.71 & 0.00 \\
        & SS  & 6.16 & \textbf{90.07} & 90.09 & \textbf{13.07} & \textbf{65.36} & 87.10 & \textbf{24.33} \\ \midrule
        \multirow{4}{*}{Defocus blur \cite{Hendrycks_ICLR_2018}} & CE & 31.83 & 94.46 & \textbf{68.56} & 31.47 & 50.13 & \textbf{99.15} & 1.07 \\
		& Focal \cite{Lin_ICCV_2017} & 31.83 & 94.98  & 66.87 & 33.24 & 51.28 & 96.70 & 3.26 \\
		& TCP \cite{Corbiere_NIPS_2019} & 31.83 & 93.50 & 64.67 & 36.05 & 54.27 & 96.71 & 4.35 \\
		& SS & 31.83 & \textbf{90.18} & 57.95 & \textbf{48.80} & \textbf{64.34} & 77.79 & \textbf{37.29} \\
		\bottomrule	
	\end{tabular}}
\end{table}

\begin{figure}[!b]
	\centering
	\subfloat[]{\includegraphics[width=0.32\textwidth]{fig/risk/risk_vit_vit} \label{fig:risk_vit}} \hfill
	\subfloat[]{\includegraphics[width=0.30\textwidth]{fig/analysis/loss} \label{fig:abl_loss}} \hfill
	\subfloat[]{\includegraphics[width=0.32\textwidth]{fig/analysis/tpr_tnr} \label{fig:abl_tpr_tnr}} 
	\caption{\label{fig:anal_abl}
    	Analyses based on \textlangle ViT, ViT\textrangle. (a) are the curves of risk vs. coverage. Selective risk represents the percentage of errors in the remaining validation set for a given coverage. (b) are the curves of loss vs. $\alpha^{-}$. (c) are TPR and TNR against various $\alpha^{-}$.
    	}
\end{figure}

\noindent\textbf{Selective Risk Analysis}.
Risk-coverage curve is an important technique for analyzing trustworthiness through the lens of the rejection mechanism in the classification task \cite{Corbiere_NIPS_2019,Geifman_NIPS_2017}. 
In the context of predicting trustworthiness, selective risk is the empirical loss that takes into account the decisions, \ie to trust or not to trust the prediction. 
Correspondingly, coverage is the probability mass of the non-rejected region. As can see in \figref{fig:risk_vit}, the proposed loss leads to significantly lower risks, compared to the other loss functions.
We present the risk-coverage curves w.r.t. all the combinations of oracles and classifiers in \appref{sec:risk}.
They consistently exhibit similar pattern.

\noindent\textbf{Ablation Study}.
In contrast to the compared loss functions, the proposed loss has more hyperparameters to be determined, \ie $\alpha^{+}$ and $\alpha^{-}$.
As the proportion of correct predictions is usually larger than that of incorrect predictions, we would prioritize $\alpha^{-}$ over $\alpha^{+}$ such that the oracle is able to recognize a certain amount of incorrect predictions.
In other words, we first search for $\alpha^{-}$ by freezing $\alpha^{+}$, and then freeze $\alpha^{-}$ and search for $\alpha^{+}$.
\figref{fig:abl_loss} and \ref{fig:abl_tpr_tnr} show how the loss, TPR, and TNR vary with various $\alpha^{-}$. In this analysis, the combination \textlangle ViT, ViT\textrangle~ is used and $\alpha^{+}=1$.
We can see that $\alpha^{-}=3$ achieves the optimal trade-off between TPR and TNR.
We follow a similar search strategy to determine $\alpha^{+}=2$ and $\alpha^{-}=5$ for training the oracle with ResNet backbone.
% With the classifier ViT and the ViT based oracle, we show how the performance vary when $\alpha^{+}$ and $\alpha^{-}$ change.  

\noindent\textbf{Effects of Using $z=\bm{w}^{\top}\bm{x}^{out}+b$}.
Using the signed distance as $z$, \ie $z = \frac{\bm{w}^{\top} \bm{x}^{out}+b}{\|\bm{w}\|}$, has a geometric interpretation as shown in \figref{fig:workflow_a}.
However, the main-stream models \cite{He_CVPR_2016,Tan_ICML_2019,Dosovitskiy_ICLR_2021} use $z=\bm{w}^{\top}\bm{x}^{out}+b$. 
Therefore, we provide the corresponding results in appendix \ref{sec:appd_z}, which are generated by the proposed loss taking the output of the linear function as input.
In comparison with the results of using $z = \frac{\bm{w}^{\top} \bm{x}^{out}+b}{\|\bm{w}\|}$, using $z=\bm{w}^{\top}\bm{x}^{out}+b$ yields comparable scores of FPR-95\%-TPR, AUPR-Error, AUPR-Success, and AUC.
Also, TPR and TNR are moderately different between $z = \frac{\bm{w}^{\top} \bm{x}^{out}+b}{\|\bm{w}\|}$ and $z=\bm{w}^{\top}\bm{x}^{out}+b$, when $\alpha^{+}$ and $\alpha^{-}$ are fixed.
This implies that TPR and TNR are sensitive to $\|\bm{w}\|$. 
% \REVISION{We discuss the reason in \appref{sec:effect_normalization}.}
% 
\REVISION{
This is because the normalization by $\|w\|$ would make $z$ more dispersed in value than the variant without normalization. 
In other words, the normalization leads to long-tailed distributions while no normalization leads to short-tailed distributions. 
Given the same threshold, TNR (TPR) is determined by the location of the distribution of negative (positive) examples and the extent of short/long tails. 
Our analysis on the histograms generated with and without $\|w\|$ normalization verifies this point.
}

% \noindent\textbf{Learning with Class Weights}. We witness the imbalancing characteristics in the learning task for predicting trustworthiness. Table xx shows that one of most common learning techniques with imbalanced data, \ie using class weights, is not effective. The reason is that applying class weights to the loss function, \eg cross entropy, it only scale up the graph along y-axis. However, the long tail regions still slow down the move of the features w.r.t. false positive or false negative towards the well-classified regions.

% \noindent\textbf{Separability between Distributions of Correct Predictions and Incorrect Predictions}.
% As observed in \figref{fig:histogram_part}, the confidences w.r.t. correct and incorrect predictions follow Gaussian-like distributions.
% Hence, we can compute the separability between the distributions of correct and incorrect predictions from a probabilistic perspective.
% % There are two common tools to achieve the goal, \ie Kullback–Leibler (KL) divergence \cite{Kullback_AMS_1951} and Bhattacharyya distance \cite{Bhattacharyya_JSTOR_1946}.
% Given the distribution of correct predictions $\mathcal{N}_{1}(\mu_{1}, \sigma^{2}_{1})$ and the distribution of correct predictions $\mathcal{N}_{2}(\mu_{2}, \sigma^{2}_{2})$, we use the average Kullback–Leibler (KL) divergence $\bar{d}_{KL}(\mathcal{N}_{1}, \mathcal{N}_{2})$ \cite{Kullback_AMS_1951} and Bhattacharyya distance $d_{B}(\mathcal{N}_{1}, \mathcal{N}_{2})$ \cite{Bhattacharyya_JSTOR_1946} to measure the separability. More details and the quantitative results are reported in \appref{sec:separability}. In short, the proposed loss leads to larger separability than the baseline loss functions. This implies that the proposed loss is more effective to differentiate incorrect predictions from correct predictions.

\noindent\textbf{Steep Slope Loss vs. Class-Balanced Loss}.
We compare the proposed loss to the class-balanced loss \cite{Cui_CVPR_2019}, which is based on a re-weighting strategy.
The results are reported in \appref{sec:cbloss}.
Overall, the proposed loss outperforms the class-balanced loss, which implies that the imbalance characteristics of predicting trustworthiness is different from that of imbalanced data classification.

% KL divergence is used to measure the difference between two distributions \cite{Cantu_Springer_2004,Luo_TNNLS_2020}, while Bhattacharyya distance is used to measure the similarity of two probability distributions. Given two Gaussian distributions $\mathcal{N}_{1}(\mu_{1}, \sigma^{2}_{1})$ and $\mathcal{N}_{2}(\mu_{2}, \sigma^{2}_{2})$, we use the averaged KL divergence, \ie $\bar{d}_{KL}(\mathcal{N}_{1}, \mathcal{N}_{2}) = (d_{KL}(\mathcal{N}_{1}, \mathcal{N}_{2}) + d_{KL}(\mathcal{N}_{2}, \mathcal{N}_{1}))/2$, where $d_{KL}(\mathcal{N}_{1}, \mathcal{N}_{2})=\log\frac{\sigma_{2}}{\sigma_{1}}+\frac{\sigma_{1}^{2}+(\mu_{1}-\mu_{2})^{2}}{2\sigma_{2}^{2}}-\frac{1}{2}$ is not symmetrical. On the other hand, Bhattacharyya distance is defined as $d_{B}(\mathcal{N}_{1}, \mathcal{N}_{2})=\frac{1}{4}\ln \left( \frac{1}{4} \left( \frac{\sigma^{2}_{1}}{\sigma^{2}_{2}}+\frac{\sigma^{2}_{2}}{\sigma^{2}_{1}}+2 \right) \right) + \frac{1}{4} \left( \frac{(\mu_{1}-\mu_{2})^{2}}{\sigma^{2}_{1}+\sigma^{2}_{2}} \right)$. A larger $\bar{d}_{KL}$ or $d_{B}$ indicates that the two distributions are further away from each other.


% We hypothesize that $x$ w.r.t. positive and negative samples both follow Gaussian distributions. The discriminativeness of features is an important characteristic that correlates to the performance, \eg accuracy. We are interested in measures of separability of feature distributions, which reflect the discriminativeness from a probabilistic perspective. There are two common tools to achieve the goal, \ie Kullback–Leibler (KL) divergence \cite{Kullback_AMS_1951} and Bhattacharyya distance \cite{Bhattacharyya_JSTOR_1946}. Usually, KL divergence is used to measure the difference between two distributions \cite{Cantu_Springer_2004,Luo_TNNLS_2020}, while Bhattacharyya distance is used to measure the similarity of two probability distributions. Given two Gaussian distributions $\mathcal{N}_{1}(\mu_{1}, \sigma^{2}_{1})$ and $\mathcal{N}_{2}(\mu_{2}, \sigma^{2}_{2})$, we use an averaged KL divergence as in this work, \ie $\bar{d}_{KL}(\mathcal{N}_{1}, \mathcal{N}_{2}) = (d_{KL}(\mathcal{N}_{1}, \mathcal{N}_{2}) + d_{KL}(\mathcal{N}_{2}, \mathcal{N}_{1}))/2$, where $d_{KL}(\mathcal{N}_{1}, \mathcal{N}_{2})$ is the KL divergence between $\mathcal{N}_{1}$ and $\mathcal{N}_{2}$ (not symmetrical). On the other hand, Bhattacharyya distance is defined as $d_{B}(\mathcal{N}_{1}, \mathcal{N}_{2})=\frac{1}{4}\ln \left( \frac{1}{4} \left( \frac{\sigma^{2}_{1}}{\sigma^{2}_{2}}+\frac{\sigma^{2}_{2}}{\sigma^{2}_{1}}+2 \right) \right) + \frac{1}{4} \left( \frac{(\mu_{1}-\mu_{2})^{2}}{\sigma^{2}_{1}+\sigma^{2}_{2}} \right)$. In this work, we use Bhattacharyya coefficient that measures the amount of overlap between two distributions, instead of Bhattacharyya distance. Bhattacharyya coefficient is defined as $c_{B}(\mathcal{N}_{1}, \mathcal{N}_{2}) = \exp(-d_{B}(\mathcal{N}_{1}, \mathcal{N}_{2}))$. $c_{B} \in [0,1]$, where 1 indicates a full overlap and 0 indicates no overlap.

% \noindent\textbf{Semantics Difference between Predicting Trustworthiness and Classification}. As we use ViT for both the oracle and classifier, it is interesting to find out what features are leaned for predicting trustworthiness, in comparison to the features learned for classification. Hence, we compute the $l_{1}$ and $l_{2}$ distances between the features generated by the learned oracle and the features generated by the pre-trained classifier. The features are the inputs to the last layer of ViT, \ie 768-dimensional vectors.

% The mean and standard deviation of distances over all the samples in the training and validation sets are provided in \tabref{tbl:anal_diff}. Note that a smaller distance indicates higher similarity between two features. Overall, the mean of distances w.r.t. the three loss functions are large, but the focal loss yields the smallest averaged distance, which implies that the oracle learned with the focal yields the most similar features as the ones generated by the pre-trained classifier. One of possible reasons is that the focal loss prohibits the oracle training.

% Comparison of classifier backbone and oracle backbone

% Per class accuracy, precision, recall, F1

% \noindent\textbf{Taking Features as Input}
% \figref{fig:anal_featinput} shows the distributions of discriminative features generated by a multi-layer perceptron (MLP),, which plays as an oracle. The MLP takes the features generated by the classifier, instead of images, as input. The MLP-based oracle is training on the training set and is evaluated on the validation set. The figure shows that the oracle barely distinguish between positives and negatives. Because all the features are on the right-hand side of the decision boundary $x=0$.

% focal loss vs proposed

% \begin{figure}[!t]
	\centering
	\subfloat{\includegraphics[width=0.32\textwidth]{fig/analysis/anal_featinput_ce}    } \hfill
	\subfloat{\includegraphics[width=0.32\textwidth]{fig/analysis/anal_featinput_focal}    } \hfill
	\subfloat{\includegraphics[width=0.32\textwidth]{fig/analysis/anal_featinput_ss}    } \\
	\caption{\label{fig:anal_featinput}
    	Analysis of taking the features of the classifier as input to the oracle on the ImageNet validation set. In this experiment, ViT is used for both the oracle backbone and the classifier. The features are 768-dimensional vectors. The classifier is used in the evaluation mode in the experiment.
    % 	\REVISION{\textit{Baseline} indicates ResNet GEM.}
    	}
\end{figure}

% \begin{table}[!t]
	\centering
	\caption{\label{tbl:anal_diff}
	    Analysis of the difference of the output features between the classifier backbone and the oracle backbone in terms of $l_{1}$ and $l_{2}$ distances. The common backbone is ViT. The oracle backbone is trained for predicting trustworthiness, while the classifier backbone is pre-trained for classification.
	}
	\adjustbox{width=1\columnwidth}{
	\begin{tabular}{L{7ex} C{14ex} C{14ex} C{14ex} C{14ex}}
		\toprule
		& \multicolumn{2}{c}{Training} & \multicolumn{2}{c}{Validation} \\
		\cmidrule(lr){2-3} \cmidrule(lr){4-5}
		Loss & $l_{1}$ & $l_{2}$ & $l_{1}$ & $l_{2}$ \\
		\cmidrule(lr){1-1} \cmidrule(lr){2-2} \cmidrule(lr){3-3} \cmidrule(lr){4-4} \cmidrule(lr){5-5}
		CE & 74.0674$\pm$23.9773 & 3.4074$\pm$1.0967 & 78.4107$\pm$24.9338 & 3.6051$\pm$1.1402 \\
        Focal & 29.0901$\pm$8.5641 & 1.3527$\pm$0.3933 & 30.6497$\pm$8.9262 & 1.4240$\pm$0.4100 \\
        SS & 70.1997$\pm$32.8220 & 3.2129$\pm$1.4973 & 77.3162$\pm$33.4536 & 3.5378$\pm$1.5271 \\
		\bottomrule	
	\end{tabular}}
\end{table}

% \noindent\textbf{Ablation Study}. With the classifier ViT and the ViT based oracle, we show how the performance vary when $\alpha^{+}$ and $\alpha^{-}$ change.  

% \noindent\textbf{Generalization to Unseen Classifier}.
% As the oracle is trained by observing what a classifier predicts the label for an image, the knowledge learned in this way highly correlates to the behaviours of the classifier. It is interesting to know how the knowledge learned by the oracle generalizes to other unseen classifiers. To this end, we use the ViT based oracle that is trained with a ViT classifier to predict the trustworthiness of a ResNet-50 on the adversarial validation set, which is the most challenging in the three sets. 
% For the proposed loss, we use $\alpha^{+}=1$ and $\alpha^{-}=3$ for the oracle that is based on ViT's backbone, while we use $\alpha^{+}=2$ and $\alpha^{-}=5$ for the oracle that is based on ResNet's backbone.



\section{Conclusion}
Our theoretical results prove limitations on the ability of any edge independent graph generative model to produce networks that match the high triangle densities of real-world graphs, while still generating a diverse set of networks, with low model overlap. These results match empirical findings that popular edge independent models indeed systematically underestimate triangle density, clustering coefficient, and related measures. Despite the popularity of edge independent models, many non-independent models, such as graph RNNs \cite{YouYingRen:2018} have been proposed. An interesting future direction would be to study the representative power and limitations of such models, giving general theoretical results that provide a foundation for the study of graph generative models.

\clearpage
\bibliographystyle{plain}
\bibliography{neurips_2021}

\clearpage
\appendix

\section{Exact Embeddings in the CELL Model}

Recently, Rendsburg et al \cite{rendsburgnetgan} propose the CELL graph generator: a major simplification of the NetGAN algorithm for \cite{bojchevski2018netgan}, which gives comparable performance, much faster runtimes, and helps clarify the key components of  the generator. CELL uses a simple low-rank factorization model. Here we prove that, when its rank parameter is $k$, the CELL model can `memorize' any graph with degree bounded by $O(k)$. This allows the model to trivially produce distributions with very high expected triangle densities. However, as our main results show, this inherently requires memorization and high overlap. 

Our result can be viewed as an extension of the results of \cite{ChanpuriyaMuscoSotiropoulos:2020}, which considers a different edge independent model. The proof techniques are very similar.
Interestingly, our result seem to indicate that the good generalization of CELL in link prediction tests may mostly be due to the fact that this model is not fully optimized, to the point of memorizing the input. %It is not due to regularization or bounded expressiveness of the model itself.
%
% which seems to generate random networks with comparable properties to an seed input graph and exhibits good generalization in link prediction tests -- i.e., given a subset of training edges, the algorithm generates a graph which contains the remaining edges of the graph with good probability.

%Here we argue that any  generalization performance  of CELL is based on incomplete optimization. If the model is fully  optimized, it can produce a very low-dimension embedding that generates the true input graph with probability $1$, obviating its usefulness as a graph generator and any generalization properties.

%\todo{Insert more complete description of CELL.}

\noindent \textbf{The CELL Model.} We first describe the CELL model introduced in \cite{rendsburgnetgan}.
\begin{enumerate}
\item Given a graph adjacency matrix $A \in \{0,1\}^{n \times n}$, let
\begin{align}\label{eq:cell}
W^\star = \min_{\substack{W \in \R^{n \times n} \\\rank(W) \le k}}\quad \sum_{i,j=1}^n A_{ij} \log \sigma_{rows}(W)_{ij},
\end{align}
where $\sigma_{rows}(W)$ applies a softmax rowwise to $W$ -- ensuring that each row of $\sigma_{rows}(W)$ sums to $1$. 
\item Let $P^\star = \sigma_{rows}(W^\star)$ and let $\pi \in \R^n$ be the eigenvector satisfying $\pi^T P^\star = \pi^T$.
\item Let $P = \max(diag(\pi) P^*, (diag(\pi) P^\star)^T)$.
\item Generate $G \sim G(P)$.
\end{enumerate}
Note that the last step described above is slightly different than the approach taken in CELL. Rather than use an edge-independent model as in Def. \ref{def:ei}, they form $G$ by sampling edges without replacement, with probability proportional to the entries in $P$. They also insure that at least one edge is sampled adjacent to every node. However, this distinction is minor.

\noindent \textbf{Unconstrained Optimum.}
We first show that, if the rank constraint in \eqref{eq:cell} is removed, then the optimal $W^\star$ has $\sigma_{rows}(W^\star) = P^\star =  D^{-1} A$, where $D$ is the diagonal degree matrix. At this minimum, we can check that $\pi_i = d_i$, the degree of the $i^{th}$ node, and thus $diag(\pi)  = D$ and $P = A$. That is, the model simply outputs the input graph with probability $1$.  
\begin{theorem}[CELL Optimum]\label{thm:piecewise}
The unconstrained CELL objective function \eqref{eq:cell} is minimized when $\sigma_{rows}(W) = D^{-1} A$. At this minimum, the edge independent model $P$ is simply $A$. That is, the model just returns the input graph with probability $1$.
\end{theorem}
\begin{proof}
%\todo{Prove this. I am confident it should hold. Proof will first I think use the fact that $\log$ is convex? This should give that $\sigma_{rows}(W) = P$ is the unconstrained minimizer. Since basically in row $i$ you should want to balance the large values of the softmax to all be $1/d_i$. Proving that at the minimum $A^\dagger = A$ is just working through the steps of generating $A^\dagger$ when $\sigma_{rows}(W) = P$.}
It suffices to consider the $i^{th}$ row of $W$ for each $i \in [n]$, since the objective function of \eqref{eq:cell} breaks down rowwise. Let $w_i,a_i \in \R^n$ be the $i^{th}$ rows of $\sigma_{rows}(W)$ and $A$ respectively. Note that $w_i$ is a probability vector, with $w_i(j) \ge 0$ for all $j$ and $\sum_{j=1}^n w_i(j) = 1$.

We seek to minimize $\sum_{j=1}^n A_{ij} \log [w_i(j)].$ We need to show that this objective is minimized when $w_i = 1/d_i \cdot a_i$ -- i.e., when $w_i$ places mass $1/d_i$ at each nonzero entry in $a_i$ $1/d_i \cdot a_i$ is the $i^{th}$ row of $D^{-1}A$, so applying this argument to all $i$ gives that $\sigma_{rows}(W) = D^{-1} A$ is the overall minimizer. Assume for the sake of contradiction that there is some other minimizer $w^\star \neq 1/d_i \cdot a_i$. Since $\sum_{j=1}^n w^\star(j) = 1$, we must have $w^\star(j) \le 1/d_i$ for some $j$ where $a_i = 1$. In turn, there must be some $j'$ with either (1) $w^\star(j') \ge 0$ and $a_i(j') = 0$ or (2) $w^\star(j') \ge 1/d_i$ and $a_i(j') = 1$. In case (1), clearly moving $w^\star(j')$ mass from $j'$ to $j$ will decrease the objective function. In case (2), due to the concavity of the log function, moving $w^\star(j') - 1/d_i$ mass from $j'$ to $j$ will also decrease the objective function. Thus, $w^\star$ cannot be a minimizer, completing the proof.
%
%We know that $\sigma_{rows}(W)_{ij} \le 1$ for all $j$ and further that $\sum_{j=1}^n \sigma_{rows}(W)_{ij} = 1$. 
\end{proof}

\noindent \textbf{Rank-Constrained Optimum.}
We next show that the unconstrained optimum of $\sigma_{rows}(W) = D^{-1} A$, which leads to CELL memorizing the input graph (Thm. \ref{thm:piecewise}) can be achieved even with the rank constraint of \eqref{eq:cell}, as long as $k \ge 2\Delta+1$, where $\Delta$ is the maximum degree of the input graph. 
%We next show that we can achieve an essentially exact factorization.
\begin{theorem}[CELL Exact Factorization]\label{thm:exact}
If $A$ is an adjacency matrix with maximum degree $\Delta$, there is a rank $2\Delta+1$ matrix $W$ with 
$$\sigma_{rows}(W) = D^{-1}A + E$$
where $\norm{E}_2 \le \epsilon$. Note that  the rank of $W$ does not depend on $\epsilon$, and so we can drive $\epsilon \rightarrow 0$ and find a rank-$2\Delta+1$ $W$ which is arbitrarily close to minimizing \eqref{eq:cell} and thus produces $P$ which is arbitrarily close to $A$.
\end{theorem}
%
%Note that in the above conjecture, the rank of $W$ does not depend on $\epsilon$, and so we can drive $\epsilon$ arbitrarily small and find rank-$2\Delta$ $W$ with is arbitrarily close to minimizing \eqref{eq:cell} and thus produces $A^\dagger$ which is arbitrarily close to $A$. \todo{Maybe include a corollary on this?}
%
%Note that in the NetGAN Without GAN paper, the rank parameter generally tends to be chosen way higher than $2\Delta$. See Section C.8 here: \url{https://proceedings.icml.cc/static/paper_files/icml/2020/4540-Supplemental.pdf}. Thus assuming the Theorem is correct, it would imply that any generalization here is just due to not fully optimizing the objective.
%
\begin{proof}

Let $V \in \R^{n \times 2\Delta+1}$ be the Vandermonde matrix with $V_{t,j} = t^{j-1}$. For any $x \in \R^{2\Delta+1}$, $[Vx](t) = \sum_{j = 1}^{2\Delta+1} x({j}) \cdot t^{j-1}$. That is: $Vx$ is a degree $2\Delta$  polynomial evaluated at the integers $t = 1,\ldots,n$.

Let  $a_i$ be the $i^{th}$ row of $A$. Note that $a_i$ has at most $\Delta$ nonzeros whose positions we denote by $t_1,t_2,\ldots,t_{d_i}$.
To prove the theorem, for each row $a_i$, we will construct a polynomial $Vx_i$ which has the \emph{same positive value} at each $t_1,t_2,\ldots,t_{d_i}$ and is negative all all other integers $t$. Then, we will let $X \in \R^{2\Delta +1 \times n}$ be the matrix with columns $x_{i}$ and $W = (VX)^T $. Note that $\rank(W) \le 2\Delta+1$, and $W$ is equal to a fixed positive value whenever A is one and negative whenever it is zero. If we scale $W$ by a very large number (which does not affect its rank), we will have $\sigma_{rows}(W)$ arbitrarily close to $D^{-1} A$, since the rowwise softmax will place equal probability on each positive entry in row $i$ of $W$ and arbitrarily close to $0$ probability on each negative. So the row will exactly have $d_i$ at the nonzero entries of $a_i$, entries each equal to $1/d_i$.

It remains to exhibit the polynomial need to construct $W$. We start by constructing a polynomial of degree $2\Delta$ that is positive on each nonzero position $t_1,t_2,\ldots,t_{d_i}$ of $a_i$ and negative at all other indices. Later we will modify this polynomial to have the same positive value at each nonzero position of $a_i$. %. %As a warm-up, We first show that we can choose $x_i$ so that $sign(V x_i) = a_i$ where the $sign$ function is applied entrywise, and outputs $0$ for negative numbers, $1$ for non-negative numbers. 
%To do this we equivalently must find a polynomial which is positive at all integers $t$ with $a_i(t) = 1$ and negative at all $t$ with $a_i(t) = 0$.  Say $a_i$ is nonzero at positions $t_1,t_2,\ldots,t_{d_i}$.
 Let $r_{j,L}$ and $r_{j,U}$ be any values with $t_{j} -1 < r_{j,L} < t_j$ and $t_{j} < r_{j,U} < t_{j} +1$. Consider the polynomial with roots at each $r_{j,L}$ and $r_{j,U}$ -- this polynomial has $2 d_i \le 2\Delta$ roots and so degree at most $2\Delta$. It will flip signs just at each  $r_{j,L}$ and $r_{j,U}$, and will in fact have the same sign at  $t_1,t_2,\ldots,t_{d_i}$ (either positive or negative). Simply negativing the coefficients we can ensure that this sign is positive, while it is negative at all other indices, giving the result. % the sign to be positive, we have the result. %This argument gives Theorem 6 of \cite{chanpuriya2020node}.


The polynomial above can be written as $p(t) = \prod_{j=1}^{d_i} (t-r_{j,U}) (t-r_{j,L})$. Choose $r_{j,U} = t_j + \epsilon w_j$ and $r_{j,L} = t_j - \epsilon w_j$, where $\epsilon$ is arbitrarily small and $w_j$ is a weight chosen specifically for $t_j$ which we'll set later. We have for any $k = 1,\cdots, d_i$,
\begin{align*}
\lim_{\epsilon \rightarrow 0} \frac{p(t_k)}{\epsilon^2} &= \lim_{\epsilon \rightarrow 0} \frac{\prod_{j=1}^\Delta (t_k-t_j+\epsilon w_j) (t_k-t_j - \epsilon w_j)}{\epsilon^2}\\
& =  \lim_{\epsilon \rightarrow 0} \frac{- \epsilon^2 w_k^2 \cdot \prod_{j\neq k} (t_k - t_j)^2}{\epsilon^2}\\
& = -w_k^2 \cdot \prod_{j\neq k} (t_k - t_j)^2.
\end{align*}
This, if we set $w_k = \frac{1}{\prod_{j\neq k} (t_k - t_j)}$, in the limit as $\epsilon \rightarrow 0$ we will have $p(t_k)/\epsilon^2 = -1$. If we negate and scale the polynomially appropriately (which doesn't change its degree) we will have $p(t_k)$ arbitrarily close to one for each nonzero index $t_k$, and negative for each zero index. This gives the theorem.
\end{proof}

\end{document}
