
\documentclass[10pt]{article}


\usepackage{amsmath,amssymb,amsfonts}

\usepackage{tikz}
\usepackage{listings}
%\usepackage{subfig}
\usepackage{subfigure}
\lstloadlanguages{Matlab}

\usepackage{curves}
\usepackage{xypic}  %commu
\usepackage[mathscr]{eucal}
\usepackage[pdftex]{hyperref}
\usepackage[all]{xy}
%\usepackage{showkeys}
\usepackage{graphics,graphicx}
\usepackage{epsfig}
%\usepackage[dvips]{epsfig}
\usepackage{color}
\definecolor{red-}{rgb}{1.0,0.0,0.0}
\definecolor{grey}{rgb}{0.6, 0.6, 0.6}
\definecolor{brown}{rgb}{0.5,0.2,0.0}
\definecolor{brown-}{rgb}{0.0,0.1,1.0}
\definecolor{green-}{rgb}{0.0, 0.6, 0.0}
\definecolor{gold}{rgb}{0.8,0.7,0.0}
\definecolor{black}{rgb}{0.0,0.0,0.0}
\definecolor{DarkGreen}{rgb}{0.0,0.3,0.2}
\definecolor{LightGreen}{rgb}{0.8,1.0, 0.8}
\definecolor{yellow}{rgb}{0.9,0.9,0.0}
\definecolor{blue-}{rgb}{0.0,0.1,1.0}

\def\red#1{{\textcolor{red-}{#1}}}
\def\grey#1{{\textcolor{grey}{#1}}}
\def\green#1{{\textcolor{green-}{#1}}}
\def\dgreen#1{{\textcolor{DarkGreen}{#1}}}
\def\brown#1{{\textcolor{brown-}{#1}}}
\def\gold#1{{\textcolor{gold}{\bf #1}}}
\def\black#1{{\textcolor{black}{\bf #1}}}
\def\brown#1{{\textcolor{brown}{\bf #1}}}
\def\yellow#1{{\textcolor{yellow}{#1}}}
\def\blue#1{{\textcolor{blue-}{#1}}}



\usepackage{bm}  
\usepackage{cancel}

%\usepackage{srcltx}
\font\tenmsb=msbm10
\font\sevenmsb=msbm7
\font\fivemsb=msbm5

%%%
\newfam\msbfam
\textfont\msbfam=\tenmsb
\scriptfont\msbfam=\sevenmsb
\scriptscriptfont\msbfam=\fivemsb
\def\Bbb#1{{\fam\msbfa #1}}


\font\teneufm=eufm10
\font\seveneufm=eufm7
\font\fiveeufm=eufm5
\newfam\eufmfam
\textfont\eufmfam=\teneufm
\scriptfont\eufmfam=\seveneufm
\scriptscriptfont\eufmfam=\fiveeufm
\def\frak#1{{\fam\eufmfam\relax#1}}
\let\goth\frak

%\newcommand\Area{\rule[-4mm]{0mm}{10mm}}
\newcommand\Area{\rule[-4mm]{0mm}{10mm}}
%\newarrow{IncluTo}{boldhook}--->

% for diagrams
\usepackage[all]{xy}

%%%%

\newcommand\qed{{\hspace*{\fill}Q.E.D.\vskip12pt plus 1pt}}
\newcommand\Pic[1]{\hbox{\rm Pic(}#1\hbox{\rm )}}
\newcommand\genus{{geometric genus\ }}
\newcommand\length{\hbox{\rm length}}
\newcommand\quadrato{{\hspace*{\fill}$\Box$\vskip12pt plus 1pt}}

\newcommand\sA{{\cal A}}
\newcommand\sB{{\cal B}}
\newcommand\sC{{\cal C}}
\newcommand\sD{{\cal D}}
\newcommand\sE{{\cal E}}
\newcommand\sF{{\cal F}}
\newcommand\sG{{\cal G}}
\newcommand\sH{{\cal H}}
\newcommand\sJ{{\cal J}}
\newcommand\sK{{\cal K}}
\newcommand\sL{{\cal L}}
\newcommand\sM{{\cal M}}
\newcommand\sN{{\cal N}}
\newcommand\sO{{\cal O}}
\newcommand\sP{{\cal P}}
\newcommand\sQ{{\cal Q}}
\newcommand\sR{{\cal R}}
\newcommand\sS{{\cal S}}
\newcommand\sT{{\cal T}}
\newcommand\sU{{\cal U}}
\newcommand\sW{{\cal W}}
\newcommand\sX{{\cal X}}
\newcommand\sV{{\cal V}}
\newcommand\sZ{{\cal Z}}

\newcommand\sca{{\mathscr A}}
\newcommand\scc{{\mathscr C}}
\newcommand\scr{{\mathscr R}}
\newcommand\scf{{\mathscr F}}
\newcommand\sce{{\mathscr E}}
\newcommand\scg{{\mathscr G}}
\newcommand\scl{{\mathscr L}}
\newcommand\scd{{\mathscr D}}
\newcommand\sco{{\mathscr O}}
\newcommand\sch{{\mathscr H}}
\newcommand\scv{{\mathscr V}}
\newcommand\scy{{\mathscr Y}}
\newcommand\scj{{\mathscr J}}
\newcommand\scp{{\mathscr P}}




\newcommand\gra{\alpha}
\newcommand\grb{\beta}
\newcommand\grg{{\mathsf H}}
\newcommand\grd{\delta}
\newcommand\gre{\epsilon}
\newcommand\grx{\chi}
\newcommand\grk{\kappa}
\newcommand\grl{\lambda}
\newcommand\vphi{\varphi}
\newcommand\grr{\rho}
\newcommand\grs{\sigma}
\newcommand\grt{\tau}
\newcommand\grz{\zeta}
\newcommand\gro{\omega}

\newcommand\MA{{\mathbb A}}
\newcommand\nat{{\mathbb N}}
\newcommand\rat{{\mathbb Q}}
\newcommand\reals{{\mathbb R}}
\newcommand\comp{{\mathbb C}}
\newcommand\zed{{\mathbb Z}}
\newcommand\eff{{\mathbb F}}
\newcommand\pn[1]{{\mathbb P}^{#1}}
\newcommand\proj[1]{{\mathbb P}({#1})}
\newcommand\pnc{{\mathbb P}_{\mathbb C}}
\newcommand\Gr{{\mathbb G}}
\newcommand\Xr{{\mathbb X}}
\newcommand{\x}{{\bm x}}
\newcommand{\alfa}{{\bm \alpha}}

\newcommand\bfm{\boldmath} 

\newcommand\fm{{\frak m}}
\newcommand\fri{{\frak I}}
\newcommand\fj{{\frak K}}
\newcommand\ft{{\frak t}}
\newcommand\alb[1]{\rm{ALB}({#1})}



\newcommand\hatS{{\widehat{S}}}
\newcommand\hatd{{\widehat{d}}}



\newcommand\proof{\noindent{\em Proof.}\ \ } \newcommand\mult{\mbox{\rm mult}}
\def\fine{\hfill\vrule height5 true pt width 5 true pt depth 0 pt}
\newcommand\Sing{\mbox{\rm Sing}}
\newcommand\ms{\medskip}



\newcommand\N{{\mathbb N}}
\newcommand\Q{{\mathbb Q}}
\newcommand\R{{\mathbb R}}
\newcommand\C{{\mathbb C}}
\newcommand\Z{{\mathbb Z}}
\newcommand{\be}{\begin{equation}}
\newcommand{\ee}{\end{equation}}

\newcommand{\epsbold}{{\bm \varepsilon}}
\def\Jac{\mathop{\rm Jac}\nolimits}

\newtheorem{theorem}{Theorem}[section]
\newtheorem{lemma}[theorem]{Lemma}
\newtheorem{corollary}[theorem]{Corollary}
\newtheorem{conjecture}[theorem]{Conjecture}
\newtheorem{clam}[theorem]{Claim}
\newtheorem{prop}[theorem]{Proposition}
\newtheorem{no}[theorem]{Notation}
\newtheorem{question}[theorem]{Question}
\newtheorem{problem}[theorem]{Problem}
\newtheorem{definition}[theorem]{Definition}
\newtheorem{re}[theorem]{Remark}
\newtheorem{defre}[theorem]{Definition--Remark}
\newtheorem{pargrph}[theorem]{}
\newtheorem{examp}[theorem]{Example}
\newtheorem{CT}[theorem]{ConjecturalTheorem}
\newtheorem{MM}[theorem]{ }
\newtheorem{lede}[theorem]{Lemma--Definition}
\newtheorem{probdef}[theorem]{Problem--Definition}
\newtheorem{ex}[theorem]{Esercizio}
\newtheorem{propdef}[theorem]{Proposition--Definition}


\textwidth15cm
\textheight22.5cm
\hoffset=-14mm
\voffset=-13mm
\makeatletter
\ifnum\@ptsize=0\addtolength{\hoffset}{-0.3cm}\fi
\ifnum\@ptsize=2\addtolength{\hoffset}{0.5cm}\fi\sloppy


\def\cocoa{{\hbox{\rm C\kern-.13em o\kern-.07em C\kern-.13em o\kern-.15em A}}}



\newenvironment{rem*}{\begin{re}\em}{\end{re}}
\newenvironment{example*}{\begin{examp}\em}{\end{examp}}
\newenvironment{definition*}{\begin{definition}\em}{\end{definition}}
\newenvironment{probdef*}{\begin{probdef}\em}{\end{probdef}}
\newenvironment{question*}{\begin{question}\em}{\end{question}}
\newenvironment{prgrph}{\indent\begin{pargrph}\em}{\end{pargrph}}
\newenvironment{prgrph*}[1]{\indent\begin{pargrph}{\bf #1.}\em\
}{\end{pargrph}}
\newenvironment{defre*}{\begin{defre}\em}{\end{defre}}
\newenvironment{MM*}{\begin{MM}\em}{\end{MM}}
\newenvironment{ex*}{\begin{ex}\em}{\end{ex}}
\newenvironment{propdef*}{\begin{propdef}\em}{\end{propdef}}


%%%%%%%%%%%%%%%%%%%%%%%%%%%%%%
%%%%%%%%%%%%%%%%%%%%%%%%%%%%%%
\begin{document}




\title{Corrigendum for   ``Almost vanishing polynomials \\and an  application  to the Hough transform"
\footnote{2010
{\em Mathematics Subject Classification}. Primary  26C10, 15A60, 14Q10;
Secondary 14H50  \newline
%68T10 \newline
\indent{{\em Keywords and phrases.}  Weighted matrix norm; hypersurface;  crossing area conditions} }}
\author{ Maria-Laura Torrente and Mauro C. Beltrametti }

\date{}
\maketitle

%\begin{abstract}
%\end{abstract}
%\section*{Introduction}

In this note we correct a technical error occurred in \cite{TB}.
%where several time the H\"older inequality was wrongly used, forgetting a transpose.
 This affects the bounds given in  that paper, 
even though  the structure and the logic of all proofs remain  fully unchanged. The error is due to a repeated wrong use  of 
H\"older's inequality (a transpose of a matrix  was missed). The first time it occurs is  in the first inequality of formula $(8)$ of \cite{TB}.   Indeed, the correct version of that formula is:

\begin{eqnarray*}\label{diseq2}
 |\Jac_f(p) (p^*-p)^t| & = &  |\Jac_f(p) \sE^{-1} \sE (p^*-p)^t| \nonumber \\ 
 &\le&  \|\big(\Jac_f(p) \sE^{-1}\big)^t\|_1 \|\sE (p^*-p)^t\|_\infty \nonumber \\ 
 &\le& \|\sE^{-1}\Jac_f(p)^t \|_1 \le  \|\Jac_f(p)^t\|_1 \|\sE^{-1}\|_1.
\end{eqnarray*}

%\smallskip


 We refer to our paper \cite{TB}, using the same notation.
Here, we confine ourselves to state the correct versions of 
%the bounds given in \cite{TB}. The following 
 Proposition 2.1,  Proposition 2.5, Proposition 3.2, and Proposition 4.3 of \cite{TB}, respectively. Up to the error pointed out and corrected as above, the proofs go parallel to those in~\cite{TB}. In the following, $P$  denotes the multivariate polynomial ring $\R[\x]=\R[x_1,\ldots,x_n]$.

\bigskip
\bigskip
\bigskip
%\begin{prop}\label{PropNonPassa} 
\noindent{\bf Proposition 2.1}
{\em Let $f=f(\x)$ be a non-constant polynomial of~$P$,
let $p$ be a point of~$\R^n$, and let ${\bf C}(p) \subseteq {\bf B}(p)$
be an $(\infty,\epsbold)$-unit cell   centered at $p$. If
$$
| f(p)| > \|\Jac_f(p)^t\|_1 \epsbold_{\rm max} +\frac{n}{2}{\mathsf H} \epsbold_{\rm max}^2 =:B_1,
$$
then the hypersurface  of equation $f=0$ does not cross ${\bf C}(p).$
}
\bigskip

%\begin{prop}\label{PropNonPassa2}
\noindent{\bf Proposition 2.5}
{\em Let $f(\x)$ be a degree $\geq 2$   polynomial of~$P$.
Let $p$ be a point of $\R^n$ and let
${\bf C}(p) \subseteq {\bf B}(p)$
be an $(\infty,\epsbold)$-unit cell  centered at $p$. If
$$
|f(p)| > \|{\rm Jac}_f(p)^t\|_1 \epsbold_{\rm max} + \frac{n}{2}  \|H_f(p)\|_\infty \epsbold_{\rm max}^2:=B_1',
$$
then  the hypersurface of equation $f=0$ does not cross ${\bf C}(p)$ neglecting contributions of order ${\rm O}(\epsbold_{\rm max} ^3)$.
}
\bigskip

%\begin{prop}\label{PropPassa} 
\noindent{\bf Proposition 3.2}
{\em Let $f=f(\x)$ be a degree $\geq 2$ polynomial of~$P$,
let $p$ be a point of $\R^n$ such that $\Jac_f(p)$ is not the zero vector, 
and let ${\bf C}(p) \subseteq {\bf B}(p)$
be an $(\infty,\epsbold)$-unit cell  centered at $p$.
Let $R$ be a positive real number such that 
$R < \min \big\{\epsbold_{\rm min}, \frac{\|\Jac_f(p)\|_1}{{\mathsf H}}\big\}$.
Set $c:=\max\{2, \sqrt{n}\}$. If 
$$
|f(p)| < \frac{2R}{\mathsf J(c+ n^{5/2}{\mathsf H} {\mathsf J} R)}  =: B_2,
$$
then the hypersurface of equation $f=0$ crosses ${\bf C}(p)$.
}

\bigskip
%\begin{prop}\label{PropPassa2}
\noindent{\bf Proposition 4.3}
{\em Let $f=f(\x)$ be a degree $\geq 2$ polynomial of~$P$,
let $p$ be a point of $\R^n$ such that  Jacobian $\Jac_f(p)$ and the Hessian matrix $H_f(p)$ are nontrivial, 
and let ${\bf C}(p) \subseteq {\bf B}(p)$
be an $(\infty,\epsbold)$-unit cell  centered at $p$.
Let $R$ be a positive real number such that
$R < \min \left \{\epsbold_{\rm min}, \frac{\|\Jac_f(p)\|_1}{n^2 \|H_f(p)\|_\infty} \right\}$, 
let $c:=\max\{2, \sqrt{n}\}$ and set 
$$
\Theta:=  \|\Jac_f^\dagger(p)\|_\infty+  n^2(1+2 \sqrt{n}) \frac{\|H_f(p)\|_\infty}{\|\Jac_f(p)\|_1^2} R.
$$
If 
\begin{equation*}
|f(p)| < \frac{2R}{\Theta (c+  n^{9/2} \|H_f(p)\|_\infty \Theta R)} =:B_2',
\end{equation*}
then the hypersurface of equation $f=0$ crosses ${\bf C}(p)$  neglecting order ${\rm O}(R^2)$ contributions.
}
\bigskip
%\section{Concluding remarks }\label{CE}
%\addtocounter{subsection}{1}\setcounter{theorem}{0}

\noindent{\bf{Remark.}} More accurate and general  estimates \cite{TBS}, when specialized to the case of hypersurfaces and $\infty$-norm,  allow us to improve the bounds above, this also assuring   that the applications discussed in \cite[Section 6]{TB} still remain  meaningful. Precisely, in \cite[Theorem 3.2]{TBS}, we can in fact show that  the bound $B_2$ goes as $\frac{1}{n}$ instead of $\frac{1}{n^{5/2}}$, weakening the assumption. 
Similarly, one can shows that the bound $B_2'$ in \cite[Theorem 4.6]{TBS} goes as $\frac{1}{n^{5/2}}$ instead of $\frac{1}{n^{9/2}}$. 



\small{


\begin{thebibliography}{10}


\bibitem{TB} M. Torrente and M.C. Beltrametti, Almost  vanishing polynomials and an application to the Hough transform, J.  Algebra Appl. {\bf 13}(8), (2014), 1450057 [39 pages].

\bibitem{TBS} M. Torrente, M.C. Beltrametti and J.R. Sendra, $r$-norm bounds  and metric properties for zero loci of real analytic functions,  preprint, 2017.


\end{thebibliography}

}



\bigskip
\bigskip

\noindent
 M. Torrente, Istituto di Matematica Applicata e Tecnologie Informatiche - Consiglio Nazionale delle Ricerche, 
Via de Marini, 6,
I-16149 Genova, Italy. e-mail {\tt torrente@dima.unige.it}


\smallskip


\noindent M.C. Beltrametti,
Dipartimento di Matematica,
Universit\`a di Genova,
Via Dodecaneso 35,
I-16146 Genova, Italy. e-mail {\tt beltrame@dima.unige.it}




\end{document}




















