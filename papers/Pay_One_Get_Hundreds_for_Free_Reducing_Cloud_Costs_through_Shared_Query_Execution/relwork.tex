\section{Related Work}
\label{sec:rel-work}

Sharing computation among multiple concurrent queries was first explored in the
context of multi-query optimization (MQO)%
~\cite{Finkelstein:1982:CEA:582353.582400,Sellis:1988:MO:42201.42203}. The basic
idea consists of, given a set of queries, reducing the computational costs by performing 
shared expressions once, materializing them temporarily, and reusing them for solving the remainder of the queries.
Thus, the evaluation of common subexpressions 
is carried out only once. This approach was later
extended to benefit from query result caches \cite{qrc}, materialized/cached
views \cite{mcv}, and intermediate query results \cite{iqr1, iqr2}. However,
MQO does not use all
sharing potential.

More recently, a new line of work has developed ways to exploit sharing
opportunities such as sharing disk or memory bandwidth among queries without
common subexpressions.  For example, StagedDB \cite{stageddb} and QPipe
\cite{Harizopoulos:2005:QSP:1066157.1066201} use a simultaneous pipelining
technique to share work among queries that arrive within a certain time window.
MonetDB \cite{Zukowski:2007:CSD:1325851.1325934} and CoScan \cite{CoScan2011}
use cooperative scans where queries are dynamically scheduled together to
reduce the aggregated amount of I/O operations.  IBM UDB
\cite{Lang:2007:IBM:1325851.1325999} performs dynamic scan group and adaptive
throttling of scan speeds to suit a set of concurrent queries. CJoin
\cite{Candea:2009:SPJ:1687627.1687659} uses an always-on plan of join operators
to execute the joins of all concurrent queries. IBM Blink
\cite{Raman:2008:CQP:1546682.1547130} and Crescando
\cite{Giannikis:2010:CRE:1807167.1807326} answer multiple queries in one table
scan sharing disk and main-memory bandwidth. Data\-Path
\cite{Arumugam:2010:DSD:1807167.1807224} uses a push-based instead of a
pull-based model for a data-centric query processing to facilitate sharing of
concurrent queries.  SharedDB \cite{Giannikis:2012:SKO:2168651.2168654}
achieves predictable performance for highly concurrent workloads by query
grouping and using a global query plan to execute them.  MQJoin \cite{mqjoin}
efficiently shares the join execution for hundreds of concurrent queries.
These approaches significanly improve performance and demonstrate the potential
of sharing in many common workloads. However, they require significant changes
to existing database engines, thereby limiting their applicability
if modifying an existing system is not an option.  

Similarly to \cite{Candea:2009:SPJ:1687627.1687659, Arumugam:2010:DSD:1807167.1807224, Giannikis:2012:SKO:2168651.2168654, mqjoin}, our approach focuses on enabling
work sharing at run-time using an operator-centric approach, i.e., each operator process
a group of queries, thus exploiting both work and data commonalities at each operator.
To accomplish this, we annotate
intermediate results to obtain a high level of sharing for queries without
common subexpressions. 
The main distinction from previous work is that
we achieve this high degree of sharing solely through \sql rewriting, i.e.,
without requiring either modifications to the underlying engine or vendor
support. 
The goal in this paper is to explore the extent to which shared execution
pays-off and whether it can be implemented atop black-box query processing
engines such as those found in the cloud. 
In a related thesis~\cite{Wolf2017}, we explore enabling on-premise database
systems to support shared workload execution for some operators.  The results
of this paper extend this preliminary work.

%This makes our approach applicable to cloud systems. Moreover, our
%approach also uses a global query plan for executing a group of queries, but
%common expressions are materialized to adapt such global plan for executing it
%in \qaas systems.

%On the other hand, the shared workload approach is based on a {\it
%    batched} query execution model, where multiple concurrent queries can share
%most of the work needed to go over the data. This degree of sharing depends on
%how much {\it data sharing} (e.g., through scan operations over base data) and
%{\it work sharing} (e.g., through common subexpressions) opportunities exist
%among them~\cite{Harizopoulos:2005:QSP:1066157.1066201}.

%% Explain about shared workload
%Batching queries together to share computation across multiple concurrent
%queries was first introduced by
%SharedDB~\cite{Giannikis:2012:SKO:2168651.2168654}, which builds on ideas from
%multi-query optimization and data stream processing.  More specifically, it
%compiles multiple queries together into a single global plan, to which new
%requests subscribe for getting their results in a push-oriented model.
