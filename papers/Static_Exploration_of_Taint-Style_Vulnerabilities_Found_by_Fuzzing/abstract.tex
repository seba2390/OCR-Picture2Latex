Taint-style vulnerabilities comprise a majority of fuzzer discovered program faults.
These vulnerabilities usually manifest as memory access violations caused by tainted program input.
Although fuzzers have helped uncover a majority of taint-style vulnerabilities in software to date, they are limited by (i) extent of test coverage; and (ii) the availability of fuzzable test cases.
Therefore, fuzzing alone cannot provide a high assurance that all taint-style vulnerabilities have been uncovered.

In this paper, we use static template matching to find recurrences of fuzzer-discovered vulnerabilities.
To compensate for the inherent incompleteness of template matching, we implement a simple yet effective match-ranking algorithm that uses test coverage data to focus attention on those matches that comprise untested code.
We prototype our approach using the Clang/LLVM compiler toolchain and use it in conjunction with {\it afl-fuzz}, a modern coverage-guided fuzzer.
Using a case study carried out on the Open vSwitch codebase, we show that our prototype uncovers corner cases in modules that lack a fuzzable test harness.
Our work demonstrates that static analysis can effectively complement fuzz testing, and is a useful addition to the security assessment tool-set.
Furthermore, our techniques hold promise for increasing the effectiveness of program analysis and testing, and serve as a building block for a hybrid vulnerability discovery framework.