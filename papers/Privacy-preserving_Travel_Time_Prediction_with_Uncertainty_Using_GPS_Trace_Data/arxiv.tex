\documentclass[10pt,journal,compsoc]{IEEEtran}
\IEEEoverridecommandlockouts
\usepackage[linesnumbered,ruled,vlined]{algorithm2e}
\usepackage[justification=centering]{caption}
\usepackage{dblfloatfix}   
\usepackage{url}
\def\UrlBreaks{\do\/\do-}
\usepackage{breakurl}
\usepackage[breaklinks]{hyperref}
\usepackage{diagbox,threeparttable,mwe} 
\usepackage[numbers,sort&compress]{natbib}
\usepackage{amsmath,amssymb,enumitem,esvect}
\usepackage{algorithmic}
\usepackage{graphicx, wrapfig}
\usepackage{textcomp}
\usepackage{xcolor, comment, setspace,empheq}
\newtheorem{defn}{Definition}
\setlength{\parindent}{0pt}
\newcommand{\M}{\mathcal{M}}
\newcommand{\E}{\mathbb{E}}
\newcommand{\V}{\mathbb{V}}
\newcommand{\f}{\mathbf{f}}
\newcommand{\s}{\mathbf{s}}
\newcommand{\x}{\mathbf{x}}
\newcommand{\y}{\mathbf{y}}
\newcommand{\ts}{\mathbf{t}}
\newcommand{\eGI}{$\epsilon$-geo-indistinguishability}
\newcommand{\GI}{geo-indistinguishability }

\setlength{\parskip}{3pt}
\allowdisplaybreaks
\def\BibTeX{{\rm B\kern-.05em{\sc i\kern-.025em b}\kern-.08em
    T\kern-.1667em\lower.7ex\hbox{E}\kern-.125emX}}
    
    
\begin{document}

\title{Privacy-preserving Travel Time Prediction with Uncertainty Using GPS Trace Data\\
\author{Fang Liu$^*$,  Dong Wang$^*$,  Zhengquan Xu
\thanks{$^*$Co-first authors.  F. Liu is Professor in the Department of Applied and Computational Mathematics and Statistics, University of Notre Dame, Notre Dame, IN, 46556, USA  (corresponding author; e-mail: Fang.Liu.131@nd.edu); D. Wang  (e-mail: dwang22@nd.edu)  is a doctoral student and Z. Xu  (e-mail: xuzq@whu.edu.cn) is Professor in the Department of State Key Laboratory of Information Engineering
in Surveying, Mapping and Remote Sensing, Wuhan University, Wuhan, 430079, China.}\vspace{-18pt}
}}
\maketitle
\vspace{-36pt}
\begin{abstract}
The rapid growth of GPS technology and mobile devices has led to a massive accumulation of location data, bringing considerable benefits to individuals and society. One of the major usages of such data is travel time prediction, a typical service provided by GPS navigation devices and apps.  Meanwhile, the constant collection and analysis of the individual location data also pose unprecedented privacy threats. We leverage the notion of geo-indistinguishability, an extension of differential privacy to the location privacy setting, and propose a procedure for privacy-preserving travel time prediction without collecting actual individual GPS trace data. We propose new concepts to examine the impact of geo-indistinguishability-based sanitization on the usefulness of GPS traces and provide analytical and experimental utility analysis for privacy-preserving travel time prediction. We also propose new metrics  to measure the adversary error in learning individual GPS traces from the collected sanitized data. Our experiment results suggest that the proposed procedure provides travel time prediction with  satisfactory accuracy at reasonably small privacy costs.
\end{abstract} \vspace{-3pt}
\begin{IEEEkeywords}
 differential privacy, geo-indistinguishability, effective number of mapped full trajectories, usefulness, usable trajectory, continuous positioning degree, average distance
\end{IEEEkeywords}

\begin{figure*}[!t]
\centering \includegraphics[width=1\textwidth]{all3V3.png}
\vspace{-18pt}
\caption{Privacy Protection Strategies for Analysis using GPS Data }\label{fig:adversary}\vspace{-12pt}\end{figure*}  

\vspace{-12pt}
\section{Introduction}\label{sec:intro}\vspace{-3pt}
\subsection{Motivation and Problem}\label{sec:problem}
\vspace{-3pt}
The rapid growth of GPS technology and mobile devices has led to a quick accumulation of massive location data. Analysis and understanding of the data have brought enormous benefits to individuals and society.  Meanwhile, collection and processing of location data can easily expose personal behaviors, interests, social relations, or other private information, especially if combined with other data sources. \citet{de2013unique} studied 15-month location data from 1.5 million people and found that as little as 4 space-time points can uniquely identify 95\% individuals.  Meanwhile, users are often not fully aware of privacy risks from sharing their location data with service providers and how their data are used \cite{coppens2014privacy, kessler2018geoprivacy}.

One  important application of GPS data is Travel time Prediction with Uncertainty (TPU). TPU examines how quickly a person arrives at a destination with a certain level of confidence. It is important for transportation and urban planning  and a typical route planning service provided by navigation systems and mapping apps.  TPU often relies on continuous collection and processing of users' travel trajectories and  thus exposes data contributors to privacy threats from adversaries, honest-but-curious (e.g., the service provider itself) and malicious, as depicted in Fig. \ref{fig:adversary}(a)). 

To our best knowledge,  there is no work focusing on privacy-preserving TPU (PP-TPU) analysis. We display in Fig.~\ref{fig:adversary}(b) and \ref{fig:adversary}(c) two possible strategies for PP-TPU. Fig. \ref{fig:adversary}(b) focuses on sanitizing aggregated statistics calculated from the actual user data, say via a differentially private randomization mechanism such as the Laplace \cite{dwork2006calibrating} or the Gaussian mechanisms \cite{dwork2014algorithmic,liu2018generalized}). This strategy  mitigates the privacy threats from the adversaries who aim to learn something new  about their targets from the released aggregate information, but it cannot manage the privacy risk brought by the adversaries who have access to the original data, such as the service provider itself. In Fig.~\ref{fig:adversary}(c), sanitization occurs during data collection; that is, the true individual responses go through a sanitization mechanism locally before being shared with a third party. As a result, the true responses are only known to the users themselves. 

We aim to develop a PP-TPU procedure that  implements the strategy in Fig.~\ref{fig:adversary}(c), leveraging the state-of-the-art notions and sanitization mechanisms in data privacy research as stepping stones to achieve our goal. 

\vspace{-12pt}\subsection{Related Work}\vspace{-6pt}
Data encryption, anonymization, and obfuscation are common frameworks for controlling the privacy risks incurred by location data collection and sharing. The PP-TPU procedure we propose can be regarded as a data obfuscation approach. Below we provide a brief overview of each framework,  analyze their limitations and challenges, and state the rationale for us adopting the data obfuscation framework to develop the PP-TPU procedure. 

Location encryption uses cryptographic techniques to mitigate privacy risks in location data \cite{zhong2007louis, popa2011privacy, zhu2011applaus, mascetti2011privacy, li2019privacy,zhu2019traffic, zhang2019privacy}. This is a common approach for protecting individual privacy but can be costly in terms of computation and resource \cite{zhou2019privacy}. Furthermore,  data, once decrypted,  are no longer private to those who have the authority to access the data; the privacy:utility ratio from the data user perspective is either 100:0\% or 0:100\%, corresponding to the two states of encryption and decryption, respectively.  These two extreme options of data access often do not meet the practical needs for data sharing. Indeed, a non-zero small privacy cost is often acceptable in practice so to create more options between the two extremes and  share information with more data users.

Data anonymization and obfuscation  provide options between the two extremes. These concepts focus on privacy-preserving data processing and analysis via methods such as data coarsening, removal of identifiers,  reporting dummy locations via randomization mechanisms, among others.  The key issue in these approaches is to strike a good balance between privacy loss and data utility (the higher the privacy loss, the more utility there is in the anonymized or obfuscated data relative to the original data).  

Several formal privacy concepts have been developed to attain  anonymization for general data, such as $k$-anonymity  \cite{samarati2001protecting, sweeney2002k} and $l$-diversity \cite{machanavajjhala2007diversity}. Both concepts have been adapted and applied in the location privacy setting (e.g.,  \cite{gruteser2003anonymous,kido2005protection,gedik2007protecting, ghinita2007prive,lu2008pad,chow2009casper,gkoulalas2010providing,niu2014achieving,fei2017k,zhao2018illia,wang2019achieving, zhang2019caching} for $k$-anonymity, \cite{terrovitis2008privacy, liu2009query, tu2018protecting} for $l$-diversity). Though the concepts are intuitive, neither $k$-anonymity nor $l$-diversity  involves randomization; it has been shown that adversaries may still learn sensitive information or re-identify individuals from anonymized location data \cite{de2013unique,wang2018privacy}. 

Differential privacy (DP) \cite{dwork2006calibrating} involves randomization and is a mathematically robust conceptual for data obfuscation. DP bas been proved to be robust against a wide range of adversary attacks \cite{nissim2017differential, dwork2017exposed}, processes properties such as privacy loss composability \cite{mcsherry2007mechanism} and immunity to post-processing \cite{dwork2014algorithmic} that facilitate practical implementation, and has quickly become the mainstream in privacy research and applications. Big tech companies (e.g., Apple, Google, IBM) and government agencies (Census2020) also adopt DP or its variants to collect or release data. The classical DP concept is also applicable in location data analysis \cite{asada2019and, bavadekar2020google, aktay2020google, xiao2015protecting}.  


Conceptual extensions of DP for location privacy also exist, among which  geo-indistinguishability  (GI) \cite{andres2013geo} is perhaps the most popular. GI has been explored in a wide range of location privacy applications.  \citet{andres2013geo} propose the planar Laplace mechanism  to perturb the 2-dimensional location data.    \citet{chatzikokolakis2015constructing} extend GI by developing an elastic indistinguishability metric that adapts the amount of injected noises according to the area density. \citet{cunha2019clustering} propose a clustering  mechanism for continuous location traces clustering application.  \citet{shi2019deep} apply  GI to preserve location information of passengers in vehicles of transportation companies.   \citet{qian2020privacy} propose a GI task allocation  mechanism to preserve location privacy in mobile crowdsensing applications. \citet{qiu2020location} design a strategy to minimize the loss in quality of service  due to GI obfuscation. \citet{shi2020quantitative} present a closed-form relationship between localization accuracy and the  GI privacy level. \citet{takagi2020poster} propose  Geo-Graph-Indistinguishability that extends DP to the setting of location privacy on road networks. \citet{ren2020egeoindis} present the ``Expanding GI'' framework to protect the privacy of vehicle locations by abstracting maps as bitmaps and utilizing linear programming to control information loss. 

In summary, despite the various applications of GI in  practical problems, its potential in TPU analysis has not been explored to our knowledge. The reasons for us adopting the GI concept for developing a PP-TPU procedure rather than other  privacy protection frameworks for location data are as follows.  First, our GI-based PP-TPU procedure avoids the limitations of encryption-based approaches,  thanks to the tunable parameter (privacy budget/loss $\epsilon$) in GI, and provides options between the two extremes of zero utility/full protection ($\epsilon\!\rightarrow\!0$) and full utility/zero protection ($\epsilon\!\rightarrow\!\infty$). Second, while PP-TPU can leverage the classical DP concept to release aggregate location information, it often requires the collection of actual individual data (Fig.~\ref{fig:adversary}(b)). In contrast, GI provides a framework to perform TPU analysis without collecting  actual individual GPS records (Fig.~\ref{fig:adversary}(c)). This strategy, compared to Fig.~\ref{fig:adversary}(b), better protects  users' privacy and boost users' confidence in data sharing  as the data sanitization is performed locally on users' own devices; in addition, users can pick $\epsilon$ themselves depending on how willing they are to share their data. Third, GI follows similar mathematical reasoning as DP and many desirable properties of DP are also applicable to GI such as composability and immunity to post-processing that are important to meeting the specific challenges in the development of PP-TPU procedures.  Specifically, the TPU analysis requires the collection of multiple 3-tuple GPS records (2-dimensional location coordinates and timestamp)  from a single trip in a traveler. Due to the sequential composition of privacy loss, the overall privacy loss can become unrealistically high  to obtain a useful sanitized trajectory, or the sanitized trajectory is useless if the overall cost is kept reasonable. Besides, the GPS records, after sanitization, have to make sense in the 3-dimensional spatiotemporal space and for the actual road networks on which TPU is performed. 

\vspace{-9pt}\subsection{Our Contribution}\vspace{-3pt}
Our PP-TPU procedure balances the trade-off between the privacy risk from collecting  individual travel trajectories and the utility of sanitized trajectories, while being mindful of its practical feasibility. The conceptual and methodological contributions and the potential practical impacts of the procedure are summarized as follows.
\vspace{-3pt}
\begin{itemize}[leftmargin=9pt]
\item The procedure takes into account privacy loss composability when sanitizing multiple GPS records per trajectory. To limit the overall privacy loss, it curbs the number of GPS records collected per trajectory and leverages public information of maps to filter out unusable trajectories.
\item  We propose new concepts of \emph{usable set of travel trajectories}, \emph{effective number of mapped full trajectories}, \emph{usefulness}, and different distance deviation measures to quantify the utility of sanitized GPS records and trajectories for PP-TPU.  
\item  We propose new metrics \emph{continuous positioning degree} and \emph{average distance} to quantify the adversary error in learning individual trajectories given sanitized GPS records.
\item We examine the feasibility of the proposed PP-TPU procedure by quantifying the trade-off between privacy loss and the utility of sanitized GPS records and traces analytically and empirically, providing insights on choosing privacy loss parameters in different application scenarios.
\item The  procedure is easy to implement. Service providers and GPS navigation systems and apps may use it to collect user location data and provide TPU service with guaranteed privacy protection.
\end{itemize}


%---------------------------------------
\vspace{-9pt}\section{Preliminaries}\label{sec:prelim}
\vspace{-3pt} \subsection{Map Matching of GPS Records}\vspace{-3pt}
TPU starts with the collection of GPS records that contain the spatial-temporal information of a traveler and then matches the GPS locations with physical road network maps. Each GPS record contains the location  $P_i$ (latitude and longitude coordinates) and timestamp $t_i$ information. Due to satellite signal blockage, multi-path effects, and other factors that may affect GPS signals,  collected GPS location information is not always accurate.  Commonly direct projections of GPS coordinates may not correspond to any meaningful real map coordinate, and road mapping algorithms often involve some level of approximation.  Fig. \ref{fig:gps mapping} shows an example of the shortest path map-matching algorithm  \cite{orda1990shortest} that project 3 GPS registration points onto the physical road network.
\begin{figure}[!htb]\vspace{-9pt}
\centering \includegraphics[width=0.45\textwidth]{GPSmappingV2.png}\vspace{-3pt}
\caption{Shortest path map matching}\label{fig:gps mapping}
\end{figure}\vspace{-15pt}   

%-----------------------
\subsection{Travel Time Prediction with Uncertainty}\vspace{-3pt}
Various approaches to travel time prediction have  been  developed. The na\"ive travel time prediction outputs a single projected travel time value, but the reachable time-space range of a traveler is rather stochastic, due to the dynamic nature of  human  behaviors, traffics, etc \cite{noland1995travel}. Studies \cite{chen2017most} have shown that individuals, when facing uncertainty in travel time,  tend to avoid the risk of lateness and often reserve some time to ensure that they can arrive on time with a high level of confidence \cite{de2013unique}. It is also important for transportation and urban planning to understand the uncertainty around travel time for infrastructure development and designs, among others. 

The analysis of TPU aims to obtain $f(t)$, the probability density function (pdf) of travel time $t$ spent over a trip with starting point A and destination B.  The probability that the destination B can be reached within time $b$ can be easily obtained from the cumulative density function (CDF) of $t$, that is, $p\!=\!F_t(b)\!=\!\int_{0}^{b}f(t)dt$. For example, suppose $b\!=\!10$ minutes and $p\!=\!0.9$, then there is a 90$\%$ chance of arriving at the destination within 10 minutes. In reality, $f(t)$ is unknown and is often estimated by an empirical $\hat{f}(t)$ based on collected individual travel data, such as via the GPS. 


\vspace{-6pt}\subsection{Differential Privacy and Geo-indistinguishability}\vspace{-3pt}
Differential privacy (DP) is a state-of-the-art privacy protection model that guarantees privacy for released information in mathematically rigorous terms. 
\begin{defn}[\textbf{$\epsilon$-differential privacy} \cite{dwork2006calibrating}]\label{def:DP}
A randomization mechanism $\M$ is $\epsilon$-differentially private, if for any pair data sets $X$ and $X'$ that differ by one record and every possible outcome set $\Omega$ to a query,
\begin{equation}
\Pr[\M(X)\in\Omega]\le{e^{\epsilon}}\cdot{\Pr[\M(X')\in{\Omega}]},
\end{equation}
\end{defn}\vspace{-6pt}
where  $\epsilon>0$ is the privacy budget or loss parameter. The smaller $\epsilon$ is, the more privacy protection there is on the individuals in the data set. $X$ and $X'$ differ by one record may refer to the case that $X$ and $X'$ are of the same size but differ in the attribute values in exactly one record, or the case that $X'$ is one record more than $X$ or vice versa. 

The classical DP in Definition \ref{def:DP} provides a mathematical model for privacy guarantees when releasing  aggregate statistics from a group of individuals. The local DP \cite{localDP2011, duchi2013local} is an extension of the classical DP to a single user's data and can be used to develop mechanisms for sanitizing individual responses rather than aggregate results, with privacy.
\begin{defn}[\textbf{$\epsilon$-local differential privacy} \cite{kasiviswanathan2011can, duchi2013local}]\label{def:localDP} 
A randomization mechanism $\M$ provides $\epsilon$ local DP if
\begin{equation} 
\Pr[\M(x)\in\Omega]\le{e^{\epsilon}}\cdot{\Pr[\M(x')\in{\Omega}]}.
\end{equation}
for all pairs of possible data $x$ and $x'$ from an individual and all possible output subset $\Omega$ from $\M$.
\end{defn}
The local DP implies that even if an adversary has access to the  sanitized  personal responses from a randomization mechanism that satisfies local DP, the adversary is still unable to learn much new about the user's actual responses. 

GI is an extension of DP to location privacy and aims at releasing individual location records. In that sense, GI is more similar to the local DP concept than the classical DP; but all three concepts are based on similar mathematical formulations.  The formal definition of GI is given below.
\begin{defn}[\textbf{Geo-indistinguishability} \cite{andres2013geo}]\label{def:GI}
Let $d(P,P')$  denote the Euclidean distance between any two distinct locations $P$ and $P'$, and $\epsilon$ be the unit-distance privacy loss. A randomization mechanism $\M$ satisfies GI iff for all possible released location $P^*$, $\gamma>0$, and any possible pair of $P$ and $P'$ within the radius of $\gamma$, 
\begin{equation}\label{Eq:geoDP}
\Pr(\M(P)=P^*|P)\le e^{\epsilon\gamma}\cdot\Pr(\M(P')=P^*|P').
\end{equation}
\end{defn}\vspace{-3pt}
In other words, $\M$ in Eq (\ref{Eq:geoDP}) enjoys $(\epsilon\gamma)$-privacy for any specified $\gamma$, and  the probability of distinguishing any two locations with a radius of $\gamma$, given the released location $P^*$, is $e^{\epsilon\gamma}$ times the probability  when  not  having $P^*$. For a fixed $\epsilon$, the larger $\gamma$ is, the larger the privacy loss $(\epsilon\gamma)$ will be.  For example, Tom is standing in Times Square in NYC and looking for a restaurant for lunch. He sends a request to a service provider for a list of restaurants nearby. However, he does not want to disclose his exact location and chooses to release a perturbed location $P^*$ via GI with $\epsilon=0.1$ per mile. The probability the service provider will identify his true location within a radius of $\gamma=1$ mile given the perturbed location information is at most 1.1 folds of the probability when  not having the information, and at most 148 folds within a radius of $\gamma=50$ miles. In the latter case, though the probability of distinguishing locations given $P^*$ dramatically increases compared to not having $P^*$, it is not practically alarming from a privacy perspective as the increase is caused by the large $\gamma$ rather than a large $\epsilon$. In other words, the service provider will have great confidence that Tom is in NYC given the released $P^*$, but little confidence in pinpointing exactly where in NYC. If it were the combination of $\epsilon=50$ and $\gamma=1$, then the probability the service provider identifying Tom's true location within a radius of 1 mile would increase by 248 folds  given the perturbed location information, constituting a disastrous situation in privacy.

The planar Laplace mechanism can be used to achieve $\epsilon$-GI by perturbing location information in polar coordinates.
\begin{defn}[\textbf{polar Laplace mechanism} \cite{andres2013geo}]\label{def:polar}
The sanitized location $P^*$,  given the actual location $P$ with coordinates $(x,y)$ in the Euclidean space, satisfies GI with coordinates
\begin{equation}\label{Eq:xsys}
(x^*,y^*)=(x+r\cos(\theta), y+r\sin(\theta)), 
\end{equation}
where  the joint distribution of $R$ and $\theta$ is
\begin{equation}\label{Eq:planar}
f(r,\theta)=\epsilon^{2}re^{-\epsilon r}/(2\pi).
\end{equation}
\end{defn}\vspace{-6pt}
Eq (\ref{Eq:planar}) implies $R$ and $\theta$ are independently distributed and 
\begin{align}
r&\sim\mbox{gamma}(2,\epsilon)=r\epsilon^2 e^{-\epsilon r}\label{Eq:r}\\
\theta&\sim\mbox{Unif}(0,2\pi)=1/(2\pi).\label{Eq:theta}
\end{align}
In summary, to generate a sanitized location $P^*$, one may draw $R$ from the gamma distribution with shape 2 and scale $\epsilon^{-1}$ and $\theta$ from Unif($0,2\pi$), and then calculate the coordinates of $P^*$ in the Euclidean space per Eq (\ref{Eq:xsys}).


%----------------------------------------------
\vspace{-9pt}\section{Privacy-preserving TPU with GI}\label{sec:method}\vspace{-3pt}
Applied to the collection of GPS records, the GI notion can help protect individual privacy on several types of information: an individual location at a given time point, the travel trajectory of an individual over a time period, and any derived information from the collected sanitized trajectories, such as TPU. 

In what follows, we present a procedure to achieve PP-TPU in the framework of GI, taking into the composability of privacy costs from disclosing multiple location points from a trajectory and leveraging public knowledge of maps and road networks to improve the utility of PP-TPU on a given target road. We also examine the accuracy of sanitized information relative to the original  information; analyze the privacy guarantees of the proposed procedure, along with newly proposed metrics for quantifying adversary errors. 

\vspace{-9pt}
\subsection{Proposed PP-TPU Procedure}\label{subsec:method}\vspace{-3pt}
We propose  a new PP-TPU procedure. The PP-TPU procedure sanitizes GPS records locally via the planar Laplace mechanism to guarantee GI before the information is shared with the service provider (the strategy in Fig \ref{fig:adversary}(c)). This approach mitigates the privacy risks of learning new private information about an individual from the collected GPS records for various types of adversaries,  as only the users themselves possess the true responses. We also take several measures to improve the accuracy of the proposed PP-TPU procedure and to quantify  the utility of sanitized trajectories, detailed below.

First, given a fixed total per-trajectory privacy cost, we limit the number of records to be collected per traveler so that the sanitization of each location record does not inject too much noise to render the sanitized record useless.

Second, we filter out  non-usable trajectories for the PP-TPU  given a target route $R$. Due to the sanitization noise injected to satisfy GI at each location, the travel direction between two consecutive time points may be opposite to route $R$, which has  a pre-specified direction. Not to bias the total travel distance, we keep the  sanitized locations as is, as long as they can be mapped onto route $R$, but attach a sign to indicate the travel direction consistency with route $R$, namely, positive distance if the traveling direction  is consistent with the direction of route $R$, negative if opposite, and 0 if the two mapped locations completely overlap. After the complete set of the GPS records from the traveler is mapped, we sum the signed distances on $R$ for the traveler. If the summed distance is negative, then the trajectory is not usable, as defined in Definition \ref{def:usable}.
\begin{defn}[usable trajectory]\label{def:usable} A usable trajectory given a target route is a trajectory that satisfies the following two conditions: 1) at least two consecutive locations are mapped onto the target route $R$; 2) the total travel distance summed over distance segments calculated from the mapped coordinates on $R$ is non-negative.  The set of usable trajectories is the \emph{usable set $\mathcal{U}$}.
\end{defn}
\vspace{-3pt}
Third, we provide users an option to weigh different trajectories for their various levels of contribution towards the TPU on a given target route $R$. The motivation behind this is as follows. It is very likely that not all the GPS records will be mapped to route $R$, even if the traveler stays on $R$ all the time at least for the period of interest, for a few reasons. First,  GPS information is not always accurate due to satellite signal blockage and multipath effects, causing difficulty in road matching. Second, road mapping procedures themselves often involve approximation and errors. Third, with the additional randomness introduced by the GI sanitization, the location accuracy will further decrease. Therefore, each trajectory may have a different number of GPS records mapped onto $R$, some of which are consecutive in times and others are not. When calculating the travel distance on $R$ for a traveler, it makes sense to only count the distances between the locations at two consecutive time points if both are mapped onto  $R$. One way to formulate the weight is to let it be proportional to how much a sanitized mapped trajectory overlaps with the target route.
\begin{defn}[trajectory weight]\label{def:weight} Denote by $d^*_i$ the travel distance for traveler $i$ on the target route $R$ of length $d$ from the usable set $\mathcal{U}$. The weight that traveler $i$ carries in the TPU is  $w_i=d^*_i/d$.
\end{defn} \vspace{-3pt}

Algorithm \ref{alg:mechanism} lists the steps of our proposed PP-TPU procedure, with the above three utility-improvement measures implemented in various stages of the procedure.  \vspace{-6pt}
\begin{algorithm}
\caption{The PP-TPU Procedure}\label{alg:mechanism}
\SetAlgoLined
\SetKwInOut{Input}{input}
\SetKwInOut{Output}{output}
\Input{GPS location coordinates $(x_{ij},y_{ij})$ with timestamp $\tau_{ij}$ for $i\!=\! 1,\ldots,K$ trajectories and $j\!=\!1,\ldots, n_i (\le n$ the maximum records per trajectory); per-trajectory privacy budget $\epsilon_i$; target route with total distance $d$.} 
\Output{sanitized travel time $\mathbf{t}^*$, trajectory weight $\mathbf{w}$.} 
Usable set $\mathcal{U}\leftarrow\emptyset$\; 
\For{$i=1,\ldots,K$}{
$d^*_i\leftarrow0$; $\delta t_i\leftarrow0$\;
\For{$j=1,\ldots, n_i$}{
Perturb $P_j=(x_{ij},y_{ij})$ via the planar Laplace mechanism in Eq (\ref{Eq:xsys}) with privacy budget $\epsilon_i/n_i$ to yield $P^*_j=(x^*_{ij},y^*_{ij})$ at time $\tau_{ij}$\;
Map $P^*_j$ onto the area map to obtain the map coordinates $Q^*_j$\;
\If{($Q^*_{j-1}, Q^*_j$) for $j>1$ fall on Route $R$}{
Calculate the signed Euclidean distance $d_{ij}$ between  $Q^*_{j-1}$ and $Q^*_j$\;
$d^*_i\!\leftarrow\! d^*_i+d^*_{ij}$; $\delta t_i\!\leftarrow\! \delta t_i+(\tau_{ij}-\tau_{i,j-1})$\;} 
}
\If{$d^*_i\ge0$}{
$\mathcal{U}\leftarrow\mathcal{U}\cup i$\;
Calculate speed $s^*_i\!\!=\!d^*_i/(\delta t_i)$, predicted travel time $t^*_i=d/s^*_i$, and weight $w_i=d^*_i/d$.}
}
\end{algorithm} 
\vspace{-3pt}
With the output weights $\mathbf{w}$ from Algorithm \ref{alg:mechanism}, we can calculate the effective number of mapped full trajectories to provide an overall metric on the impact of mapping  and sanitation of GPS records on the TPU on a target road.
\begin{defn}[effective number of mapped full trajectories]\label{def:eff} The effective  number of mapped  full trajectories is $K_{\text{eff}}=\sum_{i\in\mathcal{U}}w_i$.
\end{defn}
Since $w_i\in[0,1]$, $K_{\text{eff}}\le|\mathcal{U}|$, where $|\mathcal{U}|$ is the number of trajectories in  $\mathcal{U}$. $|\mathcal{U}|$ in turn is $\le K$, where $K$ is the number of raw GPS trajectories before mapping, sanitation, and filtering out. $K_{\text{eff}}$ in a PP-TPU depends on $\epsilon$, the number of  GPS trajectories $K$ before mapping, and the pattern and complexity of the road networks onto which the GPS records are projected.   Besides using weights to calculate $K_{\text{eff}}$, we can also incorporate the weights in the TPU by define a weighted version of $f^*_w(t)$. For example, we may sample $K_{\text{eff}}$ travel times from set ($t^*_1,\ldots,t^*_{|\mathcal{U}|}$) with the sampling probabilities proportional to $\mathbf{w}=\{w_1,\ldots,w_{|\mathcal{U}|}\}$ and obtain an empirical $\hat{f}^*_w(t)$ based on the samples.


%-------------------------------------------------------------
\vspace{-6pt}\subsection{Accuracy of Sanitized Information}\label{sec:accuracy}\vspace{-3pt}
As mentioned above, road mapping procedures per se involve approximation and errors, the quantification of which is challenging and case-dependent. As such, we focus on the accuracy of the perturbed GPS records relative to their original, instead of on the mapped coordinates. It is reasonable to assume that if sanitized and original GPS records are close, so are their mapped locations. 

We quantify  the closeness between a  sanitized GPS location vs its original using the ``usefulness'' definition \cite{andres2013geo}. A location perturbation mechanism is $(\alpha, \delta)$-usefulness if the distance between the sanitized  and original locations is $\le\alpha$ with a probability of $1-\delta$, for every original location. For example, for a unit-distance privacy budget $\epsilon=2$, the probability that a sanitized location via the planar Laplace mechanism is within $\alpha=1.5$ units of the original location is $1-\delta=0.8$, calculated directly from the CDF of gamma(2, 1.5). In other words, the planar Laplace mechanism  of $\epsilon=2$ GI is $(1.5,0.2)$-useful for sanitizing locations. We plot the relationships between $\alpha$ and $1-\delta$ for a range of $\epsilon$ values for the planar Laplace mechanism  in Fig. \ref{fig:useful}(a). 

In addition, we may assess the accuracy of the distance between two  sanitized  locations.  Denote by $(x_j,y_j)$ and  $(x_{j'},y_{j'})$ the coordinates  of two recorded GPS locations at times $\tau_j$ and $\tau_{j'}$, respectively. The sanitized coordinates for the two locations via the planar Laplace mechanism in Eq (\ref{Eq:planar})  are respectively, 
\begin{align}
\begin{cases}
x^*_j=x_j+r\cos(\theta),\; y^*_j=y_j+r\sin(\theta)\\
x^*_{j'}=x_{j'}+r'\cos(\theta'),\; y^*_{j'}=y_{j'}+r'\sin(\theta')
\end{cases},
\end{align}
the distance between which can be calculated by the Euclidean distance 
\begin{align}
&d^{*2}_{jj'} = (x^*_j\!-x^*_{j'})^2 +(y^*_j\!-y^*_{j'})^2 = d^2_{jj'} +\Delta_{jj'},\mbox{ where}\label{Eq:dgeneral}\\
&\Delta_{jj'}\!=\!r^2\!+\!r'^2\!-\! 2rr'(\cos(\theta_j)\cos(\theta_{j'})+\sin(\theta)\sin(\theta'))+\notag\\ 
&\qquad\quad2(x_j-x_{j'})(r'\cos(\theta')-r\cos(\theta))+\notag\\ 
&\qquad\quad2(y_j-y_{j'})(r'\sin(\theta')-r\sin(\theta)),\label{Eq:delta}
\end{align}
and $d_{jj'}$ is the Euclidean distance between the original GPS records at times $\tau_j$ and $\tau_{j'}$. $\Delta_{jj'}$ can be regarded as the bias of the squared sanitized distance from the original distance,  $d^*_{jj'}$ conditional on  $d_{jj'}$  is a random variable as $r,r',\theta,\theta'$ are all random variables. We propose two metrics to examine the accuracy of $d^*_{jj'}$ relative to $d_{jj'}$. 
\begin{figure*}[t]
\includegraphics[width=1\textwidth, height=0.18\textheight]{useful_all.png}\vspace{-3pt}
\caption{Usefulness analysis on perturbed GPS location (a) and distances (b) to (d)} \label{fig:useful}\vspace{-9pt}
\end{figure*} 


For the first metric,  we define $(d,\alpha,\delta)$-usefulness for sanitized distances, in a similar manner   to the $(\alpha,\delta)$-usefulness in general \cite{blum} and for sanitized locations \cite{andres2013geo}.
\begin{defn}[\textbf{$(d,\alpha,\delta)$-usefulness of sanitized distance}] A randomization mechanism is $(d,\alpha,\delta)$-useful, if there is a probability of $1-\delta$ that  the sanitized distance $d^*$ satisfies $|d^*\!/d-1|<\alpha$ for every pair of locations with a distance of at least $d$.
\end{defn}\vspace{-3pt}
$\alpha$ is the relative error of the sanitized $d^*$ to the original $d$.  The smaller $\alpha$ and the larger $\delta$ are for a given $d$, the more useful the mechanism is in terms of distance preservation. Figs. \ref{fig:useful}(b) to \ref{fig:useful}(d) depict the relationship  between $\alpha$ and $\delta$ when the original distance $d$ is 5, 10, and 20 at different levels of unit-distance privacy cost $\epsilon$. As $d$ increases, $\delta$ decreases for the same $\alpha$. From the plots, we can claim that there is a 80\% probability that the distance $d^*$ between the perturbed locations via the planar Laplace mechanism of $\epsilon=1$ GI is within $\pm 25\%$ of $d\ge10$; in other words, the mechanism is $(10,  0.25, 0.2)$-useful at $\epsilon=1$. Similarly, we may also claim the mechanism is  $(5, 0.5, 0.2)$-useful for $\epsilon=1$, and $(5, 1.0, 0.3)$-useful for $\epsilon=0.5$, etc.


For the second utility  metric on sanitized distances, we calculate the expected \%deviation  $\E(d^*_{jj'}/d_{jj'}\!-\!1)$  and the \% root mean squared deviation (\%RMSD) $\sqrt{\E(d^*_{ij}/d_{ij}\!-\!1)^2}$ of the  sanitized distance from the original distance, respectively. Eqs (\ref{Eq:dgeneral}) and (\ref{Eq:delta}) suggest there is no closed-form expression for either of them; but we can always examine the numerical deviations for a given scenario. Table \ref{tab:bias} lists the expected \%deviation and \%RMSD in distance for different scenarios of $\epsilon$ and $d$.  As expected, the larger $\epsilon$ or the larger $d$ is, the smaller the \%deviation is.  Also listed in the table is the  expected \%deviation in squared distance  $\E(d^{*2})/d^2)\!-\!1$, which has a closed-form solution.  Specifically, $\E(r^2)=\E^2(r)+\V(r)=6\epsilon^{-2}$ ($\V$ denotes variance), so is $\E(r'^2)$; since $2(x_j-x_{j'})(\E(r')\E(\cos(\theta'))-\E(r)\E(\cos(\theta))+2(y_j\!-\!y_{j'})(\E(r')\E(\sin(\theta'))\!-\!\E(r)\E(\sin(\theta))=0$, then  
\begin{align}
\E(d^{*2}_{jj'}-d^2_{jj'})&= 12\epsilon^{-2}=O(\epsilon^{-2});\label{eqn:deviance}\\
\mbox{and } \E(d^{*2}_{jj'}/d^2_{jj'}-1)&= 12/(d_{jj'}\epsilon)^{2}. \label{eqn:pdeviance}
\end{align}
Eq (\ref{eqn:deviance}) indicates that, in expectation, the squared distance between two sanitized GPS locations always deviates from the squared original distance by a constant $12\epsilon^{-2}$ for a given $\epsilon$, regardless of $d_{jj'}$; however, Eq (\ref{eqn:pdeviance}) implies that the deviation is not meaningful for large $d_{jj'}$.

\vspace{-9pt}\subsection{Privacy Guarantee and Adversary Error} \label{sec:AE}\vspace{-3pt}
As illustrated in Fig \ref{fig:adversary}(c), the proposed PP-TPU procedure is based on sanitized GPS trajectory data, mitigating the privacy risk from both the honest-and-curious and malicious adversaries. The employed privacy model, GI, is an extension of the notion of DP to location settings with a
\begin{table}[!htb]
\caption{Expected \%deviation and \%RMSD in distance, and expected \%deviation in squared distance}
\label{tab:bias}\vspace{-6pt}
\centering
\resizebox{!}{0.125\columnwidth}{
\begin{tabular}{c@{\hspace{0.3\tabcolsep}}|@{\hspace{0.5\tabcolsep}} c@{\hspace{0.75\tabcolsep}}c@{\hspace{0.75\tabcolsep}}c@{\hspace{0.75\tabcolsep}}|@{\hspace{0.5\tabcolsep}} 
c@{\hspace{1\tabcolsep}}c@{\hspace{0.5\tabcolsep}}c|@{\hspace{0.5\tabcolsep}}
c@{\hspace{1\tabcolsep}}c@{\hspace{1\tabcolsep}}c@{\hspace{0.5\tabcolsep}} }
\hline
& 
\multicolumn{3}{@{\hspace{0.1\tabcolsep}}c@{\hspace{0.2\tabcolsep}}|}{$\left(\frac{\E(d^{*})}{d}\!-\!1\!\right)(\%)^\dagger$}&
\multicolumn{3}{@{\hspace{0.1\tabcolsep}}c@{\hspace{0.2\tabcolsep}}|}{$\sqrt{\E(\frac{d^*}{d}\!-\!1)^2}(\%)^\dagger$}&
\multicolumn{3}{c}{$\left(\frac{\E(d^{*2})}{d^2}\!-\!1\!\right) (\%)^\ddagger$} \\
\cline{1-10}
$d$  & 50 & 100 & 200 & 50 & 100 & 200 & 50 & 100 & 200 \\
\hline
$\epsilon\!=\!0.01$ & 5.00 & 2.09 & 0.75 & 6.17 & 2.80 & 1.23 & 48 & 12 & 3\\
$\epsilon\!=\!0.05$ & 0.51 & 0.13 & 0.03 & 0.95 & 0.47 & 0.24 & 1.92 & 0.48 & 0.12\\
$\epsilon\!=\!0.25$ & 0.02 & 0.00 & 0.00 & 0.19 & 0.10 & 0.05 &  0.0768 & 0.0192 & 0.0048 \\
\hline
\end{tabular}}
\begin{tabular}{l}
\footnotesize $^\dagger$ numerical results; $^\ddagger$ analytical results via  Eq. (\ref{eqn:pdeviance}). \hspace{0.75in}\textcolor{white}{.}\\
\hline
\end{tabular}\vspace{-12pt}
\end{table}
similar mathematical concept for controlling privacy loss when sharing information. DP is known to provide "provable privacy protection against a wide range of potential attacks, including those currently unforeseen'' \cite{nissim2017differential,dwork2017exposed}.  The proposed PP-TPU procedure in Sec \ref{subsec:method} protects several types of spatial-temporal information: the location of a traveler at a given time point, a travel trajectory of the traveler for a given time period, any calculated statistics from the trajectory (e.g, travel distance, travel speed) per the immunity property of DP and GI against post-processing. We examine each yielded privacy protection type below in detail, especially in the case  of a travel trajectory.

First, per the definition of GI in Definition \ref{def:GI}, the probability of distinguishing the true location $P$ from any other locations with a radius of $\gamma$, given the released perturbed location $P^*$ increases by $e^{\epsilon\gamma}-1$ folds compared to the probability  when not having $P^*$. In other words, the same privacy guarantees and indistinguishability as illustrated in Definition \ref{def:GI}  apply to the GPS records collected at each timestamp for the PP-TPU. 

Second, the proposed PP-TPU procedure protects the privacy of a collected travel trajectory over a time period. Though each of the location records  on the trajectory is perturbed via the planar Laplace mechanism has a straightforward interpretation on indistinguishability as presented above, how to quantify the adversary error in the learning of the original trajectory based on the released sanitized trajectory is less studied. Below we propose two metrics --  the \emph{ average distance (AD)} and  the \emph{consecutive positioning degree (CPD)} -- to quantify the adversary error and assess the effectiveness of a randomization procedure in protecting travel trajectory privacy.  We apply both metrics to examine the adversary error in the experiments in Sec. \ref{sec:exp}.
\begin{defn}[average distance]\label{def:etd}
The average distance (AD) between the sanitized and  original mapped travel trajectories on a  road network is the averaged distance between the two sets of mapped locations at the same set of timestamps from the two trajectories.
\end{defn}
We may calculate the AD empirically as follows. The pair of original and sanitized coordinates of the mapped trajectories $i$ ($i=1,\ldots,K$) are $\{(x_{ij},y_{ij})\}$  and  $\{(x^*_{ij},y^*_{ij})\}$, respectively, at time $\{\tau_{ij}\}$ for $j=1,\ldots,n_i$. The AD is given by
\begin{equation}\label{eqn:AD}
\textstyle K^{-1}\sum_{i=1}^K\left(n_i^{-1}\sum_{j=1}^{n_i}  d((x_{ij},y_{ij}), (x^*_{ij},y^*_{ij}))\right).
\end{equation}
Given a set $\{n_i\}_{i=1,\ldots,K}$, the larger AD, the larger the adversary error and the more difficult it is to recover the original trajectory from the sanitized trajectory (the reason that we define AD rather than ``total distance'' -- the summed distances between the GPS locations from two trajectories is that AD corrects for $n_i$, which may differ by trajectory).

\begin{defn}[consecutive positioning degree]\label{def:CPD}
The consecutive positioning degree (CPD) $p(l)$ is a probability  distribution  of correctly identified $l$ consecutive locations on a trajectory based on the released sanitized trajectory with $n$ GPS records, for $l=0,\ldots,n$. The expected value of correctly identified positions out of $n$ is $n_c=\sum_{l=0}^n l\times p(l)$.
\end{defn}

We choose to examine $p(l)$, the  distribution of correctly identified  \emph{consecutive} positions instead of correctly identified positions $p(m)$ for $m=0,\ldots,n$ (regardless of whether they are consecutive or not) because the former  would be regarded by many as more revealing of travel trajectory and carrying more privacy concern than latter. How to define ``correctly identified positions'' is up to the user.  One approach is hard-thresholding. Specifically, we choose a clip radius $C$. If the sanitized location falls within the circle of radius $C$  centered at the original location, then it is claimed as correct positioning. The smaller $C$ is, the harder it is to meet the criterion, but the more meaningful ``correct'' is. Since each location on a trajectory is perturbed independently via the polar Laplace mechanism, with the hard-thresholding rule, the probability of correctly identifying a location can be determined analytically, which is $p=F(C;2,\epsilon/n)$, where $n$ is the number of recorded positions on a GPS trajectory and $F$ is the CDF of gamma($2,\epsilon/n$). 

The number of correctly identified positions $m$ given $p$ follows $m\sim$ Binomial$(n,p)$. As for the distribution of CPD $l$, we can leverage Binomial$(n,p)$ to express $p(l)$ analytically when $n$ is small, but $p(l)$ for $1\le l< n-k$ with small $k\ge0$ becomes less tractable as $n$ increases considering that a trajectory may contain multiple location strings of different $l$. For example, a GPS trace with $n=10$ records may have 2 occurrences of $l=1$, 1  occurrence of $l=2$, and 1  occurrence of $l=3$. For cases where analytical calculation of $p(l)$ becomes difficult, we design Algorithm \ref{alg:CPD} that uses Monte Carlo (MC) simulations to calculate $p(l)$. Though the algorithm is presented with the hard-thresholding rule for correct positioning (line 4), the steps are applicable to other definitions of correct positioning. $n_i^{(l)}$ in the algorithm  refers to the frequency distribution $l$ in trajectory $i$, its average over $K$ trajectories gives the MC estimate $p(l)$. The algorithm also outputs $\bar{n}_c$, the MC estimate of the expected  value  of  correctly identified positions $n_c$ in Definition \ref{def:CPD}.

%------------------------------------
\vspace{-9pt}\section{Experiments}\label{sec:exp}\vspace{-3pt}
We conduct four experiments to investigate empirically the impact of sanitization of GPS trajectories on the utility of TPU in four road network scenarios. In each experiment, there is a pre-specified target route on which the TPU 
\setlength{\textfloatsep}{0pt}
\begin{algorithm}
\caption{Calculation of CPD $p(l)$}\label{alg:CPD}
\SetAlgoLined
\SetKwInOut{Input}{input}
\SetKwInOut{Output}{output}
\Input{$K$ GPS trajectories and their sanitized counterparts with $n$ records per trajectory; clip radius $C$} 
\Output{$n^{(l)}_i$ for $i=1,\ldots,K$; $p(l)=\sum_{i=1}^K n^{(l)}_i\left(\sum_{l=0}^n\sum_{i=1}^K n^{(l)}_i\right)^{-1}$;  $\bar{n}_c=K^{-1}\sum_{i=1}^K \sum_{l=0}^n(n_i^{(l)}\times l)$.}
\For{$i=1,\ldots,K$}{
\For{$j=1,\ldots,n$}{
Calculate the distance $d_{ij}$ between sanitized location $P^*_{ij}$ and original location $P_{ij}$\;
If $d_{ij}\le C$, then $e_{ij}=1$; else $e_{ij}=0$\; 
Let $e_{i0}=0$ and $e_{i,n+1}=0$\;}
If $e_{ij'}\!=\!0\;\forall j'\!=\!1,\ldots,n$, then  $n^{(0)}_i=1$; else $n^{(0)}_i=0$\;
If $e_{ij'}\!=\!1\;\forall j'\!=\!1,\ldots,n$, then  $n^{(n)}_i=1$; else $n^{(n)}_i=0$\;
\For{$l=1,\ldots,n-1$}{
$n^{(l)}_i\leftarrow0$\;
\For{$j=1,\ldots,n-l+1$}{
\If{$(e_{ij'}\!=\!1\;\forall j'\!=\!j,\ldots,j\!+\!l\!-\!1)
\;\&\; (e_{i,j-1} =0) \;\&\; (e_{i,j+l} =0$)}
{$n^{(l)}_i \leftarrow n^{(l)}_i+1$}}
}}
\end{algorithm} 
analysis is performed.  We examine the utility of PP-TPU for a range $\epsilon$ values and assess the adversary error in learning individual trajectories. Though a privacy-preserving travel time distribution may also be obtained by sanitizing the original empirical distribution via a DP mechanism, as illustrated in Fig.~\ref{fig:adversary}(b), the server needs to collect the actual individual GPS locations, and the sanitization is processed on the server. Therefore, this approach does not provide the same privacy guarantees as the decentralized and local approach (Fig.~\ref{fig:adversary}(c)) taken by Algorithm \ref{alg:mechanism}. Since it is impossible to match the level of privacy protection between the two approaches, the utility comparison would not be as meaningful; therefore, we choose not to compare our PP-TPU approach with the DP-based approach in the experiments.   


\vspace{-9pt}\subsection{Experiment Settings}\label{sec:setting}\vspace{-3pt}
In Experiment 1, the simulated  road network contains a single road. In Experiments 2, the simulated  road network contains three parallel roads with one being the target route.  In Experiment 3, the road network is around a large  roundabout in the town of Creteil in France (Fig \ref{cologne}(a)); the target route AB is about 1.5 kilometers long. In Experiment 4, we examine a region in the San Francisco Bay Area (Fig \ref{cologne}(b)); the target road AB is about 50 kilometers long. 
\begin{figure}[!htb]
\includegraphics[height=0.2\textwidth, angle=90]{roundaboutmap.png}
\includegraphics[width=0.27\textwidth]{sanF.png}\\
\centering (a) experiment 3 \hspace{0.7in} (b) experiment 4 \\
\centering\hspace{0.1in} Creteil, France  \hspace{0.7in}  San Francisco, USA\\
\vspace{-6pt}
\caption{Area maps in Experiments 3 and 4 (source: Google Map). AB is the target route for TPU in each experiment.} \label{cologne}
\end{figure} 

The GPS trajectory data in Experiments 1 (1,000 trips) and 2 (1,000 trips on the target road) are simulated as follows. We first simulated speeds from the inverse Weibull distribution with mean $\mu=24$ meter per second and variance $\sigma^2=8$ (the values are chosen to mimic some common real traffic speed distributions). Each simulated speed corresponds to one trip, on which 10 location records were generated at a fixed timestamp of every $\tau=20$ seconds, leading to travel trajectories of different lengths, depending on the speed. The vehicular mobility trace data in Experiment 3 \cite{vehicular} are  downloadable from \url{http://vehicular-mobility-trace.github.io/} and  contains 857,136  sets of location coordinates per second from around 5102 trips during the morning rush hour (7 to 9 AM), simulated based on real data. We randomly chose 1,000 trips within the rectangle bounded by the coordinates of the ends points A and B of the target route. The dataset in Experiment 4 \cite{sanf} contains real  mobility traces of taxi cabs and is downloadable from \url{http://crawdad.org/epfl/mobility/20090224/index.html}. It contains the GPS coordinates of approximately 500 taxis  over 30 days. For this experiment, we used a subset of 30,900 location-time GPS records over the morning  rush  hours (8 to 9 AM) from  419 trips.  In Experiments 3 and 4, we set the maximum number of GPS records per trip at 10 so to control the privacy loss per traveler. If a traveler has $\le10$ records, we used all of them; otherwise, we randomly sampled 10 records or  had  10 records spaced equally over the trajectory if there were enough records to allow that.

\vspace{-9pt}\subsection{Sanitization and Implementation Details}\vspace{-3pt}
The GPS records were sanitized via the  planar Laplace mechanism and projected into the road map in each experiment using the shortest path algorithm. The PP-TPU was then conducted via algorithm \ref{alg:mechanism} in each experiment. For the GI sanitization, we set the per-location per-meter privacy loss at 0.005, 0.01, 0.03, 0.05, and 0.08 in all 4 experiments. Since the maximum of GPS records per trip is 10, the total privacy cost for releasing a trajectory is $\le 0.05, 0.1, 0.3, 0.5, 0.8$, respectively.   

Fig \ref{fig:road} presents some examples of  sanitized GPS records and mapped travel trajectory on road networks given the GPS records. Take Experiment 2 as an example. Road 1 is the target road for TPU analysis. If there was no privacy concern, the three travelers would share their GPS records (blue circles) with the service provider who would project the records via a mapping algorithm onto the road network and use usable travel trajectories on road 1 to calculate travel time and carry out TPU. In this case, the mapped trajectories (cyan lines) fall on the target road for all three travelers. For PP-TPU, the service provider collect only sanitized versions (red squares) of the original GPS records; the mapping procedure and TPU analysis are the same as in the non-private setting. Since the sanitized GPS records deviate from their original counterparts, it is almost certain the trajectories after mapping also deviate from the original. For traveler 1, all ten sanitized GPS records are mapped onto road 1 and can be used for the subsequent travel time calculation. For traveler 2, eight out of the ten  sanitized GPS records are mapped on road 1 and two on the nearby road 2.  The eight records on road 1 form two location strings of length $l=4$ and $l=2$, respectively, that are used for the subsequent PP-TPU analysis. For traveler 3, three out of the ten sanitized GPS records are mapped onto road 1 but none of the two are consecutive in time, so traveler 3 does not contribute toward the PP-TPU. In summary, out of the sanitized trajectories from the three travelers, only those from travelers 1 and 2 contribute to $\mathcal{U}$.
\begin{figure}[!htb]
\vspace{-6pt}\centering
 Experiment 2 \\
\includegraphics[width=0.48\textwidth, height=0.23\textwidth]{2.png}\\
\vspace{3pt} 
Experiment 1 \hspace{0.3in} Experiment 3 \hspace{0.3in} Experiment 4\\
\includegraphics[width=0.48\textwidth, height=0.22\textwidth]{135.png}\\
\includegraphics[width=0.45\textwidth]{legendroad.png}
\vspace{-6pt}
\caption{Examples of  sanitized GPS records and mapped travel trajectories at per-trajectory privacy cost of $\epsilon=0.1$}\label{fig:road}
\vspace{-6pt}
\end{figure}

%-----------------------------------  
\vspace{-16pt} \subsection{Utility and PP-TPU Results}\label{sec:utility}\vspace{-5pt}
Fig \ref{fig:CDF} presents the empirical CDFs of the privacy-preserving travel times in the four experiments.  As expected, the sanitization deviates the travel time distribution $\hat{f}^*(t)$ from the original $\hat{f}(t)$; the smaller per-trajectory privacy cost $\epsilon$ is, the more deviation there is. At $\epsilon\ge0.3$, $\hat{f}^*(t)$ is close to $\hat{f}(t)$ and satisfactory  utility can be reached for PP-TPU in all experiments.  From the CDF curves, we can read how quickly a traveler arrives at the destination with a certain level of confidence, and vice versa. For example, in Experiment 4, there is an 80\% probability that a traveler finishes the trip AB within 100 minutes if $\epsilon=0.5$ is used.  In addition to the unweighted TPU in Fig \ref{fig:CDF}, we also performed the weighted TPU analysis; the results are presented in Fig \ref{fig:weight}. A similar overall trend across $\epsilon$ is observed as in the non-weighted setting. In experiments 1 and 2, the weighting seems to affect $\hat{f}^*(t)$ more for smaller $\epsilon$, and the left tail of $\hat{f}^*(t)$ (smaller $t$) is more sensitive to the weighting than the right tail. In experiments 3 and 4, the weighted distributions are similar to the unweighted version across all $\epsilon$.
\begin{figure}[!htb]
\centering
\hspace{0.3in} Experiment 1 \hspace{0.8in} Experiment 2 \\
\includegraphics[width=0.245\textwidth]{CDF1.png}\hspace{-4pt}
\includegraphics[width=0.245\textwidth]{CDF2.png}\hspace{-4pt}
\hspace{0.3in} Experiment 3 \hspace{0.8in}  Experiment 4\\
\includegraphics[width=0.49\textwidth]{real35cdf.png}
\includegraphics[width=0.4\textwidth]{CDFlegend.png}
\vspace{-6pt}\caption{PP-TPU}\label{fig:CDF}\vspace{-9pt}
\end{figure}
\begin{figure}[!htb]
\centering
\hspace{0.3in} Experiment 1 \hspace{0.8in}  Experiment 2\\
\includegraphics[width=0.24\textwidth, height=0.22\textwidth, trim={6pt 0 0 0}, clip]{exp1weigihted.png}
\includegraphics[width=0.24\textwidth, height=0.22\textwidth, trim={6pt 0 0 0}, clip]{exp2weigihted.png}\\
\hspace{0.4in} Experiment 3 \hspace{0.8in}  Experiment 4\\
\includegraphics[width=0.24\textwidth, height=0.22\textwidth, trim={6pt 0 0 0}, clip]{real3weighted.png}
\includegraphics[width=0.24\textwidth, height=0.22\textwidth, trim={6pt 0 0 0}, clip]{real4weighted.png}\\
\vspace{3pt}
\includegraphics[width=0.4\textwidth]{CDFlegend.png}\vspace{-6pt}
\caption{Weighted PP-TPU} \label{fig:weight}
\end{figure} 

Table \ref{tab:Keff} presents the effective number of mapped full trajectories $K_{\text{eff}}$.   Due to the inherent error of the mapping algorithm, not every GPS record can be mapped onto the actual route where the traveler is on, or yield a sensible trajectory after mapping. Therefore, $K_{\text{eff}}$ is smaller than the number of trips even without any GI sanitization. With the GI sanitization and as $\epsilon$ decreases, $K_{\text{eff}}$ further decreases, as expected.
\begin{table}[!htb]\vspace{-6pt}
\caption{Effective number of mapped full trajectories $K_{\text{eff}}$}\label{tab:Keff}\vspace{-6pt}
\resizebox{1\columnwidth}{!}{\centering
\begin{tabular}
{c@{\hspace{0.5\tabcolsep}}|c@{\hspace{0.5\tabcolsep}}| c@{\hspace{0.5\tabcolsep}}c@{\hspace{0.9\tabcolsep}} c@{\hspace{0.9\tabcolsep}}c@{\hspace{0.9\tabcolsep}} c@{\hspace{0.9\tabcolsep}}|c@{\hspace{0.5\tabcolsep}}| c@{\hspace{0.9\tabcolsep}}c@{\hspace{0.5\tabcolsep}}}
\hline
&  &\multicolumn{5}{c|}{$\epsilon$} & original (no & \# trips\\
\cline{3-7}
& experiment & 0.05& 0.1 & 0.3 & 0.5 & 0.8 & sanitization) &\\ 
\hline 
&  1 & 792 & 853 & 873 & 889 & 892&901&1,000\\
&  2 & 682 & 721 &820  &834   & 845 & 876 &1,000\\
$K_{\text{eff}}$ &  3 &  229 & 314 &435  &460   & 478 & 513 &1,000\\
& 4 & 45 & 49 &52  &52   & 53 & 53 &419\\
\hline
\end{tabular}}\vspace{-6pt}
\end{table}

In summary, we can draw the following conclusions from the utility analysis in this subsection. (1) The quality of the PP-TPU analysis relates to the type and structure of the road network onto which the GPS records are mapped; some road networks are more sensitive to $\epsilon$ than others in the utility of sanitized trajectories. (2) The difference between the unweighted and weighted TPU analysis diminishes as  $\epsilon$ increases. (3) The CDFs of the privacy-preserving travel time in the 4 experiments are similar to  the original CDFs with the per-trajectory $\epsilon$ as small as $\approx0.3$, so is the effective number of mapped full trajectories, implying useful TPU analysis can be achieved with satisfactory privacy guarantees. 


%------------------------------------------------
\vspace{-12pt}\subsection{Adversary Error}\vspace{-3pt}
Table \ref{tab:adversary} shows the expected AD between a sanitized and its original mapped trajectories calculated via Eq (\ref{eqn:AD}). Note that the 100 repeats were generated differently for experiments 1 and 2 vs. experiments 3 and 4 because the former two are synthetic data while the latter two  are quasi-real and real datasets, respectively. Specifically,  in experiments 1 and 2, we generated 100 GPS data sets per the simulation setting in Sec \ref{sec:setting}; in experiments 3 and 4, the 100 repeats were obtained by performing 100 sets of sanitization on a fixed GPS dataset in each experiment. As a result, the variability of AD  comes from two sources -- sampling error and sanitation error -- in experiments 1 and 2 and contains only the sanitization error in experiments 3 and 4.

The first observation is that the smaller $\epsilon$ is, the larger the distance is, as expected. Second, the AD value varies across the experiments for the same $\epsilon$, which makes sense, as the AD works with the distance between a pair of locations on a map and the road network matters. Given that the road networks differ in the four experiments, it is not surprising that the AD varies by experiment. Third, the adversary error measured by the AD at $\epsilon\le0.3$ is sufficiently large per location on a trajectory for each experiment ($\ge30$ meters). 
\begin{table}[!htb]
\caption{Mean (SD) average distance between mapped locations on sanitized and original trajectories (100 repeats)}\label{tab:adversary} \vspace{-6pt}
\resizebox{1\columnwidth}{!}{\centering
\begin{tabular}
{c@{\hspace{0.2\tabcolsep}}|c@{\hspace{0.2\tabcolsep}}| c@{\hspace{0.1\tabcolsep}}c@{\hspace{0.5\tabcolsep}} c@{\hspace{0.1\tabcolsep}}c@{\hspace{0.9\tabcolsep}} c@{\hspace{0.9\tabcolsep}}c@{\hspace{0.9\tabcolsep}} c@{\hspace{0.9\tabcolsep}}c@{\hspace{0.1\tabcolsep}}}
\hline
&  &\multicolumn{5}{c}{$\epsilon$}  \\
\cline{3-7}
& experiment & 0.05& 0.1 & 0.3 & 0.5 & 0.8 \\ 
\hline 
&  1 &180 (6.7) &87 (2.0) &30 (0.9)  & 18 (0.3) &11 (0.1) \\
AD$^\dagger$  &  2 &1189 (16.4) &814 (6.8) &435 (4.9)  & 341 (4.5) &296(2.4)\\
(meters)&  3 & 739 (24.1)& 438 (32.3) &185 (31.4)  & 121 (5.4) &99 (2.1)\\
& 4 & 214 (21.5) &108 (11.4)  &38 (3.7)   & 23 (2.1)  &13 (1.4)\\
\hline
\end{tabular}}
\end{table}

Fig. \ref{fig:cpd} presents the probability distributions of CPD $l$ and the correctly identified positions $m$ (whether consecutive or not) for three different clip radius $C$  (20, 40, and 80 meters) when the number of records per trajectory $n=10$ for different $\epsilon$. Since all 4 experiments used the same $n$ and $\epsilon$ value, the results in Fig. \ref{fig:cpd} apply to all four experiments. As expected, as $C$ increases (the criterion for claiming correct positioning loosens) or as per-trajectory $\epsilon$ increases, the adversary's accuracy for correctly identifying more positions and more consecutive positions increases. In the case of $C=80$ meters -- a rather relaxed criterion for correct identification, the probability of identifying 10 positions out of 10 is $>80\%$. The probability decreases to $\sim10\%$ for $C\!=\!40$ meters and  $\sim0\%$ for $C\!=\!20$ meters.  The plots also illustrate the differences between CPD $l$ and the number of correctly identified locations $m$. For example, for $C=20$, $\Pr(l\!=\!6)$ is close to 0\%, but $\Pr(m\!=\!6)$ is $\sim20\%$, regardless of whether the 6 positions are consecutive or not.
\begin{figure}[!htb]
\centering\vspace{-9pt}
\includegraphics[width=0.48\textwidth,height=0.5\textheight]{cpd.png}\vspace{-3pt}
\caption{Probability distributions of CPD $l$ (left column) and correctly identified positions $m$ (right column)} \label{fig:cpd}\vspace{-9pt}
\end{figure} 

In summary,  we  can draw the following conclusions from the adversary error analysis in this subsection. (1) The magnitude of the adversary error closely relates to the road network type and structure. (2) The adversary error in reconstructing a trajectory from the sanitized trajectory around $\epsilon\le0.3$ is sufficiently large per the measures of AD and CPD to not pose serious privacy threats. (3) Taken together with the observations in the utility analysis,  a good trade-off between the PP-TPU utility and privacy protection can be achieved at per-trajectory $\epsilon\approx0.3$ with $\!\le\!10$ GPS records per trajectory in these 4 experiments. Users of the PP-TPU procedure can run similar analysis and choose an $\epsilon$ that leads to a good balance between utility and privacy protection for their specific problems.  

%-------------------------------------------------------------
\vspace{-9pt}\section{Conclusions}\vspace{-6pt}
This paper addresses privacy-preserving TPU analyses. We employ the notation of GI to protect individual GPS spatial-temporal records and the subsequent TPU analysis. The proposed PP-TPU procedure can be adopted by service providers (e.g., mobile phone companies, GPS navigator apps) at the GPS data collection stage. We define the effective number of mapped full trajectories, the  usefulness concept, and different types of deviations in distance measures based on sanitized GPS records to quantify the utility of the sanitized trajectories. We also propose the concepts of average distance and consecutive positioning degree to assess the adversary error based on released GPS trajectory records.  Our analytical results and empirical studies suggest that it is feasible to employ the GI concept to collect and release GPS information for TPU analysis while guaranteeing location privacy for the individuals who contribute their GPS data.  Our future work will look into incorporating the dependency among the location points on the same travel trajectory and better utilizing the public road network maps to develop new randomization mechanisms of better utility without comprising privacy.

\vspace{-12pt}
\section*{Acknowledgments}\vspace{-3pt}
Fang Liu is supported by NSF Grant \#1717417 and Dong Wang is supported by the  China Scholarships Council program (NO. 201906270230) and NSFC Grant \#41971407. We also thank the editor, associate editor, and five reviewers for their useful comments and suggestions on the manuscript.

\vspace{-12pt}
\bibliographystyle{IEEEtranN}
%\bibliography{mylib}
\documentclass[amsmath,amssymb,aps,pra,superscriptaddress,twocolumn]{revtex4-2}
\usepackage[subpreambles=true]{standalone}
\usepackage{lmodern}
\usepackage{stmaryrd}
\usepackage{graphicx,color}
\usepackage{subfigure}
\usepackage{units}
\usepackage{verbatim}
\usepackage{dsfont}
\usepackage{bm}
\usepackage{mathrsfs}
\usepackage{hyperref}
\hypersetup{colorlinks=true, linkcolor=black, citecolor=black, urlcolor=black}
\usepackage{amsmath,amssymb,amsthm,mathrsfs,bbm,bm,mathtools}
\usepackage{comment}
\usepackage{enumerate}
\usepackage{braket}
\usepackage{todonotes}
\usepackage[titletoc,title]{appendix}
\usepackage{xypic}
\usepackage{multirow}
\usepackage{soul}
\usepackage{bibunits}

\newcommand{\Red}[1]{\textcolor{red}{#1}}
\newcommand{\m}{\vspace{0.2in}}
\newcommand{\mm}{\vspace{0.09in}}
\newcommand{\missingcite}{{\color{magenta} [~]}}
\newtheorem{definition}{Definition}
\newtheorem{remark}{Remark}
\newtheorem{theorem}{Theorem}
\newtheorem{proposition}[theorem]{Proposition}
\newtheorem{lemma}[theorem]{Lemma}
\newtheorem{coro}[theorem]{Corollary}
\newtheorem{conjecture}[theorem]{Conjecture}
\newtheorem{innercustomlemma}{Lemma}
\newenvironment{customlemma}[1]
  {\renewcommand\theinnercustomlemma{#1}\innercustomlemma}
  {\endinnercustomlemma}

\newtheorem{innercustomcor}{Corollary}
\newenvironment{customcor}[1]
  {\renewcommand\theinnercustomcor{#1}\innercustomcor}
  {\endinnercustomcor}

\begin{document}
\title{Erratum: Observing a changing Hilbert-space inner product}
\author{Salini Karuvade}
\affiliation{Centre for Engineered Quantum Systems, School of Physics, 
University of Sydney, Sydney, New South Wales 2006, Australia}
\author{Abhijeet Alase }
\affiliation{Centre for Engineered Quantum Systems, School of Physics, 
University of Sydney, Sydney, New South Wales 2006, Australia}

\author{Jacob L.\ Barnett}
\affiliation{Perimeter Institute for Theoretical Physics, 31 Caroline Street North, Waterloo, Ontario N2J 2Y5, Canada}
\affiliation{Department of Physics \& Astronomy, University of Waterloo, Waterloo, Ontario
N2L 3G1, Canada}
\author{Barry C.\ Sanders}
\affiliation{Institute for Quantum Science and Technology, University of Calgary, Alberta T2N 1N4, Canada}

\maketitle

\onecolumngrid
% \begin{widetext}
Lemma~2 
in Appendix~A of our original paper,
which was
first presented in 1992~\cite{SGH92},
is incorrect as we explain in this erratum.
This lemma should be replaced by the following lemma and corollary.
\begin{customlemma}{2}
A bounded positive-definite operator, $\eta\in\mathcal{B}\left(\mathscr{H}\right)$, is invertible if and only if $\eta$ is surjective. 
\end{customlemma} 
\begin{customcor}{2.1}\label{cor}
Let $\eta\in\mathcal{B}\left(\mathscr{H}\right)$ 
be a surjective, positive-definite operator.
Then, $\eta^{-1}$ is bounded and positive-definite.
\end{customcor} 

\noindent Neither paper articulated a complete set of constraints required for the invertibility of the metric 
operator $\eta$.
Specifically, the inverse of a bounded positive-definite operator, $\eta\in\mathcal{B}\left(\mathscr{H}\right)$,
in the absence of the surjectivity constraint
is not necessarily bounded. 

A simple counter-example  to the original Lemma 2 is now presented.
Consider a separable Hilbert space $\mathscr{H}$ with a countably infinite orthonormal basis, $\{\ket{e_j}, j \in \mathbb{N}\}$. Consider the bounded positive-definite operator 
\begin{equation}\label{eq:cex}
    \eta: \ket{e_j} \mapsto \frac{1}{j}\ket{e_j} \quad \forall j \in \mathbb{N}.
\end{equation}
On the dense subspace of $\mathscr{H}$ spanned by finite linear combinations of 
$\{\ket{e_j}, j \in \mathbb{N}\}$ the inverse of $\eta$ is given
by 
\begin{equation}
    \eta^{-1}:\ket{e_j} \mapsto j\ket{e_j} \quad \forall j \in \mathbb{N},
\end{equation}
which is clearly not bounded
on the whole $\mathscr{H}$.



For $\eta$ to be invertible, we need to enforce an additional constraint,
namely that the range of $\eta$ coincides with $\mathscr{H}$, or in other words, $\eta$ is surjective.
Below we prove the corrected version of Lemma~2, which is stated above.
\begin{proof}
An operator is injective if and only if its kernel contains only the zero vector. With this fact in mind, a straightforward proof by contradiction demonstrates the injectivity of all positive-definite operators (if  $\ket{\psi}\in\ker(\eta) \setminus \bm{0}$, then 
$\braket{\psi|\eta|\psi}=0$, thereby violating the positive-definiteness of $\eta$). Together with the supposition that $\eta$ is surjective, we find that $\eta$ is a bijection. Thus, the bounded inverse theorem \cite[Thm.~14.5.1]{Narici2010} implies $\eta^{-1}\in\mathcal{B}\left(\mathscr{H}\right)$.
\end{proof}


Corollary~\ref{cor} now follows from
$\braket{\psi|\eta^{-1}|\psi} = (\bra{\psi}\eta^{-1})\eta(\eta^{-1}\ket{\psi}) > 0$ 
for any $\ket{\psi} \in \mathscr{H}$,
where we used the positive-definiteness of $\eta$ in the last inequality.
We remark that self-adjointness of the metric operator follows from 
its positive-definiteness by polarization identity~\cite{BB03},
and therefore need not be assumed separately as in the original 
version of Lemma~2. 
Similarly, if $\eta\in \mathcal{B}(\mathscr{H})$ is positive-definite and surjective,
then $\eta^{-1} \in \mathcal{B}(\mathscr{H})$ is positive-definite
by Corollary~\ref{cor} above, and therefore self-adjoint.


Invertibility of the metric operator also guarantees that $\mathscr{H}_\eta$ defined in
the original paper is a Hilbert space with respect to the modified inner product 
$\braket{\bullet|\bullet}_\eta$~\cite[Appendix A]{SGH92}. 
Subsequently, surjectivity of $\eta$,
in addition to boundedness and positive-definiteness, is required for defining 
the change in representation  $\mathcal{R}_\eta$ [Eq.~(2) of the original paper] and 
the inner-product changing operation $\mathcal{E}_{\eta}$Eq.~(4) of the original paper].


Our results concerning the simulation of PT-symmetric Hamiltonians hold without any
additional assumptions.
This is because we exclusively work with unbroken PT-symmetric Hamiltonians in finite dimensions. The rank-nullity theorem implies that all injective operators with finite-dimensional domains are additionally surjective. Therefore, Lemma~2 as originally stated in the paper is valid
for any finite-dimensional Hilbert space $\mathscr{H}$. 

We also note that for any unbroken PT-symmetric Hamiltonian in 
a separable, infinite dimensional Hilbert space $\mathscr{H}$, an $\eta \in \mathcal{B}(\mathscr{H})$
satisfying the quasi-Hermiticity condition 
that is both surjective and positive-definite can be constructed~\cite[Thm.~4.3]{Kar22}
(see also~\cite[Eq.~(23)]{BiOrthogonal}).
This is a consequence of the fact that unbroken PT-symmetric Hamiltonians in infinite dimensions are defined 
to have eigenvectors forming a Riesz basis, in addition to these vectors 
being invariant under the action of the PT operator~\cite{Mos10b}.


In conclusion, amendment to Lemma~2 mandates that the operations $\mathcal{R}_\eta$ [Eq.~(2) of the original paper] and $\mathcal{E}_\eta$ [Eq.~(4) of the original paper] are to be defined only under the additional constraint that $\eta$ is surjective. 
The rest of our original paper is correct and remains unchanged after these modifications.

\ 

\paragraph*{Author Contribution Statement:} JLB pointed out the error in Lemma 2 of the original paper and provided the counter example given in this erratum. SK and AA developed the correct formulation of Lemma 2. SK and AA wrote the draft of the erratum with feedback from JLB and BCS.

\pagebreak

\begin{center}
{\bf \large Observing a changing Hilbert space inner product} 

\ 

{Salini Karuvade, Abhijeet Alase, and Barry C. Sanders}

{\small \it Institute for Quantum Science and Technology, University of Calgary,\\
2500 University Drive NW, Calgary, Alberta T2N 1N4, Canada}

\ 

\setlength{\fboxsep}{0pt}%
\setlength{\fboxrule}{0pt}%

\noindent\fbox{%
    \parbox{0.8\textwidth}{%
\small        
In quantum mechanics,
physical states are represented by rays in Hilbert space~$\mathscr H$,
which is a vector space imbued by an inner product~$\langle\,|\,\rangle$,
whose physical meaning arises as the overlap~$\langle\phi|\psi\rangle$
for~$\ket\psi$ a pure state (description of preparation)
and~$\bra\phi$ a projective measurement. However,
current quantum theory does not formally address the consequences of 
a changing inner product during the interval between preparation and measurement.
We establish a theoretical framework for such a changing inner product,
which we show is consistent with standard quantum mechanics.
Furthermore, we show that this change is described by a quantum operation,
which is tomographically observable,
and we elucidate how our result is strongly related to the 
exploding topic of PT-symmetric quantum mechanics.
We explain how to realize experimentally a changing inner product for a 
qubit in terms of a qutrit protocol with a unitary channel.
    }%
}

\end{center}

\twocolumngrid
Hilbert-space inner product is fundamental to quantum mechanics (QM), and its physicality relates 
to norm through the Born interpretation and to fidelity and distinguishability through its complex angle~\cite{SN21}.
The uniqueness of the inner product associated to a quantum system has come under scrutiny 
following the advent of PT-symmetric QM.
PT-symmetric systems are described by non-Hermitian Hamiltonians invariant under the combined action of parity~(P) and time~(T) 
inversion symmetries~\cite{BB98,BBJ02,BBJ03,Ben05}, and they are predicted to exhibit novel physical phenomena which
have been simulated on a variety of experimental platforms~\cite{RME+10,SLZ+11,BDG+12,POS+14,ZZS+16,XZB+17,EMK+18,WLG+19,ZLW+20}. 
These phenomena have been explained by observing that non-Hermitian Hamiltonians with 
unbroken PT symmetry are Hermitian with respect to a different Hilbert-space inner product~\cite{BBJ02,Mos03a,MECM08,JMCN19}.
Changing Hilbert-space inner-product is valuable for certain quantum information processing (QIP) tasks~\cite{Cro15} 
such as non-orthogonal state discrimination~\cite{BBC+13}, cloning~\cite{ZWX+20} and quantum algorithms~\cite{BBJM07,Mos09},
but perfunctory applications have led to counter-factual conclusions~\cite{Cro15,Pat14,CCC14} 
including violation of the no-signalling principle~\cite{YHFL14}. 
Our aim is to prescribe the correct procedure for changing Hilbert-space inner product and
to devise an experiment to validate our prescription.



Consistency  of a changing Hilbert-space inner product
with standard QM  and the 
unobservability of such a change in closed systems 
have been investigated.
A C$^*$-algebraic approach shows that
a set of non-Hermitian operators comprises the
observables of a
quantum mechanical system if and only if the operators are Hermitian with respect to
a new Hilbert-space inner product~\cite{SGH92}.
Such a modified inner product is the key to proving the equivalence
of PT-symmetric QM with 
the Dirac-von Neumann formulation of QM in the case of closed systems i.e., systems in which 
every time evolution is a unitary operation~\cite{BBJ02,Mos03a,Mos10a,Mos10b,Zno15,Mos18,ZWG19,JMCN19}.
Furthermore, 
this equivalence implies that any change in inner product is unobservable in experiments on closed systems~\cite{Bro16}.
Therefore, the above proposals that use the inner-product change for QIP tasks
as well as the counter-factual claims are not applicable to closed systems.



Evolution generated by PT-symmetric Hamiltonians has been implemented experimentally
for applications including sensing~\cite{LZO+16,COZ+17,HHW+17}, cloaking~\cite{ZFZ+13,SFA15}
and unidirectional propagation~\cite{RKEC10}.
These experiments simulate PT-symmetric dynamics on classical~\cite{RME+10,SLZ+11,POS+14,ZZS+16} or
quantum~\cite{BDG+12,XZB+17,LHL+19,XQW+19} systems by balancing loss and gain.
Another way to simulate  PT-symmetric Hamiltonians with real spectra is by 
dilating the non-unitary propagator to a non-local unitary operator over multiple subsystems, which has been demonstrated 
on qubit systems~\cite{GS08a,GS08b,ZHL13,TWY+16,KAU17,HKW18,WLG+19,XWZ+19,GZL+21}.
However, none of these simulation strategies involve effecting a change 
of inner product.


PT-symmetric Hamiltonians and a changing Hilbert-space inner product
are known to be consistent with standard QM for closed systems,
but they are not yet known to be consistent for open systems.
To solve these outstanding problems, we 
construct an operational framework, consistent with the C$^*$-algebraic formulation of QM,
which accommodates a change in inner product between preparation and measurement.
Furthermore, neither PT symmetry nor a changing Hilbert-space inner product are observable in
closed systems, but could be observable in open systems~\cite{Bro16}.
We show our change in inner product is implemented by a quantum operation
(henceforth assumed to be completely positive and trace non-increasing),
which can be observed using tomography.
Next we connect our framework to the burgeoning topic of PT-symmetric QM
by explaining how an inner-product-changing quantum operation can be 
used to implement PT-symmetric dynamics in an open system.
Finally, at the empirical level, we describe a potential experimental simulation for changing the inner product 
of a qubit by subjecting a qutrit to unitary evolution and 
neglecting the third Hilbert-space dimension during preparation and measurement but not during evolution.
We also extend this simulation procedure to $d$-dimensional systems. 



To construct the operational framework for changing the inner product associated to a quantum system
between preparation and measurement,
we adopt the C$^*$-algebraic framework of QM~\cite{Str08},
which provides freedom in representing 
a given system on different Hilbert spaces following the Gel'fand-Naimark-Segal (GNS) construction~\cite{GN43,Seg47}.  
We employ this representation freedom 
first to construct representations of the C$^*$ algebra on a pair of Hilbert spaces 
whose inner products are related by a given metric operator~$\eta$. We then
define the change in inner product by~$\eta$ as the identity
isomorphism between the two Hilbert spaces.
To operationalize the change in inner product, we use commutative diagrams that connect 
this isomorphism to a quantum operation 
between the bounded operators on the two Hilbert spaces and 
finally observe that the quantum operation induces an observable physical transformation on the system.

In the operational approach, the operators of a quantum system form a unital C${}^*$ algebra
$\mathcal{A}=\{A\}$, which is equipped with a ${}^*$ operation that 
captures the notion of adjoint. The algebra~$\mathcal{A}$ is representable on a 
possibly infinite dimensional 
Hilbert space~$\mathscr H=(\mathscr{V},\langle\,\vert\,\rangle)$, 
comprising a complete vector space~$\mathscr V$
and an inner product~$\langle\,\vert\,\rangle$, which is a non-degenerate sesquilinear form.
In Fig.~\ref{fig:commutativediagram}(a),
observables are self-adjoint elements of~$\mathcal{A}$ and correspond to allowed measurements.
A representation of $\mathcal{A}$ is a product-preserving linear map 
\begin{equation}
\label{eq:pi*dagger}
\mathcal{A} \stackrel{\pi}{\to} \mathcal{B}(\mathscr{H}):
\pi(A^*) = (\pi(A))^\dagger,
\end{equation}
where~$\mathcal{B}(\mathscr H)$ denotes the space of bounded linear operators
acting on~$\mathscr H$ and $^\dagger$ denotes the Hermitian conjugate.
Such a representation can be obtained using the GNS construction~\cite{GN43,Seg47}.
Product preservation ensures that if $I$ is the identity operator in $\mathcal{A}$, then
$\pi(I)$ is the identity operator in~$\mathcal{B}(\mathscr H)$.
An operator $M\in \mathcal{B}(\mathscr H)$ satisfying $M^\dagger = M$ is called a self-adjoint or a Hermitian operator.
\begin{figure}
\begin{center}
\includegraphics[width = \columnwidth]{Fig1.pdf}
\end{center}
\caption{
(a)~Diagram illustrating the relation between the ${}^{*}$ operation of $\mathcal{A}$ and the 
$\dagger$ operation of $\mathcal{B}(\mathscr{H})$ under the representation $\pi$.
(b) Commutative diagram depicting the change of representation from $\pi$ to $\pi_{\eta}$ under $\mathcal{R}_\eta$. Operationally, 
$\mathcal{R}_\eta$ represents a trivial transformation, i.e., no change, in $S$. (c)~Commutative diagram illustrating the relation between 
the maps $\mathcal{F}_\eta$,
$\mathcal{E}_\eta$ and $\mathcal{I}_{\eta}$. }
\label{fig:commutativediagram}
\end{figure}

We now define states and explain how to represent states as operators on Hilbert space.
States correspond to allowed preparations of the system
(Fig.~\ref{fig:commutativediagram}(b)).
A state $\omega$ is a positive linear functional on $\mathcal{A}$
that is normalized, i.e. $\omega(I) = 1$.
This definition is extended to include any subnormalized positive linear functional,
i.e.\ $\omega(I)\le 1$, which corresponds to probabilistic preparation in the state 
$\omega/\omega(I)$ with probability $\omega(I)$~\cite{Gud79}.
Supernormalized positive linear functionals are not valid states
according to this probabilistic interpretation~\footnote{
We remark that supernormalized functionals are valid states in some other frameworks
for describing system evolution under non-Hermitian Hamiltonians~\cite{Mos07,GKN10,UGRM12}
}.
Let $\mathcal{D}(\mathscr{H}):= \{\rho:\rho \ge 0,\rho=\rho^\dagger,
\text{tr}(\rho)\leq 1\} \subset \mathcal{B}(\mathscr{H})$ denote the set of density operators
with $\rho\in \mathcal{D}(\mathscr H)$ representing a state~$\omega$ 
by~$\rho\stackrel{\!\!{}^{\#}\!\!\pi}{\mapsto}\omega$
if and only if the expectation value~tr$(\rho\pi(A)) = \omega(A)\,\forall A$.
As ${}^\#\pi$ is uniquely determined by $\pi$, we say
that $\omega$ is represented by $\rho\in \mathcal{D}(\mathscr H)$ under $\pi$. 
We denote by  $\mathcal{S} = \{\omega\}$, the set of all states that are represented by density operators under $\pi$.
For the special case of pure~$\omega$,
$\rho = \ket{\psi}\bra{\psi}$ for some $\ket{\psi}\in \mathscr H$ with $\braket{\psi | \psi} \le 1$.
The transformation $\ket{\psi}\stackrel{\operatorname{lift}}{\mapsto}\ket{\psi}\bra{\psi}$ relates Hilbert-space
vectors to the density operators in~$\mathcal{D}(\mathscr H)$~\footnote{
The map lift is defined to act only on the normalized and subnormalized vectors in $\mathscr{H}$.
The domain of lift in Fig.~\ref{fig:commutativediagram}(c) is shown to be $\mathscr{H}$ for simplicity, but is specified 
rigorously in Appendix~\ref{sec:liftmap}.
}.


Now that we have explained states and their representations,
we now discuss changing representation
to being over a different Hilbert space.
Given a self-adjoint positive-definite metric operator~$\eta\in\mathcal{B}(\mathscr{H})$,
a new Hilbert space~$\mathscr{H}_\eta = (\mathscr{V},\langle\,\vert\,\rangle_\eta)$ can be constructed~\cite{SGH92} 
such that the inner products of the two Hilbert spaces are related by
$\langle \bullet\,\vert\,\bullet\rangle_\eta := \langle\bullet\vert\,\eta\bullet\rangle$.
A representation $\pi_{\eta}$ on $\mathscr{H}_{\eta}$ can be constructed 
through~(see Appendix~\ref{sec:newHilbertspace})
\begin{equation}
\pi(\bullet)\stackrel{\mathcal{R}_{\eta}}{\mapsto}\pi_{\eta}(\bullet):=R\pi(\bullet)R^{-1},
\mathcal{R}_\eta:\mathcal{B}(\mathscr H)\to\mathcal{B}(\mathscr H_\eta),
\end{equation}
where $R = \eta^{\nicefrac{-1}{2}}$ is a linear isometry from $\mathscr H$ to $\mathscr H_\eta$.
We note that this isometry has been used to prove that Hamiltonians with unbroken PT symmetry 
are consistent with standard QM, in the case of closed systems~\cite{GS08a,Mos10a,Mos10b,Zno15,Cro15}.
We refer to the quantum channel $\mathcal{R}_{\eta}$ as a `change in representation' (Fig.~\ref{fig:commutativediagram}b).
In representation~$\pi_{\eta}$, the state $\omega\stackrel{\;{}^{\#}\!\!\pi_{\eta}}{\mapsfrom}\mathcal{R}_{\eta}(\rho)$ 
such that tr$(\mathcal{R}_{\eta}(\rho)\pi_{\eta}(A)) = \omega(A)\forall A$,
and $\mathcal{R}_{\eta}(\rho) \ne \rho$ in general. 
For any pure state~$\omega$, $\exists\ket\psi \in \mathscr{H}_{\eta}$ such that $\omega\stackrel{\;{}^{\#}\!\!\pi_{\eta}}{\mapsfrom}\ket{\psi}\bra{\psi}
\eta\stackrel{\operatorname{lift}_{\eta}}{\mapsfrom}\ket{\psi}$~(see Appendix~\ref{sec:liftmap}).
As 
\begin{equation}
\label{eq:differentreps}
    \text{tr}(\rho \pi(A)) = \text{tr}(\mathcal{R}_{\eta}(\rho) \pi_\eta(A)) \;\;\forall \omega,A,
\end{equation}
representations $\pi$ and~$\pi_{\eta}$ are physically,
i.e. observationally, indistinguishable. 
The right-hand side of Eq.~\eqref{eq:differentreps} 
can also be interpreted as preparation (state) described in $\pi$
followed by a change in representation from~$\pi$ to~$\pi_\eta$ 
effected by $\mathcal{R}_{\eta}$ and finally measurement (observable) described in~$\pi_\eta$.
Change in representation between preparation and measurement sets the stage for 
our definition of change in inner product. 

We define a change in inner product by $\eta$
to be the identity isomorphism~$\mathcal{I}_{\eta}:\mathscr H \to \mathscr{H}_{\eta}$ such that
every $\ket{\psi} \mapsto \ket{\psi}$.
For any pair $\ket{\psi},\ket{\phi}\in\mathscr{H}$,
the inner product between the pair of transformed vectors $\mathcal{I}_{\eta}\ket{\psi},\ \mathcal{I}_{\eta}\ket{\phi}$ 
is $\braket{\psi|\eta|\phi}$,
and the change is trivial if $\eta = \pi(I)$; i.e.\ 
for all pairs $\ket{\psi},\ket{\phi}$, $\braket{\psi|\eta|\phi} = \braket{\psi|\phi}$.
Our definition is motivated by proposals to effect PT-symmetric evolution and measurement by 
changing the Hilbert-space inner product~\cite{BBJM07,BBC+13} but without a 
prescription for making such changes operationally or mathematically.
Next we explain separately, for the cases $\eta \le \pi(I)$ and $\eta \nleq \pi(I)$,
how the isomorphism $\mathcal{I}_\eta$ can be physically realized as a quantum operation.


The change in inner product by $\eta \le \pi(I)$ is physically realizable 
via the operation 
\begin{equation}\label{eq:Etilde}
    \mathcal{E}_\eta:\mathcal{B}\left(\mathscr H\right) \to \mathcal{B}\left(\mathscr{H}_{\eta}\right):M \mapsto M\eta,
\end{equation}
which is not trace-preserving for $\eta \ne \pi(I)$~(see Appendix~\ref{sec:channelEeta}).
The operation  $\mathcal{E}_\eta$ mimics the action of $\mathcal{I}_{\eta}$ at the level of density operators,
because $\ket{\psi} \stackrel{\operatorname{lift}}{\mapsto}\ket{\psi}\bra{\psi}$
whereas~$\mathcal{I}_{\eta}\ket{\psi} \stackrel{\operatorname{lift}_{\eta}}{\mapsto}\ket{\psi}\bra{\psi}\eta = 
\mathcal{E}_\eta(\ket{\psi}\bra{\psi})$.
The operation $\mathcal{E}_\eta$ induces a linear map 
$\mathcal{F}_\eta:\mathcal{S} \to \mathcal{S}$
such that, for any pure $\omega\in \mathcal{S}$, both $\omega$ and $\mathcal{F}_\eta(\omega)$ are represented by the 
same $\ket{\psi}$ under the representations $\pi$ and $\pi_{\eta}$ respectively.
However, $\omega$ and $\mathcal{F}_\eta(\omega)$ are not necessarily the same state (Fig.~\ref{fig:commutativediagram}c).
Even in the special case where the two states differ by a scaling factor,
they are inequivalent in our setting.
The expectation value of $I$ with respect to 
$\mathcal{F}_\eta(\omega)$ gives the success probability
of the inner-product changing quantum operation on the state $\omega$.
$\mathcal{E}_\eta$ can be implemented experimentally 
by lossy purity-preserving operations,
i.e., operations that are not necessarily deterministic and 
transform the set of pure states into itself.
In the Heisenberg picture,
the operators transform according to the map
\begin{equation}\label{eq:Edual}
\mathcal{E}^{\rm op}_\eta:\mathcal{B}\left(\mathscr H\right) \to \mathcal{B}\left(\mathscr{H}_{\eta}\right):M \mapsto \eta M.
\end{equation}
This transformation $\mathcal{E}^{\rm op}_\eta$ could modify  
commutator relations as we show in Appendix~\ref{sec:commutation}.


In the case $\eta \nleq \pi(I)$,
$\mathcal{E}_\eta$
is completely positive
but trace-increasing for some $\rho\in\mathcal{B}(\mathscr{H})$ 
and hence not a quantum operation.
In such cases, a scaled version of change in inner product can be implemented in the following way:
choose $\kappa\in(0,1)$ such that $\kappa\eta\leq \pi(I)$ and observe that 
$\mathcal{E}_{\kappa\eta} = \kappa\mathcal{E}_\eta$ with 
$\mathcal{E}_{\kappa\eta}$ a quantum operation. 
Therefore, $\mathcal{E}_{\kappa\eta}$  implements change in inner product by $\eta \nleq \pi(I)$
up to a scaling factor $\kappa$.
Such a scaled version of change in inner product is useful to reverse the effect of operation 
$\mathcal{E}_\eta$ when $\eta \leq \pi(I)$.
In this case, the isomorphism $\mathcal{I}_{\eta^{-1}}:\mathscr{H}_\eta\to\mathscr{H}$
reverses the change in inner product and the corresponding $\mathcal{E}_{\eta^{-1}}$
is not a valid operation because  $\eta^{-1} \ge \pi_{\eta}(I)$.
Nevertheless, we can choose $\kappa = \nicefrac{1}{\|\eta^{-1}\|}$, 
where $\|\bullet\|$ denotes the operator norm~\cite{Con07}, 
and observe that $\mathcal{E}_{\kappa\eta^{-1}}\circ\mathcal{E}_{\eta}(\rho)=\kappa \rho$ 
for all $\rho\in \mathcal{B}(\mathscr{H})$.
Therefore, the operation   
$\mathcal{E}_{\kappa\eta^{-1}}:\mathcal{B}(\mathscr{H}_{\eta})\to\mathcal{B}(\mathscr{H})$ 
reverses, with probability $\kappa$, the change in inner product by $\eta$. 


The metric operator $\eta$ can be estimated via quantum process tomography~\cite{CN97} 
for $\eta\leq \pi(I)$,
or $\kappa \eta$ if otherwise.
The change in inner product by $\eta\leq \pi(I)$ is implemented via the operation 
(Fig.~\ref{fig:Etilde})
\begin{equation}
\label{eq:F}
\mathcal{E}_\eta= \mathcal{R}_{\eta}\circ \mathcal{G}_\eta,\,    
\mathcal{G}_\eta:\mathcal{B}(\mathscr H)\to \mathcal{B}(\mathscr H):M \mapsto \eta^{\nicefrac{1}{2}}M\eta^{\nicefrac{1}{2}},
\end{equation}
for the Kraus rank-1 operation $\mathcal{G}_\eta$.
Then the Kraus operator $\eta^{\nicefrac{1}{2}}$ and therefore $\eta$
can be estimated by quantum process tomography for trace non-increasing channels~\cite{BSS+10}.
In the other case $\eta\nleq \pi(I)$,
the change in inner product is implemented by the operation  
$\mathcal{E}_{\kappa\eta}$ from which $\kappa \eta$ is estimated similarly; however, 
the above procedure does not yield $\kappa$ and $\eta$ separately.
\begin{figure}
    \begin{center}
    \includegraphics[width = 0.7\columnwidth]{Fig2.pdf}
    \end{center}
    \caption{Commutative diagram showing the action of $\mathcal{F}_\eta$ decomposed in terms of $\mathcal{G}_\eta$ and $\mathcal{R}_\eta$.}
    \label{fig:Etilde}
\end{figure}

We now discuss how to implement dynamics generated by 
a diagonalizable Hamiltonian $H_{\rm PT}$ with unbroken PT symmetry in finite dimensions
over a time $t \ge 0$,
by building on our framework for changing inner product.
The dynamical transformation generated by~$H_{\rm PT}$ is
\begin{equation}
\label{eq:UPT}
    \rho \stackrel{\mathcal{U}_{\rm PT}}{\mapsto} \kappa U_{\rm{PT}}\rho U^\dagger_{\rm{PT}}, \; U_{\rm{PT}} := \text{e}^{-\text{i}H_{\rm {PT}}t/\hbar},
\end{equation}
for some $\kappa\in(0,1)$, where both $\rho$, $\mathcal{U}_{\rm PT}(\rho)\in\mathcal{D}(\mathscr{H})$ represent states under $\pi$.
We show that this dynamics can be implemented by first changing the inner product, 
then applying an appropriate unitary channel and finally reversing the change in inner product.
To explain this sequence, we consider an arbitrary pure state represented by $\ket{\psi}\in \mathscr{H}$,
which is to be transformed to that represented by $\sqrt{\kappa}U_{\rm PT}\ket{\psi}\in \mathscr{H}$ 
(lower row of Fig.~\ref{fig:PTdynamics}).
Next, we compute a metric operator $\eta\leq \pi(I)\in \mathcal{B}(\mathscr H)$
that satisfies the quasi-Hermiticity condition
\begin{equation}\label{eq:quasiHermitian}
H^\dagger_{\rm {PT}} = \eta H_{\rm {PT}} \eta^{-1};
\end{equation}
the existence of such an $\eta$ is guaranteed 
as $H_{\rm PT}$ has unbroken PT symmetry~\cite{Mos02}.
The Hamiltonian $H_{\rm PT}$ is self-adjoint with respect to the inner product of the new Hilbert space $\mathscr{H}_{\eta}$.
Therefore, $U_{\rm PT}$ represents unitary dynamics on $\mathscr{H}_{\eta}$,
which constitutes the second step of the sequence.
Prior to implementing~$U_{\rm PT}$, we transform $\ket{\psi}\in\mathscr{H}$ to $\ket{\psi}\in\mathscr{H}_{\eta}$ 
via a change in inner product using $\mathcal{I}_\eta$.
Finally, the transformation from $U_{\rm PT}\ket{\psi}\in\mathscr{H}_{\eta}$ to $\sqrt{\kappa}U_{\rm PT}\ket{\psi}\in\mathscr{H}$
is equivalent to reversing the change in inner product using  $\mathcal{I}_{\eta^{-1}}$ with probability $\kappa$. 
This sequence extends to general mixed states by the application of lift, lift${}_\eta$ maps 
and linearity (upper row of Fig.~\ref{fig:PTdynamics});
here  $ \widetilde{\mathcal{U}}_{\rm{PT}}\in\mathcal{B}\left(\mathscr{H}_\eta\right)$ 
is the unitary channel satisfying
$\operatorname{lift}_\eta\left(U_{\rm PT}\ket{\psi}\right) =  \widetilde{\mathcal{U}}_{\rm{PT}}\left(\operatorname{lift}_\eta\left(\ket{\psi}\right)\right)$,
for all $\ket{\psi}\in\mathscr{H}_\eta$.

\begin{figure}
\begin{center}
\includegraphics[width = \columnwidth]{Fig3.pdf}
\end{center}
\caption{Diagram showing implementation of a PT-symmetric dynamics using change in inner product and unitary dynamics.}
\label{fig:PTdynamics}
\end{figure}

PT-symmetric dynamics in Eq.~\eqref{eq:UPT} can be expressed as a sequence of channels 
acting exclusively on $\mathcal{B}(\mathscr{H})$, thereby paving the way for experimental simulation 
of PT-symmetric systems.
Following the upper row of Fig.~\ref{fig:PTdynamics}, 
we start by expressing $\mathcal{U}_{\rm{PT}}$ (Eq.~\eqref{eq:UPT})
as  $\mathcal{U}_{\rm{PT}}= \mathcal{E}_{\kappa \eta^{-1}} \circ  \widetilde{\mathcal{U}}_{\rm{PT}} \circ \mathcal{E}_\eta$.
Similar to~Eq.\ \eqref{eq:F},
we express the reverse change in inner product as 
$\mathcal{E}_{\kappa\eta^{-1}}= \mathcal{G}_{\kappa\eta^{-1}}\circ \mathcal{R}_{\kappa\eta^{-1}}$,
for  $\mathcal{G}_{\kappa\eta^{-1}}: \mathcal{B}(\mathscr{H})\to \mathcal{B}(\mathscr{H}): 
\mathcal{G}_{\kappa\eta^{-1}}(M) = \kappa \eta^{-\nicefrac{1}{2}}M\eta^{\nicefrac{1}{2}} $ and
$\mathcal{R}_{\kappa\eta^{-1}}: \mathcal{B}(\mathscr{H}_{\eta})\to \mathcal{B}(\mathscr{H})$ is the 
channel effecting the change in representation form $\pi_\eta$ to $\pi$.
We then rewrite~$\mathcal{U}_{\rm{PT}}$ as
\begin{equation}\label{eq:PTUnitaryoperational}
    \mathcal{U}_{\rm{PT}}= {\mathcal{G}}_{\kappa \eta^{-1}} \circ (\mathcal{R}_{\kappa\eta^{-1}} \circ \widetilde{\mathcal{U}}_{\rm{PT}} \circ \mathcal{R}_{\eta})\circ \mathcal{G}_\eta,
\end{equation}
which is the desired decomposition. The channels ${\mathcal{G}}_{\kappa \eta^{-1}}$, $ \mathcal{G}_\eta$ have 
single Kraus operators $\sqrt{\kappa}\eta^{-\nicefrac{1}{2}}$ and $\eta^{\nicefrac{1}{2}}$ respectively.
The maps $\mathcal{R}_{\eta}$,  $\mathcal{R}_{\kappa\eta^{-1}}$
only effect change in representation and operationally are equivalent to no change. 
Finally, the transformation $\mathcal{R}_{\kappa\eta^{-1}} \circ \widetilde{\mathcal{U}}_{\rm{PT}} \circ \mathcal{R}_{\eta}$
implements a channel 
corresponding to the unitary Kraus-operator $\eta^{\nicefrac{1}{2}}U_{\rm PT}\eta^{\nicefrac{-1}{2}}$ acting
on~$\mathscr{H}$, generated by the Hamiltonian 
\begin{equation}\label{eq:selfadjointHPT}
    h_{\rm PT} = \eta^{\nicefrac{1}{2}}H_{\rm PT}\eta^{\nicefrac{-1}{2}} \in \mathcal{B}(\mathscr{H}),
\end{equation}
which can be verified to be self-adjoint, i.e.\ $h_{\rm PT}^\dagger = h_{\rm PT}$, 
using the quasi-Hermiticity condition in Eq.~\eqref{eq:quasiHermitian}.

We now design a qutrit procedure for an agent to simulate successfully the change in inner product by $\eta \le \pi(I)$ 
of a qubit system with algebra $\mathcal{A}$, which is represented on a 
two-dimensional Hilbert space $\mathscr{H}_2$ by $\pi$.
Our procedure, which shall simulate the operation $\mathcal{G}_\eta$ (Fig.~\ref{fig:Etilde}),
uses a unitary operation on the three-dimensional Hilbert space $\mathscr{H}_3=\mathscr{H}_2 \oplus \mathscr{H}_1$ 
followed by a projective measurement on to $\mathscr{H}_2$ and postselection, as we now explain.
For any $\eta \le \pi(I)$, we first construct the metric operator
\begin{equation}
\label{eq:tildeeta}
\tilde{\eta}:=\frac{1}{\|\eta\|}\eta
\implies\mathcal{G}_\eta = \|\eta\|\mathcal{G}_{\tilde{\eta}},
\end{equation}
and the unitary operator $U_{\tilde{\eta}}\in \mathcal{B}(\mathscr{H}_3)$ that satisfies
\begin{equation}
\label{eq:simulatingFeta}
    \mathcal{G}_{\tilde{\eta}}(\rho)\oplus \bm{0} = PU_{\tilde{\eta}}\sigma U_{\tilde{\eta}}^\dagger P,\, \sigma:=\rho\oplus \bm{0},
    \;\forall \rho\in \mathcal{B}(\mathscr{H}_2),
\end{equation}
where $P$ is the orthogonal projector on~$\mathscr{H}_2$. 
The matrix representation of~$U_{\tilde{\eta}}$ is (see Appendix~\ref{sec:matrixUeta})
\begin{equation}
\label{eq:Ueta}
    \left[U_{\tilde{\eta}}\right]
    = \begin{pmatrix}
    \left[\tilde{\eta}\right]^{\frac12} &\bm{u} \\ -\text{e}^{\text{i} \theta}\bar{\bm{u}}^{\top} &\text{e}^{\text{i}\theta}r 
    \end{pmatrix},
    \text{spec}\left(\tilde{\eta}^{\nicefrac12}\right)=\{1,r\},
    \theta\in [0,2\pi),
\end{equation}
where~$[\;]$ denotes matrix representation,
$\bm{u}$ is the eigenvector of $[\tilde{\eta}]^{\nicefrac12}$ with eigenvalue $r$
and $\| \bm{u}\| = \sqrt{1-r^2}$.
Furthermore,
$\bar{\bm{u}}^{\top}$ is the Hermitian conjugate of the vector $\bm{u}$.
Both~$\theta$ and the global phase of $\bm{u}$ are free parameters.
The qutrit unitary operator $U_{\tilde{\eta}}$ is part of the overall 
simulation procedure (Eq.~\eqref{eq:simulatingFeta}).

Now we explain how an agent 
can sequentially apply each operator in Eq.~\eqref{eq:simulatingFeta}
to simulate $\mathcal{G}_{\eta}$ (Fig.~\ref{fig:flowchart}).
The agent is provided with a description of $2\times2$ matrix~$[\eta]$,
in the logical basis~$\{\ket0,\ket1\}$ 
and a quantum state $\sigma$ (Eq.~\eqref{eq:simulatingFeta}).
The task is to generate the state
$\left({\mathcal{G}_{\eta}(\rho)\oplus \bm{0}}\right)/{{\rm tr}\left(\mathcal{G}_{\eta}(\rho)\oplus \bm{0}\right)}$
with probability ${\rm tr}\left(\mathcal{G}_{\tilde{\eta}}(\rho)\oplus \bm{0}\right)$.
The agent first computes $\left[\tilde{\eta}\right]$ (Eq.~\eqref{eq:tildeeta}) and
$\left[U_{\tilde{\eta}}\right]$ (Eq.~\eqref{eq:Ueta})
and then applies physical operations corresponding to~$U_{\tilde{\eta}}$ on~$\sigma$ followed by projective measurement~$P$ (Eq.~\eqref{eq:simulatingFeta}).
For non-zero measurement outcome,
which occurs with probability ${\rm tr}\left(\mathcal{G}_{\tilde{\eta}}(\rho)\oplus \bm{0}\right)$,
the post-measurement state obtained is
(Eq.~\eqref{eq:tildeeta})
\begin{equation}
   \frac{\mathcal{G}_{\tilde{\eta}}(\rho)\oplus \bm{0}}{{\rm tr}(\mathcal{G}_{\tilde{\eta}}(\rho)\oplus \bm{0})} = \frac{\mathcal{G}_\eta(\rho)\oplus \bm{0}}{{\rm tr}(\mathcal{G}_\eta(\rho)\oplus \bm{0})}.
\end{equation}
The agent discards the state if the measurement outcome is zero.
This concludes the simulation procedure.
The agent may further estimate the success probability ${\rm tr}\left(\mathcal{G}_{\tilde{\eta}}(\rho)\oplus \bm{0}\right)$,
if required, by repeating the simulation procedure on a large number of copies
of $\sigma$ provided to them and then calculating the ratio of non-zero measurement outcomes to the total
number of copies used~\footnote{
The ratio of non-zero measurement outcomes to the total
number of copies approaches the success probability by the law of large numbers~\cite{DKLM05}.
Our setting assumes that multiple copies of the state $\sigma$
are provided to the agent by an external agent who has the knowledge of
$\sigma$, and not prepared by the agent implementing the inner-product changing channel, say, by cloning.}.


In Appendix~\ref{sec:qubitPTsymmetry}, we provide an explicit 
procedure to simulate the dynamics (Eq.~\eqref{eq:UPT}) 
of the qubit PT-symmetric Hamiltonian 
~\cite{BB98},
by sequentially applying the operators in Eq.~\eqref{eq:PTUnitaryoperational} and by using the qutrit 
simulation procedure to implement~$\mathcal{G}_{\eta},\mathcal{G}_{\kappa\eta^{-1}}$.
In Appendix~\ref{sec:dsimulation}, we design a simulation procedure, similar to our qutrit procedure given above, 
for changing the inner product of a $d$-dimensional 
system using a $2d$-dimensional system for any positive integer~$d$. Furthermore, we
use our procedure to simulate the dynamics of a $d$-dimensional PT-symmetric Hamiltonian
by using only $2d$ dimensions, instead of $d^3$ dimensions as required in the Stinespring dilation approach \cite{Sti95}.

\begin{figure}
\begin{center}
\includegraphics[width = \columnwidth]{Flowchart.pdf}
\end{center}
\vspace{-0.5cm}
\caption{Simulation of the application of $\mathcal{G}_{{\eta}}$ on $\rho$ with success probability
${\rm tr}\left(\mathcal{G}_{\tilde{\eta}}(\rho)\oplus \bm{0}\right)$}. 
\label{fig:flowchart}
\end{figure}

We also design a scheme to verify tomographically whether a prover
can perform an arbitrary change of inner product using our qutrit simulation procedure.
Input to the verification scheme is a threshold function $D_{\rm th}:\mathcal{B}(\mathscr{H}_2)\to (0,1)$ given as a black-box.
The output is `accept' if $\|\mathcal{G}_\eta\oplus \bm{0} - \hat{\mathcal{G}}_\eta\|_{1\to 1}\leq D_{\rm th}(\eta)$ 
or `reject' otherwise, where $\hat{\mathcal{G}}_\eta:\mathcal{B}(\mathscr{H}_3)\to \mathcal{B}(\mathscr{H}_3)$ 
represents a tomographic reconstruction of the qutrit process implemented by the prover, 
$\mathcal{G}_\eta\oplus \bm{0}$ extends the action of $\mathcal{G}_\eta$ to $B(\mathscr{H}_3)$
and $\|\bullet\|_{1 \to 1}$ is the induced Schatten $(1\to 1)$-norm~\cite{Pau03}.
The verifier supplies to the prover a randomly chosen valid $\eta$, a positive integer 
$N$ sufficiently large for the process tomography~\cite{STM11}
and copies of the quantum states $\sigma_i$ encoding $\rho_i$ on demand,
where $\{\rho_i\}$ is chosen based on the tomography procedure in use. 
The prover returns $N$ copies of the qutrit states on which the change of inner product is successful
 as well as the success ratios for each $\rho_i$,
both of which are used by the verifier to reconstruct $\hat{\mathcal{G}_\eta}$.
To ensure that the verifier does not accept the process 
performed by a dishonest prover implementing only
qubit-unitary channels and randomly discarding the system,
it suffices to set the threshold to 
$D_{\rm th}(\eta) = \nicefrac{1}{3}(\lambda_1-\lambda_2)$,
where $\lambda_1>\lambda_2>0$ are the eigenvalues of $\eta$ (see Appendix~\ref{sec:threshold}). 


In conclusion, we have three major results.
First, we have operationalized Hilbert-space inner-product change in a way that is both observable and fully compatible with axiomatic quantum mechanics. 
Physically we can understand this inner-product change as 
a lossy quantum operation effecting a change in norm.
This lossy operation
is reminiscent of how superluminality is reconciled by electromagnetic absorption~\cite{SKC93},
with loss in our case forbidding past counterfactual claims.
Consistency of our work is proven using C$^*$ algebra and representations.
Alternatively,
our claims can be verified experimentally by conducting two physically distinct experiments.
One experiment is for the lower-dimensional lossy quantum operation 
and the other experiment is for the higher-dimensional unitary channel with both realizations yielding the same success ratio and 
measurement statistics for a given task.
Our theory fully explains unbroken PT-symmetric quantum mechanics in all its forms as being about changing Hilbert-space inner product and observing its consequences.
Our scheme for simulating qubit PT-symmetric Hamiltonians 
only requires one extra Hilbert-space dimension and no interaction with the environment,
which eliminates the requirements for 
multiple subsystems and entangling operations used in existing schemes~\cite{GS08a,WLG+19,TWY+16,WLG+19,XWZ+19,GZL+21}.
We also show how to simulate $d$-dimensional ($d\geq 2$) PT-symmetric Hamiltonians using
$2d$ dimensions, as opposed to using $d^3$ dimensions in the Stinespring dilation approach.
Our results open possibilities for simulating PT-symmetric dynamics on new experimental platforms, 
such as transmons, where high fidelity qutrit-unitary operations have already been demonstrated~\cite{BRS+20,KYSL20}. 



\acknowledgements 
This project is supported by the Government of Alberta and 
by the Natural Sciences and Engineering Research Council of Canada (NSERC).
S.\ K.\ is grateful for a University of Calgary Eyes High International Doctoral Scholarship and an Alberta Innovates Graduate Student Scholarship.
A.\ A.\ acknowledges support through a Killam 2020 Postdoctoral Fellowship.

\begin{appendix}

\section{Constructing a representation of the C\texorpdfstring{$^{*}$}{*} algebra on the Hilbert space with a different inner product}\label{sec:newHilbertspace}
In this section, we show the construction of the Hilbert space $\mathscr{H}_\eta$ with inner product related 
to that of $\mathscr{H}$ by the metric
operator $\eta$,
the construction of a ${}^*$-representation of the C$^{*}$ algebra $\mathcal{A}$ on this new Hilbert space, 
and
finally the representation of states in $\mathcal{S}$ using density operators on $\mathscr{H}_\eta$.


\subsection{Constructing a new Hilbert space from the metric operator}
For a possibly infinte dimensional Hilbert space~$\mathscr{H}$, we denote by $\mathcal{L}(\mathscr{H})$ and $\mathcal{B}(\mathscr{H})$ the algebra of 
linear and bounded linear operators on~$\mathscr{H}$ respectively. We also denote by $\mathcal{D}\left(\mathscr{H}\right) := 
\{\rho \in \mathcal{B}(\mathscr{H}) : \rho \ge 0, \rho^\dagger = \rho, \text{tr}\rho\le 1 \}$ the set 
of density operators acting on~$\mathscr{H}_\eta$. 
\begin{definition}[\cite{Con07}]
The adjoint of an operator $A \in \mathcal{B}(\mathscr{H})$ is the unique operator $A^\dagger\in \mathcal{B}(\mathscr{H})$ satisfying
\begin{equation}
    \braket{\phi|A|\psi} = \overline{\braket{\psi|A^\dagger|\phi}} \;\; \forall \ket{\phi},\ket{\psi} \in \mathscr{H}.
\end{equation} 
The operator $A$ is self-adjoint if $A = A^\dagger$.
\end{definition}
\begin{definition}[\cite{Con07}]
An operator $A \in \mathcal{L}(\mathscr{H})$ is positive-definite if 
\begin{equation}
    \braket{\phi|A|\phi} >0  \;\; \forall \ket{\phi} \in \mathscr{H},\;\ket{\phi} \ne 0.
\end{equation} 
\end{definition}
The following theorem is adapted from the Appendix A of Ref.~\cite{SGH92}.
\begin{theorem}
\label{thm:Heta}
For any Hilbert space $\mathscr{H} = \left(\mathscr{V},\langle\,\vert\,\rangle\right)$ 
and a self-adjoint, positive-definite operator $\eta \in \mathcal{B}(\mathscr{H})$, 
\begin{enumerate}
    \item the sesquilinear form 
    \begin{equation}
    \label{eq:newip}
        \langle\bullet|\bullet \rangle_\eta := \langle\bullet|\eta|\bullet\rangle
    \end{equation}
    is non-degenerate, therefore an inner product on $\mathscr{V}$.
    \item The vector space $\mathscr{V}$ is complete with respect to the norm induced by the inner product 
    $\braket{\bullet|\bullet}'$, therefore $\mathscr{H}_\eta = \left(\mathscr{V},\langle\,\vert\,\rangle_\eta\right)$  is a Hilbert space.
\end{enumerate}
\end{theorem}

\subsection{Constructing a \texorpdfstring{${}^{*}$}{*}-representation on the new Hilbert space}

We now construct a ${}^*$-representation of the algebra $\mathcal{A}$ on the new Hilbert space $\mathscr{H}_\eta$ constructed in Theorem~\ref{thm:Heta}.
In the following, $\mathscr{H}$ and $\mathscr{H}_\eta$ are two Hilbert spaces with their inner product related by
the metric operator $\eta$ as in Theorem~\ref{thm:Heta}, $\mathcal{A}$ is a C${}^*$ algebra of operators and
$\pi:\mathcal{A} \to \mathcal{B}(\mathscr{H})$ is a ${}^*$-representation of $\mathcal{A}$.
We first establish some results required for constructing such a new representation. The following lemma,
which establishes the inverse of the metric operator $\eta$, is adapted from the Appendix A of Ref.~\cite{SGH92}.
\begin{lemma}
Any self-adjoint and positive-definite operator $\eta \in \mathcal{B}(\mathscr{H})$
is invertible. 
Furthermore, the inverse $\eta^{-1}\in \mathcal{B}(\mathscr{H})$ is self-adjoint and positive-definite.
\end{lemma}
We next show that the bounded operator spaces on $\mathscr{H}$ and $\mathscr{H}_\eta$ coincide.
\begin{lemma}\label{lem:boundedopspaces}
    $M\in\mathcal{B}\left(\mathscr{H}\right)$ if and only if $M\in\mathcal{B}\left(\mathscr{H}_\eta\right)$.
\end{lemma}
\begin{proof}
    Let $\|\bullet\|,\|\bullet\|_\eta$ respectively denote the operator norms in~$\mathscr{H}$, $\mathscr{H}_\eta$.
    From Eq.~\eqref{eq:newip}, 
    \begin{equation}\label{eq:twonorms}
        \left\|M\right\|_\eta = \left\|\eta^{\nicefrac{1}{2}}M\eta^{\nicefrac{-1}{2}}\right\|,\, \forall\, M\in\mathcal{B}(\mathscr{H}).
    \end{equation}
    If $M\in \mathcal{B}\left(\mathscr{H}\right)$, then
    \begin{equation}
        \left\|M\right\|_\eta \leq \left\|\eta^{\nicefrac{1}{2}}\right\|\cdot\left\|M\right\|\cdot\left\|\eta^{\nicefrac{-1}{2}}\right\|<\infty
    \end{equation}
    and therefore, $M\in \mathcal{B}\left(\mathscr{H}_\eta\right)$. 
    To verify the reverse implication, note that
    \begin{eqnarray}
        \left\|M\right\| &=& \left\|\eta^{\nicefrac{-1}{2}}\left(\eta^{\nicefrac{1}{2}}M\eta^{\nicefrac{-1}{2}}\right)\eta^{\nicefrac{1}{2}}\right\|\nonumber \\
        &\leq & \left\|\eta^{\nicefrac{-1}{2}}\right\|\cdot \left\|M\right\|_\eta\cdot  \left\|\eta^{\nicefrac{1}{2}}\right\| 
        <\infty,
    \end{eqnarray}
     for all $M\in\mathcal{B}\left(\mathscr{H}_\eta\right) $.
\end{proof}
The next lemma relates ${}^\dagger$ to ${}^\ddagger$, with the latter denoting the adjoint with respect to the inner product $\braket{\bullet|\bullet}'$ of $\mathscr{H}_\eta$.
\begin{lemma}
For any $M \in \mathcal{B}(\mathscr{H})$, $M^\ddagger = \eta^{-1}M^\dagger \eta$. Additionally, $\eta^\ddagger = \eta$.
\end{lemma}
\begin{proof}
By definition of $\ddagger$, we have 
\begin{eqnarray}
\label{eq:ddagger}
  \langle \phi|M|\psi\rangle_\eta = \overline{\langle \psi|M^\ddagger|\phi\rangle_\eta},\,
  &\forall&\, \ket{\psi},\ket{\phi}\in \mathscr{V}, \nonumber\\
  &\forall& \, M\in \mathcal{B}(\mathscr{H}).  
\end{eqnarray}
Using Eq.~\eqref{eq:newip}, 
\begin{eqnarray}
\label{eq:ddaggerformula}
    \langle \phi|M|\psi\rangle_\eta &=  \langle \phi|\eta M|\psi\rangle = \overline{\langle\psi |M^\dagger\eta|\phi\rangle} =  
    \overline{\langle\psi |\eta^{-1}M^\dagger\eta|\phi\rangle_\eta}, \nonumber\\
    &\forall\, \ket{\psi},\ket{\phi}\in \mathscr{V}, \, \forall \, M\in \mathcal{B}\left(\mathscr{H}\right).
\end{eqnarray}
Then $M^\ddagger = \eta^{-1}M^\dagger\eta$ follows from the comparison of Eq.~\eqref{eq:ddagger} and Eq.~\eqref{eq:ddaggerformula},
and $\eta^\ddagger = \eta$ can be obtained by substituting $M=\eta$ in this relation.
\end{proof}

We are now ready for the construction of a ${}^*$-representation of $\mathcal{A}$ on $\mathscr{H}_\eta$.
\begin{theorem}
\label{lem:pieta}
Let $\pi:\mathcal{A} \to \mathcal{B}(\mathscr{H})$ be a ${}^*$-representation of a C${}^*$ algebra $\mathcal{A}$.
Then $\pi_{\eta}: \mathcal{A} \to \mathcal{L}(\mathscr{H}_\eta): A\mapsto \eta^{\nicefrac{-1}{2}}\pi(A) \eta^{\nicefrac{1}{2}} $
is a ${}^*$-representation of $\mathcal{A}$ on $\mathscr{H}_\eta$.
\end{theorem}
\begin{proof}
The range of the map $\pi_{\eta}$ is $\mathcal{B}(\mathscr{H}_\eta)$, which follows from the fact that 
 $\eta^{1/2},\eta^{-1/2}, \pi(A) \in \mathcal{B}\left(\mathscr{H}_\eta\right)$, following Lemma~\ref{lem:boundedopspaces}
 and definition of $\pi$, for all $A\in\mathcal{A}$. 
The map $\pi_{\eta}$ is linear by construction, and it is product preserving
because $\pi_{\eta}(AB) = \eta^{\nicefrac{-1}{2}}\pi(AB)\eta^{\nicefrac{1}{2}} = \eta^{\nicefrac{-1}{2}}\pi(A)\left(\eta^{\nicefrac{1}{2}}\eta^{\nicefrac{-1}{2}}\right)\pi(B)\eta^{\nicefrac{1}{2}} = \pi_{\eta}(A)\pi_{\eta}(B)$. Therefore $\pi_\eta$ is a representation. 
The representation $\pi_{\eta}$ is also ${}^{*}$-preserving, as 
\begin{eqnarray}
    \pi_{\eta}(A^*) &=& \eta^{\nicefrac{-1}{2}}\pi\left(A^*\right)\eta^{\nicefrac{1}{2}} = \eta^{\nicefrac{-1}{2}}\pi(A)^\dagger \eta^{\nicefrac{1}{2}} 
    = \eta^{\nicefrac{1}{2}}\pi(A)^\ddagger \eta^{\nicefrac{-1}{2}} \nonumber \\
    &=&\left(\eta^{\nicefrac{-1}{2}}\pi(A) \eta^{\nicefrac{1}{2}}\right)^\ddagger = \pi_{\eta}(A)^\ddagger.  
\end{eqnarray}
Therefore $\pi_{\eta}$ is a  ${}^{*}$-representation of $\mathcal{A}$ on~$\mathscr{H}_\eta$.
\end{proof}


\subsection{Representing states on the new Hilbert space}
\label{sec:liftmap}
We now characterize the set of states represented by the set of density operators~$\mathcal{D}\left(\mathscr{H}_\eta\right)$
and construct vector representation of the pure states under $\pi_\eta$.
Recall that ${}^{\#}\pi:\rho \mapsto \omega$ such that $\omega(A) = \text{tr}(\rho \pi(A))\  \forall A \in \mathcal{A}$. 
The map ${}^{\#}\pi_\eta$ is defined analogously for the $\pi_\eta$ representation. 
In the following lemma
we show how the density operators acting on $\mathscr{H}$ and $\mathscr{H}_\eta$ are related,
which we further use to prove that the set of states represented under 
$\pi_\eta$ coincides with that represented under $\pi$, namely $\mathcal{S}$. 


\begin{lemma}
   An operator $\rho_\eta \in\mathcal{D}\left(\mathscr{H}_\eta\right) $ if and only if
   \begin{equation}\label{eq:rho_eta}
       \rho_\eta = \eta^{\nicefrac{-1}{2}}\rho \eta^{\nicefrac{1}{2}}
   \end{equation}
    for some  $\rho \in\mathcal{D}\left(\mathscr{H}\right)$.
    \end{lemma}
\begin{proof}
Let $\{\ket{e_i}\}$ be an orthonormal basis of $\mathscr{H}$. We first show that $\{\ket{f_i}:= \eta^{\nicefrac{-1}{2}}\ket{e_i}\}$
is an orthonormal basis of $\mathscr{H}_\eta$. The orthonormality of $\{\ket{f_i}\}$ follows from 
$\braket{f_i|f_j}' = \braket{e_i|\eta^{\nicefrac{-1}{2}}\eta\eta^{\nicefrac{-1}{2}} |e_j}' = \delta_{ij}$. 
We prove that $\{\ket{f_i}\}$ is a basis by showing that any $\ket{\phi} \in \mathscr{V}$ can be expressed
as $\ket{\phi} = \sum_i \braket{f_i|\phi}'\ket{f_i}$. Let $\ket{\psi} = \eta^{\nicefrac{1}{2}}\ket{\phi}$.
Then $\ket{\psi} = \sum_i \braket{e_i|\psi}\ket{e_i}$. Now premultiplying by $\eta^{\nicefrac{-1}{2}}$ and substituting
$\ket{f_i} = \eta^{\nicefrac{-1}{2}}\ket{e_i}$, $\ket{\psi} = \eta^{\nicefrac{1}{2}}\ket{\phi}$ yields 
the desired expression $\ket{\phi} = \sum_i \braket{f_i|\phi}'\ket{f_i}$. Note that this sum is convergent because
the set $\{\ket{f_i}\}$ is orthonormal~\cite{Con07}.

To prove the forward implication, we note that Hilbert Schmidt norm of any $\rho \in\mathcal{D}\left(\mathscr{H}\right)$
is finite, i.e.\ $\sum_{i\in B}\|\rho\ket{e_i}\|^2<\infty$, and
${\rm tr}(\rho) = \sum_{i\in B} \langle e_i|\rho |e_i\rangle<\infty$; 
both these properties follow from $\rho$ being a trace-class operator~\cite{Con07}.
For $\rho_\eta\in\mathcal{B}(\mathscr{H}_\eta)$ satisfying Eq.~\eqref{eq:rho_eta}, 
\begin{eqnarray}
    \sum_{i\in B_\eta}\left\|\rho_\eta\ket{f_i}\right\|_\eta^2 &=& 
    \sum_{i\in B}\left\|\eta^{\nicefrac{-1}{2}}\rho\ket{e_i}\right\|_\eta^2 \nonumber\\ 
    &=& \sum_{i\in B}\langle e_i| (\eta^{\nicefrac{-1}{2}}\rho)^\dagger \eta (\eta^{\nicefrac{-1}{2}}\rho)|e_i\rangle \nonumber\\
    &=& \sum_{i\in B}\left\|\rho\ket{e_i}\right\|^2 <\infty.
\end{eqnarray}
Therefore, $\rho_\eta$ is a Hilbert-Schmidt operator in $\mathcal{B}(\mathscr{H}_\eta)$. 
Furthermore, 
\begin{eqnarray}\label{eq:rho_etatrace}
    {\rm tr}(\rho_\eta) &=& \sum_{i\in B_\eta}\langle f_i|\rho_\eta|f_i\rangle_\eta 
 = \sum_{i\in B}\langle e_i|\eta^{\nicefrac{-1}{2}}\eta \rho_\eta\eta^{\nicefrac{-1}{2}}|e_i\rangle \nonumber\\
   & =& \sum_{i\in B}\langle e_i| \rho|e_i\rangle  = {\rm tr}(\rho).
\end{eqnarray}
Therefore,  $\rho_\eta$ is a trace-class operator with $ {\rm tr}(\rho_\eta)\leq 1$ and hence $\rho_\eta \in\mathcal{D}\left(\mathscr{H}_\eta\right) $.
Similarly, the reverse implication that for any $\rho_\eta \in\mathcal{D}\left(\mathscr{H}_\eta\right) $, the operator
$\eta^{\nicefrac{1}{2}}\rho_\eta \eta^{\nicefrac{-1}{2}} \in \mathcal{D}\left(\mathscr{H}\right)$ is proved by 
starting with an orthonormal basis $\{\ket{f_i}\}$ for $\mathscr{H}_\eta$ and observing that 
$\{\eta^{\nicefrac{1}{2}}\ket{f_i}\}$ is an orthornormal basis for $\mathscr{H}$.

\end{proof}

We now show that both $\mathcal{D}(\mathscr{H})$ and  $\mathcal{D}(\mathscr{H}_\eta)$ represent the same set of states, $\mathcal{S}$,
under the respective~$^*$-representations.
\begin{lemma}
    $\rho_\eta\stackrel{\!\!{}^{\#}\!\pi_\eta}{\to} \omega$ if and only if $\rho\stackrel{\!\!{}^{\#}\!\pi}{\to} \omega$, where $\rho,\rho_\eta$
are related by Eq.~\eqref{eq:rho_eta}.

\end{lemma}
\begin{proof}
To prove the forward implication, note $\omega(A) = \text{tr}(\rho \pi(A)) \ \forall A$ by the definition of ${}^\#\pi$. Then
$\text{tr}(\rho \pi(A)) = \text{tr}(\rho_\eta \eta^{\nicefrac{-1}{2}}\pi(A) \eta^{\nicefrac{1}{2}})
=\text{tr}(\rho_\eta\pi_\eta(A))$ using the cyclic property of trace and the definition of $\pi_\eta$ respectively.
Therefore $\omega(A) = \text{tr}(\rho_\eta\pi_\eta(A)) \ \forall A$, and therefore, $\rho_\eta\stackrel{\!\!{}^{\#}\!\pi_\eta}{\to} \omega$.
The reverse implication can be proved by following the same steps in reverse order.

\end{proof}

Finally, we represent pure states in $\mathcal{S}$ by vectors in the Hilbert space $\mathscr{H}_\eta$.
Recall that a state $\omega \in \mathcal{S}$ has a vector-representation $\ket{\psi} \in \mathscr{H}$ under $\pi$ if 
\begin{equation}
    \omega(A) = \braket{\psi | \pi(A)|\psi} \;\; 
    \forall A \in \mathcal{A}.
\end{equation}
We now extend this definition to the representation $\pi_\eta$.
\begin{definition}
\label{def:vecrep}
A state $\omega \in \mathcal{S}$ has a vector representation $\ket{\psi} \in \mathscr{H}_\eta$ under $\pi_\eta$ if 
\begin{equation}
    \omega(A) = \braket{\psi | \pi_\eta(A)|\psi}_\eta \;\; \forall A \in \mathcal{A}.
\end{equation}
\end{definition}

Let the ball $\overline{B_1}(\mathscr{H}) := \{\ket{\psi} \in \mathscr{H}: \sqrt{\braket{\psi|\psi}} \le 1\}$ 
denote the set of normalized and subnormalized vectors in $\mathscr{H}$. 
Recall that the map $\operatorname{lift}:\overline{B_1}(\mathscr{H}) \to \mathcal{D}(\mathscr{H}):
\ket{\psi} \mapsto \ket{\psi}\bra{\psi}$
connects the vector representation $\ket{\psi}$ of a pure state $\omega$ to its density operator representation
$\ket{\psi}\bra{\psi}$ under $\pi$. 
We now construct an analogous map $\operatorname{lift}_\eta: \overline{B_1}(\mathscr{H}_\eta) \to \mathcal{D}\left(\mathscr{H}_\eta\right)$
for the representation $\pi_\eta$, where the ball 
$\overline{B_1}(\mathscr{H}_\eta) := \{\ket{\psi} \in \mathscr{H}_\eta: \sqrt{\braket{\psi|\psi}}_\eta \le 1\}$. 
\begin{definition}
The map $\operatorname{lift}_\eta: \overline{B_1}(\mathscr{H}_\eta) \to \mathcal{D}(\mathscr{H}_\eta)$ is defined to be the map that 
satisfies the following condition: for any state $\omega \in \mathcal{S}$ with vector representation
$\ket{\psi} \in \mathscr{H}_\eta$ and density operator representation $\rho_\eta \in \mathcal{D}(\mathscr{H}_\eta)$,
$\operatorname{lift}_\eta:\ket{\psi} \mapsto \rho_\eta$.
\end{definition}
We now derive the explicit action of $\operatorname{lift}_\eta$.
\begin{lemma}
\label{lem:lifteta}
The map $\operatorname{lift}_\eta$ has action 
$\operatorname{lift}_\eta:\overline{B_1}(\mathscr{H}_\eta)\to \mathcal{D}(\mathscr{H}_\eta):\ket{\psi} \mapsto \ket{\psi}\bra{\psi}\eta$.
\end{lemma}
\begin{proof}
Let $\operatorname{lift}_\eta:\ket{\psi} \mapsto \rho_\eta$ and ${}^\#\pi_\eta:\rho_\eta \mapsto \omega$. 
Following the definition of ${}^\#\pi_\eta$ and  $\operatorname{lift}_\eta$,
\begin{equation}
    \braket{\psi|\pi_\eta(A)|\psi}_\eta = \text{tr}\left(\rho_\eta\pi_\eta(A)\right) = \omega(A) \;\; \forall A.
\end{equation}
As $\braket{\psi|\pi_\eta(I)|\psi}_\eta  = \omega(I)\le 1$, $\ket{\psi} \in \overline{B_1}(\mathscr{H}_\eta)$.
Using Eq.~\eqref{eq:newip}, we get $\braket{\psi|\pi_\eta(A)|\psi}_\eta = \braket{\psi|\eta\pi_\eta(A)|\psi}$. Then using
the cyclic property of the trace, this expectation value can be expressed as
\begin{equation}
    \braket{\psi|\eta\pi_\eta(A)|\psi} = \text{tr}\left(\ket{\psi}\bra{\psi}\eta\pi_\eta(A)\right) \;\; \forall A,
\end{equation}
therefore, $\rho_\eta = \ket{\psi}\bra{\psi}\eta \in \mathcal{D}(\mathscr{H}_\eta)$.
This leads to the desired action of $\operatorname{lift}_\eta$.

\end{proof}

\section{{{Quantum operation} for implementing the changing inner product}}\label{sec:channelEeta}
In this section, we construct the quantum operation that implements the change in inner product by $\eta \le \pi(I)$. 
Change in inner product is defined by the identity isomorphism $\mathcal{I}_\eta:\mathscr{H}\to\mathscr{H}_\eta$ 
(see Fig.~1c in main text).
We now show how  $\mathcal{I}_\eta$ is extended to $\mathcal{B}(\mathscr{H}_\eta)$ through the map $\mathcal{E}_\eta$ defined below.
\begin{theorem}
Let $\eta \le \pi(I)$ and $\mathcal{E}_\eta:\mathcal{B}(\mathscr{H})\to \mathcal{L}(\mathscr{H}_\eta):M \mapsto M\eta$.
Then
\begin{enumerate}
    \item $\operatorname{range}(\mathcal{E}_\eta) \subseteq \mathcal{B}(\mathscr{H}_\eta)$,
    \item $\mathcal{E}_\eta$ satisfies the following commutative diagram:
    \begin{equation}
    \xymatrix{
    \mathcal{D}(\mathscr{H}) \ar[r]^{\mathcal{E}_\eta} &  \mathcal{D}(\mathscr{H}_\eta)   \\
    \overline{B_1}(\mathscr{H}) \ar[u]^{\operatorname{lift}} \ar[r]^{\mathcal{I}_\eta} 
      & \overline{B_1}(\mathscr{H}_\eta) \ar[u]^{\operatorname{lift}_\eta} }
    \end{equation}
    \item $\mathcal{E}_\eta$ is a quantum operation.
\end{enumerate}
\end{theorem}
\begin{proof}
To prove Statement 1, note that for any $M \in \mathcal{B}\left(\mathscr{H}\right)$, the operator $M\eta \in  \mathcal{B}\left(\mathscr{H}_\eta\right)$ because
\begin{equation}
    \|M\eta\|_\eta = \|\eta^{\nicefrac{1}{2}}\left(M\eta\right)\eta^{\nicefrac{-1}{2}}\|\leq \|\eta^{\nicefrac{1}{2}}\|^2\cdot\|M\|\le \infty,
\end{equation}
where the first equality follows from Eq.~\eqref{eq:twonorms}. 
The commutative diagram in Statement 2 follows immediately from the action of lift${}_\eta$ map
in Lemma \ref{lem:lifteta}. 

We now show that $\mathcal{E}_\eta$ is a valid quantum operation,
i.e.\ a completely-positive, trace non-increasing map.
To prove the positivity of $\mathcal{E}_\eta$,
let $M \ge 0$, so that it can be expressed as $M = A A^\dagger$~\cite{Con07}. 
Then 
$\mathcal{E}_\eta(M) = A A^\dagger \eta$, which can expressed as
$\mathcal{E}_\eta(M) = BB^\ddagger$ with $B = A\eta^{\frac12}$. 
Therefore $\mathcal{E}_\eta(M) \in \mathcal{B}(\mathscr{H}_\eta)$  
is positive if $M$ is positive, which proves the positivity of $\mathcal{E}_\eta$.

Complete positivity of $\mathcal{E}_\eta$ can be proven by showing that the map 
$\mathcal{E}_\eta \otimes \mathscr{I}_k:\mathcal{B}\left(\mathscr{H}\right)\otimes\mathcal{B}(\mathbb{C}^k) \to\mathcal{B}\left(\mathscr{H}_\eta\right)\otimes\mathcal{B}(\mathbb{C}^k)  $ 
is positive,
for every positive integer $k$,
where $\mathscr{I}_k$ denotes the identity map on $\mathcal{B}(\mathbb{C}^k)$. 
The action of the new map is given by
$\left[\mathcal{E}_\eta \otimes \mathscr{I}_k\right]\left(N\right)= N \left(\eta \otimes I_k\right)$,
with $I_k \in \mathcal{B}(\mathbb{C}^k)$
the identity operator. 
Operator 
$\left[\mathcal{E}_\eta \otimes \mathscr{I}_k\right](N)\in\mathcal{B}\left(\mathscr{H}_\eta\right)\otimes\mathcal{B}(\mathbb{C}^k) $
because $\|\eta \otimes I_k\| = \|\eta\|_\eta\cdot\|I_k\| = \|\eta\|_\eta$. 
For proving positivity,
let $N\in \mathcal{B}(\mathscr{H})\otimes \mathcal{B}(\mathbb{C}^k)$ be a positive operator,
so that $N = C C^\dagger$. Then $\left[\mathcal{E}_\eta \otimes \mathscr{I}_k\right](N) = C C^\dagger (\eta \otimes I_k)$, 
which can be expressed as $\left[\mathcal{E}_\eta \otimes \mathscr{I}_k\right] (N) =D D^\ddagger$ with
$D = C(\eta^{\frac12}\otimes I_k)$,
thereby proving positivity of $\left[\mathcal{E}_\eta \otimes \mathscr{I}_k\right](N)$ and consequently positivity of 
$\mathcal{E}_\eta \otimes \mathscr{I}_k$.

To prove that $\mathcal{E}_\eta$ is trace non-increasing, let $\rho \in \mathcal{D}(\mathscr{H})$
and note that trace is independent of the inner product (Eq.~\eqref{eq:rho_etatrace}).
We have $\text{tr}(\rho\eta) = \text{tr}(\eta^{\nicefrac{1}{2}}\rho\eta^{\nicefrac{1}{2}})$
and 
\begin{equation}\label{eq:tracedecreasing}
    \text{tr}(\eta^{\nicefrac{1}{2}}\rho\eta^{\nicefrac{1}{2}}) = 
    \text{tr}|\eta^{\nicefrac{1}{2}}\rho\eta^{\nicefrac{1}{2}}|
    \le \|\eta^{\nicefrac{1}{2}}\|^2\text{tr}|\rho| \le \text{tr}(\rho),
\end{equation}
where $|M| = \sqrt{\left(M^\dagger M\right)}$ and we used $|M| = M$ for any $M \ge 0$. 
The first inequality in Eq.~\eqref{eq:tracedecreasing} is a property of the trace norm~\cite{Con07}, and the last inequality in Eq.~\eqref{eq:tracedecreasing} follows from the fact that $\eta\leq \pi(I)$ and therefore, $\left\|\eta^{\nicefrac{1}{2}}\right\|^2\leq 1$.

\end{proof}

\section{Transformation of the Operators under Changing the Inner Product}\label{sec:commutation}
An inner product changing channel could modify the 
commutation relations between the operators. 
In this section, we demonstrate such a change
with an explicit example of a qubit system undergoing an inner product change.
Consider a qubit system undergoing change in inner product by
\begin{equation}
    \eta = \frac{1}{1 + r\sin\phi} \begin{pmatrix}
1 & -\text{i}r\sin\phi \\
\text{i}r\sin\phi & 1
 \end{pmatrix}, \quad 0\le r<1.
\end{equation}
The Pauli operators $X,Y,Z \in \mathcal{B}(\mathscr{H})$ acting on the original
Hilbert space along with the identity operator $I_2\in \mathcal{B}(\mathscr{H})$ generate the $\mathfrak{u}(2)$
algebra.
These operators transform according to Eq.~(5) in the main text
following the inner product change by $\eta$.
This transformation is given by the map $\mathcal{E}^{\rm op}_\eta$.
The transformed operators satisfy the commutation relations
\begin{align}
    &\left[\mathcal{E}^{\rm op}_\eta\left(X\right),\mathcal{E}^{\rm op}_\eta\left(Y\right)\right] =  2\text{i}a\ \mathcal{E}^{\rm op}_\eta\left(Z\right),\nonumber \\ 
    &\left[ \mathcal{E}^{\rm op}_\eta\left(I_2\right), \mathcal{E}^{\rm op}_\eta\left(Z\right) \right] = 2\text{i}(1-a)\  \mathcal{E}^{\rm op}_\eta\left(X\right), \nonumber \\ 
    & \left[\mathcal{E}^{\rm op}_\eta\left(Y\right),\mathcal{E}^{\rm op}_\eta\left(Z\right)\right] =  2\text{i}a\ \mathcal{E}^{\rm op}_\eta\left(X\right),\nonumber \\ 
    &\left[ \mathcal{E}^{\rm op}_\eta\left(I_2\right), \mathcal{E}^{\rm op}_\eta\left(X\right) \right] = -2\text{i}(1-a)\ \mathcal{E}^{\rm op}_\eta\left(Z\right), 
    \nonumber \\
    &\left[\mathcal{E}^{\rm op}_\eta\left(Z\right),\mathcal{E}^{\rm op}_\eta\left(X\right)\right] =  -2\text{i}(1-a)\ \mathcal{E}^{\rm op}_\eta\left(I_2\right)+2\text{i}a\mathcal{E}^{\rm op}_\eta\left(Y\right),\nonumber \\ 
    &\left[\mathcal{E}^{\rm op}_\eta\left(I_2\right),\mathcal{E}^{\rm op}_\eta\left(Y\right)\right] = \bm{0},
\end{align}
where $a = 1/(1+r\sin \phi)$. These commutation relations are different from 
those of $\mathfrak{u}(2)$ algebra for $r \ne 0$, or equivalently $a \ne 1$.


\section{{Matrix representation of the qutrit unitary operator that simulates
 change in inner product of a qubit system}}\label{sec:matrixUeta}
In this section, we derive the matrix representation of the qutrit unitary operator $U_{\tilde{\eta}}$
(see Eq.~(12) in main text) employed in the 
simulation of the change in inner product of a qubit system. Equation~(10) in the main text
requires that $PU_{\tilde{\eta}}P = \tilde{\eta}^{\nicefrac{1}{2}}$, so that
$U_{\tilde{\eta}}$ can be expressed as
\begin{equation}\label{eq:Utildeeta}
    [U_{\tilde{\eta}}] = \begin{pmatrix}
    \left[\tilde{\eta}\right]^{\frac12} & \bm{u} \\ {\bar{\bm{v}}}^\top &r\text{e}^{\text{i}\theta} 
    \end{pmatrix}
\end{equation}
for some vectors $\bm{u},\bm{v} \in \mathbb{C}^2$, a number $r \in [0,1]$ and a phase $\theta \in [0,2\pi)$.
The unitarity conditions $U_{\tilde{\eta}}^\dagger U_{\tilde{\eta}} = U_{\tilde{\eta}} U_{\tilde{\eta}}^\dagger = I_3$ lead to
\begin{align}
    \left[\tilde{\eta}\right] + {\bm{u}}\bar{\bm{u}}^\top  = I_2,
    \label{eq:unitary1}\\
    \left[\tilde{\eta}\right]^{\frac12}\bm{u} + r\text{e}^{{\rm i}\theta}\bm{v}=0,\label{eq:unitary2}\\   
    \bar{\bm{u}}^\top{\bm{u}}+r^2 =  1 \implies \|\bm{u}\| = 1-r^2 \label{eq:unitary3}.
\end{align}
Postmultiplying Eq.~\eqref{eq:unitary1} by $\bm{u}$ and substituting Eq.~\eqref{eq:unitary3} yields
$\left[\tilde{\eta}\right]\bm{u} = r^2\bm{u}$, which implies that $\bm{u}$ is the eigenvector
of $\left[\tilde{\eta}\right]^{\nicefrac{1}{2}}$ with eigenvalue $r$. Then Eq.~\eqref{eq:unitary2}
yields $\bm{v} = -\text{e}^{-{\rm i}\theta}\bm{u}$ as desired, 
with the global phase of $\bm{u}$ and $\theta$ being the free parameters.



\section{{Simulation of a qubit PT-symmetric Hamiltonian using single qutrit}}\label{sec:qubitPTsymmetry}

We now design a qutrit procedure that simulates the dynamics of a qubit Hamiltonian with unbroken PT symmetry.
Our design is based on the qutrit procedure for simulating the change in inner product of a qubit system, 
provided in the main text  (Fig.~4). 
We illustrate our Hamiltonian-simulation procedure using the PT-symmetric Hamiltonian $H_{\rm PT}$ from~\cite{BBJ02}. 

The matrix form of  $H_{\rm PT}$ is
\begin{equation}
\label{eq:qubitHPT}
    \left[H_{\rm PT}\right] = \begin{pmatrix}
    r\text{e}^{{\rm i}\phi} & s \\ s & r\text{e}^{-{\rm i}\phi}\end{pmatrix}, \quad s>r\sin\phi\ge 0.
\end{equation}
Dynamics generated by $H_{\rm PT}$ is denoted by the operator $\mathcal{U}_{\rm PT}$ (Eq.~(6) in main text),
\begin{equation}
    \rho \stackrel{\mathcal{U}_{\rm PT}}{\mapsto} \kappa U_{\rm{PT}}\rho U^\dagger_{\rm{PT}}, \; U_{\rm{PT}} := \text{e}^{-\text{i}H_{\rm {PT}}t/\hbar},\, \kappa = \frac{1}{\|\eta_2^{-1}\|}.
\end{equation}
As proved in the main text, $\mathcal{U}_{\rm PT}$ can be expressed as a sequence of operations acting exclusively on 
$ \mathcal{B}(\mathscr{H}_2)$,
\begin{equation}~\label{eq:simulatingUPT}
     \mathcal{U}_{\rm{PT}}= {\mathcal{G}}_{\kappa \eta^{-1}_2} \circ (\mathcal{R}_{\kappa\eta^{-1}_2} \circ \widetilde{\mathcal{U}}_{\rm{PT}} \circ \mathcal{R}_{\eta_2})\circ \mathcal{G}_{\eta_2},
\end{equation}
where
$\mathcal{R}_{\kappa\eta^{-1}} \circ \widetilde{\mathcal{U}}_{\rm{PT}} \circ \mathcal{R}_{\eta}: \mathcal{B}(\mathscr{H}_2)\to \mathcal{B}(\mathscr{H}_2)$
is the channel with unitary Kraus operator $\eta^{\nicefrac{1}{2}}U_{\rm PT}\eta^{\nicefrac{-1}{2}}$ and
the Kraus operator is generated by self-adjoint Hamiltonian $ h_{\rm PT} = \eta^{\nicefrac{1}{2}}H_{\rm PT}\eta^{\nicefrac{-1}{2}}$.

Our qutrit procedure for simulating $\mathcal{U}_{\rm PT}$ involves implementing each operation in Eq.~\eqref{eq:simulatingUPT}
using qutrit unitaries and measurements, as we now explain through Steps 1-4.
The input to the simulation procedure is $\rho \in \mathcal{B}(\mathscr{H}_2)$ 
embedded as $\sigma:= \rho\oplus \bm{0} \in \mathcal{B}(\mathscr{H}_3)$ and a time $t>0$.
The output of the procedure is the state $\frac{U_{\rm{PT}}\rho U^\dagger_{\rm{PT}}}{{\rm tr}\left( U_{\rm{PT}}\rho U^\dagger_{\rm{PT}}\right)}$
with probability $\frac{1}{\|\eta_2^{-1}\|}{\rm tr}\left( U_{\rm{PT}}\rho U^\dagger_{\rm{PT}}\right).$
The simulation steps are
\begin{enumerate}
    \item Calculate the metric operator and its inverse:
    The agent calculates $\eta_2$, $\eta_2^{-1}$ satisfying the quasi-Hermiticity condition $H^\dagger_{\rm PT} = \eta_2H_{\rm PT}\eta^{-1}_2$.
    A choice of $\eta_2$ and, therefore $\eta_2^{-1}$, is
\begin{equation}
    \left[\eta_2\right] = \frac{1}{s+r\sin\phi}\begin{pmatrix}
    s & -{\rm i}r\sin\phi \\{\rm i}r\sin\phi & s \end{pmatrix},
\end{equation}
\begin{equation}
    \left[\eta_2^{-1}\right] = \frac{1}{s-r\sin\phi}\begin{pmatrix}
    s & {\rm i}r\sin\phi \\-{\rm i}r\sin\phi & s \end{pmatrix},
\end{equation}
with $\|\eta_2\|=1$ and $\|\eta_2^{-1}\| =(s+r\sin\phi)/(s-r\sin\phi)$.

\item Simulate change in inner product by $\eta_2$: 
Agent implements the qutrit procedure (Fig.~4) to simulate $\mathcal{G}_{{\eta}_2}$ 
by setting $\tilde{\eta} = \eta_2$ and for a single copy of $\sigma$.
A choice of the qutrit unitary~$U_{{\eta}_2}$ (see Eq.~\eqref{eq:Utildeeta})
simulating the action of  $\mathcal{G}_{\eta_2}$ is
\begin{equation}\label{eq:Ueta2}
    U_{{\eta}_2} = \begin{pmatrix}
    (1+q)/2 & -{\rm i}(1-q)/2 & p \\
    {\rm i}(1-q)/2 & (1+q)/2 & -{\rm i}p \\
    -p & -{\rm i}p & q
    \end{pmatrix},
\end{equation}
where
\begin{equation}
    \quad q = \sqrt{\frac{s-r\sin\phi}{s+r\sin\phi}} = \frac{1}{\sqrt{\|\eta_2^{-1}\|}}, \  p = \sqrt{\frac{r\sin\phi}{s+r\sin\phi}}.
\end{equation}
The output of this step is the qutrit state 
$\frac{\eta_2^{\nicefrac{1}{2}}\rho\eta_2^{\nicefrac{1}{2}}}{{\rm tr}\left(\eta_2^{\nicefrac{1}{2}}\rho\eta_2^{\nicefrac{1}{2}}\right)}\oplus\bm{0} $
with probability ${\rm tr}\left(\eta_2^{\nicefrac{1}{2}}\rho\eta_2^{\nicefrac{1}{2}}\right)$.

\item Simulate the unitary evolution generated by $h_{\rm PT}$:
Agent calculates $h_{\rm PT}$ embedded in $\mathcal{B}(\mathscr{H}_3)$,
\begin{equation}
    \qquad\; [h_{\rm PT} \oplus \bm{0}] = \begin{pmatrix}    
    r\cos \phi & \sqrt{s^2 - r^2\sin^2\phi} & 0\\
    \sqrt{s^2 - r^2\sin^2\phi} & r\cos \phi & 0\\
    0 & 0 & 0\end{pmatrix},
\end{equation}
and implements the qutrit unitary operator $e^{-{\rm i}(h_{\rm PT}\oplus \bm{0})t}$,
which is equivalent to simulating the channel $
(\mathcal{R}_{\kappa\eta^{-1}_2} \circ \widetilde{\mathcal{U}}_{\rm{PT}} \circ \mathcal{R}_{\eta_2})$
in Eq.~\eqref{eq:simulatingUPT}. 
The output of this deterministic step is the qutrit state $\left(e^{-{\rm i}h_{\rm PT}t} \frac{\left(\eta_2^{\nicefrac{1}{2}}\rho\eta_2^{\nicefrac{1}{2}}\right)}
{{\rm tr}\left(\eta_2^{\nicefrac{1}{2}}\rho\eta_2^{\nicefrac{1}{2}}\right)}e^{{\rm i}h_{\rm PT}t}\right)\oplus\bm{0} $, 
provided Step 2 is successful.
\item Simulate change in inner product by $\kappa\eta_2^{-1}$:
Agent applies the qutrit procedure (Fig.~4)
to simulate $\mathcal{G}_{\kappa\eta_2^{-1}}$,
by setting $\tilde{\eta} = \kappa\eta^{-1}_2$.
Note that we have $\kappa = \frac{1}{\|\eta_2^{-1}\|} = q^2$ (Eqs.~\eqref{eq:simulatingUPT},~\eqref{eq:Ueta2}).
A choice of $U_{\kappa\eta^{-1}_2} $ is
\begin{equation}
    U_{\kappa{\eta}_2^{-1}} = \begin{pmatrix}
    (1+q)/2 & {\rm i}(1-q)/2 & p \\
    -{\rm i}(1-q)/2 & (1+q)/2 & {\rm i}p \\
    -p & {\rm i}p & q
    \end{pmatrix}.
\end{equation}
The output of this procedure is the qutrit state 
$\frac{\left(\eta^{\nicefrac{-1}{2}}_2e^{-{\rm i}h_{\rm PT}t}\eta_2^{\nicefrac{1}{2}}\rho\eta_2^{\nicefrac{1}{2}}e^{{\rm i}h_{\rm PT}t}\eta^{\nicefrac{-1}{2}}_2\right)}
{{\rm tr}\left(\eta^{\nicefrac{-1}{2}}_2e^{-{\rm i}h_{\rm PT}t}\eta_2^{\nicefrac{1}{2}}\rho\eta_2^{\nicefrac{1}{2}}e^{{\rm i}h_{\rm PT}t}\eta^{\nicefrac{-1}{2}}_2\right)}\oplus\bm{0}  = \frac{U_{\rm PT} \rho U_{\rm PT}^\dagger}{{\rm tr}\left( U_{\rm PT} \rho U_{\rm PT}^\dagger\right)} \oplus \bm{0}$ 
with probability  $\frac{{\rm tr}\left( U_{\rm PT} \rho U_{\rm PT}^\dagger\right)}{\|\eta_2^{-1}\|{\rm tr} \left(\eta_2^{\nicefrac{1}{2}}\rho\eta_2^{\nicefrac{1}{2}}\right)}$.
\end{enumerate}
Therefore, the output of the simulation procedure is the state 
$\frac{U_{\rm PT} \rho U_{\rm PT}^\dagger}{{\rm tr}\left( U_{\rm PT} \rho U_{\rm PT}^\dagger\right)} \oplus \bm{0}$
with success probability given by the combined probability of success in Steps 2,4, which is equal to
$\frac{1}{\|\eta_2^{-1}\|}{{\rm tr}\left( U_{\rm PT} \rho U_{\rm PT}^\dagger\right)}$.



\section{Simulation of change in inner product and PT-symmetric dynamics 
of a \texorpdfstring{$d$}{d}-dimensional system}~\label{sec:dsimulation}
We first explain a simulation procedure
to change the inner product of a $d$-dimensional system using $2d$ dimensions.
We assume that the algebra $\mathcal{A}$ of the system is represented on a $d$-dimensional Hilbert space $\mathscr{H}_d^{(s)}$ by $\pi$.
Similar to the qutrit simulation procedure explained in main text,
the agent simulating $\mathcal{G}_\eta$ for $\eta\leq\pi(I)$ first 
constructs the metric operator $\tilde{\eta} =\frac{1}{\|\eta\|}\eta$
and the unitary operator $U_{\tilde{\eta}}\in\mathcal{B}(\mathscr{H}_d^{(s)} \oplus \mathscr{H}_d^{(a)} )$
satisfying 
\begin{equation}\label{eq:Getad}
    \mathcal{G}_{\tilde{\eta}}(\rho)\oplus \bm{0} = PU_{\tilde{\eta}}\sigma U_{\tilde{\eta}}^\dagger P,\, \sigma:=\rho\oplus \bm{0},
    \;\forall \rho\in \mathcal{B}(\mathscr{H}_d^{(s)}),
\end{equation}
where $\bm{0}$ denotes the zero operator in $\mathcal{B}(\mathscr{H}_d^{(a)})$.
The matrix representation of a choice of $U_{\tilde{\eta}}$ is
\begin{equation}
\label{eq:Uetad}
    \left[U_{\tilde{\eta}}\right]
    = \begin{pmatrix}
    \left[\tilde{\eta}\right]^{\frac12} & \left[1-\tilde{\eta}\right]^{\frac12} \\ \left[1-\tilde{\eta}\right]^{\frac12} & - \left[\tilde{\eta}\right]^{\frac12}
    \end{pmatrix}.
\end{equation}
Agent then implements  $U_{\tilde{\eta}}$
followed by projective measurement and postselection on to the subspace $\mathscr{H}_d^{(s)}$.
All steps of the simulation procedure are similar to the qutrit simulation procedure for changing inner product
explained in the main text.

We now discuss how this simulation procedure for changing the inner product 
can be used for simulating PT-symmetric dynamics in $d$ dimensions
using a $2d$-dimensional system. Similar to the $d=2$ case discussed in Sec.\,\ref{sec:qubitPTsymmetry}, 
the input to the simulation procedure is $\rho \in \mathcal{B}(\mathscr{H}_d^{(s)})$ 
embedded as $\sigma:= \rho\oplus \bm{0} \in \mathcal{B}(\mathscr{H}_d^{(s)}\oplus\mathscr{H}_d^{(a)})$ and a time $t>0$.
The simulation steps are as follows:
\begin{enumerate}
    \item The agent calculates $\eta$ satisfying the quasi-Hermiticity condition $H^\dagger_{\rm PT} = \eta H_{\rm PT}\eta^{-1}$ with $\|\eta\|=1$.
    \item The agent implements the procedure described above to simulate $\mathcal{G}_{\eta}$ 
    by setting $\tilde{\eta} = \eta$ and
    for a single copy of $\sigma$ (Eq.~\eqref{eq:Getad}). 
    \item The agent calculates $h_{\rm PT} = \eta^{\nicefrac{1}{2}}H_{\rm PT}\eta^{-\nicefrac{1}{2}}$ 
    embedded in $\mathcal{B}(\mathscr{H}_d^{(s)}\oplus\mathscr{H}_d^{(a)})$,
    and implements the unitary operator $e^{-{\rm i}(h_{\rm PT}\oplus \bm{0})t}$.
    \item The agent applies the procedure described above to simulate $\mathcal{G}_{\kappa\eta^{-1}}$, 
    by setting $\tilde{\eta} = \kappa\eta^{-1}$ such that $\kappa = \frac{1}{\|\eta^{-1}\|}$ (Eq.~\eqref{eq:Getad}).
\end{enumerate}
The output of this procedure is the state $\frac{U_{\rm{PT}}\rho U^\dagger_{\rm{PT}}}{{\rm tr}\left( U_{\rm{PT}}\rho U^\dagger_{\rm{PT}}\right)}$
with probability $\frac{1}{\|\eta_2^{-1}\|}{\rm tr}\left( U_{\rm{PT}}\rho U^\dagger_{\rm{PT}}\right).$



\section{{Additional details on the verification scheme}}\label{sec:threshold}
We now prove a threshold distance $D_{\rm th}$ for the tomographic verification scheme, for the 
qutrit procedure simulating the change in inner product by an arbitrary $\eta$, provided in the main text.
The scheme allows
a verifier to distinguish an honest prover implementing the operation $\mathcal{G}_\eta$ 
from a dishonest prover failing to implement the same.
We assume that the dishonest prover implements only unitary operations, on the qubit subspace, drawn from the set  
$\{U_j \oplus \bm{1}: U_j\in\mathcal{B}\left(\mathscr{H}_2\right) \}$, where each $U_j$ is selected with probability $p_j$, 
and the system is discarded with probability $p:=1-\sum_jp_j<1$.
The quantum operation implemented by the dishonest prover is given by 
\begin{equation}
    \hat{\mathcal{G}}_\eta(\bullet) = \sum_jp_j \left(U_j \oplus \bm{1}\right)^\dagger\bullet \left(U_j \oplus \bm{1}\right).
\end{equation}
We now derive a lower bound for the induced Schatten $(1\to 1)$-norm distance~\cite{Pau03} between the 
inner-product changing operation~$\mathcal{G}_\eta\oplus \bm{0}$  
and the implemented operation $\hat{\mathcal{G}}_\eta$.
Note that 
\begin{equation}
    \|\mathcal{G}_\eta\oplus\bm{0}-\hat{\mathcal{G}}_\eta\|_{1\to 1} = \max_{T\in\mathcal{B}\left(\mathscr{H}_3\right)}\frac{  \|\mathcal{G}_\eta\oplus\bm{1}(T)-\hat{\mathcal{G}}_\eta(T)\|_{\rm tr}}{\|T\|_{\rm tr}},
\end{equation}
 where  $\|T\|_{\rm tr} = {\rm tr}\left(\sqrt{T^\dagger T}\right)$.
 For $T = \frac{1}{3}I_3$, $\|T\|_{\rm tr} = 1$. 
 Therefore,
 \begin{align}
      \|\mathcal{G}_\eta\oplus\bm{0}-\hat{\mathcal{G}}_\eta\|_{1\to 1} 
      &\ge 
      \frac{1}{3}\|\mathcal{G}_\eta\oplus \bm{0}(I_3)-\hat{\mathcal{G}}_\eta(I_3)\|_{\rm tr} \nonumber\\
      &= 
      \frac{1}{3}\|\eta\oplus\bm{1}-\sum_jp_j(U_j\oplus\bm{1})^\dagger I_3(U_j\oplus\bm{1})\|_{\rm tr} \nonumber\\
      &= 
       \frac{1}{3}\|\eta\oplus\bm{1}-(1-p)I_3\|_{\rm tr}.
 \end{align}
 We assume that the eigenvalues of $\eta$ are denoted by $\lambda_1,\lambda_2$.
 The eigenvalues satisfy  $1\geq\lambda_1>\lambda_2>0$ for any non-trivial $\eta$, i.e.\ $\eta\neq I_3$.
 The trace distance $\|\eta\oplus\bm{1}-(1-p)I_3\|_{\rm tr} = \vert \lambda_1-(1-p)\vert + \vert\lambda_2-(1-p)\vert$.
 For any $p\in[0,1)$, it can be further verified that $\|\eta\oplus\bm{1}-(1-p)I_3\|_{\rm tr}\geq {(\lambda_1-\lambda_2)}.$
 Therefore, 
 \begin{equation}
     D_{\rm th} :=  \frac{(\lambda_1-\lambda_2)}{3} \leq \|\mathcal{G}_\eta\oplus\bm{0}-\hat{\mathcal{G}}_\eta\|_{1\to 1}.
 \end{equation}
  The above given value for $D_{\rm th}$ allows the verifier to distinguish an honest prover from a dishonest one, provided the honest prover 
  implements the operation $\mathcal{G}_{\eta}$ with error less than $D_{\rm th}$, where error is quantified by the induced Schatten $(1\to 1)$-norm.

\end{appendix}
\bibliography{ref.bib}
\end{document}


\begin{comment}
\ifCLASSOPTIONcaptionsoff
\fi
\vspace{-24pt}
\begin{IEEEbiography}[
\raisebox{0.2\height}{\includegraphics[width=0.9in,height=1in,keepaspectratio]{FangLiu.jpg}}]{Fang Liu}
Fang Liu is Professor in Applied and Computational Mathematics and Statistics at the University of Notre Dame, Notre Dame, IN.  Prof. Liu obtained her Ph.D. from the University of Michigan, Ann Arbor. Her research interests include data privacy, differential privacy, statistical machine learning, model regularization, Bayesian statistics, and analysis of missing data.
\end{IEEEbiography}
\vskip 0pt plus -1.5fil
\vspace{-24pt}
\begin{IEEEbiography}[
\raisebox{0.2\height}{\includegraphics[width=0.9in,height=1in,keepaspectratio]{dw.png}}]{Dong Wang} Dong Wang  is currently working toward her Ph.D. degree of the State Key Laboratory of Engineering in Surveying, Mapping and Remote Sensing of Wuhan University, China.   Now she as a visiting Ph.D. student studied at the University of Note Dame.  Her research interests include data privacy and data mining.
\end{IEEEbiography}
\vskip 0pt plus -1.5fil
\vspace{-24pt} 
\begin{IEEEbiography}[
\raisebox{0.2\height}{\includegraphics[width=0.9in,height=1in,keepaspectratio]{zqx.png}}]{Zhengquan Xu} 
Zhengquan Xu is Professor of  the State Key Laboratory of Engineering in Surveying, Mapping and Remote Sensing, Wuhan University, China.
Prof. Xu obtained his Ph.D.degree  from the Hong Kong Polytechnic University.   His research interests include multimedia security, privacy, network communications, and spatial data storage.      
\end{IEEEbiography}
\end{comment}


\end{document}  
 
  
  
  
  
  
  
  
    

