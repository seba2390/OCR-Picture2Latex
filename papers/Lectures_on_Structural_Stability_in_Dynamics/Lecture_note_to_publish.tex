
%\renewcommand{\baselinestretch}{1.2}
%\headsep=-1truecm \oddsidemargin=8pt \evensidemargin=8pt
%\textwidth=16.5truecm \textheight=23truecm
%
%\setlength{\parindent}{12pt} %\setlength{\parskip}{12pt}
%\usepackage{enumerate}
%\setcounter{tocdepth}{1}

%\usepackage{makeidx}
%\makeindex


\documentclass[11pt,openany,leqno]{article}
\usepackage{amsmath,amsthm,amsfonts,amssymb,amscd,url}
\usepackage{graphics}
\usepackage[latin1]{inputenc}
\usepackage{hyperref}
\usepackage{epsfig} 


\input{xy} \xyoption{all}

\renewcommand{\baselinestretch}{1.2}
\headsep=-1truecm \oddsidemargin=8pt \evensidemargin=8pt
\textwidth=16.5truecm \textheight=23truecm
\setlength{\parindent}{12pt} %\setlength{\parskip}{12pt}
%\usepackage{showkeys} 
\usepackage{enumerate}
\setcounter{tocdepth}{1}
\usepackage{hyperref}
\usepackage{epsfig} 
\input{xy} \xyoption{all}

\begin{document}

\def\Diff{\text{Diff}}
\def\Max{\text{max}}
\def\P{\mathbb P}
\def\R{\mathbb R}
\def\T{\mathbb{T}}
\def\N{\mathbb N}
\def\Z{\mathbb Z}
\def\C{\mathbb C}
\def\D{\mathbb D}
\def\a{{\underline a}}
\def\b{{\underline b}}
\def\n{{\underline n}}
\def\Log{\text{log}}
\def\loc{\text{loc}}
\def\inta{\text{int }}
\def\det{\text{det}}
\def\exp{\text{exp}}
\def\Re{\text{Re}}
\def\lip{\text{Lip}}
\def\leb{\text{Leb}}
\def\dom{\text{Dom}}
\def\diam{\text{diam}\:}
\def\supp{\text{supp}\:}
\newcommand{\ovfork}{{\overline{\pitchfork}}}
\newcommand{\ovforki}{{\overline{\pitchfork}_{I}}}
\newcommand{\Tfork}{{\cap\!\!\!\!^\mathrm{T}}}
\newcommand{\whforki}{{\widehat{\pitchfork}_{I}}}
\newcommand{\marginal}[1]{\marginpar{{\scriptsize {#1}}}}
\def é{{\' e}}
%\def è{{\` e}}
%\def ê{\^ e}
%\def ç{\c c}
\def\sR{{\mathfrak R}}
\def\sM{{\mathfrak M}}
\def\sA{{\mathfrak A}}
\def\sB{{\mathfrak B}}
\def\sY{{\mathfrak Y}}
\def\sE{{\mathfrak E}}
\def\sP{{\mathfrak P}}
\def\sG{{\mathfrak G}}
\def\sa{{\mathfrak a}}
\def\sb{{\mathfrak b}}
\def\sc{{\mathfrak c}}
\def\se{{\mathfrak e}}
\def\sg{{\mathfrak g}}
\def\sd{{\mathfrak d}}
\def\sr{{\mathfrak {r}}}
\def\ss{{\mathfrak {s}}}
\def\sD{{\mathfrak {p}}}
\def\sp{{\mathfrak {p}}}
\def\arr{\overleftarrow}
\def\u{\underline}

\newtheorem{prop}{Proposition} [section]
\newtheorem{thm}[prop] {Theorem}
\newtheorem{conj}[prop] {Conjecture}
\newtheorem{defi}[prop] {Definition}
\newtheorem{lemm}[prop] {Lemma}

\newtheorem{prob}[prop] {Problem}

\newtheorem{sublemm}[prop] {Sub-Lemma}
\newtheorem{cor}[prop]{Corollary}
\newtheorem{theo}{Theorem}
\newtheorem{theoprime}{Theorem}
\newtheorem{Claim}[prop]{Claim}
\newtheorem{fact}[prop]{Fact}

\newtheorem{coro}[theo]{Corollary}
\newtheorem{defprop}[prop]{Definition-Proposition}
\newtheorem{propdef}[prop]{Proposition-Definition}

\newtheorem{question}[prop]{Question}
\newtheorem{conjecture}[prop]{Conjecture}

\theoremstyle{remark}
\newtheorem{exam}[prop]{Example}
\newtheorem{rema}[prop]{Remark}

\renewcommand{\thetheo}{\Alph{theo}}
\renewcommand{\thetheoprime}{\Alph{theo}$'$}
\newtheorem{propfonda}[theo]{\bf Fundamental property of the parablenders}%[section]





\title{
Lectures on Structural Stability in Dynamics
}


\author{Pierre Berger\footnote{CNRS-LAGA, Université Paris 13.}}

\date{}

\maketitle

\begin{abstract} 
These lectures  present  results and problems on the characterization of structurally stable dynamics. 
We will shed light those which do not seem to  depend on the regularity class (holomorphic or differentiable).
Furthermore, we will present some links between the problems of structural stability in dynamical systems and in singularity theory. 
%It provides some material to attack some problem at the interface between holomorphic dynamics and differentiable dynamics, and those which are at the interface between  differentiable dynamics and syngularity theory.
\end{abstract}
\tableofcontents

\section*{Introduction}

Structural stability is one of the most basic topics in dynamical systems and contains some of the hardest conjectures.
  Given a class of regularity $\mathcal C$, which can be $C^r$ for $1\le r\le \infty$ or holomorphic, and formed by diffeomorphisms or endomorphisms of a manifold $M$, the problem is to describe the \emph{structurally stable dynamics} for the class $\mathcal C$. We recall that a dynamics $f$ is $\mathcal C$-structurally stable if for any perturbation $\hat f$ of $f$ in the class $\mathcal C$, there exists a homeomorphism $h$ of $M$ so that $h\circ f= f\circ h$.  
Uniform hyperbolicity seems to provide a satisfactory way to describe the structurally stable dynamics. This observation goes back to the Fatou conjecture for quadratic maps of the Riemannian sphere in 1920 and the Smale conjecture for smooth diffeomorphisms in 1970. These conjectures have been deeply studied by many mathematicians and so they are difficult to tackle directly. 

However at the interface of one-dimensional complex dynamics and differentiable dynamics, the field of two-dimensional complex dynamics  grew up recently. It enables to study the structural stability problem thanks to ingredients of both  fields. 
Also the mathematics  developed in the 1970's for the structural stability in dynamics is very similar to the one developed for the structural stability in singularity theory. This led us to combine both in the study of the structurally stable endomorphisms,  
We will review some classical works  in these beautiful fields,  some works more recent,   and we will present  new open problems at these interfaces.


In section \ref{hyp:sec}, we will recall some elementary definitions of uniform hyperbolic theory, and we will detail a few examples of such dynamics.  

In section \ref{secStabimplieshyp} we will state several theorems and conjectures suggesting the hyperbolicity of structurally stable dynamics. In particular we will recall the seminal work of Ma\~n\'e \cite{Ma88} showing this direction in the $C^1$-category. For holomorphic dynamical systems, we will present the work of Dujardin-Lyubich \cite{LD13}  and our work with Dujardin \cite{BD14} generalizing some aspects of 
Ma\~ n\'e-Sad-Sullivan and Lyubich theorems \cite{MSS, 
Ly84} for polynomial  automorphisms of $\C^2$. 

In section \ref{hypimpliesstab}, we will present several results in the directions ``hyperbolicty $\Rightarrow$ stability". In \textsection \ref{hypimpliesstab1}, we will recall classical results, including the structural stability theorems  of Anosov \cite{An67}, Moser\cite{Mo69} and Shub \cite{Shub69}, and the proof of this direction of the $\Omega$-stability conjecture by Smale\cite{Sm68} and Przytycki  \cite{Pr77}. 
Then in \textsection \ref{hypimpliesstab2}, we will sketch the  proof of this direction of  the structural stability theorem by Robbin \cite{Ro71} and Robinson \cite{Ro76} ; and we will relate a few works leading to a generalization of the Przytycki  conjecture \cite{Pr77}, a description of the structurally stable local diffeomorphisms. Finally in \textsection \ref{hypimpliesstab3}, we will recall our conjecture with Rovella \cite{BR13} stating a description of the endomorphisms (with possibly a non-empty critical set) whose inverse limit is structurally stable, and we will state our theorem with Kocsard \cite{BK13} showing one direction of this conjecture. 
%One surprising fact is that this description does not involve singularity theory, even if they are inverse stable dynamics with non-empty critical set.

In section \ref{sectionLinks}, we will recall several results from singularity theory and we will emphasize on their similarities with those of structural stability.

In section \ref{Sec_SS_endo_w_Singu}, we will present the work
\cite{Be12} which  states sufficient conditions for a smooth map with non-empty critical set to be structurally stable.  The statement involves developments of 
Mather's theorem on Singularity Theory of composed mappings. It suggests the problem of the description of the structurally stable, surface endomorphisms among those which display singularity but satisfy the axiom A. 


\medskip
\thanks{These notes were written while I was giving lectures at Montevideo in 2009
and at the Banach Center in 2016. I am very grateful for their hospitality.
}
\section{Uniformly hyperbolic dynamical systems}\label{hyp:sec}
The theory of \emph{uniformly hyperbolic dynamical systems} was constructed in the 
1960's under the dual leadership of Smale in the USA, and Anosov and Sinai in the Soviet Union.\footnote{A few sentences of this section are taken from \cite{BY14}.}
\par
 It encompasses various examples that we shall recall: expanding maps, horseshoes, solenoid maps, Plykin attractors,  Anosov maps, DA, blenders all of which are \emph{basic pieces}. 

%We shall some element of this theory, first in the invertible set up and then in the endomorphisms case.
\subsection{Uniformly hyperbolic  diffeomorphisms}
 Let $f$ be a $C^1$-diffeomorphism $f$ of a finite dimensional manifold $M$. A compact $f$-invariant subset $\Lambda \subset M$ is 
\emph{uniformly hyperbolic} if the restriction to $\Lambda$ of the tangent bundle $TM$
splits into two continuous invariant subbundles
\[TM|\Lambda = E^s\oplus E^u,\]
$E^s$ being  uniformly contracted and $E^u$ being uniformly expanded: 
 $\exists \lambda<1$, $\exists C>0$,  
\[\|T_xf_{|E^s}^n\|<C\cdot\lambda^n\quad\mathrm{and} \quad \|T_xf_{|E^u}^{-n}\|<C\cdot\lambda^n, \quad \forall x\in \Lambda, \forall n\ge 0.\]


 





\begin{exam}[Hyperbolic periodic point]
A periodic point at which the differential has no eigenvalue of modulus $1$ is called hyperbolic. It is a sink if all the eigenvalues are  of modulus less than 1, a source if all of them are of modulus greater than 1, and a saddle otherwise. 
\end{exam}

\begin{defi} A \emph{hyperbolic attractor}  is a hyperbolic, transitive compact subset $\Lambda$ such that there exists a neighborhood  $N$ satisfying $\Lambda = \cap_{n \geq 0} f^n(N)$.
\end{defi}
\begin{exam}[Anosov] If the compact hyperbolic set is equal to the whole compact manifold, then the map is called \emph{Anosov}. 
For instance if a map $A\in SL_2(\Z)$ has both eigenvalues of modulus not equal to $1$, then it acts on the torus $\R^2/\Z^2$ as an Anosov diffeomorphism. The following linear map satisfies such a property:  
\[A:= \left[\begin{array}{cc}
2&1\\
1&1\end{array}\right]\; .\]
\end{exam}



\begin{exam}[Smale solenoid]
We consider a perturbation of the map of the filled torus 
$\mathbb T := \{(\theta, z)\R/\Z \times \mathbb C: |z|<1\}$ 
:
\[(\theta, z)\in \mathbb T \mapsto (2\theta, 0)\in  \mathbb T \;,\]
which is a diffeomorphism onto its image. This is the case of the following:  
\[(\theta, z)\in \mathbb T \mapsto (2\theta, \epsilon \cdot z+ 2\epsilon\cdot  \exp(2\pi i \theta )\in  \mathbb T \;.\]
This defines a hyperbolic attractor called the \emph{Smale solenoid}.
%\begin{figure}[h!]
%	\centering
%		\includegraphics[width=7cm]{s_Soleno_1.png}
%	\caption{the Smale Solenoid (Credit Steklov institut)}
%\end{figure}
\end{exam}
\vspace{1cm}
\begin{exam}[Derivated from Anosov (DA) and Plykin attractor]
We start with a linear Anosov of the 2-torus $\R^2/\Z^2$. It fixes the point 0.  In local coordinates $\phi$ of a neighborhood $V$ of $0$, it has the form for $0<\lambda<1$:
$$(x,y)\mapsto (\lambda x, y/\lambda).$$ 
For every $\epsilon>0$, let $\rho_\epsilon$ be a smooth function 
so that: 
\begin{itemize}
\item it is equal to $x\mapsto \lambda x$ outside of the interval $(-2\epsilon, 2\epsilon)$,
\item $\rho_\epsilon$ displays exactly three fixed point: $-\epsilon$ and $\epsilon$ which are contracting and  $0$ which is expanding. 
\end{itemize}
Let $DA$ be the map of the two torus equal to $A$ outside of $V$, and in the coordinate $\phi$ it has the form:
 $$(x,y)\mapsto (\rho_\epsilon( x), y/\lambda).$$ 

We notice that $0$ is an expanding the fixed point of DA. The complement of its repulsion basin is a hyperbolic attractor. 
   
\begin{figure}[h!]
	\centering
		\includegraphics[width=5cm]{DAbw.png}
	\caption{Derivated of Anosov (Credit Y. Coudene \cite{Co06})}
\end{figure}

The DA attractor project to a basic set of a surface attractor, the \emph{Plykin attractor}. 

\begin{figure}[h!]
	\centering
		\includegraphics[width=9cm]{Plykin.png}
	\caption{Plykin attractor (Credit S. Crovisier)}
\end{figure}
\end{exam}



Given a hyperbolic compact set $\Lambda$, for every $z\in \Lambda$, the sets 
\[W^s(z)= \{z'\in M:\; \lim_{ n\to+\infty} d(f^n(z),f^n(z'))= 0\},\]
\[W^u(z)= \{z'\in M:\; \lim_{ n\to-\infty} d(f^n(z),f^n(z'))= 0\}\]
%W^s(z)= \{z'\in M:\; d(f^n(z),f^n(z'))\to 0,\; n\to+\infty\}.\]
are called \emph{the stable and unstable manifolds} of $z$. They are immersed manifolds tangent at $z$ to respectively $E^s(z)$ and $E^u(z)$.

The  \emph{$\epsilon$-local  stable manifold} $ W^s_\epsilon(z)$ of $z$ is the connected component of $z$ in the intersection of $W^s(z)$ with a $\epsilon$-neighborhood  of $z$. The \emph{$\epsilon$-local  unstable manifold} $ W^u_\epsilon(z)$ is defined likewise.

\begin{prop}
For $\epsilon>0$ small enough, the subsets  $ W^s_\epsilon(z)$ and $ W^u_\epsilon(z)$ are $C^r$-embedded manifolds which 
depend continuously on $z$ and tangent at $z$ to respectively  $E^s(z)$ and $E^u(z)$.
\end{prop}
A nice proof of this proposition can be found in \cite{Yoccozintro}.
%\begin{proof} 
%By replacing $f$ by an iterate if necessary, we can assume that $E^s$ is $\lambda$-contracted and $E^u$ is $\lambda^{-1}$-expanded.  
%
%Let $\tilde E^u$ and $\tilde E^s$ be continuous extensions of the plane fields $E^s$ and $E^u$ to an $\epsilon$-neighborhood $U$ of $\Lambda$.  If $\epsilon$ is small enough, the following cones field:
%\[C^u(z):= \{(u,v)\in \tilde E^u\times\tilde  E^s: \; \|u\|\le \|v\|\}\] 
% is sent into itself:
% \[\forall z\in U\cap f^{-1}(U),\; Df(C^u(z))\subset C^u(f(z))\;.\]
%For every $z\in \Lambda$, we consider the space $\Gamma_z$ of proper Lipschitz submanifolds of $B(z,\epsilon)$ with tangent space space in $C^u$. We recall that the tangent space of a Lipschitz submanifold $\gamma$ at $z\in \gamma$ is the set of cluster values of the sequences of the form $t (z-z_n)/\|z-z_n\|$, among sequences $z_n\in \gamma \to z$ and $t\in \R$. 
%
% Endowed with the Hausdorff distance, $\Gamma_z$ is compact. Likewise, the space $\Gamma^0$ of continuous sections  of the bundle $\Gamma\to \Lambda$ with fibers $\Gamma_z$ at $z\in \Lambda$ is compact endowed with the uniform distance. 
%
%We notice that the following map sends the space  $\Gamma$ into itself:
%\[f^\#\colon (\gamma_z)_{z\in \Lambda} \mapsto (f(\gamma_{f^{-1}(z)})\cap B(z,\epsilon) )_{z\in \Lambda}\; .\]
%
%Furthermore, it is contracting, and so it has a unique fixed point $(W^u_\epsilon (z))_{z\in \Lambda}$ in $\Gamma^0$. 
%
%
%Let us show that these unstable manifolds are of class $C^1$.
%
%For the sake of simplicity, we assume $U\subset \R^n$. We endow The Grassmanian of $dim E^u$-plans at $z\in U$ with the metric: 
%$$d(E,F)= \sup \{\|u-v\|:\; u,v\in E\times F:\; \|u\|=\|v\|=1\}\;.$$
%We endow $GL_n(\R)$ with the subordinated norm.  
%
%%we identify the Grassmian bundle in $C^u$ with the bundle of linear maps from $\tilde E^u$ to $\tilde E^s$ endowed with the norm subordinated to the the Riemannian metric of $M$. 
%
%The continuity modulus of $Df$ is the smallest continuous function $m_f$ so that for every $z,z'\in M$, 
%$$\|D_zf-D_{z'}f\|\le m_f(d(z,z')).$$
% Likewise, the continuity modulus of $\gamma\in\Gamma^0$ is the smallest family of continuous function $ m_\gamma$ so that for all $z\in M$ and $z_1,z_2\in \gamma_z$,  
%$$d( T_{z_1}\gamma_{z},T_{z_2}\gamma_{z})\le m_\gamma(d(z_1,z_2))\;.$$
%Let us bound from above the continuity modulus of $f^\#\gamma$:
%\[
%d( T_{f(z_1)}f^\#\gamma_{f(z)},T_{f(z_2)}f^\#\gamma_{f(z)})
%= d_G(D_{z_1}f(T\gamma_{z_1}),D_{z_2}f(T\gamma_{z_2}))\]
% \[\le 
%d_G(D_{z_1}f(T\gamma_{z_1}),  D_{z_1}f(T\gamma_{z_2}))
%+
%d_G(D_{z_1}f(T\gamma_{z_2}) , D_{z_2}f(T\gamma_{z_2}))\]
% \[\le \lambda m_\gamma (\|z_1-z_2\|)+ m_f(\|z_1-z_2\|)\]
%As $d(f(z_1),f(z_2))\ge d(z_1,z_2)$, it comes that the modulus of continuity satisfies:
% \[ m_{f^\#\gamma}(r)= \inf_{z\in \Lambda,\; z_1,z_2\in \gamma_z,\; d(z_1,z_2)=r} d(T_{z_1} f^\#\gamma_z, T_{z_2} f^\#\gamma_z)
%   \le \lambda m_\gamma(r) +m_f(r).\]
%Hence $f^\#$ leaves invariant the subset of a certain continuous family of curves 
%
%Hence given $\gamma\in \Gamma^0$, the modulus of continuity of the tagent spaces of $({f^\#}^n\gamma)_n$ remains bounded and so its limit $(W^u_\epsilon (z))_{z\in \Lambda}$  is of class $C^1$.
%%Each submanifold $W^u_\epsilon (z)$ is of class $C^1$ if for every  $z'\in W^u_\epsilon (z)$, for every $u\in E^u(z)$, there exists a unique $v\in C^u(z')$  so that for every sequence $(z_n)$ converging to $z'$ with the $E^
% 
%
%\end{proof}





\begin{defi} 
A \emph{basic set}  is a compact, $f$-invariant, transitive,  uniformly hyperbolic  set $\Lambda$ which is \emph{locally maximal}:
there exists   a neighborhood $N$ of $\Lambda$ such that $\Lambda = \cap_{n\in \Z} f^n(N)$.
\end{defi}

%transitive and satisfies one of the following equivalent property:
%\begin{itemize}
%\item  $K$ is locally maximal : 
%\item $K$ has a structure of local product : for $\epsilon>0$ small enough, and any $x,y\in K$ close enough, the intersection point $W^u_{\epsilon} (x)\cap W^s_{\epsilon} (y)$ belongs to $K$.
%\item $K$ is included in the closure of the set of periodic points in $K$: $K= cl(Per(f|K))$. 
%\end{itemize}
%\end{defi}
%The equivalence of these conditions is proved in \cite{shubstab78}.


%\begin{defi} A hyperbolic set has a structure of local product if for $\epsilon>0$ small enough, and any $x,y\in K$ close enough, the intersection point $W^u_{\epsilon} (x)\cap W^s_{\epsilon} (y)$ belongs to $K$. The intersection point is denoted by $[x,y]$
%\end{defi}
%
%\begin{defi} 
%A \emph{basic set}  is a compact, $f$-invariant, uniformly hyperbolic  set $\Lambda$ which is transitive and \emph{locally maximal}: there exists   a neighborhood $N$ of $\Lambda$ such that $\Lambda = \cap_{n\in \Z} f^n(N)$. A basic set is an \emph{attractor} if the neighborhood $N$ can be chosen in such a way that $\Lambda = \cap_{n \geq 0} f^n(N)$. Such a basic set contains the unstable manifolds of its points.
%\end{defi}
%
%\begin{prop}
%A basic set has a  local product structure.
%\end{prop}
%\begin{proof} For $x,y\in K$ close enough, the points 
%$[x,y]$ has its forward orbit which remains close to the one of $x$, and its back ward orbit which remains close to the one of $x$. Hence 
%$[x,y]$ has its orbit which is included in a small neighborhood of $K$. 
%By local maximality of $K$, $[x,y]$ belongs to $K$. 
%\end{proof}

%Conversely we have:
%\begin{prop}
%If a hyperbolic set has a local product structure, then it is locally maximal and so a basic set.
%\end{prop}
%\begin{proof}
%Si une orbite est dans $W^s_\epsilon(K)$ alors elle est dans $K$. En effet, si $f^{-n}(z) \in W^s_\epsilon(K)$, alors $z$ is in $W^s_{\epsilon/\Lambda^n}(K)$. Donc $z\in K$.
%
%
%
%On a une lamination $W^s_\epsilon(K)$. On prend $n$ tq la distance de  $f^n(z)$ à $W^s_\epsilon(K)$ est maximal pour un chemin dans le cone instable.  
%
%
%Let $z$ be so that its orbit remains in a small neighborhood of $K$. 
%
%Then $W^s_\epsilon(f^n(z))$ and $W^u_\epsilon(f^n(z))$ are well defined for every $n\in \Z$. 
%
%For every $n$, let $z_n$ be close to $f^n(z)$. Let $w_n$ be the intersection point of  $W^u_\epsilon(f^n(z))$ with $W^s_\epsilon(z_n)$. 
%
%For every $n$, let us chose $z_n$ such that the distance between $w_n$ and $z_n$ is minimal. 
%
%
%
%
%For every $n$, let $z_n\in K$ be at a minimal distance to $W^s_\epsilon(f^n(z))$.  Let $n_0$ be so that this distance is maximal.
%
% Since By normal expansion of this 
%
%close to $f^n (z)$.  It comes that  $W^s_\epsilon(f^n(z))$ intersects  $W^u_\epsilon(z_n)$ at a point 
%$w_n$.  Let $n$ be so that the distance between $z_n$ and $w_n$ is minimal. By 
%
%As  $f$ is $\Lambda>1$ exmapinding along  
%
% 
%\end{proof}
%\begin{rema} 
%Usually, one defines a basic piece as a hyperbolic set included in the closure of the set of periodic points. actually the three following assertion are equivalent for every hyperbolic compact set $K$:
%\begin{itemize}
%\item  $K$ is locally maximal,
%\item $K$ has a structure of local product,
%\item $K$ is included in the closure of the periodic point. 
%\end{itemize}
%The proofs of the remaining implications involve a shadowing lemma \cite[section 8, Prop 8.13- 8.20 ]{shubstab78}.
%\end{rema}

\begin{exam}[Horseshoe] 
A \emph{horseshoe} is a basic set which is a Cantor set. 
For instance take two disjoint sub-intervals $I_+\sqcup I_- \subset [0,1]$, and let $g \colon I_+\sqcup I_-  \to [0,1]$ be a locally affine map which sends each of the intervals $I_\pm$  onto $[0,1]$.  Let $g_+$ be its inverse branch with value in $I_+$ and let $g_-$ be the other inverse branch. 
Let $f$ be a diffeomorphism of the plane whose restriction to $I_\pm \times [0,1]$ is:
\[(x,y)\in  (I_+\sqcup I_-) \times [0,1] \to 
\left\{ \begin{array}{c}
(g(x),g_+(y)) \quad \text{ if }x\in I_+\\
(g(x),g_-(y)) \quad \text{ if }x\in I_-\end{array}\right.\]
\begin{figure}[h!]
		\centering
			\includegraphics[width=7cm]{Horseshoe.pdf}
						% horseshoe.png
		\caption{Smale's Horseshoe}
	\end{figure}
\end{exam}

\begin{rema} 
Usually, one defines a basic piece as a hyperbolic set included in the closure of the set of its periodic points. Actually the three following assertion are equivalent for every uniformly hyperbolic, transitive, compact set $K$:
%transitive and satisfies one of the following equivalent property:
\begin{itemize}
\item  $K$ is locally maximal.
\item $K$ has a structure of local product : for $\epsilon>0$ small enough, and any $x,y\in K$ close enough, the intersection point $W^u_{\epsilon} (x)\cap W^s_{\epsilon} (y)$ belongs to $K$.
\item $K$ is included in the closure of the set of periodic points in $K$: $K= cl(Per(f|K))$. 
\end{itemize}
The equivalence of these conditions is proved in \cite{shubstab78}.
\end{rema}





\begin{defi}[Axiom A] A diffeomorphism whose non-wandering set is a finite union of disjoint  basic sets is called   \emph{axiom A}.
 \end{defi}
 
 \bigskip
 \begin{exam}[Morse-Smale] 
A  Morse-Smale diffeomorphism is a diffeomorphism of a surface so that its non-wandering set consists of finitely many periodic hyperbolic points, and their  stable and unstable manifolds are transverse.
\begin{figure}[h!]
	\centering
		\includegraphics[width=7cm]{Morse-Smale.pdf}
	\caption{Morse-Smale}
\end{figure}
 \end{exam}



\subsection{Uniformly hyperbolic endomorphisms}


%\section{Uniformly hyperbolic endomorphisms}
A \emph{$C^r$-endomorphism} of a manifold $M$ is a differentiable map of class $C^r$ of $M$, which is not necessarily injective, nor surjective, and that may possess points at which the differential is not onto (called critical points). The \emph{critical set} is the subset of $M$ of formed by the critical points. 

A \emph{local $C^r$-diffeomorphism} is a $C^r$-endomorphism without critical point.

A compact subset $\Lambda\subset M$ is \emph{invariant} for an endomorphism $f$ of $M$ if $f^{-1} (\Lambda)=\Lambda$. 
A compact subset $\Lambda\subset M$ is \emph{stable} for an endomorphism $f$ of $M$ if $f (\Lambda)=\Lambda$.

An invariant compact set is \emph{hyperbolic} if there exists a subbundle $E^s\subset TM \Lambda$ which is left invariant  and uniformly contracted by $Df$ and so that the action of $Df$ on $TM/E^s$ is uniformly expanding.

\begin{exam}[Expanding map]
Let $f\in End^1(M)$ and  an invariant stable, compact  subset $K$ is \emph{expanded} if there exists $n\ge 1$ s.t., for every $x\in K$,  $D_xf^n$ is invertible and with contracting inverse. When  $K=M$, $f$ is said \emph{expanding}.
\end{exam}   
\begin{exam}[Anosov endomorphism] If a hyperbolic set is equal to the whole manifold, then the endomorphism is called \emph{Anosov}. 
For instance this is the case of the dynamics on the torus $\mathbb R^2/\mathbb Z^2$ induced by a linear maps in $M_2(\Z)$ with eigenvalues of modulus not equal to 1. For instance, it the case of the following for every $n\ge 2$:
\[\left[\begin{array}{cc}
n&1\\
1&1\end{array}\right]\]
\end{exam}    

%\begin{exam}[Derivated from Anosov endomorphisms]
%\end{exam}    




The stable manifold of $z$ in a hyperbolic set $\Lambda$ of an endomorphism is defined likewise: 
\[W^s(z)= \{z'\in M:\; \lim_{ n\to+\infty} d(f^n(z),f^n(z'))= 0\}.\] 

The unstable manifold depends on the preimages.  For every orbit $\underline z= (z_n)_{n\in \Z} \in \Lambda^\Z$, has an unstable manifold:  
\[W^u(\underline z)= \{z'\in M:\; \exists (z'_n)_n \text{ orbit s.t. } \lim_{ n\to-\infty} d(z_n,z'_n)= 0\}.\]

If $f$ is a local diffeomorphism then $W^s$ and $W^u$ are immersed, but in general only $W^s$ is injectively immersed. 



\begin{defi}A hyperbolic set $\Lambda$ is a \emph{basic piece} if it is locally maximal.
\end{defi}

\begin{exam}[Blender]\label{blender}
A blender of surface endomorphism is a basic set so that $C^1$-robustly its local unstable manifold cover an open subset of the surface. 

For instance let $I_-$ and $I_+$ be two disjoint segments of $[-1,1]$, and let $Q$ be a map which sends affinely each of these segments onto $[-1,1]$. 
 This is the case for instance of the following map:
\[(x,y)\in [-1,1]^2\mapsto \left\{\begin{array}{cl}
(Q(x), (2 y+1)/3)& x\in I_+\\
 (Q(x), (2 y-1)/3)& x\in I_-\end{array}\right. \]

\begin{figure}[h!]
	\centering
		\includegraphics[width=7cm]{blenderbw.pdf}
	\caption{Blender of a surface local diffeomorphism}
\end{figure}

\end{exam}    
\begin{defi}An endomorphism satisfies  \emph{axiom A} if its non-wandering set is a finite  union of basic pieces.\end{defi}

%\begin{exam}[Generalized solenoid]
%Let $f$ be a uniformly hyperbolic $C^1$-covering of a compact manifold $K\subset M$. Hence there exists a finite covering $(U_i)_{1\le i\le n}$ of $M$ so that $f|U_i$ is a diffeomorphism onto its image. 
%Let $(\rho_i)_i$ be a partition of unity adapted to $(U_i)_i$. 
%Then for $\epsilon>0$ small, the following map has a unique attractor next to $M\times \{0\}$:
%\[(x,y)\in M\times \R^n\mapsto (f(x), \epsilon y +\sum_{I=1}^n \rho_i(x) e_i)\in M\times \R^n\;,\]  
%with $y= (y_i)_{i=1}^n$ and $(e_i)_i$ the canonical basis of $\R^n$. 
%Moroever, restricted to the product of $M$ with the unit ball of $\R^n$, it is a a diffeomorphisms onto its image. 
%\end{exam}

%\subsection{Basic properties of uniformly hyperbolic dynamics}
%
%% \bigskip
% 
%Hyperbolic attractors enjoy nice properties, which are proved in \cite{Sm} and the references therein.
%
%
%  
%\paragraph{SRB and physical measure}
%
%Let $\alpha >0$, and let $\Lambda$ be an attracting basic set for a $C^{1+\alpha}$~-diffeomorphism $f$. Then there exists a unique invariant, ergodic probability $\mu$ supported on $\Lambda$ such that its conditional measures, with respect to any measurable partition of $\Lambda$ into plaques of unstable manifolds, are absolutely continuous with respect to the Lebesgue measure class (on unstable manifolds). Such a probability is called \emph{SRB} (for Sinai-Ruelle-Bowen). It turns out that a SRB -measure is \emph{physical}: the Lebesgue measure of its basin $B(\mu)$
%\begin{equation}\tag{$\mathcal B$} B(\mu) = \{z\in M: \; \frac1n \sum_{i < n} \delta_{f^i(x)} \rightharpoonup \mu\},\end{equation} 
% is positive. Actually, up to a set of Lebesgue measure $0$, $B(\mu)$ is equal to the topological basin of $\Lambda$, i.e the set of points attracted by $\Lambda$.
%
%
%
%%\paragraph{Local product structure} Let $\Lambda$ be a basic set and let $\epsilon >0$. There exists $\delta >0$ such that, for any $x,y \in \Lambda$ with $d(x,y)<\delta$, the intersection $ W^s_\epsilon(z) \cap  W^u_\epsilon(y)$ is a single point belonging to $\Lambda$, denoted by $[x,y]$. For any $z \in \Lambda$, the map $(x,y) \to [x,y]$ is a homeomorphism from $ (W^u_{\delta /2}(z) \cap \Lambda) \times  (W^s_{\delta /2}(z) \cap \Lambda)$ onto a neighborhood of $z$ in $\Lambda$.
%%\paragraph{Closing Lemma} A basic set $\Lambda$ for a $C^1$-diffeomorphism $f$ is included in the closure of the periodic points: $\Lambda\subset cl(Per(f))$.
%%locally maximal : there exists a neighborhood $N$ of $\Lambda$ such that $\Lambda= \cap_{n\in \Z} f^n(N)$.
%
%\paragraph{Coding} A basic set $\Lambda$ for a $C^1$-diffeomorphism $f$ admits a (finite) Markov partition. This implies that its dynamics is semi-conjugated with a subshift of finite type. The semi-conjugacy is 1-1 on a generic set. Its lack of injectivity is itself coded by subshifts of finite type of smaller topological entropy. This enables to study efficiently all the invariant measures of $\Lambda$, the distribution of its periodic points, the existence and uniqueness of the maximal entropy measure, and if $f$ is $C^{1+\alpha}$, the Gibbs measures which are related to the geometry of $\Lambda$.
%
%\paragraph{Structural stability} A basic set $\Lambda$ for a $C^1$-diffeomorphism $f$ is \emph {persistent}: every  $C^1$-perturbation $f'$ of $f$ leaves invariant a basic set $\Lambda'$ which is homeomorphic to $\Lambda$, via a homeomorphism which conjugates the dynamics $f|\Lambda$ and $f'|\Lambda'$.
%
% \bigskip

%\section{Introduction to the presented developments}
%
%
%Smale wished to describe the behavior of typical orbits of a typical dynamical system. 
%For this end, he conjectured the density of axiom A in the space of $C^r$-diffeomorphisms $f$ of any compact manifold $M$. 
%
%
%\medskip 
%
%In higher dimensions,  obstructions were soon discovered by Abraham-Smale \cite{AS68} and then Shub \cite{Sh71}. The latter became a paradigmatic example of the theory of partially hyperbolic dynamical systems (that we will not describe deeply). 
%For such systems, there exists an invariant splitting $E^s\oplus E^c\oplus E^u=TM$ of the tangent space $TM$ of the manifold $M$, so that $E^s$ is more contracted than $E^c\oplus E^u$ and $E^u$ is more expanded   than $E^s\oplus E^c$. 
%
%The \emph{mostly expanded} case occurs when  $E^c$ is asymptotically expanded in the following sense:
%\[\exists m>0:\quad \limsup_{n\to \infty} \frac1n \sum \log\|D_{f^n(x)}f^{-1}|E^c\|<-m\quad \text{for } \leb. \; a.e. \; x\in M\; .\]
%Then a Theorem of Alves-Bonatti-Viana \cite{ABV00} states that there exist finitely many SRB measures whose basins cover Lebesgue a.e. $M$. 
%
%The \emph{mostly contracting} case is when the maximal invariant set $K$  of $M$ is compact, partially hyperbolic and so that for every $z\in K$, with $W^{uu}_{\loc}(z)$ the local,  strong unstable manifold of $z$ and $\leb$ its Lebesgue measure, it holds:
%\[0< \leb\{x\in W^{uu}_{loc}(z) : \limsup_{n\to \infty} \frac1n  \log\|D_{x}f^{n}|E^c(x)\|<0\}\; .\]
%Then a Theorem of Bonatti-Viana \cite{BV00} states that there are finitely many SRB probability measures and the union of their basins covers Lebesgue almost every $M$. 
%
%One can wonder about the existence of a $C^2$-open set of partially hyperbolic diffeomorphisms for which there exists a measurable splitting $E^c = E^{cs}\oplus E^{cu}$ at Lebesgue a.e. point, so that $E^{cs}$ is asymptotically contracted and $E^{cu}$ asymptotically expanded. To avoid product-like examples, one should ask $Df|E^{cs}$ not to be dominated by  $Df|E^{cu}$, i.e. there are vectors in $E^{cu}$ which are more contracted than some in $E^{cs}$. In \cite{BC14} with Carrasco, we gave one of the very  first examples of an open set of $C^2$-conservative dynamics displaying such a property. It is obtained by coupling the Chirikov standard map with an Anosov diffeomorphism of the torus. 
%This example will be described in section \ref{partialhyp}.
%\bigskip 
%
%
%Also, numerical studies by Lorenz \cite{Lorenz} and  H\'enon \cite{Henon} explored dynamical systems with hyperbolic features that did not fit into the uniformly hyperbolic theory (see fig. \ref{fighenon}).
%%even in topology $C^1$ (Bonati-Diaz/Pugh-Shub). 
%In order to include many examples such as the H\'enon's attractor, the \emph{non-uniform hyperbolic theory} is still under construction. A few examples of such systems exist \cite{Ja81, Re86, BC1, BC2, PY09} as related in section \ref{IntroNUH}. We will present in section \ref{StrongRegularity}, our contribution to Yoccoz' strong regularity program \cite{Y97} on  H\'enon-like endomorphisms \cite{berhen}. The aim of this program is to find a combinatorial definition  of non-uniformly hyperbolic transitive sets (based on developments of Yoccoz' puzzle pieces) and to show their abundances (i.e. their existence for a set of positive Lebesgue measure in some open sets of one-dimensional families of dynamics). Each of  the above attractors displays a (real or complex) Hausdorff dimension close to 1. However, we will see in section \ref{parametersection} that it seems possible to adapt our proof for  surface attractors of higher Hausdorff dimension (perhaps even the one initially conjectured by H\'enon), by using the techniques of \cite{BM13}. Furthermore, this combinatorial definition enables us to prove that Strongly Regular H\'enon-like maps display ergodic properties similar to those of uniformly hyperbolic attractors \cite{berent} (see section \ref{propStrongRegular}): their attractors support a unique measure of maximal entropy which is equi-distributed on the periodic points. Moreover we  answered a question of Carleson by showing a hyperbolic bound on the Lyapunov exponents which is uniform among every invariant measure.
%
%\begin{figure}[h]
%	\centering
%		\includegraphics[width=7cm]{Henonattractor.png}
%	\caption{The Hénon attractor cannot be hyperbolic since its attractor basin is a disk, and there is no continuous line field on a disk.}\label{fighenon}
%\end{figure}
%
%\bigskip 
%
%
%Another obstruction to Smale's dream was discovered by its student Newhouse. For $2\le r\le \infty$, he showed the existence of an open set $U$ of $C^r$-surface diffeomorphisms, so that a topologically generic  $f\in U$ displays infinitely many sinks \cite{Ne74, Ne79}. For such a dynamics $f$, the sinks accumulate on a hyperbolic set, each of them supporting a very different physical probability measure. Following Yoccoz, such a coexistence of infinitely many sinks is ``a lower bound" on the complexity of such dynamics. Even presently, we do not understand a single example of such dynamics (for instance whether Lebesgue almost every point displays a Birkhoff sum which converges). We will recall this phenomenon in  section \ref{Newhouse}. 
%
%%Part \ref{Emergence}
%From the 90's this phenomenon was conjectured to be unprobable/negligible in many conjectures \cite{TLY, Pa00, Pa05, PS95}, some of them in low dimension.   In section \ref{NewhouseTypical}, we study the opposite direction. First we showed  in \cite{BdS15} that this phenomenon has codimension at most $1/2$ in the parameter space for surface diffeomorphisms (a result which does not contradict any of these conjectures). Then we showed that for surface endomorphisms or diffeomorphisms of higher dimension, this phenomenon occurs sufficiently to be  important in the sens of Kolmogorov \cite{BE15, BE152, Be16}.  
%
% This leads us to wonder about how complex a (locally) typical dynamics can be. There are many ways to precise this question. We will investigate two directions: 
%
%(i) \emph{The description of a system by its physical measures}. In section \ref{Ermergencedef}, we will define the concept of Emergence \cite{Be16} for a dynamical system, which roughly speaking, quantifies the complexity to describe a dynamical system by means of physical measures. We will state a conjecture about it.
%
%
%(ii) \emph{The growth of the number of it periodic points}. In section \ref{ArnoldPer}, we relate a complement \cite{Be17} of the works of Marteens-de Melo-Van Strien, Kaloshin, Turaev, Gonchenko-Shilnikov-Turaev, Asaoka-Shinohara-Turaev, which gives a full answer to questions of Smale in 1967, Bowen in 1978 and Arnold in 1989, about the growth of the number of periodic points. In particular we will show that the growth can be arbitrarily fast for any dimension $\ge 2$ and any regularity $\infty \ge r\ge 2$. Furthermore, we showed in  \cite{Be17} that this occurs at every parameter of a generic family (a negative answer to a question of Arnold in 1992). 
%
%The two above results are given by a counterpart of the Bonatti-Diaz Blender for parameter families: the parablender. We will recall its definition in section \ref{parablender}. 
%
%%
%%
%%Then, for another construction displaying infinitely many sinks derivated from \cite{BD99, DNP}, we showed that the co-existence of infinitely many sinks appears at every parameter of locally generic families of smooth dynamics. This showed that coexistence of infinitely many sinks is not at all negligible in the sens of Kolmogorov.  This leads us to conjecture the existence of an open set of smooth dynamics in which a typical dynamics displays a super polynomial emergence \cite{Be15Anosov}. 
%%
%%A key tool for this program is the paradynamics: we study the action of a $C^r$-family of diffeomorphisms on $C^r$-jets of families of points. 
%%This constructs a new finite dimensional dynamics, called para-dynamics which fiber over the original dynamics. In important point is that the hyperbolic sets lift canonically to hyperbolic set of the paradynamics. 
%%For the para-dynamics the bifurcations are called para-bifurcation and the blender are called para-blender. To go further in our program on Emergence, we developed this technology, and showed a negative answer to a problem of Arnold (1992). We showed in \cite{} that  the number of periodic points grows super exponentially fast at every parameter of locally generic families of smooth dynamics.  Beyond the techniques of para-dynamics, the construction gives also complete answer to a question of Smale (1957), Bowen (1978) and Arnold (1989) by showing the existence of locally generic set of $C^r$-dynamics with a number of periodic points which grows super exponentially in any dimension $\ge 2$ and any regularity $\infty\ge r\ge 1$, after the  works of Kaloshin, Turaev, ... . 
%
% \bigskip
%
%
%
%Nevertheless, Uniform Hyperbolicity seems to provide a satisfactory way to describe the structurally stable dynamics. This observation goes back to the Fatou conjecture for quadratic maps of the Riemannian sphere in 1920 and the Smale conjecture for smooth diffeomorphisms in 1970. These conjectures have been deeply studied by many mathematicians and so they are difficult to tackle directly. 
%
%However at the interface of one-dimensional complex dynamics and differentiable dynamics, the field of two-dimensional complex dynamics  grew up recently. It enables to study the structural stability problem thanks to ingredients of both  fields. 
%Also the mathematics  developed in the 1970's for the structural stability in dynamics is very similar to the one developed for the structural stability in singularity theory. This led us to combine both in the study of the structurally stable endomorphisms. In next part \ref{partstabstruct}, we will review some works  in these beautiful fields, and we will present our contributions at these interfaces \cite{Be12, BR13, BK13, BD14}, as detailed in the sequel.



\section{Properties of structurally stable dynamics}\label{partstabstruct}
%\section{Properties of structurally stable dynamics} 
\label{secStabimplieshyp}
Let us sate some definitions and conjectures on the structural stability in the $C^r$-category, $1\le r\le \infty$ and in the holomorphic category denoted by $\mathcal H$. 
Let $\mathcal C$ be a category in $\{C^r: 1\le r\le \infty\} \cup \{\mathcal H\}$. 

 \begin{defi}[Structural stability] A $\mathcal C$-map $f$ is \emph{structurally stable}  if every $\mathcal C$-perturbation $f'$ of the dynamics is conjugated: there exists a homeomorphism $h$ of the manifold so that $h\circ f= f'\circ h$. 
\end{defi}
A weaker notion of structural stability focuses on the non-wandering set $\Omega_f$ of the dynamics $f$.
\begin{defi}[$\Omega$-stability] A $\mathcal C$-map $f$ is \emph{$\Omega$-stable}  if for every $\mathcal C$-perturbation $f'$ of $f$, the dynamics of the restriction of $f$ to $\Omega_f$ is conjugated (via a homeomorphism) to the restriction of $f'$ to its non-wandering set $\Omega_{f'}$.
\end{defi}

We recall that an axiom A diffeomorphism $f$ satisfies the \emph{strong transversality condition} if its stable and unstable manifolds intersect transversally. Here is an outstanding conjecture:
\begin{conj}[Palis-Smale  structural stability conjecture, 1970 \cite{PaSm68}]
A $\mathcal C$-diffeomorphism is structurally stable if and only if it satisfies  axiom A and the strong transversality condition. 
\end{conj}
%
%that the direction 
%We will review the different achivements in section \ref{}. We will study how to formulate this condition in the endomorphisms case in sections \ref{}. 

For complex rational maps of the sphere, this conjecture takes the form:
\begin{conj}[Fatou Conjecture, 1920]
Structurally stable quadratic map are those which  satisfy  axiom A and whose critical points are not periodic.
\end{conj}
Actually the initial Fatou conjecture stated the density of axiom A quadratic map. However, in section \ref{secStabimplieshyp},  we will recall the works of Ma\~ n\'e-Sad-Sullivan \cite{MSS} and Lyubich \cite{Ly84} showing the existence of an open and dense set of structurally stable rational maps. This implies the equivalence between the original Fatou Conjecture and the above conjecture.
 Among real quadratic maps, this conjecture\footnote{
The Fatou conjecture is implied by the Mandelbrot Locally connected (MLC) conjecture that we will not have the time to recall in this manuscript.}
 has been proved by Graczyk-Swiantek \cite{GS97} and Lyubich \cite{Ly97}. 
 
%hyperbolicity.
%By the work of \cite{MSS} and \cite{Ly84} explained in the next section, the latter is equivalent to the Celebrated Fatou conjecture (1920), which conjectured the density of axiom A rational maps among the complex quadratic maps.


The description of $\Omega$-stable maps involves the no-cycle condition. We recall that any axiom A diffeomorphisms displays a non-wandering set $\Omega$ equal to a finite union of basic pieces $\Omega= \sqcup_i \Omega_i$.  The family  $(\Omega_i)_i$ is called the \emph{spectral decomposition}.  

 \begin{defi}[No-cycle condition]
An axiom A  diffeomorphism satisfies the \emph{no-cycle condition} if given $\Omega_1,\Omega_2, \dots, \Omega_n$ in the spectral decomposition, if $W^u(\Omega_i)$ intersects $W^s(\Omega_{i+1})$ for every $i<n$ and if 
$W^u(\Omega_n)$ intersects $W^s(\Omega_{1})$, then $\Omega_1=\Omega_2= \cdots = \Omega_n$.
\end{defi}

\begin{conj}[Smale $\Omega$-Stability Conjecture, \cite{Sm68}]
A $\mathcal C$-diffeomorphisms is structurally stable if and only if it satisfies  axiom A and the no-cycle condition.
\end{conj}

If the above conjectures turn out to be true then they would display a satisfactory description of structurally stable dynamics (for the axiom A diffeomorphisms are very well understood).





Let us define the probabilistic structural stability, which is implied by the 
 $\Omega$-stability. The definition involves the regular subset $\mathcal R_f$ of $\Omega_f$. This subset is formed by the points $p\in \Omega_f$ so that for every $a\in \{s,u\}$, there exist $\epsilon>0$ and a sequence of periodic points $(p_n)_n$ satisfying:
 \begin{itemize}
 \item $(p_n)_n$ converges to $p$,
 \item $(W^a_{\epsilon}(p_n))_n$ is relatively compact in the $\mathcal C$-topology.
\end{itemize} 
We showed in \cite{BD14} thanks to Katok's closing Lemma, that the set $\mathcal R_f$ has full measure for every ergodic, hyperbolic probability measure.
\begin{defi} A $\mathcal C$-map $f$ is probabilistically structurally stable  if for every $\mathcal C$-perturbation $f'$ of $f$,  the restriction of $f$ to $\mathcal R_f$ is conjugated to the restriction of $f'$ to its regular set $\mathcal R_{f'}$.
\end{defi} 
It is rather easy to see that probabilistic structural stability implies weak stability:
\begin{defi} A map $f$ is $\mathcal C$-weakly stable  if every $\mathcal C$-perturbation $f'$ of $f$ displays only hyperbolic  periodic points.
\end{defi}

To sum it up, the above definitions are related as follows:
\begin{center}
%Structural Stability $\Rightarrow$ 
$\Omega$-Stability ${\Rightarrow}$
Probabilistic  Stability $\Rightarrow $ Weak Stability
\end{center}

\paragraph{The Lambda Lemma Conjecture.} This conjecture states that weak stability implies $\Omega$-stability. For the category of rational functions of the Riemannian sphere, this Lemma has been shown independently by Ma\~n\'e-Sad-Sullivan \cite{MSS} and Lyubich  \cite{Ly84}. 

As the space of rational functions is finite dimensional, a neighborhood of a rational function $f$ can be written as an analytic family $(f_\lambda)_{\lambda\in \D^n}$, with $\D$ the complex disk and $f_0=f$. If $(f_\lambda)_\lambda$ consists of weakly stable maps, then every periodic point $p_0$ of $f_0$ persists to as unique periodic point $p_\lambda$ for $f_\lambda$. Moreover the map $\lambda\mapsto p_\lambda$ is holomorphic. The Lambda lemma asks the following question. Given $p_0$ in closure $J^*_0$ of the set of periodic points of $f_0$, for every  sequence $(p^n_0)_n$ of periodic points converging to $p_0$, does the family $(\lambda \mapsto p^n_\lambda)_n$ converges? If yes, the\emph{ holomorphic motion is said well defined at $p_0$}.
\begin{lemm}[Lambda-Lemma, Ma\~n\'e-Sad-Sullivan \cite{MSS} and Lyubich  \cite{Ly84}]\label{lambda1}
If $(f_\lambda)_\lambda$ is weakly stable, then the holomorphic motion is well defined at every point $p_0\in J^*_0$.
\end{lemm}




We recall that every rational function $J^*$ is equal to the non-wandering set and that any attracting  periodic point displays a critical point in its basin. Furthermore if a rational function is not weakly stable, it displays a new attracting periodic point after a perturbation of the rational function. Hence the new critical point belongs to the basin of this attracting periodic point.  As the number of critical points is finite, after a finite number of perturbations the rational function turns out to be  weakly stable. This shows that weak stability is open and dense among the rational functions. By the Lambda Lemma \ref{lambda1}, this implies:
\begin{thm}[Ma\~n\'e-Sad-Sulivan  \cite{MSS}, Lyubich \cite{Ly84}]
There is an open and dense subset of rational functions of degree $d\ge2$ which are  $\Omega$-stable.
\end{thm}
This result enables them to deduce a stronger result: the density of the set of structurally stable rational functions.

We recall that a polynomial automorphism of $\C^2$ is a polynomial mapping of $\C^2$ which is invertible and whose inverse is polynomial. Among polynomial automorphisms of $\C^2$, Dujardin and Lyubich \cite{LD13} showed that the holomorphic motion is well defined on any uniformly hyperbolic compact set. We improved this result:
\begin{lemm}[Berger-Dujardin \cite{BD14}]\label{lambdalemC2}
If $(f_\lambda)_\lambda$ is a weakly stable family of polynomial automorphisms of $\C^2$,  the holomorphic motion is uniquely defined on the regular set $\mathcal R_0$ of $f_0$.
\end{lemm}
An immediate consequence of this result is that weak stability implies 
probabilistic stability for the category of polynomial automorphisms of $\C^2$. 

Unfortunately, there is no hope to get the density of $\Omega$-stable polynomial automorphisms of $\C^2$  because  in a non-empty open set \cite{Bu97} of the parameter space is formed by automorphisms displaying a wild horseshoe. However, we will see below that if none pertubations of the dynamics display a homoclinic  tangency, then the dynamics is weakly stable (under a mild hypothesis of dissipativeness). 
 
%
%We notice that this result implies that 

%\marginal{defi of hyperbolic invariant measure}
%In particular, for every $\lambda$,  the holomorphic motion defines a homeomorphisms between a subset $\mathcal R_0 \subset J^*_0$ and $\mathcal R_\lambda \subset J^*_\lambda$ which transport the hyperbolic measures of $f_0$ onto those of $f_\lambda$. The automorphism $f_0$ is called \emph{probabilitiscally stable}.

As a corollary of the techniques, we showed that one connected component of the set of weakly stable polynomial automorphisms is formed by  those which satisfy axiom A.   
%
%when an axiom A polynomial automorphisms  is perturbed to one which is not axiom A, then there exists a periodic point which bifurcate. 


\paragraph{The Ma\~n\'e  Conjecture}
In 1982, Ma\~ne conjectured in \cite{Ma82} that every $C^r$-weakly stable diffeomorphism satisfies axiom A for every $1\le r\le \infty$. He proved this conjecture for $r=1$ and surface diffeomorphisms.  Ma\~ n\'e developed this technology to prove that $C^1$-structurally stable diffeomorphisms satisfy  axiom A and the strong transversality condition in \cite{Ma88}. This work enabled also Palis to prove the same direction for the $C^1$-$\Omega$-stability conjecture \cite{Pa88}.
By developing Ma\~n\'e's works, Aoki and Hayashi proved the Ma\~n\'e conjecture for $r=1$ in any dimension \cite{Ao92,Ha92}.
 
\begin{center}
Weak Stability $\overset{\text{Man\~n\'e\; Conj.}}{\Longrightarrow}$ axiom A. 
\end{center}


 

After the next section, it will be clear for the reader that the Ma\~n\'e Conjecture implies the Lambda Lemma Conjecture in any category $\mathcal C$. 



\paragraph{A Palis Conjecture}
We recall that a hyperbolic periodic point displays a \emph{homocline tangency} if its stable manifold $W^s(p)$ is tangent to its unstable manifold. Two saddle periodic points $p,q$ display a \emph{heterocline tangency} if $W^s(p)$ intersects transversally $W^u(q)$ whereas $W^s(q)$ is tangent to $W^u(p)$ (or vice versa). It is not hard to show that if a $C^r$-map is weakly stable then it cannot display a homoclinic nor a heteroclinic tangency, for every $1\le r\le \infty$. The same is true for one dimensional complex maps. For polynomial automorphisms of $\C^2$, it is a theorem  \cite{Bu97}.



Let us recall also a famous Conjecture of Palis \cite{Pa00} which states that if a dynamics which cannot be perturbed to one which displays a homoclinic nor a heteroclinic tangency, then it satisfies axiom A:
\begin{center}
Weak Stability $\Rightarrow $ Far from tangencies $\overset{\text{Palis\; Conj.}}{\Longrightarrow}$ axiom A. 
\end{center}
In the category of $C^1$-surface diffeomorphisms, this conjecture has been proved by Pujals-Sambarino \cite{PS00}. In the category of $C^1$-diffeomorphisms of higher dimensional manifolds, a weaker version has been proved by Crovisier-Pujals \cite{CP15}.
 
We notice that the Palis conjecture implies the Ma\~n\'e conjecture and so the Lambda lemma conjecture.
 
 \paragraph{A description of structurally stable dynamics as those far from tangencies?}
This question is widely open in the $C^r$-category for $r>1$ (for $C^1$-surface diffeomorphisms it is a consequence of Ma\~n\'e's theorem). It is also correct for the category of rational functions. 
This might be correct for polynomial automorphisms of $\C^2$.  Indeed, most of the work of Dujardin-Lyubich was dedicated to prove the following result:
\begin{thm}[Dujardin-Lyubich \cite{LD13}]
Given a polynomial automorphism $f$ of (dynamical) degree $d\ge 2$ and so that $ |det\, Df_0|\cdot d^2<1$, either $f$ is weakly stable, either a perturbation of $f'$ admits a homoclinic tangency. 
 \end{thm}
From Lambda Lemma \ref{lambdalemC2} we deduced:
\begin{cor}[Berger-Dujardin \cite{LD13}]
Given a polynomial automorphism $f$ of (dynamical) degree $d\ge 2$ and so that $ |det\, Df_0|\cdot d^2<1$, either $f$ is probabilistically stable, either a perturbation of $f'$ displays a homoclinic tangency. 
 \end{cor}
 
Let us stress that this direction might be interesting since numerically we can  see some local stable and unstable manifolds and observe if they display tangencies. 
%\marginal{conjecture de Bonatti?}





%\newpage
%The first conjecture concerns the quadratic polynomial $z\mapsto z^2+c$. 
%
%\begin{conj}[Fatou] For an open and dense set of parameter $c\in \mathbb C$, the map $P_c$ satisfies  axiom A. 
% \end{conj}
%This conjecture would implies that structurally stable quadratic maps satisfies axiom A.  
%This conjecture has been shown for real parameters by Graczyk-Swiatek \cite{GS97} and Lyubich \cite{Ly97}. It is also a consequence of the celebrated MLC conjecture, claiming that the Mandelbrot set is locally connected. 
%
%We will see that structurally stable quadratic maps (and more generally rational functions of degree at least 2) are dense, and that perturbations of axiom A rational functions are structurally stable. 
%
%The second conjecture regard the differentiable diffeomorphisms. 
%  \begin{conj}[Palis-Smale ???]
%  For every $r\ge1$ and every manifold $M$, structurally stable diffeomorphisms satisfies axiom A. 
%  \end{conj}
%We will see that Ma\~ne proved this conjecture in the the case $r=1$.  

\begin{figure}[h!]
	\centering
		\includegraphics[width=11cm]{Diag_Stab_Struct.pdf}
	\caption{Summary of some Theorems and Conjectures on Structural Stability}
\end{figure}



\section{Hyperbolicity implies structural stability}\label{hypimpliesstab}

In the following subsection, we recall the proof ideas of several basic theorems showing the structural stability of subsets from hyperbolic hypotheses. %Hopefully this will help the reader to tackle problems stated in the next subsections, on the description of structurally stable endomorphisms.

\subsection{$\Omega$-stability of maps satisfying axiom A and the no-cycle condition}
%Structural stability of uniformly hyperbolic compact subset and }
\label{hypimpliesstab1}
First let us recall a generalization of the notion of structural stability for invariant subsets. 
\begin{defi}[Structurally stable subset] A compact set $\Lambda$ left invariant by a differentiable map $f$ of a manifold $M$ is \emph{structurally stable} if for every $C^r$-perturbation $f'$ of $f$, there exists a continuous injection $i\colon \Lambda \to M$ so that 
$f'\circ i = i\circ f$.
\end{defi}
We notice that $M$ is structurally stable if and only if $f$ is structurally stable. 
\begin{thm}[Anosov \cite{An67}, proof by Moser \cite{Mo69}]   \label{anosovthm}
A  uniformly hyperbolic compact set $\Lambda$ for a $C^1$-diffeomorphisms is structurally stable.
\end{thm}
\begin{proof} 
We want to solve the following equation:
\begin{equation}\tag{$\star$} f'\circ h\circ f^{-1} =  h \; .
\end{equation}
for $f'$ $C^1$-close to $f$ and $h$ $C^0$-close to the canonical inclusion $i\colon \Lambda\hookrightarrow M$. We shall use the implicit function theorem with the map:
$$\Phi \colon (h, f')\in C^0(\Lambda,M)\times C^1(M,M) \to 
f'\circ h\circ f^{-1}  \in C^0(\Lambda,M)\; .$$
We notice that $\Phi$ is a $C^1$-differentiable map of Banachic manifolds. Moreover it satisfies $\Phi(i,f)=i$.  
Hence to apply the implicit function theorem it suffices to prove that $id- \partial_h  \Phi(i,f)$ is an isomorphism.

Note that  the tangent space of the Banachic manifold $C^0(\Lambda,M)$ at the canonical inclusion $i$ is the following Banach space:
\[\Gamma := \{ \gamma \in C^0(\Lambda, TM): \forall x \in \Lambda\quad \gamma (x) \in T_{x} M\}.\]
The partial derivative of $\partial_h \Phi$ at $(i,f)$ is: 
$$\Psi:=\partial_h \Phi (i, f)\colon \sigma \in \Gamma\mapsto 
Df \circ  \sigma\circ f^{-1} \in \Gamma\; .$$
To compute the inverse of $id-\Psi$, we split $\Gamma $ into two $\Psi$-invariant subspaces $\Gamma = \Gamma^{u}\oplus\Gamma^{s}$, with:
\[\Gamma^{u}:= \{ \gamma \in C^0(\Lambda, TM): \forall x \in \Lambda\quad \gamma (x) \in E^{u}_x\}\quad \text{and}\quad \Gamma^{s}:= \{ \gamma \in C^0(\Lambda, TM): \forall x \in \Lambda\quad \gamma (x) \in E^{s}_x\}.\]

As the norm of $\Psi|\Gamma^{s}$ is less than $1$,  the map $(id-\Psi)|\Gamma^{s}$ is invertible with inverse equal to  $$\sum_{n\ge 0} (\Psi|\Gamma^{s})^n\; .$$ 

As $\Psi|\Gamma^{u}$ is invertible with contracting inverse, the map $(id-\Psi)|\Gamma^{u}$ is invertible with inverse: 
$$ -(\Psi|\Gamma^{u})\circ  (id-(\Psi|\Gamma^{u})^{-1})= - (\sum_{n\ge 1} (\Psi|\Gamma^{u})^{-n})\; .$$

Hence by the implicit function theorem, for every $f'$ $C^1$-close to $f$, there exists a continuous map $h$ $C^0$-close to $i$ which semi-conjugates the dynamics:
$$ f'\circ h = h \circ f \; .$$

As $i$ is injective and close to $h$, if $h(x)= h(y)$ then $x$ and $y$ are close. Also by semi-conjugacy, $h \circ f^n(x)= h\circ f^n(y)$ for every $n\in \Z$. Hence $f^n(x)$ is close to $f^n(y)$ for every $n$. 
By expansiveness (see below), we conclude that $x=y$ and so that $h$ is injective.
\end{proof}
%\begin{defi}A compact subset $\Lambda$ left invariant by a diffeomorphism $f$ is \emph{expansive} if there exists $\epsilon>0$ so that if two orbits $(x_n)_{n\in \Z}$ and $(y_n)_{n\in \Z}$ are uniformly $\epsilon$-close, then the two orbits are the same. 
%\end{defi}
\begin{lemm}[Expansiveness]\label{expansiveness}
Every hyperbolic compact set $\Lambda$ for a diffeomorphism is \emph{expansive}: there exists $\epsilon>0$ so that if two orbits $(x_n)_{n\in \Z}$ and $(y_n)_{n\in \Z}$ are uniformly $\epsilon$-close, then $x_0=y_0$.
%for a diffeomorphism is expansive.
\end{lemm}
\begin{proof}
First we notice that for $\epsilon$ small enough, given two such orbits,  $W^s_{2\epsilon} (y_n)$ intersects $W^u_{2\epsilon} (y_n)$  at a unique point $z_n$. We observe that $(z_n)_n$ is an orbit. As $f$ is expanding along $W^u_{2\epsilon} (y_n)$ for every $n$ and since $z_n\in W^u_{2\epsilon} (y_n)$, it comes that $z_n=y_n$ for every $n\ge 0$. Using the same argument for $f^{-1}$, it comes that $z_n=x_n$  for every $n\le 0$ and so $x_0=y_0$.
\end{proof}


The image $\Lambda(f'):= h(\Lambda)$ is called the \emph{hyperbolic continuation} of $\Lambda$. Since the density of periodic points is preserved by conjugacy, it comes: 
\begin{cor}
If $\Lambda$ is a basic piece, then its hyperbolic continuation is also a basic piece.
\end{cor}
A similar result has been proved by Shub during his thesis:
\begin{thm}[Shub \cite{Shub69}]
An expanding compact set $\Lambda$ for an endomorphisms $f$ is $C^1$-structurally stable.
\end{thm}
\begin{proof} First let us notice that $f$ is a local diffeomorphism at a neighborhood of the compact set $\Lambda$. Hence there exists $\epsilon>0$ so that for every $f'$ $C^1$-close to $f$, for every $x\in \Lambda$, the restriction $f'|B(x,\epsilon)$ is invertible. This enables us to look for a semi-conjugacy thanks to the map:
%We want to solve the following equation:
%\begin{equation}\tag{$\star$} f'^{-1}\circ h\circ f =  h 
%\end{equation}
%for $f'$ $C^1$-close to $f$ and $h$ $C^0$-close to the canonical inclusion $i\colon \Lambda\hookrightarrow M$. 
%The map 
$$\Phi \colon (h, f')\in C^0(\Lambda,M)\times C^1(M,M) \to 
(f'|B(x,\epsilon))^{-1}\circ h\circ f  \in C^0(\Lambda,M)$$
The latter is well defined and of class $C^1$ on the $\epsilon$-neighborhood of the pair of the canonical inclusion $i\colon \Lambda\hookrightarrow M$ with $f$. Furthermore, it holds $\Phi(i,f)= i$ and the following partial derivative is contracting, with $\Gamma$ the tangent space of $C^0(\Lambda,M)$ at $i$.
 $$\partial_h \Phi (i, f) \colon  \sigma \in \Gamma \to 
Df^{-1}\circ \sigma \circ f  \in \Gamma\; .$$
Thus, by the implicit function Theorem, for $f'$ $C^1$-close to $f$, 
there exists a unique solution with $h\in C^0(\Lambda,M)$ close to $i$ for the semi-conjugacy equation:
\[\Phi(h,f')=h\Leftrightarrow h\circ f= f'\circ h\;.\]


As $h$ is close to the canonical inclusion, if $h(x)=h(x')$ then $x$ and $x'$ must be close.  Also by semi-conjugacy, it holds $h(f^n(x))=h(f^n(x'))$ for every $n\ge 0$. Thus the orbits $(f^n(x))_{n\ge 0}$ 
and $(f^n(x'))_{n\ge 0}$ are uniformly close. By forward expansiveness  (see below), it comes that $x=x'$.
\end{proof}
%\begin{defi} A compact subset $\Lambda$ stable by an endomorphism $f$ is \emph{forward expansive} if there exists $\epsilon>0$ so that if two orbits $(x_n)_{n\ge 0}$ and $(y_n)_{n\ge 0}$ are uniformly $\epsilon$-close, then the two orbits are the same. 
%\end{defi}
One easily shows by a similar argument to Lemma \ref{expansiveness}:
\begin{lemm}[Forward expansiveness]
Every expanding compact set $\Lambda$ is \emph{forward expansive}:
there exists $\epsilon>0$ so that if two orbits $(x_n)_{n\ge 0}$ and $(y_n)_{n\ge 0}$ are uniformly $\epsilon$-close, then $x_0=y_0$.
\end{lemm}

The two latter theorems enable us to explain the proofs of  Smale and Przytycki on {$\Omega$-stability}. We recall that the local stable and unstable manifolds of the points of a hyperbolic set $\Lambda$ for an endomorphism $f$ (which might display a non-empty critical set) are uniquely defined, provided that:
\begin{itemize}
\item Either $f|\Lambda$ is bijective,
\item Either  $\Lambda$ is injective.
\end{itemize}
On the other hand, the local stable manifold are always uniquely defined. Hence under these assumption, by looking at their images or preimages, the following is uniquely defined for $\epsilon>0$ small enough:
\[ W^s_\epsilon (\Lambda) = \cup_{x\in \Lambda} W^s_\epsilon (x)\quad 
W^u_\epsilon (\Lambda) = \cup_{x\in \Lambda} W^u_\epsilon (x)
\quad 
W^s (\Lambda) = \cup_{n\ge 0} f^{-n} (W^s_\epsilon (\Lambda))
\; .\]
The following generalizes Smale's definion of axiom A diffeomorphisms:
\begin{defi}[Axiom A in the sens of Przytycki]
A $C^1$-endomorphism satisfies \emph{axiom A-Prz}, if  its non-wandering set $\Omega$ is equal to the closure of the set of periodic points (or equivalently locally maximal), and if it is the disjoint union of an expanding compact set with a bijective, hyperbolic compact set. 
\end{defi}
For such maps we can generalize the notion of \emph{spectral decomposition}. Indeed 
by local maximality and compactness, the non-wandering set $\Omega$ of such maps is the finite union of (maximal) transitive subsets $\Omega_i$ called \emph{basic pieces}:
\[\Omega= \sqcup_i \Omega_i\; .\]
The family  $(\Omega_i)_i$ is called the \emph{spectral decomposition} of the axiom A-Prz endomorphism. Let us generalize the  no-cycle condition for such endomorphisms. 
 \begin{defi}[No-cycle condition]
An axiom A-Prz,  $C^1$-endomorphism satisfies the \emph{no-cycle condition} if given $\Omega_1,\Omega_2, \dots, \Omega_n$ in the spectral decomposition, if $W^u_\epsilon(\Omega_i)$ intersects $W^s(\Omega_{i+1})$ for every $i<n$ and if 
$W^u_\epsilon(\Omega_n)$ intersects $W^s(\Omega_{1})$, then $\Omega_1=\Omega_2= \cdots = \Omega_n$.
\end{defi}
F. Przytycki  generalized Smale's Theorem on the $\Omega$-stability of axiom A diffeomorphisms which satisfy the no-cycle condition as follows:
% The following result ge
\begin{thm}[\cite{Sm68}, \cite{Pr77}]
If a $C^1$-endomorphism satifies axiom A-Prz and the no-cycle condition, then it is $C^1-\Omega$-stable.
\end{thm}
\begin{proof}[Sketch of proof of the Smale's $\Omega$-stability Theorem]
First let us recall that by Anosov Theorem, the non-wandering set $\Omega$ is structurally stable, and its hyperbolic continuation is still locally maximal (for a neighborhood uniformly large among an open set of perturbations of the dynamics).

Then the no-cycle condition is useful to construct a filtration $(M_i)_i$:
\begin{prop}
If an axiom A, $C^1$-diffeomorphism $f$ satisfies the no-cycle condition, then there exists a chain of open subsets:
\[\emptyset = M_0 \subset M_1\subset \cdots \subset M_N=M\]
 so that 
  $f(M_i)\Subset M_i$ and $\Omega_i \Subset M_{i}\setminus M_{i-1}$ for every $i\ge 1$.
 \end{prop}
The proof of this proposition involve Conway Theory and can be find 
in \cite[Thm 2.3 p. 9]{shubstab78}.


By using this filtration and the (uniform) local maximality of the hyperbolic continuation of the non-wandering set, one easily deduces the $\Omega$-stability. 
%
%Using the breakthrough of Ma\~ne \cite{Ma88},  Palis \cite{Pa88} proved that the converse is also true: a $C^1$-$\Omega$-stable diffeomorphisms must satisfies  axiom A and the no-cycle condition. 
%
%In the endomorphism context, F. Przytycki proved the following:
\end{proof}


%\begin{thm}[\cite{Sm68}]
%A diffeomorphism which satisfies  axiom A condition and the no-cycle condition is $C^1$-$\Omega$-stable.
%\end{thm}


\subsection{Structural stability of dynamics satisfying axiom A and  the strong transversality condition}\label{hypimpliesstab2} 
%\paragraph{Structural stability of diffeomorphisms which satisfy axiom A and the strong transversality condition}
\paragraph{Structural stability of diffeomorphisms}
We recall that an axiom A diffeomorphism satisfies the {strong transversality condition} if for any  non-wandering points $x$ and $y$, the stable manifold of $x$ is transverse to the unstable manifold of $y$.


%pointsthe stable and :
%\begin{itemize}
%\item  the non-wandering set $\Omega$ is hyperbolic,
%\item  $cl(Per(f))=\Omega$,
%\item $\forall (x,y)\in \Omega^2$, $W^s(x)\pitchfork W^u(y)$.\end{itemize}
%\end{defi}
\begin{rema} By using the inclination lemma, one easily shows that the strong transversality condition implies the no-cycle condition.
\end{rema}

 
The following theorem generalizes Anosov Theorems \ref{anosovthm}:

\begin{thm}[Robbin \cite{Ro71}, Robinson \cite{Ro76}] \label{SSTRob}
For every $r\ge 1$, the diffeomorphisms which satisfy axiom A and the strong transversality condition are $C^r$-structurally stable.\end{thm}
Let us recall that the Ma\~ne theorem \cite{Ma88} implies that a $C^1$-structurally stable diffeomorphism satisfies also  axiom A and the strong transversality condition, and so both solve the conjecture   of $C^1$-structural stability. 

We will state Conjecture \ref{ConjPrz} generalizing this theorem for local diffeomorphisms. Hopefully the following will help the reader to tackle it. 
\begin{proof}[Sketch of proof of Theorem \ref{SSTRob}]
Again we want to solve the following semi-conjugacy equation:
\begin{equation}\tag{$\star$} f'\circ h\circ f^{-1} =  h 
\end{equation}
for $f'$ $C^1$-close to $f$ and $h$ $C^0$-close to the identity of $M$. 

For  $f'=f$ and $h=id$, Equality ($\star$) is valid. The set of perturbations of the identity is isomorphic to  $\Gamma = \{ \gamma \in C^0(M, TM): \forall x \in M\quad \gamma (x) \in T_{x} M\}$ by using the exponential map (associated to a Riemannian metric on $M$). Let 
$\tilde f := u\in T_xM \mapsto \exp_{f(x)}^{-1}\circ  f \circ \exp_x(u)$.

Then Equation $(\star)$ is equivalent to:
\begin{equation}\tag{$\star\star$} \tilde f'\circ \sigma\circ f^{-1} = \sigma, \quad \text{for }\sigma \in \Gamma \quad C^0\text{-small.}
\end{equation}


As the map 
$\Phi \colon (\sigma, f')\in \Gamma \times C^1(M,M) \to 
\Phi_{f'}(\sigma)=\sigma -\tilde f'\circ \sigma \circ f^{-1}  \in \Gamma $ is of class $C^1$, 
and vanishes at $(0,f)$,  
we shall show that $\partial_h  \Phi$ is left-invertible. 

Let  
$$\Psi:=\partial_h  \Phi(0,f)\colon \sigma \in \Gamma\mapsto \sigma-
Df \circ  \sigma\circ f^{-1}   \in \Gamma \; .$$


%Let $\emptyset := M_0 \Subset M_1 \cdots \Subset M_N= M$ be a filtration adapted to the spectral decomposition  $(\Omega_i)_i$ of $\Omega$.

The following is shown in \cite{Ro71}:
\begin{prop}\label{sectionEi}
For every $i$, there exists a neighborhood $N_i$ of $\Omega_i$ and  continuous extension $E^s_i$ and $E^u_i$ of respectively $E^s|\Omega_i$ and $E^u|\Omega_i$ to $N_i$, so that:
\begin{itemize}
\item There exists a filtration $(M_i)_i$ adapted to $(\Omega_i)_i$ so that $N_i= M_i\setminus M_{i-1}$. The subsets  $(N_i)_i$ form an open covering of $M$,
\item if $x\in N_i\cap f^{-1}(N_j)$, with $j\le i$, then $Df(E^s_i(x))\subset E^s_j(f(x))$, and $Df(E^u_i(x))\supset E^u_j(f(x))$.
\end{itemize}
\end{prop}

Let $(\gamma_i)_i$ be a partition of the unity adapted to $(N_i)_i$.

For every $i$ let $p^s_i$ and $p^u_i$ be the projections onto respectively $E^s_i$ and $E^u_i$ parallely to  $E^u_i$ and $E^s_i$. 

Given $x\in M$ and $v\in T_x M$, we put 
$v_i^s:= \gamma_i \cdot p^s_i(v)$ and   $v_i^u:= \gamma_i \cdot p^u_i(v)$. 
We observe that $v= \sum_i v_i^s+v_i^u$. Thus
$Df(v)= \sum_i Df(v_i^s)+Df(v_i^u)$. 
As $(Df^n(v_i^s))_{n\ge 0}$ and $(Df^{-n}(v_i^u))_{n\ge 1}$ converge exponentially fast to $0$, we consider:

\[J\colon \sigma\in \Gamma \mapsto \sum_i \sum_{n\ge 0} Df^n(\sigma_i^s\circ f^{-n}(x))
-\sum_{n\ge 1} Df^{-n}(\sigma_i^u\circ f^{n}(x))\; .\]

We notice that $J$ is a left inverse of $\Psi$ :
$$ J\circ \Psi = id$$.

The following equations are equivalent:
$$ \Phi_{f'}(\sigma)=0\Leftrightarrow (\Phi_{f'}-\Psi)(\sigma)+\Psi(\sigma)=0,$$
$$ \Leftrightarrow J\circ (\Phi_{f'}-\Psi)(\sigma)+J\circ \Psi(\sigma)=J(0).$$
Now observe that $J(0)=0$ and $J\circ \Psi(\sigma)=\sigma$. Hence  $(\star\star)$ is equivalent to 
$$J\circ (\Psi-\Phi_{f'})(\sigma)=\sigma.$$

It is easy to see that whenever $f'$ is $C^1$-close to $f$, the map $\Phi_{f'}$ is $C^1$-close to $\Psi$ at a neighborhood of the $0$-section. Hence the map $J\circ (\Psi-\Phi_{f'})$ is contracting and sends a closed ball about the zero section into itself. The contracting mapping theorem implies the existence of a fixed point $\sigma$.
Hence $(\star)$ displays a solution $h=\exp\circ \sigma$ close to the identity in the space of continuous maps.

It remains to show that the semi-conjugacy $h$ is bijective. Contrarily to Anosov maps, in general  axiom A diffeomorphisms are not expansive and the semi-conjugacy is not uniquely defined. Hence Robbin brought  a new technique to construct a map $h$ which is bijective. He defined the following metric:
$$ d_f(x,y)= \sup_{n\in \mathbb Z} d(f^n(x),f^n(y))\;,$$
where $d$ is the Riemannian metric of the manifold $M$.

%We will explain below what is the geometric meaning of this metric. 

Let us just notice that if the semi-conjugacy  $h=\exp\circ \sigma$ satisfies that $\sigma$ is $C^0$-small and $d_f$-Lipschitz with a small constant $\eta$, then $h$ is injective.

Indeed if $h(x)=h(y)$, then by $(\star)$, $h(f^n(x))= h(f^n(y))$ for every $n$. Since $h$ is close to the identity, the orbits $(f^n(x))_n$ and $(f^n(y))_n$ are uniformely close, and so that $d_f(x,y)$ is small. As $\sigma$ is $\eta$-Lipschitz, it comes:
\[0= d(h(x),h(y))\ge d(x,y)-\eta d_f(x,y)\]    
The same holds at any $n^{th}$-iterate:
\[0=d(h(f^n(x)),h(f^n(y)))\ge d(f^n(x),f^n(y))-\eta d_f(f^n(x),f^n(y))=d(f^n(x),f^n(y))-\eta d_f(x,y)\;.\]
Let $n$ be such that $d(f^n(x),f^n(y))\ge  d_f(x,y)/2$. Then 
\[0=d(h(f^n(x)),h(f^n(y)))\ge (1-2\eta) d(f^n(x),f^n(y))\;.\]
Thus $f^n(x)=f^n(y)$ and so $x=y$.

To obtain the section $\sigma$ $d_f$-Lipschitz, Robbin assumed the diffeomorphism $f$ of class $C^2$. Then in Proposition \ref{sectionEi}, he constructs the section $(E^s_i)_i$ and  $(E^u_i)_i$ $d_f$-Lipschitz, so that the map $J$ preserves the  $d_f$-Lipschitz sections. On the other hand the map $\Psi-\Psi_{f'}$ diminishes the $d_f$-Lipschitz constant for $f'$ $C^1$-close to $f$. Therefore the map $J\circ (\Psi-\Psi_{f'})$ preserves the space of continuous sections with small $d_f$-Lipschitz constant, and so its fixed point enjoys a small $d_f$-Lipschitz constant.  

The $C^1$-case was handled by Robinson. His trick was to smooth the map $Df$ to a $C^1$-map $\tilde Df$, and to replace $Df$ by $\tilde Df$ in the definition of $\Psi$ to define $\tilde \Psi$. Then he defined  likewise $\tilde Df$-pseudo invariant sections $(\tilde E^s_i)_i$ which are $d_f$-Lipschitz. By replacing $(E^s_i)_i$ by $(\tilde E^s_i)_i$ in the definition of $J$, he defined a left inverse $\tilde J$ of $\tilde \Psi$. Then he showed likewise that the map $\tilde J\circ (\tilde \Psi-\Psi_{f'})$ admits a $C^0$-small, $d_f$-Lipschitz fixed point, which is a solution of $(\star\star)$. 
\end{proof}

\paragraph{Structural stability of covering.}
We recall that every local diffeomorphism of a compact (connected) manifold is a covering. 

F. Przytycki \cite{Pr77} introduced an example of surface covering suggesting the following \emph{strong transversality condition}.
 \begin{defi}
A covering map $f$ satisfies {\emph axiom A and the strong transversality condition} if:
\begin{enumerate}[(i)]
\item The non-wandering set is locally maximal.
\item  The non-wandering set $\Omega$ is the union of a hyperbolic set on which $f$ acts bijectively with a repulsive set.
\item $\forall x\in \Omega, \; \underline y^1,\dots,\underline  y^k)\in \overleftarrow \Omega$, the following multi-transversality condition holds:
$$W^s(x)\pitchfork W^u(\underline  y^1)\pitchfork \cdots \pitchfork W^u(\underline  y^k)\; .$$
\end{enumerate}
\end{defi}
We recall that a finite family of submanifolds $(N_i)_i$ is multi-transverse if $N_1$ and $N_2$ are transverse, $N_3$ is transverse to $N_1\cap N_2$, ..., and for every $i \ge 3$, $N_{i}$ is transverse to $N_1\cap N_2\cap \cdots \cap N_{i-1}$. We notice that $(iii)$ implies $(ii)$.

Here is a generalization of a conjecture of  Przytycki \cite{Pr77}:

\begin{conj}\label{ConjPrz}
The $C^1$-struturally stable coverings are those which satisfy  axiom A and the strong transversality condition.
\end{conj} 
The fact that structurally stable coverings are axiom A has been proved by Aoki-Moriyasu-Sumi \cite{AMS01}, and the strong transversality condition has been proved by Iglesias-Portela-Rovella. The other direction is still open in the general case.

This conjecture has been proved in two special cases. The first one solves the initial Przytycki conjecture for surface coverings:
\begin{thm}[Iglesias-Portela-Rovella \cite{IPR12}] If a covering map of a surface satisfies  axiom A and the strong transversality condition then it is $C^1$-structurally stable. 
\end{thm}
The other case is for attractor-repellor covering. 

\begin{thm}[Iglesias-Portela-Rovella \cite{IPR10}] Let $M$
be a compact manifold. If
$f$ is a $C^1$- covering map satisfying  axiom A, and so that its basic pieces are either bijective attractors or expanding sets, then $f$ is $C^1$-structurally stable. \end{thm}
The strong transversality condition for these maps is certainly satisfied since, the unstable manifolds are either included in the attractor or form open subset of the manifold. They gave the following example:
\[f\colon (z,z')\in \mathbb S^1\times \hat \C\mapsto( z^2, z/2+ z'/3),\]
where the non-wandering set consists of an expanding circle and of the Smale solenoid. 

In \cite{BK13}, we constructed $d_f$-Lipschitz plane fields for endomorphisms which satisfies  axiom A and the strong transversality condition. This might be useful to prove that under the hypothesis of Conjecture \ref{ConjPrz}, the following map has a left inverse:
\[\sigma\in \Gamma^0(TM) \mapsto \sigma - Df^{-1}\circ \sigma\circ f \in \Gamma^0(TM)\; .\]
%in order to prove this conjecture \footnote{The argument of Iglesias-Portela-Rovella  uses tubular neighborhood of basic pieces, whose existence is not clear for a piece which is  not an attractor nor of codimension 1.}



\subsection{Structural stability of the inverse limit}\label{hypimpliesstab3}



Given an endomorphism $f$ of a compact manifold $M$, the inverse limit $\overleftarrow M_f$ of $f$ is the space of orbits :
\[\overleftarrow M_f:= \big\{\underline x= (x_n)_{n\in \mathbb Z} \colon x_{n+1} = f(x_n)\big\}\;.\]
It is a closed subset of $M^\mathbb Z$, which is compact endowed with the product metric:
\[d(\underline x,\underline x')= \sum_{n\in \mathbb Z} 2^{-|n|} d(x_n,x'_n)\;.\]
We notice that the inverse limit is homemorphic to $M$ when $f$ is a homeomorphism of $M$. 

We notice also that the shift dynamics $\overleftarrow f$ acts canonically on $\overleftarrow M_f$:
$$\overleftarrow f\colon (x_n)_n \mapsto (x_{n+1})_{n}. $$
With $\pi_0\colon (x_n)_n \mapsto x_0$ the zero coordinate projection, it holds:
$$\pi_0 \circ \overleftarrow f = f\circ \pi_0.$$

From this one easily deduces that the non-wandering sets  $\arr \Omega_{f}$ and $\Omega_{f}$ of respectively $\arr f$ and $f$ satisfies the following relation:
\[ \arr \Omega_{f} = \Omega_{f}^\Z\cap \arr M_f\; .\]

\begin{defi}
The endomorphism $f$ is $C^r$-inverse limit stable if for every $C^r$-perturbation $f'$ of $f$, there exists a homeomorphism $h$ from $\overleftarrow M_f$ onto $\overleftarrow M_{f'}$ so that:
\[h\circ \overleftarrow f = \overleftarrow f'\circ h.\]
\end{defi}

We can define the unstable manifold of every point $\underline x=(x_i)_i\in \overleftarrow \Omega_f$:
\[W^u(\underline x; \overleftarrow f):=\{\underline y=(y_i)_i\in \overleftarrow M_f\colon d(x_i,y_i)\to 0,\; i\to -\infty\}\]
When $f$ satisfies axiom A, it is an actual manifold embedded in $\overleftarrow M_f$. Moreover, the $0$-coordinate projection $\pi_0$ displays a differentiable restriction $\pi_0| W^u(\underline x; \overleftarrow f)$.

On the other hand, there exists $\epsilon>0$ so that the following local stable manifold is an embedded submanifold of $M$, for every  $x\in \Omega_f$:
  \[W^s_\epsilon( x;  f):=\{y\in M\colon d(f^n(x),f^n(y))\to 0,\; n\to +\infty\}\; .\]

In \cite{BR13}, we notice that surprisingly, for certain axiom A endomorphisms, the presence of critical set (made by points with non surjective differential) does not interfere with the $C^1$-inverse structural stability.  This leads us to define:
\begin{defi} An axiom A endomorphism $f$ satisfies the weak transversality condition if for every $\underline x\in \overleftarrow \Omega_f$ and every $y\in \Omega_f$, the map 
$\pi_0|W^u(\underline x; \overleftarrow f)$ is transverse to  $W^s_\epsilon(y)$.\end{defi}

There are many examples of endomorphisms which satisfy axiom A and the weak transversality condition. For instance:
\begin{itemize}
\item any axiom A map of the one point compactification $\hat \R$ of $\R$, in particular those of the form $x\mapsto x^2+c$ and even the constant map $x\mapsto 0$.
\item if $f_1$ and $f_2$ satisfy axiom A and the weak transversality condition, then the product dynamics $(f_1,f_2)$ do so.
\item By the two latter points, note that the map $(x,y,z)\mapsto (x^2,y^2,0)$ of $\R^3$ satisfies axiom A and the weak transversality condition.
\end{itemize}
The latter map is not at all structurally stable, for its critical set is not and intersects moreover the non-wandering set. For this reason the following conjecture might sound irrealistic:
\begin{conj}[Berger-Rovella \cite{BR13}]
The $C^1$-inverse limit stable endomorphisms are  those which satisfy axiom A and the weak transversality condition.
\end{conj}
However  in \cite{BR13}, we gave many evidences of veracity of this conjecture. Then in \cite{BK13} we showed one direction of this conjecture ; the other direction is still open.
\begin{thm}[Berger-Kocksard \cite{BK13}]\label{BK}
If a $C^1$-endomorphisms of a compact manifold satisfies  axiom A and the weak transversality condition, then it is inverse limit stable. 
\end{thm}
The proof of this theorem follows the strategy of the Robbin structural stability theorem. The main difficulty is the construction of pseudo-invariant plan fields $(E_i^s)_i$  and $(E_i^u)_i$, for the endomorphisms display in general a non-empty critical set.  

\begin{figure}[h!]
	\centering
		\includegraphics[width=9cm]{EideltaS.pdf}
	\caption{Construction of $(E_i^s)_i$ for the map $(x,y,z)\mapsto (x^2,y^2,0)$ }
\end{figure}
\begin{figure}[h!]
	\centering
		\includegraphics[width=9cm]{EideltaU.pdf}
	\caption{Construction of $(E_i^u)_i$ for the map $(x,y,z)\mapsto (x^2,y^2,0)$}
\end{figure}

\section{Links between structural stability in dynamical systems and  singularity theory}\label{sectionLinks}
In the last section we saw how the inverse stability does not seem to  involve any singularity theory. 
However let us notice that if a $C^\infty$-endomorphism of a manifold $M$ is structurally stable (that  is conjugated to its perturbation via a homeomorphism of $M$), then its singularities are \emph{$C^0$-equivalently, structurally stable}:
 \begin{defi}
 %[$C^r$-\emph{equivalently, structurally stable}] 
 Let $f$ be a $C^\infty$-map from a manifold $M$ into a possibly different manifold $N$ and $r\in \{0,\infty\}$. The map $f$ is $C^r$-\emph{equivalently, structurally stable} if for every $f'$ $C^\infty$-close to $f$, there are  $h\in Diff^r(M)$ and $h'\in Diff^r(N)$ which are $C^r$-close to the identity and such that the following diagram commutes:
\[\begin{array}{lcccr} 
&&f'&&\\
 &M&\rightarrow &N&\\
 h&\uparrow &&\uparrow&h'\\
 &M&\rightarrow&N&\\
 &&f&&\end{array}.\]
\end{defi}

The equivalently, structural stability has been deeply studied, in particular by Whitney, Thom and Mather. We shall recall some of the main results, by emphasizing their similarities with those of structural stability in dynamical systems. 

 \subsection{Infinitesimal stability}
Let $M, N$ be compact manifolds.  For $r\in \{0,\infty\}$, let $\chi^r(M)$ and $\chi^r(N)$ be the space of $C^r$-sections of respectively $TM$ and $TN$.

 \begin{defi}
A Diffeomorphism  $f\in Diff^1(M)$ is $C^0$-\emph{infinitesimally stable} if the following map is surjective:
\[\sigma\in \chi^0(M)\mapsto Tf\circ \sigma- \sigma\circ f\in \chi^0(f),\]
with $\chi^0(f)$ the space of continuous sections of the pull back bundle  $f^*TM$. 
\end{defi} 
  In the Robbin-Robinson proofs of structural stability (Theorem \ref{SSTRob}), we saw the importance of the left-invertibility of $\sigma\mapsto Tf\circ \sigma- \sigma\circ f$. The latter implies the  $C^0$-{infinitesimal stability} which is equivalent to the $C^1$-structural stability:
\begin{thm}[Robin-Robinson-Ma\~ne \cite{Ro71},\cite{Ro76}, \cite{Ma88}]
The $C^0$-{infinitesimally stable} diffeomorphisms are the $C^1$-equivalently stable maps.
\end{thm}

A similar definition exists in Singularity Theory:
\begin{defi}
Let $f\in C^\infty(M,N)$ is $C^\infty$-\emph{ equivalently infinitesimally stable} if 
 the following map is surjective:
\[(\sigma, \xi)\in \chi^\infty(M)\times \chi^\infty(N)\mapsto Tf\circ \sigma- \xi\circ f\in \chi^\infty(f)\]
with $\chi^\infty(f)$ the space of $C^\infty$-sections of the pull back bundle  $f^*TM$. 
\end{defi}
It turns out to be equivalent to the $C^\infty$-equivalent stability.
\begin{thm}[Mather \cite{Ma170,Ma270,Ma370,Ma470,Ma570}]
The $C^\infty$-{ infinitesimally equivalently stable} maps are the $C^\infty$-equivalently stable maps.
\end{thm}
The latter might sound complicated to verify, but on concrete examples it is rather easy to check. That is why following Mather, it is a satisfactory description of $C^\infty$-equivalently structurally stable maps. 

 \subsection{Density structurally stable maps}

Let us point out two similar results on structural stability:
\begin{thm} [Thom, Mather \cite{Ma73, Ma76, GWPL}]\label{C0ESS}
For every manifolds $M,N$, the $C^0$-equivalently structural stable maps form an open and dense set in $C^\infty(M,N)$.
\end{thm}
Let us recall:
\begin{thm}[Ma\~ne-Sad-Sullivan \cite{MSS}, Lyubich \cite{Ly84}]
For every $d\ge 2$, the set of structurally stable rational functions is open and dense. 
\end{thm}
In both cases, we do not know  how to describe these structurally stable maps. 

Still the axiom A condition is a candidate to describe the structurally stable rational functions, since the famous Fatou conjecture (1920).  
On the other hand, there is not even a conjecture for the description of the  $C^0$-equivalently structural stable maps.
 
Following Mather, a nice way to describe the equivalently structural stable maps would be (a similar way to) the $C^\infty$-equivalently infinitesimal stability.

 Nevertheless,  Mather proved that $C^\infty$-equivalently infinitesimal stable maps  are  dense if and only if the dimensions of $M$ and $N$ are not ``nice" \cite{Ma670}. We define the nice dimensions below. Thus one has to imagine a new criteria (at least of in ``not nice" dimensions) to describe the $C^0$-equivalently structural stable maps. 

\begin{defi}[Nice dimensions]
If $m=dim \, M$ and $n= dim\, N$, the pair of dimensions $(m;n)$ is nice if and only if one of the following conditions holds:
\[\begin{array}{rcl}
n-m \ge  4 & \text{and}&  m<\frac 67 n+\frac 8 7,\\
3\ge n- m \ge  0 & \text{and}& m< \frac 6 7 n +\frac 9 7,\\
 n- m =-1& \text{and}& n <8,\\
 n- m =-2& \text{and}& n <6,\\
 n- m =-3& \text{and}& n <7.\end{array}\]
\end{defi}
We notice that if  $n:=dim\,M=dim\,N$, then the pair of dimensions $(m;n)$ is nice if and only if $n\le 8$.

Let us finally recall an open question: 
\begin{prob}
In nice dimensions,  does a 
$C^0$-equivalently structurally stable map is always $C^\infty$-equivalently structurally stable map? 
\end{prob}
\subsection{Geometries of the structural stability}
The proof of the Thom-Mather Theorem \ref{C0ESS} on the density of $C^0$-equivalently structurally stable involves the concept of stratification (by analytic or smooth submanifolds). 

Similarly, the set of stable and unstable manifolds of an axiom A diffeomorphisms form a stratification of laminations, as defined in \cite{BeMem}.  

Let us recall these definitions.

\subsubsection{Stratifications}

A \emph{ stratification} is the pair of a locally compact subset $A$ and a locally finite partition $\Sigma$ by locally compact subsets $X\subset A$, called {\emph strata}, and satisfying: 
  \[\forall (X,Y)\in \Sigma^2,\; cl(X)\cap Y\not=\emptyset\Rightarrow cl(X)\supset Y\; .\]
\[\mathrm{We\; write \; then \;} X\ge Y\; .\]

In practical, the set $A$ will be embedded into a manifold $M$, and the strata $X$ will be endowed with a structure of analytic manifold, differentiable manifold or even lamination, depending on the context. 

\subsection{Whitney Stratification}
The first use of stratification goes back to the work of Whitney to describe the algebraic  varieties. Then it has been generalized by Thom 
and Lojasiewicz for the study of analytic variety and even semi-analytic variety. 

\begin{defi}
\emph{An analytic variety} of $\R^n$  is  the zero set of an analytic function on an open subset of $\R^n$. \emph{An analytic submanifold} is a submanifold which is also an analytic variety. \emph{A semi-analytic variety} is a subset $A$ of  $\R^n$ which is covered by open subset $U$ satisfying:
\[A\cap U =  \cap_{i=1}^N\cup_{i=1}^N F_{ij}\]
with $F_{ij}$ of the form $\{q_{ij} >0\}$ or $\{q_{ij} =0\}$ and $q_{ij}$ a real analytic function on $U$. 
\end{defi}

\begin{thm}[Whitney-Lojasiewicz \cite{Lo70}]
Any semi-analytic variety $S\subset R^n$ splits into a stratification $\Sigma$ by analytic manifolds.
% Moreover this stratification is $b$-regular.
\end{thm}
  One important property of the semi-analytic category is its stability by projection from the Seidenberg Theorem: given any projection $p$ of $\R^n\to \R^p$, the image by $p$ of any  semi-analytic variety is a semi-analytic variety. 
  

  \begin{figure}
	\centering
		\includegraphics[width=6cm]{algebraicvarity.png}
	\caption{Algebraic variety $x^2+y^2+z^2+2xyz-1=0$}
\end{figure}
       

\subsection{Thom-Mather Stratification}
The following is a key step in the proof of the Thom-Mather Theorem \ref{C0ESS}. 
\begin{thm}[Thom-Mather]
For every $C^\infty$-generic map from a compact manifold $M$ into $N$, there exists a stratification on $\Sigma_M$ of $M$ and a stratification $\Sigma_N$ on $N$ such that:\begin{enumerate}[$(i)$]
\item The strata of $\Sigma_M$ and $\Sigma_N$ are smooth submanifolds,  
\item the restriction of $f$ to each stratum of $\Sigma_M$ is a submersion onto a stratum of $N$,
\item this stratification is structurally stable: for every perturbation $f'$ of $f$ there are stratifications $\Sigma_M'$ and $\Sigma_N'$ homeomorphic  to respectively  $\Sigma_M$ and $\Sigma_N$, so that $(ii)$ holds for $f'$.\end{enumerate}
\end{thm}

The proof of this theorem is extremely interesting, it involves in particular the jet space, Thom's transversality theorem and Whitney stratifications in semi-analytics geometry. %The work on the jet space influenced some ideas described in section \ref{parablender}. 



\subsection{Laminar stratification}
Analogously to singularity theory, a structurally stable $C^1$-diffeomorphism displays a stratification. 
\begin{defi}
 A \emph  {lamination} of $M$ is a locally compact subset $\mathcal L$ of $M$, which is locally homeomorphic to the product of a $\mathbb R^d$ with a locally compact set $T$, so that $(\mathbb R^d \times \{t\})_{t\in T}$ corresponds to a continuous family of submanifolds.
\end{defi}

\begin{defi} A {\emph stratification of laminations} is a stratification  whose strata are endowed with a structure of  lamination.\end{defi}


\begin{prop}[\cite{BeMem}]
Let $f$ be a diffeomorphism  $M$ which satisfies Axiom A and the strong transversality condition. Then the stable set of every basic piece $\Lambda_i$ of $f$ has a structure of lamination $X_i$ whose leaves are stable manifolds. Moreover the family $\Sigma_s:= (X_i)_i$ forms a stratification of laminations such that  $X_i\le X_j$ iff $\Lambda_i \succeq \Lambda_j$ i.e. $W^u(\Lambda_i)\cap W^s(\Lambda_j)\not = \emptyset$.\end{prop} 



%\section{Structural stability far from bifurcation}
%If a map $f$  is structurally stable or $\Omega$-stable, then its periodic point does not bifurcate. This means that for every periodic point $p$ of $f$, for every perturbation $f'$ of $f$, a saddle (resp. sink, resp. source) periodic point $p$ persists as a saddle $p'$ (resp. sink, resp. source) for $f'$. 
%
%This simple hypothesis enables Ma\~ ne  to prove that the structurally stable $C^1$-maps satisfies axiom A and the strong transversality condition, by using a certain ergodic closing lemma, the Franck Lemma and the $C^1$-closing Lemma. 
%His idea was reapplied by Palis in \cite{Pa88}, and then by  \cite{AMS01}.
%
%
%Hence every continuous family $(f_t)_t$ of perturbations of $f=f_0$ defines a continuous family of periodic point $(p_t)_t$, called hyperbolic continuation of $p=p_0$.  
%
%\begin{defi} a family of maps $(f_t)_t$ such that its periodic points do not biffurcate is called a weakly stable family. 
%\end{defi}
%
%Hence a structurally map defines a weakly stable family. Surprisingly the converse is true in many sens.
%
%\subsection{Lambda-Lemma}
%\begin{lemm}[Lambda-Lemma, Ma\~ne-Sad-Sulivan  \cite{MSS}, Lyubich \cite{Ly84}]
%If a rational function $f$ of the Riemannian sphere is weakly stable (for the family of holomorphic perturbation) then $f$ is $\Omega$-structurally stable. 
%\end{lemm}
%
%
%It is well known in holomoprhic dynamic that an attracting  periodic point has a critical point in its basin. For, otherwise the function acts as an isometry on the universal covering of the basin (by Shwartz Lemma). Consequently the number of attracting periodic points is bounded by the degree of the rational function. 
%
%
%As a bifurcation creates an attracting periodic point, the following map is upper semi continuous at the neighborhood of $f$:
%\[f\mapsto cl(Per(f))\]
%
%
%%A consequence of this Lemma is the following:
%\begin{thm}[Ma\~ne-Sad-Sulivan  \cite{MSS}, Lyubich \cite{Ly84}]
%In the space of rational functions of degree $d\ge2$, structurally stable maps forms an open and dense set.
%\end{thm}
%
%
%
%
%We recall that among quadratic map, Fatou conjecture that the structurally stable maps are exactly  axiom A with the critical point orbit disjoint from the periodic sink. 
%
%
%The previous theorem for polynomial maps of $\mathbb C^2$ is wrong because of the Newhouse phenomenon (see section ...). 
%
%
%Nevertheless, given a complex polynomial automorphism $f_0\in \mathbb C[z,z']$ (without trivial dynamics) so that its family of perturbations $(f_ \lambda)_\lambda$ in the space of perturbation of the same degree is weakly stable, does $J^*_0= cl(Per(f_0))$ is structurally stable? 
%
%We recall that every periodic point $p_0$ of $f_0$ persists as a periodic point $p_\lambda$ of the same type for $f_\lambda$. 
%Moreover the map $\lambda\mapsto p_\lambda$ is holomorphic. 
%
%Hence the question is given a sequence $(p^n_0)_n$ of periodic point converging to $p_0\in J^*$, does the family $(\lambda \mapsto p^n_\lambda)_n$ converges? If yes we say that the holomorphic motion is well defined at $p_0$. 
%
%Dujardin and Lyubich \cite{LD13} showed that the holomorphic motion is well defined for every $p^0$ in an belongs to a uniformly hyperbolic compact set. Recently, we proved:
%\begin{thm}[Berger-Dujardin \cite{BD14}]
%For every hyperbolic invariant probability measure $\mu$, the holomorphic motion is uniquely defined at $\mu$-almost every point.
%\end{thm}
%\marginal{defi of hyperbolic invariant measure}
%In particular, for every $\lambda$,  the holomorphic motion defines a homeomorphisms between a subset $\mathcal R_0 \subset J^*_0$ and $\mathcal R_\lambda \subset J^*_\lambda$ which transport the hyperbolic measures of $f_0$ onto those of $f_\lambda$. The automorphism $f_0$ is called \emph{probabilitiscally stable}.
%
%As a corollary of the techniques, we show that when an axiom A automorphism is perturbed to one which is not axiom A, then there exists a periodic point which bifurcate. 
%
%A major part of Dujardin-Lyubich was dedicated to the following result:
%\begin{thm}[Dujardin-Lyubich \cite{LD13}]
%Given a polynomial automorphism $f$ of (dynamical) degree $d\ge 2$ and so that $ |det\, Df_0|\cdot d^2<1$, either $f$ is weakly stable, either a perturbation of $f'$ admits a homoclinic tangency. 
% \end{thm}
%As a corollary we had:
%\begin{cor}[Berger-Dujardin \cite{LD13}]
%Given a polynomial automorphism $f$ of (dynamical) degree $d\ge 2$ and so that $ |det\, Df_0|\cdot d^2<1$, either $f$ is probabilisticallt stable, either a perturbation of $f'$ admits a homoclinic tangency. 
% \end{cor}
%
\section{Structural stability of endomorphisms with singularities}
\label{Sec_SS_endo_w_Singu}
We are now ready to study the structural stability of endomorphisms which display a non empty critical set. 

%Clearly, there some maps which are structurally stable and have singularities. For instance in dimension 1, if the non-wandering set of an interval map is hyperbolic and the orbit of the critical points are pair with disjoint and disjoint from the non-wandering set, then it is $C^2$-structurally stable.

In dimension $2$, Iglesias-Portela-Rovella \cite{IPR08} showed the structural stability of $C^3$-perturbations of the hyperbolic rational functions $f$ which are equivalently stable and whose critical sets do not self-intersect along their orbits, nor intersect the non-wandering set. 

In all these examples, the critical set does not self intersect along its orbit. J. Mather suggested me to generalize a study he did about structural stability of graph of maps.


Let $G:=(V,A)$ be a finite oriented graph with a manifold $M_i$ associated to each vertex $i\in V$, and with a smooth map $f_{ij}\in C^\infty(M_i,M_j)$ associated to each arrow $[i,j]\in A$ from $i$ to $j$.


For $k\in \{0,\infty\}$, such a graph is \emph{ $C^k$-structurally stable} if for every $C^\infty$-perturbation 
$(f'_{ij})_{[i,j]\in A}$ of $(f_{ij})_{[i,j]\in A}$, there exists a family of $C^k$ diffeomorphisms $(h_{i})_i\in \prod_{i\in V} Diff^k(M_i, M_i)$ such that the following diagram commutes:
\[\forall [i,j]\in A\; \begin{array}{rcccl}
& &f'_{ij} & &\\ 
&M_i & \rightarrow&M_j &\\
h_i&\uparrow& f_{ij}&\uparrow&h_j\\
&M_i & \rightarrow&M_j & \end{array}\; .\]

The graph $(V, A)$ is \emph{convergent} if for every $[i,j], [i',j']\in A$ if $i=i'$ then $j=j'$. The graph is \emph{without cycle} if for every $n\ge 1$ and every  $([i_k,i_{k+1}])_{0\le k< n}\in V^n$ it holds $i_n\not= i_0$. 


\begin{thm}[Mather]
Let $G$ be a graph of smooth proper maps, convergent and without cycle. The graph is $C^\infty$- structurally stable if the following map is surjective:

$(\sigma_i)_i\in \prod_{i\in V}\chi^\infty(M_i)\mapsto (Tf_{ij}\circ\sigma_i-\sigma_j\circ f_{ij})_{[ij]}\in \prod_{[i, j]\in A}\chi^\infty (f_{ij}).$
\end{thm}

Mather gave me an unpublished manuscript of Baas relating his proof, that I developed to study the structural stability of attractor-repellor endomorphisms with possibly a non-empty critical set. 

\begin{defi}  Let $f$ be a smooth endomorphism of a compact, non necessarily connected manifold.  The endomorphism $f$ is {\emph attractor-repellor} if it satisfies  axiom A, and its basic pieces are either expanding pieces or attractors which $f$ acts bijectively. 
\end{defi}

The following theorem generalizes all the results I know (including 
\cite{IPR08} and \cite{IPR10})  about structurally stable maps with non-empty critical set.

   
\begin{thm}[Berger \cite{Be12}] Let $f$ be an {attractor-repellor}, smooth endomorphism of a compact, non necessarily connected manifold $M$. If the following conditions are satisfied, then $f$ is $C^\infty$-structurally stable:\begin{itemize}
\item[(i)] the singularities $S$ of $f$ have their orbits that do not intersect the non-wandering set $\Omega$,
\item[(ii)] the restriction of $f$ to $M\setminus \hat \Omega$ is $C^\infty$-infinitesimally stable, with $\hat \Omega:= cl\big(\cup_{n\ge 0} f^{-n}(\Omega)\big)$. In other words, the following map is surjective:
\[\sigma \in \Gamma^\infty(M) \mapsto Df\circ \sigma -\sigma \circ f\in 
 \Gamma^\infty(f)\]
 \item[(iii)] $f$ is transverse to the stable manifold of $A$'s points: for any $y\in A$, for any point $z$ in a local stable manifold $W^s_y$ of $y$, for any $n\ge 0$, and for any $x\in f^{-n}(\{z\})$, we have: 
\[Tf^n(T_xM)+T_zW_y^s=T_zM.\]
\end{itemize} 
\end{thm}
Hypothesis $(ii)$ might seem difficult to verify, but it is not. In \cite{Be12} we apply it to many example, even for map for which the critical set does self intersect along its orbit.

It would be intersecting to investigate how the attractor-repeller could be relaxed to enjoy a greater generality. However the $C^0$-equivalently stable singularities are not well classified and so an optimal theorem is today difficult to obtain. Nevertheless, it is not the case in dimension 2. Indeed it is well known that the structurally stable singularities are locally equivalent to one of the following polynomial (called resp. fold and cusp):
\[(x,y)\mapsto (x^2, y)\quad\text{and} \quad(x,y)\mapsto (x^3+xy, y).\]
Hence here is a natural question:
\begin{prob}
Under which hypothesis an axiom A surface endomorphism with singularity is structurally stable?
\end{prob}



%\subsection{$C^1$-Weak stability implies axiom A}
%The major breakthrough is the following theorem, proved by using a certain ergodic closing lemma, the Franck Lemma and the $C1$-closing Lemma. 
%\begin{thm}[Ma\~ne]
%A $C^1$-weakly stable diffeomorphisms satisfies  axiom A. 
%\end{thm}
%
%\begin{cor}[Ma\~ne]
%A $C^1$-structurally stable maps satisfies  axiom A and the strong transversality condition. 
%\end{cor}
%\begin{cor}[Palis]
%A $C^1$-$\Omega$-stable diffeomorphisms satisfies  axiom A and the no-cycle condition.
%\end{cor}
%In the endomorphisms case the following as been shown, as part of a conjecture that we can attribute to Przytycki:
%\begin{thm}[ Aoki-Moriyasu-Sumi] The $C^1$-weakly stable covering of compact manifolds satisfies  axiom A. Moreover if the covering is $\Omega$-stable (resp. structurally stable) then its satisfies the no-cycle condition (resp. the strong transversality condition. \end{thm}
%
%At this step let us recall the following:
%\begin{conj}[Berger-Rovella]
%A $C^1$-inverse limit stable endomorphisms (wich is possibly not bijective and might have singularity) satisfies  axiom A and the weak transversality condition. 
%\end{conj}



%\part{Non-uniformly hyperbolic dynamics}\label{PartNUH}
%Non-uniform hyperbolicity is a theory in construction. The aim is to prove that many dynamics can be describe by the ergodic theory.
%
%\section{Introduction to non-uniformly hyperbolicity with J.-C. Yoccoz}
%\label{IntroNUH}
%This section is an article we wrote with J.-C. Yoccoz \cite{BY14} to introduce  a book on the notion of strong regularity, a program by Yoccoz from the 90's to prove the non uniform hyperbolicity of low dimensional systems. 
% The book will be formed by two pieces of this program, 
% the Yoccoz proof of the Jakobson Theorem \cite{Y97}, and then my proof of the abundance of non-uniformly hyperbolic H\'enon-like endomorphisms \cite{berhen}. The first page of this introduction has been dropped since already expanded in Section 1. The rest of the text has been kept intact.
% 
%\subsection{ Non-uniformly hyperbolic dynamical systems}
%
%\subsubsection{ Pesin theory}
%The natural setting for non-uniform hyperbolicity is Pesin theory \cite{BP06,LY}, from which we recall some basic concepts. We first consider the simpler settings of invertible dynamics.
%
%\par
%
%Let $f$ be a $C^{1+\alpha}$-diffeomorphism  (for some $\alpha >0$) of a compact manifold $M$ and let $\mu$ be an ergodic $f$-invariant probability measure on $M$.
%The Oseledets multiplicative ergodic theorem produces  Lyapunov exponents (w.r.t. $\mu$)
%for the tangent cocycle of $f$, and an associated $\mu$-a.e $f$-invariant splitting of the tangent bundle into characteristic subbundles.
%
%Denote by $E^s(z)$ (resp. $E^u(z)$) the sum of the characteristic subspaces associated to the negative (resp. positive) Lyapunov exponents.
%
%The \emph{  stable and unstable Pesin manifolds} are defined respectively for $\mu$-a.e. $z$ by
%
%\[W^s(z)= \{z'\in M:\limsup_{n\to+\infty} \frac1n \log d(f^n(z),f^n(z'))<0\},\]
%\[ W^u(z)= \{z'\in M: \liminf_{n\to-\infty} \frac1n \log d(f^n(z),f^n(z'))>0\}.\]
%
%They are immersed manifolds through $z$ tangent respectively at $z$ to $E^s(z)$ and $E^u(z)$.
%
%
%
%
%
%The measure $\mu$ is \emph{hyperbolic} if $0$ is not a Lyapunov exponent w.r.t. 
%$\mu$.
%Every invariant ergodic measure, which is supported on a uniformly hyperbolic compact invariant set, is hyperbolic. 
%
%\paragraph{SRB, physical measures}
%An invariant ergodic measure $\mu$ is  \emph {SRB} if the largest Lyapunov exponent is positive  and the conditional measures  of $\mu$ w.r.t. a measurable partition into plaques of  unstable manifolds are $\mu$-a.s. absolutely continuous w.r.t. the Lebesgue class (on unstable manifolds). 
%When $\mu$ is SRB and hyperbolic, it is also \emph{ physical}: its basin has positive Lebesgue measure.
%
%\par
%The paper \cite{Y98} provides a general setting where appropriate hyperbolicity hypotheses allow to construct hyperbolic SRB measures with nice statistical properties.
%
%\paragraph{Coding}
%Let $\mu$ be a $f$-invariant ergodic hyperbolic SRB measure. Then there is a partition mod.$0$ of $M$ into finitely many disjoint subsets $\Lambda_1,\ldots, \Lambda_k$, which are cyclically permuted by $f$ and such that the restriction $f^k_{| \Lambda_1}$ is metrically conjugated to a Bernoulli automorphism.
%
%Of a rather different flavor is Sarig's recent work \cite{Sa13}. For a $C^{1+\alpha}$-diffeomorphism of a compact surface
%of positive topological entropy and any $\chi >0$, he constructs a countable Markov partition for an invariant set which has full measure w.r.t. any ergodic invariant measure with metric entropy $>\chi$. The semi-conjugacy associated to this Markov partition is finite-to-one.
%
%%\paragraph{Pesin sets}
%
%%Let $f$ be a $C^2$-diffeomorphism and let $\mu$ be an ergodic $f$-invariant hyperbolic probability measure. 
% %For every $\epsilon>0$, there exists a compact set $K_\epsilon(\mu)$ called \emph{Pesin set}, satisfying the following properties:
%%\begin{itemize}
%%\item[$(o)$] the $\mu$-measure of $K_\epsilon(\mu)$ is greater than $0$.
%%\item[$(i)$] the maps $z\in K_\epsilon \mapsto E^s(z)$ and $z\in K_\epsilon \mapsto E^u(z)$ are well defined and continuous, the angle between $E^s(z)$ and $E^u(z)$ is bounded from below by a positive number independent of $z\in K_\epsilon(\mu)$,
%%\item[$(ii)$] for every $z\in K_\epsilon(\mu)$, the $\epsilon$-neighborhoods $W^s_\epsilon(z)$ and $W^u_\epsilon(z)$ of $z$ in respectively $W^s(z)$ and $W^u(z)$, are well defined and   curvature smaller than $1/\epsilon$.
%%\end{itemize}
%
%
%\paragraph{Non-invertible dynamics}
%
%One should distinguish between the non-uniformly expanding case and the case of general endomorphisms.
%
%\smallskip
%
%In the first setting, a SRB measure is simply an ergodic invariant measure whose all Lyapunov exponents are positive and which is absolutely continuous.
%
%\smallskip
%
%Defining appropriately unstable manifolds and SRB measures for general endomorphisms is more delicate. One has typically to introduce the inverse limit where the endomorphism becomes invertible. 
%
%\subsubsection{Case studies}
%
%The paradigmatic examples in low dimension can be summarized by the following table:  
%
%\begin{center}
%	\begin{tabular}{|c|c|}
%	\hline
%		Uniformly hyperbolic & Non-uniformly hyperbolic\\
%		\hline
%		Expanding maps of the circle      &  Jakobson's Theorem \\%\cite{Ja81}\\
%		Conformal expanding maps of complex tori & Rees' Theorem \\%\cite{Re86}\\
%		Attractors (Solenoid, DA, Plykin...) &  Benedicks-Carleson's Theorem \\%\cite{BC2}\\
%		Horseshoes & Non-uniformly hyperbolic horseshoes \\%\cite{PY09}\\
%		Anosov diffeomorphisms & Standard map ?\\
%		\hline
%	\end{tabular}
%\end{center}
%
%Let us recall what are these theorems, and the correspondence given by the lines of the table.
%
%%The doubling angle maps of the circle $\mathbb S^1\ni \theta\mapsto 2\theta\in \mathbb S^1$ is a semi-conjugate to the Chebychev map $x\mapsto x^2-2$ via the map $x= 2\cos \theta$. The Chebychev map is certainly non-uniformly hyperbolic. 
%%Jakobson's Theorem shows basically that most of the perturbations of the Chebychev map is non-uniformly hyperbolic.  
%
%Expanding maps of the circle may be considered as the simplest case of uniformly hyperbolic dynamics. The Chebychev quadratic polynomial $P_{-2}(x) := x^2 -2$
%on the invariant interval $[-2,2]$ has a critical point at $0$, but it is still semi-conjugated to the doubling map $\theta \mapsto 2\theta$ on the circle (through $x = 2 \cos 2 \pi \theta$). For $a \in [-2, -1]$, the quadratic polynomial $P_a(x):=x^2 +a$ leaves invariant the interval $[P_a(0), P^2_a(0)]$ which contains the critical point $0$.
%
%\begin{thm}[Jakobson \cite{Ja81}] There exists a   set $\Lambda\subset [-2,-1]$ of positive Lebesgue measure  such that for every $a\in \Lambda$ the map $P(x)=x^2+a$ leaves invariant an ergodic, hyperbolic measure which is equivalent to the Lebesgue measure on $[P_a(0), P^2_a(0)]$.
%
%%\footnote{For one a dimensional map, the SRB measure is a measure absolutely continuous w.r.t. Lebesgue.}. 
%\end{thm}
%Actually the set $\Lambda$ is nowhere dense. Indeed the set of $a \in \R$ such that $P_a$ is  axiom A is open and dense \cite{GS97, Ly97}.
%
%\bigskip
%
%Let $L$ be a lattice in $\C$ and let $c$ be a complex number such that $|c| >1$ and $c L \subset L$. Then the homothety $ z \mapsto cz$ induces an expanding map of the complex torus $\C /L$. The Weierstrass function associated to the lattice $L$ defines 
%a ramified covering of degree $2$ from $\C /L$ onto the Riemann sphere which is a semi-conjugacy  from this expanding map to a rational map of degree $|c|^2$ called a \emph{Lattes map}. For any $d \ge 2$, the set ${\rm Rat}_d$ of rational maps of degree $d$ is naturally parametrized by an open subset of $\mathbb P(\C^{2d+2})$.
%
%
%\begin{thm}[Rees \cite{Re86}] For every $d\ge 2$, there exists a subset $\Lambda \subset {\rm Rat}_d$ of positive Lebesgue measure such that every  map $R\in \Lambda$ leaves invariant an ergodic hyperbolic probability measure which is equivalent to the Lebesgue measure on the Riemann sphere.
%\end{thm}
%
%For rational maps in $\Lambda$, the Julia set is equal to the Riemann sphere.
%On the other hand, a conjecture of Fatou \cite{Mi06} claims that the set of rational maps which satisfy axiom A is open and dense in ${\rm Rat}_d$. The  restriction of such maps to their  Julia set is uniformly expanding. 
%For such maps, the Hausdorff dimension of the Julia set is  smaller than $2$. 
%
%\bigskip
%
%The (real) H\'enon family is the $2$-parameter family of polynomial diffeomorphisms of the plane defined for $a,b \in \R$, $b \ne 0$ by 
%
%\[h_{a\,b}  (x,y) = (x^2+a+y,-bx)\]
%
%Observe that $h_{a\,b}$ has constant Jacobian equal to $b$.
%For small $|b| $, there exists an interval $J(b)$ close to $[-2, -1]$ such that, for $a \in J(b)$, the H\'enon map $h_{a\,b}$ has the following properties
%\begin{itemize}
%\item $h_{a\,b}$  has two fixed points; both are hyperbolic saddle points, one, called $\beta$  with positive unstable eigenvalue, the other , called $\alpha$, with negative unstable eigenvalue;
%\item there is a trapping open region $B$ satisfying $h_{a\,b}(B) \Subset B$ which contains $\alpha$  (and therefore also its unstable manifold ).
%\end{itemize}
%
%H\'enon \cite{Henon} investigated numerically the behavior of orbits starting in $B$ for $b=-0.3$, $a=-1.4$. Such orbits apparently converged to a \textquotedblleft strange attractor \textquotedblright. 
%
%\begin{thm}[Benedicks-Carleson \cite{BC2}] For every $b<0$ close enough to $0$, there exists a set $\Lambda_b\subset J(b)$ of positive Lebesgue measure, such that for every $a \in \Lambda_b$, the maximal invariant set $\bigcap_{n\ge 0} h^n_{a\,b}(B)$ is equal to the closure of the unstable manifold $W^u(\alpha)$ and contains a dense orbit along which the derivatives of iterates grow exponentially fast.
%\end{thm}
%
%An easy topological argument insures that this maximal invariant set is never uniformly hyperbolic.
%Later Benedicks-Young \cite{BY} showed that for every such parameters $a\in \Lambda_b$ the H\'enon map $h_{a\,b}$ leaves invariant an ergodic hyperbolic SRB  measure. Such a measure is physical. Benedicks-Viana \cite{BV01} actually proved that the basin of this measure has full Lebesgue measure in the trapping region $B$. 
%
%From \cite{Ur95}, every  $a \in \Lambda_b$ is accumulated by parameter intervals exhibiting Newhouse phenomenon:  for generic parameters in these intervals, $h_{a\,b}$ has infinitely many periodic sinks in $B$. In particular, the set $\Lambda_b$ is nowhere dense.
%
% %since the closure of the parameter exhibiting Newhouse phenomena contains a neighborhood of $A_b$.
%
%%\begin{theo}[Palis-Yoccoz \cite{PY09}] 
%%Let $f_0$ be a $C^3$-diffeomorphism of $\R^2$ leaving invariant a uniformly hyperbolic Horseshoe $K$ with stable and unstable dimension $d_s,d_u$ satisfying:
%%\[
%%(d_s+d_u)^2+(\max\{d_s,d_u\})^2<d_s+d_u+\max\{d_s,d_u\}
%%\]
%%Suppose the existence of two different fixed points  $P$ and $Q$ of $K$ with have local stable and unstable manifolds $W^u(P)$ and $W^s(Q)$ which have a quadractic tangency at a point $T$. 
%%Let $(f_a)_{a\in \R}$ a smooth family of diffeomorphisms unfolding non-degenerately this quadratic tangency. 
%%Then there exist a neighborhood $B$ of $H\cup\{f^i_0(P),\, i\in\Z\}$ and a positive set $\Lambda\subset \R$ of parameters  so that for every $a\in \Lambda$ the maximal invariant of $B$ for $f_a$ has zero Lebesgue measure, and a structure called \emph{non-uniformly hyperbolic horseshoe}. Gibbs state?
%%\end{theo}
%%By using by using \cite{Ro83} and \cite{MY10}, one can prove that as far as $d_s+d_u>1$, for generic families $(f_a)_a$, the set $\Lambda$ is nowhere dense: Newhouse phenomena occurs generically.
%
%\bigskip
%
%The starting point in \cite{PY09} is a smooth diffeomorphism of a surface $M$ having a horseshoe \footnote{A horseshoe is an infinite basic set of saddle type.} $K$. It is assumed that there exist distinct fixed points $p_s,p_u \in K$ and $q \in M$ such that 
%$W^s(p_s)$ and $W^u(p_u)$ have at $q$ a quadratic heteroclinic tangency which is an isolated point of $W^s(K) \cap W^u(K)$. The authors consider a one-parameter family $(f_t)$ unfolding the tangency and study the maximal $f_t$-invariant set $L_t$
%in a neighborhood of the union of $K$ with the orbit of $q$. Writing $d_s,d_u$ for the transverse Hausdorff dimensions of $W^s(K) ,\; W^u(K)$ respectively, it was shown previously \cite{PT93} that $L_t$ is a horseshoe for most $t$ when $d_s + d_u <1$. By \cite{MY10} this is no longer true when $d_s + d_u >1$. However, when $d_s + d_u $ is only slightly larger\footnote{The exact condition is  $(d_s+d_u)^2+(\max\{d_s,d_u\})^2<d_s+d_u+\max\{d_s,d_u\}$} than $1$, some dynamical and geometric information on $L_t$ is obtained in \cite{PY09} for most values of $t$: in particular, both the stable and unstable sets for $L_t$ have Lebesgue measure $0$, and an ergodic hyperbolic $f_t$-invariant probability measure supported on $L_t$ with geometric content is constructed.
%
%\bigskip
%
%The two papers in this volume are related to these case studies.
%
%\medskip
%
%In \cite{Y97}, a proof of Jakobson's theorem is given. The main  ingredient is the concept of \emph {strong regularity} (explained below).
%
%\medskip
%
%In  \cite{B11}, a class of endomorphisms of the plane containing the H\'enon family is considered. Given any map $B\in C^2(\R^3,\R^2)$ with small $C^2$-uniform norm, one studies the one-parameter family
%\[f_{a,B}(x,y)= (x^2+a+y,0)+B(x,y,a).\]
%It is shown that there exists a set $\Lambda_B \subset \R$ of positive Lebesgue measure such that, for any $a \in \Lambda_B$, $f_{a,B}$ has an invariant ergodic hyperbolic physical SRB measure. The proof is based on an appropriate generalization of strong regularity.
%
%\subsubsection{Open problems}
%
%Linear Anosov diffeomorphisms of $\T^2$ are area-preserving and uniformly hyperbolic. In the conservative setting, a very natural case study to consider is the Chirikov-Taylor standard map family. This is a one-parameter family of area-preserving diffeomorphisms of $\T^2$ defined for $a \in \R$ by
%\[ S_a(x,y) = (2x -y + a \sin 2 \pi x, x).\]
%One form of a conjecture of Sinai ( \cite{Si94} P.144) about this family is 
%\begin{conj}\label{sinaiconj}
%There exists a set $\Lambda\subset\R$ of positive Lebesgue measure such that, for $a \in \Lambda$,  the Lebesgue measure on $\T^2$ is ergodic and hyperbolic for $S_a$. 
%\end{conj}
%
%For such parameters, the map $S_a$ cannot have any of the invariant curves produced by KAM-theory. In particular, $a$ cannot be too small.
%
%This conjecture is still completely open despite intense efforts. A weak argument in favor of this conjecture is that, when $a$ is large, the maximal invariant set in the complement of an appropriate neighborhood of the critical lines $\{x= \pm 1/4 \}$ is a uniformly hyperbolic horseshoe of dimension close to 2 \cite{Du94, BC14}.
%
%
%Actually, a large  Hausdorff dimension of the invariant sets under consideration appears to be a major difficulty on the way to prove non-uniform hyperbolicity. 
%
%For the parameters considered in  \cite{BC2} and subsequent papers, the Hausdorff dimension of the H\'enon attractor is \emph {a priori}  close to $1$. On the other hand, numerical studies \cite{RHO80} of the values $a=-1.4$, $b=-0.3$ considered by H\'enon
%indicate an (eventual) attractor of Hausdorff dimension $1.261 \pm 0.003$.
%
%\begin{prob}\label{P1}
%For every $d<2$, find an open set of smooth families  $(f_t)_t$ of smooth diffeomorphisms  of $\R^2$ such that, with positive probability on the parameter,  $f_t$ leaves invariant an ergodic hyperbolic SRB probability measure whose support has dimension at least $d$.   
%\end{prob} 
%
%One should also recall that  Carleson conjectured \cite{Ca91} that proving non-uniform hyperbolicity
%(or only the weaker conclusion of  \cite{BC2}) for a particular parameter value is in some rigorous sense undecidable.
%
%A similar problem, in the setting of non-uniformly hyperbolic horseshoes, is 
%
%\begin{prob}\label{P2}
%Prove the conclusions of \cite{PY09} for an initial horseshoe $K$ of transverse Hausdorff dimensions $d_s,d_u$ satisfying 
%\[d_s+d_u>3/2.\]    
%\end{prob} 
%
%Even the non-uniformly expanding case is still incomplete, since it regards only the case of real or complex dimension 1. A positive answer to the following problem would be a 2-dimensional generalization of Jakobson's Theorem for perturbation of the product dynamics:
%\[P_a\times P_a\colon (x,y)\mapsto (x^2+a,y^2+a).\]
%
%\begin{prob}
%Does there exist an open set of $1$-parameter smooth families $(f_a)$ of endomorphisms of the plane, accumulating on $(P_a \times P_a)_a$, with the following property: with positive probability on the parameter, $f_a$ leaves invariant
%an ergodic absolutely continuous invariant measure with two positive Lyapunov exponents.
%\end{prob}
%
%
%
%
%\subsection{Proving non-uniform hyperbolicity in low dimension}
%
%There are now many proofs of both Jakobson's theorem and Benedicks-Carleson's theorem. Broadly speaking, they rely either on a binding approach, pioneered by Benedicks-Carleson, or on a strong regularity approach, closer to Jakobson's original proof.
%Both papers in this volume follow the second approach. 
%
%In both approaches, the study of the $2$-dimensional setting depends very much on the $1$-dimensional case.
%
%
%We now explain some of the differences between the two methods.
%
%
%
%\subsubsection{ The binding approach  for quadratic maps}
%
%Benedicks-Carleson proved Jakobson's theorem by focusing on the expansion of the post-critical orbit. There are many proofs in this spirit \cite{CE80, BC1, Ts932, Ts93, Lu00}.
%
%
%One actually proves the existence of a  set $\Lambda \subset \R$ of positive Lebesgue measure such that, for $a \in \Lambda$, the quadratic map $P_a(x)=x^2+a$ satisfies the  Collet-Eckmann condition: 
%\[\liminf_{+\infty} \frac1n \log \|DP^n(a)\|>0.\] 
%This property implies the existence of an absolutely continuous ergodic invariant measure with positive Lyapunov exponent\cite{CE83}.
%
%One starts with a parameter $a_0$   such that the critical value $a_0$ of $P_{a_0}$ belongs to a repulsive periodic cycle. Then, there exists $\lambda>1$ so that 
%\begin{itemize}
%\item[$(i)$] $DP_{a_0}^n(a_0)>\lambda^n$ for every  large $n$,
%\item[$(ii)$] for every $\delta>0$, the map $P_{a_0}$ is $\lambda$-expanding on the complement of $[-\delta,\delta]$ (for an adapted metric).  
%\end{itemize}
%
%Then for every  large $M$,  for every $a$ close to $a_0$ the post-critical orbit $(P^n_a(a))_{n\le M}$ is close to $(P_{a_0}^n(a_0))_{n\le M}$ and so has a similar expansion. At the next iterations $N=M+1$, there are three possibilities:
%\begin{itemize}
%\item[$(a)$] either $P_{a}^{N}(a)$ is not in $(-\delta,\delta)$ and so the expansion will continue by $(i)$, 
%\item[$(b)$] or $P_{a}^{N}(a)$ is in $(-\delta,\delta)$ but is not too close to $0$; then there exists an integer $k<N$, called \emph{the binding time}, such that the orbits $P_{a}^{N+i}(a)$ and $P_{a}^{i}(0)$ remain close for $i\le k$ and separate for $i=k+1$. The expansion of $(DP_{a}^{i}(a))_{i <k}$ is transferred to $(DP_{a}^{i}(P^{N+1}_a(a)))_{i  < k}$ . The  logarithmic contraction at time $N$, equal to $\log |DP_a (P_a^N (a))|$, is only roughly half the logarithmic expansion during the binding period $\log |DP^{k-1}_a (P_a^{N+1} (a))|$.
%\item[$(c)$] or $P_{a}^{N}(a)$ is so close to $0$ that $(b)$ does not hold.
%\end{itemize}
%Cases $(a)$ and $(b)$ are allowed. Case $(c)$ is excluded in the parameter selection by removing the parameter $a$ for which this occurs.
%Then we can redo the same alternative with $N\leftarrow N+1$ in case $(a)$ and $N\leftarrow N+k$ in case $(b)$.
%%For the paper selection the distance corresponding to Case $(c)$ needs to decrease exponentially fast.
%
%In case $(b)$, roughly half of the original transferred logarithmic expansion is lost in the binding process. Therefore the Collet-Eckmann condition will not be satisfied if too much time is spent in iterated binding periods. To avoid this, it is asked that:
%
%
%\begin{itemize}
%\item[$(H_N)$] the total length of all the  binding periods before $N$ is small with respect to $N$.
%\end{itemize}
%Actually, when appropriately formulated, the condition $(H_N)$ implies that case $(c)$ above does not hold. Hence if $(H_N)$ holds for every $N$, the map is Collet-Eckmann. 
%
% 
%
%
%To perform the parameter selection, we look at maximal \emph{critical curves} $\gamma=(P_a^N(a))_{a\in \mathcal I}$ so that:
%\begin{itemize} 
%\item[($P_1$)] Condition $(H_n)$ holds for  every $a\in \mathcal I$ and for every $n\le N$;
%\item[($P_2$)] the binding periods in $[0,N]$ are the same for every $a\in \mathcal I$, and the integer $N$ is not part of a binding period;
%\item[($P_3$)] the length of the curve $\gamma$ is bounded from below by some uniform constant.
%\end{itemize}
%
%Such a curve is split into different pieces according to which scenario holds at time $N+1$. Pieces corresponding to scenario $(a)$ are iterated once. Pieces corresponding to scenario $(c)$ (or to scenario $(b)$, with a binding time $k$ too long to satisfy $(H_{N+k})$) are discarded. The other pieces are iterated untill the end of the corresponding binding period. These new critical curves satisfy $(P_1)$ and $(P_2)$. Property $(P3)$ is also satisfied, except for some boundary effects that are easily taken care of.
%
%\medskip
%
%
%A large deviation argument, relying on property $(P3)$, shows that the Lebesgue measure of the remaining parameters is positive (actually, a large proportion of the length of the starting parameter interval).
%
%
%
%
%\subsubsection{The binding approach for H\'enon family}
%
%There are many proofs in this spirit \cite{BC2, MV93, WY01, YW08, Ta11}. 
%
%\medskip
%
%A major difficulty of the $2$-dimensional setting is that critical points are not defined beforehand, and will only be well-defined for good parameters.
%
%Call a curve \emph{flat} if it is $C^2$-close to a segment  of $\R\times \{0\}$. Roughly speaking, given a flat segment $\gamma \subset W^u(\alpha)$ going across the critical strip $\{ \vert x \vert \leq \delta \}$, a critical point on $\gamma$ should be  a point of $\gamma$
% such that the vertical tangent vector is exponentially dilated under positive iteration, while the tangent vector to $\gamma$ is exponentially contracted.
% 
% \smallskip
% 
% In the inductive construction of good parameters, only $N$ iterations of the H\'enon map are considered at a given stage. Under the appropriate induction hypotheses, one defines an approximate critical set $\mathcal C_N$. This is a finite set of cardinality exponentially large with $N$. Each point of $\mathcal C_N$ lies on a flat segment contained in $h_{a\,b}^{\theta N} (W^u_{loc}(\alpha))$, with $\theta\sim \vert \log \vert b \vert \vert^{-1}$. 
% 
% The main problem of the induction step is to extend the exponential dilation along the finitely many critical orbits beyond time $N$. As in the $1$-dimensional case, this is automatic when the critical orbit at time $N$ lies outside of the critical strip. On the other hand, when the critical orbit at time $N$ returns to a point $z_N$ of the critical strip, one has to find, after excluding inadequate parameters, a \emph{binding} critical point $\tilde z_0$ whose initial expansion will be transferred (at some cost) to the orbit of $z_N$.
% It is here important that $z_N$ should be in \emph{tangential position}, i.e much closer to the flat segment containing $\tilde z_0$ than to $\tilde z_0$ itself.
% 
% \smallskip
% 
% To prove that the set of non-excluded parameters (at the end of the induction process) has positive Lebesgue measure, one has to investigate carefully how the whole structure of approximate critical points, analytical estimates and binding relationships survives through parameter deformation. This is certainly the  trickiest part of the method.
%
%
%
%
%\subsubsection{Puzzles and parapuzzles}
%Puzzles and parapuzzles are combinatorial structures which were first introduced in $1$-dimensional complex dynamics to study the local connectivity of Julia sets and the Mandelbrot set  \cite{Hu91,Mi92}. In real $1$-dimensional dynamics, they were instrumental in the proof that almost every quadratic map satisfies  either axiom A or the Collet-Eckmann condition \cite{Ly02, AM03}.
%
%\par
%For real Julia sets of real quadratic maps, puzzle pieces are defined as follows. Let $a$ be a parameter in $[-2,-1]$. Then the quadratic polynomial $P_a$ has two fixed points $\alpha,\beta$, both repelling, denoted so that $-\beta < \alpha < -\alpha < \beta$. The real Julia set is equal to $[-\beta, \beta]$. For $n\geq 0$, the \emph{puzzle pieces of order $n$} are the closures of the connected components of $[-\beta, \beta] \setminus P_a^{-n}(\{\alpha,-\alpha \})$.
%
%\par
%Puzzle pieces of successive orders are related in two fundamental ways: a puzzle piece of order $n$ is contained in a puzzle piece of order $n-1$, and its image is contained in a puzzle piece of order $n-1$. The combinatorics of the partition by puzzle pieces of a given order depend on the sequence of nested puzzle pieces containing the critical value. This leads to a sequence of partitions of parameter space into \emph{parapuzzle} pieces. It is a general rule of thumb that, assuming a mild level of hyperbolicity, the combinatorics and geometry of parapuzzle pieces around a given parameter $a$ are closely related to the combinatorics and geometry of puzzle pieces for $P_a$ around the critical value. 
%
%%\subsection{The strong regularity approach  for quadratic maps}\label{SR1}
%
% Let $a$ be a parameter in $[-2,-1]$. A \emph{regular interval} is a puzzle piece of some order $n>0$ which is sent diffeomorphically onto $A := [\alpha, -\alpha]$ by $P_a^n$. One also asks that the corresponding inverse branch extends to a fixed neighborhood of $A$, which insures a control of the distortion. The parameter $a$ is  \emph{regular} if the measure of the set of points in $A$ which are not contained in a regular interval of order $\leq n$ is exponentially small with $n$. A classical argument shows that regular parameters satisfy the conclusions of Jakobson's theorem.
%
%\par
%To prove that the set of regular parameters has positive Lebesgue measure, one considers a more restrictive condition called \emph{strong regularity}. Assume that the parameter is close to the Chebychev value $a_0:= -2$. Then the return time $M$ of the critical point to $A$ is large. Moreover, the complement in $A$ of a neighborhood of $0$ of approximate size $2^{-M}$ is covered by finitely many regular intervals of order $<M$, which are called \emph{simple}. The parameter $a$ is called \emph{strongly regular} if
%\begin{itemize}
%\item[$(\star)$] there exists a sequence of regular intervals $(I_j)_{j>0}$ of order $(n_j)_j$ such that   $P_a^{M+n_1+\cdots+ n_{j-1}}(a)\in I_{j}$ for all $j>0$;
%\item[$(\diamondsuit )$] most $I_j$ are simple in the sense that $\sum_{i\le j: I_i\text{ is not simple} }n_i<\!\!< \sum_{i\le j} n_i$ for all $j>0$.
%\end{itemize}
%
%The most delicate part of the proof is to establish, through a careful analysis of the puzzle structures, that strongly regular parameters are regular. Then one is able to transfer the exponential regularity estimate from puzzles in phase space to parapuzzles in parameter space. Finally, one concludes through a large deviation argument that the set of strongly regular parameters has positive Lebesgue measure.
%
%
%
%\subsubsection{ The strong regularity approach for H\'enon family}\label{SR2}
%
%
%The hyperbolic fixed point $(\alpha,0)$ of $h_{a,0}$ persists as a fixed point $P$ for $h=h_{a\, b}$, with $b$ small. 
%One denotes by $Q\approx (-\alpha,0)$ the first (transverse) intersection of the stable and unstable manifolds of $P$. Let $\mathbb S$ be the segment of $W^u(P)$ bounded by $P$ and $Q$. It is a flat curve.  Given a segment $I$ of $W^u(P)$ one denotes by $W^s_\theta (\partial I)$ the union of the $\theta$-local stable manifolds of the endpoints of $I$, with $\theta= 1/|\log b|$. 
%
%A flat curve is \emph{stretched} if its end points belong to $W^s_\theta (\partial \mathbb S)$. A \emph{puzzle piece} of a flat curve $S$  is a pair $(I,n_I)$ of a segment $I$ of $S$ sent by $f^{n_I}$ \emph{onto} a flat \emph{stretched} curve. The puzzle piece is \emph{hyperbolic} if $f^{n_I}|I $ satisfies some hyperbolicity conditions and regular if it satisfies moreover a distortion condition. A \emph{puzzle pseudo-group} is the data of a pair $(\Sigma,\mathcal Y)$ formed by a family $\Sigma$ of flat stretched curves (formed in particular by $\mathbb S$), and by a set of hyperbolic puzzle pieces $\mathcal Y$ associated to the curves of $\Sigma$ so that, $\forall (I,n_I)\in \mathcal Y$,  $h^{n_I}(I)$ is a curve of $\Sigma$. The puzzle pseudo group is \emph{regular} (and the map is \emph{regular}) if for every curve $S\in \Sigma$, the measure of the set of points in $S$ which are not contained in a regular piece of order $\leq n$ given by $\mathcal Y$ is exponentially small with $n$, and if every puzzle piece in $\mathcal Y$ of $S\in \Sigma$ persists as a puzzle puzzle in $\mathcal Y$ for $S'\in \Sigma$ nearby $S$. A classical argument shows that regular maps leave invariant an ergodic, physical SRB measure.
%
%The notion of strong regularity is also generalized to show the abundance regular maps.  A H\'enon map is \emph{strongly regular} if it preserves a combinatorial and geometrical object called \emph{puzzle algebra}. Such an object does not need the concept of critical point to be defined; it relies basically on the topology of the homoclinic tangle of $P$. 
%
%By hyperbolic continuity, the simple regular intervals persist as puzzle pieces of every flat stretched curve $S$, their complement in $S$ is 
% denoted by  $S_\square$.  
%
%
%A \emph{puzzle algera} is the data of: a puzzle pseudo-group $(\Sigma, \mathcal Y)$, a family of \textquotedblleft semi-artificial\textquotedblright\; flat stretched curves $\Sigma^\square$, and for every $S\in \Sigma \sqcup \Sigma^\square$  an admissible sequence of puzzle pieces $c(S)=(I_i,n_i)_i\in\mathcal Y^{\mathbb N}$ from $\mathbb S$ satisfying the condition $(\diamondsuit )$. \emph{Admissibility} means that the intersection $J_k(S):=\cap_{i=1}^k f^{-n_{i-1}-\cdots-n_1} (I_i)$ is a puzzle piece of $\mathbb S$ for every $k\ge 1$. One shows that $(\diamondsuit )$ implies that the local stable manifolds $W^s_{\theta }(\partial J_k)$ have their end points in $\{y>\theta 2^{-M}\}$ and $\{y<-\theta 2^{-M}\}$. Hence one can ask  $\forall k>0$, $ S\in \Sigma\sqcup \Sigma^\square$:
%\begin{itemize}
%\item[$(\star)$] the segment $S_\square$ is folded by $f^{M}$ between both components of $W^s_\theta(\partial J_k(S))$.
%\end{itemize}
%This is the main ingredient of puzzle algebras definition. One notices that $f^M(S_\square)$ is tangent to a local stable manifold of the singleton $\cap_k J_k(S)$. Conversely, from these topological conditions, some combinatorially defined puzzle pieces  turn out to be necessarily regular. They form $\mathcal Y$. Also some combinatorially defined local unstable manifolds turn out to be necessarily flat. Those which are stretched form $\Sigma$, the other are artificially stretched to form $\Sigma^\square$. 
%This combinatorial formalism is certainly the main novelty and difficulty of this proof: pure topological and combinatorial properties imply analytical properties. Then it is rather quick to prove the regularity of $(\Sigma,\mathcal Y)$ and so the regularity of strongly regular maps.
%
%To handle the parameter selection, by induction on $k$, for a $C^2$-open set of dynamics $f$, we can define combinatorially 
% a finite family of flat stretched curves $\check \Sigma_{k}$.
% Similar conditions are asked on $\check \Sigma_k$. This implies the existence and regularity of many flat stretched curves and puzzle pieces.
% When the map is strong strongly regular, every curve in  $\Sigma\sqcup \Sigma^\square$ can be approximated by a curve of $\check \Sigma_k$ for $k\ge 0$.  These combinatorial definitions enable one to follow carefully how the whole structure survives by parameter deformation. 
%\section{Strongly regular quadratic maps and H\'enon-like maps }
%\label{StrongRegularity}
%In this section we recall Yoccoz' proof of Jakobson's Theorem, and how it has been generalized in \cite{berhen} to prove Benedicks-Carleson's Theorem. 
%
%\subsection{Strongly regular quadratic maps}
%
%For $a$ greater but close to $-2$, the quadratic map $P\colon  x\mapsto x^2+a$ has two fixed points $-1\approx A_0< A_0'\approx 2$ which are hyperbolic. The segment $[-A_0', A_0']$ is sent into itself by $P$, and its boundary bounds the basin of infinity. All the points of $(-A_0',A_0')$ are sent by an iterate of $P_a$ into $\R_\se:=[A_0,- A_0]$.
%
%Yoccoz' definition of strongly regular maps is based on the position of the critical value $a$ with respect to the preimages of $A_0$. To formalize this, he used his concept of puzzle pieces.
%
%
%\subsubsection{Puzzle pieces}
%
%\begin{defi}[Piece and puzzle piece]
%A \emph{piece} $\sa= \{\R_\sa, n_\sa\}$ is the data of a segment $\R_\sa$ of $\R_\se$ and an integer $n_\sa$ so that $P^{n_\sa}| \R_\sa$ is injective. The piece $\sa$ is a \emph{puzzle piece} if $P^{n_\sa}$ sends $\R_\sa$ bijectively onto $\R_\se:=[A_0,- A_0]$. 
%\end{defi}
%For instance $\se:= \{\R_\se, 0\}$ is a puzzle piece, called \emph{neutral}. 
%
%To define the simple puzzle pieces, let us denote by $M$ the minimal integer such that $P^M(a)$ belongs to $[A_0,-A_0]$; $M$ is large since $a>-2$ is close to $-A_0'\approx -2$.
%
%For $i\ge 0$, let $A_i:= -(P|\R^+)^{-i}(-A_0)$. Note that $(A_i)_{i\ge 0}$ is decreasing and converges to $-A_0'$. Also $[A_{i+1}, A_{i}]$ is sent bijectively by $P^{i+1}_a$ onto $\R_\se$. The same holds for $[ -A_i, -A_{i+1}]$.
%\begin{figure}[h]
%    \centering
%        \includegraphics{Fig0HDR.pdf}
%%    \caption{Geometric model for some parameters of the H\'enon map. }
%\label{geometricmodele}
%\end{figure}
%
%
%By definition of $M$, the critical value $a$ belongs to  $[ A_{M}, A_{M-1}]$.
% Hence for $2\le i\le M$, there is a segment $\R_{\ss^{i}_-}\subset \R^-$ and a segment $\R_{\ss^{i}_+}\subset \R^+$ both sent bijectively by $P$ onto $[- A_{i-1},-A_{i-2}]$.
%
%\begin{defi}[Simple puzzle piece]
%The pairs of the form $\{\R_{\ss^{i}_\pm}, i\}$ for $2\le i\le M$ are puzzle pieces called \emph{simple}. There are $2(M-1)$ such pairs. The set of simple puzzle pieces is denoted by $\sY_0=\{\ss^i_\pm ; 2\le i\le M\}$.  \label{Psimple}\index{$\sY_0$}
%\end{defi}
%
%Puzzle pieces enjoy two fundamental properties:
%\begin{enumerate}
%\item Two puzzle pieces $\sa$ and $\sb$ are nested or disjoint: 
%\[\R_\sa\subset \R_\sb\text{ or } \R_\sb\subset \R_\sa \text{ or }  int\; \R_\sb\cap int\; \R_\sb=\varnothing\;.\]
%\item For every puzzle piece $\sa$, for every perturbation of the dynamics, the hyperbolic continuities of the relevant preimages of the fixed point $A_0$ define a puzzle piece for the perturbation.
%\end{enumerate}
%\subsubsection{Building puzzle pieces}
%
%The first operation is the so-called \emph{simple product} $\star$:
%\begin{defi}[$\star$-product]
%Let $\sa = \{\R_\sa,n_\sa\}$ and $\sb = \{\R_\sb, n_\sb\}$ be two  pieces. The \emph{simple product} of $\sa$ with $\sb$ is the piece
%$\sa\star \sb$ with $\R_{\sa \star \sb} =  (P^{n_\sa}|\R_\sa)^{-1}(\R_\sb)\cap \R_\sa$ and $n_{\sa\star \sb} = n_\sa +n_\sb$.
%We say that the product is \emph{suitable} if $\R_{\sa \star \sb}$ is not empty neither a singleton. 
%\end{defi}\index{Simple product $\star$}
%We notice that if $\sa$ and $\sb$ are puzzle pieces, then 
%$\sa\star \sb$ is a puzzle piece (and suitable).
%
%
%Note that the simple operation $\star$ is associative. Indeed for any puzzle pieces $\sa,\sb,\sc$, it holds:
%\[\sa\star (\sb\star \sc)= (\sa\star \sb)\star \sc=:\sa\star \sb\star \sc\; .\]
%
%We need another operation to construct pieces in the closure  $\R_\square$ of the complement of the simple pieces union in $\R_\se$:
% \[\R_\square:= cl(\R_\se\setminus \cup_{\sa\in \sY_0} \R_\sa)=P^{-1}_a([ -A_{M}, -A_{M-1}])\]\index{$\R_\square$}
% 
%This is a neighborhood of $0$ of length dominated by $2^{-M}$ when $a$ is close to $-2$.
%
%This second operation is the so-called \emph{parabolic product} $\square$. 
%
%\begin{defi}[$\square$-product]\index{Parabolic product $\square$}
%Let $\sa$ and $\sb$ be two puzzle pieces so that $\R_\sb\subsetneq \R_\sa$ and so that $P^{M+1}(0)$ belongs to $\R_\sb$.
%We notice that $P^{M+1}|\R_\square$ has two inverse branches, one $g_+$ with image into $\R^+$ and the other $g_-$ with image into $\R^-$. 
%
%We define the parabolic pieces:
%\[\square_+(\sa-\sb):=\{g_+(cl(\R_\sa\setminus  \R_\sb)), M+1+n_\sa\}
%\quad \text{and}\quad \square_-(\sa-\sb):=\{g_-(cl(\R_\sa\setminus  \R_\sb)),M+1+ n_\sa\}\]
%\end{defi}
%A parabolic piece $\sp= \square_\pm(\sa-\sb)$ is never a puzzle piece. Indeed, with $\sp := \{\R_\sp, n_\sp\}$, the segment $\R_\sp$ is sent by $P^{n_\sp}$ onto a connected component of $cl(\R_\se\setminus P^{n_\sa}(\R_\sb))\subsetneq\R_\se$.
%
%
%We notice that if $\sp$ is a parabolic piece and $\sa$ is a puzzle piece so that $\sp \star \sa$ is suitable, then $\sp \star \sa$ is a puzzle piece.
%
%%We notice that the $\star$-product extends canonically to the set of parabolic and puzzle pieces: we can make simple product between those pieces. 
%
%\subsubsection{Yoccoz' definition of strong regularity}
%
%
%
%The main ingredient of Yoccoz' definition, is to ask for the existence of a sequence of puzzle pieces $\sc = (\sa_i)_{i\ge 1}$ so that the 3 following conditions hold:
%\begin{enumerate}[$(i)$]
%\item  with $\sc_k= \sa_1\star \cdots\star \sa_k$ the first return $P^M(a)$ belongs to a nested intersection of puzzle pieces $\cap_{k\ge 1} \R_{\sc_k}$:
%\begin{equation}\tag{$SR_1$}
%P^{M+1}(0)\in \bigcap_{k\ge 1} \R_{\sc_k}\; ,
%\end{equation}\index{SR$_1$}
%\item the sequence of puzzle pieces $(\sa_i)_{i\ge 1}$ satisfies: 
%\begin{equation}\tag{$\blacklozenge$}
%\sum_{j\le i\; \sa_j\notin \sY_0} n_{\sa_j} \le e^{-\sqrt M} \sum_{j\le i-1} n_{\sa_j},\quad \forall j\le i.
% \end{equation}
%\item Every involved segment  $\R_{\sa_i}$ has a neighborhood $\hat \R_{\sa_i}$ which is sent bijectively by $P^{n_{\sa_i}}$ onto the neighborhood $[A_1,-A_1]$ of $\R_\se$.
%\end{enumerate}
%
%
%%there is a neighborhood $\hat \R_\se$ of $\R_\se$ so that which is sent bijectively by $P^{n_{\sa_i}}$ onto $\hat \R_\se$. 
%The negativity of the Schwarzian derivative of $P$ gives then a distortion bound for $P^{n_{\sa_i}}|\R_{\sa_i}$. Such a hypothesis is assumed in particular for all simple pieces in $\sY_0$.
%
%Moreover, a computation gives the existence of $c>0$ such that for every $x\in \R_\sa$, $\sa\in \sY_0$, it holds:
% \[\|\partial_x P^{n_\sa}\|\ge e^{cn_\sa}\; .\] 
%Then Equation ($\star$) and the distortion bound implies:
%\begin{equation}\tag{$\mathcal {CE}$}
%\liminf_{n\to \infty} \frac1n\log \|\partial_xP^{n}(a)\| \ge c^-:=(1-e^{-\sqrt M} )c\;.\end{equation}
%In particular, strongly regular unimodal maps satisfy the Collet-Eckmann condition.
%
%\subsubsection{Alternative definition of strong regularity}
%The existence of the interval $\hat \R_{\sa_i}$ extending $\R_{\sa_i}$ is replaced by an extra condition of $(\diamondsuit )$ on the post critical orbit, which implies the hyperbolicity of any parabolic pieces and even puzzle pieces.   
%
%% two other conditions: $h$-times and $(\diamondsuit )$. 
%
%%The definition differ from the way the distortion bound is obtained. 
%%
%%It is well known since \cite{ABV00, BC2} that another way to get a distortion bound on a piece $\sa=(\R_\sa,n_\sa)$ is to satisfies the following hyperbolic times inequality. 
%
%
%\begin{defi} A puzzle piece or a parabolic piece $\sa=\{\R_\sa,n_\sa\}$ is \emph{hyperbolic} if it satisfies the following condition:
%\begin{equation}\tag{$h-times$} 
%\forall z\in \R_\sa\text{ and }l\le n_\sa: \\
%|\partial_x P ^{n_\sa}(z)| \ge e^{\frac{c}{3} (n_\sa-l)} |\partial_x P^l(z)|\; ,
%\end{equation}\index{h-times}
%with $c:= \log 2/2$ \index{$c$}.
%\end{defi}
%
%It is straight forward to see that a $\star$-product of hyperbolic pieces is hyperbolic.
%
%Suppose that the map $P$  satisfies $(SR_1)$ with $(\sc_k)_{k\ge 1}$. We define the following countable set of symbols  $\sA:= \sY_0\sqcup\{\square_\delta(\sc_k-\sc_{k+1}): \; k\ge 0,\; \delta\in \{+,-\}\}$.\index{$\sA$}
%\begin{prop}\label{pour2564}
%Every puzzle piece $\sa$ is a simple product of pieces  in $\sA$.
%\end{prop}
%\begin{proof}
%We proceed by induction on the order of $\sa$. As the puzzle pieces  are nested or disjoint, or $\R_\sa$ is included in a simple piece $\R_\ss$ either it is included in $\R_\square$.
%
% In the first case,
%$(P^{n_\ss}(R_\sa), n_\sa-n_\ss)$ is still a puzzle piece and by induction it is a product of parabolic and simple piece $\sa_1\star \cdots \star \sa_k$. Hence $\sa= \ss\star \sa_1\star \cdots \star \sa_k$. 
%
%In the second case, $\R_\sa$ is either included in $\R^-$ or in $\R^+$. Also its first return in $\R_\se$ is $f^{M+1}(\R_\sa)$. Note that
%$\{f^{M+1}(\R_\sa),n_\sa-M-1\}$  is still a puzzle piece. Let $k\ge 0$ be the greater integer so that $f^{M+1}(\R_\sa)$ is included into $\R_{\sc_k}$. Then $\R_\sa$ is included in $\R_{\square_\pm (\sc_k-\sc_{k+1})}$. Also its image by $f^{n_{ \square_\pm (\sc_k-\sc_{k+1})}}$ is also a puzzle piece and so we can use the induction hypothesis as above to achieve the proof. \end{proof}
%\begin{defi}\index{Prime}
%A puzzle piece $ \sa$ is \emph{prime} if it is a simple puzzle piece or if there exist parabolic pieces $\sp_1,\dots, \sp_k\in \sA$ and a simple puzzle piece $\ss\in \sY_0$ so that:
%\[\sa= \sp_1\star \sp_2\star\cdots\star \sp_k\star \ss.\]
%\end{defi}
%
%Hence to obtain the hyperbolicity of any puzzle piece, it suffices to give a  combinatorial condition on the critical orbit which implies the hyperbolicity of all the simple pieces and all the parabolic pieces in $\sA$.
%%Hence to ensure that all of them satisfies a hyperbolic times inequality, it suffices that every parabolic piece and every simple piece satisfies a uniform hyperbolic times inequality. 
% This is the case if $P$ satisfies $(SR_1)$ with a sequence $\sc=(\sa_i)_i$ so that $P^{M+n_{\sc_i}}(a)\in \R_\se$ does not belong to an exponentially small neighborhood of $\partial \R_\se=\{A_0,-A_0\}$.  
% 
%To make the notation less cluttered, we denote $\ss^2_-$ and $\ss^2_+$ by respectively $\ss_-$ and $\ss_+$. These two puzzle pieces have their segment which is a neighborhood of respectively $A_0$ and $-A_0$ in $\R_\se$.
%  
%Likewise, the segments of the pieces 
%$\ss_-^{k}:=\ss_-\star  \cdots \star \ss_-$ and  $\ss_+^{k}:=\ss_+\star \ss_-^{k}$ are neighborhoods of respectively $A_0$ and $-A_0$ in $\R_\se$.
%
%The condition we ask is the following:
%\begin{equation}\tag{$\diamondsuit $}
%P^{M+1+n_{\sc_i}}(0)\notin \R_{s_-^{\aleph(i)}}\sqcup \R_{s_+^{ \aleph(i)}}\;,
%\end{equation}\index{$\diamondsuit $}
%with $\aleph(0):=\left[\frac{\log M}{6c^+}\right]$ and for $i>0$, 
%$\aleph(i):=\left[\frac{c}{6c^+}(i+M)\right]$, where $c^+:=\log 5$.\index{$aleph$@$\aleph$}\index{$c^+$}
%Such a condition implies that every parabolic pieces is hyperbolic (see Prop. \ref{Proph} below).
%
%
%The condition $(\diamondsuit)$ does hold if the sequence $\sc=(\sa_i)$ involved in $(SR_1)$ satisfies $(\blacklozenge\blacklozenge)$:
% 
%\begin{defi}\index{Common}
%A \emph{common sequence} $\sc=(\sa_i)_i$ is a sequence of puzzle pieces which  satisfies
%\begin{equation}\tag{$\blacklozenge$}
%\sum_{j\le i\; \sa_j\notin \sY_0} n_{\sa_j} \le e^{-\sqrt M} \sum_{j\le i-1} n_{\sa_j},\quad \forall j\le i.
% \end{equation}
%Moreover every pieces $\sa_i(S^i)$ is either simple or included in $\R_\square$ and for every $i\ge 0$,
%\begin{equation}\tag{$\blacklozenge\blacklozenge$} 
%\sa_i\star \cdots\star \sa_{i+\aleph (i)}\notin \{s_-^{ \aleph(i)},s_+^{ \aleph(i)}\}\; .
%\end{equation}
%
%\end{defi}
%
%
%
%
%\begin{defi} The quadratic map $P$ is \emph{strongly regular} if there exists a common sequence $\sc=(\sa_i)_{i\ge 1}$ so that:
%\begin{equation}\tag{$SR_1$}
%P^{M+1}(0)\in \cap_{k\ge 0}\R_{\sc_k},\quad \text{with } \sc_k=\sa_1\star \cdots \star \sa_k.\end{equation}
%\begin{equation}\tag{$SR_2$}
%\text{Every puzzle piece $\sa_k$ is prime.}\end{equation}
%\end{defi}
%
%As announced, we have:
%\begin{prop}[Prop 1.3 and 4.1 \cite{berhen}]\label{Proph}
%If $P$ is strongly regular, then every simple piece and parabolic piece is hyperbolic. 
%\end{prop}
%As every puzzle pieces is a $\star$-product of parabolic and simple pieces, it comes:
%\begin{cor}
%If $P$ is strongly regular, then every puzzle piece is hyperbolic.
%\end{cor}
%As for Yoccoz definition, this implies:
%\begin{cor}\label{CEP} 
%If $P$ is strongly regular, then it satisfies the Collet-Eckmann  Condition $(\mathcal {CE})$. 
%\end{cor}
%
%
%\subsection{Strongly regular H\'enon-like endomorphisms}
%
%We now consider a $C^2$-map $f:=f_{a\, B} \colon (x,y)\mapsto (P(x)+y, 0)+ B_a(x,y)$ satisfying that:
%\begin{itemize} 
%\item the parameter $a>-2$ is close to $-2$, so that the first return time $M$ of $a$ by $P$ in $\R_\se$ is large.
%\item A real number $b>0$ small w.r.t. $|a+2|$ (and is even small w.r.t. $e^{-e^{e^M}}$), which bounds the $C^0$-norm of $\det\; Df$ and the $C^2$-norm of $(x,y,a)\in [-3,3]^2\times \R\mapsto  B_a(x,y)$.\index{$b$}
%\end{itemize}
%Put $\theta:= |\log\, b|^{-1}$. \index{$\theta$} We notice that $\theta$ is small w.r.t. $e^{-e^M}$.
% 
%We observe that $f$ is $b$-close to $\hat P:=(x,y)\mapsto (x^2+a+y,0)$ which preserves the line $\R\times \{0\}$ and whose restriction therein is equal to the quadratic map $P$.  Hence, for $b$ small,  the fixed point $(A_0,0)$ for $\hat P$ persists as a fixed point $A$ of $f$. \index{$A$}
%
%The strong regularity condition is related to the topology of the homoclinic tangle of $W^s(A; f)\cup  W^u(A; f)$.
%
%To formalize this we generalize the definition of puzzle pieces for \emph{flat curves} that we will define in the sequel.  
%
%First let us notice that the (compact) local stable manifold $\{(x,y)\in\R\times [-1,\infty):  x^2+a= A_0\}$ persists as a local stable manifold $W^s_{loc}(A; f)$ for $f_{a\;B}$. With the line $\{y=2\theta\}$ and the line $\{y=-2\theta\}$, the local stable manifold  $W^s_{loc}(A; f)$ bounds a compact set diffeomorphic to a filled square denoted by $Y_\se$ (see fig. \ref{geometricmodele}).
%
%Let us denote by $\partial^s Y_\se:= Y_\se\cap W^s_{loc}(A; f)$ and 
%$\partial^u Y_\se:= Y_\se\cap \{y=\pm 2\theta\} $. 
%
%Both sets consists of two connected curves whose union is $\partial Y_\se$.
%% \index{$Y_\se$}
%
%\begin{defi}[flat stretched curve]
%A   curve $S\subset Y_\se$ is \emph{flat} if it is the graph of a $C^{1+Lip}$-function $\rho$ over an interval $I\subset \R$, with $C^{1+Lip}$-norm at most\footnote{Actually, in \cite{berhen}, we ask the flat stretched curves to be the image by a certain map $y_\se$ of a graph of a function satisfying such bounds. Nevertheless the map $y_\se$ has its $C^{1+Lip}$-norm bounded and its inverse has its $C^{1+Lip}$-norm bounded by $\theta^{-1}$. Moreover all bounds on the graph transforms will have sufficiently room so that this does not change the statement of the propositions involving the flat curves.}   $\theta$.
%\[\|\rho\|_{C^0}\le\theta,\quad \|D\rho\|_{C^0}\le\theta,
%\quad \|Lip(D\rho)\|_{C^0}\le\theta\;.\]
% 
%The flat curve $S$ is \emph{stretched} if it is included in $Y_\se$ and satisfies that $\partial S\subset \partial^s Y_\se$. 
%\end{defi}\index{Flat stretched curve}
%
%\subsubsection{Puzzle pieces}
%A puzzle piece is always associated to a flat stretched curve $S$. 
%\begin{defi} 
%A \emph{ puzzle piece} \index{Puzzle piece}
% $\sa(S)$ of $S$ is the data of:
%\begin{itemize}
%\item an integer $n_\sa$ called the \emph{order of the puzzle piece}\index{Puzzle piece2@Order of a puzzle piece},
%\item a segment $S_\sa$ of $S$ sent bijectively by $f^{n_\sa}$ to a flat stretched curve $S^\sa$.\end{itemize} 
%For instance  $\se(S):=\{S, 0\}$ is a puzzle piece called \emph{neutral}. 
%
%A piece $\sa(S)=\{S_\sa,n_\sa\}$ is \emph{hyperbolic} if the following conditions hold:
%\paragraph{\emph{h-times}}\index{h-times} For every $z\in S_\sa$, $w \in T_z S_\sa$ and every $l\le n_\sa$: $\|D_{z}f^{n_\sa}(w)\|\ge e^{\frac{c}{3} (n_\sa-l)}\cdot \|D_zf^l({w})\|$.
%\end{defi}\index{h-times} \index{Hyperbolic piece}
%We recall that $c=\log\, 2/2$.
%
%In order to define the simple puzzle piece, we assumed -- in the one dimensional case -- that  $P^{M+1}(0)$ does not belong to  $\R_{s_-^{ \aleph(0)}}\sqcup 
%\R_{s_+^{ \aleph(0)}}$. Hence the following $P$-forward invariant compact set:
%$$K:=\{A_0',-A_0'\}\cup \bigcup_{i\ge 0} \{A_i, -A_i \}\cup \bigcup_{\ss\in \sY_0} \partial \R_s$$ is at bounded distance from $0$ and so is uniformly expanding for $P$. 
%
%We remark that the set $K\times\{0\}$ is uniformly hyperbolic for $\hat P$. For $z_0=(x_0,0)\in K\times \{0\}$, put:
%%the component $$ containing $z_0$ of  
%\[W^s_{loc}(z_0;\hat P):= \{(x,y)\in\R\times [-1,\infty)\colon x^2+y=x_0^2\}\; .\]
%It is a local stable manifold of $z_0$ and its shape is an arc of parabola.
%%A local stable manifolds of $(x_0,0)\in K\times \{0\}$ is the component $W^s_{loc}(x_0; \hat P)$ containing $(x_0,0)$ of 
%%
%%
%
%By hyperbolic continuity, for $b$ sufficiently small, the family of curves $(W^s_{loc}(z_0))_{z_0\in K\times \{0\}}$ persists as a family $(W^s_{loc}(z_0; f))_{x_0\in K\times \{0\}}$ so that:
%\begin{equation}\label{lamination} f (W^s_{loc}(z_0; f))\subset W^s_{loc}(\hat P(z_0); f)\; .\end{equation}
%
%
%%The idea is to extend the $\star$-product and the $\square$-product from segment of $\R$ to families of curves close to $S_\se\times \{0\}$. 
%
%
%%Let $\theta= 1/|\log b|$, let $U$ be the $3\theta$-neighborhood of $[a, f_a(a)]\times\{0\}$ in $\R^2$. We assume $b$ so small once $M$ is fixed that 
%%$f$ sends $U$ into itself.
%
%
%
%
%Also for every $\sa\in \sY_0\cup\{e, \square\}$, the endpoints $(x_-,x_+)$ of $\R_\sa$ belong to $K$, and the curves $W^s((x_\pm,0); f)$ are sufficiently close to 
%$W^s((x_\pm,0); \hat P)$ so that they cross the strip $\R\times [-2\theta,2\theta]$ to bound a compact set $Y_\sa$ close to  $S_\sa\times \{0\}$  and diffeomorphic to a filled square (see Fig. \ref{geometricmodele}).  The set $Y_\sa$ is called the \emph{box}\footnote{Also called simple extension in \cite{berhen}.} associate to $\sa$.\index{Box}\index{$Y_\square$}\index{$Y_\se$}
%
%
% 
%\begin{figure}[h]
%    \centering
%        \includegraphics{geometricmodel3.pdf}
%    \caption{Geometric model for some parameters of the H\'enon map. }
%\label{geometricmodele}
%\end{figure}
%
%
%
%Let $\partial^uY_\sa:= Y_\sa\cap \{y=\pm 2\theta\}$ and let $\partial^s Y_\sa :=  Y_\sa \cap \cup_{\pm} W^s((x_\pm,0)  ; f)$. 
%
%We notice that by (\ref{lamination}), it holds that $f^{n_\sa }( \partial^s Y_\sa) \subset  \partial^s Y_\se$, as depicted by Fig. \ref{geometricmodele}. \\
%
%\begin{defi}[Simple pieces] As before, $\ss\in \sY_0$ denotes a simple symbol.
%For every flat stretched curve $S$, $S_\ss:=S\cap Y_\ss$. The pair $\ss(S):= \{ S_\ss,n_\ss\}$ is called a \emph{simple piece} with image the curve $S^\ss=f^{n_\ss}(S_\ss)$.
% 
% In \cite{berhen} Expl. 2.2, we show that $\ss(S)$ is a \emph{puzzle piece which is hyperbolic}.
% \label{henonsimple}
%\end{defi}
%\begin{exam}[Curves $S^{{t\! t}}$ and $(S^{t})_{t\in T_0^{\mathbb Z^-}}$]\label{T0}
%We recall that $\ss_-$ is the simple symbol so that $ Y_{\ss_-}$ contains the fixed point $A$ (and is at the right hand side of its stable manifold).
%The map $S\mapsto S^{\ss_-}$ from the space of flat stretched curves into itself is well defined and $C^1$-contracting in the space of flat stretched curves.  
%We denote by $S^{t\!t}$ its fixed point, with ${t\! t}$ denoting the constant pre-sequence $(\ss_-)_{i\le -1}\in \sY_0^{\mathbb Z^-}$. It is a half local unstable manifold of the fixed point $A$. 
% We have  also $S^{{t\! t}} =\{ z_0\in Y_\se:\; \exists (z_i)_{i\le -1}\in Y_{\ss_-}^{\mathbb Z^-}, \; z_{i+1}= f^2(z_i)\}$ since the order of $\ss_-$ is $2$.
%
%
%More generally, for $t= (\sa_i)_{i\le-1}\in \sY_0^{\mathbb Z^-}$, the set:
%\[S^{t} =\{ z_0\in Y_\se:\; \exists (z_i)_{i\le -1}\in \prod_i Y_{\sa_i}, \; z_{i+1}= f^{n_{\sa_i}}(z_i)\},\]
%is a flat stretched curve. We put $T_0:=\sY_0^{\mathbb Z^-}$. They define the family of curves $(S^t)_{t\in T_0}$. 
% \end{exam}
% 
%
%%\index{$S^{t\! t}$}
%\subsubsection{Operation $\star $ on puzzle pieces}
%Similarly to the one-dimensional case, let us define the operation $\star$ on puzzle pieces of flat stretched curves.\index{$T_0$}
%
%\begin{defi}[{Operation $\star$ on puzzle pieces}]
%Let $\sa(S):= \{S_\sa, n_\sa\}$ and $\sb(S')= \{S'_\sb, n_\sb\}$ be two puzzle pieces of $S$ and $S'$ respectively. If $S'$ is equal to $S^\sa = f^{n_\sa}(S_\sa)$, then the pair of puzzle pieces  $(\sa(S),\sb(S^\sa))$ is called \emph{suitable}.
%Then we can define the puzzle  piece $\sa\star\sb(S)=\{S_{\alpha\star \beta} , n_{\alpha\star\beta}\} $ of $S$ with:
%\[S_{\alpha\star \beta} = f^{-n_\sa}(S'_\sb)\cap S_\sa\quad \text{and}\quad 
%n_{\alpha\star\beta} = n_\alpha+n_\beta\; .\]
%
%%equal to 
%%\[\sa\star\sb(S):=\{S_{\alpha\star \beta} , n_{\alpha\star\beta}\}\; . \]
%%\{f^{-n_\sa}(S'_\sb)\cap S_\sa, n_\sa+n_\sb\}.\]
%Indeed the map $f^{n_{\sa\star \sb}}|S_{\sa\star \sb}$ is a bijection onto $S^{\sa\star \sb}:= f^{n_{\sa\star \sb}}(S_{\sa\star \sb})$. 
%\end{defi}
%%The pair of puzzle pieces $(\sa(S),\sb(S^\sa))$ is called \emph{suitable}.
% More generally, a sequence $(\sa^i(S^i))_{1\le i< k}$, for $k\in \N\cup\{\infty\}$  is called suitable if the pair of any two consecutive puzzle pieces is  suitable. 
%We can now generalize condition $(\star)$ of Yoccoz' strong regularity definition. \index{Suitable}
%\begin{defi}[Common sequence]
%For $N\in [1,\infty]$, a \emph{common sequence} \index{Common sequence} $\sc$ is a suitable  sequence of  hyperbolic puzzle pieces $\sc := (\sa_i(S^i ))_{i= 1}^{N-1}$ from $S^1:=S^{t\!t}$ which satisfies the following properties:
%\begin{equation}\tag{$\blacklozenge$}
%\sum_{j\le i\; \sa_j\notin \sY_0} n_{\sa_j} \le e^{-\sqrt M} \sum_{j\le i-1} n_{\sa_j},\quad i< N.
% \end{equation}
%Moreover every pieces $\sa_i(S^i )$ is either simple or included in $Y_\square$, and for every $i\ge 0$, 
%\begin{equation}\tag{$\blacklozenge\blacklozenge$}
%\sa_{i+1}\star\cdots\star \sa_{i+\aleph(i)}\notin \{  s_-^{ \aleph(i)}, s_+^{ \aleph(i)}\}
%\end{equation}
%\end{defi}
%
%The product $\sc_i := \sa_1\star \sa_2\star \cdots \star\sa_{i-1}\star \sa_i$ 
%is called a \emph{common product of depth $i$} \index{Common product of depth $i$}
% and it defines a pair $\sc_i(S^{t\!t})=:\{ S^{t\!t}_{\sc_i},n_{\sc_i}\}$ called a \emph{common piece}. \index{Common piece}
%
%A \emph{common piece of order $0$} is the pair equal to $\sc_0(S^{t\!t}):=\{ S^{t\!t}, 0\}= \se(S^{t\!t})$.
%
%
%Not all the puzzle pieces have their endpoints with a nice local stable manifold. Nevertheless it is the case for the common piece:
%\begin{prop}[\cite{berhen} Prop. 3.6 ]
%Each endpoint $z_\pm$ of $S^{t\!t}_{\sc_i}$ has a local stable manifold $W^s_{loc} (z_\pm; f)$ which stretches across $Y_\se$ and is $\sqrt b $-$C^2$-close to an arc of curve of the form:
%\[\{(x,y)\colon x^2+y = cst\}.\]
%With the lines $\{y=\pm2 \theta\}$, this bounds a box of $Y_\se$ denoted by $Y_{\sc_i}$. Moreover, for every $z\in Y_{\sc_i}$, the vector $Df^{n_{\sc_i}} (1,0)$ is $\theta$-close to be horizontal and of norm at least $e^{c^- n_{\sc_i}}$, with:
%\[c^-=c-\frac 1{\sqrt M}=\frac {\log 2}2-\frac 1{\sqrt M}\; .\]\index{$c^-$}
%\end{prop}
%We put $\partial^s Y_{ \sc_i}=  \cup_\pm W^s_{loc} (z_\pm; f)\cap Y_\se$ and $\partial^u Y_{ \sc_i} = \partial^u Y_\se \cap Y_{ \sc_i}$.
%
%By the above Proposition, the width of $Y_{\sc_i}$ is smaller than $2e^{-c^- n_{\sc_i}}$ times the width of $Y_\se$ and so, if $N=\infty$, the following decreasing intersection:
%\[W^s_{\sc} := \cap_{i\ge 0} Y_{ \sc_i}\;.\]
%is a $C^{1+Lip}$-curve called \emph{common stable manifold}, which is $\sqrt b$-$C^{1+Lip}$-close to  an arc of a curve of the form:
%\[\{(x,y)\colon x^2+y = cst\}.\]
% 
%
%%\begin{rema}
%%We proved that for the maps we will define (strongly regular map), the box $Y_{ \sc_i}$ do not depend on $S^0$. Actually, for any flat stretched curve  $S$, the pair $\sc_i(S):= \{S\cap  Y_{\sc_i}, n_{\sc_i}\}$ is a hyperbolic puzzle piece.  
%%\end{rema}
%
%\subsubsection{Tangency condition}
%
%Every flat stretched curve $S$ intersects $Y_\square$ at a segment $S_\square=S\cap Y_\square$. This segment is sent by $f^{M+1}$ to
%a curve $S^\square$ which is $C^2$-close to a folded curve $ \{(-Cst\cdot 4^M t^2+f^{M}_a(a), 0): t\in \R\}\cap Y_\se$.
%
%The definition of strong regularity for H\'enon-like maps supposes the existence of a family of curves $(S^t)_{t\in T^*}$ so that for each $t\in  T^*$, there exists a common sequence of puzzle pieces $\sc^t$ so that\index{$T^*$}
%
%\paragraph{$(SR_1)$} $S^{t\square}= f^{M+1} (S^t_\square)$ is tangent to $W^s_{\sc^t}$.
%
%As in dimension $1$, conditions are given on the puzzle pieces involved in the common sequences. In this two dimensional case, conditions are moreover given on the flat and stretched curves forming   $(S^t)_{t\in T^*}$.
%
%\subsubsection{Parabolic operations from tangencies}
%As in dimension $1$, if a flat stretched curve $S$ satisfies that $S^\square $ is tangent to a common stable manifold $W^s_{\sc^t}$, then we can define parabolic pieces.
%
%Indeed, then for every $i$, 
%$(f^{M+1}|S_\square)^{-1}cl(Y_{\sc_i}\setminus Y_{\sc_{i+1}})$ consists of zero or two segments. 
%
%In the latter case, we denote by $S_{\square_-(\sc_i -\sc_{i+1})}$ the left hand side segment and by $S_{\square_+(\sc_i -\sc_{i+1})}$ the right hand side segment. Furthermore, with $\sp$ a symbol in $\{\square_+(\sc_i -\sc_{i+1}),\square_-(\sc_i -\sc_{i+1})\}$ and $n_{\sp} := M+1+n_{\sc_i}$, the pair $\sp(S):=\{S_{\sp}, n_{\sp}\}$ is called a \emph{parabolic piece}. 
%
%
%%The parabolic pieces $\square_-(\sc_i -\sc_{i+1})(S)= 
%%\{S_{\square_-(\sc_i -\sc_{i+1})}, n_{\square_-(\sc_i -\sc_{i+1})}\}$ and 
%%$\square_+(\sc_i -\sc_{i+1})(S)= 
%%\{S_{\square_+(\sc_i -\sc_{i+1})}, n_{\square_-(\sc_i -\sc_{i+1})}\}$ are well defined with order $n_{\square_-(\sc_i -\sc_{i+1})}=n_{\square_-(\sc_i -\sc_{i+1})}=M+1+n_{\sc_i}$. 
%%
%% 
%%
%%Let $\sp$ be a symbol in $\{\square_+(\sc_i -\sc_{i+1}),\square_-(\sc_i -\sc_{i+1})\}$.
%%\begin{defi}[Symbolic identification]
%%The symbols $\square_+(\sc_i -\sc_{i+1})$ and $\square_-(\sc_i -\sc_{i+1})$ depend only on $Y_{\sc_i}$ and $Y_{\sc_{i+1}}$.
%%
%%In particular if for $t\not=t'$ it holds $Y_{\sc_i^t}=Y_{\sc_i^{t'}}$ and 
%%$Y_{\sc_{i+1}^t}=Y_{\sc_{i+1}^{t'}}$, then the following identifications are done $\square_+(\sc^t_i -\sc^t_{i+1})=\square_+(\sc^{t'}_i -\sc^{t'}_{i+1})$ and $\square_-(\sc^t_i -\sc^t_{i+1})=\square_-(\sc^{t'}_i -\sc^{t'}_{i+1})$.
%%\end{defi}
%%
%%
%%With $n_{\sp}=M+1+n_{\sc_i}$, the pair $\sp(S):=\{S_{\sp}, n_{\sp}\}$ is called a \emph{parabolic piece}. 
%
%  
%This pair $\sp(S)$ cannot be a puzzle piece since the curve $f^{n_\sp}(S_\sp)$ is not stretched (like in the one dimensional model).
% 
%However the curve $f^{n_\sp}(S_\sp)$ can be extended to a flat stretched curve $S^\sp$ by an algorithm given by Prop. 4.8 and 5.1 in \cite{berhen}. In particular $S^\sp\supsetneq f^{n\sp} (S_{\sp})$.
%
%
%
%
%
%
%
%\begin{defi}[Set of symbols $\sA$]
%Let $f$ which satisfies $(SR_1)$ with the flat stretched curves $(S^t)_{t\in T^*}$ and the common sequences $(\sc^t)_{t\in   T^*}$.
%
%Let 
%\[\sA:= \sY_0\cup \bigcup_{t\in  T^*}\bigcup_{i\ge 0} \{\square_+ (\sc^t_i -\sc^t_{i+1}), \square_- (\sc^t_i -\sc^t_{i+1}) \}\;.\]
%\end{defi}
%The above union over $t\in T^*$ is not disjoint by the above remark.
%As they are countably many puzzle pieces of $S^{t\! t}$, they are countably many common pieces $\sc_i$ and boxes $Y_{\sc_i}$. Thus $\sA$ is countable.
%\begin{prop}[Prop. 1.7 and 4.1 of \cite{berhen}]\label{hypersA}
%For every $t\in T^*$, every parabolic or simple piece  $\sa(S^t)$,  with $\sa\in \sA$, is hyperbolic.
%\end{prop}
%
%\begin{defi}[Suitable chain]
%Let $(S^{i})_{i=1}^n$ be a family of flat stretched curves and let $(W^s_{\sc^i})_{i=1}^n$ be a family of  common stable manifolds so that $S^{i\square }$ is tangent to  $W^s_{\sc^i}$.
% 
%For each $i$ let $\sp_i$ be a symbol either in $\sY_0$ , or parabolic obtained from $\sc^i$ (that is of the form $\square_\pm (\sc_j^i -\sc_{j+1}^i)$). 
%
%The chain of symbols $(\sp_i)_{i=1}^n$ is called suitable from $S^{1}$ if:
%\begin{enumerate}
%\item $S^{{i+1}}=(S^i)^{\sp_i}$ for every $i<n$,
%\item The segment of the pair $\sp_1(S^{1})\star \cdots \star \sp_n(S^{n})$ is not trivial (it has cardinality $>1$).
%\end{enumerate}
%The chain of symbols is complete if $\sp_n$ belongs to $\sY_0$, and incomplete otherwise. 
%The chain of symbols $(\sp_i)_i$ is \emph{prime} if $\sp_i\notin \sY_0$ for $i<n$.\index{Prime} \index{Complete}
%\end{defi}
%A corollary of Proposition \ref{hypersA} is:
%\begin{coro}
%If $(\sp_i)_{i=1}^n$ is suitable, then $\sp_1(S^{t_1})\star \cdots \star \sp_n(S^{t_n})$ is a hyperbolic piece of $S^{t_1}$. 
%\end{coro}
%
%By using that the puzzle pieces are nested or disjoint, one shows:
%\begin{prop}\label{lapropafaire}
%If  $(\sp_i)_{i=1}^n$ is suitable and complete , then  $\sp_1(S^{t_1})\star \cdots \star \sp_n(S^{t_n})$ is a hyperbolic puzzle piece of $S^{t_1}$.\end{prop}
%
%\subsubsection{Puzzle algebra and strong regularity definition}
%
%
%In Example \ref{T0}, we defined for every $t\in T_0:= \sY_0^{\mathbb Z^-}$ a flat stretched curve $S^t$. Every symbol $t$ is a pre-sequence of symbols $t=(\ss_i)_{i\le -1}$.  Given a chain of symbols $\sa_{-k}, \dots, \sa_{-1}$, let $t\cdot \sa_{-k}\cdots \sa_{-1}$ be the pre-sequence of symbols $(\sa_i')_{i\le -1}$ with 
%$\sa_{-i}'=\sa_{-i}$ for every $0\le i\le k$ and $(\sa_i')_{i\le -k-1}= t$.
%
%
%
%In \cite{berhen}, for a set of parameters $a\in P_B$ of Lebesgue measure positive,  we show the existence of a family of curves $(S^t)_{t\in T^*}$ and a family of common sequences $C= (\sc^t)_{t\in T^*}$ which are linked in the following way by the tangency condition and parabolic/simple operations. 
%
%\begin{enumerate}[$(SR_1)$]
%\item $S^{t\square}= f^{M+1} (S^t\cap Y_\square)$ is tangent to $W^s_{\sc^t}$.
%
%\[ \text{Put}\quad \sA:= \sY_0\cup \bigcup_{t\in  T^*,\quad i\ge 0} \{\square_+ (\sc^t_i -\sc^t_{i+1}), \square_- (\sc^t_i -\sc^t_{i+1}) \}\]
%%\; \text{modulo the symbolic identification}.\]
%\item For every $t\in T^*$, every puzzle piece $\sa_i(S^i)$ involved in $\sc^t=(\sa_i(S^i))_i$ is given by suitable, complete and prime chain of symbols $\underline \sa_i$ in $\sA^{(\N)}$.  
%\item The set $T^*$ is the subset of $\sA^{\mathbb Z^-}$ defined by 
%$$T^*= \{t\cdot \sp_{-n} \cdots \sp_{-1}:\; t\in T_0,\;  n\ge 0,  \; (\sp_i)_{1\le i\le n} \in \sA^n\text{ is a suitable chain from }S^{t}\}.$$ 
%For $t^*= t\cdot \sp_{-n} \cdots \sp_{-1}\in T^*$, we put  $S^{t^*}= (\cdots (S^{t})^{ \sp_{-n}} \cdots )^{\sp_{-1}}$. 
%
%\end{enumerate}
%\begin{rema}
%In ($SR_3$) the element $t\cdot \sp_{-n} \cdots \sp_{-1}$ is equal to the pre-sequence $(\sa_i)_{i\le -1}\in \sA^{\mathbb Z^-}$ defined by
% $\sa_{-i}:=\sp_{-i}$ if $1\le i\le n$, and, with $t=(\ss_i)_{i\le -1}\in T^0$,
%  $\sa_{-i}:\ss_{-i+n}$ if   $i\ge n+1$.
%\end{rema}
%
%\begin{defi}
%A map $f$ so that there exists a family of flat stretched curves $(S^t)_{t\in T^*}$ and a family of common sequences $(\sc^t)_{t\in T^*}$ satisfying  $(SR_1-SR_2-SR_3)$ 
%%$(SR_0-SR_1-SR_2-SR_3-SR_4)$
% is called \emph{strongly regular}. 
% \end{defi}
%\begin{defi}\index{$\sG$}
%Let $\sG$ be the set of finite segments of sequences in $T^*\subset 
%\sA^{\Z^-}$.
%It enjoys a structure of pseudo-monoid structure for the concatenation law $\cdot$ .  
%The parabolic operation $\square$ is another law on $\sG$.  The triplet $(\sG, \cdot , \square)$ is called a \emph{Puzzle Algebra}.
%\footnote{In \cite{berhen}, the presentation of strong regularity is different: The  set $T^*$ is presented as the disjoint union of the sets $T$ and $T^\square$, formed by the pre-sequences $t\star \sp_{-n}\star \cdots\star \sp_{-1}\in T^*$ which finish by  respectively a simple piece or a parabolic piece.
%This splits the family of curves $(S^t)_{t\in T*}$ into two subfamilies  $\Sigma= (S^t)_{t\in T}$ and $\Sigma^\square= (S^t)_{t\in T^\square}$. 
%Furthermore, the set of prime puzzle pieces of a curve $S^t$, with $t\in T$, is denoted therein by $\mathcal Y(t)$. We define also $\mathcal Y:= \sqcup_{t\in T} \mathcal Y(t)$. The quadruplet $(\Sigma,\Sigma^\square,C,\mathcal Y)$ is called a puzzle algebra. This is equivalent to the above definition.
%}  
%
%%The triplet $(\sA^{(\mathbb N)},  \star, \square)$ is 
%%The triplet data $((S^t)_{t\in T\sqcup T^\square}, 
%%\mathcal Y= \sqcup_{t} \mathcal Y(t), C=(\sc^t)_{t\in T\sqcup T^\square})$ is called a \emph{puzzle algebra}.
%\end{defi}
%
%%\begin{theo} For $B$ sufficiently $C^2$-small, the set  $\Lambda_B$ of parameters $a$ for which $f$ is strongly regular has a positive Lebesgue measure:
% %$$\leb(\Lambda_B)>0$$ 
%%\end{theo}
%
%
%
%
%
%%For $\sa\in \mathcal Y(t)$, the chain of symbols $\underline a$ such that $\sa=\underline a(S^t)$ is called the $\sA$-spelling of $\sa$.
%
%The main result of \cite{berhen} (Theorem 0.1) is the following:
%\begin{thm} Every strongly regular map leaves invariant an ergodic, physical SRB measure supported by 
%a non-uniformly hyperbolic attractor. Moreover, strongly regular maps are abundant in the following meaning:
%
%For every $\epsilon>0$, there exists $b>0$, such that for every $B$ of $C^2$ norm less than $b$, there exist $\eta>0$ and a subset $\Pi_B\subset [-2,-2+\eta]$ with $\frac{\leb \Pi_B}{\leb [-2,-2+\eta]}\ge 1-\epsilon$  such that for every $a\in \Pi_B$, the map $f_{a\, B}$ is strongly regular.
%\end{thm}
%\begin{rema}\label{violent} To fix the idea, we will suppose  the following very rough inequalities:
% $M\ge 1000$ and $-\log b\le \exp\, \exp M$.  They are sufficient for the new analytic conditions given by this work. 
% \end{rema}
% 
% \section{Parameter selection}
%\label{parametersection}
%The main theorem of \cite{berhen} is the first to take place in the $C^2$-topology (\cite{YW08} works for $C^3$-mappings at the $C^2$-neighborhood of the H\'enon map). Also it is the first one which gives the SRB construction for the endomorphism case. 
%
%I would like to shed light on another aspect of this work. How does  the combinatorial formalism enable us to handle the parameter selection, and especially how does it enable us to follow rigorously the structure when the parameter varies. 
%
%This will be the occasion to introduce our work with Moreira \cite{BM13} on nested Cantor sets; it might be helpful to study how large the Haudsdorff dimension of an abundant, strongly regular attractor can be.
%
%\subsection{$k$-Strongly regular maps and their combinatorial classes}
%
%\subsubsection{The quadratic map case}
%Let us go back to our variation of Yoccoz' definition of strongly regular quadratic maps. 
%
%To define a $k$-strongly regular map, we shall ask for the existence of 
%a common sequence so that $\sc=(\sa_i)_{k\ge i\ge 1}$ 
%$P^{M+1}(0)$ belongs to $\R_{\sc_j}$ for every $j\le k$, with $\sc_j=\sa_1\star \cdots \star \sa_k$. Nevertheless this does not imply the hyperbolicity of the parabolic pieces $\square_\pm(\sc_{k-1}-\sc_k)$ (for instance, the point $0$ belongs to this piece if the critical value $P^{M+1}(0)$ belongs to the boundary of $\R_{\sc_k}$). 
%
%Hence, we shall assume that $P^{M+1}(0)$ does not belong to $\R_{\sc_k\star \ss^{ \aleph (k)}_-}\sqcup \R_{\sc_k\star \ss^{ \aleph (k)}_+}$.
%This leads to the following definition:
%
%\begin{defi}[$k$-strongly regular quadratic maps]
%The quadratic map $P$ is \emph{$k$-strongly regular} if there exists a common sequence $\sc=(\sa_i)_{k\ge i\ge 1}$ so that:
%\begin{equation}\tag{$SR_1$}
%P^{M+1}(0)\in \cap_{k \ge j\ge 1}(\R_{\sc_j}\setminus (\R_{\sc_j\star \ss^{ \aleph (j)}_-}\sqcup \R_{\sc_j\star \ss^{ \aleph (j)}_+}) ,\quad \text{with } \sc_j=\sa_1\star \cdots \star \sa_k.
%\end{equation}
%We define the set of symbols $\sA_j$ for $j\le k-1$ as follows:
%\[\sA_0:= \sY_0,\quad \sA_{j+1}= \sA_j\cup\{\square_+(\sc_j-\sc_{j+1}), \square_-(\sc_j-\sc_{j+1})\}\; .\]
%\begin{equation}\tag{$SR_2$} \forall 1\le j\le k
%\text{ the puzzle piece $\sa_j$ is given by a prime, complete sequence of symbols in $\sA_{j-1}$.}\end{equation}
%\end{defi}
%We notice that a $k$-strongly regular map is $k'$-strongly regular for every $k'\le k$. Also the strongly regular maps are exactly the $\infty$-strongly regular maps. 
%
%
%
%
%
%Let $\sG_k$ be the set of finite segments of sequences in $T^*_k\subset 
%\sA^{\Z^-}_k$.
%It enjoys a structure of pseudo-monoid structure for the concatenation law `` $\cdot$ ".  
%
%\begin{fact}\label{uniquelydefined}
%The alphabets $(\sA_j)_{j\le k}$ and the pieces $(\sa_j)_{j\le k}$ are uniquely defined.
%\end{fact}
%\begin{proof}
%This is done by induction on $j\le k$. For $j=0$, we recall that $\sa_0=\se$ and $\sA_0=\sY_0$. Assume the uniqueness for $j<k$. By $(SR_2)$, $\sa_{j+1}$ is made by pieces in $\sA_j$ and so 
%%. By   $(\blacklozenge)$ the order of $\sa_{j+1}$ of at most 
%%$\exp(-\sqrt M) n_{\sc_j}\le \exp(-\sqrt M) (M+1) j\le j$. Hence  
%%$\sa_{j+1}$ must be a prime product of pieces in $\sA_j$, and so
%it is uniquely defined. This implies that $\sA_{j+1}$ is uniquely defined. 
%\end{proof}
%
%
%The common sequence definition is a priori not purely combinatorial since such a sequence is asked to be made of hyperbolic pieces. Nevertheless, the hyperbolicity assumption is automatic from $(SR_1)$ and $(SR_2)$. Indeed by $(SR_2)$ it suffices to show that every piece in $\sA_j$ is hyperbolic. This follows from an easy induction.
%\begin{fact}\label{hyperbolicautomatic} The hyperbolicity of the puzzle pieces involved in the common sequence of a $k$-strongly regular quadratic map is ``automatic". \end{fact}
%\begin{proof}
%Note first that $\sA_0= \sY_0$ is made by hyperbolic pieces (Prop. 1.3 \cite{berhen}). Let $0\le j\le k-1$, and assume that $\sA_j$ is made by hyperbolic pieces. To show that the pieces of $\sA_{j+1}$ are hyperbolic, it suffices to show that the pieces $\square_+(\sc_j-\sc_{j+1})$ and $\square_-(\sc_j-\sc_{j+1})$ are hyperbolic as done in Prop. 4.8 \cite{berhen}.\end{proof}
%
%
%
%Hence the $k$-strongly regular condition on quadratic maps is purely combinatorial and topological.  As given a $k$-strongly regular map $P$, the $k^{th}$-common piece $\sc_k$ is uniquely defined by Fact \ref{uniquelydefined}, we can set:
% \begin{defi}[Combinatorial interval]
%Let $w\subset \R$ be an interval so that $P_a$ is strongly regular for every $a\in w$, with  $(\sA_j(a))_{j\le k}$ as set of symbols and 
% $(\sc_j(a))_j$ as common sequences of depth $j\le k$. The interval $w$ is a combinatorial interval if $a\in w\mapsto \R_{\sc_j(a)}$ is continuous for every $j\le k$.
%
%Then all the sets $\sA_j(a)$ among $a\in w$ can be identified to a single set $\sA_j(w)$. 
%
%%Two $k$-strongly regular maps $P_a$ and $P_{a'}$ are in a same combinatorial class if for every $a_\lambda:= a+\lambda (a'-a)$ among $\lambda\in [0,1]$, 
%%the map $P_{a_\lambda}$ is $k$-strongly regular for a common piece $\sc_k(\lambda)$ of depth $k$, so that  $\lambda\mapsto \R_{\sc_k(\lambda)}$ depends continuously on $\lambda$. 
%\end{defi}
%
%\subsubsection{The H\'enon-like map case}
%We first need to substitute the notion of tangency between two curves by the notion of tangency between a curve and a box: 
%\begin{defi} A flat stretched curve $S$ is tangent to a common piece $\sc_k$ of finite depth if $S^{\square}= f^{M+1} (S_\square)$ intersects the interior of the box $Y_{\sc_k}$, and exactly one component of $\partial^s Y_{\sc_k}$.\end{defi}
%
%\begin{figure}[h]
%    \centering
%        \includegraphics{tangency_condition.pdf}
%%    \caption{Geometric model for some parameters of the H\'enon map. }
%\label{tangency_condition}
%\end{figure}
%
%A H\'enon-like map $f$ is \emph{$k$-strong regular} if there exists a family of curves $(S^t)_{t\in T^*_k}$ 
%and a family of common sequences $((\sa_j^t)_{1\le j\le k})_{t\in T^*_k}$, so that for every $j\le k$ with $\sc_j^t= \sa_1^t\star \cdots \sa^t_j$ it holds:
%\index{$T^*_k$}
%\begin{enumerate}[$(SR_1^k)$]
%\item $S^{t\square}= f^{M+1} (S^t_\square)$ is tangent to $Y_{\sc^t_j}$ but not tangent to $Y_{\sc^t_j\star \ss_+^{ \aleph(j)}}$ nor $Y_{\sc^t_j\star \ss_-^{ \aleph(j)}}$ .
%\[ \text{Put}\quad \sA_j:= \sY_0\cup \bigcup_{t\in  T^*,\quad j\ge i\ge 1} \{\square_+ (\sc^t_{i-1} -\sc^t_{i}), \square_- (\sc^t_{i-1} -\sc^t_{i}) \}\; .\]
%%\; \text{modulo the symbolic identification}.\]
%\item For every $t\in T^*_k$, for every $1\le j\le k$,
%the puzzle piece $\sa_j^t$ is given by a suitable, complete and prime chain of symbols in $\sA_{j-1}^{(\N)}$.  
%\item The set $T^*_k$ is the subset of $\sA^{\mathbb Z^-}_k$ defined by 
%$$T^*_k= \{t\cdot \sp_{-n} \cdots \sp_{-1}:\; t\in T_0,\;  n\ge 0,  \; (\sp_i)_{1\le i\le n} \in \sA^n_k\text{ is a suitable chain from }S^{t}\}.$$ 
%For $t^*= t\cdot \sp_{-n} \cdots \sp_{-1}\in T^*_k$, we put  $S^{t^*}= (\cdots (S^{t})^{ \sp_{-n}} \cdots )^{\sp_{-1}}$. 
%\end{enumerate}
%
%The triplet $(\sA_k, \square, \cdot)$ is called a \emph{$k$-puzzle algebra structure} for the strongly regular map $f$. 
%
%Note that $k$-strongly regular map is $k'$-strongly regular for every $k'<k$. 
%
%
%In Proposition 4.8 \cite{berhen} showed that:
%\begin{prop}\label{prop4.8}
%Given  a flat stretched curve $S$, $(\sa_i)_{i\le j}$, with 
%$\sc_{j-1} := \sa_1\star \cdots\star  \sa_{j-1}$ and  $\sc_j=\sc_{j-1}\cdot  \sa_j$, if the curve $S^\square= f^{M1}(S\cap Y_\square)$  is tangent to $Y_{\sc_j}$ and not $Y_{\sc_j\star \ss_+^{ \aleph(j)}}$ nor $Y_{\sc_j\star \ss_-^{ \aleph(j)}}$, then for every $\sp\in \{ \square_+ (\sc_{j-1}-\sc_j), \square_- (\sc_{j-1}-\sc_j)\}$, if $S_\sp$ is not empty, then $\sp(S)$ is a hyperbolic parabolic piece and $S^\sp$ is a well-defined flat, stretched curve.
%\end{prop}
%
%This enabled us to show (similarly to the one dimensional case):
%\begin{fact}[Prop. 5.11 \cite{berhen}]
%Every $k$-strongly regular H\'enon-like map $f$, the alphabets $(\sA_j)_{j\le k}$ and the pieces $(\sa_j^t)_{j\le k,\; t\in T_k^*}$ are uniquely defined. 
%\end{fact} 
%
%Proposition \ref{prop4.8} is very important since it implies that the hyperbolicity and the flatness involved in the strong regularity definition is automatic (among maps in a $C^2$-neighborhood of $(x^2-2+y,0)$ independent of $k$).  
% 
%Hence the $k$-strongly regular condition on H\'enon-like map is purely combinatorial and topological. This is crucial for the parameter selection, since it avoids us to check if the corresponding analytic estimates (flatness and hyperbolicity) are still satisfied for parameter deformations.
%
%Let us now consider the $C^2$-\emph{family} of maps $(f_a)_a$ of the form $f_a(x,y)=(x^2+a+y,0)+B(a,x,y)$, where $B$ is $C^2$-small. 
%
%
% \begin{defi}[Combinatorial interval]
%Let $w\subset \R$ be an interval so that $f_a$ is strongly regular for every $a\in w$ with structure $\sA_k(a)$, $T_k^*(a)$, $(\sc_k^t(a))_{t\in T_k^*(a)}$. 
% 
%The interval is \emph{$k$-combinatorial} if all the $\sA_k(a)$  are in bijection and can be identified to a finite set $\sA_k(w)$ so that:
%\begin{itemize}
%\item for every $a\in w$, Condition $(SR_3^k)$ and  $\sA_k(a)\approx \sA(w)$ define an inclusion and an identification:
%\[T^*_k(a)\subset \sA_k^{\Z^-}(a) \approx \sA_k^{\Z^-}(w)\; .\]
%The image $T^*_k(w)\subset \sA_k^{\Z^-}(w)$ of  $T^*_k(a)$ must be independent of $a$.
%\item for every $t\in T_k^*(w)$, the curve $S^t(a)$ depends continuously on $a\in w$.
%
%\item For every $a\in w$, Condition $(SR_2^k)$ states that for every $t\in T^*_k(a)\approx T^*_k(w)$ the puzzle pieces $\sa_i^t(a)$ involved in $\sc_j^t(a)= \sa_1^t(a)\star \cdots \star \sa_k^t(a)$  are in $\sA_k^{(\N)}(a)\approx \sA_k^{(\N)}(w)$. We ask that the image $\sa_i^t(w)\in \sA_k^{(\N)}(w)$ of $\sa_i^t(a)$ does not depend on $a\in w$.
%
%
%\item for every $t\in T_k^*(w)$ and $j\le k$, the box $Y_{\sc_j^t}(a)$ depends continuously on $a\in w$ for the Hausdorff distance.
% \end{itemize}
%\end{defi}
%
%We remark that a $k$-combinatorial interval is also a $k'$-combinatorial interval. This definition enables us to know which pieces and curves are well-defined along a parameter interval.
%  
%\subsection{Global geometric estimates from combinatory}
%
%In this section we show some geometric estimates on the $k$-puzzle algebras, which are satisfied by any $k$-strongly regular map and whose statements are combinatorial. They are crucial for the parameter selection. 
%
%\subsubsection{Lebesgue measure of the complement of the union of puzzle pieces of order $\le j$}
%
%\paragraph{The quadratic map case}
%
%Let $P$ be $k$-strongly regular. For $M< j\le M-1+2k$, let:
%\[\mathcal E_k^j:=\{x\in \R_\square : x\notin \R_\sa; \forall \sa\in \sA_k^{(\N)},\text{ suitable, complete sequence of symbols s.t. } \; n_{\sa}\le j\}\; .\]
%We notice that every point in $\R_\se\setminus \mathcal E_k^j$ belongs to the segment $\R_{\sa}$ of a complete, suitable  $\sA_k$'s chain $\sa$ of order $\le j$.
% By Propositions \ref{pour2564} and \ref{lapropafaire}, $\mathcal E_k^j$ consists of the points of $\R_\se$ which are not in a puzzle piece of order $\le j$ (which is necessarily hyperbolic by Fact \ref{hyperbolicautomatic}).
%
%Following Yoccoz, the following is a key estimate in his proof of Jakobson's theorem. 
%\begin{prop}\label{lebestimate}
%The Lebesgue measure of $\mathcal E_k^j$ is smaller than $e^{-\frac c4 j}$.
%\end{prop}
%\begin{proof}
%The proof is done by induction on $j\ge M$. For $j=M+1$, it holds
%$\mathcal E_k^j= \R_\square$ whose length is estimated  in \cite[Lemm. 12.2]{berhen}. 
%
%Let us assume the bound proved for $M<j<M-1+2k$. 
%Let $I$ be the maximal $i$ so that $M+1+n_{\sc_i}\le j-M$ if it exists, and $0$ otherwise. An easy computation \cite[(11) Prop. 6.15]{berhen} shows that
%the Lebesgue measure of $\R_{\sigma_I}:= (P^{M+1}|\R_\square)^{-1}(\R_{\sc_{I}})$ is smaller that $\frac12 e^{-\frac c4 j}$.
%
%We observe that 
%\[\mathcal E_k^{j+1} = \R_{\sigma_I} \cup \bigcup_{i=0}^{I-1}  \mathcal E_k^j \cap (\R_{\square_+ (\sc_i-\sc_{i+1})}\cup \R_{\square_- (\sc_i-\sc_{i+1})}) \; .\]
%The set $\mathcal E_k^j \cap \R_{\square_+ (\sc_i-\sc_{i+1})}$ is equal to 
%the preimage of $\mathcal E_k^{j-M-1-n_{\sc_i}}$ by $P^{ M+1+n_{\sc_i}}|\R_{\square_+ (\sc_i-\sc_{i+1})}$. By Proposition \ref{lapropafaire}, the hyperbolicity of the parabolic pieces and the induction hypothesis, it holds:
%\[\leb(\mathcal E_k^j \cap \R_{\square_+ (\sc_i-\sc_{i+1})})\le 
%\leb(\mathcal E_k^{j-n_{\square_+ (\sc_i-\sc_{i+1})}})e^{-\frac c3 n_{\square_+ (\sc_i-\sc_{i+1})} }\le
%e^{-\frac c4 j-\frac c{12}(i+M+1)}\; .\]
%
%Consequently, since $M$ is assumed large:
%\[\leb(\mathcal E_k^{j+1})\le  \frac12 e^{-\frac c4 j}+ e^{-\frac c4 j} \sum_{i=M+1}^{I-1}2\cdot  
%e^{-\frac c{12}i}
%\le e^{-\frac c4 j}\; .\]
%\end{proof}
%
%\paragraph{The H\'enon-like case}
%Let $f$ be a $k$-strongly regular H\'enon like map.
%
%The above upper bound on the measure of the unpuzzled set 
%holds true as well for the flat stretched curve $S^t$ of $f$. However, in contrast with the one dimensional case, the hyperbolicity of a puzzle piece does not imply a uniform distortion bound (indeed a curve may be dramatically folded along its orbit, even if it satisfies the hyperbolic time inequality). We will not detail this technical aspect of the proof (and we will not even state the uniform distortion estimate \cite[Def 2.8]{berhen}!). Let us just mention that a combinatorial way to obtain uniform distortion bound is to consider the \emph{perfect sequence}:
%\begin{defi}  Let $t\in T_k^*$. 
%A complete, suitable $\sA_k$-sequence $\sa_1\cdots \sa_m$ from $S^t$ is \emph{perfect} if for every $i$ such that $\sa_i\notin \sY_0$ it holds:
%\[\sum_{j=i+1}^m n_{\sa_j} \ge \lceil\frac{4c^+}c\rceil n_{\sa_i}\; .\] 
%\end{defi}
%As the perfect sequences are complete, they define puzzle pieces called \emph{perfect pieces}. We notice that the composition of two perfect sequences is a perfect sequence.
%
%For every $t\in T_k^*$, let $\mathcal E_k ^j (S^t)$ be the set of points which do not belong to a perfect piece of order at most $j$. 
%A development of Prop \ref{lebestimate} gives:
%\begin{prop}[Prop. 6.15 \cite{berhen}]
%For every $t\in T^*_k$, and $M\le j\le M-1+2k$, the following estimate holds true:
%\[\frac1j \log \leb\,  \mathcal E_k ^j (S^t)\le -\delta\; ,\quad \text{with } 
%\delta = \frac{c}{4(1+\lceil 4\frac{c^+}c\rceil}\; .
%\]
%\end{prop}
%\subsubsection{Hausdorff dimension}
%Let $f$ be a $k$-strongly regular H\'enon-like map. 
%We recall that $(\sG_k,\cdot) $ is the pseudo-monoid formed by the finite segments of pre-sequences in $T^*_k\subset \sA^{\Z^-}_k$.
%
%It is useful to define combinatorial metric $dist$ on $T_k^*$, so that $t\in T^*\mapsto S^t$ is Lipschitz, where the space of flat stretched curves is endowed with a uniform  $C^1$-distance. 
%
%Indeed, when the parameter varies along a $k$-combinatorial interval, this will enable us to see which curves are close to one another, for every parameter.   
%
%
%For this end, we introduced in \cite{berhen} the right divisibility $/$ on the elements of $\sG_k$. 
%
%\begin{defi}
%A word $\sa\in \sG_k$ is (right) divisible by $\sa'\in \sG_k$ and we note $\sa / \sa'$ if one of the following conditions holds:
%\begin{enumerate}[$(\mathcal D_1)$]
%\item  $\sa=  \sa'$ or $\sa'=\se$. 
%\item  $\sa$ is of the form $\square_\pm (\sc_l-\sc_{l+1})$ and satisfies $ \sc_l / \sa'$. 
%\item there is a splitting $\sa = \sa_1\cdot \sa_2\cdot \sa_3$ and $\sa' = \sa'_2\cdot \sa_3$ into words of $\sG_k$ such that $\sa_2/\sa_2'$.  
%\end{enumerate}
%\end{defi}
%The two last conditions are recursive but decrease the order $n_\sa$. Thus the right divisibility is well defined by induction on $n_\sa$.  
%\begin{prop}[Prop 5.14 \cite{berhen}]
%The right divisibility relation $/$ is a partial order on $\sG_k$. Moreover for all $\sa,\sa',\sa''\in \sG_k$ it holds:
%\begin{enumerate}[$(i)$]
%\item $\sa/\sa' \Rightarrow n_{\sa}\ge n_{\sa'}$ with equality iff $\sa=\sa'$,
%\item $\sa/\sa'$ and $\sa/\sa''$ and $n_{\sa'}\ge n_{\sa''}\Rightarrow \sa'/\sa''$,
%\item $\sa/\sa'$ and $\sa\cdot \sa'',\sa'\cdot \sa''\in \sG_k \Rightarrow \sa\cdot \sa''/ \sa'\cdot \sa''$.
%\end{enumerate}
%\end{prop}
%By properties $(i)$ and $(ii)$ we can define:
%\begin{defi}[GCD]
%The \emph{greatest common divisor} of $\sa, \sa'\in \sG_k$ is the element $\sd\in \sG_k$ dividing both $\sa$ and $\sa'$ with maximal order. We denote $\sd$ by $\sa\wedge \sa'$ and put $\nu (\sa, \sa')= n_{\sa\wedge \sa'}$.
%
%Given two different $t=\cdots \sa_i\cdots \sa_{-1},t'=\cdots \sa'_i\cdots \sa'_{-1}\in T^*_k$, let $m$ be minimal such that $\sa_{-m}\not= \sa_{-m}'$. Put 
%$N:= \max(n_{\sa_{-m}\cdots \sa_{-1}},n_{\sa'_{-m}\cdots \sa'_{-1}})$. 
%The greatest common divisor of $t,t'$ is:
%\[t\wedge t':= (\sa_{-N}\cdots \sa_{-1}\wedge \sa'_{-N}\cdots \sa'_{-1})\; .\]
%\end{defi}
%
%We are now able to define the distance $dist$ on $T_k^*$.
%\begin{defi} The following is a non-Archimedean metric on $T_k^*$:
%\[dist \colon (t,t')\mapsto b^{\nu (t,t')/4}\; .\]
%\end{defi}
%Here is the announced proposition which gives geometrical estimates from this combinatorial distance:
%\begin{prop}[Prop. 5.17n \cite{berhen}]
%The function $t\in (T_k^*,dist)\mapsto S^t$ is $1$-Lipschitz for a  $C^1$-uniform distance on the space of flat stretched curves.
%\end{prop}
%
%The following is a combinatorial counterpart to the notion of favorable times by Benedicks-Carleson \cite{BC2}, although it is used for another purpose.
%
%
%\begin{prop}[Prop. 6.5 \cite{berhen}]\label{f-times}
%For every $\sg\in \sG_k$, the family $\tau^k_\sg:=\{\sd_i\}_{i=0}^{t_\sg}$ is such that:
%\begin{itemize}
%\item[$(i)$] $\sg/\sd_{t_\sg}/ \sd_{t_{\sg-1}}/\cdots / \sd_1/ \sd_0$, with $\sd_0\in \sA_1=\sY_0\sqcup \{\square_\delta (\se-\ss); \; \ss\in \sY_0,\; \delta\in \{\pm\}\}$,
%\item[$(ii)$]  $n_{\sd_i}<n_{\sd_{i+1}}\le 2M\cdot n_{\sd_{i}}$ for $i<t_\sd$ and $n_{\sg}\le 2M\cdot n_{\sd_{t_\sg}}$,
%\item[$(iii)$] the domain of $\sd_i$ contains all the flat stretched curves of $S^t$ for $t\in T_0$, moreover  
%$t \cdot \sd_i$ belongs to $T_k^*$ for every $t\in T_0$,
%\item[$(iv)$] for all  $k'<k$ and $\sg\in \sG_k\cap \sG_{k'}$, it holds  $\tau^k_\sg=\tau^{k'}_\sg$,
%\item[$(v)$] if $\sg\in \sG_k$ is the $\sA_k$-spelling of a common piece $\sc^t_i$, with $t\in T_k^*$, then $\sd_{t_\sg}=\sg$.
%\item[$(vi)$] if $\sg,\sg'\in \sG_k$ satisfy $\sg/\sg'$ then $\tau^k_{\sg'}$ is made by the elements of $\tau^k_\sg$ of order less than $n_{\sg'}$.
%\end{itemize}
%\end{prop}
%By Property $(iv)$ we write $\tau_\sg$ instead of $\tau_\sg^k$.
%
%To bound from above the Hausdorff dimension of $T^*_k$ we consider the set 
%\[P_{j\, k}:= \{t\! t\cdot \sd\in T^*_k: \sd \in \sG_k\quad \& \quad n_\sd\le j\}\; .\]
%
%We notice that for $j\le M+ k'\le M+k$, it holds $P_{j\, k}=P_{j'\, k}$.  In that case we denote  $P_{j\, k}$ by $P_j$.
%
%From the two above propositions it comes:
%\begin{fact}\label{factHD}
%The set $P_{j\, k}$ is $\sum_{i=[j/2M]}^\infty b^{i/4}= \frac{b^{\frac{j}{8M}}}{1-\sqrt[4]b}$ dense in $T_k^*$.
%\end{fact}A last combinatorial computation gives:
%\begin{fact}\label{factHD2}
%The cardinality of $P_{j\, k}$ is at most $2^j$.
%\end{fact}
%\begin{proof}
%For every $t\in T^*_k$ there are at most two symbols in $\sA_k$ with the same order, and the order is at least $2$. Hence the cardinal of $P_{j\, k}$ is bounded by:
%\[ C_j:= \sum_{k\ge 2} Card \{ (n_i)_{1\le i\le k}\in (\Z\setminus \{-1,0,1\})^k: \sum |n_i|\le j\}\; . \]
%By induction on $j$ we assume that $C_i\le 2^i$ for every $i<j$. Then it holds:
%\[C_j\le 2+\sum_{k=2}^{j-1} 2\cdot C_{j-k} \le 2+ \sum_{k=2}^{j-1} 2^{j-k+1}=2^j\; .\]\qedhere
%\end{proof}
%
%We recall that the \emph{box dimension} of $(T_k^*,dist)$ is $\limsup_{\epsilon\to 0} -\log N(\epsilon)/\log\epsilon$, where $N(\epsilon)$ is the minimal number of  $\epsilon$-balls to cover $T_k^*$.  
%From  Facts \ref{factHD} and \ref{factHD2}, it comes:
%\begin{prop}\label{HDT*}
%The box dimension of $(T^*_k,dist)$ is at most 
%$-\frac{8 M \log2}{\log b}$.
%\end{prop}
%
%\begin{rema} We recall that first $M$ is assumed large and then $b$ is assumed small in function of $M$. Hence the box dimension is small  in function of $b$.
%\end{rema}
%\begin{rema} The above estimate is very coarse. It should be possible to define a better combinatorial distance on $T^*_k$, so that  $t\mapsto  S^t$ is $1$-Lipshitz and whose Hausdorff dimension is nearly $\log 2 /\log b$.
%When Yoccoz was preparing his last lecture at Coll\`ege de France with this work, he suggested to state that the GCD of 
%two different simple pieces $\ss^i_\pm$ and $\ss^j_\pm$ of orders $i$ and $j$, the order of the GCD should not be 0 but $\min(i,j)-1$. This should help to prove a better estimate.
%  \end{rema}
%
%
%\subsection{Ideas of proof in the Parameter selection}
%
%
%\subsubsection{Transversality}
%
%\paragraph{The one-dimensional case}
%An important fact in the prove of Jakobson's Theorem states that the motion w.r.t $a$ of the common puzzle pieces  is ``slower" than the one of $P_a^{M+1}(0)$.
%
%
%Let $\sc = (\R_\sc(a),n_\sc)$ be a common piece of $P_a$. By hyperbolicity this piece persists to a puzzle piece $\R_{\sc}(a')$ for $a'$ close to $a$. 
%
% 
%To express this difference of speed, it is easier to work in 
%$\R_\sr (a):= [-A_M,-A_{M-1}]$ which contains the critical value $a$. Indeed in this interval, the derivative of critical of value $a$  w.r.t. $a$ is obviously  1.
%We recall that $\R_\sr(a)$  is sent by $P_a^{M}$ onto $\R_\se(a)$, and we put $n_\sr :=  M$, and we consider the piece $\sr := (\R_\sr(a),n_\sr)$.  
%
%For every common piece $\sc:= (\R_{\sc}(a),n_{\sc})$ for $P_a$, let 
%$\R_{\sr \star \sc}(a):= (P_a|\R_\sr(a))^{-M}(\R_{\sc}(a))$. 
%
%\begin{prop}[Prop. 9.3\cite{berhen}]
%For every common piece $\sc$ for $P_a$, if $\R_{\sr \star \sc}(a)=: [x^-(a), x^+(a)]$, then for $M$ large, it holds:
%\[\partial_a x^\pm(a) = \frac13 +o(e^{-\sqrt M})\; .\]
%\end{prop}
%
%\paragraph{The H\'enon like case}
% Similarly, we consider the box:
%\[Y_\sr(\hat P_a):=\{(x,y)\in \R^-\times [-2\theta,2\theta]: x^2+y\in [A_{M-1}^2, A_M^2]\}\; .\] 
% It is bounded by  the lines $\{y=\pm 2\theta\}$ and by the following segments of the local stable manifold of the fixed point $(A_0,0)$ of $\hat P_a :(x,y)\mapsto (x^2+a+y,0)$.
%  \[W^s_\loc(-A_M; \hat P_a)\sqcup W^s_\loc(-A_{M-1}; \hat P_a)
%  :=   \{(x,y)\in \R^-\times [-2\theta,2\theta): x^2+y\in \{A_M^2,A_{M-1}^2\}\}\; .\]
%
%%These curves and  the lines $\{y=\pm 2\theta\}$ bound the box:
%
%For $(f_a)_a$ $C^2$-close to $(\hat P_a)_a$,  the local stable manifolds $W^s_\loc(-A_M; \hat P_a)\sqcup  W^s_\loc(-A_{M-1}; \hat P_a)$ 
%persists to $W^s_\loc(-A_M; f_a)\sqcup W^s_\loc(-A_{M-1}; f_a)$, and bounds with the line $\{y=\pm 2\theta\}$ a box $Y_\sr(a)$.  
%
%
% 
%Let $w$ be  a $k$-combinatorial interval for $(f_a)_a$ and let $a\in w$. 
%
%Given a common sequence $\sc$ we define the box 
%$Y_{\sr\star \sc} (a):=  f_a^{-M}(Y_{\sc}(a))\cap Y_\sr(a)$. 
%It is bounded by the lines $\{y= \pm 2\theta\}$ and by the arcs $\partial^s Y_{\sr\star \sc} (a)$  of local stable manifolds of $A$.
%
%
%\begin{prop}[Prop. 9.3 \cite{berhen}]For every $a_0\in w$, there exists a neighborhood $V_a$ of $a_0$ in $w$ and a $C^2$-function $\rho_{\sc}\colon V_A\times \partial^s Y_{\sr\star \sc} (a_0)\to \R^2$ such that for every $z\in \partial^sY_{\sr\star \sc} (a_0)$:
%
%\begin{enumerate}[$(i)$]
%\item $\rho_{\sc}(a_0, z)=z$,
%\item $|\partial_a \rho_{\sc}(a,z)|= \frac13 +o(e^{-\sqrt M})$ for $M$ large and $\partial_a \rho_{\sc}(a,z)$ is horizontal,
%\item $\rho_{\sc} (a, \partial^s Y_{\sr\star \sc} (a_0))$ 
%is equal to $\partial^s Y_{\sr\star \sc} (a)$ for every $a\in V_a$.
%\end{enumerate}
%\end{prop}
% 
% On the other hand, the flat stretched curves $(S^t(a))_{t\in T^*_k(w)}$ display a \emph{non-artificial} segment which is 
% $C^1$-close to depend horizontally on $a\in w$. For $t= t_0\cdot \sa\in T^*_k$ with $t_0\in T_0$, the non-artificial segment of $S^t(a)$ is $f^{n_\sa}_a(S^{t_0}_\sa(a))$. 
% 
%\begin{prop}[Prop. 10.3 \cite{berhen}]
%For every $a_0\in w$, for every $t= t_0\cdot \sa\in T^*_k(w)$ with $t_0\in T_0$, the following surface of $\R^3$ displays a tangent space which makes an angle less than $2\theta$ with the plane $\R^2\times \{0\}$.  
%\[\bigcup_{a\in w}\{a\}\times  f^{n_\sa}_a(S^{t_0}_\sa(a))\; .\]
%\end{prop}
%Hence they are folded by $f_a$ in a  fashion way which is $C^2$-close to $\{(x^2+a,0): x\in \R_e(a)\}$.
% \subsubsection{Cantors sets of positive measure and rough idea of the parameter selection} 
% 
%\paragraph{The one dimensional case}
%A necessary (and actually sufficient) condition for a quadratic map $P(a)$ to be strongly regular is that there exists a common sequence $\sc = (\sa_i)_{i\ge 1}$ so that 
%the critical value $a$ of $P_a$ belongs to the intersection point
%$x_\sc(a)$ of  $\cap_{i\ge 1} \R_{\sa_1\star \cdots \star \sa_i}(a)$.
%
%By using a large deviation argument (as in the parameter selection of  \cite{Y97}),  we show:
%\begin{prop}[Prop 3.10 \cite{berhen}]\label{prop3.10}
%Let $P_a$ be strongly regular, and let $\R_{\mathcal L}(a)\subset \R_\se$ be the set of intersection points $x_\sc(a)=\cap_{i\ge 1} \R_{\sa_1\star \cdots \star \sa_i}(a)$ of common sequence $\sc := (\sa_i)_i$. Then it holds for $M$ large:
%\[\frac{\leb \, \R_{\mathcal L}(a)}{\leb\,  \R_\se} =1+o(1)\; .\] 
%\end{prop}
%
%By transversality and the latter estimate we can believe that when the critical value $P_a^{M+1}(0)$ varies with $a$, it belongs to 
%$\mathcal L(a)$ for a positive set of parameters $a$.
%
%It is a coarse idea since when $a$ varies, the geometry of $\mathcal L(a)$ varies as well... 
%
%\paragraph{The H\'enon-like case}
%A necessary condition for a H\'enon-like map $f_a$ is that 
%each of the curves $(S^t)^\square:= f_a^{M+1}(S^t\cap Y_\square)$ is tangent to a common stable manifold $W^s_\sc(a)$. 
%  
%Actually the union $\mathcal L (a) := \cup_{\sc} W^s_{\sc}(a)$ of the local stable manifolds defined by common sequences is a Lipschitz lamination \cite[Lemma 13.9]{berhen}. Hence for $C^1$-coordinate on $Y_\se\approx \R_\se\times [-2\theta,2\theta]$, the set $\mathcal L$ corresponds to the product of a Cantor set $\R_{\mathcal L} (a)$ with $[-2\theta,2\theta]$:
%\[\mathcal L (a) \approx \R_{\mathcal L} (a)\times [-2\theta,2\theta]\]
%By Proposition 3.10 \cite{berhen}, the Lebesgue measure of 
%the Cantor set $\R_{\mathcal L}(a)$ is positive. 
%
%Also for every $t\in T^*(a)$, the curve $(S^t)^\square$ is tangent to a unique fiber $\{x_t(a)\}\times [-2\theta,2\theta]$  of  $Y_\se\approx \R_\se\times [-2\theta,2\theta]$. 
%
%Hence the tangency condition is equivalent to ask that the Cantor set
%$T^*(a)\approx\{x_t(a): t\in T^*(a)\}$ is included in the Cantor set of positive measure  $\R_{\mathcal L} (a)$. 
%
%From the transversality condition, a rough idea of the parameter selection would be to show that for a positive set of translation $\tau\in \R$, it holds that $T^*(a)+\tau\subset \R_{\mathcal L} (a)$. 
%
%\begin{figure}[h!]
%	\centering
%		\includegraphics[width=9cm]{NestedCantorSet.pdf}
%	\caption{Rough idea of the parameter selection for H\'enon-like endomorphisms}
%\end{figure}
%
%
%For many people this condition is anti-intuitive, in the sense that they did not believe that one can find two Cantor sets $K$ and $\tilde K$ so that 
%\[\leb \, \{\tau\in \R : K+\tau \subset \tilde K\}\; .\]
%
%However our main result with Moreira states a sufficient condition (which is also necessary in many cases) for this to happen. More specifically, we showed the following:
%
%\begin{thm}[Thm 2.1  \cite{BM13}]
%Let  $K\subset \R$ be a Cantor set of box dimension smaller than $d$: there exists $C_K>0$ so that $K$ can be covered by $C_K \epsilon^{-d}$ $\epsilon $-balls for every $\epsilon>0$.
%
%Let $\tilde K:=[0,\diam \tilde K]\setminus \sqcup_n (a_n, b_n)$ be a Cantor set such that with $l_n= b_n-a_n$ it holds:
%\begin{equation}\tag{$C_{1-d}$} \sum_n l_n^{1-d}<\infty\; .\end{equation}
%
%Then the set of parameters $t\in \R$ so that $K+t\subset \tilde K$ has positive measure if it holds:
%\[\sum_{n: l_n> \diam K}(\diam K+l_n)+2C_K \sum_{n: l_n \le \diam K} (l_n)^{1-d} < \diam \tilde K - \diam K\]
%
%  \end{thm}
%\begin{rema} This result enjoys many applications in Diophantine approximation, see \cite{BM13}.\end{rema}  
%  
%We recall that in Proposition \ref{HDT*} we showed that there exists $d$ small when $b$ is small s.t.  $T^*(a)$ is covered by a $C_k \epsilon^{-d}$ $\epsilon $-balls for every $\epsilon>0$.
%
%It is reminiscent in the proof of \cite{berhen} that the set $\R_{\mathcal L}(a) $ satisfies Condition $(C_{1-d})$ for $d$ sufficiently small.  
%This leads us to ask:
%
%\begin{prob}What is the minimal $p$ so that $\R_{\mathcal L} (a)$ satisfies Condition $(C_{p})$?
%\end{prob}
%
%In \cite{BM13} we studied several toy models for $\R_{\mathcal L} (a)$.  This let us dream that the case of a strongly regular theory could even contain an example of attractor of the same dimension as the one of the initial H\'enon conjecture. 
%
%Here again it is a rough idea of the parameter selection: when $a$ varies, the geometries of both $\{x_t(a): t\in T^*(a)\}$ and $\R_{\mathcal L}(a) $ vary. 
%
%To overcome this difficulty we cover the $j\ll k$-combinatorial intervals by finitely many parameter intervals, each of which is associated to point $t\in P_j$. Then we do the parameter selection at the level $k$. By working on an induction at two scales, we deal with a (locally) constant geometry (for the relevant pieces). Then we are able to evaluate the measure of the parameters removed by a large deviation argument along a tree (defined thanks to the right division). 
%
%
%%Random variable : independ (distorsion+ stretched acroos). 
%%
%%
%%solution large deviation argument along a tree.
%%\subsubsection{beyond the naive ideas}
%%
%%
%%$\odot\bigodot $
%
% \section{Ergodic properties of strongly regular maps}
%\label{propStrongRegular}
%In this section we present \cite{berent} which showed that every strongly regular H\'enon-like diffeomorphism  displays ergodic properties very similar to those of uniformly hyperbolic attractors. 
%
%\subsection{Topological Collet-Eckmann condition}
%It is well known that the Collet-Eckmann condition for unimodal map of the interval (which is implied by the Yoccoz' strong regularity condition) implies the topological Collet-Eckmann condition. The latter implies (see \cite{PRS03}) the existence of $m>0$ so that the positive Lyapunov exponent of every invariant ergodic probability measure is bounded from below by $m$.
%
%
%
% L. Carleson (as related by S. Newhouse during the first Palis-Balzan conference) asked if some non-uniformly hyperbolic H\'enon-like diffeomorphism displays such a property.  In \cite{berent}, we answered to his question:
% % The following is a two dimensional counterpart for strongly regular H\'enon-like diffeomorphisms:
%\begin{thm}[\cite{berent}]\label{MainEnt}
%For every strongly regular H\'enon-like map $f$, ergodic probability measure $\mu$ has a Lyapunov exponent greater than $\frac c3>0$.
%\end{thm}
%
%The same conclusion has been recently proved for non-uniformly hyperbolic horseshoes which appear as perturbations of the first bifurcation of H\'enon-like maps \cite{Ta13}. 
%
%
%Let us share a problem we discussed several times with R. Dujardin and M. Lyubich:
%\begin{prob}
%Define the notion of topological Collet-Eckmann complex H\'enon maps, prove the existence of new examples and show similar properties to those of \cite{PRS03}.\end{prob}
%
%
%%\subsection{Maximal netropy measure}
%%In particular we will present the results of \cite{berent} which 
%%proves the existence of $m>0$ such that for any such diffeomorphisms $f$, every invariant probability measure of $f$ has a Lyapunov exponent greater than $m$ (answering a question of L. Carleson). Moreover, it shows the existence and uniqueness of a measure of maximal entropy (answering a question of M. Lyubich and Y. Pesin).  
%%We also prove that the maximal entropy measure is equi-distributed on the periodic points and is finitarily Bernoulli (answering a question of J.P. Thouvenot). Finally, we show that the maximal entropy measure is exponentially mixing and satisfies the central limit Theorem. The proof is based on a new construction of Young tower for which the first return time coincides with the symbolic return time, and whose orbit is conjugated to a strongly positive recurrent Markov shift.
%
%\subsection{Maximal entropy measure and equi-distribution on the periodic points} 
%
%\paragraph{Definitions}
%
%Let us recall the definitions of \emph{entropy}.  For two covers $\mathcal O$ and $\mathcal O'$ of $M$, the family of intersections of a set from $\mathcal O$ with a set from $\mathcal O'$ forms a covering $\mathcal O \vee \mathcal O'$, and similarly for multiple covers.
%For any finite open cover $\mathcal O$ of $M$, let $H(\mathcal O)$ be the logarithm of the smallest number of elements of $\mathcal O$ that cover $M$.
% The following limit exists:
% \[   H(\mathcal O,f) = \lim_{n\to\infty} \frac{1}{n} H(\mathcal O\vee f^{-1}\mathcal O\vee \cdots\vee f^{-n}\mathcal O).\] 
%
%\begin{defi}[Topological entropy] The \emph{topological entropy $h(f)$ of $f$} is the supremum of $H(\mathcal O, f)$ over all finite covers $\mathcal O$ of $M$.
%\end{defi}
%
%Given a measure $\mu$, the \emph{entropy of $\mu$} is defined similarly. For a finite partition $\mathcal O$, put:
% \[   H_\mu(\mathcal O,f) = \lim_{n\to\infty} \frac{1}{n} \sum_{E\in 
% \mathcal O\vee f^{-1}\mathcal O\vee \cdots\vee f^{-n}\mathcal O} -\mu (E) \log \mu (E)\; .\]
%\begin{defi}[Metric entropy]
%The \emph{entropy $h_\mu$  of $\mu$} is the supremum of $H_\mu(\mathcal O, f)$ over all possible finite partitions $\mathcal O$ of $M$. 
% \end{defi}
% 
%From the Variational Principle, the topological entropy is the supremum of entropies of invariant probability measures:
%\[h(f)=\sup\{h_\mu(f):\mu \;\text{probability } f\text{-invariant}\}.\] 
%Therefore the topological entropy is an \emph{ergodic invariant}, {\it i.e.} it is invariant by bi-measurable conjugacy. 
%
%\begin{defi}[Maximal entropy measure] A  probability $\mu$ has \emph{maximal entropy} if $h(f)=h_\mu(f)$. The measure $\mu$ is equidistributed on the set of periodic points if $\mu$ is the limit of the following sequence:
%\[\frac1{Card\; Fix\; f^n}\sum_{z\in Fix\,f^n}\delta_z \rightharpoonup\mu\;.\]
%\end{defi}
%
%
%\bigskip
%
%
%
%%\paragraph{Examples}[Complex map]
%\begin{exam}[Uniformly hyperbolic maps]
%We recall that every  uniformly hyperbolic set $\Lambda$   admits a (finite) Markov partition. This implies that its dynamics is semi-conjugated with a subshift of finite type. The semi-conjugacy is 1-1 on a generic set. Its lack of injectivity is itself coded by subshifts of finite type of smaller topological entropy. This enables one to study efficiently all the invariant measures of $\Lambda$, to show the existence and uniqueness of the maximal entropy measure $\nu$, and to show the equidistribution of the set of  periodic points w.r.t. $\nu$.
%%\[\frac1{Card\; Fix\; f^n}\sum_{z\in Fix\,f^n}\delta_z \rightharpoonup\nu\;.\]
%\end{exam}
%\begin{exam}[Unimodal map]
%In \cite{H81}, a coding is given to prove the existence and the uniqueness of the maximal entropy measure for unimodal maps of positive entropy. This measure does not need to be equidistributed on the set of periodic points \cite{MK12} if the critical point is flat and the map finitely smooth.
%%\marginal{Smooth??}
%
%We recall that Yoccoz' strongly regular map displays a positive entropy. 
%%\marginal{Senti proved????}
%% \cite{KSVS}.
%
%\end{exam}
%
%\begin{exam}[Complex map]
%For every rational function of the Riemannian sphere, the existence and uniqueness  of a maximal entropy measure is known, since the works of Lyubich \cite{Lyu83} and \cite{Ma83}. They showed also its equidistribution on the set of periodic points. 
%
%For every complex polynomial automorphism of $\C^2$, the existence, uniqueness, and equidistribution on the set of periodic points   of a maximal entropy measure has been shown in  the work of Bedford-Lyubich-Smillie \cite{BLS93}. 
%\end{exam}
%
%\begin{exam}[$C^\infty$-case]
%In finite regularity, a measure of maximal entropy needs not exist \cite{Gu1969}.
%Nevertheless, a famous theorem of Newhouse states the existence of a maximal entropy measure for every $C^\infty$-diffeomorphism \cite{NH} of any compact manifold.  
%
%Recently Buzzi-Crovisier-Sarig announced the proof of the uniqueness of the maximal entropy measure for every surface $C^\infty$-diffeomorphism whose non-wandering set is 
%transitive and whose entropy is positive.
% This result uses the finite to one Markovian coding of Sarig \cite{Sa13} of the union of the support of invariant measures with
% Lyapunov exponents uniformly far from $0$. 
% %Also Burguet announced that such measures must be equi-distributed on the set of periodic points provided that all the Lyapunov exponents of every measure  are uniformly far from $1$.
% \end{exam}
%
%\medskip
%
%We recall that the unique abundant examples of surface diffeomorphisms which displays a transitive set with topological positive entropy are: 
%  \begin{itemize}
%  \item The non-uniformly hyperbolic horseshoe. In \cite{PY09}, a certain Markovian coding is given on the maximal invariant set, but it is not easy to see if this implies the uniqueness of the maximal entropy measure, neither if all the Lyapunov exponents of every measure  are uniformly far from $0$.
%  \item The H\'enon-like attractor. In \cite{YW}, a certain coding is given in order to prove the existence of a maximal entropy measure for H\'enon attractors of Benedicks-Carleson type, but the formalism does not seem to imply easily its uniqueness.
%    \end{itemize}
%
%
%In \cite{berent}, we showed  the existence and uniqueness of a measure of maximal entropy (this answers a question of Lyubich and Pesin) for every  strongly regular H\'enon-like diffeomorphism (with sufficiently small determinant).  Let us point out that the dynamics is only of class $C^2$ and so the existence of such a measure is not implied by Newhouse Theorem \cite{NH}.
%Furthermore we proved that the maximal entropy measure is equi-distributed on the periodic points and is finitarily Bernoulli (an answer to a question of Thouvenot).
%\begin{thm}[]\label{MainEnt2}
%Every strongly regular H\'enon-like diffeomorphisms $f$ leaves invariant a unique probability of maximal entropy $\nu$. Moreover $\nu$ is equi-distributed on the periodic points of $f$, finitarily Bernoulli, exponentially mixing and it satisfies the central limit Theorem. 
%\end{thm}
%
%
%A \emph{Bernouilli shift} is the shift dynamics of $\Sigma_N:=\{1,\dots, N\}^\mathbb Z$ endowed with the product probability $p^\mathbb Z$ spanned by a probability $p=(p_i)_{i=1}^N$ on $\{1,\dots, N\}$. The entropy of the probability $p^\mathbb Z$ is $h_{p}=-\sum_i p_i\log p_i$.
% By Ornstein and Kean-Smorodinsky isomorphism Theorems, any two Bernouilli shifts $(\Sigma_N, p^\mathbb Z)$ and $(\Sigma_{N'}, p'^{\mathbb Z})$ with the same entropy $h_{p^\Z}=h_{p'^\Z}$  are \emph{finitarily isomorphic} \cite{KS79}.  A bi-measurable isomorphism is \emph{finitary} if it and its inverse send open sets to open sets, modulo null sets.
% 
%To be \emph{finitarily Bernoulli} means that the dynamics, with respect to the maximal entropy measure, is  finitarily isomorphic  to a Bernouilli shift.
%
%
%The \emph{central limit Theorem} is that for every  
%H\"older 
%function $\psi$ of $\nu$-mean $0$, such that $\psi\not= \phi-\phi\circ f$ for any $\phi$ continuous, there exists $\sigma>0$ such that $\frac 1{\sqrt n} \sum_{i=1}^n \Psi\circ f^i$ converges in distribution (w.r.t. $\nu$)  to the normal distribution with mean zero and standard deviation $\sigma$.
%    
%The measure  $\nu$ is \emph{exponentially mixing} if there exists $0<\kappa<1$ such that for every pair of functions  of the plane $g\in L^\infty(\nu)$ and $h$ H\"older continuous, there is $C(g,h)>0$ satisfying for every $n\ge 0$:
%\[Cov_\nu(g,h\circ f^n)<C(g,h) \kappa^n,\text{\; with $Cov$ the covariance.}\] 
%
%
% The proof is based on a new construction of Young's tower for which the first return time coincides with the symbolic return time, and whose orbit is conjugated to a strongly positive recurrent Markov shift. 
%
%\subsection{Idea of proofs} 
%\subsubsection{Idea of proof of Theorem \ref{MainEnt}} 
%
%We recall that the strong regularity definition involves a countable alphabet $\sA$ and a set of pre-sequences $T^*\subset \sA^{\Z^-}$. Every $t\in T^*$ is associated to a curve $S^t$ and a common sequence $\sc^t$. The common sequence defines a nested sequence of puzzle pieces $(\sc_i^t(S^{t\!t}))_{i\ge 0}$ of $S^{t\! t}$. Local stable manifolds of the endpoints of $\sc_i^t(S^{t\!t})$ cross the domain $Y_\se$ to define a domain $Y_{\sc_i^t}\subset Y_\se$.  The intersection $W^s_{\sc^t}= \cap_i Y_{\sc_i^t}$  is a local stable manifolds. By $(SR_1)$, the curve $f^{M+1}(S^t\cap Y_\square)$ is tangent to $W^s_{\sc^t}$. 
%
%Let us denote $Y_{\square (\sc_i^t- \sc_{i+1}^t) }:= (f^{M+1}|Y_\square)^{-1}(cl(Y_{\sc_i^t}\setminus Y_{\sc_{i+1}^t}))$ and 
%$Y_{\square \sc^t}:= (f^{M+1}|Y_\square)^{-1}(W^s_{\sc^t})$.
%
%
%We observe that $\{Y_\ss: \ss\in \sY_0\}\sqcup \{ Y_{\square (\sc_i^t- \sc_{i+1}^t) }): i\ge 0\}\cup\{Y_{\square \sc^t}\}$ is a partition of $Y_\se$ (modulo the stable manifold of the fixed point $A$).
%
%\begin{figure}[h!]
%    \centering
%        \includegraphics{position2.pdf}
%    \caption{Possible shapes for $Y_{\square (\sc_i-\sc_{i+1})}$.}
%   \label{position}
%\end{figure}
%In figure \ref{position}, we depict that the  set  $Y_{\square (\sc_i^t- \sc_{i+1}^t) }$ is formed by 1, 2 or 3 connected components:
%\begin{itemize}
%\item  a (below) component $Y_{\square_b (\sc_i^t- \sc_{i+1}^t) }$ which may exist, 
%\item  either a component  $Y_{\square_a (\sc_i^t- \sc_{i+1}^t) }$ or two components  $Y_{\square_+ (\sc_i^t- \sc_{i+1}^t) }$ and $Y_{\square_- (\sc_i^t- \sc_{i+1}^t) }$.
%\end{itemize}
%
%
%If a component $Y_{\square_a (\sc_i^t- \sc_{i+1}^t) }$ occurs, then we split it 
%into two domains  $Y_{\square_+ (\sc_i^t- \sc_{i+1}^t) }$ and $Y_{\square_- (\sc_i^t- \sc_{i+1}^t) }$ (following a certain algorithm), as depicted in figure \ref{partition}. 
%\begin{figure}[h!]
%    \centering
%        \includegraphics{partition2.pdf}
%    \caption{Partition of $Y_\square$. }
%   \label{partition}
%\end{figure}
%
%Then we observe that the following set is a partition of $Y_\se$ (modulo $W^s(A)$):
% \[\mathcal P(t):=\{Y_\ss: \ss\in Y_0\}\cup \{ Y_{\square_\pm (\sc_i^t- \sc_{i+1}^t) } : i\ge 0\}\cup \{ Y_{\square_b (\sc_i^t- \sc_{i+1}^t) } : i\ge 0\}\cup\{Y_{\square \sc^t}\}\; .\]
% We denote by $\sP(t):=  Y_0\cup \{\square_\pm (\sc_i^t- \sc_{i+1}^t)  : i\ge 0\}\cup \{ \square_b (\sc_i^t- \sc_{i+1}^t) : i\ge 0\}\cup\{\square \sc^t\}$ the set of symbols associated. 
%
%We observe that $\sP(t)$ depends on $t$, and is included in $\sA\sqcup   \{ \square_b (\sc_i^t- \sc_{i+1}^t) : i\ge 0\}\cup\{\square \sc^t\}$
% \paragraph{A fibered encoding}
% 
%
% \begin{defi}\label{defregular} \index{ regular sequence} A sequence of symbols $\sg=(\sb_i)_{i=0}^m\in \sA^m$  is \emph{regular}
% if $\sg$ is   suitable  from $ S^{t\! t}$ and the following inequality holds for every $i\le m$:  
%%\begin{equation}\tag{$R$}
%\label{Rxiweak}
%\[n_{\sb_{i}}\le M+\Xi \sum_{1\le j<i} n_{\sb_j}\; ,\]
%with  $\Xi:= e^{\sqrt M}$. 
%%\end{equation}
%\end{defi}
%We denote by $\tilde \sR\subset \sA^\infty$  the set of regular sequence of infinite length.  
%
%
%For every regular sequence $\sg=\sb_0 \cdots \sb_m$, we define the set 
%\[Y_\sg:=\{z\in Y_\se:\;  f^{n_{\sb_0\cdots \sb_i}}(z)\in Y_{\sb_{i+1}},\; \forall i<m\}\]
%
%
%In proposition 2.12 \cite{berent}, we showed  that $Y_\sg$ displays  a very nice geometry  (in the same way as a common piece $\sc$ displays a box $Y_\sc$ with a nice geometry). Moreover, we show that for every $z\in Y_\sg$, for every unit vector $u$ $\theta$-close to the horizontal, it holds for every $i\le m$:
%\begin{equation}\tag{$D$} \|D_zf^{n_{\sb_0}+\cdots + n_{\sb_i}}(u)\|\ge e^{\frac c3 (n_{\sb_0}+\cdots + n_{\sb_i})}\; .\end{equation}
%% and make an angle at most $\theta$ with the horizontal. 
%
%%(it is bounded by arc of parabolas and segments of $\{y=\pm 2\theta\}$).  Moreover the horizontal direction is uniformly expanded. 
%
%\medskip
%
%In particular when $m=\infty$, the set $Y_\sg$ is a stable curve which is asymptotically normally expanded by a factor $e^{c/3}$.  The idea of the proof is to show that given any ergodic, probability measure $\mu$ (not supported by a certain uniformly compact set $K^*$ with small topological entropy), it holds that $\mu$-a.e. point $z$ is sent by a certain iterate into the union of curves:
%\[\tilde {\mathcal R}= \bigcup_{\sg\in \tilde\sR} Y_\sg\; .\]
%Then the uniform bound from below of the Lyapunov exponent follows from $(D)$.
%%the uniform normal expansion of the points in $\tilde {\mathcal R}$. 
%
%To show this given $z\in Y_\se \setminus W^s(A)$, we look at the maximal regular length $p$ of a regular sequence $\sg = \sa_0\cdots \sa_{p-1}$ so that 
%$z$ belongs to $Y_\sg$. Note that $p$ may be equal to $0$.
%
%If $p=\infty$ we are done. 
%
%Otherwise we show that $z_1:=f^{n_\sg+M+1}(z)$ belongs to $Y_{\sc_i}\subset Y_\se$ with $n_{\sc_i}\ge \Xi n_\sg$.  Hence we can define again a maximal regular sequence $\sg_1$ so that $z_1\in Y_{\sg_1}$.  And so on we continue by defining $\sg_2,...,\sg_k,...$ and $z_2, ..., z_k,...$   until we may fall into a $\tilde {\mathcal R}$. 
%
%We recall that if $z_k$ belongs  to $\tilde {\mathcal R}$ for some $k$, then we are done. Otherwise, we recall that  $n_{\sg_{i+1}}\ge \Xi n_{\sg_{i}}$. Then:
%\begin{itemize}
%\item Either $n_{g_i}= 0$ for every $i$, and so $z$ belongs to the (finite) orbit of the compact set $$K_\square:= \cap_{i\ge 0} f^{-i(M+1)}(Y_\square)\; ,$$
%\item or $(n_{\sg_{i}})_i$ grows eventually exponentially fast (with factor $\Xi$) to infinity.
%\end{itemize}
%To solve the first case we show that $K^*:= \cup_{0\le j\le M}f^j(K)$ is a uniformly hyperbolic set \cite[Prop. 2.19]{berhen} with expansion at least  $e^{c/3}$. 
%
%In the second case, we recall that $\mu$-a.e. point $z$ displays defined Lyapunov exponent. 
%
%If they are both negative, then by Katok's closing Lemma, $z$ is a periodic sink. 
%Hence an iterate of $z$ belongs to infinitely many $Y_{\sg_i}$. We chose one $\sg_i$ so that $n_{\sg_i}$ is larger than the period of $z$. This contradicts $(D)$. 
%
%If one Lyapunov exponent is non-negative, then a non-contracted direction $E^{cu}$ is well defined at $z$. 
%However $z_{i+1}$ is exponentially close to the tangency point between $f^{M+1}(S^{t\! t\cdot \sg_{i}}\cap  Y_\square)$ with $W^s_{c^{t\! t\cdot \sg_{i}}}$, and we show that  the image $D_zf^{n_{\sg_0}+M+1+ n_{\sg_1}....+M+1+n_{\sg_{i}}}(E^{cu})$ of the $E^{cu}$ is exponentially close to the tangent space at this tangency \cite[Prop. 2.7]{berent}. By $b$-contraction of $W^s_{c^{t\! t\cdot \sg_{i}}}$, it follows that  $E^{cu}$ is sufficiently contracted during a sufficiently long time. 
%This implies that the sequence $(\frac1n \log \| D_zf^{k}| E^{cu}\|)_k$ does not converge. Hence the contradiction.$\qed$
%\subsubsection{Idea of proof of Theorem \ref{MainEnt2}} 
% In the previous section we saw that any invariant measure is supported by the orbit of $\tilde{\mathcal R}$ and $K_\square$. 
% 
% It is rather easy to bound from above the topological entropy of $\hat K_\square = \cup_{i=0}^M f^i(K_\square)$ by $\log2/M$. In particular this set does not support an invariant measure with high entropy. 
% 
%Hence we shall study the measure supported by the orbit of  $\tilde{\mathcal R}= \cup_{\sg\in \tilde \sR} Y_\sg$. The set $\tilde \sR$ is included in $\sA^\N$ on which the shift dynamics acts. 
%We consider the set $\sR$ formed by the sequence $\sg\in \tilde \sR$ which come back infinitely often in $\tilde \sR$ by the shift dynamics. We put $\sE:= \tilde \sR \setminus \sR$. 
%This splits $\mathcal R$ into two sets:
%\[\mathcal E:= \cup_{\sg\in \sE} Y_g\quad \text{and}\quad \mathcal R:= \cup_{\sg \in \sR} Y_\sg\; .\]
%In section 4.2 of \cite{berhen} we use an argument based on the Hausdorff dimension (following a study close to \cite{Se03})
%and the Ledrappier-Young entropy Formula \cite{LYII85}  to show that any measure supported by $\mathcal E$ displays a small entropy. 
%
%Hence all invariant measures with substantial entropy are supported by the orbit of $\mathcal R$. In order to study them thanks to the combinatory of $\sR$, we consider for every $\sg$ its first return $\sg_1$ in  $\sR$ by the shift. This defines $\sa_1\in \sA^{(\N)}$ so that $\sa_1\cdot \sg_1=\sg$ and for $x\in Y_\sg$, an integer $N(x):= n_{\sa_1}$.  We put:
%\[f^\mathcal R:= x\in \mathcal R \mapsto f^{N(x)}(x)\in \mathcal{ R }.\]
%We notice that $f^\mathcal R$ is semi-conjugated to the first return map induced by the shift on $\sR$. Nevertheless, $f^\mathcal R$ is in general not the first return map of $\mathcal R$ into $\mathcal R$.  To obtain such a property, we consider $R:= \cap_{n\ge 0} (f^{\mathcal R})^n (\mathcal R)$. 
%
%It turns out that the orbit of $R$ supports the same measures as the orbit of $\mathcal R$ (see Prop. 3.2 \cite{berent}), and so we can indeed focus on the dynamics $f^\mathcal R| \mathcal R$. 
%
%Furthermore, a surprising property is that the first return of a point $x$ of $R$ in $R$ is exactly $N(x)$ (see Prop. 3.3. \cite{berhen}). 
%This implies  that  the inverse limit $\arr \sR$ of $\sR$ for the shift dynamics  is canonically conjugated to $f^{\mathcal R}|R$ (see Prop 3.4 \cite{berhen}). 
%
%Hence it suffices to study the combinatory of $\arr \sR$  to deduce the ergodic properties of $f^{\mathcal R}|R$ therefore any measure of $f$ with substantial entropy. This part of the proof is rather straightforward. 
%As a matter of fact we obtain a Young tower, whose first return time coincides with the combinatorial return time, and whose associated Markov chain is strongly positive recurrent. 
%
% 
% 
%\section{Dynamics derivated from the standard map}
%\label{partialhyp}
%A wished candidate in the list of paradigmatic examples of non-uniformly hyperbolic dynamics is the Chirikov-Taylor Standard map family $(S_{r})_{r\in\R}$. It is formed by conservative maps of the  $2$-torus $\T^2=\mathbb R^2/2\pi \mathbb Z^2$:
%$$S_{r}(x,y)=(2x-y+ r\sin( x),x ).$$
%
%Sinai conjectured ( \cite{Si94} P.144) that for every non-zero parameter $r$, the  Lyapunov exponents of $S_r$  are non-zero at a set of points $z\in \T^2$ of positive Lebesgue measure.  Equivalently, this conjecture states that the metric entropy of $S_r$ is positive for every $r\not=0$.  
%
%This conjecture is very hard: there is not a single $r$ for which we know that the metric entropy of $S_r$ is positive. 
%
%On the other hand, the important negative result of P. Duarte \cite{Du94} states that there exists $r_0\ge 0$ so that  for a topologically generic $r\ge r_0$, the map $S_r$ displays infinitely many elliptic islands (see fig. \ref{KAMisland}). J. De Simoi  \cite{JMD} showed that this generic set of parameters has Hausdorff dimension at least $1/4$. A. Gorodetski gave a description of a ``stochastic sea'' for a generic set of these parameters \cite{Go12}: his impressive construction showed that for such parameters there exists an increasing sequence of hyperbolic sets with Hausdorff dimension converging to $2$, and such that elliptic islands accumulate on them.
%
%
%
%
%\begin{figure}
%	\centering
%		\includegraphics{ilesKAMa=-0364.jpg}
%%  {\fig{1}{cube00001}}
%	\caption{Lyapunov exponent for the parameter $r=-0.364$ of the standard map.
%White regions are expected to be elliptic islands whereas the black regions are possibly the homoclinic web of a NUH attractor.}
%	\label{KAMisland}
%\end{figure}
%
%Another way to construct examples of non-uniformly hyperbolic dynamics is to work with those which (locally) fiber over a uniformly hyperbolic one. Such techniques have been used notably by \cite{Sh71}, \cite{Vi97}, and \cite{SW00} to produce new examples which are robustly non-uniformly hyperbolic.
%
%\begin{defi} A map $f$ of a compact manifold $M$ is  ${C}^s$-\emph{robustly non-uniformly hyperbolic} if there exists a ${C}^s$-neighborhood $U$ of $f$ such that every map $g\in U$ the following property holds true:
%For Lebesgue almost every point $z$ of $M$, there are two subbundle $E^s_z\oplus E^u_z=T_zM $ so that:
%\[\limsup \frac1n \log \| D_zf^n| E^s_z\|<0\quad \text{and} 
%\quad \liminf \frac1n \log \| (D_zf^n)^{-1}| E^u_z\|>0\; .\]
%\end{defi}
%The general expectation is that (with possibly a few more hypotheses) if $f$ is robustly non-uniformly hyperbolic (in the sense above) then $f$ leaves invariant at most countably  many physical, probability measures, whose union of the basins covers $M$ modulo a Lebesgue null set.
%Many intermediate results exist in that direction, generalizing the initial works of Alvez-Bonatti-Viana and Bonatti-Viana \cite{ABV00,BV00}. When the dynamics preserve the Lebesgue measure of $M$, this expectation is a theorem of  Pesin \cite{Pe77}: if Lebesgue a.e. point displays non-zero Lyapunov exponents, then $M$ is  covered by countably many measures, modulo a Lebesgue null set.  
%
%In \cite{AV10}, another example has been given, namely a non-hyperbolic ergodic toral automorphism for which most {symplectic} perturbations are non-uniformly hyperbolic.  The techniques developed there are also suitable for dealing with some conservative cases, pushing forward the method developed in \cite{SW00} for volume preserving diffeomorphisms.
%
%In \cite{BC14} we showed the existence of a non-hyperbolic robust conservative non-uniformly hyperbolic diffeomorphism, adding a different type of example to the above list.
%
%Let $A\in SL_2(\Z)$\ be a hyperbolic matrix with eigenvalues $\lambda<1<1/\lambda$. Consider the manifold $M=\T\times \T$\ with coordinates $m=(x,y,z,w)$, and the analytic diffeomorphism $f_N:M\rightarrow M$\ given by
%$$
%f_N(m)=(\mathbf{s}_{N}(x,y)+P_x\circ A^{N}(z,w),A^{2N}(z,w))
%$$
%where $N\ge 0$, and $P_x$\ is the projection of $\R^2$ to the $x$-axis $\R\cdot (1,0)$.
%
%
%
%
%\begin{thm}\label{thmBC14}
%%\label{main}
%There exist $N_0$ and $c>0$\ such that for every $N\geq N_0$,\ the map $f_N$ satisfies for Lebesgue a.e. $z\in \T\times \T$ and every unit vector $v\in \mathbb R^4$:
%\[\lim_{n\to \infty} \big|\frac1n \log\|D_zf_N^n(v)\|\big|>c\log N.\]
%
%Moreover the same holds for every conservative diffeomorphism in a ${C}^{2}$-neighborhood of $f_N$.
%\end{thm}
%
%Blumenthal-Xue-Young  \cite{BXY17} used a very similar argument to our proof  to prove that random perturbations of every large parameter of the standard map display a positive metric entropy . They attribute their statement to a work announced by Carleson-Spencer.  
%
%\subsection{Idea of proof}
%The initial idea goes back to a theorem of M. Viana:
%\begin{thm}[Viana \cite{Vi97}]
% For $N, s$ large and $\epsilon>0$ small, the following  map is $C^s$-robustly non-uniformly hyperbolic:
%\[V_\epsilon\colon (\theta, x)\in \R/\Z\times [-2,2]\mapsto (N\theta,  x^2-2+\epsilon (2+\sin(N\theta)))\in \R/\Z\times [-2,2]\; .\]
%\end{thm}
%
%Actually, this Theorem holds true for $s=2$.  Let us describe an idea of  proof of this theorem, and how it has been modified to proved Theorem \ref{thmBC14}. 
%\begin{proof}[Sketch of proof of Viana's Theorem]
%$\;$\\
%{\bf Adapted metric}
%For every  $\epsilon>0$ small, one can show the existence of an adapted metric $g$ on $[-2,2]$ so that 
%for every critical value $a$ which is $3\epsilon$-close to $-2$, the derivative of $P_a(x)=x^2+a$
%is greater than $3/2$ at every point but those in $[-\epsilon^2, \epsilon^2]$ at which the derivative is bounded from below by $|x|$. 
%\\
%{\bf Admissible curve} Let $\Gamma$ be the set of curves of the form:
%\[\gamma_0:=\{(\theta,x)\in [-\frac12,+\frac12)\times \R: x= x_0+\epsilon \sin \theta \}\; \text{among }x_0\in [a, a^2+a].\]
%When $N$ is large it is easy to show the existence of a $C^2$-neighborhood $N_\Gamma$ of $\Gamma$, so that for every $\gamma_0\in N_\Gamma$, the image of $V_\epsilon(\gamma_0)$ is the union of $N$-curves $\gamma_i$ in $N_\Gamma$.   
%
%For every $\gamma\in N_\Gamma$, we notice that a very small proportion of $\gamma$ intersects the  strip $\R/\Z\times [-\epsilon^2,\epsilon^2]$. 
%
%Furthermore  the Lebesgue measure of points which are less that $\eta>0$-expanded is smaller than $\sqrt[3]\eta$. This enables to prove the  following inequality:
%\[\int_{ z\in \gamma} -\log \| V_\epsilon^{-1}(z)\|d\leb(z)>\log\frac43\; .\]
%\\
%{\bf Iterations}
%We recall that every $\gamma_0\in N_\Gamma$  is sent by $V_\epsilon$ to the union of $N$ curves 
%$(\gamma_i)_{1\le i\le N}$ of $N_\Gamma$. By reapplying the above inequality, it comes:
% \[\int_{ z\in  \gamma} -\log \|(D V^2_\epsilon)^{-1}(z)\|d\leb(z)>2 \log\frac43\; .\]
%And so on, for every $k\ge 0$, it comes:
%\[\frac1k  \int_{ z\in  \gamma} \log \|(D V^k_\epsilon)^{-1}(z)\|^{-1}d\leb(z)>\log\frac43=:\sigma\; .\]
%As  $-\log \| V_\epsilon^{-1}(z)\|$ is bounded by $M=\log 2$, it holds:
%\[ \sigma \le M\cdot  \leb\{ z\in \gamma: 
%\frac1k \log \| V_\epsilon^{-k}(z)\|^{-1}\ge \frac \sigma 2 \} +
% \frac \sigma 2\cdot  \leb\{ z\in \gamma: 
%\frac1k \log \| V_\epsilon^{-k}(z)\|^{-1}\le \frac \sigma 2 \}\; .\]
%This implies the existence of a subset $A\subset \gamma$ of measure $\ge \sigma/2M$, so that every point $z\in A$ satisfies:
% \[
% \limsup_{k\ge 0} \frac1k \log \|(D V^k_\epsilon)^{-1}(z)\|^{-1}d\leb(z)\ge \frac\sigma{2}
%\; . \]
% {\bf Conclusion}
%Then it is easy to see that $A$ is equal to $\gamma$ modulo a Lebesgue null set. Indeed, otherwise we take a density point of $\gamma\setminus A$, and by Lebesgue density Theorem, there exists a segment $S\subset \gamma$  which intersects $A$ in proposition much smaller than $\sigma/2M$
%and which is send by an iterate $V^k_\epsilon$ to an admissible curve $\gamma_k$ in $V_\Gamma$. Then a proportion at least $\sigma/2M$ of $\gamma_k$ belongs to $A$.  As the segment $S$ of $\gamma$ is uniformly expanded to $\gamma_k$, a distortion bound holds true and implies a contradiction. 
%
%To conclude it suffices to show that any translation $\gamma+(0,y)$ of $\gamma$ is still in $N_\Gamma$ and so by Fubini's Theorem we showed that $A$ covers Lebesgue a.e. $\T\times [-2,2]$. 
% 
% Finally let us  observe that all the above arguments are valid for $C^2$-perturbation of $V_\epsilon$.  
%\end{proof}
%
%In the same article of M. Viana, the skew product of the H\'enon map is considered. Actually the same approach works by using the  strong dissipativeness. 
%
%In both case the proof is much more easier than Jakobson's or Benedicks-Carleson's.  That is why we developed this technique to explore the non-uniformly hyperbolic properties of the standard map in \cite{BC14}.  
% 
% \begin{proof}[Sketch of Proof of Theorem \ref{thmBC14}]
% Likewise, for every large parameter,  the standard map  displays a horizontal cone field $\chi$ which is expanded and invariant, at every point but those located in two small strips centered at $x=0$ and $x=\pi$. These strips are called critical and denoted by $C$.
%
%We chose the skew product $f_N$ of the standard map with the Anosov map so that the strong unstable manifolds cross $C$  in a set of proportional Lebesgue measure very small.  Here the strong local unstable manifolds play the role of admissible curves. 
%
%We needed to develop the above argument by considering the pairs 
%$(\gamma, X)$ of a local unstable manifold $\gamma$ of length $1$ supporting a  $\frac12$-holder vector field $X$ on it. 
%
%We showed that after a few iterations by $Df_N$, these vector fields are eventually $\frac12$-holder with small constant. This enabled us to prove thanks to a simple probabilistic argument (which uses a Markov chain like in \cite{BXY17}) that for every $k$ large, for most of the points $z\in \gamma$, 
%$D_zf^k_N(X(z))$ is in $\chi$. By integrating 
%$\log \| D_zf^{k+1}_N(X(z))\|/\| D_zf^{k}_N(X(z))\|$ along $\gamma$ we obtained a positive lower bound which is independent of $\gamma$ and $k$. Then we concluded similarly as in the above proof.
%\end{proof}
%\begin{rema}
%J. Bochi and M. Viana \cite{BV05} showed that for any closed symplectic manifold $(M,\omega)$ there exists a ${C}^1$-generic set $\mathcal{R}\subset Sym^1_{\omega}(M)$ such that for $g\in\mathcal{R}$ then either (a) at least two Lyapunov exponents of $g$ are zero Lebesgue almost everywhere,  or (b) $g$ is Anosov.
%
%We observed in \cite[Cor.7]{BC14} that  $f_N$ is symplectic and not  uniformly hyperbolic. 
%Hence this implies that the statement of  Bochi-Viana theorem does not hold true in the ${C}^2$ setting.
%\end{rema}
%\part{Emergence}\label{Emergence}
%In 2003, during one of my very first talks with my adviser,  J.-C. Yoccoz told me the following about my thesis subject:
% 
%\begin{center}``{\it
%My personal feeling is that we will not get to understand [typical]
%dynamical systems in a bounded time. Not everybody agrees with me,
%some people are optimistic, I respect their belief, but I am not.
%I am sure that a scientific revolution will come, showing that dynamics
%are much more complicated than we think they are.
%
% This should come from the study of a paradigmatic example, which does not need to
%involve complicated mathematics. For instance, Newhouse phenomena is  rather simple to show.}"
%\end{center}
%
%At first I was unhappy to hear this: I was wishing to prove positive results  exhibiting the strength of mathematics. But years after years, after exploring a few branches of dynamical systems, I became more and more convinced by this vision. I am now convinced that this point of view is already revolutionary, even if there are still a lot of works to show mathematically such a faith. 
%
%
%There are many ways to interpret  Yoccoz' quote, from logic mathematics (indecidability),
%topology of a phenomenon and the one of parameters for which it occurs  (e.g., sets which are not locally connected), the growth of the number of periodic points, etc. 
%
%
%
%We will first recall Newhouse phenomenon in section \ref{Newhouse} which provides examples of very complex dynamics.  In section \ref{NewhouseTypical} we will study the typicality of such a phenomenon. In particular we will see that this phenomenon is sufficiently typical to do not be neglected. In section \ref{Ermergencedef}, we will define the concept of Emergence for a dynamical system $f$, which roughly speaking, quantifies the complexity to describe a dynamical system by means of 
%physical, probability measures.
%
%  In section \ref{ArnoldPer}, we will see that the growth of the number of periodic points can be faster than exponential (as contrarily to what was expected by Smale in 1967, Bowen 1978, Arnold 1989-92) and this in any dimension $\ge 2$ and any regularity $2\le r\le \infty$. 
%
%The two main results presented are given by a counterpart of the Bonatti-Diaz Blender for parameter families: the parablender. We will recall its definition in section \ref{parablender}. 
%
%
%\section{Newhouse phenomenon}\label{Newhouse}
%\subsection{Newhouse's discovery of the co-existence of infinitely many sinks}
%Beside Smale asked to his student Newhouse to work on the genericity of axiom A condition, among surface diffeomorphisms (as he had conjectured), Newhouse discovered an open set of $C^2$-surface diffeomorphisms which do not satisfy  axiom A condition. His counter example involves a \emph{wild horseshoe}. 
%
%Let $K$ be a basic set for a surface diffeomorphism $f$. Let us recall that a point $z\in K$ displays a \emph{homoclinic tangency} if its stable manifold  $W^s(z ; f)$ is tangent to its unstable manifold $W^u(z ; f)$. 
%If such a point exists, we say that \emph{the horseshoe displays a homoclinic tangency}. We notice that an axiom A diffeomorphism cannot exhibit a horseshoe displaying a homoclinic tangency, for the tangency point is in the non-wandering set, but its stable and unstable directions are not transverse.
%
%\begin{defi}[Wild horseshoe]
%A basic piece is \emph{wild} if it displays a $C^2$-robust homoclinic tangency: 
%for every $C^2$-perturbation $f'$ of $f$, the hyperbolic continuation of $K$ displays a homoclinic tangency.  
%\end{defi}
%
%
%\begin{figure}[h!]
%	\centering
%		\includegraphics[width=7cm]{HorseshoeW.pdf}
%	\caption{Wild horseshoe}
%\end{figure}
%To show the existence of such wild sets, Newhouse defined the concept of stable and unstable thickness for horseshoes (that we will not recall). Then he showed that given a horseshoe $K$, if the product of the thicknesses of $K$ is greater than one and if $K$ displays a homoclinic tangency, then $K$ is wild. 
%
%Although the above sufficient condition is simple to check on many examples, it is not  a necessary condition.  Up to perturbation, a satisfactory description of wild horseshoe involves the Hausdorff dimension.
%
%First, Palis-Takens showed that if the Hausdorff dimension of a horseshoe is less than 1, then it cannot be wild. Then Moreira-Yoccoz proved the following conjecture of Palis:
%\begin{thm}[Palis, Moreira-Yoccoz]
%For every $r\ge 2$, let $K$ be a horseshoe for a $C^r$-diffeomorphism $f$ which displays a homoclinic tangency. Then there exists a $C^r$-perturbation $f'$ of $f$ so that the hyperbolic continuation of $K$ is wild if and only if 
%the Hausdorff dimension of $K$ is  $\ge 1$.
%\end{thm}
%\subsubsection{Sinks Creation}
%
%Since at least the work of Birkhoff, the following is well known:
%\begin{prop}\label{sinkcreation}
%Let $r\ge 1$ and let $f$ be a $C^r$-surface diffeomorphism ($r\ge 1$) and let $P$ be a saddle fixed point which displays a homoclinic tangency and which is area contracting ($|\det\, D_Pf |<1$), then for every $M>0$, there exists a small perturbation $f'$ of $f$ which has a sink $S$ of period at least $M$.
%\end{prop}
%
%It will be important to notice that for the original proof of this proposition,  the invariant probability measure supported by $S$ is close to the one of $P$ when $M$ is large. 
%
%\begin{defi} A \emph{topologically generic set} of a metric set $(E,d)$ is a countable intersection of open and dense sets of $E$. A \emph{locally, topologically generic set}  of $(E,d)$ is a set which  is topologically generic set in a non-empty open set of $E$.
%\end{defi}
%
%\begin{thm}[Newhouse]
%Let $r\ge 2$ and let $f$ be a $C^r$-surface diffeomorphism wich displays a wild horseshoe $K$. Suppose that $|\det Df|K|<1$. Then, there exists a neighborhood $N$ of $f$, and a topologically generic set $\mathcal R\subset N$, so that every $f'\in \mathcal R$ displays infinitely many sinks.
%\end{thm}
%\begin{proof}
%Let $N$ be the open set of perturbations $f'$ for which the hyperbolic continuation $K'$ of $K$ remains wild and the restriction of $f'$ to $K'$ contracts the volume. 
%
%Let $M\ge 0$.  By density of the periodic points in this horseshoe, for every $f'\in N$, there exists a $C^r$-perturbation $f''$ of $f'$ so that there exists a periodic point $P_M$ which displays a homoclinic tangency. 
%
%By Proposition \ref{sinkcreation}, there exists a perturbation $f'''$ of $f''$ which displays an attracting sink $S_M$ of period $\ge M$. Hence the following set is dense:
%\[O_M:= \{f'\in N: \; f'\text{ has a sink of period }\ge M\}\; ,\]
%and it is also open by hyperbolic continuation.  Hence, the following is topologically generic in $N$ and made by diffeomorphisms displaying infinitely many sinks:
%\[\mathcal R:= \bigcap_{M\ge 0} O_M\; .\] 
%\end{proof}
%
%
%In the latter proof, we can choose homoclinic periodic point $P_M$ supporting a \emph{rather} different invariant, probability measure to the previous steps $(P_k)_{k< M}$. Then the sink $S_M$ supports 
%an  invariant, probability measure close to the one of $P_M$ and \emph{rather} different to the ones of $(S_k)_k$. 
%
%Consequently a diffeomorphism in $\mathcal R$ displays infinitely many sinks, and each of which supports a \emph{rather} different invariant probability measure. In this sense, the dynamics of these maps is very complex. 
%
%That is why these dynamics seem to me difficult to be described by the 
%uniformly hyperbolic or non-uniformly hyperbolic theory. 
%
%To measure our misunderstanding of the dynamics exhibiting Newhouse phenomena, let me point out that we do not know if there is a single example of such a map for which Lebesgue almost every point display a Birkhoff sum which converge. 
%
%
%\begin{rema}\label{rema4blender}
%Actually the above proof works if we relax the area contracting hypothesis on the horseshoe $K$  to  a weaker one: the existence of an area contracting periodic point in $K$. Then it is easy to show the existence of a dense set of area contracting periodic points in the horseshoe (by using the semi-conjugacy of the horseshoe dynamics with a finite shift for instance).
%\end{rema}
%
%
%\subsection{Newhouse phenomenon from the Bonatti-Diaz blender}
%Another way to obtain robust homoclinic tangencies involves the blender. This approach has been used by Bonatti-Diaz \cite{BD99} and then by Diaz-Nogueira-Pujals \cite{DNP} to exhibit  a locally, topologically generic set formed by $C^r$-diffeomorphisms of an $n$-manifold which display infinitely, for any $\infty \ge r\ge 1$ and $n\ge 3$. 
%
%Let us present a variation of the argument of \cite{DNP}, in the surface, local diffeomorphism case. 
%
%For local diffeomorphisms, we can define the same concepts as Newhouse's, where a  blender is considered instead of the horseshoe. We recall that a blender of a surface local diffeomorphism is a basic set for an endomorphism so that the union of its unstable manifolds contains robustly a non-empty open set $O$ of the manifold. The set $O$ is called a \emph{covered domain of the blender}. 
%
%Let $r\ge 1$. A blender $K$ of a $C^r$-local diffeomorphism $f$  displays a \emph{homoclinic} tangency if there exists a point $z\in K$ so that its unstable manifold $W^u(z;f)$ is tangent to its stable manifold $W^s(z;f)$.  The blender $K$ is \emph{wild} if every $C^1$-perturbation $f'$ of $f$ exhibits a homoclinic tangency. 
%
%From remark \ref{rema4blender}, the following is straight forward:
%\begin{thm}
%Let $K$ is a blender for a $C^r$-local diffeomorphism $f$. 
%Assume that $K$ contains a periodic point $\Omega$ 
%which is area contracting and that  $K$ is wild, then there exists a $C^r$-neighborhood $N$ of the dynamics so that a $C^r$-generic diffeomorphism $f$ in $N$ displays infinitely many sinks.
%\end{thm}
%
%To accomplish the proof, it remains to prove the existence of such a local diffeomorphism $f$.  Let us cook an example of such dynamics.
%
%\begin{exam}
%\begin{figure}[h!]
%	\centering
%		\includegraphics[width=9cm]{figureCexemplePalis.pdf}
%	\caption{Diaz-Nogueira-Pujals construction}
%\end{figure}
%Let $I_+\sqcup I_-\sqcup I_c$ be three non trivial intervals of $(-1,1)\setminus \{0\}$ and let $Q\colon  I_+\sqcup I_-\sqcup I_c \to [-1,1]$ be a locally affine map which sends each of the intervals  $I_+, I_-, I_c $ onto $[-1,1]$. Let $\lambda:= \frac13 \min(\leb(I_+), \leb(I_c),\leb(I_-))$. 
%Let $K$ be the maximal invariant set of the following map:
%\[f\colon (x,y)\in I_+\sqcup I_-\sqcup I_c \times [-1,1]\mapsto \left\{\begin{array}{cl}
%(Q(x), (2 y+1)/3)& x\in I_+\\
% (Q(x), (2 y-1)/3)& x\in I_-\\
%  (Q(x), \lambda y)& x\in I_c
% \end{array}\right. \]
% We notice that it is a hyperbolic set (with vertical stable direction) and transitive. Hence, it is a basic set. Moreover, we saw in Example  \ref{blender}, that the maximal invariant set of the restriction $f|I_-\sqcup I_+\times [-1,1]$ is a blender with $0$ in its covered domain. Hence $K$ is a blender. 
% 
%Furthermore it contains an area contracting fixed point $\Omega$ in $I_c\times \{0\}$.
%
%To create a homoclinic tangency, it suffices to extend $f$ at a neighborhood $N$ of $0$ so that a local stable manifold of $\Omega$ intersects $N$ at $N\cap \{(x,x^2): x\in \R\}$. Then the blender $K$ displays a homoclinc tangency. In \cite[prop. 2.1]{BE15} we showed that $K$ is wild. \flushright{$\square$}
%\end{exam}
%The following counterpart of the Palis-Moreira-Yoccoz theorem is an open problem:
%\begin{prob}
%For every $r\ge 1$, let $K$ be a blender which displays a homoclinic tangency. Show the existence of a $C^r$-perturbation $f'$ of $f$ so that the hyperbolic continuation of $K$ is wild.
%\end{prob}
%%\begin{prob}
%%\begin{prop}[Wild property of Blender]
%%There exists a $C^1$-neighborhood $U$ of the function $x\in [-1,1]\mapsto x^2$,  there exits a $C^1$-neighborhood $V$ of $f$ so that for every $f'\in V$, every $\phi\in U$, the graph $\gamma$ of $\phi$ is tangent to an unstable manifold of the blender of $f'$.
%%\end{prop}
%%
%%
%%
%%
%%Let us recall the last construction for local diffeomorphisms of surface. For this end we consider a blender as given in example \ref{blender} $K$ together with a saddle fixed point $\Omega$ so that:
%%\begin{itemize}
%%\item The saddle point $\Omega$ is dissipative: $|\det\, D_\Omega f|<1$. 
%%\item The stable manifold of $\Omega$ has a segment which contains the tip of the parabola $x\mapsto x^2$,
%%\item  The unstable manifold of of $\Omega$ intersects transversally the stable manifold of $K$. 
%%\end{itemize}
%
%%\begin{thm}
%%For every $r\ge 1$, there exists a $C^r$-neighborhood $N$ of $f$, and a Baire generic set $\mathcal R\subset N$,  so that every $f'\in R$ has infinitely many sinks. 
%%\end{thm}
%%\begin{proof}
%%Let $f'$ be a $C^r$-perturbation of $f$. 
%%We show in 4 step that for every $M$, there exists a perturbation of $f'$ which has a sinks of period greater than $M$. 
%%\begin{enumerate}
%%\item By the Fundemantal property of the blender, there exists a local unstable manifold $W^u_{loc}(z; f')$ of the blender  which is tangent to the local stable manifold $W^s_{loc}(\Omega, f')$. 
%%\item By the inclination lemma, the unstable manifold of $W^u(\Omega, f')$ accumulates on $W^u_{loc}(z; f')$.
%%\item Hence a small perturbation $f''$ of $f'$ display a (homoclinic) tangency between $W^u(\Omega, f')$ and $W^s_{loc}(\Omega, f')$.
%%\item By Proposition \ref{sinkcreation}, for every $M\ge 0$, there exists a perturbation $f'''$ of $f''$ which has an attracting sinks of period $\ge M$. 
%%\end{enumerate}
%%Hence there exists a neighborhood $N$ of $f$, so that for every $M\ge 0$, the following set is open and dense:
%%\[O_M:= \{f'\in N: \; f'\text{ has a sink of period }\ge M\}\; .\]
%% Consequently the set $\mathcal R:= \cap_M O_M$ is Baire residual and consists of diffeomorphisms which have infinitely many sinks. 
%%\end{proof}
%
%\section{On typicality of the Newhouse phenomenon} \label{NewhouseTypical}
%
%In the last section we saw the existence of a locally, topologically generic set of dynamics which exhibit infinitely many sinks accumulating on a basic set. We saw also that we do not know how to describe these dynamics.
%
%
%\begin{center}
%\emph{However do they appear typically? Do they form a subset of dynamics which is negligible?
%}
%\end{center}
% 
%Indeed locally topologically generic does not mean typical in the probabilistic sense.  For instance, there are topologically generic subsets of the real line whose Lebesgue measure is null (e.g. the set of Liouville numbers).
%
%That is why, until recently, most of the community was rather optimistic, and  believed that the set of maps exhibiting the Newhouse phenomenon should be negligible in some sense. 
%
%
%However there is no canonical measure such as the Lebesgue measure for the Banach spaces. But, we recall that many results of abundance in non-uniformly hyperbolic dynamical system have been formulated thanks to families of dynamics.  This way is somehow similar to the concept of typicality  sketched by Kolmogorov during his famous plenary talk  in the ICM 1954. Here is a version of typicality which appears in many conjectures:
%\begin{defi}[Arnold-Kolmogorov typicality] 
%A property $\mathcal P$ on dynamics of a manifold $M$ is typical 
%if there exists a Baire generic set of $C^d$-families $(f_a)_{a\in \R^k}$ of $C^r$-dynamics so that $\mathcal P$ is satisfied by Lebesgue almost every small parameter $a$. 
%\end{defi}
%
%Hence this definition of typicality involved integers $k,d,r$. 
%\begin{defi}[$C^d$-families of $C^r$-self-mappings]
%A family  $(f_a)_{a\in \R^k}$  of $C^r$-self-mappings $f_a$ of $M$ is \emph{of class $C^d$}, if the following derivatives are well defined and continuous for every $i\le d$ and $i+j\le r$:
%\[\partial_a^i \partial_z^j f_a\colon (a,z)\in \R^k\times M\mapsto \partial_a^i \partial^j_z f_a(z)\]
%The space of $C^d$-families of $C^r$-self-mappings is endowed with the Whitney topology with respect to these derivatives. We recall that the following is an elementary open subset, among $\epsilon\in C^0(\R^k\times M, (0,\infty))$.
%\[ O((f_a)_a, \epsilon):= 
%\{(g_a)_a: \quad d(\partial_a^i \partial_z^j f_a, \partial_a^i \partial_z^j g_a)<\epsilon(a,z)\; , \quad \forall a\in \R^k,z\in M,  i\le d, i+j\le r\}\]
%\end{defi}
%
%To take into account the aforementioned examples and counter examples, there are several conjectures from Tedeschini-Lalli \& Yorke \cite{TLY}, 
% Palis \& Takens \cite{PT93}, Palis himself \cite{Pa00,Pa05,Pa08} -- most of them for low dimensional dynamical systems -- claiming the typicality of the finiteness of the number of attractors. Let us recall the following: 
%\begin{conj}[Pugh-Shub \cite{PS95}]\label{ConjPS}
%Typically (in the sens of Arnold-Kolmogorov) a diffeomorphism of a compact manifold displays at most a finite number of topological attractors (and so sinks). 
%\end{conj}
%All these conjectures aim to describe typical dynamics thanks to finitely many attractors.
%In one-dimensional dynamics or ``skew-product" of one-dimensional dynamics, the seminal works of Lyubich \cite{Ly02} , Tsujii \cite{T05}  and Kozlovsk-Shen-van Strien \cite{KSvS07} gave evidence that this should be true. In higher dimension, the general strategy proposed to prove them was to study the unfolding of stable and unstable manifolds (in analogy with Thom-Mather works in singularity Theory). 
%\subsection{On typically of Newhouse phenomenon for surface diffeomorphisms}
%These conjectures are open for surface\footnote{The topics of surface dynamics was especially important for Smale since he believed that they would reflect already many important behavior of higher dimensional dynamical systems\cite{Sm68}} diffeomorphisms.
%
%Adapting some of Newhouse's results,
%Robinson (see~\cite{Ro83}) later proved that the phenomenon takes place for a residual set of parameters in a
%one-parameter family of diffeomorphisms which non-degenerately unfold a homoclinic tangency.
%
% The first attempt towards a measure theoretic understanding of
%Newhouse phenomenon is due to Tedeschini-Lalli and Yorke (see~\cite{TLY}): they considered a one-parameter unfolding of a
%homoclinic tangency involving a linear horseshoe, and showed that the set of parameters whose corresponding
%diffeomorphism admits infinitely many periodic \emph{simple sinks} (i.e. sinks obtained with the Newhouse construction)
%is a null set for Lebesgue measure.  In the same setting (one-parameter unfolding of a homoclinic tangency involving a
%linear horseshoe), Wang (see~\cite{Wang90}) proved that the Hausdorff dimension of the parameter set of diffeomorphisms
%admitting an infinite number of periodic simple sinks is strictly positive and smaller than $\frac12$.
%
%More recently, Gorodetski and Kaloshin (see~\cite{Kaloshin}) obtained the mea\-sure-ze\-ro result in a much broader
%setting: they introduced a quantitative notion of combinatorial complexity of periodic orbit visiting a neighborhood of a
%homoclinic tangency, which they call \emph{cyclicity}\footnote{ In their terminology, \emph{simple sinks} which were
%  considered above correspond to \emph{cyclicity one sinks}.}.  Their result shows that a \emph{prevalent} dissipative
%surface diffeomorphism in a neighborhood of one exhibiting a non-degenerate homoclinic tangency has only finitely many
%sinks of cyclicity which is either bounded or negligible with respect to the period of the orbit. We will see in section \ref{ArnoldPer} that prevalence does not imply nor implied by the Arnold-Kolmogorov typicality.
%
%The techniques of~\cite{Ne79,Ro83} do not apply to conservative surface diffeomorphisms; on the other hand, a clear
%analog of the Newhouse phenomenon still occurs in a vicinity of conservative diffeomorphisms exhibiting non-degenerate
%homoclinic tangencies, with elliptic islands filling in for the r\^ole of sinks.  This result was finally established by
%Duarte and Gonchenko--Shilnikov (see~\cite{Duarte99,Gonchenko-Shilnikov} and~\cite{Duarte08} for the one-parameter
%version).  In~\cite{JMD} de-Simoi proved an analog to Tedeschini-Lalli--Yorke and Wang result for the Standard Family of conservative diffeomorphisms in the large parameters regime: the set of (sufficiently large)
%parameters for which the Standard Family admits infinitely many simple sinks has zero Lebesgue measure and its Hausdorff
%dimension is not smaller than $1/4$.
%%\footnote{ The
%  %techniques described in the paper in fact allow to prove that the Hausdorff Dimension is also not larger than $1/2$.}
%
%In \cite{BdS15}, we obtained a similar lower bound on the Hausdorff dimension for dissipative surface diffeomorphisms.  We proved that the Newhouse parameter set for a generic family of sufficiently smooth diffeomorphisms which nondegenerately
%unfolds a homoclinic tangency has Hausdorff dimension not smaller than $1/2$:
%\begin{thm}[\cite{BdS15}]
%Let $r\ge 2$ and let $(f_a)_a$ be a $Cr$-non-degenerate unfolding of a  homoclinic tangency for an area contracting saddle point.
%Then the Hausdorff dimension of the following set is at least $1/2$:
%\[\mathcal N:= \{a\in \R: \; f_a \; \text{display infinitely many sinks}\}.\] 
%\end{thm}
%  It is important to stress that our lower bound takes into account non-simple sinks (and thus does not contradict Wang's result) and moreover does not assume
%linearity of the horseshoe.  The proof of our result hinges on two crucial ingredients: the first one is an improved version of Newhouse construction of a wild hyperbolic set; the second one, proved in~\cite{MisuRen}, provides precise estimates on the length of the \emph{stability range} of a sink which is created by unfolding a homoclinic tangency via the Newhouse construction.  As a consequence
%of these results, we obtain that the Hausdorff dimension of the simple Newhouse parameter set for strongly dissipative H\'enon-like families is close to one $1/2$.  We conclude by using Palis-Takens renormalization.
%\subsection{On typically of Newhouse phenomenon for surface self-mappings and diffeomorphisms of higher dimension}
%
%
%Recently, in \cite{BE15}, a mechanism has been found to stop the unfolding for an open set of self-mappings' families. This mechanism is given by the \emph{parablender}, a counterpart of the Bonatti-Diaz Blender for parameter families.
%It will be explained in section \ref{parablender}. This enabled to prove:  
%
%\begin{thm}[\cite{BE15,BE152, Be16}]
%For every manifold $M$ of dimension at least $2$, for every $k\ge 0$, for every $1\le d\le r\le \infty$ and $d<\infty$, there exists an open set $\hat{\mathcal U}$ of $C^d$-families $(f_a)_{a}$ of $C^r$-self-mappings of $M$, so that for a generic $(f_a)_a\in \hat{\mathcal U}$,  \emph{for every} parameter $a\in \R^k$, the map $f_a$ displays infinitely many sinks.
%
%Moreover if $\dim\; M\ge 3$, then the family of self-mappings is made by diffeomorphisms. 
%\end{thm}
%In \cite{BE15, BE152} the result was proved thanks to a parametric counterpart of the wild-blender, after correction we needed $\infty \ge r>d\ge 1$. In a work in progress with Crovisier and Pujals we investigated the case where a source is put in the covered domain of a wild horseshoe. This setting motivated \cite{Be16} where the case $\infty> r=d\ge 1$ was added in the above theorem. This enables to show in 
%the work in progress that the Newhouse phenomenon is actually locally typical in the sense of Arnold-Kolmogorov among surface, finitely smooth, self-mappings.  
% 
%Such results are very disturbing since the general trend was to use the bifurcation theory to show the finiteness of attractors. Here the bifurcation theory enables one to stop the bifurcation and shows the non-typicality of the finiteness of attractors.
%
%Let us end this section by remembering that during his plenary talk at ICM 1954, Kolmogorov said about phenomena which persist for small perturbations along 1-parameter families of dynamical system (that he called stable realization):
%\begin{center}
%\emph{ An arbitrary type of behavior of a dynamical system, for which there exists at least one example of its stable realization, must from this point of view be considered essential and may not be neglected.}
%\end{center}
%From this point of view, Newhouse phenomenon is essential and  may not be neglected. In the next section we propose a way to quantify the complexity of such dynamics. 
%
%\section{Quantifying the complexity of dynamics}
%\label{Ermergencedef}
%
%I would like to propose a statistical interpretation of Yoccoz' quote: 
% \begin{prob}\label{Problemstat}
%Show the existence of an open set of deterministic dynamical systems which typically cannot be described by means of statistics.
% \end{prob}
% 
%In particular, this problem wonders if the statistical mechanics tools, introduced by Sinai in his seminal works (and those of Ruelle, Bowen ...), may describe all typical dynamical  systems.
% 
%
%The aim is not to prove that statistics never apply (they do for many systems!), but that they do not apply for many typical systems, even among the finite dimensional, deterministic differentiable dynamical systems.  
%We shall formalize this problem. For this end, we are going to define the Emergence of dynamical systems. This concept evaluates the complexity to approximate a system by statistics. 
%
%In statistic and computer science, it is standard to use the Wasserstein distance $d_{W_1}$ on the space of probability measures $\mathbb P(M)$ of a compact manifold $M$:
%\[d_{W_1}(\nu,\mu) = \sup_{\phi\in Lip^1(M,[-1,1])}\int_M \phi(x)\, d(\mu-\nu)(x)\;,\quad \forall \nu,\mu\in \mathbb P(M)\]
%where $Lip^1(M,[-1,1])$ is the space of $1$-Lipschitz functions with values in $[-1,1]$. 
%
%Given a differentiable map $f$ of $M$, $x\in M$ and $n\ge 0$, we denote by $\frac1n \sum_{k=0}^{n-1}\delta_{f^k(x)}$ the probability measure which associates to an observable $\phi\in C^0(M,\R)$ the mean $\frac1n \sum_{k=0}^{n-1}\phi({f^k(x)})$.  
%
%\begin{prop}
%Given a probability  measure $\mu$, the following functions are continuous:
%\[x\in M\mapsto d_{W^1}(\frac1n \sum_{k=0}^{n-1}\delta_{f^k(x)},\mu)\in \R\; ,\quad \forall n\]
%\end{prop}
%\begin{proof}
%We notice that it suffices to show that for every $\delta>0$, there exists $\eta>0$ such that if $x$ and $x'$ are $\eta$ distant, then 
%\[d_{W^1}(\frac1n \sum_{k=0}^{n-1}\delta_{f^k(x')},\mu)\ge d_{W^1}(\frac1n \sum_{k=0}^{n-1}\delta_{f^k(x)},\mu)-\delta\; .\]
%We recall that $Lip^1(M,[-1,1])$ endowed with $C^0$-uniform norm is compact, by Arzel\`a-Ascoli Theorem. Hence, there exists $\phi \in L^1(M,[-1,1])$ such that: $$d_{W^1}(\frac1n \sum_{k=0}^{n-1}\delta_{f^k(x)},\mu)= \frac1n \sum_{k=0}^{n-1}\phi({f^k(x)})- \int_M \phi d\mu\; .$$ As $\phi$ and  $(f^k)_{k\le n}$ are Lipschitz, there exists $\eta>0$ so that for $x'$ $\eta$-close to $x$, it holds:
% \[ \frac1n \sum_{k=0}^{n-1}\phi({f^k(x')})\ge
% \frac1n \sum_{k=0}^{n-1}\phi({f^k(x)})-\delta\Rightarrow d_{W^1}(\frac1n \sum_{k=0}^{n-1}\delta_{f^k(x')},\mu)\ge d_{W^1}(\frac1n \sum_{k=0}^{n-1}\delta_{f^k(x)},\mu)-\delta\; .\]
%% \[d_{W^1}(\frac1n \sum_{k=0}^{n-1}\delta_{f^k(x')},\mu)\ge  \frac1n \sum_{k=0}^{n-1}\phi({f^k(x')})- \int_M \phi d\mu\ge
%% \frac1n \sum_{k=0}^{n-1}\phi({f^k(x)})- \int_M \phi d\mu-\delta\]
%\end{proof}
%
%We recall that the space of probabilities over a compact manifold and endowed with the metric $d_{W^1}$ is relatively compact.
%
%Hence, given a differentiable map $f$ of $M$, we can define the \emph{Emergence $\mathcal E(f,\epsilon)$ of $f$ at scale $\epsilon>0$} as the minimal number $N$ of probability measures $\{\mu_i\}_{1\le i\le N}$  so that 
%\[\limsup_{n\to \infty} \int_{x\in M} \min_{1\le i\le N} d_{W^1}(\frac1n \sum_{k=0}^{n-1}\delta_{f^k(x)},\mu_i) \, d\leb\le \epsilon\; .\]
%
%
%\begin{defi}[Emergence]
%The Emergence is \emph{\bf F} if $\mathcal E(f,\epsilon) = O(1)$ when $\epsilon\to 0$.
%
%The Emergence is at most \emph{\bf P}  if there exists $k> 1$ so that $\mathcal E(f,\epsilon) = O(\epsilon ^{-k})$.
%
%The Emergence is \emph{\bf Sup-P} if $\limsup \frac{\log \mathcal E(f,\epsilon)}{-\log\epsilon}=+\infty$.
%\end{defi}
%We notice that the Emergence is a lower bound on 
%the complexity (in space\footnote{The number of data to store.} and in time) to approximate numerically a  dynamical system by statistics with  precision  $\epsilon$. Following, the celebrated Cobham's thesis an algorithm in Sup-P is -- in practical -- not feasible \cite{Co65}. 
%
%%Note that the Emergence is invariant by differentiable conjugacy. Also the Emergence of a product of two systems is the product of their Emergences.  
%
%\medskip
%
%\noindent{\bf Examples with {\bf F}-Emergence}
%If a dynamical system $f$ admits finitely many  ergodic attractors $(\Lambda_i,\mu_i)_{1\le i\le N}$ whose basins $(B_i)_i$ cover Lebesgue almost all the manifold, then the Emergence is bounded by $N$ (and so it is of type { F}).
%\begin{proof} By the dominated function theorem, it suffices to show that for every $i\le N$ and every $x\in B_i$,  $d_{W^1}(\frac1n \sum_{k=0}^{n-1}\delta_{f^k(x)},\mu_i)\to 0$. By compacity of $Lip^1(M,[-1,1])$, for every $n$, there exists $\phi_n \in Lip^1(M,[-1,1])$ so that: 
%\[\Delta_{ n}:= d_{W^1}(\frac1n \sum_{k=0}^{n-1}\delta_{f^k(x)},\mu_i)= \int_{M} \frac1{n} \sum_{k=0}^{n-1}\phi_{n}(f^k(x))d\leb- \int_{M} \phi_n d\mu_i \; .\]
%Let $\phi\in Lip^1(M,[-1,1])$ be a cluster value of $(\phi_n)_n$ and let $(n_j)_{j\ge 0}$   be an increasing sequence so that $\phi_{n_j}\to \phi$. Then 
%\[\Delta_{ n_j}\le  2\int_M \|\phi_{n_j}-\phi\|_{C^0}d\mu_i+\int_{M} \frac1{n_j} \sum_{k=0}^{n_j-1}\phi(f^k(x))d\leb- \int_M 
%\phi d\mu_i\to 0\; .\]
%Thus every cluster value of $(\Delta_{ n})_n$ is zero, and so this sequence converges to zero.
%\end{proof}
%\begin{rema}
%We recall that a diffeomorphism satisfying Axiom A,   an irrational rotation or a H\'enon map for Benedicks-Carleson parameters have finitely many ergodic attractors whose basins cover Lebesgue almost all the phase space $M$. Hence their Emergences are finite.
%\end{rema}
%
%\noindent{\bf Example with { P}-Emergence.}
%Let $f$ be the identity. Observe that $\mathcal E(f,\epsilon) = O(\epsilon^{-n})$ with $n$ the dimension of $M$. Hence its Emergence is polynomial. Also the Emergence of an irrational rotation on a cylinder, which is the product of systems with Emergences 1 and $O(\epsilon^{-1})$, is $O(\epsilon^{-1})$.
%% and so is of linear type. 
%
%It seems also possible to prove that the Emergence of the so-called Bowen eyes dynamics is $O(\epsilon^{-1})$. 
%
%\medskip
%
%Hence, it seems that all the well understood dynamical systems have an Emergence at most {P}.
%However, the main conjecture of this work states that those of Sup-P Emergence should not be neglected:
%\begin{conjecture}\label{mainconj}
%There exists an open set $U\subset Diff(M)$ so that a typical $f\in U$ has Emergence { Sup-P}.
%\end{conjecture} 
%Let us explain why a proof of this conjecture would solve Problem \ref{Problemstat} from the computational view point. Given a typical $f\in U$, to describe by means of statistics with precision $\epsilon$, all of its orbit, but a proportion Lebesgue measure $1-\epsilon$, we would need at least a super-polynomial number of invariant probabilities w.r.t. $\frac1\epsilon$. To find them by means of statistics, we need \emph{at least} one data for each of them, and so to do a super polynomial number of operations. By Cobham's thesis this is not feasible by a computer.   
%
%Also we notice that when the Emergence is Sup-P, the Hausdorff dimension of the set of probabilities which would model our system is infinite.
%
%Hence to find these invariant probabilities, we would not be able to use the (finite dimensional) parametric statistics, but only the non-parametric ones, whose computational cost is higher (and much more than 1 as in the above lower bound).
%
%Furthermore let us notice that the product of $f\in Diff(M)$ with the identify of compact manifold $N$ of dimension $d$ displays an Emergence $\mathcal E(f\times id_N,\epsilon)$ dominated by $\epsilon^{-d} \times \mathcal E(f,\epsilon)$. 
%Indeed the Birkhoff sum of the product of the dynamics on $M\times N$ are the product of a Birkhoff sum of $M$ with a Dirac of a point in $N$. Hence it is a product of an invariant probability measure of $id_N$ with an invariant probability measure of $M$. Thus 
%\[\mathcal E(f\times id_N,\epsilon)= \mathcal E(id_N,\epsilon)\times \mathcal E(f,\epsilon)=O(\epsilon^{-d})\times \mathcal E(f,\epsilon)\]
%During a seminar  I gave, Ledrappier asked the following question:
%\begin{question} Supppose that $G$ is a compact, Lie group of symmetries acting smoothly on $M$. Suppose that $f\in Diff(M)$ is equivariant by $f$: There exists $x\in M\mapsto \sigma_x\in Aut(G)$ such that $f(g\cdot x)= \sigma_x(g)\cdot f(x)$. Then $f$ projects on the quotient $M/G$ as a map $\check f\colon M/G\to M/G$. Is that true that $\mathcal E(f,\epsilon)$ is dominated by $\mathcal E(\check f,\epsilon)\times \epsilon^{-\dim \, G}$?
%\end{question}
%\medskip
%
%Note that Newhouse phenomenon is an example of infinite Emergence, but we do not know if we can exhibit one example of at most polynomial Emergence.  Similarly KAM theory provides examples of at least $P$-Emergence in the conservative setting.
%
%\medskip
%
%\noindent{\bf Candidates for Sup-P-Emergence.} It is perhaps  possible to construct a unimodal map with Sup-P Emergence from \cite{HK90}, or a locally $C^r$-dense set of surface diffeomorphisms with Sup-P Emergence from \cite{Ki15}.  It would be very challenging to derivate from these systems one which is moreover locally typical. 
%
%
%It is perhaps possible to make a variation of Newhouse's construction to produce a generic dynamics with { Sup-P} Emergence.   
%
%Let me mention also the concept of universal dynamics of Bonatti-Diaz \cite{BD02} and Turaev \cite{Tu15} which might produce locally Baire generic sets of diffeomorphisms with high Emergence.
%
%\medskip 
%It would be interesting to study Conjecture \ref{mainconj} w.r.t. different notions of typicality \cite{HK10} and smoothness.  Also it might be interesting to investigate the concept of Emergence for other metrics than $W_1$ on the space of invariant probability measures. 
%
%\medskip 
%
%Also it would be interesting to provide numerical evidences for such a program (from big data?). The following problem remains open.
%\begin{prob}
% Show numerical simulations depicting a (typical) dynamical systems which displays infinitely many sinks. 
%\end{prob}
%Let us point out that by definition, a Sup-P Emergent  dynamical system is very complex to describe, and so the non-existence of such pictures is  consistent with their conjectured local typicality.
% 
%  \section{Growth of the number of periodic points}\label{ArnoldPer}
% As we introduced, in some extends, the growth of the number of periodic points depicts the complexity of the dynamics studied. Let us develop this topic.
%  
%Let $f$ be a diffeomorphism (or a local diffeomorphism)  of a compact manifold $M$. We denote by  $Per_n f:= \{x\in M: f^n(x)=x\}$ the set of its $n$-periodic points. To study its cardinality, we consider also the subset $Per_n^0 f\subset Per_n f$ of isolated $n$-periodic points. We notice that the cardinality of  $Per^0_n f$ is an invariant by conjugacy. Hence it is natural to study the growth of this cardinality with $n$. 
%
%Clearly, if $f$ is a polynomial map, the cardinality of $Per^0_n f$ is bounded by the degree of $f^n$, which grows at most exponentially \cite{DNT16}. 
%
%The first study for the $C^\infty$-case goes back to Artin and Mazur \cite{AM65} who proved that there exists a dense set $\mathcal D$ in $Diff^r(M)$, $r\le \infty$, so that for every $f\in \mathcal D$, the number $Card\, Per^0_n f$  grows  at most  exponentially, i.e. , there exists  $K(f)>0$ so that:
%\begin{equation}\tag{A-M} \frac1n \log Card\, Per^0_n f\le K(f)
%%\exp(K(f)\cdot n)
%\; .\end{equation}
%
%This leads Smale \cite{Sm} and Bowen \cite{Bo78} to wonder about the relationship between  rate of growth of the number of periodic points on one hand and dynamical $\zeta$-function or topological entropy on the other hand for (topologically) generic diffeomorphisms. In particular, these questions asked whether (A-M) diffeomorphisms are generic. Finally Arnold asked the following problem:
%\begin{prob}[Smale 1967, Bowen  1978, Arnold  Pb. 1989-2 \cite{Ar00}] \label{problem1} Can the number of fixed points of the $n^{th}$ iteration of a topologically generic infinitely smooth self-mapping of a compact manifold grows, as $n$ increases, faster than any prescribed sequence $(a_n)_n$ (for some subsequence of time values n)? 
%\end{prob}
%We recall that a property is \emph{topologically generic} if it holds for a countable intersection of open and dense sets. The topology on the space of $C^\infty$-maps is the union of the ones induced  by the $C^r$-topologies $C^r(M,M)$ among $r\ge 0$ finite. 
%
%Another notion to quantify the  abundance of a phenomenon 
%was sketched by Kolmogorov during its plenary talk at the ICM 1954. Then his student Arnold formalized it as follows \cite{IL99, KH07}:
%\begin{defi}[Arnold's Typicality]\label{Deftyp}
%A property $(\mathcal P)$ on the set of $C^r$-mappings $C^r(M,M)$ of $M$ is \emph{typical} if for every $k\ge 1$, for a topologically generic $C^r$-family of $(f_a)_{a\in \R^k}$ of $C^r$-maps $f_a$,
%%\in C^r(M,M)$, 
%for Lebesgue almost every $a\in \R^k$, the map $f_a$ satisfies the property $(\mathcal P)$.
%\end{defi}
%We recall that $(f_a)_{a\in \R^k}$ is of class $C^r$ if the map $(a,z)\mapsto f_a(z)$ is of class $C^r$. When $r<\infty$, the topology on this space is equal to the compact-open $C^{r}$-topology of $C^{r}(\R^k\times M,M)$.  When $r=\infty$,  
%the topology on the space of $C^\infty$-families is the one given by the union of those induced  by the $C^{r'}$-topologies among $r'\ge 0$ finite.
%\begin{prob}[Arnold 1992-13 \cite{Ar00}]\label{problem2} Prove that a typical, smooth, self-map $f$ of a compact manifold satisfies that $(Card\, Per_n f)_n$ grows  at most exponentially fast.  
%\end{prob}
%\begin{rema}
%Many other Arnold's problems are related to this question \cite[1994-47, 1994-48, 1992-14]{Ar00}.
%\end{rema}
%%We recall that a family $(f_a)_a$ is of class $C^r$ if the map $(a,z)\mapsto f_a(z)$ is of class $C^r$. Hence the space of families of diffeomorphisms is endowed with the topology induced by the one of $C^r(\R^k\times M, M)$.  
%
%\medskip
% 
%These problems enjoy a long tradition. 
%
%In dimension 1, Martens-de Melo-van Strien \cite{MdMvS92} showed that for every $\infty\ge r\ge 2$, for an open and dense set\footnote{whose complement is the infinite codimentional manifold formed by maps with at least one flat critical point.} of $C^r$-maps the number of periodic points grows at most exponentially.
%
%Kaloshin \cite{K99} answered to a question of Artin and Mazur (in the finitely smooth case) by proving that for a dense set  $\mathcal D$ in $Diff^r(M)$, $r< \infty$, the set $Per_n f$ is finite for every $n$ (and so equal to $Per^0_n f$) and its cardinality grows  at most  exponentially fast. 
%
%However, in \cite{K00}  Kaloshin proved that for $2\le r<\infty$ and $\dim M\ge 2$,  \emph{a locally topologically generic diffeomorphism displays a fast growth of the number of periodic points}: 
%there exists an open set $U\subset Diff^r(M)$,  so that for any sequence of integers $(a_n)_n$, a generic $f\in U$ satisfies:
%\begin{equation}\tag{$\star$}
%\limsup_{n\to \infty} \frac{Card\, P_n^0(f)}{a_n}=\infty\; .\end{equation} 
% Furthermore, Bonatti-Diaz-Ficher \cite{BDF08} extended this result to the $C^1$-case in dimension $\ge 3$.  The counterpart of this result in the conservative case has been proved by Kaloshin-Saprykina in \cite{KS06}. Kaloshin theorem is based on a result by Gonchenko-Shilnikov-Turaev \cite{GST93,GST99} that surface diffeomorphisms with degenerate parabolic points form a localy dense subset of $Diff^r(M^2)$; since their proof in \cite{GST99,GST07} is valid in the $ C^\infty$-case, Kaloshin theorem immediately extends to  the $C^\infty$-case as well. Recent seminal work of Turaev \cite{T15} also implies that among $C^\infty$-surface diffeomorphisms the fast growth of the number of periodic points is locally a topologically generic property.
%
%However the $C^\infty$-case in dimension $\ge 3$ remained open\footnote{in \cite{GST93b}, Theorem 7,  Gonchenko-Shilnikov-Turaev theorem was claimed to be true for any dimension but the proof has never been published.}. In this term, our first result accomplishes the study of problem \ref{problem1}, in any smoothness $\ge 2$ and any dimension: 
%
%%The latter was proved in details by Kaloshin in \cite{K00}. Furthermore, Bonatti-Diaz-Ficher \cite{BDF08} extended this result to the $C^1$-case in dimension $\ge 3$. 
%%The counter part of this result in the conservative case has been proved by Kaloshin-Saprykina in \cite{KS06}. The $C^\infty$-case has remained open until the recent seminal work of Turaev \cite{T15}, where he showed that among $C^\infty$-surface diffeomorphisms, 
%%{the fast growth of the number of periodic point is  locally a topologically generic  property}.
%%
%% However the $C^\infty$-case in dimension $\ge 3$ remained open.
%%In this term, our first result accomplishes the study of problem \ref{problem1}, in any smoothness $\ge 2$ and any dimension: 
%%rby solving the $C^\infty$-case in any dimension.
%\begin{thm}[\cite{Be17}]\label{theoA}
%Let $\infty\ge r\ge 2$ and  let $M$ be a compact manifold of dimension $d$. 
%
%If $d=1$, Property $(AM)$ is satisfied by an open and dense set of $C^r$-self-mappings.
%
%If $d\ge 2$, there exists a (non empty) open set $U\subset Diff^r(M)$ so that given any sequence $(a_n)_n$  of integers, a topologically generic $f$ in $U$ satisfies $(\star)$.
% \end{thm}
%Actually, the proof of this theorem will be done in dimension $\ge 3$ for the diffeomorphism's case, and dimension $2$ for self-mappings. For the one dimensional case has been proved in \cite{MdMvS92} whereas the surface diffeomophism's case is proved as aforementioned.
%
%
%This result is proved following a method which contains one aspect related to the work  Asaoka-Shinohara-Turaev \cite{AST16} on the fast growth of the number of periodic points for a locally generic free group action of the interval. In our proof, basically a free group of diffeomorphisms of the circle is embedded into 
%the manifold as a normally hyperbolic fibration by circles.
%As in Asaoka-Shinohara-Turaev's approach, we consider a robust hetero-dimensional cycle given by 
%%source which is robustely covered by
% a Bonatti-Diaz Blender \cite{BD96}. Thanks to a new renormalization trick, we exibit a dense set of perturbations which display
%%surface dynamics -thanks to Bonatti-Diaz Blender \cite{BD96}. The fast growth of the number of periodic points is then proved thanks to a renormalization trick exhibiting densely 
%a parabolic dynamics on an invariant,  finite union of circles. We perturb it to 
%% that we can perturb to 
% a rotation thanks to Herman-Yoccoz' development of KAM-theorem. Then it is easy to construct a topologically generic  perturbation which exhibits a fast growth of the number of periodic points.   
%
%\medskip
%
%As there are  topologically generic sets of the real line whose Lebesgue measure is null,  a negative answer to Problem \ref{problem1} does not need to suggest a negative answer to Problem \ref{problem2}. 
%
%To provide a positive answer to Arnold Problem \ref{problem2}, Hunt and Kaloshin \cite{KH07} used a method described in \cite{GHK06} to show that for $\infty\ge r>1$, a \emph{prevalent}  $C^r$-diffeomorphism satisfies:
%\begin{equation}\tag{$\Diamond$}
%\limsup_{n\to \infty} \frac{\log P_n(f)}{n^{1+\delta}}= 0,\quad \forall \delta>0\; .\end{equation}
%The notion of prevalence was introduced by  Hunt, Sauer and York \cite{HSY92}. A property is \emph{prevalent} if \emph{roughly speaking} almost all perturbations in the embedding of a Hilbert cube at every point of a Banach space (like $C^r(M,M)$), the property holds true. We notice that  $(\Diamond)$ is satisfied for a prevalent diffeomorphism but not for a topologically generic diffeomorphism  (see other examples of mixed outcome in \cite{HK10}). 
%
%However the latter did not completely solve Arnold's problem \ref{problem2} in particular because the notion of prevalence is \emph{a priori}  independent to the notion of typicality initially meant by Arnold. Indeed his problem was formulated for typicality in the sense of definition \ref{Deftyp} (see explanation below problem 1.1.5 in \cite{ KH07}). 
%
%In this term the second and main result of this work is surprising since it provides a negative answer to  Arnold's problem \ref{problem2} in the finitly smooth case:
%
%\begin{thm}[\cite{Be17}]\label{theoB}
%Let $\infty> r\ge 1$ and $0\le k<\infty$, let $M$ be a manifold of dimension $\ge 2$, and let $(a_n)_n$ be any sequence of integers. 
%
%Then there exists a (non-empty) open set $\hat U$ of $C^r$-families $(f_a)_a$ of $C^r$-self-mappings $f_a$ of $M$ so that a topologically generic $(f_a)_a\in \hat U$ consists of maps 
%$f_a$ satisfying $(\star)$,  for every $\|a\|\le 1$.
%%, the map . 
%Moreover if $dim \, M\ge 3$, we can choose $\hat U$ to be formed by families of diffeomorphisms.
%\end{thm}
%\begin{rema}\label{remaCinfty}
%Actually, the same proof shows that the statement of Theorem \ref{theoB} holds true in the category of $C^r$-families of $C^\infty$-self-mappings.
%\end{rema}
%The proof of this theorem follows the same scheme as for Theorem \ref{theoA}, beside the fact that the blender is replaced by a new object: the $\lambda$-parablender (which generalizes  both the blender and the parablender as introduced in \cite{BE15}). A generalization to the parameter case of the renormalization trick enables us to display a dense set of families of self-mappings which leave invariant a finite union of normally hyperbolic circles on which the restrictions are constantly parabolic. Then a careful  study of the parabolic bifurcation and renormalization together with KAM-Herman-Yoccoz' Theorem enables us to perturb these families by one which exhibits  a constant family of rotations. Finally it is easy to perturb the family so that it displays a fast growth of the number of periodic point at every parameter $\|a\|\le 1$. 
%
%This last step has been implemented recently by Asaoka  \cite{As16} who showed that  for every $r\in\{\infty,w\}$, there exists a $C^r$-open set of \emph{conservative} surface diffeomorphisms in which typically in the sense of Arnold, a map displays a fast growth of the number of periodic points. Indeed, he observed that, in the conservative case, a mere application of KAM theory implies the existence of diffeomorphisms which leave  invariant and persistent circles and act on them as irrational rotations (whose angles are constant for conservative perturbations).
%
% 
%   
% 
%To conclude the presentation of these results on the growth of the number of periodic points, let me recall that Arnold's philosophy  was not to propose \emph{Problems of binary type admitting a ``yes-no" answer}, but rather to propose \emph{wide-scope programs of explorations of new mathematical (and not only mathematical) continents, where reaching new peaks reveals new perspectives, and where a preconceived formulation of problems would substantially restrict the field of investigations that
%have been caused by these perspectives. [...]
%%It is not sufficient to know whether there is a river beyond the mountain; it does remain to cross this river! 
%Evolution is more important than achieving records,} as he explained in his preface \cite{Ar00}.
%
%Let us remark that in this sense the contrast between the result of Kaloshin-Hunt and Theorem \ref{theoB} is interesting since they shed light how an answer to a question might depend on the definition of typicality. 
%
%Furthermore, the proofs of this work do not only answer questions, it also develops new tools which will certainly be useful for our program on emergence \cite{Be16}. Let us notice that the $C^\infty$-case of problem \ref{problem2}  (or conjecture 1994-47 \cite{Ar00}) remains open, although in view of Remark \ref{remaCinfty}, I would bit for a negative answer; I would even dare to propose:
%% which remains open (but in my opinion are a priori false as asked):
%\begin{conj}\label{conjprincipal}
%For every $r\in \{1,..., \infty,\omega\}$,  there exists an open set of diffeomorphisms $U\in Diff^r(M)$, so that given any $k\ge 0$, for any $C^r$-generic family $(f_a)_{a\in \R^k}$ with $f_a\in U$, for every $a$ small, the growth of the number of periodic points of $f_a$ is fast.
%\end{conj}
%%\thanks{I am thankful to Abed Bounemoura,   Hakan Eliason, Bassam Fayad, Vadim Kaloshin and Rafael Krikorian for interesting conversations. I am grateful to Sylvain Crovisier and Enrique Pujals for inspiring discussions and Dimitry Turaev for important comments on the parabolic renormalization for circle diffeomorphisms.}
%
%\subsection{A circle mapping based proof}
%
%As in Asaoka works\cite{As16}, the theorems are proved by exhibiting an invariant circle whose rotation number is Diophantine. We will see below why it is sufficient. In the surface conservative case (which is the setting of Asaoka),  a non-degenerated elliptic fixed point exhibits a robust invariant circle with constant, Diophantine  rotation number (by KAM theorem). In our dissipative case, there is no invariant circle with rotation number robustly Diophantine. To show the existence of a dense set of such dynamics we shall introduce the $\lambda$-blenders and their para-version in the next section. They enable to exhibit invariant circles with parabolic dynamics, that we perturb to diophantine rotations. 
%
%Let us recall the definitions related to one dimensional dynamics.
%
%
%%We recall that our strategy is close to Asaoka one's, 
%%We shall first recall some elements of one dimensional dynamics, and more specifically the concept of parabolic map and the KAM-Herman theorems. 
%%
%%Then we will recall some elements of hyperbolic theory, including the concepts of  blender and a new feature, the $\lambda$-blender which will be useful to construct a dense set of parabolic maps on normally hyperbolic embedded circles.
%%
%%Afterward we will generalize these concepts to parameter families. The concepts of blender and $\lambda$-blender will be generalized to the $C^r$-parablender and a new object, the $C^d$-$\lambda$-parablender. 
%%  
%%Finally we will recall the Hirsch-Pugh-Shub Theory \cite{HPS}, and its extension \cite{berlam} for the endomorphisms case, for they will be usefull for the proof of the theorems. 
%  
%\paragraph{Rotation number}
%Given a homeomorphism $g\in Diff^0(\R/\Z)$ of the circle $\R/\Z$, one defines its rotation number $\rho_{g}$ as follows.  We fix $G\in Diff^0(\R)$ a lifting of $g$ for the canonical projection $\pi\colon \R\to \R/\Z$:
%\[\pi\circ G= g\circ \pi\; .\]
%Then Poincar\'e proved that $\rho_G = \lim_{n\to \infty} G^n(0)/n$ is uniquely defined. Furthermore, $\rho_g= \pi(\rho_G)$ does not depend on the lifting $G$ of $g$. The \emph{rotation number} of $g$ is $\rho_g$. 
%
%It is easy to show that the rotation number depends continuously on $g$. 
%
%\paragraph{Maps with Diophantine rotation number}
%A number $\rho\in \R$ is \emph{Diophantine}, if there exist $\tau>0$ and $C>0$  so that for every $p,q\in \N\setminus \{0\}$ it holds:
%\[ |q \rho -p|\ge C q^{-\tau} \; .\]
%Let us recall that the set of Diophantine numbers is of full Lebesgue measure. 
%
%Here is a Yoccoz' development of Herman's theorem on Arnold Conjecture:
%\begin{thm}[Arnold-Herman-Yoccoz \cite{He79,Yo84}]\label{AHthm}
%If the rotation number of  $g\in Diff^\infty(\R/\Z)$ is a Diophantine number $\rho$, then there exists $h\in Diff^\infty(\R/\Z)$ which conjugates $g$ with the rotation $R_\rho$ of angle $\rho$:
%\[h\circ g\circ h^{-1} = R_\rho\; .\]
%Moreover, if $(g_a)_a$ is a $C^\infty$-family of diffeomorphisms with constant rotation number $\rho$ which is Diophantine,  then there exists a $C^\infty$-family $(h_a)_a$ of diffeomorphisms $h_a$  which conjugates $(g_a)_a$ with $R_\rho$:
%\[h_a\circ g_a\circ h_a^{-1} = R_\rho\; .\]
%\end{thm}
%
%Hence once we exhibit an invariant $C^\infty$-circle on which the dynamics displays a diophantine rotation number $\rho$, by the above theorem, we exhibit an invariant circle on which $f$ acts as a rotation of angle $\rho$. Then it is easy to perturb this dynamics of the circle to a root of the identity (by moving $\rho$ to a rational number). In particular an iterate of $f$ is the identity on an interval. We can perturb to create an arbitrarily large number of periodic points.
%%AN ARBITRARILY LARGE NUMBER OF PERIODIC POINTS.
%%periodic point Finally we will show that it is easy to perturb a root of the identity to a diffeomorphism with a large number of hyperbolic periodic points. 
%
%\paragraph{Parabolic maps of the circle}
%
%
%A key new idea in this work is to exhibit circle diffeomorphisms with 
%Diophantine rotation number by creating first \emph{parabolic diffeomorphisms of the circle} (they are indeed easier to exhibit densely thanks to geometrical arguments). 
%
%\begin{defi} A $C^2$-diffeomorphism $g$ of a circle $\T$ is \emph{parabolic} if there exists $p\in \T$ so that 
%\begin{itemize}
%\item The point $p$ is the unique fixed point of $g$,
%\item  The point $p$ is a non-degenerated parabolic fixed point of $g$:
%\[g(p)=p,\quad D_pg= 1\quad ,\quad D^2_p g\not= 0\quad .\] 
%\end{itemize}
%\end{defi}
%
%This idea might sound anti-intuitive since the rotation number of a parabolic map is zero. 
%
%The interest of parabolic maps of the circle is that they have a geometric definition and produce irrational rotations after perturbations. Indeed  if $g$ is a $C^r$-parabolic circle map, with $r\ge 2$, then its rotation number is $0$. Also, one sees immediately that the composition $R_\epsilon\circ g$, with $R_\epsilon$ a rotation of angle $\epsilon>0$ small, has non-zero rotation number. Hence, by continuity of the rotation number and density of Diophantine number in $\R$, we can chose $\epsilon>0$ arbitrarily small so that the rotation number $\rho(\epsilon)$ of $R_\epsilon \circ g$ is Diophantine.  This proves:
%\begin{prop}\label{prethmS4}
%For every $r\ge 2$, the set $D^r$ of $C^r$-circle maps with Diophantine rotation number accumulates on the set $P^r$ of $C^r$-parabolic maps:
%\[cl(D^r)\supset P^r\; .\]
%\end{prop}
%
%The above argument is topological. Hence the following is a non trivial extension of the latter proposition for parameter families.
%%, proved in section \ref{ProofthmS4}:
%\begin{thm}\label{thmS4}
%Let $k\in \N$ and let $V\subset \R^k$ be an open  subset and $V'\Subset V$. Given any $C^\infty$-family 
% $(g_a)_{a\in V}$  of circle maps so that for every $a\in V$ the map $g_a$ is parabolic, there exists an arbitrarily small  Diophantine number $\alpha>0$,  there exists a small $C^\infty$-perturbation $(g'_a)_a$ of $(g_a)_a$ so that:
% \begin{itemize}
% \item the rotation number of $g_a'$ is $\alpha$ for every $a\in V'$.
% \item the family $(g'_a)_{a\in V}$ is of class $C^\infty$.  
% \end{itemize}
%\end{thm}
%%\marginal{Attention il faut revenir au $C^r$}
%The proof involves the parabolic renormalization for an unfolding of $(g_a)_a$, and the Arnold-Herman-Yoccoz Theorem. 
%
%\paragraph{A trick on cocycles}
%To produce parabolic maps of the circle, we consider a normally hyperbolic fibration by circles, so that the system is roughly speaking given by an iterated function systems by finitely many circle mappings. 
%
%We wish to recover densely the following situation which produces a parabolic map $h$:
%\begin{fact}
%Let $f_1(x) =\frac23 x$ and $f_2(x) = \frac x{2x+1}$. It holds:
%\[f_2^n\circ f_1^n(x)\underset{n\to \infty}{\to} \frac x{2x+1}=:h(x)\; .\] \end{fact}
%\begin{proof} Indeed this follows from the fact that $f_2= h\circ f_1^{-1}\circ h^{-1}$, and so 
%$f_2^n= h\circ f_1^{-n}\circ h^{-1}$. Restricted to any compact subset of $\R$, the map $h^{-1} \circ f_1^{n}$ is equivalent to $f^n_1$ and so the limit follows.\end{proof}
%This observation is useful since $h$ is parabolic. We observe that if $f_1'$ and $f_2'$ are $C^2$-perturbation of $f_1$ and $f_2$, so that they fix the same point with whose eigenvalue are reciprocal, it holds also that  $f'^n_2\circ f'^n_1$ converges to a parabolic map. 
%
%Therefore, it suffices to construct an IFS whose finite compositions construct a dense set of fixed points and eigenvalues nearby $0$ and $2/3$. This is given by a $\lambda$-blender. 
%
%
% \section{Blenders, parablenders and their $\lambda$-versions}
%\label{parablender}
%In this section we recall the notion of blender,  $\lambda$-blender and parablenders  which enable to prove the results of \cite{BE15, Be16, Be17}.
%\subsubsection{Blender} A hyperbolic set $K$ of a surface local diffeomorphism $f$ is a Bonatti-Diaz' \emph{blender} \cite{BD96} if $\dim E^u=1$ and a continuous union of local unstable manifolds $\cup_{\arr z\in \arr K} W^u_{loc} (\arr z; f)$ contains robustly a non-empty open set $O$ of $M$:
%\[\bigcup_{\arr z\in \arr K} W^u_{loc} (\arr z; f')\supset O \quad ,\quad \forall f' \; \text{$C^1$-close to } f\; .\]
%The set $O$ is called a \emph{covered domain} of the blender $K$. 
%
%\begin{figure}[h]
%    \centering
%        \includegraphics[width=7cm]{blenderbw.pdf}
%    \caption{A blender of a surface.}
%%    \label{preimagedeltaa}
%\end{figure}
%
%%Blender were discovered in \cite{BD96}, and then used in \cite{BD99,DNP} to produce a locally generic set of diffeomorphisms displaying infinitely many sinks. 
%%In \cite{BE15}, the notion of blender has been adapted to local surface diffeomorphisms to produce a locally generic set of surface local diffeomorphism displaying infinitely many sinks following a similar argument to \cite{DNP}. 
%  
%  
% \begin{exam}\label{example10}
%  Let $I_{-1}\sqcup I_{+1} $ be a disjoint union of non-trivial segments in $(-1,1)$. Let $\sigma$ be a map which sends affinely each $I_{\pm 1}$ onto $[-1,1]$. Put:
%  \[f:(x,y)\in [-2,2]\times I_{-1}\sqcup I_{+1}\mapsto \left\{\begin{array}{cc}
%f_{+1}(x,y)=  (\frac23(x-1)+1, \sigma(y))& \text{if } y\in I_{+1}\\
%f_{-1}(x,y)=    (\frac23(x+1)-1, \sigma(y))& \text{if } y\in I_{-1}\end{array}\right.\]
% Let $K:=\cap_{n\ge 0} \sigma^{-n}(I_{-1}\sqcup I_{+1})$ be the maximal invariant of $\sigma$, and let $B:= [-1,1]\times K$. We notice that $B$ is a hyperbolic set for $f$ with vertical stable direction and horizontal unstable direction.  Given a pre-sequence $\underline \sb =(\sb_i)_{i\le -1}\in \{-1,1\}^{\Z^-}$ we define the local unstable manifold:
% \[W^u_{loc}(\underline \sb;f):= \bigcap_{i\ge 1} f^i( [-2,2]\times I_{\sb_{-i}})\; .\]
%We notice that for any  $C^1$-perturbation $f'$ of $f$,  the following is a hyperbolic continuation of $W^u_{loc}(\underline \sb;f)$: 
%  \[W^u_{loc}(\underline \sb;f'):= \bigcap_{i\ge 1} f'^i( [-2,2]\times I_{\sb_{-i}})\; .\]
% \end{exam} 
%  \begin{fact}\label{example10fact} $B$ is a blender for $f$, and its covered domain contains $O:= (-2/3,2/3)\times (-1,1)$. 
% \end{fact}
% \begin{proof}
%Let us notice that $B$ satisfies the following \emph{covering property}. With $O_{+1}:=[0,2/3)\times (-1,1)$ and $O_{-1}:= (-2/3,0)\times (-1,1)$, it holds:
%\[O= O_{+1}\cup O_{-1},\quad 
%cl(f_{-1}^{-1}(O_{-1})\cup f_{+1}^{-1}(O_{+1}))\subset O\; .\]
%Hence, for any perturbation of the dynamics, for any $z\in O$, there exists a preorbit $(z_i)_{i\le 0}$ so that $z_i$ belongs to $O_{\sb_i}$ for $\sb_i\in \{\pm 1\}$. With $\underline \sb= (\sb_{_i})_{i\le -1}$ we note that by shadowing
%$z\in W^u_{loc}(\underline \sb;f')$. 
%\end{proof}
%
%In higher dimension $n\ge 2$,  a hyperbolic compact set $K$ with one dimensional  unstable direction is a \emph{blender} for a $C^1$-dynamics $F$, if there exists a continuous family of local unstable manifolds $(W^u_{loc}(\arr z; F))_{\arr z\in \arr K}$ whose union intersects robustly a $C^1$-neighborhood $N$ of an $n-2$-dimensional sub-manifold $S$:
%\[\bigcup_{\arr z\in K} W^u_{loc} (\arr z; F') \cap S'\not=\emptyset \quad , \quad  \forall S'\in N\quad \forall F'\; C^1\text{ close to }F\;.\]
%
%\begin{exam} Let $F: (t,x,y)\in \R^{n-2}\times \R\times\R\mapsto (0,f(x,y))$.  We notice that $\{0\}\times B$ is a hyperbolic set of $F$ with one-dimensional unstable direction. 
%\end{exam}
%\begin{fact}\label{blendern} The hyperbolic set $\{0\}\times B$  of $F$ is a blender.
%\end{fact}
%\begin{proof}
%We notice that $\R^{n-2}\times \{0\}$ is the strong stable direction.
%  Hence $DF^{-1}$ leaves invariant the constant cone field $\chi= \{(u,v)\in \R^{n-2}\times \R^2: \|v\|\le \|u\|\}$.
%  
%Let $V$ be the set of $C^1$-submanifolds $S$ of the form $S=Graph\, \phi$ where $\phi\in C^1((-1,1)^{n-2},\R^2)$ so that  $\phi(0)\in O$ and $TS\subset S\times \chi$. 
%
%We notice that any small $C^1$-perturbation $F'$ of $F$ satisfies that $DF'^{-1}$ leaves invariant $\chi$.  Hence for every $S\in V$, if $S\cap 
%\{0\}\times \R^2 \subset \{0\}\times O_{-1}$ (resp. $\{0\}\times O_{+1}$) then a connected component $S'$ of   $F'^{-1}(S)\cap (-1,1)^{n-2} \times O$ is in $V$.
% 
% Hence, by induction, for every $F'$ $C^1$-close to $F$, for every $S\in V$, we can define a sequence of preimages $(S^i)_i$ associated to symbols $\underline \sb= (\sb_{i})_{i\le 0}$. 
%
%We notice that $( F'^i(S^i))_i$ is a nested sequence of subsets in $S$, whose intersection $\cap_{i\ge 1} F'^i(S^i)$ consists of a single point $z$. By shadowing, it comes that $z$ belongs to $W^u_{loc}(\underline \sb; F')$ and so $S$ intersects $\bigcup_{z\in K} W^u_{loc} (z; F')$. 
%\end{proof}
%
%\subsubsection{$\lambda$-blender}
%In this subsection let us introduce a blender with a special property. 
%
%Let $M$ be a manifold, $f$ a self mapping of $M$ which leaves invariant a hyperbolic compact set $K$ with one dimensional unstable direction. Let $N_G$ be an open neighborhood of $E^s|K$ in the Grassmannian bundle $GM$ of $TM$, which projects onto a neighborhood $N$ of $K$ and satisfies:
%\[\forall z\in  N\cap f^{-1}(N)\quad  D_zf^{-1}N_G(f(z))\Subset N_G(z)\; , \text{with }N_G(z)= N_G\cap GM_z\; .\]
%
%
%\begin{defi}[$\lambda$-Blender when $\dim M=2$]
%The hyperbolic set $K$  is a \emph{$\lambda$-blender} if the following condition is satisfied. There exist a continuous family of local unstable manifolds $(W^u_{loc} (\arr z; f))_{\arr z\in \arr K}$
%and  a non-empty open set  $O$ of $M\times \R$ so that for every $f'$ $C^{1}$-close to  $f$,  for every $(Q,\lambda )\in O$, there exists $\arr  z\in \arr K$, so that:
%\begin{itemize}
%\item  $Q\in W^u_{loc}(\arr z; f')$, and with $(Q_{-n})_n$  the preorbit of $Q$ associated  to $\arr z$, 
%\item  for every line $L$ in $N_G(Q)$, with $L_n:= (D_{Q_{-n}}f^{n})^{-1}(L)$,  the sequence $(\frac1n 
%\log \|D_{Q_{-n}}f^{n}|L_n\|)_n$ converges to $\lambda$. 
%\end{itemize}
%The open set $O$ is called a \emph{covered domain} for the $\lambda$-blender $K$. 
%\end{defi}
%
% \begin{exam}\label{example11}
%Let $\sB= \{-1,+1\}^2$ and let $(I_{\sb })_{\sb \in \sB}$ be four disjoint, non trivial segments in $(-1,1)$. Let $\sigma$ be a map which sends affinely each $I_{\sb}$ onto $[-1,1]$. For $\epsilon>0$ small put:
%  \[f:(x,y)\in [-2,2]\times \sqcup_{\sB} I_\sb\mapsto \begin{array}{cc}
%  \left((\frac23)^{1+\epsilon\cdot \delta'}(x-\delta)+\delta, \sigma(y)\right)& \text{if } y\in I_\sb \; \text{and } \sb =({\delta, \delta'})\end{array}\; .\]
%
% Let $K:=\cap_{n\ge 0} \sigma^{-n}(\cup_{\sb \in\sB}  I_{\sb })$ be the maximal invariant of $\sigma$, and let $B:= [-1,1]\times K$. We notice that $B$ is a hyperbolic set for $f$ with vertical stable direction and horizontal unstable direction. Let $N_G:= \{(u,v)\in \R^2: \|u\|\ge \|v\|\}$.
% 
% 
%   Given a pre-sequence $\underline \sb =(\sb_i)_{i\le -1}$ and $f'$ $C^1$-close to $f$, we define the local unstable manifold:
%  \[W^u_{loc}(\underline \sb;f'):= \bigcap_{i\ge 1} f'^i( [-2,2]\times I_{\sb_{-i}})\; .\]
% \end{exam} 
%  \begin{fact}\label{example11fact} The uniformly hyperbolic $B$ is a $\lambda$-blender for $f$, and its covered domain contains:
%   $$O:= ((-2/3,2/3)\times (-1,1))\times (\log2/3 -2\epsilon; \log2/3 +2\epsilon)\; .$$ 
% \end{fact}
% \begin{proof}
%Given $\sb= (\delta,\delta')\in \{-1,1\}^2=\sB$, let \[O_\sb = 
%\{((x,y),\lambda)\in O:\; x\cdot \delta\ge 0, (\lambda -\log2/3) \cdot \delta'\ge 0\}\; .\]
%Given a $C^1$-perturbation $f'$ of the dynamics $f$, for every $(Q,\lambda)\in  O$, given any unit vector $u\in N_G(Q)$ , we define inductively a $Df'$-preorbit $(Q_n,u_n)_n$ associated to a pre-sequence of symbols  $(\sb_n)_n$ as follows. Put $Q_0=Q$ and  $u_0= u$. For $n\le 0$, we define $\sb_{n-1}=(\delta_{n-1},\delta_{n-1}')$ such that  $\delta_{n-1}$ is the sign of first coordinate of $Q_{n}$,  and $\delta_{n-1}'$ is the sign of $\log\|u_{n}\| - n \lambda$.  
%By decreasing induction one easily verifies that $(Q_{n},\log \|u_{n}\| - n\lambda+\log2/3)\in O_{\sb_{n}}$ for every $n$.  Hence $\frac1n \log \|u_{n}\|\to \lambda$ as asked.
%
%As for the proof of Fact \ref{example11fact}, this implies  that 
%$z\in W^u_{loc} (\u \sb; f')$, with $\u \sb=(\sb_{-n})_{n\le -1}$.   
%  \end{proof}
% \begin{defi}[$\lambda$-blender when $\dim M\ge 3$]
% 
% The hyperbolic set $K$  is a \emph{$\lambda$-parablender} if the following condition is satisfied.
% 
%  There exist a continuous family of local unstable manifolds $(W^u_{loc} (\arr z; f))_{\arr z\in \arr K}$
%and  a neighborhood $O$ of a pair $(S,\lambda_0)$ of a number $\lambda_0\in \R$ with an $n-2$-dimensional $C^1$-submanifold $S$  so that, for every $(S',\lambda)\in O$, there exists  $\arr  z\in \arr K$ satisfying:
%\begin{itemize}
%\item  $W^u_{loc}(\arr z; f')$ intersects $S'$ at a point $Q$, and with $(Q_{-n})_n$  the preorbit of $Q$ associated  to $\arr z$, 
%\item  for any $(n-1)$-plane $E$ in $N_G(Q)$,  the sequence $(\frac1n 
%\log \|D_{Q_{-n}}f^{n}|E_n\|)_n$ converges to $\lambda$, with $E_n:= (D_{Q_{-n}}f^{n})^{-1}(E)$. 
%\end{itemize}
%\end{defi}
%
%\begin{exam} Let $F:= (t,x,y)\in \R^{n-2}\times \R\times\R\mapsto (0,f(x,y))$ with $f$ as in example \ref{example11}.  We notice that $\{0\}\times B$ is a hyperbolic set of $F$ with one-dimensional unstable direction.  
%\end{exam}
%In the proof of Fact \ref{blendern}, we define a $C^1$-open set $V$ of $(n-1)$-submanifolds. 
%\begin{fact}\label{fact20} The hyperbolic set $\{0\}\times B$ of $F$ is a $\lambda$-blender with $O=V\times (-\log2/3-2\epsilon,+\log2/3+2\epsilon)$ in its covered domain. \end{fact}
%The proof is done by merging the one of Facts \ref{blendern} and \ref{example11fact}, and so it is left as an exercise to the reader. 
%
%\subsection{Hyperbolic theory for  families of dynamics}
%Let us fix $k\ge 0$, $1\le r<\infty$, and a $C^r$-family $\hat f= (f_a)_a$ of $C^r$-maps of $M$.
%
%
%\subsubsection{Hyperbolic continuation for parameter family}
%It is well known that if $f_0$ has a hyperbolic fixed point $P_0$, then it persists as a hyperbolic fixed point $P_a$ of $f_a$ for every $a$ small. Moreover, the map $a\mapsto P_a$ is of class $C^r$.
%
% More generally, if  $K$ is a hyperbolic set for $f_0$ (with possibly $f_0|K$ not injective), it persists in a sense involving its inverse limit $\arr K$. Let $\arr f_0$ be the shift map on $\arr K$. 
% \begin{thm}[Th. 14 \cite{Be16}]
%% Prop 1.6 BE15, BE152}]%Quandt, berRov}]
%For every $a$ in a neighborhood $V$ of $0$, there exists a map $h_a \in C^0(\arr K; M)$ so that:
%\begin{itemize}
%\item $h_0$ is the zero-coordinate projection $ (z_i)_i\mapsto z_0$.
%\item $f_a \circ h_a= h_a\circ \arr {f_0}$ for every $a\in V$. 
%\item For every $\underline  z\in \arr K$, the map $a\in V\mapsto h_a(\underline z)$ is of class $C^r$.
%\end{itemize}
%\end{thm}
%
%The point $h_a(\underline z)$ is called the \emph{hyperbolic continuation} of $\underline z$ for $f_a$. We denote $\underline z_a\in M$ the zero-coordinate of $h_a(\underline z)$. 
% The family of sets $(K_a)_a$, with $K_a:= \{\underline z_a : \underline z \in \arr K\}$, is called the hyperbolic continuation of $K$.
%
%The local stable and unstable manifolds $W^s_{loc} (z; f_a) $ and $W^u_{loc} (\underline z; f_a) $  are canonically chosen so that they depend continuously on $a$, $z$ and $\underline z$. 
% They are called the \emph{hyperbolic continuations} of $W^s_{loc} (z; f) $ and $W^u_{loc} (\underline z; f) $ for $f_a$. 
%Let us recall:
%\begin{prop}[Prop 15 \cite{Be16}]
%%1.6 \cite{BE15,BE152}]
%\label{Wupara}
%For every $z\in  K$, the family $(W^s_{loc} ( z; f_a))_{a\in V}$ is of class $C^{r}$.
%For every $\underline z\in \arr K$, the family $(W^u_{loc} (\underline z; f_a))_{a\in V}$ is of class $C^{r}$.
%Both vary continuously with $z\in K$ and $\underline z\in \arr K$.
% \end{prop}
%
%\subsubsection{Parablender}
%The bifurcation theory studies the hyperbolic continuation of hyperbolic sets and  their local stable and unstable manifolds, to find dynamical properties.  Hence we shall study the action of $C^r$-families $\hat f=(f_a)_a$ on \emph{$C^r$-jets}. 
%
%\label{notationJdM}
%Given a $C^r$-family of points $\hat z = (z_a)_{a\in \R^k }$, its \emph{$C^r$-jet at $a_0\in \R^k$} is  $J^r_{a_0} \hat z= \sum_{j=0}^r \frac{\partial^j_a z_{a_0}}{j!}  a^{\otimes j}$.  Let $J^r_{a_0} M$ be the space of $C^r$-jets of $k$-parameters, $C^r$-families of points in $M$. 
%
%We notice that any $C^{r}$-family $\hat f = (f_a)_a$ of $C^r$-maps $f_a$ of $M$ acts canonically on $J^r_{a_0} M$ as the map:
%\[J^r_{a_0} \hat f \colon  J^r_{a_0}( z_a)_a \in J^r_{a_0}M\mapsto 
%J^r_{a_0} (f_a(z_a))_a\in J^r_{a_0} M\; .\]
%   
%The first example of parablender was given in \cite{BE15}; in \cite{BCP16} a new example of parablender was given. 
%Let us give for the first time a definition for a dynamics on a manifold $M$ of any dimension $n$.
%% in any dimension. 
%% and the concept reaches the following definition:   
%\begin{defi}[$C^r$-Parablender when $\dim M=2$]
%A family $(K_a)_a$ of blenders $K_a$
%endowed with a continuous family of local (one dimensional) unstable manifolds $(W^u_{loc}(\underline z; f_a))_{\arr z}$ 
% is a \emph{$C^r$-parablender} at $a={a_0}$ if the following condition is satisfied.
%There exists a non-empty open set  $O$ of $C^r$-families of 2-codimensional $C^r$-submanifolds 
%%$(S_a)_a$ 
%%There exists a non-empty open set  $O$ of $C^r(\R^k, M)$
% so that for every $(f'_a)_a$ $C^{r}$-close to  $(f_a)_a$,  for every 
%$(S_a)_{a\in \R^k} 
%% $\hat \gamma$
% \in O$, there exist $\underline z\in \arr K$ and a $C^r$-curve of points $\hat Q= (Q_a)_a$ in $(W^u_{loc} (\arr z; f'_a))_{a}$ and a $C^r$-curve of points $\hat P=(P_a)_a$ in $(S_a)_a$  satisfying:
%\[J^r_{a_0} \hat Q = J^r_{a_0} \hat P\; .\]
%%\gamma\] 
%%d(\gamma(a), Q_a)= o(\|a-a_0\|^r)\; .\]
%
%The open set $O$ is called a \emph{covered domain} for the $C^r$-parablender $(K_a)_a$. 
%\end{defi}
%\begin{rema} When $n=2$, the set $O$ is an open subset of family of points $(S_a)_a\in C^r(\R^k, M)$. 
%\end{rema}
%\begin{rema}
%We notice that if $J^r_{a_0}(K_a)_a:=\{J^r_a (\underline z_a)_a: \underline z\in\arr  K\}$ is a blender for $J^r_{a_0}(f_a)_a$ then   $(K_a)_a$ is a $C^r$-parablender at $a_0$ for $(f_a)_a$. We do not know if it is a necessary condition. 
%\end{rema}
%\begin{exam}[$C^r$-Parablender in dimension 2]\label{expparablender} 
%Let $\Delta_r:=  \{-1,1\}^{E_r}$ with $E_r:= \{i=(i_1,\dots, i_k)\in \{0,\dots, r\}^k: i_1+\cdots +i_k\le r\}$.  For $\delta\in \Delta_r$ we put:
%\[P_\delta (a) = \sum_{i\in E_r } \delta(i)\cdot a_{1}^{i_1}\cdots a_{k}^{i_k}\; .\]
%
%Consider ${Card\, \Delta_r}$ disjoint segments $D_r:= \sqcup _{
%\sa \in \Delta_r} I_\delta $ of $(-1,1)$.  
%Let $\sigma\colon  \sqcup _{\delta\in \Delta_r} I_\delta \to [-1,1]$ 
%be a locally affine, orientation preserving map which sends each $I_\delta$ onto $[-1,1]$. Let $( f_a)_a$ be the $k$-parameters family defined by:
%\[ f_a(x,y) \colon(x,y)\in [-3,3]\times D_r\longmapsto \begin{array}{cc}
%( \frac 23( x- P_\delta(a)) + P_\delta(a), \sigma(y)) & \text{if } y\in I_\delta \; .
%\end{array} \]
%We notice that the maximal invariant set of $ f_0$ is a blender $K$.
%
%Let us define the following subsets of the space $C^r_0(\R^k ,M)$
%of germs at $0$ of $C^r$-functions:
%\[\hat O_r:= \{\hat z\in C^r_0(\R^k ,M): J^r_0 \hat z=\sum_{i\in E_r} (x_i,y_i) \cdot a_{1}^{i_1}\cdots a_{k}^{i_k}\text{ and } |x_i|< 1,|y_i|< 2/3\}\; .\]
%\[\hat O_\delta:= \{\hat z\in C^r_0(\R^k ,M): J^r_0 \hat z=\sum_{i\in E_r} (x_i,y_i) \cdot a_{1}^{i_1}\cdots a_{k}^{i_k}: |x_i|< 1, 0\le \delta(i)\cdot y_i<  2/3\}\; .\]
%We observe that $\hat O_r=\cup_{\delta\in \Delta_r} \hat O_\delta$. Also for every $\delta \in \Delta_r$, the inverse of $J^r_0 ( f_a)_a$ maps  
%$cl(\hat O_\delta)$ into the interior of $\hat O_r$.
%Hence by proceeding as in \cite[Example 19]{Be16}, we prove that the hyperbolic continuation $( K_a)_a$ of $K$ is a $C^r$-parablender at $a_0=0$ with $\hat O$ included in its covered domain. 
%  
%
%  
%\end{exam}
%\begin{exam}[$C^r$-Parablender in dimension $n\ge 2$]\label{expparablendern} 
%Let $(f_a)_a$ be given by previous example \ref{expparablender} with parablender $(K_a)_a$ and covered domain $\hat O_r=\cup_{\delta\in \Delta_r} \hat O_\delta$. Let $\hat F=(F_a)_a$ be defined by:
%\[ F_a(x,y) \colon(t,x,y)\in (-1,1)^{n-2}\times [-3,3]\times D_r\longmapsto 
%( 0, f_a(x,y))\; .\]
%We notice that $(\{0\}\times K_a)_a$ is a family of hyperbolic sets for $(F_a)_a$. Let us show that it is a $C^r$-parablender. 
%
%%As for Fact  \ref{blendern}, 
%Let  $\hat V_r$ be the space of germs at $a=0$ of $C^r$-family $\hat \phi =(\phi_a)_a$ of $C^r$-maps $\phi_a\in C^r((-1,1)^{n-2},\R^2)$ so that:
%\begin{itemize}
%\item $( \phi_a(0))_a \in \hat O_r$,
% \item the map $(a,t)\mapsto \phi_a(t)-\phi_a(0)$ has $C^{r}$-norm smaller than $1$. 
% \end{itemize}
%We identify  $\hat V_r$ with an open set $O$ of germs at $a=0$ of $C^r$- family of  $C^r$-$(n-2)$-submanifolds of $\R^n$ by associating to $\hat \phi$ its family of graphs $(Graph\, \phi_a)_a$. Let us show that for every $\hat F'$ $C^r$-close to $\hat F$, for every $(S_a)_a\in O$, there exists $\u \delta \in \Delta^{\Z^-}$, $(P_a)_a\in (S_a)_a$ and $(Q_a)_a \in (W^u_{loc} (\u \delta; f_a))_a$ so that $J^r_0 (P_a)_a=
%J^r_0 (Q_a)_a$.
%
%
%For every $\delta\in \Delta$, we define $\hat V_\delta$ as the subset of $\hat \phi\in \hat V_r$ so that   $J^r_{0} (\phi_a(0))_a$ belongs to $\hat O_\delta$.  Note that $\hat V_r= \cup_{\delta} \hat V_\delta$. 
%Given any $C^r$-perturbation $\hat F'$ of the family $\hat F$ and every $\hat \phi\in V_\delta$ we define:
%\[ \hat \phi_{\delta} = ( \phi_{a\; \delta})_{a} \; ,\quad \text{with} \; Graph \,  \phi_{a\; \delta} = (F'_a| (-1,1)^{n-2} \times [-3,3]\times I_\delta)^{-1} Graph \,  \phi_{a}
% \quad  \forall a\text{ small}.\] 
%\begin{fact}
%For every $\hat F'$ $C^r$-close to $\hat F$, for every $\hat \phi\in \hat V_\delta$, the family $\hat \phi_\delta$ is well defined and in $\hat V_r$.
%%
%%
%%The map $\phi_{a\; \delta} $ is well defined for every $\|a\|\le \alpha$ and the $C^r$-norm $(a,t)\mapsto \phi_a(t)-\phi_a(0)$ is smaller than $\epsilon$. 
%\end{fact}
%\begin{proof}
%If $\hat F'= \hat F$, for every $\hat \phi$ in $\hat V_\delta$, the family $\hat \phi_\delta$ is well defined by transversality of the map $(a, t,x ,y)\mapsto (a,
%F_a(t,x,y))$ with the submanifold $\cup_{a\in (-1,1)^{n-2}}\{a\} \times Graph\, \phi_a$. Furthermore, the family $\hat \phi_\delta$ is equal to $(t\mapsto g_a^\delta\circ \phi_a(0))_{a\in (-1,1)^{n-2}}$,  
%where  $g_a^\delta$ is the inverse of $f_a|[-3,3]\times I_\delta$.
%As $J^d_0 (\phi_a(0))_a$ is in $\hat O_\delta$, it comes that 
%$J^d_0(g_a^\delta\circ \phi_a(0))_a$ is in $J^d_0 (g_a^\delta)_a(\hat O_\delta)\Subset \hat O_r$. Note that $\hat \phi_\delta$ is in a subset of $\hat V_r$ at positive distance to the complement of $\hat V_r$. 
%
%Hence by transversality, for every $\hat F'$ in a $C^r$-small neighborhood of $\hat F$, for every $\hat \phi\in \hat V_\delta$, the family $\hat \phi_\delta$ is in $\hat V_r$. 
%\end{proof}
%From the latter fact, for every $\hat F'$ in a $C^r$-small neighborhood of $\hat F$, for every $\hat \phi\in \hat V_r$,
%we can define a sequence $\u \delta\in \Delta^{\Z^-}$ and preimages $(\hat \phi^{n})_{n\le -1}\in V_r^{\Z^-}$ with $\hat \phi^{n}= \hat \phi^{n+1}_{\delta_{n}}$ and $\hat \phi^{0}= \hat \phi$. 
%
%
%Let $S^n_a= Graph\, \phi^n_a$   and observe that 
%$S^n_a$ is mapped into $S^{n+1}_a= Graph\, \phi^{n+1}_a$ for every $n\le -1$. 
%  
%Hence the point $P_a^n:= (0,  \phi^n_a(0))$ is well defined for $a$ small,   and by contraction of $F'^n_a|S^n_a$,
%$F'^n_a(P^{-n}_a)$ belongs to $S^0_a$ and the jets $J^r_0 (F_a^{n-k}(P_a^{-n}))_a$ are bounded for every $n\ge k\ge   0$.
%
%Hence $(J^r_0F'^n_a(P^{-n}_a))_n$ converges to the $C^r$-jet at $a=0$ of a $C^r$-curve of points $(P_{a})_a\in (S_a)_a$, and 
%displays  a preorbit $((P_{k\; a})_a)_{k\le -1} $ associated  to $\u \delta$ with bounded $C^r$-jet at $0$ for every $k\le -1$.  
%
%By the shadowing property of the hyperbolic set $J^r_0 (\{0\}\times K_a)_a$ for $J^r_0 \hat F'$, the point $J^r_0 (P_a)_a$ belongs to the unstable manifold of $J^r_0 (\{0\}\times K_a)_a$ associated to $\u \delta$. 
%In other words, there exists a $C^r$-curve $(Q_a)_a\in (W^u_{loc}(\u \delta; F_a'))_a$ so that $J^r_0(Q_a)_a= J^r_0(P_a)_a$.
%
%% For the reason as for Fact \ref{blendern}, the point $P_0$ belongs to $W^u_{loc} (\u \delta; F'_0)$. 
%%
%%Moroever if $Q_a$ is the intersection point of $W^u_{loc} (\u \delta; F'_a)$ with $S^0_a$ and for every $n\le -1$, $Q^n_a$ the intersection point of $W^u_{loc} (\sigma^{-n}(\u \delta); F'_a)$ and $S^n_a$, we notice that $Q^n_a$ is mapped to $Q^0_a$ by $F_a'^n$ and  $J^r_0( Q^n_a)_a$ remain at bounded and so at bounded distance to  $J^r_0(P^n_a)_a$ for every $n\le 0$. By contraction of $F_a'^n| S^n_a$ and its action on the $r$-jet space, it comes that  $J^r_0 (P_a^0)_a$ is equal to $J^r_0 (Q_a^0)_a$.
% \end{exam}
%\subsubsection{$\lambda$-parablender}
%Let us generalize the notion of $\lambda$-blender to its parametric version the $\lambda$-parablender for a $C^r$-family of dynamics $(f_a)_a$  of a manifold $M$ of dimension $n$. 
%
%For this end, given $(\hat z,\hat u)\in C^{r-1}(\R^k,TM)$ so that $\hat z\in  C^r(\R^k,M)$,  we consider:
%\[ \check J^r_{a_0} (\hat z,\hat u) :=  (J^r_{a_0} \hat z, J^{r-1}_{a_0} \hat u)\quad \text{and}\quad \check J^r_{a_0} TM := \{ \check J^r_{a_0} (\hat z,\hat u) : \; 
%(\hat z,\hat u)\in C^r(\R^k,TM)\}\]  
%
%We note that $D\hat f$ acts canonically on $\check J^r_{a_0} TM$ as:
%\[\check J^r_{a_0} D\hat f \circ \check J^r_{a_0}(\hat z,\hat u) = 
%\check J^r_{a_0} (\hat f\circ \hat z, D_{\hat z}\hat f(\hat u))\]
%
%Let $(K_a)_a$ be a family of $C^r$-parablenders for a $C^r$-family of dynamics $(f_a)_a$. Let $N_G$ be a neighborhood of the stable direction of $K_0$ in the Grassmanian bundle $GM$ of $TM$, which projects onto a neighborhood $N$ of $K_0$ and satisfies:
%\[\forall z\in  N\cap f_0^{-1}(N)\quad  D_zf_0^{-1}N_G(f_0(z))\Subset N_G(z)\; .\]
% 
%Let $\check J^r_0 N_G\subset \check J^r_0TM$  be the subset of jets $(J^r_0(z_a)_a, J^{r-1}_0(u_a)_a)\in \check J^r_0TM\setminus \{0\}$ so that $u_a\in N_G(z_a)$ for every $a$.
%
%Let $\hat {\mathcal V}$ be the space of $C^r$-families $\hat S = (S_a)_a$ of 2-codimentional submanifolds $S_a$. We notice that $\hat {\mathcal V}$ is equal to $C^r(\R^k, M)$ if $n=2$. 
%
%
%\begin{defi}[$C^r$-$\lambda$-Parablender]
%A family $(K_a)_a$ of blenders for $(f_a)_a$  is a \emph{$C^r$-$\lambda$-parablender} at $a={a_0}$ if the following condition is satisfied.
%
%There exists
%a continuous family of  local unstable manifolds $(W^u_{loc}(\arr z; f_a))_{\arr z\, a}$,  a non-empty open set  $O$ of $\hat{\mathcal V}
%\times C^{r-1}(\R^k, (-\infty,0))$ so that for every $(f'_a)_a$ $C^{r}$-close to  $(f_a)_a$,  for every $(\hat S, \hat \lambda)\in O$, there exist $\arr  z\in \arr K$ and a $C^r$-curve of points $\hat Q= (Q_a)_a\in (W^u_{loc}(\arr z; f'_a))_a$
%and $\hat P = (P_a)_a\in (S_a)_a$
% satisfying:
%\[J^r_{a_0} \hat P = J^r_{a_0} \hat Q\; .\]
%%\[\lim_{n\to \infty} \check J^r \sqrt[n]{\| D_{z_a}g_a^{n} (n_a)\| }= 
%%J^{r-1}(\lambda_a)_a \]
%Furthermore, for every $C^{r-1}$-family of 
%$(n-1)$-planes $(E_a)_a$ 
%%curve of unit vectors $(u_a)_a$
% in $(N_G(Q_a))_a$ and $(Q_{-n\; a},E_{-n\; a})_a$ the preimage by $(Df^n_a)_a$ of $(Q_a,E_a)_a$ associated to the preorbit $\arr z$,  it holds:
%\[\lim_{n\to \infty} \frac1n J^{r-1}_{a_0}
%(\log \| D_{Q_{-n\; a}} f'^{n}_a |E_{-n\; a}\|)_a = J^{r-1}_{a_0}
%\hat \lambda\; .\]
%%+\epsilon_n(a)\; , \quad \text{with } J^{r-1}_0\epsilon_n\text{ small when $n$ is large.}\]
%
%
%The open set $O$ is called a \emph{covered domain} for the $C^r$-$\lambda$-parablender $(K_a)_a$. 
%\end{defi}
%\begin{rema}
%In particular $(K_a)_a$ is a $C^r$-parablender and $K_0$ is a $\lambda$-blender. 
%\end{rema}
%
%%We shall state a sufficient condition for a family $(K_a)_a$ of blenders for $(f_a)_a$ to be a \emph{$C^r$-$\lambda$-parablender} at $a=0$.   The Abelian group $J^{r-1}_0\R $ acts canonically on $\check J^{r}_0N_G$ as:
%%\[\begin{array}{ccc}
%%J^{r-1}_0\R\times \check J^{r}_0N_G&\to& \check J^{r}_0N_G\\
%%
%%[J^r_0(\lambda_a)_a , \check J^{r}_0(z_a,u_a)_a]
%%&\mapsto& J^r_0(\lambda_a)_a \odot \check J^{r}_0(z_a,u_a)_a:=  \check J^{r}_0(z_a,\exp(\lambda_a) \cdot u_a)_a\end{array}\]
%%
%%Note that if $\check J^{r}_0(\hat z,\hat u)\in \check J^{r}_0N_GN_G$ is so that $z_0\in N$ and $\hat g=(g_a)_a$ is an inverse branch of $\hat f= (f_a)_a$ so that $g_0(z_0)\in N$, then the pullback $\check J^{r}_ 0D\hat g(\hat z,\hat u)$ is in $\check J^{r}_0N_G$. 
%%
%%\begin{prop} 
%%If the family of parablender $(K_a)_a$ satisfies the following covering property then it is a $\lambda$-$C^r$-parablender. 
%%\begin{itemize}
%%\item[$(\mathcal C)$]   There exist $J^{r-1}_0\hat \lambda'  \in J^{r-1}_0\R$, a non-empty compact set $ O$ of $\check J^{r} N_G$  and  a finite covering $( O_\sb)_\sb$  of $O$ so that each $ cl(O_\sb)$ is sent by an inverse branch into $int(J^{r-1}_0\hat \lambda'\odot  O)$. 
%%\end{itemize}
%%\end{prop}
%%\begin{proof} We notice that this property is stable for perturbation of the dynamics and of $J^{r-1}_0\hat \lambda$. 
%%
%%Hence for every $(\hat \gamma, \hat \sigma)\in O$, it suffices to prove the existence of $\arr z\in \arr K$ and $\hat Q=(Q_a)_a$ for $\hat f'= \hat f$ and $\hat \lambda = \hat \lambda'$. 
%%%so that $J^r_0\hat Q$ is equal to $J^r_0 \hat \gamma$ and:
%%%\[\lim_{n\to \infty} \frac1n J^{r-1}_{a_0} (
%%%\log \| Df'^{n}_a |E^s_{\arr f^{-n}(\arr  z)}\|)_a = J^{r-1}_{a_0}
%%%\hat \sigma\; .\]
%% 
%%Let $\hat \gamma=(\gamma_a)_a$ and $\hat\lambda=(\lambda_a)_a$. Inductively, we define  symbols  $\underline \sb := (\sb_i)_{i\le -1}$ of inverse branches $((g_{\sb_i\; a})_a)_i$ of $(f_a)_a$ and $(\hat \gamma_i,\hat u_i)_{i\le 0}= ((\gamma_{i\; a},u_{i\; a})_a)_{i\le 0}$ so that:
%% \[(\hat \gamma,\hat u)= (\hat \gamma_0,\hat u_0) \]
%% \[ (\hat \gamma_{i},\hat u_i)\in  O_{\sb_{i-1}}\quad \text{and}\quad  (\hat \gamma_{i-1},\hat u_{i-1})=  J^{r-1}_0\hat \lambda\odot   \check J^r g_{\sb_i} (\hat \gamma_{i},\hat u_i)\; .\]
%%This sequence of inverse branches $((g_{\sb_i\; a})_a)_i$ defines a family of local unstable manifolds $(W^u_{loc} (\underline \sb; f_a))_a$. By the same argument as for the $C^r$-parablender, there exists $\hat Q= (Q_a)_a$ in $(W^u_{loc} (\underline \sb; f_a))_a$ so that $J^r_0\hat Q= J^r_0\hat \gamma$. 
%% 
%%Note also that $J^{r-1} \hat u_{-n} =  J^{r-1} (D(g_{a\; \sb_{-n}}\cdots g_{a\; \sb_{-1}})(\exp(n\cdot \lambda(a))\cdot  u_a))_a$. Therefore, $J^{r-1}_0 (u_{-n\; a})_a$ and $J^{r-1}_0 (u_a)_a$ are bounded and satisfy: 
%%$$J^{r-1} (D_{\gamma_{-n}(a)} f^n_a (u_{-n\; a}))_a = J^{r-1}_0 (\exp(n\cdot \lambda(a)) \cdot  u_a)_a\; .$$
%%$$\Rightarrow   J^{r-1} (\log \|D_{\gamma_{-n}(a)} f^n_a (u_{-n\; a})\|)_a = J^{r-1}_0 (n\cdot \lambda(a) +\log\| u_a\|)_a\; .$$
%%As both $(\|u_{-n\; 0}\|)_a$ and $(\|u_0\|)_a$  are uniformly far from $0$, $(J^r_0 \hat u_{-n})_n$ bounded, it holds:
%%$$\Rightarrow \frac1n J^{r-1} (\log \|D_{\gamma_{-n}(a)} f^n_a |\R\cdot u_{-n\; a}\|) = J^{r-1}_0 \hat \lambda+O\left(\frac1n\right)\; . \qedhere$$  
%%\end{proof}
%
%
%
%\begin{exam}[$C^r$-$\lambda$-Parablender in dimension 2]\label{lambdaparablender} 
%Let $E_r$, $\Delta_r$ and $P_\delta$  be defined as in Example \ref{expparablender} and put $\sB:= \Delta_r\times \Delta_{r-1}$. Let $\hat O_r$, $\hat O_\delta$ be the subset of $C^r(\R^k ,M)$ defined therein. 
%%Let $E^*:= E\cap \{1,\dots, r\}^k$, where $E$ was defined in Example \ref{expparablender}. Let $\Delta^*:=  \{-1,1\}^{E^*}$. For $\delta^*\in \Delta^*$ we put:
%%\[P_\delta (a) = \sum_{i\in E^* } \delta(i)\cdot a_{1}^{i_1}\cdots a_{k}^{i_k}\]
%
%%Let $\sL:= \Delta\times \Delta^*$. 
%
% Consider $Card\, (\sB)$ disjoint segments $D:= \sqcup _{\sa \in \sB} I_\sa $ of $(-1,1)$.  
%Let $\sigma\colon  \sqcup _{\sa\in \sB} I_\sa \to [-1,1]$ 
%be a locally affine map which sends each $I_\sa$ onto $[-1,1]$. For $\epsilon>0$ small, let $(\tilde  f_a)_a$ be the $k$-parameters family defined by:
%\[\tilde f_a \colon(x,y)\in D\times [-3,3]\longmapsto \begin{array}{cc}
%( \frac23\cdot \exp(\epsilon \cdot P_{\delta'}(a))\cdot x +\frac{P_{\delta}(a)}3,\sigma(y))& \text{if } y\in I_\sa,\; \sa =(\delta, \delta') \; .
%\end{array} \]
%We notice that the maximal invariant set $\tilde K_a$ of $\tilde f_a$ in $[-3,3]\times D$ is hyperbolic and for $ \u \sb =(\sb_i)_{i\le -1}\in \sB^{\Z^-}$, with local unstable manifold $W^u_{loc}(\u \sb,\tilde f_a):= \bigcap_{i\ge 1} \tilde f^i_a( [-2,2]\times I_{\sb_{-i}})$.
% 
%Let $N_G$ be the cone field constantly equal $\{(u,v): \|u\|\le \|v\|\}$. We notice that it is backward invariant. Let:
%\[\tilde O = \hat O_r\times \{
%\hat \lambda\in C^{r-1}_0(\R^k, \R): J^{r-1}_0 \hat \lambda =  
%\log \frac23+ \sum_{i\in E_{r-1}} \lambda_i a^i:\;  \lambda_i\in [-2\epsilon,2\epsilon ]\}\; .\]\end{exam}
% 
%  \begin{fact}\label{example11factparaL} $(\tilde K_a)_a$ is a $\lambda$- $C^r$-parablender for $(\tilde f_a)_a$, and its covered domain contains $\tilde O$.
% \end{fact}
%\begin{proof} We consider the covering  $(\tilde O_\sb)_{\sb \in \sB}$ of $\tilde O$ with for every $\sb = (\delta, \delta')\in \sB$:
%  $$\tilde O_\sb :=\hat O_\delta \times 
%  \{\hat \lambda\in C^{r-1}_0(\R^k, \R): J^{r-1}_0 \hat \lambda = \log \frac23+ \sum_{i\in E_{r-1}} \lambda_i a^i:\;  \delta'_i \cdot \lambda_i\in [0,2\epsilon ]\}\; , $$
%and proceed by merging the proofs of Fact \ref{example11fact} and Example \ref{expparablender}.
%   \end{proof}
%
%\begin{exam}[$C^r$-$\lambda$-Parablender in dimension $n\ge 2$]\label{expparablendernlambda} 
%Let $(\tilde f_a)_a$ be given by previous example \ref{lambdaparablender} 
% with $\lambda$-parablender $(\tilde K_a)_a$ and covered domain $\tilde  O_r=\cup_{\sb\in \sB} \tilde  O_\sb$. Let $(\tilde F_a)_a$ be defined by:
%\[ \tilde F_a(x,y) \colon(t,x,y)\in (-1,1)^{n-2}\times [-3,3]\times D\longmapsto 
%( 0, \tilde f_a(x,y))\; .\]
%Let $\tilde B_a:= \{0\}\times \tilde K_a$ and  note that  $(\tilde B_a)_a$ is a family of hyperbolic sets for $(F_a)_a$. 
%Let $\hat V_r$ be the subspace of $C^r$-families of $(n-2)$-dimensional submanifolds defined in Example \ref{expparablendern}. 
%\begin{fact}
%The family of hyperbolic sets  $(\tilde B_a)_a$ for $(\tilde F_a)_a$ is a $\lambda$-$C^r$-parablender with covered domain:
%\[\hat V_r\times   \{\hat \lambda: J^{r-1}_0 \hat \lambda = \log \frac23+ \sum_{i\in E_{r-1}} \lambda_i a^i:\;  \lambda_i\in [-2\epsilon,2\epsilon ]\}\; .\]
%\end{fact}
%\begin{proof}
%Similarly to Fact \ref{example11factparaL}, the proof is done by merging those of Fact \ref{fact20}  and Example 
%\ref{expparablendern}.
%   \end{proof}
% \end{exam}
%
%%\begin{rema} A possible alternative prove of the main theorem would be to generalize the concept of $\lambda$-$C^r$-parablender in order to obtain not only a control on the parameter jets of points and  of the first differential, but also the $r$-first derivatives. In this manuscript we prefer to exhibit diophantine rotations since they can be useful for many other purposes. Also the necessary condition describing the open set of families should be easier to exhibit typically (to solve conjecture \ref{conjprincipal}).
%%\end{rema}
%
%%\part{A jeuter}
%%
%%
%%\section{Parablender}
%%A way to define "typical dynamics" is to consider $C^d$-generic finite dimensional submanifold of $Diff^r(M)$ (resp.  $C^r(M,M)$,...) and to consider Lebesgue almost every point of this finite dimensional submanifold.
%%
%%
%%Such a submanifold defines a \emph{generic} $C^d$-family $(f_a)_a$ of $C^r$-dynamics.
%%
%%\begin{defi}[Kolmogorov Typical]
%%A property $(P)$ is $C^d$-Kolmogorov typical if for every $C^d$-generic family of diffeomorphisms $(f_a)_a$, for Lebesgue a.e. $a$ the map $f_a$ satisfies property $(P)$. 
%%\end{defi}
%%\begin{conj}[Pugh-Shub]
%%In sens of Kolmogorov, typical diffeomorphisms of a compact manifold have finitely many attractors.
%%\end{conj}
%%In other words, given a generic family $(f_a)_a$ of diffeomorphisms parametrized by $\mathbb R^k$, for Lebesgue almost every parameter $a$, $f_a$ has finitely many attractors.
%%
%%\begin{conj}[Main global Palis Conjecture (1995 version)]
%%\begin{enumerate}[$(i)$]
%%\item There is a dense set $D$ of dynamics such that any element of $D$ has finitely many attractors whose union of basins of attraction has total probability.
%%\item  The attractors of the elements in D support a physical (SRB) measure.
%%\item  For any element in $D$ and any of its attractors, for almost all small perturbations in generic
%%$k$-parameter families of dynamics, $k \in \mathbb N$, there are finitely many attractors whose union of
%%basins is nearly (Lebesgue) equal to the basins of the initial attractors; each such perturbed
%%attractor supports a physical measure.
%%\item  Stochastic stability of attractors: the attractors of elements in $D$ are stochastically stable
%%in their basins of attraction.
%%\item  For generic finite-dimensional families of one dimensional dynamics, with total probability in parameter
%%space, the corresponding systems display attractors satisfying the properties above.
%%\end{enumerate}
%%\end{conj}
%%\begin{thm}
%%For every compact surface $M$, for every $\infty > r\ge 2$, there exist a nonempty open set 
%%$\hat U$ of $C^r$-families of $C^r$-maps of $M$, and a Baire residual set $\mathcal R$ in  $\hat U$  satisfying that:
%%\begin{itemize}
%%\item[$(\star)$]  for every $(f_a)_a\in \mathcal R$, for every $a\in[-1,1]$, the map $f_a$ has infinitely many sinks.\end{itemize}
%%\end{thm}
%%
%%\begin{thm}[\cite{Be15} ]\label{theo2}
%%For every compact manifold of dimension $\ge 3$, for every $\infty > r\ge 2$,
%%there exist a nonempty open set  $\hat U$ of $C^r$-families of $C^r$-diffeomorphisms of $M$, and a Baire residual set $\mathcal R$ in  $\hat U$  satisfying  {$(\star)$}.
%%\end{thm}
%%\bigskip\bigskip
%%
%%These theorems prove that maps with finitely many sinks are not typical in the $C^r$-Kolmogorov sens.
%%The theorems work for $k$-dimensional families, and for smooth maps provided that the regularity of the families is finite.
%
%
%%
%%\newpage



\bibliographystyle{alpha}
\bibliography{references}
\end{document}




\end{document}
