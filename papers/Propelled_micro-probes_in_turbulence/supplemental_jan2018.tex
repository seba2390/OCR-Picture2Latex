\documentclass[aps,amsmath,superscriptaddress,pre,longbibliography,10pt,nofootinbib,citesort]{revtex4-1}

\usepackage{amsmath}
\usepackage{amssymb}
%\usepackage[pdftex,bookmarks,colorlinks]{hyperref}
%\pdfoutput=1

\usepackage[bookmarks,colorlinks]{hyperref}
%\pdfoutput=1
%\usepackage[pdftex]{graphicx}
\usepackage{graphicx}
\usepackage{blindtext}
\usepackage{upgreek}
%\usepackage{subcaption}
\usepackage{calc}
\usepackage{amsmath}
\usepackage{amssymb}
\usepackage{bm}
\usepackage{subfigure}
%\usepackage{color}
\usepackage[usenames,dvipsnames,svgnames,table]{xcolor}
%biblio 

%\usepackage[resetlabels,labeled]{multibib}
%\newcites{Supp}{On Line Sources}
%\usepackage[]{biblatex} % load the package
%\addbibresource{citation.bib} % add a bib-reference file
\usepackage{xcolor}
\newcommand{\red}[1]{\textcolor{red}{#1}}
\newcommand{\blue}[1]{\textcolor{blue}{#1}}



%
%
%%----------------------------------------------------------------------------------------
%%	PACKAGES AND OTHER DOCUMENT CONFIGURATIONS
%%----------------------------------------------------------------------------------------
%
%%\documentclass[fleqn,oneside]{article} % fleqn to align equations left
%%\documentclass[twocolumn, superscriptaddress,pre]{revtex4-1}
%\documentclass[pre,10pt,a4paper,superscriptaddress,nofootinbib,onecolumn,longbibliography,showkeys]{revtex4-1}
%%\linespread{1.1} % Line spacing
%%\documentclass[pre,10pt, a4paperonecolumn]{revtex4-1}
%
%
%\usepackage{placeins} % used for /FloatBarrier
%\usepackage{hyperref} % For hyperlinks in the PDF
%\usepackage{bm}
%\usepackage{amsmath} % package for equation
%%\usepackage[LGRgreek]{mathastext} % non italic equations
%\usepackage[parfill]{parskip} %new paragraph with empty line rather than indent
%\usepackage{amssymb,latexsym} %amsthm package extended theorem environments
%\usepackage{amsthm}
%\usepackage{amsmath}
%\usepackage{mathtools}
%\usepackage{setspace}
%%%% no indent 
%%\setlength\parindent{0pt}
%%\usepackage[square,numbers]{natbib}
%\usepackage{natbib}
%\usepackage{url}
%%\usepackage{listings}
%\usepackage{float}
%\usepackage{color}
%%\usepackage{mathtools, cuted}



%----------------------------------------------------------------------------------------
%	TITLE SECTION
%----------------------------------------------------------------------------------------

\begin{document}
%\title{\vspace{-15mm}\fontsize{14pt}{8pt}\selectfont\textbf{Numerical modeling of jumping behaviour of the copepods in turbulent flows}} % Article title
\title{
%Transect Measurements in Turbulence\\
%\textit{or}
%\textcolor{blue}{Fluid and scalar turbulence along transect trajectories of propelled micro-probes}\\
%\textit{or}\\
%\textcolor{red}{Transect measurements in hydrodynamic and scalar turbulence}\\
%\textit{or}\\
%\textcolor{olive}{Statistical properties of fluid and scalar turbulence\\ along propelled micro-probe trajectories}\\
%\textit{or}\\
%Propelled micro-probes in turbulence
Supplementary Material for:\\
\textit{``Propelled micro-probes in turbulence''}
%\tt{[Enrico: I prefer the first title]}
%\textit{or}\\
%Drifting probes in Turbulence\\ 
}


\author{\textsc{E. Calzavarini}}\email[]{enrico.calzavarini@polytech-lille.fr} 
\affiliation{ \textit{Univ. Lille, CNRS, FRE 3723, LML, Laboratoire de M\'{e}canique de Lille, F 59000 Lille, France}}
\author{\textsc{Y. X. Huang}}\email[]{yongxianghuang@xmu.edu.cn} 
\affiliation{ \textit{State Key Laboratory of Marine Environmental Science, College of Ocean and Earth Sciences, Xiamen University,
Xiamen 361102, People's Republic of China}}
\author{\textsc{F. G. Schmitt}}\email[]{francois.schmitt@cnrs.fr} 
\affiliation{ \textit{Univ. Lille, CNRS, Univ. Littoral Cote d'Opale, UMR 8187, LOG,Laboratoire d'Oc\'{e}anologie et de G\'{e}oscience, F 62930 Wimereux, France}}
\author{\textsc{L. P. Wang}}\email[]{lipo.wang@sjtu.edu.cn} 
\affiliation{ \textit{UM-SJTU Joint Institute, Shanghai JiaoTong University, Shanghai, 200240, People's Republic of China}}

%----------------------------------------------------------------------------------------

\date{\today}
%\maketitle


%%%%%%%%%% Merge with supplemental materials %%%%%%%%%%
%\pagebreak
\widetext
\begin{center}
\textbf{\large Supplementary Material for:\\}
\vspace{3mm} %5mm vertical space
\textbf{\large \textit{``Propelled micro-probes in turbulence''}}\\
\vspace{3mm} %5mm vertical space
{\small
\textsc{E. Calzavarini$^1$, Y. X. Huang$^2$, F. G. Schmitt$^3$, L. P. Wang$^4$}\\
$^1$\textit{Univ. Lille, CNRS, FRE 3723, LML, Laboratoire de M\'{e}canique de Lille, F 59000 Lille, France}\\
$^2$\textit{State Key Laboratory of Marine Environmental Science, College of Ocean and Earth Sciences, Xiamen University, Xiamen 361102, People's Republic of China}\\
$^3$\textit{Univ. Lille, CNRS, Univ. Littoral Cote d'Opale, UMR 8187, LOG,Laboratoire d'Oc\'{e}anologie et de G\'{e}oscience, F 62930 Wimereux, France}\\
$^4$\textit{UM-SJTU Joint Institute, Shanghai JiaoTong University, Shanghai, 200240, People's Republic of China}\\
}
%----------------------------------------------------------------------------------------
\vspace{1mm}
\today


\end{center}
%%%%%%%%%% Merge with supplemental materials %%%%%%%%%%
%%%%%%%%%% Prefix a "S" to all equations, figures, tables and reset the counter %%%%%%%%%%
\setcounter{equation}{0}
\setcounter{figure}{0}
\setcounter{table}{0}
\setcounter{page}{1}
\setcounter{section}{0}
\setcounter{subsection}{0}
\makeatletter
\renewcommand{\theequation}{S\arabic{equation}}
\renewcommand{\thefigure}{S\arabic{figure}}
\renewcommand{\thetable}{S\arabic{table}}
%\renewcommand{\bibnumfmt}[1]{[S#1]}
%\renewcommand{\citenumfont}[1]{S#1}
%%%%%%%%%% Prefix a "S" to all equations, figures, tables and reset the counter %%%%%%%%%%


\section{Single-point statistical relations for fluid velocity and scalar}

\subsection{Variance of time derivatives}
We consider the equation of motion for the probe, $\dot{\bm{x}}_s = \bm{u}( {\bm{x}}_s(t), t)    + \bm{v}_s $,
%\begin{equation}\label{vel}
%\dot{\bm{x}}_s = \bm{u}( {\bm{x}}_s(t), t)    + \bm{v}_s 
%\end{equation}
and compute its total time derivative with respect to time, this leads to:
\begin{equation}\label{sacc}
\ddot{\bm{x}}_s = \partial_t \bm{u}( {\bm{x}}_s(t), t)  +  \dot{\bm{x}}_s  \cdot \bm{\partial} \bm{u}( {\bm{x}}_s(t), t)  = D_t \bm{u}( {\bm{x}}_s(t), t)    + \bm{v}_s \cdot \bm{\partial} \bm{u}( {\bm{x}}_s(t), t) 
\end{equation}
where $D_t (\bullet)  = \partial_t(\bullet) + \bm{u} \cdot \bm{\partial} (\bullet)$ is the material fluid derivative. 
We now compute the variance for a single cartesian component  of the equation and assume that $D_t u_i$ and $\bm{v}_s \cdot \bm{\partial} u_i$ are statistically independent, this leads to:
\begin{equation}\label{temp_derivative}
\langle \ddot{\bm{x}}_{s,i}^2 \rangle = \langle \left( D_t  u_i( {\bm{x}}_s(t), t) \right)^2 \rangle    +  \langle ( \bm{v}_s \cdot \bm{\partial} u_i( {\bm{x}}_s(t), t) )^2 \rangle. 
\end{equation}
Note that because the probes do not cluster and explore the space homogeneously one can substitute ${\bm{x}}_s(t)$ one can also write:
\begin{equation}\label{temp_derivative2}
\langle \ddot{\bm{x}}_{s,i}^2 \rangle = \langle \left( D_t  u_i( {\bm{x}}, t) \right)^2 \rangle    +  \langle ( \bm{v}_s \cdot \bm{\partial} u_i( \bm{x}, t) )^2 \rangle,
\end{equation}
where  ${\bm{x}}$ is a generic spatial coordinate, meaning that the average is taken over all position (volume average) or equivalently over a set of homogeneously distributed points (as for instance an ensemble of  Lagrangian tracers  $\bm{x}_L(t)$).
In statistical isotropic turbulent conditions the following relations hold:
$$\langle \left( D_t  u_i( {\bm{x}}, t) \right)^2 \rangle = a_0\ \epsilon/ \tau_{\eta}, \quad  \langle \left( \partial_j u_i( {\bm{x}}, t) \right)^2 \rangle = 2\ \langle \left( \partial_i  u_i( {\bm{x}}, t) \right)^2 \rangle = 2/15\ \epsilon/ \tau_{\eta}.$$
The first relation can be taken as a definition of the the Heisenberg-Yaglom constant $a_0$, while the seconds are kinematic relations which take into account also the incompressibility constraint (see Hinze book for a derivation).
% which is defined as $ \langle \ddot{\bm{x}}_{L,i}^2 \rangle = a_0\ \epsilon/ \tau_{\eta}$ and $\ddot{\bm{x}}_{L,i}= D_t u_i(\bm{x}_L(t), t)$ is the time derivative of scalar along a fluid tracer and $\epsilon = \nu \langle  (\partial_j u_i + \partial_i u_j)^2 \rangle$.
The second term  in (\ref{temp_derivative}) with the additional assumption of isotropy of the probe ensemble can be rewritten as $\frac{v_s^2}{3}  \Sigma_j \langle  (\partial_j u_i)^2  \rangle = \frac{v_s^2}{9} \epsilon/ \tau_{\eta}$. This altogether leads to equation (5) of the paper.\\

The case of the time derivative of the scalar measured by a probe $ \theta_s =\theta( {\bm{x}}_s(t), t)$ can be treated in a similar way. The following relation holds: 
\begin{equation}\label{temp_derivative}
\dot{\theta}_s = D_t \theta( {\bm{x}}_s(t), t)    + \bm{v}_s \cdot \bm{\partial} \theta( {\bm{x}}_s(t), t) 
\end{equation}
As before we compute the variance of the above quantity and assume that $D_t \theta$ and $\bm{v}_s \cdot \bm{\partial} \theta$ are statistically independent, this leads to:
\begin{equation}\label{temp_derivative_var}
\langle \dot{\theta}_s^2 \rangle = \langle \left( D_t \theta( {\bm{x}}, t) \right)^2 \rangle    +  \langle ( \bm{v}_s \cdot \bm{\partial} \theta( {\bm{x}}, t) )^2 \rangle 
\end{equation}
The first term can be  rewritten in terms of  a new quantity $b_0$, equivalent of the HY constant  for fluid acceleration defined as $ \langle \dot{\theta}_L^2 \rangle = b_0\ \epsilon_{\theta}/ \tau_{\eta}$ and $\dot{\theta}_L = D_t \theta(\bm{x}_L(t), t)$ is the time derivative of scalar along a fluid tracer and $\epsilon_{\theta} = \kappa \langle  |\partial \theta|^2 \rangle$.

The second term on the right-hand-side of (\ref{temp_derivative_var}), can be rewritten as $\frac{v_s^2}{3}  \langle |\partial \theta|^2 \rangle  $ with the assumption of isotropy of scalar field and probe set. By combining these relations we find the prediction give in equation (6) of the paper. A relation that we could check only for the case $Sc=1$.


\subsection{Flatness of time derivatives}
The above relations can be extended to the flatness factor both for the velocity and the scalar. Under the same hypothesis, the fourth order moment relations for velocity and scalar derivatives are, 
\begin{equation}\label{temp_derivative_m4_v}
\langle \ddot{x}_{s,i}^4 \rangle = \langle \left( D_t u_i( {\bm{x}}_s(t), t) \right)^4 \rangle    +  6  \langle \left( D_t u_i( {\bm{x}}_s(t), t) \right)^2 \rangle \langle ( \bm{v}_s \cdot \partial u_i( {\bm{x}}_s(t), t) )^2 \rangle + \langle ( \bm{v}_s \cdot \partial u_i( {\bm{x}}_s(t), t) )^4 \rangle, 
\end{equation}
\begin{equation}\label{temp_derivative_m4}
\langle \dot{\theta}_s^4 \rangle = \langle \left( D_t \theta( {\bm{x}}_s(t), t) \right)^4 \rangle    +  6  \langle \left( D_t \theta( {\bm{x}}_s(t), t) \right)^2 \rangle \langle ( \bm{v}_s \cdot \partial \theta( {\bm{x}}_s(t), t) )^2 \rangle + \langle ( \bm{v}_s \cdot \partial \theta( {\bm{x}}_s(t), t) )^4 \rangle. 
\end{equation}
Again we take into account the isotropy of the Eulerian fields together with the isotropy of the set of probes:

$$ \langle ( \bm{v}_s \cdot \partial u_i( {\bm{x}}_s(t), t) )^4 \rangle =  \frac{v_s^4}{3} \Sigma_j \langle(\partial_j u_i )^4 \rangle =  \frac{v_s^4}{3} \left( \langle (\partial_{\parallel} u_i )^4 \rangle + 2  \langle (\partial_{\perp} u_i )^4 \rangle \right) $$
$$ \langle ( \bm{v}_s \cdot \partial \theta( {\bm{x}}_s(t), t) )^4 \rangle = \frac{v_s^4}{3}\Sigma_j \langle (\partial_j \theta)^4 \rangle $$
This lead to the following relations for the flatness factors

\begin{equation}\label{temp_derivative_flatness}
\mathcal{F}(\ddot{x}_{s,i})= \frac{\langle \ddot{x}_{s,i}^4 \rangle}{\langle \ddot{x}_{s,i}^2 \rangle^2} = \frac{ \mathcal{F}(D_t u_i)  + \frac{6}{9 a_0} \left( \frac{v_s}{u_{\eta}}  \right)^2  + \left( \mathcal{F}(\partial_{\parallel} u_i) + 8 \mathcal{F}(\partial_{\perp} u_i) \right)   \frac{1}{675 a_0^2} \left( \frac{v_s}{u_{\eta}}  \right)^4  }{  1  + \frac{2}{9 a_0} \left( \frac{v_s}{u_{\eta}}  \right)^2  +   \frac{1}{81 a_0^2}  \left( \frac{v_s}{u_{\eta}}  \right)^4   }
\end{equation}
\begin{equation}\label{temp_derivative_flatness}
\mathcal{F}( \dot{\theta}_s)= \frac{\langle \dot{\theta}_s^4 \rangle}{\langle \dot{\theta}_s^2 \rangle^2} = \frac{ \mathcal{F}( D_t \theta)  + \frac{2 Sc}{b_0} \left( \frac{v_s}{u_{\eta}}  \right)^2  + \mathcal{F}(\partial \theta_s)\   \frac{Sc^2}{9 b_0^2} \left( \frac{v_s}{u_{\eta}}  \right)^4  }{  1  + \frac{2 Sc}{3 b_0} \left( \frac{v_s}{u_{\eta}}  \right)^2  +   \frac{Sc^2}{9 b_0^2}  \left( \frac{v_s}{u_{\eta}}  \right)^4   }
\end{equation}
where $\mathcal{F}(\partial_{\parallel} u_i)$, $\mathcal{F}(\partial_{\perp} u_i)$  and $ \mathcal{F}(\partial \theta_s)$ denote respectively the flatness factor of parallel/transverse velocity gradients and of the scalar.

The asymptotic limit, $v_s \to \infty$, of large propulsions gives respectively $\mathcal{F}(\ddot{x}_{s,i}) =  3/25\ \mathcal{F}(\partial_{\parallel} u_i) +   24/25\ \mathcal{F}(\partial_{\perp} u_i)$ and $\mathcal{F}( \dot{\theta}_s) =  \mathcal{F}(\partial \theta_s)$. Note that it is well known that the single point statistical moments of longitudinal and transverse velocity gradients are different. However while for the second moment their relation is analytically know, their relation has not been derived analytically for the fourth order moment, see \cite{PhysRevE.94.023114} for a recent discussion. In our numerics at $Re_{\lambda} = 125$ we find $\langle (\partial_{\perp} u_i) )^4\rangle  \simeq  5.86 \langle (\partial_{\parallel} u_i)^4 \rangle$, a results that is consistent with the numerical findings by \cite{ishihara_kaneda_yokokawa_itakura_uno_2007}.
The latter results implies  $\mathcal{F}(\partial_{\perp} u_i) \simeq 1.47 \mathcal{F}(\partial_{\parallel} u_i) $, from which one get the asymptotic behaviour  $\mathcal{F}(\ddot{x}_{s,i})  \sim  1.53\  \mathcal{F}(\partial_{\parallel} u_i)$.

\section{Direct Numerical Simulations}

\subsection{Flow parameters}\label{parameters}
For completeness and for future reference we report in table \ref{table:euler} the value of all the DNS parameters and the magnitude of all relevant measured turbulence scales in the performed simulations. 

\begin{table}[h]
\begin{center}
\begin{tabular}{ | c | c | c | c | c | c | c | c | c | c|  c | c | c |}
\hline
$Re_{\lambda}$  & $N^3$& $L$ & $T_{tot}$ & $u'$ & $\lambda$ & $\nu$ & $\eta$ &  $\tau_{\eta}$  & $u_{\eta}$ & $\epsilon$   &  $a_0$\\
\hline
\hline
75 &   $256^3$ & 256 & $1.2 \cdot 10^5$ & $2.44 \cdot 10^{-2}$ & 25.6  & $1/120$ &1.5  & 272 &  $5.515  \cdot 10^{-3}$ &$1.143 \cdot 10^{-7}$   & $2.17 \pm 0.01$ \\
\hline
125 & $512^3$ & 512 & $6.4 \cdot 10^4$ & $3.14 \cdot 10^{-2}$& 33.2 & $1/120$ &1.5 & 272 &  $5.515  \cdot 10^{-3}$  &$1.143 \cdot 10^{-7} $  & $2.72 \pm 0.03$\\
\hline
\end{tabular}
\end{center}
\hspace{-4.6cm}
\begin{tabular}{ | c  | c | c | c | c | c | c | c |}
\hline
$Pe_{\lambda_{\theta}}$  & $Sc$ & $\theta'$  & $\lambda_{\theta}$ & $\eta_{\theta}$ &    $\epsilon_{\theta}$  & $\kappa$ & $b_0$\\
\hline
\hline
  43 &  1 &  $3.05 \cdot 10^{-2}$  & 14.6 & 1.5   & $1.143 \cdot 10^{-7}$ & $1/120$  & $0.50 \pm 0.01$ \\
\hline
  68& 1 &    $3.80 \cdot 10^{-2}$ & 18.1  &  1.5    &$1.143 \cdot 10^{-7} $& $1/120$  & $0.48 \pm 0.01$\\
\hline
\end{tabular}
\caption{DNS parameters. All dimensional quantities are given in terms of time ($\delta t$) and space ($\delta x$) discretisation units. 
Furthermore, in the LB algorithm used we assume  $\delta t = \delta x = 1$.  
$Re_{\lambda}$ is the computed Taylor-scale based Reynolds number $Re_{\lambda} = u' \lambda / \nu$, with $u'$  the single-component root-mean-square velocity (note that $u_{rms} = \sqrt{3} u'$),  $\lambda$ is the Taylor micro scale and $\nu$ the kinematic viscosity. $N^3$ is the total number of grid points, $L$ size of the periodic box, $T_{tot}$ is the total duration of the simulation,  $\eta$ , $\tau_{\eta}$  and  $u_{\eta}$ are the Kolmogorov scales, $\epsilon$  kinetic energy dissipation rate,finally $a_0$ is the Heisenberg-Yaglom acceleration constant, which is defined as $a_0  \equiv \langle a_i^2 \rangle \tau_{\eta}^2/\eta$ , where $ \langle a_i^2 \rangle$ denotes the variance of a single cartesian component of the fluid acceleration.
Concerning the scalar:  $\kappa$ is the scalar diffusivity.  $Sc$ is the Schmidt number $Sc=\nu/\kappa$, while the Taylor-scale based P\'eclet number is  defined as $Pe_{\lambda_{\theta}} =  u' \lambda_{\theta}/\kappa$. The root-mean-square fluctuations of the scalar is denoted as $\theta'$, $\lambda_{\theta} = \sqrt{3\ \theta'^2/ \epsilon_\theta} $ is the Taylor scale equivalent for the scalar field, the Batchelor scale is denoted as $\eta_{\theta} = \eta/\sqrt{Sc}$, $\epsilon_{\theta} = \langle \kappa\ \Sigma_i (\partial_i \theta)^2 \rangle$ is the scalar energy dissipation rate.}\label{table:euler}
\end{table}
%%%%%%%%%%%%%%%%%

\subsection{A Lagrangian validation test}\label{validation}

\begin{figure}[!htb]
\begin{center}
 \includegraphics[width = 0.55\columnwidth]{zeta_comparison.eps}
  % \includegraphics[scale = 1.1]{}
\caption{Logarithmic derivative of the fourth order moments of Lagrangian fluid velocity time increments versus the second order one as in Eq. (13) of the paper. Comparison of present simulations at $Re_{\lambda} =125$ with simulations and experiments at comparable Reynolds number from Ref. \cite{Arneodo_2008}.}
\label{sfcomparison}
\end{center}
\end{figure}

In order to assess the appropriate resolution of small-scale intermittent fluctuations in the present DNS based on LB method, we compare the local scaling exponent $\zeta_{4,2}(\tau)$ with the one measured in experiment and in simulations at similar $Re_{\lambda}$ number, which were reported in Ref. \cite{Arneodo_2008}. The plot of such a comparison is reported in figure \ref{sfcomparison}, note that the 
experimental data comes from Particle Tracking Velocimetry (PTV) measurements in a turbulent Von Karman flow flow setup at $Re_{\lambda} = 124$ while, the  numerical data  at $Re_{\lambda} = 140$ and were computed by DNS particle tracking in homogeneous isotropic turbulence in a pseudo-spectral simulation in a tri-periodic cubic box.

%\bibliography{citation}


%\bibliographystyleSupp{longbibliography}
%\bibliographySupp{citation}


%merlin.mbs apsrev4-1.bst 2010-07-25 4.21a (PWD, AO, DPC) hacked
%Control: key (0)
%Control: author (0) dotless jnrlst
%Control: editor formatted (1) identically to author
%Control: production of article title (0) allowed
%Control: page (1) range
%Control: year (0) verbatim
%Control: production of eprint (0) enabled
\begin{thebibliography}{3}%
\makeatletter
\providecommand \@ifxundefined [1]{%
 \@ifx{#1\undefined}
}%
\providecommand \@ifnum [1]{%
 \ifnum #1\expandafter \@firstoftwo
 \else \expandafter \@secondoftwo
 \fi
}%
\providecommand \@ifx [1]{%
 \ifx #1\expandafter \@firstoftwo
 \else \expandafter \@secondoftwo
 \fi
}%
\providecommand \natexlab [1]{#1}%
\providecommand \enquote  [1]{``#1''}%
\providecommand \bibnamefont  [1]{#1}%
\providecommand \bibfnamefont [1]{#1}%
\providecommand \citenamefont [1]{#1}%
\providecommand \href@noop [0]{\@secondoftwo}%
\providecommand \href [0]{\begingroup \@sanitize@url \@href}%
\providecommand \@href[1]{\@@startlink{#1}\@@href}%
\providecommand \@@href[1]{\endgroup#1\@@endlink}%
\providecommand \@sanitize@url [0]{\catcode `\\12\catcode `\$12\catcode
  `\&12\catcode `\#12\catcode `\^12\catcode `\_12\catcode `\%12\relax}%
\providecommand \@@startlink[1]{}%
\providecommand \@@endlink[0]{}%
\providecommand \url  [0]{\begingroup\@sanitize@url \@url }%
\providecommand \@url [1]{\endgroup\@href {#1}{\urlprefix }}%
\providecommand \urlprefix  [0]{URL }%
\providecommand \Eprint [0]{\href }%
\providecommand \doibase [0]{http://dx.doi.org/}%
\providecommand \selectlanguage [0]{\@gobble}%
\providecommand \bibinfo  [0]{\@secondoftwo}%
\providecommand \bibfield  [0]{\@secondoftwo}%
\providecommand \translation [1]{[#1]}%
\providecommand \BibitemOpen [0]{}%
\providecommand \bibitemStop [0]{}%
\providecommand \bibitemNoStop [0]{.\EOS\space}%
\providecommand \EOS [0]{\spacefactor3000\relax}%
\providecommand \BibitemShut  [1]{\csname bibitem#1\endcsname}%
\let\auto@bib@innerbib\@empty
%</preamble>
\bibitem [{\citenamefont {Fang}\ \emph {et~al.}(2016)\citenamefont {Fang},
  \citenamefont {Zhang}, \citenamefont {Fang},\ and\ \citenamefont
  {Zhu}}]{PhysRevE.94.023114}%
  \BibitemOpen
  \bibfield  {author} {\bibinfo {author} {\bibfnamefont {L.}~\bibnamefont
  {Fang}}, \bibinfo {author} {\bibfnamefont {Y.~J.}\ \bibnamefont {Zhang}},
  \bibinfo {author} {\bibfnamefont {J.}~\bibnamefont {Fang}}, \ and\ \bibinfo
  {author} {\bibfnamefont {Y.}~\bibnamefont {Zhu}},\ }\bibfield  {title}
  {\enquote {\bibinfo {title} {Relation of the fourth-order statistical
  invariants of velocity gradient tensor in isotropic turbulence},}\ }\href
  {\doibase 10.1103/PhysRevE.94.023114} {\bibfield  {journal} {\bibinfo
  {journal} {Phys. Rev. E}\ }\textbf {\bibinfo {volume} {94}},\ \bibinfo
  {pages} {023114} (\bibinfo {year} {2016})}\BibitemShut {NoStop}%
\bibitem [{\citenamefont {Ishihra}\ \emph {et~al.}(2007)\citenamefont
  {Ishihra}, \citenamefont {Kaneda}, \citenamefont {Yokokawa}, \citenamefont
  {Itakura},\ and\ \citenamefont
  {Uno}}]{ishihara_kaneda_yokokawa_itakura_uno_2007}%
  \BibitemOpen
  \bibfield  {author} {\bibinfo {author} {\bibfnamefont {T.}~\bibnamefont
  {Ishihra}}, \bibinfo {author} {\bibfnamefont {Y.}~\bibnamefont {Kaneda}},
  \bibinfo {author} {\bibfnamefont {M.}~\bibnamefont {Yokokawa}}, \bibinfo
  {author} {\bibfnamefont {K.}~\bibnamefont {Itakura}}, \ and\ \bibinfo
  {author} {\bibfnamefont {A.}~\bibnamefont {Uno}},\ }\bibfield  {title}
  {\enquote {\bibinfo {title} {Small-scale statistics in high-resolution direct
  numerical simulation of turbulence: Reynolds number dependence of one-point
  velocity gradient statistics},}\ }\href@noop {} {\bibfield  {journal}
  {\bibinfo  {journal} {J. Fluid Mech.}\ }\textbf {\bibinfo {volume} {592}},\
  \bibinfo {pages} {335–366} (\bibinfo {year} {2007})}\BibitemShut {NoStop}%
\bibitem [{\citenamefont {Arn\'eodo}\ \emph {et~al.}(2008)\citenamefont
  {Arn\'eodo}, \citenamefont {Benzi}, \citenamefont {Berg}, \citenamefont
  {Biferale},\ and\ \citenamefont {Bodenschatz~et al.}}]{Arneodo_2008}%
  \BibitemOpen
  \bibfield  {author} {\bibinfo {author} {\bibfnamefont {A.}~\bibnamefont
  {Arn\'eodo}}, \bibinfo {author} {\bibfnamefont {R.}~\bibnamefont {Benzi}},
  \bibinfo {author} {\bibfnamefont {J.}~\bibnamefont {Berg}}, \bibinfo {author}
  {\bibfnamefont {L.}~\bibnamefont {Biferale}}, \ and\ \bibinfo {author}
  {\bibfnamefont {E.}~\bibnamefont {Bodenschatz~et al.}},\ }\bibfield  {title}
  {\enquote {\bibinfo {title} {Universal intermittent properties of particle
  trajectories in highly turbulent flows},}\ }\href@noop {} {\bibfield
  {journal} {\bibinfo  {journal} {Phys. Rev. Lett.}\ }\textbf {\bibinfo
  {volume} {100}},\ \bibinfo {pages} {254504} (\bibinfo {year}
  {2008})}\BibitemShut {NoStop}%
\end{thebibliography}%



% END OF THE DOCUMENT
\end{document}
