\subsection{Objective Function}

The objective function must be smooth and consistent. This basically means that at any point, the linearizion of the system must be a good local approximation. Practically speaking, this means that absolute value function and functions that return the maximum value of a set are absolutely forbidden. There are a few ways to get around this restriction. One option is to directly smooth the function (such as arctangent smoothing), another is to introduce extra variables and constraints, which is more complicated, but nearly exact and better for convergence. Here is an example from Bett's book \cite{betts2010practical}:

\begin{align}
	& \text{BAD: } & |x| &	\\

	& \text{GOOD: } &  min(x_1+x_2) & \quad

 \text{subject to} \quad 
  \begin{cases}
	x=x_1-x_2 \\
	x_1>=0 \\ 
	x_2>=0	
  \end{cases}

\end{align}

%\begin{figure} 
%    \centering 
%    \includegraphics[width = \columnwidth]{img/fancyAbsValue.pdf}    
%    \caption{A fancy way to deal with absolute value functions}     
%    \label{fig:fancyAbsValue}     
%\end{figure} 

\par One commonly used objective function is to use the integral of the input squared against time. This has nice smoothness and consistency problems, and is vaguely related to things that people care about in real life. If an optimization problem is not converging, then it is usually a good idea to test it with a objective function of this form.

\par Another objective function, that is used in walking robotics applications, is Cost of Transport(CoT). This is typically more challenging to implement, but is directly related to real life. Smoothing is often required to use CoT in an optimization. CoT is also difficult for the optimization algorithm because it has a term for work, which typically is a product of some state and some actuator, making it very nonlinear.

%%~~~~~~~~~~~~~~~~~~~~~~~~~~~~~~~~~~~~~~~~~~~~~~~~~~~~~~~~~~~~~~~~~~
\subsection{Integral Objective Functions}

Many objective functions have a term in them that is the integral of some continuous function. While it may be tempting to use a simple integration scheme (such as one that just sums all of the values at the grid-points), this can cause problems. The best plan is to use the same integration algorithm for you dynamics and cost function. Practically, this amounts to creating a fake `state' to represent your integrand, and then passing it through the integration algorithm. This can be awkward to implement, but it is worth it. 

