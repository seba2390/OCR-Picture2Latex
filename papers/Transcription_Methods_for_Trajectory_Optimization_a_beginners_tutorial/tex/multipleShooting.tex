\section{Multiple Shooting Methods}

Multiple shooting is a transcription method for trajectory optimization. It works by breaking the desired trajectory up into a large number of segments. The transcription algorithm then runs an independent simulation for each segment. The feasibility of the trajectory (with respect to the system dynamics) is ensured by a {\em defect} constraint which requires the initial state of each trajectory segment to match the final state of the previous segment. 

\par This is a huge improvement over the more intuitive single shooting method, as shown in Figure \ref{fig:multipleShooting}. The key difference is that in multiple shooting there are many small defects (rather than one large defect). This makes the Jacobian of the defect constraints with respect to the control sparse and nearly linear, both of which dramatically improve the performance of the underlying optimization program.


\begin{figure} 
    \centering 
    \includegraphics[width = \columnwidth]{img/multipleShooting.pdf}    
    \caption{Solving the problem of moving a particle across a vector field. Notice how the multiple shooting produces several small defect vectors, and the single shooting produces one large defect vector. Also notice how the vector field influences the locations that the point passes through, allowing the trajectory to be drawn far from the initial guess.}     
    \label{fig:multipleShooting}     
\end{figure} 


\subsection{Integration Algorithm} \label{sec:IntegrationAlgorithm}

Each segment of the trajectory in the multiple shooting algorithm is calculated by running an explicit simulation. There are a wide variety of methods for computing this simulation, some of which are better than others.

\par In general it is better to have many short segments, rather than a few long segments. In general, the system should be well approximated by a linear model over the duration of each trajectory segment. For a quick approximate solution, a low order integration method such as Euler's method is good. For accurate trajectory, it is best to use a high-order method, such as $4^{th}$ order Runge-Kutta.

\par It is always preferable to use a fixed-step integrator, rather than a variable-step integrator. This is because fixed-step methods are consistent, where as variable-step methods accurate (but not consistent). The variable-step integrator is no consistent because a small change in an input (state or control) could change the number or relative size of integration steps, thus causing a discontinuity in the output across multiple iterations of the optimization algorithm. Consistency is required by the optimization algorithm for the gradients to accurate. A variable step method might make each individual simulation more accurate, but at the expense of the entire trajectory not converging properly. For the same reason, it is important to use an explicit integration method, to avoid an iterative solve that will be present in an implicit method.

\par One problem with a fixed-step explicit integration algorithm is that it is difficult to know the accuracy of the trajectory. This can have a negative effect on the trajectory as a whole, so it is always good to check the optimal trajectory that is returned from the optimization algorithm. A simple check is to take the system and run an open-loop simulation using a high-order variable-step integration algorithm, starting from the same initial conditions and using the same control actions as the optimal trajectory. The resulting trajectory can then be compared to the optimal trajectory - they should be reasonably close, although diverging exponentially in time.

