\subsection {Dynamics}

In order to get good convergence and accuracy, it is necessary that any discontinuities in the dynamics are lined up with a discontinuity in your discritization. For example, if there is a collision in the dynamics, then the problem should be broken up into two different sections: before the collision and after the collision. These two sections should then be constrained together by the discrete impact map for the collision. Another example would the that grid-points for a piecewise linear representation must line up with the grid-points used by the integration algorithm. Failure to do these things will cause failure to converge (or very slow convergence), and inaccuracy in the result.

\par Sometimes there are discontinuities hiding in `continuous' dynamics. Two common examples are the absolute value function and a saturation function, both of which have discontinuities in the first derivative. It is too cumbersome to line up grid-points with these sort of things, so the preferred method of dealing with them is to smooth them out. Examples for these smoothing functions are given in the source code for the Hammer Example, discussed at the end of the report. This sort of discontinuity are particularly problematic when the optimal trajectory lies near the discontinuity, such as trying to minimize $abs(x)$.

\par If there are hybrid dynamics, then it is good to let the optimization algorithm pick the duration of each continuous section of the trajectory. Note that the number of midpoints assigned to each continuous section must be constant throughout the optimization. This means that you must know the order and rough duration of the continuous phases of motion a priori. This is no longer a problem if your algorithm can use iterative grid refinement, although this makes the significantly more complicated.