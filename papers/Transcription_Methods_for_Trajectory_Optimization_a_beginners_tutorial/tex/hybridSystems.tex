\section{Dealing with Hybrid Systems} \label{sec:HybridSystems}

A hybrid system is a dynamical system which has multiple phases of continuous dynamics, separated by discrete transitions.The simplest example of a hybrid system is a bouncing ball. It has a single continuous phase of motion (free fall), and a single discrete transition (collision with the ground). 

\par A more complicated hybrid system would be a walking robot. A few examples of continuous modes might be:

\begin{itemize} \setlength{\itemsep}{-4pt}%
    \item {\bf flight - } both feet in the air
    \item {\bf double stance - } both feet on the ground
    \item {\bf single stance - } one foot on the ground
\end{itemize}

There would then be a discrete transition between each of these modes. Some transitions do not alter the continuous state (such as lifting a foot off of the ground), while other transitions do change the continuous state (a foot colliding with the ground).

\par A naive way to handle such hybrid systems would be to bundle them up inside a big simulation or dynamics function and try to just make a simple transcription method work. This will almost certainly fail. The reason is that the underlying optimization algorithm is typically a smooth, gradient-based method (eg. SNOPT, IPOPT, FMINCON). The discrete transitions of the hybrid system cause the dynamics to be non-smooth. For example, a small change in one variable could cause the collision to happen at a different time, which would then cause a huge change in the defect constraints.

\par There are two generally accepted methods for dealing with hybrid systems: multi-phase methods, and through-contact methods. Multi-phase methods are faster and more accurate, but they require explicit knowledge of the sequence of transitions. Through-contact methods can deal with arbitrary sequences of contacts, but are slower and less accurate.

\par These two methods are analogous to the two different methods of simulation hybrid systems. Multi-phase transcription is like simulating a finite state machine using event detection. Through-contact is analogous to time-stepping.

\subsection{Multi-Phase Methods}

Multi-phase methods are a fairly simple extension of the basic multiple shooting or collocation algorithms. First, the user decides which sequence of phases should occur. For example, in a walking robot, this might be single stance (one foot on the ground) followed by double stance (two feet on the ground). Then the constraints are set up within each of those continuous phases exactly the same as in a simpler problem. Then a set of constraints are added to switch the continuous phases together, satisfying the transition equations.

\par One interesting side effect of this method is that the state of the system can be completely different in each phase of motion, provided that there is some sensible way to link them. This is useful, because it means that each set of continuous dynamics can be written in minimal coordinates.

\par The commercially available transcription program GPOPS II \cite{Patterson2013} provides a sophisticated interface for setting up and solving hybrid system trajectories using multi-phase methods.

\subsection{Through-Contact Methods}

Through-contact methods take a completely different approach. Instead of pushing the discontinuities in the dynamics to a special constraints between each phase, they directly handle the contact constraints at every grid point.

\par The key idea in through-contact optimization is that discontinuities are fine, provided that they are directly handled by constraints. This is done by writing the system dynamics in an impulse-based form, and leaving the contact impulses arbitrary (rather than algebraically eliminating them by assuming the contact mode). The (unknown) contact impulses are then treated as a control variable, which is subject to the following constraints at every grid point:

\begin{align*}
	&d_n > 0            \quad &\text{Contact seperation} \\
	&J_t \leq \mu J_n  \quad &\text{Contact force in friction cone} \\
	&d_n J_t = 0 \quad &\text{Contact force when touching}
\end{align*}

These constraints \footnote{These constraints form what is known as a {\em linear complementarity problem} (LCP).} will provide a unique solution for the contact impulses at every grid-point. This allows for an arbitrary sequence of contacts throughout the trajectory, which is particularly useful for complex behaviors, such as a biped crawling, or standing up from a laying down position, as shown by Mordatch et al \cite{Mordatch2012}.

\par Through-contact trajectory optimization will soon be included as part of Drake \cite{Tedrake2014}, a toolbox for doing robot control design and analysis. 

\subsection{Accuracy of each method}
\par If the sequence of contact modes is unknown, then a through contact method might be more accurate because it can find solution that would be excluded by the prescribed sequence of phases in the multi-phase method. That being said, if both methods manage to find roughly the same solution, then the multi-phase method will be far more accurate.

\par The multi-phase method can be incredibly accurate because the optimization is implicitly doing root finding for the transitions between each continuous phase of motion. In addition to that, the continuous trajectory can be easily represented by a high-order polynomial (either directly in collocation, or indirectly through a Runge-Kutta scheme in multiple shooting). These high-order trajectories are able to match the true solution very well, even with relatively large time steps, since the errors are proportional to the order of the polynomial.

\par The through-contact methods are inherently limited in their accuracy because the transitions between continuous phases of motion are constrained to grid-points, which are not able to be independently adjusted. This means that the approximation errors are proportional to the step-size, drastically limiting the ability to get high-accuracy solutions.



