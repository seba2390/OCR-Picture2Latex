\subsection {Defects}

The whole idea behind multiple shooting is to take a complicated trajectory and break it into many simple trajectories. Each of these simple trajectories can then be adjusted nearly independently of the rest of the trajectory. Each trajectory section is connected to its neighbors using a {\it defect constraint}. Starting from any grid-point along a candidate trajectory, it is possible to integrate the system dynamics forward one step. The system state after this integration can then be compared with the state according to the candidate trajectory. The difference between these two state vectors is called a defect, and the defect constraint is used to tell the optimization algorithm to make all defects equal to zero. In math:

\begin{align}
	&\text{Trajectory}  &\{x_k,u_k\} \\
	&\text{Physics}  &\hat{x}_{k+1} = f(x_k,u_k,k) \\
  	&\text{Defect}  &d_k = \hat{x}_k - x_k
\end{align}

\par This becomes slightly more difficult to implement when there are hybrid dynamics, but the math is still fundamentally the same. Suppose that you want to find a trajectory for a walking system that has two continuous phases (single-stance and double-stance) and two switching events (heel-strike and toe-off). Let's suppose that you decide to give roughly 80\% of the time to single-stance and use 100 grid-points. In this case we get: 

\begin{align}
	&f( \cdot, \cdot, \{1..79\})  & \text{single-stance dynamics} \\
	&f( \cdot, \cdot, 80)  & \text{heel-strike map} \\
	&f( \cdot, \cdot, \{81-99\})  & \text{double-stance dynamics} \\
	&f( \cdot, \cdot, \{100\})  & \text{toe-off map} \\
\end{align}

\par Note here that our math would have a problem because there is no element before $1$ or after $100$. If the trajectory does not need to be periodic, then the solution is to only compute the defects for steps $1-99$. If the trajectory is periodic, then simply allow $k$ to wrap back around to $1$ after it reaches its maximum value.
