\section{Software Specifics}

Once a trajectory optimization problem been transcribed by either multiple shooting or collocation, it must then be solved by nonlinear constrained optimization solver. Two of the most popular algorithms are SNOPT and FMINCON. The following section describe some specific details that are helpful when working with each of these programs.

\par It seems that FMINCON solves the multiple shooting problem by first finding a feasible solution, and then attempting to optimize it. This means that FMINCON can handle a worse initial guess than SNOPT, but it is a little worse at finding the true optimal solution, because it does not allow for much flexibility in the constraints. 

\par Instead of writing your own transcription algorithm, it is possible to buy a commercial version. The best available software now is probably GPOPS II, which internally calls either SNOPT or IPOPT (another solver, not discussed here). 

%%~~~~~~~~~~~~~~~~~~~~~~~~~~~~~~~~~~~~~~~~~~~~~~~~~~~~~~~~~~~~~~~~~~
\subsection{SNOPT \cite{Gill2005}}


	\par A single function call is used for the objective function and all constraints. This function returns a vector, and the first element of the vector is the value of the objective function. All of the following elements are constraints. SNOPT does not require the user to specify which constraints are linear or nonlinear - it automatically computes this.

	\par Since SNOPT combines the objective function and the constraint function, you always need to provide a value for the objective function, even if you are only solving a feasibility problem. In this case, the value of the objective function should be set to zero. More importantly, you must also specify that the bounds of the objective function are also zero. In code this looks like: \\
					\texttt{F(1)==0, Flow(1)==0, Fupp(1)==0}

	\par Do not use a constant term anywhere in your objective function or constraints. All constant terms must be moved to the boundary vectors (\texttt{Flow, Fupp}). This is because of how SNOPT internally stores and estimates the gradients - it will numerically remove any constant terms.

	\par SNOPT does not allow the user to pass any parameters to the objective function, at least when called from Matlab. One way to get around this is to create a single struct of parameters, and save it as a global variable. Once this is done, you can then just read the parameters off the global variable.


%%~~~~~~~~~~~~~~~~~~~~~~~~~~~~~~~~~~~~~~~~~~~~~~~~~~~~~~~~~~~~~~~~~~
\subsection{FMINCON \cite{MatlabOptimizationToolbox2014}}

FMINCON requires that the objective function and constraint function are in separate functions. This creates a problem, because both the constraints and objective need to evaluate the dynamics at every grid-point. To avoid doing all of the work twice, it is best to use a persistent variable (this would be a static variable in C++). Create a single function that does the integration of the dynamics, and have it store the last input state and output solution. When it is called, it checks if the state is the same as last time - if it is, then just return the previous solution.

\par There is a bug of sorts in FMINCON, which prevents you from using state bounds to constrain a state to a specific value. Basically, saying: $0<x<=0$ causes FMINCON to try to divide by a step size of machine precision when doing its finite differencing. Much better to place these sorts of things as an equality constraint, which are handled in a special way. 



%%~~~~~~~~~~~~~~~~~~~~~~~~~~~~~~~~~~~~~~~~~~~~~~~~~~~~~~~~~~~~~~~~~~
\subsection{GPOPS2 \cite{Patterson2013}}

GPOPS2 is a commercially available software which implements {\em ``a general-pupose software for solving optimal control problems using variable-order adaptive orthogonal collocation methods together with sparse nonlinear programming''}

This program is excellent at solving trajectory optimization problems which have a known sequence of contact modes, making an extremely general solver. In my experience it is generally much faster and more accurate than using a more simple multiple shooting or collocation method.
