\subsection{Integration Algorithm}

The integration algorithm is dependent on the discritization of your trajectory. If you are using a piecewise linear (or piecewise constant) representation, then the best method is to use a fixed step method, such as Euler integration or $4^th$ order Runge Kutta. Typically is it sufficient to have one integration step per grid-point, but there is no real harm in taking more steps (it will tend to increase accuracy of the dynamics, but increase computational time). 

\begin{itemize}
	\item{\bf Variable-Step vs. Fixed-Step: } It turns out that it is almost always better to use a fixed-step method when doing multiple shooting. This is because fixed-step methods are very consistent. This consistency is what helps make the optimization algorithm converge quickly. While variable-step methods might be more accurate, they lack in consistency. 

	\item{\bf High-Order vs. Low-Order: } There is some evidence that suggests that one $N^{th}$ order step is better than $N$ first order steps. It probably does not make a difference whether you use many low-order steps or few high-order steps. If searching for a very accurate solution it may be better to use a high-order method, but this address only one of several sources of error in resulting optimal trajectory.

	\item{\bf Error Checking: } Since it is not easy to know the accuracy of a fixed-step method, it is possible to check by comparing the final solution against a new simulation that uses a high-accuracy variable-step method. Integrate each section of the trajectory in series, using the final state of one section as the initial state in the next. If everything worked, the resulting trajectory should be a close match to the solution. Any errors that do appear will be compounded by the integration over the duration of the trajectory.

	\item{\bf Number of Substeps: } If $N_{Int}$ is too small, then there can be large errors in the integration, if $N_{Int}$ is too large, then the integration takes a long time. Given a fixed $N_u$ and $N_{Int}$, convergence rate increased as $N_x$ varied from $1$ (Single Shooting)  through $N_u$ (Collocation). As $N_x$ increases beyond $N_u$ the convergence does not improve and the runtime and memory requirements both increase. 

\par If you decide to use a pseudo-spectral method, then it is best to do the integration implicitly. This is done by implicitly integrating the underlying analytic function, using a weighted sum of the function value at each of the grid-points.

\end{itemize}
