

\section{Hammer Example}

I've created a simple example to demonstrate a simple multiple shooting transcription algorithm (using Matlab's FMINCON to solve the underlying optimization). In this example, the goal is to find a trajectory for hammer that is periodically striking a surface. The hammer is modeled as a point-mass pendulum and powered by a torque source. The cost of using the torque source is modeled as the integral of torque squared against time.

\par A series of plots are shown to detail the progress that the optimization algorithm makes towards the solution. On the left side of each figure is a plot of the state-space trajectory. Note that there is a discrete jump in the trajectory when the hammer strikes the surface. Below this trajectory is a plot that shows how the cost function accumulates over time. On the right side of each figure is a set of three plots, each one showing a single component of the trajectory against time (hammer angle, hammer angular rate, and torque applied to hammer). The last plot has a marker for every gridpoint in the state-space trajectory, and shows the final number of iterations. 

\par In the first plot the trajectories are jagged and have red and black lines. The red lines show the {\it defects} in the trajectory - as the optimization runs these become arbitrarily small. 

\par The full code for running this example is available on my website: \\
\footnotesize{\texttt{http://ruina.tam.cornell.edu/research/MatthewKelly}}


\FloatBarrier

\onecolumn

\begin{figure} 
    \centering 
    \includegraphics[width = \columnwidth]{img/MS_Fig_Iter_1.pdf}    
    \caption{The first iteration of the optimization program. Notice that FMINCON has added a small perturbation to the guess at the control, to help numerically calculate the Jacobian. The initial guess was a trajectory made of two straight lines, and the initial defects are clearly visible.}     
    \label{fig: MS_Fig_Iter_1}     
\end{figure} 

%\begin{figure} 
%    \centering 
 %   \includegraphics[width = \columnwidth]{img/MS_Fig_Iter_2.pdf}    
 %   \caption{}     
  %  \label{fig: MS_Fig_Iter_2}     
%\end{figure} 

\begin{figure} 
    \centering 
    \includegraphics[width = \columnwidth]{img/MS_Fig_Iter_3.pdf}    
    \caption{Even after a few iterations, the defects are no longer visible on the plot - the trajectory is nearly feasible . Notice that the optimization has a first guess at the actuation, and a non-zero cost function.}     
    \label{fig: MS_Fig_Iter_3}     
\end{figure} 

%\begin{figure} 
%    \centering %
%    \includegraphics[width = \columnwidth]{img/MS_Fig_Iter_5.pdf}    
   % \caption{}     
 %   \label{fig: MS_Fig_Iter_5}     
%\end{figure} 

\begin{figure} 
    \centering 
    \includegraphics[width = \columnwidth]{img/MS_Fig_Iter_8.pdf}    
    \caption{Now the trajectory has roughly stabilized, and the optimization program is trying to make small changes to the control to reduce the total cost of the trajectory. Interestingly, the trajectory is smooth, but the control is still fairly discontinuous. This is indicative of an solution that has not fully converged.}     
    \label{fig: MS_Fig_Iter_8}     
\end{figure} 

%\begin{figure} 
%    \centering 
%    \includegraphics[width = \columnwidth]{img/MS_Fig_Iter_13.pdf}    
 %   \caption{}     
 %   \label{fig: MS_Fig_Iter_13}     
%\end{figure} 

%\begin{figure} 
%    \centering 
%    \includegraphics[width = \columnwidth]{img/MS_Fig_Iter_21.pdf}    
%    \caption{Pretty similar to the previous iteration, but with a slightly smaller cost and smoother control function.}     
%    \label{fig: MS_Fig_Iter_21}     
%\end{figure} 

%\begin{figure} 
 %   \centering 
 %   \includegraphics[width = \columnwidth]{img/MS_Fig_Iter_34.pdf}    
 %   \caption{}     
  %  \label{fig: MS_Fig_Iter_34}     
%\end{figure} 

%\begin{figure} 
%    \centering 
%    \includegraphics[width = \columnwidth]{img/MS_Fig_Iter_55.pdf}    
%    \caption{ALmost converged, but there are still a few wiggles to get out of the control solution.}     
%    \label{fig: MS_Fig_Iter_55}     
%\end{figure} 

%\begin{figure} 
%    \centering 
%    \includegraphics[width = \columnwidth]{img/MS_Fig_Iter_89.pdf}    
 %   \caption{}     
 %   \label{fig: MS_Fig_Iter_89}     
%\end{figure} 

%\begin{figure} 
%    \centering 
 %   \includegraphics[width = \columnwidth]{img/MS_Fig_Iter_144.pdf}    
 %   \caption{}     
 %   \label{fig: MS_Fig_Iter_144}     
%\end{figure} 

\begin{figure} 
    \centering 
    \includegraphics[width = \columnwidth]{img/MS_Fig_Iter_148.pdf}    
    \caption{ After 148 iterations the optimization program terminated. Notice that all components of the trajectory are smooth, including the control function. I've put small circles over the start and end of each segment of the trajectory, which now line up almost perfectly.}     
    \label{fig: MS_Fig_Iter_148}     
\end{figure} 


\FloatBarrier