\subsection{Constraints}

In many trajectory optimization problems, there are constraints on the trajectory. It should be noted that these are different than a simple bound on the system input or state. In walking robots, common constraints are: 

\begin{itemize}
	\item reaction forces in friction cone
	\item robot above the ground
	\item robot moving forward
	\item foot hits ground from above... 
\end{itemize}

The most important thing to notice about constraints is that they must also be consistent and smooth. It isn't enough to say ``The lowest point of the foot during the step must be above the ground'' because this will pass data from a different grid-point on each function call. Rather, you need to say ``Every point of the foot trajectory must be above the ground'' (or use some very fancy smoothing in your maximization function). All rules that apply to objective functions regarding smoothness and consistency, also should be applied to calculations of constraints.

\par It is also important to minimize the number of unused constraints, provided that the problem remains smooth and consistent. One way to do this is to leave all of the constraints active, and then as you refine the solution, remove the constraints that are not being actively used. At the end of each run, check to make sure that the solution still satisfies the constraints.

