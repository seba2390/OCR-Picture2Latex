\subsection{Initialization}

Even a well-posed trajectory optimization is likely to fail with a poor initialization. One good method for initializing a trajectory is to guess a few points on the trajectory, and then fit a polynomial to these points. Then this polynomial can be differentiated once to get the first derivatives of the state, and again to get the joint accelerations. Inverse dynamics can then be used to compute the joint torques necessary to produce those accelerations. Then interpolate this rough guess to get the correct grid-points for either your multiple shooting or collocation method.

\par One thing that can go wrong with initialization is that you start the optimization with an infeasible, but locally optimal solution. This commonly occurs when you set the initial control functional to zero. This can be corrected by added a small random noise along the initial guess at the actuator. This has the added benefit of starting you optimization from a new place on each iteration, which can sometimes be helpful to detect local minima.

\par Sometimes the optimization will fail, even with a reasonable initialization, if the cost function is too complicated\footnote{Cost of transport, which is total energy used by a robot divided by it's weight multiplied by the distance travelled, is a particularly difficult cost function to optimize.}. One solution is to write the optimization to use an alternate cost function, such as torque-squared, that is very simple to optimize. Use this alternate cost function when running the first, coarse grid, optimization, and then once a feasible trajectory is found, change to the more difficult cost function. Sometimes this can cause problems with local minima, but it is a good first thing to try when debugging.

\subsection{Local Minimum}

\par Once you have a `converged' solution to the trajectory optimization problem, it is good practice to check if the solution is the 'real' (global) solution. There is no practical way to prove that you have found the global solution in most cases, but you can run the optimization from several initial guesses and check that they all converge to the same solution. Figure \ref{fig:localMinCartoon} shows a cartoon of a one-dimensional constrained optimization problem with three sub-optimal local minima and a single global minimum.
\begin{figure}
\centering
\includegraphics[width = 0.5\textwidth]{img/localMinimumCartoon.pdf}
\caption{Cartoon of a local minimum in a simple 1D constrained optimization problem. Figure inspired by Russ Tedrake's lecture notes for his Underactuated Robotics class.}
\label{fig:localMinCartoon}
\end{figure}

