\section{Introduction}

The goal of trajectory optimization is to find a trajectory (system states and control as a function of time) that minimizes (or maximizes) some cost functional, usually subject to some set of constraints. This report will focus on finding optimal trajectories for continuous (or hybrid) dynamical systems. Much of the difficulty in solving a trajectory optimization problem is converting the continuous trajectory into a series of discrete constraints that a general purpose optimization algorithm can solve. This process is called transcription.

\par One of the best references for learning about this topic is: 

 ``Practical Methods for Optimal Control and Estimation Using Nonlinear Programming'' \cite{JohnT.Betts2001}. 
 
 Another good reference, somewhat more brief, is a paper by Diehl {\em et al.} \cite{Diehl2009}. \ADDSUB{Reference to Russ Tedrake's underactuated robot lectures}.

\subsection{Method Overview}

The most straight-forward way to solve a trajectory optimization is single shooting. Imagine that you are trying to hit a target using a canon. Single shooting is analogous to firing the canon several times, applying a small correction after each shot until you finally hit the target. This is a good method for simple problems, but it rapidly becomes infeasible for even moderately complicated problems.

\par Multiple shooting is a fairly simple improvement on single shooting. Instead of running on big simulation on each iteration, it breaks the trajectory up into many segments, and runs one simulation per segment. Now, in addition to hitting the target, a new set of constraints are added to force all of the segments to line up with each other. These new constraints are called defect constraints. It turns out that multiple shooting is a huge improvement on single shooting. If implemented properly, nearly all trajectory optimization problems can be solved with this method. 

\par Collocation methods are perhaps the most sophisticated way to solve trajectory optimization problems. They are similar to multiple shooting, but there is no numerical simulation of the trajectory over each segment. Instead, The trajectory is stored as an orthogonal polynomial. A feasibly solution is ensured by matching the derivative at special points along the trajectory to the system dynamics.

\subsection{Theorem: }
\begin{quote}
  {\em If there is a flaw in the problem formulation, the optimization algorithm will find it.}\footnote{The proof is left to the student}\cite{JohnT.Betts2001}
\end{quote}