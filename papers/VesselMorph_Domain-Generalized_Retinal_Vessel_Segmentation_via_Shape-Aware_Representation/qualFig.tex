\newcommand{\rsize}{0.160}
\setlength{\tabcolsep}{0.5pt}
\renewcommand{\arraystretch}{0.5}

\begin{figure}[t]
    \centering
    \begin{tabular}{ccc|ccc}
        \multicolumn{3}{c|}{HRF ${\scriptstyle[1000\times 1000]}$} & \multicolumn{3}{c}{ROSE ${\scriptstyle[200\times 200]}$} \\
        $\mathbf{x}$ & $\mathbf{z}^I$ & $\mathbf{z}^S$ & 
        $\mathbf{x}$ & $\mathbf{z}^I$ & $\mathbf{z}^S$ \\
        \specialrule{.1em}{.05em}{.05em}
        \includegraphics[width=\rsize\linewidth]{img/hrf_diabetic_im.png} & \includegraphics[width=\rsize\linewidth]{img/hrf_diabetic_zi.png} &
        \includegraphics[width=\rsize\linewidth]{img/hrf_diabetic_zs.png} & \includegraphics[width=\rsize\linewidth]{img/rose_im.png} & \includegraphics[width=\rsize\linewidth]{img/rose_zi.png} &
        \includegraphics[width=\rsize\linewidth]{img/rose_zs.png} \\
        
        \includegraphics[width=\rsize\linewidth]{img/hrf_diabetic_pred.png} & \includegraphics[width=\rsize\linewidth]{img/hrf_diabetic_yi.png} &
        \includegraphics[width=\rsize\linewidth]{img/hrf_diabetic_ys.png} & \includegraphics[width=\rsize\linewidth]{img/rose_pred.png} & \includegraphics[width=\rsize\linewidth]{img/rose_yi.png} &
        \includegraphics[width=\rsize\linewidth]{img/rose_ys.png} \\
    \end{tabular}
    \caption{Qualitative ablation. The shown patches are $1000\times 1000$pix for HRF diabetic image and $200\times 200$pix for ROSE. \textbf{Top row:} raw image, $\mathbf{z}^I$ and $\mathbf{z}^S$. \textbf{Bottom row:} the VesselMorph segmentation and prediction from each pathway, i.e., $D^T(\Gamma(\mathbf{z}^I,\mathbf{z}^S))$, $D(\mathbf{z}^I)$, and $D(\mathbf{z}^S)$. \textbf{Red} and \textbf{green} indicate the false negative (FN) and false positive (FP), respectively. $\mathbf{z}^I$ may miss large vessels, while $\mathbf{z}^S$ may miss thin ones. The fusion provides robust performance, as can also be seen quantitatively in Supp.\ Fig.\ 1.}
    \label{tab:segment}
\end{figure}