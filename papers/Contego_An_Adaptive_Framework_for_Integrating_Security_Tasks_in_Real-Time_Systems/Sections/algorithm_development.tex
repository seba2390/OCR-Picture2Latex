 \documentclass[../rt_server_main.tex]{subfiles}

 \begin{document}

\section{Algorithm Development} \label{sec:algo}


We develop a simple scheme to obtain the security task's period (for both \pve and \ave mode) and priority-level (for \ave mode). The overall algorithm, Algorithm \ref{alg:sec_schd}, works as follows. 
 

	\renewcommand{\algorithmicforall}{\textbf{for each}}
    \renewcommand\algorithmiccomment[1]{%
 {\it /* {#1} */} %
}
\renewcommand{\algorithmicrequire}{\textbf{Input:}}
    \renewcommand{\algorithmicensure}{\textbf{Output:}}
    
   \newcommand{\Function}[1]{\textbf{function}~\textsc{#1}}
   \newcommand{\EndFunction}{\textbf{end function}}
   
  
    
		\begin{algorithm}[!ht]
        %[H]
        \algsetup{linenosize=\relsize{-0.9}\footnotesize}
  \relsize{-0.9}\footnotesize
			\begin{algorithmic}[1]
				%\begin{footnotesize}
				\REQUIRE Set of real-time tasks, $\Gamma_R$, \pve and 
               \ave mode security tasks $\Gamma_S^{pa}$ and $\Gamma_S^{ac}$, allowable priority ranges $[l_S,m]$
    \ENSURE The tuple $\left\lbrace l^*, \mathbf{T}^{\boldsymbol{pa}}, Q^{pa}, P^{pa},  \mathbf{T}^{\boldsymbol{ac}}, Q^{ac}, P^{ac} \right\rbrace$, \eg \ave mode server priority-level,  \ave and \pve mode periods of the security tasks and \ave and \pve mode server parameters if the task-set is schedulable; $\mathsf{Unschedulable}$	 otherwise	
					\vspace{0.4em}
                   	
				
                
                \STATE Obtain \pve and \ave mode parameters using the functions {\sc {PassiveModeParamSelection($\Gamma_R$, $\Gamma_S^{pa}$)}} and {\sc {ActiveModeParamSelection($\Gamma_R$, $\Gamma_S^{ac}$, $l_S$)}}
             
              \IF {$\mathsf{Solution~ Found ~in~ BOTH ~Modes}$}
                    \STATE  \COMMENT{return the parameters}
                    \STATE \textbf{return} $\left\lbrace l^*, \mathbf{T}^{\boldsymbol{pa}}, Q^{pa}, P^{pa},  \mathbf{T}^{\boldsymbol{ac}}, Q^{ac}, P^{ac} \right\rbrace$
                     
                    \ELSE
\STATE \COMMENT{not possible to integrate security tasks in the system}	                   \STATE \textbf{return} $\mathsf{Unschedulable}$  
                    \ENDIF
         \\\hrulefill           
                    %%% BEGIN PASSIVE MODE
				%\vspace*{1em}
                \vspace*{0.5em}
              \STATE \Function{PassiveModeParamSelection($\Gamma_R$, $\Gamma_S^{pa}$)}
              
               \begin{ALC@g}
        \STATE Initialize \pve mode period $T_i := T_i^{des}, \forall \tau_i \in \Gamma_S^{pa}$ 
        	\STATE Solve $\mathbf{P}\mathbf{\ref{opt:server_param_pve}}$ to obtain server parameters
					\IF {$\mathsf{Solution Found}$}
                    \STATE Solve $\mathbf{P}\mathbf{\ref{opt:period_adapt_server_pve}}$ to obtain security periods
                    \IF {$\mathsf{Solution Found}$}
                    \STATE  \COMMENT{return the parameters}
                    \STATE \textbf{return} $\mathbf{T}^{\boldsymbol{pa}}$, $Q^{pa}$, $P^{pa}$ where $Q^{pa}$, $P^{pa}$ and $\mathbf{T}^{\boldsymbol{pa}}$  are the solutions obtained by $\mathbf{P}\mathbf{\ref{opt:server_param_pve}}$ and $\mathbf{P}\mathbf{\ref{opt:period_adapt_server_pve}}$
                     
                    \ENDIF
                    \ELSE
\STATE \COMMENT{unable to integrate \pve mode security tasks}	                   \STATE \textbf{return} $\mathsf{Unschedulable}$  
                    \ENDIF
        
    \end{ALC@g}
              
    \STATE \EndFunction
    %%% END PASSIVE MODE
     \\\hrulefill 
     %%% Begin Active Mode
  %\vspace*{1em}
  \vspace*{0.5em}

    
  \STATE 
  \Function{ActiveModeParamSelection($\Gamma_R$, $\Gamma_S^{ac}$, $l_S$)}
  
  \begin{ALC@g}
  \STATE $\mathsf{Schedulable} :=$ \textbf{false}
                    \STATE Initialize \ave mode security task's period $\mathbf{T}(l')_{\forall l' \in [l_S, m]} := [T_i^{des}]_{\forall \tau_i \in \Gamma_S^{ac}}^{\mathsf{T}}$ 
					\FORALL{ priority level $l' \in [l_S, m]$
                    } 
					
					
					\STATE Solve $\mathbf{P}\mathbf{\ref{opt:server_param_ave}}$ to obtain server parameters
					\IF {$\mathsf{Solution Found}$}
					
                    \STATE Solve $\mathbf{P}\mathbf{\ref{opt:period_adapt_server_ave}}$ to obtain security periods
                    
                    \IF {$\mathsf{Solution Found}$}
					
                    \STATE \COMMENT{store the parameters for priority level $l'$ where $Q^*$, $P^*$ and $\mathbf{T}^*$  are the solutions obtained by $\mathbf{P}\mathbf{\ref{opt:server_param_ave}}$ and $\mathbf{P}\mathbf{\ref{opt:period_adapt_server_ave}}$}
                    \STATE $Q(l'):= Q^*, P(l'):=P^*, \mathbf{T}(l') := \mathbf{T}^*$ 	
                    \STATE $\mathsf{Schedulable} :=$ \textbf{true}
                    \ENDIF
                    
					
					\ENDIF
					\ENDFOR	
                    \STATE \COMMENT{obtain the parameters that provide best metric}
                    \IF {$\mathsf{Schedulable}$}
                    \STATE Find the priority-level $l^*$ from the solution vector $\mathbf{T}(l')_{\forall l' \in [l_S,m] | \text{~tasks at~} l' \text{~is~} \mathsf{schedulable}}$ that gives the maximum cumulative tightness $\eta^{ac} = \sum_{\tau_i \in \Gamma_S^{ac}} \eta_i$
                    \STATE Set $\mathbf{T}^{\boldsymbol{ac}} := \mathbf{T}(l^*)$, $Q^{ac} := Q(l^*)$, $P^{ac} := P(l^*)$
                    \STATE \COMMENT{return the parameters}
                    \STATE \textbf{return} $l^*$, $\mathbf{T}^{\boldsymbol{ac}}$, $Q^{ac}$, $P^{ac}$
                    
                    \ELSE
\STATE \COMMENT{unable to integrate \ave mode security tasks}	                   \STATE \textbf{return} $\mathsf{Unschedulable}$          
                    \ENDIF
  
 \end{ALC@g}
 
    \STATE 
    \EndFunction
 %%% Done Active Mode               
				%\end{footnotesize}
			\end{algorithmic}
            
            


            
			\caption{Feasibility Checking and Parameter Selection}
			\label{alg:sec_schd}
		\end{algorithm}
		



%We develop a simple scheme to obtain the security task's period (for both \pve and \ave mode) and priority-level (for \ave mode). The overall algorithm, \textbf{Algorithm \ref{alg:sec_schd}}, works as follows. 

To find the \pve mode parameters, we initialize the security task's period with the desired period and solve the server parameter selection problem $\mathbf{P}\mathbf{\ref{opt:server_param_pve}}$ (Lines 10--11). If there exists a solution (\eg the constraints are satisfied), we then obtain the periods of the security tasks by solving $\mathbf{P}\mathbf{\ref{opt:period_adapt_server_pve}}$ (Line 13). In the event that neither of these optimization problems returns a solution, we report the task-set as unschedulable (Line 20), since it is not possible to execute security tasks opportunistically without violating real-time constraints.

To select \ave mode parameters, the algorithm iterates through each of the acceptable priority-levels $[l_S,m]$ 
and tries to obtain the periods that maximize tightness for periodic monitoring without violating the real-time constraints (Lines 26--36). 
If there exists a solution (\eg constraints in $\mathbf{P}\mathbf{\ref{opt:server_param_ave}}$ and $\mathbf{P}\mathbf{\ref{opt:period_adapt_server_ave}}$ are mutually consistent), we store the 
solution in a candidate list. The algorithm then finds the best priority-level from the candidate solution sets that provides the maximum tightness (Line 39). 
In the event that no candidate solutions are found for any of the allowable priority ranges, the algorithm reports the task-set as unschedulable.

If \textit{both} the \pve and \ave mode tasks are schedulable, then Algorithm \ref{alg:sec_schd} returns the corresponding periods and the \ave mode priority-level (Line 4). Otherwise, the system is considered as unschedulable (Line 7) since it is not possible to integrate security tasks with desired requirements. This unschedulability result hints that the designers of the system should update system parameters (\eg the number of security tasks, desired and maximum allowable periods of the security tasks, periods of the real-time tasks, if permissible, \etc) in order to integrate security mechanisms.



 \end{document}