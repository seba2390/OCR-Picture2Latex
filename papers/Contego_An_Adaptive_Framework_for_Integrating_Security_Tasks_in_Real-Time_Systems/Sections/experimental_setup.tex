% \documentclass[../rt_server_main.tex]{subfiles}

% \begin{document}

% \subsubsection{Experimental Setup}
\subsubsection{Simulation Setup}
\label{subsec:ex_setup}

In order to generate task-sets with an even distribution of tasks, we grouped the real-time and security task-sets by base-utilization from $[0.01+0.1 \cdot i, 0.1+0.1 \cdot i]$, where $i \in \mathbb{Z} \wedge 0 \leq i \leq 9$. Each utilization group contained $500$ task-sets. In other words, a total of $5000$ task-sets were tested for each of the experiments. The utilization of the real-time and security tasks were generated by the UUniFast \cite{uunifast} algorithm and we used GGPLAB \cite{ggplab} to solve the optimization problems.

We used the parameters similar to those used in earlier research \cite{sg1,mhasan_rtss16}. In particular, 
each task-set instance contained $[3, 10]$ real-time and $[2, 5]$  security tasks in each of the modes. Each real-time task $\tau_j \in \Gamma_R$ had a period $T_j \in [10~\rm{ms}, 100~\rm{ms}]$
and we assumed $l_S = \lceil 0.4 m \rceil$.  The desired periods for the security tasks $\forall \tau_i \in \lbrace \Gamma_S^{pa} \cup \Gamma_S^{ac} \rbrace$ were selected from $[1000~\rm{ms}, 3000~\rm{ms}]$ and the maximum allowable period was assumed to be $T_i^{max} = 10 T_i^{des}$.  We considered $\omega_i = 1, ~\forall \tau_i \in \lbrace \Gamma_S^{pa} \cup \Gamma_S^{ac} \rbrace$ and the total utilization of the security tasks was assumed to be no more than $30\%$ of the real-time tasks.  

% \end{document}
