% \documentclass[../rt_server_main.tex]{subfiles}

% \begin{document}





\subsection{\remove[MH]{Overview of RTS}} \label{sec:background}

\remove[MH]{RTS are those that, apart from a requirement for functional correctness,
require that temporal properties be met as well. These temporal properties are often
presented in the form of \textit{deadlines}. The usefulness of results produced by the system
drops on the passage of a deadline. If the usefulness drops sharply then we refer to
the system as a  \textit{hard} real-time system (\eg avionics, nuclear power plants, 
anti-lock braking systems in automobiles, \etc) and if it drops in a more gradual manner 
then they are referred to as \textit{soft} real-time systems (\eg multimedia streaming, 
automated windshield wipers, \etc). %Figure \ref{fig:realtime_assumptions} lists 
As illustrated in Table \ref{tab:rts}, some
of the common properties/assumptions related to hard RTS include: periodic/sporadic execution of set of tasks, strict timing and safety requirements, bounded execution time, limited resources, \etc 

% \begin{inparaenum}[\em i\textnormal{)}]
% \item implemented as a system of periodic/sporadic tasks;
% \item stringent timing requirements;
% \item worst-case bounds are known for most loops;
% \item no dynamically loaded or self modified codes;
% \item recursion is either not used or statically bounded;
% \item memory and processing power is often limited.
% \end{inparaenum}



%\todo[inline]{Rod: bounds should be known for all loops, otherwise how do you compute WCET?}
%\hl{MH: fixed!}


RTS are often designed based on the Liu and Layland task model \cite{Liu_n_Layland1973}
that contains a set of tasks, $\Gamma$ where each task $\tau_i \in \Gamma$
has the parameters: $(C_i, T_i, D_i)$, where $C_i$ is
the worst-case execution time (WCET), $T_i$ is the period or minimum inter-arrival time, and $D_i$ is the deadline, with $D_i \leq T_i$.
Schedulability tests \cite{res_time_rts, bini2004schedulability} are used to determine
if all tasks in the system meet their respective deadlines. If they do, then the task set is deemed to be \textit{`schedulable'} and the system, \textit{safe}. }

\hl{MH: This table will be removed.}
\begin{table}[h t b] 
\centering
\caption{Properties of Majority RTS}
\label{tab:rts}
\begin{tabular}{|c p{6.5cm}|} \hline
$\bullet$ & Implemented as a system of periodic/sporadic tasks\\ \hline
$\bullet$ &  Stringent timing requirements\\ \hline
$\bullet$ & Worst-case bounds are known for all loops \\ \hline
$\bullet$ & No dynamically loaded or self modified codes \\ \hline
$\bullet$ & Recursion is either not used or statically bounded \\ \hline
$\bullet$ & Memory and processing power is often limited \\
\hline\end 
{tabular}
\end{table}



% \end{document}