 \documentclass[../rt_server_main.tex]{subfiles}

 \begin{document}

%\subsection{The Security Server} 
%\subsection{Server for Security Tasks} 



%\section{Analyzing the Overhead of Mode Changes}

\subsection{Discussion on Mode Switching} \label{subsec:mode_switch_diss}


As mentioned earlier, by default, \coolname operates in \pve mode. However, when a malicious activity is suspected, a \pve-to-\ave mode change request will be issued. Similarly,  an \ave-to-\pve mode change request will be placed if the system seems clean after fine-grained checking, or a malicious entity is found and removed. %or after a predefined timeout. 
%During any mode change \coolname allows the currently executing security tasks to terminate based on application requirements (\ie if any shared resource is acquired will be released, tasks that are in middle of some computation will be finished desired operation  and/or some less critical security tasks may aborted instantaneously) and the control then  switched from \pve to \ave mode (or vice versa) with new set of security tasks. 
%When any mode change (\ie \pve-to-\ave or \ave-to-\pve) request is initiated \coolname drives from old operating mode to a new operating mode. %The switching from \pve to \ave mode (and vice versa) imposes transitional scheduling overhead which may cause the real-time tasks to miss their deadline. In real-time scheduling analysis, phenomenon is known as the mode-change problem \cite{mc_survey, mc_pedro}. 
% It is worth mentioning that, 
 In steady-state (\eg when security tasks are executing in \pve or \ave mode), the schedulability of the real-time tasks is already guaranteed by the analysis presented in Section~\ref{sec:server_param_selec}. %In order to determine whether the real-time tasks are schedulable with mode changes (\eg including transient overhead) we perform a schedulability analysis based on earlier research \cite{mc_pedro}. The basic idea is to consider each real-time tasks $\tau_j \in \Gamma_R$ which may be affected by the transitional scheduling overhead in either \pve-to-\ave or \ave-to-\pve mode switching and analyze the worst-case response time. In what follows we analyze whether the real-time tasks are also schedulable with this transitional overhead.
 
% \subsubsection*{Switching from Passive to Active Mode}
 
When \coolname switches from \pve mode to \ave mode, the schedulability of real-time tasks will not be affected. The reason this that \textit{all} the real-time tasks are higher priority than the security tasks in \pve mode and hence do not suffer any additional interference from security tasks during mode change. Therefore, the schedulability of real-time tasks during \pve-to-\ave mode switching is already covered by steady-state analysis (Section \ref{subsec:param_pve}).
%\note[RBB]{This is not clear}
 
  %\subsubsection*{Switching from Active to Passive Mode}


During \ave-to-\pve mode switching, observe that schedulability of the real-time tasks that have a priority higher than the  server (\ie $hp_R(\tau_\mathcal{S}^{ac})$) is not affected. When the mode switch request is issued, the \ave mode server (and the security tasks) stop execution and the control is then switched to the lowest priority \pve mode server.
Note that the constraints in Eq.~(\ref{eq:ser_con3_ave}) that ensures the schedulability of the low-priority real-time tasks already captures the worst-case interference introduced by the server. Hence the server will not impose any more interference (even if the mode switch is performed in the middle of the execution of a
busy interval) on the low-priority real-time tasks than what we have calculated in the steady-state analysis (Section \ref{subsec:param_ave}). Therefore if both the \pve and \ave modes task-sets are schedulable, the system will also be schedulable with mode changes.

\begin{comment}



However it is necessary to consider the transient overhead to each real-time task $\tau_j$  that are lower priority than the server (\ie $\tau_j \in lp_R(\tau_\mathcal{S}^{ac})$). %We account this as follows \cite{mc_pedro}. 
Let us define a temporal \textit{window} $w_j(x)$, starting at the activation of real-time task $\tau_j$ and finishing when $\tau_j \in lp_R(\tau_\mathcal{S}^{ac})$ completes. Besides, let \ave-to-\pve mode change request is arrived $x$ time units after the activation of $\tau_j$. We need to consider the interference from other real-time tasks and the server released in the interval $x$. Therefore the total size of the window \cite{mc_pedro} can be expressed by the following recurrence relation: 
\begin{equation}\label{eq:ave-to-pve-window}
\begin{aligned}
	%\begin{align} \label{eq:ave-to-pve-window}
    %w_j(x) = C_j + \left\lceil \frac{x}{P^{ac}} \right\rceil Q^{ac}  &+ \sum_{\tau_h \in hp_R(\tau_j)}  \left\lceil \frac{w_j(x)}{T_h} \right\rceil C_h, ~~\forall j \in lp_R(\tau_\mathcal{S}^{ac})
    w_j^{k+1}(x) = C_j + \left\lceil \frac{x}{P^{ac}} \right\rceil Q^{ac} + \hspace*{-1em} \sum_{\tau_h \in hp_R(\tau_j)}  \left\lceil \frac{w_j^{k}(x)}{T_h} \right\rceil C_h, %\nonumber 
    \\~~\forall j \in lp_R(\tau_\mathcal{S}^{ac})
%	\end{align}
\end{aligned}
    \end{equation}
where $\left\lceil \frac{x}{P^{ac}} \right\rceil Q^{ac}$ is the interference from the server and the term $\sum\limits_{\tau_h \in hp_R(\tau_j)}  \left\lceil \frac{w_j^k(x)}{T_h} \right\rceil C_h$ represents the interference from other high-priority real-time tasks. The initial value $w_j^0(x) = C_j$ and this recurrence is either guaranteed to converge  (\ie $w_j^{k+1}(x) = w_j^{k}(x)$ for some $k$) or exceed some threshold (\ie $D_j$) \cite{mc_pedro, res_time_rts}.
%Since $\left\lceil \frac{x}{P^{ac}} \right\rceil Q^{ac}$ is a monotonically increasing function of $w_j(x)$ and $w_j(x)$ occurs both sides of Eq. (\ref{eq:ave-to-pve-window}), this equation can be re-cast an a recurrence relation as follows:
The worst-case response time of the low-priority real-time tasks $\forall \tau_j \in lp_R(\tau_\mathcal{S}^{ac})$ can be obtained as the duration of the largest time window of $w_j(x)$:
\begin{equation}
W_j = \max~ w_j(x), \quad \forall x \in [0, W_j^{ac}), ~\forall \tau_j \in lp_R(\tau_\mathcal{S}^{ac})
\end{equation}
where $W_j^{ac}$ is the worst-case response time of $\tau_j$ in \ave mode. For a given \ave mode server parameters (\eg $Q^{ac}$ and $P^{ac}$) this response time can be computed using standard worst-case response time analysis \cite{res_time_rts} similar to that of Eq.~(\ref{eq:wcrt_recurrence}). The schedulability test is then performed compared to the deadline, \ie the system is schedulable if $W_j \leq D_j, ~\forall \tau_j \in lp_R(\tau_\mathcal{S}^{ac})$.


\begin{proposition} \label{prop:posy2}
If the \ave mode task-set is schedulable, the system will also be schedulable with mode changes.
\end{proposition}
\begin{proof}
Since $w_j(x)$ is a monotonically increasing function of $x$, the iterative calculation in Eq.~(\ref{eq:ave-to-pve-window}) can be approximated by assuming the number of invocation of higher-priority real-time tasks during $W_j^{ac}$, not during the exact busy period (\eg $x$). %Since $\lceil y \rceil \leq y+1$, we  can linearize and approximate Eq.~(\ref{eq:ave-to-pve-window}) by removing the ceiling function as 
Thus %Eq.~(\ref{eq:ave-to-pve-window}) 
worst-case response time of $\tau_j$ can be rewritten as
\begin{equation} \label{eq:ave-to-pve-window_appx} 
%\begin{aligned}
 w_j^{\prime} = C_j + \left\lceil \frac{W_j^{ac}}{P^{ac}} \right\rceil Q^{ac} + \hspace*{-1em} \sum_{\tau_h \in hp_R(\tau_j)}  \left\lceil \frac{W_j^{ac}}{T_h} \right\rceil C_h, %\nonumber 
    \\~~\forall j \in lp_R(\tau_\mathcal{S}^{ac}).    
%\end{aligned}
\end{equation}
The \ave mode task-set is schedulable and hence $W_j^{ac} \leq D_j^{\prime}, ~\forall j \in lp_R(\tau_\mathcal{S}^{ac})$. Again, since $\lceil y \rceil \leq y+1$, the constraints in Eq~(\ref{eq:ser_con3_ave}) are a loose bound than the conditions in Eq.~(\ref{eq:ave-to-pve-window_appx}). If there exists a solution (\eg the real-time and security tasks are schedulable) in $\mathbf{P}\mathbf{\ref{opt:server_param_ave}}$  that implies the constraints in Eq.~(\ref{eq:ser_con3_ave}) are satisfied. Since the real-time and security tasks are schedulable, satisfying the constraints in Eq.~(\ref{eq:ser_con3_ave}) will also satisfy the condition $ w_j^{\prime} \leq D_j$ and the proof follows.

\end{proof}
\end{comment}

 \end{document}