 \documentclass[../rt_server_main.tex]{subfiles}

 \begin{document}

\section{Period Adaptation}
\label{sec:period_adapt}

As already mentioned, one fundamental problem in integrating security tasks is to determine \textit{which} security tasks will be running \textit{when}.
This brings up the challenge of determining the \textit{right periods} (\viz the minimum inter-execution times) for the security tasks. For instance, some critical security routines may be required to execute more frequently than others. However, if the period is too short (\eg the security task repeats too often) then it will use too much of the processor time and eventually lower the overall system utilization. As a result, the security mechanism itself might prove to be a hindrance to the system and reduce the overall functionality or, worse, safety. In contrast, if the period is too long, the security task may not always detect violations, since attacks could be launched between two instances of the security task.





%It might be a point of interest that 
One may wonder why we cannot assign the desired period (\eg $T_i = T_i^{des}$) in both \pve and \ave modes and set the \ave mode priority level  as $l = l_S$ so that  the security tasks can always execute with the desired frequency (\ie $F_i^{des} = \frac{1}{T_i^{des}}$) and experience less interference (\eg preemption) from real-time tasks. However, since our goal is to integrate security mechanisms in legacy systems with minimal\footnote{In \ave mode \coolname does not introduce any timing violations for the real-time tasks, but their execution might
be delayed due to interference from high-priority security tasks (\eg the tasks with priority-level $l \in [l_S, m]$).} or no perturbation, setting $T_i = T_i^{des}, ~\forall \tau_i$ in either or both mode(s) may significantly perturb the real-time scheduling order. If the schedulability of the system is not analyzed after the perturbation, some (or all) of the real-time tasks may miss their deadlines and thus the main safety requirements of the system will be threatened. The same argument is also true for \ave mode if we set $l = l_S$ (or arbitrarily from the range $[l_S, m]$)  and do not perform schedulability analysis carefully.
% \todo[inline]{Rod: I don't understand the "eventually some ... may miss deadline". Perturbing the scheduling order does not necessarily mean deadline miss.
% RBB: Why will setting $T_i = T_i^{des}, ~\forall \tau_i$ for passive mode tasks impact real-time tasks? Aren't passive security tasks of a lower priority? }
% \hl{MH: re-phrased!}

\subsubsection*{Tightness of the Monitoring}


As mentioned earlier, the actual period as well as the priority-levels of the security tasks  are unknown and we need to \textit{adapt} the periods within acceptable ranges. We measure the security of the system by means of  \textit{achievable periodic monitoring}. Let $T_i$ be the period of the security task $\tau_i \in \lbrace  \Gamma_S^{pa} \cup \Gamma_S^{ac} \rbrace$ that needs to be determined. Our goal is to minimize the gap between the achievable period $T_i$ and the desired period $T_i^{des}$ and therefore we define the following metric:
\begin{equation}
\eta_i = \frac{T_i^{des}}{T_i},
\end{equation}
that denotes the \textit{tightness} of the frequency of periodic monitoring for the security task $\tau_i$. Thus $\eta^{pa} =  \sum\limits_{\tau_i \in \Gamma_S^{pa}} \omega_i \eta_i$ and $\eta^{ac} =  \sum\limits_{\tau_i \in \Gamma_S^{ac}} \omega_i \eta_i$ denote the \textit{cumulative tightness} of the achievable periodic monitoring for \pve and \ave mode, respectively.
This monitoring frequency metric, provides for instance,  one way to trade-off security with schedulability. Recall that
if the interval between consecutive monitoring events is too large, the adversary may remain undetected and harm the system between two invocations of the security task. Again, a very frequent execution of security tasks may impact the schedulability of the real-time tasks. This metric $\eta^{(\cdot)}$ will allow us to execute the security routines with a frequency closer to the desired one while respecting the temporal constraints of the other real-time tasks.

\subsection{Problem Overview} \label{subsec:proble_overview}
% \todo[inline]{should we just say problem overview?}


One may wonder why we cannot schedule the security tasks in the same way that the existing real-time tasks are scheduled. For instance, a simple approach to integrating security tasks in \pve mode without perturbing real-time scheduling order is to execute security tasks at a \textit{lower priority} than all real-time tasks. Hence, the security routines will be executing only during slack times when no other higher-priority real-time tasks are running. Likewise, in \ave mode, security tasks can be executed at a lower priority than more critical, high-priority real-time tasks. 
Hence, the security tasks will only be executing when other real-time tasks with priority-levels higher than $l_S$ are not running. 

When both real-time and security tasks follow RM priority order, we can formulate a nonlinear optimization problem for \pve mode with the following constraints that maximizes the cumulative tightness of the frequency of periodic monitoring:

\begin{myoptimizationproblem} \label{opt:period_adapt_ns_pve}
\vspace*{-2.0em}
\begin{subequations}
\begin{align}
 %& \underset{\mathbf{T}^{\boldsymbol p}}{\operatorname{max}}  \sum_{\tau_i \in \Gamma_S^{passive}} \omega_i \frac{T_i^{des}}{T_i}  \nonumber \\[-1.0ex]
 & \hspace*{5em} \underset{\mathbf{T}^{\boldsymbol {pa}}}{\operatorname{max}}~~\eta^{pa}  \nonumber \\
\hspace*{-0.45em}\text{Subject to:}~~ \nonumber \\
\sum_{\tau_i \in \Gamma_S^{pa}}\frac{C_i}{T_i} &\leq (m+n_p)(2^{\frac{1}{m+n_p}} - 1) - \sum_{\tau_j \in \Gamma_R}\frac{C_j}{T_j} \label{eq:period_ns_const_2_pve}\\
T_i &\geq \underset{\tau_j \in \Gamma_{R} }{\operatorname{max}} T_j \quad \quad \forall \tau_i \in \Gamma_S^{pa} \label{eq:period_ns_const_rm_pve} \\
T_i^{des} &\leq T_i \leq T_i^{max}  \quad \forall \tau_i \in \Gamma_S^{pa} \label{eq:period_const_1_pve}
\end{align}
\end{subequations}
\end{myoptimizationproblem}  
\hspace{-1.9em}
where $\mathbf{T}^{\boldsymbol{pa}} = [ T_1, T_2, \cdots, T_{n_p} ]^{\mathsf{T}}$ is the optimization variable for \pve mode that needs to be determined. The constraint in Eq.~(\ref{eq:period_ns_const_2_pve}) ensures that the utilization %\remove[MH]{(\eg ratio between WCET and period)}
of the security tasks are within the remaining RM utilization bound \cite{Liu_n_Layland1973}. The RM priority order for real-time and security tasks is ensured by the constraints in Eq.~(\ref{eq:period_ns_const_rm_pve}), while  Eq.~(\ref{eq:period_const_1_pve}) ensures the restrictions on periodic monitoring. %Our earlier approach \cite{mhasan_rtss16} is similar to the \pve mode of \coolname.


Recall that in \ave mode,  we allow the security tasks to execute when the real-time tasks with priority-levels higher than $l_S$ are not running. Hence, to ensure the RM priority order in \ave mode, we need to modify the constraints in Eq.~(\ref{eq:period_ns_const_rm_pve}) as follows:
\begin{equation}
T_i \geq \underset{\tau_j \in \Gamma_{R_{hp(l_S)}} }{\operatorname{max}} T_j, \quad \forall \tau_i \in \Gamma_S^{ac} \label{eq:period_ns_const_rm_ave}
\end{equation}
where $ \Gamma_{R_{hp(l_S)}}$ represents the set of real-time tasks that are higher priority than level $l_S$. In addition, %Eq.~(\ref{eq:period_ns_const_rm_pve}), 
the constraints in Eq.~(\ref{eq:period_ns_const_2_pve}) and 
Eq.~(\ref{eq:period_const_1_pve}) also need to be updated to consider \ave mode task-sets (\eg $\Gamma_S^{ac}$) and the number of active mode security tasks ($n_a$). Thus for \ave mode we can formulate an optimization problem similar to that of $\mathbf{P}\mathbf{\ref{opt:period_adapt_ns_pve}}$ with the objective function: $\underset{\mathbf{T}^{\boldsymbol{ac}}}{\operatorname{max}} ~~  \eta^{ac}$, where $\mathbf{T}^{\boldsymbol{ac}} = [ T_1, T_2, \cdots, T_{n_a} ]^{\mathsf{T}}$ is the \ave mode optimization variable.

One of the limitations of the above approach is that the overall system utilization is limited by the RM bound which has the theoretical upper bound of processor utilization only about $\lim\limits_{n \rightarrow \infty} n(2^{\frac{1}{n}}-1) = \ln 2 \approx 69.31\%$ \cite{Liu_n_Layland1973}, where $n$ is the total number of tasks under consideration. Further, the security tasks' periods need to satisfy the constraints in Eq.~(\ref{eq:period_ns_const_rm_pve}) and Eq.~(\ref{eq:period_ns_const_rm_ave}) (for \pve and \ave modes, respectively) to follow RM priority order. In addition, instead of focusing only on optimizing the periods of the security tasks, \coolname aims to provide a \textit{unified} framework that can achieve other security aspects (\viz responsiveness). Thus we follow an alternative approach similar to one we proposed in earlier work\footnote{The approach we proposed in our earlier work \cite{mhasan_rtss16} is analogous to the \pve mode of \coolname.} \cite{mhasan_rtss16}. Specifically, we had proposed 
to use a \textit{server} \cite{server_ab_uk} to execute security tasks. Our security server is motivated by the needs of hierarchical scheduling \cite{periodic_server_qp}. Under hierarchical
scheduling, the system is composed of a set of
components (\eg real-time tasks and a security server, in our context) and each of which comprises multiple tasks or subcomponents (\eg security tasks). The server abstraction not only allows us to provide better isolation between real-time and security tasks, but also enables us to integrate additional security properties (such as responsiveness) as we discuss in the following. 
% \todo[inline]{Rod: I would use the actual bound (log 2) rather than 69\%, specially if people in this community don't knwo where this is coming from (69 seems arbitrary)}
% \hl{MH: thanks, updated!}

%\subfile{Sections/server_overview}



 \end{document}