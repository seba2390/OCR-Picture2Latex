 \documentclass[../rt_server_main.tex]{subfiles}

 \begin{document}

%\section{\remove[MH]{System and Security Model} \add[MH]{Security and System Model}} \label{sec:system_model}

\section{Security and System Model} \label{sec:system_model}

% \subfile{Sections/background}

% \hl{MH: Remove from here and move to back.}



\subsection{Attack Model}
\label{subsec:sec_model}


RTS face threats in various forms, depending on the system and the goals of an adversary. For example, adversaries may insert, eavesdrop on or modify messages exchanged by system components, may manipulate the processing of sensor inputs and actuator commands  and/or could try to modify the control flow of the system \cite{securecore}. Further, rather than try to crash the system aggressively, an intruder in reconnaissance mode may want to monitor the system behavior and gather information for later use. %\remove[RBB]{the intruder may utilize side-channels to monitor the system behavior and infer certain degree of system information}. 
For instance, an intruder may utilize side-channels to monitor the system behavior and infer system information (\eg  hardware/software architecture, user tasks and thermal profiles, \etc) that may eventually help maximize the impact  of an attack \cite{cy_side_channel}.
\begin{comment}
%\remove[RBB]{For example, the intruder may utilize side-channels to monitor the system behavior and infer certain degree of system information (\eg  hardware/software architecture, user tasks and thermal profiles, \etc) that eventually leads to the attacker actively taking control, manipulating and/or crashing the system.} 
%\remove[MH]{Besides, attackers may manipulate the processing of sensor inputs and actuator commands and/or could try to modify the control flow of the system. as well as glean sensitive information through side channels \cite{sg1, sibin_RT_security_journal}. }

%\hl{[RBB: these are not the right papers to cite here] 

% MH: any suggestion for other papers to cite here? or should we just state without any reference?}  
%(\eg hardware/software architecture, user tasks and thermal profiles, \etc) that eventually leads to the attacker actively taking control, manipulating and/or crashing the system. 

%The adversarial capabilities we consider in this work are summarized below.
%\todo{RBB: Do we need this?}
\end{comment}
While the class of attacks can be broadened, for illustrative examples let us consider the following adversarial capabilities:


\begin{enumerate}[\it i\normalfont )]

\item \textit{Integrity violation}: %We assume that the 
An adversary can get a foothold in the system \cite{cy_side_channel,taskshuffler}. For example, an adversary may insert a malicious task that respects the real-time guarantees of the system to avoid immediate detection, and/or compromise one or more existing real-time tasks. %She may use 
Such a task can be used to manipulate sensor inputs and actuator commands for instance and/or modify system behavior in undesirable ways.

%including underlying \textit{real-time OS (RTOS)} as well as the \textit{real-time applications}.

\item \textit{Denial of Service (DoS)}: The attacker may take control of the real-time task(s) and perform \textit{system-level} resource (\eg CPU, disk, memory, \etc) exhaustion.  In %addition, 
particular, when critical tasks are scheduled to run
an advanced attacker may capture I/O or network ports and perform \textit{network-level} attacks to tamper with the confidentiality and integrity (\viz safety) of the system.

% \remove[MH]{
% \item \textit{Information leakage through side-channels}:
% The adversary may also aim to learn important information by side or covert-channel attacks by simply lodging themselves in the system and extracting sensitive information. For example, the intruder may utilize side-channels to monitor the system behavior and infer certain degree of system information (\eg  hardware/software architecture, user tasks and thermal profiles, \etc) that eventually leads to the attacker actively taking control, manipulating and/or crashing the system.}
\end{enumerate}


Threats to communications are usually dealt with by integrating cryptographic protection mechanisms. From an RTS perspective this increases the execution time of existing real-time tasks~\cite{xie2007improving, lin2009static}. %\remove[MH]{ In contrast, our focus is on adversaries who want to penetrate the system and stay undetected till they are ready to launch the attacks. For example, attackers may want to stay undetected during reconnaissance to, (\textit{i}) both collect necessary information to enable their attacks and (\textit{ii}) to not alert system operators since defensive actions might make attacking the system more difficult. }
\coolname is different from earlier work  in which integration of security impacted the schedulability \cite{sibin_RT_security_journal,sg1, sg2}, required modification of the existing schedulers \cite{xie2007improving, lin2009static}, or necessitated architectural modifications \cite{slack_cornell,securecore}. In this work, we focus on incorporating security mechanisms into legacy systems in which added security tasks are \textit{not} allowed to violate the temporal requirements (in either the \pve or \ave modes) and must have \textit{minimal} impact on the schedule of existing real-time tasks.
%Besides, we assume that the scheduler is trustworthy and schedules the security tasks based on the parameters obtained by the proposed framework.


Let us consider an RTS (say an avionics electronic control unit) developed using a multi-vendor model \cite{sg2}, \viz  its components are manufactured and integrated by different vendors. For example, tasks in the system component manufactured by vendor $v_i$ are very sensitive and considered classified or mission-critical (\eg images captured by the camera on the surveillance UAV). It may be undesirable for any vendor $v_j \neq v_i$ to gain unintended information about sensitive contents, even if, say, vendor $v_j$ is trusted with control tasks for controlling the RTS. Similarly, the control laws from vendor $v_j$ may contain a proprietary algorithm and vendor $v_j$ may not want other vendors to gain knowledge about the algorithm. Protected communications and network monitoring/detection mechanisms are necessary but insufficient to deal with such threats. Therefore, \textit{additional security tasks} may need to be added into the system to deal with such threats \cite{sibin_deeply}. %\note[MH]{reference added}. 

The security mechanisms could be protection, detection or response mechanisms, depending on the system requirements. For example, a sensor measurement correlation task may be added to detect sensor manipulation, a change detection task may be added to detect intrusions or additional state-cleansing tasks \cite{sg1, sg2, sibin_RT_security_journal} can be added to deal with stealthy adversaries trying to glean sensitive information through side channels.
%Further, undesired information flows between components from different vendors may need to be prevented. 


It is worth mentioning that the addition of such security mechanisms may necessitate changes to the schedule of real-time tasks \cite{xie2007improving, lin2009static, sg1, sg2, sibin_RT_security_journal}. In contrast, \coolname aims to integrate such security tasks \textit{without} impacting the timeliness constraints (\ie schedulability) required for safe operation (in both modes) and retaining the original schedule of real-time tasks most of the time (\eg in \pve mode when security tasks are executing opportunistically with lowest priority). We highlight that rather than designing specific intrusion detection tasks that target specific attack behaviors, the generic framework proposed in this work allows one to integrate a given security mechanism (referred to as \textit{security tasks}) into the system without perturbing the system parameters (\eg period of the real-time tasks, execution order, \etc).



\subsection{Overview of \coolname} \label{subsec:timeshield_overview}


%\hl{Add fig here}
% \subsubsection{\remove[MH]{Overview of \coolname}}

   \begin{figure}[!t]
%\begin{wrapfigure}{r}{0.55\columnwidth}
%\vspace{-2.0\baselineskip}
%\centering
%\hspace*{-2.7em}
\includegraphics[width=\linewidth]{timeshield_schematic}
%\vspace*{-2.7em}
\caption{\coolname: Flow of operations depicting the \pve and \ave modes for the security tasks.}
\label{fig:flow_operation}
%\vspace{-0.08\baselineskip}
%\end{wrapfigure}
 \end{figure}


As illustrated in Fig.~\ref{fig:flow_operation}, \coolname improves the security posture of the system (that contains a set of real-time tasks) by integrating additional security tasks and allowing them to execute in two different \textit{modes} (\viz \pve and \ave). %The flow of operation of \coolname is as follows. 
If the system is deemed to be clean (\ie not compromised), security routines can execute \textit{opportunistically}\footnote{Which is also the default mode of operation.} (\eg when other real-time tasks are not running). However if any anomaly or unusual behavior is suspected, the security policy may switch to \ave mode (\eg more fine-grained checking or response) and execute with \textit{higher priority} for a \textit{limited amount of time} (since our goal is to ensure security with minimum perturbation of the scheduling order of the real-time tasks). %Besides, in \ave mode, \coolname also provide a trade-off mechanism between security and real-time requirements by adding a \textit{tolerance on deadline} for non-critical low-priority real-time tasks. 
The security routines may go back to normal (\eg~\pve) mode  if: 
\begin{itemize}
\item No anomalous activity is found within a predefined time duration, say $T^{AC}$; or
\item The intrusion is detected and malicious entities are removed (or an alarm triggered if human intervention is required). 
\end{itemize}

% \ci no anomalous activity is found within a predefined time duration say $T^{AV}$; %\textit{ii}) a predefined \ave mode monitoring duration is expired; 
% or \cii the intrusion is detected and malicious entities are removed (or an alarm may be triggered if human intervention is required). 
Although we allow the security tasks to execute with higher priority than some of the real-time tasks in \ave mode, the proposed framework ensures that the timeliness constraints (\eg deadlines) for \textit{all} of the real-time tasks are always satisfied in \textit{both} modes. By using this strategy, \coolname not only enables \textit{compatibility with legacy systems} (\eg in normal situation real-time scheduling order is not perturbed), but also provides \textit{flexibility to promptly deal with anomalous behaviors} (\ie the security tasks are promoted to higher priority so that they can experience less preemption and achieve better response times).


\subsection{System Model} \label{subsec:sec_task}

% \note[MH]{Moved from previous page}
\subsubsection{Real-Time Tasks}
In this paper we consider the widely used fixed-priority sporadic task %\footnote{Sporadic tasks can make an execution request at \textit{any} time, but with a \textit{minimum} inter-invocation interval \cite{sporadic_task}.} 
model \cite{sporadic_task}. Let us consider a uniprocessor system consisting of $m$ fixed-priority
sporadic real-time tasks $\Gamma_R = \lbrace \tau_1, \tau_2, \cdots , \tau_m \rbrace$. Each real-time task $\tau_j \in \Gamma_R$ is characterized by $(C_j, T_j , D_j)$, where $C_j$ is the WCET, $T_j$ is the minimum inter-arrival time (or period) between successive releases and $D_j$ is the relative deadline.  We assume that priorities are distinct and assigned according to the rate monotonic (RM) \cite{Liu_n_Layland1973} order. %\remove[MH]{, that is, the shorter the task period the higher the priority}. 
%use $pri(\tau_i)$ to denote the priority of task $\tau_i$. 
% We also assume that tasks have implicit deadlines, \eg $D_j = T_j$ for $\forall \tau_j \in \Gamma_R$. It is worth mentioning that the fixed-priority sporadic task model we consider here is commonly used in
% %avionics \cite{fp_in_practice_2, fp_in_practice_1})
% contemporary systems and standards such as automotive (\eg AUTOSAR \cite{autosar}), avionics (\eg ARINC 653 \cite{arinc}), POSIX real-time extensions \cite{fp_posix}, \etc

% \todo[inline]{Rod: Both AUTOSAR and ARINC 653 standards mandate fixed priority. Might want to cite.

% Rod: minor, but generally $N$ is used for number of tasks rather than $m$ :)}
% \hl{MH: thanks, added the references}


The processor utilization of $\tau_j$ is defined as $U_j = \frac{C_j}{T_j}$. Let $hp_R(\tau_j)$ and $lp_R(\tau_j)$ denote the sets of real-time tasks that have higher and lower priority than $\tau_j$, respectively. We assume that the real-time task-set $\Gamma_R$ is \textit{schedulable} by a fixed-priority preemptive scheduling algorithm. Therefore, the worst-case response time %\remove[MH]{(\eg the difference between activation time and completion time)} 
$w_i$ is less than or equal to the deadline $D_i$ and the following inequality is satisfied for all tasks $\tau_j \in \Gamma_R$:
%\begin{equation} \label{eq:wcrt}
$w_j \leq D_j$,
%\end{equation}
%In Eq.~(\ref{eq:wcrt}), 
where $w_j = w_j^{k+1} = w_j^k$ is obtained by the following recurrence relation \cite{res_time_rts}:
\begin{equation} \label{eq:wcrt_recurrence}
\begin{aligned}
w_j^0 = C_j, \quad %\\ 
w_j^{k+1} = C_j + \sum_{\tau_h \in hp_R(\tau_j)} \left\lceil \frac{w_j^k}{T_h} \right\rceil C_h.
\end{aligned}
\end{equation}
% \begin{equation} \label{eq:wcrt}
% w_i^{k+1} = C_i + \sum_{\tau_h \in hp_R(\tau_i)} \left\lceil \frac{w_i^k}{T_h} \right\rceil C_h \leq D_i
% \end{equation}
%where 
In Eq.~(\ref{eq:wcrt_recurrence}), $\sum\limits_{\tau_h \in hp_R(\tau_j)} \left\lceil \frac{w_j^k}{T_h} \right\rceil C_h$ is the worst-case interference to  $\tau_j$ due to preemption by the tasks with higher priority than $\tau_j$ (\eg $hp_R(\tau_j)$). %The recurrence starts with  $w_i^0 = C_i$ and will have a solution if $w_i = w_i^{k+1} = w_i^k$ for some $k$.  
The recurrence will have a solution if $w_j^{k+1} = w_j^k$ for some $k$.  

\subsubsection{Security Tasks}

% \todo[inline]{Rod: please clarify in this section that the security tasks also have deadline = period. Otherwise, it is not clear what you mean by ``security tasks are schedulable'' later on}

% \hl{MH: thanks, updated!}

With a view of integrating  security into the system, let us add additional fixed-priority security tasks that will be executed in \pve and \ave modes. We model \pve and \ave mode security tasks as independent \textit{sporadic tasks}. The \pve and \ave mode tasks are denoted by the sets $\Gamma_S^{pa} = \lbrace \tau_1, \tau_2, \cdots , \tau_{n_p} \rbrace$ and $\Gamma_S^{ac} = \lbrace \tau_1, \tau_2, \cdots , \tau_{n_a} \rbrace$, respectively.
%As mentioned earlier, we ensure security of the system by integrating additional security tasks. 
%Let us denote the security tasks by the set $\Gamma_S = \lbrace \tau_1, \tau_2, \cdots , \tau_n \rbrace$. 
We assume that security tasks in both modes follow RM priority order. Each security task $\tau_i \in \lbrace \Gamma_S^{pa} \cup \Gamma_S^{ac} \rbrace$ is characterized by the tuple $(C_i, T_i^{des}, T_i^{max}, \omega_i)$, where $C_i$ is the WCET, $T_i^{des}$ is the most desired period between successive releases (hence $F_i^{des} = \frac{1}{T_i^{des}}$ is the desired execution frequency of a security routine) and $T_i^{max}$ is the maximum allowable period beyond which security checking by $\tau_i$ may not be effective. The parameter $\omega_i > 0$ is a designer-provided weighting factor that may reflect the criticality of the security task\footnote{As an example, the default configuration of Tripwire \cite{tripwire}, an intrusion detection system (IDS) for Linux that we use as case study in Section \ref{subsec:implementation_BBB}, has different criticality levels (\viz weights), \ie  \textit{High} (for scanning files that are significant points of vulnerability), \textit{Medium} (for non-critical files that are of significant security impact) and so forth.} $\tau_i$. %\note[RBB]{The footnote is not clear. Do we mean to say that task moniroting highly critical files has higer criticality?}
Critical security tasks would have larger $\omega_i$. The security tasks have implicit deadlines, \eg $D_i = T_i, \forall \tau_i$ that implies security tasks should complete before %\hl{the arrival of the next instance}. 
their next monitoring instance. 
We do not make any specific assumptions about the security tasks in different modes. For instance, both \pve and \ave mode task-sets may contain completely different sets of tasks (\eg $\lbrace \Gamma_S^{pa} \cap \Gamma_S^{ac} \rbrace = \emptyset$) or may contain (partially) identical tasks with different parameters (\eg period and/or criticality requirements). 
%\footnote{Intuitively, it makes more sense that the \ave mode security tasks would require higher execution frequency, \eg $F_i^{des} > F_j^{des}, ~\forall \tau_i \in \Gamma_S^{passive}$ and $\forall \tau_j \in \Gamma_S^{passive}$.})
 
 
In \pve mode, security tasks are executed with \textit{lower priority} than the real-time tasks. Hence the security tasks do not have any impact on real-time tasks and cannot perturb the real-time scheduling order. In \ave mode, we allow the security tasks to execute with a priority higher than that of certain low priority real-time tasks. This provides us with a trade-off mechanism between security (\eg responsiveness) and system constraints (\eg scheduling order of real-time tasks). Since the task priorities are distinct, there are $m$ priority-levels for real-time tasks (indexed from $0$ to $m-1$ where level $0$ is the highest priority). Among the $m$ priority-levels, we assume that \ave mode security tasks can execute with a priority-level up to $l_S$ $(0 < l_S \leq m),~ l_S \in \mathbb{Z}$. %Notice that $l_S=m$ implies that the security tasks execute with lowest priority and allowed to run \textit{only} when other real-time tasks are not running. 

Although any period $T_i$ within the range $T_i^{des} \leq T_i \leq T_i^{max}$ is acceptable for \pve (\eg $\tau_i \in \Gamma_S^{pa}$) and \ave (\eg $\tau_i \in \Gamma_S^{ac}$) mode  security tasks, the actual period $T_i$ is not known a priori. Furthermore, for \ave mode security tasks (\eg $\tau_i \in \Gamma_S^{ac}$), we need to find out the suitable priority level $l \in [l_S, m]$. Therefore our goal is to find the \textit{suitable period} (for both \pve and \ave mode security tasks) as well as the \textit{priority-level} (for \ave mode security tasks) that achieve the best trade-off between schedulability and defense against security breaches without violating the real-time constraints.

 \end{document}