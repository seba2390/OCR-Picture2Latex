 \documentclass[../rt_server_main.tex]{subfiles}

 \begin{document}

\section{Discussion} \label{sec:discussion}

Although \coolname provides an integrated approach to guarantee safety and security in RTS, this framework can be extended in several directions. In the following, we briefly analyze \coolname against different threat models and discuss the limitations of the current framework with possible directions of improvement.

\subsection{Threat Analysis}

The security mechanism will collapse if the adversary can compromise \textit{all} the security tasks. To do so, the adversary would need to intrude into the system, remain undetected and monitor the schedule \cite{cy_side_channel} (to override the security tasks)  \textit{over a long period of time}. Guaranteeing the integrity of the security tasks is an interesting research problem by itself and will be investigated in our future work. While compromising all the security tasks could be \textit{difficult} in practice, it nevertheless would be worthwhile to harden the security posture of \coolname further by \textit{randomizing task schedules} while guaranteeing the safety of the real-time tasks by using approaches similar to one recently proposed in the literature \cite{taskshuffler}. Randomizing the schedule of real-time and security tasks reduces the determinism (and thus the predictability of security tasks' execution) and further reduce the chance of information leakage.  Randomizing task schedules in RTS, unlike traditional systems, is not straightforward since it leads to priority
inversions~\cite{Sha:1990:PIP} that, in turn, cause missed deadlines, and
hence, put the safety of the system at risk.  We intend to incorporate randomization protocols on top of \coolname in future work.


The underlying detection algorithms in security tasks could raise false positive errors that may cause the system to switch modes unnecessarily. Again, a clever adversary may remain undetected and provide a fake indication of malicious activity. This may cause \coolname to frequently switch modes thus reducing performance and availability. Although \coolname \textit{guarantees that the system will remain schedulable} (and hence safe) even with mode changes (refer to Section \ref{subsec:mode_switch_diss}), running of security tasks in \ave mode could impose additional overheads (\ie increased load as we have seen in Fig.~\ref{fig:cpu_usage}) that designers of the system may want to avoid. %Therefore, this is obviously a situation designers of the system want to avoid. Besides, 

The false positive/negative errors can be mitigated by carefully designing the detection algorithms based on application requirements. Further, we argue that forced mode changes would require an adversary to intrude in the system and \textit{remain undetected for a long time}. In practice that could be \textit{difficult} and \textit{unlikely} in the  presence of several intrusion detection tasks. 


\subsection{Limitations and Improvement}



In \coolname each
security task has a desired frequency of execution for better security
coverage. Security tasks so far have been treated as \textit{independent} and
\textit{preemptive}, but in practice, some security monitoring
may need \textit{atomicity} or non-preemptive execution. The server-based model proposed in this work can be extended to incorporate this feature. For example, when a security task needs to perform a special atomic operation, the priority of the server can be increased to a priority that is strictly higher than all (or some) of the real-time tasks. Further, if the security task running under server is not the highest-priority security task, the priority of that task itself will also be increased. If the server's capacity is exhausted while it is executing an atomic operation, we can allow the server to \textit{overrun} \cite{server_overrun}, \ie the server continues to execute at the same priority until the security checking is completed. When the server overruns, the allocated capacity at the start of the next server replenishment period is reduced by the amount of the overrun. However, schedulability analysis must be performed considering maximum blocking times of the security tasks.

Further, security tasks may
have \textit{dependencies} wherein one task depends on the output from one or more other
tasks. For example, an anomaly detection task might depend on the outputs of
multiple scanning tasks, or, the scheduling framework might need to follow
certain \textit{precedence constraints} for security tasks. In
order to ensure the integrity of monitoring security, the security application's
own binary might need to be examined first before it checks the system binary
files. %To address this, we need to consider the problem of integrating security tasks with dependencies between them. In particular, we intend to use a directed acyclic graph (DAG) to capture the dependencies and constraints among security tasks and explore scheduling DAG-based task sets to integrate such tasks into RTS. 
In that case, the {\emph cumulative tightness} of the achievable
periodic monitoring proposed in Section~\ref{sec:period_adapt} might no
longer be a reasonable metric. Constraints to ensure that the dependent security tasks are
executed often enough should be included and the optimization problem may need to be
reformulated and evaluated with different metrics.

While \textit{time-to-detect} is a useful metric, it is hard to quantify in a
comprehensive way as it depends on a number of factors such as the efficacy
of monitoring tasks, the kind of intrusion \etc~and is a lagging metric.
Identifying and designing better security metrics is an important and
challenging problem. In future work we will undertake it in the narrow context of
integrating monitoring and detection tasks into RTS. 

 \end{document}