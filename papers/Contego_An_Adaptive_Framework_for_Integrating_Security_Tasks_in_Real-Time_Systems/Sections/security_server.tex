 \documentclass[../rt_server_main.tex]{subfiles}

 \begin{document}

\section{The Security Server} \label{sec:sec_server}


%\subfile{Sections/server_overview}

The server \cite{server_ab_uk} is an abstraction that provides execution time to the
security tasks according to a predefined scheduling algorithm.  Our proposed security server is characterized by the
\textit{capacity} $Q$ and \textit{replenishment period} $P$ and works as
follows. %The server may be in two states, \eg \textit{active} and \textit{inactive}. 
The server is executed with lowest-priority in \pve mode. However, in 
\ave mode, the server can switch to any allowable priority-level\footnote{Calculation of the server priority-level is described in Section \ref{sec:algo}.} within the range $[l_S, m]$. If any security
task is activated at time $t$, %and if the server is inactive, 
then the server
is activated with capacity $Q$ and the next
replenishment time is set as $t + P$.  %If the server is already active, then the current capacity and replenishment time remain unchanged. 
When the server is
 scheduled, it executes the security tasks according to its own scheduling
policy. In this work we assume that the server schedules the security tasks using fixed-priority RM scheduling. When a
security task executes, the current available capacity is decremented
accordingly. The server can be preempted by the scheduler to service 
real-time tasks. When the server is preempted, the currently available capacity
is not decremented. If the available capacity becomes zero and some
security task has not yet finished, then the server is suspended until its next
replenishment time ($t'$). At time $t'$, the
server is recharged to its full capacity $Q$, the next replenishment time is
set as $t' + P$, and the server is executed again. When the last security task
has finished executing and there is no other pending task in the server, the
server will be suspended. Also, the server will become inactive if there are no
security tasks ready to execute. 

%Similar to earlier work \cite{mhasan_rtss16}, we do not have any strict assumption on the smallest time unit of server parameters, \eg $Q, P \in \mathbb{R}^+$. Besides, we assume that there is no task or server release jitters (in any mode as well as during the mode changes) and the security task releases are not bound to the server \cite{server_ab_uk}.

% \todo[inline]{Rod: I don't quite understand this sentence. You never defined jitter. What does it mean releases are bound ot the server?}
% \hl{MH: removed!}


\subsection{Reformulation of the Period Adaptation Problem using Servers} \label{subsec:period_adapt_server}

When security tasks execute within the server, we need to modify the constraints in the period adaption problem considering the server parameters $Q$ and $P$. In the following we briefly discuss how to customize the period adaptation problem with the inclusion of the server.



%\subsubsection{Period Adaptation in passive Mode}


 Let us use $UB_{\mathcal{S}(Q, P), \Gamma}$ to denote the utilization bound for the set of tasks $\Gamma$ executing within the server.  
When the smallest period of the task is greater than or equal to $3P - 2Q$, it has been shown \cite{serverBound_raj} that the upper bound of the utilization factor for the security tasks is given by
%\begin{equation}
$
UB_{\mathcal{S}(Q, P), \Gamma} = n \left[ \left( \tfrac{3 - \tfrac{Q}{P}}{3 - 2 \tfrac{Q}{P}} \right)^{\frac{1}{n}} - 1 \right]$,
%\end{equation} 
where $n$ is number of tasks in the set $\Gamma$.

Thus with the inclusion of the server in \pve mode, we can modify the constraints in Eqs. (\ref{eq:period_ns_const_2_pve}) and (\ref{eq:period_ns_const_rm_pve}) as follows:
\begin{subequations}
\begin{align} %\label{eq:period_const_2}
\sum_{\tau_i \in \Gamma_S^{pa}}\frac{C_i}{T_i} &\leq  n_p \left[ \left( \tfrac{3 - \tfrac{Q^{pa}}{P^{pa}}}{3 - 2 \tfrac{Q^{pa}}{P^{pa}}} \right)^{\frac{1}{n_p}} - 1 \right] \label{eq:period_const_2_pve} \\ 
T_i &\geq 3P^{pa} - 2Q^{pa}, \quad \forall \tau_i \in \Gamma_S^{pa}. \label{eq:period_const_4_pve}
\end{align}
\end{subequations}
Therefore, selection of the periods for security tasks in \pve mode %with presence of the server 
is a nonlinear constrained optimization problem that can be formulated as follows: 

\begin{myoptimizationproblem} \label{opt:period_adapt_server_pve}
\vspace*{-2.0em}
\begin{subequations}
\begin{align}
\quad \quad \underset{\mathbf{T}^{\boldsymbol{pa} }}{\operatorname{max}} \sum_{\tau_i \in \Gamma_S^{pa}} \omega_i \frac{T_i^{des}}{T_i}, ~~ %\quad  \nonumber \\
\text{Subject to:~  (\ref{eq:period_const_2_pve}), (\ref{eq:period_const_4_pve}),  (\ref{eq:period_const_1_pve})} \nonumber.
\end{align}
\end{subequations}
\end{myoptimizationproblem}
\hspace{-1.9em}
where $Q^{pa}$ and $P^{pa}$ are the server capacity and replenishment period in \pve mode, respectively. The formulation of the \pve mode period adaptation problem presented above is similar to that we proposed in earlier work \cite{mhasan_rtss16}.

Similarly, in \ave mode, the period adaptation problem can be reformulated as follows:

\begin{myoptimizationproblem} \label{opt:period_adapt_server_ave}
\vspace*{-2.0em}
\begin{subequations}
\begin{align}
\underset{\mathbf{T}^{\boldsymbol{ac}}}{\operatorname{max}} & \sum_{\tau_i \in \Gamma_S^{ac}} \omega_i \frac{T_i^{des}}{T_i} ~~ \quad  \nonumber \\
\text{Subject to:~} %\quad \quad \nonumber\\
\sum_{\tau_i \in \Gamma_S^{ac}}\frac{C_i}{T_i} &\leq  n_a \left[ \left( \tfrac{3 - \tfrac{Q^{ac}}{P^{ac}}}{3 - 2 \tfrac{Q^{ac}}{P^{ac}}} \right)^{\frac{1}{n_a}} - 1 \right] \label{eq:period_const_2_ave} \\
T_i &\geq 3P^{ac} - 2Q^{ac} \quad \forall \tau_i \in \Gamma_S^{ac} \label{eq:period_const_4_ave} \\
T_i^{des} &\leq T_i \leq T_i^{max}  \quad ~~\forall \tau_i \in \Gamma_S^{ac} \label{eq:period_const_1_ave}
\end{align}
\end{subequations}
\end{myoptimizationproblem}
\hspace{-1.9em}
where $Q^{ac}$ and $P^{ac}$ are the server capacity and replenishment period in \ave mode, respectively.



\subsection{Selection of the Server Parameters} \label{sec:server_param_selec}

The period adaptation problem illustrated in Section \ref{subsec:period_adapt_server}
 %$\mathbf{P}\mathbf{\ref{opt:period_adapt_server_pve}}$ 
  is derived based on a given set of server parameters, \eg $(Q^{(\cdot)}, P^{(\cdot)})$. However, a fundamental problem is to find a suitable pair of server capacity $Q^{(\cdot)}$ and replenishment period $P^{(\cdot)}$ that respects the real-time constraints of the tasks in the system. Our approach to selecting the server parameters in \pve and \ave mode is described below.

\subsubsection{Parameter Selection in Passive Mode} \label{subsec:param_pve}

Recall that in \pve mode, the server will execute with the lowest priority to have compatibility with existing real-time tasks. Since the security tasks execute within the server, we need to ensure the following two constraints:
%\begin{enumerate}[\it i\normalfont )]
\begin{itemize}

\item \textit{The server is schedulable}: that is  the server's capacity and interference from higher priority real-time tasks are less than the replenishment period; and

\item \textit{The security tasks are schedulable}: the minimum \textit{supply} by the server to the security tasks  is greater than the worst-case workload generated by the security tasks.

\end{itemize}
%\end{enumerate}

Note that since the server is running with lowest priority, the real-time constraints %(\eg Eq.~(\ref{eq:wcrt})) 
(\eg
$w_j \leq D_j, \forall \tau_j \in \Gamma_R$)
and the task execution order are not affected in the \pve mode.  Based on the above two constraints, we illustrate an approach for determining the server parameters by formulating it as a \textit{constraint optimization problem}.

The security server is referred to as \textit{schedulable} if the worst-case response time of the server does not exceed its replenishment period \cite{server_ab_uk}. Thus, following an approach similar to ones in  earlier work \cite{mhasan_rtss16, mn_gp},  the \textit{server schedulability constraint} can be represented as follows: 
 \begin{equation} \label{eq:ser_con1_pve}
 Q^{pa} + \Delta_{\mathcal{S}^{pa}} \leq P^{pa}
 \end{equation}
 where $\Delta_{\mathcal{S}^{pa}} = \sum\limits_{\tau_h \in hp_R(\tau_{\mathcal{S}^{pa}})} \left( \frac{P^{pa}}{T_h} + 1 \right)  C_h$ is the worst-case interference experienced by the server when preempted by the higher priority real-time tasks.
In the above equation, the set of real-time tasks with higher priority than the server (\ie $hp_R(\tau_\mathcal{S}^{pa}) = \Gamma_R$) is fixed.

Let us use $hp_S^{pa}(\tau_i)$ to denote the set of \pve mode security tasks that are higher priority than $\tau_i \in \Gamma_S^{pa}$. To ensure schedulability of the security tasks, we can derive the \textit{minimum supply} of the server delivered to the security tasks by using the periodic resource model from the literature  \cite{periodic_server_qp, mn_gp, mhasan_rtss16}. In particular, the constraints on the server supply to ensure \textit{schedulability of the security tasks} \cite{mhasan_rtss16} can be expressed as:
\begin{equation} \label{eq:ser_con2_pve}
 \frac{Q^{pa}}{P^{pa}} \left[ T_i - (P^{pa} - Q^{pa}) - \Delta_{\mathcal{S}^{pa}}  \right] \geq I_i^{pa}, ~~\forall \tau_i \in \Gamma_S^{pa} 
\end{equation}
where $I_i^{pa} = C_i + \sum\limits_{\tau_h \in hp_S^{pa}(\tau_i)} \left\lceil \frac{T_i}{T_h} \right\rceil C_h$ is the worst-case workload generated by the security task $\tau_i$ and $hp_S^{pa}(\tau_i)$ during the time interval of $T_i$. This workload is a constant for a given input. 

Since we need to ensure maximal processor utilization for the security tasks without violating the real-time constraints of the system, we define the following objective function:
%\begin{equation} \label{eq:ser_obj_pve}
$\underset{Q^{pa}, P^{pa}}{\operatorname{max}}~ \frac{Q^{pa}}{P^{pa}}$. 
%\end{equation}
With 
%the objective function in Eq.~(\ref{eq:ser_obj_pve})
this objective function
and the constraints in Eqs.~(\ref{eq:ser_con1_pve})-(\ref{eq:ser_con2_pve}), the \pve mode server parameter selection problem can be formulated as follows:


\begin{myoptimizationproblem} \label{opt:server_param_pve}
\vspace*{-2.0em}
\begin{subequations}
\begin{align}
\hspace*{5em}
\underset{Q^{pa}, P^{pa}}{\operatorname{max}}~ \frac{Q^{pa}}{P^{pa}}, \quad  %\\
\text{Subject to:~  (\ref{eq:ser_con1_pve}), (\ref{eq:ser_con2_pve})} \nonumber
\end{align}
\end{subequations}
\end{myoptimizationproblem}
\hspace{-1.9em} 
where server parameters $Q^{pa}$ and $P^{pa}$ are the optimization variables.


\subsubsection{Parameter Selection in Active Mode} \label{subsec:param_ave}

% \todo[inline]{Rod: I think somewhere in this section you have to say clearly that you are going to iterate over $l_S$. It is only clear in Sec 6, which I think is too late - it makes things confusing here}

% \hl{MH: thanks, mentioned in the following paragraph after the first bullet.}

In \ave mode, the security server is \textit{no longer the lowest priority task}. Since the server can execute with priority $l_S$, there could be up to $m - l_S$ low priority real-time tasks than that of the server. Thus we need to ensure the schedulability of the real-time tasks that are executing with a priority lower than the server. Hence, in addition to the constraints described in Section \ref{subsec:param_pve} (\ie Eqs.~(\ref{eq:ser_con1_pve})-(\ref{eq:ser_con2_pve})), we need to consider the following:

\begin{itemize}
\item \textit{The real-time tasks with lower priority than the server are schedulable}: that is, the interferences from the server and other higher priority real-time tasks do not violate the deadlines for these low-priority tasks.
\end{itemize}

\begin{comment}


 Recall that \coolname operates in \ave mode only in unusual situations \viz if any anomalous behavior is suspected. Also the \ave mode operation is only for \textit{limited time} to ensure better security premises (and hence overall safety) of the system. In practice, many control systems are poorly sensitive to the response time delay and can operate without any degradation on  performance  even if the jobs are completed \textit{after} the activation of the next one \cite{delay_period}. Therefore to ensure better security guarantee and responsiveness against malicious suspect, we propose to introduce a \textit{tolerance} on deadline\footnote{We allow this tolerance \textit{only} in \ave mode (possibly in an unusual/threatened situation) with a view to achieving better security guarantee that may be acceptable for many legacy RTS.} for the low-priority non-critical real-time tasks. By allowing this tolerance for the lower priority real-time tasks, we can achieve better tightness on \ave mode monitoring as we have illustrated in Section \ref{sec:evaluation}.
 
 Let us now rewrite the deadline for low-priority real-time tasks as follows:
 \begin{equation} \label{eq:dead_tol}
 D^{\prime}_j =  D_j + \epsilon_j \leq T_j, ~~\forall \tau_j \in lp_R(\tau_\mathcal{S}^{ac})
 \end{equation}
where $lp_R(\tau_\mathcal{S}^{ac})$ is the set of real-time tasks that are with lower priority than the server. This set of tasks is fixed for a given priority-level. The parameter $\epsilon_j \geq 0$ in Eq.~(\ref{eq:dead_tol}) is the deadline tolerance  for real-time tasks $\tau_j \in lp_R(\tau_\mathcal{S}^{ac})$ where $\epsilon_j = 0$ if the real-time tasks is critical to deadline and non-zero otherwise. This tolerance parameter will allow system designers to trade-off between control system performance and security requirements. For instance, for any task $\tau_j \in \Gamma_R$ the tolerance $\epsilon_j = 0$ implies that tasks are very sensitive to deadline misses and must be completed by the deadline $D_j$. 

\end{comment}


%Based on above discussion, 

We therefore define the following constraints to ensure the \textit{schedulability of the low-priority real-time tasks}:
% \begin{align}\label{eq:ser_con3_ave}
% C_j + \hspace*{-1em} \sum_{\tau_h \in hp_R(\tau_j)} \left\lceil \frac{D^{\prime}_j}{T_h} \right\rceil C_h + \left( \frac{D^{\prime}_j}{P} + 1 \right) Q \leq D^{\prime}_j,~~  \forall \tau_j \in lp_R(\tau_\mathcal{S}^{ac}) %\hspace{5em}
% \end{align}
\begin{equation}\label{eq:ser_con3_ave}
\begin{aligned}
C_j + \hspace*{-1em} \sum_{\tau_h \in hp_R(\tau_j)} \left\lceil \frac{D_j}{T_h} \right\rceil C_h + \left( \frac{D_j}{P^{ac}} + 1 \right) Q^{ac} \leq D_j,~~ \\  
\forall \tau_j \in lp_R(\tau_\mathcal{S}^{ac}) 
\end{aligned}
\end{equation}
where $\sum\limits_{\tau_h \in hp_R(\tau_j)} \left\lceil \frac{D_j}{T_h} \right\rceil C_h$ is the interference experienced by $\tau_j$ from other real-time tasks and $\left( \frac{D_j}{P^{ac}} + 1 \right) Q^{ac}$ is the worst-case interference caused to $\tau_j$ by the server in \ave mode. As illustrated in Section \ref{sec:algo}, we iterate through the allowable priority ranges (\eg $[l_S, m]$) to find the server priority in \ave mode. Note that for a given priority-level, the set of tasks $lp(\tau_\mathcal{S}^{ac})$ is predefined. Thus the only variables for the constraints in Eq.~(\ref{eq:ser_con3_ave}) are the server capacity $Q^{ac}$ and replenishment period $P^{ac}$.


Let us use $hp_S^{ac}(\tau_i)$ to denote the set of \ave mode security tasks that are higher priority than $\tau_i \in \Gamma_S^{ac}$. Just as in $\mathbf{P}\mathbf{\ref{opt:server_param_pve}}$ we can now formulate the \ave mode parameter selection problem as follows:

\begin{myoptimizationproblem} \label{opt:server_param_ave}
\vspace*{-2.0em}
\begin{subequations}
\begin{align}
\underset{Q^{ac}, P^{ac}}{\operatorname{max}}~ \frac{Q^{ac}}{P^{ac}}, ~~& % \quad \\
\text{Subject to:   (\ref{eq:ser_con3_ave})~and} %\hspace*{1em} 
\nonumber \\
Q^{ac} + \hspace*{-1em} \sum\limits_{\tau_h \in hp_R(\tau_{\mathcal{S}^{ac}})} \left( \frac{P^{ac}}{T_h} + 1 \right)  C_h &\leq P^{ac} \label{eq:ser_con1_ave}
  \\
   \frac{Q^{ac}}{P^{ac}} \left[ T_i - (P^{ac} - Q^{ac}) - \Delta_{\mathcal{S}^{ac}}  \right] &\geq I_i^{ac} ~~ \forall \tau_i \in \Gamma_S^{ac} \label{eq:ser_con2_ave}
\end{align}
\end{subequations}
\end{myoptimizationproblem}
\hspace{-1.9em} 
%where server parameters, $Q^{ac}$ and $P^{ac}$ are the optimization variables.
where the set of real-time tasks with higher priority than the server (\ie $hp_R(\tau_\mathcal{S}^{ac}) \subset \Gamma_R$) is a constant for a given priority-level and $I_i^{ac} = C_i + \sum\limits_{\tau_h \in hp_S^{ac}(\tau_i)} \left\lceil \frac{T_i}{T_h} \right\rceil C_h$ is the worst-case workload generated by the security task $\tau_i$ and $hp_S^{ac}(\tau_i)$. Note that the schedulability of the higher priority real-time tasks (\eg $\forall \tau_j \in hp_R(\tau_\mathcal{S}^{ac})$) is already ensured by definition. %(\eg Eq. (\ref{eq:wcrt})). 

\begin{remark}
The formulation of the period adaptation and server parameter selection problems are nonlinear constraint optimization problems and are nontrivial to solve in their current form. However, these problems can be transformed  into a geometric programming (GP) \cite{GP_tutorial} problem. %using an approach similar to that of presented in earlier work \cite{mhasan_rtss16}. 
In addition, it is also possible to reformulate the non-convex GP representation into equivalent convex form that can be solved using known algorithms such as \textit{interior point} \cite[Ch. 11]{boyd_book} method. For details of this reformulation, we refer the readers to Appendix. %\ref{appsec:gp}.
\end{remark}

 \end{document}