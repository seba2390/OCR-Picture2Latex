% \documentclass[../rt_server_main.tex]{subfiles}

% \begin{document}


The \ave and \pve modes parameter selection problems given in Section \ref{sec:sec_server} are constrained nonlinear optimization problems and 
not very straightforward to solve. Therefore we reformulate the optimization problems as a geometric program (GP) \cite{GP_tutorial}. A nonlinear optimization problem can be solved by GP if the problem is formulated as follows: \cite{GP_tutorial}
\begin{subequations}
\begin{align*}
\underset{\mathbf{X}}{\operatorname{min}}~ & f_0(\mathbf{x}),  \\
\text{~Subject to:~} \hspace*{2em} \nonumber \\
 f_i(\mathbf{x}) &\leq  1 \quad i = 1, \cdots, z_p,  \text{~~and~~} \\
 g_i(\mathbf{x}) &=  1 \quad i = 1, \cdots, z_m
\end{align*}
\end{subequations}
where $\mathbf{x} = [x_1, x_2, \cdots, x_z]^{\mathsf{T}}$ denotes the vector of $z$ optimization variables. The functions $f_0(\mathbf{x}), f_1(\mathbf{x}), \cdots, f_{z_p}(\mathbf{x})$ are \textit{posynomial} and $g_1(\mathbf{x}), \cdots, g_{z_m}(\mathbf{x})$ are \textit{monomial} functions, respectively. A monomial function is  expressed as
%\begin{equation}
$
g_i(\mathbf{x}) = c_i \prod\limits_{l = 1}^{L_i} x_l^{a_l},
$
%\end{equation}
where $c_i \in \mathbb{R}^+$ and $a_l \in \mathbb{R}$. %Note that the coefficient $c_i$ must be non-negative but the exponents $a_l$ can be any real number including fractional and negative. 
A posynomial function (\ie the sum of the monomials) can be represented as
%\begin{equation}
$
f_i(\mathbf{x}) = \sum\limits_{l=1}^{L_i} c_l x_1^{a_{1l}} x_2^{a_{2l}} \cdots x_z^{a_{1l}},
$
%\end{equation}
where $c_l \in \mathbb{R}^+$ and $a_{jl} \in \mathbb{R}$. %The posynomials are closed under addition, multiplication and non-negative scaling where the monomials are closed under multiplication and division.
We can maximize a non-zero posynomial objective function by minimizing its inverse. In addition, we can express the constraint $f(\cdot) < g(\cdot)$ as $\frac{f(\cdot)}{g(\cdot)} \leq 1$.

Based on the above description, we can rewrite the period adaptation problem in either mode as:



\begin{myoptimizationproblem} \label{opt:period_adapt_gp}
\vspace*{-2.0em}
\begin{subequations}
\begin{align*}
\underset{\mathbf{T}^{\boldsymbol{(\cdot)} }}{\operatorname{min}} \sum_{\tau_i \in \Gamma_S^{(\cdot)}} {\omega_i}^{-1}  {(T_i^{des})}^{-1}  T_i \hspace*{-1em} \\
\text{Subject to:~~~} \hspace*{10em} 
\nonumber \\
%\hspace*{-1em} \text{s. t.:~} 
%\hspace*{-1em}
\Big( \sum_{\tau_i \in \Gamma_S^{(\cdot)}} C_i {T_i}^{-1} \Big) \cdot   \left(n \left[ \left( \tfrac{3 - \tfrac{Q}{P}}{3 - 2 \tfrac{Q}{P}} \right)^{\frac{1}{n}} \hspace*{-1em}- 1\right]\right)^{-1} \hspace*{-1em} &\leq 1\\
(3P^{(\cdot)} - 2Q^{(\cdot)}) {T_i} ^{-1} &\leq 1,  ~\forall \tau_i \in \Gamma_S^{(\cdot)} \hspace*{2em} \\
T_i^{des} {T_i}^{-1} &\leq 1,  ~\forall \tau_i \in \Gamma_S^{(\cdot)}  \hspace*{2em} \\
 {(T_i^{max})}^{-1} T_i &\leq 1,  ~ \forall \tau_i \in \Gamma_S^{(\cdot)} 
\end{align*}
\end{subequations}
\end{myoptimizationproblem}
\hspace{-1.4em} 
where for any symbol $y^{(\cdot)}$ represents the corresponding variable in the representative mode (\eg \pve or \ave). 



The above GP formulation $\mathbf{P}\mathbf{\ref{opt:period_adapt_gp}}$ is not a convex optimization problem since the posynomials are not convex functions \cite{GP_tutorial}. However, by using logarithmic transformations (\eg representing $\tilde{T}_i = \log T_i$ and hence $T_i = e^{\tilde{T}_i}$, and replacing inequality constraints of the form $f_i(\cdot) \leq 1$ with $\log f_i(\cdot) \leq 0$), we can convert the above formulation into a convex optimization problem. This convex optimization reformulation can be solved using standard algorithms, such as \textit{interior-point} method in polynomial time \cite[Ch. 11]{boyd_book}.

The server parameter selection problem can also cast into GP as follows. The objective function (\eg the ratio between server capacity and period) can be represented as $Q^{(\cdot)} \left(P^{(\cdot)}\right)^{-1}$, which is clearly a posynomial. The server schedulability constraints (\eg Eq.~(\ref{eq:ser_con1_pve}) and Eq.~(\ref{eq:ser_con1_ave})) can be rewritten as \begin{equation} \label{eq:posy_server1}
\left(Q^{(\cdot)} + \Delta_{S^{(\cdot)}} \right) \left(P^{(\cdot)} \right)^{-1} \leq 1 
\end{equation}
where $\Delta_{S^{(\cdot)}} =
  \sum\limits_{\tau_h \in hp_R(\tau_S^{(\cdot)})} (P^{(\cdot)} + T_h) \cdot T_h^{-1} \cdot C_h $. 
  
  Likewise, the real-time task schedulability constraints for the \ave mode (\eg Eq.~(\ref{eq:ser_con3_ave})) can be represented as
  \begin{equation}
  \begin{aligned}
\Big(  C_j + \hspace*{-1em} \sum_{\tau_h \in hp_R(\tau_j)} \left\lceil \frac{D_j}{T_h} \right\rceil C_h \Big)D_j^{-1} ~~+ \\ 
\left( D_j \left(P^{ac}\right)^{-1} Q^{ac} + Q^{ac} \right) D_j^{-1} \leq 1, ~~\forall \tau_j \in lp_R(\tau_\mathcal{S}^{ac}).
  \end{aligned}
  \end{equation}
 In addition, by using the geometric mean approximation \cite[Ch. 2]{gp_comm} of posynomials we can rewrite the schedulability constraints for the security tasks (\eg Eqs.~(\ref{eq:ser_con2_pve}) and (\ref{eq:ser_con2_ave})) as follows \cite{mhasan_rtss16}:
 \begin{equation}
 \begin{aligned}
\left[P^{(\cdot)}  (Q^{(\cdot)} + I_i^{(\cdot)}) + \Delta_{S^{(\cdot)}}  Q^{(\cdot)} \right] \cdot {\left[Q^{(\cdot)} \cdot  \hat{g}(Q^{(\cdot)}, T_i) \right]}^{-1} \leq 1, \\ 
\forall \tau_i \in \Gamma_S^{(\cdot)} \hspace*{1em}
 \end{aligned}
\end{equation} 
where $\hat{g}(Q^{(\cdot)}, T_i) = {\left( \frac{Q^{(\cdot)}}{\frac{y_0}{y_0 + T_i}} \right)}^{\frac{y_0}{y_0 + T_i}} \cdot {\left( \frac{T_i}{\frac{T_i}{y_0 + T_i}} \right)}^{\frac{T_i}{y_0 + T_i}}$ and $y_0 \in \mathbb{R}^+$ is a given initial point.
  After the logarithmic transformation (\eg $\tilde{Q}_i^{(\cdot)} = \log Q_i^{(\cdot)}$, $\tilde{P}_i^{(\cdot)} = \log P_i^{(\cdot)}$ and replacing the inequality constraints $f_i(\cdot) \leq 1$ with $\log f_i(\cdot) \leq 0$), the objective function and the constraints become a standard convex optimization problem that is solvable in polynomial time.
  

  
% \end{document}