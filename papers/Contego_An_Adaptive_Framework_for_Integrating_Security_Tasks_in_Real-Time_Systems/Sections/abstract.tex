 \documentclass[../rt_server_main.tex]{subfiles}

 \begin{document}

\begin{abstract}

% \todo[inline]{Adaptive Real-Time Security (ARTS) \\
% ARTiSt \\
% ARTSy \\
% TimeGuard \\
% Contego \\
% LightStrike \\
% TimeWatch}


Embedded real-time systems (RTS) are pervasive. Many modern RTS are exposed to unknown security flaws, and threats to RTS are growing in both number and sophistication. However, until recently, cyber-security considerations were an afterthought in the design of such systems.  
Any security mechanisms integrated into RTS must \ca \textit{co-exist} with the real-time tasks in the system and \cb operate \textit{without} impacting the timing and safety constraints of the control logic. We introduce \coolname, an approach to integrating security tasks into RTS without affecting temporal requirements. \coolname is specifically designed for \textit{legacy} systems, \viz the real-time control systems in which major alterations of the system parameters for constituent tasks is not always feasible.
\coolname combines the concept of \textit{opportunistic execution} with hierarchical scheduling to maintain compatibility with legacy systems while still providing flexibility by allowing security tasks to operate in different \textit{modes}. We also define a metric to measure the effectiveness of such integration. %and propose a trade-off mechanism between security and real-time requirements. 
We evaluate \coolname using synthetic workloads as well as with an implementation on a realistic embedded platform (an open-source ARM CPU running real-time Linux).


\end{abstract}

%\todo[inline]{RBB: For abstract registration, title and author order have to be finalized. The abstract itself can be modified for full submission. Do we like the name Shield?}

 \end{document}
