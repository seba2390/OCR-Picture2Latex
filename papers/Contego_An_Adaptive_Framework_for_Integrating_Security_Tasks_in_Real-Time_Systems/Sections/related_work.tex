 \documentclass[../rt_server_main.tex]{subfiles}

 \begin{document}

\section{Related Work}

In our earlier work~\cite{mhasan_rtss16} we proposed to use a server to 
integrate security tasks and execute them opportunistically at a lower priority than real-time tasks. That
approach was useful for legacy RTS
where perturbing 
the schedule of real-time tasks was not an option -- however, the downside was longer time for detection. In contrast, \coolname can respond to anomalous activities in an adaptive manner and provide improved monitoring frequency and detection time when needed. 

A new scheduler \cite{xie2007improving} and enhancements to an existing dynamic priority scheduler \cite{lin2009static} were proposed to meet real-time requirements while maximizing the level of security achieved.
 %Different from \cite{xie2007improving, lin2009static}  we consider a fixed-priority scheduling mechanism that can impose security without any significant modification to the real-time requirements.
%a fixed-priority scheduling mechanism %where security policies are executed sporadically using a server while meeting real-time requirements. 
A state cleanup mechanism has been introduced \cite{sg1}, and further generalized \cite{sg2, sibin_RT_security_journal} such that the fixed-priority
scheduling algorithm was modified to mitigate information leakage through shared resources. Researchers  have proposed a schedule obfuscation method \cite{taskshuffler} aimed at randomizing the task schedule while providing the necessary real-time guarantees. Such randomization techniques can improve the security posture by minimizing the predictability of the deterministic RTS scheduler. %This is different  from \coolname since it works at the scheduler-level. 
Recent work \cite{slack_cornell, securecore} on dual-core based hardware/software architectural frameworks has aimed to protect RTS against security threats. However, those approaches came at the cost of reduced schedulability or may require architectural/scheduler-level modifications. In comparison, \coolname aims to integrate security \textit{without} any significant modification of the system properties and does \textit{not} violate the temporal constraints or schedulability of the real-time tasks.



%A work closely related to ours is literature \cite{mhasan_rtss16} where the authors propose to 
%In our earlier work \cite{mhasan_rtss16}, we proposed to execute security tasks with the lowest priority relative to real-time tasks (that is analogous to the \pve mode we presented here). Although this approach may be useful for existing systems since the schedulability of the real-time tasks remain unaffected, this leads to longer response time for the security tasks and thus may increase the detection time of an attack. In contrast, \coolname can adaptability response against anomalous activities by switching the mode of operation.
%\hl{TODO}

%\cite{marques2013error}

Although not in the context of security in RTS, 
there exists other work \cite{delay_period} in which
the authors statically assign the  periods for multiple independent control tasks by considering control delay as a cost metric and estimating the delay through an approximate response time analysis. In contrast, our goal is to ensure security without violating the timing constraints of the real-time tasks. Hence, instead of minimizing response time, we attempt to assign the best possible periods and priority-levels so that we can minimize the perturbation between the  achievable period and desired period for all the security tasks. 

An on-demand fault detection and recovery mechanism has been proposed~\cite{ortega_conf} in which the system can operate in different modes. Specifically, when a fault is detected, a high-assurance
controller is activated to replace the faulty high-performance
controller. While fault-tolerance may also be a design consideration, \coolname  focuses primarily on integrating mechanisms that can foil cyber-attacks. There also exist work in the context of mixed-criticality systems (MCS) where  application tasks of different criticality requirements (\eg deadline and execution time)
share same computation and/or communication resources (refer to literature \cite{mc_review} for a survey of MCS). MCS is different than the problem considered in this work due to the fact that security properties (\ie adaptive switching depending on runtime behavior or frequent execution of monitoring events for faster detection) are often different than temporal requirements (\eg satisfying deadline constraints for mixed-criticality tasks). However, the theory and concepts emerged from MCS can also be applied to the real-time security problems to further harden the security posture of future RTS.

%\note[RBB]{The following paragraph needs to be rephrased. In general the related work section may need some work}


%With a view to hardening security mechanisms by minimizing predictability of deterministic RTS scheduler, 




%It is not inconceivable that the architectural frameworks \cite{slack_cornell, securecore} as well as the randomization protocols \cite{taskshuffler} can be employed on top of \coolname to harden security mechanisms in future RTS.


%This is different  from \coolname since it works at the scheduler-level.




% \begin{table*}
% \centering
% \caption{Summary of Related Work and Comparison with \bf{\sl \coolname}}
% \begin{tabular}{p{1.5cm}|p{2.1cm}|p{1.5cm}|p{2.1cm}|p{4.0cm}}  
% \hline 
% Ref. & Problem Focus & Security Aware & Solution Approach & Remarks \\
% \hline
% \end{tabular}
% \end{table*}



 \end{document}