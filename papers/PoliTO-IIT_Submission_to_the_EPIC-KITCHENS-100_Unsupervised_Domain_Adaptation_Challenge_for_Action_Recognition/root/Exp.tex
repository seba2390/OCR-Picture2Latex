\section{Results and Discussion}
%\begin{table}[t]
%\centering
%\begin{adjustbox}{width=0.75\columnwidth, margin=0ex 1ex 0ex 0ex}
%\begin{tabular}{l|ccc}
%\toprule\noalign{\smallskip}
%\multicolumn{4}{c}{\normalsize\textsc{Domain Generalization}} \\
%\noalign{\smallskip}
%\cline{1-4}
%\noalign{\smallskip}
%  & \multicolumn{1}{c}{Target} & Verb Top-1 & Noun Top-1 \\ 
% \noalign{\smallskip} \hline
%Source Only & \xmark               & 44.39 & 69.69 \\ \hline
%EPIC\_TA3N        & \cmark               & 46.91 & 72.70 \\ \hline
%RNA         & \xmark              & \underline{47.96} & \underline{79.54} \\ \hline
%EPIC\_TA3N+RNA    & \cmark               & \textbf{50.40} & \textbf{80.47} \\ 
%\bottomrule
%\end{tabular}
%\end{adjustbox}
%\caption{}
%\label{dg}
%\end{table}



% Please add the following required packages to your document preamble:
% \usepackage[table,xcdraw]{xcolor}
% If you use beamer only pass "xcolor=table" option, i.e. \documentclass[xcolor=table]{beamer}



In Table \ref{leaderboard} we report our best performing model on the target test, achieving the \textbf{1st} position on ‘verb', \textbf{3rd} on ‘noun' and ‘action', and \textbf{1st} position on Top-5 accuracy on all categories. In Table \ref{multi-stream} (left) we show an ablation on the contribution of the proposed ensemble UDA losses, T-HNA and MEC respectively, on the official validation set. As it can be seen, they improve Top-1 accuracy on all categories by up to $2\%$, proving the effectiveness of imposing a consistency between features from different streams. 

\textit{How well do DG approaches perform? } We show in Table \ref{multi-stream} (right) the results obtained under the multi-source DG setting, when target data are not available during training. Noticeably, RNA outperforms the baseline Source Only by up to $3\%$ on Top-1 and $10\%$ on Top-5, remarking the importance of using ad-hoc alignment techniques to deal with multiple sources in order to effectively extract a domain-agnostic model. Moreover, it outperforms the very recent UDA technique TA$^3$N without accessing to target data. 
Interestingly, when combined with EPIC\_TA3N, it further improves performance, proving the complementarity of RNA to other existing UDA approaches. 
