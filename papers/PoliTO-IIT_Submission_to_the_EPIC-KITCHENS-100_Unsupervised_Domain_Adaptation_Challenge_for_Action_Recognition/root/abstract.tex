%%%%%%%%% ABSTRACT
\begin{abstract}
In this report, we describe the technical details of our submission to the EPIC-Kitchens-100 Unsupervised Domain Adaptation (UDA) Challenge in Action Recognition. 
To tackle the domain-shift which exists under the UDA setting, we first exploited a 
recent Domain Generalization (DG) technique, called
Relative Norm Alignment (RNA). It consists in designing a model able to generalize well to any unseen domain, regardless of the possibility to access target data at training time.
Then, in a second phase, we extended the approach to work on unlabelled target data, allowing the model to adapt to the target distribution in an unsupervised fashion. For this purpose, we included in our framework existing UDA algorithms, such as Temporal Attentive Adversarial Adaptation Network (TA$^3$N), jointly with new multi-stream consistency losses, namely Temporal Hard Norm Alignment (T-HNA) and Min-Entropy Consistency (MEC).
Our submission (entry ‘plnet') is visible on the leaderboard and it achieved the 1st  position for \textit{‘verb’}, and the 3rd position for both \textit{‘noun’} and \textit{‘action’}.

\end{abstract}