\chapter{Conclusion}
\label{chap:conclusion}



This thesis introduces a methodology that allows the systematic derivation of algorithms for iterative methods; the starting point for this methodology is a formal description of an iterative method in matrix form. In addition, we presented an approach for deriving properties of matrices and matrix expressions from this representation; those properties are necessary for the derivation of algorithms.

The actual derivation of algorithms consists of four major steps. First, PMEs are generated by partitioning the operands of the matrix representation and applying the derived properties to solve equations. Then, from those PMEs, loop invariants are obtained. In the third step, from each loop invariant, one loop-based algorithm is constructed. Finally, common subexpressions are eliminated, generating an even larger number of algorithms.

One of the most important aspects, and indispensable for the automatic generation of libraries, is that the derived algorithms are provably correct. This is ensured by constructing them around a proof of correctness, based on the loop invariants generated in the second step.

A conscious effort was made to ensure that the entire process is systematic, that is, each step is performed according to well defined rules and no guidance by a human expert is required. This allows the approach to be implemented as a tool that automatically generates algorithms based on a formal description of the operation. We consider this to be another important step towards the automatic generation of linear algebra libraries as envisioned by the founders of the FLAME project.

As for future work, there are a number of ways to build on the results of this thesis:

\begin{description}
\item[Implementation] Executing the presented approach by hand is a laborious and thus error-prone task, not least because it was not designed to be executed by hand. To be used productively, the presented approach should be implemented as a computer program.

\item[Stability Analysis] To asses the usefulness of the derived algorithms in practice, a stability analysis is indispensable. In \cite{Bientinesi:thesis}, it was shown that the FLAME methodology can be combined with a systematic stability analysis. The presented method should be extended in a similar way.

\item[Performance Analysis] While it is desirable to derive a large number of algorithms to find new, potentially faster variants, the task of identifying them should not be left to the user. Thus, similar to a systematic stability analysis, the system should also be able to reason about the performance of the generated algorithms and select the best ones.

\item[Matrix Representations] In this thesis, only a small number of matrix representations for iterative methods is presented. Clearly, it would be desirable to find representations of many more methods. Additionally, it might be interesting to find out if this representation reveals new insights about different iterative methods and their relations to each other.
\end{description}
