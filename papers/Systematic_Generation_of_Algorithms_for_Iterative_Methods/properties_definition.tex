\chapter{Matrix Properties}
\label{chap:appendixProperties}

In the following, we define those matrix properties used throughout the thesis that are not self-explanatory. Let $A \in \mathbb{R}^{n \times m}$ be a matrix. The elements of this matrix are denoted as $a_{ij}$ with $i \in \{0, \ldots, n-1\}$ and $j \in \{0, \ldots, m-1\}$.

%%\begin{array}{ccccc}
%%	• & • & • & • & • \\ 
%%	• & • & • & • & • \\ 
%%	• & • & • & • & • \\ 
%%	• & • & • & • & • \\ 
%%	• & • & • & • & • \\ 
%%\end{array}
%
%$
%\left(
%\begin{array}{cccc}
%a_{00} & a_{01} & a_{02} & a_{03} \\ 
%a_{10} & a_{11} & a_{12} & a_{13} \\ 
%a_{20} & a_{21} & a_{22} & a_{23} \\ 
%a_{30} & a_{31} & a_{32} & a_{33} \\ 
%\end{array}
%\right)
%$
%
%$
%\left(
%\begin{array}{cccc}
%a_{00} & a_{01} & a_{02} & \hdots \\ 
%a_{10} & a_{11} & a_{12} &  \\ 
%a_{20} & a_{21} & a_{22} &  \\ 
%\vdots &  &  & \ddots \\ 
%\end{array}
%\right)
%$
%
%$
%\left(
%\begin{array}{cccc}
%0 & 0 & 0 & \hdots \\ 
%a_{10} & 0 & 0 &  \\ 
%0 & a_{21} & 0 &  \\ 
%\vdots &  & a_{32} & \ddots \\ 
%\end{array}
%\right)
%$
%
%$
%\left(
%\begin{array}{cccc}
%0 & a_{01} & 0 & \hdots \\ 
%0 & 0 & a_{12} &  \\ 
%0 & 0 & 0 & a_{23} \\ 
%\vdots &  &  & \ddots \\ 
%\end{array}
%\right)
%$


\begin{itemize}
\item[-] Upper diagonal ($\text{\ttfamily UpperDiagonal}$): $a_{ij} = 0$ for $i + 1 \neq j$ with $n = m$. Consider the matrix below as an example.
%
$$
\left(
\begin{array}{cccc}
0 & a_{01} & 0 & 0 \\ 
0 & 0 & a_{12} & 0 \\ 
0 & 0 & 0 & a_{23} \\ 
0 & 0 & 0 & 0 \\ 
\end{array}
\right)
$$
%
\item[-] Lower diagonal ($\text{\ttfamily LowerDiagonal}$): $a_{ij} = 0$ for $i - 1 \neq j$ with $n = m$.
%
\item[-] Diagonal and rectangular ($\text{\ttfamily DiagonalR}$): $a_{ij} = 0$ for $i \neq j$ with $n \neq m$.
%
\item[-] Upper diagonal and rectangular ($\text{\ttfamily UpperDiagonalR}$): $a_{ij} = 0$ for $i + 1 \neq j$ with $n \neq m$.
%
\item[-] Lower diagonal and rectangular ($\text{\ttfamily LowerDiagonalR}$): $a_{ij} = 0$ for $i - 1 \neq j$ with $n \neq m$.
%
\item[-] Upper trapezoidal ($\text{\ttfamily UpperTrapezoidal}$): $a_{ij} = 0$ for $i > j$ with $n < m$. Thus, the following matrix is upper trapezoidal:
%
$$
\left(
\begin{array}{cccc}
a_{00} & a_{01} & a_{02} & a_{03} \\ 
0 & a_{11} & a_{12} & a_{13} \\ 
0 & 0 & a_{22} & a_{23} \\ 
\end{array}
\right)
$$
%
\item[-] Lower trapezoidal ($\text{\ttfamily LowerTrapezoidal}$): $a_{ij} = 0$ for $i < j$ with $n > m$.
%
\item[-] Upper triangular and rectangular ($\text{\ttfamily UpperTriangularR}$): $a_{ij} = 0$ for $i > j$ with $n > m$. An example of such a matrix is shown below.
%
$$
\left(
\begin{array}{ccc}
a_{00} & a_{01} & a_{02} \\ 
0 & a_{11} & a_{12} \\ 
0 & 0 & a_{22} \\ 
0 & 0 & 0 \\ 
\end{array}
\right)
$$
%
\item[-] Lower triangular and rectangular ($\text{\ttfamily LowerTriangularR}$): $a_{ij} = 0$ for $i < j$ with $n < m$.
%
\item[-] Elements on the diagonal are zero ($\text{\ttfamily ZeroDiagonal}$): $a_{ij} = 0$ for $i = j$ with $n = m$.
\end{itemize}

%\begin{itemize}
%\item[-] Upper diagonal ($\text{\ttfamily UpperDiagonal}$)
%$ \equiv
%\left\{ 
%\begin{array}{l l}
%a_{ij}		& \quad \text{if $i - 1 = j$}\\
%0		& \quad \text{otherwise}
%\end{array} \right.
%$
%\end{itemize}