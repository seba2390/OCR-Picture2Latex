\usepackage{tabularx}

\newcommand{\myarraystretch}{1.05}
\newcommand{\temparraystretch}{1.05}

\newcolumntype{I}{!{\vrule width 1.3pt}}
\newlength\savedwidth
\newcommand\whline{\noalign{\global\savedwidth\arrayrulewidth
                            \global\arrayrulewidth 1.3pt}%
           \hline
           \noalign{\global\arrayrulewidth\savedwidth}}


\newcommand{\myFlaTwoByTwo}[4]{
\renewcommand{\temparraystretch}{\arraystretch}
\renewcommand{\arraystretch}{\myarraystretch}
%\setlength{\arraycolsep}{0pt}
  \left(
	\begin{array}{c@{\;}|@{\;}c} 
	%\begin{array}{c | c} 
      #1 & #2 \\ \hline
      #3 & #4 
    \end{array}
    \renewcommand{\arraystretch}{\temparraystretch}
  \right)
}

\newcommand{\myFlaTwoByOne}[2]{
\renewcommand{\temparraystretch}{\arraystretch}
\renewcommand{\arraystretch}{\myarraystretch}
  \left(
	\begin{array}{c} 
      #1 \\ \hline
      #2  
    \end{array} 
    \renewcommand{\arraystretch}{\temparraystretch}
  \right)
}

\newcommand{\myFlaOneByTwo}[2]{
\renewcommand{\temparraystretch}{\arraystretch}
\renewcommand{\arraystretch}{\myarraystretch}
\left(
%\begin{array}{c@{\;\;}|@{\;\;}c} 
\begin{array}{c | c} 
#1 & #2
\end{array}
\renewcommand{\arraystretch}{\temparraystretch}
\right)
}

\newcommand{\myFlaTwoByTwoI}[4]{
\renewcommand{\temparraystretch}{\arraystretch}
\renewcommand{\arraystretch}{\myarraystretch}
\left( 
\begin{array}{c@{\;\;}I@{\;\;}c} 
#1 & #2 \\ \whline
#3 & #4 
\end{array}
\renewcommand{\arraystretch}{\temparraystretch}
\right)
}

\newcommand{\myFlaThreeByThree}[9]{
\renewcommand{\temparraystretch}{\arraystretch}
\renewcommand{\arraystretch}{\myarraystretch}
\left( 
%\begin{array}{c@{\;\;}|@{\;\;}c@{\;\;}|@{\;\;}c} 
\begin{array}{c | c | c}
#1 & #2 & #3 \\ \hline
#4 & #5 & #6 \\ \hline
#7 & #8 & #9
\end{array}
\renewcommand{\arraystretch}{\temparraystretch}
\right) 
}

\newcommand{\myFlaThreeByTwo}[6]{
\renewcommand{\temparraystretch}{\arraystretch}
\renewcommand{\arraystretch}{\myarraystretch}
\left( 
%\begin{array}{c@{\;\;}|@{\;\;}c@{\;\;}|@{\;\;}c} 
\begin{array}{c | c}
#1 & #2 \\ \hline
#3 & #4 \\ \hline
#5 & #6
\end{array}
\renewcommand{\arraystretch}{\temparraystretch}
\right) 
}


\newcommand{\myFlaThreeByThreeI}[9]{
\renewcommand{\temparraystretch}{\arraystretch}
\renewcommand{\arraystretch}{\myarraystretch}
\left( 
%\begin{array}{c@{\;\;}|@{\;\;}c@{\;\;}|@{\;\;}c} 
\begin{array}{c I c I c}
#1 & #2 & #3 \\ \whline
#4 & #5 & #6 \\ \whline
#7 & #8 & #9
\end{array}
\renewcommand{\arraystretch}{\temparraystretch}
\right) 
}

%% Modular 4 x 4
% T: thin line
% F: fat line


\newcommand{\FourColHeadTTT}[0]{
\renewcommand{\temparraystretch}{\arraystretch}
\renewcommand{\arraystretch}{\myarraystretch}
\left( 
%\begin{array}{c@{\;\;}|@{\;\;}c@{\;\;}|@{\;\;}c} 
\begin{array}{c | c | c | c}
}

\newcommand{\FourColHeadFFF}[0]{
\renewcommand{\temparraystretch}{\arraystretch}
\renewcommand{\arraystretch}{\myarraystretch}
\left( 
%\begin{array}{c@{\;\;}|@{\;\;}c@{\;\;}|@{\;\;}c} 
\begin{array}{c I c I c I c}
}

\newcommand{\FourColHeadFFT}[0]{
\renewcommand{\temparraystretch}{\arraystretch}
\renewcommand{\arraystretch}{\myarraystretch}
\left( 
%\begin{array}{c@{\;\;}|@{\;\;}c@{\;\;}|@{\;\;}c} 
\begin{array}{c I c I c | c}
}

\newcommand{\FourColHeadTFF}[0]{
\renewcommand{\temparraystretch}{\arraystretch}
\renewcommand{\arraystretch}{\myarraystretch}
\left( 
%\begin{array}{c@{\;\;}|@{\;\;}c@{\;\;}|@{\;\;}c} 
\begin{array}{c | c I c I c}
}

\newcommand{\FourColHeadFTF}[0]{
\renewcommand{\temparraystretch}{\arraystretch}
\renewcommand{\arraystretch}{\myarraystretch}
\left( 
%\begin{array}{c@{\;\;}|@{\;\;}c@{\;\;}|@{\;\;}c} 
\begin{array}{c I c | c I c}
}

\newcommand{\FourColHeadTFT}[0]{
\renewcommand{\temparraystretch}{\arraystretch}
\renewcommand{\arraystretch}{\myarraystretch}
\left( 
%\begin{array}{c@{\;\;}|@{\;\;}c@{\;\;}|@{\;\;}c} 
\begin{array}{c | c I c | c}
}

\newcommand{\FourColHeadTTF}[0]{
\renewcommand{\temparraystretch}{\arraystretch}
\renewcommand{\arraystretch}{\myarraystretch}
\left( 
%\begin{array}{c@{\;\;}|@{\;\;}c@{\;\;}|@{\;\;}c} 
\begin{array}{c | c | c I c}
}

\newcommand{\FourColTail}[4]{
#1 & #2 & #3 & #4
\end{array}
\renewcommand{\arraystretch}{\temparraystretch}
\right) 
}

\newcommand{\FourColRowT}[4]{
#1 & #2 & #3 & #4 \\ \hline
}

\newcommand{\FourColRowF}[4]{
#1 & #2 & #3 & #4 \\ \whline
}

%%%%%%%%%%%%%%

\newcommand{\FiveColHeadTTTT}[0]{
\renewcommand{\temparraystretch}{\arraystretch}
\renewcommand{\arraystretch}{\myarraystretch}
\left( 
%\begin{array}{c@{\;\;}|@{\;\;}c@{\;\;}|@{\;\;}c} 
\begin{array}{c | c | c | c | c}
}

\newcommand{\FiveColHeadFFFT}[0]{
\renewcommand{\temparraystretch}{\arraystretch}
\renewcommand{\arraystretch}{\myarraystretch}
\left( 
%\begin{array}{c@{\;\;}|@{\;\;}c@{\;\;}|@{\;\;}c} 
\begin{array}{c I c I c I c | c}
}

\newcommand{\FiveColHeadFTFF}[0]{
\renewcommand{\temparraystretch}{\arraystretch}
\renewcommand{\arraystretch}{\myarraystretch}
\left( 
%\begin{array}{c@{\;\;}|@{\;\;}c@{\;\;}|@{\;\;}c} 
\begin{array}{c I c | c I c I c}
}


\newcommand{\FiveColTail}[5]{
#1 & #2 & #3 & #4 & #5
\end{array}
\renewcommand{\arraystretch}{\temparraystretch}
\right) 
}

\newcommand{\FiveColRowT}[5]{
#1 & #2 & #3 & #4 & #5 \\ \hline
}

\newcommand{\FiveColRowF}[5]{
#1 & #2 & #3 & #4 & #5\\ \whline
}

%%%%%%%%%%%%%%%

\newcommand{\myFlaFourByFourA}[8]{
\renewcommand{\temparraystretch}{\arraystretch}
\renewcommand{\arraystretch}{\myarraystretch}
\left( 
%\begin{array}{c@{\;\;}|@{\;\;}c@{\;\;}|@{\;\;}c} 
\begin{array}{c | c | c | c}
#1 & #2 & #3 & #4 \\ \hline
#5 & #6 & #7 & #8 \\ \hline
}

\newcommand{\myFlaFourByFourB}[8]{
#1 & #2 & #3 & #4 \\ \hline
#5 & #6 & #7 & #8
\end{array}
\renewcommand{\arraystretch}{\temparraystretch}
\right) 
}

\newcommand{\myFlaThreeByThreeTLI}[9]{
\renewcommand{\temparraystretch}{\arraystretch}
\renewcommand{\arraystretch}{\myarraystretch}
\left( 
\begin{array}{c@{\;\;}|@{\;\;}c@{\;\;}I@{\;\;}c} 
%\begin{array}{c | c I c}
#1 & #2 & #3 \\ \hline
#4 & #5 & #6 \\ \whline
#7 & #8 & #9
\end{array}
\renewcommand{\arraystretch}{\temparraystretch}
\right) 
}

\newcommand{\myFlaThreeByThreeBRI}[9]{
\renewcommand{\temparraystretch}{\arraystretch}
\renewcommand{\arraystretch}{\myarraystretch}
\left( 
\begin{array}{c@{\;\;}I@{\;\;}c@{\;\;}|@{\;\;}c}
%\begin{array}{c I c | c}
#1 & #2 & #3 \\ \whline
#4 & #5 & #6 \\ \hline
#7 & #8 & #9
\end{array}
\renewcommand{\arraystretch}{\temparraystretch}
\right)
}

\newcommand{\myFlaThreeByThreeTRI}[9]{
\renewcommand{\temparraystretch}{\arraystretch}
\renewcommand{\arraystretch}{\myarraystretch}
\left( 
\begin{array}{c@{\;\;}I@{\;\;}c@{\;\;}|@{\;\;}c} 
#1 & #2 & #3 \\ \hline
#4 & #5 & #6 \\ \whline
#7 & #8 & #9
\end{array}
\renewcommand{\arraystretch}{\temparraystretch}
\right) 
}

\newcommand{\myFlaThreeByThreeBLI}[9]{
\renewcommand{\temparraystretch}{\arraystretch}
\renewcommand{\arraystretch}{\myarraystretch}
\left( 
\begin{array}{c@{\;\;}|@{\;\;}c@{\;\;}I@{\;\;}c} 
#1 & #2 & #3 \\ \whline
#4 & #5 & #6 \\ \hline
#7 & #8 & #9
\end{array}
\renewcommand{\arraystretch}{\temparraystretch}
\right)
}

\newcommand{\myFlaTwoByOneI}[2]{
\renewcommand{\temparraystretch}{\arraystretch}
\renewcommand{\arraystretch}{\myarraystretch}
\left( 
\begin{array}{@{\hspace{1pt}}c@{\hspace{1pt}}}
#1 \\ \whline
#2 
\end{array}
\renewcommand{\arraystretch}{\temparraystretch}
\right)
}

\newcommand{\myFlaOneByTwoI}[2]{
\renewcommand{\temparraystretch}{\arraystretch}
\renewcommand{\arraystretch}{\myarraystretch}
\left( 
\begin{array}{@{\hspace{1pt}}c@{\hspace{2pt}}I@{\hspace{2pt}}c@{\hspace{1pt}}}
#1 & #2 
\end{array}
\renewcommand{\arraystretch}{\temparraystretch}
\right)
}

\newcommand{\myFlaOneByThree}[3]{
\renewcommand{\temparraystretch}{\arraystretch}
\renewcommand{\arraystretch}{\myarraystretch}
\left( 
\begin{array}{c | c | c}
%\begin{array}{c@{\;\;}|@{\;\;}c@{\;\;}|@{\;\;}c} 
#1 & #2 & #3 
\end{array}
\renewcommand{\arraystretch}{\temparraystretch}
\right)
}

\newcommand{\myFlaOneByThreeI}[3]{
\renewcommand{\temparraystretch}{\arraystretch}
\renewcommand{\arraystretch}{\myarraystretch}
\left( 
\begin{array}{c I c I c}
%\begin{array}{c@{\;\;}|@{\;\;}c@{\;\;}|@{\;\;}c} 
#1 & #2 & #3 
\end{array}
\renewcommand{\arraystretch}{\temparraystretch}
\right)
}

\newcommand{\myFlaOneByFour}[4]{
\renewcommand{\temparraystretch}{\arraystretch}
\renewcommand{\arraystretch}{\myarraystretch}
\left( 
\begin{array}{c | c | c | c}
%\begin{array}{c@{\;\;}|@{\;\;}c@{\;\;}|@{\;\;}c} 
#1 & #2 & #3 & #4
\end{array}
\renewcommand{\arraystretch}{\temparraystretch}
\right)
}

\newcommand{\myFlaOneByFourTFF}[4]{
\renewcommand{\temparraystretch}{\arraystretch}
\renewcommand{\arraystretch}{\myarraystretch}
\left( 
\begin{array}{c | c I c I c}
%\begin{array}{c@{\;\;}|@{\;\;}c@{\;\;}|@{\;\;}c} 
#1 & #2 & #3 & #4
\end{array}
\renewcommand{\arraystretch}{\temparraystretch}
\right)
}

\newcommand{\myFlaOneByThreeLI}[3]{
\renewcommand{\temparraystretch}{\arraystretch}
\renewcommand{\arraystretch}{\myarraystretch}
\left( 
\begin{array}{c@{\;\;}|@{\;\;}c@{\;\;}I@{\;\;}c} 
#1 & #2 & #3 
\end{array}
\renewcommand{\arraystretch}{\temparraystretch}
\right)
}

\newcommand{\myFlaOneByThreeRI}[3]{
\renewcommand{\temparraystretch}{\arraystretch}
\renewcommand{\arraystretch}{\myarraystretch}
\left( 
\begin{array}{c@{\;\;}I@{\;\;}c@{\;\;}|@{\;\;}c} 
#1 & #2 & #3 
\end{array}
\renewcommand{\arraystretch}{\temparraystretch}
\right)
}

\newcommand{\myFlaThreeByOne}[3]{
\renewcommand{\temparraystretch}{\arraystretch}
\renewcommand{\arraystretch}{\myarraystretch}
\left( 
\begin{array}{c}
#1 \\ \hline
#2 \\ \hline
#3
\end{array}
\renewcommand{\arraystretch}{\temparraystretch}
\right)
}

\newcommand{\myFlaThreeByOneI}[3]{
\renewcommand{\temparraystretch}{\arraystretch}
\renewcommand{\arraystretch}{\myarraystretch}
\left( 
\begin{array}{c}
#1 \\ \whline
#2 \\ \whline
#3
\end{array}
\renewcommand{\arraystretch}{\temparraystretch}
\right)
}

\newcommand{\myFlaFourByOne}[4]{
\renewcommand{\temparraystretch}{\arraystretch}
\renewcommand{\arraystretch}{\myarraystretch}
\left( 
\begin{array}{c}
#1 \\ \hline
#2 \\ \hline
#3 \\ \hline
#4
\end{array}
\renewcommand{\arraystretch}{\temparraystretch}
\right)
}

\newcommand{\FlaTwoByTwoI}[4]{
\renewcommand{\temparraystretch}{\arraystretch}
\renewcommand{\arraystretch}{\myarraystretch}
\left( 
\begin{array}{c I c}
#1 & #2 \\ \whline
#3 & #4 
\end{array}
\renewcommand{\arraystretch}{\temparraystretch}
\right)
}

\newcommand{\FlaTwoByOneI}[2]{
\renewcommand{\temparraystretch}{\arraystretch}
\renewcommand{\arraystretch}{\myarraystretch}
\left( 
\begin{array}{c}
#1 \\ \whline
#2 
\end{array}
\renewcommand{\arraystretch}{\temparraystretch}
\right)
}

\newcommand{\FlaOneByTwoI}[2]{
\renewcommand{\temparraystretch}{\arraystretch}
\renewcommand{\arraystretch}{\myarraystretch}
\left( 
\begin{array}{c I c}
#1 & #2 
\end{array}
\renewcommand{\arraystretch}{\temparraystretch}
\right)
}

\newcommand{\FlaThreeByThreeTLI}[9]{
\renewcommand{\temparraystretch}{\arraystretch}
\renewcommand{\arraystretch}{\myarraystretch}
\left( 
\begin{array}{c | c I c}
#1 & #2 & #3 \\ \hline
#4 & #5 & #6 \\ \whline
#7 & #8 & #9
\end{array}
\renewcommand{\arraystretch}{\temparraystretch}
\right) 
}

\newcommand{\FlaThreeByThreeBRI}[9]{
\renewcommand{\temparraystretch}{\arraystretch}
\renewcommand{\arraystretch}{\myarraystretch}
\left( 
\begin{array}{c I c | c}
#1 & #2 & #3 \\ \whline
#4 & #5 & #6 \\ \hline
#7 & #8 & #9
\end{array}
\renewcommand{\arraystretch}{\temparraystretch}
\right)
}


\newcommand{\FlaThreeByOneTI}[3]{
\renewcommand{\temparraystretch}{\arraystretch}
\renewcommand{\arraystretch}{\myarraystretch}
\left( 
\begin{array}{c}
#1 \\ \hline
#2 \\ \whline
#3 
\end{array}
\renewcommand{\arraystretch}{\temparraystretch}
\right) 
}


\newcommand{\FlaThreeByOneBI}[3]{
\renewcommand{\temparraystretch}{\arraystretch}
\renewcommand{\arraystretch}{\myarraystretch}
\left( 
\begin{array}{c}
#1 \\ \whline
#2 \\ \hline
#3 
\end{array}
\renewcommand{\arraystretch}{\temparraystretch}
\right) 
}


% In math mode, 
% \FlaTwoByTwo{A}{B}
%             {C}{D}
% creates the picture
%   / A || B \
%   | ==  == |
%   \ C || D /

\newcommand{\FlaTwoByTwo}[4]{
\left( 
\begin{array}{c || c}
#1 & #2 \\ \hline \hline
#3 & #4 
\end{array} 
\right)
}

% In math mode, 
% \TwoByTwo{A}{B}
%          {C}{D}
% creates the picture
%   |A | B|
%   |-----|
%   |C | D|

\newcommand{\TwoByTwo}[4]{
  \left(
    \begin{array}{c | c}
      #1 & #2 \\ \hline
      #3 & #4 
    \end{array} 
  \right)
}




% In math mode, 
% \FlaTwoByOne{A}
%             {C}
% creates the picture
%   / A \
%   | = |
%   \ C /

\newcommand{\FlaTwoByOne}[2]{
\left( 
\begin{array}{c}
#1 \\ \hline \hline
#2 
\end{array} 
\right)
}


% \TwoByOne{A}
%          {C}
% creates the picture
%  / A \
%  | - |
%  \ C /

\newcommand{\TwoByOne}[2]{
\left( 
  \begin{array}{c}
    #1 \\ \hline
    #2 
  \end{array} 
\right)
}





% In math mode, 
% \FlaOneByTwo{A}{B}
% creates the picture
%   ( A || B )

\newcommand{\FlaOneByTwo}[2]{
\left( 
\begin{array}{c || c}
#1 & #2 
\end{array} 
\right)
}

% In math mode, 
% \FlaThreeByThreeTL{A}{B}{C}
%                   {D}{E}{F}
%                   {G}{H}{I}
% creates the picture
%   / A | B || C \
%   | -- ---  -- |
%   | D | E || F |
%   | ==  ==  == |
%   \ G | H || I /
% Notice: the TL means that the
% center block (E) is part of the
% TL quadrant, where quadrants are
% partitioned by the double lines.

\newcommand{\FlaThreeByThreeTL}[9]{
\left( 
\begin{array}{c | c || c}
#1 & #2 & #3 \\ \hline
#4 & #5 & #6 \\ \hline \hline 
#7 & #8 & #9
\end{array} 
\right) 
}

% In math mode, 
% \FlaThreeByThreeBR{A}{B}{C}
%                   {D}{E}{F}
%                   {G}{H}{I}
% creates the picture
%   / A || B | C \
%   | ==  ==  == |
%   | D || E | F |
%   | -- ---  -- |
%   \ G || H | I /
% Notice: the BR means that the
% center block (E) is part of the
% BR quadrant, where quadrants are
% partitioned by the double lines.

\newcommand{\FlaThreeByThreeBR}[9]{
\left( 
\begin{array}{c || c | c}
#1 & #2 & #3 \\ \hline \hline 
#4 & #5 & #6 \\ \hline
#7 & #8 & #9
\end{array} 
\right)
}

% In math mode, 
% \FlaThreeByThreeTR{A}{B}{C}
%                   {D}{E}{F}
%                   {G}{H}{I}
% creates the picture
%   / A || B | C \
%   | -- ---  -- |
%   | D || E | F |
%   | ==  ==  == |
%   \ G || H | I /
% Notice: the TR means that the
% center block (E) is part of the
% TR quadrant, where quadrants are
% partitioned by the double lines.

\newcommand{\FlaThreeByThreeTR}[9]{
\left( 
\begin{array}{c || c | c}
#1 & #2 & #3 \\ \hline
#4 & #5 & #6 \\ \hline \hline 
#7 & #8 & #9
\end{array} 
\right)
}


% In math mode, 
% \FlaThreeByThreeBL{A}{B}{C}
%                   {D}{E}{F}
%                   {G}{H}{I}
% creates the picture
%   / A | B || C \
%   | ==  ==  == |
%   | D | E || F |
%   | -- ---  -- |
%   \ G | H || I /
% Notice: the BL means that the
% center block (E) is part of the
% BL quadrant, where quadrants are
% partitioned by the double lines.

\newcommand{\FlaThreeByThreeBL}[9]{
\left( 
\begin{array}{c | c || c}
#1 & #2 & #3 \\ \hline \hline 
#4 & #5 & #6 \\ \hline 
#7 & #8 & #9
\end{array} 
\right)
}

% In math mode, 
% \FlaOneByThreeR{A}{B}{C}
% creates the picture
%   ( A || B | C )
% Notice: the R means that the
% center block (B) is part of the
% R(ight) submatrix, where 
% submatrices are % partitioned 
% by the double lines.

\newcommand{\FlaOneByThreeR}[3]{
\left( 
\begin{array}{c || c | c}
#1 & #2 & #3 
\end{array} 
\right)
}

% In math mode, 
% \FlaOneByThreeL{A}{B}{C}
% creates the picture
%   ( A | B || C )
% Notice: the R means that the
% center block (B) is part of the
% R(ight) submatrix, where 
% submatrices are % partitioned 
% by the double lines.

\newcommand{\FlaOneByThreeL}[3]{
\left( 
\begin{array}{c | c || c}
#1 & #2 & #3 
\end{array} 
\right)
}

% In math mode, 
% \FlaThreeByOneT{A}
%                {D}
%                {G}
% creates the picture
%   / A  \
%   | == |
%   | B  |
%   | -- |
%   \ C  /
% Notice: the T means that the
% center block (C) is part of the
% T(op) submatrix where submatrices
% are % partitioned by the double 
% lines.

\newcommand{\FlaThreeByOneT}[3]{
\left( 
\begin{array}{c}
#1 \\ \hline
#2 \\ \hline \hline 
#3 
\end{array} 
\right) 
}

% In math mode, 
% \FlaThreeByOneB{A}
%                {D}
%                {G}
% creates the picture
%   / A  \
%   | -- |
%   | B  |
%   | == |
%   \ C  /
% Notice: the B means that the
% center block (C) is part of the
% T(op) submatrix where submatrices
% are % partitioned by the double 
% lines.

\newcommand{\FlaThreeByOneB}[3]{
\left( 
\begin{array}{c}
#1 \\ \hline \hline 
#2 \\ \hline
#3 
\end{array} 
\right) 
}
