
\documentclass{article}
\usepackage
[
        a4paper,% other options: a3paper, a5paper, etc
        left=1in,
        right=1in,
        top=1in,
        bottom=1in,
        % use vmargin=2cm to make vertical margins equal to 2cm.
        % us  hmargin=3cm to make horizontal margins equal to 3cm.
        % use margin=3cm to make all margins  equal to 3cm.
]
{geometry}

\usepackage{amsmath}
\usepackage{amsfonts}
\usepackage[ruled,vlined]{algorithm2e}
\usepackage{algorithmic}% http://ctan.org/pkg/algorithms
\usepackage[utf8]{inputenc}
\usepackage{subfig}
\usepackage{xcolor}
\usepackage{graphicx}

\newcommand{\response}[1]{{\color{black}\smallskip\\ #1 }}
\newcommand{\LL}[1]{\textcolor{red}{{\bf LL:} #1}}

\title{A response to the reviews in detail}
\author{Zheng Chen, Shi Bai, Lantao Liu}
\date{}


\begin{document}
\maketitle

\noindent Dear editor and anonymous reviewers,
\medskip

As the authors of ``Efficient Map Prediction via Low-Rank Matrix Completion'' 
we wish to thank you for your time  in reviewing and identifying areas for improvement in the submission. We have made large effort to address each of the concerns, extend the work where necessary, add more explanations, and apply adjustments to all issues identified. We thank you especially because all the points raised are very constructive. 

{
\textbf{Revision Summary:} 
}

This response letter serves to document how we have specifically addressed reviewers' concerns. We do this by examining each issue and describing the resolution in the updated
submission.

\subsection*{Review 13495}
{\color{blue}
    \begin{itemize}
        \item
        I am convinced of the use of LRMC; however, the reason for choosing the SVD based
        iterative method to solve large-scale matrix problem is unclear.
        \response{This is a good point. Although SVD is able to impute the missing values well, computing a SVD for a large-scale matrix might be expensive. However, the authors of \cite{mazumder2010spectral} have shown that the SVD based method could easily handle matrices of very large dimensions by leveraging the \textit{problem structure}~(see Eq.~(5) in \cite{mazumder2010spectral}) in LRMC. Specifically, the iterative SVD based method adopts the Lanczos bidiagonalization with partial re-orthogonalization~\cite{larsen1998lanczos}, which enables the computation of a low-rank SVD as of an order that depends linearly on the matrix dimensions. We have added more explanations in the last sentence of Section. II.
        }
        
        \item
        The following papers are also closely related, and
        they are worth mentioning. The first paper~\cite{nguyen2019low} summarizes the
        basics and the application of LRMC, which can be a good
        supplement for the paper. The second paper~\cite{zhi2019continuous} is a more recent work of BHM, which is another possible baseline for time
        computation comparison.
        \response{ We appreciate the suggestion to the reference works. Both of them are very relevant to this work. We have included \cite{nguyen2019low} in Section. II. For the recent work on BHM - FastBHM~\cite{zhi2019continuous}, we agree that this work is an improved version, which reduces the time complexity by eliminating the computation of the inverse of the full covariance matrix. However, the authors do not release the code of their work and we might want to leave this to the future work.
        }
        
        \item
        The terrain mapping
        experiment in the submitted video is fascinating. However,
        it is not written in the paper, which makes this a
        drawback.
        \response{ Thanks for the kind suggestion. To show the performance of our proposed method on real terrain map, we have replaced the mapping process on road network to terrain mapping process (see Fig. 9). We believe removing the mapping process on road network in Fig.~9 does not weaken the presentation of the results in the paper since we have already shown the LRMC results on road network in Fig. 4.
        }
        
        \item
        I would like to see more
        discussion about the time performance corresponds to
        different ranks of maps in figure 6-right and figure
        8-right. With the LRMC method, the rank-2 map takes the
        least time, but the BHM method takes the most time. It
        would be interesting to see the difference between these two methods.  
        \response{}
        
\end{itemize}
}

\subsection*{Review 13499}
{\color{blue}
    \begin{itemize}
        
        \item
        I believe many more relevant works should be
        discussed and compared to better demonstrate the advantages
        of LRMC. Using low-rank matrix completion for image
        completion (the map here is a special kind of image) has
        been widely studied in the field of computer vision and
        computer graphics since nearly 10 years ago. For example,
        the well known paper~\cite{wright2010sparse}. The NO pattern is exactly
        the image denoise problem presented in Figure 4 of that
        paper.
        \response{Thanks for pointing out this. It is true that LRMC has been widely used in image restoration. In our work we do regard the 2D map as an image and we believe treating the map as an image could bring many benefits, one of which is as presented in this paper --- to allow us to predict the map structure in the same manner as to restore the image. Another possible benefit could be to select informative waypoints for robots to navigate in the same way as active sampling for matrix completion~\cite{mak2018maximum}, which might be investigated in our future work.
        }
        
        \item
        This paper also does not compare with more recent
        map prediction methods based on CNN, though those works
        ([8-13] in manuscript) are briefly discussed. It is unclear what advantage
        does LRMC brings to this problem.
        \response{It is interesting to compare the CNN based methods with our proposed method. Firstly the reason why CNN works well on map prediction in those mentioned literature is they treat the map as an image and they process the map data in a way of processing an image, which is similar to the way of our work in spirit. Secondly, we believe our work belongs to a different category than CNN based methods, which should be classified as a learning based method that requires a large-scale dataset to train the model (we mention this in the second paragraph in Section. II). However, our proposed method does not require the training data but only the observational data.
        }
        
\end{itemize}
}

\subsection*{Review 17577}
{\color{blue}
    \begin{itemize}
        \item
        In particular, I do not think Bayesian
        Hilbert maps, in their vanilla form, are designed for map
        completion or prediction tasks.
        \response{}
        
        \item
        More recent approaches such
        as~\cite{toyungyernsub2020double, tompkins2020online}  might be more relevant for the map prediction
        problem.  Given that the proposed approach implicitly makes
        use of geometric structure, it resonates with the idea of
        domain adaptation~\cite{tompkins2020online, li2019heterogeneous}.
        \response{}
        
        \item
        One of the major drawbacks of the proposed method is the
        possibility of overly-relying on the structure. I would
        like to see the performance in varying road sizes and
        shapes (e.g. curves). Can we have an experiment with
        varying sizes of areas for matrix completion and see the
        performance?
        \response{}
        
        \item
        Experimental details are quite limited. Provide information
        on the datasets rather than briefly mentioning "road
        network map extracted from an urban environmental map." 
        \response{}
\end{itemize}
}

{
\bibliographystyle{unsrt}
\bibliography{ref}
}

\end{document}