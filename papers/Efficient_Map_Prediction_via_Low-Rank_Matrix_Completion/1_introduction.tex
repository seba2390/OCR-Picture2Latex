\section{INTRODUCTION}
Mapping is a critical functionality for an autonomous robot that needs to perform tasks in an initially unknown environment where an underlying map is constructed with onboard sensors. 
However, this can be challenging if the sensor measurements are inaccurate or if the environment can only be partially measured (sensed), which are commonly seen in autonomous mapping scenarios. 
For instance, the point clouds obtained from low-end proximity sensors (e.g., LiDAR, IR, Sonar) might be extremely sparse especially in spacious environments. The point clouds might also contain many outliers (noisy points). In addition, oftentimes a mapping robot cannot visit every spot of the environment and the traversed areas may only partially cover the space with only some small sub-regions being measured. 
Given the sparse, noisy, and partial observations, it is desirable that a complete map can be efficiently predicted to well match the ground-truth environment, which is the objective of this paper. 

More formally, the map prediction aims to predict map structure based on already known regions and could be categorized into two classes: interpolation and extrapolation. Map interpolation means given some observations (measurement samples), we want to predict the values \textit{in between} the sampled places;
Map extrapolation means given some observation samples (or map structures), we want to predict the map values \textit{beyond} the sampled places. 

\begin{figure} \vspace{5pt}
  \centering
%   \subfigure[]
  	{\label{fig:real}\includegraphics[width=\linewidth]{figs/real_examples.png}}
  \caption{\small Urban and residential environments reveal strong structured patterns, such as \textit{Left}: road (or street) network and \textit{Right}: buildings layout. These environments containing linear dependent structures can be modeled as low-rank-matrix maps.} 
\label{fig:real}  
\end{figure}


In this paper, we propose to use Low-Rank Matrix Completion~(LRMC) to leverage the ill-conditioned map data to predict the whole map. LRMC exploits some special structures, such as low rank~(or linear dependency) and incoherence hidden in underlying map to perform the prediction. Two examples that possess special patterns in urban/residential environments are shown in Fig.~\ref{fig:real}.
We first illustrate how a structured environmental map satisfies important assumptions required by LRMC model. Then we qualitatively demonstrate with an urban road network map that our proposed LRMC method outperforms state-of-the-art map prediction~(or regression) method -- Bayesian Hilbert Mapping~(BHM)~\cite{senanayake2017bayesian} which is a Bayesian extension of Hilbert Mapping~(HM)~\cite{ramos2016hilbert}, where the map prediction is treated as a kernelized logistic regression problem. BHM eliminates crucial regularization parameter tunning in HM, and outputs continuous maps where the occupancy value of any arbitrary point could be queried.

We conduct exhaustive simulations on a set of maze-like complex maps and quantitatively analyze that our proposed LRMC method is superior to BHM on the maps with varying linear patterns and different rank values. 
Finally we combine the proposed real-time map prediction with representative coverage planning methods commonly used for environmental mapping and monitoring, and show that the mapping coverage convergence could be significantly improved with high mapping accuracy. 

We summarize our contributions as follows:
\begin{itemize}
    \item 
    This is the first time to examine and report that many complex urban/residential environments possess low-rank and incoherent structure, and to apply Low-Rank Matrix Completion for map prediction based on sparse, noisy, and partially observed maps.
    \item
    Our proposed Low-Rank Matrix Completion based map prediction outperforms state-of-the-art map prediction method---Bayesian Hilbert Mapping in terms of mapping accuracy and computation time and is able to perform prediction in real-time.
    \item
    We perform extensive simulations and demonstrate the remarkable effectiveness of our proposed method  which allows representative coverage planning methods to achieve faster mapping coverage convergence rates.  %commonly used in environmental mapping   can be substantially improved.
\end{itemize}
