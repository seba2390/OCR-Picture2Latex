\section{EXPERIMENTS}
The goal of this paper is to demonstrate that a LRMC based map prediction is efficient and fast and can significantly facilitate the coverage convergence in environmental monitoring. We have shown that the maze-like environments fit well into the LRMC model. We want to understand in a further step how the model behaves compared to other state-of-the-art map prediction method, i.e., Bayesian Hilbert Mapping~(BHM) in terms of TPA as well as computation time. We also want to know how commonly used coverage planning methods can benefit from real-time map prediction. In this section, we firstly conduct exhaustive comparison experiments between LRMC and BHM, under US and BUS, respectively; then we show that an outstanding coverage convergence improvement could be obtained if we equip commonly used coverage planning methods with our proposed LRMC real-time map prediction.

\subsection{Comparison between LRMC and BHM}
We compare two map prediction methods, LRMC and BHM, on a static partially observed map, that is, the sampling is given at one-time for the whole map. The ground-truth map is the same as Fig.~\ref{fig:demo}-(a) and the map size parameters are set as the ones in Section.~\ref{sec:map_completion}. We set the value of $C$ as $2.0$~(as discussed in Section.~\ref{sec:map_completion}) and the results from both methods are shown in Fig.~\ref{fig:comparison_1}. Under US pattern, we observe both methods can achieve an exact~(or almost) recovery~(as shown in the first row of Fig.~\ref{fig:comparison_1}). However, under BUS pattern, LRMC significantly outperforms BHM in terms of the prediction accuracy~(as shown in the second row of Fig.~\ref{fig:comparison_1}). LRMC leverages the linear dependency among the row/columns of the map matrix and is able to predict the structure even multiple blocks of missing occur. In contrast, although BHM has a capacity to capture local neighbouring spatial relationship by using kernelized features, it lacks the ability to capture the linear dependency brought by the low-rank structure, and thus performs worse than LRMC in maze-like environments. To validate the effectiveness of our proposed LRMC map prediction method, we consider the comparisons under both US and BUS patterns. The evaluation metrics we use in this section is TPA and \textit{computational time}~(CT). 

Under US pattern, we first generate 20 maze environments with the same rank value of 7 and apply LRMC and BHM to the 20 mazes and collect the TPA and CT values per different values of $C$~(as shown in Fig.~\ref{fig:us_acc_time_curve}). Then we generate 6 maze environments with different ranks values and again we apply the two map prediction methods and observe the trend of TPA and CT with different $C$~(as shown in Fig.~\ref{fig:us_ranks_curve}). 

In Fig.~\ref{fig:us_acc_time_curve}, it can be observed that BHM has a bit faster accuracy convergence~(per $C$) but LRMC could achieve a higher accuracy level, although no significant difference in terms of accuracy can be found for the two methods. On the other hand, CT performance clearly distinguish the two and the required computation time for LRMC is mush lower than that of BHM. Given the size of the entire map~(matrix), the computational complexity of the LRMC algorithm we use, SoftImpute~\cite{mazumder2010spectral}, is $O((m+n)r^2)$, where $m$ and $n$ are the dimensions and $r$ is the rank of the underlying ground-truth matrix. This means the LRMC model has a constant time complexity in terms of number of samples. BHM is a logistic regression based method and has linear time complexity. Therefore, from the theoretical analysis of computational complexity, we can see that LRMC is better than BHM. However, from Fig.~\ref{fig:us_acc_time_curve}, our experiments show that LRMC model has a better time performance than constant one! The reason for this is the LRMC algorithm we use relies on SVD of the matrix. SVD is friendly to dense matrices but not that efficient for sparse matrices. In our case, when the number of samples is small, the input matrix is a sparse one and SVD may use some tricks to deal with the sparse matrix(i.e., fill in the missing parts with some well designed values before doing the decomposition). However, when the number of samples increases, the input matrix will be a dense one and the computation time for SVD decreases. According to Fig.~\ref{fig:us_acc_time_curve}, we can conclude that under the experimental setup(map size, resolution, hinge points number, etc.) in this paper, BHM is far from being used in real-time applications but LRMC is able to maintain and update the global map very quickly.

In Fig.~\ref{fig:us_ranks_curve}, we compare the TPA/CT per $C$ for mazes with different ranks. It is clear to see that when the environment becomes more complex~(as the rank value increases), the accuracy of BHM decreases while LRMC maintain the same high-level prediction(almost $100\%$). From the first row of Fig.~\ref{fig:comparison_1}, we can find BHM generally performs very well but many minor errors can be found along the edges. When the environment becomes complex, the number of edges increases and thus the minor errors accumulate and this results in the increase of the total map prediction error.

Under BUS pattern, similar to what we have in US pattern, the TPA/CT per $C$ results for mazes with the same rank and different ranks are shown in Fig.~\ref{fig:bus_acc_time_curve} and Fig.~\ref{fig:bus_ranks_curve}, respectively. The comparison results are basically the same as we discussed under US pattern.

% \begin{figure}%[h!]
%   \centering
%   \begin{subfigure}[b]{0.49\linewidth}
%     \includegraphics[width=\linewidth]{figs/pls_coverage.png}
%     \caption{}
%   \end{subfigure}
%   \begin{subfigure}[b]{0.49\linewidth}
%     \includegraphics[width=\linewidth]{figs/dls_coverage.png}
%     \caption{}
%   \end{subfigure}
%   \caption{\small Coverage convergence of different methods for (a)~PLS mode and (b)~DLS mode.}
%   \label{fig:coverage}
% \end{figure}



\begin{figure}
  \centering
%   \subfigure[]
  	{\label{fig:comparison_1}\includegraphics[width=\linewidth]{figs/comparison_1.png}}
  \caption{\small Map prediction comparison between LRMC and BHM. \textit{First row}: With US pattern. (a)~Partially observed map. Predicted map results from (b)~LRMC and (c)~BHM. \textit{Second row}: With BUS pattern. (d)~Partially observed map. Predicted map results from (e)~LRMC and (f)~BHM.  } \vspace{-10pt}
\label{fig:comparison_1}  
\end{figure}

\begin{figure} 
  \centering
  	{\label{fig:us_acc_time_curve}\includegraphics[width=0.99\linewidth]{figs/us_acc_time_curve.png}}
  \caption{\small \textit{Left}: TPA performance per value of $C$ for both methods. \textit{Right}: CT performance per samples for both methods. 
  } \vspace{-10pt}
\label{fig:us_acc_time_curve}  
\end{figure}

\begin{figure}%[t]
  \centering
  	{\label{fig:us_ranks_curve}\includegraphics[width=0.99\linewidth]{figs/us_ranks_curve.png}}
  \caption{\small \textit{Left}: TPA performance per value of $C$ of both methods for different rank values. \textit{Right}: CT performance per samples of both methods for different rank values.
  } \vspace{-10pt}
\label{fig:us_ranks_curve}  
\end{figure}


\begin{figure}%[t] 
  \centering
  	{\label{fig:bus_acc_time_curve}\includegraphics[width=0.99\linewidth]{figs/bus_acc_time_curve.png}}
  \caption{\small \textit{Left}: TPA performance per value of $C$ for both methods. \textit{Right}: CT performance per samples for both methods. 
  } \vspace{-10pt}
\label{fig:bus_acc_time_curve}  
\end{figure}

\begin{figure}%[t]
  \centering
  	{\label{fig:bus_ranks_curve}\includegraphics[width=0.99\linewidth]{figs/bus_ranks_curve.png}}
  \caption{\small \textit{Left}: TPA performance per value of $C$ of both methods for different rank values. \textit{Right}: CT performance per samples of both methods for different rank values.
  } \vspace{-10pt}
\label{fig:bus_ranks_curve}  
\end{figure}

\subsection{Coverage Planning with Real-Time Map Prediction}
\subsubsection{Introduction to Experimental Methods}
We conduct coverage planning on the maze map as shown in Fig.~\ref{fig:tsp_results}-(a), which has a rank of 7. In this paper, we assume our vehicle for performing coverage planning is a point robot and no dynamics is considered. In Section.~\ref{sec:map_completion}, we have shown with a proper choice of $C$, the offline static map completion performance is outstanding and we also conclude there that a reference value for $C$ could be set to 2.0. However, in coverage planning, the observed entries are incrementally revealed since the robot needs to take time to traverse the area and only the places the robot has traversed could become observed entries in the matrix. To find a path for robot to perform the environmental monitoring, typical coverage planning methods used in real applications include: \textit{Lawn-Mower}~(LM), \textit{Global Random Sampling}~(GRS) and \textit{Traveling Salesman Problem}~(TSP).

LM: A simple but widely used classical coverage planning method. Given the size of the mapping area, a path with carefully designed interval is computed such that the whole area could be swept incrementally by the robot.

GRS: A global random sampling based method, in which at each action step, a set of points are randomly sampled in all unexplored areas and the closest one to the robot is selected as the local goal and a series of way-points are generated correspondingly. The robot only takes the first way-point for execution.

TSP: To follow the random uniform observation pattern in the LRMC model, we first uniformly sample a set of way points, then an existing TSP solver is used to find a shortest path connecting all the sampled way points. Note that being different to LM and GRS, TSP is a planning method that is compatible with our proposed LRMC model. The procedure of \textit{random uniform sampling way points} before performing TSP satisfies the assumption for the observation set $\Omega$ in LRMC while the number of way points is guided by Eq.~(\ref{eq:sampled_cond}) and a reference value for $C$ is discussed in Section.~\ref{sec:map_completion}. We will show in later experiments that because TSP based coverage planning is able to satisfy the assumptions in LRMC model, thus it outperforms other planning methods in terms of coverage convergence.

Before we start the experiment, we must determine the number of way points in TSP based planning. In a static environmental monitoring scenario, for example, in static sensor placement problem, the number of sampled observation points guided by Eq.~(\ref{eq:sampled_cond}) can be defined as:
\begin{equation}
    \label{eq:sample_num}
    N^{'}_{s} = \left \lfloor \frac{C\cdot n^{1.2}r\log n}{N} \right \rfloor
\end{equation}
where $N^{'}_{s}$ represents the sensor positions number and $N$ is the number of observation entries within the sensor's local sensing range. Each sampled point will become a sampled block when sensor starts to run. Nevertheless, in the coverage planning scenario, the robot can take observations in between a pair of connected sampled way points and may introduce additional observations. This implies we may want to reduce the number of sampled way points as suggested by Eq.~(\ref{eq:sample_num}) and we do this by introducing a scale factor $\epsilon$ to Eq.~(\ref{eq:sample_num}):
\begin{equation}
    \label{eq:new_sample_num}
    N_{s} = \left \lfloor \epsilon N^{'}_{s} \right \rfloor = \left \lfloor \epsilon \frac{C\cdot n^{1.2}r\log n}{N} \right \rfloor
\end{equation}
where $N_{s}$ is the number of sampled way points in TSP planning and here we fix the value of $C$ as 2.0. Once the points are sampled using Eq.~(\ref{eq:new_sample_num}), a shortest path will be formed using a TSP solver and thereafter our robot could start the coverage mission by following the path. We name TSP planning with different $\epsilon$ values as: $TSP\_\epsilon$, where the $\epsilon$ is the scale factor in Eq.~(\ref{eq:new_sample_num}) and will be replaced with some specific values in experiments.

\subsubsection{Coverage Quality Analysis}
We compare the coverage convergence performance~(per action step) of \textit{LM}, \textit{GRS} and $TSP\_\epsilon$, where $\epsilon \in \left \{ 0.25,~0.5,~0.75 \right \}$, on the same underlying maze map~(as shown in Fig.~\ref{fig:tsp_results}-(a)). The results are shown in Fig.~\ref{fig:coverage}. 

\begin{figure*}%[h!]
  \centering
  \begin{subfigure}[b]{0.19\linewidth}
    \includegraphics[width=\linewidth]{figs/map_truth.png}
    \caption{}
  \end{subfigure}
  \begin{subfigure}[b]{0.39\linewidth}
    \includegraphics[width=\linewidth]{figs/coverage_half_0.5.png}
    \caption{}
  \end{subfigure}
  \begin{subfigure}[b]{0.39\linewidth}
    \includegraphics[width=\linewidth]{figs/coverage_0.5.png}
    \caption{}
  \end{subfigure}
  \caption{\small (a)~Ground-truth map for coverage planning. (b)~\textit{Left}: Real sensed map and \textit{Right}: predicted map at the middle of steps for TSP\_{0.5}. (c)~\textit{Left}: Real sensed map and \textit{Right}: predicted map at the ending step for TSP\_{0.5} }
  \label{fig:tsp_results}
\end{figure*}

\begin{table*}%[h]
\caption{Coverage performance comparison} %title of the table
\centering % centering table
\begin{tabular}{c rrrrrrrrr} % creating eight columns
\hline\hline %inserting double-line
% Properties&\multicolumn{7}{c}{Sum of Extracted Bits} \\ [0.5ex]
  & \vline & 20\% & 40\% & 50\% & 60\% & 80\% & 90\% & 100\% \\
\hline % inserts single-line
LM & \vline & 153 / 69 & 310 / 141 & - / 176 & - / 212 & - / 285 & - / 320 &  - / 356~(99\%)\\ [1ex]% Entering row contents
GRS & \vline & 238 / 38 & - / 130 & - / 155 & - / 162 & - / 456 & - / 566 & - / -\\ [1ex]% [1ex] adds vertical space
TSP\_0.25 & \vline & 124 / 37 & - / 47 & - / 70 & - / \textbf{84} & - / \textbf{110} & - / \textbf{151} & - / - \\ [1ex]
TSP\_0.5 & \vline & 127 / 19 & - / 72 & - / 101 & - / 113 & - / 137 & - / 195 & - / \textbf{251}~(98\%)\\ [1ex]
TSP\_0.75 & \vline & 120  / \textbf{16} & 269 / \textbf{32} & - / \textbf{61} & - / 88 & - / 131 & - / 240 & - / \textbf{309} \\
\hline % inserts single-line
\end{tabular}
\label{tb:coverage_comparison}
\end{table*}

We first consider the coverage planning without real-time map prediction, and the results are shown in the left graph of Fig~\ref{fig:coverage}. We can see that due to the sensor limitations, none of the listed methods could perform a complete coverage planning without the functionality of map prediction. One may argue that we can perform map prediction~(or regression) on the incomplete map after the mission finishes. This off-line map prediction does work to help human obtain a finer-resolution feature map, but it does also have problems if we want to monitor the space online or the planning algorithm require an online updated fine-grain map to make decisions in real-time. We then consider the coverage planning with real-time map prediction, as shown in the right graph of Fig~\ref{fig:coverage}. With the aid of real-time map prediction, nearly all of the planning methods could obtain a complete~(almost $100\%$) feature map. In addition, the rate for coverage convergence is improved outstandingly for all of the listed coverage planning methods compared to themselves without real-time map prediction. We also claim that our proposed LRMC map prediction could be easily plugged with other planning methods that are not listed in this paper and facilitate their environmental monitoring performance.

According to the right graph of Fig~\ref{fig:coverage}, among all the coverage planning methods, TSP\_$\epsilon$ methods significantly outperform all the baseline methods in terms of coverage convergence, although TSP based method can reach different completeness level as the value of $\epsilon$ varies. Empirically a larger value of $\epsilon$ can lead to more sample points, a longer path~(and action step) from TSP, but as well as a better completeness~(i.e., TSP\_{0.75}). A proper choice of $\epsilon$ value can give us a satisfactory trade-off between completeness and required action steps. For example, if we have enough energy supply and pursuit a well completed feature map, we should select a big value of $\epsilon$, such as 0.75. On the other hand, we can have a moderate magnitude $\epsilon$, like 0.25 to achieve an acceptable coverage ratio~(i.e., 90\%), if we have limited energy or constrained allowed action steps. Even with extremely restricted number of action steps~(i.e., less than 200), with a proper $\epsilon$, our TSP based method~(i.e., TSP\_{0.25}, TSP\_{0.5}, and TSP\_{0.75}) can still achieve around $90\%$ coverage. The mapping results at the middle of steps and ending step for TSP\_{0.5} are shown in Fig.~\ref{fig:tsp_results}-(b) and Fig.~\ref{fig:tsp_results}-(c), respectively. From Fig.~\ref{fig:tsp_results}-(b), we can see that our proposed method could perform map interpolation and extrapolation simultaneously on-the-fly during the execution of environmental monitoring.

\begin{figure}[t] 
  \centering
%   \subfigure[]
  	{\label{fig:coverage}\includegraphics[width=\linewidth]{figs/coverage.png}}
  \caption{\small Coverage convergence per action step for different coverage planning methods \textit{Left}: without real-time map prediction and \textit{Right}: with real-time map prediction.
  } \vspace{-10pt}
\label{fig:coverage}  
\end{figure}

We also list the required number of action steps of all planning methods for different levels of completeness in Table~\ref{tb:coverage_comparison}. The format for the entries in Table~\ref{tb:coverage_comparison} is \textit{\# without real-time map prediction}~/~\textit{\# with real-time map prediction}. It is clear to see that the TSP\_$\epsilon$ methods can achieve remarkable reduction for required action steps at all listed completeness quantiles.








