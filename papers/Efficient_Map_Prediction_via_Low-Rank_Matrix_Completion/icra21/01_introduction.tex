\section{INTRODUCTION}
One critical functionality for autonomous vehicles in environmental monitoring is feature mapping, which aims to obtain a feature map~(e.g., terrain map, building layout map, and chemical concentration map) by leveraging the flexible mobility of the robotic platform and the measurement capability of the sensors equipped on the vehicle. The obtained feature map not only feeds back the interested values to human operators, but also is crucial for planning to make decisions if a highly intelligent autonomy is required. In this case, an accurate and fast map prediction is necessary for the robot to quickly update the global feature map in real-time. Almost all the sensors we use can be regarded inherently inaccurate due to many error factors in designing and manufacturing, and this can lead to severely degraded sensing quality, for example, the point cloud from a depth camera may contain many outliers and the one from a low-end LiDAR~(i.e., 16-beam) might be sparse. In addition, nearly all sensors have limited sensing range, which may result in large portions of the environment unobserved/missed by the robot. The limitations on sensors produce many impediments in obtaining a desired feature map, such as measurement noise and data missingness. Hence the map prediction has to quickly predict an accurate map based on those partially sampled~(observed) information. The problem becomes even more challenging in large-scale scenarios, which are frequently encountered in most real applications.

In this paper, we propose to use Low-Rank Matrix Completion~(LRMC) to leverage the partially observed map information to predict the whole feature map in an urban building layout monitoring task~(where we assume no trees or other objects occluding the buildings). We first model the urban building layout map as a maze map and carefully illustrate how the partially sampled maze environments fit into the LRMC model. Then we demonstrate that our proposed LRMC map prediction outperforms some state-of-the-art map prediction~(or regression) method, such as Bayesian Hilbert Mapping~(BHM) in terms of mapping accuracy and computation time performance. Given the experimental setups in this paper, our experiments show that BHM is far from being used in real-time applications but the LRMC is able to maintain and update the global map very quickly. Finally we adopt some representative coverage planning methods commonly used for environmental monitoring and show that with our proposed real-time map prediction, the coverage convergence(per action step) could be significantly improved without loss of map accuracy. 

We summarize our contributions as follows:
\begin{itemize}
    \item 
    We are the first to examine and illustrate that maze-like environments possess low-rank and incoherent structure.
    \item
    We are the first to apply Low-Rank Matrix Completion to predict the partially observed maze environments.
    \item
    Our proposed Low-Rank Matrix Completion based map prediction outperforms another state-of-the-art mapping method---Bayesian Hilbert Mapping in terms of mapping accuracy and computation time and is able to perform prediction in real-time.
    \item
    We provide intensive simulated experiments and demonstrate the outstanding effectiveness of our proposed method and we also show that with our proposed real-time map prediction, the coverage convergence rate~(per action step) of some coverage planning methods commonly used in environmental monitoring can be substantially improved.
\end{itemize}
