\section{EXPERIMENTS}

To validate the proposed framework, we perform extensive simulation evaluations with a variety of complex maps. 

\subsection{Experimental Setups and Metrics Design}

To account for the sensor limitations as in robotics applications, we use two measurement patterns, \textit{Noisy Observation}~(NO) and \textit{Partial Observation}~(PO), for constructing the $\Omega$. NO means the observations are imperfect due to random noises (e.g., due to factors of environments/weathers such as snows, or sensor hardware limitations such as sparse and noisy measurements); 
PO means missed or uncovered regions which are typical in environmental mapping and monitoring scenarios. 

To quantitatively compare the two methods,  we use two evaluation metrics, \textit{total prediction accuracy}~(TPA), and \textit{computational time}~(CT). TPA indicates the correct entry matches (including the observed entries) between the entire predicted map and the ground-truth map while CT measures the required time for map prediction computation. 
In addition, we also evaluate the \textit{mapping convergence rate} during the coverage planning.


\subsection{Demonstration with Road Network Mapping}
\begin{figure}
  \centering
%   \subfigure[]
  	{\label{fig:road_image}\includegraphics[width=\linewidth]{figs/road_image.png}}
  \caption{\small \textit{Left}: road network displays linear dependent structures.  \textit{Right}: The road network is extracted where 
  the dark color represents the road while others are off-road places.
  } 
\label{fig:road_image}  
\end{figure}

\begin{figure} %\vspace{-3pt}
  \centering
%   \subfigure[]
  	{\label{fig:road_comparison}\includegraphics[width=\linewidth]{figs/road_comparison.png}}
  \caption{\small Comparison between LRMC and BHM under NO~(first row) and PO~(second row) patterns, respectively. (a)\&(e)~Ground-truth maps. (b)\&(f)~Partially observed maps for NO and PO patterns. The missing parts are indicated in green patches. (c)\&(g)~Predicted maps from LRMC model. (d)\&(h)~Predicted maps from BHM.
  } \vspace{-10pt}
\label{fig:road_comparison}  
\end{figure}



We first demonstrate that our proposed LRMC method is able to perform map interpolation and extrapolation on a road network map extracted from an  urban environmental map, as shown in Fig.~\ref{fig:road_image}. 
The road surface can be detected by proximity sensors such as IR. We assume the robot is an aerial vehicle so that it can spot regions locally below it. 

The comparison of the two mapping methods on the urban road map under NO and PO patterns can be seen in Fig.~\ref{fig:road_comparison}. With NO pattern, both methods can generally predict the road network pattern~(topology), although BHM is more sensitive to the noises and thus fails to capture the exact road shape. On the other hand, with PO pattern, LRMC significantly outperforms BHM in the presence of multiple missing patches. 
Different from NO pattern, PO requires that the map prediction not only has interpolation capability, but also can perform slight extrapolation to reason the map structure within missing patches. 
LRMC leverages the linear dependency among the rows/columns of the map matrix and is able to predict the structure even multiple missing patches occur. In contrast, although BHM can capture local neighbouring spatial relationship by using kernelization, it is unable to capture the linear dependency brought by the low-rank structure, and thus performs worse than LRMC. 

\subsection{Quantitative Comparison between LRMC and BHM}

We first generate 20 differing complex maze-like layouts/environments by using the Daedalus tool~\cite{daedalus}, and consider the TPA and CT statistical comparisons under NO and PO patterns for different values of $C$. 
The results are shown in Fig.~\ref{fig:us_acc_time_curve} and Fig.~\ref{fig:bus_acc_time_curve}. 
The trend of TPA and CT in different rank-valued maps %with different $C$ 
are shown in Fig.~\ref{fig:us_ranks_curve} for NO and Fig.~\ref{fig:bus_ranks_curve} for PO, respectively.

We first analyze the results for NO pattern. In Fig.~\ref{fig:us_acc_time_curve}-\textit{Left}, it can be observed that 
LRMC could achieve a higher accuracy level. 
From Fig.~\ref{fig:us_acc_time_curve}-\textit{Left}, we want to determine a proper scale of the coefficient $C$. We can think the coefficient $C$ is actually reflecting the number of observed map samples according to Eq.~(\ref{eq:sampled_cond}). 
In Fig.~\ref{fig:us_acc_time_curve}-\textit{Left}, we can see that TPA could reach a high level and stay stable when $C\geq 1.5$.
We then compare the CT performance as shown in Fig.~\ref{fig:us_acc_time_curve}-\textit{Right}. The result clearly distinguishes the two methods. The required computation time for LRMC is much lower than that of BHM. Given the size of the entire map~(matrix), the computational complexity of the LRMC algorithm~(SoftImpute~\cite{mazumder2010spectral}) we use is $O((m+n)r^2)$, where $m$ and $n$ are the dimensions and $r$ is the rank of the underlying ground-truth matrix. This means the LRMC model has a constant time complexity regardless of the number of observations. BHM is a logistic regression based method and has linear time complexity in terms of the number of observations. 

\begin{figure} 
  \centering
  	{\label{fig:us_acc_time_curve}\includegraphics[width=0.99\linewidth]{figs/us_acc_time_curve.png}}
  \caption{\small \textit{Left}: Statistical TPA performance and \textit{Right}: CT performance per value of $C$ for LRMC and BHM under NO pattern from 20 different mazes~(with different linear dependencies but the same rank). 
  } \vspace{-10pt}
\label{fig:us_acc_time_curve}  
\end{figure}

\begin{figure}%[t]
  \centering
  	{\label{fig:us_ranks_curve}\includegraphics[width=0.99\linewidth]{figs/us_ranks_curve.png}}
  \caption{\small \textit{Left}: TPA performance and \textit{Right}: CT performance per value of $C$ of both methods under NO pattern on different maps with varying values of rank.
  } \vspace{-10pt}
\label{fig:us_ranks_curve}  
\end{figure}


\begin{figure}%[t] 
  \centering
  	{\label{fig:bus_acc_time_curve}\includegraphics[width=0.99\linewidth]{figs/bus_acc_time_curve.png}}
  \caption{\small \textit{Left}: Statistical TPA performance and \textit{Right}: CT performance per value of $C$ for LRMC and BHM under PO pattern from 20 different mazes~(with different linear dependencies but the same rank).  
  } \vspace{-10pt}
\label{fig:bus_acc_time_curve}  
\end{figure}

\begin{figure}%[t]
  \centering
  	{\label{fig:bus_ranks_curve}\includegraphics[width=0.99\linewidth]{figs/bus_ranks_curve.png}}
  \caption{\small \textit{Left}: TPA performance and \textit{Right}: CT performance per value of $C$ of both methods under PO pattern on different maps with varying values of rank.
  } \vspace{-10pt}
\label{fig:bus_ranks_curve}  
\end{figure}

Note however, from Fig.~\ref{fig:us_acc_time_curve}-\textit{Right} we can see that the LRMC model has a better CT performance than constant time! The reason for this is the LRMC algorithm we use relies on SVD of the matrix. SVD is the most efficient for  dense matrices but not that efficient for sparse ones. In our case,   at first the number of observations can be small so the input matrix is a sparse one. 
The runtime benefit will actually  increase with more observations received, which is a significant feature for real-time robot mapping applications.  

In Fig.~\ref{fig:us_ranks_curve}, we compare the TPA and CT for environments with different ranks. It is clear to see in Fig.~\ref{fig:us_ranks_curve}-\textit{Left} that when the environment becomes more complex~(as the rank value increases), the accuracy of BHM decreases while LRMC maintains the same high-level prediction~(almost $100\%$). From the first row of Fig.~\ref{fig:road_comparison}, we can find BHM generally performs well but many minor discrepancies can be found along the road edges.  When the environment becomes complex, the number of edges increases and thus the minor errors accumulate and this results in the increase of the total map prediction error. The time performances~(Fig.~\ref{fig:us_ranks_curve}-\textit{Right}) of the two methods for different rank-valued maps show the same trend as in Fig.~\ref{fig:us_acc_time_curve}-\textit{Right}.

We then consider the results under PO pattern. Similar to what we have in the NO pattern, the TPA and CT results are shown in Fig.~\ref{fig:bus_acc_time_curve} and Fig.~\ref{fig:bus_ranks_curve}, respectively. The comparison results follow similar trends to the NO pattern. 

\subsection{Coverage Planning with Real-Time Map Prediction}

We investigate if the proposed map prediction can facilitate and improve existing environmental mapping and coverage planning methods. 
Our evaluations are based on three most prevalent strategies. 
The first planning method is the lawnmower (LM) planning: a simple but the most widely used coverage planning method in environmental surveying and monitoring.  Specifically, the traversal patterns and resolutions are pre-determined such that the whole area could be swept incrementally by the robot.
The second planning is the myopic planning (MP) which is typically based on greedy choices (or best-first actions). Here at each action step, a set of points are randomly sampled in all unexplored areas and the closest one to the robot is selected as the local goal. 
The third planning is the non-myopic customized $TSP\_\epsilon$ as described in Section~\ref{sect:plan}, 
where $\epsilon \in \left \{ 0.25,~0.5,~0.75 \right \}$.

\begin{figure*} \vspace{-3pt}
  \centering
%   \subfigure[]
  	{\label{fig:road_tsp}\includegraphics[width=0.9\linewidth]{figs/road_tsp.png}}
  \caption{\small Environmental mapping of an urban road map using TSP\_{0.5}. The path is calculated by using an adaptive \textit{k}-opt TSP planner.
  In each of (a), (b), (c) and (d),  the \textit{Left} figure denotes the observed map while \textit{Right} is the predicted map using LRMC model, respectively. 
  } \vspace{-10pt}
\label{fig:road_tsp}  
\end{figure*}
% \begin{figure*} %\vspace{-3pt}
%   \centering
% %   \subfigure[]
%   	{\label{fig:terrain_tsp}\includegraphics[width=0.95\linewidth]{figs/terrain_tsp.png}}
%   \caption{\small {\color{blue} Environmental mapping of an terrain map using TSP\_{0.5}. The path~(yellow lines) is calculated by using an adaptive \textit{k}-opt TSP planner.
%   In each of (a), (b), (c) and (d),  the \textit{Left} figure denotes the observed map while \textit{Right} is the predicted map using LRMC model, respectively. }
%   } \vspace{-10pt}
% \label{fig:terrain_tsp}  
% \end{figure*}

\begin{figure}[t] 
  \centering
%   \subfigure[]
  	{\label{fig:coverage}\includegraphics[width=\linewidth]{figs/coverage.png}}
  \caption{\small Coverage convergence per action step for different coverage planning methods \textit{Left}: without real-time map prediction and \textit{Right}: with real-time map prediction.
  } \vspace{-10pt}
\label{fig:coverage}  
\end{figure}

The mapping process with non-myopic planning on the urban road network map is demonstrated in Fig.~\ref{fig:road_tsp}, where the real observed (sensed) map and corresponding predicted map from LRMC at four different action steps are presented. The results are consistent with that shown in Fig.~\ref{fig:road_comparison}.

To statistically evaluate the methods, we first consider the coverage planning with only raw observations but no map prediction. (As mentioned earlier, real-time predicted complete maps might not be available due to expensive computation cost in large environments.) The results are shown in Fig.~\ref{fig:coverage}-\textit{Left}, in which none of the listed methods could complete the map coverage within the given number of steps.
This also implies the necessity of real-time map prediction, 
and the results with map prediction are shown in Fig.~\ref{fig:coverage}-\textit{Right}. 
We can see that, 
within the same or less number of steps, nearly all of the planning methods could obtain a complete~(almost $100\%$) mapping coverage with real-time map prediction. 
In addition, the rate for coverage convergence is also remarkably improved for all of the listed coverage planning methods.

According to Fig.~\ref{fig:coverage}-\textit{Right}, among all the coverage planning methods, TSP\_$\epsilon$ methods significantly outperform other methods in terms of coverage convergence, although TSP based method can reach different completeness level as the value of $\epsilon$ varies. 
Empirically a larger value of $\epsilon$ can lead to more observations, a longer path~(and action steps), but also a better level of completeness (i.e., TSP\_{0.75}). A proper choice of $\epsilon$ value can give us a satisfactory trade-off between completeness and required action steps. 
Even with extremely restricted number of action steps~(i.e., less than 200), with a proper $\epsilon$, the TSP based methods can still achieve around $90\%$ coverage. The mapping results at different action steps by TSP\_{0.5} are shown in Fig.~\ref{fig:road_tsp}-(a), (b), (c) and (d), where we can see that our proposed method is able to perform map interpolation and extrapolation simultaneously on the fly during the execution of environmental mapping.

We also list the required number of action steps of all planning methods for different levels of completeness in Table~\ref{tb:coverage_comparison}. It is clear to see that with the proposed real-time LRMC map prediction all coverage planning methods can obtain a substantial improvement of coverage quality, and the required action steps at all listed completeness quantiles are remarkably reduced compared to the results without real-time map prediction.

\begin{table}\vspace{2pt}%[h]
\caption{\small Required number of action steps for varying levels of coverage completeness for different methods. The format for the entries indicates \textit{\# without real-time map prediction}~/~\textit{\# with real-time map prediction}, where `-' means the corresponding method fails to finish the respective completeness level.} %title of the table
\centering % centering table
{%\footnotesize
\begin{tabularx}{0.85\linewidth}{ccccc}%{c rrrrrrrrr} % creating eight columns
\hline\hline %inserting double-line
% Properties&\multicolumn{7}{c}{Sum of Extracted Bits} \\ [0.5ex]
   & 20\% & 50\%  & 90\% & 100\% \\[0.5ex]
\hline % inserts single-line
LM  & 153 / 69 & - / 176 & - / 320 &  - / 356~(99\%)\\ [1ex]% Entering row contents
MP & 238 / 38 & - / 155 & - / 566 & - / -\\ [1ex]% [1ex] adds vertical space
TSP\_0.25  & 124 / 37 & - / 70 & - / \textbf{151} & - / - \\ [1ex]
TSP\_0.5  & 127 / 19 & - / 101 & - / 195 & - / \textbf{251}~(98\%)\\ [1ex]
TSP\_0.75  & 120  / \textbf{16} & - / \textbf{61} &- / 240 & - / \textbf{309} \\
\hline % inserts single-line
\end{tabularx}
}
\label{tb:coverage_comparison} \vspace{-10pt}
\end{table}








