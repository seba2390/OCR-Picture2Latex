\section{EXPERIMENTS}

%Through extensive evaluations, we demonstrate that the LRMC based map prediction is highly efficient and can significantly facilitate the coverage convergence in environmental mapping. 
% 
To validate the proposed framework, we perform extensive simulation evaluations with a variety of complex maps. 

\subsection{Experimental Setups and Metrics Design}

To account for the sensor limitations as in robotics applications, we use two measurement patterns, \textit{Noisy Observation}~(NO) and \textit{Partial Observation}~(PO), for constructing the $\Omega$. NO means the observations are imperfect due to random noises (e.g., due to factors of environments/weathers such as snows, or sensor hardware limitations such as sparse and noisy measurements); %sampled in entire mapping space at random. 
PO means missed or uncovered regions which are typical in environmental mapping and monitoring scenarios. 
%Then we quantitatively show that LRMC outperforms BHM and is able to facilitate coverage mapping process in general maze maps, which model the map types that possess linear dependency among spatial dimensions. Particularly for those maps satisfying low-rank and incoherent structures, the LRMC model has been proven to recover a partially observed map with theoretical guarantee. 

To quantitatively compare the two methods,  we use two evaluation metrics, \textit{total prediction accuracy}~(TPA), and \textit{computational time}~(CT). TPA indicates the correct entry matches (including the observed entries) between the entire predicted map and the ground-truth map while CT measures the required time for map prediction computation. 
In addition, we also evaluate the \textit{mapping convergence rate} during the coverage planning.


\subsection{Demonstration with Road Network Mapping}
\begin{figure}
  \centering
%   \subfigure[]
  	{\label{fig:road_image}\includegraphics[width=\linewidth]{figs/road_image.png}}
  \caption{\small \textit{Left}: road network displays linear dependent structures.  \textit{Right}: The road network is extracted where 
  %to obtain a binary map that is similar to maze maps, although the noises make the map not be rigorously low-rank. 
  the dark color represents the road while others are off-road places.
  } 
\label{fig:road_image}  
\end{figure}

\begin{figure} %\vspace{-3pt}
  \centering
%   \subfigure[]
  	{\label{fig:road_comparison}\includegraphics[width=\linewidth]{figs/road_comparison.png}}
  \caption{\small Comparison between LRMC and BHM under NO~(first row) and PO~(second row) patterns, respectively. (a)\&(e)~Ground-truth maps. (b)\&(f)~Partially observed maps for NO and PO patterns. The missing parts are indicated in green patches. (c)\&(g)~Predicted maps from LRMC model. (d)\&(h)~Predicted maps from BHM.
  } \vspace{-10pt}
\label{fig:road_comparison}  
\end{figure}



We first demonstrate that our proposed LRMC method is able to perform map interpolation and extrapolation on a road network map extracted from an  urban environmental map, as shown in Fig.~\ref{fig:road_image}. 
The road surface can be detected by proximity sensors such as IR. We assume the robot is an aerial vehicle so that it can spot regions locally below it. 

The comparison of the two mapping methods on the urban road map under NO and PO patterns can be seen in Fig.~\ref{fig:road_comparison}. With NO pattern, both methods can generally predict the road network pattern~(topology), although BHM is more sensitive to the noises and thus fails to capture the exact road shape. On the other hand, with PO pattern, LRMC significantly outperforms BHM in the presence of multiple missing patches. 
Different from NO pattern, PO requires that the map prediction not only has interpolation capability, but also can perform slight extrapolation to reason the map structure within missing patches. 
LRMC leverages the linear dependency among the rows/columns of the map matrix and is able to predict the structure even multiple missing patches occur. In contrast, although BHM can capture local neighbouring spatial relationship by using kernelization, it is unable to capture the linear dependency brought by the low-rank structure, and thus performs worse than LRMC. 

%From the second row of Fig.~\ref{fig:road_comparison}, we can see our LRMC model is able to achieve this and maintain the same accuracy level as itself under NO pattern. In contrast, BHM suffers from missing patches and perform even worse than itself under NO pattern.

% The results from both methods are shown in Fig.~\ref{fig:road_comparison}, in which the ground-truth map is the same as Fig.~\ref{fig:road_comparison}-(a). Under NO pattern~(first row), we observe both methods can achieve an exact or close-to-exact recovery. However, under PO pattern~(second row), LRMC significantly outperforms BHM in terms of the prediction accuracy. LRMC leverages the linear dependency among the row/columns of the map matrix and is able to predict the structure even multiple missing patches occur. In contrast, although BHM has a capacity to capture local neighbouring spatial relationship by using kernelization, it is unable to capture the linear dependency brought by the low-rank structure, and thus performs worse than LRMC. 



%\footnote{\url{http://www.astrolog.org/labyrnth/daedalus.htm}}

% The goal of this paper is to demonstrate that a LRMC based map prediction is fast and efficient and can significantly facilitate the coverage convergence in urban environmental mapping. In this section, we first %show that the urban structured environments fit well into the LRMC model. Then we 
% conduct exhaustive comparison experiments between LRMC and BHM, under NO and PO, respectively; lastly we show that an outstanding coverage convergence rate improvement could be obtained if we equip commonly used coverage planning methods with our proposed LRMC real-time map prediction.

%\subsection{Assumptions Evaluation}

% To illustrate our considered structured environments satisfy the assumptions required by LRMC model, different maze maps with varying ranks are generated by Daedalus\footnote{\url{http://www.astrolog.org/labyrnth/daedalus.htm}} and are shown in Fig~\ref{fig:mazes_1}. Each map has a dimension of $20m \times 20m$ with a resolution of $0.1m$, which results in a matrix with dimension of $200\times 200$. The rank values are listed in Table~\ref{tb:maze_prop}. It is easy to observe that the value of rank basically reflects the complexity of the environment. The higher the rank is, the more complex the environment is. A typical urban building layout map~(see Fig.~\ref{fig:real}) has a rank below 100, which is significantly smaller compared to the original matrix dimension which could be multiple magnitudes higher (e.g., $10^4$ or above). 
% %would be similar to the maze with rank of 2, which is a very simple maze map. 
% %In this paper, we consider a more complicated environment that is similar to the maze with rank of 7~(as shown in Fig.~\ref{fig:mazes_1}-(b)). 
% %Compared with the dimension of 200, 7 is a much smaller value and therefore enables the corresponding matrix to be called as a low-rank matrix. Even the maze environment becomes super complex, such as the one shown in Fig.~\ref{fig:mazes_1}-(f), the rank value is still relatively small compared with the matrix dimension. 
% Along with the rank values, we also list the coherence value computed by using Eq.~(\ref{eq:coherence}) for each maze. Although the coherence increases proportionally to the rank, they all stay at a low-level where we can be convinced they all are incoherent. Hence we can conclude that maze environments generally possess the low-rank and low-coherence properties. 

% \begin{figure} \vspace{-3pt}
%   \centering
% %   \subfigure[]
%   	{\label{fig:mazes_1}\includegraphics[width=0.9\linewidth]{figs/mazes_with_ranks.png}}
%   \caption{\small Maze environments with different ranks. The value of 1.0 represents the featured space while 0.2 is non-featured space. Here we use 0 to represent missing parts, the color-map range from 0.2 (non-featured space) to 1.0 (featured space).
%   } \vspace{-10pt}
% \label{fig:mazes_1}  
% \end{figure}

% \begin{table}[t]
% \caption{Matric properties of maze environments} %title of the table
% \centering % centering table
% \begin{tabularx}{\linewidth}{YYYYYYY}%{c rrrrrrr} % creating eight columns
% \hline\hline %inserting double-line
% % Properties&\multicolumn{7}{c}{Sum of Extracted Bits} \\ [0.5ex]
%   & (a) & (b) & (c) & (d) & (e) & (f)\\
% \hline % inserts single-line
% Rank & 2 & 7 & 10 & 15 & 18 & 23\\ % Entering row contents
% Coherence & 0.0200 & 0.0400 & 0.0625 & 0.0833 & 0.1000 & 0.1250\\ % [1ex] adds vertical space
% \hline % inserts single-line
% \end{tabularx}
% \label{tb:maze_prop}
% \end{table}


\subsection{Quantitative Comparison between LRMC and BHM}
% We first compare two map prediction methods, LRMC and BHM, on a static partially observed map.  That is, the observation set is given at one-time for the whole map. To account for the sensor limitations as in robotics applications, we use two measurement patterns, \textit{Noisy Observation}~(NO) and \textit{Partial Observation}~(PO), for constructing the $\Omega$, respectively. NO means the observations are imperfect due to random noises (e.g., due to factors of environments/weathers such as snows, or sensor hardware limitations such as sparse and noisy measurements); %sampled in entire mapping space at random. 
% PO means missed or uncovered regions which are typical in environmental mapping and monitoring scenarios. 
%there will be multiple regions are missed by the robot due to the limited sensing capability and PO models this situation.

% The results from both methods are shown in Fig.~\ref{fig:demo}, in which the ground-truth map is the same as Fig.~\ref{fig:demo}-(a). Under NO pattern~(first row), we observe both methods can achieve an exact or close-to-exact recovery. However, under PO pattern~(second row), LRMC significantly outperforms BHM in terms of the prediction accuracy. LRMC leverages the linear dependency among the row/columns of the map matrix and is able to predict the structure even multiple missing patches occur. In contrast, although BHM has a capacity to capture local neighbouring spatial relationship by using kernelization, it is unable to capture the linear dependency brought by the low-rank structure, and thus performs worse than LRMC. 

%To further validate the effectiveness of the proposed LRMC map prediction method, 
% To quantitatively compare the two methods,  we use two evaluation metrics, \textit{total prediction accuracy}~(TPA), and \textit{computational time}~(CT). TPA indicates the correct entry matches (including the observed entries) between the entire predicted map and the ground-truth map while CT measures the required time for map prediction computation. 

We first generate 20 differing complex maze-like layouts/environments by using the Daedalus tool~\cite{daedalus}, and consider the TPA and CT statistical comparisons under NO and PO patterns for different values of $C$. 
%Then we generate %6 
%another set of differing environments with different rank values by Daedalus~\cite{daedalus} and again we apply the two map prediction methods and observe the trend of TPA and CT with different $C$. 
%The TPA and CT statistical comparisons under NO and PO patterns %for different values of $C$ 
The results are shown in Fig.~\ref{fig:us_acc_time_curve} and Fig.~\ref{fig:bus_acc_time_curve}. 
The trend of TPA and CT in different rank-valued maps %with different $C$ 
are shown in Fig.~\ref{fig:us_ranks_curve} for NO and Fig.~\ref{fig:bus_ranks_curve} for PO, respectively.

We first analyze the results for NO pattern. In Fig.~\ref{fig:us_acc_time_curve}-\textit{Left}, it can be observed that 
%BHM has a bit faster accuracy convergence~(per $C$) but 
LRMC could achieve a higher accuracy level. 
%The high accuracy is known as an extremely important metric to evaluate the mapping performance.
%especially in mapping scenarios where the robot is subject to collision due to its limited distance to obstacles. %, although no significant difference in terms of accuracy can be found for the two methods. 
From Fig.~\ref{fig:us_acc_time_curve}-\textit{Left}, we want to determine a proper scale of the coefficient $C$. We can think the coefficient $C$ is actually reflecting the number of observed map samples according to Eq.~(\ref{eq:sampled_cond}). 
%The higher the coefficient is, the more the observations are and the better the prediction is. 
In Fig.~\ref{fig:us_acc_time_curve}-\textit{Left}, we can see that TPA could reach a high level and stay stable when $C\geq 1.5$.
% below is too engineering and trivial, so removed.
%In real applications, the random sensing may show varying patterns each time and to be robust to this randomness, we select a value of 2.0 as a reference value for $C$ to achieve a desired prediction in the environments with a rank value of 7~(this is also reflected under PO pattern as shown in Fig.~\ref{fig:bus_acc_time_curve}-\textit{Left} later). 
We then compare the CT performance as shown in Fig.~\ref{fig:us_acc_time_curve}-\textit{Right}. The result clearly distinguishes the two methods. The required computation time for LRMC is much lower than that of BHM. Given the size of the entire map~(matrix), the computational complexity of the LRMC algorithm~(SoftImpute~\cite{mazumder2010spectral}) we use is $O((m+n)r^2)$, where $m$ and $n$ are the dimensions and $r$ is the rank of the underlying ground-truth matrix. This means the LRMC model has a constant time complexity regardless of the number of observations. BHM is a logistic regression based method and has linear time complexity in terms of the number of observations. 
Note however, from Fig.~\ref{fig:us_acc_time_curve}-\textit{Right} we can see that the LRMC model has a better CT performance than constant time! The reason for this is the LRMC algorithm we use relies on SVD of the matrix. SVD is the most efficient for  dense matrices but not that efficient for sparse ones. In our case,   at first the number of observations can be small so the input matrix is a sparse one. 
The runtime benefit will actually  increase with more observations received, which is a significant feature for real-time robot mapping applications.  
%and SVD may use some tricks to deal with the sparse matrix(i.e., fill in the missing parts with some well designed values before doing the decomposition). However, when the number of samples increases, the input matrix will be a dense one and the computation time for SVD decreases. 

% \begin{figure} %\vspace{-3pt}
%   \centering
% %   \subfigure[]
%   	{\label{fig:demo}\includegraphics[width=\linewidth]{figs/new_demo.png}}
%   \caption{\small First row: Map completion with \textit{Noisy Observation}. (a)~Ground-truth maze map. (b)~Partially observed map. The missing parts are indicated by green color (has value of $0.5$). (c)~The predicted map using LRMC model. (d)~The predicted map using BHM. Second row: Map completion with \textit{Partial Observations}. Note that color map is shared between subplots.
%   } \vspace{-10pt}
% \label{fig:demo}  
% \end{figure}

\begin{figure} 
  \centering
  	{\label{fig:us_acc_time_curve}\includegraphics[width=0.99\linewidth]{figs/us_acc_time_curve.png}}
  \caption{\small \textit{Left}: Statistical TPA performance and \textit{Right}: CT performance per value of $C$ for LRMC and BHM under NO pattern from 20 different mazes~(with different linear dependencies but the same rank). 
  } \vspace{-10pt}
\label{fig:us_acc_time_curve}  
\end{figure}

\begin{figure}%[t]
  \centering
  	{\label{fig:us_ranks_curve}\includegraphics[width=0.99\linewidth]{figs/us_ranks_curve.png}}
  \caption{\small \textit{Left}: TPA performance and \textit{Right}: CT performance per value of $C$ of both methods under NO pattern on different maps with varying values of rank.
  } \vspace{-10pt}
\label{fig:us_ranks_curve}  
\end{figure}


\begin{figure}%[t] 
  \centering
  	{\label{fig:bus_acc_time_curve}\includegraphics[width=0.99\linewidth]{figs/bus_acc_time_curve.png}}
  \caption{\small \textit{Left}: Statistical TPA performance and \textit{Right}: CT performance per value of $C$ for LRMC and BHM under PO pattern from 20 different mazes~(with different linear dependencies but the same rank).  
  } \vspace{-10pt}
\label{fig:bus_acc_time_curve}  
\end{figure}

\begin{figure}%[t]
  \centering
  	{\label{fig:bus_ranks_curve}\includegraphics[width=0.99\linewidth]{figs/bus_ranks_curve.png}}
  \caption{\small \textit{Left}: TPA performance and \textit{Right}: CT performance per value of $C$ of both methods under PO pattern on different maps with varying values of rank.
  } \vspace{-10pt}
\label{fig:bus_ranks_curve}  
\end{figure}



In Fig.~\ref{fig:us_ranks_curve}, we compare the TPA and CT for environments with different ranks. It is clear to see in Fig.~\ref{fig:us_ranks_curve}-\textit{Left} that when the environment becomes more complex~(as the rank value increases), the accuracy of BHM decreases while LRMC maintains the same high-level prediction~(almost $100\%$). From the first row of Fig.~\ref{fig:road_comparison}, we can find BHM generally performs well but many minor discrepancies can be found along the road edges.  When the environment becomes complex, the number of edges increases and thus the minor errors accumulate and this results in the increase of the total map prediction error. The time performances~(Fig.~\ref{fig:us_ranks_curve}-\textit{Right}) of the two methods for different rank-valued maps show the same trend as in Fig.~\ref{fig:us_acc_time_curve}-\textit{Right}.

We then consider the results under PO pattern. Similar to what we have in the NO pattern, the TPA and CT results are shown in Fig.~\ref{fig:bus_acc_time_curve} and Fig.~\ref{fig:bus_ranks_curve}, respectively. The comparison results follow similar trends to the NO pattern. 

% According to above discussions, we can conclude that under the experimental setup (map size, resolution, hinge points number, etc.) in this paper, although both methods can achieve a similar map prediction accuracy, BHM is incompetent in real-time applications but LRMC is able to maintain and update the global map very quickly.

% \begin{figure*}%[h!]
%   \centering
%     \includegraphics[width=\linewidth]{figs/tsp_results.png}
%   \caption{\small (a)~Ground-truth map for coverage planning. (b)~\textit{Left}: Real sensed map and \textit{Right}: predicted map in the middle of mapping with TSP\_{0.5}. (c)~\textit{Left}: Real sensed map and \textit{Right}: predicted map towards the ending of mapping with TSP\_{0.5}. }
%   \label{fig:tsp_results}
% \end{figure*}

\subsection{Coverage Planning with Real-Time Map Prediction}

We investigate if the proposed map prediction can facilitate and improve existing environmental mapping and coverage planning methods. 
Our evaluations are based on three most prevalent strategies. 
The first planning method is the lawnmower (LM) planning: a simple but the most widely used coverage planning method in environmental surveying and monitoring.  Specifically, the traversal patterns and resolutions are pre-determined such that the whole area could be swept incrementally by the robot.
The second planning is the myopic planning (MP) which is typically based on greedy choices (or best-first actions). Here at each action step, a set of points are randomly sampled in all unexplored areas and the closest one to the robot is selected as the local goal. 
%and a series of way-points are subsequently generated correspondingly. %The robot only takes the first way-point for execution.
The third planning is the non-myopic customized $TSP\_\epsilon$ as described in Section~\ref{sect:plan}, 
%We compare the coverage convergence performance~(per action step) of \textit{LM}, \textit{MP} and our proposed non-myopic planning --- $TSP\_\epsilon$, 
where $\epsilon \in \left \{ 0.25,~0.5,~0.75 \right \}$. %, on the same underlying maps.   Fig.~\ref{fig:tsp_results}-(a) demonstrates a scenario. The results of coverage convergence for different methods are shown in Fig.~\ref{fig:coverage}. 

%I mean only raw measurements, no post-processing(map prediction or map regression) is performed. Here I want to compare the environment modeling results with and without real-time map prediction and emphasize the use of real-time map prediction can really speedup the coverage convergence.




\begin{figure*} %\vspace{-3pt}
  \centering
%   \subfigure[]
  	{\label{fig:road_tsp}\includegraphics[width=\linewidth]{figs/road_tsp.png}}
  \caption{\small Environmental mapping of an urban road map using TSP\_{0.5}. The path~(yellow lines) is calculated by using an adaptive \textit{k}-opt TSP planner. %The mapping results at one quarter, half, three quarters, and ending action step are shown in (a), (b), (c) and (d), 
  In each of (a), (b), (c) and (d),  the \textit{Left} figure denotes the observed map while \textit{Right} is the predicted map using LRMC model, respectively. 
  } \vspace{-10pt}
\label{fig:road_tsp}  
\end{figure*}

\begin{figure}[t] 
  \centering
%   \subfigure[]
  	{\label{fig:coverage}\includegraphics[width=\linewidth]{figs/coverage.png}}
  \caption{\small Coverage convergence per action step for different coverage planning methods \textit{Left}: without real-time map prediction and \textit{Right}: with real-time map prediction.
  } \vspace{-10pt}
\label{fig:coverage}  
\end{figure}

%We also show the advantage brought by LRMC to environmental modeling. 
The mapping process with non-myopic planning on the urban road network map is demonstrated in Fig.~\ref{fig:road_tsp}, where the real observed (sensed) map and corresponding predicted map from LRMC at four different action steps are presented. The results are consistent with that shown in Fig.~\ref{fig:road_comparison}. %The LRMC model can perform interpolation and extrapolation simultaneously during the execution of the mapping process and this could substantially accelerate coverage mapping progress and might be beneficial to decision makings for potential online adaptive planning.

To statistically evaluate the methods, we first consider the coverage planning with only raw observations but no map prediction. (As mentioned earlier, real-time predicted complete maps might not be available due to expensive computation cost in large environments.) The results are shown in Fig.~\ref{fig:coverage}-\textit{Left}, in which none of the listed methods could complete the map coverage within the given number of steps.
This also implies the necessity of real-time map prediction, 
%One may argue that some off-line methods can perform the prediction after the mission finishes. However, off-line methods lack flexibility and adaptivity to facilitate online decision making.
%One may argue that we can perform map prediction~(or regression) on the incomplete map after the mission finishes. This off-line map prediction does work to help human obtain a finer-resolution feature map, but it does also have problems if we want to monitor the space online or the planning algorithm require an online updated fine-grain map to make decisions in real-time. 
%Therefore, we have to consider to aid coverage planning with real-time map prediction, 
and the results with map prediction are shown in Fig.~\ref{fig:coverage}-\textit{Right}. 
We can see that, 
within the same or less number of steps, nearly all of the planning methods could obtain a complete~(almost $100\%$) mapping coverage with real-time map prediction. 
In addition, the rate for coverage convergence is also remarkably improved for all of the listed coverage planning methods. %compared to themselves without real-time map prediction. 
%We also claim that our proposed LRMC map prediction could be easily plugged with other planning methods that are not listed in this paper and facilitate their environmental mapping performance.

According to Fig.~\ref{fig:coverage}-\textit{Right}, among all the coverage planning methods, TSP\_$\epsilon$ methods significantly outperform other methods in terms of coverage convergence, although TSP based method can reach different completeness level as the value of $\epsilon$ varies. 
Empirically a larger value of $\epsilon$ can lead to more observations, a longer path~(and action steps), but also a better level of completeness (i.e., TSP\_{0.75}). A proper choice of $\epsilon$ value can give us a satisfactory trade-off between completeness and required action steps. 
%For example, if we have enough energy supply and pursue a well completed map, we should select a big value of $\epsilon$, such as 0.75. Otherwise, we can have a moderate magnitude $\epsilon$, like 0.25 to achieve an acceptable coverage ratio~(i.e., 90\%) if we have limited energy or constrained allowed action steps. 
Even with extremely restricted number of action steps~(i.e., less than 200), with a proper $\epsilon$, the TSP based methods can still achieve around $90\%$ coverage. The mapping results at different action steps by TSP\_{0.5} are shown in Fig.~\ref{fig:road_tsp}-(a), (b), (c) and (d), where we can see that our proposed method is able to perform map interpolation and extrapolation simultaneously on the fly during the execution of environmental mapping.

We also list the required number of action steps of all planning methods for different levels of completeness in Table~\ref{tb:coverage_comparison}. It is clear to see that with the proposed real-time LRMC map prediction all coverage planning methods can obtain a substantial improvement of coverage quality, and the required action steps at all listed completeness quantiles are remarkably reduced compared to the results without real-time map prediction.

\begin{table}\vspace{2pt}%[h]
\caption{\small Required number of action steps for varying levels of coverage completeness for different methods. The format for the entries indicates \textit{\# without real-time map prediction}~/~\textit{\# with real-time map prediction}, where `-' means the corresponding method fails to finish the respective completeness level.} %title of the table
\centering % centering table
{%\footnotesize
\begin{tabularx}{0.85\linewidth}{ccccc}%{c rrrrrrrrr} % creating eight columns
\hline\hline %inserting double-line
% Properties&\multicolumn{7}{c}{Sum of Extracted Bits} \\ [0.5ex]
   & 20\% & 50\%  & 90\% & 100\% \\[0.5ex]
\hline % inserts single-line
LM  & 153 / 69 & - / 176 & - / 320 &  - / 356~(99\%)\\ [1ex]% Entering row contents
MP & 238 / 38 & - / 155 & - / 566 & - / -\\ [1ex]% [1ex] adds vertical space
TSP\_0.25  & 124 / 37 & - / 70 & - / \textbf{151} & - / - \\ [1ex]
TSP\_0.5  & 127 / 19 & - / 101 & - / 195 & - / \textbf{251}~(98\%)\\ [1ex]
TSP\_0.75  & 120  / \textbf{16} & - / \textbf{61} &- / 240 & - / \textbf{309} \\
\hline % inserts single-line
\end{tabularx}
}
\label{tb:coverage_comparison} \vspace{-10pt}
\end{table}








