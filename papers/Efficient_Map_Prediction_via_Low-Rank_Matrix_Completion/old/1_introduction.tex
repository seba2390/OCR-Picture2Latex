\section{INTRODUCTION}
One critical task for autonomous vehicles is coverage planning, which aims to cover/monitor a certain area of interest to obtain a feature map~(e.g., terrain map, building layout map, and chemical concentration map) by leveraging the flexible mobility of the robotic platform and the sensing capability of the sensors equipped on the vehicle. There have been intensive applications of coverage planning, including floor cleaning~\cite{oh2004complete, palacin2004building}, agriculture~\cite{hameed2013optimized, zhou2014agricultural, jin2010optimal}, structure inspection~\cite{englot2012sampling, ellefsen2017multiobjective}, terrain reconstruction~\cite{krotkov1994terrain, torres2016coverage}, oil spills cleaning~\cite{song2013adaptive}, and environmental monitoring~\cite{he2019autonomous}. Aerial vehicles become particularly popular for performing coverage planning missions due to its high agility, low cost and unique down-facing field of view from the air. However, aerial robots notoriously suffer from the limited on-board resources (e.g, power supply), which may result in a partial coverage for a single aerial vehicle if the region to be explored is relatively large. The vast majority of prior works assume the robot is energy-free when considering the feature map coverage problem. Only recent few works consider energy consumption limitations~\cite{di2016coverage, modares2017ub, cabreira2018energy} and consequent partial coverage~\cite{ellefsen2017multiobjective, papachristos2016distributed, jensen2020near}. Although~\cite{di2016coverage, modares2017ub, cabreira2018energy} consider to optimize coverage path with respect to power assumption, most of them still require the robot has enough energy to achieve resolution-complete coverage. Given limited power/flight time/path length, while \cite{di2016coverage, modares2017ub, cabreira2018energy} optimize the path to achieve a trade-off between their own limitations and coverage, they end their jobs with an incomplete feature map and fail to mention how to leverage the partially revealed information to predict the whole map. In this paper, we propose to use Low-Rank Matrix Completion~(LRMC) to leverage the partially observed map information to predict the whole feature map provided that we are dealing with a scenario where a single aerial vehicle is assigned to perform an urban building layout monitoring task~(where we assume no trees or other objects occluding the buildings) and the given power might be only enough to conduct a partial coverage planning. We carefully illustrate how urban environment terrain monitoring task fits into the LRMC model and demonstrate that partially revealed information can still provide sufficient clues to predict the uncovered regions if the underlying feature map possesses some intrinsic structure(e.g., low-rank and incoherent structures) in intensive experiments with two local sensing modes: \textit{perfect local sensing} and \textit{degraded local sensing}.

We summarize our contributions as follows:
\begin{itemize}
    \item 
    We are the first to examine and illustrate that maze-like environments possess low-rank and incoherent structure.
    \item
    We are the first to apply Low-Rank Matrix Completion to predict the partially observed maze environments.
    \item
    We are the first to combine Low-Rank Matrix Completion with TSP to significantly enhance the energy-constrained coverage planning.
    \item
    We provide intensive simulated experiments and demonstrate the outstanding effectiveness of our proposed method compared with other alternate coverage planning solutions. We emphasize that our method is particularly suited with real scenarios, where local sensed map may contain many missing parts due to the sensor limitations.
\end{itemize}
