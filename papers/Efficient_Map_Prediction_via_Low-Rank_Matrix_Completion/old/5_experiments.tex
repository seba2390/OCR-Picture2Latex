\section{EXPERIMENTS}
The goal of this paper is to demonstrate that a LRMC based map predictive coverage planning is useful and necessary for coverage scenarios in maze-like environments, particularly when the sensory raw measurements are noisy and sparse and the given energy is quite limited. We have shown that the maze-like environments fit well into the LRMC model. We want to understand in a further step how the model behaves in the context of coverage planning, instead of static discrete sampling as shown in Fig.~\ref{fig:demo}, and how our proposed method performs comparing with other alternative existing coverage planning techniques. We firstly describe our experimental setups and introduce all the methods involved in experiments and then conduct exhaustive analysis of coverage behaviors for different baseline methods and our proposed method with two sensing modes: PLS and DLS, respectively.

\begin{figure*}%[h!]
  \centering
  \begin{subfigure}[b]{0.19\linewidth}
    \includegraphics[width=\linewidth]{figs/map_truth.png}
    \caption{}
  \end{subfigure}
  \begin{subfigure}[b]{0.39\linewidth}
    \includegraphics[width=\linewidth]{figs/pls_1.0.png}
    \caption{}
  \end{subfigure}
  \begin{subfigure}[b]{0.39\linewidth}
    \includegraphics[width=\linewidth]{figs/dls_1.0.png}
    \caption{}
  \end{subfigure}
  \caption{\small (a)~Ground-truth maze map. (b)\textit{Left}: Real sensed map by TSP\_$1.0$ with PLS mode. The red points are random uniform sampled points while the yellow poly-lines are the shortest path found by TSP. The green blocks represent the missing parts in the map. \textit{Right}: Predicted map using the LRMC model. (c)\textit{Left}: Real sensed map by TSP\_$1.0$ with DLS mode. The color legends have the same meaning with the ones in (b). \textit{Right}: Predicted map using the LRMC model.}
  \label{fig:tsp_results}
\end{figure*}

\begin{table*}%[h]
\caption{Coverage performance comparison} %title of the table
\centering % centering table
\begin{tabular}{c rrrrrrrrr} % creating eight columns
\hline\hline %inserting double-line
% Properties&\multicolumn{7}{c}{Sum of Extracted Bits} \\ [0.5ex]
  & \vline & 20\% & 40\% & 50\% & 60\% & 80\% & 90\% & 100\% \\
\hline % inserts single-line
LM & \vline & 70 / 81 & 142 / 170 & 177 / 214 & 214 / 255 & 286 / 342 & 324 / - &  358 / -\\ [1ex]% Entering row contents
GR & \vline & 74 / 89 & 136 / 196 & 212 / 273 & 254 / 357 & 367 / - & 450 / - & - / -\\ [1ex]% [1ex] adds vertical space
FF & \vline & 85 / - & 166 / - & 205 / - & 257 / - & 365 / - & 418 / - & 500 / - \\ [1ex]
TSP\_0.75 & \vline & 34 / \textbf{29} & 70 / 53 & 86 / \textbf{68} & 105 / \textbf{86} & 155 / 131 & 176 / 146 & - / - \\ [1ex]
TSP\_1.0 & \vline & \textbf{32} / 33 & \textbf{60} / \textbf{68} & \textbf{61} / 86 & \textbf{71} / 105 & 143 / \textbf{141} & 167 / \textbf{157} & - / - \\ [1ex]
TSP\_1.25 & \vline & 34  / 32 & 70 / 67 & 84 / 83 & 114 / 101 & \textbf{137} / 156 & \textbf{165} / 180 & \textbf{241} / \textbf{235} \\
\hline % inserts single-line
\end{tabular}
\label{tb:coverage_comparison}
\end{table*}

\subsection{Experimental Setup}
We conduct our experiments based on the maze map as shown in Fig.~\ref{fig:tsp_results}-(a), which has a rank of 11. In this paper, we assume our vehicle for performing coverage mission is a point robot and no dynamics is considered. In Section.~\ref{sec:map_completion}, we have shown with a proper choice of $C$, the offline static map completion performance is outstanding and we also conclude there that a reference value for $C$ could be set to 2.0. However, in coverage planning, the observation entries are incrementally revealed since the robot needs to take time to traverse the area and only the places the robot has traversed could become observed entries in the matrix. To find a path for robot to traverse while following the random uniform observation pattern in the LRMC model, we first uniformly sample a set of points, then the sampled points are post-processed by an existing Traveling Salesman Problem~(TSP) solver to find a shortest path connecting all the sampled points. At first glance, the number of sampled points should be identical to the number of sampled observation blocks in the example in Section.~\ref{sec:map_completion}, where the samples number is defined as:
\begin{equation}
    \label{eq:sample_num}
    N^{'}_{s} = \left \lfloor \frac{C\cdot n^{1.2}r\log n}{N} \right \rfloor
\end{equation}
where $N^{'}_{s}$ represents the samples number and $N$ is the number of observation entries within the robot local sensing range. Each sampled point will become a sampled block when the robot arrives at there. Nevertheless, in the coverage planning scenario, the robot can take observations in between a pair of connected sampled points and may introduce additional observations. This implies we may want to reduce the number of sampled points and we do this by introducing a scale factor $\epsilon$ to Eq.~(\ref{eq:sample_num}):
\begin{equation}
    \label{eq:new_sample_num}
    N_{s} = \left \lfloor \epsilon N^{'}_{s} \right \rfloor = \left \lfloor \epsilon \frac{C\cdot n^{1.2}r\log n}{N} \right \rfloor
\end{equation}
where $N_{s}$ is the new samples number and here we fix the value of $C$ as 2.0.

Once the points are sampled using Eq.~(\ref{eq:new_sample_num}), a shortest path will be formed using a TSP solver and thereafter our robot could start the coverage mission by following the path.

\begin{figure}%[h!]
  \centering
  \begin{subfigure}[b]{0.49\linewidth}
    \includegraphics[width=\linewidth]{figs/pls_coverage.png}
    \caption{}
  \end{subfigure}
  \begin{subfigure}[b]{0.49\linewidth}
    \includegraphics[width=\linewidth]{figs/dls_coverage.png}
    \caption{}
  \end{subfigure}
  \caption{\small Coverage convergence of different methods for (a)~PLS mode and (b)~DLS mode.}
  \label{fig:coverage}
\end{figure}

\subsection{Introduction to Experimental Methods}
We now describe the baseline methods and our proposed method with different choices of parameters. The baseline methods we want to beat include: \textit{Lawn-Mower}~(LM) based coverage planning, \textit{Global Random Sampling}~(GRS) based coverage planning and \textit{Feature Frontier}~(FF) based coverage planning. 

\textit{LM}: A simple but widely used classical coverage planning method. Given the size of the mapping area, a path with carefully designed interval is computed such that the whole area could be swept incrementally by the robot.

\textit{GRS}: A global random sampling based method, in which at each action step, a set of points are randomly sampled in all unexplored areas and the closest one to the robot is selected as the local goal and a series of way-points are generated correspondingly. The robot only takes the first way-point for execution.

\textit{FF}: A feature frontier based method. The feature frontier here is defined as the intersections between the current explored areas boundary and the featured spaces. The featured spaces are defined as the places identified by an aerial robot from the air as having buildings. %This method tends to lead the robot to wander above the featured spaces and can easily stuck in local space if the local sensing is sparse and noisy due to the super sensitivity of the frontiers to noises.

Our proposed map predictive coverage planning method is based on TSP and hence we name our proposed method with different $\epsilon$ values as: $TSP\_\epsilon$, where the $\epsilon$ is the scale factor in Eq.~(\ref{eq:new_sample_num}) and will be replaced with some specific values in experiments.

\subsection{Coverage Quality Analysis}
We compare the coverage convergence performance~(per action step) of \textit{LM}, \textit{GRS}, \textit{FF} and $TSP\_\epsilon$, where $\epsilon \in \left \{ 0.25,~0.5,~0.75,~1.0,~1.25,~1.5,~1.75 \right \}$, for two local sensing modes, PLS and DLS on the same underlying maze map~(as shown in Fig.~\ref{fig:tsp_results}-(a)), respectively. The results are shown in Fig.~\ref{fig:coverage}. 

We first consider the coverage planning under PLS mode, where we assume the sensed local map for each action step is perfect, and the results are shown in Fig~\ref{fig:coverage}-(a). We can see that due to the functionality of prediction, our proposed TSP\_$\epsilon$ methods significantly outperform all the baseline methods in terms of coverage convergence, although TSP based method can reach different completeness level as the value of $\epsilon$ varies. Empirically a larger value of $\epsilon$ can lead to more sample points, a longer path~(and action step) from TSP, but as well as a better completeness~(i.e., TSP\_{1.5} and TSP\_(1.75)). A proper choice of $\epsilon$ value can give us a satisfactory trade-off between completeness and required action steps. For example, if we have enough energy supply and pursuit a well completed feature map, we should select a big value of $\epsilon$, such as 1.5 and 1.75. On the other hand, we can have a moderate magnitude $\epsilon$, like 0.75, 1.0 and 1.25 to achieve an acceptable coverage ratio~(i.e., 90\%), if we have limited energy or constrained allowed action steps. Even with extremely restricted number of action steps~(i.e., 110, 130, and 180), with a proper $\epsilon$, our TSP based method~(i.e., TSP\_{0.25}, TSP\_{0.5}, and TSP\_{0.75}) can still achieve $80\%(\pm 10\%)$ coverage. Although all baseline methods~(LM, GRS, and FF) can achieve good completeness, they require excessive action steps~(357, 603+, 500 for achieving 100\%~(or almost) coverage). If the maximum allowed number of actions is set as 130, none of the baseline methods could achieve a coverage that is more than 40\%.

In addition to the PLS mode, we also consider the coverage performance under DLS mode, which resembles the sensing quality in real hardware. The results are shown in Fig~\ref{fig:coverage}-(b). An exception for baseline methods is FF fails to perform the coverage mission because the detection of feature frontier is sensitive to the noises and heavily depends on map quality. In DLS mode, the obtained map contains many missing parts due to the sensor limits and noise. The robot is stuck in local area since there are always some feature frontiers are detected~(suppose the robot is within some places having featured spaces). Solving this problem may involve more robust frontier detection tricks, which are out of the scope of this paper. Other two baselines, LM and GRS, suffer from the bad sensing quality and have much slower convergence than themselves in PLS mode. On the contrary, our proposed TSP\_$\epsilon$ method is naturally robust to missing-ness and keep the same convergence performance as in PLS mode. We show the feature maps obtained at ending step of TSP\_{1.0} for PLS and DLS in Fig.~\ref{fig:tsp_results}-(b) and Fig.~\ref{fig:tsp_results}-(c), respectively.

We also list the required number of action steps of all baseline methods and our proposed TSP based method with three mild $\epsilon$ values for different levels of completeness in Table~\ref{tb:coverage_comparison}. The format for the entries in Table~\ref{tb:coverage_comparison} is $\#_{PLS} / \#_{DLS}$, which presents the required action numbers for PLS and DLS. It is clear to see that our TSP\_$\epsilon$ can achieve remarkable reduction for required action steps at all listed completeness quantiles.








