\section{RELATED WORK}
Our work on map prediction based coverage planning is related to classical coverage planning and map structure prediction. In classical coverage planning, the most efforts are made on development of planning algorithms, while quite few works consider the potential benefits from combining map prediction. However, most of the current map prediction techniques rely on deep learning~\cite{caley2019deep, pronobis2017learning, katyal2018occupancy, shrestha2019learned, saroya2020online}, requiring extremely large datasets to achieve a reasonable results in certain types of environments. We want to avoid the data-demanding deep learning based methods and leverage Low-Rank Matrix Completion to enable the functionality of map prediction. We survey related efforts in those directions. 

\textbf{Classic Coverage Planning.} Classic approaches for coverage planning mainly focus on planning algorithms and seek to achieve a resolution-complete coverage. Many factors are considered in planning to improve the coverage performance, such as non-holonomic mobility constraints~\cite{kan2020online}, submodularity-based multi-agent network~\cite{sun2017submodularity}, and hexagonal cell decomposition~\cite{paull2012sensor}. Most of prior works~\cite{oh2004complete}-\cite{he2019autonomous},~\cite{kan2020online}-\cite{paull2012sensor} assume that the robot has enough power to complete the mission. Some recent works~\cite{ellefsen2017multiobjective},~\cite{di2016coverage}-\cite{cabreira2018energy}, \cite{papachristos2016distributed}-\cite{jensen2020near},~\cite{li2019coverage} start to consider the coverage planning under the energy constraints. While they take the power constraints into consideration, they lack solutions for the situations where the mission ends up with an incomplete feature map. Note that none of all prior coverage planning works consider to use feature map prediction to enhance the coverage planning.

\textbf{Map Prediction.} Map prediction aims to predict map structure based on already revealed regions. Most of the current map structure predictions base their methods on deep neural networks (DNNs), which require massive training data. In~\cite{caley2019deep}, a database of building blueprints is used to train a DNN to predict exit locations of buildings. A single and universal model of the robot's spatial environment is learned by combining Sum-Product Networks~(SPNs) and deep learning in~\cite{pronobis2017learning}. An occupancy map representations of sensor data is predicted for future robot motions using DNNs in~\cite{katyal2018occupancy}. A state-of-the-art generative DNN is used in~\cite{shrestha2019learned} to predict unknown regions of a partially explored map and then the prediction is used to enhance the exploration in an information-theoretic manner. Some special map structure, such as topological maps that resemble the sparse subterranean tunnel networks, are predicted using a convolutional neural network~(CNN) in~\cite{saroya2020online}. 

\textbf{Low-Rank Matrix Completion.} In the past two decades, the Low-Rank Matrix Completion~(LRMC) problem has been widely studied. The first theoretically guaranteed exact LRMC algorithm is proposed in~\cite{candes2009exact}, where any $n \times n$ incoherent matrices of rank r are proven to be exactly recovered from $C n^{1.2}r\log n$ uniformly randomly sampled entries with high probability through solving a convex problem of nuclear norm minimization~(NNM). Subsequent works~\cite{candes2010power, chen2015incoherence, gross2011recovering, recht2011simpler} refine the provable completion results following the NNM based method. However, since all of the algorithms mentioned above are based on second order methods~\cite{liu2010interior}, they can become extremely expensive if the matrix dimension is large~\cite{cai2010singular}. Some first order based methods~\cite{cai2010singular, ji2009accelerated, mazumder2010spectral} are developed to overcome the limitations of the second-order based methods. They solve the nuclear norm minimization problem in an iterative manner and rely on Singular Value Decomposition~(SVD) of matrices and are suited to large-scale matrix completion problems.