We study a variant of online convex optimization (OCO) with feedback delay and nonlinear switching (movement) cost. In recent years, the problem of online convex optimization with (linear) switching cost has received considerable attention, e.g., \citep{bansal_et_al:LIPIcs:2015:5297,10.1007/978-3-319-89441-6_13,10.1145/2796314.2745854,shi2020online,goel2019beyond} and the references therein.  In this setting an online learner iteratively picks an action $y_t$ and then suffers a convex hitting cost $f_t(y_t)$ and a (linear) switching cost $c(y_t,y_{t-1},\cdots,y_{t-p})$, depending on current and previous $p$ actions. This type of online optimization with memory has deep connections to convex body chasing \citep{bubeck2019competitively,argue2020chasing,sellke2020chasing,bubeck2020chasing} and has wide applications in areas such as power systems \citep{badiei2015online,kim2016online,li2018using}, electric vehicle charging \citep{chen2012iems,kim2016online}, cloud computing~\cite{liu2014pricing,chen2016using,10.1145/2796314.2745854}, and online control~\citep{goel2019online,li2019online,shi2020online,li2020online,lin2021perturbation}. 

Our work aims to generalize the online convex optimization literature in two directions, motivated by two limitations of the classical setting that prevent applications of the results in some important situations.  First, in the classical setting, the online learner observes the hitting cost function $f_t$ \textit{before} picking the action $y_t$. However, in many applications, such as trajectory tracking problems in robotics, $f_t$ is revealed after a multi-round delay due to communication and process delays, i.e., multiple rounds of actions must be taken \textit{without} feedback on their hitting costs. Delay is known to be very challenging in practice and \cite{shi2020online} shows that even \textit{one-step} delay requires non-trivial algorithmic modifications. The impact of multi-round delay has been recognized as a challenging open question for the design of online algorithms~\citep{shi2020online,joulani2013online,shamir2017online} and broadly in applications. For example, \cite{shi2021neural} highlights that a three-step delay (around $30$ milliseconds) can already cause catastrophic crashes in drone tracking control using a standard controller without algorithmic adjustments for delay. 

Second, the classical online convex optimization setting allows only \emph{linear} forms of switching cost functions, where the switching cost $c$ is some (squared) norm of a linear combination of current and previous actions, e.g.,  $y_t-y_{t-1}$ \cite{goel2019beyond, goel2019online, chen2018smoothed} or $y_t-\sum_{i=1}^pC_iy_{t-i}$ \cite{shi2020online}. However, in many practical scenarios the costs to move from $y_{t-1}$ to $y_t$ are non-trivial nonlinear functions. For example, consider $y_t\in\mathbb{R}$ as the vertical velocity of a drone in a velocity control task. Hovering the drone (i.e., holding the position such that $y_t=0,\forall t$) is not free, due to gravity. In this case, the cost to move from $y_{t-1}$ to $y_t$ is $(y_t-y_{t-1}+g(y_{t-1}))^2$ where the nonlinear term $g(y_{t-1})$ accounts for the gravity and aerodynamic drag \cite{shi2019neural}. 
Such non-linearities create significant algorithmic challenges because (i) in contrast to the linear setting, small movement between decisions does not necessarily imply small switching cost (e.g., the aforementioned drone control example), and (ii) a small error in a decision can lead to large non-linear penalties in the switching cost in future steps, which is further amplified by the multi-round delay. Addressing such challenges is well-known to be a challenging open question for the design of online algorithms. 

Additional motivation for our focus on delay and nonlinear switching cost comes from the emerging connection between online convex optimization and online control. The notion of a switching cost in online optimization parallels the control cost in optimal control theory in that both characterize the cost to steer the state $y_t$. Inspired by this analogy, recent papers have used similar algorithms and techniques in the two settings, e.g., \citep{agarwal2019online,li2020online}, and shown reductions between online control and online convex optimization \citep{goel2019online,li2019online,shi2020online,agarwal2019online,simchowitz2020improper,shi2021meta,yu2020power}. These results are highly provocative -- suggesting a deep connection -- but also limited in terms of the generality of the control settings that can be considered. All the existing results focus on linear dynamical systems without delay. The question of how general a connection can be made between online control and online convex optimization remains unanswered.  As we show in this paper, the incorporation of delay and nonlinear switching costs into online convex optimization significantly generalizes the set of control problems that can be reduced to online optimization. 


\subsection{Contributions}
This paper addresses the three open questions highlighted above.  We provide the first competitive algorithm for online convex optimization with feedback delay and nonlinear switching costs, and show a reduction between a class of nonlinear online control models with delay and online convex optimization with feedback delay and nonlinear switching cost. 

More specifically, we propose a novel setting of online optimization where the hitting cost suffers $k$-round delayed feedback and the switching cost is nonlinear. This setting generalizes prior work on online convex optimization with switching costs (e.g., \citep{goel2019beyond,shi2020online}).  In this setting, we propose a new algorithm, Iterative Regularized Online Balanced Descent (iROBD) and prove that it maintains a dimension-free constant competitive ratio that is $O(L^{2k})$, where $L$ is the Lipschitz constant of the non-linear switching cost and $k$ is the delay.  This is the first constant competitive algorithm in the case of either feedback delay or nonlinear switching cost and we show, via lower bounds, that the dependencies on both $L$ and $k$ are tight.  These lower bounds further serve to emphasize the algorithmic difficulties created by delay and non-linear switching costs.

The design of iROBD deals with the $k$-round delay via a novel iterative process of estimating the unknown cost function optimistically, i.e., iteratively assuming that the unknown cost functions will lead to minimal cost for the algorithm.  This approach is different than a one-shot approach focused on the whole trajectory of unknown cost functions, and the iterative nature is crucial for bounding the competitive ratio.  In particular, the key idea to our competitive ratio proof is to bound the error that accumulates in the iterations by leveraging a Lipschitz property on the nonlinear component of the switching cost. This analytic approach is novel and a contribution in its own right. 

Finally, we show that iROBD is constant competitive for the control of linear dynamical systems with squared costs and general adversarial disturbances as well as a class of nonlinear dynamics, via a novel reduction between such systems and online optimization with feedback delay and nonlinear switching costs.  This reduction represents a significant generalization of the results in \citep{goel2019online,shi2020online}, which each have significant limitations on the dynamics where they apply.  Our new reduction highlights that state disturbances can be connected to delay and nonlinear dynamics can be connected to nonlinear switching costs; thus highlighting the difficulties each creates for competitive control.