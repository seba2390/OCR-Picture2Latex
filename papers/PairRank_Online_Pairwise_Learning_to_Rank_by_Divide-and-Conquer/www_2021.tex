\documentclass[sigconf]{acmart}
% \documentclass[draft]{acmart}

\usepackage[linesnumbered, ruled]{algorithm2e}
% \renewcommand{\baselinestretch}{0.99}
\usepackage[utf8]{inputenc}
\usepackage[english]{babel}
\usepackage{graphicx}  % for sub-figure
\usepackage{subcaption}  % for sub-figure
\usepackage{amsthm}
\usepackage{appendix}
\usepackage{mathtools}
\settopmatter{authorsperrow=4}


\theoremstyle{definition}
\newtheorem{definition}{Definition}
\newtheorem{theorem}{Theorem}
\newtheorem{corollary}{Corollary}
\newtheorem{lemma}{Lemma}
\newtheorem{assumption}{Assumption}
\newtheorem{proposition}{Proposition}
\theoremstyle{remark}
\newtheorem*{remark}{Remark}
\newcommand{\me}{\mathrm{e}}
\usepackage{enumitem}
\newenvironment{hproof}{%
  \renewcommand{\proofname}{Proof Sketch}\proof}{\endproof}

\def \bx {\mathbf{x}}
\def \xij {\mathbf{x}_{ij}}
\def \xijt {{\mathbf{x}_{ij}^t}}
\def \xjit {{\mathbf{x}_{ji}^t}}
\def \xijs {{\mathbf{x}_{ij}^s}}
\def \xmns {{\mathbf{x}_{mn}^s}}
\def \xmnt {{\mathbf{x}_{mn}^t}}
\def \ymns {{\mathbf{y}_{mn}^s}}
\def \ymnt {{\mathbf{y}_{mn}^t}}
\def \yijs {{\mathbf{y}_{ij}^s}}
\def \yijt {{\mathbf{y}_{ij}^t}}
\def \tp  {t^\prime}
\def \sub {\scriptscriptstyle}
\def \btheta {\boldsymbol{\theta}}
\def \bTheta {\boldsymbol{\Theta}}
\def \bR {\mathbb{R}}
\def \bM {\mathbf{M}}
\def \bZ {\mathbf{Z}}
\def \bU {\mathbf{U}}
\def \bG {\mathbf{G}}
\def \bSigma {\mathbf{\Sigma}}
% \renewcommand{\osum}[2]{\mathop{\sum{#2}}_{#1}}

\DeclareMathOperator*{\argmax}{argmax}
\DeclareMathOperator*{\argmin}{argmin}


%%
%% \BibTeX command to typeset BibTeX logo in the docs
\AtBeginDocument{%
  \providecommand\BibTeX{{%
    \normalfont B\kern-0.5em{\scshape i\kern-0.25em b}\kern-0.8em\TeX}}}


%% Rights management information.  This information is sent to you
%% when you complete the rights form.  These commands have SAMPLE
%% values in them; it is your responsibility as an author to replace
%% the commands and values with those provided to you when you
%% complete the rights form.
\setcopyright{acmcopyright}
\copyrightyear{2021}
\acmYear{2021} 
\acmConference[WWW '21]{Proceedings of the Web Conference 2021}{April 19--23, 2021}{Ljubljana, Slovenia}
\acmBooktitle{Proceedings of the Web Conference 2021 (WWW '21), April 19--23, 2021, Ljubljana, Slovenia}
\acmPrice{}
\acmDOI{10.1145/3442381.3449972}
\acmISBN{978-1-4503-8312-7/21/04}


%%
%% end of the preamble, start of the body of the document source.
\begin{document}
%%
%% The "title" command has an optional parameter,
%% allowing the author to define a "short title" to be used in page headers.
\title{PairRank: Online Pairwise Learning to Rank by Divide-and-Conquer}
\newcommand{\model}{{PairRank}}


\author{Yiling Jia}
\affiliation{%
  \institution{University of Virginia}
  \city{Charlottesville}
  \state{VA}
  \country{USA}
}
\email{yj9xs@virginia.edu}

\author{Huazheng Wang}
\affiliation{%
  \institution{University of Virginia}
  \city{Charlottesville}
  \state{VA}
    \country{USA}
}
\email{hw7ww@virginia.edu}

\author{Stephen Guo}
\affiliation{%
  \institution{Walmart Labs}
  \city{Sunnyvale}
  \state{CA}
  \country{USA}
}
\email{sguo@walmartlabs.com}

\author{Hongning Wang}
\affiliation{%
  \institution{University of Virginia}
  \city{Charlottesville}
  \state{VA}
  \country{USA}
}
\email{hw5x@virginia.edu}


%% By default, the full list of authors will be used in the page
%% headers. Often, this list is too long, and will overlap
%% other information printed in the page headers. This command allows
%% the author to define a more concise list
%% of authors' names for this purpose.
% \renewcommand{\shortauthors}{Trovato and Tobin, et al.}

%%
%% The abstract is a short summary of the work to be presented in the
%% article.
\begin{abstract}
Online Learning to Rank (OL2R) eliminates the need of explicit relevance annotation by directly optimizing the rankers from their interactions with users. However, the required exploration drives it away from successful practices in offline learning to rank, which limits OL2R's empirical performance and practical applicability.
In this work, we propose to estimate a pairwise learning to rank model online. In each round, candidate documents are partitioned and ranked according to the model's confidence on the estimated pairwise rank order, and exploration is only performed on the uncertain pairs of documents, i.e., \emph{divide-and-conquer}.  
Regret directly defined on the number of mis-ordered pairs is proven, which connects the online solution's theoretical convergence with its expected ranking performance. Comparisons against an extensive list of OL2R baselines on two public learning to rank benchmark datasets demonstrate the effectiveness of the proposed solution.
\end{abstract}

%%
%% The code below is generated by the tool at http://dl.acm.org/ccs.cfm.
%% Please copy and paste the code instead of the example below.
%%

\begin{CCSXML}
<ccs2012>
   <concept>
       <concept_id>10002951.10003317.10003338.10003343</concept_id>
       <concept_desc>Information systems~Learning to rank</concept_desc>
       <concept_significance>500</concept_significance>
       </concept>
   <concept>
       <concept_id>10003752.10010070.10010071.10011194</concept_id>
       <concept_desc>Theory of computation~Regret bounds</concept_desc>
       <concept_significance>300</concept_significance>
       </concept>
   <concept>
       <concept_id>10003752.10003809.10010047.10010048</concept_id>
       <concept_desc>Theory of computation~Online learning algorithms</concept_desc>
       <concept_significance>300</concept_significance>
       </concept>
   <concept>
       <concept_id>10003752.10003809.10011254.10011257</concept_id>
       <concept_desc>Theory of computation~Divide and conquer</concept_desc>
       <concept_significance>300</concept_significance>
       </concept>
 </ccs2012>
\end{CCSXML}

\ccsdesc[500]{Information systems~Learning to rank}
\ccsdesc[500]{Theory of computation~Online learning algorithms}
\ccsdesc[500]{Theory of computation~Divide and conquer}
\ccsdesc[500]{Theory of computation~Regret bounds}

\keywords{Online learning to rank; divide and conquer; regret analysis}

\maketitle

\section{Introduction}  \label{sec:introduction}

\newcommand\inexpIntro[3]{#1?(#2,#3).}
\newcommand\rinexpIntro[3]{*#1?(#2,#3).}
\newcommand\outexpIntro[3]{#1!(#2,#3).}
\newcommand\outatomIntro[3]{#1!(#2,#3)}

We propose a fully automated method for proving termination of \(\pi\)-calculus processes.
Although there have been a lot of studies on termination analysis for the \(\pi\)-calculus
and related calculi~\cite{Deng06IC,Demangeon07,SangiorgiTermination,KobayashiHybrid,Yoshida04IC,DBLP:journals/jlp/DemangeonHS10,Venet98SAS}, most of them have been rather theoretical,
and there have been surprisingly little efforts in developing  fully automated termination
verification methods and tools based on them. To our knowledge,
Kobayashi's \typical{}~\cite{TyPiCal,KobayashiHybrid} is the only exception that
can prove termination of \(\pi\)-calculus processes (extended with natural numbers)
fully automatically, but its termination analysis is quite limited (see Section~\ref{sec:relatedwork}).

Our method is based on a reduction to termination analysis for sequential programs:
we translate a \(\pi\)-calculus process \(P\) to a sequential program \(S_P\), so that
if \(S_P\) is terminating, so is \(P\). The reduction allows us to use
powerful, mature methods and tools
for termination analysis of sequential programs~\cite{heizmann2016ultimate,freqterm,DBLP:conf/lics/PodelskiR04,Kuwahara2014Termination,DBLP:journals/cacm/CookPR11}.

The idea of the translation is to convert a chain of communications on replicated input
channels to a chain of recursive function calls of the target sequential program.
Let us consider the following Fibonacci process:
\begin{align*}
    & \rinexpIntro{\fib}{n}{r}
        \ifexp{n<2}{ \soutatom{r}{1} \\ &\quad}
                   { \nuexp{s_1} \nuexp{s_2} (\outatomIntro{\fib}{n-1}{s_1} \PAR \outatomIntro{\fib}{n-2}{s_2} \PAR \sinexp{s_1}{x}\sinexp{s_2}{y}\soutatom{r}{x+y}) \\}
    & \PAR \outatomIntro{\fib}{m}{r}
\end{align*}
Here, the process
$\rinexpIntro{\fib}{n}{r} \ldots$ is a function server that computes the \(n\)-th Fibonacci number
in parallel and returns the result to \(r\),
and $\outatom{\fib}{m}{r}$ sends a request for computing the \(m\)-th Fibonacci number;
those who are not familiar with the syntax of the \(\pi\)-calculus may wish to consult
Section~\ref{sec:targetlanguage} first.
To prove that the process above is terminating for any integer \(m\),
it suffices to show that there is no infinite chain of communications on $\fib$:
\[
    \fib(m,r) \to \fib(m_1,r_1) \to \fib(m_2,r_2) \to \cdots.
\]
We convert the process above to the following program:\footnote{The actual translation
  given later is a little more complex.}
\begin{verbatim}
 let rec fib(n) = if n<2 then () else (fib(n-1) [] fib(n-2)) in
 fib(m)
\end{verbatim}
Here, \texttt{[]} represents the non-deterministic choice.
Note that, although the calculation of Fibonacci numbers is not preserved,
for each chain of communications on \texttt{fib}, there is a corresponding
sequence of recursive calls:
\[
\mathtt{fib}(m) \to \mathtt{fib}(m_1) \to \mathtt{fib}(m_2) \to \cdots.
\]
Thus, the termination of the sequential program above implies the termination of
the original process.
As shown in the example above, (i) each communication on a replicated input channel
is converted to a function call, (ii) each communication on a non-replicated input
channel is just removed (or, in the actual translation, replaced by a call of
a trivial function defined by \(f(\seq{x})=(\,)\)), and (iii) parallel composition
is replaced by a non-deterministic choice.
We formalize the translation outlined above and prove its correctness.

The basic translation sketched above sometimes loses too much information.
For example, consider the following process:
\begin{align*}
    & \rinexpIntro{\pre}{n}{r} \soutatom{r}{n-1} \\
    & \PAR \rinexpIntro{f}{n}{r} \ifexp{n<0}{ \soutatom{r}{1} }
                                       { \nuexp{s} (\outatomIntro{\pre}{n}{s} \PAR \sinexp{s}{x}\outatomIntro{f}{x}{r}) } \\
    & \PAR \outatomIntro{f}{m}{r}
\end{align*}
The translation sketched above would yield:
\begin{verbatim}
  let pred(n) = n-1 in
  let rec f(n) = if n<0 then () else (pred(n) [] f(*)) in
  f(m)
\end{verbatim}
Here, \texttt{*} represents a non-deterministic integer: since we have removed
the input $\sinatom{s}{x}$, we do not have information about the value of \( x \).
As a result, the sequential program above is non-terminating, although the original
process is terminating.
To remedy this problem, we also refine the basic translation above by using a refinement
type system for the \(\pi\)-calculus. Using the refinement type system,
we can infer that the value of \(x\) in the original process is less than \(n\),
so that we can refine the definition of \texttt{f} to:
\begin{verbatim}
 let rec f(n) = ... else (pred(n) [] let x=* in assume(x<n);f(x))
\end{verbatim}
The target program is now terminating, from which
we can deduce that the original process is also terminating.
We have implemented an automated tool based on the refined translation above.

The contributions of this paper are summarized as follows.
\begin{itemize}
\item The formalization of the basic translation from the \(\pi\)-calculus
  (extended with integers) to sequential programs, and a proof of its correctness.
\item The formalization of a refined translation based on a refinement type system.
\item An implementation of the refined translation, including automated refinement type
  inference based on CHC solving, and experiments to evaluate the effectiveness of
  our method.
\end{itemize}

The rest of this paper is structured as follows.
Section~\ref{sec:targetlanguage} introduces the source and target languages
of our translation.
Section~\ref{sec:approach} 
formalizes the basic translation, and proves its correctness.
Section~\ref{sec:refinement} refines the basic translation by using a refinement type system.
Section~\ref{sec:implementation} reports an implementation and experiments.
Section~\ref{sec:relatedwork} discusses related work,
and Section~\ref{sec:conclusion} concludes the paper.

\textbf{Related work}:
% Object detection related datasets/algo in non-medical domain
% Locally labeled CXR dataset
A few CXR datasets have localized abnormality annotations \cite{shih2019augmenting,filice2020crowdsourcing,jaeger2014two} that are curated manually. These are high quality gold standard ground truth datasets but tend to be smaller in scale (< 30,000 images) and have a narrow coverage, with typically only 1-2 labels. In addition, since most labeling efforts only have abnormality semantics attached, no direct relationships with the affected anatomical locations are available. 

%MEHDI: repeated concepts from above. I am removing the following: 

%The lack of anatomic semantics in the annotation is a limitation for complex multi-modal clinical reasoning work, e.g., differential diagnosis, since clinicians often integrate information along anatomical lines, and for downstream report generation tasks, which often requires describing not only the abnormality but also correctly communicate the location of the abnormalities (and medical devices) to the receiving clinicians. 

Two recent CXR datasets have labels for anatomies described in the reports. In \cite{datta2020dataset}, a small manually annotated dataset (2000 reports) included 10 abnormalities that are individually associated with 29 unique spatial locations (anatomies) at the report level. Another CXR dataset has automatically extracted abnormality and anatomy labels as disconnected concepts that are only correlated at the study level from  160,000 reports using a supervised NLP algorithm \cite{bustos2020padchest}. This was trained on a smaller set of manually annotated data. Neither datasets contain localized annotations for the associated CXR images, nor any comparison relation annotations between sequential exams, both of which are available in the Chest ImaGenome dataset. In Table \ref{tab:related}, we present a comparison of our Chest ImagGenome dataset with other datasets available in the literature.

% Table -- Kashyap

% MEdical imaging datasets to go here: Discussed that we will only focus on cxr datasets that are available for this paper. 
% \caption{\color{red} Kashyap, feel free to continue with the table. We should remove the questionmarks and add a line for our dataset (since all others are not graph). For longer text, using abbreviations and explaining them in the caption often works better. If fill in the values is not possible, it is better to remove the table altogether.}


\begin{table}[t!]
\caption{Summary of existing chest X-ray datasets}
\resizebox{\textwidth}{!}{%
\begin{tabular}{@{}lllllllll@{}}
\toprule
\textbf{Dataset} & \textbf{Annotation Level} & \textbf{Annotation Method} & \textbf{Num Labels} & \textbf{Anatomy Labeled} & \textbf{Graph} & \textbf{Dataset Size} & \textbf{Temporal Labels} & \textbf{Reports} \\ \midrule
SIIM-ACR Pneumothorax Segmentation \cite{filice2020crowdsourcing} & Segmentation & Manual + augmented & 1 & No & No & 12,047 & No & No \\
RSNA Pneumonia Detection Challenge   \cite{shih2019augmenting} & Bounding Boxes & Manual & 1 & No & No & 30,000 & No & No \\
Indiana University Chest X-ray collection \cite{demner2016preparing} & Global & Automated & 10 & No & No & 3,813 & No & Yes \\
NIH CXR dataset \cite{wang2017chestx} & Global & Automated & 14 & No & No & 112,120 & No & No \\
PLCO \cite{team2000prostate} & Global & Automated & 24 & Yes & No & 236,000 & Yes & No \\
Stanford CheXpert \cite{irvin2019chexpert} & Global & Automated & 14 & No & No & 224,316 & No & No \\
MIMIC-CXR \cite{johnson2019mimic} & Global & Automated & 14 & No & No & 377,110 & No & Yes \\
Dutta \cite{datta2020dataset} & Global & Manual & 10 & Yes & Yes & 2,000 & No & Yes \\
PadChest \cite{bustos2020padchest} & Global & Manual + automated & 297 & Yes & No & 160,868 & No & Yes \\
Montgomery County Chest X-ray   \cite{jaeger2014two} & Segmentation & Manual & 1 & Yes & No & 138 & No & No \\
Shenzen Hospital Chest X-ray   \cite{jaeger2014two} & Segmentation & Manual & 1 & Yes & No & 662 & No & No \\  \hline \hline
\textbf{Chest ImaGenome} & Bounding Boxes & Automated & 131 & Yes & Yes & 242,072 & Yes & Yes \\
\bottomrule
\end{tabular}%
}
\label{tab:related}
\vspace{-0.4cm}
\end{table}
% removed (Derived from MIMIC-CXR \cite{johnson2019mimic}) % makes table really small

\section{Method}
%!TEX root = main.tex
\section{Problem Definition and Notations}
\label{sec:problem}







% In this section, we will first describe key concepts and notations used in this paper, and formally define our problem. Then we will use a case study to make our idea of story tree more concrete.

% \subsection{Problem Definition and Notations}
% \label{subsec:problem-define}

We first present some definitions of key concepts in the top-down hierarchy: \textit{topic} $\rightarrow$ \textit{story} $\rightarrow$ \textit{event} to be used in this paper.

\begin{definition}
  \textit{Event}: an event $\mathcal{E}$ is a set of one or several documents that contain highly similar information.
\end{definition}

\begin{definition}
  \textit{Story}: a story $\mathcal{S}$ is a tree of events that revolve around a group of specific persons and happen at certain places during specific times. A directed edge from event $\mathcal{E}_1$ to $\mathcal{E}_2$ indicates a temporal evolution or a logical connection from $\mathcal{E}_1$ to $\mathcal{E}_2$.
\end{definition}

\begin{definition}
  \textit{Topic}: a topic consists of a set of stories that are highly correlated or similar to each other.
  \vspace{-1mm}
\end{definition}


Each topic may contain multiple story trees, and each story tree consists of multiple logically connected events.
In our work, events (instead of news documents) are the smallest atomic units. Each event is also assumed to belong to a single story and contains partial information about that story.
For instance, considering the topic \textit{American presidential election}, \textit{2016 U.S. presidential election} is a story within this topic, and  \textit{Trump and Hilary's first television debate} is an event within this story.


We now introduce some notations and describe our problem formally. Given a news document stream $D = \{ \mathcal{D}_1, \mathcal{D}_2, \ldots, \mathcal{D}_t,\ldots \}$, where $\mathcal{D}_t$ is the set of news documents collected on time period $t$, our objective is to: a) cluster all news documents $D$ into a set of events $E = \{ \mathcal{E}_1, \ldots, \mathcal{E}_{|E|} \}$, and b) connect the extracted events to form a set of stories $S = \{ \mathcal{S}_1, ..., \mathcal{S}_{|S|} \}$. Each story $\mathcal{S} = (E, L)$ contains a set of events $E$ and a set of links $L$, where $L_{i,j} := <\mathcal{E}_i, \mathcal{E}_j>$ denotes a directed link from event $\mathcal{E}_i$ to $\mathcal{E}_j$, which indicates a temporal evolution or logical connection relationship.

%We now illustrate our problem with an example. (A example Fig) Fig... shows ...
Furthermore, we require the events and story trees to be extracted in an online or incremental manner. That is, we extract events from each $\mathcal D_t$ individually when the news corpus $\mathcal D_t$ arrives in time period $t$, and \emph{merge} the discovered events into the existing story trees that were found at time $t-1$. This is a unique strength of our scheme as compared to prior work, since we do not need to repeatedly process older documents and can deliver  a set of evolving yet logically consistent story trees to users.  

% \subsection{Case Study}
% \label{subsec:case-study}

\begin{figure}
\includegraphics[width=3.4in]{figure/StoryStructures}
\caption{Different structures to characterize a story.}
\vspace{-2mm}
\label{fig:storyStructures}
\vspace{-2mm}
\end{figure}

For example, Fig.~\ref{fig:CaseStudy} illustrates the story tree of ``2016 U.S. presidential election''. The story contains $20$ nodes, where each node indicates an event in 2016 U.S. election, and each link indicates a temporal evolution or a logical connection between two events. %For example, event $19$ says America votes to elect new president, and event $20$ says Donald Trump is elected president. 
The index number on each node represents the event sequence over the timeline. There are $6$ paths within this story tree, where the path $1 \rightarrow 20$ indicates the whole presidential election process, branch $3 \rightarrow 6$ is about Hilary's health conditions, branch $7 \rightarrow 13$ talks about television debates, $14 \rightarrow 18$ depicts the investigation into Hilary's ``mail door'', etc. As we can see, by modeling the evolutionary and logical structure of a story into a story tree, users can easily grasp the logic of news stories and learn the main information quickly. 


Let us represent each story by an empty root node $s$ from which the story is originated, and denote each event by an event node $e$. The events in a story can be organized in one of the following four structures shown in Fig. \ref{fig:storyStructures}: a) a flat structure that does not include dependencies between events; b) a timeline structure that organizes events by their timestamps; c) a graph structure that checks the connection between all pairs of events and maintains a subset of most strong connections; d) a tree structure, which represents a story's evolving structure by a tree.  

Compared with a tree structure, sorting events by timestamps omits the logical connection between events, while using directed acyclic graphs to model event dependencies without considering the evolving consistency of the whole story can leads to unnecessary connections between events.
Through extensive user experience studies in Sec.~\ref{sec:eval}, we show that tree structures are the most effective way to represent breaking news stories as compared to other structures, including the more complex graph structures. 

\begin{algorithm}[!ht]
\begin{algorithmic}[1]
\Require Query workload $Q$, event stream $I$, \app\ graph $G$, hash table of snapshots $S$
\Ensure Hash table of results $R$ 
\State $G \leftarrow \emptyset$, $S, R \leftarrow$ empty hash tables
\ForAll {event $e \in I$ with $e.type=E$} 
    \State $//$ \textbf{\app\ graph construction}
    \ForAll {$q \in Q$ \text{ with event types }T}
        \ForAll {$E' \in T,\ E' \neq E$}
            \State $G_{E'} \leftarrow \mathit{getGraphlet}(G,E')$,
            $G_{E'}.\mathit{active} \leftarrow \mathit{false}$
        \EndFor
    \EndFor
    \If {\textbf{not} $G_E.\mathit{active}$}
        \State $G_E \leftarrow \mathit{createGraphlet()}$, $G_{E}.\mathit{active} \leftarrow \mathit{true}$,
        $G \leftarrow G \cup G_E$
        \If {$G_E.\mathit{shared}$ by $Q_E \subseteq Q$}
            $x \leftarrow \mathit{createSnapshot()}$ 
            \ForAll {$q \in Q_E$}
                \ForAll{$E' \in \mathit{pt}(E,q), E' \neq E$}
                    \State $G_{E'} \leftarrow \mathit{getGraphlet}(G,E')$
                    \State $S(x,q) \leftarrow S(x,q) + sum(G_{E'},q)$ \hspace{0.5cm}$//$ Eq.~5
                \EndFor
            \EndFor
        \EndIf    
    \EndIf
    \State insert $e$ into $G_E$
    \State $//$ \textbf{Trend count computation}
    \If {$G_E.\mathit{shared}$ by $Q_E \subseteq Q$}
        \If {$\forall q \in Q_E\ pe(e,q)$ are identical}
            \State $count(e,Q_E) \leftarrow count(e,q)$ \hspace{2.3cm}$//$ Eq.~2
        \Else\ $y \leftarrow \mathit{createSnapshot()}$, $count(e,Q_E) = y$
            \ForAll {$q \in Q_E$}
                $S(y,q) \leftarrow count(e,q)$ \hspace{0.2cm}$//$ Eq.~2
            \EndFor
          \EndIf
    \Else\ $count(e,q)$ \hspace{5.2cm}$//$ Eq.~2
    \EndIf
    \ForAll{$q \in Q$}
  	    \If {$E \in \mathit{end}(q)$} 
  		    $R(q) \leftarrow R(q) + count(e,q)$ $//$ Eq.~3
        \EndIf
    \EndFor
\EndFor
\State \Return $R$
\end{algorithmic}
\caption{\app\ shared online trend aggregation}
\label{algo:snapshot-propagation}
\end{algorithm}

\subsection{Regret Analysis}

We theoretically analyze the regret of \model{}, which is defined by the cumulative number of mis-ordered pairs in its proposed ranked list till round $T$. 
The key in analyzing this regret is to quantify how fast the model achieves certainty about its pairwise preference estimation in candidate documents. First, we define $E_t$ as the success event at round $t$:
\begin{equation*}
    E_t = \big\{ \forall (i, j) \in [L_t]^2, |\sigma({\mathbf{x}_{ij}^t}^\top\hat{\btheta}_t) - \sigma({\mathbf{x}_{ij}^t}^\top\btheta^*) | \leq \alpha_t\Vert\mathbf{x}_{ij}^t\Vert_{\bM_t^{-1}}\big\}.
\end{equation*}

Intuitively, $E_t$ is the event that the estimated $\hat{\theta}_t$ is ``close'' to the optimal model $\theta^*$ at round $t$. According to Lemma~\ref{lemma:cb}, it is easy to reach the following conclusion,

\begin{corollary}On the event $E_t$, it holds that $\sigma({\mathbf{x}_{ij}^t}^\top\btheta^*) > {1}/{2}$ if $(i, j) \in \mathcal{E}_c^t$.
\label{col}
\end{corollary} 

\model{} suffers regret as a result of misplacing a pair of documents, i.e., swapping a pair already in the correct order. Based on Corollary~\ref{col}, under event $E_t$, the certain rank order identified by \model{} is consistent with the ground-truth. 
As our partition design always places documents under a certain rank order into distinct blocks, under event $E_t$ the ranking order across blocks is consistent with the ground-truth. In other words, regret only occurs when ranking documents within each block.

% Let $\Delta_{\min} = \min_{t\in T}\min_{(m, n) \in [L_t]^2}| \sigma({\xmnt}^\top\btheta^*) - \frac{1}{2}|$, which represents the smallest gap between any pair of documents associated to the same query over time (across all queries). 

To analyze the regret caused by random shuffling within each block, we need the following technical lemma derived from random matrix theory. We adapted it from Equation (5.23) of Theorem 5.39 from \cite{vershynin2010introduction}.

\begin{lemma}
\label{lemma:matrix}
Let $A \in \mathbb{R}^{n \times d}$ be a matrix whose rows $A_i$ are independent sub-Gaussian isotropic random vectors in $\mathbb{R}^d$ with parameter $\sigma$, namely $\mathbb{E}[\text{exp}(x^\top (A_i - \mathbb{E}[A_i])] \leq \text{exp}(\sigma^2\Vert x \Vert ^2 / 2)$ for any $x \in \mathbb{R}^d$. Then, there exist positive universal constants $C_1$ and $C_2$ such that, for every $t \geq 0$, the following holds with probability at least $1 - 2\text{exp}(-C_2t^2), \text{where } \epsilon = \sigma(C_1\sqrt{d/n} + {t}/{\sqrt{n}})$: $\Vert A^\top A/n - \mathbf{I}_d\Vert \leq \text{max}\{\epsilon, \epsilon^2\}$.
\end{lemma}

The detailed proof can be found in \cite{vershynin2010introduction}. We should note the condition in Lemma~\ref{lemma:matrix} is not hard to satisfy in OL2R: at every round, the ranker is serving a potentially distinct query; and even for the same query, different documents might be returned at different times. This gives the ranker a sufficient chance to collect informative observations for model estimation.  Based on Lemma \ref{lemma:matrix}, we have the following lemma, which provides a tight upper bound of the probability that \model{}'s estimation of the pairwise preference is an uncertain rank order.

\begin{lemma}
At round $t \geq t^\prime$, with $\delta_1 \in (0, \frac{1}{2})$, $\delta_2 \in (0, \frac{1}{2})$, $\beta \in (0, \frac{1}{2})$, and $C_1$, $C_2$ defined in Lemma~\ref{lemma:matrix}, under event $E_t$, the following holds with probability at least $1 - \delta_2$: $\forall (i, j) \in [L_t]^2$, $\mathbb{P}\big((i, j) \in \mathcal{E}_u^t\big) \leq \frac{8k_{\mu}^2\Vert\xijt\Vert_{\bM_t^{-1}}^2}{(1 - 2\beta) c_{\mu}^2\Delta_{\min}^2}\log{\frac{1}{\delta_1}}$, with 
$t^\prime = \Big(\frac{c_1\sqrt{d} + c_2\sqrt{\log(\frac{1}{\delta_2})} + abd\sqrt{\frac{o_{\text{max}}u^2}{d\lambda}}}{\lambda_{\text{min}}(\bSigma)}\Big)^2 + \frac{2ab\log({1}/{\delta_1^2}) + 8a\lambda Q^2 - \lambda}{\lambda_{\text{min}}(\bSigma)}$, 
where $\Delta_{\min} = \min\limits_{t\in T, (i, j) \in [L_t]^2}| \sigma({\xijt}^\top\btheta^*) - \frac{1}{2}|$ representing the smallest gap of pairwise difference between any pair of documents associated to the same query over time (across all queries), $a = {4k_{\mu}^2 u^2}/({\beta^2c_{\mu}^2\Delta_{\text{min}}^2})$, and $b = R^2 + 4\sqrt{\lambda}QR$, 
\label{lemma:uncertain}
\end{lemma}


% \begin{lemma} On the event $E_t$, with $\delta \in (0, \frac{1}{2})$, $\forall (i, j) \in [L_t]^2$, $\mathbb{P}\big((i, j) \in \mathcal{E}_u^t\big) \leq \frac{\Vert\mathbf{x}_{ij}\Vert_{M_t^{-1}}^2}{8 c_{\mu}^2\Delta_{\min}^2}\log{\frac{1}{\delta}}$.
% \label{lemma:uncertain}
% \end{lemma}
\begin{hproof}
According to the definition of certain rank order, a pairwise estimation $\sigma(\xijt^\top\hat{\btheta})$ is certain if and only if $|\sigma(\xijt^\top\hat{\btheta}) - 1/2| \geq \alpha_t\Vert \xijt\Vert_{\mathbf{M}_{t}^{-1}}$. By the reverse triangle inequality, the probability can be upper bounded by $\mathbb{P}\big(\big| |\sigma(\xijt^\top\hat{\btheta}) - \sigma(\xijt^\top\btheta^*)| - |\sigma(\xijt^\top\btheta^*) - 1/2|\big| \geq \alpha_t\Vert \xijt\Vert_{\mathbf{M}_{t}^{-1}}\big)$, which can be further bounded by Theorem 1 in \cite{abbasi2011improved}. The key in this proof is to obtain a tighter bound on the uncertainty of \model{}'s parameter estimation compared to the bound determined by $\delta_1$ in Lemma~\ref{lemma:cb}, such that its confidence interval on a pair of documents' comparison at round $t$ will exclude the possibility of flipping their ranking order, i.e., the lower confidence bound of this pairwise estimation is above 0.5.
\end{hproof}

In each round of result serving, as the model $\hat{\btheta}_t$ would not change until next round, the expected number of uncertain rank orders, denoted as $N_t=|\mathcal{E}_{u}^t|$, can be estimated by the summation of the uncertain probabilities over all possible pairwise comparisons under the current query $q_t$, e.g., $\mathbb{E}[N_t] = \frac{1}{2} (\sum_{(i, j) \in [L_t]^2}  \mathbb{P}((i, j) \in \mathcal{E}_u^t)$.

Denote $p_{t}$ as the probability that the user examines all documents in $\tau_t$ at round $t$, and let $p^* = \min_{1\leq t \leq T} p_{t}$ be the minimal probability that all documents in a query are examined over time. We present the upper regret bound of \model{} as follows.
\begin{theorem}
\label{theorem}
Assume pairwise query-document feature vector $\xijt$ under query $q_t$, where $(i, j) \in [L_t]^2$ and $t \in [T]$, satisfies Proposition 1. With  $\delta_1 \in (0, \frac{1}{2})$, $\delta_2 \in (0, \frac{1}{2})$, $\beta \in (0, \frac{1}{2})$, the $T$-step regret of \model{} is upper bounded by:
\begin{align*}
    R_T 
     \leq& R^{\prime} + (1 - \delta_1)(1 - \delta_2) {p^*}^{-2}\left( 2adL_{\max}\log(1 + \frac{o_{\max}Tu^2}{2d\lambda}) + aw\right)^2
\end{align*}
where $R^{\prime} = \tp L_{\max}^2 + (T - t')\big(\delta_2L_{\max}^2 - (1- \delta_2)\delta_1 L_{\max}^2\big)$, with $\tp$ and $a$ defined in Lemma \ref{lemma:uncertain}, and $w = \sum\nolimits_{s=\tp}^T ({(L_{\max}^2 - 2L_{\max})u^2 }/({\lambda_{\min}(\bM_s)})$, and $L_{\max}$ representing the maximum number of document associated to the same query over time.
% $t'$ satisfies Lemma 2, and 
By choosing $\delta_1 = \delta_2 = 1/T$, we have the expected regret at most $R_T \leq O(d\log^4(T))$.
\end{theorem}

\begin{hproof}
The detailed proof is provided in the appendix. Here, we only provide the key ideas behind our regret analysis.
The regret is first decomposed into two parts: $R^\prime$ represents the regret when either $E_t$ or Lemma~\ref{lemma:uncertain} does not hold, in which the regret is out of our control, and we use the maximum number of pairs associated to a query over time, $L_{\text{max}}$ to compute the regret. The second part corresponds to the cases when both events happen. Then, the instantaneous regret at round $t$ can be bounded by
\begin{align}
    r_t = \mathbb{E} \big[K(\tau_t, \tau_t^*)\big] = \sum\nolimits_{i=1}^{d_t}\mathbb{E}\big[\frac{(N_i^t + 1)N_i^t}{2}\big] \leq \mathbb{E}\big[\frac{N_t(N_t + 1)}{2}\big]
\end{align}
where $N_i^t$ denotes the number of uncertain rank orders in block $\mathcal{B}_i^t$ at round $t$, and $N_t$ denotes the total number of uncertain rank orders.
From the last inequality, it follows that in the worst case where the $N_t$ uncertain rank orders are placed into the same block and thus generate at most $({N_t^2 + N_t})/{2}$ mis-ordered pairs with random shuffling. This is because based on the blocks created by \model{}, with $N_t$ uncertain rank orders in one block, this block can at most have $N_t + 1$ documents. Then, the cumulative number of mis-ordered pairs can be bounded by the probability of observing uncertain rank orders in each round, which shrinks rapidly with more observations over time.
\end{hproof}

\begin{remark}[1] 
By choosing $\delta = 1/T$, the theorem shows the expected regret increases at a rate of $\mathcal{O}({\log^4{T}})$. In this analysis, we provide a gap-dependent regret upper bound of \model{}, where the gap $\Delta_{\min}$ characterizes the hardness of sorting the $L_t$ candidate documents at round $t$. As the matrix $M_t$ only contains information from observed document pairs, we adopt the probability of a ranked list is fully observed to tackle with the partial feedback \cite{kveton2015combinatorial, kveton2015tight}, which is a constant independent of $T$.
\end{remark}

\begin{remark}[2]
Our regret is defined over the number of mis-ordered pairs, which is the \emph{first} pairwise regret analysis for an OL2R algorithm, to the best of our knowledge. As we discussed before, existing OL2R algorithms optimize their own metrics, which can hardly link to any conventional rank metrics. As shown in \cite{Wang2018Lambdaloss}, most classical ranking evaluation metrics, such as ARP and NDCG, are based on pairwise document comparisons. Our regret analysis of \model{} connects its theoretical property with such metrics, which has been later confirmed in our empirical evaluations.   
\end{remark}

\newcommand{\twomoons}{{\tt Twomoons}}
\newcommand{\gauss}{{\tt Gauss}}
\newcommand{\sculpture}{{\tt Sculpture}}
\newcommand{\baseline}{{\tt Baseline}}
\newcommand{\MM}{{\tt MsgPassing}}
\newcommand{\blackboard}{{\tt Blackboard}}
\newcommand{\ncut}{\text{ncut}}
\newcommand{\chensays}[2][]{\textcolor{blue} {\textsc{Jiecao #1:} \emph{#2}}}

\section{Experiments}
In this section we present experimental results for  graph clustering in the message passing and blackboard models. We will compare the following three algorithms. (1) \baseline: each site sends all the data to the coordinator directly; (2) \MM: our algorithm in the message passing model (Section~\ref{sec:gcmessage}); (3) 
\blackboard: our algorithm in  the blackboard model (Section~\ref{sec:bb}).


%Since both of our algorithms are crucially based on the use of spectral scarification, our main focus in the experiments is to investigate to what extend the quality of the spectral clustering algorithms will be affected by using spectral sparsification, the saving of communication costs by using spectral sparsificaion, ...
%
%
%The goal of this experiment is not to demonstrate the effectiveness of the spectral clustering algorithm. We mainly want to investigate the following, 
%\begin{itemize}
%\item to what extend the quality of clustered results will be affected by using spectral sparsification.
%\item saving of communication costs by using spectral sparsifier.
%\item the affect of constants in algorithms of the message passing/blackboard model.
%\end{itemize}
%
%
%\subsection{The Setup}
%\paragraph{Reference Algorithms}
%We compare different algorithms in our experiment.

%Note that we can also run \MM~ in the blackboard model.

Besides giving the visualized results of these algorithms on various datasets, we also measure the qualities of the results via the {\em normalized cut}, defined as 
\[
\ncut(A_1, \ldots, A_{k}) = \frac{1}{2}\sum_{i\in[k]}\frac{w(A_i, V\backslash A_i)}{\vol(A_i)},
\]
 which is a standard objective function to be minimized for spectral clustering algorithms. 
%We will compare the communication costs of these algorithms in different settings.

%We also compare the total communication costs of different algorithms/models. As the unit does not matter in our case, we normalize all communication costs by the cost of \baseline.  Whenever possible, we will visualize the clustered results.

We implemented the algorithms using multiple languages, including Matlab, Python and C++. Our experiments were conducted on an IBM NeXtScale nx360 M4 server, which is equipped with 2 Intel Xeon E5-2652 v2 8-core processors, 32GB RAM and 250GB local storage.


\subsection{Datasets.}
We test the algorithms in the following real and synthetic datasets, which is visualized in \figref{visualization}.


\begin{figure}[h]
     \centering
     \subfigure[\twomoons]{\includegraphics[width=0.23\textwidth]{twomoons-14000-original.png}\label{fig:twomoons}}
     ~~
     \subfigure[\gauss]{\includegraphics[width=0.23\textwidth]{gauss-10000-original.png}\label{fig:gauss}}
     ~~
     \subfigure[\sculpture]{\includegraphics[width=0.13\textwidth,height=0.16\textwidth]{sculpture-11680-original.jpg}\label{fig:sculpture}}
     \caption{Visualization of the datasets for our experiments.}
     \label{fig:visualization}
\end{figure}



\vspace{-1mm}
\begin{itemize}
\item \twomoons : this dataset contains $n=14,000$ coordinates in $\mathbb{R}^2$. We consider each point to be a vertex. For any two vertices $u, v$, we add an edge with weight $w(u,v) = \exp\{-\|u-v\|_2^2/\sigma^2\}$ with $\sigma = 0.1$ when one vertex is among the $7000$-nearest points of the other.  This construction results in a graph with about $110,000,000$ edges.

\item  \gauss : this dataset contains $n = 10,000$ points in $\mathbb{R}^2$. There are $4$ clusters in this dataset, each generated using a Gaussian distribution. We construct a complete graph as the similarity graph.  For any two vertices $u, v$, we define the weight $w(u,v) = \exp\{-\|u-v\|_2^2/\sigma^2\}$ with $\sigma = 1$. The resulting graph has about $100,000,000$ edges.

\item \sculpture : a photo of \textit{The Greek Slave}~\footnote{Available in e.g., \url{http://artgallery.yale.edu/collections/objects/14794}}. We use an $80\times 150$ version of this photo where each pixel is viewed as a vertex. To construct a similarity graph, we map each pixel to a point in $\mathbb{R}^5$, i.e., $(x, y, r, g, b)$, where the latter three coordinates are the RGB values. For any two vertices $u, v$, we  put an edge between $u, v$ with weight $w(u,v) = \exp\{-\|u-v\|_2^2/\sigma^2\}$ with $\sigma = 0.5$ if one of $u, v$ is among the $5000$-nearest points of the other. This results in a graph with about $70,000,000$ edges.
\end{itemize}
\vspace{-1mm}
In the distributed model edges are randomly partitioned across $s$ sites. 

%\vspace{-1.5mm}



\subsection{Results on clustering quality}
%{\em Quality.} \
\begin{figure*}[ht]
     \centering
     \subfigure[\baseline]{\includegraphics[width=0.2\textwidth]{twomoons-14000-original-clustered.png}\label{fig:twomoons-clustered-original}}
     \subfigure[\MM]{\includegraphics[width=0.2\textwidth]{twomoons-14000-sparsify-clustered-15.png}\label{fig:twomoons-clustered-sparsify}}
     \subfigure[\blackboard]{\includegraphics[width=0.2\textwidth]{twomoons-14000-chain-clustered.png}\label{fig:twomoons-clustered-chain}}
     \caption*{\twomoons, $k = 2$;}

\subfigure[\baseline]{\includegraphics[width=0.2\textwidth]{gauss-10000-original-clustered.png}\label{fig:gauss-clustered-original}}
     \subfigure[\MM]{\includegraphics[width=0.2\textwidth]{gauss-10000-sparsify-clustered-15.png}\label{fig:gauss-clustered-sparsify}}
     \subfigure[\blackboard]{\includegraphics[width=0.2\textwidth]{gauss-10000-chain-clustered.png}\label{fig:gauss-clustered-chain}}
     \caption*{\gauss, $k = 4$}


     \subfigure[\baseline]{\includegraphics[width=0.2\textwidth,height=0.2\textwidth]{sculpture-11680-original-clustered.png}\label{fig:sculpture-clustered-original}}  
     \subfigure[\MM]{\includegraphics[width=0.2\textwidth,height=0.2\textwidth]{sculpture-11680-sparsify-clustered-15.png}\label{fig:sculpture-clustered-sparsify}}
     \subfigure[\blackboard]{\includegraphics[width=0.2\textwidth,height=0.2\textwidth]{sculpture-11680-chain-clustered.png}\label{fig:sculpture-clustered-chain}}
     \caption*{\sculpture, $k = 3$. }


     
     \caption{Visualization of the results on \twomoons, \gauss\ and \sculpture. In the message passing model each site samples $5 n$ edges; in the blackboard model all sites jointly sample $10n$ edges (in \twomoons~ and \gauss) or $20n$ edges (in \sculpture) and the chain has length $18$. $s = 15$.}
     \label{fig:quality-1}
\end{figure*}

We visualize the clustered results for 
the \twomoons, \gauss\ and \sculpture\ in Figure~\ref{fig:quality-1}.
% and visualize the clustered results for \gauss\ and \sculpture in Figure~\ref{fig:quality-2}.
It can be seen that \baseline, \MM\ and \blackboard\ give results of very similar qualities.  For simplicity, here we only present the visualization for $s=15$. Similar results were observed when we varied the values of $s$.  
%\he{To Qin: Do you plan to have two titles (Results \& Quality)?}


% \begin{figure*}[h]
%      \centering
% \subfigure[\baseline]{\includegraphics[width=0.3\textwidth]{gauss-10000-original-clustered.png}\label{fig:gauss-clustered-original}}
%      \subfigure[\MM]{\includegraphics[width=0.3\textwidth]{gauss-10000-sparsify-clustered-15.png}\label{fig:gauss-clustered-sparsify}}
%      \subfigure[\blackboard]{\includegraphics[width=0.3\textwidth]{gauss-10000-chain-clustered.png}\label{fig:gauss-clustered-chain}}
%      \caption*{\gauss, $k = 4$}


%      \subfigure[\baseline]{\includegraphics[width=0.2\textwidth]{sculpture-11680-original-clustered.png}\label{fig:sculpture-clustered-original}}  
%      \subfigure[\MM]{\includegraphics[width=0.2\textwidth]{sculpture-11680-sparsify-clustered-15.png}\label{fig:sculpture-clustered-sparsify}}
%      \subfigure[\blackboard]{\includegraphics[width=0.2\textwidth]{sculpture-11680-chain-clustered.png}\label{fig:sculpture-clustered-chain}}
%      \caption*{\sculpture, $k = 3$. }

%      \caption{Visualization of results on \gauss\ and \sculpture; in the message passing model each site samples $5 n$ edges; in the blackboard model all sites jointly sample $10n$ (in \gauss) or $20n$ (in \sculpture) edges and the chain has length $18$.}
%      \label{fig:quality-2}
% \end{figure*}


We also compare the normalized cut (ncut) values of the clustering results of different algorithms.  The results are presented in Figure \ref{fig:quality}. In all datasets, the ncut values of different algorithms are very close. The ncut value of \MM\ slightly decreases when we increase the value of $s$, while the ncut value of \blackboard\ is independent of $s$.
%We comment that in general, it is difficult to compare \MM\ and \blackboard\ directly because they are affected by different parameters.


\begin{figure*}[!ht]
  \centering
  \subfigure[\twomoons]{\includegraphics[width=0.33\textwidth]{twomoons-14000-ncut.png}\label{fig:twomoons-quality}}\hspace*{-1.1em}
  \subfigure[\gauss]{\includegraphics[width=0.31\textwidth]{gauss-10000-ncut.png}\label{fig:gauss-quality}}\hspace*{-1.1em}
  \subfigure[\sculpture]{\includegraphics[width=0.31\textwidth]{sculpture-11680-ncut.png}\label{fig:sculpture-quality}}\hspace*{-1.1em}
  \subfigure{\includegraphics[width=0.14\textwidth]{legend.png}}
     \caption{Comparisons on normalized cuts. In the message passing model, each site samples $5n$ edges; in each round of the algorithm in the blackboard model, all sites jointly sample $10n$ edges (in \twomoons~and \gauss) or $20n$ edges (in \sculpture) edges and the chain has length $18$.}
     \label{fig:quality}
\end{figure*}

%\textcolor{red}{To Jiecao: Can you put the color lines indicating baseline, message passing, and blackboard within one row in Pic 2? Withthis we can save some space.}

%\vspace{-1.5mm}

\subsection{Results on communication costs} 
\begin{figure*}[!ht]
     \centering
     \subfigure[\twomoons]{\includegraphics[width=0.3\textwidth]{twomoons-14000-communication.png}\label{fig:twomoons-communication}}
     \subfigure[\gauss]{\includegraphics[width=0.3\textwidth]{gauss-10000-communication.png}\label{fig:gauss-communication}}
     \subfigure[\sculpture]{\includegraphics[width=0.3\textwidth]{sculpture-11680-communication.png}\label{fig:sculpture-communication}}


     \subfigure[\twomoons]{\includegraphics[width=0.32\textwidth]{twomoons-14000-communication-2.png}\label{fig:twomoons-communication-2}}
     \subfigure[\gauss]{\includegraphics[width=0.32\textwidth]{gauss-10000-communication-2.png}\label{fig:gauss-communication-2}}
     \subfigure[\sculpture]{\includegraphics[width=0.32\textwidth]{sculpture-11680-communication-2.png}\label{fig:sculpture-communication-2}}
     \caption{Comparisons on communication costs. In the message passing model, each site samples $5n$ edges; in each round of the algorithm in the blackboard model, all sites jointly sample $10n$ (in \twomoons~and \gauss) or $20n$ (in \sculpture) edges and the chain has length $18$. }
     \label{fig:communication}
\end{figure*}

We compare the communication costs of different algorithms in Figure \ref{fig:communication}. We observe that while achieving similar clustering qualities as \baseline, both \MM\ and \blackboard\ are significantly more communication-efficient (by one or two orders of magnitudes in our experiments). We also notice that the value of $s$ does not affect the communication cost of \blackboard, while the communication cost of \MM\ grows almost linearly with $s$; when $s$ is large, \MM\ uses significantly more communication than \blackboard. These confirm our theory.  %In Figure~\ref{fig:mm-const} and Figure~\ref{fig:blackboard-const}   in Appendix~\ref{sec:parameters} we present how the performance of \MM\ and \blackboard\ are affected by their parameters.

%
%
%\vspace{-1.5mm}
%\paragraph{Summary.}  From our experimental results we conclude that \MM\ and \blackboard\ achieve similar clustering quality as the native algorithm \baseline, while significantly reduce the communication cost.  When the number of sites is large, \blackboard\ is more communication efficient than \MM, as predicted by our theory.



\subsection{Parameters in \MM\ and \blackboard}
\label{sec:parameters}

Figure \ref{fig:mm-const} shows in \MM how the value of ncut is affected by the number of sites and the number of edges sampled in each site. 
Here, each site samples $cn$ edges. 
When $c=3$ and $s=1$, the ncut value diverges in all datasets. This is because with such a small $c$, the algorithm does not generate a valid sparsifier. In general, increasing $c$ or $s$ will slightly decrease the ncut value. But once they are above some thresholds, the ncut values of \MM\ and \baseline\ become very close.

Figure \ref{fig:blackboard-const} shows in \blackboard  how the ncut value is affected by the number of iterations and the number of edges sampled. When the number of iterations is set to be $5$, ncut values diverge in all datasets. This is because we cannot expect to generate a valid sparsifier by using such few iterations. It can be seen from \ref{fig:bb-gauss-constant} that for a fixed $c$, performing more iterations will help to reduce ncut values. From the same figure, one can also conclude that for fixed iterations, increasing $c$ also helps to reduce the ncut values.



\begin{figure*}[h!t]
     \centering
     \subfigure[\twomoons]{\includegraphics[width=0.3\textwidth]{twomoons-c.png}\label{fig:mm-twomoons-constant}}
     \subfigure[\gauss~dataset]{\includegraphics[width=0.3\textwidth]{gauss-c.png}\label{fig:mm-gauss-constant}}
     \subfigure[\sculpture]{\includegraphics[width=0.3\textwidth]{sculpture-c.png}\label{fig:mm-sculpture-constant}}
     \caption{The pictures above show the $\ncut$ values with respect to the values of $c$ and $s$ for the \MM\ algorithm. Here  
 each site samples $c n$ edges.}
     \label{fig:mm-const}
\end{figure*}


\begin{figure*}[h!t]
     \centering
     \subfigure[\twomoons]{\includegraphics[width=0.3\textwidth]{twomoons-iter.png}\label{fig:bb-twomoons-constant}}
     \subfigure[\gauss]{\includegraphics[width=0.3\textwidth]{gauss-iter.png}\label{fig:bb-gauss-constant}}
     \subfigure[\sculpture]{\includegraphics[width=0.3\textwidth]{sculpture-iter.png}\label{fig:bb-sculpture-constant}}
     \caption{The pictures above show how the $\ncut$ values are affected by the number of iterations and the value of $c$ for the \blackboard\ algorithm. Here 
all sites jointly sample $c n$ edges. }
     \label{fig:blackboard-const}
\end{figure*}







\section{Conclusion}
Existing OL2R solutions suffer from slow convergence and sub-optimal performance due to inefficient exploration and limited optimization strategies. Motivated by the success of offline models, we propose to estimate a pairwise learning to rank model on the fly, named as \model{}. 
Based on the model's pairwise order estimation confidence, exploration is performed only on the pairs where the ranker is still uncertain, i.e., \emph{divide-and-conquer}. 
We prove a sub-linear upper regret bound defined on the number of mis-ordered pairs, which directly links \model{}'s convergence with classical ranking evaluations.  
Our empirical experiments support our regret analysis and demonstrate significant improvement of \model{} over several state-of-the-art OL2R baselines. 

Our effort sheds light on moving more powerful offline learning to rank solutions online. Currently, our work is based on a single layer RankNet for analysis purposes. Following recent efforts of convergence analysis in deep learning \cite{zhou2019neural}, it is possible to extend \model{} with deep ranking models and directly optimize rank-based metrics (such as NDCG). Furthermore, most OL2R solutions focus on population-level ranker estimation; thanks to the improved learning efficiency by \model{}, it is possible for us to study individual-level ranking problems, e.g., personalized OL2R.

\section{Acknowledgements}
We want to thank the reviewers for their insightful comments. This work is based upon work supported by National Science Foundation under grant IIS-1553568 and IIS-1618948, and Google Faculty Research Award.


%%
%% The acknowledgments section is defined using the "acks" environment
%% (and NOT an unnumbered section). This ensures the proper
%% identification of the section in the article metadata, and the
%% consistent spelling of the heading.
% \begin{acks}
% To Robert, for the bagels and explaining CMYK and color spaces.
% \end{acks}

\bibliographystyle{ACM-Reference-Format}
\bibliography{reference}

\appendix
% This version of CVPR template is provided by Ming-Ming Cheng.
% Please leave an issue if you found a bug:
% https://github.com/MCG-NKU/CVPR_Template.

\documentclass[review]{cvpr}
%\documentclass[final]{cvpr}

\usepackage{times}
\usepackage{epsfig}
\usepackage{graphicx}
\usepackage{amsmath}
\usepackage{amssymb}
\usepackage{xcolor}
\usepackage{multirow}
\usepackage{multicol}
\usepackage{booktabs}
\usepackage{subcaption}
\usepackage{float}
\newcommand{\iie}{\emph{i.e.}}

% Include other packages here, before hyperref.

% If you comment hyperref and then uncomment it, you should delete
% egpaper.aux before re-running latex.  (Or just hit 'q' on the first latex
% run, let it finish, and you should be clear).
\usepackage[pagebackref=true,breaklinks=true,colorlinks,bookmarks=false]{hyperref}


\def\cvprPaperID{1430} % *** Enter the CVPR Paper ID here
\def\confYear{CVPR 2022}
%\setcounter{page}{4321} % For final version only


\begin{document}

%%%%%%%%% TITLE
\title{Supplementary for Decoupled One Stage Action Detection Network}

\author{First Author\\
Institution1\\
Institution1 address\\
{\tt\small firstauthor@i1.org}
% For a paper whose authors are all at the same institution,
% omit the following lines up until the closing ``}''.
% Additional authors and addresses can be added with ``\and'',
% just like the second author.
% To save space, use either the email address or home page, not both
\and
Second Author\\
Institution2\\
First line of institution2 address\\
{\tt\small secondauthor@i2.org}
}

\maketitle

%%%%%%%%% BODY TEXT

\section{Additional experiments}

\subsection{Detection performance (action agnostic)}
As a one-stage method, we use the person bounding boxes generated by our network instead of an off-the-shelf detector. We now investigate in detail at the detection performance. We report the performance in Table~\ref{detection} with a standard 0.5 IOU threshold. The off-the-shelf detector is Faster R-CNN framework with ResNeXt-101-FPN backbone from maskrcnn-benchmark, which is widely applied in two-stage methods [\textcolor{green}{48}, \textcolor{green}{38}, \textcolor{green}{26}]. The model is first pre-trained on ImageNet, then fine-tuned on MSCOCO dataset, and finally fine-tuned on AVA dataset for higher person detection precision. We can see that our person detection result is still lower than the specialized detector, which is the crucial reason that the performance of one-stage methods cannot surpass the state-of-the-art two-stage methods.

\subsection{Backbone modification}
We are the first to use transformer-based backbone, Swin-B, in this task. A tough nut is how to maintain a larger spatial resolution of the feature map due to the overlarge spatial downsampling rate in the original version. The downsampling of Swin-B is mainly from the patch merging layer followed by a swin transformer block. We propose two schemes to modify the Swin-B: (i) removing the last patch merging layer and its following swin transformer block; (ii) just removing the last patch merging layer and modifying the dimension of the weights of the last swin transformer block. Note that the last swin transformer block in scheme (ii) cannot be initialized from a pre-trained model and can only be trained from scratch. Their results are presented in Table \ref{backbone}. Scheme (i) is slightly higher than scheme (ii) and contains fewer parameters. Thus, we adopt scheme (i) in our method.

\section{Qualitative results}
We randomly visualize some cases of our model in Figure \ref{qualitative}. Our model is able to exploit the person-context relationships to recognize interaction categories such as ``watch sb" and ``listen to sb", which are inherently hard if only focus on the actor, as shown in the first row of Figure \ref{qualitative}. There are two failure detection cases in the second row, which shows that our detection is hard to handle crowded and dark scenes. 


\begin{table} [t]
\begin{center}
\small
%\resizebox{!}{1.1cm}{
\begin{tabular}{c|c}
\toprule
Method & mAP (IOU@0.5)  \\
\midrule
Off-the-shelf detector & 93.5 \\
Ours & 89.9 \\
\bottomrule
\end{tabular}
\end{center}
% \vspace{-5mm}
\caption{We perform classification-agnostic evaluation to evaluate the performance of our person detection and compare it with the off-the-shelf detector.}
% \vspace{-1mm}
\label{detection}
\end{table}

\begin{table} [t]
\begin{center}
\small
%\resizebox{!}{1.1cm}{
\begin{tabular}{c|c}
\toprule
Scheme & mAP (IOU@0.5)  \\
\midrule
 i & 28.8 \\
 ii & 28.5 \\
\bottomrule
\end{tabular}
\end{center}
\vspace{-5mm}
\caption{Comparison of backbone modification schemes.}
\vspace{-3mm}
\label{backbone}
\end{table}

\begin{figure}[t]
%\includegraphics[width=1.0\linewidth]{LaTeX/supplement.png}
\includegraphics[width=1.0\linewidth]{latex/qualitative.png}
\centering
\caption{Per category results for the proposed network and the baseline model on the validation set of AVA dataset.}
\label{qualitative}
%\vspace{-4mm}
\end{figure}

\begin{figure*}[!t]
%\includegraphics[width=1.0\linewidth]{LaTeX/supplement.png}
\includegraphics[width=0.95\linewidth]{latex/figure.jpeg}
\centering
\caption{Per category results for the proposed network and the baseline model on the validation set of AVA dataset.}
\label{visualization}
\end{figure*}

\section{Per category analysis}
The per category results for our method and the baseline (the full model without TransPC) are shown in Figure \ref{visualization}. Our method improves the baseline performance in about 50 out of 60 classes. We can see that the categories getting the largest performance boost are from interaction categories, e.g., ``hand clap", ``work on a computer", ``smoke", and ``listen to sb", which require more attention on the supporting actors and context.





% {\small
% \bibliographystyle{ieee_fullname}
% %\bibliography{egbib}
% }

\end{document}


\end{document}
\endinput

