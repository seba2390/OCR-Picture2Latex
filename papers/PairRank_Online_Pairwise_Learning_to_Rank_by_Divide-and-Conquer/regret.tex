\subsection{Regret Analysis}

We theoretically analyze the regret of \model{}, which is defined by the cumulative number of mis-ordered pairs in its proposed ranked list till round $T$. 
The key in analyzing this regret is to quantify how fast the model achieves certainty about its pairwise preference estimation in candidate documents. First, we define $E_t$ as the success event at round $t$:
\begin{equation*}
    E_t = \big\{ \forall (i, j) \in [L_t]^2, |\sigma({\mathbf{x}_{ij}^t}^\top\hat{\btheta}_t) - \sigma({\mathbf{x}_{ij}^t}^\top\btheta^*) | \leq \alpha_t\Vert\mathbf{x}_{ij}^t\Vert_{\bM_t^{-1}}\big\}.
\end{equation*}

Intuitively, $E_t$ is the event that the estimated $\hat{\theta}_t$ is ``close'' to the optimal model $\theta^*$ at round $t$. According to Lemma~\ref{lemma:cb}, it is easy to reach the following conclusion,

\begin{corollary}On the event $E_t$, it holds that $\sigma({\mathbf{x}_{ij}^t}^\top\btheta^*) > {1}/{2}$ if $(i, j) \in \mathcal{E}_c^t$.
\label{col}
\end{corollary} 

\model{} suffers regret as a result of misplacing a pair of documents, i.e., swapping a pair already in the correct order. Based on Corollary~\ref{col}, under event $E_t$, the certain rank order identified by \model{} is consistent with the ground-truth. 
As our partition design always places documents under a certain rank order into distinct blocks, under event $E_t$ the ranking order across blocks is consistent with the ground-truth. In other words, regret only occurs when ranking documents within each block.

% Let $\Delta_{\min} = \min_{t\in T}\min_{(m, n) \in [L_t]^2}| \sigma({\xmnt}^\top\btheta^*) - \frac{1}{2}|$, which represents the smallest gap between any pair of documents associated to the same query over time (across all queries). 

To analyze the regret caused by random shuffling within each block, we need the following technical lemma derived from random matrix theory. We adapted it from Equation (5.23) of Theorem 5.39 from \cite{vershynin2010introduction}.

\begin{lemma}
\label{lemma:matrix}
Let $A \in \mathbb{R}^{n \times d}$ be a matrix whose rows $A_i$ are independent sub-Gaussian isotropic random vectors in $\mathbb{R}^d$ with parameter $\sigma$, namely $\mathbb{E}[\text{exp}(x^\top (A_i - \mathbb{E}[A_i])] \leq \text{exp}(\sigma^2\Vert x \Vert ^2 / 2)$ for any $x \in \mathbb{R}^d$. Then, there exist positive universal constants $C_1$ and $C_2$ such that, for every $t \geq 0$, the following holds with probability at least $1 - 2\text{exp}(-C_2t^2), \text{where } \epsilon = \sigma(C_1\sqrt{d/n} + {t}/{\sqrt{n}})$: $\Vert A^\top A/n - \mathbf{I}_d\Vert \leq \text{max}\{\epsilon, \epsilon^2\}$.
\end{lemma}

The detailed proof can be found in \cite{vershynin2010introduction}. We should note the condition in Lemma~\ref{lemma:matrix} is not hard to satisfy in OL2R: at every round, the ranker is serving a potentially distinct query; and even for the same query, different documents might be returned at different times. This gives the ranker a sufficient chance to collect informative observations for model estimation.  Based on Lemma \ref{lemma:matrix}, we have the following lemma, which provides a tight upper bound of the probability that \model{}'s estimation of the pairwise preference is an uncertain rank order.

\begin{lemma}
At round $t \geq t^\prime$, with $\delta_1 \in (0, \frac{1}{2})$, $\delta_2 \in (0, \frac{1}{2})$, $\beta \in (0, \frac{1}{2})$, and $C_1$, $C_2$ defined in Lemma~\ref{lemma:matrix}, under event $E_t$, the following holds with probability at least $1 - \delta_2$: $\forall (i, j) \in [L_t]^2$, $\mathbb{P}\big((i, j) \in \mathcal{E}_u^t\big) \leq \frac{8k_{\mu}^2\Vert\xijt\Vert_{\bM_t^{-1}}^2}{(1 - 2\beta) c_{\mu}^2\Delta_{\min}^2}\log{\frac{1}{\delta_1}}$, with 
$t^\prime = \Big(\frac{c_1\sqrt{d} + c_2\sqrt{\log(\frac{1}{\delta_2})} + abd\sqrt{\frac{o_{\text{max}}u^2}{d\lambda}}}{\lambda_{\text{min}}(\bSigma)}\Big)^2 + \frac{2ab\log({1}/{\delta_1^2}) + 8a\lambda Q^2 - \lambda}{\lambda_{\text{min}}(\bSigma)}$, 
where $\Delta_{\min} = \min\limits_{t\in T, (i, j) \in [L_t]^2}| \sigma({\xijt}^\top\btheta^*) - \frac{1}{2}|$ representing the smallest gap of pairwise difference between any pair of documents associated to the same query over time (across all queries), $a = {4k_{\mu}^2 u^2}/({\beta^2c_{\mu}^2\Delta_{\text{min}}^2})$, and $b = R^2 + 4\sqrt{\lambda}QR$, 
\label{lemma:uncertain}
\end{lemma}


% \begin{lemma} On the event $E_t$, with $\delta \in (0, \frac{1}{2})$, $\forall (i, j) \in [L_t]^2$, $\mathbb{P}\big((i, j) \in \mathcal{E}_u^t\big) \leq \frac{\Vert\mathbf{x}_{ij}\Vert_{M_t^{-1}}^2}{8 c_{\mu}^2\Delta_{\min}^2}\log{\frac{1}{\delta}}$.
% \label{lemma:uncertain}
% \end{lemma}
\begin{hproof}
According to the definition of certain rank order, a pairwise estimation $\sigma(\xijt^\top\hat{\btheta})$ is certain if and only if $|\sigma(\xijt^\top\hat{\btheta}) - 1/2| \geq \alpha_t\Vert \xijt\Vert_{\mathbf{M}_{t}^{-1}}$. By the reverse triangle inequality, the probability can be upper bounded by $\mathbb{P}\big(\big| |\sigma(\xijt^\top\hat{\btheta}) - \sigma(\xijt^\top\btheta^*)| - |\sigma(\xijt^\top\btheta^*) - 1/2|\big| \geq \alpha_t\Vert \xijt\Vert_{\mathbf{M}_{t}^{-1}}\big)$, which can be further bounded by Theorem 1 in \cite{abbasi2011improved}. The key in this proof is to obtain a tighter bound on the uncertainty of \model{}'s parameter estimation compared to the bound determined by $\delta_1$ in Lemma~\ref{lemma:cb}, such that its confidence interval on a pair of documents' comparison at round $t$ will exclude the possibility of flipping their ranking order, i.e., the lower confidence bound of this pairwise estimation is above 0.5.
\end{hproof}

In each round of result serving, as the model $\hat{\btheta}_t$ would not change until next round, the expected number of uncertain rank orders, denoted as $N_t=|\mathcal{E}_{u}^t|$, can be estimated by the summation of the uncertain probabilities over all possible pairwise comparisons under the current query $q_t$, e.g., $\mathbb{E}[N_t] = \frac{1}{2} (\sum_{(i, j) \in [L_t]^2}  \mathbb{P}((i, j) \in \mathcal{E}_u^t)$.

Denote $p_{t}$ as the probability that the user examines all documents in $\tau_t$ at round $t$, and let $p^* = \min_{1\leq t \leq T} p_{t}$ be the minimal probability that all documents in a query are examined over time. We present the upper regret bound of \model{} as follows.
\begin{theorem}
\label{theorem}
Assume pairwise query-document feature vector $\xijt$ under query $q_t$, where $(i, j) \in [L_t]^2$ and $t \in [T]$, satisfies Proposition 1. With  $\delta_1 \in (0, \frac{1}{2})$, $\delta_2 \in (0, \frac{1}{2})$, $\beta \in (0, \frac{1}{2})$, the $T$-step regret of \model{} is upper bounded by:
\begin{align*}
    R_T 
     \leq& R^{\prime} + (1 - \delta_1)(1 - \delta_2) {p^*}^{-2}\left( 2adL_{\max}\log(1 + \frac{o_{\max}Tu^2}{2d\lambda}) + aw\right)^2
\end{align*}
where $R^{\prime} = \tp L_{\max}^2 + (T - t')\big(\delta_2L_{\max}^2 - (1- \delta_2)\delta_1 L_{\max}^2\big)$, with $\tp$ and $a$ defined in Lemma \ref{lemma:uncertain}, and $w = \sum\nolimits_{s=\tp}^T ({(L_{\max}^2 - 2L_{\max})u^2 }/({\lambda_{\min}(\bM_s)})$, and $L_{\max}$ representing the maximum number of document associated to the same query over time.
% $t'$ satisfies Lemma 2, and 
By choosing $\delta_1 = \delta_2 = 1/T$, we have the expected regret at most $R_T \leq O(d\log^4(T))$.
\end{theorem}

\begin{hproof}
The detailed proof is provided in the appendix. Here, we only provide the key ideas behind our regret analysis.
The regret is first decomposed into two parts: $R^\prime$ represents the regret when either $E_t$ or Lemma~\ref{lemma:uncertain} does not hold, in which the regret is out of our control, and we use the maximum number of pairs associated to a query over time, $L_{\text{max}}$ to compute the regret. The second part corresponds to the cases when both events happen. Then, the instantaneous regret at round $t$ can be bounded by
\begin{align}
    r_t = \mathbb{E} \big[K(\tau_t, \tau_t^*)\big] = \sum\nolimits_{i=1}^{d_t}\mathbb{E}\big[\frac{(N_i^t + 1)N_i^t}{2}\big] \leq \mathbb{E}\big[\frac{N_t(N_t + 1)}{2}\big]
\end{align}
where $N_i^t$ denotes the number of uncertain rank orders in block $\mathcal{B}_i^t$ at round $t$, and $N_t$ denotes the total number of uncertain rank orders.
From the last inequality, it follows that in the worst case where the $N_t$ uncertain rank orders are placed into the same block and thus generate at most $({N_t^2 + N_t})/{2}$ mis-ordered pairs with random shuffling. This is because based on the blocks created by \model{}, with $N_t$ uncertain rank orders in one block, this block can at most have $N_t + 1$ documents. Then, the cumulative number of mis-ordered pairs can be bounded by the probability of observing uncertain rank orders in each round, which shrinks rapidly with more observations over time.
\end{hproof}

\begin{remark}[1] 
By choosing $\delta = 1/T$, the theorem shows the expected regret increases at a rate of $\mathcal{O}({\log^4{T}})$. In this analysis, we provide a gap-dependent regret upper bound of \model{}, where the gap $\Delta_{\min}$ characterizes the hardness of sorting the $L_t$ candidate documents at round $t$. As the matrix $M_t$ only contains information from observed document pairs, we adopt the probability of a ranked list is fully observed to tackle with the partial feedback \cite{kveton2015combinatorial, kveton2015tight}, which is a constant independent of $T$.
\end{remark}

\begin{remark}[2]
Our regret is defined over the number of mis-ordered pairs, which is the \emph{first} pairwise regret analysis for an OL2R algorithm, to the best of our knowledge. As we discussed before, existing OL2R algorithms optimize their own metrics, which can hardly link to any conventional rank metrics. As shown in \cite{Wang2018Lambdaloss}, most classical ranking evaluation metrics, such as ARP and NDCG, are based on pairwise document comparisons. Our regret analysis of \model{} connects its theoretical property with such metrics, which has been later confirmed in our empirical evaluations.   
\end{remark}