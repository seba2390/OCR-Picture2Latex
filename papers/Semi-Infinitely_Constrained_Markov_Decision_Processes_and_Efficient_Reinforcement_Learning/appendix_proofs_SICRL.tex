\section{Omitted Proofs in Section \ref{Section_Theory_SICRL}}
\label{Appendix_Proofs_SICRL}
First, we define some additional notations. 
Given a stationary policy $\pi$, we define the value function $V_\diamond^\pi(s)=\EB\paren{\sum_{t=0}^\infty \gamma^t r(s_t,a_t)|s_0=s}$, $V_\diamond^\pi=(V_\diamond^\pi(s_1), \ldots, V_\diamond^\pi(s_{|\gS|}))^\top\in \RB ^{|\gS|}$.
Thus we have $V^\pi_\diamond(\mu)=\mu^\top V_\diamond^\pi$.
Here $\diamond$ represents either the reward $r$ or cost $c_y$.
We use $Q_\diamond^\pi(s,a):=\EB\paren{\sum_{t=0}^\infty \gamma^t \diamond(s_t,a)}$ and $Q_\diamond^\pi=(Q_\diamond^\pi(s_1,a_1),...,Q_\diamond^\pi(s_{\gS},a_{|\gA|}))\in\RB^{|\gS|\cdot|\gA|}$ to denote the state-action value function. 
The local variance is defined as $\Var_P(V_\diamond^\pi)(s,a)=\EB_{s^\prime\sim P(\cdot|s,a)}(V_\diamond^\pi(s^\prime)-P(\cdot|s,a)V_\diamond^\pi)^2$.
We view $\Var_P(V^\pi)$ as vectors of length $|\gS|\cdot|\gA|$. 
We overload notation and let $P$ also refer to a matrix of size $(|\gS|\cdot |\gA|)\times |\gS|$, where the entry $P_{(s, a), s^{\prime}}$ is equal to $P(s^\prime|s,a)$. 
We also define $P^\pi$ to be the transition matrix on state-action pairs induced by a stationary policy $\pi$, namely:
$$P_{(s, a),\left(s^{\prime}, a^{\prime}\right)}^{\pi}:=P\left(s^{\prime}| s, a\right) \pi\left(a^{\prime} |s^{\prime}\right).
$$
We use $\widetilde V_\diamond^{\pi}(s), \widetilde Q_\diamond^\pi(s,a), {\Var_{\widetilde P}}(\widetilde{V}_\diamond^\pi)(s,a), \widetilde V_\diamond^\pi, \widetilde Q_\diamond^\pi,{\Var_{\widetilde P}}(\widetilde{V}^\pi), \widetilde P, \widetilde P^\pi$ to denote the value function, state-action value function, local variance, vector of the value function, vector of the state-action value function, vector of local variance, transition matrix, transition matrix on state-action pairs w.r.t. SICMDP $\widetilde M$, respectively.

\begin{lemma}\label{Lemma_Iteration_Comlexity_General}
    Suppose for all $t\in\{1,...,T\}$, 
    $$
    \frac{1}{1-\gamma}\sum_{s, a,s^\prime}z^{(t)}(s,a,s^\prime)c_{y^{(t)}}(s,a)- u_{y^{(t)}}\geq \max_{y\in Y}\left[\frac{1}{1-\gamma}\sum_{s, a,s^\prime}z^{(t)}(s,a,s^\prime)c_{y}(s,a)- u_{y}\right]-\epsilon,
    $$
    Then if we set $\eta=\epsilon$ and $T=O\left(\left[\frac{\mathrm{diam}(Y)|\gS|^2|\gA|}{(1-\gamma)\epsilon}\right]^m \right)$,
    then SI-CRL is guaranteed to output a $2\epsilon$-optimal solution of Problem~\ref{Problem_Optimistic_ELSIP}.
\end{lemma}
\proof{Proof of Lemma~\ref{Lemma_Iteration_Comlexity_General}}
For the convenience of presentation, then the problem can be written as
$$
\begin{aligned}
    \max_{z\in Z}\ &z^\top r\\
    \text{s.t.}\ &z^\top c_y\leq u_y,\ \forall y\in Y.
\end{aligned}
$$
Here $r,c\in[0,1]^{|\gS|^2|\gA|}$, $Z\subset\RB^{|\gS|^2|\gA|}$ is a feasible set defined by constraints other than the semi-infinite one.
Let $f(y,z)=z^\top c_y-u_y$, we note $f(y,z)$ is Lipschitz w.r.t. $y$ and the Lipschitz coefficient is $\beta:=\frac{2|\gS|^2|\gA|L_y}{1-\gamma}$.
WLOG, we also assume $Y$ is contained in a $\|\cdot\|_\infty$ ball with radius $R$ with $R\leq \frac{\mathrm{diam}(Y)}{2}$.
At iteration $t<T$, if we have $f(y^{(t)}, z^{(t)})\leq\epsilon$, then the algorithm terminates and we obtain a $2\epsilon$-optimal solution of Problem~\ref{Problem_Optimistic_ELSIP}.
Else we have $f(y^{(t)},z^{(t)})>\epsilon$.
Since $f(z,y)$ is $\beta$-Lipshitz in y, we can conclude $\forall z$, if $f(z,y)>\epsilon$ and $f(z,y^\prime)<0$, then $\|y-y^\prime\|_\infty>\epsilon/\beta$.
Define $B^{(t)}=\{\|y-y^{(t)}\|_\infty\leq \epsilon/2\beta\}$, as $f(y^{(t)},z^{(t)})>\epsilon$ and $f(y^{(t^\prime)},z^{(t)})\leq 0$, $t^\prime=1,...,t-1$, we have $B^{(t)}\cap\left(\cup_{t^\prime=1}^{t-1} B^{(t^\prime)}\right)=\emptyset$. 
Then by induction one may conclude $\{B^{(t^\prime)},t^\prime=1,...,t\}$ forms a $\epsilon/2\beta$-packing of $Y$.
Noting the fact that the $\epsilon/2\beta$-packing number of $Y$ is less than $\left(\frac{2R\beta}{\epsilon}\right)^m$, we complete the proof.

\endproof

\proof{Proof of Theorem~\ref{Theorem_Iteration_Complexity_Random_Search}.}
Since we have Lemma~\ref{Lemma_Iteration_Comlexity_General}, we only need to ensure that with probability at least $1-\delta$, 
for all $t\in\{1,...,T\}$, 
    $$
    \frac{1}{1-\gamma}\sum_{s, a,s^\prime}z^{(t)}(s,a,s^\prime)c_{y^{(t)}}(s,a)- u_{y^{(t)}}\geq \max_{y\in Y}\left[\frac{1}{1-\gamma}\sum_{s, a,s^\prime}z^{(t)}(s,a,s^\prime)c_{y}(s,a)- u_{y}\right]-\epsilon.
    $$
We adopt the notations introduced in the proof of Lemma~\ref{Lemma_Iteration_Comlexity_General}.
At the $t$ th iteration, let $y^*:=\argmax_{y\in Y}\left[\frac{1}{1-\gamma}\sum_{s, a,s^\prime}z^{(t)}(s,a,s^\prime)c_{y}(s,a)- u_{y}\right]$.
As $f(y,z)$ is $\beta$-Lipschitz w.r.t. $y$, then it suffices to ensure that with probability at least $1-\delta/T$ there exist $i\in\{1,...,M\}$ such that $\|y_i-y^*\|_\infty\leq \epsilon/\beta$.
As long as $\epsilon/\beta\leq \epsilon_0$, we simply need
$$
\PB\left(\exists i\in\{1,...,M\}, \|y_i-y^*\|_\infty\leq \epsilon/\beta\right)=1-\left(1-\left(\frac{\epsilon}{\beta R}\right)^m\right)^M\geq 1-\frac{\delta}{T}.
$$
The proof can be completed by basic algebra operations.
\endproof

\proof{Proof of Theorem~\ref{Theorem_Iteration_Complexity_Projected_GD}.}
Since we have Lemma~\ref{Lemma_Iteration_Comlexity_General}, we only need to ensure that for all $t\in\{1,...,T\}$, 
    $$
    \frac{1}{1-\gamma}\sum_{s, a,s^\prime}z^{(t)}(s,a,s^\prime)c_{y^{(t)}}(s,a)- u_{y^{(t)}}\geq \max_{y\in Y}\left[\frac{1}{1-\gamma}\sum_{s, a,s^\prime}z^{(t)}(s,a,s^\prime)c_{y}(s,a)- u_{y}\right]-\epsilon.
    $$
We adopt the notations introduced in the proof of Lemma~\ref{Lemma_Iteration_Comlexity_General}.
By Theorem 3.2 in \cite{bubeck2015convex}, the statement above is satisfied as long as $T_{PGA}\geq \frac{\beta^2R^2}{\epsilon^2}$.
The proof can be completed by basic algebra operations.
\endproof

\begin{lemma}\label{Lemma_Crude_Bound}
If Assumption \ref{Assumption_Two_Nonzero} is true and $M\in M_\delta$, we have 
$$
\left\|Q_r^\pi-\widetilde Q_r^\pi\right\|_\infty\leq \frac{2\gamma}{(1-\gamma)^2}\sqrt{\frac{\log 2/\delta}{2n}}
$$
\end{lemma}
\proof{Proof of Lemma~\ref{Lemma_Crude_Bound}.}
Given a stationary policy $\pi$, if Assumption \ref{Assumption_Two_Nonzero} is true and $M\in M_\delta$, 
$$\left\|\widetilde P(\cdot|s,a) -P(\cdot|s,a)\right\|_1 \leq 2\sqrt{\frac{\log 2/\delta}{2n}},\forall s\in \gS,a\in \gA,
$$
which implies
$$
\left\|(P-\widetilde P)V_r^\pi\right\|_\infty\leq \frac{2}{1-\gamma}\sqrt{\frac{\log 2/\delta}{2n}}.
$$
Then we have
$$\begin{aligned}
    \left\|Q_r^\pi-\widetilde Q_r^\pi\right\|_\infty&= \left\|\gamma\left(I-\gamma \widetilde{P}^{\pi}\right)^{-1}(P-\widetilde{P}) V_r^{\pi}\right\|_\infty\\
    &\leq \frac{\gamma}{1-\gamma}\left\|(P-\widetilde P)V_r^\pi\right\|_\infty\\
    &\leq \frac{2\gamma}{(1-\gamma)^2}\sqrt{\frac{\log 2/\delta}{2n}}
\end{aligned}
$$
\endproof

\begin{lemma}\label{Lemma_Bound_Variance}
Given a stationary policy $\pi$, when Assumption \ref{Assumption_Two_Nonzero} is true and $M\in M_\delta$, we have
$$\Var_{P}(V_r^\pi)\leq 2\Var_{\widetilde P}(\widetilde{V}_r^\pi) + \frac{6}{(1-\gamma)^2}\sqrt{\frac{\log 2/\delta}{2n}}+\frac{8\gamma^2}{(1-\gamma)^4}\frac{\log 2/\delta}{2n}.
$$
\end{lemma}
\proof{Proof of Lemma~\ref{Lemma_Bound_Variance}.}
For simplicity of notation, we drop the dependence on $\pi$.
By definition,
$$\begin{aligned}
\Var_P(V_r) &= \Var_P(V_r)-\Var_{\widetilde P}(V_r)+\Var_{\widetilde P}(V_r)\\
&= P(V_r)^2-(PV_r)^2-\widetilde P(V_r)^2 + (\widetilde P V_r)^2+\Var_{\widetilde P}(V_r)\\
&=(P-\widetilde P)(V_r)^2-\left[(PV_r)^2-(\widetilde PV_r)^2\right]+\Var_{\widetilde P}(V_r),
\end{aligned}
$$
where $(\cdot)^2$ means element-wise squares.
When Assumption \ref{Assumption_Two_Nonzero} is true and $M\in M_\delta$, by Lemma~\ref{Lemma_Crude_Bound},
$$
\begin{aligned}
\|(P-\widetilde P)(V_r)^2\|_\infty&\leq \frac{2}{(1-\gamma)^2}\sqrt{\frac{\log 2/\delta}{2n}}\\
\left\|\left[(PV_r)^2-(\widetilde PV_r)^2\right]\right\|_\infty&\leq \|PV_r +\widetilde PV_r\|_\infty\|PV_r -\widetilde PV_r\|_\infty\\
&\leq \frac{2}{1-\gamma}\left\|PV_r -\widetilde PV_r\right\|_\infty\\
&\leq \frac{4}{(1-\gamma)^2}\sqrt{\frac{\log 2/\delta}{2n}}.
\end{aligned}
$$
We also have
$$
\begin{aligned}
\Var_{\widetilde P}(V_r)&=\Var_{\widetilde P}(V_r-\widetilde{V_r}+\widetilde V_r)\\
&\leq 2\Var_{\widetilde P}(V_r-\widetilde{V_r}) + 2\Var_{\widetilde P} (\widetilde V_r)\quad \text{(AM–GM inequality)}\\
&\leq 2\left\|V_r-\widetilde{V_r}\right\|_\infty^2+2\Var_{\widetilde P} (\widetilde V_r)\\
&\leq \frac{8\gamma^2}{(1-\gamma)^4}\frac{\log 2/\delta}{2n}+2\Var_{\widetilde P} (\widetilde V_r)\quad \text{(Lemma \ref{Lemma_Crude_Bound})}.
\end{aligned}
$$
Therefore, we can get
$$\Var_{P}(V_r^\pi)\leq 2\Var_{\widetilde P}(\widetilde{V_r^\pi}) + \frac{6}{(1-\gamma)^2}\sqrt{\frac{\log 2/\delta}{2n}}+\frac{8\gamma^2}{(1-\gamma)^4}\frac{\log 2/\delta}{2n}.
$$
\endproof

\begin{lemma}\label{Lemma_Distance_between_P_tilde_P}
Let $p,\tilde p,\hat p\in[0,1]$ satisfy
$$
\begin{aligned}
|p-\hat p|&\leq \min\brc{\sqrt{\frac{2 \hat p(1-\hat p) \log 4/\delta}{n}}+\frac{4\log 4/\delta}{n},\sqrt{\frac{ \log 2/\delta}{2 n}}}\\
|\tilde p-\hat p|&\leq \min\brc{\sqrt{\frac{2 \hat p(1-\hat p) \log 4/\delta}{n}}+\frac{4\log 4/\delta}{n},\sqrt{\frac{ \log 2/\delta}{2 n}}}.
\end{aligned}
$$
Then
$$
|p-\tilde p|\leq \sqrt{\frac{8 p(1- p) \log 4/\delta}{n}}+4\paren{\frac{ \log 4/\delta}{n}}^{3/4}+\frac{8\log 4/\delta}{n}
$$
\end{lemma}

\proof{Proof of Lemma~\ref{Lemma_Distance_between_P_tilde_P}.}
Assume WLOG that $\hat p\geq p$.
Therefore,
$$\begin{aligned}
    |p-\hat p|&\leq \sqrt{\frac{2 p(1- p) \log 4/\delta}{n}}+\sqrt{\frac{2 (\hat p- p)(1- p) \log 4/\delta}{n}}+\frac{4\log 4/\delta}{n}\\
    &\leq \sqrt{\frac{2 p(1- p) \log 4/\delta}{n}}+\sqrt{\frac{2 \sqrt{\frac{ \log 2/\delta}{2 n}} \log 4/\delta}{n}}+\frac{4\log 4/\delta}{n}\\
    &\leq \sqrt{\frac{2 p(1-p) \log 4/\delta}{n}}+2^{1/4}\paren{\frac{ \log 4/\delta}{n}}^{3/4}+\frac{4\log 4/\delta}{n}.
\end{aligned}
$$
Similarly, we have
$$|\tilde p-\hat p|\leq \sqrt{\frac{2 p(1-p) \log 4/\delta}{n}}+2^{1/4}\paren{\frac{ \log 4/\delta}{n}}^{3/4}+\frac{4\log 4/\delta}{n}.
$$
Thus we may complete the proof using triangular inequality.
\endproof

\begin{lemma}\label{Lemma_Quasi_Bernstein}
Given a stationary policy $\pi$, suppose Assumption \ref{Assumption_Two_Nonzero} is true and $M\in M_\delta$, then 
$$
|(P-\widetilde P)V_r^\pi|\preceq \sqrt{\frac{8\Var_P(V_r^\pi)\log 4/\delta}{n}}+\frac{4}{1-\gamma}\paren{\frac{\log 4/\delta}{n}}^{3/4}+\frac{8\log 4/\delta}{n(1-\gamma)},
$$
where $\preceq$ means every element of LHS is less than or equal to the its counterpart in RHS.
\end{lemma}

\proof{Proof of Lemma~\ref{Lemma_Quasi_Bernstein}.}
Let $p=P(sa^+|s,a),\tilde p=\tilde P(sa^+|s,a)$. Applying Lemma \ref{Lemma_Distance_between_P_tilde_P} yields 
$$|p-\tilde p|\leq \sqrt{\frac{8 p(1-p) \log 4/\delta}{n}}+4\paren{\frac{ \log 4/\delta}{n}}^{3/4}+\frac{8\log 4/\delta}{n}.
$$
Assume WLOG that $V_r^\pi(sa^+)\geq V_r^\pi(sa^-) $.
Therefore we have
$$
\begin{aligned}
|(P(\cdot|s,a)-\tilde{P}(\cdot|s,a))^\top V_r^\pi|\leq &\sqrt{\frac{8 p(1- p) \log 4/\delta}{n}}(V_r^\pi(sa^+)-V_r^\pi(sa^-))+\frac{4}{1-\gamma}\paren{\frac{ \log 4/\delta}{n}}^{3/4}\\
&+\frac{8\log 4/\delta}{n(1-\gamma)}.
\end{aligned}
$$
Since
$$
\begin{aligned}
p(1- p)(V_r^\pi(sa^+)- V_r^\pi(sa^-))^2&=[ p V_r^\pi(sa^+)^2+(1- p) V_r^\pi(sa^-)^2]-[ p V_r^\pi(sa^+)+(1- p) V_r^\pi(sa^-)]^2\\
&=\Var_P(V_r^\pi)
\end{aligned}
$$
We may get
$$
|(P(\cdot|s,a)-\widetilde{P}(\cdot|s,a))^\top  V_r^\pi|\leq \sqrt{\frac{8\Var_P(V_r^\pi)(s,a)\log 4/\delta}{n}}+\frac{4}{1-\gamma}\paren{\frac{\log 4/\delta}{n}}^{3/4}+\frac{8\log 4/\delta}{n(1-\gamma)},
$$
which completes the proof.
\endproof

\begin{lemma}\label{Lemma_Bound_on_V_Same_Pi}
Given a stationary policy $\pi$, suppose Assumption \ref{Assumption_Two_Nonzero} is true and $M\in M_\delta$, then we have
$$\left\|V_r^\pi -\widetilde{V_r}^\pi\right\|_\infty\leq \frac{4}{(1-\gamma)^{3/2}}\sqrt{\frac{\log 4/\delta}{n}} +\frac{4\sqrt{6}}{(1-\gamma)^2}\left(\frac{\log 4/\delta}{n}\right)^{3/4}+ \frac{8}{(1-\gamma)^4}\left(\frac{\log 4/\delta}{n}\right)^{3/2}
$$
\end{lemma}
\proof{Proof of Lemma~\ref{Lemma_Bound_on_V_Same_Pi}.}
From Lemma \ref{Lemma_Simulation_Lemma}, Lemma \ref{Lemma_Norm_of_Inf_Horizon_Expectation}, Lemma \ref{Lemma_Bound_of_Weighted_Variance}, Lemma \ref{Lemma_Quasi_Bernstein} and the fact that $\left(I-\gamma \widetilde{P}^{\pi}\right)^{-1}$ has positive entries, we know
$$
\begin{aligned}
\left\|Q^\pi-\widetilde {Q}^\pi\right\|_\infty&=\gamma\left\|\left(I-\gamma \widetilde{P}^{\pi}\right)^{-1}(P-\widetilde{P}) V_r^{\pi}\right\|_\infty\\
&\leq \sqrt{\frac{8\log 4/\delta}{n}}\left\|\left(I-\gamma\widetilde{P}^{\pi}\right)^{-1}\sqrt{\Var_{P}(V_r^\pi)}\right\|_\infty+\frac{4}{(1-\gamma)^2}\left(\frac{\log 4/\delta}{n}\right)^{3/4}+\frac{8}{(1-\gamma)^2}\left(\frac{\log 4/\delta}{n}\right)\\
&\leq \sqrt{\frac{16\log 4/\delta}{n}}\left\|\left(I-\gamma\widetilde{P}^{\pi}\right)^{-1}\sqrt{\Var_{\widetilde P}(\widetilde{V_r}^\pi)}\right\|_\infty +\frac{4\sqrt{6}}{(1-\gamma)^2}\left(\frac{\log 4/\delta}{n}\right)^{3/4}+ \frac{8}{(1-\gamma)^3}\left(\frac{\log 4/\delta}{n}\right)\\
&\leq \frac{4}{(1-\gamma)^{3/2}}\sqrt{\frac{\log 4/\delta}{n}} +\frac{4\sqrt{6}}{(1-\gamma)^2}\left(\frac{\log 4/\delta}{n}\right)^{3/4}+ \frac{8}{(1-\gamma)^3}\left(\frac{\log 4/\delta}{n}\right).
\end{aligned}
$$
The proof is completed since $\left\|V_r^\pi -\widetilde{V_r}^\pi\right\|_{\infty}\leq\left\|Q^\pi -\widetilde{Q}^\pi\right\|_{\infty}$ by definitions.
\endproof

\begin{lemma}\label{Lemma_Bound_on_V_Same_Pi_Leading}
Suppose Assumption \ref{Assumption_Two_Nonzero} is true and $n>\frac{6\log 4/\delta}{(1-\gamma)^{5/2}}$, then with probability at least $1-2|\gS|^2|\gA|\delta$, we have
$$
\begin{aligned}
\left\|V_r^\pi -\widetilde{V_r}^\pi\right\|_\infty&\leq 12\sqrt{\frac{\log 4/\delta}{n(1-\gamma)^3}}\\
\left\|C^\pi -\widetilde{C}_y^\pi\right\|_\infty&\leq 12\sqrt{\frac{\log 4/\delta}{n(1-\gamma)^3}},\forall y\in Y\\
\end{aligned}
$$
\end{lemma}

\proof{Proof of Lemma~\ref{Lemma_Bound_on_V_Same_Pi_Leading}.}
When Assumption \ref{Assumption_Two_Nonzero} is true and $M\in M_\delta$, it follows from Lemma \ref{Lemma_Bound_on_V_Same_Pi} that
$$
\left\|V_r^\pi -\widetilde{V_r}^\pi\right\|_\infty\leq \frac{4}{(1-\gamma)^{3/2}}\sqrt{\frac{\log 4/\delta}{n}} +\frac{4\sqrt{6}}{(1-\gamma)^2}\left(\frac{\log 4/\delta}{n}\right)^{3/4}+ \frac{8}{(1-\gamma)^3}\left(\frac{\log 4/\delta}{n}\right).
$$
And by setting $n>\max\left\{\frac{36\log4/\delta}{(1-\gamma)^2},\frac{4\log4/\delta}{(1-\gamma)^3}\right\}$ we will get
$$\left\|V_r^\pi -\widetilde{V_r}^\pi\right\|_\infty\leq 12\sqrt{\frac{\log 4/\delta}{n(1-\gamma)^3}}.
$$
Similar arguments can be applied to bound $\left\|C_y^\pi-\widetilde{C}_y^\pi\right\|_\infty$. Since by Theorem \ref{Theorem_Feasible} we have 
$$\PB(M\in M_\delta)\geq 1-2|\gS|^2|\gA|\delta,
$$
the proof is completed.
\endproof


\proof{Proof of Theorem \ref{Lemma_Bound_on_V}.}
By Lemma \ref{Lemma_Bound_on_V_Same_Pi}, we know that with probability $1-2|\gS|^2|\gA|\delta$,
$$\begin{aligned}
\left\|V_r^{\tilde \pi}-\tilde V_r^{\tilde \pi}\right\|_\infty&\leq12\sqrt{\frac{\log 4/\delta}{n(1-\gamma)^3}}\\
\left\|V_r^{\pi^*}-\tilde V_r^{\pi^*}\right\|_\infty&\leq 12\sqrt{\frac{\log 4/\delta}{n(1-\gamma)^3}}.
\end{aligned}
$$
Thus
$$\begin{aligned}
|V_r^{\tilde \pi}(\mu)-\tilde V_r^{\tilde \pi}(\mu)|&\leq12\sqrt{\frac{\log 4/\delta}{n(1-\gamma)^3}}\\
|V_r^{\pi^*}(\mu)-\tilde V_r^{\pi^*}(\mu)|&\leq12\sqrt{\frac{\log 4/\delta}{n(1-\gamma)^3}}.
\end{aligned}
$$ 
Noting that $\tilde V_r^{\tilde\pi}(\mu)\geq\tilde V_r^{\pi^*}(\mu)$, we may get
$$
\begin{aligned}
V_r^{\pi^*}(\mu)-V_r^{\tilde \pi}(\mu)&\leq V_r^{\pi^*}(\mu)-\tilde V_r^{\pi^*}(\mu)+\tilde V_r^{\tilde \pi}(\mu)-V_r^{\tilde \pi}(\mu)\\
&\leq|V_r^{\pi^*}(\mu)-\tilde V_r^{\pi^*}(\mu)| + |\tilde V_r^{\tilde \pi}(\mu)-V_r^{\tilde \pi}(\mu)|\\
&\leq 24\sqrt{\frac{\log 4/\delta}{n(1-\gamma)^3}}.
\end{aligned}
$$

Similarly, when
$$|C_y^{\tilde \pi}(\mu)-\tilde C_y^{\tilde \pi}(\mu)|\leq12\sqrt{\frac{\log 4/\delta}{n(1-\gamma)^3}},\forall y\in Y,
$$
we may get 
$$C_y^{\tilde \pi}(\mu) - u_y \leq 12\sqrt{\frac{\log 4/\delta}{n(1-\gamma)^3}},\forall y\in Y.
$$
since $\tilde C_y^{\tilde \pi}(\mu)\leq u_y$.
\endproof
% \proof{Proof of Theorem \ref{Theorem_Sample_Complexity}.}
% Theorem \ref{Theorem_Sample_Complexity} is a direct corollary of Theorem \ref{Lemma_Bound_on_V}.
% \endproof

\proof{Proof of Theorem \ref{Theorem_Sample_Complexity_General}.}
The proof is nearly identical to the proof of Theorem 3 in \cite{LATTIMORE2014125}. The idea is to augment each state/action pair of the original MDP with $|\gS|-2$ states in the form of a binary tree as pictured in the diagram below. 

\begin{figure}[!htb]
    \centering
    \includegraphics[width=0.3\textwidth]{img/binary_tree.pdf}
    \label{Figure_Binary_Tree}
\end{figure}

The intention of the tree is to construct a SICMDP, $\bar M=\langle \bar \gS,\gA,Y,\bar P,\bar r,\bar c,u,\mu,\bar\gamma\rangle$ that with appropriate transition probabilities is functionally equivalent to $M$ while satisfying Assumption \ref{Assumption_Two_Nonzero}.
The rewards and costs in the added states are set to zero.
Since the tree has depth $d=O(\log_2|\gS|)$, it now takes $d$ time steps in the augmented SICMDP to change states once in the original SICMDP.
Therefore we must also rescale the discount factor $\bar \gamma$ by setting $\bar\gamma<\gamma^d$.
Now we have
$$
\begin{aligned}
|\bar \gS|&= O(|\gS|^2|\gA|)\\
\frac{1}{1-\bar \gamma}&=\frac{\log |\gS|}{1-\gamma}.
\end{aligned}
$$
Then we complete the proof by applying results in Theorem \ref{Theorem_Sample_Complexity}.
\endproof

\proof{Proof of Theorem \ref{Theorem_Sample_Complexity_General_Measure}.}
By Theorem \ref{Theorem_Chernoff_Inequality}, we have for any fixed $(s,a)\in \gS\times \gA$
$$
\begin{aligned}
\PB(n(s,a)<m\nu_{\min}/2)&\leq \PB(n(s,a)<m\nu(s,a)/2)\\
&\leq e^{-m\nu(s,a)}\left(\frac{em\nu(s,a)}{em\nu(s,a)/2}\right)^{\frac{m\nu(s,a)}{2}}\\
&=\left(\sqrt{\frac{e}{2}}\right)^{-\nu(s,a)m}\\
&\leq \left(\sqrt{\frac{e}{2}}\right)^{-\nu_{\min}m}
\end{aligned}
$$
\endproof
Let $m=\frac{2}{1-\log 2}\frac{\log 2|\gS||\gA|/\delta}{\nu_{\min}}$, we have $\PB(n(s,a)\geq m\nu_{\min}))\geq 1-\delta/2|\gS||\gA|$.
Therefore, with probability at least $1-\delta/2$, we can get
$$
n(s,a)>m\nu_{\min}, \forall (s,a)\in \gS\times \gA.
$$
Then our problem is reduced to the case that the offline dataset is generated by generative models.
The proof is completed by using results in Theorem \ref{Theorem_Sample_Complexity_General}.
