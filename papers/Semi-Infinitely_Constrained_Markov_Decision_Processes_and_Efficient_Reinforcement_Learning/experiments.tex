% We perform a set of numerical experiments to illustrate the SICMDP model and validate the SI-CRL algorithm.
We design two numerical examples: discharge of sewage and ship route planning.
Through a set of numerical experiments, we illustrate the SICMDP model and validate the efficacy of our proposed algorithms.
In particular, we find that the SICMDP framework greatly outperforms the CMDP baseline obtained by naively discretizing the original problem in modeling problems like Example~\ref{Example_Time_Evolving}, ~\ref{Example_Uncertain}.
We highlight that in the example of ship route planning the SI-CPO algorithm is adept at efficiently solving complex reinforcement learning tasks using modern deep reinforcement learning approaches.
% We implement our methods with Python and LP problems are solved using a full-featured university version of Gurobi \cite{gurobi}.
% Details of our implementation can be found in the appendix.
% All the experiments are run on a workstation with 8 CPUs and no GPU.
% % \subsection{Toy SICMDP}
% We consider a most simple SICMDP with $|S|=2$, $|A|=2$ and $Y=[0, 1]$. 
% Its MDP part is specified in Figure \ref{Figure_Toy_Env}, where $p\in(0.5, 1)$ and $\tau\ll 1$ is a small positive number. 
% For each $\gamma\in(0, 1)$, we design Lipschitz $c_y$ and $u_y$ such that the optimal policy takes $a_0$ with probability 0.9 and 0.5 on $s^0$ and $s^1$, respectively.
% For details of the construction of Toy SICMDP, one may refer to the appendix.
% To make our numerical results more reliable, we repeat all experiments in this subsection for $30$ times and report the average results.
% % \liangyu{what is the value of $L$? The definition of $c_y$ and $u_y$ should be explicitly stated (maybe in the appendix).}
% % \begin{figure}[htbp]
% %     \centering
% %     \includegraphics[width=0.7\linewidth]{img/toy_mdp_env.pdf}
% %     \caption{MDP part of Toy SICMDP: The triple means (action, probability, reward). The agent should always take action $a_0$ in both states if it sets aside the constraints.}
% %     \label{Figure_Toy_Env}
% % \end{figure}
% \begin{figure}[htbp]
% \begin{minipage}[h]{0.48\linewidth}
%     \centering
%     \includegraphics[height=35mm, width=70mm]{img/toy_mdp_env.pdf}
%     \caption{MDP part of Toy SICMDP: The triple means (action, probability, reward). The agent should always take action $a_0$ in both states if it sets aside the constraints.}
%     \label{Figure_Toy_Env}
% \end{minipage}
% \hspace{.15in}
% \begin{minipage}[h]{0.48\linewidth}
%     \centering
%     \includegraphics[height=39.5mm, width=39.5mm]{img/pollution_env.pdf}
%     \caption{(Discharge of Sewage) The satellite image is from NASA and is only for illustrative purposes. The icons represent the locations of the sewage outfalls.}
%     \label{Figure_Discharge_Env}
% \end{minipage}
%     % \caption{Left: (Toy SICMDP) MDP part of Toy SICMDP: The triple means (action, probability, reward). The agent should always take action $a_0$ in both states if it sets aside the constraints. Right: (Discharge of Sewage) The satellite image is from NASA (https://commons.wikimedia.org/wiki/File:L\"{u} Cshun\_Port,2010,08.jpg) and only for illustrative purpose. The icons represent locations of the sewage outfalls.}
%     % \label{Figure_Env}
% \end{figure}
% First, we would like to check the efficacy of the SI-CRL algorithm.
% We set $T$ sufficiently large such that the algorithm is guaranteed to converge. 
% %and suppose an offline dataset generated by a generative model is available.
% Then we gradually increase $m$, the size of the dataset, and see whether SI-CRL can recover the pre-defined optimal policy.
% The results are shown in Figure \ref{Figure_Toy_Gamma_Nu_Error}.
% % \ref{Figure_Toy_Logn_Error}.
% It can be noticed that as $m$ gets larger, the error term converges to zero, showing that our SI-CRL algorithm may effectively solve reinforcement learning problems for SICMDPs.
% % \begin{figure}[htbp]
% %     \centering
% %     \includegraphics[width=0.6\linewidth]{img/logm_error.pdf}
% %     \caption{(Toy SICMDP) The policy returned by SI-CRL converges to the optimal solution as the dataset gets larger. The error term is defined as $\max\mybigbrace*{V^{\pi^*}(\mu)-V^{\hat\pi}(\mu),\sup_{y\in Y} C^{\hat\pi}_y(\mu)-u_y}$. Here we set $\delta=\frac{0.005}{|S|^2|A|}$. And the dataset we use is generated by a generative model.}
% %     \label{Figure_Toy_Logn_Error}
% % \end{figure}
% Second, we would like to validate the theoretical results in Section \ref{Section_Theory_SICRL}.
% Specifically, we investigate the sample complexity of SI-CRL for a fixed $(\epsilon,\delta)$ (See Definition \ref{Definition_PAC}) when $\gamma$ and $\nu_{\min}$ vary.
% $T$ is set to be sufficiently large as in the previous experiment.
% We present the results in Figure \ref{Figure_Toy_Gamma_Nu_Error}.
% The %relationship between the 
% logarithm of sample complexity vs. the transformed parameter of interest  is approximately linear with slope 1, which indicates our sample complexity bounds are correct and tight.
% \begin{figure}[htbp]
% \begin{minipage}[t]{0.32\linewidth}
%     \centering
%     \includegraphics[height=4cm, width=\linewidth]{img/logm_error.pdf}
% \end{minipage}
% \begin{minipage}[t]{0.32\linewidth}
%     \centering
%     \includegraphics[height=4cm, width=\linewidth]{img/transform_gamma_logm.pdf}
% \end{minipage}
% \begin{minipage}[t]{0.32\linewidth}
%     \centering
%     \includegraphics[height=4cm, width=\linewidth]{img/nu_log2m.pdf}
% \end{minipage}
% \hspace{-0.5cm}
%     \caption{(Toy SICMDP) Left: The policy returned by SI-CRL converges to the optimal solution as the dataset gets larger. The error term is defined as $\max\mybigbrace*{V^{\pi^*}(\mu)-V^{\hat\pi}(\mu),\sup_{y\in Y} C^{\hat\pi}_y(\mu)-u_y}$, the dataset is generated by generative models. Middle: Sample complexity of SI-CRL with varying $\gamma$; the dataset is generated by generative models. Right: Sample complexity of SI-CRL with varying $\nu_{\min}$; the dataset is generated by a probability measure. Here we set $\epsilon=0.01, \delta=\frac{0.005}{|S|^2|A|}$. Straight lines are obtained by linear regression.}
%     \label{Figure_Toy_Gamma_Nu_Error}    
% \end{figure}
% % \begin{figure}[htbp]
% % \begin{minipage}[t]{0.49\linewidth}
% %     \centering
% %     \includegraphics[width=\linewidth]{img/transform_gamma_logm.pdf}
% % \end{minipage}
% % \begin{minipage}[t]{0.49\linewidth}
% %     \centering
% %     \includegraphics[width=\linewidth]{img/nu_log2m.pdf}
% % \end{minipage}
% %     \caption{(Toy SICMDP) Left: Sample complexity of SI-CRL with varying $\gamma$; the dataset is generated by generative models. Right: Sample comlexity of SI-CRL with varying $\nu_{\min}$; the dataset is generated by a probability measure. Here we set $\epsilon=0.01, \delta=\frac{0.005}{|S|^2|A|}$. The straight line is obtained by linear regression.}
% %     \label{Figure_Toy_Gamma_Nu_Error}    
% % \end{figure}
% \vspace{-0.4cm}
\subsection{Discharge of Sewage}
\label{Experiment_Discharge_of_Sewage}
We consider a tabular sequential decision-making problem called discharge of sewage that is adapted from the literature of environmental science \citep{gorr1972optimal}.
Assume there are $|\gS|$ sewage outfalls in a region $[0, 1]^2$, and at each time point only one single outfall is active.
The active outfall would cause pollution in nearby areas, and the impact would decrease with Euclidean distance. 
Hence our state is the current active outfall. 
Given the current active outfall, the available actions are to switch to $|\gA|$ neighboring outfalls or do nothing.
Each switch would receive a negative reward representing the switching cost.
We need to figure out a switching policy to avoid over-pollution at each location of the region while minimizing the switching cost.
Clearly, this problem can be formulated as a SICMDP model with $Y=[0,1]^2$ and corresponding $c_y$ and $u_y$.
Specifically, we use $c_y(s, a)=c_y(s)=1/(1+\|y-s\|^2_2)$, where $s$ represents the position of the state (outfall).
Given a target state-occupancy measure $d$ we define $u_y=(1+\Delta)\sum_{s\in S} d(s)c_y(s)$, where $\Delta$ is a small positive number. 
% The feasibility of the resulting SICMDP is not guaranteed even if $\Delta=0$, we reject those infeasible environments and re-sample. 
The SICMDP would be nontrivial if we choose a suitable $\Delta$.
% For details on the construction of the Discharge of Sewage, one may refer to the appendix.
In the following numerical experiments, we assume that an offline dataset generated by a generative model is available.
% \liangyu{Explain how the "switch" works (how the transition dynamic is defined). Need explicit definition of $c_y,u_y$, and justify our design.}
% We use $\gamma=0.9$ and $|A|=4$ in the following experiments.
% \begin{figure}[htbp]
% \begin{minipage}
%     \includegraphics[width=0.4\linewidth]{img/pollution_env.pdf}
%     \caption{(Discharge of Sewage) The icons represent locations of the sewage outfalls.
%     The satellite image is from NASA (https://commons.wikimedia.org/wiki/File:L\"{u} Cshun\_Port,2010,08.jpg) and only for illustrative purpose.}
%     \label{Figure_Sewage_Env}
% \end{minipage}
% \begin{minipage}
%     \includegraphics[width=0.4\linewidth]{img/deep_pollution.jpg}
%     \caption{(Ship Route Planning) The island represents the ecological critical point. The green dashed line represents a feasible route, while the red dash-dot-dot line represents a more efficient but ecologically infeasible route.
%     The satellite image is from NASA (https://eol.jsc.nasa.gov/SearchPhotos/photo.pl?mission=ISS010\&roll=E\&frame=6732) and only for illustrative purpose.}
%     \label{Figure_Route_Env}
% \end{minipage}
% \end{figure}
\begin{figure}[htb]
\begin{minipage}[htb]{0.45\linewidth}
    \centering
    % \includegraphics[height=50mm, width=50mm]{img/node_base.pdf}
    \includegraphics[height=60mm, width=60mm]{img/pollution_env.pdf}
    \caption{(Discharge of Sewage) The icons represent locations of the sewage outfalls.
    The satellite image is from NASA and only for illustrative purpose.}
    \label{Figure_Sewage_Env}  
\end{minipage}
\hspace{.15in}
\begin{minipage}[htb]{0.45\linewidth}
    \centering
    % \includegraphics[height=50mm, width=50mm]{img/time_base_bar.pdf}
    \vspace{2.9cm}
    \includegraphics[height=60mm, width=60mm]{img/deep_pollution.jpg}
    \caption{(Ship Route Planning) The island represents the ecological critical point. The green dashed line represents a feasible route, while the red dash-dot-dot line represents a more efficient but ecologically infeasible route.
    The satellite image is from NASA and only for illustrative purpose.}
    \label{Figure_Route_Env}  
\end{minipage}
\end{figure}

First, we compare our SI-CRL algorithm with a naive discretization baseline~\ref{Remark_Baseline}.
In the baseline method, we only consider the constraints on a grid of $Y$ containing $N_{\text{baseline}}$ points, which allows us to model Discharge of Sewage as a standard CMDP problem with $N_{\text{baseline}}$ constraints.
The CMDP problem is then solved by the algorithm proposed in \cite{efroni2020explorationexploitation}.
Details of our implementation can be found in Appendix \ref{Appendix_Detials_of_Experiments}.
We visualize the quality of solutions of our proposed method and baseline method in Figure \ref{Figure_Sewage_Heat}.
It can be found that when $T=N_{\text{baseline}}$, the policy obtained by our proposed methods is of far better quality than the policy obtained by the baseline methods.

\begin{figure}[htbp]
\begin{minipage}[t]{0.45\linewidth}
    \centering
    % \includegraphics[height=4cm,width=4.5cm]{img/heat_exp.pdf}
    \includegraphics[height=6cm,width=6cm]{img/new_tabular_heat_SICRL.pdf}
\end{minipage}
% \begin{minipage}[t]{0.49\linewidth}
%     \centering
%     % \includegraphics[height=4cm,width=4.5cm]{img/heat_base.pdf}
%     \includegraphics[height=6cm,width=6.5cm]{img/new_tabular_heat_CRL.pdf}
% \end{minipage}
\begin{minipage}[t]{0.45\linewidth}
    \centering
    % \includegraphics[height=4cm,width=4.5cm]{img/heat_base.pdf}
    \includegraphics[height=6cm,width=6cm]{img/three_heat_base_small.pdf}
\end{minipage}
\begin{minipage}[t]{0.08\linewidth}
    \centering
    % \includegraphics[height=4cm,width=4.5cm]{img/heat_base.pdf}
    \includegraphics[height=5.8cm,width=0.725cm]{img/sicrl_color_bar.pdf}
\end{minipage}
    \caption{(Discharge of Sewage) Visualization of violation of constraints using SI-CRL (left) and baseline (right). The heat refers to the number $\log\paren*{(V^{\hat\pi}_{c_y}(\mu)-u_y)_++5\times10^{-6}}-\log(5\times10^{-6})$. Larger numbers mean a more serious violation of constraints. The red cross icons in the left two subfigures represent the $T=N_{\text{baseline}}=9$ checkpoints selected by the algorithms.}
    \label{Figure_Sewage_Heat}    
\end{figure}

\begin{figure}[htb]
\begin{minipage}[htb]{0.48\linewidth}
    \centering
    % \includegraphics[height=50mm, width=50mm]{img/node_base.pdf}
    \includegraphics[height=70mm, width=70mm]{img/new_tabular_plot_SICRL_iter.pdf}
    \caption{(Discharge of Sewage) Averaged error term of our proposed method and the baseline method over 100 seeds when $T$ and $N_{\text{baseline}}$ vary. ($\delta=\frac{0.005}{|S|^2|A|}$, $m$ sufficiently large)}
    \label{Figure_Sewage_Node}  
\end{minipage}
\hspace{.15in}
\begin{minipage}[htb]{0.48\linewidth}
    \centering
    % \includegraphics[height=50mm, width=50mm]{img/time_base_bar.pdf}
    \vspace{-0.1cm}
    \includegraphics[height=70mm, width=70mm]{img/new_tabular_bar_SICRL_iter.pdf}
    \caption{(Discharge of Sewage) Averaged time consumption of our method and the CMDP baseline to get a solution of given accuracy over 100 seeds. ($\delta=\frac{0.005}{|S|^2|A|}$, $m$ sufficiently large)}
    \label{Figure_Sewage_time}  
\end{minipage}
\end{figure}


An anti-intuitive phenomenon is that although in our method we need to deal with multiple LP problems while in the baseline we only solve one single LP problem, our method is still more time-efficient than the CMDP baseline.
Figure \ref{Figure_Sewage_time} indicates that our method takes less time to get a solution of given accuracy, which is evaluated by the error term $\max\brc{V^{\pi^*}_r(\mu)-V^{\hat\pi}_r(\mu),\sup_{y\in Y}V_{c_y}^{\hat \pi}(\mu)-u_y}$.
% $\sup_{y\in Y}V_{c_y}^{\hat \pi}(\mu)-u_y$.
The reason is that in SI-CRL we can solve LP problems with a dual simplex method, thus re-optimization after adding a new constraint is much faster than re-solving the LP problem from scratch~\citep{koberstein2005dual}.
And our method needs far fewer active constraints to attain the same accuracy as the baseline methods, see Figure \ref{Figure_Sewage_Node}.

\begin{figure}[htbp]
\begin{minipage}[t]{0.45\linewidth}
    \centering
        \vspace{0cm}
    \includegraphics[height=6cm,width=6cm]{img/new_tabular_heat_SICPO.pdf}
\end{minipage}
\begin{minipage}[t]{0.45\linewidth}
    \centering
        \vspace{0cm}
    % \includegraphics[height=4cm,width=4.5cm]{img/heat_base.pdf}
    % \includegraphics[height=4cm,width=4.5cm]{img/new_tabular_heat_CRPO.pdf}
    \includegraphics[height=6cm,width=6cm]{img/tabular_po_heat_base_no_color_bar.pdf}
\end{minipage}
\begin{minipage}[t]{0.08\linewidth}
    \centering
        \vspace{0cm}
    % \includegraphics[height=4cm,width=4.5cm]{img/heat_base.pdf}
    % \includegraphics[height=4cm,width=4.5cm]{img/new_tabular_heat_CRPO.pdf}
    \includegraphics[height=6.1cm,width=0.7625cm]{img/sicpo_color_bar.pdf}
\end{minipage}
    \caption{(Discharge of Sewage) Visualization of violation of constraints using SI-CPO (left) and the naive discretization baseline with $N_{\text{baseline}}=500$ solved by CRPO (right). The heat refers to the number $18(V^{\hat\pi}_{c_y}(\mu)-u_y)_+$. Larger numbers mean a more serious violation of constraints.}
    \label{Figure_Sewage_Heat_PG}    
\end{figure}

\begin{figure}[htb]
\begin{minipage}[htb]{0.96\linewidth}
    \centering
    \includegraphics[height=80mm, width=80mm]{img/tabular_pg_total_error.pdf}
    \caption{(Discharge of Sewage) Error term of SI-CPO and baselines versus the number of iterations.
    The solid line is the error term averaged over 20 random seeds.
    And we also provide the according error bars.}
    \label{Figure_Sewage_SICPO_Node}  
\end{minipage}
% \hspace{.15in}
\end{figure}



We also compare our SI-CPO algorithm with the pre-mentioned discretization baseline.
In this numerical experiment, we use softmax policy parametrization.
Our SI-CPO algorithm is instantiated with sample-based NPG as the policy optimization subroutine, a finite-horizon Monte-Carlo estimator as the policy evaluation subroutine, and random search with a grid of 100 points as the optimization subroutine.
Here the size of the random grid in each iteration is 100. 
For a fairer comparison, here the CMDP resulting from discretizing $Y$ is solved by CRPO \citep{xu2021crpo}.
% a type of primal policy optimization method for solving CMDPs.
This is also equivalent to a naive version of our SI-CPO algorithm where the inner problem is solved by searching over a fixed grid. (See Remark~\ref{Remark_random_search_vs_fixed_search}).
One may find the details of the implementation of our methods as well as the baselines in Appendix \ref{Appendix_Detials_of_Experiments}.

\begin{table}[t]
    \centering
\begin{tabular}{ccccc}
\hline
 &SI-CPO  & $N_{\text{baseline}}=250$ & $N_{\text{baseline}}=500$ & $N_{\text{baseline}}=1000$ \\
 \hline
time per iteration (s) & $0.19\pm 0.02$ & $0.23\pm 0.02$ & $0.30\pm 0.03$ & $0.45\pm 0.04$\\
\hline
\end{tabular}
\caption{(Discharge of Sewage) Time consumption of each iteration in SI-CPO and baselines.}
\label{Table_time_discharge}
\end{table}

The visualization of the solutions' quality can be found in Figure~\ref{Figure_Sewage_Heat_PG}, which shows that the policy obtained by SI-CPO is better than the policy obtained by the baseline solved by CRPO.
% We
In Figure~\ref{Figure_Sewage_SICPO_Node} we compare the convergence performance of SI-CPO to baselines that naively discretize $Y$ into grids with different sizes (different $N_{\text{baseline}}$s).
The resulting CMDPs are also solved by CRPO.
We may observe that the SI-CPO algorithm achieves a more rapid convergence measured by the number of iterations than all the naive discretization baselines no matter how large the grid is.
Also, Table~\ref{Table_time_discharge} suggests that the time consumption of a single iteration of SI-CPO is comparable to baseline methods.
% \begin{figure}
%     \centering
%     \includegraphics[width=0.6\linewidth]{img/node_base.pdf}
%     \caption{(Discharge of Sewage) Error term of our proposed method and the baseline method when $T$ and $N_{\text{baseline}}$ vary. Here we set $\delta=\frac{0.005}{|S|^2|A|}$ and $m$ sufficiently large.}
%     \label{Figure_Sewage_Node}
% \end{figure}
% \begin{figure}
%     \centering
%     \includegraphics[width=0.6\linewidth]{img/time_base_bar.pdf}
%     \caption{(Discharge of Sewage) Time consumption of our method and the CMDP baseline to get a solution of given accuracy. Here we set $\delta=\frac{0.005}{|S|^2|A|}$) and $m$ is set to be sufficiently large.}
%     \label{Figure_Sewage_time}
% \end{figure}

\subsection{Ship Route Planning}
To demonstrate the power of the SICMDP model and our proposed algorithms, we design a more complex continuous control problem with continuous state space named ship route planning.
This numerical example tackles a challenging task in maritime science \citep{wan2016four, wan2016pollution}: planning ship routes while ensuring their negative environmental impacts under an adaptive threshold.
Consider a ship sailing in a 2-dimensional area represented by the unit square $[0,1]^2$. 
At each time step $t$, the state of the ship is represented by its current position $s_t\in[0,1]^2$ and the action it takes is represented by the next heading angle $a_t\in[0,2\pi)$.
Given an outset $O\in [0,1]^2$ and a destination $D\in[0,1]^2$, at each time step $t$, we receive a negative reward $r(s_t)=-0.1 \times (\|s_t-D \|_2+1)$, and after we arrive at $D$ we will receive a large positive reward $5$.
The most efficient route is apparently a straight line.
However, we must take into account additional environmental concerns.
Specifically, the ship positioned at $s$ would cause pollution $c_y(s)=e^{-20\|y-s\|_2}$ to position $y$.
$c_y$ is designed to account for the greater pollution impact on areas closer to the ship.
The adaptive threshold of pollution is defined by $u_y=0.015+0.005\times e^{20\|y-\text{MPA}\|_2}$, where $\text{MPA}\in [0, 1]^2$ is an environmentally critical point that has special ecological significance, such as a habitat of endangered species or a natural heritage priority site.
The design of $u_y$ reflects the principle that we implement more strict pollution restrictions for nearer positions from the environmentally critical point $\text{MPA}$.
We would like to complement that due to the existence of a terminal state (the ship's destination), we set the discount factor $\gamma=1$.
Figure~\ref{Figure_Route_Env} provides a visual explanation of this numerical example.

We study the performance of an actor-critic version of SI-CPO called SI-CPPO in this example.
In SI-CPPO, the policy optimization subroutine is PPO, the policy evaluation subroutine is TD-learning and the optimization subroutine is a trust-region method \citep{conn2000trust}.
Both the policy and the value estimator are parametrized by deep neural networks.
We still consider the naive discretization baseline where the CMDPs resulting from discretization are solved by CRPO.
The implementation details of SI-CPPO and the baseline can be found in Appendix \ref{Appendix_Detials_of_Experiments}.
Figure~\ref{Figure_Route_Heat} is a visualization of the solutions attained via SI-CPPO and the discretization baseline.
While the baseline fails to generate a feasible route, SI-CPPO manages to plan a route that is both feasible and efficient.
We demonstrate the convergence performance of SI-CPPO and baselines with various $N_{\text{baseline}}$ in Figure~\ref{Figure_Route_Reward} and Figure~\ref{Figure_Route_Violat}.
It is shown that the convergence of baselines is very slow and the curves oscillate a lot.
And simply increasing $N_{\text{baseline}}$ does not help.
In contrast, our SI-CPPO algorithm rapidly converges to the optimal solution.
Also, Table~\ref{Table_time_route} shows that a single iteration of SI-CPPO consumes a similar amount of time compared with baseline methods.

\begin{table}[t]
    \centering
\begin{tabular}{ccccc}
\hline
 &SI-CPPO  & $N_{\text{baseline}}=250$ & $N_{\text{baseline}}=500$ & $N_{\text{baseline}}=1000$ \\
 \hline
time per iteration (s) & $11.39\pm 1.43$ & $8.91\pm 1.50$ & $10.05\pm 1.72$ & $11.78\pm 1.74$\\
\hline
\end{tabular}
\caption{(Discharge of Sewage) Time consumption of each iteration in SI-CPO and baselines.}
\label{Table_time_route}
\end{table}



\begin{figure}[htbp]
\begin{minipage}[t]{0.45\linewidth}
    \centering
    \vspace{0cm}
    % \includegraphics[height=4cm,width=4.5cm]{img/heat_exp.pdf}
    \includegraphics[height=6cm,width=6cm]{img/heat_SICPPO.pdf}
\end{minipage}
\begin{minipage}[t]{0.45\linewidth}
    \centering
    \vspace{0cm}
    % \includegraphics[height=4cm,width=4.5cm]{img/heat_base.pdf}
    \includegraphics[height=6cm,width=6cm]{img/heat_crpo_no_color_bar.pdf}
\end{minipage}
\begin{minipage}[t]{0.08\linewidth}
    \centering
    \vspace{0cm}
    % \includegraphics[height=4cm,width=4.5cm]{img/heat_base.pdf}
    \includegraphics[height=6.1cm,width=0.7625cm]{img/deep_color_bar.pdf}
\end{minipage}
    \caption{(Ship Route Planning) Visualization of routes and violation of constraints using SI-CPPO (left) and naive discretization with $N_{\text{baseline}}=1000$ (right). The heat refers to the number $5(V^{\hat\pi}_{c_y}(\mu)-u_y)_+$. Larger numbers mean a more serious violation of constraints. The green dashed line represents a feasible route induced by the SI-CPPO policy, while the red dash-dot line represents an infeasible route induced by the baseline policy. The blue icons in the center represent the ecologically critical points.}
    \label{Figure_Route_Heat}    

\begin{figure}[H]
\begin{minipage}[htb]{0.48\linewidth}
    \centering
    % \includegraphics[height=50mm, width=50mm]{img/node_base.pdf}
    \includegraphics[height=70mm, width=70mm]{img/deep_plot_SICPPO_Reward.pdf}
    \caption{(Ship Route Planning) Cumulative reward of SI-CPPO and baselines versus the number of iterations.
    The solid line is the cumulative reward averaged over 20 random seeds.
    And we also provide the according error bars.}
    \label{Figure_Route_Reward}  
\end{minipage}
\hspace{.15in}
\begin{minipage}[htb]{0.48\linewidth}
    \centering
    % \includegraphics[height=50mm, width=50mm]{img/time_base_bar.pdf}
    \includegraphics[height=70mm, width=70mm]{img/deep_plot_SICPPO_MaxConstraintViolation.pdf}
    \caption{(Ship Route Planning) Maximum constraint violation of SI-CPPO and baselines versus the number of iterations.
    The solid line is the maximum constraint violation averaged over 20 random seeds.
    And we also provide the according error bars.}
    \label{Figure_Route_Violat}  
\end{minipage}
\end{figure}
\end{figure}





