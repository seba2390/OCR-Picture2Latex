\documentclass[11pt]{article}

%% \usepackage[utf8]{inputenc} % allow utf-8 input
%% \usepackage[T1]{fontenc}    % use 8-bit T1 fonts

%
%
%
%\newcounter{example}[section]
%\renewcommand{\theexample}{\nthesection.\arabic{example}}
%\newenvironment{example}{
%     \refstepcounter{example}
%     {\vspace{1ex} \noindent\bf  Example  \theexample:}}{
%     \eop\vspace{1ex}} %\hspace*{\fill}\vspace*{1ex}}
%
%\newcounter{definition}[section]
%\renewcommand{\thedefinition}{\nthesection.\arabic{definition}}
%\newenvironment{definition}{
%     \refstepcounter{definition}
%     {\vspace{1ex} \noindent\bf  Definition  \thedefinition:}}{
%     \eop\vspace{1ex}} %\hspace*{\fill}\vspace*{1ex}}
%
%\newcounter{property}[section]
%\renewcommand{\theproperty}{\nthesection.\arabic{property}}
%\newenvironment{property}{
%     \refstepcounter{property}
%     {\vspace{1ex} \noindent\bf  Property  \theproperty:}}{
%     \eop\vspace{1ex}} %\hspace*{\fill}\vspace*{1ex}}
%
%
%\newcounter{theorem}[section]
%\renewcommand{\thetheorem}{\nthesection.\arabic{theorem}}
%\newenvironment{theorem}{\begin{em}
%        \refstepcounter{theorem}
%        {\vspace{1ex} \noindent\bf  Theorem  \thetheorem:}}{
%        \end{em}\eop\vspace{1ex}} %\hspace*{\fill}\vspace*{1ex}}
%
%\newcounter{lemma}[section]
%\renewcommand{\thelemma}{\nthesection.\arabic{lemma}}
%\newenvironment{lemma}{\begin{em}
%        \refstepcounter{lemma}
%        {\vspace{1ex}\noindent\bf Lemma \thelemma:}}{
%        \end{em}\eop\vspace{1ex}} %\hspace*{\fill}\vspace*{1ex}}
%
%\newcounter{proposition}[section]
%\renewcommand{\theproposition}{\nthesection.\arabic{proposition}}
%\newenvironment{proposition}{\begin{em}
%        \refstepcounter{proposition}
%        {\vspace{1ex}\noindent\bf Proposition \theproposition:}}{
%        \end{em}\eop\vspace{1ex}} %\hspace*{\fill}\vspace*{1ex}}
%
%\newcounter{corollary}[section]
%\renewcommand{\thecorollary}{\nthesection.\arabic{corollary}}
%\newenvironment{corollary}{\begin{em}
%        \refstepcounter{corollary}
%        {\vspace{1ex}\noindent\bf Corollary \thecorollary:}}{
%        \end{em}\eop\vspace{1ex}} %\hspace*{\fill}\vspace*{1ex}}
%
%\newcounter{remark}[section]
%\renewcommand{\theremark}{\nthesection.\arabic{remark}}
%\newenvironment{remark}{\begin{em}
%        \refstepcounter{remark}
%        {\vspace{1ex}\noindent\bf Remark \theremark:}}{
%        \end{em}\eop\vspace{1ex}} %\hspace*{\fill}\vspace*{1ex}}

% Small Title
%\newcommand{\stitle}[1]{\vspace{1ex} \noindent{\bf #1}}
\newcommand{\etitle}[1]{\vspace{1ex} \noindent{\underline{\em #1}}}
%\newcommand{\kw}[1]{{\ensuremath {\mathsf{#1}}}\xspace}


\newcommand{\ra}{\rightarrow}
\newcommand{\la}{\leftarrow}

\newcommand{\MCS}{{\sl MCS}\xspace}
\newcommand{\MCSs}{{\sl MCS}s\xspace}
\newcommand{\MCSv}{{\sl MCSv}\xspace}
\newcommand{\MCSe}{{\sl MCSe}\xspace}
\newcommand{\mcs}{\kw{mcs}}
\newcommand{\gsim}{\kw{sim}}
\newcommand{\asim}{\kw{score}}
\newcommand{\ME}{\kw{ME}}
\newcommand{\UN}{\kw{UN}}

%\renewcommand{\algorithmicrequire}{\textbf{Input:}}
%\renewcommand{\algorithmicensure}{\textbf{Output:}}

\newcommand{\smatch} {\kw{match}}
\newcommand{\getanchor} {\kw{anchor}-\kw{selection}}
\newcommand{\getM} {\kw{initM}}
\newcommand{\refine} {\kw{refine}}
\newcommand{\initialize} {\kw{anchor}-\kw{expansion}}

\newcommand{\btupframe} {\kw{Bottom}-\kw{up} \kw{Framework} }
\newcommand{\finddensetruss} {\kw{FindDenseTruss}}
\newcommand{\findmindensetruss} {\kw{FindMinDenseTruss}}
\newcommand{\findktruss} {\kw{FindKTruss}}

\newcommand{\buildktindex} {\kw{BuildKTIndex}}


\newcommand{\topdframe} {\kw{Top}-\kw{Down} \kw{Framework} }
%\newcommand{\topdkt} {\kw{Top}-\kw{Down}-\kw{KT} \kw{Search} }
\newcommand{\topdkt} {\kw{Improved}-\kw{KT} \kw{Search} }


\newcommand{\basic} {\kw{Basic}}
\newcommand{\kt} {\kw{KT}}
\newcommand{\ktd} {\kw{KTd}}
\newcommand{\ktdb} {\kw{KTdb}}
\newcommand{\ktb} {\kw{KTb}}
\newcommand{\ktbs} {\kw{KTbs}}
\newcommand{\rsgq} {\kw{RSGQ}}
\newcommand{\rsgqd} {\kw{RSGQ_D}}
\newcommand{\rsgqls} {\kw{RSGQ_{LS}}}
\newcommand{\rsgqs} {\kw{RSGQ_S}}
\newcommand{\dbase} {\kw{RSGQ_D}-\kw{exact}}
\newcommand{\lsbase} {\kw{RSGQ_{LS}}-\kw{exact}}
\newcommand{\getkccore} {\kw{ExactKCCore}}
\newcommand{\greedykccore} {\kw{GreedyKCCore}}
\newcommand{\rangeq} {\kw{RangeQuery}}
\newcommand{\diameterprune} {\kw{DiameterPrune}}
\newcommand{\dcdtree} {\kw{RSGQ_D}-\kw{cdtree}}
\newcommand{\lscdtree} {\kw{RSGQ_{LS}}-\kw{cdtree}}
\newcommand{\scdtree} {\kw{RSGQ_{S}}-\kw{cdtree}}
\newcommand{\plexrelated} {\kw{Plex Related Algorithms}}
\newcommand{\cdtree} {\kw{cd}-\kw{tree}}
\newcommand{\cdtrees} {\kw{cd}-\kw{trees}}

\newcommand{\tdtree} {\kw{td}-\kw{tree}}


\newcommand{\model} {\mathcal{M}}
\newcommand{\task} {\mathcal{T}}
\newcommand{\query} {\mathcal{Q}}
\newcommand{\support} {\mathcal{S}}
\newcommand{\community} {\mathcal{C}}
\newcommand{\neighbor} {\mathcal{N}}

\newcommand{\Fagg}{f_{\mathcal{A}}}
\newcommand{\Fcom}{f_{\mathcal{C}}}
\newcommand{\loss}{\mathcal{L}}
\newcommand{\relu}{\mathsf{ReLU}}
\newcommand{\sigmoid}{\mathsf{sigmoid}}
\newcommand{\softmax}{\mathsf{softmax}}


%dataset
\newcommand{\Reddit} {\kw{Reddit}}
\newcommand{\Cora} {\kw{Cora}}
\newcommand{\Citeseer} {\kw{Citeseer}}
\newcommand{\Arxiv} {\kw{Arxiv}}
\newcommand{\DBLP} {\kw{DBLP}}
\newcommand{\Facebook} {\kw{Facebook}}
\newcommand{\Coraciteseer} {\kw{Cora2Cite}}
\newcommand{\Citeseercora} {\kw{Cite2Cora}}

%metrics
\newcommand{\Fone} {\kw{F1}}
\newcommand{\Pre} {\kw{Pre}}
\newcommand{\Rec} {\kw{Rec}}
\newcommand{\Acc} {\kw{Acc}}
\newcommand{\distance} {\kw{dist}}
\newcommand{\Kernel} {\mathcal{K}_\theta}

% Task type
%baselines
\newcommand{\SGSC}{{$\texttt{SGSC}$}\xspace}
\newcommand{\SGDC}{{$\texttt{SGDC}$}\xspace}
\newcommand{\MGOD}{{$\texttt{MGOD}$}\xspace}
\newcommand{\MGDD}{{$\texttt{MGDD}$}\xspace}


%baselines
\newcommand{\ATC}{{$\texttt{ATC}$}\xspace}
\newcommand{\ACQ}{{$\texttt{ACQ}$}\xspace}
\newcommand{\Featrans}{{$\texttt{FeatTrans}$}\xspace}
\newcommand{\MAML}{{$\texttt{MAML}$}\xspace}
\newcommand{\Supervise}{{$\texttt{Supervised}$}\xspace}
\newcommand{\CGNPIP}{{$\texttt{CGNP-IP}$}\xspace}
\newcommand{\CTC}{{$\texttt{CTC}$}\xspace}
\newcommand{\CGNPMLP}{{$\texttt{CGNP-MLP}$}\xspace}
\newcommand{\CGNPGNN}{{$\texttt{CGNP-GNN}$}\xspace}
\newcommand{\ICSGNN}{{$\texttt{ICS-GNN}$}\xspace}
\newcommand{\PN}{{$\texttt{GPN}$}\xspace}
\newcommand{\Reptile}{{$\texttt{Reptile}$}\xspace}
\newcommand{\AQDGNN}{{$\texttt{AQD-GNN}$}\xspace}

\newcommand{\GCN}{{\sl GCN}\xspace}
\newcommand{\GNN}{{\sl GNN}\xspace}
\newcommand{\GIN}{{\sl GIN}\xspace}
\newcommand{\GAT}{{\sl GAT}\xspace}
\newcommand{\SAGE}{{\sl GraphSAGE}\xspace}


% \usepackage{enumerate}
\usepackage{enumitem}
\newtheorem{claim}{Claim}[section]
\newtheorem{lemma}[claim]{Lemma}
\newtheorem{assumption}{Assumption}
\newtheorem{theorem}{Theorem}
\newtheorem{proposition}{Proposition}
\newtheorem{corollary}{Corollary}
\theoremstyle{definition}
\newtheorem{definition}{Definition}
\newtheorem{example}{Example}
\theoremstyle{remark}
\newtheorem{remark}{Remark}


% \onehalfspacing

\usepackage{xr}
% \externaldocument{main}

\title{Semi-Infinitely Constrained Markov Decision Processes and Efficient Reinforcement Learning}

%% \title{How many labelers do you have? \\ A look at gold-standard labels and
%%   their weaknesses}

% \title{Modeling Aggregation and Uncertainty in Modern Data Analysis}

\author{Liangyu Zhang\thanks{Academy of Advanced Interdisciplinary Studies, Peking University; email: \texttt{zhangliangyu@pku.edu.cn}.} 
\and
Yang Peng\thanks{School of Mathematical Sciences, Peking University; email: \texttt{pengyang@pku.edu.cn}.}
\and
Wenhao Yang\thanks{Academy of Advanced Interdisciplinary Studies, Peking University; email: \texttt{yangwenhaosms@pku.edu.cn}.}
\and
Zhihua Zhang\thanks{School of Mathematical Sciences, Peking University; email: \texttt{zhzhang@math.pku.edu.cn}.}
}


\linespread{1.5}
\begin{document}
\maketitle

\begin{abstract}
  We propose a novel generalization of constrained Markov decision processes (CMDPs) that we call the \emph{semi-infinitely constrained Markov decision process} (SICMDP).
% In particular, we in a SICMDP model impose 
Particularly, %in an SICMDP model, 
we consider a continuum of constraints instead of a finite number of constraints as in the case of ordinary CMDPs.
We also devise two reinforcement learning algorithms
for SICMDPs that we call SI-CRL and SI-CPO.
SI-CRL is a model-based reinforcement learning algorithm.
Given an estimate of the transition model, we first transform the reinforcement learning problem into a linear semi-infinitely programming (LSIP) problem and then use the dual exchange method in the LSIP literature to solve it.
SI-CPO is a policy optimization algorithm.
Borrowing the ideas from the cooperative stochastic approximation approach, we make alternative updates to the policy parameters to maximize the reward or minimize the cost.
To the best of our knowledge, we are the first to apply tools from semi-infinitely programming (SIP) to solve constrained reinforcement learning problems.
We present theoretical analysis for SI-CRL and SI-CPO, identifying their iteration complexity and sample complexity.
We also conduct extensive numerical examples to illustrate the SICMDP model and demonstrate that our proposed algorithms are able to solve complex sequential decision-making tasks leveraging modern deep reinforcement learning techniques.
\end{abstract}

% \tableofcontents
\section{Introduction}
Reinforcement learning has achieved great success in areas such as Game-playing \citep{silver2018general,vinyals2019grandmaster}, robotics \cite{kober2013reinforcement}, large language models \citep{ouyang2022training}, etc.
However, due to safety concerns or physical limitations, in some real-world reinforcement learning problems, we must consider additional constraints that may influence the optimal policy and the learning process \citep{garcia2015comprehensive}.
% For example, a robotic arm must not take actions that may cause harm to itself or the environments.
A standard framework to handle such cases is the constrained Markov Decision Process (CMDP) \citep{altman1999constrained}.
Within the CMDP framework, the agent has to maximize
the expected cumulative reward while
obeying a finite number of constraints, which are usually in the form of expected cumulative cost criteria.

However, we are sometimes concerned with the problem with a continuum of constraints.
For example,
the constraints we meet might be time-evolving or subject to uncertain parameters, which
cannot be formulated as an ordinary CMDP
(see Examples \ref{Example_Time_Evolving} and  \ref{Example_Uncertain}).
In this paper we would study a generalized CMDP  
to address the above problem.  Because the constraints are not only infinite-number but also lie
in a continuous set,
the generalization is not trivial. Fortunately, we find that we can borrow the idea behind semi-infinite programming (SIP) \citep{remez1934determination, hettich1993semi} to deal with the semi-infinite constraints.
Accordingly, we propose \emph{semi-infinitely constrained Markov decision processes} (SICMDPs)
as a novel complement to the ordinary CMDP framework.
%More specifically,  an SICMDP model %, we consider 
%contains a continuum of constraints whereas an ordinary CMDP contains a finite number of constraints. 

%This generalization is natural but not trivial. However, we can brows the idea  
%The idea is quite natural and can be backtracked
%to the practice of extending linear programming to linear semi-infinite programming (LSIP) %\cite{remez1934determination, GobernaLSIO1998}.
%In addition, 
%As a complementary approach to the ordinary CMDP framework, 
%SICMDP can be used to model these problems  which cannot be described by a finite number of constraints
%that are not covered by .
%For example,
%the restrictions we consider can be time-evolving or subject to uncertain parameters
%, thus
%cannot be described by a finite number of constraints but a continuum of constraints 
%(see Examples \ref{Example_Time_Evolving} and  \ref{Example_Uncertain}).

We also present two reinforcement learning algorithms to solve SICMDPs called SI-CRL and SI-CPO, respectively.
SI-CRL is a model-based reinforcement learning algorithm designed for tabular cases, and SI-CPO is a policy optimization algorithm for non-tabular cases.
% and analyze its performance both theoretically and empirically.
The main challenge is that we need to deal with a continuum of constraints, thus reinforcement learning algorithms for ordinary CMDPs do not work anymore.
In SI-CRL, we tackle this difficulty by first transforming the reinforcement learning problem to an equivalent LSIP problem, which can then be solved using methods in the LSIP literature like the dual exchange methods \citep{Hu1990,reemtsen1998numerical}.
In SI-CPO, we resort to the idea of cooperative stochastic approximation developed in \cite{lan2020algorithms, wei2020comirror}.
As far as we know, we are the first to introduce tools from semi-infinitely programming (SIP) into the reinforcement learning community for solving constrained reinforcement learning problems.

% To the best of our knowledge, we are the first to apply tools from semi-infinitely programming (SIP) to solve reinforcement learning problems.
Furthermore, we give theoretical analysis for both SI-CRL and SI-CPO.
We decompose the error of SI-CRL into two parts: the statistical error from approximating the true SICMDP with an offline dataset and the optimization error due to the fact that the solution of the LSIP problem obtained by the dual exchange method is inexact.
On the optimization side, we show that the iteration complexity of SI-CRL is $O\left(\left\{\mathrm{diam}(Y)L\sqrt{|\gS|^2|\gA|m}/\left[(1-\gamma)\epsilon\right]\right\}^m\right)$.
On the statistical side, we show that the sample complexity of SI-CRL is $\widetilde O\left(\frac{|S|^2|A|^2}{\epsilon^2(1-\gamma)^3}\right)$ if the offline dataset is generated by a generative model, and $\widetilde O\left(\frac{|S||A|}{\nu_{\min} \epsilon^2(1-\gamma)^3}\right)$ if the dataset is generated by a probability measure $\nu$ as considered in \cite{chen2019information}.
Here $\widetilde O$ means that all logarithm terms are discarded.
For SI-CPO, things become a little more complicated because other than the statistical error and the optimization error, we also need to consider the function approximation error, which comes from imperfect policy parametrizations.
It is shown if the function approximation error can be controlled to $O(\epsilon)$ order, the iteration complexity of SI-CPO is $\widetilde{O}\left(\frac{1}{\epsilon^2(1-\gamma)^6}\right)$ and the sample complexity of SI-CPO is $\widetilde{O}(\frac{1}{\epsilon^4(1-\gamma)^{10}})$.
Here our iteration complexity bound is equivalent to a typical $\widetilde O(1/\sqrt{T})$ global convergence rate.

We perform a set of numerical experiments to illustrate the SICMDP model and validate our proposed algorithms.
Specifically, we examine two numerical examples, namely the discharge of sewage and ship route planning.
Through the discharge of sewage example, we show the advantage of the SICMDP framework over the CMDP baseline obtained by naive discretization in modeling realistic sequential decision-making problems.
Moreover, we demonstrate the effectiveness of the SI-CRL and SI-CPO algorithms in such tabular environments. 
In the ship route planning example, we illustrate the benefits of the SICMDP framework and the ability of the SI-CPO algorithm to address complex continuous control tasks involving continuous state spaces with modern deep reinforcement learning techniques.

% In summary, our contributions are listed as follows.
% First, we present the SICMDP model, which can be viewed as a generalization of the ordinary CMDP model.
% Second, we propose an algorithm to perform reinforcement learning for SICMDPs, which is called SI-CRL, and we believe that we are the first to apply tools from SIP
% to solve reinforcement learning problems.
% Third, we give a theoretical analysis of SI-CRL and identify both its sample complexity and iteration complexity.
% In addition, we perform numerical experiments to illustrate the SICMDP model and validate the SI-CRL algorithm.
% \{This paragraph can be removed!!! \}





\section{Related work}
\textbf{Related work}:
% Object detection related datasets/algo in non-medical domain
% Locally labeled CXR dataset
A few CXR datasets have localized abnormality annotations \cite{shih2019augmenting,filice2020crowdsourcing,jaeger2014two} that are curated manually. These are high quality gold standard ground truth datasets but tend to be smaller in scale (< 30,000 images) and have a narrow coverage, with typically only 1-2 labels. In addition, since most labeling efforts only have abnormality semantics attached, no direct relationships with the affected anatomical locations are available. 

%MEHDI: repeated concepts from above. I am removing the following: 

%The lack of anatomic semantics in the annotation is a limitation for complex multi-modal clinical reasoning work, e.g., differential diagnosis, since clinicians often integrate information along anatomical lines, and for downstream report generation tasks, which often requires describing not only the abnormality but also correctly communicate the location of the abnormalities (and medical devices) to the receiving clinicians. 

Two recent CXR datasets have labels for anatomies described in the reports. In \cite{datta2020dataset}, a small manually annotated dataset (2000 reports) included 10 abnormalities that are individually associated with 29 unique spatial locations (anatomies) at the report level. Another CXR dataset has automatically extracted abnormality and anatomy labels as disconnected concepts that are only correlated at the study level from  160,000 reports using a supervised NLP algorithm \cite{bustos2020padchest}. This was trained on a smaller set of manually annotated data. Neither datasets contain localized annotations for the associated CXR images, nor any comparison relation annotations between sequential exams, both of which are available in the Chest ImaGenome dataset. In Table \ref{tab:related}, we present a comparison of our Chest ImagGenome dataset with other datasets available in the literature.

% Table -- Kashyap

% MEdical imaging datasets to go here: Discussed that we will only focus on cxr datasets that are available for this paper. 
% \caption{\color{red} Kashyap, feel free to continue with the table. We should remove the questionmarks and add a line for our dataset (since all others are not graph). For longer text, using abbreviations and explaining them in the caption often works better. If fill in the values is not possible, it is better to remove the table altogether.}


\begin{table}[t!]
\caption{Summary of existing chest X-ray datasets}
\resizebox{\textwidth}{!}{%
\begin{tabular}{@{}lllllllll@{}}
\toprule
\textbf{Dataset} & \textbf{Annotation Level} & \textbf{Annotation Method} & \textbf{Num Labels} & \textbf{Anatomy Labeled} & \textbf{Graph} & \textbf{Dataset Size} & \textbf{Temporal Labels} & \textbf{Reports} \\ \midrule
SIIM-ACR Pneumothorax Segmentation \cite{filice2020crowdsourcing} & Segmentation & Manual + augmented & 1 & No & No & 12,047 & No & No \\
RSNA Pneumonia Detection Challenge   \cite{shih2019augmenting} & Bounding Boxes & Manual & 1 & No & No & 30,000 & No & No \\
Indiana University Chest X-ray collection \cite{demner2016preparing} & Global & Automated & 10 & No & No & 3,813 & No & Yes \\
NIH CXR dataset \cite{wang2017chestx} & Global & Automated & 14 & No & No & 112,120 & No & No \\
PLCO \cite{team2000prostate} & Global & Automated & 24 & Yes & No & 236,000 & Yes & No \\
Stanford CheXpert \cite{irvin2019chexpert} & Global & Automated & 14 & No & No & 224,316 & No & No \\
MIMIC-CXR \cite{johnson2019mimic} & Global & Automated & 14 & No & No & 377,110 & No & Yes \\
Dutta \cite{datta2020dataset} & Global & Manual & 10 & Yes & Yes & 2,000 & No & Yes \\
PadChest \cite{bustos2020padchest} & Global & Manual + automated & 297 & Yes & No & 160,868 & No & Yes \\
Montgomery County Chest X-ray   \cite{jaeger2014two} & Segmentation & Manual & 1 & Yes & No & 138 & No & No \\
Shenzen Hospital Chest X-ray   \cite{jaeger2014two} & Segmentation & Manual & 1 & Yes & No & 662 & No & No \\  \hline \hline
\textbf{Chest ImaGenome} & Bounding Boxes & Automated & 131 & Yes & Yes & 242,072 & Yes & Yes \\
\bottomrule
\end{tabular}%
}
\label{tab:related}
\vspace{-0.4cm}
\end{table}
% removed (Derived from MIMIC-CXR \cite{johnson2019mimic}) % makes table really small

\section{The SICMDP Model}
\section{The \MakeLowercase{i}W\MakeLowercase{inr}NFL model}
\label{sec:model}

In this section we are going to present the data we used to develop our in-game probability model as well as the design details of {\method}. 

{\bf Data: }In order to perform our analysis we utilize a dataset collected from NFL's Game Center for all the regular season games between the seasons 2009 and 2016. 
We access the data using the Python {\tt nflgame} API \cite{nflgame}. 
The dataset includes detailed play-by-play information for every game that took place during these seasons. 
This information is used to obtain the state of the game that will drive the design of {\method}. 
In total, we collected information for 2,048 regular season games and a total of 338,294 snaps/plays. 

{\bf Model: }
{\method} is based on a logistic regression model that calculates the probability of the home team winning given the current status of the game as: 

\begin{equation}
\Pr(H=1| \mathbf{x})= \frac{\exp(\mathbf{\weight}^T\cdot\mathbf{x})}{1+\exp(\mathbf{\weight}^T\cdot\mathbf{x})}
\label{eq:reg}
\end{equation}
where $H$ is the dependent random variable of our model representing whether the home team wins or not, $\mathbf{x}$ is the vector with the independent variables, while the coefficient vector $\mathbf{\weight}$ includes the weights for each independent variable and is estimated using the corresponding data.  
For a game of infinite duration a linear model could be a very good approximation.  
However, the boundary effects from the finite duration of a game create several non-linearities \cite{winston2012mathletics}.  
For this reason, we enhance our model - using the same set of features - with a Support Vector Machine classifier with radial kernel for the last three minutes of regulation.  
In order to obtain a probability output from the SVM classifier, we further use Platt's scaling \cite{platt1999probabilistic}: 

\begin{equation}
\Pr(H=1| \mathbf{x})= \frac{1}{1+\exp{(Af(x)+B)}}
\label{eq:platt}
\end{equation}
where $f(x)$ is the uncalibrated value produced by the SVM classifier: 

\begin{equation}
f(x) = \sum_{i} (\alpha_i y_i k(\mathbf{x}_i\cdot\mathbf{x}))+ b
\label{eq:svm}
\end{equation}
where $k(\mathbf{x},\mathbf{x}')$ is the kernel used for the SVM.   
Figure \ref{fig:iwinrNFL} depicts the simple flow chart of {\method}. 


\begin{figure}[t]
\begin{center}
\includegraphics[scale=0.35]{plots/iwinrNFL.pdf}%\vspacecap
 \caption{{\method} includes a linear and a non-linear component.}
 \label{fig:iwinrNFL}
\end{center}
\end{figure}

In order to describe the status of the game we use the following variables:

\begin{enumerate}
\item {\bf Ball Possession Team:} This binary feature captures whether the home or the visiting team has the ball possession
\item {\bf Score Differential:} This feature captures the current score differential (home - visiting)
\item {\bf Timeouts Remaining:} This feature is represented by two independent variables - one for the home and one for the away team - and they capture the number of timeouts remaining for each of the teams
%\item {\bf Quarter:} This feature captures the current quarter of the game
%\item {\bf Time Remaining:} This feature captures the time (in seconds) remaining for the current quarter to end
\item {\bf Time Elapsed: } This feature captures the time elapsed since the beginning of the game
\item {\bf Down:} This feature represents the down of the team in possession
\item {\bf Field Position:} This feature captures the distance covered by the team in possession from their own yard line
\item {\bf Yards-to-go:} This variables represents the number of yards needed for a first down
\item {\bf Ball Possession Time: } This variable captures the time that the offensive unit of the home team is on the field 
\item {\bf Ranking Differential: } This variable represents the difference of the win percentage for the two team (home - visiting)
\end{enumerate}

The last independent variable is representative of the power ranking difference between the two teams. 
Most of the existing models that include such a variable are using the Vegas line spread for each game.  
We choose not to do so for the following reason.  
The objective of the Vegas line is not to predict game outcomes but rather distribute money across the different bets.  
Exactly because of this objective the line is changing during the week before the game.  
While this line can change due to new information for the competing teams (e.g., injury updates), the line is mainly changing when a particular team has accumulated the majority of the bets. 
In this case it will also be hard to choose which line to use (e.g., the opening, the closing or some average of them).  
Therefore, we choose to use the win percentage differential of the two teams as an indicator of their strength (even though this has its own issues given the uneven schedule in NFL).  
However, note that if one would like to use the point spread as a variable this can be easily incorporated in the model. 
Table \ref{tab:iwinrnfl} presents the coefficients of the logistic regression model of {\method} with standardized independent variables for better comparisons. 


\begin{table}[ht]
\begin{center}
\def\sym#1{\ifmmode^{#1}\else\(^{#1}\)\fi}
\begin{tabular}{l*{1}{c}}
\toprule
                    &\multicolumn{1}{c}{(1)}\\
                    &\multicolumn{1}{c}{Winner}\\
\midrule
Possession Team (H)         &      0.41\sym{***}\\
                    &     (49.19)         \\
\addlinespace
Score Differential           &      3.59\sym{***}\\
                    &    (247.34)         \\
\addlinespace
Home Timeouts           &     0.12\sym{***}\\
                    &      (8.74)         \\
\addlinespace
Away Timeouts           &     -0.11\sym{***}\\
                    &    (-12.47)         \\
\addlinespace
Ball Possession Time  &     -0.05.\\
                    &    (-1.66)         \\
\addlinespace
Time Lapsed       &   -0.05.\\
                    &      (-1.66)         \\
\addlinespace
Down                &   -0.01         \\
                    &      (0.04)         \\
\addlinespace
Field Position            &   0.02\sym{**} \\
                    &      (2.71)         \\
\addlinespace
Yards-to-go                &  -0.01         \\
                    &      (0.23)         \\
\addlinespace
Rating differential         &       0.75\sym{***}\\
                    &     (80.47)         \\
\addlinespace
Intercept            &       0.57\sym{*}\\
                    &    (2.09)         \\
\midrule
Observations        &      338,294         \\
\bottomrule
\multicolumn{2}{l}{\footnotesize \textit{t} statistics in parentheses}\\
\multicolumn{2}{l}{\footnotesize \sym{$_.$} \(p<0.1\), \sym{*} \(p<0.05\), \sym{**} \(p<0.01\), \sym{***} \(p<0.001\)}\\
\end{tabular}
\end{center}
\caption{Standardized logisitic regression coefficients for {\method}.}
\label{tab:iwinrnfl}
\end{table}


As we can see, as one might have expected the current scoring differential exhibits the strongest correlation with the in-game win probability.  
The only factors that do not appear to be statistically significant predictors of the dependent variable are the down and the yards-to-go. 
Even though the corresponding coefficients are negative as one might have expected (e.g., being at an earlier down gives you more chances to advance the ball), they are not significant in estimating the win probability. 
On the contrary, all else being equal timeouts appear to be quiet important since they can help a team stop the clock, while teams with better win percentage appear to have an advantage as well, since this can be a sign of a better team. 
In the following section we provide a detailed evaluation of {\method}.
\section{Algorithms}
In this section, we present two reinforcement learning algorithms called semi-infinitely constrained reinforcement learning (SI-CRL) and semi-infinitely constrained policy optimization (SI-CPO), respectively.
SI-CRL is a model-based reinforcement learning algorithm that can solve tabular SICMDP in a sample-efficient way.
The SI-CPO algorithm is a policy optimization algorithm and it works for large-scale SICMDPs where we can use complex function approximators such as deep neural networks to approximate the policy and the value function.
\subsection{The SI-CRL Algorithm}\label{Section_SICRL}
From a high-level point of view, the SI-CRL algorithm is a semi-infinite version of the algorithms proposed in \cite{ijcai2021-347, efroni2020explorationexploitation}.
In the first stage, SI-CRL takes an offline dataset $\{(s_i, a_i, s_i^\prime)|i=1, 2, \ldots, m\}$ as input and generates an empirical estimate $\widehat P$ of the true transition dynamic $P$.
Then the algorithm constructs a confidence set (the optimistic set) according to $\widehat P$ that would cover the true SICMDP with high probability.
For each policy $\pi$ we would only view its return as the largest possible return in SICMDPs in the confidence set.
% Since when the true SICMDP lies in the confidence set we would overestimate the return of each policy $\pi$, 
This method is also called the optimistic approach.
In the second stage, we reformulate the problem as an LSIP problem and find the optimistic policy $\hat \pi$ using an LSIP solver.
It can be shown that the resulting policy $\hat\pi$ is guaranteed to be nearly optimal, and the theoretical analysis can be found in Section \ref{Section_Theory_SICRL}.

Now we give a more detailed description of SI-CRL.
First, the empirical estimate $\widehat P$ is calculated as:
$\widehat P(s^\prime|s, a):=\frac{n(s, a, s^\prime)}{\max\paren{1,n(s,a)}}$,
where $n(s,a,s^\prime) :=\sum_{i=1}^m \mathbf{1}\{s_i=s, a_i=a, s_i^\prime=s^\prime\}$ and $n(s,a)=\sum_{s^\prime} n(s,a,s^\prime)$.
The reason why we do not directly plug $\widehat P$ into Problem \eqref{Problem_SICMDP_LSIP} and solve the resulting LSIP problem is due to the fact that there is no guarantee that the LSIP problem w.r.t.\ $\widehat P$ is feasible.
To address this issue, we construct an optimistic set $M_\delta$ such that with high probability the true SICMDP $M$ lies in $M_\delta$.
In particular, $M_\delta$ is defined via the empirical Bernstein's bound and the Hoeffding's bound \citep{LATTIMORE2014125}:
%\begingroup
%\small
\begin{align*}
M_\delta :=& \Big\{\langle \gS,\gA,Y,P^\prime,r,c,u,\mu, \gamma \rangle\colon  |P^\prime(s^\prime|s,a)-\widehat P(s^\prime|s,a)| \leq d_\delta(s,a,s^\prime), \forall s, s^\prime\in \gS, a\in \gA \Big\},
\end{align*}
%\endgroup
where 
% $$\begin{aligned}
% d_\delta(s,a,s^\prime):=&\min\bigg\{\sqrt{[2\widehat P(s^\prime|s,a)(1-\widehat P(s^\prime|s,a))\log(4/\delta)]/n(s,a,s^\prime)}\\
% &+4\log (4/\delta)/n(s,a,s^\prime), \sqrt{\log (2/\delta)/2n(s,a,s^\prime)}\bigg\}.
% \end{aligned}
% $$
\begin{align*}
d_\delta(s,a,s^\prime):=&\min\left\{\sqrt{\frac{2\widehat P(s^\prime|s,a)(1 {-} \widehat P(s^\prime|s,a))\log(4/\delta)}{n(s,a,s^\prime)}}+\frac{4\log (4/\delta)}{n(s,a,s^\prime)}, \; \sqrt{\frac{\log (2/\delta)}{2n(s,a,s^\prime)}}\right\}.
\end{align*}


The next step is to solve the optimistic planning problem:
\begin{equation}\label{Problem_Optimistic}
\begin{aligned}
\max_{M^\prime\in M_\delta,\pi}\ V_r^{\pi,M^\prime}(\mu),\quad
\text{s.t.}\ V_{c_y}^{\pi,M^\prime}(\mu) \leq u_y,\ \forall y\in Y,
\end{aligned}
\end{equation}
where the superscript $M^\prime$ denotes that the expectation is taken w.r.t.\ SICMDP $M^\prime$.
\begin{theorem}\label{Theorem_Feasible}
Suppose $n\geq 3$. With probability at least $1-2|\gS|^2|\gA|\delta$, we have that $M\in M_\delta$, and Problem (\ref{Problem_Optimistic}) is feasible.
\end{theorem}

\proof {Proof of Theorem~\ref{Theorem_Feasible}}
See Appendix~\ref{Appendix_Proofs_4}.
\endproof

Note that the optimization variables include both $M^\prime$ and $\pi$, and LSIP reformulations like Problem (\ref{Problem_SICMDP_LSIP}) would no longer be possible. 
Instead, we shall introduce the state-action-state occupancy measure $z(s,a,s^\prime)$.
In particular, assuming $z_{P,\pi}(s,a,s^\prime):=P(s^\prime|s,a)q_\pi(s,a)$, we have $P(s^\prime|s,a)=\frac{z_{P,\pi}(s,a,s^\prime)}{\sum_{x\in \gS}z_{P,\pi}(s,a,x)}$, and $\pi(a|s)=\frac{\sum_{s^\prime\in \gS}z_{P,\pi}(s,a,s^\prime)}{\sum_{s^\prime\in \gS,a^\prime\in \gA}z_{P,\pi}(s,a^\prime,s^\prime)}$. 
Problem (\ref{Problem_Optimistic}) can be reformulated as the following extended LSIP problem:

\begingroup
\small
\begin{equation}\label{Problem_Optimistic_ELSIP}
\begin{aligned}
    \max_{z}\ &\sum_{s, a,s^\prime}z(s,a,s^\prime)r(s,a) \\
    \text{s.t.}\ &\frac{1}{1-\gamma}\sum_{s, a,s^\prime}z(s,a,s^\prime)c_y(s,a)\leq u_y,\ \forall y\in Y, \\
    &z(s,a,s^\prime)\leq (\widehat P(s^\prime|s,a)+d_\delta(s,a,s^\prime))\sum_{x\in \gS} z(s,a,x), \forall s,s^\prime,\ a\in \gA, \\
    &z(s,a,s^\prime)\geq (\widehat P(s^\prime|s,a)-d_\delta(s,a,s^\prime))\sum_{x\in \gS} z(s,a,x), \forall s,s^\prime\in \gS,\ a\in \gA, \\
    &\sum_{x\in \gS,b\in \gA}z(s,b,x)=(1-\gamma)\mu(s)+\gamma\sum_{x\in \gS,b\in \gA}z(x,b,s), \forall s\in \gS, \\
    &z\succeq 0.
\end{aligned}
\end{equation}
\endgroup

However, compared to LP problems, LSIP problems are typically harder to solve and there are no all-purpose LSIP solvers.
Here, we choose the simple yet effective dual exchange methods \citep{Hu1990,reemtsen1998numerical} to solve Problem~\ref{Problem_Optimistic_ELSIP}.
The SI-CRL algorithm can be summarized in Algorithm~\ref{Algorithm_SICRL}.
A key ingredient of Algorithm~\ref{Algorithm_SICRL} is solving the inner-loop optimization problem 
$$
\max_{y\in Y} \sum_{s, a,s^\prime}z(s,a,s^\prime)c_y(s,a)-u_y.
$$
We can obtain different versions of SI-CRL algorithm by choosing different optimization subroutines to solve the inner-loop problem above. 
If $c_y$ and $u_y$ satisfy conditions like concavity and smoothness, then the inner problem can be solved using methods like projected subgradient ascent \citep{bubeck2015convex}.
If the inner problem is ill-posed, we may still solve it using methods like random search \citep{solis1981minimization, andradottir2015review}.
\begin{algorithm}[htb]
   \caption{SI-CRL}
   \label{Algorithm_SICRL}
\begin{algorithmic}
   \STATE {\bfseries Input:} state space $\gS$, action space $\gA$, dataset $\{(s_i,a_i,s_i^\prime)|i=1,2,...,m\}$, reward function $r$, a continuum of cost function $c$, index set $Y$, value for constraints $u$, discount factor $\gamma$, tolerance $\eta$, maximum iteration number $T$.
   \FOR{each $(s,a,s^\prime)$ tuple}
   \STATE Set $\widehat P(s^\prime|s,a):=\frac{\sum_{i=1}^m \ind\{s_i=s,a_i=a,s_i^\prime=s^\prime\}}{\max\paren{1,\sum_{i=1}^m \ind\{s_i=s,a_i=a\}}}$
   \ENDFOR
   \STATE Initialize $Y_0=\{y_0\}$
   \FOR{$t=1$ {\bfseries to} $T$}
   \STATE Use an LP solver to solve a finite version of Problem (\ref{Problem_Optimistic_ELSIP}) by only considering constraints in $Y_0$ and store the solution as $z^{(t)}$.
   \STATE Find $y^{(t)}\approx\argmax_{y\in Y} \sum_{s, a,s^\prime}z^{(t)}(s,a,s^\prime)c_y(s,a)-u_y$.
   \IF {$\sum_{s, a,s^\prime}z(s,a,s^\prime)c_{y^{(t)}}(s,a)-u_{y^{(t)}} \leq\eta$}
   \STATE  Set $z^{(T)}=z^{(t)}$.
   \STATE  {\bfseries BREAK}
   \ENDIF
   \STATE Add $y^{(t)}$ to $Y_0$.
   \ENDFOR
   \FOR{each $(s,a)$ pair}
   \STATE Set $\hat\pi(a|s)=\frac{\sum_{s^\prime}z^{(T)}(s,a,s^\prime)}{\sum_{s^\prime,a^\prime}z^{(T)}(s,a^\prime,s^\prime)}$.
   \ENDFOR
   \STATE {\bfseries RETURN} $\hat{\pi}$.
\end{algorithmic}
\end{algorithm}



\subsection{The SI-CPO Algorithm}\label{Section_SICPO}
In SI-CPO, we borrow ideas from the cooperative stochastic approximation \citep{lan2020algorithms, wei2020comirror} to deal with the infinitely many constraints.
At a certain iteration, the SI-CPO algorithm first determines whether the constraint violation is below some tolerance or not.
It then performs a single step of policy optimization along the direction of maximizing the value of reward if the constraint violation is below some tolerance;
or performs a single step of policy optimization along the direction of minimizing the value of some cost corresponding to a violated constraint.

We now describe the SI-CPO algorithm in more detail.
We follow the convention to define the parameterized policy class as $\{\pi_\theta,\theta\in\Theta\subset\RB^d\}$ and use $\pi^{(t)}$ in short of $\pi_{\theta^{(t)}}$, $V_\diamond^{(t)}$ in short of $V_\diamond^{\pi^{(t)}}$ for ease of notation.
Here $\diamond$ represents either the reward $r$ or some cost $c_y$.
Suppose at the $t$-th iteration our policy parameter is $\theta^{(t)}$, then we first construct an estimate $\widehat V^{(t)}_{c_y}(\mu)$ using some policy evaluation subroutine.
Next, we are to solve a subproblem using some optimization subroutine
$$
y^{(t)}=\argmax_y\ \widehat V_{c_y}^{\pi^{(t)}}(\mu)-u_y.
$$
If ${\widehat V_{c^{(t)}}^{\pi^{(t)}}(\mu)-u_{y^{(t)}}\leq \eta}$, where $c^{(t)}:=c_{y^{(t)}}$ and $\eta> 0$ is a threshold of tolerance, we say the constraint violation is small and add the time index $t$ to the ``good set" $\gB$.
Then we perform a step of update with a policy optimization subroutine to maximize the value of reward $V_r^{(t)}(\mu)$ to get $\theta^{(t+1)}$.
Else, we first add the time index $t$ to the ``bad set" $\gN$.
Next, we find the violated constraint ${V_{c^{(t)}}^{\pi^{(t)}}(\mu)-u_{y^{(t)}}>\eta}$, and perform a step of update with a policy optimization subroutine to minimize the value of cost $V_{c^{(t)}}^{\pi^{(t)}}(\mu)$ to get $\theta^{(t+1)}$.
After $T$ iterations, we draw $\hat\theta$ uniformly from the set ${\{\theta^{(t)},t\in\gB\}}$, as return the policy ${\hat\pi=\pi_{\hat\theta}}$.
The procedure of SI-CPO  is summarized in Algorithm~\ref{Algorithm_SICPO}.


We can get different instances of the SI-CPO algorithms by making different choices of the subroutines aforementioned.
Specifically, the policy optimization subroutine can be any policy optimization algorithm like policy gradient~(PG) \citep{sutton1999policy}, natural policy gradient~(NPG) \citep{kakade2001natural}, trust-region policy gradient~(TRPO) \cite{schulman2015trust}, or proximal policy optimization~(PPO) \citep{schulman2017proximal}.
The policy evaluation subroutine can be chosen as Monte-Carlo policy evaluation algorithms \citep{curtiss1954theoretical} or various TD-learning algorithms \citep{sutton1988learning, dann2014policy}.
We may also integrate the policy optimization subroutine and the policy evaluation subroutine into actor-critic-type algorithms \citep{konda1999actor}.
The optimization subroutine can be any optimization algorithm suitable for the problem instance, like the case in Algorithm~\ref{Algorithm_SICRL}.
% It is shown in Section~\ref{Section_Theory_SICPO} that if we use sample-based NPG \cite{agarwal2021theory} as the policy optimization subroutine, a finite-horizon Monte-Carlo estimator as the policy evaluation subroutine and either random search or projected subgradient ascent as the optimization subroutine, then $\hat\pi$ is guaranteed to be a globally near-optimal policy.
% We also empirically show in Section~\ref{Section_Experiment} the efficacy of SI-CPO in solving complex sequential decision-making problems with deep neural networks if we use PPO as the policy optimization subroutine, a value network trained in a TD style as the policy evaluation subroutine, and random search as the optimization subroutine.

\begin{algorithm}[htb]
   \caption{SI-CPO}
   \label{Algorithm_SICPO}
\begin{algorithmic}
   \STATE {\bfseries Input:} state space $\gS$, action space $\gA$, reward function $r$, a continuum of cost function $c$, index set $Y$, value for constraints $u$, discount factor $\gamma$, learning rate $\alpha$, tolerance $\eta$, maximum iteration number $T$.
   \STATE Initialize $\gB=\emptyset$, $\gN=\emptyset$, $\theta^{(0)}=\theta_0\in\Theta$.
   \FOR{$t=0,...,T-1$}
   \STATE Obtain $\widehat V_{c_y}^{\pi^{(t)}}(\mu)$ via a policy evaluation subroutine.
   \STATE Use an optimization subroutine to solve ${\max_y\ \widehat V_{c_y}^{\pi^{(t)}}(\mu)-u_y}$, and set ${y^{(t)}\approx\argmax_y \widehat V_{c_y}^{\pi^{(t)}}(\mu)-u_y}$, $c^{(t)}=c_{y^{(t)}}$.
   \IF {$\widehat V_{c^{(t)}}^{\pi^{(t)}}(\mu)-u_{y^{(t)}}\leq \eta$}
   \STATE  Perform a step of policy update to maximize $V_r^{\pi^{(t)}}(\mu)$ to get $\pi^{(t+1)}$. Specifically,
   ${\theta^{(t+1)}=\theta^{(t)}+\alpha\hat w^{(t)}}.$
%   \IF {$t\geq s$}
   \STATE Add $t$ to $\gB$
%   \ENDIF
   \ELSE 
   \STATE  Perform a step of policy update to minimize $V_{c^{(t)}}^{\pi^{(t)}}(\mu)$ to get $\pi^{(t+1)}$. Specifically, ${\theta^{(t+1)}=\theta^{(t)}-\alpha\hat w^{(t)}}.$
%   \IF {$t\geq s$}
   \STATE Add $t$ to $\gN$
%   \ENDIF
   \ENDIF
   \ENDFOR
   \STATE {\bfseries RETURN} $\hat\pi=\pi_{\hat\theta}$, where $\hat\theta\sim\mathrm{Unif}\left(\{\theta^{(t)}, t\in\gB\}\right)$.
\end{algorithmic}
\end{algorithm}


\section{Theoretical Analysis}
\subsection{Theoretical Analysis of SI-CRL}\label{Section_Theory_SICRL}
We give PAC-type bounds for SI-CRL under different settings.
The error of SI-CRL is decomposed into two parts: the optimization error from the fact that the solution of (\ref{Problem_Optimistic}) obtained by the dual exchange method is inexact and the statistical error from approximating Problem (\ref{Problem_SICMDP}) with Problem (\ref{Problem_Optimistic}).
On the optimization side, we show that if the inner maximization problem w.r.t.\ $y$ is solved via random search or projected subgradient ascent, the dual exchange method would produce an $\epsilon$-optimal solutions (see Definition \ref{Definition_Optimal_Solution})
%\footnotemark 
when the number of iterations $T=O\left(\left[\frac{\mathrm{diam}(Y)|\gS|^2|\gA|}{(1-\gamma)\epsilon}\right]^m \right)$.
% , where $L_y$ is the Lipschitz constant defined in Assumption \ref{Assumption_Lipschitz}.

On the statistical side, our goal is to determine how many samples are required to make SI-CRL an $(\epsilon, \delta)$-optimal~(see Definition \ref{Definition_PAC}) when Problem (\ref{Problem_Optimistic}) can be solved exactly, i.e., we want to find the sample complexity of SI-CRL  (see Definition \ref{Definition_PAC}).
%\footnotetext{The $(\epsilon, \delta)$-optimality would be defined in Definition \ref{Definition_PAC}}
We show that the sample complexity of SI-CRL is $\widetilde O\left(\frac{|\gS|^2|\gA|^2}{\epsilon^2(1-\gamma)^3}\right)$ if the dataset we use is generated by a generative model, and $\widetilde O\left(\frac{|\gS||\gA|}{\nu_{\min} \epsilon^2(1-\gamma)^3}\right)$ if the dataset we use is generated by a probability measure $\nu$ defined on the space $\gS\times \gA$ and $P(\cdot|s,a)$ as considered in \cite{chen2019information}.
% \wenhao{Why do we use $\Omega(\cdot)$ represent upper bound? $\Omega(\cdot)$ is used to show lower bound while $O(\cdot)$ is used to show upper bound.}
% \liangyu{Resolved.}
Here $\widetilde O$ means that all logarithm terms are discarded, and $\nu_{\min}:=\min_{\nu(s,a)>0}\nu(s,a)$.
% It can be noted that the order of our sample complexity bound increases by a factor of $|\gS||\gA|$ compared to that of ordinary discounted MDP \citep{azar2013minimax}.
We will present our theoretical analysis in more detail in the following part of this section.
%\footnotetext{The $\epsilon$-optimal solutions is defined in Definition \ref{Definition_Optimal_Solution}}


\subsubsection{Preliminaries}
%In addition to the notation defined in Sections \ref{Section_SICMDP} and \ref{Section_Algorithm}, 
% Given a stationary policy $\pi$, we define the value function $V^\pi(s)=\EB\paren*{\sum_{t=0}^\infty \gamma^t r(s_t,a_t)|s_0=s}$, $V^\pi=(V^\pi(s_1), \ldots, V^\pi(s_{|\gS|}))^\top\in \RB ^{|\gS|}$.
% Thus we have $V^\pi(\mu)=\mu^\top V^\pi$.
% Similarly, we define the expected cost $C_y^\pi(s)=\EB\paren*{\sum_{t=1}^\infty \gamma^t c_y(s_t,a_t)|s_0=s}$, $C_y^\pi=(C_y^\pi(s_1), \ldots, C_y^\pi(s_{|\gS|}))^\top\in \RB ^{|\gS|}$,  thus $C_y^\pi(\mu)=\mu^\top C_y^\pi$. 
% Suppose $\tilde\pi,\widetilde M$ are the solution of Problem ($\ref{Problem_Optimistic}$) and $\widetilde M=\langle \gS,\gA,Y,\widetilde P,r,c,u,\mu,\gamma \rangle$.
% For a given stationary policy $\pi$, $\widetilde V_r^\pi(\mu)$, $\widetilde V_{c_y}^\pi(\mu)$, to represent the value function, expected cost, of SICMDP $\widetilde M$, respectively.

Let $\pi^*$ denote the optimal policy.
An $(\epsilon,\delta)$-optimal policy is defined as follows. 
\begin{definition}\label{Definition_PAC}
An RL algorithm is called $(\epsilon,\delta)$-optimal for $\epsilon,\delta>0$ if with probability at least $1-\delta$ it can return a policy $\pi$ such that
$$
\begin{aligned}
V_r^{\pi^*}(\mu)-V_r^{\pi}(\mu) &\leq \epsilon;\quad
V_{c_y}^{\pi}(\mu) - u_y  \leq \epsilon, \forall y\in Y.
\end{aligned}
$$
% \wenhao{There is no randomness of a given policy $\pi$. $\delta$ can be removed. }
% \liangyu{Resolved.}
\end{definition}
An $\epsilon$-optimal solution of Problem (\ref{Problem_Optimistic}) is defined as \begin{definition}\label{Definition_Optimal_Solution}
A stationary policy $\hat\pi$ is called an $\epsilon$-optimal solution of Problem (\ref{Problem_Optimistic}) for $\epsilon>0$ if 
$$
\begin{aligned}
|V_r^{\hat\pi}(\mu)-V_r^{\tilde\pi}(\mu)| &\leq \epsilon \quad \mbox{and} \quad
|V_{c_y}^{\hat\pi}(\mu) - u_y|  \leq \epsilon,  \forall y\in Y \\
\end{aligned}
$$
hold simultaneously.
\end{definition}

Unless otherwise specified, we assume that $\forall (s,a)\in \gS\times \gA$, $c_y(s,a)$ is $L_y$-Lipschitz in $y$ w.r.t.\ $\|\cdot\|_\infty$.
We also assume that $u_y$ is $L_y$-Lipschitz in $y$ w.r.t.\ $\|\cdot\|_\infty$.
The assumptions can be formally stated as:
\begin{assumption}\label{Assumption_Lipschitz}
$c_y(s,a)$ and $u_y$ are Lipschitz in $y$ w.r.t.\ $\|\cdot\|_\infty$, i.e., $\exists L_y>0$ s.t. $\forall y,y^\prime\in Y, (s,a)\in \gS\times \gA, |c_y(s,a)-c_{y^\prime}(s,a)|\leq L_y\|y-y^\prime\|_\infty, 
|u_y-u_{y^\prime}|\leq L_y\|y-y^\prime\|_\infty$.
\end{assumption}
The Lipschitz assumption is usually necessary when dealing with a semi-infinitely constrained problem \citep{still2001discretization,Hu1990}.
And this assumption is indeed quite mild because $Y$ is a compact set.

We say an offline dataset $\{(s_i,a_i,s_i^\prime)|i=1, 2, \ldots, n\}$ to be generated by a generative model if we sample according to $P(\cdot|s,a)$ for each $(s,a)$-pair $n_0=n/|\gS||\gA|$ times and record the results in the dataset.
We say an offline dataset to be generated by probability measure $\nu$ and $P(\cdot|s,a)$ if $(s_i,a_i)\stackrel{i.i.d.}{\sim} \nu$ and $s_i^\prime\sim P(\cdot|s_i,a_i)$.

We solve the inner-loop problem in Algorithm~\ref{Algorithm_SICRL} with random search or projected gradient ascent.
The idea of random search is simple.
For an objective $f(y)$ defined on domain $Y$, we form a random grid of $Y$ consisting of $M$ grid points and select the grid point with the largest objective value.
The precise definition can be found in Algorithm~\ref{Algorithm_random_search} in Appendix~\ref{Appendix_Algorithm}.
The projected subgradient ascent is defined in a standard way \citep{bubeck2015convex}.
The precise definition can be found in Algorithm~\ref{Algorithm_projected_GD} in Appendix~\ref{Appendix_Algorithm}.

\subsubsection{Iteration Complexity of SI-CRL}

We give the iteration complexity of SI-CRL, i.e., how many iterations are required to output an $\epsilon$-optimal solution of Problem (\ref{Problem_Optimistic}) when near-optimal solutions of the inner-loop optimization problems can be obtained.
Our result is similar to Theorem 4 in \cite{Hu1990}.
Specifically, we consider two different cases: 1) we make no assumption of the constraint and use random search to solve the inner-loop problem; 2) we assume the constraint is concave and use projected subgradient ascent to solve the inner-loop problem.
% The random search algorithm and the projected subgradient ascent algorithm are defined by Algorithm~\ref{Algorithm_random_search} and Algorithm~\ref{Algorithm_projected_GD} in Appendix~\ref{Appendix_Algorithm}, respectively.

Before we give the iteration complexity of the case of random search, we make the following assumption to ensure technical rigor.
\begin{assumption}\label{Assumption_regular_maxima}
     For any $(s,a)\in\gS\times\gA$ and weight $v\in \RB^{\gS\times\gA}$, let $y_0\in\arg\max_{y\in Y} (v^\top  c_y-u_y)$. Then $\exists y_0$ such that
    $$
    \{y:\|y-y_0\|_\infty\leq \epsilon_0\}\subset Y.
    $$
\end{assumption}
Assumption~\ref{Assumption_regular_maxima} guarantees any possible solution of the inner-loop problem lies in the interior of $Y$.
\begin{theorem}\label{Theorem_Iteration_Complexity_Random_Search}
Suppose we use random search to solve the inner-loop problem of the SI-CRL algorithm, then if we set the size of random grid $M=O\left(\frac{\log(\delta/T)}{\log \left(1-((1-\gamma)\epsilon/|\gS|^2|\gA|\mathrm{diam}(Y))^m\right)}\right)$, $T=O\left(\left[\frac{\mathrm{diam}(Y)|\gS|^2|\gA|}{(1-\gamma)\epsilon}\right]^m \right)$, SI-CRL would output a $\epsilon$-optimal solution of Problem~\ref{Problem_Optimistic_ELSIP} with probability at least $1-\delta$.
Here we require $\epsilon\leq \frac{2|\gS|^2|\gA|L_y\epsilon_0}{1-\gamma}$.
\end{theorem}

\proof{Proof of Theorem~\ref{Theorem_Iteration_Complexity_Random_Search}.}
See Appendix~\ref{Appendix_Proofs_SICRL}.
\endproof

To derive theoretical guarantees for the case of projected subgradient ascent, we need the following assumption of concavity.

\begin{assumption}\label{Assumption_concave_constraint}
     For any $(s,a)\in\gS\times\gA$, $c_y(s,a)$ is concave in $y$. In addition, $u_y$ is convex in $y$.
\end{assumption}

\begin{theorem}\label{Theorem_Iteration_Complexity_Projected_GD}
Suppose we use projected gradient ascent to solve the inner-loop problem of the SI-CRL algorithm, then if we set the iteration number of the projected subgradient ascent $T_{PGA}=O\left(\frac{|\gS|^4|\gA|^2\mathrm{diam}(Y)^2}{(1-\gamma)^2\epsilon^2}\right)$, $T=O\left(\left[\frac{\mathrm{diam}(Y)|\gS|^2|\gA|}{(1-\gamma)\epsilon}\right]^m \right)$, SI-CRL would output a $\epsilon$-optimal solution of Problem~\ref{Problem_Optimistic_ELSIP}.
\end{theorem}

\proof{Proof of Theorem~\ref{Theorem_Iteration_Complexity_Projected_GD}.}
See Appendix~\ref{Appendix_Proofs_SICRL}.
\endproof

% \begin{remark}
The most crucial part of our proof is a $\epsilon$-packing argument.
Suppose we can get a $\epsilon/2$-optimal solution to the inner-loop problem by either random search of projected subgradient ascent and set the tolerance $\eta=\epsilon/2$.
By the assumption of Lipschitzness and the construction of the SI-CRL algorithm, for any $t\leq T$, either the SI-CRL algorithm has already terminated and we obtain a $\epsilon$-optimal solution to Problem~\ref{Problem_Optimistic_ELSIP}, or $\{B^{(t^\prime)},t=1,...,t\}$ forms a packing of $Y$.
Here $B^{(t^\prime)}:=\{y:\|y-y^{(t^\prime)}\|_\infty\leq \epsilon/2\beta\}$, and $\beta$ is some Lipschitz coefficient.
Then we may draw the conclusion by noting that the maximum iteration number of SI-CRL is no larger than the $\epsilon/2\beta$-packing number of $Y$.
We find that \cite{Hu1990} also used similar techniques to derive their convergence rate, although they assume the inner-loop problem can always be solved exactly.
% \end{remark}

\begin{remark}
    The iteration complexity of the SI-CRL algorithm grows with $m$ in an exponential manner.
    Thus from a theoretical viewpoint, the SI-CRL algorithm is no better than the naive discretization method mentioned in Remark~\ref{Remark_Baseline}.
    However, we find SI-CRL is far more efficient than the naive method in empirical evaluations.
    Perhaps it is because our bound of iteration complexity is obtained by the packing argument and not tight enough.
    Hopefully, the bound can be tightened by a refined analysis of the dynamics of $\{(y^{(t)}, z^{(t)}),t=1,...,T\}$.
\end{remark}

\subsubsection{Sample Complexity of SI-CRL}
%To begin with, 
We consider the case where the offline dataset we use is generated by a generative model.
First, we consider a restricted setting as in \cite{LATTIMORE2014125} where for each $(s,a)$-pair in the true SICMDP there are at most two possible next-states and provide the sample complexity bound.
Then we will drop Assumption \ref{Assumption_Two_Nonzero} using the same strategy as in \cite{LATTIMORE2014125} and derive the sample complexity bound of the general case.
\begin{assumption}\label{Assumption_Two_Nonzero}
The true unknown SICMDP $M$ satisfies $P(s^\prime|s,a)=0$ for all but two $s^\prime\in \gS$ denoted as $sa^+$ and  $sa^-\in \gS$.
\end{assumption}


Although Assumption \ref{Assumption_Two_Nonzero} seems quite restrictive, we argue that it is necessary to establish sharp sample complexity bound, as shown in \cite{LATTIMORE2014125}.
Specifically, without this assumption the ``quasi-Bernstein bound'' (Lemma \ref{Lemma_Quasi_Bernstein}) will not hold, thus we may not be able to get the $\widetilde O((1-\gamma)^{-3})$ bound.

\begin{lemma}\label{Lemma_Bound_on_V}
Suppose Assumption \ref{Assumption_Two_Nonzero} holds, and the dataset we use is generated by a generative model with $n/|\gS||\gA|=n_0>\max\left\{\frac{36\log4/\delta}{(1-\gamma)^2}, \frac{4\log4/\delta}{(1-\gamma)^3}\right\}$. Then with probability $1-2|\gS|^2|\gA|\delta$, we have that
$$\begin{aligned}
V_r^{\pi^*}(\mu)-V_r^{\tilde\pi}(\mu)\leq 24\sqrt{\frac{\log 4/\delta}{{n_0}(1-\gamma)^3}};\quad
V_{c_y}^{\tilde \pi}(\mu) - u_y \leq 12\sqrt{\frac{\log 4/\delta}{{n_0}(1-\gamma)^3}}, \; \forall y\in Y.
\end{aligned}
$$
Here $\tilde\pi$ is the exact solution of Problem~\ref{Problem_Optimistic}.
\end{lemma}
\proof{Proof of Lemma~\ref{Lemma_Bound_on_V}.}
See Appendix~\ref{Appendix_Proofs_SICRL}.

\begin{theorem}\label{Theorem_Sample_Complexity}
Suppose Assumption \ref{Assumption_Two_Nonzero} holds, the dataset we use is generated by a generative model and Problem \ref{Problem_Optimistic} can be solved exactly. Then when $n=O\left(\frac{|\gS||\gA|\log \paren{8|\gS|^2|\gA|/\delta}}{\epsilon^2(1-\gamma)^3}\right)$, SI-CRL is $(\epsilon,\delta)$-optimal.
\end{theorem}
\proof{Proof of Theorem~\ref{Theorem_Sample_Complexity}.}
Theorem~\ref{Theorem_Sample_Complexity} is a direct consequence of Lemma~\ref{Lemma_Bound_on_V}.

\begin{theorem}\label{Theorem_Sample_Complexity_General}
Suppose the dataset we use is generated by a generative model and Problem \ref{Problem_Optimistic} can be solved exactly. Then when $n=O\left(\frac{|\gS|^2|\gA|^2\paren{\log|\gS|}^3\log \paren{8|\gS|^4|\gA|^3/\delta}}{\epsilon^2(1-\gamma)^3}\right)$, a modification of SI-CRL is $(\epsilon,\delta)$-optimal.
\end{theorem}
\proof{Proof of Theorem~\ref{Theorem_Sample_Complexity_General}.}
See Appendix~\ref{Appendix_Proofs_SICRL}.


% \begin{remark}
Our proof strategy is similar to \cite{LATTIMORE2014125}. 
However, to get a $\widetilde O((1-\gamma)^{-3})$ bound \cite{LATTIMORE2014125} used a tedious recursion argument.
We greatly simplify the proof and achieve improvements in log terms (by a factor of $(\log(\frac{|\gS|}{\epsilon(1-\gamma)}))^2$) using sharper bounds on local variances of MDPs developed in \cite{pmlr-v125-agarwal20b}.
% \end{remark}

% \begin{remark}\label{Remark_Sample_Complexity_Assumption}
% Although Assumption \ref{Assumption_Two_Nonzero} seems quite restrictive, we argue that it is necessary to establish sharp sample complexity bound, as shown in \cite{LATTIMORE2014125}.
% Specifically, without this assumption the ``quasi-Bernstein bound'' (Lemma \ref{Lemma_Quasi_Bernstein}) will not hold, thus we may not be able to get the $\widetilde O((1-\gamma)^{-3})$ bound.
% \end{remark}

\begin{remark}\label{Remark_Sample_Complexity_General_Dependence_on_Constraints}
It can be noted that our sample complexity bound does not rely on the constraint set $Y$.
This is because we consider the setting where $r$ and $c_y$ are known deterministic functions and the only source of randomness comes from estimating the unknown transition dynamic using an offline dataset.
In other words, the constraints do not make the problem more difficult in the statistical sense.
\end{remark}

\begin{remark}\label{Remark_Modification}
Here ``a modification of SI-CRL" stands for the following procedure: first we transform the original SICMDP to a new SICMDP satisfying Assumption~\ref{Assumption_Two_Nonzero}, then we run SI-CRL to solve the new SICMDP.
One may refer to the proof in Appendix~\ref{Appendix_Proofs_SICRL} for more details.
\end{remark}

Now we generalize our results to the case where the offline dataset is generated by a probability measure.
\begin{theorem}\label{Theorem_Sample_Complexity_General_Measure}
Suppose the dataset we use is generated by a probability measure $\nu$ and Problem \ref{Problem_Optimistic} can be solved exactly. Then when $m=O\left(\frac{|\gS||\gA|\paren{\log|\gS|}^3\log \paren{8|\gS|^4|\gA|^3/\delta}}{\nu_{\min} \epsilon^2(1-\gamma)^3}\right)$, a modification of SI-CRL is $(\epsilon,\delta)$-optimal.
\end{theorem}
\proof{Proof of Theorem~\ref{Theorem_Sample_Complexity_General_Measure}.}
See Appendix~\ref{Appendix_Proofs_SICRL}.
\subsection{Theoretical Analysis of SI-CPO}\label{Section_Theory_SICPO}
In this section, we present theoretical guarantees of SI-CPO.
We consider a version of the SI-CPO algorithm, where we use sample-based NPG \citep{agarwal2021theory} as the policy optimization subroutine, a finite-horizon Monte-Carlo estimator as the policy evaluation subroutine, and either random search or projected subgradient ascent as the optimization subroutine.
It is shown that when the function approximation error $\epsilon_{bias}$ is of the same order with $\epsilon$, our proposed algorithm takes $\widetilde{O}\left(\frac{1}{\epsilon^2(1-\gamma)^6}\right)$ iterations and make $\widetilde{O}\left(\frac{1}{\epsilon^4(1-\gamma)^{10}}\right)$ interactions with the environment to achieve an $\epsilon$-level of averaged suboptimality with high probability.
This corresponds to a $\widetilde{O}(1/\sqrt{T})$ globally convergence rate, which is typical for NPG-based policy optimization algorithms.
We will give a detailed description of the considered version of the SI-CPO algorithm as well as our technical assumptions in Section~\ref{Subsection_SICPO_Prem} and present the theoretical results in Sections~\ref{Subsection_SICPO_Iteration_Complexity} and~\ref{Subsection_SICPO_Sample_Complexity}.

\subsubsection{Preliminaries}\label{Subsection_SICPO_Prem}
Recall the policy $\pi$ is parameterized by $\theta\in\Theta\subset\RB^d$ (denoted by $~\pi_\theta$).
We make the following assumptions about the parameterized policy class.
\begin{assumption}[Differentiable policy class]\label{Assumption_differentiable}
$\Pi$ can be parametrized as $\Pi_\theta=\{\pi_\theta|\theta\in\RB^d\}$, such that for all $s\in\gS$,~$a\in\gA$, $\log_\theta\pi(s|a)$ is a differentiable function of $\theta$.
\end{assumption}

\begin{assumption}[Lipschitz policy class]\label{Assumption_Lipschitz_policy}
For all $s\in\gS$,~$a\in\gA$, $\log\pi_\theta(s|a)$ is a $L_\pi$-Lipschitz function of $\theta$, i.e.,
$$
\|\nabla_\theta\log\pi_\theta(s|a)\|_{2}\leq L_\pi,\forall s\in\gS,a\in\gA,\theta\in \RB^d.
$$
\end{assumption}

\begin{assumption}[Smooth policy class]\label{Assumption_smooth}
For all $s\in\gS$,~$a\in\gA$, $\log\pi_\theta(s|a)$ is a $\beta$-smooth function of $\theta$, i.e.,
$$
\|\nabla_\theta\log\pi_\theta(s|a)-\nabla_\theta\log\pi_{\theta^\prime}(s|a)\|_{2}\leq \beta\|\theta-\theta^\prime\|_2,\forall s\in\gS,a\in\gA,\theta,\theta^\prime\in \RB^d.
$$
\end{assumption}

\begin{assumption}[Positive semidefinite Fisher information]\label{Assumption_PSD_Fisher}
    For all $\theta\in\RB^d$,
    $$F(\theta):=\EB_{(s,a)\sim\nu_\theta}[\nabla_\theta\log\pi_\theta(a|s)\nabla_\theta\log\pi_\theta(a|s)^\top]\succeq\mu_F I_d.
    $$
\end{assumption}
The assumptions above are standard in the literature of policy optimizations \citep{agarwal2021theory}.
We also assume the parametrization realizes good function approximation in terms of transferred compatible function approximation errors, which is first introduced by \cite{agarwal2021theory}.
The error term can be close to zero if the policy class is rich \citep{wang2019neural} or the underlying MDP has low-rank structures \citep{jiang2017contextual}.
\begin{assumption}[Bounded function approximation error]\label{Assumption_func_approx_err}
The transferred compatible function approximation errors satisfies that $\forall t\in\{1,...,T\}$
$$
\begin{aligned}
\min_w E^{\nu^{(t)}}(r,\theta^{(t)},w)&\leq \epsilon_{\text{bias}}\\
\min_w E^{\nu^{(t)}}(c_y,\theta^{(t)},w)&\leq \epsilon_{\text{bias}}\ \forall y\in Y,\\
\end{aligned}
$$
where $\nu^{(t)}$ denotes the state-action occupancy measure induced by policy $\pi^{(t)}$. 
The transferred compatible function approximation errors are defined as:
$$
E^{\nu}(\diamond,\theta,w):=\EB_{(s,a)\sim\nu}(A^{\pi_\theta}_\diamond(s,a)-w^\top\nabla_\theta\log\pi_\theta(a|s))^2.
$$
\end{assumption}

Besides, we also assume the weights to minimize the transferred compatible function approximation errors are bounded.

\begin{assumption}[Bounded Weight]\label{Assumption_est_err}
For any $t\in\{1,...,T\}$, $\forall y\in Y$,
$$
\left\|\underset{w}{\mathrm{argmin}} E^{\nu^{(t)}}(r,\theta^{(t)},w)\right\|_2^2\leq W^2, \left\|\underset{w}{\mathrm{argmin}} E^{\nu^{(t)}}(c_y,\theta^{(t)},w)\right\|_2^2\leq W^2.
$$
\end{assumption}

In the theoretical analysis of SI-CPO, we consider an instance of SI-CPO where we use a sample-based version of NPG \citep{agarwal2021theory} as the policy optimization subroutine, a fixed-horizon Monte-Carlo estimator as the policy evaluation subroutine, and either random search or projected subgradient ascent as the optimization subroutine.
In the NPG algorithm, we use the following natural policy gradient $w^{(t)}$ to update the policy parameters:
$$
w^{(t)}:=F(\theta^{(t)})^\dagger\EB_{(s,a)\sim\nu^{(t)}}(A^{\pi^{(t)}}_\diamond(s,a)\nabla_\theta\pi_\theta(a|s)).
$$
Here $\diamond$ can be either the reward $r$ or some cost function $c_y$.
However, for most RL problems it is computationally prohibitive to evaluate $F(\theta)^\dagger$, and $\EB_{(s,a)\sim\nu^{(t)}}(A^{\pi^{(t)}}_\diamond(s,a)\nabla_\theta\pi_\theta(a|s))$ are usually unknown to the algorithm.
Therefore, we instead use a sample-based estimate of $w^{(t)}$, which can be obtained by solving the following optimization problem by running $K_{sgd}$ steps of stochastic gradient descent:
$$
\hat w^{(t)}\approx\frac{1}{1-\gamma}\arg\min_w E^{\nu^{(t)}}(b,\theta^{(t)},w),
$$
recall that $E^{\nu^{(t)}}(\diamond,\theta^{(t)},w)$ is the transferred function approximation error defined in Assumption~\ref{Assumption_func_approx_err}.
The precise definition of sample-based NPG can be found in Algorithm~\ref{Algorithm_sample_based_NPG} in Appendix~\ref{Appendix_Algorithm}.

As for policy evaluation, we choose to use a Monte-Carlo estimator with a fixed horizon $H$.
The idea is very simple, in each episode we run the target policy $\pi$ for $H$ steps, and record the return
$$
G_i=\sum_{k=0}^{H-1}\gamma^k c_y(s_k,a_k).
$$
The procedure is repeated for $K_{eval}$ times and we take the average as an estimate of $V_{c_y}^{\pi^{(t)}}(\mu)$.
Compared with the more commonly used unbiased Monte-Carlo estimate
$$\widetilde{G}_i=\sum_{k=0}^{H^\prime-1}c_y(s_k,a_k)
$$
where $H^\prime$ is no longer fixed and drawn from an exponential distribution $\mathrm{Exp(\lambda)}$, $G_i$ does introduce bias, but it also has the advantage of being sub-gaussian.
Moreover, the bias term is always bounded by $\frac{\gamma^H}{1-\gamma}$, which decays exponentially as we choose larger $H$s.

\subsubsection{Iteration complexity of SI-CPO}\label{Subsection_SICPO_Iteration_Complexity}

The following two theorems give the iteration complexity of the SI-CPO algorithm when we use either random search or projected subgradient ascent to solve the inner-loop problem.

\begin{theorem}\label{Theorem_random_search_SICPO}
Suppose we use random search to solve the inner-loop problem and Assumption~\ref{Assumption_regular_maxima} holds. 
If we set $\alpha=1/\sqrt{T}$, $\eta=\epsilon+\frac{1}{(1-\gamma)^{3/2}}\sqrt{\left\|\frac{\nu^*}{\nu_0}\right\|_\infty\epsilon_{bias}}$, and 
${K_{sgd}=\widetilde{O}\left(\frac{1}{\epsilon_{bias}^2(1-\gamma)^4}\right)}$ , $ H=O\left(\frac{\log(1-\gamma)+\min\{\log(\epsilon_{bias}),\log(\epsilon)\}}{\log\gamma}\right)$, $K_{eval}=\widetilde{O}\left(\frac{1}{\epsilon^2(1-\gamma)^2}\right)$, ${M=\widetilde{O}\left(\frac{(\mathrm{diam}(Y))^m}{\epsilon^m(1-\gamma)^m}\right)}$, ${T=\widetilde{O}\left(\frac{1}{\epsilon^2(1-\gamma)^6}\right)}$, then we have with probability $1-2\delta$,
$$
    \frac{1}{|\gB|}\sum_{t\in\gB}(V_r^*(\mu)-V_r^{(t)}(\mu))\leq \epsilon+\frac{1}{(1-\gamma)^{3/2}}\sqrt{\left\|\frac{\nu^*}{\nu_0}\right\|_\infty\epsilon_{bias}},
$$
and $\forall t\in\gB$
$$ \sup_{y\in Y}\left[V_{c_y}^{(t)}(\mu)-u_y\right]\leq 2\epsilon+\frac{1}{(1-\gamma)^{3/2}}\sqrt{\left\|\frac{\nu^*}{\nu_0}\right\|_\infty\epsilon_{bias}}.
$$
for any $\epsilon<2\epsilon_0 L_y/(1-\gamma)$.
\end{theorem}
\proof {Proof of Theorem~\ref{Theorem_random_search_SICPO}}
See Appendix~\ref{Appendix_Proofs_SICPO}.
\endproof

\begin{theorem}\label{Theorem_PGA_SICPO}
Suppose we use projected subgradient ascent to solve the inner-loop problem and Assumption~\ref{Assumption_concave_constraint} holds. 
If we set $\alpha=1/\sqrt{T}$, $\eta=\epsilon+\frac{1}{(1-\gamma)^{3/2}}\sqrt{\left\|\frac{\nu^*}{\nu_0}\right\|_\infty\epsilon_{bias}}$, and 
${K_{sgd}=\widetilde{O}\left(\frac{1}{\epsilon_{bias}^2(1-\gamma)^4}\right)}$ , $ H=O\left(\frac{\log(1-\gamma)+\min\{\log(\epsilon_{bias}),\log(\epsilon)\}}{\log\gamma}\right)$, $K_{eval}=\widetilde{O}\left(\frac{1}{\epsilon^2(1-\gamma)^2}\right)$, ${T_{PGA}=O\left(\frac{[\mathrm{diam}(Y)]^2}{\epsilon^2(1-\gamma)^2}
    \right)}$, ${T=\widetilde{O}\left(\frac{1}{\epsilon^2(1-\gamma)^6}\right)}$, then we have with probability $1-\delta$,
$$
    \frac{1}{|\gB|}\sum_{t\in\gB}(V_r^*(\mu)-V_r^{(t)}(\mu))\leq \epsilon+\frac{1}{(1-\gamma)^{3/2}}\sqrt{\left\|\frac{\nu^*}{\nu_0}\right\|_\infty\epsilon_{bias}},
$$
and $\forall t\in\gB$
$$ \sup_{y\in Y}\left[V_{c_y}^{(t)}(\mu)-u_y\right]\leq 2\epsilon+\frac{1}{(1-\gamma)^{3/2}}\sqrt{\left\|\frac{\nu^*}{\nu_0}\right\|_\infty\epsilon_{bias}}.
$$
\end{theorem}
\proof {Proof of Theorem~\ref{Theorem_random_search_SICPO}}
See Appendix~\ref{Appendix_Proofs_SICPO}.
\endproof

In our proof, we focus on the event that the policy evaluation subroutine returns accurate estimates of $V^{(t)}_{c_y}(\mu)$ and the sample-based NPG generates a near-optimal solution of $\min_w E^{\nu^{(t)}}(\diamond,\theta^{(t)},w)$.
We show that this event happens with high probability.
When it happens, with carefully chosen tolerance threshold $\eta$, either the "good set" $\gB$ is large or the policies in $\gB$ perform equally well to the optimal policy $\pi^*$ on average, \textit{i.e.} $\sum_{t\in\gB}(V^{(t)}_r(\mu)-V^*_r(\mu))\geq 0$.
As long as $\gB$ is large enough, we may further conclude that  $\frac{1}{|\gB|}\sum_{t\in\gB}|V^{(t)}_r(\mu)-V^*_r(\mu)|$ is small by typical analysis techniques of NPG \citep{agarwal2021theory}.
Recalling that the constraint violations of policies in $\gB$ are small as long as the inner-loop optimization problems are effectively solved, we complete our proof.

Our ideas of proof are similar to \cite{wei2020comirror, xu2021crpo}.
However, \cite{wei2020comirror} focused on the semi-infinitely constrained convex problems and we focus on the semi-infinitely constrained RL problems.
Moreover, their theoretical results are in the form of bounds on expectations, while ours are in the form of high probability bounds.
Our work is different with \cite{xu2021crpo} in the sense that they address finitely constrained RL problems, and restrict their analysis to two specific forms of policy parametrizations, whereas we consider general policy parametrizations.

\begin{remark}
The error terms of SI-CPO can be attributed to three sources: the function approximation error, the statistical error, and the optimization error.
When we say SI-CPO converges to the globally optimal policy $\pi^*$ at a $\widetilde {O}(1/\sqrt{T})$ rate we mean that if we use a near-perfect parameterized policy class, estimate $V^{(t)}_{c_y}(\mu)$ and the natural policy gradient with adequate data and solve the inner-loop problem with sufficient accuracy, then the averaged error term of SI-CPO has a $\widetilde O(1/\sqrt{T})$ order with high probability.
\end{remark}

\begin{remark}\label{Remark_random_search_vs_fixed_search}
    When solving the inner-loop problem, an alternative approach to random search is to search over a fixed grid of $Y$.
    This is equivalent to a version of naive discretization: we first transform the SICMDP to a finitely constrained MDP by discretizing $Y$, and then solve the resulting problem with CRPO \citep{xu2021crpo}.
    From a theoretical viewpoint, random search is no better than the gird search since both need to search over a $\widetilde{O}((\mathrm{diam}(Y)/\epsilon)^m)$-sized grid to ensure $\epsilon$-optimality.
    However, in numerical experiments we find that the approach based on random search is far more efficient that the approach based on grid search.
    The reasons can be two-fold: 1) in the theoretical analysis we must give guarantees for the hardest problem instances, but real-world problem settings may contain structures that can be exploited by random search \citep{bergstra2012random}; 2) in random search the random grid are generated in an independent way in each iteration, which can reduce the bias introduced by replacing the constraint set $Y$ with a fixed finite grid.
\end{remark}

\subsubsection{Sample complexity of SI-CPO}\label{Subsection_SICPO_Sample_Complexity}

\begin{corollary}\label{Corollary_Sample_Complexity_SICPO}
SI-CPO need to make $\widetilde O{\left(\frac{1}{\epsilon^2\min\{\epsilon^2,\epsilon_{bias}^2\}(1-\gamma)^{10}}\right)}$ interactions with the environments to ensure with high probability
$$
    \frac{1}{|\gB|}\sum_{t\in\gB}(V_r^*(\mu)-V_r^{(t)}(\mu))\leq \epsilon+\frac{1}{(1-\gamma)^{3/2}}\sqrt{\left\|\frac{\nu^*}{\nu_0}\right\|_\infty\epsilon_{bias}},
$$
and $\forall t\in\gB$
$$ \sup_{y\in Y}\left[V_{c_y}^{(t)}(\mu)-u_y\right]\leq 2\epsilon+\frac{1}{(1-\gamma)^{3/2}}\sqrt{\left\|\frac{\nu^*}{\nu_0}\right\|_\infty\epsilon_{bias}}.
$$
\end{corollary}
\proof{Proof of Corollary~\ref{Corollary_Sample_Complexity_SICPO}}
This corollary is a direct consequence of Theorem~\ref{Theorem_random_search_SICPO} as the sample complexity is of the order $T \cdot H\cdot(K_{eval}+K_{sgd})$.
Note that the sample complexity bound is independent of how we solve the inner-loop problem.
\endproof

Our sample complexity bound is of the order $\widetilde{O}\left(\frac{1}{\epsilon^4(1-\gamma)^{10}}\right)$.
This is worse than typical sample complexity bounds for sample-based NPG such as the $\widetilde{O}\left(\frac{1}{\epsilon^4(1-\gamma)^8}\right)$ in \cite{agarwal2021theory}.
This does not mean that the constrained problem is statistically harder or our bounds are loose.
The difference comes from that their goal is to ensure accuracy in expectation, while our goal is to ensure accuracy with high probability.
Therefore, in our algorithm, we need to run more SGD iterations (corresponding to larger $K_{sgd}$) to find a better estimate of the natural gradient.

\section{Numerical Experiments}\label{Section_Experiment}

\section{Experiments}\label{sec:experiments}
We validate our approach using multiple datasets containing real-life data from the fields of criminal risk assessment, credit, lending, and college admissions. In each of the datasets we select a binary feature and treat it as the protected attribute (e.g., race or gender), which is the feature we require our trained classifier to behave fairly upon. Our proposed method performs well on all of these datasets, succeeding in removing unfairness almost entirely, at a very modest price in terms of accuracy.


\begin{table*}[h]
\centering
\resizebox{\textwidth}{!}{
\def\arraystretch{1.2}

\begin{tabular}{c c c | c | c | c || c | c | c || c | c | c |}

\cline{4-12}
&&&
\multicolumn{9}{ c| }{\textbf{COMPAS Dataset}}
\\ \cline{4-12}
&&&
\multicolumn{3}{ c|| }{\textbf{FPR Considerations}}&
\multicolumn{3}{ c|| }{\textbf{FNR Considerations}}&
\multicolumn{3}{ c| }{\textbf{Both Considerations}}
\\ \cline{4-12}
&&&
 $\mathbf{Acc.}$ &  $\mathbf{D_{FPR}}$ &  $\mathbf{D_{FNR}}$ &  $\mathbf{Acc.}$ &  $\mathbf{D_{FPR}}$ &  $\mathbf{D_{FNR}}$ &  $\mathbf{Acc.}$ &  $\mathbf{D_{FPR}}$ &  $\mathbf{D_{FNR}}$
\\  \cline{4-12}
\vspace*{-0.5ex}
\\ \cline{1-2} \cline{4-12}
\multicolumn{1}{ |c  }{} &
\multicolumn{1}{ c|  }{  \textbf{Our Method (AVD Penalizers)}}  &&
$\mathbf{0.660}$    &  $\mathbf{0.01}$  &  $0.04$ &
$\mathbf{0.653}$    &  $0.02$   &  $\mathbf{0.04}$ &
$\mathbf{0.654}$    &  $\mathbf{0.02}$  &  $\mathbf{0.04}$
\\ \cline{1-2} \cline{4-12}
\multicolumn{1}{ |c  }{} &
\multicolumn{1}{ c|  }{  \textbf{Our Method (SD Penalizers)}}  &&
$\mathbf{0.664}$    &  $\mathbf{0.02}$  &  $0.09$ &
$\mathbf{0.661}$    &  $0.05$   &  $\mathbf{0.03}$ &
$\mathbf{0.661}$    &  $\mathbf{0.02}$  &  $\mathbf{0.03}$
\\ \cline{1-2} \cline{4-12}
\multicolumn{1}{ |c  }{} &
\multicolumn{1}{ c|  }{  Zafar et al.~(\citeyear{disparatemistreatment})}  &&
$0.660$    &   $0.06$    &   $0.14$  &
$0.662$    &   $0.03$    &   $0.10$  &
$0.661$    &   $0.03$    &   $0.11$
\\ \cline{1-2} \cline{4-12}
\multicolumn{1}{ |c  }{} &
\multicolumn{1}{ c|  }{  Zafar et al. Baseline~(\citeyear{disparatemistreatment})}  &&
$0.643$    &   $0.03$    &   $0.11$  &
$0.660$    &   $0.00$    &   $0.07$  &
$0.660$    &   $0.01$    &   $0.09$
\\ \cline{1-2} \cline{4-12}
\multicolumn{1}{ |c  }{} &
\multicolumn{1}{ c|  }{  Hardt et al.~(\citeyear{hardt})}  &&
$0.659$    &  $0.02$    &   $0.08$  &
$0.653$    &  $0.06$   &    $0.01$  &
$0.645$    &  $0.01$   &    $0.01$
\\ \cline{1-2} \cline{4-12}
\multicolumn{1}{ |c  }{} &
\multicolumn{1}{ c|  }{  \textbf{Vanilla Regularized Logistic Regression}}  &&
$\mathbf{0.672}$    &   $\mathbf{0.20}$    &   $\mathbf{0.30}$  &
$\mathbf{0.672}$    &   $\mathbf{0.20}$    &   $\mathbf{0.30}$  &
$\mathbf{0.672}$    &   $\mathbf{0.20}$    &   $\mathbf{0.30}$
\\ \cline{1-2} \cline{4-12}
\end{tabular}
}
\vspace{3mm}
\caption{Performance comparison on the COMPAS dataset. For the approaches in bold -- Accuracy, FPR difference and FNR difference are evaluated on the test set, averaging over five runs and using a 70-30 training/test split. The performance of the remaining three approaches is stated as reported in Zafar et al.~(\citeyear{disparatemistreatment}).} \label{table:comparison_results}
\end{table*}



\begin{figure*}[b]
  \includegraphics[scale=0.6]{compas0-400.png}
  \caption{COMPAS Dataset. Accuracy, FPR difference ($\mathbf{D_{FPR}}$), and FNR difference ($\mathbf{D_{FNR}}$) (all evaluated on the test set) of the learned classifier, as a function of the weight $c=c_1 = c_2 \geq 0$ placed on the fairness penalizer terms. On the left we use the Absolute Value Difference (AVD) penalizer, and the Squared Difference (SD) penalizer on the right, both as presented in Section~\ref{regularization}. ``Relaxed FPR/FNR Diff.'' plots the value of the relevant penalization term.} %In this particular run, parameters chosen for the absolute value relaxation were: $c=80, q_c=60$, and for the squared relaxation: $c=220, q_c=30$.}
  \label{fig:compas}
\end{figure*}


\subsection{Implementation}
\textbf{Our method} 
%We instantiate our method in the following way: Given dataset $Q$, we split it randomly into a training set $S$ (which we will use for learning) and a test set $T$ (which we will only use for reporting performance). 
For the purpose of comparison with  Zafar et al.~(\citeyear{disparatemistreatment}) and Hardt et al.~\cite{hardt} on the COMPAS data, we use a parameter $c$ to induce three possible combinations of weights on the FPR and FNR penalization terms: $c = c_1$ and $c_2 = 0$; $c_1 = 0$ and $c = c_2$; and $c = c_1 = c_2$. For the other three datasets, we consider only $c = c_1 = c_2$.\footnote{The reason for varying the values of $c$ in the training phase is since we shifted to a proxy problem, in which we rely on the distance from the decision boundary rather the actual classifications. 
%Our hope is that there is no need for a worst-case cross validation between all of the combinations of $c_1, c_2, c_3$, and that the training scheme we propose is sufficient. 
It is possible, of course, that even better results are attainable using our scheme with other combinations of $c_1, c_2$, and $q$.} To explore the accuracy/fairness trade-off curve for the relaxed optimization problem~(\ref{eq:2}), we train for different values of $c$, starting at $c=0$ (which is just standard logistic regression), and growing gradually.



Given a dataset $Q$ and fixing a $d_1, d_2 \in \{0, 1\}$ of interest, we use the following training scheme:
\begin{enumerate}
\item Split $Q$ at random into training set $S$ and test set $T$.
\item For each $c$, perform cross-validation on $S$ to select the corresponding best value $q_c$ for the regularization parameter.
\item For each $(c,q_c)$, let $\theta_c = \argmin\limits_{\theta} \text{Proxy}(\theta;S,c,c,q_c)$.
\item Select $\theta^* \in \argmin\limits_{\theta_c} \text{Objective}(\theta_c;S,d_1,d_2)$.
\item Evaluate performance using $\theta^*$ on test set $T$.
\end{enumerate}
We report the average of five such runs, each with a fresh training-test split.




%We instantiate our method by solving the relaxed optimization problem~(\ref{eq:2}), in place of the original, non-convex problem~(\ref{eq:1}).  
%We test our approach with three different combinations of weights on the penalization terms:
%\katrina{What are the $d$, and how are they related to the $c$s?}
%\begin{enumerate}
%\item FPR considerations only: $d_1 = 1, d_2 = 0$.
%\item FNR considerations only: $d_1 = 0, d_2 = 1$.
%\item Both FPR, FNR considerations, assigned similar significance: $d_1 = 1, d_2 = 1$.
%\end{enumerate}
%One could, of course, pick any other combination of the FPR and FNR penalty weights.

%\katrina{I don't understand how the below is distinct from the list above}
%Learning is done by training the parameters of a logistic regressor to solve~\ref{eq:2}, while picking the value of $c_1, %c_2$ as the following:
%\begin{enumerate}
%\item FPR considerations only: $c_1 = c \geq 0$, $c_2 = 0$.
%\item FNR considerations only: $c_1 = 0$, $c_2 = c \geq 0$.
%\item Both FPR, FNR considerations, assigned similar significance: $c_1 = c_2 = c \geq 0$
%\end{enumerate}



% We then cross-validate to pick the best $c_3$ (the weight on the standard $\ell_2$-regularization term) given $c$.\footnote{The reason for varying the values of $c$ in the training phase is since we shifted to a proxy problem, in which we rely on the distance from the decision boundary rather the actual classifications. 
%Our hope is that there is no need for a worst-case cross validation between all of the combinations of $c_1, c_2, c_3$, and that the training scheme we propose is sufficient. 
%It is possible, of course, that even better results are attainable using our scheme with other combinations of $c_1, c_2, c_3$.} For each such combination, we report results as the averages of multiple \katrina{how many?} different runs, each time splitting data randomly into training and test sets.
%\yahav{We need to shorten this description.}

We solve the relaxed convex optimization problem using the CVXPY solver. Due to stability issues with large training sets, we use a train/test split of 30-70 on the larger datasets, rather than 70-30 as on the COMPAS dataset\footnote{The code implementing our method can be found at https://github.com/jjgold012/lab-project-fairness}.

%
%
%We then report the results (as evaluated on the test set) attained by a regressor $\theta \in \mathbb{R}^d$ that minimizes (on the training set $S$) a weighted combination of the $0$-$1$ loss and the differences in FPR and FNR across populations:
%\begin{equation*}
%\begin{aligned}
%&\underset{\theta}{\text{argmin}}
%& & L_{S}^{0\text{-}1}(\theta) \\
%&&& + d_1|FPR_{A=0}(\theta;S)-FPR_{A=1}(\theta;S)| \\
%&&& + d_2|FNR_{A=0}(\theta;S)-FNR_{A=1}(\theta;S)|
%\end{aligned}
%\end{equation*}
%
%\katrina{What is $d_1$ vs. $c_1$ etc.?}



%For classification, we decided use a standard cut-off threshold of $c=0.5$. There are of course, further possible interactions between the FPR, FNR considerations, and picking a certain cut-off level. These are not straightforward, since  these interactions are data-specific. 



%allows for flexibility in picking the values of $c_1, c_2$, which reflect the significance we wish to place on the objectives of achieving accuracy, equal FPR, and equal FNR. As for $c_3$, we will want to find the value of it that achieves the best results, for any combined objective of accuracy and fairness defined by a specific selection of $c_1,c_2$. Therefore, given a specific selection of $c_1, c_2$, we apply cross-validation to select the value of $c_3$. 




We briefly describe the other algorithmic approaches to which we compare:\\
\textbf{Zafar et al.}~(\citeyear{disparatemistreatment}) performs optimization by considering a proxy for the bias: the covariance between the samples' sensitive attributes and the signed distance between the feature vectors of misclassified users and the classifier decision boundary.\\
\textbf{Zafar et al. Baseline}~(\citeyear{disparatemistreatment}) tries to enforce equal FP/FN rates on the different groups by introducing different penalties for misclassified data points with different sensitive attribute values during the training phase.\\
\textbf{Hardt et al.}~(\citeyear{hardt}) performs post-processing on a standard trained (unfair) logistic regressor, picking different decision thresholds for different groups, and possibly adding randomization.


\subsection{Experimental Results}

In what follows, we use the following notation, given a trained classifier $\hat{Y}$:
\begin{align*}
\mathbf{D_{FPR}}&=\left|FPR_{A=0}(\hat{Y})-FPR_{A=1}(\hat{Y})\right| \\ 
\mathbf{D_{FNR}}&=\left|FNR_{A=0}(\hat{Y})-FNR_{A=1}(\hat{Y})\right|
\end{align*}
The values $FPR_{A=0}(\hat{Y})$, $FPR_{A=1}(\hat{Y})$, $FNR_{A=0}(\hat{Y})$, $FNR_{A=1}(\hat{Y})$ are reported as evaluated on the test set.

\paragraph{The COMPAS Dataset\footnote{https://github.com/propublica/compas-analysis}} The Correctional Offender Management Profiling for Alternative Sanctions (COMPAS) records from Broward County, Florida 2013-2014, made available online by ProPublica, are perhaps the best-studied data in the context of fairness.  The goal in this scenario is to successfully predict recidivism within two years, based on features such as age, gender, race, number of prior offenses, and charge degree. The dataset contains 5,278 samples. The protected attribute in this scenario is race, where $A$ indicates black or white. We filtered the dataset using the same features as Zafar et al.~(\citeyear{disparatemistreatment}), to allow for comparison.

%\begin{table}[h]
%\centering
%\begin{tabularx}{\columnwidth}{c|c|c|c}
%\hline
%  &  Recid. ($y = 1$)        & No Recid.  ($y = 0$)       & Total \\ \hline
%Black &  $ 1661   $ & $ 1514 $ &  $ 3175 $ \\ \hline
%White &  $ 822   $  & $1281  $ &  $ 2103 $ \\ \hline
%Total &  $ 2483  $  & $2795 $ &  $ 5278 $ \\\hline
%\end{tabularx}
%\caption{Statistics of the ProPublica COMPAS data.} \label{table:compas-stats}
%\label{tab:stats}
%\end{table}
%\vspace{-1em}

%\begin{table}[h]
%\centering
%\begin{tabularx}{\columnwidth}{c|c}
%\hline
%Feature  &  Description \\ \hline
%Age Category &  $<25$, between $25$ and $45$, $>45$ \\
%Gender &  Male or Female \\
%Race &  White or Black \\
%Priors Count &  0--37 \\
%Charge Degree &  Misconduct or Felony \\
%\hline
%2-year-recid. & Whether or not the  \\
%(target feature)  & defendant recidivated within two years
%\end{tabularx}
%\caption{Description of features used from ProPublica COMPAS data.} \label{table:compas-features}
%\label{tab:features}
%\end{table}




\begin{table*}[t]
\centering
\caption{A description of the datasets used, along with parameters of the training procedure used for each.}
\label{table:datasets_description}
\begin{adjustbox}{max width=\textwidth}
\begin{tabular}{|l|l|l|l|l|l|l|l|}
\hline
\textbf{Dataset} & \textbf{No. Samples} & \textbf{No. Features} & \textbf{Train/Test Split} & \textbf{No. Repetitions} & \textbf{No. Folds in CV} & \textbf{Protected Feature} & \textbf{Target Variable} \\ \hline
COMPAS           & 5,278                     & 5                          & 70-30                     & 5                        & 5                                 & Race                       & 2-Year-Recidivism        \\ \hline
Adult            & 30,162                    & 10                         & 30-70                     & 5                        & 5                                 & Gender                     & Income Over/Under 50K    \\ \hline
Default          & 30,000                    & 23                         & 30-70                     & 5                        & 3                                 & Gender                     & Defaulting On Payments   \\ \hline
Admissions       & 20,839                    & 17                         & 30-70                     & 5                        & 3                                 & Race                       & Passing Bar Exam         \\ \hline
\end{tabular}
\end{adjustbox}
\end{table*}


\begin{table*}[t]
\centering
\resizebox{\textwidth}{!}{
\def\arraystretch{1.2}

\begin{tabular}{c c c | c | c | c || c | c | c || c | c | c |}

\cline{4-12}
&&&
\multicolumn{3}{ c|| }{\textbf{Adult Dataset}}&
\multicolumn{3}{ c|| }{\textbf{Default Dataset}}&
\multicolumn{3}{ c| }{\textbf{Admissions Dataset}}
\\ \cline{4-12}
%&&&
%\multicolumn{3}{ c|| }{\textbf{Both Considerations}}&
%\multicolumn{3}{ c|| }{\textbf{Both Considerations}}&
%\multicolumn{3}{ c| }{\textbf{Both Considerations}}
%\\ \cline{4-12}
&&&
 $\mathbf{Acc.}$ &  $\mathbf{D_{FPR}}$ &  $\mathbf{D_{FNR}}$ &  $\mathbf{Acc.}$ &  $\mathbf{D_{FPR}}$ &  $\mathbf{D_{FNR}}$ &  $\mathbf{Acc.}$ &  $\mathbf{D_{FPR}}$ &  $\mathbf{D_{FNR}}$
\\  \cline{4-12}
\vspace*{-0.5ex}
\\ \cline{1-2} \cline{4-12}
\multicolumn{1}{ |c  }{} &
\multicolumn{1}{ c|  }{  \textbf{Our Method (AVD Penalizers)}}  &&
$\mathbf{0.776}$    &  $\mathbf{0.00}$  &  $\mathbf{0.04}$ &
$\mathbf{0.807}$    &  $\mathbf{0.00}$   &  $\mathbf{0.01}$ &
$\mathbf{0.950}$    &  $\mathbf{0.01}$  &  $\mathbf{0.00}$
\\ \cline{1-2} \cline{4-12}
\multicolumn{1}{ |c  }{} &
\multicolumn{1}{ c|  }{  \textbf{Our Method (SD Penalizers)}}  &&
$\mathbf{0.783}$    &  $\mathbf{0.00}$  &  $\mathbf{0.09}$ &
$\mathbf{0.806}$    &  $\mathbf{0.01}$   &  $\mathbf{0.02}$ &
$\mathbf{0.950}$    &  $\mathbf{0.00}$  &  $\mathbf{0.00}$
\\ \cline{1-2} \cline{4-12}
\multicolumn{1}{ |c  }{} &
\multicolumn{1}{ c|  }{  \textbf{Vanilla Regularized Logistic Regression}}  &&
$\mathbf{0.800}$    &   $\mathbf{0.08}$    &   $\mathbf{0.39}$  &
$\mathbf{0.807}$    &   $\mathbf{0.01}$    &   $\mathbf{0.05}$  &
$\mathbf{0.951}$    &   $\mathbf{0.16}$    &   $\mathbf{0.02}$
\\ \cline{1-2} \cline{4-12}
\end{tabular}
}
\vspace{3mm}
\caption{Performance on the Adult, Loan Default, and Admissions datasets, penalizing for both FPR and FNR difference. Accuracy, FPR difference and FNR difference are evaluated on the test set, averaging over five runs and using a 30-70 training/test split.} \label{table:comparison_results_rest}
\end{table*}


In Table~\ref{table:comparison_results}, we compare the performance of our approach with that of three other techniques from the literature. Each method was trained based on logistic regression.  As a basis for comparison, we also present the performance of vanilla logistic regression, absent fairness considerations, with the regularization parameter selected via cross-validation.\footnote{Zafar et al.~(\citeyear{disparatemistreatment}) do not incorporate regularization in any of the approaches they report.}
%Results are reported as the averages of 5 different runs \katrina{Is that still correct?}, each time splitting data evenly and randomly into training and test sets. 
Results for Zafar et al., Zafar et al. baseline, and Hardt et al. appear here as reported in Zafar et al.~(\citeyear{disparatemistreatment}).\footnote{Our method selects the classifier based on the training set only and reports its performance over the test set. Results for the three other approaches, reported by Zafar et al.~(\citeyear{disparatemistreatment}), are based on tuning parameters after seeing the trade-off curve over the test set, and reporting according to the best selection of these parameters.}
%\katrina{Perhaps here is the right place for a footnote about the discrepancy with the Zafar baseline}

We find that the vanilla logistic regressor (absent fairness considerations) results in significant unfairness, as $\mathbf{D_{FPR}}=0.20$, and $\mathbf{D_{FNR}}=0.30$. The overall accuracy of this classifier measured on the test set was $0.672$.\footnote{Zafar et al.~(\citeyear{disparatemistreatment}) report a slightly different baseline of: Accuracy = 0.668, $\mathbf{D_{FPR}}=0.18$, $\mathbf{D_{FNR}}=0.30$.} Our SD penalization approach empirically achieves approximately the same accuracy as the Zafar et al.~(\citeyear{disparatemistreatment}) approach, with significantly better fairness. It is difficult to compare fairness-accuracy tradeoffs with the Hardt et al.~(\citeyear{hardt}) approach, since their accuracy is significantly lower than ours. A more direct comparison is possible by noting that our learned classifier can be post-processed to improve its fairness at a direct cost to accuracy. Hence, we can achieve accuracy of $0.659$ with $\mathbf{D_{FPR}} = \mathbf{D_{FNR}} = 0.01$, which compares very favorably with the Hardt et al. accuracy rate of 0.645 given the same FPR and FNR rates.\footnote{For completeness, we note that using a 50-50 training-test split (again not using the test set for parameter selection), our method (SD, both considerations) produces a classifier that provides: Accuracy = 0.659, $\mathbf{D_{FPR}} = 0.01, \mathbf{D_{FNR}} = 0.05$. This classifier can be post-processed to achieve rates of: Accuracy = 0.655, $\mathbf{D_{FPR}} = \mathbf{D_{FNR}} = 0.01$.}

Figure \ref{fig:compas} illustrates the accuracy/fairness trade-offs achievable using our scheme. Increasing the weight $c$ on the proxy fairness penalizers results in reducing their magnitude. The figure also illustrates how our relaxed penalizers succeed in tracking the real FPR and FNR differences. 
%
%
%\katrina{Must rewrite the following paragraph}
%We observe that our method succeeds in eliminating unfairness almost completely on the COMPAS dataset, while retaining most of the accuracy, when compared to the vanilla logistic regression. We achieve very low difference rates when penalizing for achieving each of the FPR and FNR criteria individually, and also for both. We achieve preferable results comparing to Zafar et al. and Zafar et al. baseline in all 3 scenarios, and also comparing to Hardt et al. in the settings of false positive/false negative considerations only. In the setting of both considerations - The Hardt et al. method removes a larger portion of the unfairness, however it results in major accuracy loss as it achieves accuracy rate of 0.645 in comparison to our method which results in accuracy of 0.665, retaining most of the original accuracy rate while removing most of the unfairness.




%The Hardt et al.~\cite{hardt} approach as reported removes a smaller portion of the bias in the different scenarios, however for FP/FN constraints alone, it provides higher accuracy rates. The Zafar et al.~(\citeyear{disparatemistreatment}) approach as reported retains significant bias (in most cases), but in some cases  achieves slightly superior accuracy rates to the methods above. 

%These performance comparisons are incomplete in the sense that each of the compared techniques has the potential to trade off between accuracy and fairness, using some degree of parameter tuning; what we report here is only one point on the achievable trade-off frontier for each algorithm. The ``correct'' trade-off, and, in particular, the best manner in which to weigh unfairness in the FPR against unfairness in the FNR, are matters of opinion. We have chosen to report our method's performance under parameters designed to very aggressively mitigate unfairness, at some cost to the accuracy.

%It would certainly be desirable to evaluate these and other approaches to fair learning on other datasets and on different tasks, particularly on larger datasets, which might afford both greater accuracy and better bias-reduction. The present empirical evaluations, however, suggest that our regularization-based approach provides a new tool worthy of consideration---we succeed in almost entirely eliminating bias on the hold-out set, at a modest price in terms of accuracy.

%Due to the fact that our true objective includes the original non-convex penalization terms, our approach does not carry any formal guarantees. However, the ease of implementation, generality, and empirical results are encouraging. Figure~\ref{fig:test1} illustrates the rate of convergence to a fair, accurate classifier on this dataset.
%In terms of computation costs, given that at each iteration we must calculate the gradient according to the FPR and FNR regularizers, we are required to predict the labels for the entire training set at each step. 
%However, this does not pose a computational burden, as it is already required by the (classic) gradient descent algorithm in our logistic regressor fitting scheme. Furthermore, when given a sufficiently large dataset (one or two orders of magnitude larger than the one currently available for the COMPAS scores data), this could be relaxed to sampling only a mini-batch of samples from the training data set at each iteration (much as is done in stochastic gradient descent).






\subsection{Additional Datasets}


Table~\ref{table:datasets_description} provides summary statistics on each of the datasets on which we tested our approach. We also briefly describe the datasets below. 


{\bf The Adult Dataset}\footnote{http://archive.ics.uci.edu/ml/datasets/Adult} is based on 1994 US Census data. The task we consider is to predict whether the income of each individual is over or under 50K dollars per year, based on features such as occupation, marital status, and education. The protected attribute selected in this task is gender. 

{\bf The Loan Default Dataset}\footnote{{\scriptsize https://archive.ics.uci.edu/ml/datasets/default+of+credit+card+clients}}
contains data regrading Taiwanese credit card users. The task we consider is to predict whether an individual will default on payments, based on features such as history of past payments, age, and the amount of given credit. The protected attribute is gender.

{\bf The Admissions Dataset}\footnote{http://www2.law.ucla.edu/sander/Systemic/Data.htm}
contains records of law school students who went on to take the bar exam. The task we consider is to predict whether a student will pass the exam based on features such as LSAT score, undergraduate GPA, and family income. The protected attribute is set to race.

Table~\ref{table:comparison_results_rest} describes the performance of our approach on these datasets, and Figures~\ref{fig:adult},~\ref{fig:default}, and~\ref{fig:lawschool} illustrate the fairness-accuracy trade-offs we achieve in each context. Overall, we see that unfairness is nearly eliminated while accuracy remains quite high. The dataset on which accuracy suffers most under our approach is the Adult dataset, which is also the dataset on which the vanilla regression is the most unfair.


\begin{figure*}[]
  \includegraphics[scale=0.6]{adult0-800.png}
  \caption{Adult Dataset. Fairness-Accuracy tradeoffs, as in Figure~\ref{fig:compas}.}
  \label{fig:adult}  
\end{figure*}



\begin{figure*}[]
  \includegraphics[scale=0.6]{default0-50.png}
  \caption{Loan Default Dataset. Fairness-Accuracy tradeoffs, as in Figure~\ref{fig:compas}.}
  \label{fig:default}
\end{figure*}



\begin{figure*}[]
  \includegraphics[scale=0.6]{admissions0-400.png}
  \caption{Admissions Dataset. Fairness-Accuracy tradeoffs, as in Figure~\ref{fig:compas}.}
  \label{fig:lawschool}
\end{figure*}




\section{Conclusion}

\begin{comment}
\begin{figure}
\includegraphics[width=\linewidth]{figs/beyond_tss_lesion.pdf}
\caption[]{End-to-End runtime lesion study of the entire MNIST dataset and the FMA featurized music dataset. Each of DROP's contributions provides a runtime improvement.}
\label{fig:beyond_lesion}
\end{figure}
\end{comment}



\section{Conclusion}
\label{sec:conclusion}

Advanced data analytics techniques must scale to rising data volumes. 
DR techniques offer a powerful toolkit when processing these datasets, with PCA frequently outperforming popular techniques in exchange for high computational cost. 
In response, we propose DROP, a new dimensionality reduction optimizer. 
DROP combines progressive sampling, progress estimation, and online aggregation to identify high quality low dimensional bases via PCA without processing the entire dataset by balancing the runtime of downstream tasks and achieved dimensionality. 
Thus, DROP provides a first step in bridging the gap between quality and efficiency in end-to-end DR for downstream \red{analytics}. 

%We revisit canonical operators for time series dimensionality reduction and the measurement study of~\cite{keogh-study}, and show that PCA is more effective than popular alternatives in the data mining literature often by a margin of over $2\times$ on average on gold-standard time series benchmark data sets with respect to output data dimension. More surprisingly, we empirically demonstrate that a small number of samples are sufficient to accurately characterize directions of maximum variance and obtain a high-quality low-dimensional transformation.




\section{Acknowledgement}
% This work has been supported by the National Key Research and Development Project of China (No. 2020AAA0104400). 
The authors would like to thank Mr. Hao Jin for helpful discussions.

\bibliography{ref}
\bibliographystyle{abbrvnat}
\newpage

\newpage
\newpage
\appendix
\section{Omitted Proofs in Section \ref{Section_SICRL}}
\label{Appendix_Proofs_4}
\proof{Proof of Theorem \ref{Theorem_Feasible}.}
By Lemma \ref{Theorem_Empirical_Bernstein}, 
$$
\PB\paren{|P(s^\prime|s,a)-\hat P(s^\prime|s,a)|\leq \sqrt{\frac{2\hat P(s^\prime|s,a)(1-\hat P(s^\prime|s,a))\log4/\delta}{n}}+\frac{4\log 4/\delta}{n}}\geq 1-\delta.
$$
By Lemma \ref{Theorem_Hoeffding_Inequality}, 
$$
\PB\paren{|P(s^\prime|s,a)-\hat P(s^\prime|s,a)|\leq \sqrt{\frac{\log 2/\delta}{2n}}}\geq  1-\delta.
$$
Combining the two inequalities in a union bound, we have:
$$
\PB\paren{|P(s^\prime|s,a)-\hat P(s^\prime|s,a)|\leq d_\delta(s,a,s^\prime)}\geq  1-2\delta.
$$
Again we apply the union bound argument to get:
$$
\PB(M\in M_\delta)=\PB\paren{|P(s^\prime|s,a)-\hat P(s^\prime|s,a)|\leq d_\delta(s,a,s^\prime),\forall s,s^\prime\in \gS,a\in \gA}\geq  1-2|\gS|^2|\gA|\delta.
$$
Finally, Problem (\ref{Problem_Optimistic}) is feasible as long as $P\in M_\delta$ because of Assumption $\ref{Assumption_Feasible}$.
\endproof
\section{Omitted Proofs in Section \ref{Section_Theory_SICRL}}
\label{Appendix_Proofs_SICRL}
First, we define some additional notations. 
Given a stationary policy $\pi$, we define the value function $V_\diamond^\pi(s)=\EB\paren{\sum_{t=0}^\infty \gamma^t r(s_t,a_t)|s_0=s}$, $V_\diamond^\pi=(V_\diamond^\pi(s_1), \ldots, V_\diamond^\pi(s_{|\gS|}))^\top\in \RB ^{|\gS|}$.
Thus we have $V^\pi_\diamond(\mu)=\mu^\top V_\diamond^\pi$.
Here $\diamond$ represents either the reward $r$ or cost $c_y$.
We use $Q_\diamond^\pi(s,a):=\EB\paren{\sum_{t=0}^\infty \gamma^t \diamond(s_t,a)}$ and $Q_\diamond^\pi=(Q_\diamond^\pi(s_1,a_1),...,Q_\diamond^\pi(s_{\gS},a_{|\gA|}))\in\RB^{|\gS|\cdot|\gA|}$ to denote the state-action value function. 
The local variance is defined as $\Var_P(V_\diamond^\pi)(s,a)=\EB_{s^\prime\sim P(\cdot|s,a)}(V_\diamond^\pi(s^\prime)-P(\cdot|s,a)V_\diamond^\pi)^2$.
We view $\Var_P(V^\pi)$ as vectors of length $|\gS|\cdot|\gA|$. 
We overload notation and let $P$ also refer to a matrix of size $(|\gS|\cdot |\gA|)\times |\gS|$, where the entry $P_{(s, a), s^{\prime}}$ is equal to $P(s^\prime|s,a)$. 
We also define $P^\pi$ to be the transition matrix on state-action pairs induced by a stationary policy $\pi$, namely:
$$P_{(s, a),\left(s^{\prime}, a^{\prime}\right)}^{\pi}:=P\left(s^{\prime}| s, a\right) \pi\left(a^{\prime} |s^{\prime}\right).
$$
We use $\widetilde V_\diamond^{\pi}(s), \widetilde Q_\diamond^\pi(s,a), {\Var_{\widetilde P}}(\widetilde{V}_\diamond^\pi)(s,a), \widetilde V_\diamond^\pi, \widetilde Q_\diamond^\pi,{\Var_{\widetilde P}}(\widetilde{V}^\pi), \widetilde P, \widetilde P^\pi$ to denote the value function, state-action value function, local variance, vector of the value function, vector of the state-action value function, vector of local variance, transition matrix, transition matrix on state-action pairs w.r.t. SICMDP $\widetilde M$, respectively.

\begin{lemma}\label{Lemma_Iteration_Comlexity_General}
    Suppose for all $t\in\{1,...,T\}$, 
    $$
    \frac{1}{1-\gamma}\sum_{s, a,s^\prime}z^{(t)}(s,a,s^\prime)c_{y^{(t)}}(s,a)- u_{y^{(t)}}\geq \max_{y\in Y}\left[\frac{1}{1-\gamma}\sum_{s, a,s^\prime}z^{(t)}(s,a,s^\prime)c_{y}(s,a)- u_{y}\right]-\epsilon,
    $$
    Then if we set $\eta=\epsilon$ and $T=O\left(\left[\frac{\mathrm{diam}(Y)|\gS|^2|\gA|}{(1-\gamma)\epsilon}\right]^m \right)$,
    then SI-CRL is guaranteed to output a $2\epsilon$-optimal solution of Problem~\ref{Problem_Optimistic_ELSIP}.
\end{lemma}
\proof{Proof of Lemma~\ref{Lemma_Iteration_Comlexity_General}}
For the convenience of presentation, then the problem can be written as
$$
\begin{aligned}
    \max_{z\in Z}\ &z^\top r\\
    \text{s.t.}\ &z^\top c_y\leq u_y,\ \forall y\in Y.
\end{aligned}
$$
Here $r,c\in[0,1]^{|\gS|^2|\gA|}$, $Z\subset\RB^{|\gS|^2|\gA|}$ is a feasible set defined by constraints other than the semi-infinite one.
Let $f(y,z)=z^\top c_y-u_y$, we note $f(y,z)$ is Lipschitz w.r.t. $y$ and the Lipschitz coefficient is $\beta:=\frac{2|\gS|^2|\gA|L_y}{1-\gamma}$.
WLOG, we also assume $Y$ is contained in a $\|\cdot\|_\infty$ ball with radius $R$ with $R\leq \frac{\mathrm{diam}(Y)}{2}$.
At iteration $t<T$, if we have $f(y^{(t)}, z^{(t)})\leq\epsilon$, then the algorithm terminates and we obtain a $2\epsilon$-optimal solution of Problem~\ref{Problem_Optimistic_ELSIP}.
Else we have $f(y^{(t)},z^{(t)})>\epsilon$.
Since $f(z,y)$ is $\beta$-Lipshitz in y, we can conclude $\forall z$, if $f(z,y)>\epsilon$ and $f(z,y^\prime)<0$, then $\|y-y^\prime\|_\infty>\epsilon/\beta$.
Define $B^{(t)}=\{\|y-y^{(t)}\|_\infty\leq \epsilon/2\beta\}$, as $f(y^{(t)},z^{(t)})>\epsilon$ and $f(y^{(t^\prime)},z^{(t)})\leq 0$, $t^\prime=1,...,t-1$, we have $B^{(t)}\cap\left(\cup_{t^\prime=1}^{t-1} B^{(t^\prime)}\right)=\emptyset$. 
Then by induction one may conclude $\{B^{(t^\prime)},t^\prime=1,...,t\}$ forms a $\epsilon/2\beta$-packing of $Y$.
Noting the fact that the $\epsilon/2\beta$-packing number of $Y$ is less than $\left(\frac{2R\beta}{\epsilon}\right)^m$, we complete the proof.

\endproof

\proof{Proof of Theorem~\ref{Theorem_Iteration_Complexity_Random_Search}.}
Since we have Lemma~\ref{Lemma_Iteration_Comlexity_General}, we only need to ensure that with probability at least $1-\delta$, 
for all $t\in\{1,...,T\}$, 
    $$
    \frac{1}{1-\gamma}\sum_{s, a,s^\prime}z^{(t)}(s,a,s^\prime)c_{y^{(t)}}(s,a)- u_{y^{(t)}}\geq \max_{y\in Y}\left[\frac{1}{1-\gamma}\sum_{s, a,s^\prime}z^{(t)}(s,a,s^\prime)c_{y}(s,a)- u_{y}\right]-\epsilon.
    $$
We adopt the notations introduced in the proof of Lemma~\ref{Lemma_Iteration_Comlexity_General}.
At the $t$ th iteration, let $y^*:=\argmax_{y\in Y}\left[\frac{1}{1-\gamma}\sum_{s, a,s^\prime}z^{(t)}(s,a,s^\prime)c_{y}(s,a)- u_{y}\right]$.
As $f(y,z)$ is $\beta$-Lipschitz w.r.t. $y$, then it suffices to ensure that with probability at least $1-\delta/T$ there exist $i\in\{1,...,M\}$ such that $\|y_i-y^*\|_\infty\leq \epsilon/\beta$.
As long as $\epsilon/\beta\leq \epsilon_0$, we simply need
$$
\PB\left(\exists i\in\{1,...,M\}, \|y_i-y^*\|_\infty\leq \epsilon/\beta\right)=1-\left(1-\left(\frac{\epsilon}{\beta R}\right)^m\right)^M\geq 1-\frac{\delta}{T}.
$$
The proof can be completed by basic algebra operations.
\endproof

\proof{Proof of Theorem~\ref{Theorem_Iteration_Complexity_Projected_GD}.}
Since we have Lemma~\ref{Lemma_Iteration_Comlexity_General}, we only need to ensure that for all $t\in\{1,...,T\}$, 
    $$
    \frac{1}{1-\gamma}\sum_{s, a,s^\prime}z^{(t)}(s,a,s^\prime)c_{y^{(t)}}(s,a)- u_{y^{(t)}}\geq \max_{y\in Y}\left[\frac{1}{1-\gamma}\sum_{s, a,s^\prime}z^{(t)}(s,a,s^\prime)c_{y}(s,a)- u_{y}\right]-\epsilon.
    $$
We adopt the notations introduced in the proof of Lemma~\ref{Lemma_Iteration_Comlexity_General}.
By Theorem 3.2 in \cite{bubeck2015convex}, the statement above is satisfied as long as $T_{PGA}\geq \frac{\beta^2R^2}{\epsilon^2}$.
The proof can be completed by basic algebra operations.
\endproof

\begin{lemma}\label{Lemma_Crude_Bound}
If Assumption \ref{Assumption_Two_Nonzero} is true and $M\in M_\delta$, we have 
$$
\left\|Q_r^\pi-\widetilde Q_r^\pi\right\|_\infty\leq \frac{2\gamma}{(1-\gamma)^2}\sqrt{\frac{\log 2/\delta}{2n}}
$$
\end{lemma}
\proof{Proof of Lemma~\ref{Lemma_Crude_Bound}.}
Given a stationary policy $\pi$, if Assumption \ref{Assumption_Two_Nonzero} is true and $M\in M_\delta$, 
$$\left\|\widetilde P(\cdot|s,a) -P(\cdot|s,a)\right\|_1 \leq 2\sqrt{\frac{\log 2/\delta}{2n}},\forall s\in \gS,a\in \gA,
$$
which implies
$$
\left\|(P-\widetilde P)V_r^\pi\right\|_\infty\leq \frac{2}{1-\gamma}\sqrt{\frac{\log 2/\delta}{2n}}.
$$
Then we have
$$\begin{aligned}
    \left\|Q_r^\pi-\widetilde Q_r^\pi\right\|_\infty&= \left\|\gamma\left(I-\gamma \widetilde{P}^{\pi}\right)^{-1}(P-\widetilde{P}) V_r^{\pi}\right\|_\infty\\
    &\leq \frac{\gamma}{1-\gamma}\left\|(P-\widetilde P)V_r^\pi\right\|_\infty\\
    &\leq \frac{2\gamma}{(1-\gamma)^2}\sqrt{\frac{\log 2/\delta}{2n}}
\end{aligned}
$$
\endproof

\begin{lemma}\label{Lemma_Bound_Variance}
Given a stationary policy $\pi$, when Assumption \ref{Assumption_Two_Nonzero} is true and $M\in M_\delta$, we have
$$\Var_{P}(V_r^\pi)\leq 2\Var_{\widetilde P}(\widetilde{V}_r^\pi) + \frac{6}{(1-\gamma)^2}\sqrt{\frac{\log 2/\delta}{2n}}+\frac{8\gamma^2}{(1-\gamma)^4}\frac{\log 2/\delta}{2n}.
$$
\end{lemma}
\proof{Proof of Lemma~\ref{Lemma_Bound_Variance}.}
For simplicity of notation, we drop the dependence on $\pi$.
By definition,
$$\begin{aligned}
\Var_P(V_r) &= \Var_P(V_r)-\Var_{\widetilde P}(V_r)+\Var_{\widetilde P}(V_r)\\
&= P(V_r)^2-(PV_r)^2-\widetilde P(V_r)^2 + (\widetilde P V_r)^2+\Var_{\widetilde P}(V_r)\\
&=(P-\widetilde P)(V_r)^2-\left[(PV_r)^2-(\widetilde PV_r)^2\right]+\Var_{\widetilde P}(V_r),
\end{aligned}
$$
where $(\cdot)^2$ means element-wise squares.
When Assumption \ref{Assumption_Two_Nonzero} is true and $M\in M_\delta$, by Lemma~\ref{Lemma_Crude_Bound},
$$
\begin{aligned}
\|(P-\widetilde P)(V_r)^2\|_\infty&\leq \frac{2}{(1-\gamma)^2}\sqrt{\frac{\log 2/\delta}{2n}}\\
\left\|\left[(PV_r)^2-(\widetilde PV_r)^2\right]\right\|_\infty&\leq \|PV_r +\widetilde PV_r\|_\infty\|PV_r -\widetilde PV_r\|_\infty\\
&\leq \frac{2}{1-\gamma}\left\|PV_r -\widetilde PV_r\right\|_\infty\\
&\leq \frac{4}{(1-\gamma)^2}\sqrt{\frac{\log 2/\delta}{2n}}.
\end{aligned}
$$
We also have
$$
\begin{aligned}
\Var_{\widetilde P}(V_r)&=\Var_{\widetilde P}(V_r-\widetilde{V_r}+\widetilde V_r)\\
&\leq 2\Var_{\widetilde P}(V_r-\widetilde{V_r}) + 2\Var_{\widetilde P} (\widetilde V_r)\quad \text{(AM–GM inequality)}\\
&\leq 2\left\|V_r-\widetilde{V_r}\right\|_\infty^2+2\Var_{\widetilde P} (\widetilde V_r)\\
&\leq \frac{8\gamma^2}{(1-\gamma)^4}\frac{\log 2/\delta}{2n}+2\Var_{\widetilde P} (\widetilde V_r)\quad \text{(Lemma \ref{Lemma_Crude_Bound})}.
\end{aligned}
$$
Therefore, we can get
$$\Var_{P}(V_r^\pi)\leq 2\Var_{\widetilde P}(\widetilde{V_r^\pi}) + \frac{6}{(1-\gamma)^2}\sqrt{\frac{\log 2/\delta}{2n}}+\frac{8\gamma^2}{(1-\gamma)^4}\frac{\log 2/\delta}{2n}.
$$
\endproof

\begin{lemma}\label{Lemma_Distance_between_P_tilde_P}
Let $p,\tilde p,\hat p\in[0,1]$ satisfy
$$
\begin{aligned}
|p-\hat p|&\leq \min\brc{\sqrt{\frac{2 \hat p(1-\hat p) \log 4/\delta}{n}}+\frac{4\log 4/\delta}{n},\sqrt{\frac{ \log 2/\delta}{2 n}}}\\
|\tilde p-\hat p|&\leq \min\brc{\sqrt{\frac{2 \hat p(1-\hat p) \log 4/\delta}{n}}+\frac{4\log 4/\delta}{n},\sqrt{\frac{ \log 2/\delta}{2 n}}}.
\end{aligned}
$$
Then
$$
|p-\tilde p|\leq \sqrt{\frac{8 p(1- p) \log 4/\delta}{n}}+4\paren{\frac{ \log 4/\delta}{n}}^{3/4}+\frac{8\log 4/\delta}{n}
$$
\end{lemma}

\proof{Proof of Lemma~\ref{Lemma_Distance_between_P_tilde_P}.}
Assume WLOG that $\hat p\geq p$.
Therefore,
$$\begin{aligned}
    |p-\hat p|&\leq \sqrt{\frac{2 p(1- p) \log 4/\delta}{n}}+\sqrt{\frac{2 (\hat p- p)(1- p) \log 4/\delta}{n}}+\frac{4\log 4/\delta}{n}\\
    &\leq \sqrt{\frac{2 p(1- p) \log 4/\delta}{n}}+\sqrt{\frac{2 \sqrt{\frac{ \log 2/\delta}{2 n}} \log 4/\delta}{n}}+\frac{4\log 4/\delta}{n}\\
    &\leq \sqrt{\frac{2 p(1-p) \log 4/\delta}{n}}+2^{1/4}\paren{\frac{ \log 4/\delta}{n}}^{3/4}+\frac{4\log 4/\delta}{n}.
\end{aligned}
$$
Similarly, we have
$$|\tilde p-\hat p|\leq \sqrt{\frac{2 p(1-p) \log 4/\delta}{n}}+2^{1/4}\paren{\frac{ \log 4/\delta}{n}}^{3/4}+\frac{4\log 4/\delta}{n}.
$$
Thus we may complete the proof using triangular inequality.
\endproof

\begin{lemma}\label{Lemma_Quasi_Bernstein}
Given a stationary policy $\pi$, suppose Assumption \ref{Assumption_Two_Nonzero} is true and $M\in M_\delta$, then 
$$
|(P-\widetilde P)V_r^\pi|\preceq \sqrt{\frac{8\Var_P(V_r^\pi)\log 4/\delta}{n}}+\frac{4}{1-\gamma}\paren{\frac{\log 4/\delta}{n}}^{3/4}+\frac{8\log 4/\delta}{n(1-\gamma)},
$$
where $\preceq$ means every element of LHS is less than or equal to the its counterpart in RHS.
\end{lemma}

\proof{Proof of Lemma~\ref{Lemma_Quasi_Bernstein}.}
Let $p=P(sa^+|s,a),\tilde p=\tilde P(sa^+|s,a)$. Applying Lemma \ref{Lemma_Distance_between_P_tilde_P} yields 
$$|p-\tilde p|\leq \sqrt{\frac{8 p(1-p) \log 4/\delta}{n}}+4\paren{\frac{ \log 4/\delta}{n}}^{3/4}+\frac{8\log 4/\delta}{n}.
$$
Assume WLOG that $V_r^\pi(sa^+)\geq V_r^\pi(sa^-) $.
Therefore we have
$$
\begin{aligned}
|(P(\cdot|s,a)-\tilde{P}(\cdot|s,a))^\top V_r^\pi|\leq &\sqrt{\frac{8 p(1- p) \log 4/\delta}{n}}(V_r^\pi(sa^+)-V_r^\pi(sa^-))+\frac{4}{1-\gamma}\paren{\frac{ \log 4/\delta}{n}}^{3/4}\\
&+\frac{8\log 4/\delta}{n(1-\gamma)}.
\end{aligned}
$$
Since
$$
\begin{aligned}
p(1- p)(V_r^\pi(sa^+)- V_r^\pi(sa^-))^2&=[ p V_r^\pi(sa^+)^2+(1- p) V_r^\pi(sa^-)^2]-[ p V_r^\pi(sa^+)+(1- p) V_r^\pi(sa^-)]^2\\
&=\Var_P(V_r^\pi)
\end{aligned}
$$
We may get
$$
|(P(\cdot|s,a)-\widetilde{P}(\cdot|s,a))^\top  V_r^\pi|\leq \sqrt{\frac{8\Var_P(V_r^\pi)(s,a)\log 4/\delta}{n}}+\frac{4}{1-\gamma}\paren{\frac{\log 4/\delta}{n}}^{3/4}+\frac{8\log 4/\delta}{n(1-\gamma)},
$$
which completes the proof.
\endproof

\begin{lemma}\label{Lemma_Bound_on_V_Same_Pi}
Given a stationary policy $\pi$, suppose Assumption \ref{Assumption_Two_Nonzero} is true and $M\in M_\delta$, then we have
$$\left\|V_r^\pi -\widetilde{V_r}^\pi\right\|_\infty\leq \frac{4}{(1-\gamma)^{3/2}}\sqrt{\frac{\log 4/\delta}{n}} +\frac{4\sqrt{6}}{(1-\gamma)^2}\left(\frac{\log 4/\delta}{n}\right)^{3/4}+ \frac{8}{(1-\gamma)^4}\left(\frac{\log 4/\delta}{n}\right)^{3/2}
$$
\end{lemma}
\proof{Proof of Lemma~\ref{Lemma_Bound_on_V_Same_Pi}.}
From Lemma \ref{Lemma_Simulation_Lemma}, Lemma \ref{Lemma_Norm_of_Inf_Horizon_Expectation}, Lemma \ref{Lemma_Bound_of_Weighted_Variance}, Lemma \ref{Lemma_Quasi_Bernstein} and the fact that $\left(I-\gamma \widetilde{P}^{\pi}\right)^{-1}$ has positive entries, we know
$$
\begin{aligned}
\left\|Q^\pi-\widetilde {Q}^\pi\right\|_\infty&=\gamma\left\|\left(I-\gamma \widetilde{P}^{\pi}\right)^{-1}(P-\widetilde{P}) V_r^{\pi}\right\|_\infty\\
&\leq \sqrt{\frac{8\log 4/\delta}{n}}\left\|\left(I-\gamma\widetilde{P}^{\pi}\right)^{-1}\sqrt{\Var_{P}(V_r^\pi)}\right\|_\infty+\frac{4}{(1-\gamma)^2}\left(\frac{\log 4/\delta}{n}\right)^{3/4}+\frac{8}{(1-\gamma)^2}\left(\frac{\log 4/\delta}{n}\right)\\
&\leq \sqrt{\frac{16\log 4/\delta}{n}}\left\|\left(I-\gamma\widetilde{P}^{\pi}\right)^{-1}\sqrt{\Var_{\widetilde P}(\widetilde{V_r}^\pi)}\right\|_\infty +\frac{4\sqrt{6}}{(1-\gamma)^2}\left(\frac{\log 4/\delta}{n}\right)^{3/4}+ \frac{8}{(1-\gamma)^3}\left(\frac{\log 4/\delta}{n}\right)\\
&\leq \frac{4}{(1-\gamma)^{3/2}}\sqrt{\frac{\log 4/\delta}{n}} +\frac{4\sqrt{6}}{(1-\gamma)^2}\left(\frac{\log 4/\delta}{n}\right)^{3/4}+ \frac{8}{(1-\gamma)^3}\left(\frac{\log 4/\delta}{n}\right).
\end{aligned}
$$
The proof is completed since $\left\|V_r^\pi -\widetilde{V_r}^\pi\right\|_{\infty}\leq\left\|Q^\pi -\widetilde{Q}^\pi\right\|_{\infty}$ by definitions.
\endproof

\begin{lemma}\label{Lemma_Bound_on_V_Same_Pi_Leading}
Suppose Assumption \ref{Assumption_Two_Nonzero} is true and $n>\frac{6\log 4/\delta}{(1-\gamma)^{5/2}}$, then with probability at least $1-2|\gS|^2|\gA|\delta$, we have
$$
\begin{aligned}
\left\|V_r^\pi -\widetilde{V_r}^\pi\right\|_\infty&\leq 12\sqrt{\frac{\log 4/\delta}{n(1-\gamma)^3}}\\
\left\|C^\pi -\widetilde{C}_y^\pi\right\|_\infty&\leq 12\sqrt{\frac{\log 4/\delta}{n(1-\gamma)^3}},\forall y\in Y\\
\end{aligned}
$$
\end{lemma}

\proof{Proof of Lemma~\ref{Lemma_Bound_on_V_Same_Pi_Leading}.}
When Assumption \ref{Assumption_Two_Nonzero} is true and $M\in M_\delta$, it follows from Lemma \ref{Lemma_Bound_on_V_Same_Pi} that
$$
\left\|V_r^\pi -\widetilde{V_r}^\pi\right\|_\infty\leq \frac{4}{(1-\gamma)^{3/2}}\sqrt{\frac{\log 4/\delta}{n}} +\frac{4\sqrt{6}}{(1-\gamma)^2}\left(\frac{\log 4/\delta}{n}\right)^{3/4}+ \frac{8}{(1-\gamma)^3}\left(\frac{\log 4/\delta}{n}\right).
$$
And by setting $n>\max\left\{\frac{36\log4/\delta}{(1-\gamma)^2},\frac{4\log4/\delta}{(1-\gamma)^3}\right\}$ we will get
$$\left\|V_r^\pi -\widetilde{V_r}^\pi\right\|_\infty\leq 12\sqrt{\frac{\log 4/\delta}{n(1-\gamma)^3}}.
$$
Similar arguments can be applied to bound $\left\|C_y^\pi-\widetilde{C}_y^\pi\right\|_\infty$. Since by Theorem \ref{Theorem_Feasible} we have 
$$\PB(M\in M_\delta)\geq 1-2|\gS|^2|\gA|\delta,
$$
the proof is completed.
\endproof


\proof{Proof of Theorem \ref{Lemma_Bound_on_V}.}
By Lemma \ref{Lemma_Bound_on_V_Same_Pi}, we know that with probability $1-2|\gS|^2|\gA|\delta$,
$$\begin{aligned}
\left\|V_r^{\tilde \pi}-\tilde V_r^{\tilde \pi}\right\|_\infty&\leq12\sqrt{\frac{\log 4/\delta}{n(1-\gamma)^3}}\\
\left\|V_r^{\pi^*}-\tilde V_r^{\pi^*}\right\|_\infty&\leq 12\sqrt{\frac{\log 4/\delta}{n(1-\gamma)^3}}.
\end{aligned}
$$
Thus
$$\begin{aligned}
|V_r^{\tilde \pi}(\mu)-\tilde V_r^{\tilde \pi}(\mu)|&\leq12\sqrt{\frac{\log 4/\delta}{n(1-\gamma)^3}}\\
|V_r^{\pi^*}(\mu)-\tilde V_r^{\pi^*}(\mu)|&\leq12\sqrt{\frac{\log 4/\delta}{n(1-\gamma)^3}}.
\end{aligned}
$$ 
Noting that $\tilde V_r^{\tilde\pi}(\mu)\geq\tilde V_r^{\pi^*}(\mu)$, we may get
$$
\begin{aligned}
V_r^{\pi^*}(\mu)-V_r^{\tilde \pi}(\mu)&\leq V_r^{\pi^*}(\mu)-\tilde V_r^{\pi^*}(\mu)+\tilde V_r^{\tilde \pi}(\mu)-V_r^{\tilde \pi}(\mu)\\
&\leq|V_r^{\pi^*}(\mu)-\tilde V_r^{\pi^*}(\mu)| + |\tilde V_r^{\tilde \pi}(\mu)-V_r^{\tilde \pi}(\mu)|\\
&\leq 24\sqrt{\frac{\log 4/\delta}{n(1-\gamma)^3}}.
\end{aligned}
$$

Similarly, when
$$|C_y^{\tilde \pi}(\mu)-\tilde C_y^{\tilde \pi}(\mu)|\leq12\sqrt{\frac{\log 4/\delta}{n(1-\gamma)^3}},\forall y\in Y,
$$
we may get 
$$C_y^{\tilde \pi}(\mu) - u_y \leq 12\sqrt{\frac{\log 4/\delta}{n(1-\gamma)^3}},\forall y\in Y.
$$
since $\tilde C_y^{\tilde \pi}(\mu)\leq u_y$.
\endproof
% \proof{Proof of Theorem \ref{Theorem_Sample_Complexity}.}
% Theorem \ref{Theorem_Sample_Complexity} is a direct corollary of Theorem \ref{Lemma_Bound_on_V}.
% \endproof

\proof{Proof of Theorem \ref{Theorem_Sample_Complexity_General}.}
The proof is nearly identical to the proof of Theorem 3 in \cite{LATTIMORE2014125}. The idea is to augment each state/action pair of the original MDP with $|\gS|-2$ states in the form of a binary tree as pictured in the diagram below. 

\begin{figure}[!htb]
    \centering
    \includegraphics[width=0.3\textwidth]{img/binary_tree.pdf}
    \label{Figure_Binary_Tree}
\end{figure}

The intention of the tree is to construct a SICMDP, $\bar M=\langle \bar \gS,\gA,Y,\bar P,\bar r,\bar c,u,\mu,\bar\gamma\rangle$ that with appropriate transition probabilities is functionally equivalent to $M$ while satisfying Assumption \ref{Assumption_Two_Nonzero}.
The rewards and costs in the added states are set to zero.
Since the tree has depth $d=O(\log_2|\gS|)$, it now takes $d$ time steps in the augmented SICMDP to change states once in the original SICMDP.
Therefore we must also rescale the discount factor $\bar \gamma$ by setting $\bar\gamma<\gamma^d$.
Now we have
$$
\begin{aligned}
|\bar \gS|&= O(|\gS|^2|\gA|)\\
\frac{1}{1-\bar \gamma}&=\frac{\log |\gS|}{1-\gamma}.
\end{aligned}
$$
Then we complete the proof by applying results in Theorem \ref{Theorem_Sample_Complexity}.
\endproof

\proof{Proof of Theorem \ref{Theorem_Sample_Complexity_General_Measure}.}
By Theorem \ref{Theorem_Chernoff_Inequality}, we have for any fixed $(s,a)\in \gS\times \gA$
$$
\begin{aligned}
\PB(n(s,a)<m\nu_{\min}/2)&\leq \PB(n(s,a)<m\nu(s,a)/2)\\
&\leq e^{-m\nu(s,a)}\left(\frac{em\nu(s,a)}{em\nu(s,a)/2}\right)^{\frac{m\nu(s,a)}{2}}\\
&=\left(\sqrt{\frac{e}{2}}\right)^{-\nu(s,a)m}\\
&\leq \left(\sqrt{\frac{e}{2}}\right)^{-\nu_{\min}m}
\end{aligned}
$$
\endproof
Let $m=\frac{2}{1-\log 2}\frac{\log 2|\gS||\gA|/\delta}{\nu_{\min}}$, we have $\PB(n(s,a)\geq m\nu_{\min}))\geq 1-\delta/2|\gS||\gA|$.
Therefore, with probability at least $1-\delta/2$, we can get
$$
n(s,a)>m\nu_{\min}, \forall (s,a)\in \gS\times \gA.
$$
Then our problem is reduced to the case that the offline dataset is generated by generative models.
The proof is completed by using results in Theorem \ref{Theorem_Sample_Complexity_General}.

\section{Omitted Proofs in Section \ref{Section_Theory_SICPO}}
\label{Appendix_Proofs_SICPO}

\begin{lemma}\label{Lemma_basic_expansion}
We have
$$
\begin{aligned}
&(1-\gamma)\alpha\sum_{t\in \gB} (V_r^*(\mu)-V_r^{(t)}(\mu))+(1-\gamma)\alpha\sum_{t\in\gN}\left(\eta-\left|\widehat V_{c^{(t)}}^{(t)}(\mu)-V_{c^{(t)}}^{(t)}(\mu)\right|\right)\\
&\leq \log|\gA|+\alpha\sum_{t\in\gB}\sqrt{E^{\nu^*}\left(r, \theta^{(t)},\hat w^{(t)}\right)}+\alpha\sum_{t\in\gN}\sqrt{E^{\nu^*}\left(c^{(t)}, \theta^{(t)},\hat w^{(t)}\right)}+\frac{\beta\alpha^2}{2}\sum_{t=0}^{T-1}\|\hat w^{(t)}\|_2^2.
\end{aligned}
$$
\end{lemma}
\proof{Proof of Lemma~\ref{Lemma_basic_expansion}}
Note that using Assumption~\ref{Assumption_smooth} and Taylor's expansion we have
$$
\log \frac{\pi_t(a | s)}{\pi_{t+1}(a | s)}+\nabla_{\theta} \log \pi_{t}(a | s)^\top\left(\theta_{t+1}-\theta_t\right) \leq \frac{\beta}{2}\left\|\theta_{t+1}-\theta_t\right\|^{2}.
$$
and $\theta^{(t+1)}-\theta^{(t)}=\alpha\hat w^{(t)}$.
As a result, suppose $t\in\gB$,
$$
\begin{aligned}
&\EB_{s\sim d^{\pi^*}}(D_{\operatorname{KL}}(\pi^*(\cdot|s)\|\pi^{(t)}(\cdot|s))-D_{\operatorname{KL}}(\pi^*(\cdot|s)\|\pi^{(t+1)}(\cdot|s)))\\
&=-\EB_{(s,a)\sim\nu^*}\log\frac{\pi^{(t)}(a|s)}{\pi^{(t+1)}(a|s)}\\
&\geq\alpha\EB_{(s,a)\sim\nu^*}[\nabla_{\theta} \log \pi^{(t)}(a | s)^\top \hat w^{(t)}]-\frac{\beta\eta_\theta^2}{2}\|\hat w^{(t)}\|^2_2\\
&=\alpha\EB_{(s,a)\sim\nu^*} A_r^{(t)}(s,a)+\alpha \EB_{(s,a)\sim\nu^*}\left[\nabla_\theta\log \pi^{(t)}(a|s)^\top \hat w^{(t)}-A_r^{(t)}(s,a)\right]-\frac{\beta\alpha^2}{2}\|\hat w^{(t)}\|^2_2\\
&\geq (1-\gamma)\alpha(V_r^*(\mu)-V_r^{(t)}(\mu))-\alpha\sqrt{E^{\nu^*}\left(r, \theta^{(t)},\hat w^{(t)}\right)}
-\frac{\beta\alpha^2}{2}\|\hat w^{(t)}\|^2_2\\
\end{aligned}
$$

The second last inequality is true due to the performance difference lemma, the last inequality is true due to Jensen's inequality and definition of the transferred function approximation error.
Rearranging terms yields
$$
\begin{aligned}
&(1-\gamma)\alpha(V_r^*(\mu)-V_r^{(t)}(\mu))\\
&\leq \EB_{s\sim d^{\pi^*}}(D_{\operatorname{KL}}(\pi^*(\cdot|s)\|\pi^{(t)}(\cdot|s))-D_{\operatorname{KL}}(\pi^*(\cdot|s)\|\pi^{(t+1)}(\cdot|s)))+\alpha\sqrt{E^{\nu^*}\left(r, \theta^{(t)},\hat w^{(t)}\right)}
+\frac{\beta\alpha^2}{2}\|\hat w^{(t)}\|^2_2.
\end{aligned}
$$
Similarly, suppose $t\in\gN$, we have $\theta^{(t+1)}-\theta^{(t)}=-\alpha\hat w^{(t)}$
$$
\begin{aligned}
&(1-\gamma)\alpha(V_{c^{(t)}}^{(t)}(\mu)-V_{c^{(t)}}^*(\mu))\\
&\leq \EB_{s\sim d^{\pi^*}}(D_{\operatorname{KL}}(\pi^*(\cdot|s)\|\pi^{(t)}(\cdot|s))-D_{\operatorname{KL}}(\pi^*(\cdot|s)\|\pi^{(t+1)}(\cdot|s)))+\alpha\sqrt{E^{\nu^*}\left(c^{(t)}, \theta^{(t)},\hat w^{(t)}\right)}
+\frac{\beta\alpha^2}{2}\|\hat w^{(t)}\|^2_2.
\end{aligned}
$$
Then we may get
$$
\begin{aligned}
&(1-\gamma)\alpha\sum_{t\in \gB} (V_r^*(\mu)-V_r^{(t)}(\mu))+(1-\gamma)\alpha\sum_{t\in \gN}\left(V_{c^{(t)}}^{(t)}(\mu)-V_{c^{(t)}}^*(\mu)\right)\\
&\leq \sum_{t=0}^{T-1} \EB_{s\sim d^{\pi^*}}(D_{\operatorname{KL}}(\pi^*(\cdot|s)\|\pi^{(t)}(\cdot|s))-D_{\operatorname{KL}}(\pi^*(\cdot|s)\|\pi^{(t+1)}(\cdot|s)))\\
&\quad+\alpha\sum_{t\in\gB}\sqrt{E^{\nu^*}\left(r, \theta^{(t)},\hat w^{(t)}\right)}+\alpha\sum_{t\in\gN}\sqrt{E^{\nu^*}\left(c^{(t)}, \theta^{(t)},\hat w^{(t)}\right)}+\frac{\beta\alpha^2}{2}\sum_{t=0}^{T-1}\|\hat w^{(t)}\|_2^2\\
&\leq \log|\gA|+\alpha\sum_{t\in\gB}\sqrt{E^{\nu^*}\left(r, \theta^{(t)},\hat w^{(t)}\right)}+\alpha\sum_{t\in\gN}\sqrt{E^{\nu^*}\left(c^{(t)}, \theta^{(t)},\hat w^{(t)}\right)}+\frac{\beta\alpha^2}{2}\sum_{t=0}^{T-1}\|\hat w^{(t)}\|_2^2.
\end{aligned}
$$
Since for $t\in\gN$
$$
\begin{aligned}
V_{c^{(t)}}^{(t)}(\mu)-V_{c^{(t)}}^*(\mu)&\geq \widehat V_{c^{(t)}}^{(t)}(\mu)-V_{c^{(t)}}^*(\mu)-\left|\widehat V_{c^{(t)}}^{(t)}(\mu)-V_{c^{(t)}}^{(t)}(\mu)\right|\\
&\geq u_{y_*^{(t)}}+\eta-u_{y_*^{(t)}}-\left|\widehat V_{c^{(t)}}^{(t)}(s)-V_{c^{(t)}}^{(t)}(\mu)\right|\\
&=\eta-\left|\widehat V_{c^{(t)}}^{(t)}(s)-V_{c^{(t)}}^{(t)}(\mu)\right|
\end{aligned}
$$
we may obtain the conclusion.
\endproof

\begin{lemma}\label{Lemma_SGD_convergence}
We have for $t\in\gB$, $\forall\delta\in(0,1)$, with probability at least $1-\delta$,
for $t\in\gB$
$$
\begin{aligned}
    & E^{\nu^*}(r,\theta^{(t)}, \hat w^{(t)})\\
    &\leq \frac{1}{1-\gamma}\left\|\frac{\nu^*}{\nu_0}\right\|_\infty\left(\epsilon_{bias}+C\frac{(4L_\pi^2W+8L_\pi/(1-\gamma))^2}{(1-\gamma)^2\mu_F}\frac{\log(T/\delta)}{K_{sgd}}+\frac{4\gamma^H}{1-\gamma}\left(\frac{1}{1-\gamma}+WL_\pi\right)\right),
\end{aligned}
$$
for $t\in\gN$,
$$
\begin{aligned}
     &E^{\nu^*}(c^{(t)},\theta^{(t)}, \hat w^{(t)})\\
     &\leq \frac{1}{1-\gamma}\left\|\frac{\nu^*}{\nu_0}\right\|_\infty\left(\epsilon_{bias}+C\frac{(4L_\pi^2W+8L_\pi/(1-\gamma))^2}{(1-\gamma)^2\mu_F}\frac{\log(T/\delta)}{K_{sgd}}+\frac{4\gamma^H}{1-\gamma}\left(\frac{1}{1-\gamma}+WL_\pi\right)\right),
\end{aligned}
$$
where $\nu_0$ is the uniform distribution on $\gS\times\gA$.
\end{lemma}

\proof{Proof of Lemma~\ref{Lemma_SGD_convergence}}
First we show for $t\in\gB$, with high probability
$$
\begin{aligned}
    &E^{\nu^*}(r,\theta^{(t)}, \hat w^{(t)})\\
    &\leq \frac{1}{1-\gamma}\left\|\frac{\nu^*}{\nu_0}\right\|_\infty\left(\epsilon_{bias}+C\frac{(4L_\pi^2W+8L_\pi/(1-\gamma))^2}{(1-\gamma)^2\mu_F}\frac{\log(T/\delta)}{K_{sgd}}+\frac{4\gamma^H}{1-\gamma}\left(\frac{1}{1-\gamma}+WL_\pi\right)\right).
\end{aligned}
$$
We have:
$$
\begin{aligned}
&E^{\nu^*}(r,\theta^{(t)}, \hat w^{(t)})\\
&\leq \left\|\frac{\nu^{*}}{\nu^{(t)}}\right\|_{\infty} E^{\nu^{(t)}}(r,\theta^{(t)}, \hat w^{(t)})\\
&\leq \frac{1}{1-\gamma} \left\|\frac{\nu^{*}}{\nu_{0}}\right\|_{\infty} E^{\nu^{(t)}}(r,\theta^{(t)}, \hat w^{(t)})\\
&=\frac{1}{1-\gamma}\left\|\frac{\nu^{*}}{\nu_{0}}\right\|_{\infty}\left( \min_w E^{\nu^{(t)}}(r,\theta^{(t)},w)+E^{\nu^{(t)}}\left(r,\theta^{(t)}, \hat w^{(t)}\right)-\min_w E^{\nu^{(t)}}(r,\theta^{(t)},w)\right)\\
&\leq \frac{1}{1-\gamma}\left\|\frac{\nu^{*}}{\nu_{0}}\right\|_{\infty}\left(\epsilon_{bias}+E^{\nu^{(t)}}\left(r, \theta^{(t)}, \hat w^{(t)}\right)-\min_{w\in B(0,W,\|\cdot\|_2)} E^{\nu^{(t)}}(r,\theta^{(t)},w)\right),
\end{aligned}
$$
the last step is due to Assumption~\ref{Assumption_func_approx_err} and Assumption~\ref{Assumption_est_err}.
Now we define a proximal loss function:
$$
\widetilde{E}^{\nu^{(t)}}(r,\theta^{(t)},w):=\EB_{(s,a)\sim\nu}(\widetilde{A}^{\pi^{(t)}}_r(s,a)-w^\top\nabla_\theta\log\pi_\theta(a|s))^2,
$$
where
$$
\begin{aligned}
    \widetilde{A}^{(t)}_r(s,a):&=\widetilde{Q}^{\pi^{(t)}}_r(s,a)-\widetilde{V}^{\pi^{(t)}}_r(s),\\
    \widetilde{Q}^{\pi^{(t)}}_r(s,a):&=\EB\left(\sum_{t=0}^H\gamma^t r(s_t,a_t)\mid (s_0,a_0)=(s,a)\right),\\
    \widetilde{V}^{\pi^{(t)}}_r(s):&=\EB\left(\sum_{t=0}^H\gamma^t r(s_t,a_t)\mid s_0=s\right).\\
\end{aligned}
$$
Recall that
$$
\begin{aligned}
&\widetilde{E}^{\nu^{(t)}}(r,\theta^{(t)},w)\\
&=\EB_{(s,a)\sim\nu^{(t)}}(\widetilde{A}^{(t)}(s,a)-w^\top\nabla_\theta\log\pi_{\theta^{(t)}}(a|s))^2\\    &=w^\top F(\theta^{(t)})w-2\sum_{s,a}\nu^{(t)}(s,a)w^\top \nabla_\theta\log\pi_{\theta^{(t)}}(a|s)\widetilde{A}^{(t)}(s,a)+\sum_{s,a}\nu^{(t)}(s,a)\widetilde{A}^{(t)}(s,a)^2.
\end{aligned}
$$
According to Assumption~\ref{Assumption_PSD_Fisher}, $E^{\nu^{(t)}}(r,\theta^{(t)},w)$ is $(1-\gamma)^2\mu_F$-strongly convex.
The full gradient is
$$G(w)=2F(\theta^{(t)})w-2\sum_{s,a}\nu^{(t)}(s,a)\widetilde{A}^{(t)}(s,a)\nabla_\theta\log\pi_{\theta^{(t)}}(a|s),
$$
and the stochastic gradient we use is 
$$\widehat G(w)=2\nabla_\theta\log\pi_\theta(A|S)\nabla_\theta\log\pi_\theta(A|S)^\top w-2\widehat A^{\pi_\theta}(S,A)\nabla_\theta\log\pi_\theta(A|S),
$$
where $(S,A)\sim \nu^{(t)}$.
Then we have $\EB\widehat G(w)=G(w)$ and
$$
\|G(w)-\widehat G(w)\|_2\leq 2L_\pi^2W+\frac{4L_\pi}{1-\gamma}.
$$
Set $\eta_k=\frac{2}{(1-\gamma)^2\mu_F(k+1)}$, $\gamma_k=\frac{2k}{K(K+1)}$, by Theorem C.3 in \cite{Harvey2019Simple}, for any $\delta\in(0,1)$, with probability at least $1-\delta$ we have,
$$
\widetilde{E}^{\nu^{(t)}}\left(r, \theta^{(t)}, \hat w^{(t)}\right)-\min_{w\in B(0,W,\|\cdot\|_2)} \widetilde{E}^{\nu^{(t)}}(r,\theta^{(t)},w)\leq C\frac{(4L_\pi^2W+8L_\pi/(1-\gamma))^2}{(1-\gamma)^2\mu_F}\frac{\log(1/\delta)}{K_{sgd}}.
$$
Here $C$ is a universal constant.
We may finish the bound by noting that
$$
\sup_{w\in B(0,W,\|\cdot\|_2)}\left|\widetilde{E}^{\nu^{(t)}}\left(r, \theta^{(t)}, \hat w^{(t)}\right)-E^{\nu^{(t)}}\left(r, \theta^{(t)}, \hat w^{(t)}\right)\right|\leq \frac{2\gamma^H}{1-\gamma}\left(\frac{1}{1-\gamma}+WL_\pi\right).
$$
For $t\in\gN$, the inequality holds by similar arguments as above.
And we would like to emphasize that $c^{(t)}$ is determined a priori and thus can be viewed as a fixed cost.
Our final conclusion can be obtained by a union bound argument.
% Set $\delta=\frac{1}{4L_\pi^2}$, by Theorem 1 in \cite{bach2013non} we may get:
% $$
% \EB\left[E^{\nu^{(t)}}\left(r,\theta^{(t)}, \hat w^{(t)}\right)-\min_w E^{\nu^{(t)}}(r,\theta^{(t)},w)\right]\leq \frac{2(\sigma\sqrt{d}+L_\pi W)^2}{K}.
% $$
% $\sigma$ is defined as:
% $$
% \EB_{(s,a)\sim\nu^{(t)}}\left[G_r^{(t)}\left(G_r^{(t)}\right)^\top\right]\preceq \sigma^2 F(\theta^{(t)}),
% $$
% where $G_{r}^{(t)}:=2\left(\left(w_{r}^{*}\right)^\top \nabla_{\theta} \log \pi^{(t)}(a \mid s)-\widehat{A}_{r}^{(t)}(s, a)\right) \nabla_{\theta} \log \pi^{(t)}(a \mid s)$ and $w_{r}^*:=\underset{w}{\mathrm{argmin}} E^{\nu^{(t)}}(r,\theta^{(t)},w).$
% We have $\sigma\leq 2WL_\pi +\frac{2}{1-\gamma}$.
% For $t\in\gN$, we may note that for any fixed $y\in Y$, $\theta$, and corresponding NPG update $\hat w$,
% $$
% \EB_{\hat w}E^{\nu^*}(c_y,\theta, \hat w)
% \leq \frac{1}{1-\gamma}\left\|\frac{\nu^*}{\nu_0}\right\|_\infty\left(\epsilon_{bias}+\frac{2\left(2\sqrt{d}WL_\pi+\frac{2\sqrt{d}}{1-\gamma}+WL_\pi\right)^2}{K}\right).
% $$ 
% The inquality holds by similar arguments as above.
% And the expectation is taken w.r.t. the randomness of $\hat w$.
% Therefore, we may get
% $$
% \EB E^{\nu^*}(c^{(t)},\theta^{(t)}, \hat w^{(t)})
% \leq \frac{1}{1-\gamma}\left\|\frac{\nu^*}{\nu_0}\right\|_\infty\left(\epsilon_{bias}+\frac{2\left(2\sqrt{d}WL_\pi+\frac{2\sqrt{d}}{1-\gamma}+WL_\pi\right)^2}{K}\right).
% $$ 
\endproof

\begin{lemma}\label{Lemma_whp_basic_expansion}
For any $\delta\in(0,1)$, let $A$ denote the event
$$
\begin{aligned}
&(1-\gamma)\alpha\sum_{t\in \gB} (V_r^*(\mu)-V_r^{(t)}(\mu))\\
&+(1-\gamma)\alpha|\gN|\left(\eta-\frac{1}{(1-\gamma)\sqrt{2K_{eval}}}\left(1+\sqrt{\log{(4T/\delta)}+m\log(4\mathrm{diam}(Y)L_y\sqrt{2K_{eval}}+\mathrm{diam}(Y))}\right)-\frac{\gamma^H}{1-\gamma}\right)\\
&\leq \log|\gA|+\frac{\beta\alpha^2 TW^2}{2(1-\gamma)^2}\\
&+\alpha T\sqrt{\frac{1}{1-\gamma}\left\|\frac{\nu^*}{\nu_0}\right\|_\infty\left(\epsilon_{bias}+C\frac{(4L_\pi^2W+8L_\pi/(1-\gamma))^2}{(1-\gamma)^2\mu_F}\frac{\log(T/\delta)}{K_{sgd}}+\frac{4\gamma^H}{1-\gamma}\left(\frac{1}{1-\gamma}+WL_\pi\right)\right)}.
\end{aligned}
$$
Then we have $\PB(A)\geq 1-\delta$.
\end{lemma}
\proof{Proof of Lemma~\ref{Lemma_whp_basic_expansion}}
Let $A_1$ denote the event that 
for $t\in\gB$
$$
\begin{aligned}
     &E^{\nu^*}(r,\theta^{(t)}, \hat w^{(t)})\\
     &\leq \frac{1}{1-\gamma}\left\|\frac{\nu^*}{\nu_0}\right\|_\infty\left(\epsilon_{bias}+C\frac{(4L_\pi^2W+8L_\pi/(1-\gamma))^2}{(1-\gamma)^2\mu_F}\frac{\log(2T/\delta)}{K_{sgd}}+\frac{4\gamma^H}{1-\gamma}\left(\frac{1}{1-\gamma}+WL_\pi\right)\right),
\end{aligned}
$$
for $t\in\gN$,
$$
\begin{aligned}
     &E^{\nu^*}(c^{(t)},\theta^{(t)}, \hat w^{(t)})\\
     &\leq \frac{1}{1-\gamma}\left\|\frac{\nu^*}{\nu_0}\right\|_\infty\left(\epsilon_{bias}+C\frac{(4L_\pi^2W+8L_\pi/(1-\gamma))^2}{(1-\gamma)^2\mu_F}\frac{\log(2T/\delta)}{K_{sgd}}+\frac{4\gamma^H}{1-\gamma}\left(\frac{1}{1-\gamma}+WL_\pi\right)\right).
\end{aligned}
$$
By Lemma~\ref{Lemma_SGD_convergence}, $\PB(A_1)\geq 1-\delta/2$.
Also note that we always have $\|\hat w^{(t)}\|_2^2\leq \frac{W^2}{(1-\gamma)^2}$, together with Lemma~\ref{Lemma_basic_expansion} lead to that conditioned on $A_1$,
$$
\begin{aligned}
&(1-\gamma)\alpha\sum_{t\in \gB} (V_r^*(\mu)-V_r^{(t)}(\mu))+(1-\gamma)\alpha\sum_{t\in\gN}\left(\eta-\left|\widehat V_{c^{(t)}}^{(t)}(\mu)-V_{c^{(t)}}^{(t)}(\mu)\right|\right)\\
&\leq \log|\gA|+\frac{\beta\alpha^2 TW^2}{2(1-\gamma)^2}\\
&\alpha T\sqrt{\frac{1}{1-\gamma}\left\|\frac{\nu^*}{\nu_0}\right\|_\infty\left(\epsilon_{bias}+C\frac{(4L_\pi^2W+8L_\pi/(1-\gamma))^2}{(1-\gamma)^2\mu_F}\frac{\log(2T/\delta)}{K_{sgd}}+\frac{4\gamma^H}{1-\gamma}\left(\frac{1}{1-\gamma}+WL_\pi\right)\right)}.
\end{aligned}
$$
It remains to bound $\left|\widehat V_{c^{(t)}}^{(t)}(\mu)-V_{c^{(t)}}^{(t)}(\mu)\right|$.
Let $A_2$ denote the event that $\forall t\in\{0,...,T-1\}$,
 $$
    \left|\widehat V_{c^{(t)}}^{(t)}(\mu)-V_{c^{(t)}}^{(t)}(\mu)\right|\leq \frac{1}{(1-\gamma)\sqrt{2K_{eval}}}\left(1+\sqrt{\log{(4T/\delta)}+m\log(4\mathrm{diam}(Y)L_y\sqrt{2K_{eval}}+\mathrm{diam}(Y))}+\frac{\gamma^H}{1-\gamma}\right)
    $$
By Lemma~\ref{Lemma_WHP_Bound_Evaluation_Uniform_y}, $\PB(A_2)\geq \delta/2$.
Thus conditioned on the event $A_1\cap A_2$,
$$
\begin{aligned}
&(1-\gamma)\alpha\sum_{t\in \gB} (V_r^*(\mu)-V_r^{(t)}(\mu))\\
&+(1-\gamma)\alpha|\gN|\left(\eta-\frac{1}{(1-\gamma)\sqrt{2K_{eval}}}\left(1+\sqrt{\log{(4T/\delta)}+m\log(4\mathrm{diam}(Y)L_y\sqrt{2K_{eval}}+\mathrm{diam}(Y))}\right)-\frac{\gamma^H}{1-\gamma}\right)\\
&\leq \log|\gA|+\frac{\beta\alpha^2 TW^2}{2(1-\gamma)^2}\\
&+\alpha T\sqrt{\frac{1}{1-\gamma}\left\|\frac{\nu^*}{\nu_0}\right\|_\infty\left(\epsilon_{bias}+C\frac{(4L_\pi^2W+8L_\pi/(1-\gamma))^2}{(1-\gamma)^2\mu_F}\frac{\log(2T/\delta)}{K_{sgd}}+\frac{4\gamma^H}{1-\gamma}\left(\frac{1}{1-\gamma}+WL_\pi\right)\right)}.
\end{aligned}
$$
We complete the proof by noting $\PB(A_1\cap A_2)\geq 1-\delta$.
\endproof

\begin{lemma}\label{Lemma_well_define_B}
    Suppose 
    $$\begin{aligned}
        &\frac{T}{2}\left(\eta-\frac{1}{(1-\gamma)\sqrt{2K_{eval}}}\left(1+\sqrt{\log{(4T/\delta)}+m\log(4\mathrm{diam}(Y)L_y\sqrt{2K_{eval}}+\mathrm{diam}(Y))}\right)-\frac{\gamma^H}{1-\gamma}\right)\\
        &>\frac{\log|\gA|}{(1-\gamma)\alpha}+\frac{\beta\alpha TW^2}{2(1-\gamma)^3}\\
        &+\frac{ T}{1-\gamma}\sqrt{\frac{1}{1-\gamma}\left\|\frac{\nu^*}{\nu_0}\right\|_\infty\left(\epsilon_{bias}+C\frac{(4L_\pi^2W+8L_\pi/(1-\gamma))^2}{(1-\gamma)^2\mu_F}\frac{\log(2T/\delta)}{K_{sgd}}+\frac{4\gamma^H}{1-\gamma}\left(\frac{1}{1-\gamma}+WL_\pi\right)\right)},
    \end{aligned}
    $$
Then $\gB\not=\emptyset$. Then for any $\delta\in(0,1)$, conditioned on event $A$, either a). $|\gB|\geq T/2$ or b). $\sum_{t\in\gB}(V_r^*(\mu)-V_r^{(t)}(\mu))< 0$.
\end{lemma}
\proof{Proof of Lemma~\ref{Lemma_well_define_B}}
    First we set $\sum_{t\in\gB}(V_r^*(\mu)-V_r^{(t)}(\mu))= 0$ if $\gB=\emptyset$. Now we would show $|\gB|\geq T/2$ as long as $\sum_{t\in\gB}(V_r^*(\mu)-V_r^{(t)}(\mu))\geq 0$, which also precludes the possibility that $\gB=\emptyset$.
    Note that when $\sum_{t\in\gB}(V_r^*(\mu)-V_r^{(t)}(\mu))\geq 0$ and $A$ happens,
    $$
    \begin{aligned}
    &|\gN|\left(\eta-\frac{1}{(1-\gamma)\sqrt{2K_{eval}}}\left(1+\sqrt{\log{(4T/\delta)}+m\log(4\mathrm{diam}(Y)L_y\sqrt{2K_{eval}}+\mathrm{diam}(Y))}\right)-\frac{\gamma^H}{1-\gamma}\right)\\
    &\leq\frac{\log|\gA|}{(1-\gamma)\alpha}+\frac{\beta\alpha TW^2}{2(1-\gamma)^3}\\
    &+\frac{ T}{1-\gamma}\sqrt{\frac{1}{1-\gamma}\left\|\frac{\nu^*}{\nu_0}\right\|_\infty\left(\epsilon_{bias}+C\frac{(4L_\pi^2W+8L_\pi/(1-\gamma))^2}{(1-\gamma)^2\mu_F}\frac{\log(2T/\delta)}{K_{sgd}}+\frac{4\gamma^H}{1-\gamma}\left(\frac{1}{1-\gamma}+WL_\pi\right)\right)}.
    \end{aligned}
    $$
    Since
    $$
    \begin{aligned}
    &\frac{T}{2}\left(\eta-\frac{1}{(1-\gamma)\sqrt{2K_{eval}}}\left(1+\sqrt{\log{(4T/\delta)}+m\log(4\mathrm{diam}(Y)L_y\sqrt{2K_{eval}}+\mathrm{diam}(Y))}\right)-\frac{\gamma^H}{1-\gamma}\right)\\
    &>\frac{\log|\gA|}{(1-\gamma)\alpha}+\frac{\beta\alpha TW^2}{2(1-\gamma)^3}\\
    &+\frac{ T}{1-\gamma}\sqrt{\frac{1}{1-\gamma}\left\|\frac{\nu^*}{\nu_0}\right\|_\infty\left(\epsilon_{bias}+C\frac{(4L_\pi^2W+8L_\pi/(1-\gamma))^2}{(1-\gamma)^2\mu_F}\frac{\log(2T/\delta)}{K_{sgd}}+\frac{4\gamma^H}{1-\gamma}\left(\frac{1}{1-\gamma}+WL_\pi\right)\right)}.
    \end{aligned}
    $$ we can conclude $|\gN|\leq T/2$ and $|\gB|\geq T/2$.
\endproof

\begin{lemma}\label{Lemma_reward_bound}
    Let $\alpha=1/\sqrt{T}$. Suppose we choose
    $$K_{eval}=\widetilde{O}\left(\frac{1}{\eta^2(1-\gamma)^2}\right)
    ,~H=O\left(\log\left(\frac{\eta(1-\gamma)}{\gamma}\right)\right)$$ and
    $$\begin{aligned}
    \eta&> \frac{4\log |\gA|}{\sqrt{T}(1-\gamma)}+\frac{2\beta W^2}{\sqrt{T}(1-\gamma)^3}\\
    &+\frac{4}{1-\gamma}\sqrt{\frac{1}{1-\gamma}\left\|\frac{\nu^*}{\nu_0}\right\|_\infty\left(\epsilon_{bias}+C\frac{(4L_\pi^2W+8L_\pi/(1-\gamma))^2}{(1-\gamma)^2\mu_F}\frac{\log(2T/\delta)}{K_{sgd}}+\frac{4\gamma^H}{1-\gamma}\left(\frac{1}{1-\gamma}+WL_\pi\right)\right)}
    \end{aligned}
    $$,
    then we have for any $\delta\in(0,1)$, conditioned on event $A$,
    $$
    \begin{aligned}
        &\frac{1}{|\gB|}\sum_{t\in\gB}(V_r^*(\mu)-V_r^{(t)}(\mu)) \leq \frac{2\log|\gA|}{\sqrt{T}(1-\gamma)}+\frac{\beta W^2}{\sqrt{T}(1-\gamma)^3}\\
       &+\frac{2}{1-\gamma}\sqrt{\frac{1}{1-\gamma}\left\|\frac{\nu^*}{\nu_0}\right\|_\infty\left(\epsilon_{bias}+C\frac{(4L_\pi^2W+8L_\pi/(1-\gamma))^2}{(1-\gamma)^2\mu_F}\frac{\log(2T/\delta)}{K_{sgd}}+\frac{4\gamma^H}{1-\gamma}\left(\frac{1}{1-\gamma}+WL_\pi\right)\right)}.
    \end{aligned}
    $$
    Here $\widetilde{O}$ means we discard any logarithmic terms. 
\end{lemma}
\proof{Proof of Lemma~\ref{Lemma_reward_bound}}
    When we choose $K_{eval}=\widetilde{O}\left(\frac{m}{\eta^2(1-\gamma)^2}\right)$ and $H=O\left(\log\left(\frac{\eta(1-\gamma)}{\gamma}\right)\right)$, we have
    $$
    \eta-\frac{1}{(1-\gamma)\sqrt{2K_{eval}}}\left(1+\sqrt{\log{(4T/\delta)}+m\log(4\mathrm{diam}(Y)L_y\sqrt{2K_{eval}}+\mathrm{diam}(Y))}\right)-\frac{\gamma^H}{1-\gamma}
    >\frac{\eta}{2}$$
    Setting
    $$\begin{aligned}
    \eta&> \frac{4\log |\gA|}{\sqrt{T}(1-\gamma)}+\frac{2\beta W^2}{\sqrt{T}(1-\gamma)^3}\\
    &+\frac{4}{1-\gamma}\sqrt{\frac{1}{1-\gamma}\left\|\frac{\nu^*}{\nu_0}\right\|_\infty\left(\epsilon_{bias}+C\frac{(4L_\pi^2W+8L_\pi/(1-\gamma))^2}{(1-\gamma)^2\mu_F}\frac{\log(2T/\delta)}{K_{sgd}}+\frac{4\gamma^H}{1-\gamma}\left(\frac{1}{1-\gamma}+WL_\pi\right)\right)}
    \end{aligned}
    $$
    results in
   $$\begin{aligned}
        &\frac{T}{2}\left(\eta-\frac{1}{(1-\gamma)\sqrt{2K_{eval}}}\left(1+\sqrt{\log{(4T/\delta)}+m\log(4\mathrm{diam}(Y)L_y\sqrt{2K_{eval}}+\mathrm{diam}(Y))}\right)-\frac{\gamma^H}{1-\gamma}\right)\\
        &>\frac{\log|\gA|}{(1-\gamma)\alpha}+\frac{\beta\alpha TW^2}{2(1-\gamma)^3}\\
        &\frac{ T}{1-\gamma}\sqrt{\frac{1}{1-\gamma}\left\|\frac{\nu^*}{\nu_0}\right\|_\infty\left(\epsilon_{bias}+C\frac{(4L_\pi^2W+8L_\pi/(1-\gamma))^2}{(1-\gamma)^2\mu_F}\frac{\log(2T/\delta)}{K_{sgd}}+\frac{4\gamma^H}{1-\gamma}\left(\frac{1}{1-\gamma}+WL_\pi\right)\right)},
    \end{aligned}
    $$
    Then we may use Lemma~\ref{Lemma_well_define_B} to conclude either a). $|\gB|\geq T/2$ or b). $\sum_{t\in\gB}(V_r^*(\mu)-V_r^{(t)}(\mu))< 0$.
    If $a)$, the conclusion holds trivially.
    Else via the combination of b) and the fact that we are in event $A$ we may obtain
$$
    \begin{aligned}
        &\frac{1}{|\gB|}\sum_{t\in\gB}(V_r^*(\mu)-V_r^{(t)}(\mu)) \leq \frac{2\log|\gA|}{\sqrt{T}(1-\gamma)}+\frac{\beta W^2}{\sqrt{T}(1-\gamma)^3}\\
       &+\frac{2}{1-\gamma}\sqrt{\frac{1}{1-\gamma}\left\|\frac{\nu^*}{\nu_0}\right\|_\infty\left(\epsilon_{bias}+C\frac{(4L_\pi^2W+8L_\pi/(1-\gamma))^2}{(1-\gamma)^2\mu_F}\frac{\log(2T/\delta)}{K_{sgd}}+\frac{4\gamma^H}{1-\gamma}\left(\frac{1}{1-\gamma}+WL_\pi\right)\right)}.
    \end{aligned}
    $$
    
\endproof

\begin{lemma}\label{Lemma_constraint_violation_random_search}
    Suppose Assumption~\ref{Assumption_regular_maxima} holds and we use random search (Algorithm~\ref{Algorithm_random_search}) to solve the inner-loop problem.
    Let $B$ denote the event that for any $t\in\{1,...,T\}$, 
    $$\begin{aligned}
    &\max_{y\in Y} \left[V_{c_y}^{(t)}(\mu)-u_y\right]-\left[\widehat V_{c_{y^{(t)}}}^{(t)}(\mu)-u^{(t)}\right]\\
    &\leq \frac{1}{(1-\gamma)\sqrt{2K_{eval}}}\left(1+\sqrt{\log{(4T/\delta)}+m\log(4\mathrm{diam}(Y)L_y\sqrt{2K_{eval}}+\mathrm{diam}(Y))}\right)+\frac{\gamma^H}{1-\gamma}+2\epsilon_{con},
    \end{aligned}$$
    We have $\PB(B\cap A_2)\geq 1-\delta$ as long as $M>\log(2T/\delta)/\log(1-\epsilon_{con}^m/[\operatorname{vol}(Y)(2L_y)^m(1-\gamma)^m])    
    $
\end{lemma}
\proof{Proof of Lemma~\ref{Lemma_constraint_violation_random_search}}
    According to Lemma~\ref{Lemma_WHP_Bound_Evaluation_Uniform_y}, conditioned on event $A_2$, we have $\forall t\in\{1,...,T\}$
     $$
    \sup_{y\in Y}\left|\widehat V_{c_y}^{(t)}(\mu)-V_{c_y}^{(t)}(\mu)\right|\leq \frac{1}{(1-\gamma)\sqrt{2K_{eval}}}\left(1+\sqrt{\log{(4T/\delta)}+m\log(4\mathrm{diam}(Y)L_y\sqrt{2K_{eval}}+\mathrm{diam}(Y))}\right)+\frac{\gamma^H}{1-\gamma}
    $$
    as long as $K_{eval}$ is large enough.
    Therefore, let $y^*\in\arg\max_{y\in Y} \left[V_{c_y}^\pi(\mu)-u_y\right]$, 
    $$
    \left|V_{c_{y^*}}^\pi(\mu)-\widehat{V}_{c_{y^*}}^\pi(\mu)\right|\leq \frac{1}{(1-\gamma)\sqrt{2K_{eval}}}\left(1+\sqrt{\log{(4T/\delta)}+m\log(4\mathrm{diam}(Y)L_y\sqrt{2K_{eval}}+\mathrm{diam}(Y))}\right)+\frac{\gamma^H}{1-\gamma}
    $$
    By Assmption~\ref{Assumption_Lipschitz}, as long as $\exists i\in\{1,...,M\}$, $\|y_i-y^*\|_\infty\leq \epsilon_{con}(1-\gamma)/2L_y$,
    we have 
    $$\left[\widehat V_{c_{y^*}}^{(t)}(\mu)-u_{ y^*}\right]-\max_{i\in\{1,...,M\}}\left[\widehat V_{c_{y_i}}^\pi(\mu)-u_{y_i}\right]\leq \epsilon_{con}.$$
     According to Assumption~\ref{Assumption_regular_maxima}, when $\epsilon_{con}(1-\gamma)/2L_y<\epsilon_0$
    $$
    \PB(\exists i\in\{1,...,M\}, \|y_i-\hat y\|_\infty\leq \epsilon_{con}(1-\gamma)/2L_y)\geq 1-\left(1-\frac{\epsilon_{con}^m(1-\gamma)^m}{\operatorname{vol}(Y)(2L_y)^m}\right)^M.
    $$
    If $M\geq M(\epsilon_{con},\beta)$, we have
    $$\PB\left(\left[\widehat V_{c_{y^*}}^{(t)}(\mu)-u_{ y^*}\right]-\max_{i\in\{1,...,M\}}\left[\widehat V_{c_{y_i}}^\pi(\mu)-u_{y_i}\right]\leq \epsilon\right)\geq 1-\beta,$$ 
    where we define
    $$M(\epsilon_{con},\beta)=\frac{\log\beta}{\log\left(1-\frac{\epsilon_{con}^m(1-\gamma)^m}{\operatorname{vol}(Y)(2L_y)^m}\right)}.
    $$
    Setting $M=M(\epsilon_{con},\delta/2T)$ and applying the union bound argument yields the result.
\endproof

\begin{lemma}\label{Lemma_constraint_violation_PGA}
    Suppose Assumption~\ref{Assumption_concave_constraint} holds and we use projected subgradient ascent (Algorithm~\ref{Algorithm_projected_GD}) to solve the inner-loop problem.
    Then 
    $$\begin{aligned}
    &\max_{y\in Y} \left[V_{c_y}^{(t)}(\mu)-u_y\right]-\left[\widehat V_{c_{y^{(t)}}}^{(t)}(\mu)-u^{(t)}\right]\\
    &\leq \frac{1}{(1-\gamma)\sqrt{2K_{eval}}}\left(1+\sqrt{\log{(4T/\delta)}+m\log(4\mathrm{diam}(Y)L_y\sqrt{2K_{eval}}+\mathrm{diam}(Y))}\right)+\frac{\gamma^H}{1-\gamma}+\epsilon_{con},
    \end{aligned}$$
    as long as $T_{PGA}>\frac{4[\mathrm{diam}(Y)]^2L_y^2}{\epsilon_{con}^2(1-\gamma)^2}
    $.
\end{lemma}
\proof{Proof of Lemma~\ref{Lemma_constraint_violation_PGA}}
Noting that the objective is $\frac{2L_y}{1-\gamma}$-Lipschitz, we may complete the proof by directly applying Theorem 3.2 in \cite{bubeck2015convex}.
\endproof
\proof{Proof of Theorem~\ref{Theorem_random_search_SICPO}}
Conditioned on event $A=A_1\cap A_2$, setting $K_{sgd}, H, K_{eval}, T, \alpha, \eta$ as above yields
$$\begin{aligned}
    \eta&> \frac{4\log |\gA|}{\sqrt{T}(1-\gamma)}+\frac{2\beta W^2}{\sqrt{T}(1-\gamma)^3}\\
    &+\frac{4}{1-\gamma}\sqrt{\frac{1}{1-\gamma}\left\|\frac{\nu^*}{\nu_0}\right\|_\infty\left(\epsilon_{bias}+C\frac{(4L_\pi^2W+8L_\pi/(1-\gamma))^2}{(1-\gamma)^2\mu_F}\frac{\log(2T/\delta)}{K_{sgd}}+\frac{4\gamma^H}{1-\gamma}\left(\frac{1}{1-\gamma}+WL_\pi\right)\right)}.
    \end{aligned}
$$
Now we use Lemma~\ref{Lemma_reward_bound} to get
$$
    \begin{aligned}
        &\frac{1}{|\gB|}\sum_{t\in\gB}(V_r^*(\mu)-V_r^{(t)}(\mu)) \leq \frac{2\log|\gA|}{\sqrt{T}(1-\gamma)}+\frac{\beta W^2}{\sqrt{T}(1-\gamma)^3}\\
       &+\frac{2}{1-\gamma}\sqrt{\frac{1}{1-\gamma}\left\|\frac{\nu^*}{\nu_0}\right\|_\infty\left(\epsilon_{bias}+C\frac{(4L_\pi^2W+8L_\pi/(1-\gamma))^2}{(1-\gamma)^2\mu_F}\frac{\log(2T/\delta)}{K_{sgd}}+\frac{4\gamma^H}{1-\gamma}\left(\frac{1}{1-\gamma}+WL_\pi\right)\right)}\\
       &\leq \epsilon+\frac{1}{(1-\gamma)^{3/2}}\sqrt{\left\|\frac{\nu^*}{\nu_0}\right\|_\infty\epsilon_{bias}}.
    \end{aligned}
    $$
According to Lemma~\ref{Lemma_constraint_violation_random_search}, conditioned on event $A_2\cap B$, as long as we set $M$ as in the statement of the theorem, we have $\forall t\in\{1,...,T\}$
$$\begin{aligned}
    &\sup_{y\in Y} \left[V_{c_y}^{(t)}(\mu)-u_y\right]-\left[\widehat V_{c_{y^{(t)}}}^{(t)}(\mu)-u^{(t)}\right]\\
    &\leq \frac{1}{(1-\gamma)\sqrt{2K_{eval}}}\left(1+\sqrt{\log{(4T/\delta)}+m\log(4\mathrm{diam}(Y)L_y\sqrt{2K_{eval}}+\mathrm{diam}(Y))}\right)+\frac{\gamma^H}{1-\gamma}+\epsilon/2,
    \end{aligned}$$
Thus for $t\in \gB$, 
$$ \sup_{y\in Y}\left[V_{c_y}^{(t)}(\mu)-u_y\right]\leq 2\epsilon+\frac{1}{(1-\gamma)^{3/2}}\sqrt{\left\|\frac{\nu^*}{\nu_0}\right\|_\infty\epsilon_{bias}}.
$$
We complete the proof by noting that
$$
P(A\cap B)\geq 1-(1-P(A))-(1-P(A_2\cap B))\geq 1-2\delta.
$$
\endproof

\proof{Proof of Theorem~\ref{Theorem_PGA_SICPO}}
Conditioned on event $A=A_1\cap A_2$, setting $K_{sgd}, H, K_{eval}, T, \alpha, \eta$ as above yields
$$\begin{aligned}
    \eta&> \frac{4\log |\gA|}{\sqrt{T}(1-\gamma)}+\frac{2\beta W^2}{\sqrt{T}(1-\gamma)^3}\\
    &+\frac{4}{1-\gamma}\sqrt{\frac{1}{1-\gamma}\left\|\frac{\nu^*}{\nu_0}\right\|_\infty\left(\epsilon_{bias}+C\frac{(4L_\pi^2W+8L_\pi/(1-\gamma))^2}{(1-\gamma)^2\mu_F}\frac{\log(2T/\delta)}{K_{sgd}}+\frac{4\gamma^H}{1-\gamma}\left(\frac{1}{1-\gamma}+WL_\pi\right)\right)}.
    \end{aligned}
$$
Now we use Lemma~\ref{Lemma_reward_bound} to get
$$
    \begin{aligned}
        &\frac{1}{|\gB|}\sum_{t\in\gB}(V_r^*(\mu)-V_r^{(t)}(\mu)) \leq \frac{2\log|\gA|}{\sqrt{T}(1-\gamma)}+\frac{\beta W^2}{\sqrt{T}(1-\gamma)^3}\\
       &+\frac{2}{1-\gamma}\sqrt{\frac{1}{1-\gamma}\left\|\frac{\nu^*}{\nu_0}\right\|_\infty\left(\epsilon_{bias}+C\frac{(4L_\pi^2W+8L_\pi/(1-\gamma))^2}{(1-\gamma)^2\mu_F}\frac{\log(2T/\delta)}{K_{sgd}}+\frac{4\gamma^H}{1-\gamma}\left(\frac{1}{1-\gamma}+WL_\pi\right)\right)}\\
       &\leq \epsilon+\frac{1}{(1-\gamma)^{3/2}}\sqrt{\left\|\frac{\nu^*}{\nu_0}\right\|_\infty\epsilon_{bias}}.
    \end{aligned}
    $$
According to Lemma~\ref{Lemma_constraint_violation_PGA}, conditioned on event $A_2$, as long as we set $T_{PGA}$ as in the statement of the theorem, we have $\forall t\in\{1,...,T\}$
$$\begin{aligned}
    &\sup_{y\in Y} \left[V_{c_y}^{(t)}(\mu)-u_y\right]-\left[\widehat V_{c_{y^{(t)}}}^{(t)}(\mu)-u^{(t)}\right]\\
    &\leq \frac{1}{(1-\gamma)\sqrt{2K_{eval}}}\left(1+\sqrt{\log{(4T/\delta)}+m\log(4\mathrm{diam}(Y)L_y\sqrt{2K_{eval}}+\mathrm{diam}(Y))}\right)+\frac{\gamma^H}{1-\gamma}+\epsilon/2,
    \end{aligned}$$
Thus for $t\in \gB$, 
$$ \sup_{y\in Y}\left[V_{c_y}^{(t)}(\mu)-u_y\right]\leq 2\epsilon+\frac{1}{(1-\gamma)^{3/2}}\sqrt{\left\|\frac{\nu^*}{\nu_0}\right\|_\infty\epsilon_{bias}}.
$$
\endproof


\begin{lemma}\label{Lemma_WHP_Bound_Evaluation_Uniform_y}
    For any $\delta\in(0,1)$, with probability at least $1-\delta$,
    $$
    \sup_{y\in Y}\left|\widehat V_{c_y}^{(t)}(\mu)-V_{c_y}^{(t)}(\mu)\right|\leq \frac{1}{(1-\gamma)\sqrt{2K_{eval}}}\left(1+\sqrt{\log{(2/\delta)}+m\log(4\mathrm{diam}(Y)L_y\sqrt{2K_{eval}}+\mathrm{diam}(Y))}\right)+\frac{\gamma^H}{1-\gamma}
    $$
    as long as $L_y\sqrt{2K_{eval}}>1$.
\end{lemma}
\proof{Proof of Lemma~\ref{Lemma_WHP_Bound_Evaluation_Uniform_y}}
Assumption~\ref{Assumption_Lipschitz} implies
$$
\left|\widehat V_{c_y}^{(t)}(\mu)-\widetilde{V}_{c_y}^{(t)}(\mu)\right|-\left|\widehat V_{c_{y^\prime}}^{(t)}(\mu)-\widetilde{V}_{c_{y^\prime}}^{(t)}(\mu)\right|\leq \frac{2L_y}{1-\gamma}\|y-y^\prime\|_{\infty}.
$$
Let $N_{\epsilon}:=\{y_1,...,y_N\}$ be a $\epsilon$ cover w.r.t. $\|\cdot\|_{\infty}$ of $Y$, Example 5.8 in \cite{wainwright2019high}. We have
$$
\log N\leq m\log(2\mathrm{diam}(Y)/\epsilon+\mathrm{diam}(Y)).
$$
Combining Theorem~\ref{Theorem_Hoeffding_Inequality} and the arguments of union bound we may have that for any $\delta\in(0,1)$, with probability at least $1-\delta$,
 $$
    \sup_{y\in N_{\epsilon/2}}\left|\widehat V_{c_y}^{(t)}(\mu)-\widetilde{V}_{c_y}^{(t)}(\mu)\right|\leq \frac{1}{1-\gamma}\sqrt{\frac{\log(2N/\delta)}{2K_{eval}}}.
    $$
Then we may get
$$
\begin{aligned}
    \sup_{y\in Y}\left|\widehat V_{c_y}^{(t)}(\mu)-\widetilde{V}_{c_y}^{(t)}(\mu)\right|&\leq
    \frac{2L_y\epsilon}{1-\gamma}+
    \frac{1}{1-\gamma}\sqrt{\frac{\log(2/\delta)+m\log(2\mathrm{diam}(Y)/\epsilon+\mathrm{diam}(Y))}{2K_{eval}}}.
\end{aligned}
    $$
Here we set $\epsilon=\frac{1}{2L_y\sqrt{2K_{eval}}}$ and get
$$
    \sup_{y\in Y}\left|\widehat V_{c_y}^{(t)}(\mu)-\widetilde{V}_{c_y}^{(t)}(\mu)\right|\leq \frac{1}{(1-\gamma)\sqrt{2K_{eval}}}\left(1+\sqrt{\log{(2/\delta)}+m\log(4\mathrm{diam}(Y)L_y\sqrt{2K_{eval}}+\mathrm{diam}(Y))}\right).
$$
Since
$$
\sup_{y\in Y}\left|V_{c_y}^{(t)}(\mu) -\widetilde{V}_{c_y}^{(t)}(\mu)\right|\leq \frac{\gamma^H}{1-\gamma},
$$
we complete the proof.
\endproof
\section{Auxiliary Lemmas}

\begin{lemma}[Empirical Bernstein Inequality]
\label{Theorem_Empirical_Bernstein}
Suppose $n\geq 3$, $\{X_1,...,X_n\}$ be $n$ i.i.d. random variables with values in $[0,1]$. 
Let $\delta>0$. 
Then with probability at least $1-\delta$ we have
$$
\left|\EB X_1-\frac{\sum_{i=1}^n X_i}{n}\right|\leq\sqrt{\frac{2\mathbb{V}_n(X_{1:n})\log4/\delta}{n}}+\frac{4\log 4/\delta}{n},
$$
where $\mathbb{V}_n(X_{1:n}):=\frac{1}{n(n-1)}\sum_{i,j}\frac{(X_i-X_j)^2}{2}$ denotes the empirical variance of the dataset $\{X_1,...,X_n\}$.
\end{lemma}
\proof{Proof of Lemma~\ref{Theorem_Empirical_Bernstein}.}
See Theorem 11 in \cite{maurer2009empirical}.
\endproof

\begin{lemma}[Hoeffding's Inequality]
\label{Theorem_Hoeffding_Inequality}
Suppose $\{X_1,...,X_n\}$ be $n$ i.i.d. random variables with values in $[0,1]$.
Let $\delta>0$. 
Then with probability at least $1-\delta$ we have
$$
\left|\EB X_1-\frac{\sum_{i=1}^n X_i}{n}\right|\leq\sqrt{\frac{\log 2/\delta}{2n}}
$$
\end{lemma}
\proof{Proof of Lemma~\ref{Theorem_Hoeffding_Inequality}.}
See Theorem 2.2.6 in \cite{vershynin_2018}.
\endproof

\begin{lemma}\label{Lemma_Simulation_Lemma}
For any policy $\pi$ and transition probabilities $P$, $\widetilde{P}$, we have that
$$Q^{\pi}-\widetilde{Q}^{\pi}=\gamma\left(I-\gamma \widetilde{P}^{\pi}\right)^{-1}(P-\widetilde{P}) V^{\pi}
$$
\end{lemma}
\proof{Proof of Lemma~\ref{Lemma_Simulation_Lemma}.}
See Lemma 2 in \cite{pmlr-v125-agarwal20b}.
\endproof

\begin{lemma}\label{Lemma_Norm_of_Inf_Horizon_Expectation}
For any policy $\pi$, any transition probability $P$ and any vector $v\in \RB^{|\gS|\cdot|\gA|}$, we have 
$$\left\|\left(I-\gamma P^{\pi}\right)^{-1} v\right\|_{\infty} \leq\|v\|_{\infty} /(1-\gamma).
$$
\end{lemma}
\proof{Proof of Lemma~\ref{Lemma_Norm_of_Inf_Horizon_Expectation}.}
See Lemma 3 in \cite{pmlr-v125-agarwal20b}.
\endproof

\begin{lemma}\label{Lemma_Bound_of_Weighted_Variance}
For any policy $\pi$ and any transition probability $P$, we have
$$\left\|(I-\gamma P^\pi)^{-1} \sqrt{\Var_P^\pi}\right\|_\infty\leq\sqrt{\frac{2}{(1-\gamma)^3}},
$$
where $\sqrt{\cdot}$ is defined as the element-wise square root.
\end{lemma}
\proof{Proof of Lemma~\ref{Lemma_Bound_of_Weighted_Variance}.}
See Lemma 4 in \cite{pmlr-v125-agarwal20b}.
\endproof

\begin{theorem}[Chernoff's Inequality]
\label{Theorem_Chernoff_Inequality}
Let $X_i$ be independent Bernoulli random variables with parameter $p_i$. 
Consider their sum $S_N=\sum_{i=1}^N X_i$ and denote its mean by $\mu=\EB S_N$. 
Then, for any $t<\mu$, we have
$$
\PB\paren{S_N<t}\leq e^{-\mu}\paren{\frac{e\mu}{t}}^t.
$$
\end{theorem}
\proof{Proof of Theorem~\ref{Theorem_Chernoff_Inequality}.}
See \cite{vershynin_2018}.
\endproof
\appendix
\section{Additional Experiments}
\subsection{Additional Experiments on Subpopulation Shifts}
\subsubsection{Dataset Details}
\label{sec:app_sub_data}

\textbf{Colored MNIST (CMNIST)}: We classify MNIST digits from 2 classes, where classes 0 and 1 indicate original digits (0,1,2,3,4) and (5,6,7,8,9). The color is treated as a spurious attribute. Concretely, in the training set, the proportion between red samples and green samples is 8:2 in class 0, while the proportion is set as 2:8 in class 1. In the validation set, the proportion between green and red samples is 1:1 for all classes. In the test set, the proportion between green and red samples is 1:9 in class 0, while the ratio is 9:1 in class 1. The data sizes of train, validation, and test sets are 30000, 10000, and 20000, respectively. Follow~\cite{arjovsky2019invariant}, we flip labels with probability 0.25.

\textbf{Waterbirds}~\citep{sagawa2019distributionally}: The Waterbirds dataset aims to classify birds as ``waterbird" or ``landbird", where each bird image is spuriously associated with the background ``water" or ``land". 
Waterbirds is a synthetic dataset where each image is composed by pasting a bird image sampled from CUB dataset~\citep{WahCUB_200_2011} to a background drawn from the Places dataset~\cite{zhou2017places}. 
The bird categories in CUB are stratified as land birds or water birds. Specifically, the following bird species are selected to construct the waterbird class: albatross, auklet, cormorant, frigatebird, fulmar, gull, jaeger, kittiwake, pelican, puffin, tern, gadwall, grebe, mallard, merganser, guillemot, or Pacific loon. All other bird species are combined as the landbird class. We define (land background, waterbird) and (water background, landbird) are minority groups. There are 4,795 training samples while only 56 samples are ``waterbirds on land" and 184 samples are ``landbirds on water". The remaining training data include 3,498 samples from ``landbirds on land", and 1,057 samples from ``waterbirds on water".

\textbf{CelebA}~\citep{liu2015faceattributes,sagawa2019distributionally}: 
For the CelebA data~\citep{liu2015faceattributes}, we follow the data preprocess procedure from~\cite{sagawa2019distributionally}.
CelebA defines a image classification task where the input is a face image of celebrities and the classification label is its corresponding hair color --  ``blond” or ``not blond.” The label is spuriously correlated with gender, i.e., male or female. In CelebA, the minority groups are (blond, male) and (not blond, female). The number of samples for each group are 71,629 ``dark hair, female'', 66,874 ``dark hair, male", 22,880 ``blond hair, female", 1,387 ``blond hair, male". 


\textbf{CivilComments}~\citep{borkan2019nuanced,koh2021wilds}: 
We use CivilComments from the WILDS benchmark~\citep{koh2021wilds}. 
CivilComments is a text classification task, aiming to predict whether an online comment is toxic or non-toxic. 
The spurious domain identifications are defined as the demographic features, including male, female, LGBTQ, Christian, Muslim, other religion, Black, and White. 
CivilComments contains 450,000 comments collected from online articles. The number of samples for training, validation, and test are 269,038, 45,180, and 133,782, respectively. The readers may kindly refer to Table 17 in~\cite{koh2021wilds} for the detailed group information. 


\subsubsection{Training Details}
\label{sec:app_sub_training}
We adopt pre-trained ResNet-50~\citep{he2016deep} and BERT~\citep{sanh2019distilbert} as the model for image data (i.e., CMNIST, Waterbirds, CelebA) and text data (i.e., CivilComments), respectively. In each training iteration, we sample a batch of data per group. 
For intra-label LISA, we randomly apply mixup on sample batches with the same labels but different domains. For intra-domain LISA, we instead apply mixup on sample batches with the same domain but different labels. The interpolation ratio $\lambda$ is sampled from the distribution $\mathrm{Beta}(2,2)$. All hyperparameters are selected via cross-validation and are listed in Table~\ref{tab:hyperameter_sub}.

\subsubsection{Additional Results}
In this section, we have added the full results of subpopulation shifts in Table~\ref{tab:subpopulation_main_full} and Table~\ref{tab:subpopulation_ablation_full}.

\begin{table}[ht]
    \centering
    \small
    \caption{Hyperparameter settings for the subpopulation shifts.}
    \label{tab:hyperameter_sub}
    \begin{tabular}{l|cccc}
    \toprule
        Dataset & CMNIST & Waterbirds & CelebA & CivilComments  \\
        \midrule
        Learning rate &  1e-3 & 1e-3 &  1e-4 & 1e-5\\
        Weight decay & 1e-4 & 1e-4 & 1e-4 & 0 \\
        Scheduler & n/a & n/a & n/a & n/a \\ 
        Batch size & 16 & 16 & 16 & 8 \\
        Type of mixup & mixup & mixup & CutMix & ManifoldMix \\
        Architecture & ResNet50 & ResNet50 & ResNet50 & DistilBert \\
        Optimizer & SGD & SGD & SGD & Adam  \\
        Maximum Epoch & 300 & 300 & 50 & 3 \\
        Strategy sel. prob. $p_{sel}$ & 0.5 & 0.5 & 0.5 & 1.0\\
        \bottomrule
    \end{tabular}
\end{table}


\begin{table*}[h]
\small
\caption{Full results of subpopulation shifts with standard deviation. All the results are performed with three random seed.}
\label{tab:subpopulation_main_full}
\begin{center}
\begin{tabular}{l|cc|cc}
\toprule
\multirow{2}{*}{} & \multicolumn{2}{c|}{CMNIST} & \multicolumn{2}{c}{Waterbirds} \\
& Avg. & Worst  & Avg. & Worst \\\midrule
ERM & 27.8 $\pm$ 1.9\% & 0.0 $\pm$ 0.0\% & 97.0 $\pm$ 0.2\% & 63.7 $\pm$ 1.9\%\\
UW  & 72.2 $\pm$ 1.1\% & 66.0 $\pm$ 0.7\% &  95.1 $\pm$ 0.3\% & 88.0 $\pm$ 1.3\% \\
IRM & 72.1 $\pm$ 1.2\% &  70.3 $\pm$ 0.8\% &  87.5 $\pm$ 0.7\% & 75.6 $\pm$ 3.1\% \\
IB-IRM & 72.2 $\pm$ 1.3\% & 70.7 $\pm$ 1.2\% & 88.5 $\pm$ 0.6\%  & 76.5 $\pm$ 1.2 \% \\
V-REx & 71.7 $\pm$ 1.2\% & 70.2 $\pm$ 0.9\% & 88.0 $\pm$ 1.0\% & 73.6 $\pm$ 0.2\% \\
Coral & 71.8 $\pm$ 1.7\%  & 69.5 $\pm$ 0.9\% & 90.3 $\pm$ 1.1\% & 79.8 $\pm$ 1.8\% \\
GroupDRO & 72.3 $\pm$ 1.2\% & 68.6 $\pm$ 0.8\% & 91.8 $\pm$ 0.3\% & \textbf{90.6 $\pm$ 1.1\%} \\
DomainMix & 51.4 $\pm$ 1.3\% & 48.0 $\pm$ 1.3\% & 76.4 $\pm$ 0.3\% & 53.0 $\pm$ 1.3\% \\
Fish &  46.9 $\pm$ 1.4\% & 35.6 $\pm$ 1.7\% & 85.6 $\pm$ 0.4\% & 64.0 $\pm$ 0.3\%  \\
\midrule
\textbf{LISA} & 74.0 $\pm$ 0.1\% & \textbf{73.3 $\pm$ 0.2\%} & 91.8 $\pm$ 0.3\% & 89.2 $\pm$ 0.6\%  \\\midrule\midrule
& \multicolumn{2}{c|}{CelebA} & \multicolumn{2}{c}{CivilComments} \\
 & Avg. & Worst & Avg. & Worst \\\midrule
ERM  & 94.9 $\pm$ 0.2\% & 47.8 $\pm$ 3.7\% & 92.2 $\pm$ 0.1\% & 56.0 $\pm$ 3.6\% \\
UW & 92.9 $\pm$ 0.2\% & 83.3 $\pm$ 2.8\% & 89.8 $\pm$ 0.5\% & 69.2 $\pm$ 0.9\% \\
IRM &  94.0 $\pm$ 0.4\% & 77.8 $\pm$ 3.9\% &  88.8 $\pm$ 0.7\% & 66.3 $\pm$ 2.1\% \\
IB-IRM & 93.6 $\pm$ 0.3\% & 85.0 $\pm$ 1.8\% &  89.1 $\pm$ 0.3\% & 65.3 $\pm$ 1.5\%  \\ 
V-REx & 92.2 $\pm$ 0.1\% & 86.7 $\pm$ 1.0\%  & 90.2 $\pm$ 0.3\% & 64.9 $\pm$ 1.2\% \\
Coral &  93.8 $\pm$ 0.3\% & 76.9 $\pm$ 3.6\% & 88.7 $\pm$ 0.5\% & 65.6 $\pm$ 1.3\% \\
GroupDRO & 92.1 $\pm$ 0.4\% & 87.2 $\pm$ 1.6\% & 89.9 $\pm$ 0.5\% & 70.0 $\pm$ 2.0\% \\
DomainMix  & 93.4 $\pm$ 0.1\% & 65.6 $\pm$ 1.7\% & 90.9 $\pm$ 0.4\% & 63.6 $\pm$ 2.5\% \\
Fish &  93.1 $\pm$ 0.3\%  &  61.2 $\pm$ 2.5\% & 89.8 $\pm$ 0.4\% & 71.1 $\pm$ 0.4\%  \\\midrule
\textbf{LISA (ours)} & 92.4 $\pm$ 0.4\% & \textbf{89.3 $\pm$ 1.1\%} & 89.2 $\pm$ 0.9\% & \textbf{72.6 $\pm$ 0.1\%}\\
\bottomrule
\end{tabular}
\end{center}
\end{table*}


\begin{table*}[h]
\small
\caption{Full table of the comparison between LISA and other substitute mixup strategies in subpopulation shifts. UW represents upweighting.}
\vspace{-1em}
\label{tab:subpopulation_ablation_full}
\begin{center}
\begin{tabular}{l|cc|cc}
\toprule
\multirow{2}{*}{} & \multicolumn{2}{c|}{CMNIST} & \multicolumn{2}{c}{Waterbirds} \\
& Avg. & Worst  & Avg. & Worst   \\\midrule
ERM & 27.8 $\pm$ 1.9\% & 0.0 $\pm$ 0.0\% & 97.0 $\pm$ 0.2\% & 63.7 $\pm$ 1.9\%\\
Vanilla mixup & 32.6 $\pm$ 3.1\% & 3.1 $\pm$ 2.4\% &  81.0 $\pm$ 0.2\% & 56.2 $\pm$ 0.2\% \\
Vanilla mixup + UW & 72.2 $\pm$ 0.7\% & 71.8 $\pm$ 0.1\% & 92.1 $\pm$ 0.1\% & 85.6 $\pm$ 1.0\% \\
In-group Group & 33.6 $\pm$ 1.9\% & 24.0 $\pm$ 1.1\% & 88.7 $\pm$ 0.3\%  & 68.0 $\pm$ 0.4\% \\
In-group + UW & 72.6 $\pm$ 0.1\% & 71.6 $\pm$ 0.2\% & 91.4 $\pm$ 0.6\% & 87.1 $\pm$ 0.6\% \\
\midrule
\textbf{LISA (ours)} & 74.0 $\pm$ 0.1\% &\textbf{73.3 $\pm$ 0.2\%} & 91.8 $\pm$ 0.3\% & \textbf{89.2 $\pm$ 0.6\%}  \\
\midrule\midrule
& \multicolumn{2}{c|}{CelebA} & \multicolumn{2}{c}{CivilComments} \\
& Avg. & Worst & Avg. & Worst \\\midrule
ERM & 94.9 $\pm$ 0.2\% & 47.8 $\pm$ 3.7\% & 92.2 $\pm$ 0.1\% & 56.0 $\pm$ 3.6\% \\
Vanilla mixup & 95.8 $\pm$ 0.0\% & 46.4 $\pm$ 0.5\% & 90.8 $\pm$ 0.8\% & 67.2 $\pm$ 1.2\% \\
Vanilla mixup + UW & 91.5 $\pm$ 0.2\% & 88.0 $\pm$ 0.3\% & 87.8 $\pm$ 1.2\% & 66.1 $\pm$ 1.4\%  \\
Within Group & 95.2 $\pm$ 0.3\% & 58.3 $\pm$ 0.9\% & 90.8 $\pm$ 0.6\% & 69.2 $\pm$ 0.8\% \\
Within Group + UW & 92.4 $\pm$ 0.4\%  & 87.8 $\pm$ 0.6\% & 84.8 $\pm$ 0.7\% & 69.3 $\pm$ 1.1\% \\
\midrule
\textbf{LISA (ours)} & 92.4 $\pm$ 0.4\% & \textbf{89.3 $\pm$ 1.1\%} & 89.2 $\pm$ 0.9\% & \textbf{72.6 $\pm$ 0.1\%}\\
\bottomrule
\end{tabular}
\end{center}
\end{table*}


\subsection{Additional Experimental Settings on Domain Shifts}
\subsubsection{Dataset Details}
\label{sec:app_domain_data}

In this section, we provide detailed descriptions of datasets used in the experiments of domain shifts and report the data statistics in Table~\ref{tab:domian_data}.
\paragraph{Camelyon17}
We use Camelyon17 from the WILDS benchmark~\citep{koh2021wilds,bandi2018detection}, which provides $450,000$ lymph-node scans sampled from $5$ hospitals. 
Camelyon17 is a medical image classification task where the input $x$ is a $96\times 96$ image and the label $y$ is whether there exists tumor tissue in the image. The domain $d$ denotes the hospital that the patch was taken from. The training dataset is drawn from the first $3$ hospitals, while out-of-distribution validation and out-of-distribution test datasets are sampled from the $4$-th hospital and $5$-th hospital respectively. 

\paragraph{FMoW}
The FMoW dataset is from the WILDS benchmark~\citep{koh2021wilds,christie2018functional} --- a satellite image classification task which includes $62$ classes and $80$ domains ($16$ years x $5$ regions). Concretely, the input $x$ is a $224 \times 224$ RGB satellite image, the label $y$ is one of the $62$ building or land use categories, and the domain $d$ represents the year that the image was taken as well as its corresponding geographical region -- Africa, the Americas, Oceania, Asia, or Europe. 
The train/test/validation splits are based on the time when the images are taken. Specifically, images taken before 2013 are used as the training set. Images taken between 2013 and 2015 are used as the validation set. Images taken after 2015 are used for testing. 



\paragraph{RxRx1}
RxRx1~\citep{koh2021wilds,taylor2019rxrx1} from the WILDS benchmark is a cell image classification task. In the dataset, some cells have been genetically perturbed by siRNA. The goal of RxRx1 is to predict which siRNA that the cells have been treated with. Concretely, the input $x$ is an image of cells obtained by fluorescent microscopy, the label $y$ indicates which of the $1,139$ genetic treatments the cells received, and the domain $d$ denotes the experimental batches. Here, $33$ different batches of images are used for training, where each batch contains one sample for each class. The out-of-distribution validation set has images from $4$ experimental batches. The out-of-distribution test set has $14$ experimental batches. The average accuracy on out-of-distribution test set is reported.

\paragraph{Amazon}
Each task in the Amazon benchmark~\citep{koh2021wilds,ni2019justifying} is a multi-class sentiment classification task. The input $x$ is the text of a review, the label $y$ is the corresponding star rating ranging from 1 to 5, and the domain $d$ is the corresponding reviewer. The training set has $245,502$ reviews from $1,252$ reviewers, while the out-of-distribution validation set has $100,050$ reviews from another $1,334$ reviewers. The out-of-distribution test set also has $100,050$ reviews from the rest $1,252$ reviewers. We evaluate the models by the 10th percentile of per-user accuracies in the test set.



\paragraph{MetaShift} 
We use the MetaShift~\citep{metadataset}, which is derived from Visual Genome~\citep{krishnavisualgenome}. 
MetaShift leverages the natural heterogeneity of Visual Genome to provide many distinct data distributions for a given class (e.g. “cats with cars” or “cats in bathroom” for the “cat” class). A key feature of MetaShift is that it provides explicit explanations of the dataset correlation and a distance score to measure the degree of distribution shift between any pair of sets.

We adopt the “Cat vs. Dog” task in MetaShift, where we evaluate the model on the “dog(\emph{shelf})” domain with 306 images, and the “cat(\emph{shelf})” domain with 235 images. The training data for the “Cat” class is the cat(\emph{sofa + bed}), including cat(\emph{sofa}) domain and cat(\emph{bed}) domain. MetaShift provides 4 different sets of training data for the “Dog” class in an increasingly challenging order, i.e., increasing the amount of distribution shift. Specifically, dog(\emph{cabinet + bed}), dog(\emph{bag + box}), dog(\emph{bench + bike}), dog(\emph{boat + surfboard}) are selected for training, where their corresponding distances to dog(\emph{shelf}) are 0.44, 0.71, 1.12, 1.43.





\subsubsection{Training Details}
\label{sec:app_domain_training}
Follow WILDS~\citet{koh2021wilds}, we adopt pre-trained DenseNet121~\citep{huang2017densely} for Camelyon17 and FMoW datasets, ResNet-50~\citep{he2016deep} for RxRx1 and MetaShift datasets, and DistilBert~\citep{sanh2019distilbert} for Amazon datasets.

In each training iteration, we first draw a batch of samples $B_1$ from the training set. With $B_1$, we then select another sample batch $B_2$ with same labels as $B_1$ for data interpolation. The interpolation ratio $\lambda$ is drawn from the distribution $\mathrm{Beta}(2,2)$. We use the same image transformers as~\citet{koh2021wilds}, and all other hyperparameters are selected via cross-validation and are listed in Table~\ref{tab:domain_parameter}.



\begin{table}[ht]
    \centering
    \small
    \caption{Hyperparameter settings for the domain shifts.}
    \begin{tabular}{l|ccccc}
    \toprule
        Dataset & Camelyon17 & FMoW & RxRx1 & Amazon & MetaShift \\
        \midrule
        Learning rate & 1e-4 & 1e-4 & 1e-3 & 2e-6 & 1e-3 \\
        Weight decay & 0 & 0 & 1e-5 & 0 & 1e-4 \\
        Scheduler & n/a & n/a & 
        \shortstack{Cosine Warmup} & n/a & n/a \\
        Batch size & 32 & 32 & 72 & 8 & 16 \\
        Type of mixup & CutMix & CutMix & CutMix & ManifoldMix & CutMix \\
        Architecture & DenseNet121 & DenseNet121 & ResNet50 & DistilBert & ResNet50 \\
        Optimizer & SGD & Adam & Adam & Adam & SGD \\
        Maximum Epoch & 2 & 5 & 90 & 3 & 100\\
        Strategy sel. prob. $p_{sel}$ & 1.0 & 1.0 & 1.0 & 1.0 & 1.0\\
        \bottomrule
    \end{tabular}
    \label{tab:domain_parameter}
\end{table}
















\subsection{Strength of Spurious Correlation}
\label{sec:app_spurious_strength}
\yao{In Section~\ref{sec:prelim}, the spurious correlation is defined as the association between the domain $d$ and label $y$, measured by  Cramér's V ~\citep{cramer2016mathematical}. Specifically, let $k_{y,d}$ be the number of samples from domain $d$ with label $y$. The Cramér's V is formulated as 
\begin{equation}
\label{eq:strength_spurious}
    V=\sqrt{\frac{\chi^2}{N\min(|Y-1|, |D-1|)}}=\sqrt{\frac{\sum_{y\in \mathcal{Y}, d\in \mathcal{D}}\frac{(k_{y,d}-\tilde k_{y,d})^2}{\tilde k_{y,d}}}{N\min(|\mathcal{Y}|-1|, |\mathcal{D}|-1|)}},
\end{equation}
where $N$ represents the number of samples in the entire dataset and $\tilde k_{y,d}=\frac{\sum_{y\in \mathcal{Y}} k_{y,d}\sum_{d\in \mathcal{D}}{ k_{y,d}}}{\sum_{y\in \mathcal{Y},d \in \mathcal{D}} k_{y,d}}$. Cramér's V varies from 0 to 1 and higher Cramér's V represents stronger correlation.}

\yao{According to Eq.~\eqref{eq:strength_spurious}, we calculate the strength of spurious correlations on all datasets used in the experiments and report the results in Table~\ref{tab:spurious_strength}. Compared with other datasets, the Cramér's V on Camelyon17, FMoW and RxRx1 are significantly smaller, indicating weaker spurious correlations.}


\begin{table*}[h]
\small
\caption{Analysis of the strength of spurious correlations on datasets with subpopulation shifts or domain shifts.}
\label{tab:spurious_strength}
\begin{center}
\begin{tabular}{cccc|ccccc}
\toprule
\multicolumn{4}{c|}{Subpopulation Shifts} & \multicolumn{5}{c}{Domain Shifts}\\
CMNIST & Waterbirds & CelebA & CivilComments & Camelyon17 & FMoW & RxRx1 & Amazon & MetaShift \\\midrule
0.6000 & 0.8672 & 0.3073 & 0.8723 & 0.0004 & 0.1114 & 0.0067 & 0.3377 & 0.4932 \\
\bottomrule
\end{tabular}
\end{center}
\end{table*}





\subsection{Results on Datasets without Spurious Correlations}
\label{sec:app_no_spurious}
In order to analyze the factors that lead to the performance gains of LISA, we conduct experiments on datasets without spurious correlations. To be more specific, we balance the number of samples for each group under the subpopulation shifts setting. The results of ERM, Vanilla mixup and LISA on CMNIST, Waterbirds and CelebA are reported in Table \ref{tab:app_no_spurious}. The results show that LISA performs similarly compared with ERM when datasets do not have spurious correlations. If there exists any spurious correlation, LISA significantly outperforms ERM. Another interesting finding is that Vanilla mixup outperforms LISA and ERM without spurious correlations, while LISA achieves the best performance with spurious correlations. This finding strengthens our conclusion that the performance gains of LISA are from eliminating spurious correlations rather than simple data augmentation.


\begin{table}[h]
\small
\caption{Results on datasets without spurious correlations. LISA performs similarly to ERM when there are no spurious correlations. However, Vanilla mixup outperforms LISA and ERM when there are no spurious correlations while underperforms LISA on datasets with spurious correlations. The results further strengthen our claim that the performance gains of LISA are not from simple data augmentation.}
\label{tab:app_no_spurious}

\begin{center}
\setlength{\tabcolsep}{1.3mm}{
\begin{tabular}{l|ccc}
    \toprule
        Dataset & CMNIST & Waterbirds & CelebA \\
        \midrule
        ERM & 73.67\% & 88.07\% & 86.11\% \\
        Vanilla mixup & 74.28\% & 88.23\% & 88.89\% \\
        LISA & 73.18\% & 87.05\% & 87.22\% \\
        \bottomrule
    \end{tabular}
}
\end{center}

\end{table}

\subsection{Additional Invariance Analysis}
\subsubsection{Additional Metrics of Invariant Predictor Analysis}
\label{sec:app_additional_predictor_invariance}
In Table~\ref{tab:additional_invariance_predictor}, we report two additional metrics to measure the invariance of predictors -- Risk Variance and Gradient Norm, which is defined as:
\begin{itemize}[leftmargin=*]
    \item \textbf{Risk Variance ($\mathrm{IP}_{var}$)}. Motivated by~\citet{krueger2021out}, we use the variance of test risks across all domains to measure the invariance, which is defined as $\mathrm{IP}_{var}=\mathrm{Var}(\{\mathcal{R}_1(\theta),\ldots, \mathcal{R}_D(\theta)\})$, where $D$ represents the number of test domains and $\mathcal{R}_d(\theta)$ represents the risk of domain $d$.
    \item \textbf{Gradient Norm ($\mathrm{IP}_{norm}$)}. Follow IRMv1~\cite{arjovsky2019invariant}, we use the gradient norm of the classifier to measure the optimality of the dummy classifier at each domain $d$. Assume the classifier is parameterized by $w$, $\mathrm{IP}_{norm}$ is defined as: $\mathrm{IP}_{norm}=\frac{1}{|\mathcal{D}|}\sum_{d\in \mathcal{D}}\| \nabla_{w|w=1.0} \mathcal{R}_d (\theta)\|^2$.
\end{itemize}


\begin{table}[h]
\small
\caption{Additional Invariance Metrics for Invariant Predictor Analysis. We report the results of risk variance ($\mathrm{IP}_{var}$) and gradient norm ($\mathrm{IP}_{norm}$), where smaller values indicate stronger invariance.}
\label{tab:additional_invariance_predictor}
\begin{center}
\setlength{\tabcolsep}{0.9mm}{
\begin{tabular}{l|c|c|c|c|c|c|c|c}
\toprule
& \multicolumn{4}{c|}{$\mathrm{IP}_{var} \downarrow$} & \multicolumn{4}{c}{$\mathrm{IP}_{norm} \downarrow$} \\\cmidrule{2-9}
  & CMNIST & Waterbirds & Camelyon  & MetaShift & CMNIST & Waterbirds  & Camelyon & MetaShift\\\midrule
ERM  & 12.0486 & 0.2456 & 0.0150 & 1.8824  & 1.1162 & 1.5780  & 1.2959 & 1.0914\\
Vanilla mixup & 0.2769 & 0.1465 & 0.0180 & 0.2659  & 1.5347 & 1.8631 & 0.3993 & 0.1985 \\
IRM & 0.0112 & 0.1243 & 0.0201 & 0.8748 & 0.0908 & 0.9798  & 0.5266 & 0.2320 \\
IB-IRM & 0.0072 & 0.2069 & 0.0329 & 0.5680 & 0.6225 & 0.8814 & 0.6890 & 0.1683\\
V-REx & 0.0056 & 0.1257 & 0.0106 & 0.4220 & 0.0290 & 0.8329 & 0.9641 & 0.3680 \\
\midrule
\textbf{LISA (ours)} & \textbf{0.0012} & \textbf{0.0016} & \textbf{9.97e-5} & \textbf{0.2387} & \textbf{0.0039}  & \textbf{0.0538} & \textbf{0.3081} & \textbf{0.1354} \\ 
\bottomrule
\end{tabular}
}
\end{center}
\end{table}

Comparing LISA to other invariant learning methods, the results of $\mathrm{IP}_{var}$ and $\mathrm{IP}_{norm}$ further confirm that LISA does indeed improve predictor invariance.

\subsubsection{Analysis of Learned Invariant Representations}
\label{app:sec_representation_invariance}
\yao{In this section, we use pairwise divergence of representations ($\mathrm{IR}_{kl}$) to measure representation-level invariance. Specifically, assume the representation before classifier of each sample $(x_i, y_i, d)$ is $h_{i,d}$, we compute the KL divergence of the distribution of representations. Similarly, kernel density estimation is also used to estimate the probability density function $P(h_d^y)$ of representations from domain $d$ with label $y$. Formally, $\mathrm{IR}_{kl}$ is defined as $\mathrm{IR}_{kl}=\frac{1}{|\mathcal{Y}||\mathcal{D}|^2}\sum_{y\in \mathcal{Y}}\sum_{d',d\in \mathcal{D}}\mathrm{KL}(P(h_D^y\mid D=d)|P(h_{D}^y\mid D=d'))$. Smaller $\mathrm{IR}_{kl}$ values indicate more invariant representations with respect to the labels. We report the results on CMNIST, Waterbirds, Camelyon17 and MetaShift in Table~\ref{tab:invariance}. Our key observations are: (1) Compared with ERM, LISA learns stronger representation-level invariance. The potential reason is that a stronger 
invariant predictor implicitly includes stronger invariance representation; (2) LISA provides more invariant representations than other regularization-based invariant predictor learning methods, i.e., IRM, IB-IRM, V-REx, showing its capability in learning stronger invariance.}

\begin{table}[h]
\small
\caption{Results of representation-level invariance $\mathrm{IR}_{kl}$ ($\times 10^8$ for CMNIST), where smaller $\mathrm{IR}_{kl}$ value denotes stronger invariance.}
\centering
\label{tab:invariance}
\begin{tabular}{l|c|c|c|c}
\toprule
  & CMNIST & Waterbirds & Camelyon17 & MetaShift \\\midrule
ERM & 1.683 & 3.592 & 8.213 & 0.632 \\
Vanilla mixup & 4.392 & 3.935 & 7.786 & 0.634 \\
IRM  & 1.905 & 2.413 & 8.169 & 0.627  \\
IB-IRM & 3.178 & 3.306 & 8.824 & 0.646 \\
V-REx & 3.169 & 3.414 & 8.838 & 0.617\\
\midrule
\textbf{LISA (ours)} & \textbf{0.421} & \textbf{1.912} & \textbf{7.570} & \textbf{0.585} \\
\bottomrule
\end{tabular}
\end{table}

\yao{Besides the quantitative analysis, follow Appendix C in~\citet{lee2019meta}, we visualize the hidden representations for all test samples and the decision boundary on Waterbirds and illustrate the results in Figure~\ref{fig:boundary}. Compared with other methods, the representations of samples with the same label that learned by LISA are closer regardless of their domain information, which further demonstrates the promise of LISA in producing invariant representations.}
\begin{figure}[h]
\centering
  \includegraphics[width=0.7\textwidth]{figures/boundary_visualization.pdf}
  \vspace{-0.5em}
  \caption{Visualization of sample representations and decision boundaries on Waterbirds dataset.}\label{fig:boundary}
\end{figure}

\subsection{Full Results of WILDS data}
\label{sec:app_domain_full_results}
Follow~\citet{koh2021wilds}, we reported more results on WILDS datasets in Table~\ref{tab:camelyon_full} - Table~\ref{tab:amazon_full}, including validation performance and the results of other metrics. According to these additional results, we could see that LISA outperforms other baseline approaches in all scenarios. Particularly, we here discuss two additional findings: (1) In Camelyon dataset, the test data is much more visually distinctive compared with the validation data, resulting in the large gap ($\sim10\%$) between validation and test performance of ERM (see Table~\ref{tab:camelyon_full}). However, LISA significantly reduces the performance gap between the validation and test sets, showing its promise in improving OOD robustness; (2) In Amazon dataset, though LISA performs worse than ERM in average accuracy, it achieves the best accuracy at the 10th percentile, which is regarded as a more common and important metric to evaluate whether models perform consistently well across all users~\citep{koh2021wilds}.


\begin{table*}[h]
\small
\caption{Full Results of Camelyon17. We report both validation accuracy and test accuracy.}
\label{tab:camelyon_full}
\begin{center}
\begin{tabular}{l|cc}
\toprule
 & Validation Acc. & Test Acc.\\\midrule
ERM & 84.9 $\pm$ 3.1\% & 70.3 $\pm$ 6.4\%  \\
IRM  & \textbf{86.2 $\pm$ 1.4\%} & 64.2 $\pm$ 8.1\% \\
IB-IRM & 80.5 $\pm$ 0.4\% & 68.9 $\pm$ 6.1\% \\
V-REx  & 82.3 $\pm$ 1.3\% & 71.5 $\pm$ 8.3\% \\
Coral & \textbf{
86.2 $\pm$ 1.4\%} & 59.5 $\pm$ 7.7\% \\
GroupDRO & 85.5 $\pm$ 2.2\% & 68.4 $\pm$ 7.3\% \\
DomainMix & 83.5 $\pm$ 1.1\% & 69.7 $\pm$ 5.5\% \\
Fish & 83.9 $\pm$ 1.2\% & 74.7 $\pm$ 7.1\% \\
\midrule
\textbf{LISA (ours)} & 81.8 $\pm$ 1.3\% & \textbf{77.1 $\pm$ 6.5\%} \\
\bottomrule
\end{tabular}
\end{center}
\end{table*}

\begin{table*}[h]
\small
\caption{Full Results of FMoW. Here, we report the average accuracy and the worst-domain accuracy on both validation and test sets.}
\label{tab:fmow_full}
\begin{center}
\begin{tabular}{l|cc|cc}
\toprule
\multirow{2}{*}{}  & \multicolumn{2}{c}{Validation} & \multicolumn{2}{c}{Test}\\
& Avg. Acc. & Worst Acc. & Avg. Acc. & Worst Acc. \\\midrule
ERM &  \textbf{59.5 $\pm$ 0.37\%} & 48.9 $\pm$ 0.62\% & \textbf{53.0 $\pm$ 0.55\%} & 32.3 $\pm$ 1.25\% \\
IRM  & 57.4 $\pm$ 0.37\% & 47.5 $\pm$ 1.57\% & 50.8 $\pm$ 0.13\% & 30.0 $\pm$ 1.37\% \\
IB-IRM & 56.1 $\pm$ 0.48\% & 45.0 $\pm$ 0.62\% & 49.5 $\pm$ 0.49\% & 28.4 $\pm$ 0.90\% \\
V-REx  & 55.3 $\pm$ 1.75\% & 44.7 $\pm$ 1.31\% & 48.0 $\pm$ 0.64\% & 27.2 $\pm$ 0.78\%\\
Coral & 56.9 $\pm$ 0.25\% & 47.1 $\pm$ 0.43\% & 50.5 $\pm$ 0.36\% & 31.7 $\pm$ 1.24\% \\
GroupDRO & 58.8 $\pm$ 0.19\% & 46.5 $\pm$ 0.25\% & 52.1 $\pm$ 0.50\% & 30.8 $\pm$ 0.81\% \\
DomainMix & 58.6 $\pm$ 0.29\% & 48.9 $\pm$ 1.15\% & 51.6 $\pm$ 0.19\% & 34.2 $\pm$ 0.76\% \\
Fish & 57.8 $\pm$ 0.15\% & \textbf{49.5 $\pm$ 2.34\%} & 51.8 $\pm$ 0.32\% & 34.6 $\pm$ 0.18\% \\
\midrule
\textbf{LISA (ours)} & 58.7 $\pm$ 0.92\% & 48.7 $\pm$ 0.74\% & \textbf{52.8 $\pm$ 0.94\%} & \textbf{35.5 $\pm$ 0.65\%} \\
\bottomrule
\end{tabular}
\end{center}
\end{table*}


\begin{table*}[h]
\small
\caption{Full Results of RxRx1. ID: in-distribution; OOD: out-of-distribution}
\label{tab:rr1_full}
\begin{center}
\begin{tabular}{l|ccc}
\toprule
 & Validation Acc. & Test ID Acc. & Test OOD Acc.\\\midrule
ERM & 19.4 $\pm$ 0.2\% & 35.9 $\pm$ 0.4\% & 29.9 $\pm$ 0.4\% \\
IRM  & 5.6 $\pm$ 0.4\% & 9.9 $\pm$ 1.4\% & 8.2 $\pm$ 1.1\% \\
IB-IRM & 4.3 $\pm$ 0.7\% & 7.9 $\pm$ 0.5\% & 6.4 $\pm$ 0.6\% \\
V-REx  & 5.2 $\pm$ 0.6\% & 9.3 $\pm$ 0.9\% & 7.5 $\pm$ 0.8\%\\
Coral & 18.5 $\pm$ 0.4\% & 34.0 $\pm$ 0.3\% & 28.4 $\pm$ 0.3\% \\
GroupDRO & 15.2 $\pm$ 0.1\% & 28.1 $\pm$ 0.3\% & 23.0 $\pm$ 0.3\% \\
DomainMix & 19.3 $\pm$ 0.7\% & 39.8 $\pm$ 0.2\% & 30.8 $\pm$ 0.4\% \\
Fish & 7.5 $\pm$ 0.6\% & 12.7 $\pm$ 1.9\% & 10.1 $\pm$ 1.5\% \\
\midrule
\textbf{LISA (ours)} & \textbf{20.1 $\pm$ 0.4\%} & \textbf{41.2 $\pm$ 1.0\%} & \textbf{31.9 $\pm$ 0.8\%}  \\
\bottomrule
\end{tabular}
\end{center}
\end{table*}


\begin{table*}[h]
\small
\caption{Full Results of Amazon. Both the average accuracy and the 10th Percentile accuracy are reported.}
\label{tab:amazon_full}
\begin{center}
\begin{tabular}{l|cc|cc}
\toprule
\multirow{2}{*}{}  & \multicolumn{2}{c}{Validation} & \multicolumn{2}{c}{Test}\\
& Avg. Acc. & 10-th Per. & Avg. Acc. & 10-th Per. Acc. \\\midrule
ERM &  \textbf{72.7 $\pm$ 0.1\%} & \textbf{55.2 $\pm$ 0.7\%} & 71.9 $\pm$ 0.1\%  & 53.8 $\pm$ 0.8\% \\
IRM  & 71.5 $\pm$ 0.3\% & 54.2 $\pm$ 0.8\% & 70.5 $\pm$ 0.3\% & 52.4 $\pm$ 0.8\% \\
IB-IRM & 72.4 $\pm$ 0.4\% & \textbf{55.1 $\pm$ 0.6\%} & \textbf{72.2 $\pm$ 0.3\%} & 53.8 $\pm$ 0.7\% \\
V-REx  & 72.7 $\pm$ 1.2\% & 53.8 $\pm$ 0.7\% & 71.4 $\pm$ 0.4\% & 53.3 $\pm$ 0.0\% \\
Coral & 72.0 $\pm$ 0.3\% & 54.7 $\pm$ 0.0\% & 70.0 $\pm$ 0.6\% & 52.9 $\pm$ 0.8\% \\
GroupDRO & 70.7 $\pm$ 0.6\% & 54.7 $\pm$ 0.0\% & 70.0 $\pm$ 0.6\% & 53.3 $\pm$ 0.0\% \\
DomainMix & 71.9 $\pm$ 0.2\% & 54.7 $\pm$ 0.0\% & 71.1 $\pm$ 0.1\% & 53.3 $\pm$ 0.0\% \\
Fish & 72.5 $\pm$ 0.0\% & 54.7 $\pm$ 0.0\% & 71.7 $\pm$ 0.1\% & 53.3 $\pm$ 0.0\% \\
\midrule
\textbf{LISA (ours)} & 71.6 $\pm$ 0.4\% & \textbf{55.1 $\pm$ 0.6\%} & 70.8 $\pm$ 0.3\% & \textbf{54.7 $\pm$ 0.0\%} \\
\bottomrule
\end{tabular}
\end{center}
\end{table*}
\section{Omitted Algorithms}\label{Appendix_Algorithm}

\begin{algorithm}[htb!]
   \caption{Random Search}
   \label{Algorithm_random_search}
\begin{algorithmic}
   \STATE {\bfseries Input:} Objective function $f\colon Y\to\RB$, where $Y$ is a compact subset of $\RB^m$.
   \STATE Sample $y_1,...,y_M\stackrel{i.i.d.}{\sim}\mathrm{Unif}(Y)$.
   \STATE {\bfseries RETURN} $\hat y=y_{i_0}$, $i_0=\argmax_{i\in\{1,...,M\}}f(y_i)$.
\end{algorithmic}
\end{algorithm}

\begin{algorithm}[htb!]
   \caption{Projected Subgradient Ascent}
   \label{Algorithm_projected_GD}
\begin{algorithmic}
   \STATE {\bfseries Input:} Objective function $f\colon Y\to\RB$, where $Y$ is a compact subset of $\RB^m$, the maximum number of iterations $T_{PGA}$.
   \STATE Initialize: set $y_0$ as an arbitrary element of $Y$, learning rate $\alpha=\frac{\mathrm{diam}(Y)}{L_y\sqrt{T}}$.
   \FOR {$t=0,...,T_{PGA}-1$}
   \STATE $t_{t+0.5}=y_t+\alpha g_t$, where $g_t$ is a subgradient of $f$ at $y_t$.
   \STATE $y_{t+1}=\argmin_{y\in Y} \|y-y_{t+0.5}\|$.
   \ENDFOR
   \STATE {\bfseries RETURN} $\hat y=\frac{1}{T_{PGA}}\sum_{t=1}^{T_{PGA}} y_t$.
\end{algorithmic}
\end{algorithm}

\begin{algorithm}[htb!]
   \caption{Sample-based NPG}
   \label{Algorithm_sample_based_NPG}
\begin{algorithmic}
   \STATE {\bfseries Input:} state space $\gS$, action space $\gA$, a criterion function $b$ (Can be the reward function $r$ or cost function $c_y$ for some fixed $y$), discount factor $\gamma$, policy $\pi_\theta$, number of paths $K_{sgd}$, fixed horizon $H$, upper bound of parameters' norm $W$, learning rate $\{\eta_k\}$, weight $\{\gamma_k\}$
   \FOR{$k=0$ {\bfseries to} $K_{sgd}-1$}
      \STATE Draw $(s,a)\sim \nu$, with $\nu(s,a)=d^{\pi_\theta}(s)\pi_\theta(a|s)$.
   \STATE Execute policy $\pi_\theta$ from $(s,a)$ for $H$ steps, then construct the estimators as
   $$
   \begin{aligned}
   \widehat Q^{\pi_\theta}(s,a)=\sum^{H-1}_{k=0} \gamma^k b(s_k,a_k),\ \text{where } (s_0,a_0)=(s,a).
   \end{aligned}
   $$
   \STATE Execute policy $\pi_\theta$ from $s$ for $H$ steps, then construct the estimators as
   $$
   \widehat V^{\pi_\theta}(s,a)=\sum^{H-1}_{k=0} \gamma^k b(s_k,a_k),\ \text{where } s_0=s.
   $$
   \STATE Set $\widehat A^{\pi_\theta}(s,a)=\widehat Q^{\pi_\theta}(s,a)-\widehat V^{\pi_\theta}(s)$.
   \STATE Perform an iteration of projected SGD: $w^{(k+1)}=\operatorname{Proj}_{B(0,W,\|\cdot\|_2)}(w^{(k)}-\eta_k G^{(k)})$ with
   $$
   \begin{aligned}
      G^{(k)}&=2({w^{(k)}}^\top\nabla_\theta\log\pi_\theta(a|s)-\widehat A^{\pi_\theta}(s,a))\nabla_\theta\log\pi_\theta(a|s),\\
   \end{aligned}
   $$
   and $B(0,W,\|\cdot\|_2):=\{w\in\RB^d|\|w\|_2\leq W\}$.
   \ENDFOR
   \STATE {\bfseries RETURN} $\sum_{k=1}^K \gamma_k w^{(k)}$ as a NPG update direction at $\pi_\theta$ w.r.t. criterion function $b$.
\end{algorithmic}
\end{algorithm}




\end{document}
