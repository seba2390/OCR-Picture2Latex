\documentclass[11pt]{article}

%% \usepackage[utf8]{inputenc} % allow utf-8 input
%% \usepackage[T1]{fontenc}    % use 8-bit T1 fonts

\usepackage{geometry}
\geometry{verbose,tmargin=1in,bmargin=1in,lmargin=1in,rmargin=1in}
\usepackage{setspace}
\usepackage{amsmath, amssymb, amsfonts, bm, mathtools}
\usepackage{amsthm}
\usepackage[dvipsnames]{xcolor}
\definecolor{darkblue}{rgb}{0,0,.5}
\usepackage{graphicx}
\usepackage{subfigure}

\usepackage[numbers]{natbib}

\usepackage[colorlinks=true,allcolors=darkblue]{hyperref}       % hyperlinks
\usepackage{url}            % simple URL typesetting
\usepackage{booktabs}       % professional-quality tables
\usepackage{amsfonts}       % blackboard math symbols
\usepackage{nicefrac}       % compact symbols for 1/2, etc.
\usepackage{microtype}      % microtypography

\allowdisplaybreaks

\usepackage{float}
\usepackage{multirow}
\usepackage{footnote}
\usepackage{dsfont}
% \usepackage{mathabx}

\usepackage{algorithm}
\usepackage{algorithmic}
\usepackage{nicefrac}

\usepackage{tikz}
\usepackage{overpic}

\usepackage{dsfont}
\usepackage{hyperref}
\usepackage[capitalize]{cleveref}
\usepackage{crossreftools}

\newcommand{\ind}{\mathds{1}}
\newcommand{\var}{\mathsf{Var}}
\newcommand{\E}{\mathbb{E}}
\newcommand{\mc}[1]{\mathcal{#1}}
\newcommand{\brc}[1]{\left\{{#1}\right\}}
\newcommand{\paren}[1]{\left({#1}\right)} % parentheses
\newcommand{\brk}[1]{\left[{#1}\right]} % bracket
\newcommand{\norm}[1]{\left\|{#1}\right\|} % norm
\newcommand{\abs}[1]{\left|{#1}\right|} % norm
\newcommand{\what}[1]{\widehat{#1}}
\newcommand{\sgn}{\mathsf{sign}}
\newcommand{\normal}{\mathsf{N}}
\newcommand{\bindist}{\mathsf{Binomial}}
\newcommand{\matrixnorm}[1]{\left|\!\left|\!\left|{#1}
	\right|\!\right|\!\right|} % Matrix norm with three bars

\newcommand{\opnorm}[1]{\matrixnorm{#1}_{\rm op}} % Operator norm with three bars
\newcommand{\<}{\langle} % Angle brackets
\renewcommand{\>}{\rangle}

\DeclareMathOperator*{\argmax}{argmax}
\DeclareMathOperator*{\argmin}{argmin}

\newcommand{\simiid}{\stackrel{\textup{iid}}{\sim}}

\newcommand{\cd}{\stackrel{d}{\rightarrow}}
\newcommand{\cp}{\stackrel{p}{\rightarrow}}
\newcommand{\cas}{\stackrel{a.s.}{\rightarrow}}

%%%%% NEW MATH DEFINITIONS %%%%%

\def\ceil#1{\lceil #1 \rceil}
\def\floor#1{\lfloor #1 \rfloor}

% Graph
\def\gA{{\mathcal{A}}}
\def\gB{{\mathcal{B}}}
\def\gC{{\mathcal{C}}}
\def\gD{{\mathcal{D}}}
\def\gE{{\mathcal{E}}}
\def\gF{{\mathcal{F}}}
\def\gG{{\mathcal{G}}}
\def\gH{{\mathcal{H}}}
\def\gI{{\mathcal{I}}}
\def\gJ{{\mathcal{J}}}
\def\gK{{\mathcal{K}}}
\def\gL{{\mathcal{L}}}
\def\gM{{\mathcal{M}}}
\def\gN{{\mathcal{N}}}
\def\gO{{\mathcal{O}}}
\def\gP{{\mathcal{P}}}
\def\gQ{{\mathcal{Q}}}
\def\gR{{\mathcal{R}}}
\def\gS{{\mathcal{S}}}
\def\gT{{\mathcal{T}}}
\def\gU{{\mathcal{U}}}
\def\gV{{\mathcal{V}}}
\def\gW{{\mathcal{W}}}
\def\gX{{\mathcal{X}}}
\def\gY{{\mathcal{Y}}}
\def\gZ{{\mathcal{Z}}}

\def\CB{{\mathbb C}}
\def\RB{{\mathbb R}}
\def\EB{{\mathbb E}}
\def\ZB{{\mathbb Z}}
\def\PB{{\mathbb P}}
\def\OB{{\mathbb O}}

\def\st{\mathsf{s.t.}}
\def\vect{\mathsf{vec}}
\def\etal{{\em et al.\/}\,}
\def\ie{{\em i.e.\/}}


\newcommand{\KL}{D_{\textup{KL}}}
% Wolfram Mathworld says $L^2$ is for function spaces and $\ell^2$ is for vectors
% But then they seem to use $L^2$ for vectors throughout the site, and so does
% wikipedia.

\makeatletter
\long\def\@makecaption#1#2{
  \vskip 0.8ex
  \setbox\@tempboxa\hbox{\small {\bf #1:} #2}
  \parindent 1.5em  %% How can we use the global value of this???
  \dimen0=\hsize
  \advance\dimen0 by -3em
  \ifdim \wd\@tempboxa >\dimen0
  \hbox to \hsize{
    \parindent 0em
    \hfil 
    \parbox{\dimen0}{\def\baselinestretch{0.96}\small
      {\bf #1.} #2
      %%\unhbox\@tempboxa
    } 
    \hfil}
  \else \hbox to \hsize{\hfil \box\@tempboxa \hfil}
  \fi
}
\makeatother

\newcommand{\lipnorm}[1]{\norm{#1}_{\textup{Lip}}}

\newcommand{\half}{\frac{1}{2}}

\newcommand{\conv}{\textup{Conv}}

\newcommand{\red}[1]{\textcolor{red}{#1}}
\newcommand{\blue}[1]{\textcolor{blue}{#1}}
\newcommand{\darkblue}[1]{\textcolor{darkblue}{#1}}

\newcommand{\Var}{\mathrm{Var}}

% \usepackage{enumerate}
\usepackage{enumitem}
\newtheorem{claim}{Claim}[section]
\newtheorem{lemma}[claim]{Lemma}
\newtheorem{assumption}{Assumption}
\newtheorem{theorem}{Theorem}
\newtheorem{proposition}{Proposition}
\newtheorem{corollary}{Corollary}
\theoremstyle{definition}
\newtheorem{definition}{Definition}
\newtheorem{example}{Example}
\theoremstyle{remark}
\newtheorem{remark}{Remark}


% \onehalfspacing

\usepackage{xr}
% \externaldocument{main}

\title{Semi-Infinitely Constrained Markov Decision Processes and Efficient Reinforcement Learning}

%% \title{How many labelers do you have? \\ A look at gold-standard labels and
%%   their weaknesses}

% \title{Modeling Aggregation and Uncertainty in Modern Data Analysis}

\author{Liangyu Zhang\thanks{Academy of Advanced Interdisciplinary Studies, Peking University; email: \texttt{zhangliangyu@pku.edu.cn}.} 
\and
Yang Peng\thanks{School of Mathematical Sciences, Peking University; email: \texttt{pengyang@pku.edu.cn}.}
\and
Wenhao Yang\thanks{Academy of Advanced Interdisciplinary Studies, Peking University; email: \texttt{yangwenhaosms@pku.edu.cn}.}
\and
Zhihua Zhang\thanks{School of Mathematical Sciences, Peking University; email: \texttt{zhzhang@math.pku.edu.cn}.}
}


\linespread{1.5}
\begin{document}
\maketitle

\begin{abstract}
  We propose a novel generalization of constrained Markov decision processes (CMDPs) that we call the \emph{semi-infinitely constrained Markov decision process} (SICMDP).
% In particular, we in a SICMDP model impose 
Particularly, %in an SICMDP model, 
we consider a continuum of constraints instead of a finite number of constraints as in the case of ordinary CMDPs.
We also devise two reinforcement learning algorithms
for SICMDPs that we call SI-CRL and SI-CPO.
SI-CRL is a model-based reinforcement learning algorithm.
Given an estimate of the transition model, we first transform the reinforcement learning problem into a linear semi-infinitely programming (LSIP) problem and then use the dual exchange method in the LSIP literature to solve it.
SI-CPO is a policy optimization algorithm.
Borrowing the ideas from the cooperative stochastic approximation approach, we make alternative updates to the policy parameters to maximize the reward or minimize the cost.
To the best of our knowledge, we are the first to apply tools from semi-infinitely programming (SIP) to solve constrained reinforcement learning problems.
We present theoretical analysis for SI-CRL and SI-CPO, identifying their iteration complexity and sample complexity.
We also conduct extensive numerical examples to illustrate the SICMDP model and demonstrate that our proposed algorithms are able to solve complex sequential decision-making tasks leveraging modern deep reinforcement learning techniques.
\end{abstract}

% \tableofcontents
\section{Introduction}
% !TEX root = ../arxiv.tex

Unsupervised domain adaptation (UDA) is a variant of semi-supervised learning \cite{blum1998combining}, where the available unlabelled data comes from a different distribution than the annotated dataset \cite{Ben-DavidBCP06}.
A case in point is to exploit synthetic data, where annotation is more accessible compared to the costly labelling of real-world images \cite{RichterVRK16,RosSMVL16}.
Along with some success in addressing UDA for semantic segmentation \cite{TsaiHSS0C18,VuJBCP19,0001S20,ZouYKW18}, the developed methods are growing increasingly sophisticated and often combine style transfer networks, adversarial training or network ensembles \cite{KimB20a,LiYV19,TsaiSSC19,Yang_2020_ECCV}.
This increase in model complexity impedes reproducibility, potentially slowing further progress.

In this work, we propose a UDA framework reaching state-of-the-art segmentation accuracy (measured by the Intersection-over-Union, IoU) without incurring substantial training efforts.
Toward this goal, we adopt a simple semi-supervised approach, \emph{self-training} \cite{ChenWB11,lee2013pseudo,ZouYKW18}, used in recent works only in conjunction with adversarial training or network ensembles \cite{ChoiKK19,KimB20a,Mei_2020_ECCV,Wang_2020_ECCV,0001S20,Zheng_2020_IJCV,ZhengY20}.
By contrast, we use self-training \emph{standalone}.
Compared to previous self-training methods \cite{ChenLCCCZAS20,Li_2020_ECCV,subhani2020learning,ZouYKW18,ZouYLKW19}, our approach also sidesteps the inconvenience of multiple training rounds, as they often require expert intervention between consecutive rounds.
We train our model using co-evolving pseudo labels end-to-end without such need.

\begin{figure}[t]%
    \centering
    \def\svgwidth{\linewidth}
    \input{figures/preview/bars.pdf_tex}
    \caption{\textbf{Results preview.} Unlike much recent work that combines multiple training paradigms, such as adversarial training and style transfer, our approach retains the modest single-round training complexity of self-training, yet improves the state of the art for adapting semantic segmentation by a significant margin.}
    \label{fig:preview}
\end{figure}

Our method leverages the ubiquitous \emph{data augmentation} techniques from fully supervised learning \cite{deeplabv3plus2018,ZhaoSQWJ17}: photometric jitter, flipping and multi-scale cropping.
We enforce \emph{consistency} of the semantic maps produced by the model across these image perturbations.
The following assumption formalises the key premise:

\myparagraph{Assumption 1.}
Let $f: \mathcal{I} \rightarrow \mathcal{M}$ represent a pixelwise mapping from images $\mathcal{I}$ to semantic output $\mathcal{M}$.
Denote $\rho_{\bm{\epsilon}}: \mathcal{I} \rightarrow \mathcal{I}$ a photometric image transform and, similarly, $\tau_{\bm{\epsilon}'}: \mathcal{I} \rightarrow \mathcal{I}$ a spatial similarity transformation, where $\bm{\epsilon},\bm{\epsilon}'\sim p(\cdot)$ are control variables following some pre-defined density (\eg, $p \equiv \mathcal{N}(0, 1)$).
Then, for any image $I \in \mathcal{I}$, $f$ is \emph{invariant} under $\rho_{\bm{\epsilon}}$ and \emph{equivariant} under $\tau_{\bm{\epsilon}'}$, \ie~$f(\rho_{\bm{\epsilon}}(I)) = f(I)$ and $f(\tau_{\bm{\epsilon}'}(I)) = \tau_{\bm{\epsilon}'}(f(I))$.

\smallskip
\noindent Next, we introduce a training framework using a \emph{momentum network} -- a slowly advancing copy of the original model.
The momentum network provides stable, yet recent targets for model updates, as opposed to the fixed supervision in model distillation \cite{Chen0G18,Zheng_2020_IJCV,ZhengY20}.
We also re-visit the problem of long-tail recognition in the context of generating pseudo labels for self-supervision.
In particular, we maintain an \emph{exponentially moving class prior} used to discount the confidence thresholds for those classes with few samples and increase their relative contribution to the training loss.
Our framework is simple to train, adds moderate computational overhead compared to a fully supervised setup, yet sets a new state of the art on established benchmarks (\cf \cref{fig:preview}).

\section{Related work}
\section{Related Work}\label{sec:related}
 
The authors in \cite{humphreys2007noncontact} showed that it is possible to extract the PPG signal from the video using a complementary metal-oxide semiconductor camera by illuminating a region of tissue using through external light-emitting diodes at dual-wavelength (760nm and 880nm).  Further, the authors of  \cite{verkruysse2008remote} demonstrated that the PPG signal can be estimated by just using ambient light as a source of illumination along with a simple digital camera.  Further in \cite{poh2011advancements}, the PPG waveform was estimated from the videos recorded using a low-cost webcam. The red, green, and blue channels of the images were decomposed into independent sources using independent component analysis. One of the independent sources was selected to estimate PPG and further calculate HR, and HRV. All these works showed the possibility of extracting PPG signals from the videos and proved the similarity of this signal with the one obtained using a contact device. Further, the authors in \cite{10.1109/CVPR.2013.440} showed that heart rate can be extracted from features from the head as well by capturing the subtle head movements that happen due to blood flow.

%
The authors of \cite{kumar2015distanceppg} proposed a methodology that overcomes a challenge in extracting PPG for people with darker skin tones. The challenge due to slight movement and low lighting conditions during recording a video was also addressed. They implemented the method where PPG signal is extracted from different regions of the face and signal from each region is combined using their weighted average making weights different for different people depending on their skin color. 
%

There are other attempts where authors of \cite{6523142,6909939, 7410772, 7412627} have introduced different methodologies to make algorithms for estimating pulse rate robust to illumination variation and motion of the subjects. The paper \cite{6523142} introduces a chrominance-based method to reduce the effect of motion in estimating pulse rate. The authors of \cite{6909939} used a technique in which face tracking and normalized least square adaptive filtering is used to counter the effects of variations due to illumination and subject movement. 
The paper \cite{7410772} resolves the issue of subject movement by choosing the rectangular ROI's on the face relative to the facial landmarks and facial landmarks are tracked in the video using pose-free facial landmark fitting tracker discussed in \cite{yu2016face} followed by the removal of noise due to illumination to extract noise-free PPG signal for estimating pulse rate. 

Recently, the use of machine learning in the prediction of health parameters have gained attention. The paper \cite{osman2015supervised} used a supervised learning methodology to predict the pulse rate from the videos taken from any off-the-shelf camera. Their model showed the possibility of using machine learning methods to estimate the pulse rate. However, our method outperforms their results when the root mean squared error of the predicted pulse rate is compared. The authors in \cite{hsu2017deep} proposed a deep learning methodology to predict the pulse rate from the facial videos. The researchers trained a convolutional neural network (CNN) on the images generated using Short-Time Fourier Transform (STFT) applied on the R, G, \& B channels from the facial region of interests.
The authors of \cite{osman2015supervised, hsu2017deep} only predicted pulse rate, and we extended our work in predicting variance in the pulse rate measurements as well.

All the related work discussed above utilizes filtering and digital signal processing to extract PPG signals from the video which is further used to estimate the PR and PRV.  %
The method proposed in \cite{kumar2015distanceppg} is person dependent since the weights will be different for people with different skin tone. In contrast, we propose a deep learning model to predict the PR which is independent of the person who is being trained. Thus, the model would work even if there is no prior training model built for that individual and hence, making our model robust. 

%
\section{The SICMDP Model}
Online convex optimization with memory has emerged as an important and challenging area with a wide array of applications, see \citep{lin2012online,anava2015online,chen2018smoothed,goel2019beyond,agarwal2019online,bubeck2019competitively} and the references therein.  Many results in this area have focused on the case of online optimization with switching costs (movement costs), a form of one-step memory, e.g., \citep{chen2018smoothed,goel2019beyond,bubeck2019competitively}, though some papers have focused on more general forms of memory, e.g., \citep{anava2015online,agarwal2019online}. In this paper we, for the first time, study the impact of feedback delay and nonlinear switching cost in online optimization with switching costs. 

An instance consists of a convex action set $\mathcal{K}\subset\mathbb{R}^d$, an initial point $y_0\in\mathcal{K}$, a sequence of non-negative convex cost functions $f_1,\cdots,f_T:\mathbb{R}^d\to\mathbb{R}_{\ge0}$, and a switching cost $c:\mathbb{R}^{d\times(p+1)}\to\mathbb{R}_{\ge0}$. To incorporate feedback delay, we consider a situation where the online learner only knows the geometry of the hitting cost function at each round, i.e., $f_t$, but that the minimizer of $f_t$ is revealed only after a delay of $k$ steps, i.e., at time $t+k$.  This captures practical scenarios where the form of the loss function or tracking function is known by the online learner, but the target moves over time and measurement lag means that the position of the target is not known until some time after an action must be taken. 
To incorporate nonlinear (and potentially nonconvex) switching costs, we consider the addition of a known nonlinear function $\delta$ from $\mathbb{R}^{d\times p}$ to $\mathbb{R}^d$ to the structured memory model introduced previously.  Specifically, we have
\begin{align}
c(y_{t:t-p}) = \frac{1}{2}\|y_t-\delta(y_{t-1:t-p})\|^2,    \label{e.newswitching}
\end{align}
where we use $y_{i:j}$ to denote either $\{y_i, y_{i+1}, \cdots, y_j\}$ if $i\leq j$, or  $\{y_i, y_{i-1}, \cdots, y_j\}$ if $i > j$ throughout the paper. Additionally, we use $\|\cdot\|$ to denote the 2-norm of a vector or the spectral norm of a matrix.

In summary, we consider an online agent that interacts with the environment as follows:
% \begin{inparaenum}[(i)] 
\begin{enumerate}%[leftmargin=*]
    \item The adversary reveals a function $h_t$, which is the geometry of the $t^\mathrm{th}$ hitting cost, and a point $v_{t-k}$, which is the minimizer of the $(t-k)^\mathrm{th}$ hitting cost. Assume that $h_t$ is $m$-strongly convex and $l$-strongly smooth, and that $\arg\min_y h_t(y)=0$.
    \item The online learner picks $y_t$ as its decision point at time step $t$ after observing $h_t,$  $v_{t-k}$.
    \item The adversary picks the minimizer of the hitting cost at time step $t$: $v_t$. 
    \item The learner pays hitting cost $f_t(y_t)=h_t(y_t-v_t)$ and switching cost $c(y_{t:t-p})$ of the form \eqref{e.newswitching}.
\end{enumerate}

The goal of the online learner is to minimize the total cost incurred over $T$ time steps, $cost(ALG)=\sum_{t=1}^Tf_t(y_t)+c(y_{t:t-p})$, with the goal of (nearly) matching the performance of the offline optimal algorithm with the optimal cost $cost(OPT)$. The performance metric used to evaluate an algorithm is typically the \textit{competitive ratio} because the goal is to learn in an environment that is changing dynamically and is potentially adversarial. Formally, the competitive ratio (CR) of the online algorithm is defined as the worst-case ratio between the total cost incurred by the online learner and the offline optimal cost: $CR(ALG)=\sup_{f_{1:T}}\frac{cost(ALG)}{cost(OPT)}$.

It is important to emphasize that the online learner decides $y_t$ based on the knowledge of the previous decisions $y_1\cdots y_{t-1}$, the geometry of cost functions $h_1\cdots h_t$, and the delayed feedback on the minimizer $v_1\cdots v_{t-k}$. Thus, the learner has perfect knowledge of cost functions $f_1\cdots f_{t-k}$, but incomplete knowledge of $f_{t-k+1}\cdots f_t$ (recall that $f_t(y)=h_t(y-v_t)$).

Both feedback delay and nonlinear switching cost add considerable difficulty for the online learner compared to versions of online optimization studied previously. Delay hides crucial information from the online learner and so makes adaptation to changes in the environment more challenging. As the learner makes decisions it is unaware of the true cost it is experiencing, and thus it is difficult to track the optimal solution. This is magnified by the fact that nonlinear switching costs increase the dependency of the variables on each other. It further stresses the influence of the delay, because an inaccurate estimation on the unknown data, potentially magnifying the mistakes of the learner. 

The impact of feedback delay has been studied previously in online learning settings without switching costs, with a focus on regret, e.g., \citep{joulani2013online,shamir2017online}.  However, in settings with switching costs the impact of delay is magnified since delay may lead to not only more hitting cost in individual rounds, but significantly larger switching costs since the arrival of delayed information may trigger a very large chance in action.  To the best of our knowledge, we give the first competitive ratio for delayed feedback in online optimization with switching costs. 

We illustrate a concrete example application of our setting in the following.

\begin{example}[Drone tracking problem]
\label{example:drone} \emph{
Consider a drone with vertical speed $y_t\in\mathbb{R}$. The goal of the drone is to track a sequence of desired speeds $y^d_1,\cdots,y^d_T$ with the following tracking cost:}
\begin{equation}
    \sum_{t=1}^T \frac{1}{2}(y_t-y^d_t)^2 + \frac{1}{2}(y_t-y_{t-1}+g(y_{t-1}))^2,
\end{equation}
\emph{where $g(y_{t-1})$ accounts for the gravity and the aerodynamic drag. One example is $g(y)=C_1+C_2\cdot|y|\cdot y$ where $C_1,C_2>0$ are two constants~\cite{shi2019neural}. Note that the desired speed $y_t^d$ is typically sent from a remote computer/server. Due to the communication delay, at time step $t$ the drone only knows $y_1^d,\cdots,y_{t-k}^d$.}

\emph{This example is beyond the scope of existing results in online optimization, e.g.,~\cite{shi2020online,goel2019beyond,goel2019online}, because of (i) the $k$-step delay in the hitting cost $\frac{1}{2}(y_t-y_t^d)$ and (ii) the nonlinearity in the switching cost $\frac{1}{2}(y_t-y_{t-1}+g(y_{t-1}))^2$ with respective to $y_{t-1}$. However, in this paper, because we directly incorporate the effect of delay and nonlinearity in the algorithm design, our algorithms immediately provide constant-competitive policies for this setting.}
\end{example}


\subsection{Related Work}
This paper contributes to the growing literature on online convex optimization with memory.  
Initial results in this area focused on developing constant-competitive algorithms for the special case of 1-step memory, a.k.a., the Smoothed Online Convex Optimization (SOCO) problem, e.g., \citep{chen2018smoothed,goel2019beyond}. In that setting, \citep{chen2018smoothed} was the first to develop a constant, dimension-free competitive algorithm for high-dimensional problems.  The proposed algorithm, Online Balanced Descent (OBD), achieves a competitive ratio of $3+O(1/\beta)$ when cost functions are $\beta$-locally polyhedral.  This result was improved by \citep{goel2019beyond}, which proposed two new algorithms, Greedy OBD and Regularized OBD (ROBD), that both achieve $1+O(m^{-1/2})$ competitive ratios for $m$-strongly convex cost functions.  Recently, \citep{shi2020online} gave the first competitive analysis that holds beyond one step of memory.  It holds for a form of structured memory where the switching cost is linear:
$
    c(y_{t:t-p})=\frac{1}{2}\|y_t-\sum_{i=1}^pC_iy_{t-i}\|^2,
$
with known $C_i\in\mathbb{R}^{d\times d}$, $i=1,\cdots,p$. If the memory length $p = 1$ and $C_1$ is an identity matrix, this is equivalent to SOCO. In this setting, \citep{shi2020online} shows that ROBD has a competitive ratio of 
\begin{align}
    \frac{1}{2}\left( 1 + \frac{\alpha^2 - 1}{m} + \sqrt{\Big( 1 + \frac{\alpha^2 - 1}{m}\Big)^2 + \frac{4}{m}} \right),
\end{align}
when hitting costs are $m$-strongly convex and $\alpha=\sum_{i=1}^p\|C_i\|$. 


Prior to this paper, competitive algorithms for online optimization have nearly always assumed that the online learner acts \emph{after} observing the cost function in the current round, i.e., have zero delay.  The only exception is \citep{shi2020online}, which considered the case where the learner must act before observing the cost function, i.e., a one-step delay.  Even that small addition of delay requires a significant modification to the algorithm (from ROBD to Optimistic ROBD) and analysis compared to previous work. 

As the above highlights, there is no previous work that addresses either the setting of nonlinear switching costs nor the setting of multi-step delay. However, the prior work highlights that ROBD is a promising algorithmic framework and our work in this paper extends the ROBD framework in order to address the challenges of delay and non-linear switching costs. Given its importance to our work, we describe the workings of ROBD in detail in Algorithm~\ref{robd}. 

\begin{algorithm}[t!]
  \caption{ROBD \citep{goel2019beyond}}
  \label{robd}
\begin{algorithmic}[1]
  \STATE {\bfseries Parameter:} $\lambda_1\ge0,\lambda_2\ge0$
  \FOR{$t=1$ {\bfseries to} $T$}
  \STATE {\bfseries Input:} Hitting cost function $f_t$, previous decision points $y_{t-p:t-1}$
  \STATE $v_t\leftarrow\arg\min_yf_t(y)$
  \STATE $y_t\leftarrow\arg\min_yf_t(y)+\lambda_1c(y,y_{t-1:t-p})+\frac{\lambda_2}{2}\|y-v_t\|^2_2$
  \STATE {\bfseries Output:} $y_t$
  \ENDFOR
   
\end{algorithmic}
\end{algorithm}

Another line of literature that this paper contributes to is the growing understanding of the connection between online optimization and adaptive control. The reduction from adaptive control to online optimization with memory was first studied in \citep{agarwal2019online} to obtain a sublinear static regret guarantee against the best linear state-feedback controller, where the approach is to consider a disturbance-action policy class with some fixed horizon.  Many follow-up works adopt similar reduction techniques \citep{agarwal2019logarithmic, brukhim2020online, gradu2020adaptive}. A different reduction approach using control canonical form is proposed by \citep{li2019online} and further exploited by \citep{shi2020online}. Our work falls into this category.  The most general results so far focus on Input-Disturbed Squared Regulators, which can be reduced to online convex optimization with structured memory (without delay or nonlinear switching costs).  As we show in \Cref{Control}, the addition of delay and nonlinear switching costs leads to a significant extension of the generality of control models that can be reduced to online optimization. 
\section{Algorithms}
In this section, we present two reinforcement learning algorithms called semi-infinitely constrained reinforcement learning (SI-CRL) and semi-infinitely constrained policy optimization (SI-CPO), respectively.
SI-CRL is a model-based reinforcement learning algorithm that can solve tabular SICMDP in a sample-efficient way.
The SI-CPO algorithm is a policy optimization algorithm and it works for large-scale SICMDPs where we can use complex function approximators such as deep neural networks to approximate the policy and the value function.
\subsection{The SI-CRL Algorithm}\label{Section_SICRL}
From a high-level point of view, the SI-CRL algorithm is a semi-infinite version of the algorithms proposed in \cite{ijcai2021-347, efroni2020explorationexploitation}.
In the first stage, SI-CRL takes an offline dataset $\{(s_i, a_i, s_i^\prime)|i=1, 2, \ldots, m\}$ as input and generates an empirical estimate $\widehat P$ of the true transition dynamic $P$.
Then the algorithm constructs a confidence set (the optimistic set) according to $\widehat P$ that would cover the true SICMDP with high probability.
For each policy $\pi$ we would only view its return as the largest possible return in SICMDPs in the confidence set.
% Since when the true SICMDP lies in the confidence set we would overestimate the return of each policy $\pi$, 
This method is also called the optimistic approach.
In the second stage, we reformulate the problem as an LSIP problem and find the optimistic policy $\hat \pi$ using an LSIP solver.
It can be shown that the resulting policy $\hat\pi$ is guaranteed to be nearly optimal, and the theoretical analysis can be found in Section \ref{Section_Theory_SICRL}.

Now we give a more detailed description of SI-CRL.
First, the empirical estimate $\widehat P$ is calculated as:
$\widehat P(s^\prime|s, a):=\frac{n(s, a, s^\prime)}{\max\paren{1,n(s,a)}}$,
where $n(s,a,s^\prime) :=\sum_{i=1}^m \mathbf{1}\{s_i=s, a_i=a, s_i^\prime=s^\prime\}$ and $n(s,a)=\sum_{s^\prime} n(s,a,s^\prime)$.
The reason why we do not directly plug $\widehat P$ into Problem \eqref{Problem_SICMDP_LSIP} and solve the resulting LSIP problem is due to the fact that there is no guarantee that the LSIP problem w.r.t.\ $\widehat P$ is feasible.
To address this issue, we construct an optimistic set $M_\delta$ such that with high probability the true SICMDP $M$ lies in $M_\delta$.
In particular, $M_\delta$ is defined via the empirical Bernstein's bound and the Hoeffding's bound \citep{LATTIMORE2014125}:
%\begingroup
%\small
\begin{align*}
M_\delta :=& \Big\{\langle \gS,\gA,Y,P^\prime,r,c,u,\mu, \gamma \rangle\colon  |P^\prime(s^\prime|s,a)-\widehat P(s^\prime|s,a)| \leq d_\delta(s,a,s^\prime), \forall s, s^\prime\in \gS, a\in \gA \Big\},
\end{align*}
%\endgroup
where 
% $$\begin{aligned}
% d_\delta(s,a,s^\prime):=&\min\bigg\{\sqrt{[2\widehat P(s^\prime|s,a)(1-\widehat P(s^\prime|s,a))\log(4/\delta)]/n(s,a,s^\prime)}\\
% &+4\log (4/\delta)/n(s,a,s^\prime), \sqrt{\log (2/\delta)/2n(s,a,s^\prime)}\bigg\}.
% \end{aligned}
% $$
\begin{align*}
d_\delta(s,a,s^\prime):=&\min\left\{\sqrt{\frac{2\widehat P(s^\prime|s,a)(1 {-} \widehat P(s^\prime|s,a))\log(4/\delta)}{n(s,a,s^\prime)}}+\frac{4\log (4/\delta)}{n(s,a,s^\prime)}, \; \sqrt{\frac{\log (2/\delta)}{2n(s,a,s^\prime)}}\right\}.
\end{align*}


The next step is to solve the optimistic planning problem:
\begin{equation}\label{Problem_Optimistic}
\begin{aligned}
\max_{M^\prime\in M_\delta,\pi}\ V_r^{\pi,M^\prime}(\mu),\quad
\text{s.t.}\ V_{c_y}^{\pi,M^\prime}(\mu) \leq u_y,\ \forall y\in Y,
\end{aligned}
\end{equation}
where the superscript $M^\prime$ denotes that the expectation is taken w.r.t.\ SICMDP $M^\prime$.
\begin{theorem}\label{Theorem_Feasible}
Suppose $n\geq 3$. With probability at least $1-2|\gS|^2|\gA|\delta$, we have that $M\in M_\delta$, and Problem (\ref{Problem_Optimistic}) is feasible.
\end{theorem}

\proof {Proof of Theorem~\ref{Theorem_Feasible}}
See Appendix~\ref{Appendix_Proofs_4}.
\endproof

Note that the optimization variables include both $M^\prime$ and $\pi$, and LSIP reformulations like Problem (\ref{Problem_SICMDP_LSIP}) would no longer be possible. 
Instead, we shall introduce the state-action-state occupancy measure $z(s,a,s^\prime)$.
In particular, assuming $z_{P,\pi}(s,a,s^\prime):=P(s^\prime|s,a)q_\pi(s,a)$, we have $P(s^\prime|s,a)=\frac{z_{P,\pi}(s,a,s^\prime)}{\sum_{x\in \gS}z_{P,\pi}(s,a,x)}$, and $\pi(a|s)=\frac{\sum_{s^\prime\in \gS}z_{P,\pi}(s,a,s^\prime)}{\sum_{s^\prime\in \gS,a^\prime\in \gA}z_{P,\pi}(s,a^\prime,s^\prime)}$. 
Problem (\ref{Problem_Optimistic}) can be reformulated as the following extended LSIP problem:

\begingroup
\small
\begin{equation}\label{Problem_Optimistic_ELSIP}
\begin{aligned}
    \max_{z}\ &\sum_{s, a,s^\prime}z(s,a,s^\prime)r(s,a) \\
    \text{s.t.}\ &\frac{1}{1-\gamma}\sum_{s, a,s^\prime}z(s,a,s^\prime)c_y(s,a)\leq u_y,\ \forall y\in Y, \\
    &z(s,a,s^\prime)\leq (\widehat P(s^\prime|s,a)+d_\delta(s,a,s^\prime))\sum_{x\in \gS} z(s,a,x), \forall s,s^\prime,\ a\in \gA, \\
    &z(s,a,s^\prime)\geq (\widehat P(s^\prime|s,a)-d_\delta(s,a,s^\prime))\sum_{x\in \gS} z(s,a,x), \forall s,s^\prime\in \gS,\ a\in \gA, \\
    &\sum_{x\in \gS,b\in \gA}z(s,b,x)=(1-\gamma)\mu(s)+\gamma\sum_{x\in \gS,b\in \gA}z(x,b,s), \forall s\in \gS, \\
    &z\succeq 0.
\end{aligned}
\end{equation}
\endgroup

However, compared to LP problems, LSIP problems are typically harder to solve and there are no all-purpose LSIP solvers.
Here, we choose the simple yet effective dual exchange methods \citep{Hu1990,reemtsen1998numerical} to solve Problem~\ref{Problem_Optimistic_ELSIP}.
The SI-CRL algorithm can be summarized in Algorithm~\ref{Algorithm_SICRL}.
A key ingredient of Algorithm~\ref{Algorithm_SICRL} is solving the inner-loop optimization problem 
$$
\max_{y\in Y} \sum_{s, a,s^\prime}z(s,a,s^\prime)c_y(s,a)-u_y.
$$
We can obtain different versions of SI-CRL algorithm by choosing different optimization subroutines to solve the inner-loop problem above. 
If $c_y$ and $u_y$ satisfy conditions like concavity and smoothness, then the inner problem can be solved using methods like projected subgradient ascent \citep{bubeck2015convex}.
If the inner problem is ill-posed, we may still solve it using methods like random search \citep{solis1981minimization, andradottir2015review}.
\begin{algorithm}[htb]
   \caption{SI-CRL}
   \label{Algorithm_SICRL}
\begin{algorithmic}
   \STATE {\bfseries Input:} state space $\gS$, action space $\gA$, dataset $\{(s_i,a_i,s_i^\prime)|i=1,2,...,m\}$, reward function $r$, a continuum of cost function $c$, index set $Y$, value for constraints $u$, discount factor $\gamma$, tolerance $\eta$, maximum iteration number $T$.
   \FOR{each $(s,a,s^\prime)$ tuple}
   \STATE Set $\widehat P(s^\prime|s,a):=\frac{\sum_{i=1}^m \ind\{s_i=s,a_i=a,s_i^\prime=s^\prime\}}{\max\paren{1,\sum_{i=1}^m \ind\{s_i=s,a_i=a\}}}$
   \ENDFOR
   \STATE Initialize $Y_0=\{y_0\}$
   \FOR{$t=1$ {\bfseries to} $T$}
   \STATE Use an LP solver to solve a finite version of Problem (\ref{Problem_Optimistic_ELSIP}) by only considering constraints in $Y_0$ and store the solution as $z^{(t)}$.
   \STATE Find $y^{(t)}\approx\argmax_{y\in Y} \sum_{s, a,s^\prime}z^{(t)}(s,a,s^\prime)c_y(s,a)-u_y$.
   \IF {$\sum_{s, a,s^\prime}z(s,a,s^\prime)c_{y^{(t)}}(s,a)-u_{y^{(t)}} \leq\eta$}
   \STATE  Set $z^{(T)}=z^{(t)}$.
   \STATE  {\bfseries BREAK}
   \ENDIF
   \STATE Add $y^{(t)}$ to $Y_0$.
   \ENDFOR
   \FOR{each $(s,a)$ pair}
   \STATE Set $\hat\pi(a|s)=\frac{\sum_{s^\prime}z^{(T)}(s,a,s^\prime)}{\sum_{s^\prime,a^\prime}z^{(T)}(s,a^\prime,s^\prime)}$.
   \ENDFOR
   \STATE {\bfseries RETURN} $\hat{\pi}$.
\end{algorithmic}
\end{algorithm}



\subsection{The SI-CPO Algorithm}\label{Section_SICPO}
In SI-CPO, we borrow ideas from the cooperative stochastic approximation \citep{lan2020algorithms, wei2020comirror} to deal with the infinitely many constraints.
At a certain iteration, the SI-CPO algorithm first determines whether the constraint violation is below some tolerance or not.
It then performs a single step of policy optimization along the direction of maximizing the value of reward if the constraint violation is below some tolerance;
or performs a single step of policy optimization along the direction of minimizing the value of some cost corresponding to a violated constraint.

We now describe the SI-CPO algorithm in more detail.
We follow the convention to define the parameterized policy class as $\{\pi_\theta,\theta\in\Theta\subset\RB^d\}$ and use $\pi^{(t)}$ in short of $\pi_{\theta^{(t)}}$, $V_\diamond^{(t)}$ in short of $V_\diamond^{\pi^{(t)}}$ for ease of notation.
Here $\diamond$ represents either the reward $r$ or some cost $c_y$.
Suppose at the $t$-th iteration our policy parameter is $\theta^{(t)}$, then we first construct an estimate $\widehat V^{(t)}_{c_y}(\mu)$ using some policy evaluation subroutine.
Next, we are to solve a subproblem using some optimization subroutine
$$
y^{(t)}=\argmax_y\ \widehat V_{c_y}^{\pi^{(t)}}(\mu)-u_y.
$$
If ${\widehat V_{c^{(t)}}^{\pi^{(t)}}(\mu)-u_{y^{(t)}}\leq \eta}$, where $c^{(t)}:=c_{y^{(t)}}$ and $\eta> 0$ is a threshold of tolerance, we say the constraint violation is small and add the time index $t$ to the ``good set" $\gB$.
Then we perform a step of update with a policy optimization subroutine to maximize the value of reward $V_r^{(t)}(\mu)$ to get $\theta^{(t+1)}$.
Else, we first add the time index $t$ to the ``bad set" $\gN$.
Next, we find the violated constraint ${V_{c^{(t)}}^{\pi^{(t)}}(\mu)-u_{y^{(t)}}>\eta}$, and perform a step of update with a policy optimization subroutine to minimize the value of cost $V_{c^{(t)}}^{\pi^{(t)}}(\mu)$ to get $\theta^{(t+1)}$.
After $T$ iterations, we draw $\hat\theta$ uniformly from the set ${\{\theta^{(t)},t\in\gB\}}$, as return the policy ${\hat\pi=\pi_{\hat\theta}}$.
The procedure of SI-CPO  is summarized in Algorithm~\ref{Algorithm_SICPO}.


We can get different instances of the SI-CPO algorithms by making different choices of the subroutines aforementioned.
Specifically, the policy optimization subroutine can be any policy optimization algorithm like policy gradient~(PG) \citep{sutton1999policy}, natural policy gradient~(NPG) \citep{kakade2001natural}, trust-region policy gradient~(TRPO) \cite{schulman2015trust}, or proximal policy optimization~(PPO) \citep{schulman2017proximal}.
The policy evaluation subroutine can be chosen as Monte-Carlo policy evaluation algorithms \citep{curtiss1954theoretical} or various TD-learning algorithms \citep{sutton1988learning, dann2014policy}.
We may also integrate the policy optimization subroutine and the policy evaluation subroutine into actor-critic-type algorithms \citep{konda1999actor}.
The optimization subroutine can be any optimization algorithm suitable for the problem instance, like the case in Algorithm~\ref{Algorithm_SICRL}.
% It is shown in Section~\ref{Section_Theory_SICPO} that if we use sample-based NPG \cite{agarwal2021theory} as the policy optimization subroutine, a finite-horizon Monte-Carlo estimator as the policy evaluation subroutine and either random search or projected subgradient ascent as the optimization subroutine, then $\hat\pi$ is guaranteed to be a globally near-optimal policy.
% We also empirically show in Section~\ref{Section_Experiment} the efficacy of SI-CPO in solving complex sequential decision-making problems with deep neural networks if we use PPO as the policy optimization subroutine, a value network trained in a TD style as the policy evaluation subroutine, and random search as the optimization subroutine.

\begin{algorithm}[htb]
   \caption{SI-CPO}
   \label{Algorithm_SICPO}
\begin{algorithmic}
   \STATE {\bfseries Input:} state space $\gS$, action space $\gA$, reward function $r$, a continuum of cost function $c$, index set $Y$, value for constraints $u$, discount factor $\gamma$, learning rate $\alpha$, tolerance $\eta$, maximum iteration number $T$.
   \STATE Initialize $\gB=\emptyset$, $\gN=\emptyset$, $\theta^{(0)}=\theta_0\in\Theta$.
   \FOR{$t=0,...,T-1$}
   \STATE Obtain $\widehat V_{c_y}^{\pi^{(t)}}(\mu)$ via a policy evaluation subroutine.
   \STATE Use an optimization subroutine to solve ${\max_y\ \widehat V_{c_y}^{\pi^{(t)}}(\mu)-u_y}$, and set ${y^{(t)}\approx\argmax_y \widehat V_{c_y}^{\pi^{(t)}}(\mu)-u_y}$, $c^{(t)}=c_{y^{(t)}}$.
   \IF {$\widehat V_{c^{(t)}}^{\pi^{(t)}}(\mu)-u_{y^{(t)}}\leq \eta$}
   \STATE  Perform a step of policy update to maximize $V_r^{\pi^{(t)}}(\mu)$ to get $\pi^{(t+1)}$. Specifically,
   ${\theta^{(t+1)}=\theta^{(t)}+\alpha\hat w^{(t)}}.$
%   \IF {$t\geq s$}
   \STATE Add $t$ to $\gB$
%   \ENDIF
   \ELSE 
   \STATE  Perform a step of policy update to minimize $V_{c^{(t)}}^{\pi^{(t)}}(\mu)$ to get $\pi^{(t+1)}$. Specifically, ${\theta^{(t+1)}=\theta^{(t)}-\alpha\hat w^{(t)}}.$
%   \IF {$t\geq s$}
   \STATE Add $t$ to $\gN$
%   \ENDIF
   \ENDIF
   \ENDFOR
   \STATE {\bfseries RETURN} $\hat\pi=\pi_{\hat\theta}$, where $\hat\theta\sim\mathrm{Unif}\left(\{\theta^{(t)}, t\in\gB\}\right)$.
\end{algorithmic}
\end{algorithm}


\section{Theoretical Analysis}
\subsection{Theoretical Analysis of SI-CRL}\label{Section_Theory_SICRL}
We give PAC-type bounds for SI-CRL under different settings.
The error of SI-CRL is decomposed into two parts: the optimization error from the fact that the solution of (\ref{Problem_Optimistic}) obtained by the dual exchange method is inexact and the statistical error from approximating Problem (\ref{Problem_SICMDP}) with Problem (\ref{Problem_Optimistic}).
On the optimization side, we show that if the inner maximization problem w.r.t.\ $y$ is solved via random search or projected subgradient ascent, the dual exchange method would produce an $\epsilon$-optimal solutions (see Definition \ref{Definition_Optimal_Solution})
%\footnotemark 
when the number of iterations $T=O\left(\left[\frac{\mathrm{diam}(Y)|\gS|^2|\gA|}{(1-\gamma)\epsilon}\right]^m \right)$.
% , where $L_y$ is the Lipschitz constant defined in Assumption \ref{Assumption_Lipschitz}.

On the statistical side, our goal is to determine how many samples are required to make SI-CRL an $(\epsilon, \delta)$-optimal~(see Definition \ref{Definition_PAC}) when Problem (\ref{Problem_Optimistic}) can be solved exactly, i.e., we want to find the sample complexity of SI-CRL  (see Definition \ref{Definition_PAC}).
%\footnotetext{The $(\epsilon, \delta)$-optimality would be defined in Definition \ref{Definition_PAC}}
We show that the sample complexity of SI-CRL is $\widetilde O\left(\frac{|\gS|^2|\gA|^2}{\epsilon^2(1-\gamma)^3}\right)$ if the dataset we use is generated by a generative model, and $\widetilde O\left(\frac{|\gS||\gA|}{\nu_{\min} \epsilon^2(1-\gamma)^3}\right)$ if the dataset we use is generated by a probability measure $\nu$ defined on the space $\gS\times \gA$ and $P(\cdot|s,a)$ as considered in \cite{chen2019information}.
% \wenhao{Why do we use $\Omega(\cdot)$ represent upper bound? $\Omega(\cdot)$ is used to show lower bound while $O(\cdot)$ is used to show upper bound.}
% \liangyu{Resolved.}
Here $\widetilde O$ means that all logarithm terms are discarded, and $\nu_{\min}:=\min_{\nu(s,a)>0}\nu(s,a)$.
% It can be noted that the order of our sample complexity bound increases by a factor of $|\gS||\gA|$ compared to that of ordinary discounted MDP \citep{azar2013minimax}.
We will present our theoretical analysis in more detail in the following part of this section.
%\footnotetext{The $\epsilon$-optimal solutions is defined in Definition \ref{Definition_Optimal_Solution}}


\subsubsection{Preliminaries}
%In addition to the notation defined in Sections \ref{Section_SICMDP} and \ref{Section_Algorithm}, 
% Given a stationary policy $\pi$, we define the value function $V^\pi(s)=\EB\paren*{\sum_{t=0}^\infty \gamma^t r(s_t,a_t)|s_0=s}$, $V^\pi=(V^\pi(s_1), \ldots, V^\pi(s_{|\gS|}))^\top\in \RB ^{|\gS|}$.
% Thus we have $V^\pi(\mu)=\mu^\top V^\pi$.
% Similarly, we define the expected cost $C_y^\pi(s)=\EB\paren*{\sum_{t=1}^\infty \gamma^t c_y(s_t,a_t)|s_0=s}$, $C_y^\pi=(C_y^\pi(s_1), \ldots, C_y^\pi(s_{|\gS|}))^\top\in \RB ^{|\gS|}$,  thus $C_y^\pi(\mu)=\mu^\top C_y^\pi$. 
% Suppose $\tilde\pi,\widetilde M$ are the solution of Problem ($\ref{Problem_Optimistic}$) and $\widetilde M=\langle \gS,\gA,Y,\widetilde P,r,c,u,\mu,\gamma \rangle$.
% For a given stationary policy $\pi$, $\widetilde V_r^\pi(\mu)$, $\widetilde V_{c_y}^\pi(\mu)$, to represent the value function, expected cost, of SICMDP $\widetilde M$, respectively.

Let $\pi^*$ denote the optimal policy.
An $(\epsilon,\delta)$-optimal policy is defined as follows. 
\begin{definition}\label{Definition_PAC}
An RL algorithm is called $(\epsilon,\delta)$-optimal for $\epsilon,\delta>0$ if with probability at least $1-\delta$ it can return a policy $\pi$ such that
$$
\begin{aligned}
V_r^{\pi^*}(\mu)-V_r^{\pi}(\mu) &\leq \epsilon;\quad
V_{c_y}^{\pi}(\mu) - u_y  \leq \epsilon, \forall y\in Y.
\end{aligned}
$$
% \wenhao{There is no randomness of a given policy $\pi$. $\delta$ can be removed. }
% \liangyu{Resolved.}
\end{definition}
An $\epsilon$-optimal solution of Problem (\ref{Problem_Optimistic}) is defined as \begin{definition}\label{Definition_Optimal_Solution}
A stationary policy $\hat\pi$ is called an $\epsilon$-optimal solution of Problem (\ref{Problem_Optimistic}) for $\epsilon>0$ if 
$$
\begin{aligned}
|V_r^{\hat\pi}(\mu)-V_r^{\tilde\pi}(\mu)| &\leq \epsilon \quad \mbox{and} \quad
|V_{c_y}^{\hat\pi}(\mu) - u_y|  \leq \epsilon,  \forall y\in Y \\
\end{aligned}
$$
hold simultaneously.
\end{definition}

Unless otherwise specified, we assume that $\forall (s,a)\in \gS\times \gA$, $c_y(s,a)$ is $L_y$-Lipschitz in $y$ w.r.t.\ $\|\cdot\|_\infty$.
We also assume that $u_y$ is $L_y$-Lipschitz in $y$ w.r.t.\ $\|\cdot\|_\infty$.
The assumptions can be formally stated as:
\begin{assumption}\label{Assumption_Lipschitz}
$c_y(s,a)$ and $u_y$ are Lipschitz in $y$ w.r.t.\ $\|\cdot\|_\infty$, i.e., $\exists L_y>0$ s.t. $\forall y,y^\prime\in Y, (s,a)\in \gS\times \gA, |c_y(s,a)-c_{y^\prime}(s,a)|\leq L_y\|y-y^\prime\|_\infty, 
|u_y-u_{y^\prime}|\leq L_y\|y-y^\prime\|_\infty$.
\end{assumption}
The Lipschitz assumption is usually necessary when dealing with a semi-infinitely constrained problem \citep{still2001discretization,Hu1990}.
And this assumption is indeed quite mild because $Y$ is a compact set.

We say an offline dataset $\{(s_i,a_i,s_i^\prime)|i=1, 2, \ldots, n\}$ to be generated by a generative model if we sample according to $P(\cdot|s,a)$ for each $(s,a)$-pair $n_0=n/|\gS||\gA|$ times and record the results in the dataset.
We say an offline dataset to be generated by probability measure $\nu$ and $P(\cdot|s,a)$ if $(s_i,a_i)\stackrel{i.i.d.}{\sim} \nu$ and $s_i^\prime\sim P(\cdot|s_i,a_i)$.

We solve the inner-loop problem in Algorithm~\ref{Algorithm_SICRL} with random search or projected gradient ascent.
The idea of random search is simple.
For an objective $f(y)$ defined on domain $Y$, we form a random grid of $Y$ consisting of $M$ grid points and select the grid point with the largest objective value.
The precise definition can be found in Algorithm~\ref{Algorithm_random_search} in Appendix~\ref{Appendix_Algorithm}.
The projected subgradient ascent is defined in a standard way \citep{bubeck2015convex}.
The precise definition can be found in Algorithm~\ref{Algorithm_projected_GD} in Appendix~\ref{Appendix_Algorithm}.

\subsubsection{Iteration Complexity of SI-CRL}

We give the iteration complexity of SI-CRL, i.e., how many iterations are required to output an $\epsilon$-optimal solution of Problem (\ref{Problem_Optimistic}) when near-optimal solutions of the inner-loop optimization problems can be obtained.
Our result is similar to Theorem 4 in \cite{Hu1990}.
Specifically, we consider two different cases: 1) we make no assumption of the constraint and use random search to solve the inner-loop problem; 2) we assume the constraint is concave and use projected subgradient ascent to solve the inner-loop problem.
% The random search algorithm and the projected subgradient ascent algorithm are defined by Algorithm~\ref{Algorithm_random_search} and Algorithm~\ref{Algorithm_projected_GD} in Appendix~\ref{Appendix_Algorithm}, respectively.

Before we give the iteration complexity of the case of random search, we make the following assumption to ensure technical rigor.
\begin{assumption}\label{Assumption_regular_maxima}
     For any $(s,a)\in\gS\times\gA$ and weight $v\in \RB^{\gS\times\gA}$, let $y_0\in\arg\max_{y\in Y} (v^\top  c_y-u_y)$. Then $\exists y_0$ such that
    $$
    \{y:\|y-y_0\|_\infty\leq \epsilon_0\}\subset Y.
    $$
\end{assumption}
Assumption~\ref{Assumption_regular_maxima} guarantees any possible solution of the inner-loop problem lies in the interior of $Y$.
\begin{theorem}\label{Theorem_Iteration_Complexity_Random_Search}
Suppose we use random search to solve the inner-loop problem of the SI-CRL algorithm, then if we set the size of random grid $M=O\left(\frac{\log(\delta/T)}{\log \left(1-((1-\gamma)\epsilon/|\gS|^2|\gA|\mathrm{diam}(Y))^m\right)}\right)$, $T=O\left(\left[\frac{\mathrm{diam}(Y)|\gS|^2|\gA|}{(1-\gamma)\epsilon}\right]^m \right)$, SI-CRL would output a $\epsilon$-optimal solution of Problem~\ref{Problem_Optimistic_ELSIP} with probability at least $1-\delta$.
Here we require $\epsilon\leq \frac{2|\gS|^2|\gA|L_y\epsilon_0}{1-\gamma}$.
\end{theorem}

\proof{Proof of Theorem~\ref{Theorem_Iteration_Complexity_Random_Search}.}
See Appendix~\ref{Appendix_Proofs_SICRL}.
\endproof

To derive theoretical guarantees for the case of projected subgradient ascent, we need the following assumption of concavity.

\begin{assumption}\label{Assumption_concave_constraint}
     For any $(s,a)\in\gS\times\gA$, $c_y(s,a)$ is concave in $y$. In addition, $u_y$ is convex in $y$.
\end{assumption}

\begin{theorem}\label{Theorem_Iteration_Complexity_Projected_GD}
Suppose we use projected gradient ascent to solve the inner-loop problem of the SI-CRL algorithm, then if we set the iteration number of the projected subgradient ascent $T_{PGA}=O\left(\frac{|\gS|^4|\gA|^2\mathrm{diam}(Y)^2}{(1-\gamma)^2\epsilon^2}\right)$, $T=O\left(\left[\frac{\mathrm{diam}(Y)|\gS|^2|\gA|}{(1-\gamma)\epsilon}\right]^m \right)$, SI-CRL would output a $\epsilon$-optimal solution of Problem~\ref{Problem_Optimistic_ELSIP}.
\end{theorem}

\proof{Proof of Theorem~\ref{Theorem_Iteration_Complexity_Projected_GD}.}
See Appendix~\ref{Appendix_Proofs_SICRL}.
\endproof

% \begin{remark}
The most crucial part of our proof is a $\epsilon$-packing argument.
Suppose we can get a $\epsilon/2$-optimal solution to the inner-loop problem by either random search of projected subgradient ascent and set the tolerance $\eta=\epsilon/2$.
By the assumption of Lipschitzness and the construction of the SI-CRL algorithm, for any $t\leq T$, either the SI-CRL algorithm has already terminated and we obtain a $\epsilon$-optimal solution to Problem~\ref{Problem_Optimistic_ELSIP}, or $\{B^{(t^\prime)},t=1,...,t\}$ forms a packing of $Y$.
Here $B^{(t^\prime)}:=\{y:\|y-y^{(t^\prime)}\|_\infty\leq \epsilon/2\beta\}$, and $\beta$ is some Lipschitz coefficient.
Then we may draw the conclusion by noting that the maximum iteration number of SI-CRL is no larger than the $\epsilon/2\beta$-packing number of $Y$.
We find that \cite{Hu1990} also used similar techniques to derive their convergence rate, although they assume the inner-loop problem can always be solved exactly.
% \end{remark}

\begin{remark}
    The iteration complexity of the SI-CRL algorithm grows with $m$ in an exponential manner.
    Thus from a theoretical viewpoint, the SI-CRL algorithm is no better than the naive discretization method mentioned in Remark~\ref{Remark_Baseline}.
    However, we find SI-CRL is far more efficient than the naive method in empirical evaluations.
    Perhaps it is because our bound of iteration complexity is obtained by the packing argument and not tight enough.
    Hopefully, the bound can be tightened by a refined analysis of the dynamics of $\{(y^{(t)}, z^{(t)}),t=1,...,T\}$.
\end{remark}

\subsubsection{Sample Complexity of SI-CRL}
%To begin with, 
We consider the case where the offline dataset we use is generated by a generative model.
First, we consider a restricted setting as in \cite{LATTIMORE2014125} where for each $(s,a)$-pair in the true SICMDP there are at most two possible next-states and provide the sample complexity bound.
Then we will drop Assumption \ref{Assumption_Two_Nonzero} using the same strategy as in \cite{LATTIMORE2014125} and derive the sample complexity bound of the general case.
\begin{assumption}\label{Assumption_Two_Nonzero}
The true unknown SICMDP $M$ satisfies $P(s^\prime|s,a)=0$ for all but two $s^\prime\in \gS$ denoted as $sa^+$ and  $sa^-\in \gS$.
\end{assumption}


Although Assumption \ref{Assumption_Two_Nonzero} seems quite restrictive, we argue that it is necessary to establish sharp sample complexity bound, as shown in \cite{LATTIMORE2014125}.
Specifically, without this assumption the ``quasi-Bernstein bound'' (Lemma \ref{Lemma_Quasi_Bernstein}) will not hold, thus we may not be able to get the $\widetilde O((1-\gamma)^{-3})$ bound.

\begin{lemma}\label{Lemma_Bound_on_V}
Suppose Assumption \ref{Assumption_Two_Nonzero} holds, and the dataset we use is generated by a generative model with $n/|\gS||\gA|=n_0>\max\left\{\frac{36\log4/\delta}{(1-\gamma)^2}, \frac{4\log4/\delta}{(1-\gamma)^3}\right\}$. Then with probability $1-2|\gS|^2|\gA|\delta$, we have that
$$\begin{aligned}
V_r^{\pi^*}(\mu)-V_r^{\tilde\pi}(\mu)\leq 24\sqrt{\frac{\log 4/\delta}{{n_0}(1-\gamma)^3}};\quad
V_{c_y}^{\tilde \pi}(\mu) - u_y \leq 12\sqrt{\frac{\log 4/\delta}{{n_0}(1-\gamma)^3}}, \; \forall y\in Y.
\end{aligned}
$$
Here $\tilde\pi$ is the exact solution of Problem~\ref{Problem_Optimistic}.
\end{lemma}
\proof{Proof of Lemma~\ref{Lemma_Bound_on_V}.}
See Appendix~\ref{Appendix_Proofs_SICRL}.

\begin{theorem}\label{Theorem_Sample_Complexity}
Suppose Assumption \ref{Assumption_Two_Nonzero} holds, the dataset we use is generated by a generative model and Problem \ref{Problem_Optimistic} can be solved exactly. Then when $n=O\left(\frac{|\gS||\gA|\log \paren{8|\gS|^2|\gA|/\delta}}{\epsilon^2(1-\gamma)^3}\right)$, SI-CRL is $(\epsilon,\delta)$-optimal.
\end{theorem}
\proof{Proof of Theorem~\ref{Theorem_Sample_Complexity}.}
Theorem~\ref{Theorem_Sample_Complexity} is a direct consequence of Lemma~\ref{Lemma_Bound_on_V}.

\begin{theorem}\label{Theorem_Sample_Complexity_General}
Suppose the dataset we use is generated by a generative model and Problem \ref{Problem_Optimistic} can be solved exactly. Then when $n=O\left(\frac{|\gS|^2|\gA|^2\paren{\log|\gS|}^3\log \paren{8|\gS|^4|\gA|^3/\delta}}{\epsilon^2(1-\gamma)^3}\right)$, a modification of SI-CRL is $(\epsilon,\delta)$-optimal.
\end{theorem}
\proof{Proof of Theorem~\ref{Theorem_Sample_Complexity_General}.}
See Appendix~\ref{Appendix_Proofs_SICRL}.


% \begin{remark}
Our proof strategy is similar to \cite{LATTIMORE2014125}. 
However, to get a $\widetilde O((1-\gamma)^{-3})$ bound \cite{LATTIMORE2014125} used a tedious recursion argument.
We greatly simplify the proof and achieve improvements in log terms (by a factor of $(\log(\frac{|\gS|}{\epsilon(1-\gamma)}))^2$) using sharper bounds on local variances of MDPs developed in \cite{pmlr-v125-agarwal20b}.
% \end{remark}

% \begin{remark}\label{Remark_Sample_Complexity_Assumption}
% Although Assumption \ref{Assumption_Two_Nonzero} seems quite restrictive, we argue that it is necessary to establish sharp sample complexity bound, as shown in \cite{LATTIMORE2014125}.
% Specifically, without this assumption the ``quasi-Bernstein bound'' (Lemma \ref{Lemma_Quasi_Bernstein}) will not hold, thus we may not be able to get the $\widetilde O((1-\gamma)^{-3})$ bound.
% \end{remark}

\begin{remark}\label{Remark_Sample_Complexity_General_Dependence_on_Constraints}
It can be noted that our sample complexity bound does not rely on the constraint set $Y$.
This is because we consider the setting where $r$ and $c_y$ are known deterministic functions and the only source of randomness comes from estimating the unknown transition dynamic using an offline dataset.
In other words, the constraints do not make the problem more difficult in the statistical sense.
\end{remark}

\begin{remark}\label{Remark_Modification}
Here ``a modification of SI-CRL" stands for the following procedure: first we transform the original SICMDP to a new SICMDP satisfying Assumption~\ref{Assumption_Two_Nonzero}, then we run SI-CRL to solve the new SICMDP.
One may refer to the proof in Appendix~\ref{Appendix_Proofs_SICRL} for more details.
\end{remark}

Now we generalize our results to the case where the offline dataset is generated by a probability measure.
\begin{theorem}\label{Theorem_Sample_Complexity_General_Measure}
Suppose the dataset we use is generated by a probability measure $\nu$ and Problem \ref{Problem_Optimistic} can be solved exactly. Then when $m=O\left(\frac{|\gS||\gA|\paren{\log|\gS|}^3\log \paren{8|\gS|^4|\gA|^3/\delta}}{\nu_{\min} \epsilon^2(1-\gamma)^3}\right)$, a modification of SI-CRL is $(\epsilon,\delta)$-optimal.
\end{theorem}
\proof{Proof of Theorem~\ref{Theorem_Sample_Complexity_General_Measure}.}
See Appendix~\ref{Appendix_Proofs_SICRL}.
\subsection{Theoretical Analysis of SI-CPO}\label{Section_Theory_SICPO}
In this section, we present theoretical guarantees of SI-CPO.
We consider a version of the SI-CPO algorithm, where we use sample-based NPG \citep{agarwal2021theory} as the policy optimization subroutine, a finite-horizon Monte-Carlo estimator as the policy evaluation subroutine, and either random search or projected subgradient ascent as the optimization subroutine.
It is shown that when the function approximation error $\epsilon_{bias}$ is of the same order with $\epsilon$, our proposed algorithm takes $\widetilde{O}\left(\frac{1}{\epsilon^2(1-\gamma)^6}\right)$ iterations and make $\widetilde{O}\left(\frac{1}{\epsilon^4(1-\gamma)^{10}}\right)$ interactions with the environment to achieve an $\epsilon$-level of averaged suboptimality with high probability.
This corresponds to a $\widetilde{O}(1/\sqrt{T})$ globally convergence rate, which is typical for NPG-based policy optimization algorithms.
We will give a detailed description of the considered version of the SI-CPO algorithm as well as our technical assumptions in Section~\ref{Subsection_SICPO_Prem} and present the theoretical results in Sections~\ref{Subsection_SICPO_Iteration_Complexity} and~\ref{Subsection_SICPO_Sample_Complexity}.

\subsubsection{Preliminaries}\label{Subsection_SICPO_Prem}
Recall the policy $\pi$ is parameterized by $\theta\in\Theta\subset\RB^d$ (denoted by $~\pi_\theta$).
We make the following assumptions about the parameterized policy class.
\begin{assumption}[Differentiable policy class]\label{Assumption_differentiable}
$\Pi$ can be parametrized as $\Pi_\theta=\{\pi_\theta|\theta\in\RB^d\}$, such that for all $s\in\gS$,~$a\in\gA$, $\log_\theta\pi(s|a)$ is a differentiable function of $\theta$.
\end{assumption}

\begin{assumption}[Lipschitz policy class]\label{Assumption_Lipschitz_policy}
For all $s\in\gS$,~$a\in\gA$, $\log\pi_\theta(s|a)$ is a $L_\pi$-Lipschitz function of $\theta$, i.e.,
$$
\|\nabla_\theta\log\pi_\theta(s|a)\|_{2}\leq L_\pi,\forall s\in\gS,a\in\gA,\theta\in \RB^d.
$$
\end{assumption}

\begin{assumption}[Smooth policy class]\label{Assumption_smooth}
For all $s\in\gS$,~$a\in\gA$, $\log\pi_\theta(s|a)$ is a $\beta$-smooth function of $\theta$, i.e.,
$$
\|\nabla_\theta\log\pi_\theta(s|a)-\nabla_\theta\log\pi_{\theta^\prime}(s|a)\|_{2}\leq \beta\|\theta-\theta^\prime\|_2,\forall s\in\gS,a\in\gA,\theta,\theta^\prime\in \RB^d.
$$
\end{assumption}

\begin{assumption}[Positive semidefinite Fisher information]\label{Assumption_PSD_Fisher}
    For all $\theta\in\RB^d$,
    $$F(\theta):=\EB_{(s,a)\sim\nu_\theta}[\nabla_\theta\log\pi_\theta(a|s)\nabla_\theta\log\pi_\theta(a|s)^\top]\succeq\mu_F I_d.
    $$
\end{assumption}
The assumptions above are standard in the literature of policy optimizations \citep{agarwal2021theory}.
We also assume the parametrization realizes good function approximation in terms of transferred compatible function approximation errors, which is first introduced by \cite{agarwal2021theory}.
The error term can be close to zero if the policy class is rich \citep{wang2019neural} or the underlying MDP has low-rank structures \citep{jiang2017contextual}.
\begin{assumption}[Bounded function approximation error]\label{Assumption_func_approx_err}
The transferred compatible function approximation errors satisfies that $\forall t\in\{1,...,T\}$
$$
\begin{aligned}
\min_w E^{\nu^{(t)}}(r,\theta^{(t)},w)&\leq \epsilon_{\text{bias}}\\
\min_w E^{\nu^{(t)}}(c_y,\theta^{(t)},w)&\leq \epsilon_{\text{bias}}\ \forall y\in Y,\\
\end{aligned}
$$
where $\nu^{(t)}$ denotes the state-action occupancy measure induced by policy $\pi^{(t)}$. 
The transferred compatible function approximation errors are defined as:
$$
E^{\nu}(\diamond,\theta,w):=\EB_{(s,a)\sim\nu}(A^{\pi_\theta}_\diamond(s,a)-w^\top\nabla_\theta\log\pi_\theta(a|s))^2.
$$
\end{assumption}

Besides, we also assume the weights to minimize the transferred compatible function approximation errors are bounded.

\begin{assumption}[Bounded Weight]\label{Assumption_est_err}
For any $t\in\{1,...,T\}$, $\forall y\in Y$,
$$
\left\|\underset{w}{\mathrm{argmin}} E^{\nu^{(t)}}(r,\theta^{(t)},w)\right\|_2^2\leq W^2, \left\|\underset{w}{\mathrm{argmin}} E^{\nu^{(t)}}(c_y,\theta^{(t)},w)\right\|_2^2\leq W^2.
$$
\end{assumption}

In the theoretical analysis of SI-CPO, we consider an instance of SI-CPO where we use a sample-based version of NPG \citep{agarwal2021theory} as the policy optimization subroutine, a fixed-horizon Monte-Carlo estimator as the policy evaluation subroutine, and either random search or projected subgradient ascent as the optimization subroutine.
In the NPG algorithm, we use the following natural policy gradient $w^{(t)}$ to update the policy parameters:
$$
w^{(t)}:=F(\theta^{(t)})^\dagger\EB_{(s,a)\sim\nu^{(t)}}(A^{\pi^{(t)}}_\diamond(s,a)\nabla_\theta\pi_\theta(a|s)).
$$
Here $\diamond$ can be either the reward $r$ or some cost function $c_y$.
However, for most RL problems it is computationally prohibitive to evaluate $F(\theta)^\dagger$, and $\EB_{(s,a)\sim\nu^{(t)}}(A^{\pi^{(t)}}_\diamond(s,a)\nabla_\theta\pi_\theta(a|s))$ are usually unknown to the algorithm.
Therefore, we instead use a sample-based estimate of $w^{(t)}$, which can be obtained by solving the following optimization problem by running $K_{sgd}$ steps of stochastic gradient descent:
$$
\hat w^{(t)}\approx\frac{1}{1-\gamma}\arg\min_w E^{\nu^{(t)}}(b,\theta^{(t)},w),
$$
recall that $E^{\nu^{(t)}}(\diamond,\theta^{(t)},w)$ is the transferred function approximation error defined in Assumption~\ref{Assumption_func_approx_err}.
The precise definition of sample-based NPG can be found in Algorithm~\ref{Algorithm_sample_based_NPG} in Appendix~\ref{Appendix_Algorithm}.

As for policy evaluation, we choose to use a Monte-Carlo estimator with a fixed horizon $H$.
The idea is very simple, in each episode we run the target policy $\pi$ for $H$ steps, and record the return
$$
G_i=\sum_{k=0}^{H-1}\gamma^k c_y(s_k,a_k).
$$
The procedure is repeated for $K_{eval}$ times and we take the average as an estimate of $V_{c_y}^{\pi^{(t)}}(\mu)$.
Compared with the more commonly used unbiased Monte-Carlo estimate
$$\widetilde{G}_i=\sum_{k=0}^{H^\prime-1}c_y(s_k,a_k)
$$
where $H^\prime$ is no longer fixed and drawn from an exponential distribution $\mathrm{Exp(\lambda)}$, $G_i$ does introduce bias, but it also has the advantage of being sub-gaussian.
Moreover, the bias term is always bounded by $\frac{\gamma^H}{1-\gamma}$, which decays exponentially as we choose larger $H$s.

\subsubsection{Iteration complexity of SI-CPO}\label{Subsection_SICPO_Iteration_Complexity}

The following two theorems give the iteration complexity of the SI-CPO algorithm when we use either random search or projected subgradient ascent to solve the inner-loop problem.

\begin{theorem}\label{Theorem_random_search_SICPO}
Suppose we use random search to solve the inner-loop problem and Assumption~\ref{Assumption_regular_maxima} holds. 
If we set $\alpha=1/\sqrt{T}$, $\eta=\epsilon+\frac{1}{(1-\gamma)^{3/2}}\sqrt{\left\|\frac{\nu^*}{\nu_0}\right\|_\infty\epsilon_{bias}}$, and 
${K_{sgd}=\widetilde{O}\left(\frac{1}{\epsilon_{bias}^2(1-\gamma)^4}\right)}$ , $ H=O\left(\frac{\log(1-\gamma)+\min\{\log(\epsilon_{bias}),\log(\epsilon)\}}{\log\gamma}\right)$, $K_{eval}=\widetilde{O}\left(\frac{1}{\epsilon^2(1-\gamma)^2}\right)$, ${M=\widetilde{O}\left(\frac{(\mathrm{diam}(Y))^m}{\epsilon^m(1-\gamma)^m}\right)}$, ${T=\widetilde{O}\left(\frac{1}{\epsilon^2(1-\gamma)^6}\right)}$, then we have with probability $1-2\delta$,
$$
    \frac{1}{|\gB|}\sum_{t\in\gB}(V_r^*(\mu)-V_r^{(t)}(\mu))\leq \epsilon+\frac{1}{(1-\gamma)^{3/2}}\sqrt{\left\|\frac{\nu^*}{\nu_0}\right\|_\infty\epsilon_{bias}},
$$
and $\forall t\in\gB$
$$ \sup_{y\in Y}\left[V_{c_y}^{(t)}(\mu)-u_y\right]\leq 2\epsilon+\frac{1}{(1-\gamma)^{3/2}}\sqrt{\left\|\frac{\nu^*}{\nu_0}\right\|_\infty\epsilon_{bias}}.
$$
for any $\epsilon<2\epsilon_0 L_y/(1-\gamma)$.
\end{theorem}
\proof {Proof of Theorem~\ref{Theorem_random_search_SICPO}}
See Appendix~\ref{Appendix_Proofs_SICPO}.
\endproof

\begin{theorem}\label{Theorem_PGA_SICPO}
Suppose we use projected subgradient ascent to solve the inner-loop problem and Assumption~\ref{Assumption_concave_constraint} holds. 
If we set $\alpha=1/\sqrt{T}$, $\eta=\epsilon+\frac{1}{(1-\gamma)^{3/2}}\sqrt{\left\|\frac{\nu^*}{\nu_0}\right\|_\infty\epsilon_{bias}}$, and 
${K_{sgd}=\widetilde{O}\left(\frac{1}{\epsilon_{bias}^2(1-\gamma)^4}\right)}$ , $ H=O\left(\frac{\log(1-\gamma)+\min\{\log(\epsilon_{bias}),\log(\epsilon)\}}{\log\gamma}\right)$, $K_{eval}=\widetilde{O}\left(\frac{1}{\epsilon^2(1-\gamma)^2}\right)$, ${T_{PGA}=O\left(\frac{[\mathrm{diam}(Y)]^2}{\epsilon^2(1-\gamma)^2}
    \right)}$, ${T=\widetilde{O}\left(\frac{1}{\epsilon^2(1-\gamma)^6}\right)}$, then we have with probability $1-\delta$,
$$
    \frac{1}{|\gB|}\sum_{t\in\gB}(V_r^*(\mu)-V_r^{(t)}(\mu))\leq \epsilon+\frac{1}{(1-\gamma)^{3/2}}\sqrt{\left\|\frac{\nu^*}{\nu_0}\right\|_\infty\epsilon_{bias}},
$$
and $\forall t\in\gB$
$$ \sup_{y\in Y}\left[V_{c_y}^{(t)}(\mu)-u_y\right]\leq 2\epsilon+\frac{1}{(1-\gamma)^{3/2}}\sqrt{\left\|\frac{\nu^*}{\nu_0}\right\|_\infty\epsilon_{bias}}.
$$
\end{theorem}
\proof {Proof of Theorem~\ref{Theorem_random_search_SICPO}}
See Appendix~\ref{Appendix_Proofs_SICPO}.
\endproof

In our proof, we focus on the event that the policy evaluation subroutine returns accurate estimates of $V^{(t)}_{c_y}(\mu)$ and the sample-based NPG generates a near-optimal solution of $\min_w E^{\nu^{(t)}}(\diamond,\theta^{(t)},w)$.
We show that this event happens with high probability.
When it happens, with carefully chosen tolerance threshold $\eta$, either the "good set" $\gB$ is large or the policies in $\gB$ perform equally well to the optimal policy $\pi^*$ on average, \textit{i.e.} $\sum_{t\in\gB}(V^{(t)}_r(\mu)-V^*_r(\mu))\geq 0$.
As long as $\gB$ is large enough, we may further conclude that  $\frac{1}{|\gB|}\sum_{t\in\gB}|V^{(t)}_r(\mu)-V^*_r(\mu)|$ is small by typical analysis techniques of NPG \citep{agarwal2021theory}.
Recalling that the constraint violations of policies in $\gB$ are small as long as the inner-loop optimization problems are effectively solved, we complete our proof.

Our ideas of proof are similar to \cite{wei2020comirror, xu2021crpo}.
However, \cite{wei2020comirror} focused on the semi-infinitely constrained convex problems and we focus on the semi-infinitely constrained RL problems.
Moreover, their theoretical results are in the form of bounds on expectations, while ours are in the form of high probability bounds.
Our work is different with \cite{xu2021crpo} in the sense that they address finitely constrained RL problems, and restrict their analysis to two specific forms of policy parametrizations, whereas we consider general policy parametrizations.

\begin{remark}
The error terms of SI-CPO can be attributed to three sources: the function approximation error, the statistical error, and the optimization error.
When we say SI-CPO converges to the globally optimal policy $\pi^*$ at a $\widetilde {O}(1/\sqrt{T})$ rate we mean that if we use a near-perfect parameterized policy class, estimate $V^{(t)}_{c_y}(\mu)$ and the natural policy gradient with adequate data and solve the inner-loop problem with sufficient accuracy, then the averaged error term of SI-CPO has a $\widetilde O(1/\sqrt{T})$ order with high probability.
\end{remark}

\begin{remark}\label{Remark_random_search_vs_fixed_search}
    When solving the inner-loop problem, an alternative approach to random search is to search over a fixed grid of $Y$.
    This is equivalent to a version of naive discretization: we first transform the SICMDP to a finitely constrained MDP by discretizing $Y$, and then solve the resulting problem with CRPO \citep{xu2021crpo}.
    From a theoretical viewpoint, random search is no better than the gird search since both need to search over a $\widetilde{O}((\mathrm{diam}(Y)/\epsilon)^m)$-sized grid to ensure $\epsilon$-optimality.
    However, in numerical experiments we find that the approach based on random search is far more efficient that the approach based on grid search.
    The reasons can be two-fold: 1) in the theoretical analysis we must give guarantees for the hardest problem instances, but real-world problem settings may contain structures that can be exploited by random search \citep{bergstra2012random}; 2) in random search the random grid are generated in an independent way in each iteration, which can reduce the bias introduced by replacing the constraint set $Y$ with a fixed finite grid.
\end{remark}

\subsubsection{Sample complexity of SI-CPO}\label{Subsection_SICPO_Sample_Complexity}

\begin{corollary}\label{Corollary_Sample_Complexity_SICPO}
SI-CPO need to make $\widetilde O{\left(\frac{1}{\epsilon^2\min\{\epsilon^2,\epsilon_{bias}^2\}(1-\gamma)^{10}}\right)}$ interactions with the environments to ensure with high probability
$$
    \frac{1}{|\gB|}\sum_{t\in\gB}(V_r^*(\mu)-V_r^{(t)}(\mu))\leq \epsilon+\frac{1}{(1-\gamma)^{3/2}}\sqrt{\left\|\frac{\nu^*}{\nu_0}\right\|_\infty\epsilon_{bias}},
$$
and $\forall t\in\gB$
$$ \sup_{y\in Y}\left[V_{c_y}^{(t)}(\mu)-u_y\right]\leq 2\epsilon+\frac{1}{(1-\gamma)^{3/2}}\sqrt{\left\|\frac{\nu^*}{\nu_0}\right\|_\infty\epsilon_{bias}}.
$$
\end{corollary}
\proof{Proof of Corollary~\ref{Corollary_Sample_Complexity_SICPO}}
This corollary is a direct consequence of Theorem~\ref{Theorem_random_search_SICPO} as the sample complexity is of the order $T \cdot H\cdot(K_{eval}+K_{sgd})$.
Note that the sample complexity bound is independent of how we solve the inner-loop problem.
\endproof

Our sample complexity bound is of the order $\widetilde{O}\left(\frac{1}{\epsilon^4(1-\gamma)^{10}}\right)$.
This is worse than typical sample complexity bounds for sample-based NPG such as the $\widetilde{O}\left(\frac{1}{\epsilon^4(1-\gamma)^8}\right)$ in \cite{agarwal2021theory}.
This does not mean that the constrained problem is statistically harder or our bounds are loose.
The difference comes from that their goal is to ensure accuracy in expectation, while our goal is to ensure accuracy with high probability.
Therefore, in our algorithm, we need to run more SGD iterations (corresponding to larger $K_{sgd}$) to find a better estimate of the natural gradient.

\section{Numerical Experiments}\label{Section_Experiment}
In this section we conduct comprehensive experiments to emphasise the effectiveness of DIAL, including evaluations under white-box and black-box settings, robustness to unforeseen adversaries, robustness to unforeseen corruptions, transfer learning, and ablation studies. Finally, we present a new measurement to test the balance between robustness and natural accuracy, which we named $F_1$-robust score. 


\subsection{A case study on SVHN and CIFAR-100}
In the first part of our analysis, we conduct a case study experiment on two benchmark datasets: SVHN \citep{netzer2011reading} and CIFAR-100 \cite{krizhevsky2009learning}. We follow common experiment settings as in \cite{rice2020overfitting, wu2020adversarial}. We used the PreAct ResNet-18 \citep{he2016identity} architecture on which we integrate a domain classification layer. The adversarial training is done using 10-step PGD adversary with perturbation size of 0.031 and a step size of 0.003 for SVHN and 0.007 for CIFAR-100. The batch size is 128, weight decay is $7e^{-4}$ and the model is trained for 100 epochs. For SVHN, the initial learinnig rate is set to 0.01 and decays by a factor of 10 after 55, 75 and 90 iteration. For CIFAR-100, the initial learning rate is set to 0.1 and decays by a factor of 10 after 75 and 90 iterations. 
%We compared DIAL to \cite{madry2017towards} and TRADES \citep{zhang2019theoretically}. 
%The evaluation is done using Auto-Attack~\citep{croce2020reliable}, which is an ensemble of three white-box and one black-box parameter-free attacks, and various $\ell_{\infty}$ adversaries: PGD$^{20}$, PGD$^{100}$, PGD$^{1000}$ and CW$_{\infty}$ with step size of 0.003. 
Results are averaged over 3 restarts while omitting one standard deviation (which is smaller than 0.2\% in all experiments). As can be seen by the results in Tables~\ref{black-and_white-svhn} and \ref{black-and_white-cifar100}, DIAL presents consistent improvement in robustness (e.g., 5.75\% improved robustness on SVHN against AA) compared to the standard AT 
%under variety of attacks 
while also improving the natural accuracy. More results are presented in Appendix \ref{cifar100-svhn-appendix}.


\begin{table}[!ht]
  \caption{Robustness against white-box, black-box attacks and Auto-Attack (AA) on SVHN. Black-box attacks are generated using naturally trained surrogate model. Natural represents the naturally trained (non-adversarial) model.
  %and applied to the best performing robust models.
  }
  \vskip 0.1in
  \label{black-and_white-svhn}
  \centering
  \small
  \begin{tabular}{l@{\hspace{1\tabcolsep}}c@{\hspace{1\tabcolsep}}c@{\hspace{1\tabcolsep}}c@{\hspace{1\tabcolsep}}c@{\hspace{1\tabcolsep}}c@{\hspace{1\tabcolsep}}c@{\hspace{1\tabcolsep}}c@{\hspace{1\tabcolsep}}c@{\hspace{1\tabcolsep}}c@{\hspace{1\tabcolsep}}c}
    \toprule
    & & \multicolumn{4}{c}{White-box} & \multicolumn{4}{c}{Black-Box}  \\
    \cmidrule(r){3-6} 
    \cmidrule(r){7-10}
    Defense Model & Natural & PGD$^{20}$ & PGD$^{100}$  & PGD$^{1000}$  & CW$^{\infty}$ & PGD$^{20}$ & PGD$^{100}$ & PGD$^{1000}$  & CW$^{\infty}$ & AA \\
    \midrule
    NATURAL & 96.85 & 0 & 0 & 0 & 0 & 0 & 0 & 0 & 0 & 0 \\
    \midrule
    AT & 89.90 & 53.23 & 49.45 & 49.23 & 48.25 & 86.44 & 86.28 & 86.18 & 86.42 & 45.25 \\
    % TRADES & 90.35 & 57.10 & 54.13 & 54.08 & 52.19 & 86.89 & 86.73 & 86.57 & 86.70 &  49.50 \\
    $\DIAL_{\kl}$ (Ours) & 90.66 & \textbf{58.91} & \textbf{55.30} & \textbf{55.11} & \textbf{53.67} & 87.62 & 87.52 & 87.41 & 87.63 & \textbf{51.00} \\
    $\DIAL_{\ce}$ (Ours) & \textbf{92.88} & 55.26  & 50.82 & 50.54 & 49.66 & \textbf{89.12} & \textbf{89.01} & \textbf{88.74} & \textbf{89.10} &  46.52  \\
    \bottomrule
  \end{tabular}
\end{table}


\begin{table}[!ht]
  \caption{Robustness against white-box, black-box attacks and Auto-Attack (AA) on CIFAR100. Black-box attacks are generated using naturally trained surrogate model. Natural represents the naturally trained (non-adversarial) model.
  %and applied to the best performing robust models.
  }
  \vskip 0.1in
  \label{black-and_white-cifar100}
  \centering
  \small
  \begin{tabular}{l@{\hspace{1\tabcolsep}}c@{\hspace{1\tabcolsep}}c@{\hspace{1\tabcolsep}}c@{\hspace{1\tabcolsep}}c@{\hspace{1\tabcolsep}}c@{\hspace{1\tabcolsep}}c@{\hspace{1\tabcolsep}}c@{\hspace{1\tabcolsep}}c@{\hspace{1\tabcolsep}}c@{\hspace{1\tabcolsep}}c}
    \toprule
    & & \multicolumn{4}{c}{White-box} & \multicolumn{4}{c}{Black-Box}  \\
    \cmidrule(r){3-6} 
    \cmidrule(r){7-10}
    Defense Model & Natural & PGD$^{20}$ & PGD$^{100}$  & PGD$^{1000}$  & CW$^{\infty}$ & PGD$^{20}$ & PGD$^{100}$ & PGD$^{1000}$  & CW$^{\infty}$ & AA \\
    \midrule
    NATURAL & 79.30 & 0 & 0 & 0 & 0 & 0 & 0 & 0 & 0 & 0 \\
    \midrule
    AT & 56.73 & 29.57 & 28.45 & 28.39 & 26.6 & 55.52 & 55.29 & 55.26 & 55.40 & 24.12 \\
    % TRADES & 58.24 & 30.10 & 29.66 & 29.64 & 25.97 & 57.05 & 56.71 & 56.67 & 56.77 & 24.92 \\
    $\DIAL_{\kl}$ (Ours) & 58.47 & \textbf{31.19} & \textbf{30.50} & \textbf{30.42} & \textbf{26.91} & 57.16 & 56.81 & 56.80 & 57.00 & \textbf{25.87} \\
    $\DIAL_{\ce}$ (Ours) & \textbf{60.77} & 27.87 & 26.66 & 26.61 & 25.98 & \textbf{59.48} & \textbf{59.06} & \textbf{58.96} & \textbf{59.20} & 23.51  \\
    \bottomrule
  \end{tabular}
\end{table}


% \begin{table}[!ht]
%   \caption{Robustness comparison of DIAL to Madry et al. and TRADES defense models on the SVHN dataset under different PGD white-box attacks and the ensemble Auto-Attack (AA).}
%   \label{svhn}
%   \centering
%   \begin{tabular}{llllll|l}
%     \toprule
%     \cmidrule(r){1-5}
%     Defense Model & Natural & PGD$^{20}$ & PGD$^{100}$ & PGD$^{1000}$ & CW$_{\infty}$ & AA\\
%     \midrule
%     $\DIAL_{\kl}$ (Ours) & $\mathbf{90.66}$ & $\mathbf{58.91}$ & $\mathbf{55.30}$ & $\mathbf{55.12}$ & $\mathbf{53.67}$  & $\mathbf{51.00}$  \\
%     Madry et al. & 89.90 & 53.23 & 49.45 & 49.23 & 48.25 & 45.25  \\
%     TRADES & 90.35 & 57.10 & 54.13 & 54.08 & 52.19 & 49.50 \\
%     \bottomrule
%   \end{tabular}
% \end{table}


\subsection{Performance comparison on CIFAR-10} \label{defence-settings}
In this part, we evaluate the performance of DIAL compared to other well-known methods on CIFAR-10. 
%To be consistent with other methods, 
We follow the same experiment setups as in~\cite{madry2017towards, wang2019improving, zhang2019theoretically}. When experiment settings are not identical between tested methods, we choose the most commonly used settings, and apply it to all experiments. This way, we keep the comparison as fair as possible and avoid reporting changes in results which are caused by inconsistent experiment settings \citep{pang2020bag}. To show that our results are not caused because of what is referred to as \textit{obfuscated gradients}~\citep{athalye2018obfuscated}, we evaluate our method with same setup as in our defense model, under strong attacks (e.g., PGD$^{1000}$) in both white-box, black-box settings, Auto-Attack ~\citep{croce2020reliable}, unforeseen "natural" corruptions~\citep{hendrycks2018benchmarking}, and unforeseen adversaries. To make sure that the reported improvements are not caused by \textit{adversarial overfitting}~\citep{rice2020overfitting}, we report best robust results for each method on average of 3 restarts, while omitting one standard deviation (which is smaller than 0.2\% in all experiments). Additional results for CIFAR-10 as well as comprehensive evaluation on MNIST can be found in Appendix \ref{mnist-results} and \ref{additional_res}.
%To further keep the comparison consistent, we followed the same attack settings for all methods.


\begin{table}[ht]
  \caption{Robustness against white-box, black-box attacks and Auto-Attack (AA) on CIFAR-10. Black-box attacks are generated using naturally trained surrogate model. Natural represents the naturally trained (non-adversarial) model.
  %and applied to the best performing robust models.
  }
  \vskip 0.1in
  \label{black-and_white-cifar}
  \centering
  \small
  \begin{tabular}{cccccccc@{\hspace{1\tabcolsep}}c}
    \toprule
    & & \multicolumn{3}{c}{White-box} & \multicolumn{3}{c}{Black-Box} \\
    \cmidrule(r){3-5} 
    \cmidrule(r){6-8}
    Defense Model & Natural & PGD$^{20}$ & PGD$^{100}$ & CW$^{\infty}$ & PGD$^{20}$ & PGD$^{100}$ & CW$^{\infty}$ & AA \\
    \midrule
    NATURAL & 95.43 & 0 & 0 & 0 & 0 & 0 & 0 &  0 \\
    \midrule
    TRADES & 84.92 & 56.60 & 55.56 & 54.20 & 84.08 & 83.89 & 83.91 &  53.08 \\
    MART & 83.62 & 58.12 & 56.48 & 53.09 & 82.82 & 82.52 & 82.80 & 51.10 \\
    AT & 85.10 & 56.28 & 54.46 & 53.99 & 84.22 & 84.14 & 83.92 & 51.52 \\
    ATDA & 76.91 & 43.27 & 41.13 & 41.01 & 75.59 & 75.37 & 75.35 & 40.08\\
    $\DIAL_{\kl}$ (Ours) & 85.25 & $\mathbf{58.43}$ & $\mathbf{56.80}$ & $\mathbf{55.00}$ & 84.30 & 84.18 & 84.05 & \textbf{53.75} \\
    $\DIAL_{\ce}$ (Ours)  & $\mathbf{89.59}$ & 54.31 & 51.67 & 52.04 &$ \mathbf{88.60}$ & $\mathbf{88.39}$ & $\mathbf{88.44}$ & 49.85 \\
    \midrule
    $\DIAL_{\awp}$ (Ours) & $\mathbf{85.91}$ & $\mathbf{61.10}$ & $\mathbf{59.86}$ & $\mathbf{57.67}$ & $\mathbf{85.13}$ & $\mathbf{84.93}$ & $\mathbf{85.03}$  & \textbf{56.78} \\
    $\TRADES_{\awp}$ & 85.36 & 59.27 & 59.12 & 57.07 & 84.58 & 84.58 & 84.59 & 56.17 \\
    \bottomrule
  \end{tabular}
\end{table}



\paragraph{CIFAR-10 setup.} We use the wide residual network (WRN-34-10)~\citep{zagoruyko2016wide} architecture. %used in the experiments of~\cite{madry2017towards, wang2019improving, zhang2019theoretically}. 
Sidelong this architecture, we integrate a domain classification layer. To generate the adversarial domain dataset, we use a perturbation size of $\epsilon=0.031$. We apply 10 of inner maximization iterations with perturbation step size of 0.007. Batch size is set to 128, weight decay is set to $7e^{-4}$, and the model is trained for 100 epochs. Similar to the other methods, the initial learning rate was set to 0.1, and decays by a factor of 10 at iterations 75 and 90. 
%For being consistent with other methods, the natural images are padded with 4-pixel padding with 32-random crop and random horizontal flip. Furthermore, all methods are trained using SGD with momentum 0.9. For $\DIAL_{\kl}$, we balance the robust loss with $\lambda=6$ and the domains loss with $r=4$. For $\DIAL_{\ce}$, we balance the robust loss with $\lambda=1$ and the domains loss with $r=2$. 
%We also introduce a version of our method that incorporates the AWP double-perturbation mechanism, named DIAL-AWP.
%which is trained using the same learning rate schedule used in ~\cite{wu2020adversarial}, where the initial 0.1 learning rate decays by a factor of 10 after 100 and 150 iterations. 
See Appendix \ref{cifar10-additional-setup} for additional details.

\begin{table}[ht]
  \caption{Black-box attack using the adversarially trained surrogate models on CIFAR-10.}
  \vskip 0.1in
  \label{black-box-cifar-adv}
  \centering
  \small
  \begin{tabular}{ll|c}
    \toprule
    \cmidrule(r){1-2}
    Surrogate (source) model & Target model & robustness \% \\
    % \midrule
    \midrule
    TRADES & $\DIAL_{\ce}$ & $\mathbf{67.77}$ \\
    $\DIAL_{\ce}$ & TRADES & 65.75 \\
    \midrule
    MART & $\DIAL_{\ce}$ & $\mathbf{70.30}$ \\
    $\DIAL_{\ce}$ & MART & 64.91 \\
    \midrule
    AT & $\DIAL_{\ce}$ & $\mathbf{65.32}$ \\
    $\DIAL_{\ce}$ & AT  & 63.54 \\
    \midrule
    ATDA & $\DIAL_{\ce}$ & $\mathbf{66.77}$ \\
    $\DIAL_{\ce}$ & ATDA & 52.56 \\
    \bottomrule
  \end{tabular}
\end{table}

\paragraph{White-box/Black-box robustness.} 
%We evaluate all defense models using Auto-Attack, PGD$^{20}$, PGD$^{100}$, PGD$^{1000}$ and CW$_{\infty}$ with step size 0.003. We constrain all attacks by the same perturbation $\epsilon=0.031$. 
As reported in Table~\ref{black-and_white-cifar} and Appendix~\ref{additional_res}, our method achieves better robustness compared to the other methods. Specifically, in the white-box settings, our method improves robustness over~\citet{madry2017towards} and TRADES by 2\% 
%using the common PGD$^{20}$ attack 
while keeping higher natural accuracy. We also observe better natural accuracy of 1.65\% over MART while also achieving better robustness over all attacks. Moreover, our method presents significant improvement of up to 15\% compared to the the domain invariant method suggested by~\citet{song2018improving} (ATDA).
%in both natural and robust accuracy. 
When incorporating 
%the double-perturbation mechanism of 
AWP, our method improves the results of $\TRADES_{\awp}$ by almost 2\%.
%and reaches state-of-the-art results for robust models with no additional data. 
% Additional results are available in Appendix~\ref{additional_res}.
When tested on black-box settings, $\DIAL_{\ce}$ presents a significant improvement of more than 4.4\% over the second-best performing method, and up to 13\%. In Table~\ref{black-box-cifar-adv}, we also present the black-box results when the source model is taken from one of the adversarially trained models. %Then, we compare our model to each one of them both as the source model and target model. 
In addition to the improvement in black-box robustness, $\DIAL_{\ce}$ also manages to achieve better clean accuracy of more than 4.5\% over the second-best performing method.
% Moreover, based on the auto-attack leader-board \footnote{\url{https://github.com/fra31/auto-attack}}, our method achieves the 1st place among models without additional data using the WRN-34-10 architecture.

% \begin{table}
%   \caption{White-box robustness on CIFAR-10 using WRN-34-10}
%   \label{white-box-cifar-10}
%   \centering
%   \begin{tabular}{lllll}
%     \toprule
%     \cmidrule(r){1-2}
%     Defense Model & Natural & PGD$^{20}$ & PGD$^{100}$ & PGD$^{1000}$ \\
%     \midrule
%     TRADES ~\cite{zhang2019theoretically} & 84.92  & 56.6 & 55.56 & 56.43  \\
%     MART ~\cite{wang2019improving} & 83.62  & 58.12 & 56.48 & 56.55  \\
%     Madry et al. ~\cite{madry2017towards} & 85.1  & 56.28 & 54.46 & 54.4  \\
%     Song et al. ~\cite{song2018improving} & 76.91 & 43.27 & 41.13 & 41.02  \\
%     $\DIAL_{\ce}$ (Ours) & $ \mathbf{90}$  & 52.12 & 48.88 & 48.78  \\
%     $\DIAL_{\kl}$ (Ours) & 85.25 & $\mathbf{58.43}$ & $\mathbf{56.8}$ & $\mathbf{56.73}$ \\
%     \midrule
%     $\DIAL_{\kl}$+AWP (Ours) & $\mathbf{85.91}$ & $\mathbf{61.1}$ & - & -  \\
%     TRADES+AWP \cite{wu2020adversarial} & 85.36 & 59.27 & 59.12 & -  \\
%     % MART + AWP & 84.43 & 60.68 & 59.32 & -  \\
%     \bottomrule
%   \end{tabular}
% \end{table}


% \begin{table}
%   \caption{White-box robustness on MNIST}
%   \label{white-box-mnist}
%   \centering
%   \begin{tabular}{llllll}
%     \toprule
%     \cmidrule(r){1-2}
%     Defense Model & Natural & PGD$^{40}$ & PGD$^{100}$ & PGD$^{1000}$ \\
%     \midrule
%     TRADES ~\cite{zhang2019theoretically} & 99.48 & 96.07 & 95.52 & 95.22 \\
%     MART ~\cite{wang2019improving} & 99.38  & 96.99 & 96.11 & 95.74  \\
%     Madry et al. ~\cite{madry2017towards} & 99.41  & 96.01 & 95.49 & 95.36 \\
%     Song et al. ~\cite{song2018improving}  & 98.72 & 96.82 & 96.26 & 96.2  \\
%     $\DIAL_{\kl}$ (Ours) & 99.46 & 97.05 & 96.06 & 95.99  \\
%     $\DIAL_{\ce}$ (Ours) & $\mathbf{99.49}$  & $\mathbf{97.38}$ & $\mathbf{96.45}$ & $\mathbf{96.33}$ \\
%     \bottomrule
%   \end{tabular}
% \end{table}


% \paragraph{Attacking MNIST.} For consistency, we use the same perturbation and step sizes. For MNIST, we use $\epsilon=0.3$ and step size of $0.01$. The natural accuracy of our surrogate (source) model is 99.51\%. Attacks results are reported in Table~\ref{black-and_white-mnist}. It is worth noting that the improvement margin is not conclusive on MNIST as it is on CIFAR-10, which is a more complex task.

% \begin{table}
%   \caption{Black-box robustness on MNIST and CIFAR-10 using naturally trained surrogate model and best performing robust models}
%   \label{black-box-mnist-cifar}
%   \centering
%   \begin{tabular}{lllllll}
%     \toprule
%     & \multicolumn{3}{c}{MNIST} & \multicolumn{3}{c}{CIFAR-10} \\
%     \cmidrule(r){2-4} 
%     \cmidrule(r){5-7}  
%     Defense Model & PGD$^{40}$ & PGD$^{100}$ & PGD$^{1000}$ & PGD$^{20}$ & PGD$^{100}$ & PGD$^{1000}$ \\
%     \midrule
%     TRADES ~\cite{zhang2019theoretically} & 98.12 & 97.86 & 97.81 & 84.08 & 83.89 & 83.8 \\
%     MART ~\cite{wang2019improving} & 98.16 & 97.96 & 97.89  & 82.82 & 82.52 & 82.47 \\
%     Madry et al. ~\cite{madry2017towards}  & 98.05 & 97.73 & 97.78 & 84.22 & 84.14 & 83.96 \\
%     Song et al. ~\cite{song2018improving} & 97.74 & 97.28 & 97.34 & 75.59 & 75.37 & 75.11 \\
%     $\DIAL_{\kl}$ (Ours) & 98.14 & 97.83 & 97.87  & 84.3 & 84.18 & 84.0 \\
%     $\DIAL_{\ce}$ (Ours)  & $\mathbf{98.37}$ & $\mathbf{98.12}$ & $\mathbf{98.05}$  & $\mathbf{89.13}$ & $\mathbf{88.89}$ & $\mathbf{88.78}$ \\
%     \bottomrule
%   \end{tabular}
% \end{table}



% \subsubsection{Ensemble attack} In addition to the white-box and black-box settings, we evaluate our method on the Auto-Attack ~\citep{croce2020reliable} using $\ell_{\infty}$ threat model with perturbation $\epsilon=0.031$. Auto-Attack is an ensemble of parameter-free attacks. It consists of three white-box attacks: APGD-CE which is a step size-free version of PGD on the cross-entropy ~\citep{croce2020reliable}. APGD-DLR which is a step size-free version of PGD on the DLR loss ~\citep{croce2020reliable} and FAB which  minimizes the norm of the adversarial perturbations, and one black-box attack: square attack which is a query-efficient black-box attack ~\citep{andriushchenko2020square}. Results are presented in Table~\ref{auto-attack}. Based on the auto-attack leader-board \footnote{\url{https://github.com/fra31/auto-attack}}, our method achieves the 1st place among models without additional data using the WRN-34-10 architecture.

%Additional results can be found in Appendix ~\ref{additional_res}.

% \begin{table}
%   \caption{Auto-Attack (AA) on CIFAR-10 with perturbation size $\epsilon=0.031$ with $\ell_{\infty}$ threat model}
%   \label{auto-attack}
%   \centering
%   \begin{tabular}{lll}
%     \toprule
%     \cmidrule(r){1-2}
%     Defense Model & AA \\
%     \midrule
%     TRADES ~\cite{zhang2019theoretically} & 53.08  \\
%     MART ~\cite{wang2019improving} & 51.1  \\
%     Madry et al. ~\cite{madry2017towards} & 51.52    \\
%     Song et al.   ~\cite{song2018improving} & 40.18 \\
%     $\DIAL_{\ce}$ (Ours) & 47.33  \\
%     $\DIAL_{\kl}$ (Ours) & $\mathbf{53.75}$ \\
%     \midrule
%     DIAL-AWP (Ours) & $\mathbf{56.78}$ \\
%     TRADES-AWP \cite{wu2020adversarial} & 56.17 \\
%     \bottomrule
%   \end{tabular}
% \end{table}


% \begin{table}[!ht]
%   \caption{Auto-Attack (AA) Robustness (\%) on CIFAR-10 with $\epsilon=0.031$ using an $\ell_{\infty}$ threat model}
%   \label{auto-attack}
%   \centering
%   \begin{tabular}{cccccc|cc}
%     \toprule
%     % \multicolumn{8}{c}{Defence Model}  \\
%     % \cmidrule(r){1-8} 
%     TRADES & MART & Madry & Song & $\DIAL_{\ce}$ & $\DIAL_{\kl}$ & DIAL-AWP  & TRADES-AWP\\
%     \midrule
%     53.08 & 51.10 & 51.52 &  40.08 & 47.33  & $\mathbf{53.75}$ & $\mathbf{56.78}$ & 56.17 \\

%     \bottomrule
%   \end{tabular}
% \end{table}

% \begin{table}[!ht]
% \caption{$F_1$-robust measurement using PGD$^{20}$ attack in white-box and black-box settings on CIFAR-10}
%   \label{f1-robust}
%   \centering
%   \begin{tabular}{ccccccc|cc}
%     \toprule
%     % \multicolumn{8}{c}{Defence Model}  \\
%     % \cmidrule(r){1-8} 
%     Defense Model & TRADES & MART & Madry & Song & $\DIAL_{\kl}$ & $\DIAL_{\ce}$ & DIAL-AWP  & TRADES-AWP\\
%     \midrule
%     White-box & 0.659 & 0.666 & 0.657 & 0.518 & $\mathbf{0.675}$  & 0.643 & $\mathbf{0.698}$ & 0.682 \\
%     Black-box & 0.844 & 0.831 & 0.846 & 0.761 & 0.847 & $\mathbf{0.895}$ & $\mathbf{0.854}$ &  0.849 \\
%     \bottomrule
%   \end{tabular}
% \end{table}

\subsubsection{Robustness to Unforeseen Attacks and Corruptions}
\paragraph{Unforeseen Adversaries.} To further demonstrate the effectiveness of our approach, we test our method against various adversaries that were not used during the training process. We attack the model under the white-box settings with $\ell_{2}$-PGD, $\ell_{1}$-PGD, $\ell_{\infty}$-DeepFool and $\ell_{2}$-DeepFool \citep{moosavi2016deepfool} adversaries using Foolbox \citep{rauber2017foolbox}. We applied commonly used attack budget 
%(perturbation for PGD adversaries and overshot for DeepFool adversaries) 
with 20 and 50 iterations for PGD and DeepFool, respectively.
Results are presented in Table \ref{unseen-attacks}. As can be seen, our approach  gains an improvement of up to 4.73\% over the second best method under the various attack types and an average improvement of 3.7\% over all threat models.


\begin{table}[ht]
  \caption{Robustness on CIFAR-10 against unseen adversaries under white-box settings.}
  \vskip 0.1in
  \label{unseen-attacks}
  \centering
%   \small
  \begin{tabular}{c@{\hspace{1.5\tabcolsep}}c@{\hspace{1.5\tabcolsep}}c@{\hspace{1.5\tabcolsep}}c@{\hspace{1.5\tabcolsep}}c@{\hspace{1.5\tabcolsep}}c@{\hspace{1.5\tabcolsep}}c@{\hspace{1.5\tabcolsep}}c}
    \toprule
    Threat Model & Attack Constraints & $\DIAL_{\kl}$ & $\DIAL_{\ce}$ & AT & TRADES & MART & ATDA \\
    \midrule
    \multirow{2}{*}{$\ell_{2}$-PGD} & $\epsilon=0.5$ & 76.05 & \textbf{80.51} & 76.82 & 76.57 & 75.07 & 66.25 \\
    & $\epsilon=0.25$ & 80.98 & \textbf{85.38} & 81.41 & 81.10 & 80.04 & 71.87 \\\midrule
    \multirow{2}{*}{$\ell_{1}$-PGD} & $\epsilon=12$ & 74.84 & \textbf{80.00} & 76.17 & 75.52 & 75.95 & 65.76 \\
    & $\epsilon=7.84$ & 78.69 & \textbf{83.62} & 79.86 & 79.16 & 78.55 & 69.97 \\
    \midrule
    $\ell_{2}$-DeepFool & overshoot=0.02 & 84.53 & \textbf{88.88} & 84.15 & 84.23 & 82.96 & 76.08 \\\midrule
    $\ell_{\infty}$-DeepFool & overshoot=0.02 & 68.43 & \textbf{69.50} & 67.29 & 67.60 & 66.40 & 57.35 \\
    \bottomrule
  \end{tabular}
\end{table}


%%%%%%%%%%%%%%%%%%%%%%%%% conference version %%%%%%%%%%%%%%%%%%%%%%%%%%%%%%%%%%%%%
\paragraph{Unforeseen Corruptions.}
We further demonstrate that our method consistently holds against unforeseen ``natural'' corruptions, consists of 18 unforeseen diverse corruption types proposed by \citet{hendrycks2018benchmarking} on CIFAR-10, which we refer to as CIFAR10-C. The CIFAR10-C benchmark covers noise, blur, weather, and digital categories. As can be shown in Figure \ref{corruption}, our method gains a significant and consistent improvement over all the other methods. Our method leads to an average improvement of 4.7\% with minimum improvement of 3.5\% and maximum improvement of 5.9\% compared to the second best method over all unforeseen attacks. See Appendix \ref{corruptions-apendix} for the full experiment results.


\begin{figure}[h]
 \centering
  \includegraphics[width=0.4\textwidth]{figures/spider_full.png}
%   \caption{Summary of accuracy over all unforeseen corruptions compared to the second and third best performing methods.}
  \caption{Accuracy comparison over all unforeseen corruptions.}
  \label{corruption}
\end{figure}


%%%%%%%%%%%%%%%%%%%%%%%%% conference version %%%%%%%%%%%%%%%%%%%%%%%%%%%%%%%%%%%%%

%%%%%%%%%%%%%%%%%%%%%%%%% Arxiv version %%%%%%%%%%%%%%%%%%%%%%%%%%%%%%%%%%%%%
% \newpage
% \paragraph{Unforeseen Corruptions.}
% We further demonstrate that our method consistently holds against unforeseen "natural" corruptions, consists of 18 unforeseen diverse corruption types proposed by \cite{hendrycks2018benchmarking} on CIFAR-10, which we refer to as CIFAR10-C. The CIFAR10-C benchmark covers noise, blur, weather, and digital categories. As can be shown in Figure  \ref{spider-full-graph}, our method gains a significant and consistent improvement over all the other methods. Our approach leads to an average improvement of 4.7\% with minimum improvement of 3.5\% and maximum improvement of 5.9\% compared to the second best method over all unforeseen attacks. Full accuracy results against unforeseen corruptions are presented in Tables \ref{corruption-table1} and \ref{corruption-table2}. 

% \begin{table}[!ht]
%   \caption{Accuracy (\%) against unforeseen corruptions.}
%   \label{corruption-table1}
%   \centering
%   \tiny
%   \begin{tabular}{lcccccccccccccccccc}
%     \toprule
%     Defense Model & brightness & defocus blur & fog & glass blur & jpeg compression & motion blur & saturate & snow & speckle noise  \\
%     \midrule
%     TRADES & 82.63 & 80.04 & 60.19 & 78.00 & 82.81 & 76.49 & 81.53 & 80.68 & 80.14 \\
%     MART & 80.76 & 78.62 & 56.78 & 76.60 & 81.26 & 74.58 & 80.74 & 78.22 & 79.42 \\
%     AT &  83.30 & 80.42 & 60.22 & 77.90 & 82.73 & 76.64 & 82.31 & 80.37 & 80.74 \\
%     ATDA & 72.67 & 69.36 & 45.52 & 64.88 & 73.22 & 63.47 & 72.07 & 68.76 & 72.27 \\
%     DIAL (Ours)  & \textbf{87.14} & \textbf{84.84} & \textbf{66.08} & \textbf{81.82} & \textbf{87.07} & \textbf{81.20} & \textbf{86.45} & \textbf{84.18} & \textbf{84.94} \\
%     \bottomrule
%   \end{tabular}
% \end{table}


% \begin{table}[!ht]
%   \caption{Accuracy (\%) against unforeseen corruptions.}
%   \label{corruption-table2}
%   \centering
%   \tiny
%   \begin{tabular}{lcccccccccccccccccc}
%     \toprule
%     Defense Model & contrast & elastic transform & frost & gaussian noise & impulse noise & pixelate & shot noise & spatter & zoom blur \\
%     \midrule
%     TRADES & 43.11 & 79.11 & 76.45 & 79.21 & 73.72 & 82.73 & 80.42 & 80.72 & 78.97 \\
%     MART & 41.22 & 77.77 & 73.07 & 78.30 & 74.97 & 81.31 & 79.53 & 79.28 & 77.8 \\
%     AT & 43.30 & 79.58 & 77.53 & 79.47 & 73.76 & 82.78 & 80.86 & 80.49 & 79.58 \\
%     ATDA & 36.06 & 67.06 & 62.56 & 70.33 & 64.63 & 73.46 & 72.28 & 70.50 & 67.31 \\
%     DIAL (Ours) & \textbf{48.84} & \textbf{84.13} & \textbf{81.76} & \textbf{83.76} & \textbf{78.26} & \textbf{87.24} & \textbf{85.13} & \textbf{84.84} & \textbf{83.93}  \\
%     \bottomrule
%   \end{tabular}
% \end{table}


% \begin{figure}[!ht]
%   \centering
%   \includegraphics[width=9cm]{figures/spider_full.png}
%   \caption{Accuracy comparison with all tested methods over unforeseen corruptions.}
%   \label{spider-full-graph}
% \end{figure}
% %%%%%%%%%%%%%%%%%%%%%%%%% Arxiv version %%%%%%%%%%%%%%%%%%%%%%%%%%%%%%%%%%%%%
%%%%%%%%%%%%%%%%%%%%%%%%% Arxiv version %%%%%%%%%%%%%%%%%%%%%%%%%%%%%%%%%%%%%

\subsubsection{Transfer Learning}
Recent works \citep{salman2020adversarially,utrera2020adversarially} suggested that robust models transfer better on standard downstream classification tasks. In Table \ref{transfer-res} we demonstrate the advantage of our method when applied for transfer learning across CIFAR10 and CIFAR100 using the common linear evaluation protocol. see Appendix \ref{transfer-learning-settings} for detailed settings.

\begin{table}[H]
  \caption{Transfer learning results comparison.}
  \vskip 0.1in
  \label{transfer-res}
  \centering
  \small
\begin{tabular}{c|c|c|c}
\toprule

\multicolumn{2}{l}{} & \multicolumn{2}{c}{Target} \\
\cmidrule(r){3-4}
Source & Defence Model & CIFAR10 & CIFAR100 \\
\midrule
\multirow{3}{*}{CIFAR10} & DIAL & \multirow{3}{*}{-} & \textbf{28.57} \\
 & AT &  & 26.95  \\
 & TRADES &  & 25.40  \\
 \midrule
\multirow{3}{*}{CIFAR100} & DIAL & \textbf{73.68} & \multirow{3}{*}{-} \\
 & AT & 71.41 & \\
 & TRADES & 71.42 &  \\
%  \midrule
% \multirow{3}{}{SVHN} & DIAL &  &  & \multirow{3}{}{-} \\
%  & Madry et al. &  &  &  \\
%  & TRADES &  &  &  \\ 
\bottomrule
\end{tabular}
\end{table}


\subsubsection{Modularity and Ablation Studies}

We note that the domain classifier is a modular component that can be integrated into existing models for further improvements. Removing the domain head and related loss components from the different DIAL formulations results in some common adversarial training techniques. For $\DIAL_{\kl}$, removing the domain and related loss components results in the formulation of TRADES. For $\DIAL_{\ce}$, removing the domain and related loss components results in the original formulation of the standard adversarial training, and for $\DIAL_{\awp}$ the removal results in $\TRADES_{\awp}$. Therefore, the ablation studies will demonstrate the effectiveness of combining DIAL on top of different adversarial training methods. 

We investigate the contribution of the additional domain head component introduced in our method. Experiment configuration are as in \ref{defence-settings}, and robust accuracy is based on white-box PGD$^{20}$ on CIFAR-10 dataset. We remove the domain head from both $\DIAL_{\kl}$, $\DIAL_{\awp}$, and $\DIAL_{\ce}$ (equivalent to $r=0$) and report the natural and robust accuracy. We perform 3 random restarts and omit one standard deviation from the results. Results are presented in Figure \ref{ablation}. All DIAL variants exhibits stable improvements on both natural accuracy and robust accuracy. $\DIAL_{\ce}$, $\DIAL_{\kl}$, and $\DIAL_{\awp}$ present an improvement of 1.82\%, 0.33\%, and 0.55\% on natural accuracy and an improvement of 2.5\%, 1.87\%, and 0.83\% on robust accuracy, respectively. This evaluation empirically demonstrates the benefits of incorporating DIAL on top of different adversarial training techniques.
% \paragraph{semi-supervised extensions.} Since the domain classifier does not require the class labels, we argue that additional unlabeled data can be leveraged in future work.
%for improved results. 

\begin{figure}[ht]
  \centering
  \includegraphics[width=0.35\textwidth]{figures/ablation_graphs3.png}
  \caption{Ablation studies for $\DIAL_{\kl}$, $\DIAL_{\ce}$, and $\DIAL_{\awp}$ on CIFAR-10. Circle represent the robust-natural accuracy without using DIAL, and square represent the robust-natural accuracy when incorporating DIAL.
  %to further investigate the impact of the domain head and loss on natural and robust accuracy.
  }
  \label{ablation}
\end{figure}

\subsubsection{Visualizing DIAL}
To further illustrate the superiority of our method, we visualize the model outputs from the different methods on both natural and adversarial test data.
% adversarial test data generated using PGD$^{20}$ white-box attack with step size 0.003 and $\epsilon=0.031$ on CIFAR-10. 
Figure~\ref{tsne1} shows the embedding received after applying t-SNE ~\citep{van2008visualizing} with two components on the model output for our method and for TRADES. DIAL seems to preserve strong separation between classes on both natural test data and adversarial test data. Additional illustrations for the other methods are attached in Appendix~\ref{additional_viz}. 

\begin{figure}[h]
\centering
  \subfigure[\textbf{DIAL} on natural logits]{\includegraphics[width=0.21\textwidth]{figures/domain_ce_test.png}}
  \hspace{0.03\textwidth}
  \subfigure[\textbf{DIAL} on adversarial logits]{\includegraphics[width=0.21\textwidth]{figures/domain_ce_adversarial.png}}
  \hspace{0.03\textwidth}
    \subfigure[\textbf{TRADES} on natural logits]{\includegraphics[width=0.21\textwidth]{figures/trades_test.png}}
    \hspace{0.03\textwidth}
    \subfigure[\textbf{TRADES} on adversarial logits]{\includegraphics[width=0.21\textwidth]{figures/trades_adversarial.png}}
  \caption{t-SNE embedding of model output (logits) into two-dimensional space for DIAL and TRADES using the CIFAR-10 natural test data and the corresponding PGD$^{20}$ generated adversarial examples.}
  \label{tsne1}
\end{figure}


% \begin{figure}[ht]
% \centering
%   \begin{subfigure}{4cm}
%     \centering\includegraphics[width=3.3cm]{figures/domain_ce_test.png}
%     \caption{\textbf{DIAL} on nat. examples}
%   \end{subfigure}
%   \begin{subfigure}{4cm}
%     \centering\includegraphics[width=3.3cm]{figures/domain_ce_adversarial.png}
%     \caption{\textbf{DIAL} on adv. examples}
%   \end{subfigure}
  
%   \begin{subfigure}{4cm}
%     \centering\includegraphics[width=3.3cm]{figures/trades_test.png}
%     \caption{\textbf{TRADES} on nat. examples}
%   \end{subfigure}
%   \begin{subfigure}{4cm}
%     \centering\includegraphics[width=3.3cm]{figures/trades_adversarial.png}
%     \caption{\textbf{TRADES} on adv. examples}
%   \end{subfigure}
%   \caption{t-SNE embedding of model output (logits) into two-dimensional space for DIAL and TRADES using the CIFAR-10 natural test data and the corresponding adversarial examples.}
%   \label{tsne1}
% \end{figure}



% \begin{figure}[ht]
% \centering
%   \begin{subfigure}{6cm}
%     \centering\includegraphics[width=5cm]{figures/domain_ce_test.png}
%     \caption{\textbf{DIAL} on nat. examples}
%   \end{subfigure}
%   \begin{subfigure}{6cm}
%     \centering\includegraphics[width=5cm]{figures/domain_ce_adversarial.png}
%     \caption{\textbf{DIAL} on adv. examples}
%   \end{subfigure}
  
%   \begin{subfigure}{6cm}
%     \centering\includegraphics[width=5cm]{figures/trades_test.png}
%     \caption{\textbf{TRADES} on nat. examples}
%   \end{subfigure}
%   \begin{subfigure}{6cm}
%     \centering\includegraphics[width=5cm]{figures/trades_adversarial.png}
%     \caption{\textbf{TRADES} on adv. examples}
%   \end{subfigure}
%   \caption{t-SNE embedding of model output (logits) into two-dimensional space for DIAL and TRADES using the CIFAR-10 natural test data and the corresponding adversarial examples.}
%   \label{tsne1}
% \end{figure}



\subsection{Balanced measurement for robust-natural accuracy}
One of the goals of our method is to better balance between robust and natural accuracy under a given model. For a balanced metric, we adopt the idea of $F_1$-score, which is the harmonic mean between the precision and recall. However, rather than using precision and recall, we measure the $F_1$-score between robustness and natural accuracy,
using a measure we call
%We named it
the
\textbf{$\mathbf{F_1}$-robust} score.
\begin{equation}
F_1\text{-robust} = \dfrac{\text{true\_robust}}
{\text{true\_robust}+\frac{1}{2}
%\cdot
(\text{false\_{robust}}+
\text{false\_natural})},
\end{equation}
where $\text{true\_robust}$ are the adversarial examples that were correctly classified, $\text{false\_{robust}}$ are the adversarial examples that were miss-classified, and $\text{false\_natural}$ are the natural examples that were miss-classified.
%We tested the proposed $F_1$-robust score using PGD$^{20}$ on CIFAR-10 dataset in white-box and black-box settings. 
Results are presented in Table~\ref{f1-robust} and demonstrate that our method achieves the best $F_1$-robust score in both settings, which supports our findings from previous sections.

% \begin{table}[!ht]
%   \caption{$F_1$-robust measurement using PGD$^{20}$ attack in white and black box settings on CIFAR-10}
%   \label{f1-robust}
%   \centering
%   \begin{tabular}{lll}
%     \toprule
%     \cmidrule(r){1-2}
%     Defense Model & White-box & Black-box \\
%     \midrule
%     TRADES & 0.65937  & 0.84435 \\
%     MART & 0.66613  & 0.83153  \\
%     Madry et al. & 0.65755 & 0.84574   \\
%     Song et al. & 0.51823 & 0.76092  \\
%     $\DIAL_{\ce}$ (Ours) & 0.65318   & $\mathbf{0.88806}$  \\
%     $\DIAL_{\kl}$ (Ours) & $\mathbf{0.67479}$ & 0.84702 \\
%     \midrule
%     \midrule
%     DIAL-AWP (Ours) & $\mathbf{0.69753}$  & $\mathbf{0.85406}$  \\
%     TRADES-AWP & 0.68162 & 0.84917 \\
%     \bottomrule
%   \end{tabular}
% \end{table}

\begin{table}[ht]
\small
  \caption{$F_1$-robust measurement using PGD$^{20}$ attack in white and black box settings on CIFAR-10.}
  \vskip 0.1in
  \label{f1-robust}
  \centering
%   \small
  \begin{tabular}{c
  @{\hspace{1.5\tabcolsep}}c @{\hspace{1.5\tabcolsep}}c @{\hspace{1.5\tabcolsep}}c @{\hspace{1.5\tabcolsep}}c
  @{\hspace{1.5\tabcolsep}}c @{\hspace{1.5\tabcolsep}}c @{\hspace{1.5\tabcolsep}}|
  @{\hspace{1.5\tabcolsep}}c
  @{\hspace{1.5\tabcolsep}}c}
    \toprule
    % \cmidrule(r){8-9}
     & TRADES & MART & AT & ATDA & $\DIAL_{\ce}$ & $\DIAL_{\kl}$ & $\DIAL_{\awp}$ & $\TRADES_{\awp}$ \\
    \midrule
    White-box & 0.659 & 0.666 & 0.657 & 0.518 & 0.660 & \textbf{0.675} & \textbf{0.698} & 0.682 \\
    Black-box & 0.844 & 0.831 & 0.845 & 0.761 & \textbf{0.890} & 0.847 & \textbf{0.854} & 0.849 \\ 
    \bottomrule
  \end{tabular}
\end{table}


\section{Conclusion}
% \vspace{-0.5em}
\section{Conclusion}
% \vspace{-0.5em}
Recent advances in multimodal single-cell technology have enabled the simultaneous profiling of the transcriptome alongside other cellular modalities, leading to an increase in the availability of multimodal single-cell data. In this paper, we present \method{}, a multimodal transformer model for single-cell surface protein abundance from gene expression measurements. We combined the data with prior biological interaction knowledge from the STRING database into a richly connected heterogeneous graph and leveraged the transformer architectures to learn an accurate mapping between gene expression and surface protein abundance. Remarkably, \method{} achieves superior and more stable performance than other baselines on both 2021 and 2022 NeurIPS single-cell datasets.

\noindent\textbf{Future Work.}
% Our work is an extension of the model we implemented in the NeurIPS 2022 competition. 
Our framework of multimodal transformers with the cross-modality heterogeneous graph goes far beyond the specific downstream task of modality prediction, and there are lots of potentials to be further explored. Our graph contains three types of nodes. While the cell embeddings are used for predictions, the remaining protein embeddings and gene embeddings may be further interpreted for other tasks. The similarities between proteins may show data-specific protein-protein relationships, while the attention matrix of the gene transformer may help to identify marker genes of each cell type. Additionally, we may achieve gene interaction prediction using the attention mechanism.
% under adequate regulations. 
% We expect \method{} to be capable of much more than just modality prediction. Note that currently, we fuse information from different transformers with message-passing GNNs. 
To extend more on transformers, a potential next step is implementing cross-attention cross-modalities. Ideally, all three types of nodes, namely genes, proteins, and cells, would be jointly modeled using a large transformer that includes specific regulations for each modality. 

% insight of protein and gene embedding (diff task)

% all in one transformer

% \noindent\textbf{Limitations and future work}
% Despite the noticeable performance improvement by utilizing transformers with the cross-modality heterogeneous graph, there are still bottlenecks in the current settings. To begin with, we noticed that the performance variations of all methods are consistently higher in the ``CITE'' dataset compared to the ``GEX2ADT'' dataset. We hypothesized that the increased variability in ``CITE'' was due to both less number of training samples (43k vs. 66k cells) and a significantly more number of testing samples used (28k vs. 1k cells). One straightforward solution to alleviate the high variation issue is to include more training samples, which is not always possible given the training data availability. Nevertheless, publicly available single-cell datasets have been accumulated over the past decades and are still being collected on an ever-increasing scale. Taking advantage of these large-scale atlases is the key to a more stable and well-performing model, as some of the intra-cell variations could be common across different datasets. For example, reference-based methods are commonly used to identify the cell identity of a single cell, or cell-type compositions of a mixture of cells. (other examples for pretrained, e.g., scbert)


%\noindent\textbf{Future work.}
% Our work is an extension of the model we implemented in the NeurIPS 2022 competition. Now our framework of multimodal transformers with the cross-modality heterogeneous graph goes far beyond the specific downstream task of modality prediction, and there are lots of potentials to be further explored. Our graph contains three types of nodes. while the cell embeddings are used for predictions, the remaining protein embeddings and gene embeddings may be further interpreted for other tasks. The similarities between proteins may show data-specific protein-protein relationships, while the attention matrix of the gene transformer may help to identify marker genes of each cell type. Additionally, we may achieve gene interaction prediction using the attention mechanism under adequate regulations. We expect \method{} to be capable of much more than just modality prediction. Note that currently, we fuse information from different transformers with message-passing GNNs. To extend more on transformers, a potential next step is implementing cross-attention cross-modalities. Ideally, all three types of nodes, namely genes, proteins, and cells, would be jointly modeled using a large transformer that includes specific regulations for each modality. The self-attention within each modality would reconstruct the prior interaction network, while the cross-attention between modalities would be supervised by the data observations. Then, The attention matrix will provide insights into all the internal interactions and cross-relationships. With the linearized transformer, this idea would be both practical and versatile.

% \begin{acks}
% This research is supported by the National Science Foundation (NSF) and Johnson \& Johnson.
% \end{acks}

\section{Acknowledgement}
% This work has been supported by the National Key Research and Development Project of China (No. 2020AAA0104400). 
The authors would like to thank Mr. Hao Jin for helpful discussions.

\bibliography{ref}
\bibliographystyle{abbrvnat}
\newpage

\newpage
\newpage
\appendix
\section{Omitted Proofs in Section \ref{Section_SICRL}}
\label{Appendix_Proofs_4}
\proof{Proof of Theorem \ref{Theorem_Feasible}.}
By Lemma \ref{Theorem_Empirical_Bernstein}, 
$$
\PB\paren{|P(s^\prime|s,a)-\hat P(s^\prime|s,a)|\leq \sqrt{\frac{2\hat P(s^\prime|s,a)(1-\hat P(s^\prime|s,a))\log4/\delta}{n}}+\frac{4\log 4/\delta}{n}}\geq 1-\delta.
$$
By Lemma \ref{Theorem_Hoeffding_Inequality}, 
$$
\PB\paren{|P(s^\prime|s,a)-\hat P(s^\prime|s,a)|\leq \sqrt{\frac{\log 2/\delta}{2n}}}\geq  1-\delta.
$$
Combining the two inequalities in a union bound, we have:
$$
\PB\paren{|P(s^\prime|s,a)-\hat P(s^\prime|s,a)|\leq d_\delta(s,a,s^\prime)}\geq  1-2\delta.
$$
Again we apply the union bound argument to get:
$$
\PB(M\in M_\delta)=\PB\paren{|P(s^\prime|s,a)-\hat P(s^\prime|s,a)|\leq d_\delta(s,a,s^\prime),\forall s,s^\prime\in \gS,a\in \gA}\geq  1-2|\gS|^2|\gA|\delta.
$$
Finally, Problem (\ref{Problem_Optimistic}) is feasible as long as $P\in M_\delta$ because of Assumption $\ref{Assumption_Feasible}$.
\endproof
\section{Omitted Proofs in Section \ref{Section_Theory_SICRL}}
\label{Appendix_Proofs_SICRL}
First, we define some additional notations. 
Given a stationary policy $\pi$, we define the value function $V_\diamond^\pi(s)=\EB\paren{\sum_{t=0}^\infty \gamma^t r(s_t,a_t)|s_0=s}$, $V_\diamond^\pi=(V_\diamond^\pi(s_1), \ldots, V_\diamond^\pi(s_{|\gS|}))^\top\in \RB ^{|\gS|}$.
Thus we have $V^\pi_\diamond(\mu)=\mu^\top V_\diamond^\pi$.
Here $\diamond$ represents either the reward $r$ or cost $c_y$.
We use $Q_\diamond^\pi(s,a):=\EB\paren{\sum_{t=0}^\infty \gamma^t \diamond(s_t,a)}$ and $Q_\diamond^\pi=(Q_\diamond^\pi(s_1,a_1),...,Q_\diamond^\pi(s_{\gS},a_{|\gA|}))\in\RB^{|\gS|\cdot|\gA|}$ to denote the state-action value function. 
The local variance is defined as $\Var_P(V_\diamond^\pi)(s,a)=\EB_{s^\prime\sim P(\cdot|s,a)}(V_\diamond^\pi(s^\prime)-P(\cdot|s,a)V_\diamond^\pi)^2$.
We view $\Var_P(V^\pi)$ as vectors of length $|\gS|\cdot|\gA|$. 
We overload notation and let $P$ also refer to a matrix of size $(|\gS|\cdot |\gA|)\times |\gS|$, where the entry $P_{(s, a), s^{\prime}}$ is equal to $P(s^\prime|s,a)$. 
We also define $P^\pi$ to be the transition matrix on state-action pairs induced by a stationary policy $\pi$, namely:
$$P_{(s, a),\left(s^{\prime}, a^{\prime}\right)}^{\pi}:=P\left(s^{\prime}| s, a\right) \pi\left(a^{\prime} |s^{\prime}\right).
$$
We use $\widetilde V_\diamond^{\pi}(s), \widetilde Q_\diamond^\pi(s,a), {\Var_{\widetilde P}}(\widetilde{V}_\diamond^\pi)(s,a), \widetilde V_\diamond^\pi, \widetilde Q_\diamond^\pi,{\Var_{\widetilde P}}(\widetilde{V}^\pi), \widetilde P, \widetilde P^\pi$ to denote the value function, state-action value function, local variance, vector of the value function, vector of the state-action value function, vector of local variance, transition matrix, transition matrix on state-action pairs w.r.t. SICMDP $\widetilde M$, respectively.

\begin{lemma}\label{Lemma_Iteration_Comlexity_General}
    Suppose for all $t\in\{1,...,T\}$, 
    $$
    \frac{1}{1-\gamma}\sum_{s, a,s^\prime}z^{(t)}(s,a,s^\prime)c_{y^{(t)}}(s,a)- u_{y^{(t)}}\geq \max_{y\in Y}\left[\frac{1}{1-\gamma}\sum_{s, a,s^\prime}z^{(t)}(s,a,s^\prime)c_{y}(s,a)- u_{y}\right]-\epsilon,
    $$
    Then if we set $\eta=\epsilon$ and $T=O\left(\left[\frac{\mathrm{diam}(Y)|\gS|^2|\gA|}{(1-\gamma)\epsilon}\right]^m \right)$,
    then SI-CRL is guaranteed to output a $2\epsilon$-optimal solution of Problem~\ref{Problem_Optimistic_ELSIP}.
\end{lemma}
\proof{Proof of Lemma~\ref{Lemma_Iteration_Comlexity_General}}
For the convenience of presentation, then the problem can be written as
$$
\begin{aligned}
    \max_{z\in Z}\ &z^\top r\\
    \text{s.t.}\ &z^\top c_y\leq u_y,\ \forall y\in Y.
\end{aligned}
$$
Here $r,c\in[0,1]^{|\gS|^2|\gA|}$, $Z\subset\RB^{|\gS|^2|\gA|}$ is a feasible set defined by constraints other than the semi-infinite one.
Let $f(y,z)=z^\top c_y-u_y$, we note $f(y,z)$ is Lipschitz w.r.t. $y$ and the Lipschitz coefficient is $\beta:=\frac{2|\gS|^2|\gA|L_y}{1-\gamma}$.
WLOG, we also assume $Y$ is contained in a $\|\cdot\|_\infty$ ball with radius $R$ with $R\leq \frac{\mathrm{diam}(Y)}{2}$.
At iteration $t<T$, if we have $f(y^{(t)}, z^{(t)})\leq\epsilon$, then the algorithm terminates and we obtain a $2\epsilon$-optimal solution of Problem~\ref{Problem_Optimistic_ELSIP}.
Else we have $f(y^{(t)},z^{(t)})>\epsilon$.
Since $f(z,y)$ is $\beta$-Lipshitz in y, we can conclude $\forall z$, if $f(z,y)>\epsilon$ and $f(z,y^\prime)<0$, then $\|y-y^\prime\|_\infty>\epsilon/\beta$.
Define $B^{(t)}=\{\|y-y^{(t)}\|_\infty\leq \epsilon/2\beta\}$, as $f(y^{(t)},z^{(t)})>\epsilon$ and $f(y^{(t^\prime)},z^{(t)})\leq 0$, $t^\prime=1,...,t-1$, we have $B^{(t)}\cap\left(\cup_{t^\prime=1}^{t-1} B^{(t^\prime)}\right)=\emptyset$. 
Then by induction one may conclude $\{B^{(t^\prime)},t^\prime=1,...,t\}$ forms a $\epsilon/2\beta$-packing of $Y$.
Noting the fact that the $\epsilon/2\beta$-packing number of $Y$ is less than $\left(\frac{2R\beta}{\epsilon}\right)^m$, we complete the proof.

\endproof

\proof{Proof of Theorem~\ref{Theorem_Iteration_Complexity_Random_Search}.}
Since we have Lemma~\ref{Lemma_Iteration_Comlexity_General}, we only need to ensure that with probability at least $1-\delta$, 
for all $t\in\{1,...,T\}$, 
    $$
    \frac{1}{1-\gamma}\sum_{s, a,s^\prime}z^{(t)}(s,a,s^\prime)c_{y^{(t)}}(s,a)- u_{y^{(t)}}\geq \max_{y\in Y}\left[\frac{1}{1-\gamma}\sum_{s, a,s^\prime}z^{(t)}(s,a,s^\prime)c_{y}(s,a)- u_{y}\right]-\epsilon.
    $$
We adopt the notations introduced in the proof of Lemma~\ref{Lemma_Iteration_Comlexity_General}.
At the $t$ th iteration, let $y^*:=\argmax_{y\in Y}\left[\frac{1}{1-\gamma}\sum_{s, a,s^\prime}z^{(t)}(s,a,s^\prime)c_{y}(s,a)- u_{y}\right]$.
As $f(y,z)$ is $\beta$-Lipschitz w.r.t. $y$, then it suffices to ensure that with probability at least $1-\delta/T$ there exist $i\in\{1,...,M\}$ such that $\|y_i-y^*\|_\infty\leq \epsilon/\beta$.
As long as $\epsilon/\beta\leq \epsilon_0$, we simply need
$$
\PB\left(\exists i\in\{1,...,M\}, \|y_i-y^*\|_\infty\leq \epsilon/\beta\right)=1-\left(1-\left(\frac{\epsilon}{\beta R}\right)^m\right)^M\geq 1-\frac{\delta}{T}.
$$
The proof can be completed by basic algebra operations.
\endproof

\proof{Proof of Theorem~\ref{Theorem_Iteration_Complexity_Projected_GD}.}
Since we have Lemma~\ref{Lemma_Iteration_Comlexity_General}, we only need to ensure that for all $t\in\{1,...,T\}$, 
    $$
    \frac{1}{1-\gamma}\sum_{s, a,s^\prime}z^{(t)}(s,a,s^\prime)c_{y^{(t)}}(s,a)- u_{y^{(t)}}\geq \max_{y\in Y}\left[\frac{1}{1-\gamma}\sum_{s, a,s^\prime}z^{(t)}(s,a,s^\prime)c_{y}(s,a)- u_{y}\right]-\epsilon.
    $$
We adopt the notations introduced in the proof of Lemma~\ref{Lemma_Iteration_Comlexity_General}.
By Theorem 3.2 in \cite{bubeck2015convex}, the statement above is satisfied as long as $T_{PGA}\geq \frac{\beta^2R^2}{\epsilon^2}$.
The proof can be completed by basic algebra operations.
\endproof

\begin{lemma}\label{Lemma_Crude_Bound}
If Assumption \ref{Assumption_Two_Nonzero} is true and $M\in M_\delta$, we have 
$$
\left\|Q_r^\pi-\widetilde Q_r^\pi\right\|_\infty\leq \frac{2\gamma}{(1-\gamma)^2}\sqrt{\frac{\log 2/\delta}{2n}}
$$
\end{lemma}
\proof{Proof of Lemma~\ref{Lemma_Crude_Bound}.}
Given a stationary policy $\pi$, if Assumption \ref{Assumption_Two_Nonzero} is true and $M\in M_\delta$, 
$$\left\|\widetilde P(\cdot|s,a) -P(\cdot|s,a)\right\|_1 \leq 2\sqrt{\frac{\log 2/\delta}{2n}},\forall s\in \gS,a\in \gA,
$$
which implies
$$
\left\|(P-\widetilde P)V_r^\pi\right\|_\infty\leq \frac{2}{1-\gamma}\sqrt{\frac{\log 2/\delta}{2n}}.
$$
Then we have
$$\begin{aligned}
    \left\|Q_r^\pi-\widetilde Q_r^\pi\right\|_\infty&= \left\|\gamma\left(I-\gamma \widetilde{P}^{\pi}\right)^{-1}(P-\widetilde{P}) V_r^{\pi}\right\|_\infty\\
    &\leq \frac{\gamma}{1-\gamma}\left\|(P-\widetilde P)V_r^\pi\right\|_\infty\\
    &\leq \frac{2\gamma}{(1-\gamma)^2}\sqrt{\frac{\log 2/\delta}{2n}}
\end{aligned}
$$
\endproof

\begin{lemma}\label{Lemma_Bound_Variance}
Given a stationary policy $\pi$, when Assumption \ref{Assumption_Two_Nonzero} is true and $M\in M_\delta$, we have
$$\Var_{P}(V_r^\pi)\leq 2\Var_{\widetilde P}(\widetilde{V}_r^\pi) + \frac{6}{(1-\gamma)^2}\sqrt{\frac{\log 2/\delta}{2n}}+\frac{8\gamma^2}{(1-\gamma)^4}\frac{\log 2/\delta}{2n}.
$$
\end{lemma}
\proof{Proof of Lemma~\ref{Lemma_Bound_Variance}.}
For simplicity of notation, we drop the dependence on $\pi$.
By definition,
$$\begin{aligned}
\Var_P(V_r) &= \Var_P(V_r)-\Var_{\widetilde P}(V_r)+\Var_{\widetilde P}(V_r)\\
&= P(V_r)^2-(PV_r)^2-\widetilde P(V_r)^2 + (\widetilde P V_r)^2+\Var_{\widetilde P}(V_r)\\
&=(P-\widetilde P)(V_r)^2-\left[(PV_r)^2-(\widetilde PV_r)^2\right]+\Var_{\widetilde P}(V_r),
\end{aligned}
$$
where $(\cdot)^2$ means element-wise squares.
When Assumption \ref{Assumption_Two_Nonzero} is true and $M\in M_\delta$, by Lemma~\ref{Lemma_Crude_Bound},
$$
\begin{aligned}
\|(P-\widetilde P)(V_r)^2\|_\infty&\leq \frac{2}{(1-\gamma)^2}\sqrt{\frac{\log 2/\delta}{2n}}\\
\left\|\left[(PV_r)^2-(\widetilde PV_r)^2\right]\right\|_\infty&\leq \|PV_r +\widetilde PV_r\|_\infty\|PV_r -\widetilde PV_r\|_\infty\\
&\leq \frac{2}{1-\gamma}\left\|PV_r -\widetilde PV_r\right\|_\infty\\
&\leq \frac{4}{(1-\gamma)^2}\sqrt{\frac{\log 2/\delta}{2n}}.
\end{aligned}
$$
We also have
$$
\begin{aligned}
\Var_{\widetilde P}(V_r)&=\Var_{\widetilde P}(V_r-\widetilde{V_r}+\widetilde V_r)\\
&\leq 2\Var_{\widetilde P}(V_r-\widetilde{V_r}) + 2\Var_{\widetilde P} (\widetilde V_r)\quad \text{(AM–GM inequality)}\\
&\leq 2\left\|V_r-\widetilde{V_r}\right\|_\infty^2+2\Var_{\widetilde P} (\widetilde V_r)\\
&\leq \frac{8\gamma^2}{(1-\gamma)^4}\frac{\log 2/\delta}{2n}+2\Var_{\widetilde P} (\widetilde V_r)\quad \text{(Lemma \ref{Lemma_Crude_Bound})}.
\end{aligned}
$$
Therefore, we can get
$$\Var_{P}(V_r^\pi)\leq 2\Var_{\widetilde P}(\widetilde{V_r^\pi}) + \frac{6}{(1-\gamma)^2}\sqrt{\frac{\log 2/\delta}{2n}}+\frac{8\gamma^2}{(1-\gamma)^4}\frac{\log 2/\delta}{2n}.
$$
\endproof

\begin{lemma}\label{Lemma_Distance_between_P_tilde_P}
Let $p,\tilde p,\hat p\in[0,1]$ satisfy
$$
\begin{aligned}
|p-\hat p|&\leq \min\brc{\sqrt{\frac{2 \hat p(1-\hat p) \log 4/\delta}{n}}+\frac{4\log 4/\delta}{n},\sqrt{\frac{ \log 2/\delta}{2 n}}}\\
|\tilde p-\hat p|&\leq \min\brc{\sqrt{\frac{2 \hat p(1-\hat p) \log 4/\delta}{n}}+\frac{4\log 4/\delta}{n},\sqrt{\frac{ \log 2/\delta}{2 n}}}.
\end{aligned}
$$
Then
$$
|p-\tilde p|\leq \sqrt{\frac{8 p(1- p) \log 4/\delta}{n}}+4\paren{\frac{ \log 4/\delta}{n}}^{3/4}+\frac{8\log 4/\delta}{n}
$$
\end{lemma}

\proof{Proof of Lemma~\ref{Lemma_Distance_between_P_tilde_P}.}
Assume WLOG that $\hat p\geq p$.
Therefore,
$$\begin{aligned}
    |p-\hat p|&\leq \sqrt{\frac{2 p(1- p) \log 4/\delta}{n}}+\sqrt{\frac{2 (\hat p- p)(1- p) \log 4/\delta}{n}}+\frac{4\log 4/\delta}{n}\\
    &\leq \sqrt{\frac{2 p(1- p) \log 4/\delta}{n}}+\sqrt{\frac{2 \sqrt{\frac{ \log 2/\delta}{2 n}} \log 4/\delta}{n}}+\frac{4\log 4/\delta}{n}\\
    &\leq \sqrt{\frac{2 p(1-p) \log 4/\delta}{n}}+2^{1/4}\paren{\frac{ \log 4/\delta}{n}}^{3/4}+\frac{4\log 4/\delta}{n}.
\end{aligned}
$$
Similarly, we have
$$|\tilde p-\hat p|\leq \sqrt{\frac{2 p(1-p) \log 4/\delta}{n}}+2^{1/4}\paren{\frac{ \log 4/\delta}{n}}^{3/4}+\frac{4\log 4/\delta}{n}.
$$
Thus we may complete the proof using triangular inequality.
\endproof

\begin{lemma}\label{Lemma_Quasi_Bernstein}
Given a stationary policy $\pi$, suppose Assumption \ref{Assumption_Two_Nonzero} is true and $M\in M_\delta$, then 
$$
|(P-\widetilde P)V_r^\pi|\preceq \sqrt{\frac{8\Var_P(V_r^\pi)\log 4/\delta}{n}}+\frac{4}{1-\gamma}\paren{\frac{\log 4/\delta}{n}}^{3/4}+\frac{8\log 4/\delta}{n(1-\gamma)},
$$
where $\preceq$ means every element of LHS is less than or equal to the its counterpart in RHS.
\end{lemma}

\proof{Proof of Lemma~\ref{Lemma_Quasi_Bernstein}.}
Let $p=P(sa^+|s,a),\tilde p=\tilde P(sa^+|s,a)$. Applying Lemma \ref{Lemma_Distance_between_P_tilde_P} yields 
$$|p-\tilde p|\leq \sqrt{\frac{8 p(1-p) \log 4/\delta}{n}}+4\paren{\frac{ \log 4/\delta}{n}}^{3/4}+\frac{8\log 4/\delta}{n}.
$$
Assume WLOG that $V_r^\pi(sa^+)\geq V_r^\pi(sa^-) $.
Therefore we have
$$
\begin{aligned}
|(P(\cdot|s,a)-\tilde{P}(\cdot|s,a))^\top V_r^\pi|\leq &\sqrt{\frac{8 p(1- p) \log 4/\delta}{n}}(V_r^\pi(sa^+)-V_r^\pi(sa^-))+\frac{4}{1-\gamma}\paren{\frac{ \log 4/\delta}{n}}^{3/4}\\
&+\frac{8\log 4/\delta}{n(1-\gamma)}.
\end{aligned}
$$
Since
$$
\begin{aligned}
p(1- p)(V_r^\pi(sa^+)- V_r^\pi(sa^-))^2&=[ p V_r^\pi(sa^+)^2+(1- p) V_r^\pi(sa^-)^2]-[ p V_r^\pi(sa^+)+(1- p) V_r^\pi(sa^-)]^2\\
&=\Var_P(V_r^\pi)
\end{aligned}
$$
We may get
$$
|(P(\cdot|s,a)-\widetilde{P}(\cdot|s,a))^\top  V_r^\pi|\leq \sqrt{\frac{8\Var_P(V_r^\pi)(s,a)\log 4/\delta}{n}}+\frac{4}{1-\gamma}\paren{\frac{\log 4/\delta}{n}}^{3/4}+\frac{8\log 4/\delta}{n(1-\gamma)},
$$
which completes the proof.
\endproof

\begin{lemma}\label{Lemma_Bound_on_V_Same_Pi}
Given a stationary policy $\pi$, suppose Assumption \ref{Assumption_Two_Nonzero} is true and $M\in M_\delta$, then we have
$$\left\|V_r^\pi -\widetilde{V_r}^\pi\right\|_\infty\leq \frac{4}{(1-\gamma)^{3/2}}\sqrt{\frac{\log 4/\delta}{n}} +\frac{4\sqrt{6}}{(1-\gamma)^2}\left(\frac{\log 4/\delta}{n}\right)^{3/4}+ \frac{8}{(1-\gamma)^4}\left(\frac{\log 4/\delta}{n}\right)^{3/2}
$$
\end{lemma}
\proof{Proof of Lemma~\ref{Lemma_Bound_on_V_Same_Pi}.}
From Lemma \ref{Lemma_Simulation_Lemma}, Lemma \ref{Lemma_Norm_of_Inf_Horizon_Expectation}, Lemma \ref{Lemma_Bound_of_Weighted_Variance}, Lemma \ref{Lemma_Quasi_Bernstein} and the fact that $\left(I-\gamma \widetilde{P}^{\pi}\right)^{-1}$ has positive entries, we know
$$
\begin{aligned}
\left\|Q^\pi-\widetilde {Q}^\pi\right\|_\infty&=\gamma\left\|\left(I-\gamma \widetilde{P}^{\pi}\right)^{-1}(P-\widetilde{P}) V_r^{\pi}\right\|_\infty\\
&\leq \sqrt{\frac{8\log 4/\delta}{n}}\left\|\left(I-\gamma\widetilde{P}^{\pi}\right)^{-1}\sqrt{\Var_{P}(V_r^\pi)}\right\|_\infty+\frac{4}{(1-\gamma)^2}\left(\frac{\log 4/\delta}{n}\right)^{3/4}+\frac{8}{(1-\gamma)^2}\left(\frac{\log 4/\delta}{n}\right)\\
&\leq \sqrt{\frac{16\log 4/\delta}{n}}\left\|\left(I-\gamma\widetilde{P}^{\pi}\right)^{-1}\sqrt{\Var_{\widetilde P}(\widetilde{V_r}^\pi)}\right\|_\infty +\frac{4\sqrt{6}}{(1-\gamma)^2}\left(\frac{\log 4/\delta}{n}\right)^{3/4}+ \frac{8}{(1-\gamma)^3}\left(\frac{\log 4/\delta}{n}\right)\\
&\leq \frac{4}{(1-\gamma)^{3/2}}\sqrt{\frac{\log 4/\delta}{n}} +\frac{4\sqrt{6}}{(1-\gamma)^2}\left(\frac{\log 4/\delta}{n}\right)^{3/4}+ \frac{8}{(1-\gamma)^3}\left(\frac{\log 4/\delta}{n}\right).
\end{aligned}
$$
The proof is completed since $\left\|V_r^\pi -\widetilde{V_r}^\pi\right\|_{\infty}\leq\left\|Q^\pi -\widetilde{Q}^\pi\right\|_{\infty}$ by definitions.
\endproof

\begin{lemma}\label{Lemma_Bound_on_V_Same_Pi_Leading}
Suppose Assumption \ref{Assumption_Two_Nonzero} is true and $n>\frac{6\log 4/\delta}{(1-\gamma)^{5/2}}$, then with probability at least $1-2|\gS|^2|\gA|\delta$, we have
$$
\begin{aligned}
\left\|V_r^\pi -\widetilde{V_r}^\pi\right\|_\infty&\leq 12\sqrt{\frac{\log 4/\delta}{n(1-\gamma)^3}}\\
\left\|C^\pi -\widetilde{C}_y^\pi\right\|_\infty&\leq 12\sqrt{\frac{\log 4/\delta}{n(1-\gamma)^3}},\forall y\in Y\\
\end{aligned}
$$
\end{lemma}

\proof{Proof of Lemma~\ref{Lemma_Bound_on_V_Same_Pi_Leading}.}
When Assumption \ref{Assumption_Two_Nonzero} is true and $M\in M_\delta$, it follows from Lemma \ref{Lemma_Bound_on_V_Same_Pi} that
$$
\left\|V_r^\pi -\widetilde{V_r}^\pi\right\|_\infty\leq \frac{4}{(1-\gamma)^{3/2}}\sqrt{\frac{\log 4/\delta}{n}} +\frac{4\sqrt{6}}{(1-\gamma)^2}\left(\frac{\log 4/\delta}{n}\right)^{3/4}+ \frac{8}{(1-\gamma)^3}\left(\frac{\log 4/\delta}{n}\right).
$$
And by setting $n>\max\left\{\frac{36\log4/\delta}{(1-\gamma)^2},\frac{4\log4/\delta}{(1-\gamma)^3}\right\}$ we will get
$$\left\|V_r^\pi -\widetilde{V_r}^\pi\right\|_\infty\leq 12\sqrt{\frac{\log 4/\delta}{n(1-\gamma)^3}}.
$$
Similar arguments can be applied to bound $\left\|C_y^\pi-\widetilde{C}_y^\pi\right\|_\infty$. Since by Theorem \ref{Theorem_Feasible} we have 
$$\PB(M\in M_\delta)\geq 1-2|\gS|^2|\gA|\delta,
$$
the proof is completed.
\endproof


\proof{Proof of Theorem \ref{Lemma_Bound_on_V}.}
By Lemma \ref{Lemma_Bound_on_V_Same_Pi}, we know that with probability $1-2|\gS|^2|\gA|\delta$,
$$\begin{aligned}
\left\|V_r^{\tilde \pi}-\tilde V_r^{\tilde \pi}\right\|_\infty&\leq12\sqrt{\frac{\log 4/\delta}{n(1-\gamma)^3}}\\
\left\|V_r^{\pi^*}-\tilde V_r^{\pi^*}\right\|_\infty&\leq 12\sqrt{\frac{\log 4/\delta}{n(1-\gamma)^3}}.
\end{aligned}
$$
Thus
$$\begin{aligned}
|V_r^{\tilde \pi}(\mu)-\tilde V_r^{\tilde \pi}(\mu)|&\leq12\sqrt{\frac{\log 4/\delta}{n(1-\gamma)^3}}\\
|V_r^{\pi^*}(\mu)-\tilde V_r^{\pi^*}(\mu)|&\leq12\sqrt{\frac{\log 4/\delta}{n(1-\gamma)^3}}.
\end{aligned}
$$ 
Noting that $\tilde V_r^{\tilde\pi}(\mu)\geq\tilde V_r^{\pi^*}(\mu)$, we may get
$$
\begin{aligned}
V_r^{\pi^*}(\mu)-V_r^{\tilde \pi}(\mu)&\leq V_r^{\pi^*}(\mu)-\tilde V_r^{\pi^*}(\mu)+\tilde V_r^{\tilde \pi}(\mu)-V_r^{\tilde \pi}(\mu)\\
&\leq|V_r^{\pi^*}(\mu)-\tilde V_r^{\pi^*}(\mu)| + |\tilde V_r^{\tilde \pi}(\mu)-V_r^{\tilde \pi}(\mu)|\\
&\leq 24\sqrt{\frac{\log 4/\delta}{n(1-\gamma)^3}}.
\end{aligned}
$$

Similarly, when
$$|C_y^{\tilde \pi}(\mu)-\tilde C_y^{\tilde \pi}(\mu)|\leq12\sqrt{\frac{\log 4/\delta}{n(1-\gamma)^3}},\forall y\in Y,
$$
we may get 
$$C_y^{\tilde \pi}(\mu) - u_y \leq 12\sqrt{\frac{\log 4/\delta}{n(1-\gamma)^3}},\forall y\in Y.
$$
since $\tilde C_y^{\tilde \pi}(\mu)\leq u_y$.
\endproof
% \proof{Proof of Theorem \ref{Theorem_Sample_Complexity}.}
% Theorem \ref{Theorem_Sample_Complexity} is a direct corollary of Theorem \ref{Lemma_Bound_on_V}.
% \endproof

\proof{Proof of Theorem \ref{Theorem_Sample_Complexity_General}.}
The proof is nearly identical to the proof of Theorem 3 in \cite{LATTIMORE2014125}. The idea is to augment each state/action pair of the original MDP with $|\gS|-2$ states in the form of a binary tree as pictured in the diagram below. 

\begin{figure}[!htb]
    \centering
    \includegraphics[width=0.3\textwidth]{img/binary_tree.pdf}
    \label{Figure_Binary_Tree}
\end{figure}

The intention of the tree is to construct a SICMDP, $\bar M=\langle \bar \gS,\gA,Y,\bar P,\bar r,\bar c,u,\mu,\bar\gamma\rangle$ that with appropriate transition probabilities is functionally equivalent to $M$ while satisfying Assumption \ref{Assumption_Two_Nonzero}.
The rewards and costs in the added states are set to zero.
Since the tree has depth $d=O(\log_2|\gS|)$, it now takes $d$ time steps in the augmented SICMDP to change states once in the original SICMDP.
Therefore we must also rescale the discount factor $\bar \gamma$ by setting $\bar\gamma<\gamma^d$.
Now we have
$$
\begin{aligned}
|\bar \gS|&= O(|\gS|^2|\gA|)\\
\frac{1}{1-\bar \gamma}&=\frac{\log |\gS|}{1-\gamma}.
\end{aligned}
$$
Then we complete the proof by applying results in Theorem \ref{Theorem_Sample_Complexity}.
\endproof

\proof{Proof of Theorem \ref{Theorem_Sample_Complexity_General_Measure}.}
By Theorem \ref{Theorem_Chernoff_Inequality}, we have for any fixed $(s,a)\in \gS\times \gA$
$$
\begin{aligned}
\PB(n(s,a)<m\nu_{\min}/2)&\leq \PB(n(s,a)<m\nu(s,a)/2)\\
&\leq e^{-m\nu(s,a)}\left(\frac{em\nu(s,a)}{em\nu(s,a)/2}\right)^{\frac{m\nu(s,a)}{2}}\\
&=\left(\sqrt{\frac{e}{2}}\right)^{-\nu(s,a)m}\\
&\leq \left(\sqrt{\frac{e}{2}}\right)^{-\nu_{\min}m}
\end{aligned}
$$
\endproof
Let $m=\frac{2}{1-\log 2}\frac{\log 2|\gS||\gA|/\delta}{\nu_{\min}}$, we have $\PB(n(s,a)\geq m\nu_{\min}))\geq 1-\delta/2|\gS||\gA|$.
Therefore, with probability at least $1-\delta/2$, we can get
$$
n(s,a)>m\nu_{\min}, \forall (s,a)\in \gS\times \gA.
$$
Then our problem is reduced to the case that the offline dataset is generated by generative models.
The proof is completed by using results in Theorem \ref{Theorem_Sample_Complexity_General}.

\section{Omitted Proofs in Section \ref{Section_Theory_SICPO}}
\label{Appendix_Proofs_SICPO}

\begin{lemma}\label{Lemma_basic_expansion}
We have
$$
\begin{aligned}
&(1-\gamma)\alpha\sum_{t\in \gB} (V_r^*(\mu)-V_r^{(t)}(\mu))+(1-\gamma)\alpha\sum_{t\in\gN}\left(\eta-\left|\widehat V_{c^{(t)}}^{(t)}(\mu)-V_{c^{(t)}}^{(t)}(\mu)\right|\right)\\
&\leq \log|\gA|+\alpha\sum_{t\in\gB}\sqrt{E^{\nu^*}\left(r, \theta^{(t)},\hat w^{(t)}\right)}+\alpha\sum_{t\in\gN}\sqrt{E^{\nu^*}\left(c^{(t)}, \theta^{(t)},\hat w^{(t)}\right)}+\frac{\beta\alpha^2}{2}\sum_{t=0}^{T-1}\|\hat w^{(t)}\|_2^2.
\end{aligned}
$$
\end{lemma}
\proof{Proof of Lemma~\ref{Lemma_basic_expansion}}
Note that using Assumption~\ref{Assumption_smooth} and Taylor's expansion we have
$$
\log \frac{\pi_t(a | s)}{\pi_{t+1}(a | s)}+\nabla_{\theta} \log \pi_{t}(a | s)^\top\left(\theta_{t+1}-\theta_t\right) \leq \frac{\beta}{2}\left\|\theta_{t+1}-\theta_t\right\|^{2}.
$$
and $\theta^{(t+1)}-\theta^{(t)}=\alpha\hat w^{(t)}$.
As a result, suppose $t\in\gB$,
$$
\begin{aligned}
&\EB_{s\sim d^{\pi^*}}(D_{\operatorname{KL}}(\pi^*(\cdot|s)\|\pi^{(t)}(\cdot|s))-D_{\operatorname{KL}}(\pi^*(\cdot|s)\|\pi^{(t+1)}(\cdot|s)))\\
&=-\EB_{(s,a)\sim\nu^*}\log\frac{\pi^{(t)}(a|s)}{\pi^{(t+1)}(a|s)}\\
&\geq\alpha\EB_{(s,a)\sim\nu^*}[\nabla_{\theta} \log \pi^{(t)}(a | s)^\top \hat w^{(t)}]-\frac{\beta\eta_\theta^2}{2}\|\hat w^{(t)}\|^2_2\\
&=\alpha\EB_{(s,a)\sim\nu^*} A_r^{(t)}(s,a)+\alpha \EB_{(s,a)\sim\nu^*}\left[\nabla_\theta\log \pi^{(t)}(a|s)^\top \hat w^{(t)}-A_r^{(t)}(s,a)\right]-\frac{\beta\alpha^2}{2}\|\hat w^{(t)}\|^2_2\\
&\geq (1-\gamma)\alpha(V_r^*(\mu)-V_r^{(t)}(\mu))-\alpha\sqrt{E^{\nu^*}\left(r, \theta^{(t)},\hat w^{(t)}\right)}
-\frac{\beta\alpha^2}{2}\|\hat w^{(t)}\|^2_2\\
\end{aligned}
$$

The second last inequality is true due to the performance difference lemma, the last inequality is true due to Jensen's inequality and definition of the transferred function approximation error.
Rearranging terms yields
$$
\begin{aligned}
&(1-\gamma)\alpha(V_r^*(\mu)-V_r^{(t)}(\mu))\\
&\leq \EB_{s\sim d^{\pi^*}}(D_{\operatorname{KL}}(\pi^*(\cdot|s)\|\pi^{(t)}(\cdot|s))-D_{\operatorname{KL}}(\pi^*(\cdot|s)\|\pi^{(t+1)}(\cdot|s)))+\alpha\sqrt{E^{\nu^*}\left(r, \theta^{(t)},\hat w^{(t)}\right)}
+\frac{\beta\alpha^2}{2}\|\hat w^{(t)}\|^2_2.
\end{aligned}
$$
Similarly, suppose $t\in\gN$, we have $\theta^{(t+1)}-\theta^{(t)}=-\alpha\hat w^{(t)}$
$$
\begin{aligned}
&(1-\gamma)\alpha(V_{c^{(t)}}^{(t)}(\mu)-V_{c^{(t)}}^*(\mu))\\
&\leq \EB_{s\sim d^{\pi^*}}(D_{\operatorname{KL}}(\pi^*(\cdot|s)\|\pi^{(t)}(\cdot|s))-D_{\operatorname{KL}}(\pi^*(\cdot|s)\|\pi^{(t+1)}(\cdot|s)))+\alpha\sqrt{E^{\nu^*}\left(c^{(t)}, \theta^{(t)},\hat w^{(t)}\right)}
+\frac{\beta\alpha^2}{2}\|\hat w^{(t)}\|^2_2.
\end{aligned}
$$
Then we may get
$$
\begin{aligned}
&(1-\gamma)\alpha\sum_{t\in \gB} (V_r^*(\mu)-V_r^{(t)}(\mu))+(1-\gamma)\alpha\sum_{t\in \gN}\left(V_{c^{(t)}}^{(t)}(\mu)-V_{c^{(t)}}^*(\mu)\right)\\
&\leq \sum_{t=0}^{T-1} \EB_{s\sim d^{\pi^*}}(D_{\operatorname{KL}}(\pi^*(\cdot|s)\|\pi^{(t)}(\cdot|s))-D_{\operatorname{KL}}(\pi^*(\cdot|s)\|\pi^{(t+1)}(\cdot|s)))\\
&\quad+\alpha\sum_{t\in\gB}\sqrt{E^{\nu^*}\left(r, \theta^{(t)},\hat w^{(t)}\right)}+\alpha\sum_{t\in\gN}\sqrt{E^{\nu^*}\left(c^{(t)}, \theta^{(t)},\hat w^{(t)}\right)}+\frac{\beta\alpha^2}{2}\sum_{t=0}^{T-1}\|\hat w^{(t)}\|_2^2\\
&\leq \log|\gA|+\alpha\sum_{t\in\gB}\sqrt{E^{\nu^*}\left(r, \theta^{(t)},\hat w^{(t)}\right)}+\alpha\sum_{t\in\gN}\sqrt{E^{\nu^*}\left(c^{(t)}, \theta^{(t)},\hat w^{(t)}\right)}+\frac{\beta\alpha^2}{2}\sum_{t=0}^{T-1}\|\hat w^{(t)}\|_2^2.
\end{aligned}
$$
Since for $t\in\gN$
$$
\begin{aligned}
V_{c^{(t)}}^{(t)}(\mu)-V_{c^{(t)}}^*(\mu)&\geq \widehat V_{c^{(t)}}^{(t)}(\mu)-V_{c^{(t)}}^*(\mu)-\left|\widehat V_{c^{(t)}}^{(t)}(\mu)-V_{c^{(t)}}^{(t)}(\mu)\right|\\
&\geq u_{y_*^{(t)}}+\eta-u_{y_*^{(t)}}-\left|\widehat V_{c^{(t)}}^{(t)}(s)-V_{c^{(t)}}^{(t)}(\mu)\right|\\
&=\eta-\left|\widehat V_{c^{(t)}}^{(t)}(s)-V_{c^{(t)}}^{(t)}(\mu)\right|
\end{aligned}
$$
we may obtain the conclusion.
\endproof

\begin{lemma}\label{Lemma_SGD_convergence}
We have for $t\in\gB$, $\forall\delta\in(0,1)$, with probability at least $1-\delta$,
for $t\in\gB$
$$
\begin{aligned}
    & E^{\nu^*}(r,\theta^{(t)}, \hat w^{(t)})\\
    &\leq \frac{1}{1-\gamma}\left\|\frac{\nu^*}{\nu_0}\right\|_\infty\left(\epsilon_{bias}+C\frac{(4L_\pi^2W+8L_\pi/(1-\gamma))^2}{(1-\gamma)^2\mu_F}\frac{\log(T/\delta)}{K_{sgd}}+\frac{4\gamma^H}{1-\gamma}\left(\frac{1}{1-\gamma}+WL_\pi\right)\right),
\end{aligned}
$$
for $t\in\gN$,
$$
\begin{aligned}
     &E^{\nu^*}(c^{(t)},\theta^{(t)}, \hat w^{(t)})\\
     &\leq \frac{1}{1-\gamma}\left\|\frac{\nu^*}{\nu_0}\right\|_\infty\left(\epsilon_{bias}+C\frac{(4L_\pi^2W+8L_\pi/(1-\gamma))^2}{(1-\gamma)^2\mu_F}\frac{\log(T/\delta)}{K_{sgd}}+\frac{4\gamma^H}{1-\gamma}\left(\frac{1}{1-\gamma}+WL_\pi\right)\right),
\end{aligned}
$$
where $\nu_0$ is the uniform distribution on $\gS\times\gA$.
\end{lemma}

\proof{Proof of Lemma~\ref{Lemma_SGD_convergence}}
First we show for $t\in\gB$, with high probability
$$
\begin{aligned}
    &E^{\nu^*}(r,\theta^{(t)}, \hat w^{(t)})\\
    &\leq \frac{1}{1-\gamma}\left\|\frac{\nu^*}{\nu_0}\right\|_\infty\left(\epsilon_{bias}+C\frac{(4L_\pi^2W+8L_\pi/(1-\gamma))^2}{(1-\gamma)^2\mu_F}\frac{\log(T/\delta)}{K_{sgd}}+\frac{4\gamma^H}{1-\gamma}\left(\frac{1}{1-\gamma}+WL_\pi\right)\right).
\end{aligned}
$$
We have:
$$
\begin{aligned}
&E^{\nu^*}(r,\theta^{(t)}, \hat w^{(t)})\\
&\leq \left\|\frac{\nu^{*}}{\nu^{(t)}}\right\|_{\infty} E^{\nu^{(t)}}(r,\theta^{(t)}, \hat w^{(t)})\\
&\leq \frac{1}{1-\gamma} \left\|\frac{\nu^{*}}{\nu_{0}}\right\|_{\infty} E^{\nu^{(t)}}(r,\theta^{(t)}, \hat w^{(t)})\\
&=\frac{1}{1-\gamma}\left\|\frac{\nu^{*}}{\nu_{0}}\right\|_{\infty}\left( \min_w E^{\nu^{(t)}}(r,\theta^{(t)},w)+E^{\nu^{(t)}}\left(r,\theta^{(t)}, \hat w^{(t)}\right)-\min_w E^{\nu^{(t)}}(r,\theta^{(t)},w)\right)\\
&\leq \frac{1}{1-\gamma}\left\|\frac{\nu^{*}}{\nu_{0}}\right\|_{\infty}\left(\epsilon_{bias}+E^{\nu^{(t)}}\left(r, \theta^{(t)}, \hat w^{(t)}\right)-\min_{w\in B(0,W,\|\cdot\|_2)} E^{\nu^{(t)}}(r,\theta^{(t)},w)\right),
\end{aligned}
$$
the last step is due to Assumption~\ref{Assumption_func_approx_err} and Assumption~\ref{Assumption_est_err}.
Now we define a proximal loss function:
$$
\widetilde{E}^{\nu^{(t)}}(r,\theta^{(t)},w):=\EB_{(s,a)\sim\nu}(\widetilde{A}^{\pi^{(t)}}_r(s,a)-w^\top\nabla_\theta\log\pi_\theta(a|s))^2,
$$
where
$$
\begin{aligned}
    \widetilde{A}^{(t)}_r(s,a):&=\widetilde{Q}^{\pi^{(t)}}_r(s,a)-\widetilde{V}^{\pi^{(t)}}_r(s),\\
    \widetilde{Q}^{\pi^{(t)}}_r(s,a):&=\EB\left(\sum_{t=0}^H\gamma^t r(s_t,a_t)\mid (s_0,a_0)=(s,a)\right),\\
    \widetilde{V}^{\pi^{(t)}}_r(s):&=\EB\left(\sum_{t=0}^H\gamma^t r(s_t,a_t)\mid s_0=s\right).\\
\end{aligned}
$$
Recall that
$$
\begin{aligned}
&\widetilde{E}^{\nu^{(t)}}(r,\theta^{(t)},w)\\
&=\EB_{(s,a)\sim\nu^{(t)}}(\widetilde{A}^{(t)}(s,a)-w^\top\nabla_\theta\log\pi_{\theta^{(t)}}(a|s))^2\\    &=w^\top F(\theta^{(t)})w-2\sum_{s,a}\nu^{(t)}(s,a)w^\top \nabla_\theta\log\pi_{\theta^{(t)}}(a|s)\widetilde{A}^{(t)}(s,a)+\sum_{s,a}\nu^{(t)}(s,a)\widetilde{A}^{(t)}(s,a)^2.
\end{aligned}
$$
According to Assumption~\ref{Assumption_PSD_Fisher}, $E^{\nu^{(t)}}(r,\theta^{(t)},w)$ is $(1-\gamma)^2\mu_F$-strongly convex.
The full gradient is
$$G(w)=2F(\theta^{(t)})w-2\sum_{s,a}\nu^{(t)}(s,a)\widetilde{A}^{(t)}(s,a)\nabla_\theta\log\pi_{\theta^{(t)}}(a|s),
$$
and the stochastic gradient we use is 
$$\widehat G(w)=2\nabla_\theta\log\pi_\theta(A|S)\nabla_\theta\log\pi_\theta(A|S)^\top w-2\widehat A^{\pi_\theta}(S,A)\nabla_\theta\log\pi_\theta(A|S),
$$
where $(S,A)\sim \nu^{(t)}$.
Then we have $\EB\widehat G(w)=G(w)$ and
$$
\|G(w)-\widehat G(w)\|_2\leq 2L_\pi^2W+\frac{4L_\pi}{1-\gamma}.
$$
Set $\eta_k=\frac{2}{(1-\gamma)^2\mu_F(k+1)}$, $\gamma_k=\frac{2k}{K(K+1)}$, by Theorem C.3 in \cite{Harvey2019Simple}, for any $\delta\in(0,1)$, with probability at least $1-\delta$ we have,
$$
\widetilde{E}^{\nu^{(t)}}\left(r, \theta^{(t)}, \hat w^{(t)}\right)-\min_{w\in B(0,W,\|\cdot\|_2)} \widetilde{E}^{\nu^{(t)}}(r,\theta^{(t)},w)\leq C\frac{(4L_\pi^2W+8L_\pi/(1-\gamma))^2}{(1-\gamma)^2\mu_F}\frac{\log(1/\delta)}{K_{sgd}}.
$$
Here $C$ is a universal constant.
We may finish the bound by noting that
$$
\sup_{w\in B(0,W,\|\cdot\|_2)}\left|\widetilde{E}^{\nu^{(t)}}\left(r, \theta^{(t)}, \hat w^{(t)}\right)-E^{\nu^{(t)}}\left(r, \theta^{(t)}, \hat w^{(t)}\right)\right|\leq \frac{2\gamma^H}{1-\gamma}\left(\frac{1}{1-\gamma}+WL_\pi\right).
$$
For $t\in\gN$, the inequality holds by similar arguments as above.
And we would like to emphasize that $c^{(t)}$ is determined a priori and thus can be viewed as a fixed cost.
Our final conclusion can be obtained by a union bound argument.
% Set $\delta=\frac{1}{4L_\pi^2}$, by Theorem 1 in \cite{bach2013non} we may get:
% $$
% \EB\left[E^{\nu^{(t)}}\left(r,\theta^{(t)}, \hat w^{(t)}\right)-\min_w E^{\nu^{(t)}}(r,\theta^{(t)},w)\right]\leq \frac{2(\sigma\sqrt{d}+L_\pi W)^2}{K}.
% $$
% $\sigma$ is defined as:
% $$
% \EB_{(s,a)\sim\nu^{(t)}}\left[G_r^{(t)}\left(G_r^{(t)}\right)^\top\right]\preceq \sigma^2 F(\theta^{(t)}),
% $$
% where $G_{r}^{(t)}:=2\left(\left(w_{r}^{*}\right)^\top \nabla_{\theta} \log \pi^{(t)}(a \mid s)-\widehat{A}_{r}^{(t)}(s, a)\right) \nabla_{\theta} \log \pi^{(t)}(a \mid s)$ and $w_{r}^*:=\underset{w}{\mathrm{argmin}} E^{\nu^{(t)}}(r,\theta^{(t)},w).$
% We have $\sigma\leq 2WL_\pi +\frac{2}{1-\gamma}$.
% For $t\in\gN$, we may note that for any fixed $y\in Y$, $\theta$, and corresponding NPG update $\hat w$,
% $$
% \EB_{\hat w}E^{\nu^*}(c_y,\theta, \hat w)
% \leq \frac{1}{1-\gamma}\left\|\frac{\nu^*}{\nu_0}\right\|_\infty\left(\epsilon_{bias}+\frac{2\left(2\sqrt{d}WL_\pi+\frac{2\sqrt{d}}{1-\gamma}+WL_\pi\right)^2}{K}\right).
% $$ 
% The inquality holds by similar arguments as above.
% And the expectation is taken w.r.t. the randomness of $\hat w$.
% Therefore, we may get
% $$
% \EB E^{\nu^*}(c^{(t)},\theta^{(t)}, \hat w^{(t)})
% \leq \frac{1}{1-\gamma}\left\|\frac{\nu^*}{\nu_0}\right\|_\infty\left(\epsilon_{bias}+\frac{2\left(2\sqrt{d}WL_\pi+\frac{2\sqrt{d}}{1-\gamma}+WL_\pi\right)^2}{K}\right).
% $$ 
\endproof

\begin{lemma}\label{Lemma_whp_basic_expansion}
For any $\delta\in(0,1)$, let $A$ denote the event
$$
\begin{aligned}
&(1-\gamma)\alpha\sum_{t\in \gB} (V_r^*(\mu)-V_r^{(t)}(\mu))\\
&+(1-\gamma)\alpha|\gN|\left(\eta-\frac{1}{(1-\gamma)\sqrt{2K_{eval}}}\left(1+\sqrt{\log{(4T/\delta)}+m\log(4\mathrm{diam}(Y)L_y\sqrt{2K_{eval}}+\mathrm{diam}(Y))}\right)-\frac{\gamma^H}{1-\gamma}\right)\\
&\leq \log|\gA|+\frac{\beta\alpha^2 TW^2}{2(1-\gamma)^2}\\
&+\alpha T\sqrt{\frac{1}{1-\gamma}\left\|\frac{\nu^*}{\nu_0}\right\|_\infty\left(\epsilon_{bias}+C\frac{(4L_\pi^2W+8L_\pi/(1-\gamma))^2}{(1-\gamma)^2\mu_F}\frac{\log(T/\delta)}{K_{sgd}}+\frac{4\gamma^H}{1-\gamma}\left(\frac{1}{1-\gamma}+WL_\pi\right)\right)}.
\end{aligned}
$$
Then we have $\PB(A)\geq 1-\delta$.
\end{lemma}
\proof{Proof of Lemma~\ref{Lemma_whp_basic_expansion}}
Let $A_1$ denote the event that 
for $t\in\gB$
$$
\begin{aligned}
     &E^{\nu^*}(r,\theta^{(t)}, \hat w^{(t)})\\
     &\leq \frac{1}{1-\gamma}\left\|\frac{\nu^*}{\nu_0}\right\|_\infty\left(\epsilon_{bias}+C\frac{(4L_\pi^2W+8L_\pi/(1-\gamma))^2}{(1-\gamma)^2\mu_F}\frac{\log(2T/\delta)}{K_{sgd}}+\frac{4\gamma^H}{1-\gamma}\left(\frac{1}{1-\gamma}+WL_\pi\right)\right),
\end{aligned}
$$
for $t\in\gN$,
$$
\begin{aligned}
     &E^{\nu^*}(c^{(t)},\theta^{(t)}, \hat w^{(t)})\\
     &\leq \frac{1}{1-\gamma}\left\|\frac{\nu^*}{\nu_0}\right\|_\infty\left(\epsilon_{bias}+C\frac{(4L_\pi^2W+8L_\pi/(1-\gamma))^2}{(1-\gamma)^2\mu_F}\frac{\log(2T/\delta)}{K_{sgd}}+\frac{4\gamma^H}{1-\gamma}\left(\frac{1}{1-\gamma}+WL_\pi\right)\right).
\end{aligned}
$$
By Lemma~\ref{Lemma_SGD_convergence}, $\PB(A_1)\geq 1-\delta/2$.
Also note that we always have $\|\hat w^{(t)}\|_2^2\leq \frac{W^2}{(1-\gamma)^2}$, together with Lemma~\ref{Lemma_basic_expansion} lead to that conditioned on $A_1$,
$$
\begin{aligned}
&(1-\gamma)\alpha\sum_{t\in \gB} (V_r^*(\mu)-V_r^{(t)}(\mu))+(1-\gamma)\alpha\sum_{t\in\gN}\left(\eta-\left|\widehat V_{c^{(t)}}^{(t)}(\mu)-V_{c^{(t)}}^{(t)}(\mu)\right|\right)\\
&\leq \log|\gA|+\frac{\beta\alpha^2 TW^2}{2(1-\gamma)^2}\\
&\alpha T\sqrt{\frac{1}{1-\gamma}\left\|\frac{\nu^*}{\nu_0}\right\|_\infty\left(\epsilon_{bias}+C\frac{(4L_\pi^2W+8L_\pi/(1-\gamma))^2}{(1-\gamma)^2\mu_F}\frac{\log(2T/\delta)}{K_{sgd}}+\frac{4\gamma^H}{1-\gamma}\left(\frac{1}{1-\gamma}+WL_\pi\right)\right)}.
\end{aligned}
$$
It remains to bound $\left|\widehat V_{c^{(t)}}^{(t)}(\mu)-V_{c^{(t)}}^{(t)}(\mu)\right|$.
Let $A_2$ denote the event that $\forall t\in\{0,...,T-1\}$,
 $$
    \left|\widehat V_{c^{(t)}}^{(t)}(\mu)-V_{c^{(t)}}^{(t)}(\mu)\right|\leq \frac{1}{(1-\gamma)\sqrt{2K_{eval}}}\left(1+\sqrt{\log{(4T/\delta)}+m\log(4\mathrm{diam}(Y)L_y\sqrt{2K_{eval}}+\mathrm{diam}(Y))}+\frac{\gamma^H}{1-\gamma}\right)
    $$
By Lemma~\ref{Lemma_WHP_Bound_Evaluation_Uniform_y}, $\PB(A_2)\geq \delta/2$.
Thus conditioned on the event $A_1\cap A_2$,
$$
\begin{aligned}
&(1-\gamma)\alpha\sum_{t\in \gB} (V_r^*(\mu)-V_r^{(t)}(\mu))\\
&+(1-\gamma)\alpha|\gN|\left(\eta-\frac{1}{(1-\gamma)\sqrt{2K_{eval}}}\left(1+\sqrt{\log{(4T/\delta)}+m\log(4\mathrm{diam}(Y)L_y\sqrt{2K_{eval}}+\mathrm{diam}(Y))}\right)-\frac{\gamma^H}{1-\gamma}\right)\\
&\leq \log|\gA|+\frac{\beta\alpha^2 TW^2}{2(1-\gamma)^2}\\
&+\alpha T\sqrt{\frac{1}{1-\gamma}\left\|\frac{\nu^*}{\nu_0}\right\|_\infty\left(\epsilon_{bias}+C\frac{(4L_\pi^2W+8L_\pi/(1-\gamma))^2}{(1-\gamma)^2\mu_F}\frac{\log(2T/\delta)}{K_{sgd}}+\frac{4\gamma^H}{1-\gamma}\left(\frac{1}{1-\gamma}+WL_\pi\right)\right)}.
\end{aligned}
$$
We complete the proof by noting $\PB(A_1\cap A_2)\geq 1-\delta$.
\endproof

\begin{lemma}\label{Lemma_well_define_B}
    Suppose 
    $$\begin{aligned}
        &\frac{T}{2}\left(\eta-\frac{1}{(1-\gamma)\sqrt{2K_{eval}}}\left(1+\sqrt{\log{(4T/\delta)}+m\log(4\mathrm{diam}(Y)L_y\sqrt{2K_{eval}}+\mathrm{diam}(Y))}\right)-\frac{\gamma^H}{1-\gamma}\right)\\
        &>\frac{\log|\gA|}{(1-\gamma)\alpha}+\frac{\beta\alpha TW^2}{2(1-\gamma)^3}\\
        &+\frac{ T}{1-\gamma}\sqrt{\frac{1}{1-\gamma}\left\|\frac{\nu^*}{\nu_0}\right\|_\infty\left(\epsilon_{bias}+C\frac{(4L_\pi^2W+8L_\pi/(1-\gamma))^2}{(1-\gamma)^2\mu_F}\frac{\log(2T/\delta)}{K_{sgd}}+\frac{4\gamma^H}{1-\gamma}\left(\frac{1}{1-\gamma}+WL_\pi\right)\right)},
    \end{aligned}
    $$
Then $\gB\not=\emptyset$. Then for any $\delta\in(0,1)$, conditioned on event $A$, either a). $|\gB|\geq T/2$ or b). $\sum_{t\in\gB}(V_r^*(\mu)-V_r^{(t)}(\mu))< 0$.
\end{lemma}
\proof{Proof of Lemma~\ref{Lemma_well_define_B}}
    First we set $\sum_{t\in\gB}(V_r^*(\mu)-V_r^{(t)}(\mu))= 0$ if $\gB=\emptyset$. Now we would show $|\gB|\geq T/2$ as long as $\sum_{t\in\gB}(V_r^*(\mu)-V_r^{(t)}(\mu))\geq 0$, which also precludes the possibility that $\gB=\emptyset$.
    Note that when $\sum_{t\in\gB}(V_r^*(\mu)-V_r^{(t)}(\mu))\geq 0$ and $A$ happens,
    $$
    \begin{aligned}
    &|\gN|\left(\eta-\frac{1}{(1-\gamma)\sqrt{2K_{eval}}}\left(1+\sqrt{\log{(4T/\delta)}+m\log(4\mathrm{diam}(Y)L_y\sqrt{2K_{eval}}+\mathrm{diam}(Y))}\right)-\frac{\gamma^H}{1-\gamma}\right)\\
    &\leq\frac{\log|\gA|}{(1-\gamma)\alpha}+\frac{\beta\alpha TW^2}{2(1-\gamma)^3}\\
    &+\frac{ T}{1-\gamma}\sqrt{\frac{1}{1-\gamma}\left\|\frac{\nu^*}{\nu_0}\right\|_\infty\left(\epsilon_{bias}+C\frac{(4L_\pi^2W+8L_\pi/(1-\gamma))^2}{(1-\gamma)^2\mu_F}\frac{\log(2T/\delta)}{K_{sgd}}+\frac{4\gamma^H}{1-\gamma}\left(\frac{1}{1-\gamma}+WL_\pi\right)\right)}.
    \end{aligned}
    $$
    Since
    $$
    \begin{aligned}
    &\frac{T}{2}\left(\eta-\frac{1}{(1-\gamma)\sqrt{2K_{eval}}}\left(1+\sqrt{\log{(4T/\delta)}+m\log(4\mathrm{diam}(Y)L_y\sqrt{2K_{eval}}+\mathrm{diam}(Y))}\right)-\frac{\gamma^H}{1-\gamma}\right)\\
    &>\frac{\log|\gA|}{(1-\gamma)\alpha}+\frac{\beta\alpha TW^2}{2(1-\gamma)^3}\\
    &+\frac{ T}{1-\gamma}\sqrt{\frac{1}{1-\gamma}\left\|\frac{\nu^*}{\nu_0}\right\|_\infty\left(\epsilon_{bias}+C\frac{(4L_\pi^2W+8L_\pi/(1-\gamma))^2}{(1-\gamma)^2\mu_F}\frac{\log(2T/\delta)}{K_{sgd}}+\frac{4\gamma^H}{1-\gamma}\left(\frac{1}{1-\gamma}+WL_\pi\right)\right)}.
    \end{aligned}
    $$ we can conclude $|\gN|\leq T/2$ and $|\gB|\geq T/2$.
\endproof

\begin{lemma}\label{Lemma_reward_bound}
    Let $\alpha=1/\sqrt{T}$. Suppose we choose
    $$K_{eval}=\widetilde{O}\left(\frac{1}{\eta^2(1-\gamma)^2}\right)
    ,~H=O\left(\log\left(\frac{\eta(1-\gamma)}{\gamma}\right)\right)$$ and
    $$\begin{aligned}
    \eta&> \frac{4\log |\gA|}{\sqrt{T}(1-\gamma)}+\frac{2\beta W^2}{\sqrt{T}(1-\gamma)^3}\\
    &+\frac{4}{1-\gamma}\sqrt{\frac{1}{1-\gamma}\left\|\frac{\nu^*}{\nu_0}\right\|_\infty\left(\epsilon_{bias}+C\frac{(4L_\pi^2W+8L_\pi/(1-\gamma))^2}{(1-\gamma)^2\mu_F}\frac{\log(2T/\delta)}{K_{sgd}}+\frac{4\gamma^H}{1-\gamma}\left(\frac{1}{1-\gamma}+WL_\pi\right)\right)}
    \end{aligned}
    $$,
    then we have for any $\delta\in(0,1)$, conditioned on event $A$,
    $$
    \begin{aligned}
        &\frac{1}{|\gB|}\sum_{t\in\gB}(V_r^*(\mu)-V_r^{(t)}(\mu)) \leq \frac{2\log|\gA|}{\sqrt{T}(1-\gamma)}+\frac{\beta W^2}{\sqrt{T}(1-\gamma)^3}\\
       &+\frac{2}{1-\gamma}\sqrt{\frac{1}{1-\gamma}\left\|\frac{\nu^*}{\nu_0}\right\|_\infty\left(\epsilon_{bias}+C\frac{(4L_\pi^2W+8L_\pi/(1-\gamma))^2}{(1-\gamma)^2\mu_F}\frac{\log(2T/\delta)}{K_{sgd}}+\frac{4\gamma^H}{1-\gamma}\left(\frac{1}{1-\gamma}+WL_\pi\right)\right)}.
    \end{aligned}
    $$
    Here $\widetilde{O}$ means we discard any logarithmic terms. 
\end{lemma}
\proof{Proof of Lemma~\ref{Lemma_reward_bound}}
    When we choose $K_{eval}=\widetilde{O}\left(\frac{m}{\eta^2(1-\gamma)^2}\right)$ and $H=O\left(\log\left(\frac{\eta(1-\gamma)}{\gamma}\right)\right)$, we have
    $$
    \eta-\frac{1}{(1-\gamma)\sqrt{2K_{eval}}}\left(1+\sqrt{\log{(4T/\delta)}+m\log(4\mathrm{diam}(Y)L_y\sqrt{2K_{eval}}+\mathrm{diam}(Y))}\right)-\frac{\gamma^H}{1-\gamma}
    >\frac{\eta}{2}$$
    Setting
    $$\begin{aligned}
    \eta&> \frac{4\log |\gA|}{\sqrt{T}(1-\gamma)}+\frac{2\beta W^2}{\sqrt{T}(1-\gamma)^3}\\
    &+\frac{4}{1-\gamma}\sqrt{\frac{1}{1-\gamma}\left\|\frac{\nu^*}{\nu_0}\right\|_\infty\left(\epsilon_{bias}+C\frac{(4L_\pi^2W+8L_\pi/(1-\gamma))^2}{(1-\gamma)^2\mu_F}\frac{\log(2T/\delta)}{K_{sgd}}+\frac{4\gamma^H}{1-\gamma}\left(\frac{1}{1-\gamma}+WL_\pi\right)\right)}
    \end{aligned}
    $$
    results in
   $$\begin{aligned}
        &\frac{T}{2}\left(\eta-\frac{1}{(1-\gamma)\sqrt{2K_{eval}}}\left(1+\sqrt{\log{(4T/\delta)}+m\log(4\mathrm{diam}(Y)L_y\sqrt{2K_{eval}}+\mathrm{diam}(Y))}\right)-\frac{\gamma^H}{1-\gamma}\right)\\
        &>\frac{\log|\gA|}{(1-\gamma)\alpha}+\frac{\beta\alpha TW^2}{2(1-\gamma)^3}\\
        &\frac{ T}{1-\gamma}\sqrt{\frac{1}{1-\gamma}\left\|\frac{\nu^*}{\nu_0}\right\|_\infty\left(\epsilon_{bias}+C\frac{(4L_\pi^2W+8L_\pi/(1-\gamma))^2}{(1-\gamma)^2\mu_F}\frac{\log(2T/\delta)}{K_{sgd}}+\frac{4\gamma^H}{1-\gamma}\left(\frac{1}{1-\gamma}+WL_\pi\right)\right)},
    \end{aligned}
    $$
    Then we may use Lemma~\ref{Lemma_well_define_B} to conclude either a). $|\gB|\geq T/2$ or b). $\sum_{t\in\gB}(V_r^*(\mu)-V_r^{(t)}(\mu))< 0$.
    If $a)$, the conclusion holds trivially.
    Else via the combination of b) and the fact that we are in event $A$ we may obtain
$$
    \begin{aligned}
        &\frac{1}{|\gB|}\sum_{t\in\gB}(V_r^*(\mu)-V_r^{(t)}(\mu)) \leq \frac{2\log|\gA|}{\sqrt{T}(1-\gamma)}+\frac{\beta W^2}{\sqrt{T}(1-\gamma)^3}\\
       &+\frac{2}{1-\gamma}\sqrt{\frac{1}{1-\gamma}\left\|\frac{\nu^*}{\nu_0}\right\|_\infty\left(\epsilon_{bias}+C\frac{(4L_\pi^2W+8L_\pi/(1-\gamma))^2}{(1-\gamma)^2\mu_F}\frac{\log(2T/\delta)}{K_{sgd}}+\frac{4\gamma^H}{1-\gamma}\left(\frac{1}{1-\gamma}+WL_\pi\right)\right)}.
    \end{aligned}
    $$
    
\endproof

\begin{lemma}\label{Lemma_constraint_violation_random_search}
    Suppose Assumption~\ref{Assumption_regular_maxima} holds and we use random search (Algorithm~\ref{Algorithm_random_search}) to solve the inner-loop problem.
    Let $B$ denote the event that for any $t\in\{1,...,T\}$, 
    $$\begin{aligned}
    &\max_{y\in Y} \left[V_{c_y}^{(t)}(\mu)-u_y\right]-\left[\widehat V_{c_{y^{(t)}}}^{(t)}(\mu)-u^{(t)}\right]\\
    &\leq \frac{1}{(1-\gamma)\sqrt{2K_{eval}}}\left(1+\sqrt{\log{(4T/\delta)}+m\log(4\mathrm{diam}(Y)L_y\sqrt{2K_{eval}}+\mathrm{diam}(Y))}\right)+\frac{\gamma^H}{1-\gamma}+2\epsilon_{con},
    \end{aligned}$$
    We have $\PB(B\cap A_2)\geq 1-\delta$ as long as $M>\log(2T/\delta)/\log(1-\epsilon_{con}^m/[\operatorname{vol}(Y)(2L_y)^m(1-\gamma)^m])    
    $
\end{lemma}
\proof{Proof of Lemma~\ref{Lemma_constraint_violation_random_search}}
    According to Lemma~\ref{Lemma_WHP_Bound_Evaluation_Uniform_y}, conditioned on event $A_2$, we have $\forall t\in\{1,...,T\}$
     $$
    \sup_{y\in Y}\left|\widehat V_{c_y}^{(t)}(\mu)-V_{c_y}^{(t)}(\mu)\right|\leq \frac{1}{(1-\gamma)\sqrt{2K_{eval}}}\left(1+\sqrt{\log{(4T/\delta)}+m\log(4\mathrm{diam}(Y)L_y\sqrt{2K_{eval}}+\mathrm{diam}(Y))}\right)+\frac{\gamma^H}{1-\gamma}
    $$
    as long as $K_{eval}$ is large enough.
    Therefore, let $y^*\in\arg\max_{y\in Y} \left[V_{c_y}^\pi(\mu)-u_y\right]$, 
    $$
    \left|V_{c_{y^*}}^\pi(\mu)-\widehat{V}_{c_{y^*}}^\pi(\mu)\right|\leq \frac{1}{(1-\gamma)\sqrt{2K_{eval}}}\left(1+\sqrt{\log{(4T/\delta)}+m\log(4\mathrm{diam}(Y)L_y\sqrt{2K_{eval}}+\mathrm{diam}(Y))}\right)+\frac{\gamma^H}{1-\gamma}
    $$
    By Assmption~\ref{Assumption_Lipschitz}, as long as $\exists i\in\{1,...,M\}$, $\|y_i-y^*\|_\infty\leq \epsilon_{con}(1-\gamma)/2L_y$,
    we have 
    $$\left[\widehat V_{c_{y^*}}^{(t)}(\mu)-u_{ y^*}\right]-\max_{i\in\{1,...,M\}}\left[\widehat V_{c_{y_i}}^\pi(\mu)-u_{y_i}\right]\leq \epsilon_{con}.$$
     According to Assumption~\ref{Assumption_regular_maxima}, when $\epsilon_{con}(1-\gamma)/2L_y<\epsilon_0$
    $$
    \PB(\exists i\in\{1,...,M\}, \|y_i-\hat y\|_\infty\leq \epsilon_{con}(1-\gamma)/2L_y)\geq 1-\left(1-\frac{\epsilon_{con}^m(1-\gamma)^m}{\operatorname{vol}(Y)(2L_y)^m}\right)^M.
    $$
    If $M\geq M(\epsilon_{con},\beta)$, we have
    $$\PB\left(\left[\widehat V_{c_{y^*}}^{(t)}(\mu)-u_{ y^*}\right]-\max_{i\in\{1,...,M\}}\left[\widehat V_{c_{y_i}}^\pi(\mu)-u_{y_i}\right]\leq \epsilon\right)\geq 1-\beta,$$ 
    where we define
    $$M(\epsilon_{con},\beta)=\frac{\log\beta}{\log\left(1-\frac{\epsilon_{con}^m(1-\gamma)^m}{\operatorname{vol}(Y)(2L_y)^m}\right)}.
    $$
    Setting $M=M(\epsilon_{con},\delta/2T)$ and applying the union bound argument yields the result.
\endproof

\begin{lemma}\label{Lemma_constraint_violation_PGA}
    Suppose Assumption~\ref{Assumption_concave_constraint} holds and we use projected subgradient ascent (Algorithm~\ref{Algorithm_projected_GD}) to solve the inner-loop problem.
    Then 
    $$\begin{aligned}
    &\max_{y\in Y} \left[V_{c_y}^{(t)}(\mu)-u_y\right]-\left[\widehat V_{c_{y^{(t)}}}^{(t)}(\mu)-u^{(t)}\right]\\
    &\leq \frac{1}{(1-\gamma)\sqrt{2K_{eval}}}\left(1+\sqrt{\log{(4T/\delta)}+m\log(4\mathrm{diam}(Y)L_y\sqrt{2K_{eval}}+\mathrm{diam}(Y))}\right)+\frac{\gamma^H}{1-\gamma}+\epsilon_{con},
    \end{aligned}$$
    as long as $T_{PGA}>\frac{4[\mathrm{diam}(Y)]^2L_y^2}{\epsilon_{con}^2(1-\gamma)^2}
    $.
\end{lemma}
\proof{Proof of Lemma~\ref{Lemma_constraint_violation_PGA}}
Noting that the objective is $\frac{2L_y}{1-\gamma}$-Lipschitz, we may complete the proof by directly applying Theorem 3.2 in \cite{bubeck2015convex}.
\endproof
\proof{Proof of Theorem~\ref{Theorem_random_search_SICPO}}
Conditioned on event $A=A_1\cap A_2$, setting $K_{sgd}, H, K_{eval}, T, \alpha, \eta$ as above yields
$$\begin{aligned}
    \eta&> \frac{4\log |\gA|}{\sqrt{T}(1-\gamma)}+\frac{2\beta W^2}{\sqrt{T}(1-\gamma)^3}\\
    &+\frac{4}{1-\gamma}\sqrt{\frac{1}{1-\gamma}\left\|\frac{\nu^*}{\nu_0}\right\|_\infty\left(\epsilon_{bias}+C\frac{(4L_\pi^2W+8L_\pi/(1-\gamma))^2}{(1-\gamma)^2\mu_F}\frac{\log(2T/\delta)}{K_{sgd}}+\frac{4\gamma^H}{1-\gamma}\left(\frac{1}{1-\gamma}+WL_\pi\right)\right)}.
    \end{aligned}
$$
Now we use Lemma~\ref{Lemma_reward_bound} to get
$$
    \begin{aligned}
        &\frac{1}{|\gB|}\sum_{t\in\gB}(V_r^*(\mu)-V_r^{(t)}(\mu)) \leq \frac{2\log|\gA|}{\sqrt{T}(1-\gamma)}+\frac{\beta W^2}{\sqrt{T}(1-\gamma)^3}\\
       &+\frac{2}{1-\gamma}\sqrt{\frac{1}{1-\gamma}\left\|\frac{\nu^*}{\nu_0}\right\|_\infty\left(\epsilon_{bias}+C\frac{(4L_\pi^2W+8L_\pi/(1-\gamma))^2}{(1-\gamma)^2\mu_F}\frac{\log(2T/\delta)}{K_{sgd}}+\frac{4\gamma^H}{1-\gamma}\left(\frac{1}{1-\gamma}+WL_\pi\right)\right)}\\
       &\leq \epsilon+\frac{1}{(1-\gamma)^{3/2}}\sqrt{\left\|\frac{\nu^*}{\nu_0}\right\|_\infty\epsilon_{bias}}.
    \end{aligned}
    $$
According to Lemma~\ref{Lemma_constraint_violation_random_search}, conditioned on event $A_2\cap B$, as long as we set $M$ as in the statement of the theorem, we have $\forall t\in\{1,...,T\}$
$$\begin{aligned}
    &\sup_{y\in Y} \left[V_{c_y}^{(t)}(\mu)-u_y\right]-\left[\widehat V_{c_{y^{(t)}}}^{(t)}(\mu)-u^{(t)}\right]\\
    &\leq \frac{1}{(1-\gamma)\sqrt{2K_{eval}}}\left(1+\sqrt{\log{(4T/\delta)}+m\log(4\mathrm{diam}(Y)L_y\sqrt{2K_{eval}}+\mathrm{diam}(Y))}\right)+\frac{\gamma^H}{1-\gamma}+\epsilon/2,
    \end{aligned}$$
Thus for $t\in \gB$, 
$$ \sup_{y\in Y}\left[V_{c_y}^{(t)}(\mu)-u_y\right]\leq 2\epsilon+\frac{1}{(1-\gamma)^{3/2}}\sqrt{\left\|\frac{\nu^*}{\nu_0}\right\|_\infty\epsilon_{bias}}.
$$
We complete the proof by noting that
$$
P(A\cap B)\geq 1-(1-P(A))-(1-P(A_2\cap B))\geq 1-2\delta.
$$
\endproof

\proof{Proof of Theorem~\ref{Theorem_PGA_SICPO}}
Conditioned on event $A=A_1\cap A_2$, setting $K_{sgd}, H, K_{eval}, T, \alpha, \eta$ as above yields
$$\begin{aligned}
    \eta&> \frac{4\log |\gA|}{\sqrt{T}(1-\gamma)}+\frac{2\beta W^2}{\sqrt{T}(1-\gamma)^3}\\
    &+\frac{4}{1-\gamma}\sqrt{\frac{1}{1-\gamma}\left\|\frac{\nu^*}{\nu_0}\right\|_\infty\left(\epsilon_{bias}+C\frac{(4L_\pi^2W+8L_\pi/(1-\gamma))^2}{(1-\gamma)^2\mu_F}\frac{\log(2T/\delta)}{K_{sgd}}+\frac{4\gamma^H}{1-\gamma}\left(\frac{1}{1-\gamma}+WL_\pi\right)\right)}.
    \end{aligned}
$$
Now we use Lemma~\ref{Lemma_reward_bound} to get
$$
    \begin{aligned}
        &\frac{1}{|\gB|}\sum_{t\in\gB}(V_r^*(\mu)-V_r^{(t)}(\mu)) \leq \frac{2\log|\gA|}{\sqrt{T}(1-\gamma)}+\frac{\beta W^2}{\sqrt{T}(1-\gamma)^3}\\
       &+\frac{2}{1-\gamma}\sqrt{\frac{1}{1-\gamma}\left\|\frac{\nu^*}{\nu_0}\right\|_\infty\left(\epsilon_{bias}+C\frac{(4L_\pi^2W+8L_\pi/(1-\gamma))^2}{(1-\gamma)^2\mu_F}\frac{\log(2T/\delta)}{K_{sgd}}+\frac{4\gamma^H}{1-\gamma}\left(\frac{1}{1-\gamma}+WL_\pi\right)\right)}\\
       &\leq \epsilon+\frac{1}{(1-\gamma)^{3/2}}\sqrt{\left\|\frac{\nu^*}{\nu_0}\right\|_\infty\epsilon_{bias}}.
    \end{aligned}
    $$
According to Lemma~\ref{Lemma_constraint_violation_PGA}, conditioned on event $A_2$, as long as we set $T_{PGA}$ as in the statement of the theorem, we have $\forall t\in\{1,...,T\}$
$$\begin{aligned}
    &\sup_{y\in Y} \left[V_{c_y}^{(t)}(\mu)-u_y\right]-\left[\widehat V_{c_{y^{(t)}}}^{(t)}(\mu)-u^{(t)}\right]\\
    &\leq \frac{1}{(1-\gamma)\sqrt{2K_{eval}}}\left(1+\sqrt{\log{(4T/\delta)}+m\log(4\mathrm{diam}(Y)L_y\sqrt{2K_{eval}}+\mathrm{diam}(Y))}\right)+\frac{\gamma^H}{1-\gamma}+\epsilon/2,
    \end{aligned}$$
Thus for $t\in \gB$, 
$$ \sup_{y\in Y}\left[V_{c_y}^{(t)}(\mu)-u_y\right]\leq 2\epsilon+\frac{1}{(1-\gamma)^{3/2}}\sqrt{\left\|\frac{\nu^*}{\nu_0}\right\|_\infty\epsilon_{bias}}.
$$
\endproof


\begin{lemma}\label{Lemma_WHP_Bound_Evaluation_Uniform_y}
    For any $\delta\in(0,1)$, with probability at least $1-\delta$,
    $$
    \sup_{y\in Y}\left|\widehat V_{c_y}^{(t)}(\mu)-V_{c_y}^{(t)}(\mu)\right|\leq \frac{1}{(1-\gamma)\sqrt{2K_{eval}}}\left(1+\sqrt{\log{(2/\delta)}+m\log(4\mathrm{diam}(Y)L_y\sqrt{2K_{eval}}+\mathrm{diam}(Y))}\right)+\frac{\gamma^H}{1-\gamma}
    $$
    as long as $L_y\sqrt{2K_{eval}}>1$.
\end{lemma}
\proof{Proof of Lemma~\ref{Lemma_WHP_Bound_Evaluation_Uniform_y}}
Assumption~\ref{Assumption_Lipschitz} implies
$$
\left|\widehat V_{c_y}^{(t)}(\mu)-\widetilde{V}_{c_y}^{(t)}(\mu)\right|-\left|\widehat V_{c_{y^\prime}}^{(t)}(\mu)-\widetilde{V}_{c_{y^\prime}}^{(t)}(\mu)\right|\leq \frac{2L_y}{1-\gamma}\|y-y^\prime\|_{\infty}.
$$
Let $N_{\epsilon}:=\{y_1,...,y_N\}$ be a $\epsilon$ cover w.r.t. $\|\cdot\|_{\infty}$ of $Y$, Example 5.8 in \cite{wainwright2019high}. We have
$$
\log N\leq m\log(2\mathrm{diam}(Y)/\epsilon+\mathrm{diam}(Y)).
$$
Combining Theorem~\ref{Theorem_Hoeffding_Inequality} and the arguments of union bound we may have that for any $\delta\in(0,1)$, with probability at least $1-\delta$,
 $$
    \sup_{y\in N_{\epsilon/2}}\left|\widehat V_{c_y}^{(t)}(\mu)-\widetilde{V}_{c_y}^{(t)}(\mu)\right|\leq \frac{1}{1-\gamma}\sqrt{\frac{\log(2N/\delta)}{2K_{eval}}}.
    $$
Then we may get
$$
\begin{aligned}
    \sup_{y\in Y}\left|\widehat V_{c_y}^{(t)}(\mu)-\widetilde{V}_{c_y}^{(t)}(\mu)\right|&\leq
    \frac{2L_y\epsilon}{1-\gamma}+
    \frac{1}{1-\gamma}\sqrt{\frac{\log(2/\delta)+m\log(2\mathrm{diam}(Y)/\epsilon+\mathrm{diam}(Y))}{2K_{eval}}}.
\end{aligned}
    $$
Here we set $\epsilon=\frac{1}{2L_y\sqrt{2K_{eval}}}$ and get
$$
    \sup_{y\in Y}\left|\widehat V_{c_y}^{(t)}(\mu)-\widetilde{V}_{c_y}^{(t)}(\mu)\right|\leq \frac{1}{(1-\gamma)\sqrt{2K_{eval}}}\left(1+\sqrt{\log{(2/\delta)}+m\log(4\mathrm{diam}(Y)L_y\sqrt{2K_{eval}}+\mathrm{diam}(Y))}\right).
$$
Since
$$
\sup_{y\in Y}\left|V_{c_y}^{(t)}(\mu) -\widetilde{V}_{c_y}^{(t)}(\mu)\right|\leq \frac{\gamma^H}{1-\gamma},
$$
we complete the proof.
\endproof
\section{Auxiliary Lemmas}

\begin{lemma}[Empirical Bernstein Inequality]
\label{Theorem_Empirical_Bernstein}
Suppose $n\geq 3$, $\{X_1,...,X_n\}$ be $n$ i.i.d. random variables with values in $[0,1]$. 
Let $\delta>0$. 
Then with probability at least $1-\delta$ we have
$$
\left|\EB X_1-\frac{\sum_{i=1}^n X_i}{n}\right|\leq\sqrt{\frac{2\mathbb{V}_n(X_{1:n})\log4/\delta}{n}}+\frac{4\log 4/\delta}{n},
$$
where $\mathbb{V}_n(X_{1:n}):=\frac{1}{n(n-1)}\sum_{i,j}\frac{(X_i-X_j)^2}{2}$ denotes the empirical variance of the dataset $\{X_1,...,X_n\}$.
\end{lemma}
\proof{Proof of Lemma~\ref{Theorem_Empirical_Bernstein}.}
See Theorem 11 in \cite{maurer2009empirical}.
\endproof

\begin{lemma}[Hoeffding's Inequality]
\label{Theorem_Hoeffding_Inequality}
Suppose $\{X_1,...,X_n\}$ be $n$ i.i.d. random variables with values in $[0,1]$.
Let $\delta>0$. 
Then with probability at least $1-\delta$ we have
$$
\left|\EB X_1-\frac{\sum_{i=1}^n X_i}{n}\right|\leq\sqrt{\frac{\log 2/\delta}{2n}}
$$
\end{lemma}
\proof{Proof of Lemma~\ref{Theorem_Hoeffding_Inequality}.}
See Theorem 2.2.6 in \cite{vershynin_2018}.
\endproof

\begin{lemma}\label{Lemma_Simulation_Lemma}
For any policy $\pi$ and transition probabilities $P$, $\widetilde{P}$, we have that
$$Q^{\pi}-\widetilde{Q}^{\pi}=\gamma\left(I-\gamma \widetilde{P}^{\pi}\right)^{-1}(P-\widetilde{P}) V^{\pi}
$$
\end{lemma}
\proof{Proof of Lemma~\ref{Lemma_Simulation_Lemma}.}
See Lemma 2 in \cite{pmlr-v125-agarwal20b}.
\endproof

\begin{lemma}\label{Lemma_Norm_of_Inf_Horizon_Expectation}
For any policy $\pi$, any transition probability $P$ and any vector $v\in \RB^{|\gS|\cdot|\gA|}$, we have 
$$\left\|\left(I-\gamma P^{\pi}\right)^{-1} v\right\|_{\infty} \leq\|v\|_{\infty} /(1-\gamma).
$$
\end{lemma}
\proof{Proof of Lemma~\ref{Lemma_Norm_of_Inf_Horizon_Expectation}.}
See Lemma 3 in \cite{pmlr-v125-agarwal20b}.
\endproof

\begin{lemma}\label{Lemma_Bound_of_Weighted_Variance}
For any policy $\pi$ and any transition probability $P$, we have
$$\left\|(I-\gamma P^\pi)^{-1} \sqrt{\Var_P^\pi}\right\|_\infty\leq\sqrt{\frac{2}{(1-\gamma)^3}},
$$
where $\sqrt{\cdot}$ is defined as the element-wise square root.
\end{lemma}
\proof{Proof of Lemma~\ref{Lemma_Bound_of_Weighted_Variance}.}
See Lemma 4 in \cite{pmlr-v125-agarwal20b}.
\endproof

\begin{theorem}[Chernoff's Inequality]
\label{Theorem_Chernoff_Inequality}
Let $X_i$ be independent Bernoulli random variables with parameter $p_i$. 
Consider their sum $S_N=\sum_{i=1}^N X_i$ and denote its mean by $\mu=\EB S_N$. 
Then, for any $t<\mu$, we have
$$
\PB\paren{S_N<t}\leq e^{-\mu}\paren{\frac{e\mu}{t}}^t.
$$
\end{theorem}
\proof{Proof of Theorem~\ref{Theorem_Chernoff_Inequality}.}
See \cite{vershynin_2018}.
\endproof
% \section{Details of Numerical Experiments}
\section{Experimental Setup}
\label{Appendix_Detials_of_Experiments}
We use PyTorch, RLlib, and Gurobi framework to implement the algorithms in our work, and the codes are run on Nvidia RTX Titan GPUs and Intel Xeon Gold 6132 CPUs with 252GB memory. The code of our algorithms and construction of the corresponding environments has been released on
\href{https://github.com/pengyang7881187/SICMDP-new}{https://github.com/pengyang7881187/SICMDP-new}.

% \subsection{Construction of Toy SICMDP}
% % \liangyu{Specify the function $f$ we actually used!}

% Without loss of generality, assume $Y=[0, 3]$. We split $Y=[0, 3]$ to $Y_1=[0, 1]$, $Y_2=[1, 2]$ and $Y_3=[2, 3]$. Intuitively, in $Y_1$ ($Y_3$), we restrict the agent to take action $a_0$ in $s^0$ ($s^1$) with the given probability from the target policy $\tilde{\pi}$. (Recall, taking action $a_0$ is always better if we set aside the constraints.) We introduce $Y_2$ to obtain $L$-Lipschitz of $c_y$ and $u_y$. 
% Assume $f\colon [-0.5, 0.5] \rightarrow \mathbb{R}$ is a continuous-differentiable even function, with unique maximum point $0$ and zero point $0.5$, we use $f(x)=(1+\cos(2\pi x))\cos(2\pi x)$ in practice. Let $c_y(s^0, a_0)=f(y-0.5)$ for $y\in Y_1$, $c_y(s^1, a_0)=f(y-2.5)$ for $y\in Y_3$, and $c\equiv 0$ otherwise.
% For each $\gamma\in(0, 1)$, we define $u^1=C^{\tilde{\pi}}_{0.5}>0$ and $u^1=C^{\tilde{\pi}}_{2.5}>0$. Let $u_y\equiv u^1$ in $Y_1$, $u_y \equiv u^2$ in $Y_3$ and we make linear interpolate $u_y$ in $Y_2$.

% So far, we have constructed Lipschitz $c_y$ and $u_y$. The only active constraints are $C^{\pi}_{0.5}\leq u^1$ and $C^{\pi}_{2.5}\leq u^2$ and the optimal policy is the known target policy $\tilde{\pi}$.

\subsection{Details of Construction of Discharge of Sewage}
We generate this environment randomly with $|\gS|=8$, $|\gA|=4$ and $\gamma=0.9$. In each experiment, we sample the environments several times and report the average result.

Positions of sewage outfalls, transition dynamics and rewards are sampled uniformly on $Y$, probability simplex and $[0, 1]$ respectively. The state-occupancy measure $d$ is then generated by the uniform policy and $\Delta$ is set to $10^{-6}$.

The optimal policy $\pi^*$ used is obtained by solving a corresponding linear programming with true transition dynamic $P$ and a fine grid of $Y$ of size $10^6$. 
% Solving such a linear programming with many constraints is very slow as shown in Figure \ref{Figure_Sewage_time}.

% Assume $f\colon [0, +\infty) \rightarrow [0, +\infty)$ is a continuous-differentiable decreasing function, we use $f(x)=\frac{1}{1+x^2}$ in practice. Let $c_y(s, a)=c_y(s)=f(\|y-s\|_2)$, where $s$ also represents the position of the state (outfall).

% Given a target state-occupancy measure $d$ which can be generated randomly or specified in advance, we define $u_y=(1+\Delta)\sum_{s\in S} d(s)c_y(s)$, where $\Delta$ is a small positive number. The feasibility of the resulting SICMDP is not guaranteed even if $\Delta=0$, we reject those infeasible environments and re-sample. The SICMDP would be nontrivial if we choose a suitable $\Delta$.

\subsection{Implementation of SI-CRL}
We would like to clarify how to solve the maximization problem to generate new $y_t$ in $t$th iteration in SI-CRL. Since the $u_y$ can be non-convex in $y$ and evaluating $\widehat V^\pi_{c_y}-u_y$ for multiple $y$ is much cheaper in the model-based setting with small $|\gS|$ and $|\gA|$ than solving the linear programming, we choose to solve the maximization problem by brute force.
Specifically, we first create a grid of $Y$ of size $10^5$, and then find the grid point with max objective.
This method works well since in the problems we consider $Y$ is of low dimensions.

When solving the linear programming, we do not force Gurobi to use dual simplex method in SI-CRL algorithm, and find that it achieves an even better re-optimization performance in practice. 

\subsection{Implementation of SI-CPO}
The parameterized policy class is chosen as the softmax policy: $\pi_\theta(a|s)=e^{\theta_{s,a}}/\sum_{a^\prime\in \gA}e^{\theta_{s,a^\prime}}$, where $\theta\in\RB^{|\gS||\gA|}$ is initialized as $0$. This policy class satisfies the assumptions of the theoretical analysis of SI-CPO.

We use learning rate $\alpha=1$, tolerance $\eta=0.013$, maximum iteration
number $T=10000$ for both SI-CPO and baseline. SI-CPO and baseline use sample-based NPG, Algorithm \ref{Algorithm_sample_based_NPG}, as the policy optimization subroutine, and share the same hyper-parameters: number of evaluation paths $K_{eval}=100000$, number of training paths $K_{sgd}=1000$, fixed horizon $H=100$, upper bound of parameters' norm $W=1000$, constant learning rate $1$ and weight $\gamma_k=2k/K_{sgd}(K_{sgd}+1)$.

In $t$th iteration in SI-CPO, we sample $100$ points uniformly in $Y$ and find the best one to solve the maximization problem approximately as Algorithm \ref{Algorithm_random_search}. In the model-free setting, evaluating $\widehat V^\pi_{c_y}$ is much more computationally expensive than the model-based setting, hence brute force is impractical. Additionally, the random search method can yield a better policy than the one utilizing constraint optimization algorithm in practice. 

\subsection{Details of Construction of Ship Route Planning}
We fix the outset $O=(0, 0)$, destination $D=(1, 1)$, environmentally critical point $MPA=(\frac{1}{2}, \frac{1}{2})$ and we assume the ship sails with constant speed $0.1$. $c_y$ and $u_y$ are designed to make sure the trajectory along the curve $y=x^4$ satisfies the constraint.

\subsection{Implementation of SI-CPPO}
The actor, constraint critic and reward critic networks have two hidden layers of size 512 with tanh non-linearities without sharing layers for both SI-CPPO and baseline. We update the networks using the Adam optimizer with learning rate $10^{-4}$. SI-CPPO and baseline are modified from the standard implementation of PPO in RLlib and share the same hyper-parameters. See more details in Algorithm \ref{Algorithm_SICPPO}, note that we use the generalized advantage estimation in practice instead of the one-step estimation in the pseudo-code.

In $t$th iteration in SI-CPPO, we use a trust-region method to solve the maximization problem based on the trajectories collected by $4$ roll-out workers. We call scipy.optimize.minimize to implement it, if the optimization process fails to converge, we switch to use random search with $100$ points uniformly in $Y$ temporarily. Empirically, the optimization process seldom fails and this strategy outperforms random search.

\begin{algorithm}[htb!]
   \caption{SI-CPPO}
   \label{Algorithm_SICPPO}
\begin{algorithmic}
   \STATE {\bfseries Input:} state space $\gS$, action space $\gA$, reward function $r$, a continuum of cost function $c$, index set $Y$, value for constraints $u$, discount factor $\gamma$, batch size $B$, tolerance $\eta$, maximum iteration number $T$.
   \STATE Initialize policy network $\pi^{(0)}$, reward critic network $V_r^{(0)}$ and constraint critic network $V^{(0)}_c$.
   \FOR{$t=0,...,T-1$}
   \STATE Sample $B$ trajectories $\gB^{(t)}$ using policy network $\pi^{(t)}$.
   \STATE Obtain Monte-Carlo estimator $\widehat V_{c_y}^{\pi^{(t)}}(\mu)$ based on $\gB^{(t)}$.
   \STATE Use an optimization subroutine to solve ${\max_y\ \widehat V_{c_y}^{\pi^{(t)}}(\mu)-u_y}$, and set ${y^{(t)}\approx\argmax_y \widehat V_{c_y}^{\pi^{(t)}}(\mu)-u_y}$, $c^{(t)}=c_{y^{(t)}}$.
   \IF {$\widehat V_{c^{(t)}}^{\pi^{(t)}}(\mu)-u_{y^{(t)}}\leq \eta$}
   \STATE  Obtain advantage estimation of reward using reward critic network: 
   
   $\hat{A}^{(t)}(s_\tau,a_\tau)=\left(r_\tau+\gamma V_r^{(t)}(s_{\tau+1}) \right)-V_r^{(t)}(s_{\tau})$, $\forall (s_\tau, a_\tau, r_\tau, s_{\tau+1})\in \gB^{(t)}$.
   \ELSE 
   \STATE  Obtain advantage estimation of constraint at $y^{(t)}$ using constraint critic network: 
   
   $\hat{A}^{(t)}(s_\tau,a_\tau)=V^{(t)}_{c^{(t)}}(s_{\tau})-\left(c_{y^{(t)},\tau}+\gamma V^{(t)}_{c^{(t)}}(s_{\tau+1}) \right)$,  $\forall(s_\tau, a_\tau, r_\tau, s_{\tau+1})\in \gB^{(t)}$.
%   \IF {$t\geq s$}
%   \ENDIF
   \ENDIF
   \STATE Update policy network $\pi^{(t)}$, reward critic network $V_r^{(t)}$ and constraint critic network $V^{(t)}_c$ to $\pi^{(t+1)}$, $V_r^{(t+1)}$ and $V^{(t+1)}_c$ respectively with the above advantage estimation via the standard proximal policy optimization algorithm.
   \ENDFOR
   \STATE {\bfseries RETURN} $\hat\pi=\pi^{(T)}$.
\end{algorithmic}
\end{algorithm}
% The parameterized policy class is chosen as the softmax policy: $\pi_\theta(a|s)=e^{\theta_{s,a}}/\sum_{a^\prime\in \gA}e^{\theta_{s,a^\prime}}$, where $\theta\in\RB^{|\gS||\gA|}$ is initialized as $0$. This policy class satisfies the assumptions of the theoretical analysis of SI-CPO.

% We use learning rate $\alpha=0.1$, tolerance     $\eta=0.013$, maximum iteration
% number $T=10000$ for both SI-CPO and baseline. 

% In $t$th iteration in SI-CPO, we sample $100$ points uniformly in $Y$ and find the best one to solve the maximization problem approximately as \ref{Algorithm_random_search}. In the model-free setting, evaluating $\widehat V^\pi_{c_y}$ is much more computationally expensive than the model-based setting, hence brute force is impractical. Additionally, the random search method can yield a better policy than the one utilizing constraint optimization algorithm in practice. 


\section{Omitted Algorithms}\label{Appendix_Algorithm}

\begin{algorithm}[htb!]
   \caption{Random Search}
   \label{Algorithm_random_search}
\begin{algorithmic}
   \STATE {\bfseries Input:} Objective function $f\colon Y\to\RB$, where $Y$ is a compact subset of $\RB^m$.
   \STATE Sample $y_1,...,y_M\stackrel{i.i.d.}{\sim}\mathrm{Unif}(Y)$.
   \STATE {\bfseries RETURN} $\hat y=y_{i_0}$, $i_0=\argmax_{i\in\{1,...,M\}}f(y_i)$.
\end{algorithmic}
\end{algorithm}

\begin{algorithm}[htb!]
   \caption{Projected Subgradient Ascent}
   \label{Algorithm_projected_GD}
\begin{algorithmic}
   \STATE {\bfseries Input:} Objective function $f\colon Y\to\RB$, where $Y$ is a compact subset of $\RB^m$, the maximum number of iterations $T_{PGA}$.
   \STATE Initialize: set $y_0$ as an arbitrary element of $Y$, learning rate $\alpha=\frac{\mathrm{diam}(Y)}{L_y\sqrt{T}}$.
   \FOR {$t=0,...,T_{PGA}-1$}
   \STATE $t_{t+0.5}=y_t+\alpha g_t$, where $g_t$ is a subgradient of $f$ at $y_t$.
   \STATE $y_{t+1}=\argmin_{y\in Y} \|y-y_{t+0.5}\|$.
   \ENDFOR
   \STATE {\bfseries RETURN} $\hat y=\frac{1}{T_{PGA}}\sum_{t=1}^{T_{PGA}} y_t$.
\end{algorithmic}
\end{algorithm}

\begin{algorithm}[htb!]
   \caption{Sample-based NPG}
   \label{Algorithm_sample_based_NPG}
\begin{algorithmic}
   \STATE {\bfseries Input:} state space $\gS$, action space $\gA$, a criterion function $b$ (Can be the reward function $r$ or cost function $c_y$ for some fixed $y$), discount factor $\gamma$, policy $\pi_\theta$, number of paths $K_{sgd}$, fixed horizon $H$, upper bound of parameters' norm $W$, learning rate $\{\eta_k\}$, weight $\{\gamma_k\}$
   \FOR{$k=0$ {\bfseries to} $K_{sgd}-1$}
      \STATE Draw $(s,a)\sim \nu$, with $\nu(s,a)=d^{\pi_\theta}(s)\pi_\theta(a|s)$.
   \STATE Execute policy $\pi_\theta$ from $(s,a)$ for $H$ steps, then construct the estimators as
   $$
   \begin{aligned}
   \widehat Q^{\pi_\theta}(s,a)=\sum^{H-1}_{k=0} \gamma^k b(s_k,a_k),\ \text{where } (s_0,a_0)=(s,a).
   \end{aligned}
   $$
   \STATE Execute policy $\pi_\theta$ from $s$ for $H$ steps, then construct the estimators as
   $$
   \widehat V^{\pi_\theta}(s,a)=\sum^{H-1}_{k=0} \gamma^k b(s_k,a_k),\ \text{where } s_0=s.
   $$
   \STATE Set $\widehat A^{\pi_\theta}(s,a)=\widehat Q^{\pi_\theta}(s,a)-\widehat V^{\pi_\theta}(s)$.
   \STATE Perform an iteration of projected SGD: $w^{(k+1)}=\operatorname{Proj}_{B(0,W,\|\cdot\|_2)}(w^{(k)}-\eta_k G^{(k)})$ with
   $$
   \begin{aligned}
      G^{(k)}&=2({w^{(k)}}^\top\nabla_\theta\log\pi_\theta(a|s)-\widehat A^{\pi_\theta}(s,a))\nabla_\theta\log\pi_\theta(a|s),\\
   \end{aligned}
   $$
   and $B(0,W,\|\cdot\|_2):=\{w\in\RB^d|\|w\|_2\leq W\}$.
   \ENDFOR
   \STATE {\bfseries RETURN} $\sum_{k=1}^K \gamma_k w^{(k)}$ as a NPG update direction at $\pi_\theta$ w.r.t. criterion function $b$.
\end{algorithmic}
\end{algorithm}




\end{document}
