%!TEX root = 0_main.tex
\setcounter{equation}{0}
\renewcommand{\theequation}{A\arabic{equation}}
\section{Derivation of Equation~(\ref{eq:r}) in Section \ref{sec:overview}}
\label{sec:appendix}
In Section \ref{sec:overview}, the derivation of \eqref{eq:r} from \eqref{eq:action} is omitted.
In this Appendix, the omitted derivation is presented.
\begin{align}
  &\phantom{=} \argmin_{u_{i+1}} \min_{u_{i+2}, \ldots} \rob(\square \varphi, \mathcal{M}(\left[(0, u_0), (\Delta_T, u_1), \ldots \right]), 0) \tag{\ref{eq:action}} \\
  &= \argmin_{u_{i+1}} \min_{u_{i+2}, \ldots} \min_{t \in \mathbb R} \rob(\varphi, \mathcal{M}(\left[(0, u_0), (\Delta_T, u_1), \ldots \right]), t)\\
  &\sim \argmin_{u_{i+1}} \min_{u_{i+2}, \ldots} \min_{k = i+1, i+2, \ldots} \rob(\varphi, \mathcal{M}(\left[(0, u_0), \ldots, (k\Delta_T, u_k) \right]), k\Delta_T) \label{eq:disc}\\
  &\sim \argmin_{u_{i+1}} \min_{u_{i+2}, \ldots} \left[ - \log \left\{ 1 + \sum_{k=i+1}^\infty \{ e^{- \rob(\varphi, \mathcal{M}(\left[(0, u_0), \ldots, (k\Delta_T, u_k) \right]), k\Delta_T)} - 1\} \right\} \right] \label{eq:logsum}\\
  &= \argmax_{u_{i+1}} \max_{u_{i+2}, \ldots} \sum_{k=i+1}^\infty \{ e^{- \rob(\varphi, \mathcal{M}(\left[(0, u_0), \ldots, (k\Delta_T, u_k) \right]), k\Delta_T)} - 1\} \tag{\ref{eq:r}}
\end{align}
\eqref{eq:action} is the equation which derives the next input $u_{i+1}$.
\eqref{eq:disc} uses the fact $\varphi$ is past-dependent and \eqref{eq:logsum} uses an approximation of minimum by the log-sum-exp function~\cite{cook2011basic}.
Finally, the last equation \eqref{eq:r} is the same equation \eqref{eq:r} in Section \ref{sec:overview}.
This completes the derivation.

%\section{Definitions}
%
%\begin{definition}[System]
%  Informally speaking, we consider systems which takes continuous inputs from \emph{input space} $R^N$ and outputs continuous outputs into \emph{output space} $R^M$.
%  To make simulation of systems possible, we choose \emph{sampling time instants}
%  $\mathbf{t} = t_0, t_1, \ldots, t_n, \ldots$ and only consider inputs and outputs of those time instant in $\mathbf{t}$ ($\mathbf{t}$ is often omitted to be mentioned).
%
%  Formally, we use the following definition of systems.
%  An \emph{(finite) input trace} $\mathbf{x} = x_0, x_1, \ldots, x_n$, where $x_i\in\mathbb R^N$ for $i\in[0,n]$ is a sequence of elements of the \emph{input space} $\mathbb R^N$.
%
%  An \emph{(finite) output trace} $\mathbf{y} = y_0, y_1, \ldots, y_n$ is a sequence of elements of the \emph{system states} $\mathbb R^M$, i.e., for $j\in [0,n], y_j\in\mathbb R^M$.
%
%  If an (input/output) trace $\mathbf{z}_1$ is a prefix of $\mathbf{z}_2$, then trace $\mathbf{z}_1$ is said to be a prefix of $\mathbf{z}_2$.
%
%  \stodo{In the following definition of system, there is no time information involved. It is not clear what is a system state. However, the time and state information of a system is referred in the next definition, which makes our paper not self-contained. Yoriyuki, please help make the definition of the system complete.}
%  \ytodo{I modified the definition above.
%  Is is better?
%  The time information is committed to mention as stated above.
%  }
%  \stodo{My comments are about definition (reactive) system. We need to explain clearly what is time, and what is a state clearly. Such indirect explanation makes it hard to understand for audience. Moreover, the time and state information of a system is referred in the following definition, but not defined in the definition of system, which makes our paper not self-contained.}
%  \ytodo{I do not understand what you are saying.  If you want to change the text please suggest the alternative writing.}
%  A \emph{(reactive) system} $\mathbf{f}$ is a map from finite input traces $x_1, \ldots, x_{n-1}$ to (finite) output trace $y_1, \ldots, y_n$ such that for two traces $\mathbf{x}_1$, $\mathbf{x}_2 \in \mathbf{x}$, if $\mathbf{x}_1$ is a prefix of $\mathbf{x}_2$, $\mathbf{f}(\mathbf{x}_1)$ is a prefix of $\mathbf{f}(\mathbf{x}_2)$.
%\end{definition}
%%
%$\mathbf{f}$ induces a map from \emph{infinite} traces $\mathbf{x}$ to $\mathbf{y}$. Therefore, we use the same notation $\mathbf{f}$ for infinite traces.
%
%\stodo{In the following definition, (1)clearly specify what is n (2)$p(y_n)$ is not defined. (3) Is there a typo in the 6th formula?, should be $\diamond_I \phi$?}
%\ytodo{updated}
%\stodo{I only see the update for (3), not (1) and (2).}
%
%\begin{definition}[Metric Temporal Logic (MTL) formula]
%  A \emph{metric temporal logic (MTL)} formula $\phi$ is defined by a BNF as follows.
%  \begin{equation}
%    \phi ::= p \ \mid\  \phi \wedge \phi \ \mid\ \phi \vee \phi \ \mid \ \neg \phi \ \mid\  \square_I \phi \ \mid\ \diamond_I \phi \ \mid\ \phi \ \mathcal{U}_I \psi \ \mid\ \phi \ \mathcal{S}_I \psi \ \mid\ X\phi \ \mid\ P\phi
%  \end{equation}
%  where $p$ is an atomic formula and $I\in \mathbb R \times \mathbb R $ represents a time interval, where $\mathbb R$ is the set of real numbers.
%  We identify atomic formulas $p$ and subsets $p \subseteq \mathbb R^M$ of  system states.
%  For simplicity, we omit $I$ if it is the time interval $[0, \infty]$.
%  For $\mathcal{U}$ and $\mathcal{S}$ operations, we assume $I$ is non-negative.
%\end{definition}
%
%\begin{definition}
%  Let $\phi$ be a MTL-formula.
%  Let $\mathbf{t} = t_0, t_1, \ldots, t_n, \ldots$ be an infinite sequence of sampling time of the system states.
%  Let $\mathbf{y} = y_0, y_1, \ldots, y_n, \ldots$ be system states of time instants $\mathbf{t}$, respectively.
%  The relation $\mathbf{y}, n \models \phi$ (which represents $\phi$ holds at $n$ on the trace $\mathbf{y}$) is defined recursively on $\phi$ as follows.
%  \begin{flushleft}
%  \begin{align*}
%    \mathbf{y}, n \models p &\iff y_n \in p\\
%    \mathbf{y}, n \models \phi_1 \wedge \phi_2 &\iff \mathbf{y}, n \models \phi_1 \text{ and } \mathbf{y}, n \models \phi_2\\
%    \mathbf{y}, n \models \phi_1 \vee \phi_2 &\iff \mathbf{y}, n \models \phi_1 \text{ or } \mathbf{y}, n \models \phi_2\\
%    \mathbf{y}, n \models \neg \phi &\iff \neg (\mathbf{y}, n \models \phi)\\
%    \mathbf{y}, n \models \square_I \phi &\iff \forall n'\text{ such that } t_{n'} - t_n \in I, \mathbf{y}, n' \models \phi\\
%    \mathbf{y}, n \models \diamond_I \phi &\iff \exists n' \text{ such that } t_{n'} - t_n \in I, \mathbf{y}, n' \models \phi\\
%    \mathbf{y}, n \models \phi \ \mathcal{U}_I \psi &\iff
%    \begin{gathered}
%      \exists n', t_{n'} - t_n \in I \text{ such that } \mathbf{y}, n' \models \psi \text{ and }\\
%      \forall n'', n \leq n'' < n', \mathbf{y}, n'' \models \phi
%    \end{gathered}\\
%    \mathbf{y}, n \models \phi \ \mathcal{S}_I \psi &\iff
%    \begin{gathered}
%      \exists n', t_n - t_{n'} \in I \text{ such that } \mathbf{y}, n' \models \psi \text{ and }\\
%    \forall n'',  n' < n'' \leq n, \mathbf{y}, n'' \models \phi
%    \end{gathered}\\
%    \mathbf{y}, n \models X\phi &\iff \mathbf{y}, n+1 \models \phi\\
%    \mathbf{y}, n \models P\phi &\iff n \geq 1 \text{ and } \mathbf{y}, n-1 \models \phi
%  \end{align*}
%  \end{flushleft}
%\end{definition}
%
%\stodo{Please add explanations verbally for the following: (1) $rob(\mathbf{y}, n, \phi)$, (2)$dist(y, y_n)$, and give an example of this.}
%\subsection{Robustness of metric temporal logic formulas (MTLs)}
%
%Next, robustness $\rob(\mathbf{y}, n, \phi)$ of a MTL formula $\phi$ for a output trace $y$ at the instant $t_n$ is defined as the measure of how ``robust'' $\phi$ holds.
%For the atomic formula $p$, the robustness $\rob(\mathbf{y}, n, p)$ is defined as the infimum of the distance of the point $y$ which does not satisfies $p$ from $y_n$.
%The distance between two states $\dist(y, y_n)$ can be any metric, but in this paper we focus on Euclidian metric.
%For example, if $y_n = 0$ and $p = \{ y \mid y < 1 \}$, $\rob(\mathbf{y}, n, p) = \dist(y_n, y) = 1$.
%\begin{definition}[Robustness]
%  Let $\phi$ be a MTL-formula.
%  The robustness function $\rob(\mathbf{y}, n, \phi)$ over infinite traces $\mathbf{y} = y_0, y_1, \ldots, y_n, \ldots$ is defined as follows.
%  \begin{align}
%    \rob(\mathbf{y}, n, p) &= \inf \{ \dist(y, y_n) \mid y \not\in p \}\\
%    \rob(\mathbf{y}, n, \phi_1 \wedge \phi_2) &= \min(\rob(\mathbf{y}, n, \phi_1), \rob(\mathbf{y}, n, \phi_2))\\
%    \rob(\mathbf{y}, n, \neg \phi) &= - \rob(\mathbf{y}, n, \phi)\\
%    \rob(\mathbf{y}, n, \phi_1 \vee \phi_2) &= \max(\rob(\mathbf{y}, n, \phi_1), \rob(\mathbf{y}, n, \phi_2))\\
%    \rob(\mathbf{y}, n, \square_I \phi) &= \min \{ \rob(\mathbf{y}, n', \phi) \mid t_{n'} - t_n \in I \}\\
%    \rob(\mathbf{y}, n, \diamond_I \phi) &= \max \{ \rob(\mathbf{y}, n', \phi) \mid t_{n'} - t_n \in I \}\\
%    \rob(\mathbf{y}, n, \phi \ \mathcal{U}_I \psi) &= \max_{n' \text{ s.t. } t_{n'} - t_n \in I} \min(\rob(\mathbf{y}, n', \psi), \min_{n'' = n}^{n'-1} \rob(\mathbf{y}, n'', \phi))\\
%    \rob(\mathbf{y}, n, \phi \ \mathcal{S}_I \psi) &= \max_{n' \text{ s.t. } t_{n} - t_n' \in I} \min(\rob(\mathbf{y}, n', \psi), \min_{n'' = n'+1}^{n} \rob(\mathbf{y}, n'', \phi))\\
%    \rob(\mathbf{y}, n, X \phi) &= \rob(\mathbf{y}, n+1, \phi)\\
%    \rob(\mathbf{y}, n, P \phi) &=
%    \begin{cases}
%        \rob(\mathbf{y}, n-1, \phi) & \text{ if } n \geq 1\\
%        -\infty & n = 0
%    \end{cases}
%  \end{align}
%  where $\dist$ is the distance between two states.
%  For the empty set $\emptyset$, $\min \emptyset = \infty$ and $\max \emptyset = -\infty$.
%\end{definition}
%
%
%\begin{definition}[Horizon]
%We assign a \emph{horizon} $\hrz(\phi)$ for each MTL-formula as follows.
%  \begin{align}
%    \hrz(p) &= 0\\
%    \hrz(\phi \wedge \psi) = \hrz(\phi \vee \psi) &= \max(\hrz(\phi), \hrz(\psi))\\
%    \hrz(\neg \phi) &= \hrz(\phi)\\
%    \hrz(\square_I \phi) = \hrz(\diamond_I \phi) &= \hrz(\phi) + \sup I\\
%    \hrz(\phi \ \mathcal{U}_I \psi) &= \max(\hrz(\phi) + \sup I, \hrz(\psi) + \sup I)\\
%    \hrz(\phi \ \mathcal{S}_I \psi) &= \max(\hrz(\phi), \hrz(\psi))\\
%    \hrz(X\phi) &= \hrz(\phi) + 1\\
%    \hrz(P\phi) &= \min(0, \hrz(\phi) - 1)
%  \end{align}
%  If $\hrz(\phi) = 0$, we call $\phi$ \emph{past dependendt}.
%\end{definition}
%
%\stodo{Typo in Definition 7, formula 29.}
%\begin{definition}[Shift operator]
%  For a MTL-formula $\phi$, $\sht(\phi, t)$ is defined as follows.
%  \begin{align}
%    \sht(p, t) &= \square_{[-t, -t]} p\\
%    \sht(\neg \phi, t) &= \neg \sht(\phi, t)\\
%    \sht(\phi_1 \wedge \phi_2, t) &= \sht(\phi_1) \wedge \sht(\phi_2)\\
%    \sht(\phi_1 \vee \phi_2, t) &= \sht(\phi_1) \vee \sht(\phi_2)\\
%    \sht(\square_I \phi, t) &= \square_{I - t} \phi\\
%    \sht(\diamond_I \phi, t) &= \square_{I - t} \phi\\
%    \sht(\phi \ \mathcal{U}_I \psi, t) &= \square_{[-t, -t]} \phi \ \mathcal{U}_I \psi\\
%    \sht(\phi \ \mathcal{S}_I \psi, t) &= \phi \ \mathcal{S}_{I+t} \psi\\
%    \sht(X\phi, t) &= \square_{[-t, -t]} X\phi\\
%    \sht(P\phi, t) &= \square_{[-t, -t]} P\phi
%  \end{align}
%\end{definition}
