\documentclass{llncs}
\usepackage{times}
\usepackage[title]{appendix}
\usepackage{graphicx}
\usepackage{amsmath, amssymb}
\usepackage{multirow}
\usepackage{algpseudocode}
\usepackage{algorithmicx,algorithm}

\newcommand{\action}{$\mathcal A$\space}
\newcommand{\probkernal}{$\mathcal P_0$\space}
\newcommand{\state}{$\mathcal X$\space}
\newcommand{\status}{s\space}
\newcommand{\vspeed}{$\mathtt{v}$\space}
\newcommand{\ATmodel}{\mathbf{AT}}
\newcommand{\PTCmodel}{\mathbf{PTC}}
\newcommand{\STL}{\mathrm{STL}}
\newcommand{\SA}{\mathrm{SA}}
\newcommand{\CE}{\mathrm{CE}}
\newcommand{\AAAC}{\mathbf{A3C}}
\newcommand{\DQN}{\mathbf{DDQN}}

% \newcommand{\ytodo}[1]{\todo[inline,color=blue!30]{#1}}
% \newcommand{\stodo}[1]{\todo[inline,color=red!30]{#1}}
% \newcommand{\ttodo}[1]{\todo[color=red!30]{#1}}
% \newcommand{\ytodo}[1]{}
% \newcommand{\stodo}[1]{}
% \newcommand{\ttodo}[1]{}

\DeclareMathOperator{\rob}{\rho}
\DeclareMathOperator*{\minimize}{\mathsf{minimize}}
\DeclareMathOperator{\dist}{\mathsf{dist}}
\DeclareMathOperator{\hrz}{\mathsf{fr}}
\DeclareMathOperator{\reward}{\mathsf{reward}}
\DeclareMathOperator{\hst}{\mathsf{pr}}
\DeclareMathOperator{\sht}{\mathsf{shift}}
\DeclareMathOperator*{\argmin}{arg\,min}
\DeclareMathOperator*{\argmax}{arg\,max}
\DeclareMathOperator{\UNTIL}{\mathcal{U}}
\DeclareMathOperator{\SINCE}{\mathcal{S}}

\algnewcommand\algorithmicinput{\textbf{input:}}
\algnewcommand\INPUT{\item[\algorithmicinput]}
\algnewcommand\algorithmicparameters{\textbf{parameters:}}
\algnewcommand\PARAMETERS{\item[\algorithmicparameters]}
\algnewcommand\algorithmicoutput{\textbf{output:}}
\algnewcommand\OUTPUT{\item[\algorithmicoutput]}

\title{Falsification of Cyber-Physical Systems Using Deep Reinforcement Learning}

\author{
Takumi Akazaki\inst{1}\inst{2} \and Shuang Liu\inst{3} \and Yoriyuki Yamagata\inst{4} \and Yihai Duan\inst{3} \and Jianye Hao\inst{3}
}

\institute{The University of Tokyo \\
\and
Japan Society for the Promotion of Science \\
\and
School of Software, Tianjin University \\
\and
National Institute of Advanced Industrial Science and Technology (AIST)
}


\begin{document}
\maketitle

\begin{abstract}
With the rapid development of software and distributed computing,  \emph{Cyber-Physical Systems} (CPS) are widely adopted in many application areas, e.g., smart grid, autonomous automobile. It is difficult to detect defects in CPS models due to the complexities involved in the software and physical systems. To find defects in CPS models efficiently, robustness guided falsification of CPS is introduced. Existing methods use several optimization techniques to generate counterexamples, which falsify the given properties of a CPS. However those methods may require a large number of simulation runs to find the counterexample and is far from practical. In this work, we explore state-of-the-art \emph{Deep Reinforcement Learning (DRL)} techniques
%, i.e., Asynchronous Advanced Acctor Critic (A3C) and Double Deep-Q Network (DDQN),
to reduce the number of simulation runs required to find such counterexamples. We %introduce our method and
report our method and the preliminary evaluation results.
\end{abstract}


\begin{figure}[t]
\begin{center}
   \includegraphics[width=1.0\linewidth]{figures/nas_comp_v3}
\end{center}
   \vspace{-4mm}
   \caption{The comparison between NetAdaptV2 and related works. The number above a marker is the corresponding total search time measured on NVIDIA V100 GPUs.}
\label{fig:nas_comparison}
\end{figure}

\section{Introduction}
\label{sec:introduction}

Neural architecture search (NAS) applies machine learning to automatically discover deep neural networks (DNNs) with better performance (e.g., better accuracy-latency trade-offs) by sampling the search space, which is the union of all discoverable DNNs. The search time is one key metric for NAS algorithms, which accounts for three steps: 1) training a \emph{super-network}, whose weights are shared by all the DNNs in the search space and trained by minimizing the loss across them, 2) training and evaluating sampled DNNs (referred to as \emph{samples}), and 3) training the discovered DNN. Another important metric for NAS is whether it supports non-differentiable search metrics such as hardware metrics (e.g., latency and energy). Incorporating hardware metrics into NAS is the key to improving the performance of the discovered DNNs~\cite{eccv2018-netadapt, Tan2018MnasNetPN, cai2018proxylessnas, Chen2020MnasFPNLL, chamnet}.


There is usually a trade-off between the time spent for the three steps and the support of non-differentiable search metrics. For example, early reinforcement-learning-based NAS methods~\cite{zoph2017nasreinforcement, zoph2018nasnet, Tan2018MnasNetPN} suffer from the long time for training and evaluating samples. Using a super-network~\cite{yu2018slimmable, Yu_2019_ICCV, autoslim_arxiv, cai2020once, yu2020bignas, Bender2018UnderstandingAS, enas, tunas, Guo2020SPOS} solves this problem, but super-network training is typically time-consuming and becomes the new time bottleneck. The gradient-based methods~\cite{gordon2018morphnet, liu2018darts, wu2018fbnet, fbnetv2, cai2018proxylessnas, stamoulis2019singlepath, stamoulis2019singlepathautoml, Mei2020AtomNAS, Xu2020PC-DARTS} reduce the time for training a super-network and training and evaluating samples at the cost of sacrificing the support of non-differentiable search metrics. In summary, many existing works either have an unbalanced reduction in the time spent per step (i.e., optimizing some steps at the cost of a significant increase in the time for other steps), which still leads to a long \emph{total} search time, or are unable to support non-differentiable search metrics, which limits the performance of the discovered DNNs.

In this paper, we propose an efficient NAS algorithm, NetAdaptV2, to significantly reduce the \emph{total} search time by introducing three innovations to \emph{better balance} the reduction in the time spent per step while supporting non-differentiable search metrics:

\textbf{Channel-level bypass connections (mainly reduce the time for training and evaluating samples, Sec.~\ref{subsec:channel_level_bypass_connections})}: Early NAS works only search for DNNs with different numbers of filters (referred to as \emph{layer widths}). To improve the performance of the discovered DNN, more recent works search for DNNs with different numbers of layers (referred to as \emph{network depths}) in addition to different layer widths at the cost of training and evaluating more samples because network depths and layer widths are usually considered independently. In NetAdaptV2, we propose \emph{channel-level bypass connections} to merge network depth and layer width into a single search dimension, which requires only searching for layer width and hence reduces the number of samples.

\textbf{Ordered dropout (mainly reduces the time for training a super-network, Sec.~\ref{subsec:ordered_droput})}: We adopt the idea of super-network to reduce the time for training and evaluating samples. In previous works, \emph{each} DNN in the search space requires one forward-backward pass to train. As a result, training multiple DNNs in the search space requires multiple forward-backward passes, which results in a long training time. To address the problem, we propose \emph{ordered dropout} to jointly train multiple DNNs in a \emph{single} forward-backward pass, which decreases the required number of forward-backward passes for a given number of DNNs and hence the time for training a super-network.

\textbf{Multi-layer coordinate descent optimizer (mainly reduces the time for training and evaluating samples and supports non-differentiable search metrics, Sec.~\ref{subsec:optimizer}):} NetAdaptV1~\cite{eccv2018-netadapt} and MobileNetV3~\cite{Howard_2019_ICCV}, which utilizes NetAdaptV1, have demonstrated the effectiveness of the single-layer coordinate descent (SCD) optimizer~\cite{book2020sze} in discovering high-performance DNN architectures. The SCD optimizer supports both differentiable and non-differentiable search metrics and has only a few interpretable hyper-parameters that need to be tuned, such as the per-iteration resource reduction. However, there are two shortcomings of the SCD optimizer. First, it only considers one layer per optimization iteration. Failing to consider the joint effect of multiple layers may lead to a worse decision and hence sub-optimal performance. Second, the per-iteration resource reduction (e.g., latency reduction) is limited by the layer with the smallest resource consumption (e.g., latency). It may take a large number of iterations to search for a very deep network because the per-iteration resource reduction is relatively small compared with the network resource consumption. To address these shortcomings,  we propose the \emph{multi-layer coordinate descent (MCD) optimizer} that considers multiple layers per optimization iteration to improve performance while reducing search time and preserving the support of non-differentiable search metrics.

Fig.~\ref{fig:nas_comparison} (and Table~\ref{tab:nas_result}) compares NetAdaptV2 with related works. NetAdaptV2 can reduce the search time by up to $5.8\times$ and $2.4\times$ on ImageNet~\cite{imagenet_cvpr09} and NYU Depth V2~\cite{nyudepth} respectively and discover DNNs with better performance than state-of-the-art NAS works. Moreover, compared to NAS-discovered MobileNetV3~\cite{Howard_2019_ICCV}, the discovered DNN has $1.8\%$ higher accuracy with the same latency.


%!TEX root = 0_main.tex
\section{Preliminary}\label{sec:preliminary}
%We briefly introduce the preliminary concepts used in our work.

%\vspace{4mm}
%\subsection{Metric Temporal Logic}
%\subsubsection{Robustness guided falsification}
\subsubsection{{Robustness guided falsification}}
%
In this paper, we employ a variant of \emph{Signal Temporal Logic ($\STL$)} defined in~\cite{bartocci2018specification}. The syntax is defined in the equation (\ref{eq:MTL}),
\begin{equation}\label{eq:MTL}
  \varphi ::= v \sim c \mid p \mid \neg \varphi \mid
  \varphi_1 \vee \varphi_2
  \mid \varphi_1 \UNTIL_I \varphi_2
  \mid \varphi_1 \SINCE_I \varphi_2
\end{equation}
where $v$ is \emph{real} variable,
$c$ is a rational number,
$p$ is atomic formula,
$\sim \in \{<, \leq\}$
and $I$ is an interval over non-negative real numbers.
%\ttodo{Propositional variables are omitted.
%  I think we need them to treat True in the definition of box operator,
%  and gears in our experiment.}
If $I$ is $[0, \infty]$, $I$ is omitted.
We also use other common abbreviations,
e.g., $\square_I \varphi \equiv \mathsf{True} \UNTIL_I \varphi$ and
$\boxminus_I \varphi \equiv \mathsf{True} \SINCE_I \varphi$.
For a given formula $\varphi$, an output signal $\mathbf{x}$ and time $t$, we adopt the notation of work~\cite{bartocci2018specification} and denote the \emph{robustness degree} of output signal $\mathbf{x}$ satisfying $\varphi$ at time $t$ by $\rob(\varphi, \mathbf{x}, t)$.
%Note that,
%intuitively,
%the robustness degree $\rob(\varphi, x, t)$
%stands for
%how ``robust'' the signal $x$ satisfies the formula $\varphi$ at time $t$.

%% Robustness~\cite{TLF_CPS_2013} of a MTL formula $\phi$ for a output trace $y$ at the time instant $t_n$ is defined as
%% \todo{Does n means $t_n$ in function rob()?}
%% $\rob(\mathbf{y}, n, \phi)$ , and is a measure of how
%% ``robust'' $\phi$ holds.
%% For atomic formula $p$, the robustness $\rob(\mathbf{y}, n, p)$ is defined as the infimum of the distance of the point $y$ which does not satisfies $p$ from $y_n$.
%% \todo{$y_n$ means $y_t$? or $y_tn$? We either use t or n, be consistent}
%% %The distance between two states $\dist(y, y_n)$ can be any metric, but
%% \textcolor{red}{
%% In this paper we use Euclidian metric to define the distance between two states $\dist(y, y_n)$.
%% For example, if $y_n = 0$ and $p = \{ y \mid y < 1 \}$, $\rob(\mathbf{y}, n, p) = \dist(y_n, y) = 1$.}
%% %

We also adopt the notion of \emph{future-reach} $\hrz(\varphi)$ and
\emph{past-reach} $\hst(\varphi)$ following~\cite{DBLP:conf/rv/HoOW14}.
%% The \emph{horizon} $\hrz(\phi)$ of a MTL formula $\phi$ is the time in future which is required to determine the truth value of the formula $\phi$.
%Generally speaking, for a given formula $\varphi$,
Intuitively, $\hrz(\varphi)$ is the time in future which is required to determine the truth value of formula $\varphi$, and $\hst(\varphi)$ is the time in past.
For example,
$\hrz(p) = 0$, $\hrz(\square_{[0, 3]}p) = 3$ and $\hrz(\boxminus_{[0,3]}p) = 0$.
%%Similarly we define the \emph{history} $\hst(\varphi)$.
Similarly, for past-reach,
$\hst(p) = 0$, $\hst(\square_{[0, 3]}p) = 0$, $\hst(\boxminus_{[0,3]}p) = 3$.

%% \begin{lemma}[Robustness of a past dependent formula]\label{lem:rob-past}
%%   Let $\mathbf{y}$ be a finite trace of system states $y_0, \ldots, y_n$.
%%   Let $\overline{\mathbf{y}}_1$ and $\overline{\mathbf{y}}_2$ be two infinite extensions of $\mathbf{y}$.
%%   If $\phi$ is past dependent,
%%   \begin{equation}
%%     \rob(\overline{\mathbf{y}}_1, t, \phi) = \rob(\overline{\mathbf{y}}_2, t, \phi)
%%   \end{equation}
%% \end{lemma}

%% By Lemma \ref{lem:rob-past}, robustness of a past dependent formula at instant $n$ is completely determined by $y_0, \ldots, y_n$.
%% Therefore, we use the notation $\rob(\mathbf{y}, t, \phi)$ for robustness of a past-dependent formula $\phi$ on a finite trace $\mathbf{y}$.

%\subsection{Past-dependent life-long property falsification}
In this paper, we focus on a specific class of the temporal logic formula called \emph{life-long property}.
%to employ our approach.

\begin{definition}[life-long property]
  A \emph{life-long property} is an $\STL$ formula $\psi \equiv \square \varphi$ where $\hrz(\varphi),
  \hst(\varphi)$ are finite.
  If $\hrz(\varphi) = 0$, we call $\psi$ \emph{past-dependent life-long property}.

\end{definition}

%% %TODO life-long property, not past-dependent
%% \begin{definition}[Past-dependent life-long property]
%%   A \emph{life-long property} is an $\mathsf{MTL}$ formula $\psi \equiv \square \varphi$ where $\varphi$ only has a finite horizon.
%%   In particular, if $\varphi$ only contains bounded modal operators, $\psi$ is a life-long property.
%%   If $\varphi$ is past-dependent, then $\square \varphi$ is called \emph{past-dependent life-long property}.
%% \end{definition}


%\vspace{4mm}
\subsubsection{Reinforcement Learning}
%\subsection{Reinforcement learning}
%\subsubsection{Reinforcement Learning}
Reinforcement learning is one of machine learning techniques in which an agent learns the structure of the environment based on observations, and maximizes the rewards by acting according to the learnt knowledge.
% Reinforcement learning is first proposed and used in the domain of audio and image processing to improve the analysis performance. Reinforcement learning has shown its power and potential in training AlphaGo Zero~\cite{AlphaGo0}, which became the world's best Go player in 40 days, from scratch.
%In this work, we are using reinforcement learning techniques to reduce the accelerate the process of finding the counterexample, which falsifies the robustness property defined for a CPS.
%In particular, we adopt Asynchronous Advantage Actor-Critic (A3C) and Double Deep Q Network (DDQN) in our problem.
%
%Fig. \ref{fig:RL} shows the standard setting of reinforcement learning.
%%
%
%\begin{figure}
%  \centering
%  \scriptsize
%  \includegraphics[scale=0.67]{fig/RL.pdf}
%  \caption{Reinforcement learning setting}
%  \label{fig:RL}
%  \vspace{-7mm}
%\end{figure}
%
%%
The standard setting of a reinforcement learning problem consists of an agent and an environment. %, as shown in Fig.~\ref{fig:arch}.
The agent observes the current state and reward from the environment, and returns the next action to the environment.
%
%%
%%%
% Reinforcement learning is often formulated as a \emph{Markov decision process (MDP)}~\cite{Szepesvari2010}.
% A MDP is a triple $\mathcal M = (\mathcal X, \mathcal A, \mathcal P_0)$.
% \state is a set of states, \action is a set of actions and \probkernal is the transition probability kernel.
% A transition probability kernel \probkernal assigns a probability distribution (over $\mathcal X \times \mathbb R$, which is a distribution over the next states and the reward when the agent takes an action $a$ at the state $x$.), to each state-action pair $(x, a) \in \mathcal X \times \mathcal A$.
The goal of reinforcement learning is for each step $n$, given the sequence of previous states $x_0, \ldots, x_{n-1}$, rewards $r_1, \ldots, r_{n}$ and actions $a_0, \ldots, a_{n-1}$, generate an action $a_n$, which maximizes expected value of the sum of rewards:
%\begin{equation}
 $ r = \sum_{k = n}^\infty \gamma^k r_{k+1}$
%\end{equation}
, where $0 < \gamma \leq 1$ is a discount factor.
%
% For each state $x \in \mathcal X$, $V^*(x)$ is used to denote the highest achievable expected value of reward $r$, when $x_0 = x$.
%There are different kinds of reinforcement learning algorithms proposed in the literature. These approaches mainly falls into 2 different categories, i.e., value based and policy based, categorized by types of agents.
%
% \emph{$Q$-learning} \cite{} is the representative method for value-based reinforcement learning algorithm.
% For each action-state pair $(x, a)$, let \emph{optimal action-value function} $Q^*(x, a)$ be the highest achievable expected value of $r$ when $x_0 = x$ and $a_0 = a$.
% Once the value of $Q^*$ is known, the optimal strategy is to choose action $a$ which maximizes $Q^*(x, a)$ for the current state $x$ (following the greedy policy).
% One approach of reinforcement learning is to directly estimate $Q^*$ and use this estimated value to determine best actions.
% This approach is called \emph{$Q$-learning} \cite{}.
%
% \emph{actor-critic} method is the representative method for the other kind of approaches, policy-based algorithms.
% A \emph{stochastic stationary policy} (or just \emph{policy}) $\pi$ maps states in \state to probability distributions over actions in \action.
% The set of all policies is denoted by $\Pi_{\mathrm{stat}}$.
% Each policy $\pi$ gives rise to a \emph{Markov reward process (MRP)} $\mathcal M = (\mathcal X, \mathcal P_0)$.
% In a MRP, the state makes transitions as a Markovian process and generates a sequence of rewards $r_1, r_2, \ldots$.
% The action-value function $Q^\pi$ is defined by
% \begin{equation}
%   Q^\pi(x, a) = \mathbf E \left[ \sum_{t = 0}^\infty \gamma^t R_{t+1} \middle| x_0 = x, a_0 = a \right]
% \end{equation}
% where $\mathbf E$ signifies the expect value.
% An actor-critic method works as follows.
% First, it starts with a random policy $\pi_0$ and the ``actor'' follows $\pi_0$ some duration of time.
% Then, the the ``critic'' estimates $Q^{\pi_0}$ by the results of the run.
% In the next phase, the a greedy policy $\pi_1$ determined by estimated $Q^{\pi_0}$ is generated and the actor follows $\pi_1$.
% The actor-critic method repeats this process.
%
%%
%%%
Deep reinforcement learning is a reinforcement learning technique which uses a \emph{deep neural network} for learning.  % to represent a $Q$-function and/or a policy $\pi$.
In this work, we particularly adopted 2 state-of-the-art deep reinforcement learning algorithms, i.e., \emph{Asynchronous Advantage Actor-Critic} (A3C)~\cite{Mnih2016} and \emph{Double Deep Q Network} (DDQN)~\cite{pmlr-v48-gu16}.
% We briefly review these methods in the following.
% %
%
% \noindent \textbf{A3C: Asynchronous Advantage Actor-Critic}
% Asynchronous Advantage Actor-Critic (A3C)~\cite{Mnih2016} utilizes multiple processes to accelerate the training process. All processes run the same training algorithm and the information is collected by a central process. In this way, the algorithm and train models much faster.
%
% \noindent \textbf{Double Deep Q Network}
% DQN~\cite{mnih2013playing} combines CNN and Q-learning. A CNN network is used to analyze the Q-value. DQN can is proposed to solve large problems, which is hard to tackle with the normal table-based Q-learning algorithms.
%
%The output of the DQN is the best action to take in the current state and it's corresponding q-value. Moreover, to avoid the problem that may be caused by over estimation, we adopt double DQN  (DDQN) which uses Current Q-network to select actions and older Q-network to evaluate actions. Current Q-network use max to find the best action, overestimation may happen here. But another Q-network which doesn't use max evaluates q-value of the selected action. It may high or low. So over estimation is solved.

% Double DQN is an improvement of DQN, it has two networks to conduct action selection and Q-value evaluation separately. Double DQN learns faster and can avoid the over-estimation problem of DQN.
%The loss function is
%\begin{equation}
% L(w) =  \mathbb{E}\left[(r+\gamma\mathop {\max }\limits_{a'} Q(s',a',w) - Q(s,a,w))^2\right]
%\end{equation}
%
%DQN is short for Deep Q network\cite{mnih2013playing}. Q-learning is a Reinforcement Learning algorithm which for finite states and actions. Practical problems may have too many states therefore Q-learning has some limits. So q-values based on neural network comes true.
%
%\begin{algorithm}
%  \caption{Deep Q-learning with Experience Replay}
%  \begin{algorithmic}
%  \State Initialize replay memory D to capacity N % \State
%  \State Initialize action-value function Q with random weights
%    \For{episode = 1;M}
%      \State Initialise sequence $s_1 = \left\{ x_1 \right\}$ and preprocessed sequenced $\phi_1 = \phi(s_1) $
%    \EndFor

%  \end{algorithmic}
%\end{algorithm}
%

%%!TEX root = 0_main.tex
\section{Motivating Example}\label{sec:example}

(... Explanation of an ARCH2014 example ...)

\begin{definition}[System]
  An \emph{(finite) input trace} $\mathbf{x} = x_0, x_1, \ldots, x_n$ is a sequence of elements of the \emph{input space} $\mathbb R^N$ together with the sequence of sampling time instants $\mathbf{t} = t_0, t_1, \ldots, t_n, \ldots$.
  $\mathbf{t}$ is often omitted to be mentioned.

  An \emph{(finite) output trace} $\mathbf{y} = y_0, y_1, \ldots, y_n$ is a sequence of elements of the \emph{system states} $\mathbb R^M$.
  If an (input/output) trace $\mathbf{z}_1$ and its sampling time instants $\mathbf{t}_1$ are prefixes of $\mathbf{z}_2$ and $\mathbf{t}_2$ respectively, the trace $\mathbf{z}_1$ is a prefix of $\mathbf{z}_2$.

  A \emph{(reactive) system} $\mathbf{f}$ is a map from finite input traces $x_1, \ldots, x_{n-1}$ to $y_1, \ldots, y_n$ such that if $\mathbf{x}_1$ is a prefix of $\mathbf{x}_2$, $\mathbf{f}(\mathbf{x}_1)$ is a prefix of $\mathbf{f}(\mathbf{x}_2)$.
\end{definition}

Because of the definition, $\mathbf{f}$ induces a map from \emph{infinite} traces $\mathbf{x}$ to $\mathbf{y}$.
We use the same notation $\mathbf{f}$ for this map.

(... Intuitive explanation and references ...)
\begin{definition}[Metric Temporal Logic (MTL) formula]
  A \emph{metric temporal logic (MTL)} formula $\phi$ is defined by a BNF as follows.
  \begin{equation}
    \phi ::= p \ \mid\  \phi \wedge \phi \ \mid\ \phi \vee \phi \ \mid \ \neg \phi \ \mid\  \square_I \phi \ \mid\ \diamond_I \phi \ \mid\ \phi \ \mathcal{U}_I \psi \ \mid\ \phi \ \mathcal{S}_I \psi \ \mid\ X\phi \ \mid\ P\phi
  \end{equation}
  where $p$ is an atomic formula and $I$ is any interval on $\mathbb R$.
  We omit $I$ if $I = [0, \infty]$.
  For $\mathcal{U}$ and $\mathcal{S}$, we assume $I$ is non-negative.
\end{definition}

\begin{definition}
  Let $\phi$ be a MTL-formula.
  Let $\mathbf{t} = t_0, t_1, \ldots, t_n, \ldots$ be an infinite sequence of sampling time of the system states.
  Let $\mathbf{y} = y_0, y_1, \ldots, y_n, \ldots$ be system states of time instants $\mathbf{t}$ respectively.
  The relation $\mathbf{y}, n \models \phi$ (read that $\phi$ holds at $n$ on the trace $\mathbf{y}$) is defined recursively on $\phi$ as follows.
  \begin{align}
    \mathbf{y}, n \models p &\iff p(y_n)\\
    \mathbf{y}, n \models \phi_1 \wedge \phi_2 &\iff \mathbf{y}, n \models \phi_1 \text{ and } \mathbf{y}, n \models \phi_2\\
    \mathbf{y}, n \models \phi_1 \vee \phi_2 &\iff \mathbf{y}, n \models \phi_1 \text{ or } \mathbf{y}, n \models \phi_2\\
    \mathbf{y}, n \models \neg \phi &\iff \neg (\mathbf{y}, n \models \phi)\\
    \mathbf{y}, n \models \square_I \phi &\iff \forall n'\text{ such that } t_{n'} - t_n \in I, \mathbf{y}, n' \models \phi\\
    \mathbf{y}, n \models \square_I \phi &\iff \exists n' \text{ such that } t_{n'} - t_n \in I, \mathbf{y}, n' \models \phi\\
    \mathbf{y}, n \models \phi \ \mathcal{U}_I \psi &\iff
    \begin{gathered}
      \exists n', t_{n'} - t_n \in I \text{ such that } \mathbf{y}, n' \models \psi \text{ and }\\
       n \leq \forall n'' < n', \mathbf{y}, n'' \models \phi
    \end{gathered}\\
    \mathbf{y}, n \models \phi \ \mathcal{S}_I \psi &\iff
    \begin{gathered}
      \exists n', t_n - t_{n'} \in I \text{ such that } \mathbf{y}, n' \models \psi \text{ and }\\
      n' < \forall n'' \leq n, \mathbf{y}, n'' \models \phi
    \end{gathered}\\
    \mathbf{y}, n \models X\phi &\iff \mathbf{y}, n+1 \models \phi\\
    \mathbf{y}, n \models P\phi &\iff n \geq 1 \text{ and } \mathbf{y}, n-1 \models \phi
  \end{align}
\end{definition}

(... Intuitive explanation and references ...)
\begin{definition}[Robustness]
  Let $\phi$ be a MTL-formula.
  The robustness function $\rob(\mathbf{y}, n, \phi)$ over infinite traces $\mathbf{y} = y_0, y_1, \ldots, y_n, \ldots$ is defined as follows.
  \begin{align}
    \rob(\mathbf{y}, n, p) &= \min \{ \dist(y, y_n) \mid \neg p(y) \}\\
    \rob(\mathbf{y}, n, \phi_1 \wedge \phi_2) &= \min(\rob(\mathbf{y}, n, \phi_1), \rob(\mathbf{y}, n, \phi_2))\\
    \rob(\mathbf{y}, n, \neg \phi) &= - \rob(\mathbf{y}, n, \phi)\\
    \rob(\mathbf{y}, n, \phi_1 \vee \phi_2) &= \max(\rob(\mathbf{y}, n, \phi_1), \rob(\mathbf{y}, n, \phi_2))\\
    \rob(\mathbf{y}, n, \square_I \phi) &= \min \{ \rob(\mathbf{y}, n', \phi) \mid t_{n'} - t_n \in I \}\\
    \rob(\mathbf{y}, n, \diamond_I \phi) &= \max \{ \rob(\mathbf{y}, n', \phi) \mid t_{n'} - t_n \in I \}\\
    \rob(\mathbf{y}, n, \phi \ \mathcal{U}_I \psi) &= \max_{n' \text{ s.t. } t_{n'} - t_n \in I} \min(\rob(\mathbf{y}, n', \psi), \min_{n'' = n}^{n'-1} \rob(\mathbf{y}, n'', \phi))\\
    \rob(\mathbf{y}, n, \phi \ \mathcal{S}_I \psi) &= \max_{n' \text{ s.t. } t_{n} - t_n' \in I} \min(\rob(\mathbf{y}, n', \psi), \min_{n'' = n'+1}^{n} \rob(\mathbf{y}, n'', \phi))\\
    \rob(\mathbf{y}, n, X \phi) &= \rob(\mathbf{y}, n+1, \phi)\\
    \rob(\mathbf{y}, n, P \phi) &=
    \begin{cases}
        \rob(\mathbf{y}, n-1, \phi) & \text{ if } n \geq 1\\
        -\infty & n = 0
    \end{cases}
  \end{align}
  where $\dist$ is a distance between two states.
  For the empty set $\emptyset$, $\min \emptyset = \infty$ and $\max \emptyset = -\infty$.
\end{definition}

\begin{definition}[Horizon]
  For each MTL-formula, we assign the \emph{horizon} $\hrz(\phi)$.
  \begin{align}
    \hrz(p) &= 0\\
    \hrz(\phi \wedge \psi) = \hrz(\phi \vee \psi) &= \max(\hrz(\phi), \hrz(\psi))\\
    \hrz(\neg \phi) &= \hrz(\phi)\\
    \hrz(\square_I \phi) = \hrz(\diamond_I \phi) &= \hrz(\phi) + \sup I\\
    \hrz(\phi \ \mathcal{U}_I \psi) &= \max(\hrz(\phi) + \sup I, \hrz(\psi) + \sup I)\\
    \hrz(\phi \ \mathcal{S}_I \psi) &= \max(\hrz(\phi), \hrz(\psi))\\
    \hrz(X\phi) &= \hrz(\phi) + 1\\
    \hrz(P\phi) &= \min(0, \hrz(\phi) - 1)
  \end{align}
  If $\hrz(\phi) = 0$, we call $\phi$ \emph{past dependendt}.
\end{definition}

\begin{lemma}[Robustness of a past dependent formula]\label{lem:rob-past}
  Let $\mathbf{y}$ be a finite trace of system states $y_0, \ldots, y_n$.
  Let $\overline{\mathbf{y}}_1$ and $\overline{\mathbf{y}}_2$ be two infinite extensions of $\mathbf{y}$.
  If $\phi$ is past dependent,
  \begin{equation}
    \rob(\overline{\mathbf{y}}_1, t, \phi) = \rob(\overline{\mathbf{y}}_2, t, \phi)
  \end{equation}
\end{lemma}

By Lemma \ref{lem:rob-past}, robustness of a past dependent formula at instant $n$ is completely determined by $y_0, \ldots, y_n$.
Therefore, we use the notation $\rob(\mathbf{y}, t, \phi)$ for robustness of a past-dependent formula $\phi$ on a finite trace $\mathbf{y}$.

%!TEX root = 0_main.tex
\section{Our Approach}\label{sec:overview}

\begin{algorithm}[tp]
\scriptsize
  \caption{Falsification for $\psi = \square \varphi$ by reinforcement learning}
  \label{algo:RLfalsification}
  \begin{algorithmic}[1]
    \INPUT A past-dependent life-long property $\psi = \square \varphi$, a system $\mathcal{M}$,
    an agent $\mathcal{A}$
    \OUTPUT A counterexample input signal $\mathbf{u}$ if exists
    \PARAMETERS A step time $\Delta_T$, the end time $T_{\mathsf{end}}$, the maximum number of the episode $N$
 %   \Ensure something
    \For{$\mathsf{numEpisode} \gets$ $1$ to $N$}
    \State $i \gets 0$, $r \gets 0$, $x$ be the initial (output) state of $\mathcal{M}$
    \State $\mathbf{u}$ be the empty input signal sequence
    \While{$i \Delta_T < T_{\mathsf{end}}$}
    \State $u \gets \mathcal{A}.\mathsf{step}(x, r)$,  $\mathbf{u} \gets \mathsf{append}(\mathbf{u}, (i \Delta_T, u))$
    \Comment choose the next input by the agent
    \State $\mathbf{x} \gets \mathcal{M}(\mathbf{u})$, $x \gets \mathbf{x}((i+1)\Delta_T)$
    \Comment simulate, observe the new output state
    \State $r \gets \reward(\mathbf{x}, \psi)$
    \State $i \gets i+1$
    \Comment calculate the reward by following eq.~(\ref{def:reward})
    \EndWhile
    \If{$\mathbf{x} \not\models \psi$}
    %% \Then
    \Return $\mathbf{u}$ as a falsifying input
    \EndIf
    \State $\mathcal{A}.\mathsf{reset}(x, r)$
    \EndFor
   \end{algorithmic}
\end{algorithm}
%
% In this section, we describe our method
% of enforcing the CPS model to falsify the given STL specification
% by reinforcement learning.

\subsection{Overview of our algorithm}\label{subsec:algorithm}
Let us consider the falsification problem to find a counterexample of the life-long property $\psi \equiv \square \varphi$.
If the output signal is infinitely long to past and future directions, $\psi$ is logically equivalent to a past-dependent life-long property $\square \boxminus_{[\hrz(\varphi), \hrz(\varphi)]} \varphi$.
In general, the output signal is not infinitely long to some direction but using this conversion we convert all life-long properties to past-dependent life-long properties.
Our evaluation in Section \ref{sec:exp} suggests that this approximation does not adversely affect the performance.
%
Therefore, assume $\psi$ is a past-dependent life-long property, we generate an input signal $\mathbf{u}$ for system $\mathcal{M}$,
such that the corresponding output signal $\mathcal{M}(\mathbf{u})$ does not satisfy $\psi$.

In our algorithm,
we fix the simulation time to be $T_{\mathsf{end}}$
and
call one simulation until time $T_\mathsf{end}$
an \emph{episode} in conformance with the reinforcement learning terminology.
We fix the discretization of time to a positive real number $\Delta_T$.
%For an agent $\mathcal{A}$,
%in each episode,
%it generates an input signal $\mathbf{u}(t)$
%%adaptively
%based on the observed current system output and the reward.
%More precisely,
The agent $\mathcal{A}$ generates the piecewise-constant input signal
$\mathbf{u} = \big[(0, u_0), (\Delta_T, u_{1}), (2\Delta_T, u_{2}), \dots \big]$
by iterating the following steps:

%\begin{enumerate}
%\item
% At time $i \Delta_T$ ($i=0,1,\dots$),
%  the agent $\mathcal{A}$ choose the next input value $u_i$.
%  The generated input signal is extended to
%  $\mathbf{u} = \big[(0, u_0), \dots, (i\Delta_T, u_i) \big]$. \\
%\item
%  Our algorithm obtains the corresponding output signal $\mathbf{x} = \mathcal{M}(\mathbf{u})$
%  by stepping forward one simulation on the model $\mathcal{M}$
%  from time $i \Delta_T$ to $(i+1) \Delta_T$ with input $u_{i}$. \\
%\item
% Let $x_{i+1} = \mathbf{x}((i+1)\Delta_T)$ be the new (observed) state (i.e., output) of the system. \\
%\item
% We compute reward $r_{i+1}$ by $\reward(\varphi, \mathbf{x}, (i+1)\Delta_T)$ (defined in Section \ref{subsec:reward}). \\
%\item
% $\mathcal{A}$ updates its action based on the new state $x_{i+1}$ and the reward $r_{i+1}$.
%\end{enumerate}

(1) At time $i \Delta_T$ ($i=0,1,\dots$),
  the agent $\mathcal{A}$ chooses the next input value $u_i$.
  The generated input signal is extended to
  $\mathbf{u} = \big[(0, u_0), \dots, (i\Delta_T, u_i) \big]$. \\
%\item
\indent (2) Our algorithm obtains the corresponding output signal $\mathbf{x} = \mathcal{M}(\mathbf{u})$
  by stepping forward one simulation on the model $\mathcal{M}$
  from time $i \Delta_T$ to $(i+1) \Delta_T$ with input $u_{i}$. \\
%\item
\indent (3) Let $x_{i+1} = \mathbf{x}((i+1)\Delta_T)$ be the new (observed) state (i.e., output) of the system. \\
%\item
\indent (4) We compute reward $r_{i+1}$ by $\reward(\varphi, \mathbf{x}, (i+1)\Delta_T)$ (defined in Section \ref{subsec:reward}). \\
%\item
\indent (5) The agent $\mathcal{A}$ updates its action based on the new state $x_{i+1}$ and reward $r_{i+1}$.

At the end of each episode,
we obtain the output signal trajectory $\mathbf{x}$,
and check whether it satisfies the property $\psi = \square \varphi$ or not.
If it is falsified, return the current input signal $\mathbf{u}$ as a counterexample.
Otherwise, we discard the current generated signal input
and restart the episode from the beginning.

The complete algorithm of our approach is shown in Algorithm~\ref{algo:RLfalsification}.
The method call $\mathcal{A}.\mathsf{step}(x, r)$ represents
the agent $\mathcal{A}$ push the current state reward pair ($x$, $r$) into its memory
and returns the next action $u$ (the input signal in the next step).
The method call $\mathcal{A}.\mathsf{reset}(x, r)$ notifies the agent that the current episode is completed, and returns the current state and reward.
%\ttodo{This sentence seems misleadning as is discussed in our skype chat.}
Function $\reward(\mathbf{x}, \psi)$ calculates the reward based on Definition~\ref{def:reward}.



\subsection{Reward definition for life-long property falsification}\label{subsec:reward}
Our goal is to find the input signal $\mathbf{u}$ to the
system $\mathcal{M}$ which minimizes $\rob(\psi, \mathcal{M}(\mathbf{u}), 0)$ where $\psi = \square \varphi$ and $\rho$ is a robustness.
We determine $u_0, u_1, \ldots$ in a greedy way.
Assume that $u_0, \ldots, u_i$ are determined.
$u_{i+1}$ can be determined by
\begin{align}
\scriptstyle
\label{eq:action}
  u_{i+1} &= \argmin_{u_{i+1}} \min_{u_{i+2}, \ldots} \rob(\square \varphi, \mathcal{M}(\left[(0, u_0), (\Delta_T, u_1), \ldots \right]), 0) \\
  &\sim \argmax_{u_{i+1}} \max_{u_{i+2}, \ldots} \sum_{k=i+1}^\infty \{ e^{- \rob(\varphi, \mathcal{M}(\left[(0, u_0), \ldots, (k\Delta_T, u_k) \right]), k\Delta_T)} - 1\} \label{eq:r}
\end{align}
The detailed derivation steps can be found in Appendix~\ref{sec:appendix}.
%\eqref{eq:disc} uses the fact $\varphi$ is past-dependent and \eqref{eq:logsum} uses an approximation of minimum by the log-sum-exp function~\cite{cook2011basic}.

In our reinforcement learning base approach, we use discounting factor $\gamma=1$ and reward $r_i = e^{- \rob(\varphi, \mathcal{M}(\left[(0, u_0), \ldots, (i\Delta_T, u_i)\right]), i\Delta_T)} - 1$  to approximately compute action $u_{i+1}$, from $u_0, \ldots, u_i$, $\mathcal{M}(\left[(0, u_0), \ldots, (i\Delta_T, u_i) \right])$ and $r_1, \ldots, r_i$.
%\ttodo{No guarantee that u is approximately computed.
%  We cannot estimate the approximation error.
%  I prefer we claim
%  ``we use the discounting factor and reward to hopefully compute the approximation of the next action u...''
%}
%If the reward $r_i = e^{- \rob(\varphi, \mathcal{M}(\left[(0, u_0), \ldots, (i\Delta_T, u_i)\right]), i\Delta_T)} - 1$ and discounting factor $\gamma=1$ are used, we expect a reinforcement learning algorithm
%approximately computes $u_{i+1}$ as an action from $u_0, \ldots, u_i$, $\mathcal{M}(\left[(0, u_0), \ldots, (i\Delta_T, u_i) \right])$ and $r_1, \ldots, r_i$.

\begin{definition}[reward]\label{def:reward}
  Let $\psi \equiv \square \varphi$ be a past-dependent formula
  and $\mathbf{x} = \mathcal{M}(\mathbf{u})$ be a finite length signal until the time $t$.
  We define the reward $\reward(\psi, \mathbf{x})$ as
  \begin{equation}\label{eq:reward}
    \reward(\psi, \mathbf{x}) =
      \exp(- \rob(\varphi, \mathbf{x}, t)) - 1
  \end{equation}
\end{definition}

%!TEX root = 0_main.tex
\section{Preliminary Results}\label{sec:exp}
%To evaluate the efficiency and effectiveness of our work, we conduct experiments with well known CPS models in Matlab/Simulink.
%We compare our reinforcement learning based technique with existing methods, i.e., simulated annealing and cross entropy based methods, and analyze the results.
%\ytodo{Mention Breach when we finish the comparison with Breach}

%We discuss our implementation and report our preliminary evaluation result in this paper.

%\subsection{Implementation}
%\vspace{5mm}
\noindent\textbf{Implementation}
%\subsubsection{Implementation}
\begin{figure}[tp]
\centering
\scriptsize
\includegraphics[scale=0.44]{./fig/Architecture.pdf}
\caption{Architecture of our system}
\label{fig:arch}
\vspace{-7mm}
\end{figure}
%
The overall architecture of our system is shown in Fig.~\ref{fig:arch}.
Our implementation consists of three components, i.e., input generation, output handling and simulation.
The input generation component adopts reinforcement learning techniques and is implemented based on the ChainerRL library~\cite{ChainerRL}.
We use default hyper-parameters in the library or sample programs without change.
The output handling component conducts reward calculation using dp-TaliRo~\cite{S-TaliRo}.
The simulation is conducted with Matlab/Simulink models, which are encapsulated by the openAI gym library~\cite{1606.01540}.
%The output of the system is normalized (linearly mapped) into the interval $[-1, 1]\subseteq \mathbb{R}$, since deep reinforcement learning has the best performance for normalized inputs.
%The robustness and reward are calculated using normalized outputs.
%All input to the system are first mapped into the interval $[-1, 1]$ and then linearly transformed to the actual range, which the system accepts.
%%%
%%
%
%To falsify a CPS model, we run the model step by step using a fixed sampling rate, which is a period of the time obeyed by the model.
%The reward is calculated based on the trace of states, from the starting state to the current state of the run, based on the formula defined in~\ref{eq:reward}.
%For each step, the state of the model and the reward are used as input to the reinforcement learning agent.
%The reinforcement learning agent then returns the next action to take.
%Using the suggested action as an input, we restart the simulation cycle.
%
%%
%%%
%To calculate the robustness value of the monitoring formula, we use a function in S-TaliRo for calculating robustness.
%S-TaliRo provides two kinds of functions to calculate robustness,  one is a MATLAB function \verb|dp-taliro| and its variants; the other is a runtime monitoring function, which is a Simulink block.
%Our experiment uses the \verb|dp-taliro| function to calculate robustness, \textcolor{red}{because...}.
%However, \verb|dp-taliro| does not support past-dependent formulas.
%Therefore, to calculate robustness, we revert the order of the trace, i.e., from the current state to the beginning state.
%We also revert the time flow of the monitoring formula.
%For example, $\square_{[-3, 0]}p$ is converted to $\square_{[0, 3]}p$ and $\square_{[-5, -3]}p \rightarrow \square_{[-3, 0]}q$ is converted to $\square_{[3, 5]}p \rightarrow \square_{[0, 3]}q$.
%However, this conversion is not always possible.
%The until operator $\mathcal{U}_I$, the next operator $X$ and nested temporal operators seem resist from being converted.
%Therefore, our implementation can only handle limited properties.
%\stodo{The sentence ``The until operator $\mathcal{U}_I$, the next operator $X$ and nested temporal operators seem resist from being converted.
%Therefore, our implementation can only handle limited properties.'' uses the word ``seem'', which is not good. We need to make assured claims, i.e., give clear claims, whether those operators resist from being converted or not. }

%Then, we input the current state and the calculated reward to a reinforcement learning agent.
%The reinforcement learning agent then returns the next \emph{action} to take.
%Using the suggested action as an input, we restart and execute the simulation until the next step.
%We assume that the input (\textcolor{red}{to the simulation model?}) is constant during each step of the simulation.
%
%%%
%Finally, if the simulation reaches the end, we calculate robustness of the property which we want to falsify using the full trajectory.
%We use \verb|dp-taliro| for this purpose.
%%%
%
%but we do not revert the order of the trace nor the time flow of the formula as the case of the monitoring formula, because
%
%If the robustness of the property is negative, we successfully falsify the property and terminate the process.
%Otherwise, we reset the simulation and signifies the reinforcement learning agent at the end of the \emph{episode}.

%Our implementation has a large overhead in computation time.
%This is mostly due to the fact that, we utilized step-based reinforcement learning algorithms, and after getting the feedback from the RL agent, we stop and restart the simulation for each sampling cycle.
%Another reason is that reinforcement learning agents are implemented by Python while robustness calculation and simulation are implemented by MATLAB/Simulink.
%This creates another overhead to interpolate Python and MATLAB runtime.

%\ytodo{The part below is better to be moved to the preliminary.}
%\noindent\textbf{Double DQN}
%We adopt DDQN to accelerate the process of choosing inputs which may falsify the robustness properties. We describe the implementation of DDQN in our approach.  For the Autonomous system, the inputs to the model are the values of throttle and break, which are normalized to the range of ...; and the outputs of the model are Engine speed $\omega$ (RPM) and vehicle speed $\mathbf(v)$ (mph).In order to use DDQN to find the counterexample, which falsifies the robustness property of the autonomous model, we need to find a way to link the robustness value with the rewards calculated by DDQN.
%In Deep Q Network~\cite{mnih2013playing},the outputs are q-values of each action available at the current state. However, the actions taken (throttle and break value pairs) by the autonomous model is contentious, which results in an infinite input space. To solve this problem, we use Normalized Advantage Functions (NAF)~\cite{pmlr-v48-gu16}, which leads into Advantage. The relationship between Q(q-value),A(advantage) and V(value) is
%\begin{equation}
% Q(s,a) = A(s,a) + V(s)
%\end{equation}
%Then we can limit A less than 0, and select the A which equals 0.

%   The output of the DQN is the best action to take in the current state and it's corresponding q-value. Moreover, to avoid the problem that may be caused by over estimation, we adopt double DQN which uses Current Q-network to select actions and older Q-network to evaluate actions. Current Q-network use max to find the best action, overestimation may happen here. But another Q-network which doesn't use max evaluates q-value of the selected action. It may high or low. So over estimation is solved.


%\subsection{Evaluation Settings}
\vspace{4mm}
\noindent \textbf{Evaluation Settings}
%\subsubsection{Evaluation Settings}
We use a widely adopted CPS model, automatic transmission control system ($\ATmodel$) ~\cite{bardh2014benchmarks}, to evaluate our method.
%
%The system model is from a public demonstration of modeling an automatic transmission control with the Stateflow~\cite{ATinSF}
%%~\footnote{https://mathworks.com/products/stateflow.html}
%package in MATLAB.
%
$\ATmodel$ has throttle and brake as input ports, and the output ports are the vehicle velocity $v$, the engine rotation speed $\omega$ and the current gear state $g$.
%Although the size of the model is relatively small comparing the actual systems in industry, but its dynamics contains both discrete and continuous values.
%Therefore, it is suitable as a benchmark of falsification on CPSs.
%
We conduct our evaluation with the formulas in Table~\ref{tab:formulas}.
Formulas $\varphi_1$--$\varphi_6$ are rewriting of  $\varphi^{AT}_1$--$\varphi^{AT}_6$ in benchmark~\cite{bardh2014benchmarks} into life-long properties in our approach.
%We do not use the original benchmark because we focus on life-long properties.
In addition, we propose three new formulas $\varphi_{7}$--$\varphi_{9}$.
%\textcolor{red}{For all formulas, we tune parameters in the formulas such that they are difficult to falsify, therefore can differentiate performance of each method.}
%
For each formula $\varphi_1$--$\varphi_9$, we compare the performance of our approaches (A3C, DDQN), with the baseline algorithms, i.e., simulated annealing ($\SA$) and cross entropy ($\CE$).
%
For each property, we run the falsification procedure 20 times.
For each falsification procedure, we execute simulation episodes up to 200 times and measure the number of simulation episodes required to falsify the property.
If the property cannot be falsified within 200 episodes, the procedure fails.
%We record whether each falsification procedure is successful or not.
%
% Further, for fair comparison, we change sampling rate and choose the best performing rate for each combination of a method and formula.
% Here, the best means the smallest median of the number of simulations runs required to falsify the formula.
% If the tie occurs, we further compares success rates of the falsification procedure.
% Finally, if still we cannot decide we use the media of the execution time of the falsification processes.
%
We observe that $\Delta_{T}$ may strongly affect the performance of each algorithm.
%For example, simulated annealing tends to perform badly if we use high sampling rate.
%On the other hand, reinforcement learning based methods are not affected by high sampling rate.
%Based on the above observation,
Therefore, we
%choose the $\Delta_{T}$ which gives the best performance (prioritized by numEpisode and success rate) of each algorithm.
vary $\Delta_{T}$ (among \{1, 5, 10\} for $\varphi_1$--$\varphi_6$ and \{5, 10\} for $\varphi_7$--$\varphi_9$~\footnote{Experiment with $\Delta_{T}=1$ for $\varphi_7$--$\varphi_9$ shows bad performance and did not terminate in 5 days.}) and report the setting (of $\Delta_{T}$) which leads to the best performance (the least episode number and highest success rate) for each algorithm.
%We change $\Delta_{T}$ from 1, 5, 10 for $\varphi_1$--$\varphi_6$ and 5, 10 for $\varphi_7$--$\varphi_9$.
%$\Delta_{T}$ 1 for $\varphi_5$--$\varphi_7$ is omitted because of the performance reason.



\begin{table}[tp]
  \centering
  \begin{minipage}[t]{.48\textwidth}
    \centering
    \scriptsize
    \begin{tabular}{c||c}
      id & Formula\\
      \hline
      \hline
      $\varphi_1$ & $\square \omega \leq \overline{\omega}$\\
      $\varphi_2$ & $\square (v \leq \overline{v} \wedge \omega \leq \overline{\omega})$\\
      $\varphi_3$
      & $\square ((g_2 \wedge \diamond_{[0, 0.1]} g_1) \rightarrow \square_{[0.1, 1.0]} \neg g_2)$\\
      $\varphi_4$
      & $\square ((\neg g_1 \wedge \diamond_{[0, 0.1]} g_1) \rightarrow \square_{[0.1, 1.0]} g_1)$\\
      $\varphi_5$
      & $\square \bigwedge_{i=1}^4 ((\neg g_i \wedge \diamond_{[0, 0.1] g_i}) \rightarrow \square_{[0.1, 1.0]} g_i)$\\
    \end{tabular}
  \end{minipage}
  \begin{minipage}[t]{.48\textwidth}
    \centering
    \scriptsize
    \begin{tabular}{c||c}
      id & Formula\\
      \hline
      \hline
      $\varphi_6$
      & $\square (\square_{[0, t_1]} \omega \leq \overline\omega \rightarrow \square_{[t_1, t_2]} v \leq \overline{v})$\\
      $\varphi_7$
      & $\square v \leq \overline{v}$\\
      $\varphi_8$
      & $\square \diamond_{[0,25]} \neg (\underline{v} \leq v \leq  \overline{v})$\\
      $\varphi_9$
      & $\square \neg \square_{[0,20]} (\neg g_4 \wedge \omega \geq \overline{\omega})$\\
    \end{tabular}
  \end{minipage}
  \caption{The list of the evaluated properties on $\ATmodel$.}
  \label{tab:formulas}
  \vspace{-5mm}
\end{table}
\begin{table}[tp]
  \centering
    \centering
    \scriptsize
    \begin{tabular}[t]{c||c c c c|c c c c|c c c c|}
      id & \multicolumn{4}{|c|}{$\Delta_T$} & \multicolumn{4}{|c|}{Success rate} & \multicolumn{4}{|c|}{$\mathsf{numEpisode}$}\\
      \hline
      & $\AAAC$ & $\DQN$ & $\SA$ & $\CE$ & $\AAAC$ & $\DQN$ & $\SA$ & $\CE$ & $\AAAC$ & $\DQN$ & $\SA$ & $\CE$ \\
      \hline
      \hline
      $\varphi_1$ & 5 & 1 & 10 & 5 & $\textbf{100\%}^*$ & $\textbf{100\%}^*$ & 65.0\% & 10.0\% & $\textbf{12.0}^{**}$ & 28.0 & 118.5 & 200.0\\
      $\varphi_2$ & 5 & 1 & 10 & 5 & 95.0\% & $\textbf{100\%}^*$ & 65.0\% & 10.0\% & $\textbf{20.0}^{**}$ & 27.0 & 118.5 & 200.0\\
      $\varphi_3$ & 1 & 1 & 1 & 1 & \textbf{90.0\%} & 5.0\% & 20.0\% & 85.0\% & 46.0 & 200.0 & 200.0 & \textbf{26.5}\\
      $\varphi_4$ & 1 & 1 & 1 & 1 & \textbf{90.0\%} & 5.0\% & 20.0\% & 85.0\% & 63.5 & 200.0 & 200.0 & $\textbf{26.5}^*$\\
      $\varphi_5$ & 1 & 1 & 1 & 1 & \textbf{90.0\%} & 5.0\% & 20.0\% & 85.0\% & 73.0 & 200.0 & 200.0 & \textbf{26.5}\\
      $\varphi_6$ & 10 & 5 & 10 & 10 & $\textbf{100\%}^*$ & $\textbf{100\%}^*$ & 70.0\% & 50.0\% & $\textbf{2.5}^{**}$ & 4.0 & 160.5 & 119.0\\
      $\varphi_7$ & 5 & 5 & 10 & 10 & 80.0\% & $\textbf{100\%}^{**}$ & 0.0\% & 0.0\% & $\textbf{45.0}^{**}$ & 57.5 & 200.0 & 200.0\\
      $\varphi_8$ & 10 & 10 & 10 & 10 & 75.0\% & \textbf{100\%} & 90.0\% & 65.0\% & 52.0 & 37.5 & 83.0 & \textbf{35.5}\\
      $\varphi_9$ & 10 & 10 & 10 & 10 & 95.0\% & $\textbf{100\%}^{**}$ & 15.0\% & 5.0\% & 49.5 & $\textbf{12.0}^{**}$ & 200.0 & 200.0 \\
      \hline
  \end{tabular}
  \caption{The experimental result on $\ATmodel$.}
  \label{tab:ARCH2014}
   \vspace{-5mm}
\end{table}



%\subsection{Evaluation Results}\label{sec:result}
\vspace{4mm}
\noindent \textbf{Evaluation Results}
%\subsubsection{Evaluation Results}
The preliminary results are presented in Table.~\ref{tab:ARCH2014}.
%The algorithms indicated by bold face are our approach and others are baselines.
%%
%
%$\SA$ is simulated annealing and $\CE$ is cross entropy method.
%
%%
The $\Delta_{T}$ columns indicate the best performing $\Delta_{T}$ for each algorithm.
The ``Success rate'' columns indicate the success rate of falsification process.
The ``numEpisode'' columns show the median (among the 20 procedures) of the number of simulation episodes required to falsify the formula.
If the falsification procedure fails, we consider the number of simulation episodes to be the maximum allowed episodes (200).
We use median since the distribution of the number of simulation episodes tends to be skewed.
%Another reason is that our sample size is relatively small so that we need to avoid the effects of outliers.

The best results (success rate and numEpisode) of each formula are highlighted in bold.
If the difference between the best entry of our methods and the best entry of the baseline methods is statistically significant by Fisher's exact test and the Mann Whitney U-test~\cite{corder2014nonparametric}, we mark the best entry with $*$ ($p < 0.05$) or $**$ ($p < 0.001$), respectively.
%\todo{This is again a little misleading.  We do not compare, for example, A3C and DDQN.  Only between the best one among our method and the best one among baselines.}
%To test the success rate, we use Fisher's exact test of independence~\cite{corder2014nonparametric}.
%To test iteration numbers, we use the Mann-Whitney U-test~\cite{corder2014nonparametric}.

As shown in Table~\ref{tab:ARCH2014}, RL based methods almost always outperforms baseline methods on success rate, which means RL based methods are more likely to find the falsified inputs with a limited number of episodes.
This is because RL based methods learn knowledge from the environment and generate input signals adaptively during the simulations.
%On the other hand, the result on iteration numbers is more mixed.
Among the statistically significant results of numEpisode, our methods are best for five cases ($\varphi_1, \varphi_2,\varphi_6,\varphi_7,\varphi_9$), while the baseline methods are best for one case ($\varphi_4$).
%Although, for $\varphi_3$--$\varphi_5$ and $\varphi_8$ the cross entropy method is best in episode numbers, only in one case the result is statistically significant.
For the case of $\varphi_4$, it is likely because that
all variables in this formula take discrete values,
thus, reinforcement learning is less effective.
%the reward in reinforcement learning tends to be constant.
%\todo{Still A3C performs not so bad.  Also what is the reason of good performance of CE?}
Further, DDQN tends to return extreme values as actions,
which are not solutions to falsify $\varphi_3$--$\varphi_5$.
This explains poor performance of DDQN for the case of $\varphi_3$--$\varphi_5$.

%% \begin{table}
%% \centering
%% \scriptsize
%% \begin{tabular}{|c|c|c|c|c|c|}
%% \hline
%% Property id \& Formula & Algorithm & sampling rate & falsified? & iteration number\\
%% \hline
%% \hline
%% \multirow{4}{*}{$\varphi_1 \colon \square ((g_2 \wedge \diamond_{[0, 0.1]} g_1) \rightarrow \square_{[0.1, 1.0]} \neg g_2)$}
%%               & $\AAAC$ & 1 & \textbf{90.0\%} & 46.0 \\
%% 							& $\DQN$ & 1 & 5.0\% & 200.0 \\
%%               & $\SA$  & 1 & 20.0\% & 200.0 \\
%% 							& $\CE$  & 1 & 85.0\% & \textbf{26.5} \\

%% \hline
%% \multirow{4}{*}{$\varphi_2 \colon \square ((\neg g_1 \wedge \diamond_{[0, 0.1]} g_1) \rightarrow \square_{[0.1, 1.0]} g_1)$}
%%               & $\AAAC$ & 1 & \textbf{90.0\%} & 63.5 \\
%%               & $\DQN$ & 1 & 5.0\% & 200.0 \\
%% 							& $\SA$ & 1 & 20.0\% & 200.0 \\
%% 							& $\CE$  & 1 & 85.0\% & $\textbf{26.5}^*$ \\
%% \hline
%% \multirow{4}{*}{$\varphi_3 \colon \square \bigwedge_{i=1}^4 ((\neg g_i \wedge \diamond_{[0, 0.1] g_i}) \rightarrow \square_{[0.1, 1.0]} g_i)$}
%%               & $\AAAC$ & 1 & 75.0\% & 73.0 \\
%%               & $\DQN$ & 1 & 10.0\% & 200.0 \\
%% 							& $\SA$ & 1 & 20.0\% & 200.0 \\
%% 							& $\CE$ & 1 & \textbf{85.0\%} & \textbf{26.5} \\
%% \hline
%% \multirow{4}{*}{$\varphi_4 \colon \square (\square_{[0, t_1]} \omega \leq \overline\omega \rightarrow \square_{[t_1, t_2]} v \leq \overline{v})$}
%%               & $\AAAC$ & 10 & $\textbf{100\%}^*$ & $\textbf{2.5}^{**}$ \\
%%               & $\DQN$ & 5 & $\textbf{100\%}^*$ & 4.0 \\
%%               & $\SA$ & 10 & 70.0\% & 160.5  \\
%% 							& $\CE$ & 10 & 50.0\% & 119.0 \\
%% \hline
%% \multirow{4}{*}{$\varphi_5 \colon \square v \leq \overline{v}$}
%%               & $\AAAC$ & 5 & 80.0\% & $\textbf{45.0}^{**}$ \\
%%               & $\DQN$ & 5 & $\textbf{100}\%^{**}$ & 57.5 \\
%%               & $\SA$ & 10 & 0.0\% & 200.0 \\
%%               & $\CE$ & 10 & 0.0\% & 200.0 \\

%% \hline
%% \multirow{4}{*}{$\varphi_{6} \colon \square \diamond_{[0,25]} \neg (\underline{v} \leq v \leq  \overline{v})$}
%%               & $\AAAC$ & 10 & 75.0\% & 52.0 \\
%%               & $\DQN$ & 10 & \textbf{100}\% & 37.5 \\
%% 							& $\SA$ & 10 & 90.0\% & 83.0 \\
%% 							& $\CE$ & 10 & 65.0\% & \textbf{35.5} \\
%% \hline
%% \multirow{4}{*}{$\varphi_{7} \colon \square \neg \square_{[0,20]} (\mathrm{gearLow} \wedge \mathrm{highRPM})$}
%% 							& $\AAAC$ & 10 & 95.0\% & 49.5 \\
%%               & $\DQN$ & 10 & $\textbf{100\%}^{**}$ & $\textbf{12.0}^{**}$ \\
%% 							& $\SA$ & 10 & 15.0\% & 200.0 \\
%% 							& $\CE$ & 10 & 5.0\% & 200.0 \\
%% \hline
%% \end{tabular}
%% \caption{The experimental result on $\ATmodel$.}
%% \label{tab:ARCH2014}
%% \end{table}


%
% \begin{table}
%   \centering
%   \begin{minipage}[t]{.4\textwidth}
%     \centering
%     \scriptsize
%     \begin{tabular}[t]{c||c|c|c|c}
%       id & Algorithm & $\Delta_T$ & falsified? & $\mathsf{numEpisode}$\\
%       \hline
%       \hline
%       \multirow{4}{*}{$\varphi_1$}
%       & $\AAAC$ & 1 & \textbf{90.0\%} & 46.0 \\
%       & $\DQN$ & 1 & 5.0\% & 200.0 \\
%       & $\SA$  & 1 & 20.0\% & 200.0 \\
%       & $\CE$  & 1 & 85.0\% & \textbf{26.5} \\
%       \hline
%       \multirow{4}{*}{$\varphi_2$}
%       & $\AAAC$ & 1 & \textbf{90.0\%} & 63.5 \\
%       & $\DQN$ & 1 & 5.0\% & 200.0 \\
%       & $\SA$ & 1 & 20.0\% & 200.0 \\
%       & $\CE$  & 1 & 85.0\% & $\textbf{26.5}^*$ \\
%       \hline
%       \multirow{4}{*}{$\varphi_3$}
%       & $\AAAC$ & 1 & 75.0\% & 73.0 \\
%       & $\DQN$ & 1 & 10.0\% & 200.0 \\
%       & $\SA$ & 1 & 20.0\% & 200.0 \\
%       & $\CE$ & 1 & \textbf{85.0\%} & \textbf{26.5} \\
%       \hline
%       \multirow{4}{*}{$\varphi_4$}
%       & $\AAAC$ & 10 & $\textbf{100\%}^*$ & $\textbf{2.5}^{**}$ \\
%       & $\DQN$ & 5 & $\textbf{100\%}^*$ & 4.0 \\
%       & $\SA$ & 10 & 70.0\% & 160.5  \\
%       & $\CE$ & 10 & 50.0\% & 119.0 \\
%     \end{tabular}
%   \end{minipage}
%   \begin{minipage}[t]{.4\textwidth}
%     \centering
%     \scriptsize
%     \begin{tabular}[t]{c||c|c|c|c}
%       id & Algorithm & $\Delta_T$ & falsified? & $\mathsf{numEpisode}$\\
%       \hline
%       \hline
%       \multirow{4}{*}{$\varphi_5$}
%       & $\AAAC$ & 5 & 80.0\% & $\textbf{45.0}^{**}$ \\
%       & $\DQN$ & 5 & $\textbf{100}\%^{**}$ & 57.5 \\
%       & $\SA$ & 10 & 0.0\% & 200.0 \\
%       & $\CE$ & 10 & 0.0\% & 200.0 \\
%       \hline
%       \multirow{4}{*}{$\varphi_{6}$}
%       & $\AAAC$ & 10 & 75.0\% & 52.0 \\
%       & $\DQN$ & 10 & \textbf{100}\% & 37.5 \\
%       & $\SA$ & 10 & 90.0\% & 83.0 \\
%       & $\CE$ & 10 & 65.0\% & \textbf{35.5} \\
%       \hline
%       \multirow{4}{*}{$\varphi_{7}$}
%       & $\AAAC$ & 10 & 95.0\% & 49.5 \\
%       & $\DQN$ & 10 & $\textbf{100\%}^{**}$ & $\textbf{12.0}^{**}$ \\
%       & $\SA$ & 10 & 15.0\% & 200.0 \\
%       & $\CE$ & 10 & 5.0\% & 200.0 \\
%     \end{tabular}
%   \end{minipage}
%   \caption{The experimental result on $\ATmodel$.}
%   \label{tab:ARCH2014}
% \end{table}

%\section{Related Work}\label{sec:related_work}

\subsection{Testing methods for hybrid systems}

\cite{kapinski2003systematic,zhao2003generating,dang2009coverage}

Coverage based testing~\cite{esposito2004adaptive,bhatia2004incremental,branicky2006sampling}

Robust simulation trajectory~\cite{Donze:2007:SSU:1760804.1760822,girard2006verification,julius2007robust,Lerda:2008:VSC:1423033.1423059}

\subsection{Model checking of hybrid automata}

Model checking hybrid automata~\cite{Alur:1995:AAH:202379.202381,Henzinger:1995:WDH:225058.225162,henzinger1997hytech}

Monte-Carlo model checking~\cite{grosu2005monte}

Statistical model checking~\cite{Younes:2006:SPM:1182767.1182770,Clarke:2009:SMC:1533832.1533850,clarke2011statistical,Zuliani:2010:BSM:1755952.1755987}

\subsection{Robustness guided falsification of CPSs}

There have been works that are proposed to conduct verification on Cyber-Physical Systems (CPS).
Plaku et al.~\cite{Plaku:2009:FLS:1532891.1532932}...
Houssam~\cite{TLF_CPS_2013} proposed to find counterexamples in CPS through global optimization of robustness metrics using Monte-Carlo technique that conduct a random walk on over the space of inputs. They also build a Matlab toolbox, S-TaliRo~\cite{S-TaliRo}, which enables searches for trajectories of minimal robustness in Simulink/Stateflow diagrams.

\subsection{Controller synthesis}
%do we really need this catagory of related work? We may not adopt examples all from controller systems I suppose?
?

\section{Conclusion and Future Work}\label{sec:conclusion}
In this paper, we report an approach which adopts reinforcement learning algorithms to solve the problem of robustness-guided falsification of CPS systems. We implement our approach in a prototype tool and conduct preliminary evaluations with a widely adopted CPS system. The evaluation results show that our method can reduce the number of episodes to find the falsifying input. As a future work, we plan to extend the current work to explore more reinforcement learning algorithms and evaluate our methods on more CPS benchmarks. 
\bibliographystyle{abbrv}
\bibliography{ref}

\begin{appendix}
\newpage
\appendix
\section{Pricing equations}
\subsection{Credit default swap}
\label{CDS_pricing}
A credit default swap (CDS) is a contract designed to exchange credit risk of a Reference Name (RN) between a Protection Buyer (PB) and a Protection Seller (PS). PB makes periodic coupon payments to PS conditional on no default of RN, up to the nearest payment date, in the exchange for receiving from PS the loss given RN's default.

Consider a CDS contract written on the first bank (RN), denote its price $C_1(t, x)$.\footnote{For the CDS contracts written on the second bank, the similar expression could be provided by analogy.} We assume that the coupon is paid continuously and equals to $c$. Then, the value of a standard CDS contract can be given (\cite{BieleckiRutkowski}) by the solution of  (\ref{kolm_1})--(\ref{kolm_2})  with $\chi(t, x) = c$ and terminal condition
\begin{equation*}
	\psi(x) = 
	\begin{cases}
		1 - \min(R_1, \tilde{R}_1(1)), \quad (x_1, x_2) \in D_2, \\
		1 - \min(R_1, \tilde{R}_1(\omega_2)), \quad (x_1, x_2) \in D_{12}, \\		
	\end{cases}
\end{equation*}
where $\omega_2 = \omega_2(x)$ is defined in (\ref{term_cond}) and 
\begin{equation*}
	\tilde{R}_1(\omega_2) = \min \left[1, \frac{A_1(T) +  \omega_2 L_{2 1}(T)}{L_1(T) + \omega_2 L_{12}(T)}\right].
\end{equation*}
Thus, the pricing problem for CDS contract on the first bank is
\begin{equation}
\begin{aligned}
		& \frac{\partial}{\partial t} C_1(t, x) + \mathcal{L} C_1(t, x) = c, \\
		& C_1(t, 0, x_2) = 1 - R_1, \quad C_1(t, \infty, x_2) = -c(T-t), \\
		& C_1(t, x_1, 0) = \Xi(t, x_1) = 
		\begin{cases}
			c_{1,0}(t, x_1), & x_1 \ge \tilde{\mu}_1, \\
			1-R_1, & x_1 < \tilde{\mu}_i,
		\end{cases} \quad C_1(t, x_1, \infty) = c_{1,\infty}(t, x_1),\\
		& C_1(T, x) = \psi(x) = 
	\begin{cases}
		1 - \min(R_1, \tilde{R}_1(1)), \quad (x_1, x_2) \in D_2, \\
		1 - \min(R_1, \tilde{R}_1(\omega_2)), \quad (x_1, x_2) \in D_{12}, \\		
	\end{cases}
\end{aligned}
\end{equation}
where $c_{1,0}(t, x_1)$ is the solution of the following boundary value problem:
\begin{equation}
\begin{aligned}
		& \frac{\partial}{\partial t} c_{1, 0}(t, x_1) + \mathcal{L}_1 c_{1, 0}(t, x_1) = c, \\
		& c_{1, 0}(t, \tilde{\mu}_1^{<}) = 1 - R_1, \quad c_{1, 0}(t, \infty) = -c(T-t), \\
		& c_{1, 0}(T, x_1) = (1 - R_1) \mathbbm{1}_{\{\tilde{\mu}_1^{<} \le x_1 \le \tilde{\mu}_1^{=}\}}, 
\end{aligned}
\end{equation}
and $c_{1,\infty}(t, x_1)$ is the solution of the following boundary value problem
\begin{equation}
\begin{aligned}
		& \frac{\partial}{\partial t} c_{1, \infty}(t, x_1) + \mathcal{L}_1 c_{1, \infty}(t, x_1) = c, \\
		& c_{1, \infty}(t, 0) = 1 - R_1, \quad c_{1, \infty}(t, \infty) = -c(T-t), \\
		& c_{1, \infty}(T, x_1) = (1 - R_1) \mathbbm{1}_{\{x_1 \le \mu_1^{=}\}}.
\end{aligned}
\end{equation}

\subsection{First-to-default swap}
An FTD contract refers to a basket of reference names (RN). Similar to a regular CDS, the Protection Buyer (PB) pays a regular coupon payment $c$ to the Protection Seller (PS) up to the first default of any of the RN in the basket or maturity time $T$. In return, PS compensates PB the loss caused by the first default.

Consider the FTD contract referenced on $2$ banks, and denote its price $F(t, x)$. We assume that the coupon is paid continuously and equals to $c$. Then, the value of FTD contract can be given (\cite{LiptonItkin2015}) by the solution of  (\ref{kolm_1})--(\ref{kolm_2})  with $\chi(t, x) = c$ and terminal condition
\begin{equation*}
	\psi(x) = \beta_0  \mathbbm{1}_{\{x \in D_{12}\}} + \beta_1 \mathbbm{1}_{\{x \in D_{1}\}} + \beta_2 \mathbbm{1}_{\{x \in D_{2}\}},
\end{equation*}
where
\begin{equation*}
	\begin{aligned}
		\beta_0 = 1 - \min[\min(R_1, \tilde{R}_1(\omega_2), \min(R_2, \tilde{R}_2(\omega_1)], \\
		\beta_1 = 1 - \min(R_2, \tilde{R}_2(1)), \quad \beta_2 = 1 - \min(R_1, \tilde{R}_1(1)),
	\end{aligned}
\end{equation*}
and
\begin{equation*}
	\tilde{R}_1(\omega_2) = \min \left[1, \frac{A_1(T) +  \omega_2 L_{2 1}(T)}{L_1(T) + \omega_2 L_{12}(T)}\right], \quad \tilde{R}_2(\omega_1) = \min \left[1, \frac{A_2(T) +  \omega_1 L_{1 2}(T)}{L_2(T) + \omega_1 L_{21}(T)}\right].
\end{equation*}
with $\omega_1 = \omega_1(x)$ and $\omega_2 = \omega_2(x)$ defined in (\ref{term_cond}).

Thus, the pricing problem for a FTD contract is
\begin{equation}
\begin{aligned}
		& \frac{\partial}{\partial t} F(t, x) + \mathcal{L} F(t, x) = c, \\
		& F(t, x_1, 0) = 1 - R_2,  \quad F(t, 0, x_2) = 1 - R_1, \\
		& F(t, x_1, \infty) = f_{2,\infty}(t, x_1), \quad F(t, \infty, x_2) = f_{1,\infty}(t, x_2), \\
		& F(T, x) = \beta_0  \mathbbm{1}_{\{x \in D_{12}\}} + \beta_1 \mathbbm{1}_{\{x \in D_{1}\}} + \beta_2 \mathbbm{1}_{\{x \in D_{2}\}},
\end{aligned}
\end{equation}
where $f_{1,\infty}(t, x_1)$ and $f_{2,\infty}(t, x_2)$ are the solutions of the following boundary value problems
\begin{equation}
\begin{aligned}
		& \frac{\partial}{\partial t} f_{i, \infty}(t, x_i) + \mathcal{L}_i f_{i, \infty}(t, x_i) = c, \\
		& f_{i, \infty}(t, 0) = 1 - R_i, \quad f_{i, \infty}(t, \infty) = -c(T-t), \\
		& f_{1, \infty}(T, x_i) = (1 - R_i) \mathbbm{1}_{\{x_i \le \mu_i^{=}\}}.
\end{aligned}
\end{equation}

\subsection{Credit and Debt Value Adjustments for CDS}

Credit Value Adjustment and Debt Value Adjustment can be considered either unilateral or bilateral. For unilateral counterparty risk, we need to consider only two banks (RN, and PS for CVA and PB for DVA), and a two-dimensional problem can be formulated, while bilateral counterparty risk requires a three-dimensional problem, where Reference Name, Protection Buyer, and Protection Seller are all taken into account. We follow \cite{LiptonSav} for the pricing problem formulation but include jumps and mutual liabilities, which affects the boundary conditions.

\paragraph{Unilateral CVA and DVA}
The Credit Value Adjustment represents the additional price associated with the possibility of a counterparty's default. Then, CVA can be defined as
\begin{equation}
	V^{CVA} = (1- R_{PS}) \mathbb{E}[\mathbbm{1}_{\{\tau^{PS} < \min(T, \tau^{RN}) \}} (V_{\tau^{PS}}^{CDS})^{+} \, | \mathcal{F}_t],
\end{equation}
where $R_{PS}$ is the recovery rate of PS, $\tau^{PS}$ and $\tau^{RN}$ are the default times of PS and RN, and $V_t^{CDS}$ is the price of a CDS without counterparty credit risk.

We associate $x_1$ with the Protection Seller and $x_2$ with the Reference Name, then CVA can be given by the solution of  (\ref{kolm_1})--(\ref{kolm_2})  with $\chi(t, x) = 0$ and $\psi(x) = 0$. Thus,
\begin{equation}
\begin{aligned}
		& \frac{\partial}{\partial t} V^{CVA}+ \mathcal{L} V^{CVA} = 0, \\
		& V^{CVA}(t, 0, x_2) = (1 - R_{PS}) V^{CDS}(t, x_2)^{+}, \quad V^{CVA}(t, x_1, 0) = 0, \\
		& V^{CVA}(T, x_1, x_2) = 0.
\end{aligned}
\end{equation}

Similar, Debt Value Adjustment represents the additional price associated with the default and defined as
\begin{equation}
	V^{DVA} = (1- R_{PB}) \mathbb{E}[\mathbbm{1}_{\{\tau^{PB} < \min(T, \tau^{RN}) \}} (V_{\tau^{PB}}^{CDS})^{-} \, | \mathcal{F}_t],
\end{equation}
where $R_{PB}$ and $\tau^{PB}$ are the recovery rate and default time of the protection buyer.

Here, we associate $x_1$ with the Protection Buyer and $x_2$ with the Reference Name, then, similar to CVA,  DVA can be given by the solution of  (\ref{kolm_1})--(\ref{kolm_2}),
\begin{equation}
\begin{aligned}
		& \frac{\partial}{\partial t} V^{DVA}+ \mathcal{L} V^{DVA} = 0, \\
		& V^{DVA}(t, 0, x_2) = (1 - R_{PB}) V^{CDS}(t, x_2)^{-}, \quad V^{DVA}(t, x_1, 0) = 0, \\
		& V^{DVA}(T, x_1, x_2) = 0.
\end{aligned}
\end{equation}

\paragraph{Bilateral CVA and DVA}

When we defined unilateral CVA and DVA, we assumed that either protection  buyer, or protection seller are risk-free. Here we assume that they are both risky. Then, 
The Credit Value Adjustment represents the additional price associated with the possibility of counterparty's default and defined as
\begin{equation}
	V^{CVA} = (1 - R_{PS}) \mathbb{E}[\mathbbm{1}_{\{\tau^{PS} < \min(\tau^{PB}, \tau^{RN}, T)\}} (V^{CDS}_{\tau^{PS}})^{+} \, | \mathcal{F}_t],
\end{equation} 

Similar, for DVA
\begin{equation}
	V^{DVA} = (1 - R_{PB}) \mathbb{E}[\mathbbm{1}_{\{\tau^{PB} < \min(\tau^{PS}, \tau^{RN}, T)\}} (V^{CDS}_{\tau^{PB}})^{-} \, | \mathcal{F}_t],
\end{equation} 


We associate $x_1$ with protection seller, $x_2$ with protection buyer, and $x_3$ with reference name. Here, we have a three-dimensional process. Applying three-dimensional version of (\ref{kolm_1})--(\ref{kolm_2}) with $\psi(x) = 0, \chi(t, x) = 0$, we get
\begin{equation}
	\label{CVA_pde}
\begin{aligned}
		& \frac{\partial}{\partial t} V^{CVA} + \mathcal{L}_3 V^{CVA} = 0, \\
		& V^{CVA}(t, 0, x_2, x_3) = (1 - R_{PS}) V^{CDS}(t, x_3)^{+}, \\
		& V^{CVA}(t, x_1, 0, x_3 ) = 0, \quad V^{CVA}(t, x_1, x_2, 0)  = 0, \\
		& V^{CVA}(T, x_1, x_2, x_3) = 0,
\end{aligned}
\end{equation}
and
\begin{equation}
\label{DVA_pde}
\begin{aligned}
		& \frac{\partial}{\partial t} V^{DVA} + \mathcal{L}_3 V^{DVA} = 0, \\
		& V^{DVA}(t, 0, x_2, x_3) = (1 - R_{PB}) V^{CDS}(t, x_3)^{-}, \\
		& V^{DVA}(t, x_1, 0, x_3 ) = 0, \quad V^{DVA}(t, x_1, x_2, 0)  = 0, \\
		& V^{DVA}(T, x_1, x_2, x_3) = 0,
\end{aligned}
\end{equation}
where $\mathcal{L}_3 f$ is the three-dimensional infinitesimal generator.



\end{appendix}
\end{document}
