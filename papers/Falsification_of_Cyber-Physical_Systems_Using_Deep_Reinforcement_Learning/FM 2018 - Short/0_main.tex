\documentclass{llncs}
\usepackage{times}
\usepackage[title]{appendix}
\usepackage{graphicx}
\usepackage{amsmath, amssymb}
\usepackage{multirow}
\usepackage{algpseudocode}
\usepackage{algorithmicx,algorithm}

\newcommand{\action}{$\mathcal A$\space}
\newcommand{\probkernal}{$\mathcal P_0$\space}
\newcommand{\state}{$\mathcal X$\space}
\newcommand{\status}{s\space}
\newcommand{\vspeed}{$\mathtt{v}$\space}
\newcommand{\ATmodel}{\mathbf{AT}}
\newcommand{\PTCmodel}{\mathbf{PTC}}
\newcommand{\STL}{\mathrm{STL}}
\newcommand{\SA}{\mathrm{SA}}
\newcommand{\CE}{\mathrm{CE}}
\newcommand{\AAAC}{\mathbf{A3C}}
\newcommand{\DQN}{\mathbf{DDQN}}

% \newcommand{\ytodo}[1]{\todo[inline,color=blue!30]{#1}}
% \newcommand{\stodo}[1]{\todo[inline,color=red!30]{#1}}
% \newcommand{\ttodo}[1]{\todo[color=red!30]{#1}}
% \newcommand{\ytodo}[1]{}
% \newcommand{\stodo}[1]{}
% \newcommand{\ttodo}[1]{}

\DeclareMathOperator{\rob}{\rho}
\DeclareMathOperator*{\minimize}{\mathsf{minimize}}
\DeclareMathOperator{\dist}{\mathsf{dist}}
\DeclareMathOperator{\hrz}{\mathsf{fr}}
\DeclareMathOperator{\reward}{\mathsf{reward}}
\DeclareMathOperator{\hst}{\mathsf{pr}}
\DeclareMathOperator{\sht}{\mathsf{shift}}
\DeclareMathOperator*{\argmin}{arg\,min}
\DeclareMathOperator*{\argmax}{arg\,max}
\DeclareMathOperator{\UNTIL}{\mathcal{U}}
\DeclareMathOperator{\SINCE}{\mathcal{S}}

\algnewcommand\algorithmicinput{\textbf{input:}}
\algnewcommand\INPUT{\item[\algorithmicinput]}
\algnewcommand\algorithmicparameters{\textbf{parameters:}}
\algnewcommand\PARAMETERS{\item[\algorithmicparameters]}
\algnewcommand\algorithmicoutput{\textbf{output:}}
\algnewcommand\OUTPUT{\item[\algorithmicoutput]}

\title{Falsification of Cyber-Physical Systems Using Deep Reinforcement Learning}

\author{
Takumi Akazaki\inst{1}\inst{2} \and Shuang Liu\inst{3} \and Yoriyuki Yamagata\inst{4} \and Yihai Duan\inst{3} \and Jianye Hao\inst{3}
}

\institute{The University of Tokyo \\
\and
Japan Society for the Promotion of Science \\
\and
School of Software, Tianjin University \\
\and
National Institute of Advanced Industrial Science and Technology (AIST)
}


\begin{document}
\maketitle

\begin{abstract}
With the rapid development of software and distributed computing,  \emph{Cyber-Physical Systems} (CPS) are widely adopted in many application areas, e.g., smart grid, autonomous automobile. It is difficult to detect defects in CPS models due to the complexities involved in the software and physical systems. To find defects in CPS models efficiently, robustness guided falsification of CPS is introduced. Existing methods use several optimization techniques to generate counterexamples, which falsify the given properties of a CPS. However those methods may require a large number of simulation runs to find the counterexample and is far from practical. In this work, we explore state-of-the-art \emph{Deep Reinforcement Learning (DRL)} techniques
%, i.e., Asynchronous Advanced Acctor Critic (A3C) and Double Deep-Q Network (DDQN),
to reduce the number of simulation runs required to find such counterexamples. We %introduce our method and
report our method and the preliminary evaluation results.
\end{abstract}


\section{Introduction}

Scientific literature is most commonly available in the form of PDFs, which pose challenges for accessibility \citep{NielsenPDFStillUnfit, Bigham2016AnUT}. When researchers, students, and other individuals who are blind or low vision (BLV) interact with scientific PDFs through screen readers, the availability of document structure tags, labeled reading order, labeled headers, and image alt-text are necessary to facilitate these interactions. However, these features must be painstakingly added by authors using proprietary software tools, and as a result, are often missing from papers. Low vision or dyslexic readers who interact with PDFs through screen magnification or text-to-speech may also find the complexity of certain academic paper PDF formats challenging, e.g., non-linear layout can interrupt the flow of text in a magnifying tool. Inaccessible paper PDFs can lead to high cognitive overload, frustration, and abandonment of reading for BLV readers. 

Unfortunately, we find that the majority of scientific PDFs lack basic accessibility features. We estimate based on a sample of \numpdfs PDFs from multiple fields of study that only around \percaccessible of paper PDFs released in the last decade satisfy all of the aforementioned accessibility requirements. 
Accessibility challenges for academic PDFs are largely due to three factors: (1) the complexity of the PDF file format, which make it less amenable to certain accessibility features, (2) the dearth of tools, especially non-proprietary tools, for creating accessible PDFs, and (3) the dependency on volunteerism from the community with minimal support or enforcement \citep{Bigham2016AnUT}. The intent of the PDF file format is to support faithful visual representation of a document for printing, a goal that is inherently divergent from that of document representation for the purposes of accessibility. Though some professional organizations like the Association for Computing Machinery (ACM) have encouraged PDF accessibility through standards and writing guidelines,\footnote{\href{https://www.acm.org/publications/authors/submissions}{https://www.acm.org/publications/authors/submissions}} uptake among academic publishers and disciplines more broadly has been limited. 

While policy changes help, the fact remains that most academic PDFs produced today, and historically, are inaccessible, yet remain as the dominant way to read those papers. A long-range solution will necessitate buy-in from multiple stakeholders---publishers, authors, readers, technologists, granting agencies, and the like. But in the interim, there are technological solutions that can be offered as a sort of ``band-aid'' to the problem. We use this paper to offer an in-depth qualitative and quantitative description of the problem as it stands, and to introduce one such technological solution: the \scially system that automatically extracts semantic information from paper PDFs and re-renders this content in the form of an accessible HTML document. Though the process is imperfect and can introduce errors, we demonstrate the ability of the rendered HTMLs to reduce cognitive load and facilitate in-paper navigation and interactions for BLV users. 

The goals and contributions of this paper are three-fold:

\begin{enumerate}
    \item We characterize the state of academic-paper PDF accessibility by estimating the degree of adherence to accessibility criteria for papers published in the last decade (2010--2019), and describe correlations between year, field of study, PDF typesetting software, and PDF accessibility.
    \item We propose an automated approach for extracting the content of academic PDFs and displaying this content in a more accessible HTML document format. We build a prototype that re-renders 12 million PDFs in HTML, and describe the design decisions, features, and quality of the renders (assessed as faithfulness to the source PDF). We perform expert grading of the rendered HTML and report an error analysis. A demo of our system is available at \href{https://scia11y.org/}{scia11y.org}, which makes available 1.5M HTML renders of open access PDFs.
    \item We conduct an exploratory user study with \numusers BLV scholars to better understand the challenges they experience when reading academic papers and how our proposed tool might augment their current workflow. During the study, we ask users to interact with the prototype and offer feedback for its improvement. We perform open coding of interviews to identify existing reading challenges, coping mechanisms, as well as positive and negative responses to prototype features. We summarize the findings of this user study into a set of design recommendations.
\end{enumerate}

Our analysis reveals that PDF accessibility adherence is low across all fields of study. Of the five accessibility criteria we assess, only \percaccessible of the PDFs we assess demonstrate full compliance. Though compliance for several criteria seems to be increasing over time, author awareness and contribution to accessibility remains low, as Alt-text has the lowest compliance of the five criteria at between 5--10\% (Alt-text is the only criterion of the five that \textit{requires} author intervention in all cases using current tools). We also find that typesetting software is strongly associated with accessibility compliance, with LaTeX and publishing software like Arbortext APP producing low compliance PDFs, while Microsoft Word is generally associated with higher compliance.


\begin{figure}[t!]
    \centering
    \includegraphics[width=\textwidth]{figures/pipeline.png}
    \caption{A schematic for creating the \scially HTML render from a paper PDF. Starting with the raw two-column PDF on the left, S2ORC \citep{lo-wang-2020-s2orc} is used to extract title, authors, abstract, section headers, body text, and references. S2ORC also identifies links between inline citations and references to figures and table objects. DeepFigures \citep{Siegel2018ExtractingSF} is used to extract figures and tables, along with their captions. The output of these two models are merged with metadata from the Semantic Scholar API. Heuristics are used to construct a table of contents, to insert figures and tables in the appropriate places in the text, and to repair broken URLs. We add HTML headers as illustrated (header tags for sections, paragraph tags for body text, and figure tags for figures and tables); highlighted components (table of contents and links in references) are not in the PDF and novel navigational features that we introduce to the HTML render. An example HTML render of parts of a paper document is show to the right (actual render is single column, which is split here for presentation).}
    \label{fig:pipeline}
    \Description{A schematic diagram showing the components of the SciA11y pipeline. An image of a paper PDF is on the left. Red boxes on the PDF image highlight the text components from the paper, with an arrow pointing to a box that says "S2ORC extracts: title, authors, abstract, section headers, body paragraphs, and references." A blue box on the PDF image highlights a figure, with an arrow pointing to a box that says "DeepFigures extracts: figures, figure captions, tables, and table titles/captions." A box below "S2ORC extracts" and "DeepFigures extracts" says "Additional content: metadata from Semantic Scholar API, table of contents, figures and tables inserted at first mention, and links between references and text." Arrows from all three boxes point into a larger box that describes the SciA11y prototype, where HTML tags are inserted around various blocks of text extracted from the PDF. On the right of all this is a screen capture of an example HTML render, showing how the semantic content from the PDF is represented as a single-column HTML page for easy reading.}
\end{figure}

To offset the reading challenges of inaccessible papers for BLV researchers, we propose and test the \scially system for rendering academic PDFs into accessible HTML documents. As shown in Figure~\ref{fig:pipeline}, our prototype integrates several machine learning text and vision models to extract the structure and semantic content of papers. The content is represented as an HTML document with headings and links for navigation, figures and tables, as well as other novel features to assist in document structure understanding. Our evaluation of the \scially system identifies common classes of extraction problems, and finds that though many papers exhibit some extraction errors, the majority (55\%) have no major problems that impact readability, and another 32\% have only some problems that impact readability.

Through our user study, we identify numerous challenges faced by BLV users when reading paper PDFs, including some that affect the whole document or limit navigation, and many that affect the ability of the reader to understand text or various elements of a paper like math content or tables. Responses to \scially were positive; participants especially liked navigation features such as headings, the table of contents, and bidirectional links between inline citations and references. Of the extraction errors in \scially, missed or incorrectly extracted headings were the most problematic, as these impact the user's ability to navigate between sections and fully trust the system. All users reported being likely to use the system in the future. When asked how the system might be integrated into their workflow, one participant replied ``I think it would become the workflow.'' Another participant said, ``for unaccessible PDFs, this is life-changing.'' We condense these findings into a set of recommendations for designing and engineering accessible reading systems (Section~\ref{sec:designrecs}). Most importantly, documents should be structured to match a reader's mental model, objects should be properly tagged, and care should be taken to reduce the reader's cognitive load and increase trust in the system. Features that emulate the external memory that visual layout provides to sighted users can be especially beneficial.

This paper is organized as follows. Following a description of related work in Section \ref{sec:related_work}, we first provide a meta-scientific analysis of the current state of academic PDF accessibility in Section \ref{sec:sos}. In Section \ref{sec:pdf2html}, we document our pipeline for converting PDF to HTML and describe the \scially prototype for rendering papers. An evaluation of HTML render quality and faithfulness is provided in Section \ref{sec:evaluation}. Section \ref{sec:user_study} describes our user study and findings. 
We recognize that no PDF extraction system is perfect, and many open research challenges remain in improving these systems. However, based on our findings, we believe \scially can dramatically improve screen reader navigation of most papers compared to PDFs, and is well-positioned to assist BLV researchers with many of their most common reading use cases. Our hope is that a system such as \scially can improve BLV researcher access to the content of academic papers, and that these design recommendations can be leveraged by others to create better, more faithful, and ultimately more usable tools and systems for scholars in the BLV community.

%!TEX root = 0_main.tex
\section{Preliminary}\label{sec:preliminary}
%We briefly introduce the preliminary concepts used in our work.

%\vspace{4mm}
%\subsection{Metric Temporal Logic}
%\subsubsection{Robustness guided falsification}
\subsubsection{{Robustness guided falsification}}
%
In this paper, we employ a variant of \emph{Signal Temporal Logic ($\STL$)} defined in~\cite{bartocci2018specification}. The syntax is defined in the equation (\ref{eq:MTL}),
\begin{equation}\label{eq:MTL}
  \varphi ::= v \sim c \mid p \mid \neg \varphi \mid
  \varphi_1 \vee \varphi_2
  \mid \varphi_1 \UNTIL_I \varphi_2
  \mid \varphi_1 \SINCE_I \varphi_2
\end{equation}
where $v$ is \emph{real} variable,
$c$ is a rational number,
$p$ is atomic formula,
$\sim \in \{<, \leq\}$
and $I$ is an interval over non-negative real numbers.
%\ttodo{Propositional variables are omitted.
%  I think we need them to treat True in the definition of box operator,
%  and gears in our experiment.}
If $I$ is $[0, \infty]$, $I$ is omitted.
We also use other common abbreviations,
e.g., $\square_I \varphi \equiv \mathsf{True} \UNTIL_I \varphi$ and
$\boxminus_I \varphi \equiv \mathsf{True} \SINCE_I \varphi$.
For a given formula $\varphi$, an output signal $\mathbf{x}$ and time $t$, we adopt the notation of work~\cite{bartocci2018specification} and denote the \emph{robustness degree} of output signal $\mathbf{x}$ satisfying $\varphi$ at time $t$ by $\rob(\varphi, \mathbf{x}, t)$.
%Note that,
%intuitively,
%the robustness degree $\rob(\varphi, x, t)$
%stands for
%how ``robust'' the signal $x$ satisfies the formula $\varphi$ at time $t$.

%% Robustness~\cite{TLF_CPS_2013} of a MTL formula $\phi$ for a output trace $y$ at the time instant $t_n$ is defined as
%% \todo{Does n means $t_n$ in function rob()?}
%% $\rob(\mathbf{y}, n, \phi)$ , and is a measure of how
%% ``robust'' $\phi$ holds.
%% For atomic formula $p$, the robustness $\rob(\mathbf{y}, n, p)$ is defined as the infimum of the distance of the point $y$ which does not satisfies $p$ from $y_n$.
%% \todo{$y_n$ means $y_t$? or $y_tn$? We either use t or n, be consistent}
%% %The distance between two states $\dist(y, y_n)$ can be any metric, but
%% \textcolor{red}{
%% In this paper we use Euclidian metric to define the distance between two states $\dist(y, y_n)$.
%% For example, if $y_n = 0$ and $p = \{ y \mid y < 1 \}$, $\rob(\mathbf{y}, n, p) = \dist(y_n, y) = 1$.}
%% %

We also adopt the notion of \emph{future-reach} $\hrz(\varphi)$ and
\emph{past-reach} $\hst(\varphi)$ following~\cite{DBLP:conf/rv/HoOW14}.
%% The \emph{horizon} $\hrz(\phi)$ of a MTL formula $\phi$ is the time in future which is required to determine the truth value of the formula $\phi$.
%Generally speaking, for a given formula $\varphi$,
Intuitively, $\hrz(\varphi)$ is the time in future which is required to determine the truth value of formula $\varphi$, and $\hst(\varphi)$ is the time in past.
For example,
$\hrz(p) = 0$, $\hrz(\square_{[0, 3]}p) = 3$ and $\hrz(\boxminus_{[0,3]}p) = 0$.
%%Similarly we define the \emph{history} $\hst(\varphi)$.
Similarly, for past-reach,
$\hst(p) = 0$, $\hst(\square_{[0, 3]}p) = 0$, $\hst(\boxminus_{[0,3]}p) = 3$.

%% \begin{lemma}[Robustness of a past dependent formula]\label{lem:rob-past}
%%   Let $\mathbf{y}$ be a finite trace of system states $y_0, \ldots, y_n$.
%%   Let $\overline{\mathbf{y}}_1$ and $\overline{\mathbf{y}}_2$ be two infinite extensions of $\mathbf{y}$.
%%   If $\phi$ is past dependent,
%%   \begin{equation}
%%     \rob(\overline{\mathbf{y}}_1, t, \phi) = \rob(\overline{\mathbf{y}}_2, t, \phi)
%%   \end{equation}
%% \end{lemma}

%% By Lemma \ref{lem:rob-past}, robustness of a past dependent formula at instant $n$ is completely determined by $y_0, \ldots, y_n$.
%% Therefore, we use the notation $\rob(\mathbf{y}, t, \phi)$ for robustness of a past-dependent formula $\phi$ on a finite trace $\mathbf{y}$.

%\subsection{Past-dependent life-long property falsification}
In this paper, we focus on a specific class of the temporal logic formula called \emph{life-long property}.
%to employ our approach.

\begin{definition}[life-long property]
  A \emph{life-long property} is an $\STL$ formula $\psi \equiv \square \varphi$ where $\hrz(\varphi),
  \hst(\varphi)$ are finite.
  If $\hrz(\varphi) = 0$, we call $\psi$ \emph{past-dependent life-long property}.

\end{definition}

%% %TODO life-long property, not past-dependent
%% \begin{definition}[Past-dependent life-long property]
%%   A \emph{life-long property} is an $\mathsf{MTL}$ formula $\psi \equiv \square \varphi$ where $\varphi$ only has a finite horizon.
%%   In particular, if $\varphi$ only contains bounded modal operators, $\psi$ is a life-long property.
%%   If $\varphi$ is past-dependent, then $\square \varphi$ is called \emph{past-dependent life-long property}.
%% \end{definition}


%\vspace{4mm}
\subsubsection{Reinforcement Learning}
%\subsection{Reinforcement learning}
%\subsubsection{Reinforcement Learning}
Reinforcement learning is one of machine learning techniques in which an agent learns the structure of the environment based on observations, and maximizes the rewards by acting according to the learnt knowledge.
% Reinforcement learning is first proposed and used in the domain of audio and image processing to improve the analysis performance. Reinforcement learning has shown its power and potential in training AlphaGo Zero~\cite{AlphaGo0}, which became the world's best Go player in 40 days, from scratch.
%In this work, we are using reinforcement learning techniques to reduce the accelerate the process of finding the counterexample, which falsifies the robustness property defined for a CPS.
%In particular, we adopt Asynchronous Advantage Actor-Critic (A3C) and Double Deep Q Network (DDQN) in our problem.
%
%Fig. \ref{fig:RL} shows the standard setting of reinforcement learning.
%%
%
%\begin{figure}
%  \centering
%  \scriptsize
%  \includegraphics[scale=0.67]{fig/RL.pdf}
%  \caption{Reinforcement learning setting}
%  \label{fig:RL}
%  \vspace{-7mm}
%\end{figure}
%
%%
The standard setting of a reinforcement learning problem consists of an agent and an environment. %, as shown in Fig.~\ref{fig:arch}.
The agent observes the current state and reward from the environment, and returns the next action to the environment.
%
%%
%%%
% Reinforcement learning is often formulated as a \emph{Markov decision process (MDP)}~\cite{Szepesvari2010}.
% A MDP is a triple $\mathcal M = (\mathcal X, \mathcal A, \mathcal P_0)$.
% \state is a set of states, \action is a set of actions and \probkernal is the transition probability kernel.
% A transition probability kernel \probkernal assigns a probability distribution (over $\mathcal X \times \mathbb R$, which is a distribution over the next states and the reward when the agent takes an action $a$ at the state $x$.), to each state-action pair $(x, a) \in \mathcal X \times \mathcal A$.
The goal of reinforcement learning is for each step $n$, given the sequence of previous states $x_0, \ldots, x_{n-1}$, rewards $r_1, \ldots, r_{n}$ and actions $a_0, \ldots, a_{n-1}$, generate an action $a_n$, which maximizes expected value of the sum of rewards:
%\begin{equation}
 $ r = \sum_{k = n}^\infty \gamma^k r_{k+1}$
%\end{equation}
, where $0 < \gamma \leq 1$ is a discount factor.
%
% For each state $x \in \mathcal X$, $V^*(x)$ is used to denote the highest achievable expected value of reward $r$, when $x_0 = x$.
%There are different kinds of reinforcement learning algorithms proposed in the literature. These approaches mainly falls into 2 different categories, i.e., value based and policy based, categorized by types of agents.
%
% \emph{$Q$-learning} \cite{} is the representative method for value-based reinforcement learning algorithm.
% For each action-state pair $(x, a)$, let \emph{optimal action-value function} $Q^*(x, a)$ be the highest achievable expected value of $r$ when $x_0 = x$ and $a_0 = a$.
% Once the value of $Q^*$ is known, the optimal strategy is to choose action $a$ which maximizes $Q^*(x, a)$ for the current state $x$ (following the greedy policy).
% One approach of reinforcement learning is to directly estimate $Q^*$ and use this estimated value to determine best actions.
% This approach is called \emph{$Q$-learning} \cite{}.
%
% \emph{actor-critic} method is the representative method for the other kind of approaches, policy-based algorithms.
% A \emph{stochastic stationary policy} (or just \emph{policy}) $\pi$ maps states in \state to probability distributions over actions in \action.
% The set of all policies is denoted by $\Pi_{\mathrm{stat}}$.
% Each policy $\pi$ gives rise to a \emph{Markov reward process (MRP)} $\mathcal M = (\mathcal X, \mathcal P_0)$.
% In a MRP, the state makes transitions as a Markovian process and generates a sequence of rewards $r_1, r_2, \ldots$.
% The action-value function $Q^\pi$ is defined by
% \begin{equation}
%   Q^\pi(x, a) = \mathbf E \left[ \sum_{t = 0}^\infty \gamma^t R_{t+1} \middle| x_0 = x, a_0 = a \right]
% \end{equation}
% where $\mathbf E$ signifies the expect value.
% An actor-critic method works as follows.
% First, it starts with a random policy $\pi_0$ and the ``actor'' follows $\pi_0$ some duration of time.
% Then, the the ``critic'' estimates $Q^{\pi_0}$ by the results of the run.
% In the next phase, the a greedy policy $\pi_1$ determined by estimated $Q^{\pi_0}$ is generated and the actor follows $\pi_1$.
% The actor-critic method repeats this process.
%
%%
%%%
Deep reinforcement learning is a reinforcement learning technique which uses a \emph{deep neural network} for learning.  % to represent a $Q$-function and/or a policy $\pi$.
In this work, we particularly adopted 2 state-of-the-art deep reinforcement learning algorithms, i.e., \emph{Asynchronous Advantage Actor-Critic} (A3C)~\cite{Mnih2016} and \emph{Double Deep Q Network} (DDQN)~\cite{pmlr-v48-gu16}.
% We briefly review these methods in the following.
% %
%
% \noindent \textbf{A3C: Asynchronous Advantage Actor-Critic}
% Asynchronous Advantage Actor-Critic (A3C)~\cite{Mnih2016} utilizes multiple processes to accelerate the training process. All processes run the same training algorithm and the information is collected by a central process. In this way, the algorithm and train models much faster.
%
% \noindent \textbf{Double Deep Q Network}
% DQN~\cite{mnih2013playing} combines CNN and Q-learning. A CNN network is used to analyze the Q-value. DQN can is proposed to solve large problems, which is hard to tackle with the normal table-based Q-learning algorithms.
%
%The output of the DQN is the best action to take in the current state and it's corresponding q-value. Moreover, to avoid the problem that may be caused by over estimation, we adopt double DQN  (DDQN) which uses Current Q-network to select actions and older Q-network to evaluate actions. Current Q-network use max to find the best action, overestimation may happen here. But another Q-network which doesn't use max evaluates q-value of the selected action. It may high or low. So over estimation is solved.

% Double DQN is an improvement of DQN, it has two networks to conduct action selection and Q-value evaluation separately. Double DQN learns faster and can avoid the over-estimation problem of DQN.
%The loss function is
%\begin{equation}
% L(w) =  \mathbb{E}\left[(r+\gamma\mathop {\max }\limits_{a'} Q(s',a',w) - Q(s,a,w))^2\right]
%\end{equation}
%
%DQN is short for Deep Q network\cite{mnih2013playing}. Q-learning is a Reinforcement Learning algorithm which for finite states and actions. Practical problems may have too many states therefore Q-learning has some limits. So q-values based on neural network comes true.
%
%\begin{algorithm}
%  \caption{Deep Q-learning with Experience Replay}
%  \begin{algorithmic}
%  \State Initialize replay memory D to capacity N % \State
%  \State Initialize action-value function Q with random weights
%    \For{episode = 1;M}
%      \State Initialise sequence $s_1 = \left\{ x_1 \right\}$ and preprocessed sequenced $\phi_1 = \phi(s_1) $
%    \EndFor

%  \end{algorithmic}
%\end{algorithm}
%

%%!TEX root = 0_main.tex
\section{Motivating Example}\label{sec:example}

(... Explanation of an ARCH2014 example ...)

\begin{definition}[System]
  An \emph{(finite) input trace} $\mathbf{x} = x_0, x_1, \ldots, x_n$ is a sequence of elements of the \emph{input space} $\mathbb R^N$ together with the sequence of sampling time instants $\mathbf{t} = t_0, t_1, \ldots, t_n, \ldots$.
  $\mathbf{t}$ is often omitted to be mentioned.

  An \emph{(finite) output trace} $\mathbf{y} = y_0, y_1, \ldots, y_n$ is a sequence of elements of the \emph{system states} $\mathbb R^M$.
  If an (input/output) trace $\mathbf{z}_1$ and its sampling time instants $\mathbf{t}_1$ are prefixes of $\mathbf{z}_2$ and $\mathbf{t}_2$ respectively, the trace $\mathbf{z}_1$ is a prefix of $\mathbf{z}_2$.

  A \emph{(reactive) system} $\mathbf{f}$ is a map from finite input traces $x_1, \ldots, x_{n-1}$ to $y_1, \ldots, y_n$ such that if $\mathbf{x}_1$ is a prefix of $\mathbf{x}_2$, $\mathbf{f}(\mathbf{x}_1)$ is a prefix of $\mathbf{f}(\mathbf{x}_2)$.
\end{definition}

Because of the definition, $\mathbf{f}$ induces a map from \emph{infinite} traces $\mathbf{x}$ to $\mathbf{y}$.
We use the same notation $\mathbf{f}$ for this map.

(... Intuitive explanation and references ...)
\begin{definition}[Metric Temporal Logic (MTL) formula]
  A \emph{metric temporal logic (MTL)} formula $\phi$ is defined by a BNF as follows.
  \begin{equation}
    \phi ::= p \ \mid\  \phi \wedge \phi \ \mid\ \phi \vee \phi \ \mid \ \neg \phi \ \mid\  \square_I \phi \ \mid\ \diamond_I \phi \ \mid\ \phi \ \mathcal{U}_I \psi \ \mid\ \phi \ \mathcal{S}_I \psi \ \mid\ X\phi \ \mid\ P\phi
  \end{equation}
  where $p$ is an atomic formula and $I$ is any interval on $\mathbb R$.
  We omit $I$ if $I = [0, \infty]$.
  For $\mathcal{U}$ and $\mathcal{S}$, we assume $I$ is non-negative.
\end{definition}

\begin{definition}
  Let $\phi$ be a MTL-formula.
  Let $\mathbf{t} = t_0, t_1, \ldots, t_n, \ldots$ be an infinite sequence of sampling time of the system states.
  Let $\mathbf{y} = y_0, y_1, \ldots, y_n, \ldots$ be system states of time instants $\mathbf{t}$ respectively.
  The relation $\mathbf{y}, n \models \phi$ (read that $\phi$ holds at $n$ on the trace $\mathbf{y}$) is defined recursively on $\phi$ as follows.
  \begin{align}
    \mathbf{y}, n \models p &\iff p(y_n)\\
    \mathbf{y}, n \models \phi_1 \wedge \phi_2 &\iff \mathbf{y}, n \models \phi_1 \text{ and } \mathbf{y}, n \models \phi_2\\
    \mathbf{y}, n \models \phi_1 \vee \phi_2 &\iff \mathbf{y}, n \models \phi_1 \text{ or } \mathbf{y}, n \models \phi_2\\
    \mathbf{y}, n \models \neg \phi &\iff \neg (\mathbf{y}, n \models \phi)\\
    \mathbf{y}, n \models \square_I \phi &\iff \forall n'\text{ such that } t_{n'} - t_n \in I, \mathbf{y}, n' \models \phi\\
    \mathbf{y}, n \models \square_I \phi &\iff \exists n' \text{ such that } t_{n'} - t_n \in I, \mathbf{y}, n' \models \phi\\
    \mathbf{y}, n \models \phi \ \mathcal{U}_I \psi &\iff
    \begin{gathered}
      \exists n', t_{n'} - t_n \in I \text{ such that } \mathbf{y}, n' \models \psi \text{ and }\\
       n \leq \forall n'' < n', \mathbf{y}, n'' \models \phi
    \end{gathered}\\
    \mathbf{y}, n \models \phi \ \mathcal{S}_I \psi &\iff
    \begin{gathered}
      \exists n', t_n - t_{n'} \in I \text{ such that } \mathbf{y}, n' \models \psi \text{ and }\\
      n' < \forall n'' \leq n, \mathbf{y}, n'' \models \phi
    \end{gathered}\\
    \mathbf{y}, n \models X\phi &\iff \mathbf{y}, n+1 \models \phi\\
    \mathbf{y}, n \models P\phi &\iff n \geq 1 \text{ and } \mathbf{y}, n-1 \models \phi
  \end{align}
\end{definition}

(... Intuitive explanation and references ...)
\begin{definition}[Robustness]
  Let $\phi$ be a MTL-formula.
  The robustness function $\rob(\mathbf{y}, n, \phi)$ over infinite traces $\mathbf{y} = y_0, y_1, \ldots, y_n, \ldots$ is defined as follows.
  \begin{align}
    \rob(\mathbf{y}, n, p) &= \min \{ \dist(y, y_n) \mid \neg p(y) \}\\
    \rob(\mathbf{y}, n, \phi_1 \wedge \phi_2) &= \min(\rob(\mathbf{y}, n, \phi_1), \rob(\mathbf{y}, n, \phi_2))\\
    \rob(\mathbf{y}, n, \neg \phi) &= - \rob(\mathbf{y}, n, \phi)\\
    \rob(\mathbf{y}, n, \phi_1 \vee \phi_2) &= \max(\rob(\mathbf{y}, n, \phi_1), \rob(\mathbf{y}, n, \phi_2))\\
    \rob(\mathbf{y}, n, \square_I \phi) &= \min \{ \rob(\mathbf{y}, n', \phi) \mid t_{n'} - t_n \in I \}\\
    \rob(\mathbf{y}, n, \diamond_I \phi) &= \max \{ \rob(\mathbf{y}, n', \phi) \mid t_{n'} - t_n \in I \}\\
    \rob(\mathbf{y}, n, \phi \ \mathcal{U}_I \psi) &= \max_{n' \text{ s.t. } t_{n'} - t_n \in I} \min(\rob(\mathbf{y}, n', \psi), \min_{n'' = n}^{n'-1} \rob(\mathbf{y}, n'', \phi))\\
    \rob(\mathbf{y}, n, \phi \ \mathcal{S}_I \psi) &= \max_{n' \text{ s.t. } t_{n} - t_n' \in I} \min(\rob(\mathbf{y}, n', \psi), \min_{n'' = n'+1}^{n} \rob(\mathbf{y}, n'', \phi))\\
    \rob(\mathbf{y}, n, X \phi) &= \rob(\mathbf{y}, n+1, \phi)\\
    \rob(\mathbf{y}, n, P \phi) &=
    \begin{cases}
        \rob(\mathbf{y}, n-1, \phi) & \text{ if } n \geq 1\\
        -\infty & n = 0
    \end{cases}
  \end{align}
  where $\dist$ is a distance between two states.
  For the empty set $\emptyset$, $\min \emptyset = \infty$ and $\max \emptyset = -\infty$.
\end{definition}

\begin{definition}[Horizon]
  For each MTL-formula, we assign the \emph{horizon} $\hrz(\phi)$.
  \begin{align}
    \hrz(p) &= 0\\
    \hrz(\phi \wedge \psi) = \hrz(\phi \vee \psi) &= \max(\hrz(\phi), \hrz(\psi))\\
    \hrz(\neg \phi) &= \hrz(\phi)\\
    \hrz(\square_I \phi) = \hrz(\diamond_I \phi) &= \hrz(\phi) + \sup I\\
    \hrz(\phi \ \mathcal{U}_I \psi) &= \max(\hrz(\phi) + \sup I, \hrz(\psi) + \sup I)\\
    \hrz(\phi \ \mathcal{S}_I \psi) &= \max(\hrz(\phi), \hrz(\psi))\\
    \hrz(X\phi) &= \hrz(\phi) + 1\\
    \hrz(P\phi) &= \min(0, \hrz(\phi) - 1)
  \end{align}
  If $\hrz(\phi) = 0$, we call $\phi$ \emph{past dependendt}.
\end{definition}

\begin{lemma}[Robustness of a past dependent formula]\label{lem:rob-past}
  Let $\mathbf{y}$ be a finite trace of system states $y_0, \ldots, y_n$.
  Let $\overline{\mathbf{y}}_1$ and $\overline{\mathbf{y}}_2$ be two infinite extensions of $\mathbf{y}$.
  If $\phi$ is past dependent,
  \begin{equation}
    \rob(\overline{\mathbf{y}}_1, t, \phi) = \rob(\overline{\mathbf{y}}_2, t, \phi)
  \end{equation}
\end{lemma}

By Lemma \ref{lem:rob-past}, robustness of a past dependent formula at instant $n$ is completely determined by $y_0, \ldots, y_n$.
Therefore, we use the notation $\rob(\mathbf{y}, t, \phi)$ for robustness of a past-dependent formula $\phi$ on a finite trace $\mathbf{y}$.

%!TEX root = 0_main.tex
\section{Our Approach}\label{sec:overview}

\begin{algorithm}[tp]
\scriptsize
  \caption{Falsification for $\psi = \square \varphi$ by reinforcement learning}
  \label{algo:RLfalsification}
  \begin{algorithmic}[1]
    \INPUT A past-dependent life-long property $\psi = \square \varphi$, a system $\mathcal{M}$,
    an agent $\mathcal{A}$
    \OUTPUT A counterexample input signal $\mathbf{u}$ if exists
    \PARAMETERS A step time $\Delta_T$, the end time $T_{\mathsf{end}}$, the maximum number of the episode $N$
 %   \Ensure something
    \For{$\mathsf{numEpisode} \gets$ $1$ to $N$}
    \State $i \gets 0$, $r \gets 0$, $x$ be the initial (output) state of $\mathcal{M}$
    \State $\mathbf{u}$ be the empty input signal sequence
    \While{$i \Delta_T < T_{\mathsf{end}}$}
    \State $u \gets \mathcal{A}.\mathsf{step}(x, r)$,  $\mathbf{u} \gets \mathsf{append}(\mathbf{u}, (i \Delta_T, u))$
    \Comment choose the next input by the agent
    \State $\mathbf{x} \gets \mathcal{M}(\mathbf{u})$, $x \gets \mathbf{x}((i+1)\Delta_T)$
    \Comment simulate, observe the new output state
    \State $r \gets \reward(\mathbf{x}, \psi)$
    \State $i \gets i+1$
    \Comment calculate the reward by following eq.~(\ref{def:reward})
    \EndWhile
    \If{$\mathbf{x} \not\models \psi$}
    %% \Then
    \Return $\mathbf{u}$ as a falsifying input
    \EndIf
    \State $\mathcal{A}.\mathsf{reset}(x, r)$
    \EndFor
   \end{algorithmic}
\end{algorithm}
%
% In this section, we describe our method
% of enforcing the CPS model to falsify the given STL specification
% by reinforcement learning.

\subsection{Overview of our algorithm}\label{subsec:algorithm}
Let us consider the falsification problem to find a counterexample of the life-long property $\psi \equiv \square \varphi$.
If the output signal is infinitely long to past and future directions, $\psi$ is logically equivalent to a past-dependent life-long property $\square \boxminus_{[\hrz(\varphi), \hrz(\varphi)]} \varphi$.
In general, the output signal is not infinitely long to some direction but using this conversion we convert all life-long properties to past-dependent life-long properties.
Our evaluation in Section \ref{sec:exp} suggests that this approximation does not adversely affect the performance.
%
Therefore, assume $\psi$ is a past-dependent life-long property, we generate an input signal $\mathbf{u}$ for system $\mathcal{M}$,
such that the corresponding output signal $\mathcal{M}(\mathbf{u})$ does not satisfy $\psi$.

In our algorithm,
we fix the simulation time to be $T_{\mathsf{end}}$
and
call one simulation until time $T_\mathsf{end}$
an \emph{episode} in conformance with the reinforcement learning terminology.
We fix the discretization of time to a positive real number $\Delta_T$.
%For an agent $\mathcal{A}$,
%in each episode,
%it generates an input signal $\mathbf{u}(t)$
%%adaptively
%based on the observed current system output and the reward.
%More precisely,
The agent $\mathcal{A}$ generates the piecewise-constant input signal
$\mathbf{u} = \big[(0, u_0), (\Delta_T, u_{1}), (2\Delta_T, u_{2}), \dots \big]$
by iterating the following steps:

%\begin{enumerate}
%\item
% At time $i \Delta_T$ ($i=0,1,\dots$),
%  the agent $\mathcal{A}$ choose the next input value $u_i$.
%  The generated input signal is extended to
%  $\mathbf{u} = \big[(0, u_0), \dots, (i\Delta_T, u_i) \big]$. \\
%\item
%  Our algorithm obtains the corresponding output signal $\mathbf{x} = \mathcal{M}(\mathbf{u})$
%  by stepping forward one simulation on the model $\mathcal{M}$
%  from time $i \Delta_T$ to $(i+1) \Delta_T$ with input $u_{i}$. \\
%\item
% Let $x_{i+1} = \mathbf{x}((i+1)\Delta_T)$ be the new (observed) state (i.e., output) of the system. \\
%\item
% We compute reward $r_{i+1}$ by $\reward(\varphi, \mathbf{x}, (i+1)\Delta_T)$ (defined in Section \ref{subsec:reward}). \\
%\item
% $\mathcal{A}$ updates its action based on the new state $x_{i+1}$ and the reward $r_{i+1}$.
%\end{enumerate}

(1) At time $i \Delta_T$ ($i=0,1,\dots$),
  the agent $\mathcal{A}$ chooses the next input value $u_i$.
  The generated input signal is extended to
  $\mathbf{u} = \big[(0, u_0), \dots, (i\Delta_T, u_i) \big]$. \\
%\item
\indent (2) Our algorithm obtains the corresponding output signal $\mathbf{x} = \mathcal{M}(\mathbf{u})$
  by stepping forward one simulation on the model $\mathcal{M}$
  from time $i \Delta_T$ to $(i+1) \Delta_T$ with input $u_{i}$. \\
%\item
\indent (3) Let $x_{i+1} = \mathbf{x}((i+1)\Delta_T)$ be the new (observed) state (i.e., output) of the system. \\
%\item
\indent (4) We compute reward $r_{i+1}$ by $\reward(\varphi, \mathbf{x}, (i+1)\Delta_T)$ (defined in Section \ref{subsec:reward}). \\
%\item
\indent (5) The agent $\mathcal{A}$ updates its action based on the new state $x_{i+1}$ and reward $r_{i+1}$.

At the end of each episode,
we obtain the output signal trajectory $\mathbf{x}$,
and check whether it satisfies the property $\psi = \square \varphi$ or not.
If it is falsified, return the current input signal $\mathbf{u}$ as a counterexample.
Otherwise, we discard the current generated signal input
and restart the episode from the beginning.

The complete algorithm of our approach is shown in Algorithm~\ref{algo:RLfalsification}.
The method call $\mathcal{A}.\mathsf{step}(x, r)$ represents
the agent $\mathcal{A}$ push the current state reward pair ($x$, $r$) into its memory
and returns the next action $u$ (the input signal in the next step).
The method call $\mathcal{A}.\mathsf{reset}(x, r)$ notifies the agent that the current episode is completed, and returns the current state and reward.
%\ttodo{This sentence seems misleadning as is discussed in our skype chat.}
Function $\reward(\mathbf{x}, \psi)$ calculates the reward based on Definition~\ref{def:reward}.



\subsection{Reward definition for life-long property falsification}\label{subsec:reward}
Our goal is to find the input signal $\mathbf{u}$ to the
system $\mathcal{M}$ which minimizes $\rob(\psi, \mathcal{M}(\mathbf{u}), 0)$ where $\psi = \square \varphi$ and $\rho$ is a robustness.
We determine $u_0, u_1, \ldots$ in a greedy way.
Assume that $u_0, \ldots, u_i$ are determined.
$u_{i+1}$ can be determined by
\begin{align}
\scriptstyle
\label{eq:action}
  u_{i+1} &= \argmin_{u_{i+1}} \min_{u_{i+2}, \ldots} \rob(\square \varphi, \mathcal{M}(\left[(0, u_0), (\Delta_T, u_1), \ldots \right]), 0) \\
  &\sim \argmax_{u_{i+1}} \max_{u_{i+2}, \ldots} \sum_{k=i+1}^\infty \{ e^{- \rob(\varphi, \mathcal{M}(\left[(0, u_0), \ldots, (k\Delta_T, u_k) \right]), k\Delta_T)} - 1\} \label{eq:r}
\end{align}
The detailed derivation steps can be found in Appendix~\ref{sec:appendix}.
%\eqref{eq:disc} uses the fact $\varphi$ is past-dependent and \eqref{eq:logsum} uses an approximation of minimum by the log-sum-exp function~\cite{cook2011basic}.

In our reinforcement learning base approach, we use discounting factor $\gamma=1$ and reward $r_i = e^{- \rob(\varphi, \mathcal{M}(\left[(0, u_0), \ldots, (i\Delta_T, u_i)\right]), i\Delta_T)} - 1$  to approximately compute action $u_{i+1}$, from $u_0, \ldots, u_i$, $\mathcal{M}(\left[(0, u_0), \ldots, (i\Delta_T, u_i) \right])$ and $r_1, \ldots, r_i$.
%\ttodo{No guarantee that u is approximately computed.
%  We cannot estimate the approximation error.
%  I prefer we claim
%  ``we use the discounting factor and reward to hopefully compute the approximation of the next action u...''
%}
%If the reward $r_i = e^{- \rob(\varphi, \mathcal{M}(\left[(0, u_0), \ldots, (i\Delta_T, u_i)\right]), i\Delta_T)} - 1$ and discounting factor $\gamma=1$ are used, we expect a reinforcement learning algorithm
%approximately computes $u_{i+1}$ as an action from $u_0, \ldots, u_i$, $\mathcal{M}(\left[(0, u_0), \ldots, (i\Delta_T, u_i) \right])$ and $r_1, \ldots, r_i$.

\begin{definition}[reward]\label{def:reward}
  Let $\psi \equiv \square \varphi$ be a past-dependent formula
  and $\mathbf{x} = \mathcal{M}(\mathbf{u})$ be a finite length signal until the time $t$.
  We define the reward $\reward(\psi, \mathbf{x})$ as
  \begin{equation}\label{eq:reward}
    \reward(\psi, \mathbf{x}) =
      \exp(- \rob(\varphi, \mathbf{x}, t)) - 1
  \end{equation}
\end{definition}

%!TEX root = 0_main.tex
\section{Preliminary Results}\label{sec:exp}
%To evaluate the efficiency and effectiveness of our work, we conduct experiments with well known CPS models in Matlab/Simulink.
%We compare our reinforcement learning based technique with existing methods, i.e., simulated annealing and cross entropy based methods, and analyze the results.
%\ytodo{Mention Breach when we finish the comparison with Breach}

%We discuss our implementation and report our preliminary evaluation result in this paper.

%\subsection{Implementation}
%\vspace{5mm}
\noindent\textbf{Implementation}
%\subsubsection{Implementation}
\begin{figure}[tp]
\centering
\scriptsize
\includegraphics[scale=0.44]{./fig/Architecture.pdf}
\caption{Architecture of our system}
\label{fig:arch}
\vspace{-7mm}
\end{figure}
%
The overall architecture of our system is shown in Fig.~\ref{fig:arch}.
Our implementation consists of three components, i.e., input generation, output handling and simulation.
The input generation component adopts reinforcement learning techniques and is implemented based on the ChainerRL library~\cite{ChainerRL}.
We use default hyper-parameters in the library or sample programs without change.
The output handling component conducts reward calculation using dp-TaliRo~\cite{S-TaliRo}.
The simulation is conducted with Matlab/Simulink models, which are encapsulated by the openAI gym library~\cite{1606.01540}.
%The output of the system is normalized (linearly mapped) into the interval $[-1, 1]\subseteq \mathbb{R}$, since deep reinforcement learning has the best performance for normalized inputs.
%The robustness and reward are calculated using normalized outputs.
%All input to the system are first mapped into the interval $[-1, 1]$ and then linearly transformed to the actual range, which the system accepts.
%%%
%%
%
%To falsify a CPS model, we run the model step by step using a fixed sampling rate, which is a period of the time obeyed by the model.
%The reward is calculated based on the trace of states, from the starting state to the current state of the run, based on the formula defined in~\ref{eq:reward}.
%For each step, the state of the model and the reward are used as input to the reinforcement learning agent.
%The reinforcement learning agent then returns the next action to take.
%Using the suggested action as an input, we restart the simulation cycle.
%
%%
%%%
%To calculate the robustness value of the monitoring formula, we use a function in S-TaliRo for calculating robustness.
%S-TaliRo provides two kinds of functions to calculate robustness,  one is a MATLAB function \verb|dp-taliro| and its variants; the other is a runtime monitoring function, which is a Simulink block.
%Our experiment uses the \verb|dp-taliro| function to calculate robustness, \textcolor{red}{because...}.
%However, \verb|dp-taliro| does not support past-dependent formulas.
%Therefore, to calculate robustness, we revert the order of the trace, i.e., from the current state to the beginning state.
%We also revert the time flow of the monitoring formula.
%For example, $\square_{[-3, 0]}p$ is converted to $\square_{[0, 3]}p$ and $\square_{[-5, -3]}p \rightarrow \square_{[-3, 0]}q$ is converted to $\square_{[3, 5]}p \rightarrow \square_{[0, 3]}q$.
%However, this conversion is not always possible.
%The until operator $\mathcal{U}_I$, the next operator $X$ and nested temporal operators seem resist from being converted.
%Therefore, our implementation can only handle limited properties.
%\stodo{The sentence ``The until operator $\mathcal{U}_I$, the next operator $X$ and nested temporal operators seem resist from being converted.
%Therefore, our implementation can only handle limited properties.'' uses the word ``seem'', which is not good. We need to make assured claims, i.e., give clear claims, whether those operators resist from being converted or not. }

%Then, we input the current state and the calculated reward to a reinforcement learning agent.
%The reinforcement learning agent then returns the next \emph{action} to take.
%Using the suggested action as an input, we restart and execute the simulation until the next step.
%We assume that the input (\textcolor{red}{to the simulation model?}) is constant during each step of the simulation.
%
%%%
%Finally, if the simulation reaches the end, we calculate robustness of the property which we want to falsify using the full trajectory.
%We use \verb|dp-taliro| for this purpose.
%%%
%
%but we do not revert the order of the trace nor the time flow of the formula as the case of the monitoring formula, because
%
%If the robustness of the property is negative, we successfully falsify the property and terminate the process.
%Otherwise, we reset the simulation and signifies the reinforcement learning agent at the end of the \emph{episode}.

%Our implementation has a large overhead in computation time.
%This is mostly due to the fact that, we utilized step-based reinforcement learning algorithms, and after getting the feedback from the RL agent, we stop and restart the simulation for each sampling cycle.
%Another reason is that reinforcement learning agents are implemented by Python while robustness calculation and simulation are implemented by MATLAB/Simulink.
%This creates another overhead to interpolate Python and MATLAB runtime.

%\ytodo{The part below is better to be moved to the preliminary.}
%\noindent\textbf{Double DQN}
%We adopt DDQN to accelerate the process of choosing inputs which may falsify the robustness properties. We describe the implementation of DDQN in our approach.  For the Autonomous system, the inputs to the model are the values of throttle and break, which are normalized to the range of ...; and the outputs of the model are Engine speed $\omega$ (RPM) and vehicle speed $\mathbf(v)$ (mph).In order to use DDQN to find the counterexample, which falsifies the robustness property of the autonomous model, we need to find a way to link the robustness value with the rewards calculated by DDQN.
%In Deep Q Network~\cite{mnih2013playing},the outputs are q-values of each action available at the current state. However, the actions taken (throttle and break value pairs) by the autonomous model is contentious, which results in an infinite input space. To solve this problem, we use Normalized Advantage Functions (NAF)~\cite{pmlr-v48-gu16}, which leads into Advantage. The relationship between Q(q-value),A(advantage) and V(value) is
%\begin{equation}
% Q(s,a) = A(s,a) + V(s)
%\end{equation}
%Then we can limit A less than 0, and select the A which equals 0.

%   The output of the DQN is the best action to take in the current state and it's corresponding q-value. Moreover, to avoid the problem that may be caused by over estimation, we adopt double DQN which uses Current Q-network to select actions and older Q-network to evaluate actions. Current Q-network use max to find the best action, overestimation may happen here. But another Q-network which doesn't use max evaluates q-value of the selected action. It may high or low. So over estimation is solved.


%\subsection{Evaluation Settings}
\vspace{4mm}
\noindent \textbf{Evaluation Settings}
%\subsubsection{Evaluation Settings}
We use a widely adopted CPS model, automatic transmission control system ($\ATmodel$) ~\cite{bardh2014benchmarks}, to evaluate our method.
%
%The system model is from a public demonstration of modeling an automatic transmission control with the Stateflow~\cite{ATinSF}
%%~\footnote{https://mathworks.com/products/stateflow.html}
%package in MATLAB.
%
$\ATmodel$ has throttle and brake as input ports, and the output ports are the vehicle velocity $v$, the engine rotation speed $\omega$ and the current gear state $g$.
%Although the size of the model is relatively small comparing the actual systems in industry, but its dynamics contains both discrete and continuous values.
%Therefore, it is suitable as a benchmark of falsification on CPSs.
%
We conduct our evaluation with the formulas in Table~\ref{tab:formulas}.
Formulas $\varphi_1$--$\varphi_6$ are rewriting of  $\varphi^{AT}_1$--$\varphi^{AT}_6$ in benchmark~\cite{bardh2014benchmarks} into life-long properties in our approach.
%We do not use the original benchmark because we focus on life-long properties.
In addition, we propose three new formulas $\varphi_{7}$--$\varphi_{9}$.
%\textcolor{red}{For all formulas, we tune parameters in the formulas such that they are difficult to falsify, therefore can differentiate performance of each method.}
%
For each formula $\varphi_1$--$\varphi_9$, we compare the performance of our approaches (A3C, DDQN), with the baseline algorithms, i.e., simulated annealing ($\SA$) and cross entropy ($\CE$).
%
For each property, we run the falsification procedure 20 times.
For each falsification procedure, we execute simulation episodes up to 200 times and measure the number of simulation episodes required to falsify the property.
If the property cannot be falsified within 200 episodes, the procedure fails.
%We record whether each falsification procedure is successful or not.
%
% Further, for fair comparison, we change sampling rate and choose the best performing rate for each combination of a method and formula.
% Here, the best means the smallest median of the number of simulations runs required to falsify the formula.
% If the tie occurs, we further compares success rates of the falsification procedure.
% Finally, if still we cannot decide we use the media of the execution time of the falsification processes.
%
We observe that $\Delta_{T}$ may strongly affect the performance of each algorithm.
%For example, simulated annealing tends to perform badly if we use high sampling rate.
%On the other hand, reinforcement learning based methods are not affected by high sampling rate.
%Based on the above observation,
Therefore, we
%choose the $\Delta_{T}$ which gives the best performance (prioritized by numEpisode and success rate) of each algorithm.
vary $\Delta_{T}$ (among \{1, 5, 10\} except for the cases of $\AAAC$ and $\DQN$ for $\varphi_7$--$\varphi_9$ among we use \{5, 10\}~\footnote{These methods with $\Delta_{T}=1$ for $\varphi_7$--$\varphi_9$ shows bad performance and did not terminate in 5 days.}) and report the setting (of $\Delta_{T}$) which leads to the best performance (the least episode number and highest success rate) for each algorithm.
%We change $\Delta_{T}$ from 1, 5, 10 for $\varphi_1$--$\varphi_6$ and 5, 10 for $\varphi_7$--$\varphi_9$.
%$\Delta_{T}$ 1 for $\varphi_5$--$\varphi_7$ is omitted because of the performance reason.



\begin{table}[tp]
  \centering
  \begin{minipage}[t]{.48\textwidth}
    \centering
    \scriptsize
    \begin{tabular}{c||c}
      id & Formula\\
      \hline
      \hline
      $\varphi_1$ & $\square \omega \leq \overline{\omega}$\\
      $\varphi_2$ & $\square (v \leq \overline{v} \wedge \omega \leq \overline{\omega})$\\
      $\varphi_3$
      & $\square ((g_2 \wedge \diamond_{[0, 0.1]} g_1) \rightarrow \square_{[0.1, 1.0]} \neg g_2)$\\
      $\varphi_4$
      & $\square ((\neg g_1 \wedge \diamond_{[0, 0.1]} g_1) \rightarrow \square_{[0.1, 1.0]} g_1)$\\
      $\varphi_5$
      & $\square \bigwedge_{i=1}^4 ((\neg g_i \wedge \diamond_{[0, 0.1] g_i}) \rightarrow \square_{[0.1, 1.0]} g_i)$\\
    \end{tabular}
  \end{minipage}
  \begin{minipage}[t]{.48\textwidth}
    \centering
    \scriptsize
    \begin{tabular}{c||c}
      id & Formula\\
      \hline
      \hline
      $\varphi_6$
      & $\square (\square_{[0, t_1]} \omega \leq \overline\omega \rightarrow \square_{[t_1, t_2]} v \leq \overline{v})$\\
      $\varphi_7$
      & $\square v \leq \overline{v}$\\
      $\varphi_8$
      & $\square \diamond_{[0,25]} \neg (\underline{v} \leq v \leq  \overline{v})$\\
      $\varphi_9$
      & $\square \neg \square_{[0,20]} (\neg g_4 \wedge \omega \geq \overline{\omega})$\\
    \end{tabular}
  \end{minipage}
  \caption{The list of the evaluated properties on $\ATmodel$.}
  \label{tab:formulas}
  \vspace{-5mm}
\end{table}
\begin{table}[tp]
  \centering
    \centering
    \scriptsize
    \begin{tabular}[t]{c||c c c c|c c c c|c c c c|}
      id & \multicolumn{4}{|c|}{$\Delta_T$} & \multicolumn{4}{|c|}{Success rate} & \multicolumn{4}{|c|}{$\mathsf{numEpisode}$}\\
      \hline
      & $\AAAC$ & $\DQN$ & $\SA$ & $\CE$ & $\AAAC$ & $\DQN$ & $\SA$ & $\CE$ & $\AAAC$ & $\DQN$ & $\SA$ & $\CE$ \\
      \hline
      \hline
      $\varphi_1$ & 5 & 1 & 10 & 5 & $\textbf{100}\%^*$ & $\textbf{100}\%^*$ & 65.0\% & 10.0\% & $\textbf{16.5}^{**}$ & 24.5 & 118.5 & 200.0\\
      $\varphi_2$ & 5 & 1 & 10 & 5 & $\textbf{100}\%^*$ & $\textbf{100}\%^*$ & 65.0\% & 10.0\% & $\textbf{11.5}^{**}$ & 27.5 & 118.5 & 200.0\\
      $\varphi_3$ & 1 & 1 & 1 & 1 & 75.0 & 5.0\% & 20.0\% & \textbf{85.0}\% & 44.0 & 200.0 & 200.0 & \textbf{26.5}\\
      $\varphi_4$ & 1 & 1 & 1 & 1 & 75.0 & 10.0\% & 20.0\% & \textbf{85.0}\% & 67.5 & 200.0 & 200.0 & $\textbf{26.5}^{*}$\\
      $\varphi_5$ & 1 & 1 & 1 & 1 & \textbf{100}\% & \textbf{100}\% & \textbf{100}\% & \textbf{100}\% & \textbf{1.0} & 2.0 & \textbf{1.0} & \textbf{1.0} \\
      $\varphi_6$ & 10 & 10 & 10 & 10 & $\textbf{100}\%^*$ & $\textbf{100}\%^*$ & 70.0\% & 50.0\% & $\textbf{3.5}^{**}$ & $\textbf{3.5}^{**}$ & 160.5 & 119.0\\
      $\varphi_7$ & 5 & 5 & 1 & 1 & 65.0\% & $\textbf{100}\%^{**}$ & 0.0\% & 0.0\% & 125.0 & $\textbf{63.0}^{**}$ & 200.0 & 200.0\\
      $\varphi_8$ & 10 & 10 & 10 & 1 & 80.0\% & \textbf{95.0}\% & 90.0\% & 75.0\% & 72.0 & 52.0 & 83.0 & \textbf{21.0}\\
      $\varphi_9$ & 10 & 10 & 10 & 10 & 95.0\% & $\textbf{100}\%^{**}$ & 15.0\% & 5.0\% & 46.0 & $\textbf{12.0}^{**}$ & 200.0 & 200.0 \\
      \hline
  \end{tabular}
  \caption{The experimental result on $\ATmodel$.}
  \label{tab:ARCH2014}
   \vspace{-5mm}
\end{table}



%\subsection{Evaluation Results}\label{sec:result}
\vspace{4mm}
\noindent \textbf{Evaluation Results}
%\subsubsection{Evaluation Results}
The preliminary results are presented in Table.~\ref{tab:ARCH2014}.
%The algorithms indicated by bold face are our approach and others are baselines.
%%
%
%$\SA$ is simulated annealing and $\CE$ is cross entropy method.
%
%%
The $\Delta_{T}$ columns indicate the best performing $\Delta_{T}$ for each algorithm.
The ``Success rate'' columns indicate the success rate of falsification process.
The ``numEpisode'' columns show the median (among the 20 procedures) of the number of simulation episodes required to falsify the formula.
If the falsification procedure fails, we consider the number of simulation episodes to be the maximum allowed episodes (200).
We use median since the distribution of the number of simulation episodes tends to be skewed.
%Another reason is that our sample size is relatively small so that we need to avoid the effects of outliers.

The best results (success rate and numEpisode) of each formula are highlighted in bold.
If the difference between the best entry of our methods and the best entry of the baseline methods is statistically significant by Fisher's exact test and the Mann Whitney U-test~\cite{corder2014nonparametric}, we mark the best entry with $*$ ($p < 0.05$) or $**$ ($p < 0.001$), respectively.
%\todo{This is again a little misleading.  We do not compare, for example, A3C and DDQN.  Only between the best one among our method and the best one among baselines.}
%To test the success rate, we use Fisher's exact test of independence~\cite{corder2014nonparametric}.
%To test iteration numbers, we use the Mann-Whitney U-test~\cite{corder2014nonparametric}.

As shown in Table~\ref{tab:ARCH2014}, RL based methods almost always outperforms baseline methods on success rate, which means RL based methods are more likely to find the falsified inputs with a limited number of episodes.
This is because RL based methods learn knowledge from the environment and generate input signals adaptively during the simulations.
%On the other hand, the result on iteration numbers is more mixed.
Among the statistically significant results of numEpisode, our methods are best for five cases ($\varphi_1, \varphi_2,\varphi_6,\varphi_7,\varphi_9$), while the baseline methods are best for one case ($\varphi_4$).
%Although, for $\varphi_3$--$\varphi_5$ and $\varphi_8$ the cross entropy method is best in episode numbers, only in one case the result is statistically significant.
For the case of $\varphi_4$, it is likely because that
all variables in this formula take discrete values,
thus, reinforcement learning is less effective.
%the reward in reinforcement learning tends to be constant.
%\todo{Still A3C performs not so bad.  Also what is the reason of good performance of CE?}
Further, DDQN tends to return extreme values as actions,
which are not solutions to falsify $\varphi_3$ and $\varphi_4$.
This explains poor performance of DDQN for the case of $\varphi_3$ and $\varphi_4$.

%% \begin{table}
%% \centering
%% \scriptsize
%% \begin{tabular}{|c|c|c|c|c|c|}
%% \hline
%% Property id \& Formula & Algorithm & sampling rate & falsified? & iteration number\\
%% \hline
%% \hline
%% \multirow{4}{*}{$\varphi_1 \colon \square ((g_2 \wedge \diamond_{[0, 0.1]} g_1) \rightarrow \square_{[0.1, 1.0]} \neg g_2)$}
%%               & $\AAAC$ & 1 & \textbf{90.0\%} & 46.0 \\
%% 							& $\DQN$ & 1 & 5.0\% & 200.0 \\
%%               & $\SA$  & 1 & 20.0\% & 200.0 \\
%% 							& $\CE$  & 1 & 85.0\% & \textbf{26.5} \\

%% \hline
%% \multirow{4}{*}{$\varphi_2 \colon \square ((\neg g_1 \wedge \diamond_{[0, 0.1]} g_1) \rightarrow \square_{[0.1, 1.0]} g_1)$}
%%               & $\AAAC$ & 1 & \textbf{90.0\%} & 63.5 \\
%%               & $\DQN$ & 1 & 5.0\% & 200.0 \\
%% 							& $\SA$ & 1 & 20.0\% & 200.0 \\
%% 							& $\CE$  & 1 & 85.0\% & $\textbf{26.5}^*$ \\
%% \hline
%% \multirow{4}{*}{$\varphi_3 \colon \square \bigwedge_{i=1}^4 ((\neg g_i \wedge \diamond_{[0, 0.1] g_i}) \rightarrow \square_{[0.1, 1.0]} g_i)$}
%%               & $\AAAC$ & 1 & 75.0\% & 73.0 \\
%%               & $\DQN$ & 1 & 10.0\% & 200.0 \\
%% 							& $\SA$ & 1 & 20.0\% & 200.0 \\
%% 							& $\CE$ & 1 & \textbf{85.0\%} & \textbf{26.5} \\
%% \hline
%% \multirow{4}{*}{$\varphi_4 \colon \square (\square_{[0, t_1]} \omega \leq \overline\omega \rightarrow \square_{[t_1, t_2]} v \leq \overline{v})$}
%%               & $\AAAC$ & 10 & $\textbf{100\%}^*$ & $\textbf{2.5}^{**}$ \\
%%               & $\DQN$ & 5 & $\textbf{100\%}^*$ & 4.0 \\
%%               & $\SA$ & 10 & 70.0\% & 160.5  \\
%% 							& $\CE$ & 10 & 50.0\% & 119.0 \\
%% \hline
%% \multirow{4}{*}{$\varphi_5 \colon \square v \leq \overline{v}$}
%%               & $\AAAC$ & 5 & 80.0\% & $\textbf{45.0}^{**}$ \\
%%               & $\DQN$ & 5 & $\textbf{100}\%^{**}$ & 57.5 \\
%%               & $\SA$ & 10 & 0.0\% & 200.0 \\
%%               & $\CE$ & 10 & 0.0\% & 200.0 \\

%% \hline
%% \multirow{4}{*}{$\varphi_{6} \colon \square \diamond_{[0,25]} \neg (\underline{v} \leq v \leq  \overline{v})$}
%%               & $\AAAC$ & 10 & 75.0\% & 52.0 \\
%%               & $\DQN$ & 10 & \textbf{100}\% & 37.5 \\
%% 							& $\SA$ & 10 & 90.0\% & 83.0 \\
%% 							& $\CE$ & 10 & 65.0\% & \textbf{35.5} \\
%% \hline
%% \multirow{4}{*}{$\varphi_{7} \colon \square \neg \square_{[0,20]} (\mathrm{gearLow} \wedge \mathrm{highRPM})$}
%% 							& $\AAAC$ & 10 & 95.0\% & 49.5 \\
%%               & $\DQN$ & 10 & $\textbf{100\%}^{**}$ & $\textbf{12.0}^{**}$ \\
%% 							& $\SA$ & 10 & 15.0\% & 200.0 \\
%% 							& $\CE$ & 10 & 5.0\% & 200.0 \\
%% \hline
%% \end{tabular}
%% \caption{The experimental result on $\ATmodel$.}
%% \label{tab:ARCH2014}
%% \end{table}


%
% \begin{table}
%   \centering
%   \begin{minipage}[t]{.4\textwidth}
%     \centering
%     \scriptsize
%     \begin{tabular}[t]{c||c|c|c|c}
%       id & Algorithm & $\Delta_T$ & falsified? & $\mathsf{numEpisode}$\\
%       \hline
%       \hline
%       \multirow{4}{*}{$\varphi_1$}
%       & $\AAAC$ & 1 & \textbf{90.0\%} & 46.0 \\
%       & $\DQN$ & 1 & 5.0\% & 200.0 \\
%       & $\SA$  & 1 & 20.0\% & 200.0 \\
%       & $\CE$  & 1 & 85.0\% & \textbf{26.5} \\
%       \hline
%       \multirow{4}{*}{$\varphi_2$}
%       & $\AAAC$ & 1 & \textbf{90.0\%} & 63.5 \\
%       & $\DQN$ & 1 & 5.0\% & 200.0 \\
%       & $\SA$ & 1 & 20.0\% & 200.0 \\
%       & $\CE$  & 1 & 85.0\% & $\textbf{26.5}^*$ \\
%       \hline
%       \multirow{4}{*}{$\varphi_3$}
%       & $\AAAC$ & 1 & 75.0\% & 73.0 \\
%       & $\DQN$ & 1 & 10.0\% & 200.0 \\
%       & $\SA$ & 1 & 20.0\% & 200.0 \\
%       & $\CE$ & 1 & \textbf{85.0\%} & \textbf{26.5} \\
%       \hline
%       \multirow{4}{*}{$\varphi_4$}
%       & $\AAAC$ & 10 & $\textbf{100\%}^*$ & $\textbf{2.5}^{**}$ \\
%       & $\DQN$ & 5 & $\textbf{100\%}^*$ & 4.0 \\
%       & $\SA$ & 10 & 70.0\% & 160.5  \\
%       & $\CE$ & 10 & 50.0\% & 119.0 \\
%     \end{tabular}
%   \end{minipage}
%   \begin{minipage}[t]{.4\textwidth}
%     \centering
%     \scriptsize
%     \begin{tabular}[t]{c||c|c|c|c}
%       id & Algorithm & $\Delta_T$ & falsified? & $\mathsf{numEpisode}$\\
%       \hline
%       \hline
%       \multirow{4}{*}{$\varphi_5$}
%       & $\AAAC$ & 5 & 80.0\% & $\textbf{45.0}^{**}$ \\
%       & $\DQN$ & 5 & $\textbf{100}\%^{**}$ & 57.5 \\
%       & $\SA$ & 10 & 0.0\% & 200.0 \\
%       & $\CE$ & 10 & 0.0\% & 200.0 \\
%       \hline
%       \multirow{4}{*}{$\varphi_{6}$}
%       & $\AAAC$ & 10 & 75.0\% & 52.0 \\
%       & $\DQN$ & 10 & \textbf{100}\% & 37.5 \\
%       & $\SA$ & 10 & 90.0\% & 83.0 \\
%       & $\CE$ & 10 & 65.0\% & \textbf{35.5} \\
%       \hline
%       \multirow{4}{*}{$\varphi_{7}$}
%       & $\AAAC$ & 10 & 95.0\% & 49.5 \\
%       & $\DQN$ & 10 & $\textbf{100\%}^{**}$ & $\textbf{12.0}^{**}$ \\
%       & $\SA$ & 10 & 15.0\% & 200.0 \\
%       & $\CE$ & 10 & 5.0\% & 200.0 \\
%     \end{tabular}
%   \end{minipage}
%   \caption{The experimental result on $\ATmodel$.}
%   \label{tab:ARCH2014}
% \end{table}

%\section{Related Work}\label{sec:related_work}

\subsection{Testing methods for hybrid systems}

\cite{kapinski2003systematic,zhao2003generating,dang2009coverage}

Coverage based testing~\cite{esposito2004adaptive,bhatia2004incremental,branicky2006sampling}

Robust simulation trajectory~\cite{Donze:2007:SSU:1760804.1760822,girard2006verification,julius2007robust,Lerda:2008:VSC:1423033.1423059}

\subsection{Model checking of hybrid automata}

Model checking hybrid automata~\cite{Alur:1995:AAH:202379.202381,Henzinger:1995:WDH:225058.225162,henzinger1997hytech}

Monte-Carlo model checking~\cite{grosu2005monte}

Statistical model checking~\cite{Younes:2006:SPM:1182767.1182770,Clarke:2009:SMC:1533832.1533850,clarke2011statistical,Zuliani:2010:BSM:1755952.1755987}

\subsection{Robustness guided falsification of CPSs}

There have been works that are proposed to conduct verification on Cyber-Physical Systems (CPS).
Plaku et al.~\cite{Plaku:2009:FLS:1532891.1532932}...
Houssam~\cite{TLF_CPS_2013} proposed to find counterexamples in CPS through global optimization of robustness metrics using Monte-Carlo technique that conduct a random walk on over the space of inputs. They also build a Matlab toolbox, S-TaliRo~\cite{S-TaliRo}, which enables searches for trajectories of minimal robustness in Simulink/Stateflow diagrams.

\subsection{Controller synthesis}
%do we really need this catagory of related work? We may not adopt examples all from controller systems I suppose?
?

In this work, we propose sub-character tokenization and conduct comprehensive experiments to illustrate its advantage over existing tokenization methods. 
%
% The idea of SubChar tokenization is simple: we encode every Chinese character into a short sequence of symbols and then construct the vocabulary on the encoded sequences with sub-word tokenization. 
%
Compared to treating each individual character as a token (CharTokenizer) or directly running sub-word tokenization on the raw Chinese text (sub-word tokenizer), our SubChar tokenizers not only perform competitively on downstream NLU tasks, more importantly, they can be much more efficient and robust.
%
We conduct a series of ablation and analysis to understand the reasons why SubChar tokenizers are more efficient, as well as the impact of linguistic and non-linguistic encoding.
%
Given the advantages of our SubChar tokenizers, we believe that they are better alternatives to all existing Chinese tokenizers, especially in applications where efficiency and robustness are critical. 
%
% We will release well-documented code, tokenizers, and pretrained models for easy usage. 
%
It is possible that our approach can be useful for other morphologically poor languages and more complicated methods could be developed based on SubChar tokenization for even better performance. We leave these interesting directions for future exploration.
%
On a broader level, our work makes an important attempt in developing more tailored methods for a language drastically different from English with promising results. We believe that this is a crucial future direction for the community given the language diversity in the world.
%
We hope that our work can inspire more such work in order to benefit language technology users from different countries and cultures.


% In this paper, we have explored three linguistically informed tokenization methods motivated by the unique linguistic characteristics of the Chinese writing system. Specifically, we find that pronunciation-based and glyph-based tokenizers can match or outperform the conventional Chinese tokenizers and Chinese word segmentation is not a useful addition for the tokenizer.
% Moreover, we find that our glyph-based tokenizers achieve large gains on noisy input as compared to the baselines, while our pronunciation-based tokenizers obtain limited success. This highlights the potential advantage of our proposed methods in real-life scenarios with noisy data.
% We believe that our work sets an important example of exploiting the unique linguistic property of a language beyond English to develop more tailored techniques, which should be an important direction for the global NLP community. 



\bibliographystyle{abbrv}
\bibliography{ref}

\begin{appendix}
\newpage
\section{Dataset Visualizations}
\label{sec:app_dataset_visuals}

%%%%%%
%%
%%
\subsection{Examples of each view class}
\newcommand{\BC}{0.33}
\setlength{\tabcolsep}{0.1cm}
\begin{figure}[!h]
\begin{tabular}{c c c c}
    PLAX  & PSAX & OTHER 
    \\
    \includegraphics[width=\BC\textwidth]{figures/small_appendix/Appendix_PLAX1.jpg}
    &
    \includegraphics[width=\BC\textwidth]{figures/small_appendix/Appendix_PSAX1.jpg}
    &
    \includegraphics[width=\BC\textwidth]{figures/small_appendix/Appendix_Other1.jpg}
    &
   
    \\
    
    \includegraphics[width=\BC\textwidth]{figures/small_appendix/Appendix_PLAX2.jpg}
    &
    \includegraphics[width=\BC\textwidth]{figures/small_appendix/Appendix_PSAX2.jpg}
    &
    \includegraphics[width=\BC\textwidth]{figures/small_appendix/Appendix_Other2.jpg}
    &
   
     \\
     
     \includegraphics[width=\BC\textwidth]{figures/small_appendix/Appendix_PLAX3.jpg}
    &
    \includegraphics[width=\BC\textwidth]{figures/small_appendix/Appendix_PSAX3.jpg}
    &
    \includegraphics[width=\BC\textwidth]{figures/small_appendix/Appendix_Other3.jpg}
    &
   
     \\
     
     \includegraphics[width=\BC\textwidth]{figures/small_appendix/Appendix_PLAX4.jpg}
    &
    \includegraphics[width=\BC\textwidth]{figures/small_appendix/Appendix_PSAX4.jpg}
    &
    \includegraphics[width=\BC\textwidth]{figures/small_appendix/Appendix_Other4.jpg}
    &
   
    \end{tabular}	
    \caption{Examples of images for each possible view label in our dataset. \emph{From left to right:} Four examples of peristernal long axis (PLAX) view, four examples of peristernal short axis (PSAX) view, and four examples of other kinds of view in our ``Other'' class. }
    \label{fig:VIEW_SAMPLES_APPENDIX}
\end{figure}

%%%%%%
%%
%%
\newpage
\subsection{Examples of each view for a Severe AS patient}
\newcommand{\BA}{0.33}
\setlength{\tabcolsep}{0.1cm}
\begin{figure}[!h]
\begin{tabular}{c c c c}
    PLAX  & PSAX & OTHER 
    \\
    \includegraphics[width=\BA\textwidth]{figures/small_appendix/SevereAS_11112007_PLAX1.jpg}
    &
    \includegraphics[width=\BA\textwidth]{figures/small_appendix/SevereAS_11112007_PSAX1.jpg}
    &
    \includegraphics[width=\BA\textwidth]{figures/small_appendix/SevereAS_11112007_Other1.jpg}
    &
    
    \\
    
    \includegraphics[width=\BA\textwidth]{figures/small_appendix/SevereAS_11112007_PLAX2.jpg}
    &
    \includegraphics[width=\BA\textwidth]{figures/small_appendix/SevereAS_11112007_PSAX2.jpg}
    &
    \includegraphics[width=\BA\textwidth]{figures/small_appendix/SevereAS_11112007_Other2.jpg}
    &
   
     \\
     
     \includegraphics[width=\BA\textwidth]{figures/small_appendix/SevereAS_11112007_PLAX3.jpg}
    &
    \includegraphics[width=\BA\textwidth]{figures/small_appendix/SevereAS_11112007_PSAX3.jpg}
    &
    \includegraphics[width=\BA\textwidth]{figures/small_appendix/SevereAS_11112007_Other3.jpg}
    &
  
    \end{tabular}	
    \caption{Examples of images from a patient with Severe AS in our dataset. \emph{From left to right:} Three examples of parasternal long axis (PLAX) view, three examples of parasternal short axis (PSAX) view, and three examples of other kinds of view in our ``Other'' class. }
    \label{fig:PatientSevereAS}
\end{figure}


%%%%%%
%%
%%
\newpage
\subsection{Examples of each view for a No AS patient}
\newcommand{\BB}{0.33}
\setlength{\tabcolsep}{0.1cm}
\begin{figure}[!h]
\begin{tabular}{c c c c}
    PLAX  & PSAX & OTHER 
    \\
    \includegraphics[width=\BB\textwidth]{figures/small_appendix/NoAS_1996889_PLAX1.jpg}
    &
    \includegraphics[width=\BB\textwidth]{figures/small_appendix/NoAS_1996889_PSAX1.jpg}
    &
    \includegraphics[width=\BB\textwidth]{figures/small_appendix/NoAS_1996889_Other1.jpg}
    &
    
    \\
    
    \includegraphics[width=\BB\textwidth]{figures/small_appendix/NoAS_1996889_PLAX2.jpg}
    &
    \includegraphics[width=\BB\textwidth]{figures/small_appendix/NoAS_1996889_PSAX2.jpg}
    &
    \includegraphics[width=\BB\textwidth]{figures/small_appendix/NoAS_1996889_Other2.jpg}
    &
   
     \\
     
     \includegraphics[width=\BB\textwidth]{figures/small_appendix/NoAS_1996889_PLAX3.jpg}
    &
    \includegraphics[width=\BB\textwidth]{figures/small_appendix/NoAS_1996889_PSAX3.jpg}
    &
    \includegraphics[width=\BB\textwidth]{figures/small_appendix/NoAS_1996889_Other3.jpg}
    &
  
    \end{tabular}	
    \caption{Examples of images from a patient with No AS in our dataset. \emph{From left to right:} Three examples of parasternal long axis (PLAX) view, three examples of parasternal short axis (PSAX) view, and three examples of other kinds of view in our ``Other'' class. }
    \label{fig:PatientNoAS}
\end{figure}



\newpage 
\section{Further Results}

\subsection{Assessment of ensembling}

Table~\ref{tab:best_single_checkpoint_VS_ensemble_FS_echo260} compares using a single checkpoint (one point estimate of neural network weight vector $\theta$) to using an ensemble of parameters aggregated from the last 25 checkpoints (one per epoch).

\begin{table}[!h]
    \centering
    \begin{tabular}{c|cccc|c}
    \textit{Diagnosis classification} & Split 1  & Split 2 & Split 3 & Split 4 & Average\\
    \hline
    Best single checkpoint  & 61.81 & 59.79 & 56.05 & 64.21 & 60.46\\
    Ensemble  & 62.95 & 61.03 & 56.58 & 63.84 & \textbf{61.13}
	\\ \hline
    \textit{View classification}  &   &  &  &  & 
    \\ \hline
    Best single checkpoint  & 93.03 & 93.24 & 92.39 & 93.79 & 93.11\\
    Ensemble  & 92.37 & 93.24 & 93.72 & 93.87 & \textbf{93.30}\\
    \end{tabular}
    \caption{Comparing best single checkpoint performance with ensemble performance on \textbf{Full-size \datasetName-156-52}}
    \label{tab:best_single_checkpoint_VS_ensemble_FS_echo260}
\end{table}


%%%%%%
%%
%%
\subsection{Patient-level diagnosis performance on bonus heldout set}

Table~\ref{tab:diagnosis classification patient unlabeled_heldout_174} examines the performance of the best labeled-set-only methods and MixMatch methods on the 174 patient studies that have diagnosis but no view labels.
 While the images used here were originally included in the unlabeled training set (which was used to train SSL methods like MixMatch), the diagnosis labels were not provided at all during training time. 
 We thus still believe this is an authentic test of generalization given the scarcity of labeled data available for our task.
 Of course, additional independent evaluation (especially from another institution) is needed.

\begin{table}[!h]
    \centering
    \begin{tabular}{l l l|rrrr|c}
    Pretrain & Method & Voting
    & Split 1  & Split 2 & Split 3 & Split 4 & average\\
    \hline
    & Basic WRN & Simple average & 76.73 & 75.25 & 76.87 & 81.88 & 77.68\\
    & Basic WRN & View-prioritized & 73.63 & 83.21 & 79.70 & 80.08 & 79.18\\
    %SSL & FS & MixMatch & Priority view + confidence & 94.58 & 84.17 & 77.50 & 92.5 & 87.19\\
    \hline
    & MixMatch & Simple average & 85.32 & 76.29 & 74.14 & 79.95 & 78.93\\
    view & MixMatch & Simple average & 83.36 & 77.96 & 75.61 & 81.37 & 79.58\\
    & MixMatch & View-prioritized & 83.27 & 83.76 & 82.34 & 82.83 & \textbf{83.05}\\
    view & MixMatch & View-prioritized & 82.53 & 86.15 & 79.62 & 83.27 & 82.89\\
    %view & MixMatch & LR with view-priority & 80.42 & 84.24 & 76.58 & 80.67 & 80.48\\
    %(MixMatch transfered) + MysteryMethod & NA & NA & NA\\ 
    \end{tabular}
    \caption{Patient-level AS Severity Diagnosis Classification on the \textbf{bonus heldout set} of 174 patients for whom we have diagnosis labels only (no view labels). We show balanced accuracy on models trained on each of the four folds on four \textbf{full-size \datasetName-156-52} dataset.
    }%endcaption
    \label{tab:diagnosis classification patient unlabeled_heldout_174}
\end{table}


%%%%%%
%%
%%
\subsection{Assessment of MixMatch hyperparameter sensitivity}

In Table~\ref{tab:MixMatch hyperparameters ablation study}, we consider four possible strategies for setting the hyperparameters of MixMatch, varying two  key settings for the weight on unlabeled loss $\lambda$. First, we vary whether the final value of $\lambda$ is set to its \emph{best} value among a grid of candidates (based on validation set performance), or \emph{fixed} to a constant.
Second, we vary whether $\lambda$ remains fixed over iterations throughout a training run, or is updated over iterations on a linear ramp schedule from 0 to its final target value. 

From this comparison, we see we consistent gains across splits (average gain across splits of over 1.6\% balanced accuracy) for using a delayed ramp up schedule with target value selected via grid search.

\begin{table}[!h]
    \centering
    \begin{tabular}{l l| rrrr | r}
    Final $\lambda$ value & $\lambda$ update schedule & Split 1  & Split 2 & Split 3 & Split 4 & Average\\
    \hline
    best on val & Delayed ramp-up  & 65.57 & 62.69 & 60.87 & 66.29 & 63.86\\
    best on val & Immediate ramp-up & 65.07 & 61.87 & 60.82 & 65.37 & 63.28\\
    best on val & Constant  & 65.03 & 61.52 & 58.87 & 65.22 & 62.66\\
    100 (fixed) & Constant & 63.94 & 61.79 & 58.87 & 64.35 & 62.24\\
    \end{tabular}
    \caption{Ablation study of different settings of the unlabeled loss weight $\lambda$ for MixMatch. AS severity diagnosis classification for individual images on the \textbf{full-size \datasetName-156-52} dataset. showing balanced accuracy averaged over the test sets from multiple folds (each fold’s test set contains all images from 52 patients). }%endcaption
    \label{tab:MixMatch hyperparameters ablation study}
\end{table}



%%%%%%
%%
%%
\subsection{Assessment of alternative view prioritization strategy using thresholding}


An anonymous reviewer suggested an alternative strategy for prioritizing images of relevant view.
The alternative strategy works as follows: for each image, we compute the predicted probability that the image is a ``relevant view'' (either PLAX and PSAX) by summing the probabilities of each view type.
However, instead of using this raw probability as a weight (as our chosen method does), we use a \emph{cutoff threshold} and simply average the diagnosis predictions of images whose relevant view probability is above the cutoff.
For each patient, we use the majority vote prediction of the diagnosis from the images of relevant views.
The value of the cutoff threshold is selected using the validation set to maximize balanced accuracy.

Table~\ref{tab:Suggested_Aggregation_Ablation} shows the performance of this strategy (``threshold-then-average'') on the full-size dataset.
Using this alternative prioritization strategy together with our suggested methodology for patient-level diagnosis (using MixMatch, pretraining on view), we find the average test set balanced accuracy is around 85.8\%, while the weighted average strategy in the main paper achieves over 90\% balanced accuracy. We take this as reasonably decisive evidence that a weighted average (rather than a simple cutoff) should be preferred.

\begin{table}[!h]
    \centering
    \begin{tabular}{l l l|rrrr|c}
    Pretrain & Method & Aggregation across images
    & Split 1  & Split 2 & Split 3 & Split 4 & average\\
    \hline
    & Basic WRN & Threshold-then-Average & 85.42 & 86.25 & 79.17 & 92.50 & 85.84 \\
    %SSL & FS & MixMatch & Priority view + confidence & 94.58 & 84.17 & 77.50 & 92.5 & 87.19\\
    & MixMatch & Threshold-then-Average & 83.33 & 84.17 & 77.50 & 94.58 & 84.90 \\
    view & MixMatch & Threshold-then-Averagen & 86.67 & 80.00 & 82.50 & 94.17 & 85.84\\
    %view & MixMatch & LR with view-priority & 87.08 & 82.08 & 85.00 & 88.75 & 85.73\\
    %(MixMatch transfered) + MysteryMethod & NA & NA & NA\\ 
    \end{tabular}
    \caption{Alternative view-prioritizing strategy for patient-level AS severity diagnosis classification on the \textbf{full-size \datasetName-156-52} dataset, showing balanced accuracy on the test set across multiple folds (each fold’s test set contains 52 patients).}
    %endcaption
    \label{tab:Suggested_Aggregation_Ablation}
\end{table}



%%%%%%
%%
%%
\subsection{ROC Curve of patient-level diagnosis: no AS vs. mild/moderate/severe AS}

Fig.~\ref{fig: No AS vs Some AS} shows receiver operating curves for several methods for the task of distinguishing no AS vs Some AS (which aggregates both the mild/moderate and severe levels in the 3-level diagnosis task of the main paper).

\begin{figure}[!h]
\begin{tabular}{c c}
	\includegraphics[width=0.43\textwidth]{figures/fold0_multitask_PatientLevel_NoVSSome_NormalizedPriorityStrategyClassProbabilityScore.pdf}
	&
    \includegraphics[width=0.43\textwidth]{figures/fold1_multitask_PatientLevel_NoVSSome_NormalizedPriorityStrategyClassProbabilityScore.pdf}
	\\
	(a) Split 1 & (b) Split 2
	\\
	\includegraphics[width=0.43\textwidth]{figures/fold2_multitask_PatientLevel_NoVSSome_NormalizedPriorityStrategyClassProbabilityScore.pdf}
	&
    \includegraphics[width=0.43\textwidth]{figures/fold3_multitask_PatientLevel_NoVSSome_NormalizedPriorityStrategyClassProbabilityScore.pdf}
	\\
	(c) Split 3 & (d) Split 4
\end{tabular}
    
\caption{ROC curves for binary diagnosis task (no AS vs ``mild/moderate/severe AS'') on \textbf{full-size \datasetName-156-52}.
    }%endcaption
    \label{fig: No AS vs Some AS}
\end{figure}

\section{Methodological Details}

\subsection{Image processing details}
\label{sec:removing_doppler}

\paragraph{Removing doppler images.}
In the raw data of all imagery available for an echocardiogram study, 
we obtained TIFF files that represent both cineloops and Doppler images.

We verified in our labeled set that all Doppler images have one of the following landscape aspect ratio $(831, 323)$, $(901, 384)$, $(901, 390)$, $(704, 305)$, $(831, 421)$, $(901, 469)$ or $(563, 294)$. Only the Dopplers have these aspect ratios. We thus filtered out Doppler completely via these aspect ratios. 

\paragraph{Downsizing}
The original images are provided as high-resolution TIFF format images (hundreds of pixels per side) of varying aspect ratios. Generally, we can expect that both view and diagnosis classifiers would perform better given higher-resolution input (and holding other factors the same). The main trade-off of processing higher-resolution images is increased runtime and memory requirements. In our preliminary experiments, we compared downsizing all images to a standard square aspect ratio at 3 possible sizes: 32x32, 64x64 and 128x128. We found that 64x64 achieves a good balance between model performance and computation cost. 
A prior study by \citet{madaniDeepEchocardiographyDataefficient2018} provides a more extensive study of optimal resolution size. The interested reader can refer to their work for more details. 


\subsection{Architecture Settings and Hyperparameters}
\label{sec:arch_and_hyperparameters}

\paragraph{Weighted cross-entropy for labeled loss}
To counteract the effect of class imbalance in the dataset, we use weighted cross-entropy for the labeled loss. For an input image $x$ whose true label $y$ indicates it belongs to class $c$, the weighted cross-entropy assumes the following form:
\begin{align}
\mathcal{L}^L(\theta, x) = - w_{c} \log \hat{p}_{c}(\theta, x),
\end{align}
where $\hat{p}_{c}$ is the predicted probability of class $c$. The weight $w_{c}$ is calculated using the training set statistics as follow:
\begin{align}
w_{c} = \frac{\prod_{k\neq c}{N_{k}}}{\sum_{j}\prod_{k \neq j}{N_{k}}}
\end{align}
where $N_{k}$ is the number of images of class $k$ in the training set.

\paragraph{Common architecture.}
Following~\citet{oliverRealisticEvaluationDeep2018}, for all considered methods, we use the \emph{same} backbone neural network architecture: a wide residual network~\citep{zagoruykoWideResidualNetworks2017} with 28 layers (WRN-28), which has total of 5,931,683 parameters.
This same network architecture is used in the original MixMatch evaluation~\citep{berthelotMixmatchHolisticApproach2019} with promising results.

\paragraph{Common training protocol.}
All SSL methods we consider follow the loss minimization framework with two primary losses (one for ``labeled'' data and one for ``unlabeled'' data) in Eq.~\eqref{eq:standard-SSL-loss-template}.
We allow every method to train for 32 epochs (where each epoch processes $2^{16}$ images, as in \citet{berthelotMixmatchHolisticApproach2019}).
Our preliminary experiments suggest that after 30 epochs all methods effectively converge in terms of validation balanced accuracy. 

\paragraph{Common regularization.}
For all methods, we expect performance will be vulnerable to overfitting, so we impose an L2-norm penalty on the weights $\theta$, also known as weight decay. Each method selects its preferred value of this penalty strength hyperparameter. We searched values in [0.0002, 0.002, 0.02].

\paragraph{Common optimization.}
We use ADAM \citep{kingma2014adam} to optimize each model.
Each method selects the value of the step size (learning rate) as a hyperparameter. We experimented with 0.002 and 0.0007
%HZ: 'performance being sensitive to learning rate' is very reasonable. But we don't have an ablation to back it. 
%We find performance is sensitive to the step size (learning rate) hyperparameter, so we perform a grid search and select the value that maximizes balanced accuracy on the validation set.

\paragraph{Hyperparameters for Pseudo-Label.}
Beyond the usual hyperparameters for our loss-minimization SSL framework, another important hyperparameter for pseudo-label is the threshold $\tau$. We find that performance is not very sensitive to the chosen $\tau$ value as long as it is within a certain range. We set $\tau$ to 0.95, as done in past literature that evaluates Pseudo-Label as an SSL method ~\citep{oliverRealisticEvaluationDeep2018,berthelotMixmatchHolisticApproach2019, berthelotRemixmatchSemisupervisedLearning2019, sohnFixmatchSimplifyingSemisupervised2020}.


\paragraph{Hyperparameters for VAT.}
Beyond the usual hyperparameters for our SSL framework, for VAT we need to select a value for $\epsilon$.
In \citet{miyatoVirtualAdversarialTraining2019}, the authors claimed that they can achieve superior performance by tuning only $\epsilon$ and fixing $\lambda$ to 1. In our experiment, we used the default $\lambda$ as in \cite{berthelotMixmatchHolisticApproach2019} and searched the value of $\epsilon$ in [2, 6, 18], together with learning rate and weight decay. We select the best hyperparameters using validation set performance. 


\paragraph{Hyperparameters for MixMatch.}
Beyond the usual hyperparameters for our SSL framework, the key hyperparameters for MixMatch include the number of augmentations $K$, the temperature $T>0$ used for sharpening, interpolation hyperparameter $\alpha$ and unlabeled loss coefficient $\lambda$. We set $K=2$, $T=0.5$, and $\alpha=0.75$ as done in \citet{berthelotMixmatchHolisticApproach2019}, and search for $\lambda$ in the range [10, 30, 75, 100, 130] using validation set. 

\paragraph{Hyperparameters for Multitask training.}
We searched $\gamma$, the hyperparameter that control the strength of the auxilliary view loss in Eq.~\eqref{eq:multitask}, in the range [10, 3, 1, 0.3, 0.1]. The best $\alpha$ is selected together with other hyperparameters on validation set. 

\end{appendix}
\end{document}
