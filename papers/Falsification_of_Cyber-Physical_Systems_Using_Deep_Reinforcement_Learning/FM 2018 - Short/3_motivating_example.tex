%!TEX root = 0_main.tex
\section{Motivating Example}\label{sec:example}

(... Explanation of an ARCH2014 example ...)

\begin{definition}[System]
  An \emph{(finite) input trace} $\mathbf{x} = x_0, x_1, \ldots, x_n$ is a sequence of elements of the \emph{input space} $\mathbb R^N$ together with the sequence of sampling time instants $\mathbf{t} = t_0, t_1, \ldots, t_n, \ldots$.
  $\mathbf{t}$ is often omitted to be mentioned.

  An \emph{(finite) output trace} $\mathbf{y} = y_0, y_1, \ldots, y_n$ is a sequence of elements of the \emph{system states} $\mathbb R^M$.
  If an (input/output) trace $\mathbf{z}_1$ and its sampling time instants $\mathbf{t}_1$ are prefixes of $\mathbf{z}_2$ and $\mathbf{t}_2$ respectively, the trace $\mathbf{z}_1$ is a prefix of $\mathbf{z}_2$.

  A \emph{(reactive) system} $\mathbf{f}$ is a map from finite input traces $x_1, \ldots, x_{n-1}$ to $y_1, \ldots, y_n$ such that if $\mathbf{x}_1$ is a prefix of $\mathbf{x}_2$, $\mathbf{f}(\mathbf{x}_1)$ is a prefix of $\mathbf{f}(\mathbf{x}_2)$.
\end{definition}

Because of the definition, $\mathbf{f}$ induces a map from \emph{infinite} traces $\mathbf{x}$ to $\mathbf{y}$.
We use the same notation $\mathbf{f}$ for this map.

(... Intuitive explanation and references ...)
\begin{definition}[Metric Temporal Logic (MTL) formula]
  A \emph{metric temporal logic (MTL)} formula $\phi$ is defined by a BNF as follows.
  \begin{equation}
    \phi ::= p \ \mid\  \phi \wedge \phi \ \mid\ \phi \vee \phi \ \mid \ \neg \phi \ \mid\  \square_I \phi \ \mid\ \diamond_I \phi \ \mid\ \phi \ \mathcal{U}_I \psi \ \mid\ \phi \ \mathcal{S}_I \psi \ \mid\ X\phi \ \mid\ P\phi
  \end{equation}
  where $p$ is an atomic formula and $I$ is any interval on $\mathbb R$.
  We omit $I$ if $I = [0, \infty]$.
  For $\mathcal{U}$ and $\mathcal{S}$, we assume $I$ is non-negative.
\end{definition}

\begin{definition}
  Let $\phi$ be a MTL-formula.
  Let $\mathbf{t} = t_0, t_1, \ldots, t_n, \ldots$ be an infinite sequence of sampling time of the system states.
  Let $\mathbf{y} = y_0, y_1, \ldots, y_n, \ldots$ be system states of time instants $\mathbf{t}$ respectively.
  The relation $\mathbf{y}, n \models \phi$ (read that $\phi$ holds at $n$ on the trace $\mathbf{y}$) is defined recursively on $\phi$ as follows.
  \begin{align}
    \mathbf{y}, n \models p &\iff p(y_n)\\
    \mathbf{y}, n \models \phi_1 \wedge \phi_2 &\iff \mathbf{y}, n \models \phi_1 \text{ and } \mathbf{y}, n \models \phi_2\\
    \mathbf{y}, n \models \phi_1 \vee \phi_2 &\iff \mathbf{y}, n \models \phi_1 \text{ or } \mathbf{y}, n \models \phi_2\\
    \mathbf{y}, n \models \neg \phi &\iff \neg (\mathbf{y}, n \models \phi)\\
    \mathbf{y}, n \models \square_I \phi &\iff \forall n'\text{ such that } t_{n'} - t_n \in I, \mathbf{y}, n' \models \phi\\
    \mathbf{y}, n \models \square_I \phi &\iff \exists n' \text{ such that } t_{n'} - t_n \in I, \mathbf{y}, n' \models \phi\\
    \mathbf{y}, n \models \phi \ \mathcal{U}_I \psi &\iff
    \begin{gathered}
      \exists n', t_{n'} - t_n \in I \text{ such that } \mathbf{y}, n' \models \psi \text{ and }\\
       n \leq \forall n'' < n', \mathbf{y}, n'' \models \phi
    \end{gathered}\\
    \mathbf{y}, n \models \phi \ \mathcal{S}_I \psi &\iff
    \begin{gathered}
      \exists n', t_n - t_{n'} \in I \text{ such that } \mathbf{y}, n' \models \psi \text{ and }\\
      n' < \forall n'' \leq n, \mathbf{y}, n'' \models \phi
    \end{gathered}\\
    \mathbf{y}, n \models X\phi &\iff \mathbf{y}, n+1 \models \phi\\
    \mathbf{y}, n \models P\phi &\iff n \geq 1 \text{ and } \mathbf{y}, n-1 \models \phi
  \end{align}
\end{definition}

(... Intuitive explanation and references ...)
\begin{definition}[Robustness]
  Let $\phi$ be a MTL-formula.
  The robustness function $\rob(\mathbf{y}, n, \phi)$ over infinite traces $\mathbf{y} = y_0, y_1, \ldots, y_n, \ldots$ is defined as follows.
  \begin{align}
    \rob(\mathbf{y}, n, p) &= \min \{ \dist(y, y_n) \mid \neg p(y) \}\\
    \rob(\mathbf{y}, n, \phi_1 \wedge \phi_2) &= \min(\rob(\mathbf{y}, n, \phi_1), \rob(\mathbf{y}, n, \phi_2))\\
    \rob(\mathbf{y}, n, \neg \phi) &= - \rob(\mathbf{y}, n, \phi)\\
    \rob(\mathbf{y}, n, \phi_1 \vee \phi_2) &= \max(\rob(\mathbf{y}, n, \phi_1), \rob(\mathbf{y}, n, \phi_2))\\
    \rob(\mathbf{y}, n, \square_I \phi) &= \min \{ \rob(\mathbf{y}, n', \phi) \mid t_{n'} - t_n \in I \}\\
    \rob(\mathbf{y}, n, \diamond_I \phi) &= \max \{ \rob(\mathbf{y}, n', \phi) \mid t_{n'} - t_n \in I \}\\
    \rob(\mathbf{y}, n, \phi \ \mathcal{U}_I \psi) &= \max_{n' \text{ s.t. } t_{n'} - t_n \in I} \min(\rob(\mathbf{y}, n', \psi), \min_{n'' = n}^{n'-1} \rob(\mathbf{y}, n'', \phi))\\
    \rob(\mathbf{y}, n, \phi \ \mathcal{S}_I \psi) &= \max_{n' \text{ s.t. } t_{n} - t_n' \in I} \min(\rob(\mathbf{y}, n', \psi), \min_{n'' = n'+1}^{n} \rob(\mathbf{y}, n'', \phi))\\
    \rob(\mathbf{y}, n, X \phi) &= \rob(\mathbf{y}, n+1, \phi)\\
    \rob(\mathbf{y}, n, P \phi) &=
    \begin{cases}
        \rob(\mathbf{y}, n-1, \phi) & \text{ if } n \geq 1\\
        -\infty & n = 0
    \end{cases}
  \end{align}
  where $\dist$ is a distance between two states.
  For the empty set $\emptyset$, $\min \emptyset = \infty$ and $\max \emptyset = -\infty$.
\end{definition}

\begin{definition}[Horizon]
  For each MTL-formula, we assign the \emph{horizon} $\hrz(\phi)$.
  \begin{align}
    \hrz(p) &= 0\\
    \hrz(\phi \wedge \psi) = \hrz(\phi \vee \psi) &= \max(\hrz(\phi), \hrz(\psi))\\
    \hrz(\neg \phi) &= \hrz(\phi)\\
    \hrz(\square_I \phi) = \hrz(\diamond_I \phi) &= \hrz(\phi) + \sup I\\
    \hrz(\phi \ \mathcal{U}_I \psi) &= \max(\hrz(\phi) + \sup I, \hrz(\psi) + \sup I)\\
    \hrz(\phi \ \mathcal{S}_I \psi) &= \max(\hrz(\phi), \hrz(\psi))\\
    \hrz(X\phi) &= \hrz(\phi) + 1\\
    \hrz(P\phi) &= \min(0, \hrz(\phi) - 1)
  \end{align}
  If $\hrz(\phi) = 0$, we call $\phi$ \emph{past dependendt}.
\end{definition}

\begin{lemma}[Robustness of a past dependent formula]\label{lem:rob-past}
  Let $\mathbf{y}$ be a finite trace of system states $y_0, \ldots, y_n$.
  Let $\overline{\mathbf{y}}_1$ and $\overline{\mathbf{y}}_2$ be two infinite extensions of $\mathbf{y}$.
  If $\phi$ is past dependent,
  \begin{equation}
    \rob(\overline{\mathbf{y}}_1, t, \phi) = \rob(\overline{\mathbf{y}}_2, t, \phi)
  \end{equation}
\end{lemma}

By Lemma \ref{lem:rob-past}, robustness of a past dependent formula at instant $n$ is completely determined by $y_0, \ldots, y_n$.
Therefore, we use the notation $\rob(\mathbf{y}, t, \phi)$ for robustness of a past-dependent formula $\phi$ on a finite trace $\mathbf{y}$.
