\begin{figure}
    \centering
    % \includegraphics[draft]{}
    \includegraphics[width=0.482\textwidth]{../pdfs/Task-Map-Shift}
    \vspace{-2em}
    \caption{\footnotesize 
    Tasks’ prompt uncertainty shifts before and after training with four analogy tasks. We visualize all the tasks on the task map with two models, \textbf{M0 (Blue)} and \textbf{M1 (Orange)}. \textbf{M0} is the same instruction-tuned model as in \autoref{fig:task-map}, which does not train on any \textit{analogy tasks}. \textbf{M1} is \textbf{M0}, further trained with four analogy tasks: \textit{task1159, task1154, task1152, task1155}.Additionally, we measure the prediction probability and prompt uncertainty of 620 irrelevant tasks and four unseen analogy tasks, \textit{task1157, task1156, task1158, task1153}, using both \textbf{M0} and \textbf{M1}, plotted in orange and blue. It can be seen that after training the model with analogy tasks (from \textbf{M0} to \textbf{M1}), the prompt uncertainty of the four unseen analogy tasks consistently decreases, while the distribution of other irrelevant tasks remains relatively unchanged.\looseness=-1
    }
    \vspace{-1em}
    \label{fig:task-map-shift}
\end{figure}