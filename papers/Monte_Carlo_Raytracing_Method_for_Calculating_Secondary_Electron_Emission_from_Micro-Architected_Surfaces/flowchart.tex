\documentclass{article}

\usepackage{tikz}
\usepackage{tikz-qtree}
\usepackage{tkz-berge}
\usetikzlibrary{intersections}
\usepackage{tkz-euclide}
\usetkzobj{all}
\usepackage{sansmath}
\usepackage{bm}
\usetikzlibrary{shadings,intersections}
\usetikzlibrary{shapes.geometric, arrows.meta}
\usepackage{tikz-3dplot}
\usetikzlibrary{calc,3d,decorations.markings,backgrounds,positioning,intersections,shapes}

\begin{document}

\def\radius{1.mm} 

\begin{figure}[h]
\tikzstyle{startstop} = [rectangle, rounded corners, minimum width=3cm, minimum height=1.5cm,text centered, font=\sffamily, draw=black, fill=red!30]
\tikzstyle{input} = [trapezium, trapezium left angle=70, trapezium right angle=110, minimum width=1cm, minimum height=1.2cm, text centered, font=\sffamily, draw=black, fill=blue!20]
\tikzstyle{process} = [rectangle, minimum width=5cm, minimum height=6cm, text centered, font=\sffamily, draw=black, fill=orange!30]
\tikzstyle{decision} = [diamond, minimum width=3cm, minimum height=1.5cm, text centered, font=\sffamily, draw=black, fill=green!30]
\tikzstyle{arrow} = [thick,->,>=stealth]
\centering
\begin{tikzpicture}[scale=0.75, node distance=1.5cm,
    every node/.style={fill=white, font=\sffamily, transform shape}, align=center]
\node (kernel)	at (0,0)  [process] {\large Calculation Kernel \\ \\ \\ \\ \large MT Intersection Algorithm \\ \large {$\downarrow$}\\ \large {$\alpha$} \\ \\ \large Source Sampling Functions \\ \large {$\gamma(N), E$}};
\node (primary)	[input, above of=kernel, yshift=4cm]    {\large Generate Primary Rays \\ \large {$(x_{p},y_{p},z_{p}), E_{p}$}};   
\node (fixed)	[above of=primary, xshift=-0.75cm,yshift=2cm] {\large fixed \\ \large direction};
\node (random)	[above of=primary, xshift=0.75cm,yshift=2cm] {\large random \\ \large orientation};                
\node (mesh)	[process, left of=kernel, xshift=-5.25cm]    {\large Geometry \\ \\ \\ \\ \large \bm{$V_{0}, V_{1}, V_{2}$} \\ \large \bm{$n$} \\ \\};
\node (decision_0)	[decision, below of=kernel, yshift=-3cm]    {\large {$i \leq N$}};  
\node (ray_i)	[startstop, below of=decision_0, yshift=-1.5cm]    {\large Ray 1 \\ \large Ray 2 \\ \large \bm{$\vdots$} \\ \large Ray i};    
\node (decision_1)	[decision, right of=kernel, xshift=3cm, yshift=1.75cm]    {\large {$z_{i} > z_{max}$}};    
\node (decision_2)	[decision, right of=kernel, xshift=3cm, yshift=-1.75cm]    {\large {$E_{i} > E_{c}$}};    
\node (yield)	[startstop, right of=decision_1, xshift=2.5cm]    {\large $\gamma_{eff}$};
\node (terminate)	[startstop, right of=decision_2, xshift=2.5cm]    {\large Terminate \\ \large Ray i};

\draw [arrow] (primary) -- (kernel);
\draw [arrow] (fixed) -- (fixed.south|-primary.north);
\draw [arrow] (random) -- (random.south|-primary.north);
\draw [arrow] (decision_1) -- node[above,pos=0.5] {\large Yes} (yield);
\draw [arrow] (decision_2) -- node[above,pos=0.5] {\large No} (terminate);
\draw [arrow] (decision_2) -- node[right,pos=0.5] {\large Yes} (decision_1);
\draw [arrow] (decision_1) |- node[right,pos=0.125] {\large No} node[pos=1] {\large {$(x_{i},y_{i},z_{i}), E_{i}$}} ([xshift=1.5cm,yshift=1cm]kernel.north) -| ([xshift=1.5cm]kernel);
\draw [arrow] (ray_i) -| (decision_2);
\draw [arrow] (kernel) -- (decision_0);
\draw [arrow,name path=line 1] (decision_0) -| node[above,pos=0.125] {\large No} ([xshift=-1cm]kernel.west) |- (primary);
\path [name path=line 2] (mesh) -- (kernel);
\path [name intersections={of = line 1 and line 2}];
\coordinate (S)  at (intersection-1);
\path [name path=circle] (S) circle(\radius);
\path [name intersections={of = circle and line 2}];
\coordinate (I1)  at (intersection-1);
\coordinate (I2)  at (intersection-2);
\draw[ thick,->,>=stealth] (I1) -- (kernel);
\draw [thick] (I2) -- (mesh);
\tkzDrawArc[color=black,thick](S,I1)(I2);
\draw [arrow] (decision_0) -- node[right,pos=0.5] {\large Yes} (ray_i);

\end{tikzpicture}
\caption{Flowchart of the Raytracing Monte Carlo Code.\label{flow}}
\end{figure}
\end{document}