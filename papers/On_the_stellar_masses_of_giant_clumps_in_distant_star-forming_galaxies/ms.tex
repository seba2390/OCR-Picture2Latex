%%
%% Beginning of file 'sample61.tex'
%%
%% Modified 2016 September
%%
%% This is a sample manuscript marked up using the
%% AASTeX v6.1 LaTeX 2e macros.
%%
%% AASTeX is now based on Alexey Vikhlinin's emulateapj.cls 
%% (Copyright 2000-2015).  See the classfile for details.

%% AASTeX requires revtex4-1.cls (http://publish.aps.org/revtex4/) and
%% other external packages (latexsym, graphicx, amssymb, longtable, and epsf).
%% All of these external packages should already be present in the modern TeX 
%% distributions.  If not they can also be obtained at www.ctan.org.

%% The first piece of markup in an AASTeX v6.x document is the \documentclass
%% command. LaTeX will ignore any data that comes before this command. The 
%% documentclass can take an optional argument to modify the output style.
%% The command below calls the preprint style  which will produce a tightly 
%% typeset, one-column, single-spaced document.  It is the default and thus
%% does not need to be explicitly stated.
%%
%%
%% using aastex version 6.1
\documentclass[twocolumn]{aastex61}

%% The default is a single spaced, 10 point font, single spaced article.
%% There are 5 other style options available via an optional argument. They
%% can be envoked like this:
%%
%% \documentclass[argument]{aastex61}
%% 
%% where the arguement options are:
%%
%%  twocolumn   : two text columns, 10 point font, single spaced article.
%%                This is the most compact and represent the final published
%%                derived PDF copy of the accepted manuscript from the publisher
%%  manuscript  : one text column, 12 point font, double spaced article.
%%  preprint    : one text column, 12 point font, single spaced article.  
%%  preprint2   : two text columns, 12 point font, single spaced article.
%%  modern      : a stylish, single text column, 12 point font, article with
%% 		  wider left and right margins. This uses the Daniel
%% 		  Foreman-Mackey and David Hogg design.
%%
%% Note that you can submit to the AAS Journals in any of these 6 styles.
%%
%% There are other optional arguments one can envoke to allow other stylistic
%% actions. The available options are:
%%
%%  astrosymb    : Loads Astrosymb font and define \astrocommands. 
%%  tighten      : Makes baselineskip slightly smaller, only works with 
%%                 the twocolumn substyle.
%%  times        : uses times font instead of the default
%%  linenumbers  : turn on lineno package.
%%  trackchanges : required to see the revision mark up and print its output
%%  longauthor   : Do not use the more compressed footnote style (default) for 
%%                 the author/collaboration/affiliations. Instead print all
%%                 affiliation information after each name. Creates a much
%%                 long author list but may be desirable for short author papers
%%
%% these can be used in any combination, e.g.
%%
%% \documentclass[twocolumn,linenumbers,trackchanges]{aastex61}

%% AASTeX v6.* now includes \hyperref support. While we have built in specific
%% defaults into the classfile you can manually override them with the
%% \hypersetup command. For example,
%%
%%\hypersetup{linkcolor=red,citecolor=green,filecolor=cyan,urlcolor=magenta}
%%
%% will change the color of the internal links to red, the links to the
%% bibliography to green, the file links to cyan, and the external links to
%% magenta. Additional information on \hyperref options can be found here:
%% https://www.tug.org/applications/hyperref/manual.html#x1-40003

%% If you want to create your own macros, you can do so
%% using \newcommand. Your macros should appear before
%% the \begin{document} command.
%%
\newcommand{\vdag}{(v)^\dagger}
\newcommand\aastex{AAS\TeX}
\newcommand\latex{La\TeX}

\usepackage{color}

%% Reintroduced the \received and \accepted commands from AASTeX v5.2
\received{}
\revised{}
\accepted{}
%% Command to document which AAS Journal the manuscript was submitted to.
%% Adds "Submitted to " the arguement.
\submitjournal{ApJL}

%% Mark up commands to limit the number of authors on the front page.
%% Note that in AASTeX v6.1 a \collaboration call (see below) counts as
%% an author in this case.
%
%\AuthorCollaborationLimit=3
%
%% Will only show Schwarz, Muench and "the AAS Journals Data Scientist 
%% collaboration" on the front page of this example manuscript.
%%
%% Note that all of the author will be shown in the published article.
%% This feature is meant to be used prior to acceptance to make the
%% front end of a long author article more manageable. Please do not use
%% this functionality for manuscripts with less than 20 authors. Conversely,
%% please do use this when the number of authors exceeds 40.
%%
%% Use \allauthors at the manuscript end to show the full author list.
%% This command should only be used with \AuthorCollaborationLimit is used.

%% The following command can be used to set the latex table counters.  It
%% is needed in this document because it uses a mix of latex tabular and
%% AASTeX deluxetables.  In general it should not be needed.
%\setcounter{table}{1}

%%%%%%%%%%%%%%%%%%%%%%%%%%%%%%%%%%%%%%%%%%%%%%%%%%%%%%%%%%%%%%%%%%%%%%%%%%%%%%%%
%%
%% The following section outlines numerous optional output that
%% can be displayed in the front matter or as running meta-data.
%%
%% If you wish, you may supply running head information, although
%% this information may be modified by the editorial offices.
\shorttitle{On the masses of clumps in distant galaxies}
\shortauthors{Dessauges-Zavadsky et al.}
%%
%% You can add a light gray and diagonal water-mark to the first page 
%% with this command:
% \watermark{text}
%% where "text", e.g. DRAFT, is the text to appear.  If the text is 
%% long you can control the water-mark size with:
%  \setwatermarkfontsize{dimension}
%% where dimension is any recognized LaTeX dimension, e.g. pt, in, etc.
%%
%%%%%%%%%%%%%%%%%%%%%%%%%%%%%%%%%%%%%%%%%%%%%%%%%%%%%%%%%%%%%%%%%%%%%%%%%%%%%%%%

%% This is the end of the preamble.  Indicate the beginning of the
%% manuscript itself with \begin{document}.

\begin{document}

\title{On the stellar masses of giant clumps in distant star-forming galaxies}

%% LaTeX will automatically break titles if they run longer than
%% one line. However, you may use \\ to force a line break if
%% you desire. In v6.1 you can include a footnote in the title.

%% A significant change from earlier AASTEX versions is in the structure for 
%% calling author and affilations. The change was necessary to implement 
%% autoindexing of affilations which prior was a manual process that could 
%% easily be tedious in large author manuscripts.
%%
%% The \author command is the same as before except it now takes an optional
%% arguement which is the 16 digit ORCID. The syntax is:
%% \author[xxxx-xxxx-xxxx-xxxx]{Author Name}
%%
%% This will hyperlink the author name to the author's ORCID page. Note that
%% during compilation, LaTeX will do some limited checking of the format of
%% the ID to make sure it is valid.
%%
%% Use \affiliation for affiliation information. The old \affil is now aliased
%% to \affiliation. AASTeX v6.1 will automatically index these in the header.
%% When a duplicate is found its index will be the same as its previous entry.
%%
%% Note that \altaffilmark and \altaffiltext have been removed and thus 
%% can not be used to document secondary affiliations. If they are used latex
%% will issue a specific error message and quit. Please use multiple 
%% \affiliation calls for to document more than one affiliation.
%%
%% The new \altaffiliation can be used to indicate some secondary information
%% such as fellowships. This command produces a non-numeric footnote that is
%% set away from the numeric \affiliation footnotes.  NOTE that if an
%% \altaffiliation command is used it must come BEFORE the \affiliation call,
%% right after the \author command, in order to place the footnotes in
%% the proper location.
%%
%% Use \email to set provide email addresses. Each \email will appear on its
%% own line so you can put multiple email address in one \email call. A new
%% \correspondingauthor command is available in V6.1 to identify the
%% corresponding author of the manuscript. It is the author's responsibility
%% to make sure this name is also in the author list.
%%
%% While authors can be grouped inside the same \author and \affiliation
%% commands it is better to have a single author for each. This allows for
%% one to exploit all the new benefits and should make book-keeping easier.
%%
%% If done correctly the peer review system will be able to
%% automatically put the author and affiliation information from the manuscript
%% and save the corresponding author the trouble of entering it by hand.

\correspondingauthor{Miroslava Dessauges-Zavadsky}
\email{miroslava.dessauges@unige.ch}

%\author[0000-0002-0786-7307]{Greg J. Schwarz}
\author{Miroslava Dessauges-Zavadsky}
\affil{Observatoire de Gen\`eve, Universit\'e de Gen\`eve, 51 Ch. des Maillettes, 1290 Versoix, Switzerland}

\author{Daniel Schaerer}
\affiliation{Observatoire de Gen\`eve, Universit\'e de Gen\`eve, 51 Ch. des Maillettes, 1290 Versoix, Switzerland}
\affiliation{CNRS, IRAP, 14 Avenue E. Belin, 31400 Toulouse, France}

\author{Antonio Cava}
\affiliation{Observatoire de Gen\`eve, Universit\'e de Gen\`eve, 51 Ch. des Maillettes, 1290 Versoix, Switzerland}

\author{Lucio Mayer}
\affiliation{Center for Theoretical Astrophysics and Cosmology, Institute for Computational Science, University of Zurich, Winterthurerstrasse 190, 8057 Z\"urich, Switzerland}
\affiliation{Physik-Institut, University of Zurich, Winterthurerstrasse 190, 8057 Z\"urich, Switzerland}

\author{Valentina Tamburello}
\affiliation{Center for Theoretical Astrophysics and Cosmology, Institute for Computational Science, University of Zurich, Winterthurerstrasse 190, 8057 Z\"urich, Switzerland}
\affiliation{Physik-Institut, University of Zurich, Winterthurerstrasse 190, 8057 Z\"urich, Switzerland}


%% Note that the \and command from previous versions of AASTeX is now
%% depreciated in this version as it is no longer necessary. AASTeX 
%% automatically takes care of all commas and "and"s between authors names.

%% AASTeX 6.1 has the new \collaboration and \nocollaboration commands to
%% provide the collaboration status of a group of authors. These commands 
%% can be used either before or after the list of corresponding authors. The
%% argument for \collaboration is the collaboration identifier. Authors are
%% encouraged to surround collaboration identifiers with ()s. The 
%% \nocollaboration command takes no argument and exists to indicate that
%% the nearby authors are not part of surrounding collaborations.

%% Mark off the abstract in the ``abstract'' environment. 
\begin{abstract}

We analyse stellar masses of clumps drawn from a compilation of star-forming galaxies at $1.1<z<3.6$. Comparing clumps selected in different ways, and in lensed or blank field galaxies, we examine the effects of spatial resolution and sensitivity on the inferred stellar masses. Large differences are found, with median stellar masses ranging from $\sim 10^9~M_{\sun}$ for clumps in the often-referenced field galaxies to $\sim 10^7~M_{\sun}$ for fainter clumps selected in deep-field or lensed galaxies. We argue that the clump masses, observed in non-lensed galaxies with a limited spatial resolution of $\sim 1$~kpc, are artificially increased due to the clustering of clumps
%the blending and cluster of clumps 
of smaller mass. Furthermore, we show that the sensitivity threshold used for the clump selection affects the inferred masses even more strongly than resolution, biasing clumps at the low mass end. 
%Together the two effects imply that very large clump stellar masses are most likely overestimated by $1-2$ orders of magnitude, in agreement with recent high-resolution simulations of disk fragmentation. 
Both improved spatial resolution and sensitivity appear to shift the clump stellar mass distribution to lower masses, qualitatively in agreement with clump masses found in recent high-resolution simulations of disk fragmentation. We discuss the nature of the most massive clumps, and we conclude that it is currently not possible to properly establish a meaningful clump stellar mass distribution from observations and to infer the existence and value of a characteristic clump mass scale.
% 
%We have compiled a sample of 242 clumps within 40 star-forming galaxies at $1.0<z<3.5$ with stellar mass measurements. They cover a wide dynamical range of stellar masses from $\sim 10^{5.5}~M_{\sun}$ to $10^{10}~M_{\sun}$. When dividing the host galaxies in three sub-samples, lensed galaxies, field galaxies with a deep clump selection, and field galaxies with a shallow clump selection, denoted by L, FD, and FS, respectively, large differences are observed between them. Whereas the ``typical'' mass of clump masses in the FS sub-sample~--~used until now as the benchmark of high-redshift clump properties~--~is very high (median $\log (M_*^{\rm clump,FS}/M_{\sun}) \sim 8.9$), the FD sub-sample has a significantly lower median clump mass (median $\log (M_*^{\rm clump,FD}/M_{\sun}) \sim 7.3$), and clumps in the two lensed galaxies show even somewhat lower masses. Comparable differences are observed when comparing the absolute rest-frame optical magnitudes of clumps in the three sub-samples. We show that it is primarily spatial resolution and sensitivity (i.e.\ imaging depth and/or clump selection threshold) effects that are driving the observed differences among the three sub-samples. Obviously, limited spatial resolution artificially boosts the clump stellar masses by clump-blending in photometric apertures, as the true physical sizes of clumps, resolved down to $\sim 100$~pc sizes in lensed galaxies, are clearly smaller than the kiloparsec resolution achieved in HST imaging of field galaxies. Apart from this effect, the sensitivity effect even more importantly affects the inferred clump stellar masses, biasing the observed clumps at the low mass end, as clearly demonstrated by the large differences observed between clumps in the field galaxy FS and FD sub-samples, while the two sub-samples suffer from the same resolution limitation. Furthermore, we show evidence for an apparent scaling of the maximum clump mass with the host galaxy mass. From all this we conclude that it is currently not possible to properly establish a meaningful clump stellar mass distribution from observations, and, in particular, to infer the existence and value of a characteristic clump mass scale. Nevertheless, the only clear indication is that both improved spatial resolution and sensitivity appear to shift the clump stellar mass distribution to lower masses, in agreement with the latest simulations of disk fragmentation. These concur to find a characteristic clump mass of a few $10^7~M_{\sun}$ at the onset of fragmentation and predict a stellar mass distribution of clumps, which very much resembles that of clumps in the L and FD sub-samples. The agreement may well be fortuitous, and, in addition, we cannot exclude the possible clump hierarchical structure.
  
%The first lesson in the tutorial is to remind authors that the AAS Journals, the Astrophysical Journal (ApJ), the Astrophysical Journal Letters (ApJL), and Astronomical Journal (AJ), all have a 250 word limit for the abstract.  If you exceed this length the Editorial office will ask you to shorten it.

\end{abstract}

%% Keywords should appear after the \end{abstract} command. 
%% See the online documentation for the full list of available subject
%% keywords and the rules for their use.
\keywords{galaxies: evolution --- galaxies: high-redshift --- galaxies: structure}

%% From the front matter, we move on to the body of the paper.
%% Sections are demarcated by \section and \subsection, respectively.
%% Observe the use of the LaTeX \label
%% command after the \subsection to give a symbolic KEY to the
%% subsection for cross-referencing in a \ref command.
%% You can use LaTeX's \ref and \label commands to keep track of
%% cross-references to sections, equations, tables, and figures.
%% That way, if you change the order of any elements, LaTeX will
%% automatically renumber them.

%% We recommend that authors also use the natbib \citep
%% and \citet commands to identify citations.  The citations are
%% tied to the reference list via symbolic KEYs. The KEY corresponds
%% to the KEY in the \bibitem in the reference list below. 

%
%______________________________________________________________

\section{Introduction}
\label{sect:introduction}

Deep {\it Hubble Space Telescope} ({\it HST}) observations and pioneering 
morphological analysis of distant star-forming galaxies have revealed that 
galaxies at the peak of the cosmic star formation activity do not follow the 
{\it Hubble} classification, but are mostly irregular and clumpy 
\citep{elmegreen05,elmegreen07,elmegreen09}. \citet{guo15} and \citet{shibuya16} 
%in their study of thousands of galaxies in various {\it HST} deep fields, 
have evaluated that at $z\gtrsim 2$ roughly 60\% of galaxies are clumpy and that 
this fraction evolves over $z\simeq 0-8$.

While the observed stellar clumps have initially been associated with 
interactions/mergers, another clump origin had to be invoked with kinematic 
studies showing a substantial proportion of $z\sim 1-2.5$ galaxies dominated by 
ordered disk rotation \citep{forster09,wisnioski15,rodrigues16}. These 
high-redshift disks, however, are very different from their local counterparts, 
being highly turbulent, thick, gas-rich, and strongly star-forming disks. They 
are subject to violent instabilities \citep{dekel09b} caused by intense inflows 
of cold gas \citep{keres05,ocvirk08,dekel09a}. Giant kpc-scale clumps with 
masses as high as $\gtrsim 10^8-10^{9.5}~M_{\sun}$ may then form during the disk 
fragmentation phase resulting from disk instabilities, as found in idealized 
simulations of isolated galaxies and cosmological simulations 
\citep{agertz09,bournaud10,bournaud14,ceverino10,ceverino12,genel12,mandelker14}. 
The produced giant clumps resemble the kpc-sized clumps observed in $z\sim 2$ 
galaxies with similar stellar masses
%appear in agreement with the stellar masses estimated for observed kpc-sized clumps in $z\sim 2$ galaxies 
\citep{forster11,guo12}, and provide support to clump formation via disk 
fragmentation.

Recently, \citet[][hereafter T15]{tamburello15} and \citet{behrendt16} performed 
numerical simulations of isolated galaxies at a significantly better spatial 
resolution, 
%a parameter known to be crucial before the onset of fragmentation in order 
necessary to capture fragmentation correctly \citep{mayer08}. They both find 
that the formation of giant clumps via disk fragmentation with masses 
$>10^8~M_{\sun}$ is {\em not} a common occurrence. They get the same
characteristic clump mass set by fragmentation, as low as a few times 
$10^7~M_{\sun}$, despite significant differences in their respective simulation 
techniques, 
%(SPH versus AMR), 
and star formation and feedback recipes only included in T15. This fragmentation 
mass is well matched with the modified Toomre mass proposed by T15 that takes 
into account nonlinear aspects of disk fragmentation \citep{boley10}. A few 
clumps can grow to larger masses ($\sim 10^{9-9.5}~M_{\sun}$) by clump-clump 
mergers and gas accretion, but they populate only the tail of the mass 
distribution and emerge after several disk orbits. The conventional Toomre mass, 
resulting from simple linear perturbation theory \citep{toomre64}, is almost an 
order of magnitude larger, and hence appears only fortuitously comparable to the 
clump high mass tail.
%emerging at later times.
A similar mass spectrum ranging from $\sim 10^{6.5}~M_{\sun}$ to 
$10^{9.5}~M_{\sun}$ is obtained for the `in situ' clumps formed in the recent 
high-resolution cosmological simulations by \citet{mandelker17}, as well as in 
the FIRE cosmological simulations by \citet{oklopcic17}.

Now that disk fragmentation simulations of different groups 
%ds - maybe not yet converged? too early to say...
%converge on finding 
find significantly lower masses for the high-redshift clumps with respect to 
previous claims,
%\footnote{But see \citet{mayer16} on the possibility that fragmentation is even further reduced with the latest generation of feedback models.}, 
it is timely to revisit the observational constraints on clump masses. In this 
Letter, we compile a sample of clumps in star-forming galaxies at $1.1<z<3.6$ 
with stellar mass measurements. 
%(Section~\ref{sect:samples}). In Section~\ref{sect:results} 
We show that a very broad range of clump masses has been derived, and find 
evidence that the derived masses and mass distributions suffer from 
limitations in both spatial resolution and sensitivity. 
%the flux limit applied on the clump selection. 
%These clump masses are then critically examined in Sect.~\ref{sect:results} to determine whether they are correct or whether they are affected by spatial resolution and sensitivity limitations. 
%Finally, in Section~\ref{sect:discussion} 
We discuss what we may infer on the true stellar mass spectrum of high-redshift 
clumps. Our simple qualitative analysis presented here highlights important 
biases affecting the intrinsic clump stellar mass estimates, which we have 
started quantitatively evaluating in our first companion paper on H$\alpha$ mock 
simulations \citep[][hereafter T16]{tamburello16} and in a detailed 
observational clump analysis within a multiple-imaged galaxy (Cava et~al.\ in 
preparation).

%
%______________________________________________________________

\section{Clump sample}
\label{sect:samples}

We have compiled a sample of clumps from the literature within clumpy 
star-forming galaxies at $z>1$, where clumps have been identified in broad-band 
{\it HST} imaging, predominantly tracing stellar emission. Our sample comprises 
a total of 241 stellar clumps hosted in 40 galaxies from \citet{forster11}, 
\citet{guo12}, \citet{adamo13}, \citet{elmegreen13}, and \citet{wuyts14}. 
These five clump datasets are described in Table~\ref{tab:statistics}. For the 
bulk of the sample (213 out of 241 clumps), 
%i.e., all clumps except those from \citet{forster11}),
%from \citet{guo12}, \citet{adamo13}, \citet{elmegreen13}, and \citet{wuyts14}), 
we have been able to recompute the clump stellar masses in a homogeneous way, 
using the original multi-band {\it HST} photometry and the updated version of 
the {\it Hyperz} photometric redshift and SED fitting code \citep{schaerer10}. 
For the remaining 28 clumps from \citet{forster11}, as observations in only one 
{\it HST}/NICMOS filter F160W are available, we have not re-analysed their 
published stellar masses, instead we rely on these estimates obtained from an 
assumed mass-to-light ratio.

The photometry of the stellar clumps from \citet{guo12} and \citet{elmegreen13} 
is based on the {\it Hubble} Ultra Deep Field observations 
\citep[HUDF,][]{beckwith06}, performed with {\it HST}/ACS in the filters F435W, 
F606W, F775W, and F850LP. \citet{guo12} also used the {\it HST}/WFC3 
observations in the filters F105W, F125W, and F160W, which were not available at 
the time of the \citet{elmegreen13} work. \citet{adamo13} have used the Cluster 
Lensing And Supernova survey with {\it Hubble} \citep[CLASH,][]{postman12} to 
analyse clumps in the filters F390W, F475W, F555W, F606W, F775W, F814W, and 
F850LP from {\it HST}/ACS, and the filters F105W, F110W, F125W, and F160W from 
{\it HST}/WFC3. And, \citet{wuyts14} had at disposal observations in the 
{\it HST}/WFC3 filters F390W, F606W, F814W, F098M, F125W, and F160W. 
%For all the studied clumpy host galaxies, the longest wavelength observations available at 1.6~$\mu$m is covering the rest-frame optical emission of the stellar clumps at their typical redshift of $z\sim 2$.
For the typical redshift $z\sim 2$ of the studied clumpy host galaxies, the 
longest wavelength observations available at 1.6~$\mu$m for all cover the 
rest-frame optical emission of the stellar clumps.

%Guo - Beckwith: ACS (B) F435W 29.4 mag, (VI) F606W 29.8, (i) F775W 29.7, (z) F850LP 29.0, $5 \sigma$, 0.35 \arsec\ diameter aperture)
%Guo - WFC3 F105W, F125W, F160W, all $\sim 29.0$ mag.
%\Elmegreen - same ACS data, no WFC3

For the SED modelling, we have adopted the \citet{bruzual03} stellar tracks at 
solar metallicity and the \citet{chabrier03} initial mass function. We have 
allowed for variable star formation histories, parametrised by exponentially 
declining models with timescales varying from 10~Myr to infinity\footnote{More 
precisely, we have used the following timescales $\tau=0.01$, 0.03, 0.05, 0.07, 
0.1, 0.3, 0.5, 0.7, 1., 3., $\infty$~Gyr.}, corresponding to a constant star 
formation rate. Nebular emission has been neglected, as in \citet{guo12} and 
\citet{elmegreen13}. With respect to these works, we find very small or no 
systematic differences with our inferred clump stellar masses\footnote{The mean 
of the logarithmic differences in stellar mass is 
$\Delta(\log(M^{\rm clump}_*/M_{\sun})) = -0.16\pm 0.15$ for clumps from 
\citet{guo12}, and $\Delta(\log(M^{\rm clump}_*/M_{\sun})) = +0.045\pm 0.45$ for 
clumps from \citet{elmegreen13}.}. We have also tested the impact on clump 
stellar masses when including or not the near-infrared {\it HST}/WFC3 photometry 
in the dataset of \citet{guo12}. We find higher stellar masses by +0.20~dex, on 
average, when the {\it HST}/WFC3 filters are omitted, as done in 
\citet{elmegreen13}. This could thus lead to a small systematic shift by a 
factor of $\sim 1.5$ between the clump masses of \citet{guo12} and 
\citet{elmegreen13}.
%but which does not affect our conclusions.
Compared to the clump stellar masses reported by \citet{adamo13}, our masses are 
higher by +0.56~dex, on average. This difference 
vanishes\footnote{$\Delta(\log(M^{\rm clump}_*/M_{\sun})) = +0.004\pm 0.46$.} if 
we include nebular emission and allow for ages younger than 10~Myr in the SED 
fits, as adopted by \citet{adamo13}. For the \citet{wuyts14} dataset, we find a 
small difference of $-0.12$~dex, on average, in clump stellar masses when 
neglecting nebular emission, as in their work. For a uniform and conservative 
comparison between all the clump datasets, we retain the clump stellar masses 
obtained from SED fits {\it without} nebular emission.

%{\bf \textcolor{red}{The estimates of the clump stellar masses from 
%\citet{forster11} are less accurate, since being derived from a single 
%{\it HST}/NICMOS F160W filter and by assuming a given mass-to-light ratio. In 
%what follows, we consider/omit (***what do we decide?) this sample of 28 clumps 
%for the sake of completeness/homogeneity in the clump stellar mass 
%measurements.}}

The wavelength coverage of the above four clump datasets is nearly identical, 
which thus enables a meaningful and nearly homogeneous comparison between these 
datasets. As a measure of the depth of the selected clumps, we examine the clump 
magnitude distributions and we list in Table~\ref{tab:statistics}, for the four 
clump datasets, the magnitudes {\it i}$_{16}$ and {\it z}$_{16}$ corresponding to 
the 16{\it th} percentile of the magnitude distribution of clumps
%we list the faint magnitude corresponding to 68\% of the magnitude distribution of clumps 
in the F775W (for HUDF) or F814W {\it i}-band and in the F850LP {\it z}-band, 
respectively. The latter corresponds to the clump selection band of 
\citet{guo12}, and the former is the second-deepest band in HUDF and CLASH. We 
also indicate the $3\,\sigma$ sensitivity limits in the {\it i}- and 
{\it z}-bands measured in 0.35\arcsec\ diameter apertures, as reported by 
\citet{beckwith06} for HUDF and \citet{postman12} for CLASH. 
%the sensitivity of the images, as measured by the $3\,\sigma$ limit measured in 0.3\arcsec\ diameter apertures.
%For the last clump dataset from \citet{forster11}, as observations in only one {\it HST}/NICMOS filter F160W are available, we do not re-analyse their data. Instead, we rely on their clump stellar mass estimates obtained from an assumed mass-to-light ratio.}
On this basis, we divide the clumpy host galaxy compilation into three main 
sub-samples, lensed galaxies, field galaxies with a deep clump selection, and 
field galaxies with a shallow clump selection, denoted hereafter by L, FD, and 
FS, respectively (see Table~\ref{tab:statistics}). 
%{\bf The sub-sample FS$^\prime$ stands for the FS sub-sample when excluding the clumps from \citet{forster11}, whose stellar masses have not been rederived in a homogeneous way.

The 40 host galaxies have redshifts ranging from $z=1.1$ to 3.6, with the bulk 
found at $1.3<z<2.6$. Their stellar masses are uniformly distributed between 
$M_*^{\rm host} \sim 10^8-10^{11}~M_{\sun}$ (see Figure~\ref{fig:clump-host}), 
with the L and FD sub-samples containing the low-mass host galaxies 
($M_*^{\rm host} \lesssim 10^{10}~M_{\sun}$) and the FS sub-sample the high-mass 
hosts ($M_*^{\rm host} \gtrsim 10^{9.8}~M_{\sun}$). Most of the hosts are on the 
main sequence at their corresponding redshift.
%Most of the host galaxies follow the redshift-dependent main sequence. 
%at their respective redshift \citep[e.g.,][]{rodighiero10}. 

%DS je ne pense pas que ce soit important ici:
%The kinematic measurements available for the \citet{forster11}, \citet{adamo13}, %and \citet{wuyts14} galaxies show for all, except two, 
%the majority\footnote{Only one field galaxy is classified as a major merger \citep{forster11}, and the lensed galaxy from \citet{wuyts14} shows disturbed kinematics suggestive of an ongoing interaction.} 
%ordered disk rotation.
%***DS Uniquement laisser si important:
%with rotation velocity over velocity dispersion ratios varying between $\upsilon_{\rm rot}/\sigma = 0.8-6$, in line with the $\upsilon_{\rm rot}/\sigma$ spread found by \citet{wisnioski15} in $0.7<z<2.7$ galaxies with comparable stellar masses.

For comparison we also consider local star clusters found in nearby galaxies 
\citep{adamo13} and two starburst galaxies \citep{bastian06,larsen02}. 
%including two starburst galaxies, the Antennae \citep{bastian06} and NGC\,6946 \citep{larsen02}. 

%The resolution obviously matters when determining the intrinsic physical size of clumps and it may also affect their stellar mass. The HST imaging achieves a spatial resolution of $0.15''$ FWHM, which still corresponds to a relatively large physical scale of $1.2-1.3~\rm kpc$ at $z\sim 1.3-2.6$. Currently, the only way to beat this limitation and reach sub-kpc scales is with the help of strong gravitational lensing. Indeed, strong lensing leads to the stretching of the source image, such that in the lensed clumpy galaxies studied in the literature spatial scales down to $\sim 120$~pc are resolved in the source plane.
%%that conserving the surface brightness produces an amplification in the total observed flux proportional to the stretching factor. Thus, depending on the amplification factor solely, physical sub-kiloparsec scales become accessible in lensed high-redshift galaxies. 
%With the \citet{elmegreen13} sub-sample, we may, on the other hand, test the effect of the flux limit applied on the clump selection. 

%
%_______________________________________________________________

\begin{deluxetable*}{lcc|c|cc}
\tablecolumns{6}
\tablecaption{Properties of existing stellar clump datasets in high-redshift galaxies \label{tab:statistics}}
\tablewidth{0pt}
\tablehead{
\colhead{References} & \colhead{Adamo+13} & \colhead{Wuyts+14} & \colhead{Elmegreen+13} & \colhead{Guo+12} & \colhead{F\"orster Schreiber+11} \\
 & \multicolumn{2}{c}{L sub-sample\tablenotemark{a}} & \colhead{FD sub-sample\tablenotemark{b}} & \multicolumn{2}{c}{FS sub-sample\tablenotemark{c}} 
}
%\colnumbers
\startdata
Number of clumps                                           & 31 & 7 & 135 & 40 & 28  \\ 
Number of host galaxies                                    & 1  & 1 & 22  & 10 & 6  \\
Redshift                                                   & 1.5 & 1.7 & $1.1-3.6$ & $1.6-2.0$ & $2.2-2.5$ \\
{\it i}$_{16}$\tablenotemark{d}                            & 29.7\tablenotemark{$\dag$} & 29.1\tablenotemark{$\ddag$} & 29.7  & 27.6  & -- \\
{\it i}-band $3\sigma$-0.35\arcsec\ limit\tablenotemark{e} & 30.5\tablenotemark{$\dag$} & 30.9\tablenotemark{$\ddag$} & 30.25 & 30.25 & -- \\
{\it z}$_{16}$\tablenotemark{d}                            & 29.7\tablenotemark{$\dag$} & --                          & 29.7  & 27.3  & -- \\
{\it z}-band $3\sigma$-0.35\arcsec\ limit\tablenotemark{e} & 29.5\tablenotemark{$\dag$} & --                          & 29.55 & 29.55 & -- \\
%
Median $\log(M_*^{\rm clump}/M_{\sun})$ & \multicolumn{2}{c|}{6.98} & 7.23 & \multicolumn{2}{c}{8.89} \\ 
%Mean $\log(M_*^{\rm clump}/M_{\sun})$   & 6.67  & 7.25   & 8.84    \\ 
%Standard deviation ($M_{\sun}$)         & 0.32  & 0.14   & 0.17    \\ 
\enddata
\tablenotetext{a}{Clumps identified in lensed galaxies.} 
\tablenotetext{b}{Clumps identified in field galaxies with a deep clump selection.}
\tablenotetext{c}{Clumps identified in field galaxies with a shallow clump selection.}
\tablenotetext{d}{Magnitudes corresponding to the 16{\it th} percentile of the magnitude distribution of clumps in the F775W (for HUDF) or F814W {\it i}-band and in the F850LP {\it z}-band, respectively.}
\tablenotetext{e}{$3\,\sigma$ sensitivity limits in the {\it i}-, respectively, {\it z}-band measured in 0.35\arcsec\ diameter apertures (from \citet{beckwith06} for HUDF and \citet{postman12} for CLASH).}
\tablenotetext{$\dag$}{Corrected for lensing, assuming a magnification factor $\mu=8$ \citep{adamo13}.}
\tablenotetext{$\ddag$}{Corrected for lensing, assuming a magnification factor $\mu=25$ \citep{sharon12}.}
\end{deluxetable*}
%
%_______________________________________________________________

\begin{figure}
\centering
\includegraphics[width=8.3cm,clip]{fig1.eps}
\caption{Normalized stellar mass distributions of local star clusters (filled yellow histogram), and three sub-samples of high-redshift clumps: clumps in lensed galaxies (L sub-sample, open green histogram), in field galaxies with a deep clump selection (FD sub-sample, hatched red histogram), and in field galaxies with a shallow clump selection (FS sub-sample, filled cyan histogram). The medians of the high-redshift clump sub-samples are shown using dotted green, dashed red, and solid cyan vertical lines, respectively. For comparison, clump mass distributions as predicted by different disk fragmentation simulations (open blue thick and thin histograms from T15 and \citet{ceverino12}, respectively) are also shown in each panel.} 
%All histograms are plotted in stellar mass bins of 0.3~dex.
\label{fig:clump-M*}
\end{figure}
%
%_______________________________________________________________

\section{Can we infer accurate clump masses at high redshift?}
\label{sect:results}

As shown in Figure~\ref{fig:clump-M*}, the distribution of stellar masses of 
clumps identified in high-redshift galaxies is very broad and ranges from 
$M_*^{\rm clump}\sim 10^{5.5}~M_{\sun}$ to $10^{10.5}~M_{\sun}$. Large 
differences are observed among the three sub-samples of high-redshift galaxies 
considered here. Whereas the ``typical'' mass of clump masses in the field 
galaxies studied by \citet{forster11} and \citet{guo12} (FS sub-sample)~--~used 
until now as the benchmark of high-redshift clump properties~--~is very high 
(median $\log (M_*^{\rm clump,FS}/M_{\sun}) = 8.89$), the \citet{elmegreen13} 
field galaxies (FD sub-sample) have a median clump mass much lower 
($\log (M_*^{\rm clump,FD}/M_{\sun}) = 7.23$), and clumps in lensed galaxies 
(L sub-sample) show even somewhat lower masses (median 
$\log (M_*^{\rm clump,FS}/M_{\sun}) = 6.98$; see Table~\ref{tab:statistics}). 
In comparison to the star clusters identified in local galaxies, the inferred 
clump masses in high-redshift galaxies are, on average, significantly higher 
than those in local galaxies also shown in Figure~\ref{fig:clump-M*}, with the 
exception of some star clusters in the most intensively star-forming nearby 
galaxies.

%The three sub-samples are also compared in Figure~\ref{fig:clump-Mag}, where we 
%show the distributions of their clumps' absolute magnitudes in the {\it B} or 
%{\it V} band. 
The absolute rest-frame {\it V}-band magnitude distributions of the three clump 
sub-samples are compared in Figure~\ref{fig:clump-Mag}. Clearly, the FS 
sub-sample has significantly brighter clumps than the FD sub-sample, although 
both are drawn from field galaxies over a similar redshift range. The clumps in 
the lensed galaxies are slightly fainter, on average, than those in the FD 
sub-sample. The differences in absolute magnitude and in stellar mass 
(Figure~\ref{fig:clump-Mag} versus Figure~\ref{fig:clump-M*}) are comparable, as 
expected, since the optical light traces stellar mass if the mass-to-light ratio 
of clumps does not vary much.

What explains the large differences found between the three sub-samples of 
high-redshift clumps? We primarily envisage {\em spatial resolution} and 
{\em sensitivity} as the main sources for these differences. 

All high-redshift clumps rely on {\it HST} imaging with the same spatial 
resolution of $\sim 0.15\arcsec$ FWHM, which corresponds to physical sizes of 
$1.2-1.3$~kpc at $z\sim 1.3-2.6$ in field/non-lensed galaxies.
%(FS and FD sub-samples). 
Obviously, limited spatial resolution can affect the measure of clump stellar 
masses, if the true physical sizes of clumps are smaller than the resolution, 
since then several clumps may be blended within the photometric aperture. This 
effect will artificially ``boost'' the flux and increase the inferred stellar 
mass of clumps. The amount of this artificial boost will depend on the clump 
true sizes, their distribution and clustering. 
%ds -- not well defined, and it's the same as clustering, no? 
%and surface filling factor, 
%as well as the spatial resolution. 
With the help of strong gravitational lensing,
%\footnote{Strong lensing leads to the stretching of the source image, that conserving the surface brightness produces an amplification in the total observed flux proportional to the stretching factor.} 
sub-kpc sizes down to $\sim 100$~pc (representing an improvement by a factor of 
10) are reached 
%at high redshift. 
in the two lensed galaxies of the L sub-sample.
%as for example used in the work by \citet{adamo13}, which makes the majority of clumps in the lensed galaxy L sub-sample. 
The finding of considerably lower clump stellar masses 
%in the lensed galaxy L sub-sample 
(Figure~\ref{fig:clump-M*} and Table~\ref{tab:statistics}) compared to the 
clump masses in the field galaxy sub-sample(s) (FS and somewhat FD) supports 
that indeed spatial resolution, and the induced blending, affects the derived 
clump masses and artificially boosts them towards high masses. 

%
%_______________________________________________________________

\begin{figure}
\centering
\includegraphics[width=8.3cm,clip]{fig2.eps}
\caption{Absolute rest-frame {\it V}-band magnitude distributions of high-redshift clumps in the L (open green histogram), FD (hatched red histogram), and FS (filled cyan histogram) sub-samples. The respective means are shown using dotted green, dashed red, and solid cyan vertical lines.} 
%All histograms are plotted in bins of 0.5~mag.
\label{fig:clump-Mag}
\end{figure}
%
%_______________________________________________________________

%Although we cannot precisely quantify this effect here, several recent studies have tried to estimate it. 
%At present it is difficult to quantify the amount of ``artificial boosting'' on clump stellar masses both observationally and from simulations. 
%Quantifying the exact amount of this ``artificial boosting'' on clump stellar masses is beyond the scope of this Letter. 
First quantitative hints of this low-resolution ``boosting'' on clump stellar 
masses have been obtained from recent simulations, which, however, focus on gas 
clumps. \citet{behrendt16}, in their simulations of a massive gas disk with one 
of the highest resolutions to date, find very small ($\sim 35$~pc in radii) and 
low gas mass fragments produced 
%by gravitational instability and 
with disk fragmentation
%characteristic masses of fragments produced by gravitational instability are of order a few times $10^7~M_{\sun}$; 
that, when mimicking observations on kpc-scales ($\rm FWHM=1.6~kpc$), appear to 
be distributed in loosely bound clusters 
%(the small-scale substructure disappears due to beam smearing)
%and to look similar to giant kpc-clumps 
with $10-100$ times larger masses.
%stellar masses as high as $\sim (1-3)\times 10^9~M_{\sun}$. 
%Other simulations from our group (T16) are briefly discussed below. 
We report similar results in T16 using our H$\alpha$ mock observations of 
simulations from T15, and we infer a $\sim 1$~kpc resolution ``boosting'' on 
100~pc-scale clump masses of less than a factor of 5.
%which does not exceed a factor of 5. 
Apart from that, \citet{fisher17},
%observational efforts in low-redshift galaxies \citet{fisher17} 
using low-redshift H$\alpha$ galaxy observations, have analysed how severely 
clump clustering increases sizes and star formation rates in limited 
$\sim 1$~kpc resolution maps.
%H$\alpha$ observations with limited $\sim 1$~kpc resolution,
%but the effect on clump stellar masses still needs to be evaluated.

Interestingly, large stellar mass differences are also observed between clumps 
in the field galaxy FS and FD sub-samples (Figure~\ref{fig:clump-M*} and 
Table~\ref{tab:statistics}), while these galaxies are all affected by the same 
$\sim 1.2$~kpc resolution limitation. Another effect than spatial resolution 
must thus be at the origin of these clump mass differences.
%between the FS and FD sub-samples. 
%They are likely due to different clump selections based on different data 
%depths, different wavebands, and/or more or less conservative 
These differences are likely due to different clump selections applied, 
resulting from different data depths, different wavebands used to identify the 
clumps, and/or more or less conservative 
%``agressive'' selection criteria
detection limits set for clumps. In fact both \citet{guo12}\footnote{The clumps 
from \citet{guo12} represent 60\% of the clumps in the field galaxy FS 
sub-sample.} 
%The remainder comes from \citet{forster11}, where clumps have been selected in the shallower {\it HST}/NICMOS F160W image.} 
and \citet{elmegreen13} used HUDF observations, but the former selected clumps 
in the F850LP {\it z}-band, whereas the latter in the F775W {\it i}-band that is 
0.7~mag deeper (see Table~\ref{tab:statistics}).
%ds c'est un autre point -  ane pas melanger ici...
%The selection of \citet{guo12} thus is clearly more restrictive, and is, in addition, confined to host galaxies with spectroscopic redshifts, which are brighter and more massive, on average, as seen in Fig.~\ref{fig:clump-host}. 
Furthermore, the clumps extracted by \citet{guo12} are limited to F850LP 
magnitudes brighter than $\sim 27.3$, well above the depth of the HUDF 
{\it z}-band image.
%well above their detection image (29.0~mxag, $5\,\sigma$ in a $0.35\arcsec$ aperture). 
In contrast, the observed magnitudes of clumps selected by \citet{elmegreen13} 
reach down to $3\,\sigma$, which can explain differences of up to $\sim 2.5$~mag 
for the faintest clumps in the FD sub-sample (see Table~\ref{tab:statistics}) 
compared to \citet{guo12}. Hence, the clump selection sensitivity threshold 
strongly affects the clump stellar masses, biasing the observed clumps at the 
low mass end.
%against low-mass clumps if too shallow.
%In addition, the study of \citet{guo12} is restricted to host galaxies with spectroscopic redshifts, which are brighter and more massive, on average, as seen in Figure~\ref{fig:clump-host}. 

%It is difficult to quantity the ``sensitivity effect'' 
%As before, we do not aim here to quantify the impact of ``sensitivity'' on the inferred clump masses. Nevertheless, this effect
The sensitivity effect appears to be more important than the spatial resolution 
effect on the inferred clump masses, 
%Although we cannot quantify this effect here, its relative strength with respect to the effect of spatial resolution seems to dominate, 
since the respective stellar mass distributions of clumps in the
\citet{elmegreen13} field galaxies limited by $\sim 1.2$~kpc resolution 
(FD sub-sample) and in the lensed galaxies (L sub-sample) end up to be very 
comparable
%\footnote{The remaining small difference observed in the medians/means of the respective stellar mass distributions of clumps in the L and FD sub-samples (factor of $\sim 4$) can be seen as a first order estimate of the effect of spatial resolution solely on the inferred clump stellar masses.} 
(Figure~\ref{fig:clump-M*} and Table~\ref{tab:statistics}), whereas clumps in 
the lensed galaxies benefit from 10 times better spatial resolution {\it and} 
similarly good sensitivities.
%both better spatial resolution which avoids blending of smaller structures and better sensitivity which is determinant for a complete/unbiased clump selection. 

In any case, the finding of clumps in lensed galaxies and in field galaxies from 
\citet{elmegreen13} with stellar masses between $\sim 10^{5.5}-10^9~M_{\sun}$, 
well below the often-quoted ``typical'' masses of giant clumps 
$\gtrsim 10^8-10^9~M_{\sun}$ inferred from observations with $\sim 1.2$~kpc 
resolution and shallower clump selection thresholds (FS sub-sample), suggests 
that the latter is systematically overestimated 
%due to the finite spatial resolution and sensitivity, 
by $1-2$ orders of magnitude (Table~\ref{tab:statistics}), or more depending on 
whether a characteristic mass scale of fragmentation exists or not (see 
Section~\ref{sect:spectrum}). The same conclusion can be drawn, when we restrict 
the FD and FS sub-samples to host galaxies with redshifts comparable to those of 
the two lensed galaxies from the L sub-sample.
%Furthermore, we would like to stress that the sensitivity effect (including both the imaging depth and/or the applied selection threshold effects) biases the clump stellar masses against low masses because of decreasing completeness with increasing flux cuts, while the spatial resolution effect, in addition, modifies/boosts the observed clump stellar mass distribution toward high masses.
In T16 we study quantitatively the effects of $\sim 1$~kpc resolution and 
shallow sensitivity on the observed clump masses using H$\alpha$ mocks. 
%mock H$\alpha$ observations of a subset of simulations from T15.
%***DS I am not willing to write the next two sentences! and to give numbers. By how much things shift depends on how much you shift the sensitivy... And we're still talking gas-selected clumps etc. I do not think we should make more strong claims. Let us show this properly in subsequent papers...***
We find that the inferred clump stellar masses can be easily overestimated by at 
least a factor of 10 due to the combination of both effects (and with the 
sensitivity effect dominating). 
%**DS I would leave this out! This 'distracts' the reader from our main result...***
%The result is an apparent characteristic stellar mass that only by chance ends up matching the Toomre mass estimated from the disk properties at the clump location.

%
%______________________________________________________________

\section{Discussion}
\label{sect:discussion}

%We now discuss what can be inferred about the maximum mass of clumps and about their true stellar mass spectrum, given the observational evidence that spatial resolution and sensitivity affect the inferred clump masses at high redshift.

%REPHRASE ***
%While the scenario with intrinsically lower stellar masses for the high-redshift star-forming clumps is attractive, it also raises a number of questions: How does the low $M_*^{\rm clump}$ mean fit with the numerical simulations of galactic disk fragmentation? What is the characteristic stellar mass of high-redshift clumps? May we conclude that the giant/massive clumps initially identified in high-redshift clumpy galaxies solely result fromselection effects? Is there a viable dependence of $M_*^{\rm clump}$ on $M_*^{\rm host}$?

%
%_______________________________________________________________

\subsection{On the existence of the most massive clumps}
\label{sect:max}
%\subsection{On the nature of giant/massive clumps: somewhere between reality and beam smearing?}
%\label{sect:bias-highM*}

Is there a maximum stellar mass for clumps, how massive, and what determines 
it? If clump stellar masses are artificially increased by the spatial 
resolution effect 
%and biased against low masses by sensitivity effects (i.e.\ imaging depth/applied clump selection threshold) 
as discussed above, our current best maximum clump mass estimate should come 
from lensed galaxies, where clump stellar masses up to $\sim 10^{8.7}~M_{\sun}$ 
are observed (see Figure~\ref{fig:clump-M*}). However, the L sub-sample is quite 
small (38 clumps) and small number statistics could bias the maximum mass 
determination of clumps (especially if the true clump mass function decreases 
rapidly towards high masses). Furthermore, the maximum clump stellar mass could 
depend on the host galaxy stellar mass 
%shown in Fig.~\ref{fig:clump-host} and 
%as already discussed by \citet{elmegreen13}.
\citep[see][]{elmegreen13}. 
%***DS: and other observers? E05? Guo?
%By definition, the clump mass cannot exceed the host galaxy mass, but there could be a physical process limiting the maximum mass of clump formation 
The fact that clumps in the FS and FD sub-samples, observed with the same 
spatial resolution in host galaxies spanning a wide range of stellar masses, 
show an increase of the upper envelope of their stellar masses with the host 
galaxy mass as illustrated in Figure~\ref{fig:clump-host}, indicates that the 
maximum clump mass indeed depends on the host mass. 
%in agreement with recent simulation predictions. 

By definition, the clump mass cannot exceed the host galaxy mass, but what 
determines the maximum clump mass? The most simple expectation is that the 
maximum clump mass is set by the fragmentation mass that is directly 
proportional to the galaxy mass and the square of its gas fraction in the linear 
perturbation theory, as described by \citet{escala08}. Otherwise, according to 
the innovative simulations of T15, the combination of a typical fragmentation 
scale and additional processes yielding the clump mass growth, such as 
clump-clump mergers, leads to a fractional stellar mass contribution of the sum 
of all clumps to the total disk stellar mass in the range of $10-15$\%, with 
little variation with disk mass. 
%This is explained by the result that 
This results from the fact that the characteristic mass scale of fragmentation 
they get is independent on disk mass. 
%***DS: ARE WE SURE aout this? IS THIS a stable result/fact??? reflects the fact}
%ds - repetition du texte ci-dessous...
%hence as the disk mass increases the only thing that can change is that the disk can be more unstable and fragments more often. 
Massive disks thus just give rise to more clumps that in turn increase the 
likelihood of clump-clump mergers, shifting the maximum stellar mass of clumps 
to larger values. 
%This could thus also explain 
Both approaches allow to explain the apparent scaling of the maximum clump 
mass with the host galaxy mass.
%***DS: Has this been discussed in some simulation paper?***

%
%_______________________________________________________________

\begin{figure}
\centering
\includegraphics[width=8.3cm,clip]{fig3.eps}\hspace{1cm}
%\includegraphics[width=8cm,clip]{fig3b.eps}
\caption{Stellar masses of high-redshift clumps plotted as a function of the stellar mass of their host galaxy. The symbols refer to different works, and the color-coding to the L, FD, and FS sub-samples, similarly to Figures~\ref{fig:clump-M*} and \ref{fig:clump-Mag}. The dotted line is the one-to-one relation.}
\label{fig:clump-host}
\end{figure}
%
%_______________________________________________________________

On the other hand, we could expect the clump properties to correlate with 
redshift, such that the more massive clumps should be found in the higher 
redshift host galaxies, since both the velocity rotation over dispersion ratio 
and the molecular gas fraction, which together control the Toomre disk stability 
criterion, have been shown to increase with redshift
\citep{wisnioski15,dessauges15}. However, no such a trend is observed, when plotting the measured clump stellar masses as a function of the redshift of their hosts.
%***DS I would suggest to leave out the fig M* versus redshift. Just say that no trend is found!
%DS - repetition, not needed
%The maximum clump masses are observed in the \citet{forster11} and \citet{guo12} host galaxies (FS sub-sample) that are the more massive hosts (as shown in the left panel), but not the higher redshift hosts.

%How such massive clumps are formed and over which timescale remains open. Recently, the numerical simulations of \citet{tamburello15} have revealed a formation mechanism of the massive clumps (with masses above the characteristic clump mass set by fragmentation) via the clump-clump mergers. As a result, more massive clumps are expected in more massive hosts, because larger numbers of clumps are formed in massive galaxies from their larger gas reservoirs that, in turn, increase the likelihood of multiple mergers and the clump mass growth. 
%If true, the observed clump stellar mass distribution would not reflect the characteristic clump mass distribution set by fragmentation.

If we assume that the simulations of T15 and \citet{mandelker17} predict 
correctly the stellar masses of the order of $\sim 10^{9-9.5}~M_{\sun}$
%the high mass tail around $10^{9-9.5}~M_{\sun}$ is not observed.} 
of the most massive clumps\footnote{This maximum clump mass is obtained for 
simulated galaxies with stellar masses up to $10^{10.6}~M_{\sun}$ (T15),
comparable to the field galaxies in the FS sub-sample ($M_*^{\rm host} \gtrsim 
10^{9.8}~M_{\sun}$). Limiting the simulations of T15 to galaxies with 
$M_*^{\rm host} < 10^{10}~M_{\sun}$ (comparable to galaxies in the FD and L 
sub-samples), leads to a clump stellar mass distribution not exceeding 
$\sim 10^{8.5}~M_{\sun}$.}, we see that still a non-negligible fraction 
%about 12\% 
%***DS is this number robust? How much does it change if you adopt $10^9$ or 9.5?
of the observed clumps in the FS sub-sample has stellar masses above the 
$10^{9.5}~M_{\sun}$ limit (Figures~\ref{fig:clump-M*} and \ref{fig:clump-host}). 
Explaining these extreme clump masses with the spatial resolution effect appears 
difficult as several very massive clumps would need to be closely clustered.
%(\citet{behrendt16} have shown that masses up to $\sim 3\times 10^9~M_{\sun}$ can be reached in artificially inflated $\sim 1$~kpc clumps).
%For instance,
In our H$\alpha$ mocks (T16), maximum stellar masses up to $\sim 3\times 
10^9~M_{\sun}$ can be reached in artificially inflated $\sim 1$~kpc clumps. 
%We have therefore examined 
When examining these extremely massive clumps from \citet{forster11} and 
\citet{guo12} individually, we find that almost all of them coincide with the 
centers of host galaxies or are located very close by, and have among the 
reddest colors. They thus appear more suggestive of galactic bulges, or alike, 
rather than genuine clumps \citep[see also][]{elmegreen09}. But, it has also been 
proposed that they could be old clumps that have migrated in the centers of 
galaxies \citep{wuyts12}. Their extreme masses remain, nevertheless, puzzling. 
%We conclude that there is no robust indication for clumps with a maximum stellar mass above $\sim 10^{9.5}~M_{\sun}$, in agreement with recent simulations. Nevertheless, 
Contributions from other processes than disk fragmentation and clump-clump 
mergers that follow, such as minor mergers or accretion of cores of disrupted 
satellites \citep[Ribeiro et~al.\ 2016, in preparation;][]{mandelker17}, can be 
an alternative way to explain star complexes with extreme masses, 
%larger than in the disk fragmentation scenario 
eventually red colors, and central galaxy positions after migration.
%therefore some giant clumps should remain extremely massive, even when better spatial resolution is achieved.

%
%_______________________________________________________________

\subsection{Is there a characteristic clump mass from observations?}
%\subsection{Determining the clump stellar mass spectrum from observations}
%\subsection{Clump stellar mass spectrum: characteristic mass versus hierarchical structure?}
\label{sect:spectrum}

%Both theoretical studies and numerical simulations have suggested the existence of a characteristic mass of clumps in high-redshift disk galaxies, which could be related to the Jeans mass, the Toomre mass, or the fragmentation mass. 
%\citep{toomre64,tamburello15}. 
%An example of a clump stellar mass distribution from the simulations of \citet{ceverino12}, predicting a high characteristic mass equivalent to the Toomre mass, is plotted in Figure~\ref{fig:clump-M*}. 
%On the observational side, 
%\citet{forster11} and \citet{guo12} have argued that the typical clump stellar masses they find (FS sub-sample) are in broad agreement with this fragmentation scale. However, this scale is predicted by linear perturbation theory, which cannot be applied at the stage when gaseous overdensities enter the nonlinear regime and form clumps (see T15).

As shown in Section~\ref{sect:results}, {\it HST} imaging has revealed 
high-redshift clumps with a wide range of stellar masses,
%covering a wide dynamical range, 
typically spread over two orders of magnitude, or significantly larger if data 
with different sensitivities and spatial resolutions are combined 
(Figure~\ref{fig:clump-M*}). Furthermore, in each clump dataset the lower 
stellar mass end is limited by the depth and spatial resolution of the available 
observations. From this we conclude that it is currently not possible to 
properly establish a meaningful clump stellar mass distribution from 
observations, and, in particular, to infer the existence and value of a 
characteristic clump mass. The only clear indication is that both improved 
sensitivity and spatial resolution shift the clump stellar mass distribution to 
lower masses that ends up to be in agreement with the latest simulations of disk 
fragmentation. Indeed, 
%the simulations from 
T15 find a characteristic clump stellar mass of $\sim 5\times 10^7~M_{\sun}$ at 
the onset of fragmentation and predict a stellar mass distribution of clumps, 
also plotted in Figure~\ref{fig:clump-M*}, which very much resembles that of 
clumps in 
%the lensed galaxy L sub-sample and in the field galaxy FD sub-sample
the L and FD sub-samples. The agreement may well be fortuitous for the reasons 
just discussed. 

If clumps are formed by disk fragmentation and molecular clouds down to several 
orders of magnitude lower mass scales are formed primarily by the same mechanism 
%as suggested by various studies 
\citep[e.g.,][]{tasker09,krumholz10}, clump formation would be hierarchical and, 
hence, one would expect clumps to continuously reveal new substructure at all 
scales \citep{elmegreen11,bournaud16}, making it impossible to assess their mass 
distribution in a resolution-independent way. 
%irrespective of the resolution. 
Observational evidence for a hierarchical star cluster structure in nearby 
galaxies is discussed by \citet{gouliermis15}. On the other end, if high-
redshift disks do possess a characteristic fragmentation mass scale as suggested 
by simulations of T15 and \citet{behrendt16}, the signature of such scale should 
be independent on spatial resolution and sensitivity once observations approach 
the corresponding scale. Convergence studies of simulations with increased 
resolution will help assess the latter mass scale robustly in the context of the 
fragmentation scenario.
%Simulations also need to better address the dependence of disk stability and fragmentation on feedback models and external triggers such as satellite perturbations and mass accretion, very frequent at high resdhift, which can both be crucial (Mayer et al. 2016). 
At the same time, larger high-redshift clump samples within deep observations, 
ideally at the best-possible spatial resolution, and a systematic analysis (with 
the same clump selection criteria), including completeness corrections, are 
needed to establish the true clump stellar mass spectrum. 
%and any possible trends with host galaxy properties.
%It is also possible that the clump formation is hierarchical and thus scale-free, as suggested for the star cluster formation in nearby galaxies \citep[see][]{gouliermis15}.

%
%______________________________________________________________

%\section{Conclusions}

%We have compiled a sample of 242 star-forming clumps at $1.0<z<3.6$ from the literature with available multi-band HST photometry. We subdivide these clumps in three sub-samples: the lensed clumps, the faint \citet{elmegreen13} clumps, and the field/non-lensed usual clumps. Surprisingly, we find that the stellar masses of the high-redshift clumps identified in the lensed plus \citet{elmegreen13} host galaxies and the field/non-lensed host galaxies, respectively, are not drawn from the same population at 100 per cent c.l. The lensed and \citet{elmegreen13} clumps probe star-forming regions with about two orders of magnitude lower stellar masses than the conventional field clumps, resembling more their local star cluster counterparts observed in the most intensively star-forming nearby galaxies. The resulting stellar mass mean is nicely representative of the characteristic fragmentation stellar mass expected by the very recent galactic disk gravitational instability simulations from T15 and \citet{behrendt16}. Nevertheless, because of severe clump sample incompleteness, we may hardly observationally constrain the clump stellar mass spectrum. Moreover, evidence for a hierarchical clump structure cannot be excluded. In summary, this illustrates the strong effects of spatial resolution and/or sensitivity of the data on the accessible clump stellar mass. In this picture, giant clumps yield unrealistic far too high masses, resulting from the blending of smaller and fainter subunits artificially smeared out owing to the lack of resolution and/or sensitivity. On the other hand, the lensed and \citet{elmegreen13} clumps tend to be biased against clumps with high stellar masses, because of the low stellar masses of their host galaxies. In presence of more massive host galaxies, massive clumps are formed. They are not produced during the disk fragmentation phase, but are post-processed during the galaxy lifetime in the multiple clump-clump mergers. Nonetheless, the most massive field clumps, more massive than $10^{9.5}~M_{\sun}$ tend preferably to be suggestive of galactic bulges or alike. Consequently, the inferred unlensed clump stellar masses do not reflect the intrinsic clump physical properties.

%
%______________________________________________________________

\acknowledgments

This work was supported by the Swiss National Science Foundation, in the context 
of the Sinergia STARFORM network on ``Star formation in galaxies from the 
Milky Way to the distant Universe''. We are grateful to Bruce G. Elmegreen and 
Yicheng Guo for sharing with us their {\it HST} clump photometry, and we warmly 
thank Bruce G. Elmegreen for very fruitful discussions.

%
%______________________________________________________________

\begin{thebibliography}{}

\bibitem[Adamo et al.(2013)]{adamo13}
Adamo, A., \"Ostlin, G., Bastian, N., et~al.\ 2013, \apj, 766, 105

\bibitem[Agertz et al.(2009)]{agertz09}
Agertz, O., Teyssier, R., \& Moore, B.\ 2009, \mnras, 397, L64

\bibitem[Bastian et al.(2006)]{bastian06}
Bastian, N., Emsellem, E., Kissler-Patig, M., \& Maraston, C.\ 2006, \aap, 445, 471

\bibitem[Beckwith et al.(2006)]{beckwith06}
Beckwith, S.~V.~W., Stiavelli, M., Koekemoer, A.~M., et~al.\ 2006, \aj, 132, 1729

\bibitem[Behrendt et al.(2016)]{behrendt16}
Behrendt, M., Burkert, A., \& Schartmann, M.\ 2016, \apjl, 819, L2

\bibitem[Boley et al.(2010)]{boley10}
Boley, A.~C., Hayfield, T., Mayer, L., \& Durisen, R.~H.\ 2010, \icarus, 207, 509

\bibitem[Bournaud et al.(2010)]{bournaud10}
Bournaud, F., Elmegreen, B.~G., Teyssier, R., Block, D.~L., \& Puerari, I.\ 2010, \mnras, 409, 1088

\bibitem[Bournaud et al.(2014)]{bournaud14}
Bournaud, F., Perret, V., Renaud, F., et~al.\ 2014, \apj, 780, 57

\bibitem[Bournaud et al.(2016)]{bournaud16}
Bournaud, F.\ 2016, \apss, 418, 355

\bibitem[Bruzual \& Charlot(2003)]{bruzual03}
Bruzual, G., \& Charlot, S.\ 2003, \mnras, 344, 1000

%\bibitem[Buitrago et al.(2008)]{buitrago08}
%Buitrago, F., Trujillo, I., Conselice, C. J., et al. 2008, ApJL, 687, L61

\bibitem[Ceverino et al.(2010)]{ceverino10}
Ceverino, D., Dekel, A., \& Bournaud, F.\ 2010, \mnras, 404, 2151

\bibitem[Ceverino et al.(2012)]{ceverino12}
Ceverino, D., Dekel, A., Mandelker, N., et~al.\ 2012, \mnras, 420, 3490

\bibitem[Chabrier(2003)]{chabrier03}
Chabrier, G.\ 2003, \pasp, 115, 763

%\bibitem[Daddi et al.(2007)]{daddi07}
%Daddi, E., Dickinson, M., Morrison, G., et~al.\ 2007, ApJ, 670, 156

\bibitem[Dekel et al.(2009a)]{dekel09a}
Dekel, A., Birnboim, Y.,  Engel, G., et~al.\ 2009a, \nat, 457, 451

\bibitem[Dekel et al.(2009b)]{dekel09b}
Dekel, A., Sari, R., \& Ceverino, D.\ 2009b, \apj, 703, 785

%\bibitem[Dessauges-Zavadsky et al.(2011)]{dessauges11}
%Dessauges-Zavadsky, M., Christensen, L., D?Odorico, S., Schaerer, D., \& Richard, J. 2011, A\&A, 533, A15

\bibitem[Dessauges-Zavadsky et al.(2015)]{dessauges15}
Dessauges-Zavadsky, M., Zamojski, M., Schaerer, D., et~al.\ 2015, \aap, 577, A50

\bibitem[Elmegreen et al.(2005)]{elmegreen05}
Elmegreen, D.~M., Elmegreen, B.~G., Rubin, D.~S., \& Schaffer, M.~A.\ 2005, \apj, 631, 85

\bibitem[Elmegreen et al.(2007)]{elmegreen07}
Elmegreen, D.~M., Elmegreen, B.~G., \& Coe, D.~A.\ 2007, \apj, 658, 763

\bibitem[Elmegreen et al.(2009)]{elmegreen09}
Elmegreen, B.~G., Elmegreen, D.~M., Fernandez, M.~X., \& Lemonias, J.~J.\ 2009, \apj, 692, 12

\bibitem[Elmegreen(2011)]{elmegreen11}
Elmegreen, B.~G.\ 2011, \apj,  737, 10

\bibitem[Elmegreen et al.(2013)]{elmegreen13}
Elmegreen, B.~G., Elmegreen, D.~M., S\'anchez, A.~J., et~al.\ 2013, \apj, 774, 86

\bibitem[Escala \& Larson(2008)]{escala08}
Escala, A., \& Larson, R.~B.\ 2008, \apjl, 685, L31

%\bibitem[Feldmann \& Mayer(2015)]{feldmann15}
%Feldmann, R., \& Mayer, L. 2015, MNRAS, 446, 1939

\bibitem[Fisher et al.(2017)]{fisher17}
Fisher, D.~B., Glazebrook, K., Damjanov, I., et~al.\ 2017, \mnras, 464, 491

\bibitem[F\"orster Schreiber et al.(2009)]{forster09}
F\"orster Schreiber, N.~M., Genzel, R., Bouch\'e, N., et~al.\ 2009, \apj, 706, 1364

\bibitem[F\"orster Schreiber et al.(2011)]{forster11}
F\"orster Schreiber, N.~M., Shapley, A.~E., Genzel, R., et~al.\ 2011, \apj, 739, 45

\bibitem[Genel et al.(2012)]{genel12}
Genel, S., Naab, T., Genzel, R., et~al.\ 2012, \apj, 745, 11

%\bibitem[Genzel et al.(2006)]{genzel06}
%Genzel, R., Tacconi, L.~J., Eisenhauer, F., et~al.\ 2006, \nat, 442, 786

%\bibitem[Genzel et al.(2011)]{genzel11}
%Genzel, R., Newman, S., Jones, T., et~al.\ 2011, ApJ, 733, 101

%\bibitem[\protect\citeauthoryear{Genzel et al.}{2015}]{genzel15}
%Genzel, R., Tacconi, L.~J., Lutz, D., et~al.\ 2015, \apj, 800, 20

\bibitem[Gouliermis et al.(2015)]{gouliermis15}
Gouliermis, D.~A., Thilker, D., Elmegreen, B.~G., et~al.\ 2015, \mnras, 452, 3508

%\bibitem[Guedes et al.(2011)]{guedes11}
%Guedes, J., Callegari, S., Madau, P., \& Mayer, L. 2011, ApJ, 742, 76

\bibitem[Guo et al.(2012)]{guo12}
Guo, Y., Giavalisco, M., Ferguson, H.~C., Cassata, P., \& Koekemoer, A.~M.\ 2012, \apj, 757, 120

\bibitem[Guo et al.(2015)]{guo15}
Guo, Y., Ferguson, H.~C., Bell, E.~F., et~al.\ 2015, \apj, 800, 39

%\bibitem[Jones et al.(2010)]{jones10}
%Jones, T.~A., Swinbank, A.~M., Ellis, R.~S., Richard, J., \& Stark, D.~P. 2010, MNRAS, 404, 1247

\bibitem[Kere{\v s} et al.(2005)]{keres05}
Kere\v s, D., Katz, N., Weinberg, D.~H., \& Dav\'e, R.\ 2005, \mnras, 363, 2

\bibitem[Krumholz \& Burkert(2010)]{krumholz10}
Krumholz, M., \& Burkert, A.\ 2010, \apj, 724, 895

\bibitem[Larsen et al.(2002)]{larsen02}
Larsen, S.~S., Efremov, Y.~N., Elmegreen, B.~G., et~al.\ 2002, \apj, 567, 896

%\bibitem[Livermore et al.(2012)]{livermore12}
%Livermore, R.~C, Jones, T., Richard, J., et~al.\ 2012, MNRAS, 427, 688

%\bibitem[Livermore et al.(2015)]{livermore15}
%Livermore, R.~C, Jones, T., Richard, J., et~al.\ 2015, MNRAS, 450, 1812

\bibitem[Mandelker et al.(2014)]{mandelker14}
Mandelker, N., Dekel, A., Ceverino, D., et~al.\ 2014, \mnras, 443, 3675

\bibitem[Mandelker et al.(2017)]{mandelker17}
Mandelker, N., Dekel, A., Ceverino, D., et~al.\ 2017, \mnras, 464, 635

\bibitem[Mayer \& Gawrysczak(2008)]{mayer08}
Mayer, L. \& Gawrysczak, A.~J.\ 2008, in ASPC 398, Extreme Solar Systems, ed. D. Fischer, F.~A. Rasio, S.~E. Thorsett, \& A. Wolszczan, 243

%\bibitem[Mayer et al.(2016)]{mayer16}
%Mayer, L., Tamburello, V., Lupi, A., et al.\ 2016, \apjl, 830, L13

%\bibitem[Noeske et al.(2007)]{noeske07}
%Noeske, K.~G., Weiner, B.~J., Faber, S.~M., et~al.\ 2007, ApJL, 660, L43

\bibitem[Ocvirk et al.(2008)]{ocvirk08}
Ocvirk, P., Pichon, C., \& Teyssier, R.\ 2008, \mnras, 390, 1326

\bibitem[Oklop\v{c}i\'{c} et al.(2017)]{oklopcic17}
Oklop\v{c}i\'{c}, A., Hopkins, P.~F., Feldmann, R., et~al.\ 2017, \mnras, 465, 952

\bibitem[Postman et al.(2012)]{postman12}
Postman, M., Coe, D., Ben\'{i}tez, N., et~al.\ 2012, \apjs, 199, 25  

%\bibitem[Rodighiero et al.(2010)]{rodighiero10}
%Rodighiero, G., Cimatti, A., Gruppioni, C., et~al.\ 2010, \aap, 518, L25

\bibitem[Rodrigues et al.(2016)]{rodrigues16}
Rodrigues, M., Hammer, F., Flores, H., Puech, M., \& Athanassoula, E.\ 2016, arXiv:1611.03499, \mnras, submitted

%\bibitem[Rozas et al.(2006)]{rozas06}
%Rozas, M., Richer, M.~G., L\'opez, J.~A., Rela\~no, M., \& Beckman, J.~E. 2006, A\&A, 455, 539

%\bibitem[Saintonge et al.(2013)]{saintonge13}
%Saintonge, A., Lutz, D., Genzel, R., et~al.\ 2013, \apj, 778, 2

%\bibitem[\protect\citeauthoryear{Schaerer \& de Barros}{2009}]{schaerer09}
%Schaerer D., de Barros S. 2009, A\&A, 502, 423

\bibitem[Schaerer \& de Barros(2010)]{schaerer10}
Schaerer, D., \& de Barros, S.\ 2010, \aap, 515, A73

\bibitem[Sharon et al.(2012)]{sharon12}
Sharon, K., Gladders, M.~D., Rigby, J.~R., et~al.\ 2012, \apj, 746, 161 

%\bibitem[Shibuya et al.(2015)]{shibuya15}
%Shibuya, T., Ouchi, M., \& Harikane, Y. 2015, ApJS, 219, 15

\bibitem[Shibuya et al.(2016)]{shibuya16}
Shibuya, T., Ouchi, M., Kubo, M., \& Harikane, Y.\ 2016, \apj, 821, 72

%\bibitem[Swinbank et al.(2012)]{swinbank12}
%Swinbank, A.~M., Smail, I., Sobral, D., et~al.\ 2012, ApJ, 760, 130 

\bibitem[Tamburello et al.(2015)]{tamburello15}
Tamburello, V., Mayer, L., Shen, S., \& Wadsley, J.\ 2015, \mnras, 453, 2490 (T15)

\bibitem[Tamburello et al.(2016)]{tamburello16}
Tamburello, V., Rahmati, A., Mayer, L., et~al.\ 2016, arXiv:1610.05304, \mnras, submitted (T16)

\bibitem[Tasker \& Tan(2009)]{tasker09}
Tasker, E.~J., \& Tan, J.~C.\ 2009, \apj, 700, 358

\bibitem[Toomre(1964)]{toomre64}
Toomre, A.\ 1964, \apj, 139, 1217

%\bibitem[Wisnioski et al.(2012)]{wisnioski12}
%Wisnioski, E., Glazebrook, K., Blake, C., et~al.\ 2012, MNRAS, 422, 3339

\bibitem[Wisnioski et al.(2015)]{wisnioski15}
Wisnioski, E., F\"orster Schreiber, N.~M., Wuyts, S., et~al.\ 2015, \apj, 799, 209

\bibitem[Wuyts et al.(2012)]{wuyts12}
Wuyts, S., F\"orster Schreiber, N.~M., Genzel, R., et~al.\ 2012, \apj, 753, 114 

\bibitem[Wuyts et al.(2014)]{wuyts14}
Wuyts, E., Rigby, J.~R., Gladders, M.~D., \& Sharon, K.\ 2014, \apj, 781, 61

\end{thebibliography}

%% This command is needed to show the entire author+affilation list when
%% the collaboration and author truncation commands are used.  It has to
%% go at the end of the manuscript.
%\allauthors

%% Include this line if you are using the \added, \replaced, \deleted
%% commands to see a summary list of all changes at the end of the article.
%\listofchanges

\end{document}

% End of file `sample61.tex'.
