\begin{figure*}[!t]
  \centering \includegraphics[width=1.00\linewidth]{figures/Fig_Sources_of_Noise_2col.png}
  \caption{\textbf{Comparison of noise sources in different image sensor pixels:}
	(a) Theoretical expressions for the three main sources of noise affecting a
conventional pixel, PF-SPAD pixel \cite{ingle2019high} and the proposed IP-SPAD
pixel are summarized in this table. Note that the IP-SPAD's sources of noise are similar
to a PF-SPAD except for quantization noise. (b) The expressions in (a) are plotted for
the case of $T=\SI{5}{\ms}$, $q=100\%$, $\sigma_r=5 e^-$, $\Phi_\text{dark}=10$ photons/second,
$\taud = \SI{150}{\ns}$, $\Delta=\SI{200}{\ps}$. The conventional sensor's saturation capacity
is set at 34,000 $e^-$ which matches the maximum possible SPAD counts of
$\lceil \nicefrac{T}{\taud} \rceil$.
Observe that the IP-SPAD soft-saturation point is at a much higher flux level
than the PF-SPAD. \label{fig:sources_of_noise}}
\end{figure*}


\section{Image Formation with Inter-Photon Timing}
\subsection{Flux Estimator \label{sec:flux_estimator}}
Consider a single IP-SPAD pixel passively capturing photons over a fixed
exposure time $T$ from a scene point\footnote{We assume that there is no scene
or camera motion so that the flux $\Phi$ stays constant over the exposure time
$T$.} with true photon flux of $\Phi$ photons per second.  After each photon
detection event, the IP-SPAD pixel goes blind for a fixed duration $\taud$
called the dead-time. During this dead-time, the pixel is reset and the pixel's
time-to-digital converter (TDC)
circuit stores a picosecond resolution timestamp of the most recent
photon detection time, and also increments the total photon count. This process
is repeated until the end of the exposure time $T$. Let $N_T \geq 2$ denote the
total number of photons detected by the IP-SPAD pixel during its fixed exposure
time, and let $X_i$ $(1\leq i \leq N_T)$ denote the timestamp of the
$i^\text{th}$ photon detection.  The measured inter-photon times
between successive photons is defined as $Y_i := X_{i+1} -
X_{i} - \taud$ (for $1 \leq i \leq N_T-1$). Note that $Y_i$'s follow an exponential
distribution. It is tempting to derive a closed-from maximum
likelihood photon flux estimator $\widehat\Phi$ for the true flux $\Phi$
using the log-likelihood function of the measured inter-photon times $Y_i$:
\begin{align}
\vspace{-0.15in}
  \log l(q\Phi; Y_1,\ldots,Y_{N_T-1}) = \log \left( \prod_{i=1}^{N_T-1}
                     q\Phi\,e^{-q\Phi Y_i} \right) \nonumber \\
  = - q\Phi \left( \sum_{n=1}^{N_T-1} Y_i \right) + (N_T-1)
                    \log q\Phi, \label{eq:loglik}
\vspace{-0.15in}
\end{align}
where $0\!\!<q<\!\!1$ is the quantum efficiency of the IP-SPAD pixel.
The maximum likelihood estimate $\widehat\Phi$ of the true photon flux is
computed by setting the derivative of Eq.~(\ref{eq:loglik}) to zero and
solving for $\Phi$:
\begin{equation}
\boxed{
  \widehat{\Phi} = \frac{1}{q}\frac{N_T-1}{X_{N_T}-X_1-(N_T-1)\taud}.
  \label{eq:flux_estimator}
}
\end{equation}

Although the above proof sketch captures the intuition of our flux
estimator, it leaves out two details. First, the total number of photons
$N_T$ is itself a random variable. Second, the times of capture of
future photons are constrained by the timestamps of preceding photon arrivals
because we operate in a finite exposure time $T$. The sequence of timestamps $Y_i$
cannot be treated as independent and identically distributed. \atul{In fact, it is
possible to show that the sequence of timestamps forms a Markov chain
\cite{rapp2019dead} where the conditional distribution of the $i^\text{th}$
inter-photon time conditioned on the previous inter-photon times is given by}:
\begin{equation}
p_{Y_i | Y_{1},\dots,Y_{i-1}}(t)=\begin{cases}
q\Phi e^{-q\Phi t} & 0< t< T_i\\
e^{-q\Phi T_i} \delta(t-T_i) & t = T_i\\
0 & \mbox{otherwise.}
\end{cases}\nonumber 
\end{equation}
Here $\delta(\cdot)$ is the Dirac delta function.  The $T_i$'s model the
shrinking effective exposure times for subsequent photon detections.
$T_1 = T$ and for $i>1$, 
$T_i = \max(0,T_{i-1} - Y_{i-1}-\taud)$.
The log-likelihood function can now be written as:
\begin{align*}
  \log l(q\Phi; Y_1,\ldots,Y_{L}) = \log \left( \prod_{i=1}^{\lceil\nicefrac{T}{\taud}\rceil} p_{Y_i | Y_{1},\dots,Y_{i-1}}(t) \right) \\
  \!\!= - q\Phi \max\left(\sum_{i=1}^{N_T}Y_i,T\!-\!N_T\taud\right) + N_T \log q\Phi.
\end{align*}
As shown in \nolink{\ref{suppl:MLE_conditional_derivation}} this likelihood function also
leads to the flux estimator given in Eq.~(\ref{eq:flux_estimator}).

We make the following key observations about the IP-SPAD flux estimator.
First, note that the estimator is only a function of the first and the last
photon timestamps, the exact times of capture of the intermediate photons do
not provide additional information.\footnote{As we show later in our hardware
implementation, in practice, it is useful to capture intermediate photon
timestamps as they allow us to calibrate for various pixel non-idealities.}
This is because photon arrivals follow Poisson statistics: the time until the
next photon arrival from the end of the previous dead-time is independent of
all preceding photon arrivals. Secondly, we note that the denominator in
Eq.~(\ref{eq:flux_estimator}) is simply the total time the IP-SPAD spends
waiting for the next photon to be captured while not in dead-time. Intuitively,
if the photon flux were extremely high, we will expect to see a photon
immediately after every dead-time duration ends, implying the denominator in
Eq.~(\ref{eq:flux_estimator}) approaches zero, hence $\widehat{\Phi}
\rightarrow \infty$.

% Please add the following required packages to your document preamble:
% \usepackage{booktabs}
% \usepackage{multirow}
%\begin{table}[]
%\begin{tabular}{@{}|c|c|c|c|c|c|c|@{}}
%\toprule
%\multirow{2}{*}{\textbf{Noise Sources}} & \multicolumn{2}{c|}{\textbf{CMOS}}                                                                             & \multicolumn{2}{c|}{\textbf{PF-SPAD}}                                        & \multicolumn{2}{c|}{\textbf{IP-SPAD [Proposed]}}                                                                        \\ \cmidrule(l){2-7} 
%                                        & \textbf{Bias} & \textbf{Variance}                                                                              & \textbf{Bias}      & \textbf{Variance}                                       & \textbf{Bias}      & \textbf{Variance}                                                                                  \\ \midrule
%Shot noise                              & -             & $\frac{\Phi}{qT}$                                                                              & -                  & $\frac{\Phi (1+q\Phi \taud)}{qT}$                        & -                  & $\frac{\Phi (1+q\Phi \taud)}{qT}$                                                                   \\
%Quantization noise                      & -             & \begin{tabular}[c]{@{}c@{}}$\approx 0$ if not saturated\\ $\infty$, if saturated.\end{tabular} & -                  & $\frac{(1+q\Phi\taud)^4}{12 q^2 T^2}$                   & -                  & $\frac{(1+q\Phi(\taud+\Delta))^2(1+q\Phi\Delta)^2}{12 q^2 T^2}$                                    \\
%Dark noise                              & -             & $\frac{\sig_r^2}{q^2 T^2}$                                                                     & $\Phi_\text{dark}$ & $\frac{\Phi_\text{dark} (1+q\Phi_\text{dark}\taud)}{qT}$ & $\Phi_\text{dark}$ & \begin{tabular}[c]{@{}c@{}}$\frac{\Phi_\text{dark} (1+q\Phi_\text{dark} \taud)}{qT}$\end{tabular} \\ \bottomrule
%\end{tabular}
%\end{table}

\begin{figure}[!t]
	\centering \includegraphics[width=0.95\columnwidth]{figures/timing_quantization.png}
	\caption{\textbf{Advantage of using photon timing over photon counts:}
	(a) Photon counts are inherently discrete. At high flux levels, even a
	small $\pm 1$ change in photon counts corresponds to a large flux uncertainty.
	(b) Inter-photon timing is inherently continuous. This leads to smaller
	uncertainty at high flux levels. The uncertainty depends on jitter and
	floating point resolution of the timing electronics. \label{fig:timing_quantization_noise}}
	\vspace{-0.15in}
\end{figure}

\begin{figure}[!t]
\centering \includegraphics[width=\columnwidth]{figures/Fig2_SNR.png}
\caption{\textbf{Effect of various IP-SPAD parameters on SNR:} We vary
different parameters to see the effect on SNR. The solid lines are theoretical
SNR curves while each dot represents a SNR average from 10 Monte Carlo
simulations. Unless otherwise noted the parameters used are $T=1$~\si{\ms},
$\taud=100$~\si{\ns}, $q=0.4$, and $\Delta=0$.  (a) As exposure time increases,
SNR increases at all brightness levels. (b) Decreasing the dead-time increases
the maximum achievable SNR, but provides little benefit in low flux.
(c) Coarser time quantization causes SNR drop-off at high flux values. (d) Our
IP-SPAD flux estimator outperforms a counts-only (PF-SPAD) flux estimator
\cite{ingle2019high} at high flux levels.\label{fig:snr}}
\vspace{-0.15in}
\end{figure}

\subsection{Sources of Noise}
Although, theoretically, the IP-SPAD scene brightness estimator in
Eq.~(\ref{eq:flux_estimator}) can recover the entire range of incident photon
flux levels, including very low and very high flux values, in practice, the
accuracy is limited by various sources of noise. To assess the performance of
this estimator, we use a signal-to-noise ratio (SNR) metric defined as
\cite{yang2011bits,ingle2019high}: 
\begin{equation}
  \mathsf{SNR}(\Phi) = 10 \log_{10} \left( \frac{\Phi^2}{\mathbf{E}[(\Phi-\widehat\Phi)^2]}\right) \label{eq:snr}
\end{equation}
Note that the denominator in Eq.~(\ref{eq:snr}) is the mean-squared-error
of the estimator $\widehat\Phi$, and is equal to the sum of the bias-squared
terms and variances of the different sources of noise.
The \emph{dynamic range} (DR) of the sensor is defined as the range of
brightness levels for which the SNR stays above a minimum specified threshold.
At extremely low flux levels, the dynamic range is limited due to IP-SPAD dark
counts which causes spurious photon counts even when no light is incident
on the pixel. This introduces a bias in $\widehat\Phi$. Since photon arrivals
are fundamentally stochastic (due to the quantum nature of light), the
estimator also suffers from Poisson noise which introduces a non-zero variance
term. Finally, at high enough flux levels, the time discretization $\Delta$
used for recording timestamps with the IP-SPAD pixel limits the maximum usable
photon flux at which the pixel can operate. Fig.~\ref{fig:sources_of_noise}(a)
shows the theoretical expression for bias and variance introduced by shot noise,
quantization noise and dark noise in an IP-SPAD pixel along with corresponding
expressions for a conventional image sensor pixel and a PF-SPAD pixel.
Fig.~\ref{fig:sources_of_noise}(b) shows example plots for these theoretical
expressions. For realistic values of $\Delta$ in the 100's of picoseconds range,
the IP-SPAD pixel has a smaller quantization noise term that allows reliable
brightness estimation at much higher flux levels than a PF-SPAD pixel. (See
\nolink{\ref{suppl:quantization}}). 


\begin{figure}[!ht]
  \centering \includegraphics[width=0.90\linewidth]{figures/Fig3_simulated_HDR_pf_ip.png}
  \caption{\textbf{Simulated HDR scene captured with a PF-SPAD (counts only)
  vs. IP-SPAD (inter-photon timing):}
  (a) Although a PF-SPAD can capture this extreme dynamic range scene in a
  single \SI{5}{\ms} exposure, extremely bright pixels such as the bulb
  filament that are beyond the soft-saturation limit appear noisy.
  (b) An IP-SPAD camera captures both dark and bright regions in a single
  exposure, including fine details around the bright bulb filament.
  In both cases, we set the SPAD pixel's quantum efficiency to 0.4, dead-time
  to \SI{150}{\ns} and an exposure time of \SI{5}{\ms}. The IP-SPAD has a time
  resolution of $\Delta=\SI{200}{\ps}$.  (Original image from
  HDRIHaven.com)\label{fig:sim_hdr}}
\end{figure}

%\begin{table}[!ht]
%\begin{center}
%\begin{tabular}{ccc}
%\cline{1-3}
%Source of noise & Bias  & Variance  \\ \cline{1-3}
%Dark counts & $\Phi_\text{dark}$ & $\frac{\Phi_\text{dark} (1+q\Phi_\text{dark}\taud)}{qT}$ \\
%Poisson noise & -  & $\frac{\Phi (1+q\Phi\taud)}{qT}$\\
%Time-discretization & -  & $\frac{(1+q\Phi (\taud + \Delta))^2(1+q\Phi \Delta)^2}{12 q^2 T^2}$\\ \cline{1-3}
%\end{tabular}
%\end{center}
%\caption{{\bf Bias and variance due IP-SPAD noise:} The brightness estimator
%for an IP-SPAD pixel suffers from three main sources of noise. The expected
%estimator mean squared error (denominator in the SNR formula) is given by
%the sum of the squared bias terms and variance expressions.
%\label{tab:sources_of_noise}}
%\end{table}

\smallskip
\noindent{\bf Quantization Noise in PF-SPAD vs. IP-SPAD:}
Conventional pixels are affected by quantization in low flux and hard
saturation (full-well capacity) limit in high flux.  In contrast, a PF-SPAD
pixel that only uses photon counts is affected by quantization noise \emph{at
extremely high flux levels due to soft-saturation} \cite{ingle2019high}. This
behavior is unique to SPADs and is quite different from conventional sensors. A
counts-only PF-SPAD pixel can measure at most $\lceil
\nicefrac{T}{\tau_\text{d}}\rceil$ photons where $T$ is the exposure time and
$\tau_\text{d}$ is the dead-time \cite{ingle2019high}.  Due to a non-linear
response curve, as shown in Fig.~\ref{fig:timing_quantization_noise}(a), a
small change of $\pm 1$ count maps to a large range of flux values. Due to the
inherently discrete nature of photon counts, even a small fluctuation (due to
shot noise or jitter) can cause a large uncertainty in the estimated flux.

% mention that counts are a fundamental limitation
% they are quantized. But timing could be improved
% as technology improves.

The proposed IP-SPAD flux estimator uses timing information which is inherently
continuous.  Even at extremely high flux levels, photon arrivals are random and
due to small random fluctuations, the time interval between the first and last
photon ($X_{N_T}-X_1$) is not exactly equal to $T$.
Fig.~\ref{fig:timing_quantization_noise}(b) shows the intuition for why
fine-grained inter-photon measurements at high flux levels can enable flux
estimation with a smaller uncertainty than counts alone. In practice, the
improvement in dynamic range compared to a PF-SPAD depends on the time
resolution, which is limited by hardware constraints like floating point
precision of the TDC electronics and timing jitter of the SPAD pixel.
Simulations in Fig.~\ref{fig:snr} suggest that even with a \SI{100}{\ps} time
resolution the 20-dB dynamic range improves by 2 orders of magnitude over using
counts alone.

\smallskip
\noindent{\bf Single-Pixel Simulations:}
We verify our theoretical SNR expression using single-pixel Monte Carlo
simulations of a single IP-SPAD pixel. For a fixed set of parameters we run 10
simulations of an IP-SPAD at 100 different flux levels ranging $10^4-10^{16}$
photons per second. Fig. \ref{fig:snr} shows the effect of various pixel
parameters on the SNR. The overall SNR can be increased by either
increasing the exposure time $T$ or decreasing the dead-time $\taud$;
both enable the IP-SPAD pixel to capture more total
photons. The maximum achievable SNR is theoretically equal to
$10\log_{10}\left(\nicefrac{T}{\taud}\right)$. The IP-SPAD
SNR degrades at high flux levels due because photon timestamps cannot be
captured with infinite resolution.  A larger floating point quantization bin
size $\Delta$ increases the uncertainty in photon timestamps. If the time bin
is large enough, there is no advantage in using the timestamp-based brightness
estimator and the performance reverts to a counts-based flux estimator
\cite{Antolovic_2018, ingle2019high}. 

%\begin{figure}[!htb]
%  \centering
%  \includegraphics[width=1.00\linewidth]{figures/Fig4_firstphoton_Small.png}
%  \caption{\textbf{Denoising passive first-photon images:} We simulate
%  the extreme case of a single photon timestamp per image pixel.  (a) Ground
%  truth image.  (b) Simulated data consisting of a single photon timestamp per
%  pixel.  (c) Inverting the each timestamp gives an extremely noisy image. (d)
%  A KPN denoiser trained on photon timestamp information, is able to
%  reconstruct bright regions and coarse object structures invisible
%  in the raw data. \label{fig:passive_first_photon_small}}
%\end{figure}



%It turns out that single-photon sensors have a free-running mode of operation:
%after each photon detection the pixel enters a recharge phase where the
%electronics are reset and then the pixel becomes ready to capture the next
%photon. The on-times are not driven completely by previous photon detections.
%The flux estimator is now given simply by: \[ \widehat \Phi = \frac{1}{\bar T -
%\tau_d} \] where $\tau_d$ is the dead-time of the pixel (and electronics)
%needed to reset it for the next photon detection.
%
%\begin{itemize}
%    \item Show simulated results for PF-SPAD with timing
%    \item Experimental hardware results
%\end{itemize}

%Maybe start with the normal multiple photons per pixel case (fixed exposure
%time). Derive per pixel flux estimator. Show HDR property.  Now start reducing
%number of photons/exposure time. Images start looking noisier. Use a DNN to
%denoise. What architecture to use?  Look up standard denoising architectures.


%\begin{equation*}
%  \widehat{\Phi}^*=\frac{N_T}{q\max(X_1+\sum_{n=2}^{N_T}X_n-X_{n-1}-\tau_d,T-N_T\tau_d)}
%\end{equation*}
%
%Note that the $\max(\cdot,\cdot)$ function takes value of exactly the time of
%darkness. The left argument is larger only if a dead time window overlaps with
%the end of the exposure, which happens with very high probability at high flux.
%At low flux the the max function will likely take the right hand argument and
%this estimator is equal to the counts based estimator of Ingle et
%al\cite{ingle2019high}. This implies that in the low flux regime a counts and
%timing based flux estimator are equivalent, but at high flux there is important
%information in photon timing. 
%
%In practice the the start and end of the exposure time may not be known
%precisely, so there may be inaccuracies in $X_1$ and $T$ leading to inaccurate
%flux estimations.  In this case, we can instead consider the time period
%between the first and last photon detection to be our exposure time reducing
%the maximum likelihood estimator to exactly Eq.~\ref{eq:flux_estimator}. 


%
%\begin{equation}
%  \widehat{\Phi} = \begin{cases}
%    \frac{1}{q}\frac{N_T}{T-N_T \tau_d} \text{ if } N_T < \frac{T}{2\tau_d},\\
%    \frac{1}{q}\frac{N_T-1}{X_{N_T} - X_1 - (N_T-1)\tau_d}.
%  \end{cases}
%\end{equation}




