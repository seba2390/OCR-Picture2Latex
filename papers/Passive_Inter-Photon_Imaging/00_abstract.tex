\begin{abstract}
%  Image sensors capable of detecting individual photons are used in many active
%  imaging systems in synchronization with a light source such as a pulsed
%  laser. Single-photon avalanche diodes (SPADs) are driving a revolution in
%  active imaging applications like 3D time-of-flight cameras and fluorescence
%  microscopy due to their unique ability to capture individual photons with
%  picosecond timing resolution. Here we show that the high resolution timing
%  information captured by a single-photon sensor encodes scene intensity
%  information and can be used to estimate photon flux completely passively,
%  without any active light source or pulsed laser. We derive a brightness
%  estimator that uses the decay time of the inter-photon timing statistics 
%  captured by a dead-time-limited single-photon sensor pixel. This brightness
%  estimator works over a wide range of photon flux levels: at extremely low
%  flux levels, down to a single photon timestamp per pixel to extremely high
%  flux levels well beyond the saturation limit of conventional image sensor
%  pixels. We experimentally demonstrate capturing scenes with a dynamic range
%  of over ten million to 1. This will have implications for a wide range of
%  applications: consumer photography, astronomy, microscopy and biomedical
%  imaging.

% ATUL: I wrote this from scratch after a long break, but I still can't make it
% not sound like the CVPR paper abstract from 2019 - HELP!

%method that circumvents these limitations by estimating
%scene brightness based on a  different characteristic of light---the random
%distribution of photons over time. 

%, from just one photon timestamp per pixel to extremely high photon flux levels, beyond the saturation limit of conventional sensors. W

Digital camera pixels measure image intensities by converting incident
light energy into an analog electrical current, and then digitizing it
into a fixed-width binary representation.  This direct measurement method,
while conceptually simple, suffers from limited dynamic range and poor
performance under extreme illumination --- electronic noise dominates under
low illumination, and pixel full-well capacity results in saturation
under bright illumination. We propose a novel intensity cue based on
measuring \textnormal{inter-photon~timing}, defined as the time delay between
detection of successive photons. Based on the statistics of inter-photon
times measured by a time-resolved single-photon sensor, we develop theory and
algorithms for a scene brightness estimator which works over extreme dynamic
range; we experimentally demonstrate imaging scenes with a dynamic range of
over ten million to one. The proposed techniques, aided by the emergence of
single-photon sensors such as single-photon avalanche diodes (SPADs) with
picosecond timing resolution, will have implications for a wide range of
imaging applications: robotics, consumer photography, astronomy, microscopy
and biomedical imaging.

\end{abstract}

% photon are quantized - it doesn't make
% sense to capture it in analog and then
% digitize it later.
% mention quanta image sensors - that is
% counting photons.
% but SPAD sensors have high timing
% resolution.

% How many photons
% do you need to capture an image? Conventional image sensors need hundreds or
% thousands of photons per pixel to produce any meaningful signal. However,
% with an emerging class of image sensors that are sensitive to individual
% photons it is now possible to capture images in extreme darkness.
% need to motiviate the timing idea but not repeat the cvpr abstract - a bit tricky here!
% \keywords{single-photon cameras, SPAD, high-dynamic-range}

