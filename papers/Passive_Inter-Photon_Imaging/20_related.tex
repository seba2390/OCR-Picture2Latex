\section{Related Work}
\noindent\textbf{High-Dynamic-Range Imaging:}
Conventional high-dynamic-range (HDR) imaging techniques using CMOS image
sensors use variable exposures to capture scenes with extreme dynamic range.
The most common method called exposure bracketing
\cite{Gupta_2013,Hasinoff2010} captures multiple images with different exposure
times; shorter exposures reliably capture bright pixels in the scene avoiding
saturation, while longer exposures capture darker pixels while avoiding photon
noise. Another technique involves use of neutral density (ND) filters of
varying densities resulting in a tradeoff between spatial resolution and
dynamic range \cite{Nayar_2000}. ND filters reduce overall sensitivity to avoid
saturation, at the cost of making darker scene pixels noisier.  In contrast, an
IP-SPAD captures scene intensities in a different way by relying on the
non-linear reciprocal relationship between inter-photon timing and scene
brightness. This gives extreme dynamic range in a single capture.

\smallskip
\noindent\textbf{Passive Imaging with Photon-Counting Sensors:}
Previous work on passive imaging with photon counting sensors relies on two
sensor technologies---SPADs and quanta-image sensors (QISs)
\cite{gnanasambandam2019megapixel}. A QIS has single-photon sensitivity but
much lower time resolution than a SPAD pixel. On the other hand, QIS pixels can
be designed with much smaller pixel pitch compared to SPAD pixels, allowing
spatial averaging to further improve dynamic range while still maintaining high
spatial resolution \cite{ma2020quanta}. SPAD-based high-dynamic range schemes
provide lower spatial resolution than the QIS-based counterparts
\cite{dutton2018high}, although, recently, megapixel SPAD arrays capable of
passive photon counting have also be developed \cite{Morimoto_2020}. Previous
work \cite{ingle2019high} has shown that passive free-running SPADs can
potentially provide several orders of magnitude improved dynamic range compared
to conventional CMOS image sensor pixel. The present work exploits the precise
\emph{timing information}, in addition to photon counts, measured by a
free-running SPAD sensor. An IP-SPAD can image scenes with even higher dynamic
range than the counts-based PF-SPAD method.

\smallskip
\noindent\textbf{Methods Relying on Photon Timing:}
The idea of using timing information for passive imaging has been explored
before for conventional CMOS image sensor pixels. A saturated CMOS pixel's
output is simply a constant and meaningless, but if the time taken to reach
saturation is also available \cite{culurciello2003biomorphic}, it provides
information about scene brightness, because a brighter scene point will reach
saturation more quickly (on average) than a dimmer scene point. The idea of
using photon timing information for HDR has also been discussed before but the
dynamic range improvements were limited by the low timing resolution of the
pixels \cite{zarghami2019high, laurenzis2019single} at which point, the photon
timing provides no additional information over photon counts.
\atul{In their seminal work on single-photon 3D imaging
under high illumination conditions,
Rapp \emph{et al.} provide rigorous theoretical derivations of photon-timing-based
maximum-likelihood flux estimators that include
dead-time effects of both the SPAD pixel and the
timing electronics \cite{rapp2019dead,rapp2021high}.
Our theoretical SNR analysis and experimental results show that such photon-timing-based flux estimators can provide extreme dynamic range for passive 2D intensity imaging.}

\smallskip
\noindent\textbf{Methods Relying on Non-linear Sensor Response:}
Logarithmic image sensors include additional pixel electronics that apply
log-compression to capture a large dynamic range. These pixels are difficult to
calibrate and require additional pixel electronics compared to conventional
CMOS image sensor pixels \cite{kavadias2000logarithmic}. A modulo-camera
\cite{zhao2015unbounded} allows a conventional CMOS pixel output to wrap around
after saturation. It requires additional in-pixel computation involving an
iterative algorithm that unwraps the modulo-compression to reconstruct the
high-dynamic-range scene. In contrast, our timing-based HDR flux estimator is a
closed-form expression that can be computed using simple arithmetic operations.
Although our method also requires additional in-pixel electronics to capture
high-resolution timing information, recent trends in SPAD technology indicate
that such arrays can be manufactured cheaply and at scale using CMOS
fabrication techniques \cite{Henderson_2019, Henderson_2019_ISSCC}.

\smallskip
\noindent\textbf{Active Imaging Methods:}
Photon timing information captured by a SPAD sensor has been exploited for
various active imaging applications like transient imaging
\cite{o2017reconstructing}, fluorescence lifetime microscopy
\cite{bruschini2019single}, 3D imaging LiDAR \cite{Kirmani_2013,
gupta2019asynchronous} and non-line-of-sight imaging \cite{kirmani2009looking,
buttafava2015non}.  Active methods capture photon timing information with
respect to a synchronized light source like a pulsed laser that illuminates the
scene. \atul{In a low
flux setting, photon timing information measured with respect to a
synchronized laser source can be used to reconstruct scene depth and reflectivity maps
\cite{Kirmani_2013}. In contrast, here we show that inter-photon timing
information captured with an asynchronous SPAD sensor can enable scene
intensity estimation not just in low but also extremely high flux levels in a
passive imaging setting.}

