\section{Future Outlook}
The analysis and experimental proof-of-concept shown in this paper were
restricted to a single IP-SPAD pixel. Recent advances in CMOS SPAD technology
that rely on 3D stacking \cite{Henderson_2019_ISSCC} can enable larger arrays
of SPAD pixels for passive imaging. This will introduce additional design
challenges and noise sources not discussed here. In
\nolink{\ref{suppl:pixel_designs}} we show some pixel architectures for an
IP-SPAD array that could be implemented in the future.

Arrays of single-photon image sensor pixels are being increasingly used for 2D
intensity imaging and 3D depth sensing
\cite{yoshida_2020,Lee_2020,Mardirosian_2020} in commercial and consumer
devices. When combined with recent advances in high-time-resolution SPAD sensor
hardware, the methods developed in this paper can enable extreme imaging
applications across various applications including consumer photography, vision
sensors for autonomous driving and robotics, and biomedical optical imaging.


%Conceptually, the idea of using photon timing information for passive imaging
%is quite general and not necessarily tied to a specific sensor technology. The
%analysis in this paper was restricted to SPADs as the technology of choice
%because SPAD arrays can be manufactured at scale using standard CMOS
%fabrication process.

%\noindent\textbf{Passive SPAD Pixel Architectures:}
%Fig.~\ref{fig:array_designs}(a) shows a pixel architecture with an efficient
%way of storing and transmitting photon timestamp data for passive imaging. It
%relies on storing the first and last photon timestamps within a single exposure
%time, together with the total photon counts.  While this increases pixel
%complexity over conventional SPAD pixel designs (see
%\nolink{\ref{suppl:pixel_designs}}), it only requires two additional data
%registers. The disadvantage of this scheme is that, depending on the total
%exposure time, the time-to-digital converter (TDC) may require a large
%full-scale range.  For example, using an exposure time in the millisecond range
%and the timestamp resolution in picoseconds, the TDC data depth will be
%$\log_2(10^{-3}/10^{-12})\approx 30$~bits.  Fig.~\ref{fig:array_designs}(b)
%shows an implementation of a brightness estimator based on inter-photon
%timing statistics. Here the IP-SPAD pixel computes a moving average of
%inter-photon times.

% sensor fusion 16x16 with RGB camera
% mixed hybrid sensor (add simulation? maybe not...)
% just a sentence... maybe a future analysis in a
% different paper
