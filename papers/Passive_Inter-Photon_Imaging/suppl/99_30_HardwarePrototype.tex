\section{Hardware Prototype\label{suppl:hardware}}
Our hardware prototype (\ref{fig:hardware_photo})
consists of a single SPAD pixel mounted on two
translation stages. Dead-time is controlled using a long cable that produces
analog delay.  After each photon detection event the SPAD pixel has to be kept
disabled for few tens of \si{\ns} to lower the probability of afterpulses
\cite{cova1991} and reset its original bias condition. Usually this dead-time
is set either using the discharge time of an R-C network or employing a digital
timing circuit, since the dead-time accuracy is not a limiting factor in
conventional SPAD applications. In case of dead-time defined using digital
timing circuits, there are implementations where its accuracy depends on the
period of an uncorrelated (with respect to photon arrival times) digital clock.
For example, a \num{100}~\si{\mega\hertz} clock frequency will limit the
accuracy of the dead-time to about \num{10}~\si{\ns}, which is too coarse to
get reliable photon flux estimates.  This is true especially at extremely high
photon flux values where photons get detected almost immediately after each
dead-time duration ends. As described in the main text, we rely on low-jitter
voltage comparators and analog delays introduced by long coaxial cables to
obtain precisely controlled dead-time durations with low jitter.

\begin{figure}[!ht]
  \centering \includegraphics[width=0.4\columnwidth]{figures/SupplFig_hardware_setup.png}
  \caption{The IP-SPAD hardware prototype consists of a single SPAD pixel mounted on
    translation stages to scan the image plane of a vari-focal lens
    (Fujinon DV3.4x3.8SA-1). Part of a $\SI{20}{\m}$ long co-axial cable used for
    generating the dead-time delay is also shown.
    \label{fig:hardware_photo}}
\end{figure}

\section{Pixel Non-idealities}
When conducting experiments with our hardware prototype we found two
non-idealities: dead-time drift and non-zero gate rise time.  
\subsection*{Dead-time Drift}\label{suppl:dead time drift}
When imaging high flux regions for extended periods of time our hardware
prototype's dead-time increases; we call this \textit{dead-time drift}.  This
is due to heating of the SPAD front-end. We calibrated each pixel position
individually by constructing an inter-photon timing histogram and using the
first non-zero bin of this histogram as an estimate of the true dead-time for
that pixel position. Experimentally, we observed that the dead-time drift is
slower than the 5ms exposure times used so this method should approximate the
true dead-time well for each pixel position. Without this correction the error
introduced by the drift dominates the denominator in
Eq.~(\ref{eq:flux_estimator}) at high flux values, limiting the dynamic range.

\begin{figure}[!ht]
  \centering \includegraphics[width=0.6\columnwidth]{figures/suppl_drift_rise.png}
  \caption{This figure shows inter-photon histograms for 4 points from the
    tunnel scene.  Notice that the histograms are not aligned on the left edge,
    indicating a drift in dead-time. We correct for this drift by taking the
    time of the first non-zero bin as the dead-time for that pixel.
    \label{suppl_fig:drift_rise}}
\end{figure}

When our single-pixel IP-SPAD stays active for long periods of time dead-time
drift becomes a problem. Suppl. Fig.~\ref{suppl_fig:drift_rise} shows
inter-photon timing histograms of four different scene points with increasing
flux levels ($1\rightarrow 4$). Notice that the histograms are not aligned on
the left edges indicating a difference in dead-times at these points. We
correct for the dead-time drift in by using a dead-time estimate for each pixel
in the final image. We set the dead-time estimate in a pixel to the smallest
inter-arrival time from that pixel, this has the effect of shifting each
pixel's inter-arrival histogram to zero.  In the tunnel scene the difference
between the longest used and shortest used dead-time is \SI{887}{\pico\second},
a variation of about 0.8\%.

\subsection*{Gate Rise Time}\label{suppl:rise time}
When the SPAD enters and exits the dead-time phase, its bias voltage
has to be quickly changed from above to below the breakdown value, and
vice-versa \cite{Buttafava2014}. The duration of these transitions is as
critical as the dead-time duration itself, and has to be short (in order to
swiftly restore the SPAD bias for the next detection) and precise (to prevent
variations in the overall dead-time duration). In our system the rise
times are on the order of \SI{200}{\pico\second}:
it translates into non-exponentially shaped inter-photon timing histograms,
especially in high flux regions. We did not find that this behavior
detrimentally effected our results; however, it has an effect similar to
slightly tone mapping bright regions downward. 

Unlike dead-time drift, the rise time behavior seems to be independent of how
long the SPAD was exposed to a high flux source. Fig.  \ref{fig:exp_hists}
shows inter-photon timestamp histograms for increasing photon flux levels. Rise
time causes these to deviate from an exponential shape at high flux levels. 

We found that this behavior made it virtually impossible to fully saturate the
SPAD pixel, that is increasing the incident flux would lead to a non-linear
increase in photons counted. We performed an experiment where a laser was
directly pointed into the SPAD active region and the power of the laser was
increased. We found that the photon counts did not saturate before the SPAD
overheated and shut itself off. 

The rise-time behaviour can by incorporated into the flux estimator derived in
\ref{suppl:MLE_conditional_derivation} using a time-varying quantum efficiency
$q(t)$. For $t<0$, $q(t)=0$ and  $\int_0^{\infty} q(t) dt \rightarrow\infty$.
When the dead time ends, the IP-SPAD pixel's $q(t)$ ramps up to its peak value.
The probability distribution of time-of-darkness, $Y_i$, can be written as:

\begin{equation}
Y_{i} \sim f_{Y_{i}}(Y_i | Y_{1}\dots Y_{i-1})=\begin{cases}
q(Y_i)\Phi e^{-\Phi\int_0^{Y_i}q(l)dl} & \mbox{for } 0< Y_i< T_i\\
e^{-\Phi \int_0^{T_i}q(l)dl} \,\cdot\, \delta(t-T_i) &\mbox{for } Y_i=T_i\\
0 & \mbox{otherwise.}
\end{cases} \label{eq:interarrival_distr_varyQ}
\end{equation}
where $T_i$ is defined in \ref{suppl:MLE_conditional_derivation}. For a series
of $N$ timestamps with times-of-darkness given by $Y_1\dots Y_{N}$, we use a
similar derivation to \ref{suppl:MLE_conditional_derivation} to find the
maximum likelihood estimator (MLE): 
\begin{equation}
  \widehat \Phi =  \frac{N}{\int^{T_{N+1}}_0 q(t)dt+\sum_{i=1}^N\int^{Y_i}_0 q(t)dt}. \label{suppl:MLE_varyQ}
\end{equation}
Eq.~(\ref{suppl:MLE_varyQ}) reduces to Eq.~(\ref{eq:MLE_cond}) if $q(t)$ is an
ideal step function. For the experimental results shown in the main text, the
IP-SPAD pixel's $q(t)$ was not precisely known so we could not apply this
correction. Future work will look at estimating $q(t)$ from inter-photon
histograms and quantifying SNR improvements from such a correction.

\clearpage
\section{Additional Results\label{suppl:additional_results}}

\begin{figure}[!htb]
  \centering
  \includegraphics[width=0.55\linewidth]{figures/Fig3_simulated_HDR.png}
  \caption{\textbf{Simulated Extreme Dynamic Range Scene:}
  This figure shows simulated extreme dynamic range images using an IP-SPAD
  camera compared with a conventional camera with different exposure settings.
  (a) A \SI{5}{\ms} exposure image with a conventional camera (full-well capacity
  34,000 and read noise 5$e^-$ has many saturated pixels. Observe that the
  bright bulb region is washed out.  (b) A short exposure image is dominated by
  shot noise in darker parts of the scene.  It becomes visible only at a much
  lower exposure setting. Since this is a simulation we were able to reduce the
  exposure time down to $5 \times 10^{-5}$ \si{\ms} which may be impossible to
  achieve with a conventional camera. In practice, this exposure can be
  achieved by, say, reducing the shutter speed to 1/16,000 and adding a 10-stop
  ND filter.  (c) A PF-SPAD camera is able to capture both dark and bright
  regions in a single exposure, but the bright bulb filament still suffers from
  noise due to the soft-saturation phenomenon. (d) Our proposed IP-SPAD method
  estimates scene brightness using high-resolution timestamps to capture both
  extremely dark and extremely bright pixels, beyond the soft-saturation limit
  of a counts-based PF-SPAD.  (Original image from HDRIHaven.com) 
    \label{fig:sim_hdr_suppl}}
\end{figure}
\clearpage

\begin{figure}[!htb]
  \centering
  \includegraphics[width=0.95\linewidth]{figures/SupplFig_shelf_shelfLed.png}
  \caption{\textbf{Experimental Extreme Dynamic Range ``Shelf'' Scene:}
  This ``Shelf'' scene shows extreme dynamic range, with a bright bulb filament
  in one of the shelves and text in the neighboring shelf which is dark and not
  directly illuminated by the light source. The bottom row of images uses a
  similar setup as the top row but also includes two bright LED lights in
  addition to the filament bulb.  The conventional camera requires three
  exposures to cover the dynamic range of this scene. The proposed IP-SPAD flux
  estimator captures the scene in a single exposure.
    \label{fig:shelf_expt}}
\end{figure}


\clearpage

\begin{figure}[!htb]
  \centering
  \includegraphics[width=0.8\linewidth]{figures/FigX_IncreasingPhotons.png}
	\caption{\textbf{Effect of Increasing Number of Photons on Denoising.}
  Some image details start appearing with as few as 10 photon timestamps per
  pixel. For example, the text on the fire-truck is visible with images
  denoised with the bilateral filter and our KPN-based denoiser.  BM3D appears
  to give less noisy results in this example but finer details are lost.
  \label{fig:passive_first_photon_expt_full}}
\end{figure}


