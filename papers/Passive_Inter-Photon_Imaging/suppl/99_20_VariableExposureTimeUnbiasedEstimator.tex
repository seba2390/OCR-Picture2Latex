\section{Image Formation}\label{suppl:MLE_conditional_derivation}

\begin{figure}[!ht]
\centering \includegraphics[width=0.65\textwidth]{figures/suppl_timeline_fig.png}
\caption{\textbf{Photon Detection Timeline:} (a) The photon timeline shows the
random variables used in the derivation of our photon flux estimator.  $X_i$'s
denote the photon arrival time with respect to the start of the exposure at
$t=0$ and $Y_i$'s denote the $i^{th}$ time-of-darkness. (b) There are two
possibilities for the final dead-time window after the last photon detection.
In high photon flux scenarios, the final dead-time ends after the end of the
exposure time $(X_{N_T}+\taud > T)$ with high probability.
\label{suppl:timeline}}
\end{figure}

Consider an IP-SPAD sensor pixel with quantum efficiency $q$ exposed to a
photon flux of $\Phi$ photons/second. Photon arrivals follow a Poisson process,
so inter-photon times follow an exponential distribution with rate $q\Phi$.
After a detection event the IP-SPAD is unable to detect photons for a period of
$\taud$. Because of the memoryless property of a Poisson process, the arrival
time of a photon after the end of a dead-time window (called the
\textit{time-of-darkness}), follows an exponential distribution with rate
$q\Phi$. The probability that no photons are detected in the exposure time $T$
is equal to $\int^\infty_T q\Phi e^{-q\Phi t} dt =e^{-q\Phi T}.$

Let the first time-of-darkness be denoted by $Y_1$. If no photons are detected,
we define $Y_1:=T$. If $Y_1<T$ it follows an exponential distribution.
Therefore, the probability density function of $Y_1$ can be written as:

\begin{equation}
Y_{1}{\sim}f_{Y_{1}}(t)=\begin{cases}
q\Phi e^{-q\Phi t} & \mbox{for } 0< t< T\\
e^{-q\Phi T} \delta(t-T)&\mbox{for } t=T\\
0 & \mbox{otherwise,}
\end{cases} \label{eq:first_photon_dist}
\end{equation}
where $\delta$ is the Dirac delta function.

Now consider $Y_2$, the second time-of-darkness. $Y_2$ is non-zero if and only
if $Y_1 \neq T$ (to be the second, there must be a first). If the second photon
is detected, $Y_2$ will be exponentially distributed. But the exposure time
interval shrinks because a time interval of $Y_1+\taud$ has elapsed due to
the first photon detection.  We define the remaining exposure time
$T_2=\max(0,T-Y_1-\tau_d$), where the $\max()$ function ensures $T_2$ is
non-negative. Then the probability distribution of $Y_2$ conditioned on $Y_1$
will be given by replacing $T$ for $T_2$ in Eq.~(\ref{eq:first_photon_dist}).
More generally, the conditional distribution of $Y_i$ can be written as:

\begin{equation}
Y_{i} \sim f_{Y_i | Y_{1}\dots Y_{i-1}}(t | Y_{1}\dots Y_{i-1})=\begin{cases}
q\Phi e^{-q\Phi t} & \mbox{for } 0< t < T_i\\
e^{-q\Phi T_i} \delta(t-T_i) &\mbox{for } t=T_i\\
0 & \mbox{otherwise,}
\end{cases} \label{eq:interarrival_distr}
\end{equation}
where,
\begin{align}
T_1 &=T\nonumber\\
T_i &= \max(0,T_{i-1} - Y_{i-1}-\taud))\nonumber\\
 &=\max\left(0,T -\sum^{i-1}_{j=1}(Y_j+\taud)\right).
\end{align}

The $T_i$'s model the fact that the effective exposure time for the $i^{th}$
photon shrinks due to preceding photon detections. Note if $Y_i = T_i$ then no
$i^\text{th}$ photon is detected and $Y_{i+1}=T_{i+1}=0$. Note that the $X_i$
in the main text are related to $Y_i$ by $X_i-X_{i-1}-\taud=:Y_i$ for $i\geq2$
and $X_1=Y_1$. Suppl.  Fig.~\ref{suppl:timeline}(a) shows  $X_i$ and $Y_i$ on a
photon timeline.

\subsection*{Maximum Likelihood Flux Estimator}
For a fixed exposure time $T$, the maximum number of possible photon detections
is $L=\big\lceil\frac{T}{\taud}\big\rceil$. Let $N$ be the number of detected
photons, then $Y_{N+1}$ will be the last possibly non-zero time-of-darkness,
and $Y_{N+2}\dots Y_L =0$ with probability $1$. The log-likelihood of the
unknown flux value given the set of time-of-darkness measurements $Y_1\dots
Y_L$ is given by:

\begin{align}
  \log l(q\Phi; Y_1,\ldots,Y_{L}) &= \log \left( \prod_{i=1}^{L} f_{Y_{i}}(Y_i | Y_{1}\dots Y_{i-1}) \right) \nonumber \\
  &= \log \left(f_{Y_{N+1}}(Y_{N+1} | Y_{1}\dots Y_{N}) \prod_{i=1}^{N} f_{Y_{i}}(Y_i | Y_{1}\dots Y_{i-1}) \right) \nonumber \\
  &= \log \left( e^{-q\Phi T_{N+1}}\prod_{i=1}^{N} q\Phi\,e^{-q\Phi Y_i} \right) \nonumber \\
  &= - q\Phi \left( T_{N+1}+\sum_{i=1}^{N}Y_i \right) + N\log q\Phi \nonumber \\
  &= - q\Phi\,\left( \max\left(0,T-\sum_{i=1}^{N}Y_i-\taud\right)+\sum_{i=1}^{N}Y_i\right) + N \log q\Phi\nonumber\\
  &= - q\Phi\, \max\left(\sum_{i=1}^{N}Y_i,T-N\taud\right) + N \log q\Phi.
  \label{eq:loglik_cond}
\end{align}
We find the maximum likelihood estimate, $\widehat{\Phi}$, by setting the
derivative of Eq.(\ref{eq:loglik_cond}) to zero and solving for $\Phi$:

\begin{align}
  - q\max\left(\sum_{i=1}^{N}Y_i,T-N\taud\right) + \frac{N}{\widehat\Phi} &= 0,
\end{align}
which gives: 
\begin{align}
  \widehat\Phi=\frac{N}{q\max\left(\sum_{n=1}^{N}Y_i,T-N\taud\right)}.\label{eq:MLE_cond}
\end{align}

The $\max()$ function can be thought of as selecting the time-of-darkness based
on whether or not the final dead-time window ends after $t=T$, see Suppl.
Fig.~\ref{suppl:timeline}(b). In practice the beginning and end of the exposure
time may not be known precisely, introducing uncertainty in $X_1$ and $T$.
Because of this we instead use an approximation:
\begin{align}
  \widehat\Phi=\frac{N-1}{q\sum_{n=2}^{N}Y_i}.\label{eq:MLE_cond_simp}
\end{align}
Plugging in $Y_i = X_i - X_{i-1} - \taud$ gives Eq.~(\ref{eq:flux_estimator})
in the main text.

%However, when the left argument of the maximum function is used there is a
%multiplicative bias \cite{Abdulaziz2001} of $\frac{N}{N-1}$ so for an unbiased
%estimator we use:

%\begin{align}
 % \hat\phi=\frac{N}{q\max(\frac{N}{N-1}\sum_{i=1}^{N}Y_i,T-N\tau_d)}.
  %\end{align}
\subsection*{Flux Estimator Variance}
Let $N$ be the number of photons detected in an exposure time $T$. Using the
law of large numbers for renewal processes we find the expectation and the
variance of $N$ to be:

\begin{align}
  \text{E}[N] = \frac{q\Phi(T+\taud)}{1+q\Phi \taud} \\
  \text{Var}[N]=\frac{q\Phi(T+\taud)}{(1+q\Phi \taud)^3}
\end{align}

In the following derivation we will assume $N$ is large enough that it can be
assumed to be constant for a given $T$. This holds, for example, when $\Phi \gg
\frac{1}{T}$.  This assumption also allows us to approximate $Y_i$'s as i.i.d.
shifted exponential random variables. We will consider the estimator in
Eq.~(\ref{eq:MLE_cond_simp}) where the sum in the denominator is given by
$S_{N_T}=Y_2+Y_3+...Y_N$ and letting $N_T=N-1$.  The final photon timestamp
$S_{N_T}$ is the sum of exponential random variables and follows a gamma
distribution:

\begin{equation}
S_{N_T}\stackrel{ }{\sim}f_{S_{N_T}}(t)=\begin{cases}
\frac{(q\Phi)^{N_T} t^{N_T-1} e^{-q\Phi t}}{(N_T-1)!} & \mbox{for } t\geq0\\
0 & \mbox{otherwise.}
\end{cases} \label{eq:gamma_interarrival_distr}
\end{equation}

The mean of $\widehat\Phi$ can be computed as: 
\begin{align}
  \text{E}\left[\frac{N_T}{q S_{N_T}}\right]&= \frac{N_T}{q}\text{E}\left[\frac{1}{S_{N_T}}\right] \nonumber\\
  &=  \frac{N_T}{q}\int^{\infty}_{0} \frac{(q\Phi)^{N_T} t^{N_T-1} e^{-q\Phi t}}{t(N_T-1)!}\, dt \nonumber\\
  &=  \frac{N_T}{q}\frac{q\phi}{N_T-1}\int^{\infty}_{0} \frac{(q\Phi)^{N_T-1} t^{N_T-2} e^{-q\Phi t}}{(N_T-2)!}\, dt\nonumber\\
  &= \frac{N_T}{N_T-1}\Phi \label{eq:bias}
\end{align}
where the last line comes from the recognizing the argument of the integral as
the p.d.f. for a gamma distribution and for large $N_T$, $\frac{N_T}{N_T-1}
\approx 1$.

The second moment of $\widehat\Phi$ is given by:
\begin{align}
  \text{E}\left[\left(\frac{N_T}{q S_{N_T}}\right)^2\right] 
  &=\frac{N_T^2}{q^2}\text{E}\left[\frac{1}{S_{N_T}^2}\right] \nonumber\\
  &=\frac{N_T^2}{q^2}\int^{\infty}_{0} \frac{(q\Phi)^{N_T} t^{N_T-1} e^{-q\Phi t}}{t^2(N_T-1)!} dt\nonumber\\
  &=\frac{(q\Phi)^2N_T^2}{q^2(N_T-1)(N_T-2)}\int^{\infty}_{0} \frac{(q\Phi)^{N_T-2} t^{N_T-3} e^{-q\Phi t}}{(N_T-3)!} dt\nonumber\\
  &=\frac{\Phi^2 N_T^2}{(N_T-1)(N_T-2)}
\end{align}
This expression is valid for $N_T>2$. The variance of $\widehat\Phi$ is given by: 
\begin{align}
  \text{Var}\left[\frac{N_T}{q S_{N_T}}\right] &=\Phi^2\frac{N_T^2}{(N_T-2)(N_T-1)}-\frac{N_T^2}{(N_T-1)^2}\Phi^2 \nonumber \\
  &=\Phi^2\frac{N_T^2}{(N_T-2)(N_T-1)^2}\nonumber\\
  &\approx \Phi^2\frac{1}{N_T}\\
  &= \Phi^2 \frac{ q\Phi\taud+1}{q\Phi(T+\taud)}\\
  &\approx \Phi \frac{ q\Phi\taud+1}{q T}
\end{align}
where we replace $N_T$ with its mean value. The last line follows if we assume
$T\gg\taud$.
Finally, the SNR is given by:
\begin{align}
  \text{SNR} &= 20\log_{10}\frac{\Phi}{\sqrt{\text{Var}[\frac{N_T}{q S_{N_T}}]}} \nonumber\\
  &= 10\log_{10}\frac{q\Phi T}{ q\Phi \taud+1}
  \label{eq:SNRTiming}
\end{align}

We make the following observations about our estimator $\widehat\Phi$:
\begin{myitemize}
  \item At high flux, when $N_T$ is large enough, Eq.~(\ref{eq:bias}) reduces to
    $\text{E}[\widehat\Phi] = \Phi$, i.e.  our estimator is unbiased.
  \item Unlike \cite{ingle2019high} which only uses photon counts $N_T$, our
    derivation explicitly accounts for individual inter-photon timing
    information captured in $S_{N_T}$.
  \item As $\Phi\rightarrow\infty$,
    $\text{SNR}\rightarrow10\log_{10}(\frac{T}{\taud})$. So at high flux the
    SNR will flatten out to a constant independent of the true flux $\Phi$.  In
    practice, the SNR drops at high flux due to time quantization, discussed
    next in \ref{suppl:quantization}.
\end{myitemize}

