\section{Time Quantization \label{suppl:quantization}}
Consider an IP-SPAD with quantum efficiency $q$, dead time $\taud$, and time
quantization $\Delta$ that detects photons over exposure time $T$. To match
our hardware prototype, the start of the dead time window is not quantized and
time stamps are quantized by $\Delta$. The quantization noise variance term
derived in previous work \cite{ingle2019high,Antolovic_2018} that relies on a
counts-only measurement model is given by:

\begin{equation}
V_\text{count-quantization} = \frac{(1+q\Phi \taud)^4}{12 q^2 T^2}.\label{suppl:count_quant} 
\end{equation}

We derive a modified quantization noise variance expression by modifying this
counts-only expression to account for two key insights gained from extensive
simulations of SNR plots for our timing-based IP-SPAD flux estimator. First, we
note that the timing-based IP-SPAD flux estimator follows a similar SNR curve
as the counts-based PF-SPAD flux estimator when  $\Delta=\taud$. Second, the
rate at which the SNR drop off moves slows after $\Delta$ exceeds $\taud$. In
this way we propose a new time quantization term:
\begin{equation}
V_\text{time-quantization} = \frac{(1+q\Phi \taud +q\Phi \Delta)^2(1+q\phi \Delta)^2}{12 q^2 T^2}.\label{suppl:quantization_noise} 
\end{equation}

Note we break the quartic term from Eq.~(\ref{suppl:count_quant}) into two
quadratic terms. The two quadratic terms strike a balance between quantization due
to counts and timing. If $\Delta=0$ then $V_\text{time quantization}$ is an order
2 polynomial with respect to $\Phi$ which leads to a constant SNR at high flux.
Also note if $\Delta=\taud$ the time quantization term is roughly equal to the
counts quantization term. We found this expression matches simulated IP-SPAD
SNR curves for a range of dead-times and exposure times.

