\section{Results}
\subsection{Simulation Results}
\noindent{\bf Simulated RGB Image Comparisons:}
Fig.~\ref{fig:sim_hdr} shows a simulated HDR scene with extreme brightness
variations between the dark text and bright bulb filament. We use a \SI{5}{\ms}
exposure time for this simulation. The PF-SPAD and
IP-SPAD both use pixels with $q=0.4$ and $\taud=\SI{150}{\ns}$.
The PF-SPAD only captures photon
counts whereas the IP-SPAD captures counts and timestamps with a resolution
of $\Delta=\SI{200}{\ps}$. Notice that the extremely bright pixels on the bulb
filament appear noisy in the PF-SPAD image. This degradation in SNR at high
flux levels is due to its soft-saturation phenomenon. The IP-SPAD, on the other
hand, captures the dark text and noise-free details in the bright bulb filament
in a single exposure. Please see supplement for additional comparisons with a
conventional camera.

%\smallskip
%\noindent{\bf Low-Light Imaging:}
%Fig.~\ref{fig:passive_first_photon_small}(a) shows a ground-truth image used
%for simulating passive first-photon timestamps shown in
%Fig.~\ref{fig:passive_first_photon_small}(b). As expected, the point-wise
%brightness estimator that computes the reciprocal of the first-photon
%timestamps gives extremely noisy result shown in
%Fig.~\ref{fig:passive_first_photon_small}(c).
%Fig.~\ref{fig:passive_first_photon_small}(d) shows the denoised image using our
%KPN-based denoiser. A comparison with conventional denoising methods like BM3D
%and bilateral filtering are shown in the supplement. Although it does not
%reconstruct any fine details and object boundaries, surprisingly, the
%KPN denoiser recovers some coarse object shapes like the sofa and the
%bright bulb on the ceiling which are barely visible in the raw data.

\begin{figure*}[!ht]
  \centering \includegraphics[width=0.99\linewidth]{figures/Fig5_tunnel2_compressed_OneRow_cmos_pf_ip.png}
  \caption{\textbf{Experimental ``Tunnel'' scene:}  (a-b) Images from a
  conventional sensor with long and short exposure times. Notice that both the speed
  limit sign and the toy figure cannot be captured simultaneously with a single
	exposure. Objects outside the tunnel appear saturated even with the
	shortest exposure time possible with our CMOS camera. (c) A PF-SPAD
  \cite{ingle2019high} only uses photon counts when estimating flux.
  Although it captures much higher dynamic range than the
  conventional CMOS camera, the bright pixels near the halogen lamp 
  appear saturated. (d) Our IP-SPAD single-pixel hardware prototype
  captures both the dark and the extremely bright regions with a single
  exposure. Observe that the fine details within the halogen
  lamp are visible. \label{fig:tunnel}}
  \vspace{-0.15in}
\end{figure*}


\subsection{Single-Pixel IP-SPAD Hardware Prototype}
\begin{figure}[!ht]
  \centering
  \includegraphics[width=\columnwidth]{figures/FigY_expt_histograms.png}
\caption{\textbf{Rise-time Non-ideality in Measured IP-SPAD Histograms:} We show
four inter-photon histograms for pixels in the experimental ``Tunnel'' scene.
The histograms of [P1] and [P2] have an ideal exponentially decaying shape.
However, at the brighter points, [P3] and [P4], the inter-photon histograms
deviate from an ideal exponential shape. This is because the IP-SPAD pixels requires
$\sim \SI{100}{\ps}$ rise time to re-activate after the end of the previous dead-time.
%(Note the different x-axis limits for the inter-photon histograms of
%[P1] and [P2] vs. [P3] and [P4].)
  \vspace{-0.25in}
\label{fig:exp_hists}}
\end{figure}


Our single-pixel IP-SPAD prototype is a modified version of a fast-gated SPAD
module \cite{Buttafava2014}. Conventional dead-time control circuits for a SPAD
rely on digital timers that have a coarse time-quantization limited by the
digital clock frequency and cannot be used for implementing an IP-SPAD. We
circumvent this limitation by using coaxial cables and low-jitter voltage
comparators to generate ``analog delays'' that enable precise control of the
dead-time with jitter limited to within a few \si{\ps}. We used a \SI{20}{\m}
long co-axial cable to get a dead-time of \SI{110}{\ns}. The measured dead-time
jitter was $\sim\SI{200}{\ps}$. This is an improvement over previous PF-SPAD
implementations \cite{ingle2019high} that relied on a digital timer circuit
whose time resolution was limited to $\sim\SI{6}{\ns}$.
%We could not get the IP-SPAD pixel to saturate with a 200mW laser shining
%straight into the pixel active area.

The IP-SPAD pixel is mounted on two translation stages that raster scan the
image plane of a variable focal length lens. The exposure time per pixel
position depends on the translation speed along each scan-line.  We capture
400$\times$400 images with an exposure time of \num{5}~\si{\ms} per pixel
position. The total capture takes $\sim\!\!\!15$ minutes. Photon timestamps are
captured with a \SI{1}{\ps} time binning using a time-correlated single-photon
counting (TCSPC) system (PicoQuant HydraHarp400). A monochrome camera
(PointGrey Technologies GS3-U3-23S6M-C) placed next to the SPAD captures
conventional camera images for comparison.  The setup is arranged carefully to
obtain approximately the same field of view and photon flux per pixel for both
the IP-SPAD and CMOS camera pixels.

%\begin{figure}[!ht]
%  \centering \includegraphics[width=0.95\columnwidth]{figures/Fig6_tunnel_timing_vs_counts.png}
%  \caption{\textbf{Advantage of photon timestamps in high flux regime:} (a) At
%  high enough flux levels a counts-based PF-SPAD flux estimator
%  \cite{ingle2019high} saturates.  (b) Our IP-SPAD pixel provides reliable flux
%  estimates by exploiting information encoded in photon timestamp
%  distributions. Observe the increased dynamic range in the bright parts of the
%  scene around the halogen lamp in the timing-based flux estimator.
%  \label{fig:tunnel_timing_vs_counts}}
%  \vspace{-0.20in}
%\end{figure}
%Gate rise time issues.
%Practical limitations long cable based delay to get precise dead-time.

\begin{figure}[!htb]
  \centering
  \includegraphics[width=0.9\columnwidth]{figures/Fig_LowPhotonCountsSmall.png}
	\caption{\textbf{IP-SPAD Imaging in Low Photon Count Regime:}
  This figure shows IP-SPAD images captured with very few photons and denoised
  with two different methods: (a) an off-the-shelf BM3D denoiser, and (b) a DNN
  denoiser based on a kernel prediction network architecture.
  Details like the text on the fire-truck are visible with as few as 10 photons per
  pixel.\label{fig:low_counts}}
  \vspace{-0.15in}
\end{figure}



\subsection{Hardware Experiment Results}
\noindent\textbf{HDR Imaging:}
Fig.~\ref{fig:tunnel} shows an experiment result using our single-pixel
raster-scanning hardware prototype. The ``Tunnel'' scene contains dark objects
like a speed limit sign inside the tunnel and an extremely bright region
outside the tunnel from a halogen lamp.  This scene has a dynamic range of over
\num[retain-unity-mantissa=false]{1e7}:\num{1}. The conventional CMOS camera
(Fig.~\ref{fig:tunnel}(a-b)), requires multiple exposures to cover this dynamic
range.  Even with the shortest possible exposure time of \SI{0.005}{\ms}, the
halogen lamp appears saturated. Our IP-SPAD prototype captures details of both
the dark regions (text on the speed limit sign) simultaneously with the bright
pixels (outline of halogen lamp tube) in a single exposure.
Fig.~\ref{fig:tunnel}(c) and (d) shows experimental comparison between a
PF-SPAD (counts-only) image \cite{ingle2019high} and the proposed IP-SPAD image
that uses photon timestamps. Observe that in extremely high flux levels (in
pixels around the halogen lamp) the PF-SPAD flux estimator fails due to the
inherent quantization limitation of photon counts. The IP-SPAD preserves 
details in these extremely bright regions, like the shape of the halogen 
lamp tube and the metal cage on the lamp.

\smallskip
\noindent\textbf{Hardware Limitations:}
The IP-SPAD pixel does not exit the dead-time duration instantaneously and in
practice it takes around \SI{100}{\ps} to transition into a fully-on state.
Representative histograms for four different locations in the experiment tunnel
scene are shown in Fig.~\ref{fig:exp_hists}. Observe that at lower flux levels
(pixels [P1] and [P2]) the inter-photon histograms follow an exponential
distribution as predicted by the Poisson model for photon arrival statistics.
However, at pixels with extremely high brightness levels (pixels [P3] and [P4]
on the halogen lamp), the histograms have a rising edge denoting the transition
phase when the pixel turns on after the end of the previous dead-time.
We also found that in practice the dead-time is not constant and exhibits a
drift over time (especially at high flux values) due to internal heating.  Such
non-idealities, if not accounted for, can cause uncertainty in photon timestamp
measurements, and limit the usability of our flux estimator in the high photon
flux regime.  Since we capture timestamps for every photon detected in a fixed
exposure time, it is possible to correct these non-idealities in
post-processing by estimating the true dead-time and rise-time from these
inter-photon timing histograms. See \nolink{\ref{suppl:dead time drift}} for
details.

\smallskip
\noindent\textbf{IP-SPAD Imaging with Low Photon Counts:}
The results so far show that precise photon timestamps from an IP-SPAD
pixel enables imaging at extremely high photon flux levels. We now show that it
is also possible to leverage timing information when the IP-SPAD pixel captures
very few photons per pixel. We simulate the low photon count regime by keeping
the first few photons and discarding the remaining photon timestamps for each
pixel in the experimental ``Tunnel'' scene. Fig.~\ref{fig:low_counts} shows
IP-SPAD images captured with as few as 1 and 10 photons per pixel and denoised
using an off-the-shelf BM3D denoiser and a deep neural network denoiser that
uses a kernel prediction network (KPN) architecture \cite{burstkpn_2018}. We
can recover intensity images with just one photon timestamp per pixel using
real data captured by our IP-SPAD hardware prototype.  Quite remarkably, with
as few as 10 photons per pixel, image details such as facial features and text
on the fire truck are discernible. Please see
\nolink{\ref{suppl:timing_usefulness}} for details about the KPN denoiser and
\nolink{\ref{suppl:additional_results}} for additional experimental results and
comparisons with other denoising algorithms.

