\section{Introduction}
Since systems are getting increasingly complex, models are needed to better comprehend their behaviors. 
Models can assist with understanding and optimizing the overall system, discovering causes and effects, measuring consequences of changes, and communicating among engineers. 
Once the system is deployed, the models used to build it should not be discarded. 
Instead, they should be integrated into a DT. 

A DT is a system that monitors its PT counterpart, and can reconfigure it \cite{Wright2020,Tao2019a}.
The DT employs calibration algorithms to update the models of the PT with estimates of the parameters.
The calibration is carried out through (co-)simulation inside the DT.
The resulting data is then used to inform experts that check whether the system is performing safely and optimally.
DTs enable advanced visualizations, reconfigurability, safety, predictability, and reduced maintenance. 
However, these benefits do not come without challenges. 

Accurately estimating the states of a PT is the basis for any other visualization and self-adaptation algorithms, but it needs to overcome noise in data, network delays, and the potential lack of sensor signals. 
Fortunately, the use of prediction models enables us to overcome these challenges. This can be achieved by correlating the measured behaviors of the PT with the one predicted by the model, as well as by estimating missing measurements based on the existing ones. 
To this end, the Kalman Filter (KF) \cite{thrun2005probabilistic} can be used. 
The KF provides a way to estimate the state of a dynamic system based on its past behavior, as it is expressed by multiple sequential measurements from sensors, that is better than the estimate obtained by using only one measurement alone.

In this tutorial, we provide an introduction and detailed derivation of the KF, assuming that the reader is familiar with basic statistics.
We use, as a running example, an incubator system \cite{feng2021}: a simple styrofoam insulated container with the ability to reach a certain temperature within a box and regulate it regardless of the content. We demonstrate the use of KF to estimate the states of the air inside the incubator and the heatbed (a heating unit inside the incubator) in the scenario that only the measurement data of the air inside the box is accessible.

The ability to predict the states of the system also enables anomaly detection. This can be done by the DT by detecting that the predictions from its models are no longer aligned with the data from the PT indicate.
In our work, we opened the lid of the incubator, which we can detect as a fault in the system. 
By comparing the states estimated by the KF and the measurements from the sensor, the fault is detected.

The rest of the paper is organized as follows: \cref{sec:incubator_system} describes the incubator system including the PT and the DT, and \cref{sec:kf} presents the derivation of the KF. \Cref{sec:anomaly_detection} demonstrates the use of KF for anomaly detection in the incubator system. Finally, we provide the conclusions in \cref{sec:conclusion}.