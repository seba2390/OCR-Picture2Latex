\section*{Abstract}

A digital twin (DT) monitors states of the physical twin (PT) counterpart and provides a number of benefits such as advanced visualizations, fault detection capabilities, and reduced maintenance cost. It is the ability to be able to detect the states inside the DT that enable such benefits. In order to estimate the desired states of a PT, we propose the use of a Kalman Filter (KF). In this tutorial, we provide an introduction and detailed derivation of the KF. We demonstrate the use of KF to monitor and anomaly detection through an incubator system. Our experimental result shows that KF successfully can detect the anomaly during monitoring.
% \hao{Needs to be finished}

\textbf{Keywords:} state estimate, kalman filter, runtime monitoring, anomaly detection
% \textbf{Keywords:} instructions, author’s kit, conference publication. (3-5 keywords separated by a comma.)
%% AUTHOR:
% This is a list of no more than five keywords that will identify your paper in indices and databases (required).
% Do not use the words “computer”, “simulation”, “model”, or “modeling”, since these are all assumed.