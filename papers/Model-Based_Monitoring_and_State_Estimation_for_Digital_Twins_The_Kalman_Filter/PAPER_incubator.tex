\section{Incubator Digital Twins} \label{sec:incubator_system}

Incubator DTs system consists of a PT and a corresponding DT. 
The DT replicates the essential features of the PT and enables some techniques increasing the value of the PT.

\paragraph{Physical Twin.}
The incubator is a system with the ability to hold a desired temperature within an insulated container.
The physical components include a controller and a plant. The controller is running on a Raspberry PI, and the plant includes a Styrofoam box, a heater, a fan, and three temperature sensors.
The overall systematic diagram of the Incubator is shown in \cref{fig:incubator}.
For more details, please refer to the report in \cite{feng2021} and the online documentation at \url{https://github.com/INTO-CPS-Association/example-incubator}.

\begin{figure}[h!]
	\centering
	\includegraphics[scale=0.85]{figures/system.pdf}
	\caption{Schematic overview of the Incubator.}
	\label{fig:incubator}
\end{figure}

\paragraph{Digital Twin}
In a DT, we highlight the following important parts \cite{Wright2020} (inspired by autonomic computing \cite{Kephart2003}): 
\begin{inparadesc} 
	\item[Data:] collected from the PT through sensors and actuators over time;
	\item[Models:] representing knowledge about different aspects of both the cyber and the physical of the PT and its environment; and
	\item[Algorithms:] representing techniques that use data and models, manipulating those to generate more data and knowledge (e.g., fault detectors, supervisory controllers, state estimators, optimizers).
\end{inparadesc} 

The data in the PT are transmitted through the Raspberry PI using a RabbitMQ server, which makes it easily accessible to other algorithms. 
In order to study the dynamics of temperature inside the incubator, a dynamic model is necessary. It is worth noting that models can be described at different levels of abstraction, and typically, accurate models are also very slow to simulate.
We experimented with different models, which can be found online, and selected the one that offered the best compromise between accuracy and performance.

Usually a physical dynamic system is modeled in the form of differential equations organized in a state-space model in continuous time.
However, in order to represent such system as a discrete process that produces samples at some finite frequency, it is necessary to transform the continuous-time state space model to a discrete-time state space model, based on a well known process called discretization.
We therefore focus on linear discrete-time dynamical systems, and refer the reader to \cite{Li2005} for details on discretization.

The general form of a linear discrete-time dynamical system is
\begin{equation} \label{equ:state_space_equation_discrete}
		{x}_k = A x_{k-1} + B u_k, \hspace{3em}
		y_k =C x_k,
\end{equation}
where $x_k$ and $x_{k-1}$ are state vectors, $u_k$ is the input vector, and $y_k$ is the measurement vector. 
The subscripts indicate the time step. 
In addition, $A$, $B$, and $C$ are matrices expressing the rule of how the system states advances and how the measurements relate to the state. 

The dynamics model described below mainly focuses on describing the temperature behaviors inside the Styrofoam box, and assume that:
the temperature inside the box is distributed uniformly; 
the box walls do not accumulate heat; 
and the specific heat capacity of the air inside the box is constant.
For the full definition of the model please refer to \cite{feng2021}.

\begin{example}\label{ex:incubator}
  The incubator system is described by the following linear discrete time system:
  \begin{equation} \label{equ:model_b}
     	\begin{bmatrix}
     		T^{h}_k\\
     		T^{b}_k
     	\end{bmatrix}
     	=
     	A
     	\begin{bmatrix}
     		T^{h}_{k-1}\\
     		T^{b}_{k-1}
     	\end{bmatrix}
     	+ B
     	\begin{bmatrix}
     		P_k\\
     		T^r_k
     	\end{bmatrix}, \hspace{3em}
    y_k =	
    	\begin{bmatrix}
    		0 & 1
    	\end{bmatrix}
    		\begin{bmatrix}
    				T^{h}_k\\
    				T^{b}_k
    		\end{bmatrix}
        = T^{b}_k,
  \end{equation}
  where $T^{h}$, $T^{b}$, and $T^{r}$ are the temperature of heater, the temperature of the air inside the incubator, and the temperature of the room, respectively. $P$ represents the power supply on/off function. $A$ and $B$ are $2$ by $2$ matrices. 
\end{example}

As can be seen from \cref{ex:incubator}, the measurements consist only of the temperature $T^{b}$, but the state of the model consists of both $T^{b}$ and $T^{h}$. 
We will now explore different ways to estimate the state in \cref{ex:incubator}.

\paragraph{State Estimation under Exact State.}
Let us assume that we know exactly the initial state of the system $T^{b}_0,T^{h}_0$, and the initial input $P_1, T^r_1$.
We can then simply apply the expressions in \cref{ex:incubator} to obtain the next predicted state $T^{b}_1,T^{h}_1$.
If we want to know the $k$-th state, and we have the history of inputs, denoted as $u_{1:k}=u_1,u_2,u_3,\ldots,u_{k-1}$, then we simply have to apply the expressions in \cref{ex:incubator} iteratively, from the given initial state.
This is no different that simulating the system.

\paragraph{State Estimation under Uncertain State with No Measurement.}
Now imagine that the initial state $T^{b}_0$ is known exactly, but $T^{h}_0$ is not known exactly. 
Instead, we know that it follows a Gaussian distribution with mean $\mu_{T^{h}_0}$, and variance $\Sigma_{T^{h}_0}$:
$
T^{h}_0 \sim \mathcal{N}(\mu_{T^{h}_0},\Sigma_{T^{h}_0}).
$
Suppose we want to estimate the most likely state $T^{b}_1,T^{h}_1$.
If we have no access to output measurements, then we could simply apply the expressions in \cref{ex:incubator} to the most likely initial state.
Since we know that the most likely output of a Gaussian process is its mean (recall that the Gaussian distribution resembles a bell-shaped distribution), we would then apply \cref{equ:model_b} to the initial state $\mu_{T^{h}_0},T^{b}_0$.

\paragraph{State Estimation under Uncertain State with Measurement.}
Following on the previous paragraph, imagine that a measurement for $T^{b}_1$ is given, in addition to the initial state $T^{b}_0$ and inputs $P_1, T^r_1$. Can we do better than just use $\mu_{T^{h}_0}$?
The answer is yes, because $T^{b}_1$ provides us with some extra information to try to find what the real initial state was.
We can insert the known values into \cref{equ:model_b} and estimate the unknown value $T^{h}_0$.
Expanding the $2 \times 2$ matrices, and focusing on the expression $T^{b}_1$, we obtain
$
T^{b}_1 = A_{21} T^{h}_0 + A_{22} T^{b}_0 + B_{21} P_1 + B_{22} T^r_1,
$
which can be solved to yield the solution to $T^{h}_0$. Then we insert the values of $T^{h}_0$, $T^{b}_0$, and inputs $P_1, T^r_1$ into \cref{equ:model_b} to acquire the value of $T^{h}_1$. 
Here we can see that $T^{h}_1$ is dependent on $T^{b}_1$. 
If the measurement $T^{b}_1$ is noisy, then $T^{h}_1$ is noisy as well. 
In addition, if the model contains process noises as well, this makes the uncertainty of $T^{h}_1$ larger. 
In practice, it is not likely for measurements and models to be noise free.
This leads to the third situation where measurements and models are noisy. 


\paragraph{State Estimation under Process Noise.}
Now imagine that there is noise in the state transition function, and in the measurements, as summarized in the following example.

\begin{example}\label{ex:incubator_noise}
  The incubator system with noise is described by the following linear discrete time system:
  \begin{equation} \label{equ:model_b_noisy}
     	\begin{bmatrix}
     		T^{h}_k\\
     		T^{b}_k
     	\end{bmatrix}
     	=
     	A
     	\begin{bmatrix}
     		T^{h}_{k-1}\\
     		T^{b}_{k-1}
     	\end{bmatrix}
     	+ B
     	\begin{bmatrix}
     		P_k\\
     		T^r_k
     	\end{bmatrix} 
      + \epsilon_k, \hspace{3em}
    y_k = T^{b}_k + \delta_k,
  \end{equation}
  where $\epsilon_k$ and $\delta_k$ are random variables representing process noise and measurement noise, satisfying $\epsilon_k \sim \mathcal{N}(0,R_k)$ and $\delta_k \sim \mathcal{N}(0,Q_k)$, respectively.
\end{example}

Given values for the initial states $T^{b}_0,T^{h}_0$, inputs $P_1, T^r_1$, and measurement $T^{b}_1$, we want to estimate the value of $T^{h}_1$.
In this case we cannot simply apply \cref{equ:model_b_noisy} because 
the measurement and the model are noisy. 

\paragraph{Goal of the Kalman Filter.}
Applied to the incubator case, the KF combines measurements $T^{b}_k$,
inputs $P_k, T^r_k$, 
Gaussian distribution parameters $\mu_{T^{h}_{k-1}},\Sigma_{T^{h}_{k-1}}$,
and a model in \cref{equ:model_b_noisy}, to estimate the parameters 
$\mu_{T^{h}_{k}},\Sigma_{T^{h}_{k}}$:
\begin{equation} \label{equ:kf_summary}
  \begin{bmatrix}
		\mu_{T^{h}_{k}}\\
		\Sigma_{T^{h}_{k}}
	\end{bmatrix}
	=
	\mathit{KF}(
  	\begin{bmatrix}
  		\mu_{T^{h}_{k-1}}\\
  		\Sigma_{T^{h}_{k-1}}
  	\end{bmatrix},
  	\begin{bmatrix}
  		P_k\\
  		T^r_k
  	\end{bmatrix},
    T^{b}_k
  )
  \text{, \ \ \ with }
  T^{h}_K \sim \mathcal{N}(\mu_{T^{h}_k},\Sigma_{T^{h}_k}).
\end{equation}
In the following section, we discuss how to derive KF. 