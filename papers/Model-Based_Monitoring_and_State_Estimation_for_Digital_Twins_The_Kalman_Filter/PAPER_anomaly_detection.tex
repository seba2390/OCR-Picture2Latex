\section{Anomaly Detection for incubator digital twin} \label{sec:anomaly_detection}

In this section we demonstrate the use of KF for anomaly detection for an incubator system. 
In the incubator system, we have the model predicting the states of the system. Furthermore, we have three sensors to measure the room and air temperature inside the incubator.
In the meanwhile, the KF is used to estimate and the states of the system using the measurement data and the predicted states from the DT. 
Furthermore, if an anomaly occurs in the system, the assumption that the process noise is Gaussian no longer holds true. This makes the predictions from KF no longer align with the sensory data from the physical twin indicate. Thus KF can be used for anomaly detection.  

In our case, we completely opened the lid of the incubator system when it was running. 
We viewed such an action as an anomaly happening in the system, because the model we used to do the state estimation only represents the behaviors of the system with a closed lid.
Opening it causes the system to violate the physical principles the model was originally built on, on, and will therefore make the KF fail to track what is happening.
The result can be seen in \cref{fig:anomaly_detection}. 
The orange line is the control signal of the heated, \enquote{high} means turning on the heated and \enquote{low} is turning off. 
Green line, purple line, and blue line are predicted state form the model, estimated state from KF, and measurement from the sensors respectively. 
\begin{figure}[h!] 
	\centering
	\includegraphics[width=\linewidth]{figures/Anomaly_Detection.png}
	\caption{results of anomaly detection for incubator digital twin}
	\label{fig:anomaly_detection}
\end{figure}

During the experiment, we opened the lid at $14:59:58$ and $15:09:19$. After opening the lid for around $1$ minute, we closed the lid at $15:00:58$ and $15:10:25$ respectively.
As can be seen in \cref{fig:anomaly_detection}, after opening the lid at $14:59:58$, the anomaly was detected since there was a discrepancy between the estimated state from KF and the sensory data. 

After closing the lid, the purple line and the blue line were merging gradually, i.e.\ the discrepancy reduced gradually. Before they converge together we opened the lid again, the discrepancy started to increase again. Should we have waited longer before opening the lid again, the discrepancy would have become virtually zero.

Through the running example, the KF successfully demonstrated the ability of state estimation. Such estimation succeeds in detecting an anomaly in the incubator system. Furthermore, this ability can be extended to other applications such as system monitoring. 

