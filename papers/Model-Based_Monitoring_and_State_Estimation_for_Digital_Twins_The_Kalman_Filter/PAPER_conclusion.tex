\section{Conclusion} \label{sec:conclusion}

In this tutorial, we provided the derivation of the KF from the perspective of probability theory. We started with a linear dynamical system. 
Based on MAP, maximizing the posterior of the state $x_k$ is objective function. We split it into a prediction phase and measurement phase for the derivation of the posteriori. 
We proved that the each of the phases resulted a Gaussian distribution. 
Thus the posterior of the estimated state is Gaussian distributed. 
Such Gaussian distribution enables us to obtain best estimation of state $x_k$ which is the expectation of the distribution. 

We demonstrated the use of KF through a running example, the incubator system. 
We modeled our incubator system as a linear dynamical system in the matrix form. Such form can be substituted into \cref{equ:state_space_equation_noise}. 
We applied the KF to estimate the states of the system and to detect a system anomaly. In the experiments, we completely opened the lid of the incubator system when it was running and the KF successfully detected the anomaly, which is illustrated in \cref{fig:anomaly_detection}.
This feature can be further used for more advance applications. 
