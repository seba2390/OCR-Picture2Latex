%%
%% This is file `sample-sigconf.tex',
%% generated with the docstrip utility.
%%
%% The original source files were:
%%
%% samples.dtx  (with options: `sigconf')
%% 
%% IMPORTANT NOTICE:
%% 
%% For the copyright see the source file.
%% 
%% Any modified versions of this file must be renamed
%% with new filenames distinct from sample-sigconf.tex.
%% 
%% For distribution of the original source see the terms
%% for copying and modification in the file samples.dtx.
%% 
%% This generated file may be distributed as long as the
%% original source files, as listed above, are part of the
%% same distribution. (The sources need not necessarily be
%% in the same archive or directory.)
%%
%% The first command in your LaTeX source must be the \documentclass command.
\documentclass[sigconf,natbib=true,anonymous=false]{acmart}

\usepackage{multirow}
\usepackage{threeparttable}
% \usepackage[ruled,vlined]{algorithm2e}
\usepackage[ruled]{algorithm2e}
\usepackage{caption}
\usepackage{subcaption}
\usepackage{makecell}
% \usepackage{graphicx}
% \usepackage{subfigure}
% \usepackage{enumitem}
% \usepackage{fancyhdr,graphicx,amsmath,amssymb}
% \usepackage[ruled,vlined]{algorithm2e}
% \usepackage{booktabs}
% % \usepackage[noend]{algpseudocode}
% % \usepackage{algorithmicx,algorithm2e}
% \usepackage{cite}
% \usepackage[numbers,sort&compress]{natbib}
\newcommand{\modelname}{MICQA}
\newcommand{\rerankname}{post-ranker}
%% NOTE that a single column version may be required for 
%% submission and peer review. This can be done by changing
%% the \doucmentclass[...]{acmart} in this template to 
%% \documentclass[manuscript,screen]{acmart}
%% 
%% To ensure 100% compatibility, please check the white list of
%% approved LaTeX packages to be used with the Master Article Template at
%% https://www.acm.org/publications/taps/whitelist-of-latex-packages 
%% before creating your document. The white list page provides 
%% information on how to submit additional LaTeX packages for 
%% review and adoption.
%% Fonts used in the template cannot be substituted; margin 
%% adjustments are not allowed.
%%
%%
%% \BibTeX command to typeset BibTeX logo in the docs
\AtBeginDocument{%
  \providecommand\BibTeX{{%
    \normalfont B\kern-0.5em{\scshape i\kern-0.25em b}\kern-0.8em\TeX}}}

%% Rights management information.  This information is sent to you
%% when you complete the rights form.  These commands have SAMPLE
%% values in them; it is your responsibility as an author to replace
%% the commands and values with those provided to you when you
%% complete the rights form.
\setcopyright{acmcopyright}
\copyrightyear{2018}
\acmYear{2018}
\acmDOI{10.1145/1122445.1122456}

%% These commands are for a PROCEEDINGS abstract or paper.
\acmConference[Woodstock '18]{Woodstock '18: ACM Symposium on Neural
  Gaze Detection}{June 03--05, 2018}{Woodstock, NY}
\acmBooktitle{Woodstock '18: ACM Symposium on Neural Gaze Detection,
  June 03--05, 2018, Woodstock, NY}
\acmPrice{15.00}
\acmISBN{978-1-4503-XXXX-X/18/06}


%%
%% Submission ID.
%% Use this when submitting an article to a sponsored event. You'll
%% receive a unique submission ID from the organizers
%% of the event, and this ID should be used as the parameter to this command.
%%\acmSubmissionID{123-A56-BU3}

%%
%% The majority of ACM publications use numbered citations and
%% references.  The command \citestyle{authoryear} switches to the
%% "author year" style.
%%
%% If you are preparing content for an event
%% sponsored by ACM SIGGRAPH, you must use the "author year" style of
%% citations and references.
%% Uncommenting
%% the next command will enable that style.
%%\citestyle{acmauthoryear}

%%
%% end of the preamble, start of the body of the document source.
\begin{document}

%%
%% The "title" command has an optional parameter,
%% allowing the author to define a "short title" to be used in page headers.
% \title{Curriculum-Guided Question Answering in Multi-Round Dialogue Systems}
\title{Multifaceted Improvements for Conversational Open-Domain Question Answering}
%%
%% The "author" command and its associated commands are used to define
%% the authors and their affiliations.
%% Of note is the shared affiliation of the first two authors, and the
%% "authornote" and "authornotemark" commands
%% used to denote shared contribution to the research.
\settopmatter{authorsperrow=3}

% \author{Tingting Liang}
% \authornote{Both authors contributed equally to this research.}
% \email{liangtt@hdu.edu.cn}
% % \orcid{1234-5678-9012}
% \author{Yixuan Jiang}
% \authornotemark[1]
% \email{webmaster@marysville-ohio.com}
% \affiliation{%
%   \institution{Institute for Clarity in Documentation}
%   \streetaddress{P.O. Box 1212}
%   \city{Dublin}
%   \state{Ohio}
%   \country{USA}
%   \postcode{43017-6221}
% }

\author{Tingting Liang}
\authornote{Both authors contributed equally to this research.}
\affiliation{%
  \institution{Hangzhou Dianzi University}
  \city{Hangzhou}
  \country{China}
}
\email{liangtt@hdu.edu.cn}

\author{Yixuan Jiang}
\authornotemark[1]
\affiliation{%
  \institution{Hangzhou Dianzi University}
  \city{Hangzhou}
  \country{China}
}
\email{jyx201050027@hdu.edu.cn}

\author{Congying Xia}
\affiliation{%
  \institution{University of Illinois at Chicago}
  \city{Chicago}
  \country{US}}
\email{cxia8@uic.edu}

\author{Ziqiang Zhao}
\affiliation{%
  \institution{Hangzhou Dianzi University}
  \city{Hangzhou}
  \country{China}
}
\email{zhaoziqiang@hdu.edu.cn}

\author{Yuyu Yin}
\authornote{Corresponding author}
\affiliation{%
  \institution{Hangzhou Dianzi University}
  \city{Hangzhou}
  \country{China}
}
\email{yinyuyu@hdu.edu.cn}

\author{Philip S. Yu}
\affiliation{%
  \institution{University of Illinois at Chicago}
  \city{Chicago}
  \country{US}}
\email{psyu@uic.edu}


%%
%% By default, the full list of authors will be used in the page
%% headers. Often, this list is too long, and will overlap
%% other information printed in the page headers. This command allows
%% the author to define a more concise list
%% of authors' names for this purpose.
\renewcommand{\shortauthors}{Liang and Jiang, et al.}

%%
%% The abstract is a short summary of the work to be presented in the
%% article.
\begin{abstract}
Open-domain question answering (OpenQA) is an important branch of textual QA which discovers answers for the given questions based on a large number of unstructured documents. Effectively mining correct answers from the open-domain sources still has a fair way to go. Existing OpenQA systems might suffer from the issues of question complexity and ambiguity, as well as insufficient background knowledge.
Recently, conversational OpenQA is proposed to address these issues with the abundant contextual information in the conversation.
Promising as it might be, there exist several fundamental limitations including the inaccurate question understanding, the coarse ranking for passage selection, and the inconsistent usage of golden passage in the training and inference phases.
To alleviate these limitations, in this paper, we propose a framework with Multifaceted Improvements for Conversational open-domain Question Answering ({\modelname}). Specifically, {\modelname} has three significant advantages. First, the proposed KL-divergence based regularization is able to lead to a better question understanding for retrieval and answer reading. Second, the added post-ranker module can push more relevant passages to the top placements and be selected for reader with a two-aspect constrains. Third, the well designed curriculum learning strategy effectively narrows the gap between the golden passage settings of training and inference, and encourages the reader to find true answer without the golden passage assistance.
Extensive experiments conducted on the publicly available dataset OR-QuAC demonstrate the superiority of {\modelname} over the state-of-the-art model in conversational OpenQA task.

%  encourages the retriever and post-ranker to find the passage with the true answer contained by itself.


\end{abstract}

%%
%% The code below is generated by the tool at http://dl.acm.org/ccs.cfm.
%% Please copy and paste the code instead of the example below.
%%
\begin{CCSXML}
<ccs2012>
 <concept>
  <concept_id>10010520.10010553.10010562</concept_id>
  <concept_desc>Computer systems organization~Embedded systems</concept_desc>
  <concept_significance>500</concept_significance>
 </concept>
 <concept>
  <concept_id>10010520.10010575.10010755</concept_id>
  <concept_desc>Computer systems organization~Redundancy</concept_desc>
  <concept_significance>300</concept_significance>
 </concept>
 <concept>
  <concept_id>10010520.10010553.10010554</concept_id>
  <concept_desc>Computer systems organization~Robotics</concept_desc>
  <concept_significance>100</concept_significance>
 </concept>
 <concept>
  <concept_id>10003033.10003083.10003095</concept_id>
  <concept_desc>Networks~Network reliability</concept_desc>
  <concept_significance>100</concept_significance>
 </concept>
</ccs2012>
\end{CCSXML}

\ccsdesc[500]{Computer systems organization~Embedded systems}
\ccsdesc[300]{Computer systems organization~Redundancy}
\ccsdesc{Computer systems organization~Robotics}
\ccsdesc[100]{Networks~Network reliability}

%%
%% Keywords. The author(s) should pick words that accurately describe
%% the work being presented. Separate the keywords with commas.
\keywords{Open-Domain, Conversational Question Answering, Curriculum Learning}

%%
%% This command processes the author and affiliation and title
%% information and builds the first part of the formatted document.
\maketitle

\section{Introduction}
\label{sec1:introduction}
\vspace{-5pt}

Language Models (LMs) have opened up a new era in Natural Language Processing (NLP) by leveraging extensive datasets and billions of parameters\,\citep{llm_survey, gpt4_report, scaling_law}. These LMs excel at In-Context Learning (ICL), generating responses based on a few demonstrations without needing further parameter adjustments\,\citep{emergent_abilities_llms, gpt3, icl_survey}. The rise of instruction-tuning has further enhanced this capability, optimizing LMs to align their outputs closely with human-specified instructions\,\citep{flan, t0, gpt3, lms_are_unsupervised_multitask_learners}. This approach has demonstrated a significant improvement in zero-shot scenarios, underscoring its importance for tackling diverse tasks.

However, instruction-tuned models often struggle with unfamiliar tasks due to limitations in their training datasets, whether the datasets are human-annotated\,\citep{ni_dataset, sni_dataset} or model-generated\,\citep{self_instruct, unnatural_ni_dataset}. Refining these datasets is essential but requires substantial effort and computational resources, highlighting the need for more efficient approaches\,\citep{flan_t5, lima}. Moreover, the depth of a model's understanding of and how they respond to instructions remains an area of active research. While recent studies have provided some insights\,\citep{do_really_follows_instructions, did_you_read_instructions}, many questions remain unanswered. Techniques such as prompt-engineering\,\citep{prompt_analysis} and utilizing diversified outputs\,\citep{self_consistency} aim to increase the quality of outputs. However, the effectiveness of these techniques often depends on the fortuitous alignment of prompts or initial conditions, making them labor-intensive since the tuning process must be tailored.

In pursuit of refining the behavior of LMs, some researchers have begun to explore the \textit{anchoring effect}\,\citep{kahneman1982judgment}—a well-known cognitive bias where initial information exerts disproportionate influence on subsequent judgments. Intriguingly, this cognitive principle has been demonstrated to extend to LMs. For example, through effective prompting, the outputs generated by LMs can be steered towards a specific intent\,\citep{jones2022capturing}. Similarly, emphasizing the first few sentences of a long context enhances the model's overall comprehension of the content\,\citep{coherence_boosting}. Given these observations on LMs—parallels that mirror human tendencies—and the influential role of initial prompts, we hypothesize that the strategic application of the anchoring effect could substantially improve LMs' fidelity to instructions.

In this work, we propose \textit{Instructive Decoding}\,(ID)\,(\autoref{fig:main}), a novel method that enhances the attention of instruction-tuned LMs towards provided instructions during the generation phase without any parameter updates. The core idea of ID is deploying \textit{noisy} variants of instructions, crafted to induce a clear \textit{anchoring effect} within the LMs, to adjust the output anchored by the original instruction. More precisely, this effect aims to steer the models toward particular, potentially sub-optimal predictions. Our range of variants spans from simple strategies such as instruction truncation and more aggressive alterations, the most extreme of which is the \textit{opposite} instruction. By intentionally introducing such deviations, ID capitalizes on the resulting disparities. Within a contrastive framework, next-token prediction logits that are influenced by the noisy instructions are systematically compared to those derived from the original instruction. This process refines the model's responses to align more closely with the intended instruction.

\begin{figure}[t!]
\centering
\vspace{-10pt}
\includegraphics[width=\textwidth]{materials/figures/fig1_main_fix.pdf}
\vspace{-10pt}
\caption{Overview of Instructive Decoding\,(ID). The example in this figure is from \texttt{task442\_com\_qa\_paraphrase\_question\_generation} in \textsc{SupNatInst}\,\citep{sni_dataset}. The original response not only fails to meet the task requirements (Question Rewriting) but also contains incorrect information\protect\footnotemark. In contrast, ID generates a more relevant response by refining its next-token predictions based on the noisy instruction (here, opposite prompting is used for ID).}
\vspace{-12pt}
\label{fig:main}
\end{figure}
%
\footnotetext{According to the \href{https://population.un.org/wpp/}{2022 U.N. Revision}, the population of USA is approximately 338.3 million as of 2022.}

\begin{wrapfigure}{r}{0.54\textwidth}
\vspace{-15pt}
    \includegraphics[width=1.0\linewidth]{materials/figures/main_result_fig.pdf}
    \vspace{-15pt}
    \caption{Zero-shot Rouge-L comparison on the \textsc{SupNatInst} heldout dataset\,\citep{sni_dataset}. Models not instruction-tuned on \textsc{SupNatInst} are in \textcolor{blue}{blue} dotted boxes, while those instruction-tuned are in \textcolor{green!50!black}{green}.} 
    \label{fig:result_overall}
\vspace{-10pt}
\end{wrapfigure}

Experiments on unseen task generalization with \textsc{SupNatInst}\,\citep{sni_dataset} and \textsc{UnNatInst}\,\citep{unnatural_ni_dataset} held-out datasets show that instruction-tuned models enhanced by ID consistently outperform baseline models across various setups. Intriguingly, T\textit{k}-XL combined with our method outperforms its larger version, T\textit{k}-XXL, with standard inference\,(\autoref{fig:result_overall}). Models not previously trained on the \textsc{SupNatInst} dataset, including Alpaca (7B) and T0 (3B), also show marked enhancements in performance. Additionally, the overall Rouge-L score of the GPT3 (175B) is strikingly competitive, closely mirroring the performance of OpenSNI (7B) when augmented with our method. 
We further observe that ID's generation exhibits increased both adherence to the instruction and an improvement in semantic quality. To provide a comprehensive understanding, we investigate the anchoring effect of noisy instructions. Our findings suggest that as the model's comprehension of the noisy instruction intensifies, the anchoring effect becomes more potent, making ID more effective. Our main contributions are as follows:
%
\begin{itemize}
    \item We introduce \textit{Instructive Decoding}\,(ID), a novel method to enhance the instruction following capabilities in instruction-tuned LMs. By using distorted versions of the original instruction, ID directs the model to bring its attention to the instruction during generation \textbf{(\Autoref{sec2:method})}.
    %
    \item We show that steering the noisy instruction towards more degrading predictions leads to improved decoding performance. Remarkably, the \textit{opposite} variant, which is designed for the most significant deviation from the original instruction yet plausible, consistently shows notable performance gains across various models and tasks (\textbf{\Autoref{sec3:experiment}}).
    %
    \item We provide a comprehensive analysis of the behavior of ID, demonstrating its efficacy from various perspectives. The generated responses via ID also improve in terms of label adherence and coherence, and contribute to mitigate the typical imbalances observed in the standard decoding process. (\textbf{\Autoref{sec4:discuss}})
\end{itemize}

% \vspace{-0.15in}
\section{Preliminary}\label{sec:preliminary}
 

% and $\{q_i\}^{c-1}_{i=1}$ represents the historical questions. 

% \subsection{Dual Encoder Retrieval Model}

% In the retrieval process, the commonly used dual-encoder model \cite{bromley1993signature} consists of the question encoder $E_Q$ and the passage encoder $E_P$, which encode the question and passage into d-dimensional vectors respectively. Many similarity functions, such as inner product and Euclidean distance, can be used to indicate the relationship between questions and passages. Some tests in \cite{karpukhin2020dense} demonstrate that these comparable functions act similarly, so the simpler inner product is selected:

% \begin{equation}\label{eq:inner}%加*表示不对公式编号
% \begin{aligned}
% \mathrm{sim}(q,p) = E_Q(q)^\mathrm{T}E_P(p)
% \end{aligned}
% \end{equation}
% Typically, $E_Q$ and $E_P$ are two pre-trained model, e.g., BERT\cite{devlin2018bert}, ALBERT\cite{lan2020albert}. We picked the lighter ALBERT, which contains less model parameters and allows us to raise the batch size to have a higher training impact.

% \subsection{Rerank and Reader}

% Based on the retrieved passages from a first-stage retriever,


% Given the top k retrieved passages, the reader assigns a passage selection score to each passage. In addition, it extracts an answer span from each passage and assigns a span score. The probabilities of a token being the starting/ending positions of an answer span and a passage being selected are defined as:

\subsection{Retriever-Reader Pipeline}
The most typical OpenQA system follows a two-stage pipeline, which has two components: one retriever and one reader. 
% Retriever 
% Deep learning techniques enable the two components to be end-to-end trainable.

\subsubsection{Retriever}\label{subsec:Retrieving and Reading}
In the retrieval process, the commonly used dual-encoder model \cite{bromley1993signature} consists of a question encoder and a passage encoder, which encode the question and passage into low-dimensional vectors, respectively. Many similarity functions, such as inner product and Euclidean distance, can be used to measure the relationship between questions and passages. The evaluation in~\cite{karpukhin2020dense} demonstrates that these functions perform comparably, so the simpler inner product is selected in our work:
\vspace{-0.05in}
\begin{equation}\label{eq:inner}%加*表示不对公式编号
\begin{aligned}
\mathrm{sim}(q,p) = E_Q(q)^\mathrm{T}E_P(p).
\end{aligned}
\end{equation}
% Typically, 
where $q$ and $p$ denote a given question and passage. $E_Q$ and $E_P$ refer to the question encoder and passage encoder which typically are two pre-trained model, e.g., BERT\cite{devlin2018bert}, ALBERT\cite{lan2020albert}. 
% We picked the lighter ALBERT, which contains less model parameters and allows us to raise the batch size to have a higher training impact. 
The retriever score is defined as the similarly of representations of the question and passage.
Given a question $q$, the retriever derives a small subset of passages with embeddings closest to $q$ from a large corpus.

\subsubsection{Reader}\label{subsec:Retrieving and Reading}
In reading phase, a conventional BERT-based extractive machine comprehension model is usually used as the reader. The retrieved passages concatenated with the corresponding questions are first encoded separately. 
The reader then extracts the start and end tokens with the max probabilities among tokens from all the the top passages.
% The reader then maximizes the probabilities of finding the true start and end tokens among tokens from all of the top passages. 
The reader score is defined as the sum of the scores of start token and end token. The answer score is the sum of retriever score and reader score.
Taking the small collection of passages generated by the retriever as input, the reader outputs the answer span with the highest answer score.

\vspace{-0.05in}
\subsection{Problem Definition}
Following the task definition in ~\cite{qu2020open}, the conversational OpenQA problem can be formulated as follows: Given the current question $q_c$, and a set of historical question-answer pairs $\{(q_i, a_i)\}^{c-1}_{i=1}$, where $q_i$ denotes the $i$-th question and $a_i$ is the related answer in a conversation, the task is to identify answer spans for the current question $q_c$ from a large corpus of articles, \emph{e.g.,} Wikipedia.

Based on the introduced two-stage pipeline, this work applies a multi-stage pipeline of one pretrained retriever, one post-ranker, and one reader. The details of the framework and training process of {\modelname} are introduced in the next section.
% Some preliminaries for the pipeline are introduced as follows.  


% \vspace{-0.05in}
\section{Model}\label{sec:model}
The overall framework of our {\modelname} is shown in Figure~\ref{fig:model}.
{\modelname} mainly consists of three components: (1) one regularized \textcolor{black}{retriever} for relevant passages discovery, (2) one \textcolor{black}{contrastive {\rerankname} for improving passage retrieval quality}, and (3) one \textcolor{black}{reader} for detecting answers from a small collection of ranked passages. These three components work as follows: With the passage representations offline encoded by the pre-trained passage encoder, the retriever outputs the top $K$ passages by operating inner product between them and the embedding of a given question generated by the question encoder. The post-ranker takes the initially retrieved top $K$ passage embeddings and question embedding as input, and selects the top $T$ of the newly ranked passages as output. The outputted passages are respectively concatenated with the given question, and the concatenations are fed into the reader to find the final answer. 
The answer score is decided by the post-ranker score and reader score, which would be introduced in details in Section~\ref{subsec:inference}. 
The retriever is pre-trained with a KL-divergence based regularization. Then the passage encoder of it is frozen and the question encoder can be fine-tuned with the other two components in the next joint training process in an end-to-end manner.
% All the three components are learnable in the end-to-end manner.
In the following part, we would describe the procedures of pre-training, joint training, and inference, along with the details of each component.

% The details of each component would be described in the following part.

% \vspace{-0.15in}
\subsection{KL-Divergence based Regularization for Retriever Pre-training}\label{sec:kl}
% 原先模型存在的问题
% 引出对比学习
% So, we introduce contrast learning to improve the accuracy of mapping functions from feature space to semantic space.
% 下标k是指当前对话是第k轮,下标i是指在一批训练样本中的第i个
% \textcolor{red}{Introduction: Considerable previous works apply the dual-encoder architecture to obtain the initial retrieval results from the large corpus of articles, including ORConvQA, which only uses the rewrite form for each question in the pre-training phase and mitigates the mismatch problem by fine-tuning the question encoder in the concurrent learning phase~\cite{qu2020open}. However, the input questions significantly influence the offline representation learning for the Wikipedia passages and the retrieval results, which could not be revised in the later joint training phase.} 
In order to improve the capability of question understanding, we propose to exploit both the original questions and question rewrites in retriever pre-training. Specifically, a regularization mechanism is proposed to force two distributions of the retrieved results generated by feeding the original questions and their rewrites into the dual-encoder to be consistent with each other by minimizing the bidirectional KL divergence between them.
% hard negative: Strengthen the contrast

Concretely, given the current question $q_c$ and its historical question-answer pairs $\{q_i, a_i\}_{i=1}^{c-1}$, we form the original question by concatenating the question-answer pairs in a history window of size $w$ with the current question. Moreover, to mitigate the issue of underspecified and ambiguous initial questions, the initial question $q_1$ is invariably considered as it makes the constructed original question closer to the question rewrite. For the question encoder, the input sequence of original question can be defined as 
$q_c^{or} = \verb|[CLS]|\;q_1 \;\verb|[SEP]|\;a_1    \;\verb|[SEP]|\;q_{c-w}\verb|[SEP]|\;a_{c-w}\;\verb|[SEP]| \cdots \verb|[SEP]|\;q_{c-1}\\
\verb|[SEP]|\;a_{c-1}\;\verb|[SEP]|\;q_c\;\verb|[SEP]|$.
% $q_c^{or} = \verb|[CLS]|\;q_1 \;\verb|[SEP]|\;q_{c-w}\;\verb|[SEP]| \cdots \verb|[SEP]|\;q_{c-1}\;\verb|[SEP]|\;q_c\;\verb|[SEP]|$.

Let $\mathcal{D} = \{(q_i^{or}, q_i^{rw}, p_i^+, p_{i,1}^-, ... ,  p_{i,n}^-)\}_{i=1}^m$ denote the training data that consists of $m$ instances. Each instance contains two forms of one question (\emph{i.e.,} original question and question rewrite), one positive passage, along with $n$ irrelevant passages $\{p_{i,j}^-\}_{j=1}^n$. These irrelevant passages used for training contain one hard negative of the given question and $n-1$ in-batch negatives which are the positive and negative samples of the other questions from the same mini-batch.
% are hard negatives of the given question. Similar to DPR~\cite{karpukhin2020dense}, the in-batch negatives which are the positive and negative samples of the other questions from the same mini-batch are also used for training.
% and at training time, we also use in-batch negatives similar to DPR\cite{karpukhin2020dense} which are the positive and negative samples of other questions.
As shown in Figure~\ref{fig:kl}, we feed the original question $q_i^{or}$ and question rewrite $q_i^{rw}$ to the question encoder $E_Q$, respectively. The derived question embeddings are matched with the passage embeddings outputted by the passage encoder $E_P$ via inner product.
We can obtain two distributions of the retrieved results through softmax operation, denoted as $P^{or}(p_i|q_i^{or})$ and $P^{rw}(p_i|q_i^{rw})$. In the retrieval pre-training phase, we try to regularize on the retrieved results by minimizing the bidirectional KL divergence between the two distributions,
% for the different forms of one question, 
which can be formulated as follows:
% For two forms of one question, we feed them to question encoder respectively. Specifically, one is the concatenation of history questions and current question, denoted as $q_k^{rt}$ = [CLS] $q_1$ [SEP] $q_{k-w} $[SEP] ... [SEP] $q_{k-1}$ [SEP] $q_k$ [SEP], another is the rewrite in CANARD of $q_k$ denoted as $q_k^{rw}$.And for passages, we feed them to passage encoder twice. Therefore, we can obtain two distributions of the model prediction, denoted as $P^{rw}(p_i^+|q_i^{rw})$ and $ P^{rt}(p_i^+|q_i^{rt}) $. Then, inspired by \cite{liang2021rdrop}, we try to regularize on the model predictions by minimizing the bidirectional Kullback-Leibler (KL) divergence between these two output distributions, which is:
\begin{small}
\begin{equation}%加*表示不对公式编号
\begin{aligned}
\mathcal{L}_{KL}^i 
= & \frac{1}{2}(\mathcal{D}_{KL}(P^{or}(p_i|q_i^{or}) || P^{rw}(p_i|q_i^{rw})) \\
& + \mathcal{D}_{KL}(P^{rw}(p_i|q_i^{rw}) || P^{or}(p_i|q_i^{or}))).
\end{aligned}
\end{equation}
\end{small}

For the main task of passage retrieval, we apply the widely used negative log likelilhood of the positive passage given two forms of one question as objective function:
% With the basic negative log-likelihood learning objective $L_{CE}^i$ of the two forward passes:
% \begin{equation}
% \begin{aligned}
% & \mathcal{L}_{NLL}^i = - \frac{1}{2}(\mathrm{log}P^{or}(p_i^+|q_i^{or}) + \mathrm{log}P^{rw}(p_i^+|q_i^{rw})),
% \end{aligned}
% \end{equation}
\begin{small}
\begin{equation}
% \belowdisplayskip= 3pt
\begin{aligned}
\mathcal{L}_{NLL}^i 
= & - \frac{1}{2}(\mathcal{L}_{NLL\_or}^i+\mathcal{L}_{NLL\_rw}^i)\\
=&- \frac{1}{2}(\mathrm{log}P^{or}(p_i^+|q_i^{or}) + \mathrm{log}P^{rw}(p_i^+|q_i^{rw})),
\end{aligned}
\end{equation}
\end{small}
\\where the probability of retrieving the positive passage can be calculated as:
\begin{small}
\begin{equation}%加*表示不对公式编号
 P^{or}(p_i^+|q_i^{or}) 
 = \frac{\mathrm{exp}(\mathrm{sim}(q_i^{or}, p_i^+))}{\mathrm{exp}(\mathrm{sim}(q_i^{or}, p_i^+))+\sum_{j=1}^n \mathrm{exp}(\mathrm{sim}(q_i^{or}, p_{i,j}^-))},
\end{equation}
\end{small}
\begin{small}
\begin{equation}%加*表示不对公式编号
% \belowdisplayskip= 1.5pt
 P^{rw}(p_i^+|q_i^{rw})
 = \frac{\mathrm{exp}(\mathrm{sim}(q_i^{rw}, p_i^+))}{\mathrm{exp}(\mathrm{sim}(q_i^{rw}, p_i^+))+\sum_{j=1}^n \mathrm{exp}(\mathrm{sim}(q_i^{rw}, p_{i,j}^-))}.
\end{equation}
\end{small}
\\The probability of retrieving those negative passages can be obtained in the same manner.

To pre-train the retriever, 
% especially the passage encoder, 
the training objective is to minimize the pre-retrieval loss $\mathcal{L}^i_{pre}$ for data ($q_i^{or}$, $q_i^{rw}$, $p_i^+$, $p_{i,1}^-$, ... ,  $p_{i,n}^-)$:
\begin{equation}%加*表示不对公式编号
% \begin{aligned}
\mathcal{L}^i_{pre} = \mathcal{L}_{NLL}^i + \alpha \mathcal{L}_{KL}^i,
% \end{aligned}
\end{equation}
where $\alpha \in [0, 1]$ is a hyperparameter used  to control $\mathcal{L}_{KL}^i$.
\begin{figure}[t]
    \centering
    % \includegraphics[width=12cm]{figure/pretrain_v4.pdf}
    \includegraphics[width=\linewidth]{figure/pretrain_v4.pdf}
    \vspace{-0.2in}
    \caption{Retriever pre-training model with KL-divergence based regularization. Two forms of one question are fed into question encoder while the related positive and hard negative are fed into passage encoder. The positive or negative passages of the other questions in the same batch are used as negatives.}
    \label{fig:kl}
    \vspace{-0.1in}
\end{figure}

% \vspace{-0.05in}
\subsection{Joint Training with Curriculum Learning}
With the passage encoder and question encoder of the retriever pre-trained, our {\modelname} jointly trains the question encoder, post-ranker, and reader with a designed curriculum learning strategy.

% \vspace{-0.05in}
\subsubsection{Retriever Loss}\label{subsec:retriever}
The pre-trained passage encoder is used to embed the open-domain passages offline and obtain a set of passage representations. 
Given the current question $q_c$, and its historical question-answer pairs $\{q_i, a_i\}_{i=c-w}^{c-1}$ within a window size $w$, the input for question encoder is constructed by concatenating the question-answer pairs with the current question as mentioned in Section~\ref{sec:kl}. With the prepared offline passage representations and outputs of question encoder, each passages would be assigned with a retrieval score computed by the inner product operation as shown in Equation~\ref{eq:inner}. We select the top $K$ passages with high retrieval scores for the \textcolor{black}{{\rerankname} and reader}.

For each question, the reformatted question for the question encoder is denoted as $q_i^{en}$.
The question encoder of the retriever is fine-tuned by optimizing the following retrieval loss:
\begin{equation}\label{eq:retriever}
    \mathcal{L}^i_{retriever}=-\mathrm{log}\frac{\mathrm{exp}(\mathrm{sim}(q_i^{en},p_i^+))}{\sum_{j=1}^{K}\mathrm{exp}(\mathrm{sim}(q_i^{en},p_{i,j}))},
\end{equation}
where $p_{i,j}$ denotes the retrieved passages regarding question $q_i$.

For a fair comparison, we omit the answers from the concatenated question in the experiment to keep the same setting with the previous work~\cite{qu2020open}.
\subsubsection{Post-Ranker with Contrastive Loss}
Since only the first $T$ of the top $K$ passages outputted by the retriever module are fed into the reader for answer extraction, a {\rerankname} is proposed to re-rank the $K$ passages to push more relevant passages be ranked in the top $T$ placement.
We build the {\rerankname} by adding a subsequent network after the pre-trained passage encoder, which takes the embeddings of $K$ passages as input. With the subsequent network, {\rerankname} is able to learn high-level feature representations for passages.
In this work, we use the linear layer as the subsequent network for simplicity. The output for the retrieved passage $p_{i,j}$ and the ranking score can be formulated as follows:
\vspace{-0.1in}
\begin{equation}
    \mathbf{d}_{i,j} = \mathrm{LinearLayer}(E_P(p_{i,j})),
\end{equation}
% \vspace{-0.1in}
\begin{equation}\label{eq:score_post}
    S_{post}(q_i^{en},p_{i,j})=\mathbf{d}_{q_i}^{\mathrm{T}}\mathbf{d}_{i,j},
\end{equation}
where $\mathbf{d}_{q_i}$ is the representation for question $q_i$ generated by the question encoder $E_Q$. $S_{post}(\cdot)$ refers to the scores generated by the {\rerankname} network.
% with $\theta$ which includes the set of parameters of the linear layer. 

To simultaneously fine-tune the question encoder and {\rerankname}, we apply the modified hinge loss combined with a distance-based contrastive loss to pose constraints for passage reranking from two aspects. The hinge loss function~\cite{santos2016attentive} is defined as: 
\begin{small}
\begin{equation}\label{eq:rerank}
    \mathcal{L}^i_{ranker}=\mathrm{max}\{0,\delta-S_{post}(q_i^{en},p_i^+)+\mathrm{max}_j\{S_{post}(q_i^{en},p_{i,j}^-)\}\},
\end{equation}
% \belowdisplayskip= 0.5pt
\end{small}
where $\delta$ is the margin of the hinge loss, $p_i^+$ and $p_{i,j}^-$ denote the positive passage and negative passages retrieved for $q_i$. 
% $\mathbf{d}_i^+$ denotes the positive passage.
For contrastive learning, we uses the triplet margin loss~\cite{weinberger2006distance} to measure the distance between positive and negative samples as follows: 
\vspace{-0.05in}
\begin{equation}\label{eq:contrastive}
    \mathcal{L}^i_{cl}=\mathrm{max}\{0,\mu+D(\mathbf{d}_{q_i},\mathbf{d}^+_i)-D(\mathbf{d}_{q_i},\mathbf{d}_{i,j}^-)\},
    \vspace{-0.05in}
\end{equation}
where $\mu$ is the margin of the triplet margin loss. We apply Euclidean distance $D$ to compute the distance between the question and passage in the representation space. The final {\rerankname} loss is defined as:
\vspace{-0.1in}
\begin{equation}\label{eq:contrastive}
    \mathcal{L}^i_{postranker}= \mathcal{L}^i_{ranker}+\beta\mathcal{L}^i_{cl},
    \vspace{-0.05in}
\end{equation}
where $\beta$ is hyperparameter that need to be determined.

\subsubsection{Reader Loss}
The neural reader of our {\modelname} is designed for predicting an correct answer to the given question. Given the top $T$ retrieved/reranked passages, the reader intends to detect an answer span from each passage with a span score. Moreover, a passage selection score is assigned to each passage. The passage selection is used to select the passages that contains the true answer of the given question~\cite{lin2018denoising,karpukhin2020dense}, which performs as a contraint to facilitate the training of reader. The span with the highest passage selection score and span score is extracted as the final answer. 
% The passage selection is used to select the passages that contains the answer of the given question by measuring the probability distribution over all the retrieved passages.
% as a reranker to reselect the passages from a small number

Our reader applies the widely used pre-trained model, such as BERT~\cite{devlin2018bert}, RoBERTa~\cite{liu2019roberta}. Given a question $q_i$ and the top $T$ previously retrieved passages, we form the question input in the same way as that for retriever described in Section~\ref{subsec:retriever}. The only difference is that the initial question $q_1$ is left out as the setting of~\cite{qu2020open}. We denote the reconstructed question as $q_i^{rd}$ and concatenate it with a retrieved passage to form the sequence as the input for the pre-trained model. Specifically, let $\mathbf{z}_{i,j,t}$ be the outputted token-level representations of the $t$-th token in the passage $p_{i,j}$ which is retrieved for question $q_i^{rd}$. $\mathbf{z}_{i,j,[\mathrm{cls}]}$ denotes the sequence-level representation for the input sequence. The scores for the $t$-th token being the start and end tokens, and a passage being selected are defined as follows:
\begin{equation}\label{eq:score_reader1}
    S_{s}(q_{i}^{rd},p_{i,j},[t]) = \mathbf{z}_{i,j,t}^\mathrm{T}\mathbf{w}_{start},
\end{equation}
\begin{equation}\label{eq:score_reader2}
    S_{e}(q_{i}^{rd},p_{i,j},[t]) = \mathbf{z}_{i,j,t}^\mathrm{T}\mathbf{w}_{end},
\end{equation}
\begin{equation}\label{eq:score_select}
    S_{select}(q_{i}^{rd},p_{i,j})=\mathbf{z}_{i,j,[\mathrm{cls}]}^\mathrm{T}\mathbf{w}_{select},
\end{equation}
% \begin{align} 
%  S_{s}(q_{i}^{rd},p_{i,j},[t]) &= \mathbf{z}_{i,j,t}^\mathrm{T}\mathbf{w}_{start}, \label{eq:score_reader1}\\ 
%  S_{e}(q_{i}^{rd},p_{i,j},[t]) &= \mathbf{z}_{i,j,t}^\mathrm{T}\mathbf{w}_{end}, \label{eq:score_reader2}\\
%  S_{select}(q_{i}^{rd},p_{i,j})&=\mathbf{z}_{i,j,[\mathrm{cls}]}^\mathrm{T}\mathbf{w}_{select}, \label{eq:score_select}
% \end{align}
where $\mathbf{w}_{start}$, $\mathbf{w}_{end}$, and $\mathbf{w}_{select}$ are trainable vectors.

The loss function for the start token prediction is defined as:
\begin{equation}
    \mathcal{L}^i_{start}=-\mathrm{log}\frac{\mathrm{exp}(S_s(q_i^{rd},p_i^+,[st_i^+]))}{\sum_{j=1}^{M}\sum_{[t]\in p_{i,j}}\mathrm{exp}(S_s(q_i^{rd},p_{i,j}, [t]))},
\end{equation}
where $[st_i^+]$ denotes the start token of the true answer in the golden passage $p_i^+$.
% and $[t]$ is the token in the top $T$ retrieved passages.
The objective function for the end token prediction denoted as $\mathcal{L}^i_{end}$ is defined in the same manner. In addition, the passage selecting loss is computed as follows:
\begin{equation}
    \mathcal{L}^i_{select}=-\mathrm{log}\frac{\mathrm{exp}(S_{select}(q_i^{rd},p_i^+))}{\sum_{j=1}^{M} \mathrm{exp}(S_{select}(q_i^{rd},p_{i,j}))}.
\end{equation}
Finally, the loss function of the reader is defined as follows:
\begin{equation}\label{eq:reader}
    \mathcal{L}_{reader}^i = \frac{1}{2}(\mathcal{L}_{start}^i+\mathcal{L}_{end}^i)+\mathcal{L}_{select}^i.
\end{equation}

\subsubsection{Joint Training with Curriculum Learning Strategy}
% \subsubsection{Simple Curriculum Learning Method}
We propose a semi-automatic curriculum learning (CL) strategy consisting of two core components: an automatic difficulty measurer and a discrete training scheduler, to improve the joint training of retriever, post-ranker, and reader.
The CL strategy aims at reducing the chance of adding golden passage during the joint training process to help the settings of training and inference be more consistent. 
It would effectively push the retriever to find the passage with the true answer contained by itself and encourage the reader to discover the correct answer without the assistance of golden passage.
To the best of our knowledge, this is the first work that designs CL strategy for conversational OpenQA joint learning.
% reduce the dependency on the golden passage in both training and inference phase.

% A general framework of curriculum learning consists of two core components: difficulty measurer and training scheduler. 
Specifically, our semi-automatic CL strategy automates the difficulty measurer by taking the question-wise training loss as criteria. A higher retriever loss is recognized a harder mode for the retriever to discover the positive passages. For the hard mode, the training scheduler of our CL strategy introduces the golden passage in the retrieval results to make the learning easier. 
Our training scheduler is like a Baby Step scheduler~\cite{bengio2009curriculum,spitkovsky2010baby} with a finer granularity as shown in Algorithm~\ref{alg:TS}.


% We use curriculum learning to reduce the chance of golden passage during concurrent learning. \cite{qu2020open} implies that golden paragraphs can be accessible during the training process to avoid poor training outcomes owing to failure to retrieve important articles. We will progressively diminish the likelihood of introducing golden passage to lessen the importance of golden passage.

% We present and compare two curriculum learning methods: decreasing the likelihood with the training step and decreasing the probability with the retriever training loss. First, we reduce the probability of adding golden passage with the training step, starting from 1 to 0. Second, we reduce the probability of adding golden passage with the retriever loss, when the loss of the retriever is high, we think it is necessary to add additional golden passage.

Formally, for each iteration $l$, let $\mathcal{P}^{(l)}_K=\{\mathcal{P}^{(l)}_i|q_i\in Q^{(l)}\}$ be the set of all the collections of top $K$ passages retrieved by the retriever to questions of the batch in the iteration. For the hard mode, the training scheduler brings the golden passage of each question to $\mathcal{P}^{(l)}_K$ to form the set of retrieved passages with the positive one, denoted as $\mathcal{P}^{(l)}_{KG}=\{\mathcal{P}^{(l)}_i\cup\{p^+_i\}|q_i\in Q^{(l)}\}$. With these retrieved passages, the final loss to be optimized is defined as:
\begin{equation}\label{eq:final}
    \mathcal{L}^{(l)}_{final} = v^{(l)}\mathcal{L}^{(l)}(\mathcal{P}_{KG}^{(l)})+(1-v^{(l)})\mathcal{L}^{(l)}(\mathcal{P}^{(l)}_{K}),
\end{equation}
where $v^{(l)}$ is determined by the difficulty measurer with the retriever loss of the previous iteration:
\begin{equation}\label{eq:v}
    v^{(l)}=\left\{
    \begin{array}{ccl}
     1,  & & {\mathcal{L}^{(l-1)}_{retriever}>\lambda_{upper}}, \\
     0,  & & {\mathcal{L}^{(l-1)}_{retriever}<\lambda_{lower}}, \\
     I(p_b), & & {\text{otherwise}},
    \end{array} \right.
\end{equation}
% \begin{equation}
% v^{(l)}=\left\{\begin{matrix}
%   1, & \mathcal{L}^{(l-1)}_{retriever}<\lambda\\
%   \\
% \end{matrix}\right.
% \end{equation}
where $\lambda_{upper}$ and $\lambda_{lower}$ are the pre-defined upper and lower thresholds. $I(p_b)$ denotes an indicator function whose output is sampled from a Bernoulli distribution:
\begin{equation}
    I(p_b)=\left\{
    \begin{array}{ccl}
     1,  & & {\text{with probability}\ p_b}, \\
     0,  & & {\text{with probability}\ 1-p_b}, 
    \end{array} \right.
\end{equation}
where the probability is evaluated by the min-max normalization denoted as $p_b=\frac{ \mathcal{L}^{(l-1)}_{retriever}-\lambda_{lower}}{\lambda_{upper} - \lambda_{lower}}$, which measures the degree of approximating upper threshold for the retriever loss.
It can be intuitively explained that if the retrieval loss of the previous iteration is close to an upper threshold $\lambda_{upper}$, the training status is regarded in the hard mode.
And the value of $v^{(l)}$ is assigned with 1 in a higher probability to select the set of retrieved passages with golden passage for training in the current iteration.

% It can be intuitively explained that if the retrieval loss of the previous iteration is greater than the threshold $\lambda$, then the training status is regarded in the hard mode. And the value of $v^{(l)}$ is assigned with 0 to select the set of retrieved passages with golden passage for the training in the current iteration.
$\mathcal{L}^{(l)}(\cdot)$ in Equation~\ref{eq:final} denotes the objective function integrated by all the three modules with the retrieved passages in the $l$-th iteration as follows:
\begin{equation}
    \mathcal{L}^{(l)} = \mathcal{L}^{(l)}_{retriever} + \mathcal{L}^{(l)}_{postranker}+ \mathcal{L}^{(l)}_{reader}.
\end{equation}
The loss of each module is averaged over the questions in the $l$-th iteration.
% where $v_i$ is decided by the retriever loss of the query example in the previous mini-batch. 
 %% This declares a command \Comment
%% The argument will be surrounded by /* ... */
\SetKwComment{Comment}{/* }{ */}

\begin{algorithm}[t]
\caption{Training Scheduler}\label{alg:TS}
% \KwData{$n \geq 0$}
% \KwResult{$y = x^n$}
\LinesNumbered
\KwIn{Training data $\mathcal{D}_{train}=\{(q_i, a_i, p_i^+)\}_{i=1}^m$, \\
\qquad \quad Iteration number $L$.}
\KwOut{A set of optimal model parameters.}

\For{$l=1,\cdots, L$}{
    Sample a batch of questions $Q^{(l)}$\\
    \For{$q_i\in Q^{(l)}$}{
        $\mathcal{P}_{i}^{(l)} \gets \mathrm{arg\,max}_{p_{i,j}}(\mathrm{sim}(q_i^{en},p_{i,j}),K)$\\
        $\mathcal{P}_{Gi}^{(l)} \gets \mathcal{P}_{i}^{(l)}\cup\{p^+_i\}$\\
        Compute $\mathcal{L}^i_{retriever}$, $\mathcal{L}^i_{postranker}$, $\mathcal{L}^i_{reader}$\\ according to Eq.\ref{eq:retriever}, Eq.\ref{eq:rerank}, Eq.\ref{eq:reader}\\
    }
    % $\mathcal{L}^{(l)}_{retriever} \gets \frac{1}{|Q^{(l)}|}\sum_i\mathcal{L}^i_{retriever}$\\
    % $\mathcal{L}^{(l)}_{retriever} \gets \mathrm{Avg}(\mathcal{L}^i_{retriever})$,
    % $\mathcal{L}^{(l)}_{rerank} \gets \mathrm{Avg}(\mathcal{L}^i_{rerank})$,
    % $\mathcal{L}^{(l)}_{reader} \gets \mathrm{Avg}(\mathcal{L}^i_{reader})$\\
    % Compute $\mathcal{L}^{(l)}_{retriever}$, $\mathcal{L}^{(l)}_{rerank}$, and $\mathcal{L}^{(l)}_{reader}$ by averaging over $Q^{(l)}$\\
    $\mathcal{L}^{(l)} \gets \frac{1}{|Q^{(l)}|}\sum_i(\mathcal{L}^{i}_{retriever} + \mathcal{L}^{i}_{postranker}+ \mathcal{L}^{i}_{reader})$\\
    $\mathcal{P}^{(l)}_K\gets\{\mathcal{P}^{(l)}_i|q_i\in Q^{(l)}\}$,\quad $\mathcal{P}^{(l)}_{KG}\gets\{\mathcal{P}^{(l)}_{Gi}|q_i\in Q^{(l)}\}$\\
    Compute the coefficient $v^{(l)}$ according to Eq.~\ref{eq:v}\\
  \eIf{$ v^{(l)}=1$}{
    $\mathcal{L}^{(l)}_{final} \gets \mathcal{L}^{(l)}(\mathcal{P}_{KG}^{(l)})$\\
  }{
      $\mathcal{L}^{(l)}_{final} \gets \mathcal{L}^{(l)}(\mathcal{P}^{(l)}_{K}),$\\
    }
    Optimize $\mathcal{L}^{(l)}_{final}$
}
\end{algorithm}


%  \eIf{$ \mathcal{L}^{(l-1)}_{retriever}<\lambda$}{
%     $\mathcal{L}^{(l)}_{final} \gets \mathcal{L}^{(l)}(\mathcal{P}_K^{(l)})$\\
%   }{
%       $\mathcal{L}^{(l)}_{final} \gets \mathcal{L}^{(l)}(\mathcal{P}^{(l)}_{KG}),$\\
%     }


% \begin{algorithm}[t]
% \LinesNumbered
% \KwIn{}
% \KwOut{A set of optimal parameters $\Theta$.}
% \For{$t<T$}{
% % Sample users for episode $\mathcal{U}_{epi}^t\leftarrow \mathrm{RandomSample}(\mathcal{U}_{train},N)$\\  
% % \ForEach{user $i\in\mathcal{U}_{epi}^t$}{
% %     % Select support instances \textcolor{red}{$\mathcal{S}_i\leftarrow \mathrm{RandomSample}(\{\},K)$}\\
% %     Select $K$ support instances to form $\mathcal{S}_i$ \\
% %     % Select query instances \textcolor{red}{$\mathcal{Q}_i\leftarrow \mathrm{RandomSample}(,M)$}\\
% %     Select $M$ query instances to form $\mathcal{Q}_i$ \\
% %     Compute representations $\mathbf{u}_{i,j}$ for review instances in $\mathcal{S}_i$ \\
% %     Estimate attentive weights $\alpha_{j,k}^i$ for review instances in $\mathcal{S}_i$\\
% %     Compute user prototype $\mathbf{p}_i \leftarrow \frac{1}{|\mathcal{S}_i|}\sum_{j\in\mathcal{S}_i}\hat{\mathbf{u}}_{i,j}$\\
% %     \ForEach{item $j\in\mathcal{Q}_i$}{
% %     Compute textual representation $\mathbf{q}_{j}^i$\\
% %     Compute memory-based representation $\mathbf{v}_j^i$\\
% %     Compute review-based rating according to Eq. (\ref{eq:rating})\\
% %     }
% % }
% % Minimize training loss according to Eq. (\ref{eq:loss})\\
% }

% \Return{$\Theta$}
   
% \caption{{\bf Training Scheduler}}
% \label{algorithm}
% \end{algorithm}
% \vspace{-0.05in}
\subsection{Inference}\label{subsec:inference}
% We follow the inference process of~\cite{qu2020open}. 
In the inference stage, for a given question $q_c$ and its historical question-answer pairs $\{q_i, a_i\}_{i=c-w}^{c-1}$ within a window size $w$, we first obtain a collection of the top $T$ relevant passages through the two consecutive modules, namely retriever and {\rerankname}. For each passage $p_{c,j}$ in the collection, we get the {\rerankname} score $S_{post}(q_c^{en},p_{c,j})$ according to Equation~\ref{eq:score_post}. The reader score includes two parts, one is the select score $S_{select}(q_{c}^{rd},p_{c,j})$ calculated by Equation~\ref{eq:score_select}, the other one is the span score which consists of the start score and end score as shown in Equations~\ref{eq:score_reader1} and~\ref{eq:score_reader2}. The span score can be obtained by:
\begin{footnotesize}
\begin{equation}
    S_{span}(q_{c}^{rd},p_{c,j},sp)=\max\limits_{sp\in p_{c,j}}\{ S_{s}(q_{c}^{rd},p_{c,j},[t_s])+ S_{e}(q_{c}^{rd},p_{c,j},[t_e])\},
\end{equation}
\end{footnotesize}
where $sp=([t_s],[t_e])$ is the answer span starting with token $[t_s]$ and ending with token $[t_e]$. Following the previous work~\cite{kenton2019bert,qu2020open},  we pick out the top 20 spans and discard the invalid predictions including the cases where the start token comes after the end token, or the predicted span overlaps with the question part of the input sequence. The final prediction score is defined as follows:
% \begin{equation}
%     S(q_{c}^{en},q_{c}^{rd},p_{c,j},sp) = S_{\theta}(q_c^{en},p_{c,j}) + S_{reader}(q_{c}^{rd},p_{c,j}) + S_{span}(q_{c}^{rd},p_{c,j},sp),
% \end{equation}
\begin{equation}
    S(q_{c}^{en},q_{c}^{rd},p_{c,j},sp) = S_{post}(q_c^{en},p_{c,j}) + S_{reader}(q_{c}^{rd},p_{c,j}, sp),
\end{equation}
where $S_{reader}(q_{c}^{rd},p_{c,j},sp) = S_{select}(q_{c}^{rd},p_{c,j}) + S_{span}(q_{c}^{rd},p_{c,j},sp)$. For the given question $q_c$ in a conversation, the answer span $sp$ in the retrieved passage $p_{c,j}$ that has the largest score is the predicted answer. 
\section{Experiments}\label{sec:exp}
\subsection{Dataset}
To evaluate the effectiveness of the proposed {\modelname}, we conduct comprehensive experiments on the public available dataset OR-QuAC~\cite{qu2020open}, which integrates three datasets including the QuAC~\cite{choi2018quac}, CANARD~\cite{elgohary2019can}, and the Wikipedia corpus.
OR-QuAC consists of totally 5,644 conversations containing 40,527 questions and answers with the rewrites of questions obtained from CANARD. The question rewrites support the \textcolor{black}{KL divergence-based regularization} for pretraining a better passage retriever. 
OR-QuAC also provides a collection of more than 11 million passages obtained from the English Wikipedia dump from 10/20/2019\footnote{ttps://dumps.wikimedia.org/enwiki/20191020/} for open-retrieval.
Table \ref{tab:dataset} summarizes the statistics of the aggregated OR-QuAC dataset. 
% We conduct experiments on the public available dataset OR-QuAC to verify our proposed {\modelname}. 
% We aim to answer the following research questions to explore the \textcolor{red}{insight} of {\modelname}:

% \begin{itemize}
%     \item \textbf{RQ1}: How does {\modelname} perform compared with the state-of-the-art methods for conversational open-domain QA task?
%     \item \textbf{RQ2}: How do different components in {\modelname} affect the performance?
%     \item \textbf{RQ3}: What are the influences of different settings of hyper-parameters?
%     % \item \textbf{RQ4}: How is the generalization capacity of {\modelname}?
%     % \item \textcolor{black}{\textbf{RQ4}: How do the representations benefit from {\modelname}?}
% \end{itemize}

% Table generated by Excel2LaTeX from sheet 'Sheet1'
\begin{table}[t]\label{tab:data}
  \centering
  \caption{Data statistics of the OR-QuAC dataset.}
    \begin{tabular}{llccc}
    \toprule
          & Items & Train & Dev   & Test \\
    \midrule
    \multirow{5}[2]{*}{Coversations} & \# Dialogs & 4,383 & 490   & 771 \\
          & \# Questions / Rewrites & 31,526 & 3430  & 5571 \\
          & \# Avg. Questions / Dialog & 7.2   & 7     & 7.2 \\
          & \# Avg. Tokens / Question & 6.7   & 6.6   & 6.7 \\
          & \# Avg. Tokens / Rewrite & 10    & 10    & 9.8 \\
    \midrule
    Wikipedia  & \# Passages & \multicolumn{3}{c}{11 million} \\
    \bottomrule
    \end{tabular}%
  \label{tab:dataset}%
\end{table}%

% Table generated by Excel2LaTeX from sheet 'joint training'
\renewcommand{\arraystretch}{1.2}
\begin{table*}[t]
  \centering
  \caption{Performance comparison of {\modelname} and baseline models. The number in the parentheses is the batch size during the retriever pre-training.}
  \vspace{-0.1in}
     \begin{tabular}{p{10.5em}cccccccccc}
    \toprule
    \multirow{2}[4]{*}{Methods} & \multicolumn{5}{c}{Development}                       & \multicolumn{5}{c}{Test} \\
\cmidrule(lr){2-6} \cmidrule(lr){7-11}     \multicolumn{1}{c}{} & F1    & HEQ-Q & HEQ-D & Rt MRR & Rt Recall & F1    & HEQ-Q & HEQ-D & Rt MRR  & Rt Recall \\
    \midrule
    DrQA~\cite{chen2017reading}  & 4.5   & 0.0   & 0.0   & 0.1151   & 0.2000  & 6.3   & 0.1   & 0.0   & 0.1574   & 0.2253 \\
    BERTserini~\cite{yang2019end} & 19.3  & 14.1  & 0.2   & 0.1767  & 0.2656  & 26.0  & 20.4  & 0.1   & 0.1784   & 0.2507 \\
    DPR (16)~\cite{karpukhin2020dense} & 25.9  & 16.4  & 0.2   & 0.3993  & 0.5440  & 26.4  & 21.3  & 0.5   & 0.1739  & 0.2447  \\
    \midrule
    % \multicolumn{13}{c}{ORConvQA (16$\times$4)~\cite{qu2020open}} \\
    % \midrule
    % -w/o hist & 24.0  & 15.2  & 0.2   & 0.4012  & 0.4472  & 0.5271  & 26.3  & 20.7  & 0.4   & 0.1979  & 0.2702  & 0.2859  \\
    ORConvQA-bert (64)~\cite{qu2020open} & 26.9  & 17.5  & 0.2   & 0.4286   & 0.5714  & 29.4  & 24.1  & 0.6   & 0.2246  & 0.3141  \\
    ORConvQA-roberta (64) & 26.5 & 17.8 & 0.2 &	0.4284  & 0.5624 & 	28.7 &	24.2 &	0.8 & 0.2330  & 0.3226 \\
    \midrule
    % \multicolumn{13}{c}{ours (16)}\\
    % \midrule
    ours-bert (16)  & 28.0 &	19.4 &	0.2 & 0.4639  & 0.6157 & 31.7 & 27.8 &	1.2 & 0.2763  & 0.3668  \\
    ours-roberta (16) & 28.1 & 19.5 & \textbf{0.4} & 0.4639 & 0.6169 & 33.4 & 29.4 & 1.7 & 0.2887  & 0.3819\\
    %  ours-roberta (16) & \textbf{28.1} & \textbf{19.5} & \textbf{0.4} & \textbf{0.4639} & \textbf{0.6169} & \textbf{33.4} & \textbf{29.4} & \textbf{1.7} & \textbf{0.2887}  & \textbf{0.3819} \\
    % \midrule
    % \multicolumn{13}{c}{ours (32)}\\
    % \midrule
    ours-bert (32) & 27.6 &	19.6 &	0.0 & \textbf{0.4675}  & 0.6236 & 32.6 & 29.1 & 0.8 & 0.3013  & 0.4130  \\
    ours-roberta (32)& \textbf{29.4}  & \textbf{20.2}  & \textbf{0.4}   & 0.4656   & \textbf{0.6248}  & \textbf{35.0}  & \textbf{30.8}  & \textbf{1.8}   & \textbf{0.3073}  & \textbf{0.4202}  \\
    \bottomrule
    \end{tabular}%
  \label{tab:overall}%
\end{table*}%


\vspace{-0.1in}
\subsection{Experimental Settings}

\subsubsection{Evaluation Metrics}
Following the evaluation protocols used in~\cite{qu2020open}, we apply the word-level F1 and the human equivalence score (HEQ) that are provided by the QuAC challenge~\cite{choi2018quac} to evaluate our {\modelname}. As the core evaluation metric for the overall performance of answer retrieval, F1 is computed by considering the portion of words in the prediction and groud truth that overlap after removing stopwords.
HEQ is used to judge whether a system's output is as good as that of an average human. It measures the percentage of examples for which system F1 exceeds or matches human F1. Here two variants are considered: the percentage of questions for which this is true (HEQ-Q), and the percentage of dialogs for which this is true for every questions in the dialog (HEQ-D).
Furthermore, we use another two metrics, Mean Reciprocal Rank (MRR) and Recall for the evaluation of the retrieval performance. MRR reflects the abilities of post-ranker to return the passages containing true answers in a high place.
Recall is indicative of post-ranker's capability of providing relevant passages for the next modules. 
For the sake of fairness, we follow ~\cite{qu2020open} and calculate the two metrics for the top $T$ passages that are retrieved for the reader.
% MRR is used to evaluate the outputs of post-ranker and the selection part 

% Furthermore, we use another two metrics, Mean Reciprocal Rank (MRR) and Recall for the evaluation of the \textcolor{red}{retriever and reranker}. Specifically, MRR is used to evaluate both the retriever and reranker. The Reciprocal Rank (RR) calculates the reciprocal of the rank at which the first positive passage is retrieved. MRR is obtained by averaging the RRs across all the queries. 
% Recall is defined as the fraction of the passages that are relevant to the query that are successfully retrieved.
% % the ratio of the number of retrieved relevant passages to the number of 
% Recall is only used to evaluate the retriever only as 
% ranking has no effect on it. 
% MRR reflects the abilities of retriever to return the passages containing true answers in a high placing.
% Recall is indicative of retriever's capability of providing relevant passages for the next modules. 
% For the sake of fairness, we follow ~\cite{qu2020open} and calculate the two metrics for the top 5 passages that are retrieved for the reader and reranker.

\vspace{-0.1in}
\subsubsection{Implementation Settings}
\begin{enumerate}
    \item \textbf{Retriever Pretraining}. As question and passage encoders, we employ two ALBERT Base models. The maximum sequence length for the question encoder is 128, while the maximum length for the passage encoder is 384. The models are trained on NVIDIA TITAN X GPU. The training batch size is set to 16, the number of training epochs is set to 12, the learning rate is set to 5e-5, the window size $w$ is set to 6 and the coefficient of KL divergence $\alpha$ is set to 0.2. Every 5,000 steps, we save checkpoints and assess on the development set provided by \cite{qu2020open}. Then we select several models that performed well on the development set and apply them on test questions and wiki passages. Finally, we select the best model for future training.
    \item \textbf{Post-ranker and Reader}. For the post-ranker, we utilize a linear layer with a size of 128 for simplicity. The number $K$ of passage embeddings it takes as input is set to 100, and the number $T$ of its outputted passages for reader is set to 5. For the reader, we apply the BERT and RoBERTa model. The max sequence length is assigned with 512. The sequence is concatenated by a question and a passage. The maximum passage length is set to 384, with the remainder reserved for the question and other tokens such as  \verb|[CLS]| and  \verb|[SEP]|.
    \item \textbf{Joint Training}. We use the pre-trained passage encoder to compute an embedding for each passage, and build a single MIPS index using FAISS\cite{JDH17} for fast retrieval. Models are jointly trained on NVIDIA TITAN X GPU. 
    The training batch is set to 2, the number of epochs is 3, the learning rate is 5e-5, and the optimizer is Adam.
    % For post-ranker, the margin of hinge loss $\delta$ and triplet margin loss $\mu$ are set to 2 and 0.5, the coefficient of contrastive loss $\beta$ is set to 0.8. For curriculum learning, the pre-defined threshold $\lambda_{upper}$ and $\lambda_{lower}$ are respectively set to 3 and 1. 
    % For all the variant of our {\modelname}, 
    The hyper-parameters including the margin of hinge loss $\delta$, triplet margin loss $\mu$, coefficient of contrastive loss $\beta$, and pre-defined threshold $\lambda_{upper}$ and $\lambda_{lower}$ are tuned based on the development set to select the optimal model for inference.
    We save checkpoints and evaluate on the development set every 5,000 steps, and then select the best model for the test set.
\end{enumerate}

\vspace{-0.1in}
\subsubsection{Baselines}
To demonstrate the effectiveness of our proposed {\modelname}, we compare it with the following state-of-the-art question answering models, including three representative OpenQA models (DrQA, BERTserini, and DPR) and one conversational OpenQA model (ORConvQA):
\begin{itemize}
    \item \textbf{DrQA}~\cite{chen2017reading} is composed of a document retriever which uses bigram hashing and TF-IDF matching to return the relevant passages for a given question, and a multi-layer RNN based document reader for answer spans detection in those retrieved passages.
    \item \textbf{BERTserini}~\cite{yang2019end} uses a BM25 retriever from Anserini\footnote{http://anserini.io/} and a BERT reader to tackle end-to-end question answering. The retriever directly identifies segments of open-domain texts and pass them to the reader. Compared to DPR, ORConvQA and our {\modelname}, it has no selection loss in the reader and benefits less from the joint learning.
    \item \textbf{DPR}~\cite{karpukhin2020dense} increases retrieval by learning dense representations instead of using typical IR methods. It innovatively proposes to introduce hard negatives in the training process of the retriever. For OpenQA, DPR consists of a dual encoder as a retriever and BERT as a reader. 
    \item \textbf{ORConvQA}~\cite{qu2020open} is first proposed to solve the conversational open-domain QA problem with the retriever, rerank, and reader pipeline. This is the key work to compare for our {\modelname}. 
    % We consider to compare with ORConvQA and its variant with the history window size assigned as 0 (ORConvQA w/o hist.).
\end{itemize}

The results of all the baselines except DPR come from the previous work~\cite{qu2020open}. The implementation setting of DPR is the same with ORConvQA, including the encoder network, window size, learning rate, and so on.

% % Table generated by Excel2LaTeX from sheet 'joint training'
% \renewcommand{\arraystretch}{1.2}
% \begin{table*}[t]
%   \centering
%   \caption{Add caption}
%   \normalsize
%     \begin{tabular}{p{8.2em}cccccccccccc}
%     \toprule
%     \multirow{2}[4]{*}{Methods} & \multicolumn{6}{c}{Development}                       & \multicolumn{6}{c}{Test} \\
% \cmidrule(lr){2-7} \cmidrule(lr){8-13}     \multicolumn{1}{c}{} & F1    & HEQ-Q & HEQ-D & Rt MRR & Rr MRR & Rt Recall & F1    & HEQ-Q & HEQ-D & Rt MRR & Rr MRR & Rt Recall \\
%     \midrule
%     DrQA~\cite{chen2017reading}  & 4.5   & 0.0   & 0.0   & 0.1151  & N/A   & 0.2000  & 6.3   & 0.1   & 0.0   & 0.1574  & N/A   & 0.2253 \\
%     BERTserini~\cite{yang2019end} & 19.3  & 14.1  & 0.2   & 0.1767  & N/A   & 0.2656  & 26.0  & 20.4  & 0.1   & 0.1784  & N/A   & 0.2507 \\
%     DPR~\cite{karpukhin2020dense} & 25.9  & 16.4  & 0.2   & 0.3993  & 0.4926  & 0.5440  & 26.4  & 21.3  & 0.5   & 0.1739  & 0.2444  & 0.2447  \\
%     \midrule
%     % \multicolumn{13}{c}{ORConvQA (16$\times$4)~\cite{qu2020open}} \\
%     % \midrule
%     % -w/o hist & 24.0  & 15.2  & 0.2   & 0.4012  & 0.4472  & 0.5271  & 26.3  & 20.7  & 0.4   & 0.1979  & 0.2702  & 0.2859  \\
%     ORConvQA-bert & 26.9  & 17.5  & 0.2   & 0.4286  & 0.5209  & 0.5714  & 29.4  & 24.1  & 0.6   & 0.2246  & 0.3127  & 0.3141  \\
%     ORConvQA-roberta & 26.5 & 17.8 & 0.2 &	0.4284 & 0.5104 & 0.5624 & 	28.7 &	24.2 &	0.8 & 0.2330 &	0.3162 & 0.3226 \\
%     \midrule
%     % \multicolumn{13}{c}{ours (16)}\\
%     % \midrule
%     ours-bert  & 28.0 &	19.4 &	0.2 & 0.4639 &	0.5526 & 0.6157 & 31.7 & 27.8 &	1.2 & 0.2763 &	0.3538 & 0.3668  \\
%     ours-roberta  & \textbf{28.1} & \textbf{19.5} & \textbf{0.4} & \textbf{0.4639} & \textbf{0.5530} & \textbf{0.6169} & \textbf{33.4} & \textbf{29.4} & \textbf{1.7} & \textbf{0.2887} & \textbf{0.3744} & \textbf{0.3819} \\
%     \bottomrule
%     \end{tabular}%
%   \label{tab:overall}%
% \end{table*}%



\vspace{-0.1in}
\subsection{Overall Results}
The overall experimental results are reported in Table~\ref{tab:overall}. The results of the baseline models are public in~\cite{qu2020open} except for DPR. The retrieval metrics denoted by ``Rt MRR'', and ``Rt Recall'' are used to evaluate the retrieval results of retriever for baselines while evaluating that of post-ranker for {\modelname}.
Generally, our {\modelname} outperforms all the baseline models. In detail, several observations can be achieved:
\begin{enumerate}
    \item Both DrQA and BERTserini achieve poor performance. The primary reason is they use the sparse retriever which can not be fine-tuned in the downstream  reader training to discover relevant passages for answer extraction. DrQA performs rather badly in answer reading as it uses RNN-based reader which does not have the strong ability of representation learning as those pre-trained language model. 
    The reader of BERTserini is similar to the other compared methods except DrQA. But it benefits less from the multi-task training process as there is no select component (reranker) in reader. 
    Compared with DrQA and BERTserini, DPR improves the retriever with the dense representation learning. The performance of DPR is limited by the batch size of pre-training.
    
    \item As the first system designed for the task of conversational OpenQA, ORConvQA provides the best performance among the baselines. ORConvQA is similar with DPR, where the difference is that it does not use hard negatives for retriever pre-training while DPR uses. The main reason ORConvQA performs better is the batch size of retriever pre-training is assigned with 64 as it uses 4 GPUs and set batch size to 16 per GPU. Actually, with the same batch size, the retriever of DPR is stronger, detailed results of which can be seen in the further analysis of Section~\ref{subsec:fa1}. 
    \item Our {\modelname} gives the significantly better performance than ORConvQA, even though our retriever is pre-trained with batch size of 16. This indicates that our {\modelname} can perform better with the lower memory cost. When the batch size is increased to 32, the performance can be further improved. The results convincingly demonstrate the effectiveness of the multifaceted improvements of KL-divergence regularization based pre-training, the added post-ranker, and the semi-automatic curriculum learning strategy.
    \item To investigate the influence of the pre-trained language model of reader, we evaluate our {\modelname} and ORConvQA based on two language models, namely BERT and RoBERTa. For ORConvQA, the results obtained by using BERT and RoBERTa are comparable. For our {\modelname}, applying RoBERTa for reader effectively improves the performance compared to using BERT. Overall, {\modelname} achieves the best performance whether with BERT or RoBERTa.
\end{enumerate}


% \renewcommand{\arraystretch}{1.5} %控制行高
% \begin{table*}[htbp]

%   \centering
%   \begin{threeparttable}
%   \caption{Main evaluation results. “Rt” and “Rr” refers to “Retriever” and “Reranker”.}
%   \label{tab:performance_comparison}
%     \begin{tabular}{ccccccccccccc}
%     \toprule
%     \multirow{2}{*}{Methods}&
%     \multicolumn{6}{c}{Dev}&\multicolumn{6}{c}{Test}\cr
%     \cmidrule(lr){2-7} \cmidrule(lr){8-13} 
%     &Rt-R&Rt-M&Rr-M&H-Q&H-D&F1&Rt-R&Rt-M&Rr-M&H-Q&H-D&F1\cr
%     \midrule
%     DrQA&0.2000&0.1151&N/A&0.0&0.0&4.5&0.2253&0.1574&N/A&0.1&0.0&6.3\cr
%     BERTserini&0.2656&0.1767&N/A&14.1&{\bf 0.2}&19.3&0.2507&0.1784&N/A&20.4&0.1&26.0\cr
%     ORConvQA w/o hist&0.5271&0.4012&0.4472&15.2&{\bf 0.2}&24.0&0.2859&0.1979&0.2702&20.7&0.4&26.3\cr
%     ORConvQA&0.5714&0.4286&0.5209&17.5&{\bf 0.2}&26.9&0.3141&0.2246&0.3127&24.1&0.6&29.4\cr
%     ours&{\bf 0.6254}&{\bf 0.4668}&{\bf 0.5666}&{\bf 20.5}&{\bf 0.2}&{\bf 29.5}&{\bf 0.3968}&{\bf 0.2950}&{\bf 0.3866}&{\bf 29.2}&{\bf 0.9}&{\bf 33.8}\cr
%     \bottomrule
%     \end{tabular}
%     \end{threeparttable}
% \end{table*}


\vspace{-0.15in}
\subsection{Ablation Studies}
% Table generated by Excel2LaTeX from sheet 'ablation'
\begin{table}[t]
  \centering
  \caption{Performance of ablation on different components. {\modelname} refers to the full system.}
  \vspace{-0.1in}
    \begin{tabular}{clllll}
    \toprule
    \multicolumn{2}{l}{Settings} & {\modelname}  & \makecell[l]{w/o KL \\retriever} & \makecell[l]{w/o \\post-ranker} & \makecell[l]{w/o \\curriculum} \\
    \midrule
    \multirow{5}[2]{*}{Dev} & F1    & \textbf{28.1} & 27.7  & 27.8  & 27.8  \\
          & HEQ-Q & \textbf{19.5} & 19.4  & 19.4  & 18.3  \\
          & HEQ-D & \textbf{0.4} & 0.0   & 0.0   & 0.0  \\
          & Rt MRR & \textbf{0.4639} & 0.4033  & 0.4604  & 0.4560  \\
        %   & Rr MRR & \textbf{0.5530} & 0.4920  & 0.5525  & 0.5564  \\
          & Rt Recall & \textbf{0.6169} & 0.5324  & 0.6157  & 0.6140  \\
    \midrule
    \multirow{5}[2]{*}{Test} & F1    & \textbf{33.4} & 31.5  & 32.2  & 31.7  \\
          & HEQ-Q & \textbf{29.4} & 29.0  & 28.2  & 27.1  \\
          & HEQ-D & \textbf{1.7} & 1.4   & 1.6   & 0.8  \\
          & Rt MRR & \textbf{0.2887} & 0.2110  & 0.2810  & 0.2812  \\
        %   & Rr MRR & \textbf{0.3744} & 0.2966  & 0.3741  & 0.3708  \\
          & Rt Recall & 0.3819  & 0.2968  & \textbf{0.3830} & 0.3764  \\
    \bottomrule
    \end{tabular}%
    \vspace{-0.15in}
  \label{tab:ablation}%
\end{table}%
To investigate the effectiveness of three improved parts in our {\modelname}, we evaluate several variants of our system. As shown in Table~\ref{tab:ablation}, once we remove one of the three components, both the retrieval and QA performance generally decrease. The detailed observations are summarized as follows:
\begin{enumerate}
    \item When we remove the KL-based regularized pre-trained retriever and use the retriever of ORConvQA as replacement, the performance drops significantly, especially the retrieval performance. It shows the importance of the KL-based regularization in our pre-trained retriever.
    \item Removing the post-ranker also brings a degradation in the overall performance. The influence is slightly smaller than that of KL-based regularization, which is probably caused by the simple linear layer used for post-ranker. This is our limitation and we plan to explore the more flexible and effective neural network for post-ranker in future work. From another perspective, some improvements can be achieved just by adding a linear layer for further passage representation learning.
    It is notable that the retrieval recall is higher than the full system. It is mainly because each question in the test set has more than one golden passages. Retrieving more golden passages does not necessarily lead to the better answer span which has the higher coverage of the ground truth answer.
    \item The variant without curriculum learning in the joint training performs worse than the full {\modelname}. By comparison, the QA performance decreases more noticeably. The reason behind is that the curriculum learning strategy encourages the reader to find correct answer with no golden passage assistance at the joint training time. It makes the reader more suitable for answer extraction in the inference phase.
    % It makes the training process more consistent with the inference.
\end{enumerate}

\begin{table}[t]
  \centering
  \caption{Results of retriever pre-training. B is batch size, Q is the form of question used in training, HD is the number of hard negatives.}
    \vspace{-0.1in}
    \begin{tabular}{cccccc}
    \toprule
          & B & Q & HD & Recall@20 & Recall@100 \\
    \midrule
    BM25  & /     & /     & /     & 0.3711  & 0.5100  \\
    \midrule
    \multicolumn{1}{c}{\multirow{2}[2]{*}{ORConvQA }} & 16    & $q^{rw}$ & 0     & 0.1672  & 0.2916  \\
          & 16$\times$4  & $q^{rw}$ & 0     & 0.3561  & 0.5034  \\
    \midrule
    DPR   & 16    & $q^{rw}$ & 1     & 0.2395  & 0.3759  \\
    \midrule
    \multirow{2}[2]{*}{ours} & 16    & $q^{rw}$/$q^{rw}$ & 1     & 0.3214  & 0.4689  \\
          & 16    & $q^{rw}$/$q^{or}$ & 1     & 0.3690  & 0.5070  \\
          & 32    & $q^{rw}$/$q^{or}$ & 1     & 0.4675  & 0.5882  \\
    \bottomrule
    \end{tabular}%
    \vspace{-0.2in}
  \label{tab:resofpretraining}%
\end{table}%

\vspace{-0.1in}
\subsection{Further Analysis}
\subsubsection{ Retriever Performance}\label{subsec:fa1}
% Results of retriever pretraining. B is batch size, Q is the form of question used in training, HD is the number of hard negatives.  
We evaluate some existing pre-trained retriever with our KL-divergence regularized retriever on the full collection of Wikipedia passages. The evaluation results are reported in Table~\ref{tab:resofpretraining}. 
As the typical sparse retriever, BM25 achieves a good retrieval performance while its QA performance is limited as it can not be fine-tuned in subsequent joint training.
Both ORConvQA and DPR retrievers take the question rewrite as input following the setting in~\cite{qu2020open}. The only difference between them is that DPR uses a TF-IDF hard negative provided by the dataset in addition to the in-batch negatives. DPR outperforms ORConvQA when the batch size is the same, which indicates that the hard negative play a key role during training. Considering the batch size of ORConvQA in~\cite{qu2020open} is 64, we also show the retrieval performance, which is improved dramatically as the number of in-batch negatives increases.
Our KL-divergence based regularized retriever uses two question forms, rewrite and original question, as input, and offers the best performance with the same batch size of 16 among these dense retrieval model. When the batch size is raised to 32, our retriever achieves a better performance and exceeds BM25. 
Moreover, to explore the advantage of the original question which is the concatenation of the historical and current questions, we evaluate our retriever by feeding two of the same rewrite. The results shows that employing two different forms of one question leads to a better performance. It is probably because the input of two same rewrites works just as a regularization while that of original question and rewrite is able to add a supervision for question understanding learning.  
% We report the main evaluation results in Table \ref{tab:resofpretraining}. During testing, questions are expressed in two ways, question rewrite $q^{wr}$ and original question $q^{or}$. Limited by the GPU size, our experiments can only be set to the case of a batch size of 16. We summarize the observations as follows:
% \begin{enumerate}
%     \item When the batch size is 16, we notice that the ORConvQA pre-training retriever performs poorly. And as the batch size is increased, the results improve dramatically, implying that raising the number of in-batch negatives aids in improving training results.
%     \item The DPR model uses the same batch size and TF-IDF hard negative as our model to make a more fair comparison. Compared with the findings of ORConvQA, we feel that negative samples play a key role during training.
%     \item Our model outperforms DPR because it was trained with two question rewrites to ensure that the distribution of retrieval outcomes was consistent. We feel that using KL divergence improves the model's stability. And the fact that our model employs two different types of questions adds to the overall effect. This shows that asking several types of questions might lead to improved retrieval outcomes by developing more accurate semantic representations.
%     \item Retrieval algorithms such as BM25 outperform our model, but the dense retriever can continue to fine-tune the question encoder in subsequent joint training, while retrieval algorithms such as BM25 are not trainable.
% \end{enumerate}




\subsubsection{Case Study for Post-Ranker}
% \begin{figure*}[t]
%      \centering
%      \begin{subfigure}[b]{0.46\textwidth}
%          \centering
%          \includegraphics[width=\textwidth]{figure/case1v2.pdf}
%          \caption{}
%          \label{fig:case1}
%      \end{subfigure}
%      \hfill
%      \begin{subfigure}[b]{0.45\textwidth}
%          \centering
%          \includegraphics[width=\textwidth]{figure/case2v2.pdf}
%          \caption{}
%          \label{fig:case2}
%      \end{subfigure}
%         \caption{Case Study}
%         \label{fig:case}
% \end{figure*}
To explore the effect of post-ranker, we count the number of golden passages before and after post-ranker in the inference phase. Concretely, the number of golden passages in the top $T$ ($T=5$) increases from 2,228 to 2,269 while the golden passages ranked between $T+1$ and $K$ ($K=100$) decreases from 1,292 to 1,251, which verifies post-ranker' ability in enforcing the relevant passages to appear in the higher place. Figure~\ref{fig:case1} shows an example of the passage retrieval results before and after post-ranker for a question. Before applying post-ranker, it can be observed that golden passage is not contained in the top 5 passages but ranked in the 14 place of the ranking list generated by the retriever. Taking the passage representations in the ranking list as input, post-ranker provides a new passage ranking where the golden passage is reranked in the top 5. Moreover, the top 5 passages of the newly generated ranking list contains more relevant content with the question and its historical questions, showing the advantage of the proposed post-ranker module in our {\modelname}.  
\begin{figure}[t]
    \centering
    \includegraphics[width=\linewidth]{figure/case2v4.png}
    \vspace{-0.3in}
    \caption{Case study of the passage retrieval results before and after post-ranker for an example question.}
    \label{fig:case1}
    \vspace{-0.15in}
\end{figure}

% \begin{figure}[t]
%     \centering
%     \includegraphics[width=\linewidth]{figure/case1v4.png}
%     % \vspace{-0.3in}
%     \caption{}
%     \label{fig:case2}
%     % \vspace{-0.25in}
% \end{figure}

\subsubsection{Impact of Curriculum Learning Strategy}
% \begin{figure}[t]
%     \centering
%     \includegraphics[width=0.5\linewidth]{figure/f1.png}
%     % \vspace{-0.3in}
%     \caption{}
%     \label{fig:f1}
%     % \vspace{-0.25in}
% \end{figure}

% \begin{figure}[t]
%     \centering
%     \includegraphics[width=0.5\linewidth]{figure/retriever_recall.png}
%     % \vspace{-0.3in}
%     \caption{}
%     \label{fig:recall}
%     % \vspace{-0.25in}
% \end{figure}

To investigate the impact of curriculum learning, we save checkpoints and evaluate on the test set every 500 steps, the results of all the checkpoints are shown in Figure~\ref{fig:curriculum}.
It can be observed that the QA metric F1 with curriculum learning increases more steadily and reached better final performance faster. 
For retrieval metric, the improvement achieved by curriculum learning is the higher final recall value, which is not quite noticeable.
It indicates that the curriculum learning strategy makes more pronounced contribution for the answer reading other than the passage retrieval, which in reasonable as the inconsistent problem addressed by curriculum learning mainly reflects in the reading part.













\vspace{-0.1in}
\section{Related Work}\label{sec:related}

This section briefly summarizes some existing works on open domain question answering and conversational question answering, which are most relevant to this work.

\vspace{-0.1in}
\subsection{Open Domain Question Answering}
%介绍open domain QA
Open-domain question answering (OpenQA) \cite{voorhees1999trec} is a task that uses a huge library of documents to answer factual queries. The two-stage design, which included a passage retriever to choose a subset of passages and a machine reader to exact answers, became popular after DrQA~\cite{chen2017reading}.

The passage retriever is an important component of OpenQA system since it searches relevant paragraphs for the next stage. 
Traditional sparse retrieval models, such as TF-IDF or BM25~\cite{robertson2009probabilistic}, have been widely adopted as retriever in OpenQA systems~\cite{chen2017reading, yang2019end, lin2018denoising}. While sparse retrieval cannot handle the case of high semantic correlation with little lexical overlap and it is untrainable, dense passage retrievers have lately gained popularity~\cite{lee2019latent, guu2020realm, karpukhin2020dense, Qu2021RocketQAAO}. 
In general, the dense retrieval model is a dual-encoder architecture that encodes both the question and the passage individually. Both encoders are trained during the retriever pre-training process. When training with the reader for the QA task, only the question encoder is normally fine-tuned.
In order to increase the retrieval impact of dense retrievers, some studies incorporate hard negatives. 
% mining to train dense retrieval models. 
BM25 top passages which do not contain answers are utilized as hard negatives~\cite{karpukhin2020dense, gao2020complementing}. And dynamic hard negatives are employed in~\cite{xiong2021approximate, zhan2021optimizing,guu2020realm}, which are the top-ranked irrelevant documents given by dense retriever during training. 
% While classic passage retrieval models such as BM25\cite{robertson2009probabilistic} depend on sparse representations, dense passage retrievers\cite{lee2019latent,guu2020realm,karpukhin2020dense} have lately gained popularity due to their ability to return paragraphs with little lexical overlap but great relevance to questions.
%引入硬负样本训练dense retriever
% Many researchers employ negative sampling techniques to train dense retrieval models in order to improve their performance. For example, BM25 top passages which do not contain answers were utilized as hard negatives by DPR\cite{karpukhin2020dense}. RocketQA\cite{Qu2021RocketQAAO} uses a well-trained cross-encoder, and passages with high confidence are chosen as negative samples. And in the training phase, dynamic hard negatives are employed in \cite{xiong2021approximate, zhan2021optimizing}, which are the top-ranked irrelevant documents given the dense retrieval parameters. 

% reader
A contemporary OpenQA system also includes a reader as a key component. Its goal is to deduce the answer to a query from a collection of documents. Existing Readers may be divided into two types: (1) Extractive readers~\cite{chen2017reading,yang2019end,karpukhin2020dense}, which anticipate an answer span from the retrieved texts, (2) Generative readers~\cite{lewis2020retrieval,izacard2021leveraging}, which produce natural language replies using sequence-to-sequence (Seq2Seq) models.

\begin{figure}[t]
     \centering
     \begin{subfigure}[b]{0.23\textwidth}
         \centering
         \includegraphics[width=\textwidth]{figure/f1.pdf}
         \caption{}
         \label{fig:curriculum_f1}
     \end{subfigure}
     \hfill
     \begin{subfigure}[b]{0.23\textwidth}
         \centering
         \includegraphics[width=\textwidth]{figure/retriever_recall.pdf}
         \caption{}
         \label{fig:curriculum_recall}
     \end{subfigure}
     \vspace{-0.15in}
        \caption{Comparison of performance improvement with / without curriculum learning.}
    \vspace{-0.2in}
        \label{fig:curriculum}
        % \vspace{-0.1in}
\end{figure}

% \vspace{-0.2in}
\subsection{Conversational Question Answering} 
% 介绍Conversational QA
Conversational Question Answering (CQA) is required to understand the given context and history dialogue to answer the question.
As a main type of CQA, Conversational Machine Reading Comprehension (CMRC)~\cite{qu2019bert, qu2019attentive, qiu2021reinforced} does the QA task with text-based corpora. 
% It could be split into two categories \cite{zaib2021conversational}: (1) sequential Knowledge-Based Question Answering (KBQA) system \cite{iyyer2017search, guo2018dialog, kacupaj2021conversational}, (2) Conversational Machine Reading Comprehension (CMRC) \cite{qu2019bert, qu2019attentive, qiu2021reinforced}. CMRC is text-based, whereas KBQA is knowledge graph-based.
For CMRC, the number of conversational history turns is critical, as context utterances that are relevant to the inquiry are valuable, while irrelevant ones may introduce additional noise. For example, \cite{qu2019bert, qu2020open} makes use of conversation history by including $K$ rounds of history turns. \cite{qu2019attentive} weights previous conversation rounds based on their contribution to the answer to the current question.

% 引入open retrieval Conversational QA
The approaches outlined above rely extensively either on the provided material or a given paragraph to extract or generate answers. 
However, this is impractical in real world since golden passage is not always available. 
Open retrieval methods which try to obtain evidence from a big collection, have lately been popular in the CMRC. 
% The approaches outlined above significantly rely on the provided material to extract or generate an answer. 
ORConvQA~\cite{qu2020open} is the first work proposing three primary modules for open-retrieval CQA: (1) a passage retriever, (2) a passage reranker, and (3) a passage reader. Given a query and its conversational history, the passage retriever retrieves the top $K$ relevant texts from a large-scale corpus. The passage reranker and reader then rerank and read the top texts to discover the correct answer. 
The research for the conversational OpenQA needs to be further explored. This work tries to improve ORConvQA from multiple aspects including regularizing retriever pre-training, incorporating post-ranking, and curriculum learning.





\section{Conclusion and Future Work}
In this study, we point out that the traditional placeholder translation method embeds the specified term into the generated translation without considering the context of the placeholder token, which potentially leads to grammatically incorrect translations.
To address this shortcoming, we proposed a flexible placeholder translation model that handles inflection when the specified term is given in the form of a lemma.
In the experiment of the Japanese-to-English translation task, we showed that the proposed model can inflect user-specified terms more accurately than the code-switching method.

Future work includes testing the proposed method on morphologically-rich languages or extending the model to handle more than one placeholder in a sentence.
Also, the proposed model still has room for improvement to learn inflection.
It is possible that we can improve the model by exploiting monolingual corpora in the target language to provide additional training signals for learning the correct inflection in context.

%%
%% The acknowledgments section is defined using the "acks" environment
%% (and NOT an unnumbered section). This ensures the proper
%% identification of the section in the article metadata, and the
%% consistent spelling of the heading.
% \begin{acks}
% To Robert, for the bagels and explaining CMYK and color spaces.
% \end{acks}

%%
%% The next two lines define the bibliography style to be used, and
%% the bibliography file.
\newpage
\bibliographystyle{ACM-Reference-Format}
\bibliography{sample-base}

%%
%% If your work has an appendix, this is the place to put it.


\end{document}
\endinput
%%
%% End of file `sample-sigconf.tex'.
