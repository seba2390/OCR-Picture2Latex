%!TEX root = paper.tex
\def\half{{\textstyle\frac{1}{2}}}
\newcommand{\smallfrac}[2]{{\textstyle \frac{#1}{#2}}}

%%%%%%%%%%%%%%%%%% Old AZ Defs below this line %%%%%%%%%%%%%%%%%%%%%%

\def\vec#1{\mathchoice%
	{\mbox{\boldmath $\displaystyle\bf#1$}}
	{\mbox{\boldmath $\textstyle\bf#1$}}
	{\mbox{\boldmath $\scriptstyle\bf#1$}}
	{\mbox{\boldmath $\scriptscriptstyle\bf#1$}}}
\def\v#1{\protect\vec #1}

\newcommand{\LabelFunc}[2]{\parbox[t]{#2\textwidth}%
	{\hspace*{\fill}{\vspace*{-0.3cm}#1}\hspace*{\fill}}}

\newcommand{\OneLabel}[1]{\LabelFunc{#1}{1}}

\newcommand{\TwoLabels}[2]{\LabelFunc{#1}{0.48}\hfill\LabelFunc{#2}{0.48}}

\newcommand{\ThreeLabels}[3]{\LabelFunc{#1}{0.31}\hfill
                             \LabelFunc{#2}{0.31}\hfill\LabelFunc{#3}{0.31}}

\newcommand{\FourLabels}[4]{\LabelFunc{#1}{0.23}\hfill\LabelFunc{#2}{0.23}\hfill
			      \LabelFunc{#3}{0.23}\hfill\LabelFunc{#4}{0.23}}

\newcommand{\FiveLabels}[5]{\LabelFunc{#1}{0.17}\hfill\LabelFunc{#2}{0.17}\hfill\LabelFunc{#3}{0.17}\hfill\LabelFunc{#4}{0.17}\hfill\LabelFunc{#5}{0.17}}
%!TEX root = paper.tex
\newcommand{\reals}{\mathbb R}
\newcommand{\D}{\mathcal D}
\newcommand{\A}{\mathcal A}
\newcommand{\B}{\mathcal B}
\newcommand{\eg}{{\it e.g.\xspace}}
\newcommand{\etc}{{\it etc\xspace}}
\newcommand{\ie}{{\it i.e.\xspace}}
\newcommand{\Bbn}{\alpha \D\cdot \|\vec{U}\|_2 + \beta}
\newcommand{\Cbn}{\alpha \Delta_\theta\cdot \|\vec{U} \|_2}
\newcommand{\Gbn}{\alpha\D \cdot \| \delta_\vec{U}\|_2}
\newcommand{\bbn}{1+ \alpha\cdot|\D \|\vec{U}\|_2 -1|}

% \newcommand{\C}{\mathcal C}
\newcommand{\Y}{\mathcal Y}
\newcommand{\Z}{\mathcal Z}
% \newcommand{\G}{\mathcal G}
\newcommand{\R}{\mathcal R}
\newcommand{\I}{\mathcal I}
\newcommand{\K}{\mathcal K}
\newcommand{\J}{\mathcal J}
\newcommand{\ex}{\mathbb E}
\newcommand{\prob}{\mathbb P}
\newcommand{\TODOK}[1]{
\ifmmode
\text{\textcolor{blue}{ }}
\else
\textcolor{blue}{ }
\fi
}
\newcommand{\TODOA}[1]{
\ifmmode
\text{\textcolor{red}{ }}
\else
\textcolor{red}{ }
\fi
}

\renewcommand{\vec}[1]{{\mathbf{#1}}}
\newcommand{\br}[1]{\left({#1}\right)}

\DeclareMathOperator{\Tr}{Tr}
% Theorem environments
\makeatletter

\usepackage{amsthm}

\makeatletter
\newtheorem*{rep@theorem}{\rep@title}
\newcommand{\newreptheorem}[2]{%
\newenvironment{rep#1}[1]{%
 \def\rep@title{#2 \ref{##1}}%
 \begin{rep@theorem}}%
 {\end{rep@theorem}}}
\makeatother


\newtheorem{theorem}{Theorem}
\newreptheorem{theorem}{Theorem}
\newtheorem{lemma}{Lemma}
\newreptheorem{lemma}{Lemma}

% \newtheorem{theorem}{Theorem}
\numberwithin{theorem}{section}
% \newtheorem{lemma}[theorem]{Lemma}
\newtheorem{proposition}[theorem]{Proposition}
\newtheorem{remark}[theorem]{Remark}
\newtheorem{corollary}[theorem]{Corollary}
\newtheorem{definition}[theorem]{Definition}
\newtheorem{conjecture}[theorem]{Conjecture}
\newtheorem{axiom}[theorem]{Axiom}
\newtheorem{assumption}{Assumption}

% Asymptotic notation
\newcommand{\bigO}[1]{{\mathcal O}\br{{#1}}}
\newcommand{\softO}[1]{\widetilde{\cal O}\br{{#1}}}
\newcommand{\Om}[1]{\Omega\br{{#1}}}
\newcommand{\softOm}[1]{\tilde\Omega\br{{#1}}}

% Defining Operator norms
\makeatletter
\newcommand{\opnorm}{\@ifstar\@opnorms\@opnorm}
% \newcommand{\@opnorms}[1]{%
%   \left|\mkern-1.5mu\left|\mkern-1.5mu\left|
%   #1
%   \right|\mkern-1.5mu\right|\mkern-1.5mu\right|
% }
% \newcommand{\@opnorm}[2][]{%
%   \mathopen{#1|\mkern-1.5mu#1|\mkern-1.5mu#1|}
%   #2
%   \mathclose{#1|\mkern-1.5mu#1|\mkern-1.5mu#1|}
% }
% \makeatother

\newcommand{\redSpace}{\vspace{-6mm}}
% \renewcommand{\baselinestretch}{0.995}


% \makeatletter
% \newcommand{\newreptheorem}[2]{\newtheorem*{rep@#1}{\rep@title}
% \newenvironment{rep#1}[1]{\def\rep@title{#2 \ref*{##1}}\begin{rep@#1}}{\end{rep@#1}}
% }
% \makeatother

% \newreptheorem{lemma}{Lemma}
% \newreptheorem{theorem}{Theorem}
% \newreptheorem{claim}{Claim}
\newcommand{\argmin}[1]{\underset{#1}{\operatorname{arg}\,\operatorname{min}}\;}

\usepackage{array}

\newcolumntype{L}{>{\arraybackslash}m{8cm}}
\newcommand{\yij}[2]{y_{i#1}^{#2}}
\newcommand{\Lf}[2]{\mathbf{Lf}(#1, #2)}

\newcommand{\pt}[2]{\mathbf{T}_{#1}^{#2}}

\newcommand{\vx}{\mathbf{x}_{i}}
\newcommand{\pr}[2]{
\mathbf{P_r}(\yij{\pt{l}{#1}}{#1}|\pt{l}{#2},\vx)
}
\newcommand{\pri}[3]{
\mathbf{P_r}(\yij{\pt{l}{#1}}{#1}=#3|\pt{l}{#2},\vx)
}
\newcommand{\lpr}[2]{
\log(\mathbf{P_r}(\yij{\pt{l}{#1}}{#1}|\pt{l}{#2},\vx))
}
\newcommand{\lpri}[3]{
\log(\mathbf{P_r}(\yij{\pt{l}{#1}}{#1}=#3|\pt{l}{#2},\vx))
}
\newcommand{\pli}[1]{
\mathbf{P_r}(y_{il}=#1|\pt{l}{D},\vx)
}
\newcommand{\ri}[1]{
\v r_{i}^{{#1}}(\pt{l}{#1})
}

\newcommand{\rp}[1]{
r^{#1}
}
\newcommand{\pl}{
\mathbf{P_r}(y_{il}|\pt{l}{D},\vx)
}

\newcommand{\povernd}{
\prod_{i=1}^{N}\prod_{l=1}^{L}
}
\newcommand{\poverd}[3]{
\left(\prod_{#1=1}^{#2} #3\right)
}
\newcommand{\sovernd}{
\sum_{i=1}^{N}\sum_{l=1}^{L}
}

\definecolor{arsenic}{rgb}{0.23, 0.27, 0.29}

\definecolor{silver}{rgb}{0.75, 0.75, 0.75}
\newcommand{\wpred}[1]{
\textcolor{silver}{#1}
}

\newcommand{\suppl}{\href{http://manikvarma.org/pubs/mittal21.pdf}{\color{blue}{supplementary material}}\xspace}

\newcommand{\code}{\href{https://github.com/Extreme-classification/DECAF}{\color{blue}{https://github.com/Extreme-classification/DECAF}}}\xspace
