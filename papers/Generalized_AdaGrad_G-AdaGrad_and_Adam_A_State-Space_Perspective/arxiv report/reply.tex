\textcolor{red}{************************}

We denote $m = \inf_{1\leq s \leq t} \mu_i(s)^2 v_i(s)^{-0.5}$ and $M = \sup_{1\leq s \leq t} \mu_i(s)^2 v_i(s)^{-0.5}$. 
Now, $$\int_1^t \mu_i(s)^2 v_i(s)^{-0.5} ds \geq \int_1^t m ds = m(t-1).$$ We consider the case $m>0$ first. If $m>0$, for $t \geq \frac{M}{m}+1$, we have $\int_1^t \mu_i(s)^2 v_i(s)^{-0.5} ds \geq M = \sup_{1\leq s \leq t} \mu_i(s)^2 v_i(s)^{-0.5}$. The inequality follows with $T_2 = \frac{M}{m}+1$. Next, we consider the case where $m=0$. From the argument at the end of the paragraph following equation (25), $\mu_i(t) = 0$ only at isolated points $t$. We can take these set of isolated points off from the range of the above integral, and its value will remain the same (integral evaluated at an isolated point is zero). But, over these new range, $\mu_i(t) > 0$, and therefore $m > 0$. Then we apply the argument from the previous case.