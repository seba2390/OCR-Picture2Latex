Context: Why now?  Why the need is so pressing or important?
Need: Why you? Why something needed to be done at all?
Task: Why me? What was undertaken to address the need
Object: [Document] What the present document does or covers
Findings: me What the work done yielded or revieled
Conclusion: you What the findings mean for the audience
Perspectives: future What the future holds, beyond this work

 A supernova's evolution into a supernova remnant is highly dependant on the structures created by the RTI.

Context: Why the need is so pressing or important?
The Rayleigh-Taylor Instability's (RTI's) effects are present in a vast range of High Energy Density (HED) length scales, spanning from supernova explosions (m$^{13}$) to inertial confinement fusion (m$^{-6}$).  In inertial confinement fusion, the RTI is known to induce mixing or turbulent transition which in turn cools the hot spot and hinders ignition.  The fine-scale features that develop on the RTI's structures may help determine if the system is mixing or transitioning to turbulence. 


Need: Why you? Why something needed to be done at all?
Despite its ubiquitous nature, the fine-scale features of the RTI are difficult to image in HED Physics.  Earlier diagnostics lacked the spatial and temporal resolution necessary to diagnose the dynamics that occur along the RTI structure.  Simulations are one method to observe this behavior, yet codes with identical initial conditions but different numerical schemes generate different fine-scale structures.  


Task: Why me? What was undertaken to address the need
A recently developed diagnostic, the Crystal Backlighter Imager (CBI), can now produce an x-ray radiograph capable of resolving the fine-scale features expected in these RT unstable systems.

Object: [Document] What the present document does or covers
This paper describes an experimental design that adapts a well- characterized National Ignition Facility (NIF) platform to accommodate the CBI diagnostic. Simulations and synthetic radiographs highlight the CBI's resolution capabilities in comparison to previous diagnostics. 


Conclusion: What the findings mean for the audience
The improved resolution of the system can provide new observations to study the RTI's involvement in mixing and the transition to turbulence in the HED regime.  


The Rayleigh-Taylor Instability's (RTI's) effects are present in a vast range of High Energy Density (HED) length scales, spanning from supernova explosions (m$^{13}$) to inertial confinement fusion (m$^{-6}$).  In inertial confinement fusion, the RTI is known to induce mixing or turbulent transition which in turn cools the hot spot and hinders ignition.  The fine-scale features that develop on the RTI's structures may help determine if the system is mixing or transitioning to turbulence. Despite its ubiquitous nature, the fine-scale features of the RTI are difficult to image in HED Physics.  Earlier diagnostics lacked the spatial and temporal resolution necessary to diagnose the dynamics that occur along the RTI structure.  Simulations are one method to observe this behavior, yet codes with identical initial conditions but different numerical schemes generate different fine-scale structures.  A recently developed diagnostic, the Crystal Backlighter Imager (CBI), can now produce an x-ray radiograph capable of resolving the fine-scale features expected in these RT unstable systems. This paper describes an experimental design that adapts a well- characterized National Ignition Facility (NIF) platform to accommodate the CBI diagnostic. Simulations and synthetic radiographs highlight the CBI's resolution capabilities in comparison to previous diagnostics. The improved resolution of the system can provide new observations to study the RTI's involvement in mixing and the transition to turbulence in the HED regime.  


