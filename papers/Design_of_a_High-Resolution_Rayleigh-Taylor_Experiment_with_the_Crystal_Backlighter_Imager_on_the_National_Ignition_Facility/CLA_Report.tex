\usepackage{blindtext}
\usepackage{graphicx}
\usepackage{color}
\usepackage{verbatim}
\usepackage{natbib}   
\usepackage{tabularx}
\usepackage[svgnames]{xcolor}
 \usepackage{hyperref}
 \usepackage{amsmath}
 \usepackage{amsfonts}
 \usepackage{array,booktabs,arydshln,xcolor}
 \usepackage[normalem]{ulem}
 \usepackage{epstopdf} \epstopdfsetup{update} 
 \usepackage[superscript,biblabel]{cite}


\section*{NIF Experiments}

\section{Motivation}


Late time nonlinear Rayleigh-Taylor (RT) unstable behavior plays a key role during multiple phases of inertial confinement fusion capsule implosions.  Imperfections present on the capsule interface causes mixing that cool the hot spot and adversely affect the outcome of the experiment.  Therefore nderstanding Rayleigh–Taylor (RT) instability is critical for conducting successful inertial confinement fusion (ICF) experiments \cite{Remington:2006,Robey:2003, RPD, Nagel:2017, Smalyuk:2017} However, due to diagnostic and facility limitations, historically it has been difficult to diagnose the dynamics that occur along the spike tip and ascertain whether mixing is solely present, or if it transitions to turbulence. 


Turbulence has been studied quite prolifically throughout the century and has yielded a wide range of definitions across a wide range of subfields of physics. \cite{RPD} \cite{dimotakis2000mixing,  dimonte2006k} However, this experiment highlights the practical aspect of understanding and diagnosing turbulence as it applies to specifically to High-Energy-Density-Physics (HEDP) and to ICF. Therefore, we claim our system is transitioning, or has transitioned, to turbulence if we achieve a sufficiently high Reynolds Number \cite{zhou2019turbulent, zhou2003progress, zhou2007unification} and are able to observe spatial scales corresponding to the Liepmann-Taylor scale \cite{RPD, dimotakis2000mixing, dimonte2006k,Robey:2003, Zhou:2017, zhou2019turbulent}.  For a system with a Reynolds number of order 10$^4$, the Liepmann-Taylor scale is of order 2 $\mu$m - a scale several orders of magnitude below previous diagnostic capabilities.





As the instability further evolves and the amplitude grows to 10 percent of the wavelength, it approaches the early nonlinear stage. It is during this stage when nonlinearities begin to affect the growth rate and make significant contributions to the interface features.\cite{Zhou:2017, robey2003onset}  Vorticity is introduced as the density gradient becomes misaligned from the pressure gradient along the length of the spike (Figure \Ref{fig:RTdiagram}) and creates vortex features that cause the spike to broaden as it grows. It is presently unknown whether turbulence is present within the vortex features \cite{jacobs1996experimental}.  The purpose of this design is to improve experimental resolution to begin to diagnose and capture turbulent transition within the vortex structure. 

  Fundamental turbulence theory defines turbulence as (i) any process that increases structure at an interface, (ii) the early nonlinear behavior of an instability at interface, (iii) the development of substantially convoluted structure at an interface, having spectral content or spatial extent far beyond those of the initial state, (iv) the appearance of an inertial range in the fluctuation spectrum, with power-law decay of the spectral energy density, and (v) the development of strong mixing as indicated by supra-linear growth of the thickness of the mixing layer in time. \cite{RPD}




However, this experiment highlights the practical aspect of understanding and diagnosing turbulence as it applies to specifically to High-Energy-Density-Physics (HEDP) and to ICF. Therefore, we claim our system is transitioning, or has transitioned, to turbulence if we achieve a sufficiently high Reynolds Number \cite{zhou2019turbulent, zhou2003progress, zhou2007unification} and are able to observe spatial scales corresponding to the Liepmann-Taylor scale \cite{RPD, dimotakis2000mixing, dimonte2006k,Robey:2003, Zhou:2017, zhou2019turbulent}.  For a system with a Reynolds number of order 10$^4$, the Liepmann-Taylor scale is of order 2 $\mu$m -- a scale several orders of magnitude below previous diagnostic capabilities.





\section{Experimental or Computational Setup}

% 1 page with 1 - 2 figures

 Numerous experiments have been conducted at the Laboratory for Laser Energetics and National Ignition Facility aimed at measuring nonlinear RT growth using x-ray radiography \cite{Nagel:2017, Robey:2003, Zhou:2017}. Yet these these platforms are optimized at measuring the gross mix width of the instability, and not the fine-scale features that develop along the spike tip.  Facility and diagnostic advancements now allow us to reach regimes and image features that may play a significant role in HED unstable systems. The experimental design described in this paper adapts the drive and diagnostic configuration of a related NIF campaign.\cite{Nagel:2017, doss2015shock} studying the RT instability in a planar geometry to the newly commissioned Crystal Backlighter Imager (CBI) to image the fine-scale features that develop on an RT spike tip. 
 
 
 This experiment utilizes a similar drive and physics package from previous successful campaigns conducted on the NIF \cite{Nagel:2017, Robey:2003}. The Vortex experiment is conducted in a rectangular shock tube and is driven by a halfraum on one end.  When the halfraum receives a laser pulse, it produces a thermal bath of soft x-rays that generates a strong shock into a physics package comprised of high density plastic and low density foam. The strong shock turns this initially solid target into a plasma and the material boundary between the plastic and foam becomes a fluid interface. The high density fluid, previously the plastic, is accelerated into the low density fluid, previously the foam, and the system becomes RT unstable.  A precise single-mode sine pattern is machined at the fluid interface to seed the RT instability with a known amplitude and wavelength such that it will achieve a sufficient degree of nonlinearity during the experiment.   After a specified time delay, additional lasers will land on a backlighter to produce the hard x-rays necessary to radiograph the evolving interface.  \\


  \begin{figure}[h]
 \begin{center}
 \includegraphics[width = 3. in]{images/Targetschematic.pdf}
 \caption{Left: Pre- shot radiograph of the target Right: Exploded view of the target package in Ref. \cite{Nagel:2017} }
 \label{fig:TargetLayout}
 \end{center}
 \end{figure}
 

\noindent \textbf{The Physics Package} The high density component, or ablator,  involves two density-matched (1.43 g/cc) plastics: polyamide-imide (PAI), C$_{22}$H$_{14}$N$_2$O$_3$, and an iodine-doped polystyrene (CHI), C$_{50}$H$_{47}$I$_3$.  Models and previous experiments suggest that the plastics behave hydrodynamically similar \cite{Nagel:2017}, but the CHI is relatively opaque to the imaging x-rays compared to the PAI.  Therefore, a 300 $\mu$m thick slice of CHI is placed at the center of the PAI to act as a tracer strip. This improves the contrast between the plastic and foam, and reduces multi-dimensional effects. The ablator compound is machined with a sinusoidal perturbation on the plastic-foam interface wavelength $\lambda$ = 200 $\mu$m and an initial amplitude of  $a_0$ =  7.5 $\mu$m and 15 $\mu$m  (15 $\mu$m and 30 $\mu$m peak-to-valley respectively) for two separate targets. A sample pre-shot radiograph and exploded view of the target package are shown in Figure \ref{fig:TargetLayout}.  In the left image in Fig. \ref{fig:TargetLayout}, a shock, indicated by the arrows, will propogate downward thereby accelerating the high density CHI ablator, PAI layer, and CHI tracer into the low density CRF foam.  The right image in Fig. \ref{fig:TargetLayout} is an exploded view of the target and highlights the CHI tracer strip location that is partially isolated from wall effects.



The 2D simulations in this report use HYDRA, a three-dimensional radiation hydrodynamics simulation code developed at Lawrence Livermore National Lab (LLNL).   HYDRA is an arbitrary Lagrangian–Eulerian (ALE) code wherein the mesh moves with the flow, while grid relaxation and advection algorithms handle the mesh tangling that can result from complex motion. However, the simulations in this report are pureley Eulerian. It is important to note that HYDRA does not have mix-model capabilities, and no sub-grid modeling- therefore the resolution is achieved with the more grid points in the initial mesh.  The simulations represent the plastic sections including the doped plastic tracer layer, undoped plastic layer, and foam section, as separate material regions,\Ref{image},  with tabulated equations of-state and opacities. The initial design simulations are primarily aimed at understanding the large-scale hydrodynamics of the experiment. Future simulations will extend to include models for small-scale hydrodynamic effects such as turbulent mixing. 






\section{Results}

% 1 page with 1 - 2 figures


Figure \ref{fig:N190213data} is the experimental data obtained.  The left image is the smaller initial amplitude and the right image is the larger initial amplitude, while both images were obtained at 46 ns. The images show the shock and ustable interface moving "upwards" toward the center of the tube.  However the shockfront is difficult to see due to the poor contrast between the shocked and un-shocked foam at this backlighter energy.  The images show a planar interface region in the middle portion of the field-of-view, suitable for planar analysis.  Curvature effects can be seen towards the edge of the tube and is assumed to be due to shock-wall interaction.

Figure \ref{fig:simoverlap} is the HYDRA simulation at t = 46 ns overlapped with the large initial amplitude experimental data.  This qualitatively shows that the simulation is in large agreement with the experimental data and is used for additional analysis. The reduced mix width shown in the simulation compared to experimental data is attributed to the lack of motion blur in the simulation image.  Simulated radiographs that incorporate motion blur to improve simulation comparison to experimental data is underway.


The spatial resolution is calculated using ray tracing as 9 $\mu$m at the object plane (Gall, private conversation). However, the image resolution is still calculated at 26 um.  Unlike earlier experiments, the majority of the contribution to this resolution is due to temporal limitations of the backlighter gate time in conjunction with the bulk flow motion.  With the backlighter gate time at 600 ps and the bulk flow estimated as 40 $\mu$m/ns from simulations, the temporal blurring is estimated at ~ 24 um.  There is additional motion within the roll-up, and t



We can estimate the time-dependent Reynolds number in the roll-up based on the roll up width from the simulations in \ref{fig:RTdiagram}. At 46 ns, the transverse flow velocity is about 5 $\mu$m /ns for the large initial amplitude. The transverse flow velocity is used because it removes velocity contributions from the bulk flow. We estimate the plasma viscosity to be about 0.001 $\mu^2$/ns, based on numerical simulations. Using a roll-up width of 30 $\mu$m, the Reynolds number obtained is Re ~ 3 $\times 10^4$, which exceeds the minimum threshold discussed in the motivation.  Although simulations indicate the system within the roll-up has transitioned to turbulence, the Leipmann-Taylor length scale is still unsresolvable. Therefore, we are unable to  indicate that we have achieved  transition to turbulence using solely this experimental data. 




Future experiments in collaboration with LLNL take a comprehensive approach of not only addressing the spatial resolution of the optic and detectors used, but also the need to minimize motion blur in the platform.  I am providing design support for the series of upcoming experiments while



\section{LDHydro}

An upcoming shotday will conduct the same experimental design and timings using the Mn-He backlighter for the SCI diagnostic.  The MnHe backlighter is at a lower backlighter energy and should improve the contrast between the shocked and unshocked foam, and between the CHI rod and the foam.  This will ideally allow us to determine the shock position with confidence.  The shock position is crucial to determine the mix width growth rate for each peak on the rod.  Simulated radiographs that plot the transmission of a given backlighter energy as a function of thickness in a cylindrically symmetric target will help determine the experimental peak positions. These simulated radiographs are presently underway. 



In future work, the mixing will be analyzed for its role
765 in reducing aerial mass flux of the shocked filament.
766 As the experiment is expected to be the upper limit as
767 the adiabatic case, this aerial mass flux may be directly
768 translated to a astrophysically relevant aerial mass flux
769 and correspondingly linked to predictions for the star
770 formation rate. Furthermore, the dependency between
771 mixing and instability related quantities such as wave772
length and Atwood number will be investigated. We
773 note that future experiments would benefit from the use
774 of larger laser platforms, such as NIF, capable of provid775
ing stronger shocks needed for higher radiation temper776
atures, along with increasing typical target dimensions
777 for late time evolution.








