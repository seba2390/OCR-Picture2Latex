%%
%% This is file `sample-sigplan.tex',
%% generated with the docstrip utility.
%%
%% The original source files were:
%%
%% samples.dtx  (with options: `sigplan')
%% 
%% IMPORTANT NOTICE:
%% 
%% For the copyright see the source file.
%% 
%% Any modified versions of this file must be renamed
%% with new filenames distinct from sample-sigplan.tex.
%%  
%% For distribution of the original source see the terms
%% for copying and modification in the file samples.dtx.
%% 
%% This generated file may be distributed as long as the
%% original source files, as listed above, are part of the
%% same distribution. (The sources need not necessarily be
%% in the same archive or directory.)
%%
%% Commands for TeXCount
%TC:macro \cite [option:text,text]
%TC:macro \citep [option:text,text]
%TC:macro \citet [option:text,text]
%TC:envir table 0 1
%TC:envir table* 0 1
%TC:envir tabular [ignore] word
%TC:envir displaymath 0 word
%TC:envir math 0 word
%TC:envir comment 0 0
%%
%%
%% The first command in your LaTeX source must be the \documentclass command.
\documentclass[9pt,sigconf,screen]{acmart}
\usepackage{algorithm}
\usepackage{algorithmic}
\usepackage{amsmath}
\usepackage{xspace}
\usepackage{siunitx}
\usepackage{subfig}
\usepackage{tikz}
%% NOTE that a single column version is required for 
%% submission and peer review. This can be done by changing
%% the \doucmentclass[...]{acmart} in this template to 
%% \documentclass[manuscript,screen,review]{acmart}
%% 
%% To ensure 100% compatibility, please check the white list of
%% approved LaTeX packages to be used with the Master Article Template at
%% https://www.acm.org/publications/taps/whitelist-of-latex-packages 
%% before creating your document. The white list page provides 
%% information on how to submit additional LaTeX packages for 
%% review and adoption.
%% Fonts used in the template cannot be substituted; margin 
%% adjustments are not allowed.
%%
%% \BibTeX command to typeset BibTeX logo in the docs
\AtBeginDocument{%
  \providecommand\BibTeX{{%
    \normalfont B\kern-0.5em{\scshape i\kern-0.25em b}\kern-0.8em\TeX}}}

%% Rights management information.  This information is sent to you
%% when you complete the rights form.  These commands have SAMPLE
%% values in them; it is your responsibility as an author to replace
%% the commands and values with those provided to you when you
%% complete the rights form.
\setcopyright{acmcopyright}
\copyrightyear{2018}
\acmYear{2018}
\acmDOI{XXXXXXX.XXXXXXX}

%% These commands are for a PROCEEDINGS abstract or paper.
\acmConference[Arxiv]{ }{Nov. 2023}{Preprint}
%
%  Uncomment \acmBooktitle if th title of the proceedings is different
%  from ``Proceedings of ...''!
%
%\acmBooktitle{Woodstock '18: ACM Symposium on Neural Gaze Detection,
%  June 03--05, 2018, Woodstock, NY} 
\acmPrice{15.00}
\acmISBN{978-1-4503-XXXX-X/18/06}


%%
%% Submission ID.
%% Use this when submitting an article to a sponsored event. You'll
%% receive a unique submission ID from the organizers
%% of the event, and this ID should be used as the parameter to this command.
%%\acmSubmissionID{123-A56-BU3}

%%
%% For managing citations, it is recommended to use bibliography
%% files in BibTeX format.
%%
%% You can then either use BibTeX with the ACM-Reference-Format style,
%% or BibLaTeX with the acmnumeric or acmauthoryear sytles, that include
%% support for advanced citation of software artefact from the
%% biblatex-software package, also separately available on CTAN.
%%
%% Look at the sample-*-biblatex.tex files for templates showcasing
%% the biblatex styles.
%%

%%
%% The majority of ACM publications use numbered citations and
%% references.  The command \citestyle{authoryear} switches to the
%% "author year" style.
%%
%% If you are preparing content for an event
%% sponsored by ACM SIGGRAPH, you must use the "author year" style of
%% citations and references.
%% Uncommenting
%% the next command will enable that style.
%%\citestyle{acmauthoryear}

%%
%% end of the preamble, start of the body of the document source.
\newcommand{\nojan}[1]{\textcolor{red}{{\sf (NS:} {\sl{#1})}}}
\newcommand{\yaman}[1]{\textcolor{blue}{{\sf (YJ:} {\sl{#1})}}}
\newcommand{\nasimeh}[1]{\textcolor{purple}{{\sf (NH:} {\sl{#1})}}}
\newcommand{\Prv}{$\mathcal{P}$\xspace}
\newcommand{\Vrf}{$\mathcal{V}$\xspace}
\newcommand{\Cir}{$\mathcal{C}$\xspace}
\newcommand\sys{$\mathsf{SPAM}$\xspace}
\newcommand*\circled[1]{\tikz[baseline=(char.base)]{
            \node[shape=circle,draw,inner sep=2pt] (char) {#1};}}

\begin{document}

%%
%% The "title" command has an optional parameter,
%% allowing the author to define a "short title" to be used in page headers.
\title{\textbf{SPAM}: Secure \& Private Aircraft Management}
% maybe ZkyOT could be Zero knowledge interconnected obfuscated transmission

%%
%% The "author" command and its associated commands are used to define
%% the authors and their affiliations.
%% Of note is the shared affiliation of the first two authors, and the
%% "authornote" and "authornotemark" commands
%% used to denote shared contribution to the research.
% \author{Ben Trovato}
% \email{trovato@corporation.com}
% \orcid{1234-5678-9012}
% \author{G.K.M. Tobin}
% \authornotemark[1]
% \affiliation{%
%   \institution{Institute for Clarity in Documentation}
%   \streetaddress{P.O. Box 1212}
%   \city{Dublin}
%   \state{Ohio}
%   \country{USA}
%   \postcode{43017-6221}
% }

% \author{Lars Th{\o}rv{\"a}ld}
% \affiliation{%
%   \institution{The Th{\o}rv{\"a}ld Group}
%   \streetaddress{1 Th{\o}rv{\"a}ld Circle}
%   \city{Hekla}
%   \country{Iceland}}
% \email{larst@affiliation.org}

% \author{Valerie B\'eranger}
% \affiliation{%
%   \institution{Inria Paris-Rocquencourt}
%   \city{Rocquencourt}
%   \country{France}
% }

\author{\textbf{Yaman Jandali*}, \textbf{Nojan Sheybani*}, \textbf{Farinaz Koushanfar} \\ 
University of California, San Diego \\ 
\tt\small{\{yeljanda, nsheybani, farinaz\}@ucsd.edu}}
\thanks{*Equal contribution}

%%
%% By default, the full list of authors will be used in the page
%% headers. Often, this list is too long, and will overlap
%% other information printed in the page headers. This command allows
%% the author to define a more concise list
%% of authors' names for this purpose.
\renewcommand{\shortauthors}{Jandali, Sheybani, and Koushanfar}

%%
%% The abstract is a short summary of the work to be presented in the
%% article.
\begin{abstract}
With the rising use of aircrafts for operations ranging from disaster-relief to warfare, there is a growing risk of adversarial attacks. Malicious entities often only require the location of the aircraft for these attacks. Current satellite-aircraft communication and tracking protocols put aircrafts at risk if the satellite is compromised, due to computation being done in plaintext. In this work, we present \sys \footnote{The name \sys (Secure \& Private Aircraft Management) is crafted with a dual meaning in mind: signifying the management system's assurance of rendering externally observable aircraft data unintelligible, resembling typical ``spam'' content.}, a private, secure, and accurate system that allows satellites to efficiently manage and maintain tracking angles for aircraft fleets without learning aircrafts' locations. \sys is built upon multi-party computation and zero-knowledge proofs to guarantee privacy and high efficiency. While catered towards aircrafts, \sys's zero-knowledge fleet management can be easily extended to the IoT, with very little overhead.
\end{abstract}

%%
%% The code below is generated by the tool at http://dl.acm.org/ccs.cfm.
%% Please copy and paste the code instead of the example below.
%%
\begin{CCSXML}
<ccs2012>
   <concept>
       <concept_id>10002978.10002979</concept_id>
       <concept_desc>Security and privacy~Cryptography</concept_desc>
       <concept_significance>500</concept_significance>
       </concept>
   <concept>
       <concept_id>10010520.10010553</concept_id>
       <concept_desc>Computer systems organization~Embedded and cyber-physical systems</concept_desc>
       <concept_significance>300</concept_significance>
       </concept>
   <concept>
       <concept_id>10010520.10010570</concept_id>
       <concept_desc>Computer systems organization~Real-time systems</concept_desc>
       <concept_significance>300</concept_significance>
       </concept>
 </ccs2012>
\end{CCSXML}

\ccsdesc[500]{Security and privacy~Cryptography}
\ccsdesc[300]{Computer systems organization~Embedded and cyber-physical systems}
\ccsdesc[300]{Computer systems organization~Real-time systems}

%%
%% Keywords. The author(s) should pick words that accurately describe
%% the work being presented. Separate the keywords with commas.
\keywords{Privacy-Preserving Computation, Zero-Knowledge Proofs, Aircraft Privacy, Multi-Party Computation, Two-Party computation}

%% A "teaser" image appears between the author and affiliation
%% information and the body of the document, and typically spans the
%% page.


% \begin{teaserfigure}
%   \includegraphics[width=\textwidth]{sampleteaser}
%   \caption{Seattle Mariners at Spring Training, 2010.}
%   \Description{Enjoying the baseball game from the third-base
%   seats. Ichiro Suzuki preparing to bat.}
%   \label{fig:teaser}
% \end{teaserfigure}

\received{20 February 2007}
\received[revised]{12 March 2009}
\received[accepted]{5 June 2009}

%%
%% This command processes the author and affiliation and title
%% information and builds the first part of the formatted document.
\maketitle

\begin{figure}[h]
  \centering
  \includegraphics[width=0.8\columnwidth]{images/sat_and_planes.pdf}
  \caption{\sys is an end-to-end aircraft management system that guarantees location privacy to both the aircraft and satellite via scalable Boolean synthesis for secure 2PC and ZKP.}
\end{figure}

\section{Introduction}
% \nojan{Citations go before periods, not after. pls fix}
% \nojan{NOTE: Export all images as PDF before including them in paper (scales better)}
% \nasimeh{intro is way too short. You need to have the following structure: -What is the problem? Motivation? Why is it important?
% - What has been done?
% - Limitations
% -what is it that we are doing?
% -contributions
% -paper organization}
% \yaman{Is this better?}
Over the past decade, the use of piloted aircrafts, as well as Unmanned Aerial Vehicles (UAVs), has significantly grown. Fleets of remotely controlled vehicles are increasingly used in disaster relief, search and rescue, warfare, and more \cite{disRelief, searchAndRescue}. The Federal Aviation Administration (FAA) of the United States of America (USA) handles approximately 16.5 million flights each year. In addition to the aircrafts, the USA’s Department of Defense operates over 11,000 Unmanned Aircraft Systems (UAS) \cite{dod}. This includes UAVs, commonly known as "drones". With the increasing number of aircrafts, not just in the USA, but globally, there is also an increased risk of one of these aircrafts  becoming the target of an adversarial attack or of “going rogue” (i.e. falling into the hands of enemy forces \cite{rogueNations}), and there have already been cases where this is suspected to have happened \cite{hartmann}. 

Many types of adversarial attacks, such as those using automated weaponry based on GPS tracking, GPS spoofing, jamming attacks, and more, rely on knowledge of an aircraft's location \cite{GPSNav}. 
% \nasimeh{"However" doesn't make sense here. Use in addition?}. 
In addition, satellite-based control systems that require an antenna to be aimed at a remotely controlled aircraft, as well as satellites that provide other types of aircrafts with communication capabilities, can have their transmissions be intercepted \cite{interceptComm}, thereby revealing the location of the tracked vehicle. For this reason, it is imperative that the computations used to determine the direction that a satellite's antenna faces be done without ultimately revealing the location of the aircraft. This can be accomplished by taking advantage of scalable methods for privacy preserving computation based on synthesis of Boolean logic \cite{tg}.

% \nasimeh{you're basically repeating the first paragraph here. Try to merge this into the first paragraph. Then add a paragraph about what has been done in terms of aircraft security and privacy.}\yaman{-->Ready for recheck}   

Works such as \cite{PPCA, hideAndSeek} offer solutions aiming to protect the privacy of a drone's location. However, this is done either by simply communicating a temporary location through a secure channel or by obfuscating the location through differential privacy, which would make it infeasible to accurately compute the trajectory from one party to the other. Akkaya et al. \cite{Akkaya} focus on private communication of data from an aircraft to a server for machine learning inference on private data, but do not address concerns for privacy of the aircraft's location. Other methods such as \cite{blockchain3} utilize blockchain for secure authentication of drones, but this approach is ineffective for privacy-preserving location management. 

% \nasimeh{the first sentence is not clear. Plus, you shouldn't refer to the works mentioned in the previous paragraph} 
Challenges of privacy preservation are further amplified by the communication channels available between aircrafts and satellite. In this work, we assume that aircrafts and satellites communicate using the standardized SATCOM network. SATCOM enables fast aircraft-satellite communication, but lacks encryption below the application layer by default. This puts any communication through SATCOM at risk of eavesdropping attacks, which has been drastically simplified due to the introduction of software defined radios (SDRs) \cite{9811060}. The solutions require different communication infrastructures or the adoption of secure channels, which increases the overhead and limits the scalability. Scalability is essential, as in many satellite-aircraft communication settings, satellites are actively communicating with many aircrafts at a time for various tasks. One of the most prominent tasks in this domain is the enforcement of certain routes or bounded areas for an aircraft to traverse (e.g. surveillance aircrafts). It is essential to ensure aircraft location privacy in these settings, as eavesdroppers and malicious adversaries that compromise the satellite could mount dangerous attacks if this location is leaked.

In this work, we propose \sys, the first automated end-to-end solution to private satellite-aircraft localization. \sys utilizes automated Boolean logic synthesis to build optimal circuits for privacy preserving functions to preserve the secrecy of both aircraft and satellite location. Using a combination of Boolean logic-based two-party computation and zero knowledge proofs, \sys enables two key tasks in satellite-aircraft localization with strong privacy guarantees: 
% \nasimeh{I don't like the circled numbers. I suggest you just use (1) or 1)}
( 1 ) \textit{Movement Tracking}: \sys enables a satellite to continuously adjust its antenna to maximize communication efficiency without revealing the satellite or aircraft's location and ( 2 ) \textit{Location Management}: \sys enables a satellite to monitor aircrafts within its network and ensure that they are staying within certain bounds, without learning the exact locations of the aircrafts.

To the best of our knowledge, we propose the first method that securely verifies the correct angle of a satellite's antenna for aircraft communication. Further, \sys is the first work to allow satellites to enforce location bounds for aircrafts in their networks, without revealing the exact location of the aircrafts. \sys not only elegantly solves these problems with very little overhead, but also defends against eavesdropping attacks on SATCOM networks due to the use of privacy-preserving computation.

% \nojan{MENTION THAT WE ASSUME SATCOM USE}

% \nojan{edit this to include location management zk stuff} To the best of our knowledge, there are no works for securely calculating the trajectory between aircrafts and/or spacecrafts, as would be necessary for maximizing communication efficiency without compromising the privacy of either party's location. Additionally, there are no systems in place for the secure assignment of aircrafts to satellite systems based on transmission domains, as would be necessary for location management in order to further maximize communication efficiency. In order to close this gap in current research, we propose \sys. 

% Although there have been works on applying privacy preservation to UAV related technology, such as for collision prevention and machine learning inference on encrypted video data, to the best of our knowledge there has been no proposed work for securely computing the trajectory of a satellite's antenna for efficient communication with an aircraft without revealing the locations of either the aircraft or the satellite. 

% In this work, we evaluate \sys's runtime and communication to show its real-world applicability.

% \vspace{-1cm}
In summary, our contributions are as follows.
\begin{itemize}
    \item Introduction of \sys, the first automated end-to-end framework for privately tracking and managing aircrafts, without revealing any location information.
    \item Enabling unintelligible tracking of location (without recompute) by introducing new 2PC-based methods for further private tracking of aircraft trajectories.
    % \item Building \sys upon novel synthesis of Boolean logic functions for two-party privacy preserving location computation.
    \item Utilizing scalable Boolean logic synthesis to design novel circuits compatible with interactive zero-knowledge proof protocols which enforces spatial bounds for aircrafts.
    \item Extensive evaluations of \sys highlight its performance at scale, while providing privacy to all parties involved.
\end{itemize}
% \yaman{TODO: This paragraph will be completed once all other sections are finalized}The remainder of this paper is structured as follows. In section \ref{prelim} we provide relevant background. In \ref{relatedWork} we review related works. In \ref{methodology} we articulate the assumptions we make and the methodology of our implementation. 

\section{Preliminaries}\label{prelim}

With increasing data-driven decision making and information sharing, there has been a growing need for privacy preserving computation to protect our sensitive data. A variety of solutions have emerged over the years, with one efficient and robust choice being multi-party computation (MPC), which allows for joint computation on private data by multiple parties while achieving provable privacy and accuracy.

\subsection{Multi-Party Computation}\label{MPCSec}

In MPC schemes, $n$ parties, each holding their own private inputs denoted as ${d_1, d_2, ..., d_n}$, perform joint computation of a public function, $F(d_1, d_2, ..., d_n)$, such that only the output of this function is revealed while the each party's input remains private. There are generally two settings of MPC -  namely \textit{semi-honest} and \textit{malicious} settings. In semi-honest settings, parties may gather information passively without straying from the designated protocol. In malicious settings, parties are able to actively pursue information both passively, as with the semi-honest setting, but also by deviating from the designated protocol to gain more information \cite{introToMPC}. 
% \nojan{need a bit of a transition here} 
In this work, we utlize two-party computation, a subset of multi-party computation.
% MPC is often implemented through the use of secret sharing schemes such as with shamir shares or arithmetic shares. An adversary would require control over a certain number of parties in order to learn the value being shared. This threshold is often denoted as $t$ such that in shamir sharing we have $t < \frac{n}{2}$ in semi-honest settings and $t < \frac{n}{3}$ in malicious settings \cite{thresholdMPC}. Using arithmetic sharing, this threshold is
\subsubsection{Two-Party Computation}
Two-party computation (2PC) was first introduced by Andrew Yao in 1986 and was later extended by Goldreich, Micali, and Wigderson to MPC \cite{2pc, mpc}. 2PC often relies on the use of garbled circuits, a topic first introduced by Yao but further formalized by Beaver et al. \cite{BMR}. Garbled circuit based evaluation involves two parties, namely a \textit{garbler} and an \textit{evaluator}. The garbler's role is in generating a circuit that describes an underlying function to be computed with both parties' inputs. The evaluator receives this circuit from the garbler and by using Oblivious Transfer \cite{OT}, the evaluator is able to garble its own input with the help of the garbler, without the garbler learning the evaluator's input. The function is then evaluated, which results in both parties obtaining the function's output without revealing any information about each other's inputs in addition to what can be inferred from the output itself. Beaver-Micali-Rogaway (BMR) is one method of implementing garbled circuits by utilizing generalized secure protocols for 2PC to compute the garbled circuit \cite{BMR}. 

% Satellite to Fleet Communication System ; Tether; SkyOT 

\subsection{Zero-Knowledge Proofs}
Zero-Knowledge Proofs (ZKPs) are a two-party cryptographic primitive between two parties: prover \Prv and verifier \Vrf. ZKPs enables \Prv to prove that the the evaluation of a computation \Cir on a private value $w$, called the witness, is valid without revealing anything about $w$. In standard ZKP schemes, \Prv convinces \Vrf that $w$ is a valid input such that $y=\mathcal{C}(x,w)$, where $x$ and $y$ are public inputs and outputs, respectively \cite{zk}.
% In \sys's setting, ZKPs can be viewed akin to maliciously-secure two-party computation.
Interactive ZKP schemes require many rounds of interaction between \Prv and \Vrf to successfully build a ZKP. Non-interactive ZKPs (NIZKs) allow for proof generation to be done in one step, but often require a rigorous trusted setup process per computation \Cir. While NIZKs have smaller proof size and faster verification, they benefit from \textit{publicly-verifiable} proofs, meaning that any \Vrf can verify the proof. 

Conversely, interactive ZKPs have larger proofs and are \textit{designated verifier}, meaning that only the verifier interacting with \Prv can verify the proof, however do not have a trusted setup process. One of the main advantages of interactive proofs is the reduced prover complexity when compared to NIZKs. This becomes especially suitable when working with resource-constrained devices, such as those in IoT settings, which we discuss in detail in section \ref{eval}. While the \textit{designated verifier} constraint can viewed as a drawback, we will explain that this is actually suitable for our system, and far outweighs the drawback of having a rigorous trusted setup process. In this work, we employ the Wolverine protocol \cite{passwordAuth}, a state-of-the-art interactive ZKP system that is notably efficient in terms of execution time, memory demands, and data transfer for \Prv. Within this protocol, the witness $w$ is verified using information-theoretic message authentication codes (IT-MACs). Computations in Wolverine are structured as either arithmetic or Boolean circuits, and are collaboratively evaluated by \Prv and \Vrf. Upon completion, \Prv reveals the output to demonstrate the validity of the proof, which in our case is proving that the evaluated output matches a pre-existing public output. 
 
\section{Related Work}\label{relatedWork}

Works such as PPCA \cite{PPCA} recognize the threats of location leakage to adversaries. With increasing investments from stakeholders such as Google and Amazon, there are more drones being used for autonomous delivery each year. The authors claim to be the first ones to develop an improved collision detection system that maintains the privacy of the drone locations rather than having each UAV openly broadcasting its location - the current norm. They do so by having UAVs share temporary locations through secure wireless connections to nearby drones that have a risk of colliding with each other. 

In Hide and Seek \cite{hideAndSeek}, the authors show that Critical Infrastructure operators are able to preserve drone privacy while detecting drones violating no-fly zone designations. They provide DiPrID, a framework which uses differential privacy to improve drone location privacy. They also propose ICARUS, which provides a solution for detecting unauthorized drones within no-fly zones. 

The authors of \cite{IODdesign} outline the lack of mechanisms for the privacy preservation of drone locations but do not provide implementation to fill this gap. \cite{jammingAttacks, attacks2} further demonstrate the need for location privacy by showcasing various attacks on drones, such as jamming and GPS spoofing, that rely on knowing a drone's location.

Works such as \cite{Akkaya, darksky} discuss the growing issues of drones capturing sensitive data and provide frameworks for enabling privacy-preserving transmission of data captured by drones, such as video recordings. However, they do not discuss the privacy of the drone itself. 

Research interest in efficient privacy preserving computation has drastically increased in recent years. Developments in the field have allowed for advancements in a variety of private computing applications such as machine learning model inference on sensitive data, credit scoring, identity verification, and more \cite{passwordAuth, chameleon, creditScoring}. A number of methods are used, such as multiparty computation and homomorphic encryption, which allow for secure computation on private data. Alongside this, zero-knowledge proofs are an emerging technology that allow users to prove attributes about their data, without revealing anything about their data \cite{zk}.

% \nasimeh{Is there only 4 related works? If so,  maybe add some related work on the applications of privacy preserving techniques in similar domains (not necessarily aircraft position but similar applications.}
% \vspace{-2mm}
\section{Methodology}\label{methodology}
% \yaman{fix flow of this}
% \nojan{first thing should be the threat models. Then we cna go into how the algorithm actually works in prose. Then we can discuss what we used to implement everything}
% \yaman{Add mp-spdz details and design decisions}
% \nojan{Fleet management stuff w/ zk}

% \begin{figure*}[h]
%   \centering
%   \includegraphics[width=\linewidth]{images/SPAM - Overview Diagram.pdf}
%   \caption{Caption goes here}
%   \Description{description goes here}
% \end{figure*}

\begin{figure*}[t]
\centering
\subfloat[Trajectory Computations: The aircraft and satellite jointly compute the unit vector describing the desired trajectory for the satellite's antenna using Two-Party Computation.]{\includegraphics[width=.4\textwidth]{images/setup.pdf}}
\hspace{1em}
\subfloat[Range Checks: Aircrafts periodically send a ZKP to prove to the satellite that they are still within range.]{\includegraphics[width=.4\textwidth]{images/zkp.pdf}}
% \hfill
% \hfill
% \subfloat[Trajectory Update: In order the optimize calculation of the updated unit vector, we take advantage the Law of Cosines which allows us to compute the new magnitude using one private multiplication instead of three.]{\includegraphics[width=.3\textwidth]{images/[Update] SPAM - Overview Diagram.pdf}}
\caption{High Level Overview of \sys.}
\label{fig:system}
\end{figure*}

Many aircrafts, including UAS or drones, are often remotely controlled through satellite systems, which aim an antenna towards them  to achieve efficient transmission. Naively, this can be accomplished through the aircraft communicating its location to the satellite.  However,  this approach exposes the aircraft's location to the satellite, introducing vulnerabilities  if the satellite is compromised by adversaries. 

\sys provides secure computation of the trajectory from a spacecraft to an aircraft through 2PC and a fleet management system built on zero-knowledge proofs, all without compromising the location of either party. We represent the trajectory using a \textit{trajectory unit vector}, which is simply a vector of magnitude one that defines the direction from the satellite to the aircraft in three dimensional space.
% An alternative would be to use angle representations, such the azimuth and elevation angles where azimuth angle describes the angle of direction with regards to the vertical axis, and the elevation angle describes the angle above the horizontal aircraft.

% \nasimeh{here you should briefly explain what is it that you do. Define the trajectory unit vector and clearly state your goal.}


\subsection{Threat Model(s)}

\sys aims to protect the privacy of the aircraft's location from the satellite, as leaking this could make the aircraft susceptible to attacks from adversaries. Alongside this, we would like to leak as little information as possible about the satellite to the aircraft, which is why two-party computation and zero knowledge proofs are employed in \sys. In our setting, we assume that both parties are \textit{semi-honest}, meaning that they will follow the given protocol, but try to learn as much information as possible with the data in hand. While not the main contribution of this work, we also consider the threat of a malicious satellite that may deviate from the protocol.

% We aim to conserve the privacy of the locations of the satellite and aircraft while computing the trajectory between them in space. Namely, the distance between the two parties should remain hidden. We assume both a semi-honest and malicious setting such that parties may follow or stray from the designated protocol. The trajectory unit vector defining the trajectory between the two parties is only revealed to the spacecraft adjusting its antenna while the aircraft learns nothing. 

% \subsection{Privacy Assumptions}
% We have two cases of privacy, one where where one party's location is private and another case where both parties' locations are private.

% \yaman{does it work to put this stuff here?}
% \subsubsection{One-Way Privacy}\label{sec:oneWay}
% We might assume the aircraft’s location to remain private while the satellite’s location may be known by the aircraft. In this setting, we are able to simply have the aircraft perform computations of the trajectory unit vector and send it to the satellite along with periodical updates. This removes the need for secure computation entirely and results in the efficient calculation of the trajectory in plain-text. 

% \subsubsection{Two-Way Privacy}\label{sec:twoWay}
% Under stricter constraints, we would like to maintain the privacy of both the aircraft and the satellite. With these more secure assumptions, we see the need for privacy preserving computation methods such as two-party computation or homomorphic encryption. Under 2PC, our two parties would be the satellite and the aircraft and they would jointly compute the unit vector conveying the trajectory from the satellite to the aircraft without revealing any additional information such as the distance between the satellite and aircraft. Because computation under privacy is significantly more expensive than plain-text computation, it is important to consider possible cost-aware algorithmic optimizations. 

\subsection{\sys Overview}
\sys consists of two core functionalities: movement tracking and location management.
The main task of \sys's movement tracking is privately updating the angle of a satellite's antenna to ensure that it is pointing at the aircraft, without revealing the exact location of the aircraft. The main task of \sys's location management is to generate and verify proofs in which aircrafts attest to the satellite that they are bound-compliant. We note that we use scalable Boolean logic synthesis \cite{emp, tg} to optimize our 2PC and ZKP circuits, ensuring state-of-the-art performance.

\subsubsection{Movement Tracking}
An aircraft or drone is equipped with a communication system that requires a satellite connection. In order to transmit information most efficiently between the aircraft and the satellite, an antenna on the satellite must point itself at the aircraft's changing location. We denote the aircraft by $p$, the satellite by $s$, and the locations of the satellite and the aircraft by $(x_s, y_s, z_s)$  and $(x_a, y_a, z_a)$, respectively. If the satellite's location is public,  we are able to simply have the aircraft perform the computation of the satellite's trajectory unit vector and send the computed vector to the satellite. This fulfills all of privacy requirements for the aircraft without the need for expensive private computation. This simple algorithm is described in Algorithm \ref{algo:oneWayAlg}.

% \subsection{Setup}


% In a setting where there is a aircraft, $p$, and a satellite, $s$, we denote the locations of the satellite and aircraft at time $t$ by $s_t$ and $p_t$ respectively.


% \subsection{Plain-text Computation}\label{PTComp}


\begin{algorithm}
\caption{Plain-text Trajectory Unit Vector Calculation}
\label{algo:oneWayAlg}

\begin{algorithmic}
\STATE \textbf{Input}:
\STATE Satellite location (sent to aircraft): $(x_s, y_s, z_s)$
\STATE aircraft location (known by aircraft): $(x_a, y_a, z_a)$
\STATE \textbf{Computed}:
\STATE Vector $v = (x_v=x_a - x_s, y_v=y_a - y_s, z_v=z_a - z_s)$
\STATE Magnitude $m = \sqrt{x_v^2 + y_v^2 + z_v^2}$
\STATE Unit vector $a = \left(\frac{x_v}{m}, \frac{y_v}{m}, \frac{z_v}{m}\right)$
\STATE \textbf{Send the Unit vector to the satellite.}
\end{algorithmic}
\end{algorithm}
% \vspace{-19mm}

% \noindent We also note that because of the low computational cost of plain-text operations, subsequent update to the location can be computed in a similar fashion. This is particularly true if the satellite's location does not change, which would mean the satellite does not need to communicate an updated location unless necessary. 

% \subsection{Private Computation}

However, under \sys's stricter privacy constraints, we assume that both of the aircraft's and satellite's locations must remain private.  We rely on 2PC to securely compute the trajectory. In this more secure setting, we perform the same computations as we have in Algorithm \ref{algo:oneWayAlg} but using secure protocols. This new formulation is described in Algorithm \ref{algo:twoWayAlg}. We build our computation using semi-honest and malicious secure two-party methods based on oblivious transfer (OT). For the semi-honest threat model, we utilize the Semi \cite{mpspdz} protocol, which utilizes OT and secret sharing over a large prime field to support semi-honest computation. For the malicious threat model, we use the MASCOT \cite{mascot} protocol, which builds off of Semi. MASCOT primarily adds MAC keys and OT correlation and consistency checks to Semi to support malicious parties.
% In both cases, the garbled circuit is generated through BMR \cite{BMR}.
The utilization of these methods enable \sys to securely compute the trajectory unit vector with the aircraft's and satellite's locations as private inputs. After secure computation, the resulting trajectory unit vector is only revealed to the satellite. This is done to protect the satellite's location, as the aircraft could triangulate the satellite's location if more than one unit trajectory vector was learned during movement.

\begin{algorithm}
\caption{Private Trajectory Unit Vector Calculation}
\label{algo:twoWayAlg}

\begin{algorithmic}
%\STATE \textbf{Input}:
\STATE Satellite location: $(x_s, y_s, z_s)$
\STATE Aircraft location: $(x_a, y_a, z_a)$
\STATE $Party 1 \leftarrow Satellite$
\STATE $Party 2 \leftarrow aircraft$
\STATE \textbf{Computed}:
\STATE $f$ $\leftarrow$ $gen\_garble\_circuit(get\_trajectory(a, b))$
\STATE $g_1 \leftarrow{garble} (x_s, y_s, z_s)$
\STATE $g_2 \leftarrow{garble} (x_a, y_a, z_a)$
\STATE $reveal\_to\_party\_1(f(g_1, g_2))$
\STATE \textbf{Note: get\_trajectory( ) is the function described by Algorithm \ref{algo:oneWayAlg}}
\end{algorithmic}

\end{algorithm}


% \subsection{Computing Updates}\label{update}

%____________________________

% We consider a  dynamic setting where the trajectory unit vector needs to be frequently updated. In One-Way privacy model where the computation is done in plain-text, computing updates is as simple as recomputing the trajectory unit vector using Algorithm \ref{algo:oneWayAlg}. However, in the Two-Way privacy model, the computation is done under 2PC protocols, which are significantly more expensive that their plain-text counterparts. Therefore, we need to optimize the update process. \nasimeh{I think the following sentece is redundant} This is especially true for private multiplications, which are significantly more expensive than additions or subtractions. 

% Assume that we are under the  Two-Way privacy model and that at time $t$ we computed the trajectory unit vector, as described in Algorithm \ref{algo:oneWayAlg} \nasimeh{again, a reason for having an algorithm for Two-Way since alg. 1 is not used for Two-way}. At each update time, the aircraft has traversed a certain distance in space, represented by a 3D vector that we denote by $v_p$ with magnitude $c=|v_p|$. Since the aircraft is aware of its own location, it knows both $v_p$ and $c$. In order to compute the new trajectory unit vector, we will need to compute the distance between the satellite and aircraft at time $t+1$, denoted by $d_{t+1}$. During our initial setup of the trajectory unit vector, this computation requires three private multiplications and one private square root. However, we are able to leverage the Law of Cosines in order to optimize this computation in our update step, as shown below:

% \begin{equation}\label{eq:eq2}
% \begin{aligned}
%     d_{t+1}^2 = c^2 + d_t^2 - 2 \cdot c \cdot d_t \cdot \cos(\sigma)
% \end{aligned}
% \end{equation}

% \noindent where $\sigma$ denotes the angle between the vector connecting the satellite to the  aircraft at time $t$ and the aircraft's movement vector from time $t$ to $t+1$. Using the trajectory unit vector at time $t$ and the aircraft's $v_p$ vector, the aircraft is able to determine $\sigma$, $2 \cdot c \cdot cos(\sigma)$, and $c^2$ through plain-text computation completely independent of the satellite. Additionally, the private value $d_t^2$ has already been computed during the previous update (or the initial setup) of the trajectory unit vector. This results in a more efficient computation because only one private multiplication needs to be computed instead of the three required for computing $x_v^2 + y_v^2 + z_v^2$ in the setup phase of Algorithm \ref{algo:oneWayAlg}.
%__________________________________

% \nasimeh{I believe the following is just basic math and not required. You can explain briefly that the trajectory can be transformed into angle, etc} In cases where we'd like to direct our satellite's antenna using angles, rather than a trajectory vector, we note that converting between the trajectory unit vector and our defining angles is trivial. For a vector described by $[x, y, z]$, this can be computed as shown in equations \ref{sec:eq1}.

% \begin{equation}\label{sec:eq1}
% \begin{aligned}
%     \text{Azimuth} & \leftarrow \arctan2(y, x) \\
%     \text{Elevation} & \leftarrow \arctan2(z, \sqrt{x^2 + y^2})
% \end{aligned}
% \end{equation}

% \noindent where the azimuth angle describes the angle of direction with regards to the vertical axis, and the elevation angle describes the angle above the horizontal aircraft. The resulting angles can be converted to degrees if needed. We also note that it is trivial to reverse this computation as shown in equations \ref{eq:eq2}.

% \begin{equation}\label{eq:eq2}
% \begin{aligned}
%     x &\leftarrow \cos(\text{elevation}) \cdot \cos(\text{azimuth}) \\
%     y &\leftarrow \cos(\text{elevation}) \cdot \sin(\text{azimuth}) \\
%     z &\leftarrow \sin(\text{elevation})
% \end{aligned}
% \end{equation}

% \noindent This demonstrates that the minimal information needed to define the direction that an antenna must take to point at an aircraft is the unit vector or elevation and azimuth angles, as these components are interchangeable - i.e. if you have one, you can easily compute the other. 

% \nasimeh{My suggestion is that instead of talking about trivial math, put an algorithm (for Two-Way model) with all of the privacy protocols you're using and add details about them. For example, you're mentioning in experiments section that you're using oblivious transfer, etc but haven't even mentioned them here. You're skipping over your main contributions and instead are talking about angle vs trajectory.}
\subsubsection{Location Management}
Satellites are often in communication with many aircrafts at a given time. In tasks, such as surveillance, a satellite would like to enforce that the aircrafts are staying within these bounds. However, the aircraft wants to ensure that their location within these bounds is not revealed, in the event of the satellite being compromised. \sys enables this with a provably private location management scheme. \sys's location management is built upon ZKPs, which enables the aircrafts to prove that they are within bounds that are predetermined and updated by the satellite, while ensuring that the specific locations of the aircrafts are not revealed in the event of the satellite being compromised.

The bounds that the satellite enforces are established in the form $[(x_{min}, x_{max}), (y_{min}, y_{max}), (z_{min}, z_{max})]$. We design an efficient Boolean circuit for the $\geq$ and $\leq$ comparisons, based on fast custom multiplexers, that is compatible with the Wolverine protocol that returns a single bit. These circuits are used to compare the aircraft's current location $(x_a, y_a, z_a)$ with the satellites' enforced bounds in zero-knowledge. As computation in Wolverine is represented as Boolean circuits, we aim to take advantage of this by performing primarily bitwise operations. The high-level algorithm we employ can be see in algorithm \ref{algo:location}. As Wolverine has optimized implementations of zero-knowledge AND gates, \sys's location management requires very little overhead.

\begin{algorithm}
\caption{Check if aircraft is in bounds}
\label{algo:location}
\begin{algorithmic}
\STATE Aircraft $\gets$ Prover\\
\STATE Satellite $\gets$ Verifier\\
\STATE \textbf{Input}:
\STATE Bounds: $[(x_{min}, x_{max}), (y_{min}, y_{max}), (z_{min}, z_{max})]$
\STATE Aircraft location (known by aircraft): $(x_a, y_a, z_a)$
\STATE \textbf{ZK Circuit}:
\STATE Bit $x$ = $x_a.geq(x_{min})$ AND $x_{a}.leq(x_{max})$
\STATE Bit $y$ = $y_a.geq(y_{min})$ AND $y_{a}.leq(y_{max})$
\STATE Bit $z$ = $z_a.geq(z_{min})$ AND $z_{a}.leq(z_{max})$
\STATE Bit $valid$ = $x$ AND $y$ AND $z$
\STATE \textbf{Return $valid$}
\end{algorithmic}

\end{algorithm}

% \nojan{start describing system}

\subsection{Implementation Details}

While there are many coordinate systems for representing location in 3D space, we chose to evaluate on the Cartesian coordinate system. Other methods, such as the use of polar coordinates, often rely on trigonometric functions to determine the distance and direction between points in space. These functions are often more expensive to compute, particularly in the realm of privacy-preserving computing. 

\sys's 2PC location tracking implementation is built using \textit{MP-SPDZ} \cite{mpspdz}. MP-SPDZ is a user-friendly library that allows for implementation of 2PC while leveraging a variety of protocols to use in the backend. This enables benchmarking of different contexts, such as in semi-honest and malicious settings. We use  \sys's zero-knowledge location management is built using the Wolverine \cite{weng2021wolverine} protocol, an efficient interactive ZKP protocol that boasts highly scalable communication and performance. For both applications, we utilize state-of-the-art scalable Boolean logic synthesis tools \cite{emp, tg} to design low-level, optimized representations of our computation.

% If we follow the first privacy model described in \ref{sec:oneWay}, then we are assuming that the location of the aircraft is hidden but the aircraft is allowed to know (or compute) the location of the satellite. Consequently

% \nasimeh{don't understand the point of the following} One of our key observations is that, assuming our satellite's antenna is pointing at the aircraft, the main point of information that we are able to keep hidden is the distance from the satellite to the aircraft. In other words, a unit vector with the appropriate trajectory from the satellite to the aircraft represents the minimal knowledge that can be known by the satellite. 

% \nasimeh{MP-SPDZ seems to be a core part of your algorithm and should be explained earlier. Putting it here and they way you're talking about it feels like you're just advertising for MP-SPDZ :)) } 

\section{Evaluation and Discussion}\label{eval}
% \textbf{Setup. } 
\subsection{Setup}
The end-to-end \sys framework is implemented in C++ and the MP-SPDZ domain specific language.
We use the EMP-Toolkit~\cite{emp} for implementation of zero-knowledge proofs.
We run all experiments on a 128GB RAM, AMD Ryzen 3990X CPU desktop.
% \noindent\textbf{Results. }
\subsection{Results}
% In our evaluation, we benchmark \sys's movement tracking in both semi-honest and malicious settings \cite{mpspdz}.  
% \nojan{Yaman: Discuss your results here and what the takeaways are. Try to come up with some explanation as to why the communication numbers don't change. Discuss that we vary bitwidth to allow for flexibility so we can adapt to different communication bandwidths (btw are you still doing the bandwidth experiments?) oh lol i get it now. communication scales with size of circuit not bitwidth, so that's why they stay the same - i figured bc of the boolean nature of things maybe it would grow but i guess not}
% \nojan{Yaman, are you still doing bandwidth experiments?}

We benchmark \sys's capabilities over several bitwidths ranging from 128 to 8 bits. We note that we do not perform separate experiments for the semi-honest and malicious threat model for the location management task as Wolverine is designed to be maliciously secure, however supports semi-honest parties. The results can be seen in Tables \ref{tab:movement} and \ref{tab:manage}. As can be seen in both applications, the communication is not affected by the bitwidths. This is due to the fact that all the specific protocols - Semi, MASCOT, and Wolverine - that are used in \sys are designed such that communication grows with respect to the size of the computation/circuit. Due to the lean, optimized algorithms we design for \sys, the communication cost is negligible in the semi-honest movement tracking and location management tasks. We see a significant increase in communication cost and runtime in the malicious location management setting, as MASCOT introduce OT correlation checks and MAC generation schemes to ensure that the protocol is completely soundly.

\begin{table}[hbt!]
  \centering
  \small
  \resizebox{\columnwidth}{!}{\begin{tabular}{ccccc}
    \toprule
    & \multicolumn{2}{c}{Semi-Honest} & \multicolumn{2}{c}{Malicious} \\
    \cmidrule(lr){2-3} \cmidrule(lr){4-5}
    \textbf{Bitwidth} & \multicolumn{1}{c}{\textbf{Runtime (s)}} & \multicolumn{1}{c}{\textbf{Comm. (MB)}} & \multicolumn{1}{c}{\textbf{Runtime (s)}} & \multicolumn{1}{c}{\textbf{Comm. (MB)}} \\
    \midrule
     128 & 0.13 & 0.75 & 3.44 & 168.40 \\
    100 & 0.09 & 0.75 & 3.18 & 168.40 \\
     64 & 0.08 & 0.75 & 3.13 & 168.40 \\
     32 & 0.08 & 0.75 & 3.12 & 168.40 \\
     16 & 0.08 & 0.75 & 2.96 & 168.40 \\
     8 & 0.06 & 0.75 & 2.95 & 168.40 \\
    \bottomrule
  \end{tabular}}
\caption{Evaluation of \sys's Movement Tracking}
  \label{tab:movement}
\end{table}

% \vspace{-6mm}

\begin{table}[hbt]
    \centering
    \resizebox{\columnwidth}{!}{\begin{tabular}{ccccc}
    \toprule
    & \multicolumn{2}{c}{Prover (\Prv)} & \multicolumn{2}{c}{Verifier (\Vrf)} \\
    \cmidrule(lr){2-3} \cmidrule(lr){4-5}
     \textbf{Bitwidth} & \textbf{Runtime (ms)} & \textbf{Comm. (kB)} & \textbf{Runtime (ms)} & \textbf{Comm. (kB)} \\
     \midrule
     128 & 53.87 & 84.54 & 44.88 & 105.12\\ 
     100 & 33.39 & 84.54 & 34.45 & 105.12\\
     64 & 24.79 & 84.54 & 23.32 & 105.12\\
     32 & 11.5 & 84.54 & 10.14 & 105.12\\
     16 & 6.71 & 84.54 & 6.92 & 105.12\\
     8 & 5.96 & 84.54 & 6.08 & 105.12\\
    \bottomrule
    \end{tabular}}
    \caption{Evaluation of \sys's Location Management}
    \label{tab:manage}
\end{table}

In semi-honest movement tracking and location management, \sys boasts very low runtime, making \sys a real-time solution for private aircraft management. While the malicious movement tracking setting leads to much higher runtime, this is a common pitfall of the malicious threat model. Due to the significantly increased amount of computation that needs to be done to ensure that a malicious party does not break the protocol, this increased runtime is unavoidable. We provide the measurements of runtime for several bitwidths, from 128 bits to 8 bits. This is done to show how \sys overhead changes based on the amount of precision that is necessary to successfully locate or prove the location of an aircraft. Based on the bandwidth and computational power that is available at a given time, the bitwidth can be adjusted to reduce the computational overhead or satellite-aircraft interaction time. 

The presented results highlight the communication and computational efficiency that \sys achieves by utilizing novel techniques combined with state-of-the-art privacy-preserving protocols. Unlike most privacy-preserving approaches, \sys is not limited by scale, as the circuit sizes and the number of parties per computation stay constant. The only minor bottleneck is communication, which only becomes a problem if the network, SATCOM in our case, has limited bandwidth. Overall, the results show that \sys is a real-time solution to private aircraft management that can scale to real-world applications.

% \noindent\textbf{Extending SPAM. }
\subsection{Extending \sys}
As a brief aside, we want to discuss the extension of \sys into the IoT - particularly the zero-knowledge location management aspect. Oftentimes, edge devices have computational and bandwidth constraints, which makes it challenging to implement privacy-preserving solutions in the IoT. Due to the lightweight communication and computational overhead that we show for \sys, we believe that this work can be easily extended to the IoT and provide a secure solution towards edge devices proving attributes of their collected data. 
%\yaman{please check, this almost seems like it should be in related work:} Additionally, there are many works which demonstrate the feasibility of zero-knowledge proof based systems for IoT devices. \cite{iot1} provides a method for ZKP-based secure authentication for IoT devices. \cite{litezkp, iot2} demonstrate approaches for efficient data exchange and authentication in resource-constrained IoT systems. 
In \sys, we have aircrafts proving that their current location is within certain pre-determined bounds using ZKP techniques.  There are many works that demonstrate the feasibility of ZKP-based systems for IoT devices \cite{iot1,litezkp, iot2}. It is therefore expected that the ZKP techniques employed in \sys can be extended to different sensor-based IoTs. For instance, users that utilize biomedical IoT edge devices can adopt the \sys approach to prove that their health vitals (e.g. heart rate) are within a certain "safe" range. As health data is often sensitive, the user would prefer that this data stays private, unless their vitals are deemed to be in an unsafe range. While this is just an example, it is clear to see that \sys's approach towards aircraft location management can be modified to span many domains within the IoT. 
%With the \sys approach, a low-overhead solution is available for IoT edge device management and private data collection.

\section{Conclusion}
In this work, we presented \sys, a privacy preserving framework for optimizing transmission efficiency and aircraft management. By leveraging methods for secure two-party computation, \sys preserves the privacy of each party's location while enabling a satellite to maximize its communication throughput by securely computing the optimal trajectory for its antennas to take. \sys also utilizes zero-knowledge proof techniques to provide a monitoring system that enables a satellite to determine if an aircraft's location falls within a predetermined range without revealing the exact location of the aircraft. Automated Boolean logic synthesis is utilized for representing computation to ensure state-of-the-art runtime and communication for our given application. While many standard methods of satellite communication provide little privacy preservation, \sys implements an automated end-to-end system for protection against both semi-honest and malicious adversaries, as well as security against eavesdropping attacks, with little overhead. 
% \nojan{also, pls cut down on the references - i really don't think we need this many and it makes our paper look weaker to have only 5 pages of content}

% \section{Acknowledgments}
% \begin{acks}
% Acknowledgements go here.
% \end{acks}

\bibliographystyle{ACM-Reference-Format}
\bibliography{sample-base}

\end{document}
