\section{Conclusion}
\label{sec:conclusion}
%
We proposed a new descriptor to characterize the flow of people in large public infrastructures.
Based on 3D data and without tracking people, we provided an occupancy map of the space---a very useful indicator of the service performance.
We proposed an unsupervised methodology to identify and cluster these occupancy maps, without knowing their number or shape/configuration \emph{a priori}.
The approach is divided into two main steps: identify regular and irregular periods/maps, and cluster the regular maps into classes corresponding to the different queue configurations.
%
This methodology proposes a continuous irregularity measure for each map. The first step of the approach uses this measure as a means to identify the most abnormal maps in the data set. 
The second step segments the regular maps, based on an estimate of the number of clusters, and then consolidates the clusters using the Normalized Subspace Inclusion (NSI) criterion.

The approach gave consistent results, independently of the quantity of data analyzed, with the descriptor and NSI having smooth deterioration in the presence of missing data.

The data originates from different days but it has a common pattern: the busiest hours occur in the same periods of the day.
Finally, irregular maps exist mainly during late night, when maintenance operations are due.

As future work, we plan to augment the model to account for more information sources, \emph{e.g.}, velocity, flight schedule, flight destiny, in order to strengthen the assessment of the service performance.
%

