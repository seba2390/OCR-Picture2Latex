%\thispagestyle{fancy}

\section{Introduction}
\label{sec:intro}
%
In this article, we address the problem of crowd monitoring in public infrastructures. We propose a sensing and data processing framework which captures crowd occupancy in these large spaces, and identifies and classifies it into meaningful spatial patterns/classes. 
We tested it in a real-life scenario in the main security X-ray screening area at Lisbon's international airport (LIS), Portugal.

Airports are transportation hubs subject to strict service-level agreements (SLA), high security risks and high operational costs. These factors put great pressure on human and physical resources, calling for tight monitoring of passenger flow within the infrastructure.
In particular, the security and identification checkpoints are critical bottlenecks in the path between the check-in and the departure gate. Besides the risk of SLA violation, these bottlenecks have great impact on operational and commercial costs. 
Our work focuses on characterizing how passengers flow while waiting for inspection in these critical checkpoints. 

The sensing infrastructure we propose consists of a network of depth cameras that provide 3D data of the covered space. A set of pre-processing algorithms anonymously detect people in the 3D point cloud and compute the 3D space occupancy map.
%
\begin{figure}[t]
\centering
\includegraphics[width=0.48\textwidth]{./imgs/xray-building-plant-v4.png}
\caption{X-ray screening queue area at the Lisbon international airport. The colored lines illustrate examples of paths in the queue area. The queue area is around $200\text{m}^2$.}
\label{fig:queue-modes}
\end{figure}
%
Fig. \ref{fig:queue-modes} displays the specific geometry of LIS X-ray area\footnote{The physical infrastructure described here no longer exists.}, where passengers enter from the right, wait for their turn and exit to the left towards individual X-ray booths. In such a large space---more than $200\text{m}^2$---several configurations emerge, shaped by inflow variations, processing capacity, planning protocols and particular events. 

Currently, the global status of the X-ray system is monitored using global counts of in-out passenger flow \cite{denman2015automatic,felkel2012comprehensive}. These global measures do not account for the internal state of the queue that reflects the instantaneous operating condition of the whole system.
Queue modeling (e.g., flow parameterization) is very difficult to do, despite some structure imposed by queue guides and a relatively controlled environment. X-ray queue configurations are set in an ad-hoc manner and often change according to discretionary decisions by operations personnel. 
Also, innumerable ``small'' local decisions impact the global system: metal detectors settings, protocols for suspicious luggage screening, passenger inspection or security personnel skills. 
All these aspects are hard to account for but affect the performance and overall state of the X-ray queue.

Our methodology for classification of passenger flow patterns identifies meaningful classes that encode space occupancy. Classes are defined by space occupancy patterns that appear regularly over time (\emph{regular} patterns) as well as sporadic patterns that correspond to ``atypical behaviours'' (\emph{irregular} patterns). 
The classes and its number are not known \emph{a priori} and the amount of outliers can be significative.
Because of the large amount of data, it is very difficult for the airport management to define the most relevant classes.

Building on previous work in the computer vision domain \cite{oat2016mfs}, we encode regularity/irregularity through linear subspace models that explicitly represent these regularity concepts \cite{soltanolkotabi2012geometric,elhamifar2013sparse}. Such subspaces are determined from data in an unsupervised way. 

Our methodology grants decision-makers with the much sought holistic measure of the ``operational state'' of the infrastructure. The methodology is general and applicable to a wide variety of public infrastructures such as railway stations or commercial shopping centers.
%

In summary, our main contributions are:
\begin{itemize}
    \item a 3D sensing infrastructure for queue data collection;
	\item a descriptor computed from 3D data of multiple cameras that captures queue patterns;
    \item an algorithm that explicitly models and identifies \emph{regular} and \emph{irregular} flow patterns and clusters the regular data into classes of ``typical'' operation modes.
\end{itemize}
%

By identifying regular ``operating states'' of the whole system, airport experts are able to rank and map them to  desirable performance indicators. 
Having these states, one can build individual and more accurate models for each of them using queueing theory, for example, seeking to improve management and security. The ultimate goal is to increase costumer satisfaction. 
This interaction with the management and operational staff is very important in order to understand the behavior of the whole system~\cite{chen2015survey}. Our methodology can be understood as a pre-processing stage whose information can feed the planning, scheduling and performance analysis systems of the airport. 

