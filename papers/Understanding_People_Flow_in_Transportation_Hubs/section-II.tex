\section{Previous Work}
\label{sec:related-work}

In complex environments such as an airport \cite{wu2013review}, the flow of passengers depends on a large number of variables: staff performance, passenger behaviour, hand luggage contents, sensitivity of metal detection systems, among others. 
Traditional approaches based on discrete event systems \cite{guizzi2009discrete,dorton2016effects} or queueing theory \cite{gilliam1979application,takakuwa2003modeling,de2013virtual} model airport activity as a function of several such variables. 
However, these methods are difficult to tune because they are highly dependent on variables that are unobservable or modeled with (unrealistic) stationary distributions (\emph{e.g.}, arrival and service rates). Also, these are based on data manually acquired within limited time frames or based on aggregated statistics.
Besides providing tools for real time data acquisition, our methodology identifies the stationary states of the queue, allowing the design of individual and more accurate models for each state.

To automatically acquire people flow data, several works track and count people with a single video camera \cite{albiol2009statistical,barandiaran2008real} or a single over-the-head RGB-D camera \cite{gao2016people,del2015versatile,fu2014scene}. 

In crowded scenes, instead of tracking people individually, several works propose solutions for activity detection or estimation of the number of people present in the scene.
For example, Convolutional Neural Networks (CNN) are employed in \cite{zhao2016crossing} to compute a per-pixel crowd counting map in order to estimate the number of people crossing a line.  
\cite{zhang2015cross} improves crowd counting by training a CNN using two related objectives, crowd density and count, to obtain a better local optimum for both.
Alternatively, \cite{zhang2016single} proposes a multi-column CNN to map an image to its crowd density map.
In \cite{ma2013crossing} and \cite{mukherjee2015unique}, the authors use features derived from optical flow to train a system for counting people in regions of interest at crowded places. 
The optical acceleration and histogram of optical flow gradients can be combined to detect abnormal objects or speed violation in pedestrian scenes \cite{nallaivarothayan2014mrf}.
A probabilistic approach is employed by Wang \etal, \cite{wang2009unsupervised}, to cluster moving pixels and video segments in order to model atomic activities and interactions in crowded places.
In \cite{roshtkhari2013online}, the detection of abnormal behaviors is done with a spatio-temporal analysis of a densely sampled video volumes.

%
Clustering individual trajectories is another common approach to analyze the flow of people.
In \cite{cheriyadat2008detecting}, feature point tracks are clustered and dominant trajectories are identified by fitting polynomials to cluster mean points.
Approach \cite{lei2016robust} proposes the Robust K-means algorithm for clustering data that is less sensitive to the initialization of the $K$ clusters.
K-means and agglomerative clustering are used in \cite{kalayeh2015understanding} to cluster trajectories into common patterns. 
Alternatively, in \cite{morris2008learning}, trajectories are clustered with Fuzzy C-means and modeled with HMM, for trajectory analysis. Also, local distance and similarity measures are frequently employed to cluster and analyze trajectories and flow data \cite{liao2005clustering, hou2016repeatability}.

%
In \cite{wu2011real}, the authors propose a system for monitoring queues by tracking people using a network of over-the-head video cameras. 
A solution to integrate existing monitoring technology is proposed in \cite{denman2015automatic}. It studies methodologies for people counting and individual tracking for queue monitoring and behavior analysis. 

In our scenario, the placement of the sensors is limited by the infrastructure, leading to large scene perspective distortion, people occlusion and rendering the registration of video cameras unfeasible. 
Moreover, due to privacy concerns, the use of video cameras is not allowed in many locations, including the security checkpoint at Lisbon Airport.
This precludes the use of the referenced approaches, which, besides using RGB cameras, assume: high ceilings, proposing over-the-head solutions \cite{gao2016people, del2015versatile,wu2011real}; access to airport sensing infrastructure \cite{denman2015automatic}; existence of reliable individual trajectories \cite{cheriyadat2008detecting,kalayeh2015understanding,morris2008learning, candamo2010understanding}; absence of outliers% in the data
, as irregular patterns \cite{lei2016robust}; and large amounts of training data \cite{zhang2015cross, zhang2016single, zhao2016crossing}.

To the best of our knowledge, there is no work proposed for unsupervised identification and classification of people flow patterns from depth data.
Our approach is applicable in a large range of scenarios because it does not require a tracking scheme.
Also, we bring the novelty of testing our methodology with $14$ days of real data acquired at an international airport.

