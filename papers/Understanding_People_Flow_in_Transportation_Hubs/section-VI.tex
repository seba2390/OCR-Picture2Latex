\section{Experimental Results}
\label{sec:results}
In this section, we present quantitative results for synthetic data and qualitative results for real data, with periods of one day and fourteen days. %and four days during Christmas time. 
We used seven depth cameras to cover the X-ray queue area at the Lisbon international airport. The descriptor used integrates $5min$ occupancy.
%
\subsection{Synthetic Data}
%
To quantitatively assess the performance of our method and to compare it with other clustering methods, we perform a set of experiments with synthetic data. 
The classes are unknown to the algorithm and the data is corrupted with noise and outliers.

We consider a squared space of $400m^2$, with a total of 8 doors and several possible paths between those doors (Fig. \ref{fig:synth_maps}). At each time instant, we randomly generate a simulated passenger at a given door. Each passenger moves along a set of predefined paths (subject to perturbations), with speed $s\backsim\mathcal{N}(1.4ms^{-1},0.05)$. Similar to the method for people detection used in the real setup \cite{carvalho2016detecting}, objects are correctly detected $96\%$ of time. 
Outlier trajectories are created with passengers partially following one of the regular paths but then temporarily moving away to go through random points in space, returning finally to the normal course. 
We build $5min$ occupancy maps as explained in Section \ref{sec:occ_map}, with one binary map per second.
Fig. \ref{fig:synth} shows some examples of regular and irregular occupancy maps created with this synthetic data. 
These data recreates the real setting, with a similar large space with several doors and paths, passengers moving at the average human speed and occupancy maps with the same characteristics.
In the following experiments, we create data matrices with 10 clusters, corrupted with several percentages of outliers. For each such percentage, we run 20 simulated experiments with 576 maps each. 
%
%
\begin{figure}[bht]
\centering
\subfloat[Regular maps]{\includegraphics[width=0.45\textwidth]{./imgs/synth_maps_doors.png}
\label{fig:synth_maps}}\\
\hspace{0.3mm}
\subfloat[Outliers]{\includegraphics[width=0.44\textwidth]{./imgs/synth_out_v2.png}
\label{fig:synth_out}}
\caption{Examples of occupancy maps generated from synthetic data: (a) regular maps, with the 8 doors illustrated in the two left maps; (b) outlier maps. Each map integrates data from a period of $5min$.}
\label{fig:synth}
\end{figure}
%

We compare our method with three well established clustering approaches: \emph{K-Means} \cite{lloyd1982least}, \emph{K-Medoids} \cite{kaufman1987clustering} and DBSCAN \cite{ester1996density}. 
We include K-Means because it is widely used, however, it is inadequate for this application, performing poorly.
We evaluate the clustering error, defined as the rate between incorrectly classified points and the total number of points. In this comparison, we consider only the error in the regular maps, meaning we do not assess the explicit identification of irregular maps. 
We input the correct number of clusters to K-Means and K-Medoids, and use angle between points as distance function.
%
Table \ref{tab:ec} shows the average clustering error as a function of the percentage of outliers. DBSCAN accounts for outliers in the data but, similar to K-Means, performs poorly with this data. K-Medoids achieves small errors for lower percentages but loses accuracy as the number of outliers increases.
On the other hand, our approach is able to achieve error close to zero for all percentages of outliers. Note that our error is higher with $20\%$ of outliers because we set $p=0.50$ and, for some experiments, that value of $p$ removes classes from the regular data. With a smaller $p$, we can also achieve zero error for this amount of outliers.
%
\begin{table}[hbt]
\caption{Average clustering error as a function of the percentage of outliers, $p_{out}$. For each trial with K-Means and K-Medoids, we used the best of 10 replicates. For our method, we report the error for $p=0.50$, NSI threshold $0.93$ and $\gamma = 17$ clusters.}
\label{tab:ec}
\centering
\begin{tabular}{llllllllll}
\toprule
$p_{out}$& 0.20&0.30&0.40&0.50&0.60\\
\midrule
K-Means &0.284&0.334&0.388&0.393&0.402\\
DBSCAN&0.206&0.223&0.228&0.223&0.212\\
K-Medoids&\textbf{0}&0.030&0.030&0.067&0.104\\
Proposed Method  &0.022&\textbf{0.009}&\textbf{0.004}&\textbf{0.003}&\textbf{0.005}\\
\bottomrule
\end{tabular}
%}
\end{table}

In Fig. \ref{fig:ec_p_splits}, we evaluate the clustering error as a function of the parameter $p$, percentage of outliers and as function of number of clusters in which the regular maps are segmented, $\gamma$. 
%
As we show, our method has the best performance for a large range of values of parameter $p$. 
To achieve this, we must ensure that all outliers are removed, $p > \%outliers$, and that none of the clusters is removed from the regular data.
%
On the other hand, even if not all outliers are removed, the split and merge strategy improves the results because the remaining outliers have a reduced impact on the original clusters. 
%
\begin{figure}[bht]
\centering
\includegraphics[width=0.48\textwidth]{./imgs/ec_p_splits_in_and_out.png}
\caption{Clustering error versus parameter $p$ for three levels of outliers. Top plot: segmenting data in 10 clusters. 
Middle plot: segmenting in 14 clusters. Bottom plot: segmenting in 17 clusters. In each plot we show the error for three percentages of outliers, $p_{out} \in \{0.20, 0.30, 0.40\}$.}
\label{fig:ec_p_splits}
\end{figure}
%

%
\subsection{Real Data: Analyzing One Day}
%
\label{sec:one-day}
%
Here we present in detail the results of each step of our methodology applied to real data of one day. Fig. \ref{fig:c-16-03} shows the $\mathbf{C}$ matrix obtained by solving \eqref{eq:mfs} for this day.
The block diagonal shape of $\mathbf{C}$ suggests that classes appear in sequence and do not change often during the day.

\begin{figure}[ht]
\centering
\subfloat[Coefficients matrix $\mathbf{C}$.]{\includegraphics[width=0.33\textwidth]{./imgs/C_orig_w_guides.pdf}
\label{fig:c-16-03}}\\
\subfloat[Clusters from the Spectral Clustering with $\gamma=5$.]{\includegraphics[width=0.33\textwidth]{./imgs/spc-50_guides.pdf}
\label{fig:spc-16-03}}\\
\subfloat[Clusters after merging with NSI.]{\includegraphics[width=0.33\textwidth]{./imgs/nsi-50_guides.pdf}
\label{fig:nsic-16-03}}\\
\hspace{-0.5mm}
\subfloat[Final clusters.]{\includegraphics[width=0.35\textwidth]{./imgs/final-50_guides.pdf}
\label{fig:final-16-03}}
\caption{Results for one day of data: (a) coefficients matrix $\mathbf{C}$; (b) over-segmentation after labeling $50\%$ of the maps as irregular; result for $\gamma=5$, with one color per cluster (white for irregular maps); (c) consolidated clusters obtained using the NSI criterion, with one color per cluster, and some of the corresponding regular maps; (c) final clusters, obtained after classifying maps initially labeled as irregular. The gray vertical dashed line in all sub-figures allows to relate coefficients in $\mathbf{C}$ with clustering results.}
\label{fig:16-03}
\end{figure}
%
Fig. \ref{fig:irreg} shows the \emph{irregularity} for this sequence of maps, given this matrix $\mathbf{C}$.
After ranking the maps, we remove the $50\%$ with highest irregularity.
Next, we partition the \emph{regular} data ($50\%$ with lowest error) in $5$ clusters using spectral clustering\footnote{The adjacency matrix for the Spectral Clustering is $\mathbf{C}+\mathbf{C}^T$.}. 
Fig. \ref{fig:spc-16-03} shows the obtained labels, with one color per cluster and irregular maps in white.
%
By consolidating these clusters with the NSI criterion, we obtained three clusters (green, yellow and red labels) corresponding to three different queue configurations, depicted in Fig. \ref{fig:nsic-16-03}.
Finally, we classified the $50\%$ maps previously labeled as irregular to assess if they belong to any regular clusters.
Fig. \ref{fig:final-16-03} shows the final labels, where $11\%$ of the maps remain labeled \emph{irregular}.

During the periods 00:00$am$ - 06:15$am$ and 07:45$pm$ - 11:59$pm$, the queue had low inflow of passengers and people would go straight from the entrance to the exit of the queue (green label in Fig.\! \ref{fig:final-16-03}). 
The period with largest passenger inflow occurs between 06:15$am$ and 07:15$am$ (red label). In this period, the queue structure is more complex, occupying a larger area to accommodate all passengers. 
Yellow label corresponds to a configuration with moderate flow.
Irregular maps, with white label, occur mainly in the night periods (with low passenger inflow), when cleaning and maintenance operations are performed.

%
%
\subsection{Real Data: Analyzing an Extended Time Period}
\label{sec:larger-periods}
%
\begin{figure*}[tbh]
\centering
\includegraphics[width=0.8\textwidth]{./imgs/14_labels_system_off_largo.pdf}
\caption{Results of our methodology for 14 days of data, with $\mathbf{X} \in \mathbb{R}^{d \times 3583}$. Days are shown separately for easier analysis. Each queue configuration is labeled with a different color, with white corresponding to irregular periods. Day 3 has a large period without labeling, identified in gray, because the acquisition system was off.}
\label{fig:nsic-14days}
\end{figure*}
Fig. \ref{fig:nsic-14days} shows results for $14$ days of operation. The color labeling is shown for each day separately, although we applied the methodology to all data, with $\mathbf{X} \in \mathbb{R}^{d \times 3583}$.
These colors correspond to the same configurations of Fig. \ref{fig:16-03}.
White label is associated with irregular or empty maps\footnote{Empty maps were not included in the data matrix.}.

Although all days are different from each other, periods with more inflow look similar: between 06:00$am$ and 07:00$am$ or between 06:00$pm$ and 07:00$pm$. 
Irregular periods of time, as the ones depicted in Fig. \ref{fig:irreg-14days}, occur mainly in the early or late hours of the day and represent $12\%$ of the fourteen days, a time lapse of around $39$ hours\footnote{Day $3$ has a large period of time without labeling because the acquisition system was off.}.
Without identifying the outliers, the clusters computed by common approaches are contaminated and the corresponding subspaces may not represent the meaningful classes.

\begin{figure}[bht]
\centering
\includegraphics[width=0.38\textwidth]{./imgs/irreg-maps14.pdf}
\caption{Some of the irregular maps identified within a 14 day period.}
\label{fig:irreg-14days}
%\label{fig:pcl-rx}
\end{figure}
%

As we previously showed in Fig. \ref{fig:countings-maps}, similar passenger count at specific periods corresponds to different queue patterns. Fig. \ref{fig:counts-vs-labels} shows the same occurs for full days.
\begin{figure}[bht]
\centering
\includegraphics[width=0.4\textwidth]{./imgs/counts_vs_labels_final.pdf}
\caption{Counting and labeling for two different days. Although the counting is similar, the days are different.}
\label{fig:counts-vs-labels}
\end{figure}
%
This is due to several factors affecting the state of the queue, such as the number of open X-ray gates, number of staff members operating or luggage contents, for example. 
In other words, for the same number of passengers, the service can be operated differently.
