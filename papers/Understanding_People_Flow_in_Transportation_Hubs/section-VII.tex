\section{Discussion}
\label{sec:disc}
%%
We propose an unsupervised approach with few parameters to setup.
The first parameter is the period of time covered by each occupancy map, \emph{e.g.}, $30s, 1min, 5min$. 
This period of time should be chosen depending on the activity we want to capture with the descriptor.
For these experiments, we empirically chose $5min$ maps, that showed to be a good trade-off between computational cost and data filtering.
Another parameter is the period of time in which we search for clusters. 
In this paper, we use one day and two weeks but we can choose shorter or longer periods.
Our methodology presents consistent results and similar clusters can be identified whether analyzing a day by itself or combined with other days, including smaller clusters. 

Fig. \ref{fig:affinity-missing} shows the affinity matrix for nine clusters in a period where one of the (seven) cameras broke. For each cluster, we show the mean of its maps.
Clusters $7$, $8$ and $9$ are similar except in the area where the camera failed, in the middle left side of $9$. Similarly, cluster $6$ corresponds to the same queue state as clusters $4$ and $5$. 
Our method degrades gracefully in the presence of perturbations and clusters $6$ and $9$ have high affinity with the corresponding configuration classes.

Although the methodology presented in this paper is unsupervised, when working with real data and operational scenarios, interaction with end users/experts is very useful.
Because of the large amount of data, airport experts are not able to describe classes \emph{a priori}.
By providing only representative patterns, our methodology works as a filter and saves users from the overwhelming load of interpreting all data.
When presented with the relevant patterns, the airport experts recognize the classes as meaningful and understand the affinity matrix.
Finally, with these classes, and applying queueing and discrete events theory, we are able to compute individual and more accurate models for each state.
\begin{figure}[bht]
\centering
\includegraphics[width=0.38\textwidth]{./imgs/nsi2_parula_final_numbers.png}
\caption{NSI affinity matrix including maps with missing data. The characteristics of the descriptor and NSI allow for a smooth degradation of the affinity value when missing data.}
\label{fig:affinity-missing}
\end{figure}
%


