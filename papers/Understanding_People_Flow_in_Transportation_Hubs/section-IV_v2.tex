\section{Queue Pattern versus Throughput}
\label{sec:params}
%
In this section, we show that using passenger counts \cite{denman2015automatic}, \cite{felkel2012comprehensive}, is not enough to characterize the occupancy of the area being monitored.

Based on the people detection procedure, we are able to count people passing in a small area at the exit of the queue (Fig. \ref{fig:queue-box}). 
In this scenario, we have $4\%$ error, similar to the classification error obtained in the original paper \cite{carvalho2016detecting}.
%
\begin{figure}[bht]
\centering
\includegraphics[width=0.23\textwidth]{./imgs/queue-box_v3.pdf}
\caption{Illustration of a counting box at the exit of a queue.}
\label{fig:queue-box}
\end{figure}
%
Fig. \ref{fig:dashboard_valid} shows the number of passengers at the entrance (blue line) and at the exit (red line) of the queue during one day. 
The passenger count at the entrance is provided by the boarding pass scanning system. However, at the queue exit, our system counts people without distinction between passengers and staff, therefore, the difference between the two counts is due not only to the detection error, but also due to staff passing the exit and the delay between the two counting systems (one is at the entrance and the other at the exit).
%
\begin{figure}[bth]
\centering
\subfloat[]{\includegraphics[width=0.34\textwidth]{./imgs/xray_countings_gt_no_flights.png}
}
\caption{Number of people passing by the X-ray queue detected with our approach versus ticket validation count. Differences between the two are due to the ground truth being based in boarding pass validation at the entrance of the queue and the count being relative to all the people, including staff, at the exit of the queue. Our detection methodology is able to capture the outflow variation.}
\label{fig:dashboard_valid}
\end{figure}
%
Despite these sources of error, we can see in Fig. \ref{fig:dashboard_valid} that our detection procedure successfully captures the outflow of the queue.
However, the count of passengers is not sufficient to identify the queue state.

Fig. \ref{fig:countings-maps} shows four occupancy maps corresponding to four time instants. 
\begin{figure}[bth]
\centering
\subfloat[]{\includegraphics[width=0.38\textwidth]{./imgs/countings-maps.pdf}
}\\
\caption{X-ray exit count versus occupancy maps. Similar counts may correspond to different queue configurations.}
\label{fig:countings-maps}
\end{figure}
%
The periods around 07:00$am$ and 12:00$pm$ have very similar counts, however, the queue configurations are different. 
The state of the queue is dependent on many factors, including, for example, the relation between the queue inflow, service performance and the number of X-ray operating gates. 
Since these indicators are not available, the outflow of the queue is not enough to determine the queue/service state. Our solution relies on the occupancy map to capture the usage of the space.
%


