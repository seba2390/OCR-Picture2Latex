\section{Empirical Study Setup}
\huaxun{Not ready yet.}
We conducted an empirical study by collecting a set of real configuration compatibility issues from open source Android apps, aiming to understand (1) \textit{the common root causes} and (2) \textit{the consequences} of configuration compatibility issues. 
Exploring these two research questions can serve as a guide for the design of automatic issue detection tool. 
This section presents our methodology towards dataset collection and analysis method to answer the above research questions.

\subsection{Dataset Collection}
In order to answer the above two research questions, we need to collect the following two types of data from real Android apps: (1) bug reports, and (2) bug-related code revisions and discussions. 
To this end, we searched for suitable research subjects on F-Droid\cite{fdroid}, which is a famous website with a collection of open-source apps in high quality. 
We further filtered research subjects that meet the following criteria: (1) including a public issue tracking system, (2) receiving more than 500 stars on GitHub, and (3) pushing the latest git commit occurred within the 6 months. 
We chose these three criteria because the configuration compatibility issues in these selected subjects will tend have a greater impact on app users. 
We adopted the above three criteria and finally found that 71 apps meet our requirements.

In order to locate the configuration compatibility issues affecting 71 candidate apps, we use the following two types of keywords to search in the code repository: 
(1) A change log containing keywords related to compatibility issues. Since the Android framework source code on different devices is different and not open source, this article only considers the open source framework source code in AOSP, that is, compatibility issues related to API version evolution. We use "API level" and the version number of the Android system "Android i", where i represents an integer and is used as a keyword to search; If "-vXX" appears in the path of the changed file, this identifier means that the relevant xml file can only be used after API level XX, so as to avoid compatibility problems.
(2) In order to detect compatibility issues related to configuration, we use the following three keywords : "Xml", to detect XML-related issues, "resource", "AndroidManifest", these two Android configuration files.

We used the above two types of keywords, and a total of 3,136 code revisions were selected out of 71 apps. For each collected code revision, we carefully check the change log and code changes written by the original app developers in the code revision to confirm whether the code revision is related to compatibility issues in configuration files. Through the above steps, we have collected and found 136 configuration compatibility issues in 66 apps. These 136 issues together work as the dataset for our empirical research.


\subsection{Issue-Analysis Methodologies}

For the 136 configuration compatibility issues we collected, we performed the following tasks:
\begin{itemize}
	\item We first identify the xml tags and attributes that induce configuration compatibility issues; 
	\item We use issue-inducing xml tag and attributes as keywords to search for related discussions in online forums; 
	\item We use attributes and xml tags as a guide in the update history of the Android framework to understand how the issue occurs down to the level of code changes in Android framework.
\end{itemize}


We follow the methods described above to analyze our research results and classify the root causes and consequences of the configuration compatibility issues.

