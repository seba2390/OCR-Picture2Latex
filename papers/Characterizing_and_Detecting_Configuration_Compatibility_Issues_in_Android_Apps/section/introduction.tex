
\section{Introduction}
\label{sec:1}

The Android framework provides a flexible XML configuration environment, which is widely used by developers to control Android components' behaviors or even the entire apps, such as defining User Interface (UI) structures of~the apps' layout and declaring the required system permissions,~and so on.

Android continuously evolves to meet different market demands, resulting in
successive releases of thirty
different API
levels since its launch~\cite{androiddevelopers}.
Each API level introduces functional changes to the Android configurations to cater for revised requirements.
Such functional changes can cause the same configuration element in an Android
app to manifest inconsistent behaviors at different API levels. We refer to
such inconsistencies as \textbf{configuration compatibility issues}, which can
lead to poor user experience.
\sethlcolor{yellow}
\begin{figure}[t]
	\centering
	\includegraphics[width=0.5\textwidth]{./img/configurationfile.pdf}
	\caption{
		%\huaxun{Condition \#3: Revise crash stack trace at API level 21, add inconsistent look-and-feel at API level 22.}
		{A real-world example of configuration compatibility issues reported in \texttt{BUG 1486200} of gecko-dev.}}
	\label{fig:configurationfile}
\end{figure}

Figure~\ref{fig:configurationfile} shows a real-world example of a configuration compatibility issue (i.e. \texttt{BUG 1486200}~\cite{geckoissue}) reported by an open-source Android browser engine project called gecko-dev~\cite{gecko}, which has received 2.1K+ stars in GitHub~\cite{github}.
Lines 3-8 specify a \texttt{<selector>} configuration element to create a color
state list object.
It has a child tag named~\texttt{<item>} containing a set of attributes
such as \texttt{android:color} in Line~6.
{The configuration element triggers a compatibility issue, where the app normally works at an API level $\geq$ 23 but throws \texttt{ArrayIndexOutOfBoundsException} at API level 21 and manifests inconsistent look-and-feel at API level 22. The issue was caused by the different implementations among API levels 21, 22, and 23 to process the attribute value of \texttt{android:color} from XML configuration files in the Android framework (see Section~\ref{sec:motivating_example} for more details).}
In our empirical study, we observe that among the 196 real-world configuration
compatibility issues in 43 popular Android apps, 89 (45.4\%) of them can induce app crashes at certain API levels, while a further 88 (44.9\%) can induce inconsistent look-and-feel across different
API levels. This indicates that, in practice, configuration compatibility issues can induce severe problems in Android apps.

Nevertheless, uncovering configuration compatibility issues caused by such changes is non-trivial.
Specifically, our investigation of 200 top-ranked Android apps found that each of them on average specifies 25,991.6 configuration elements spreading across 663.1 XML files.
Typically, Android apps are designed to run on a range of API levels. It is expensive to design tests to check whether all these configuration elements and their attributes are handled as intended across these diverse API levels.
As a result, configuration compatibility issues can easily be missed by app developers.
Although the related official Android documentation, such as Android Developers~\cite{androiddevelopers} and Android API Differences Reports~\cite{differencereport}, records the information of configuration changes that can result in compatibility issues, such documentation can miss many configuration changes that manifest runtime inconsistencies. 
Even if the relevant changes are documented, the documentation can be overlooked by developers.
Thus, an automatic tool to help detect configuration compatibility issues is helpful.

Existing techniques on detecting software misconfigurations~\cite{rabkin2011static, xu2013not, behrang2015users,xu2016early, dong2016orplocator, chen2020understanding, toman2016staccato, reisner2010using} and Android incompatibilities~\cite{fazzini2017automated,ki2019mimic, wei2016taming, wei2018understanding, wei2019pivot, huang2018understanding, li2018cid, he2018understanding, li2018elegant} are not effective in identifying such configuration compatibility issues and pinpointing the root cause.
For example, the issue between API levels 22 and 23 in Figure~\ref{fig:configurationfile} is triggered by using different API calls with different guarded conditions to load the value of \texttt{android:color} in the XML tag~\texttt{<item>}. %\civi{Cannot follow ``color state list''}
Accurate identification of the root cause requires path-sensitive analysis, which can be expensive.
However, if we consider the existence of a compatibility issue whenever there are implementation differences on how the Android framework processes XML attributes, many issues identified are spurious as many code changes are irrelevant to compatibility issues.
It is non-trivial to analyze code changes that can trigger configuration compatibility issues in the Android framework with the large codebase (i.e., 4M+ LOC at API level 30) and a huge number of code changes in history (i.e., 250 git development branches with changes from 566K+ commits until April 2021).
So far, no prior works have studied the common patterns of code changes in the Android framework that induce configuration compatibility.
To fill such gaps, we conducted an empirical study linking the root causes of real configuration compatibility issues in 43 popular open-source Android apps to the code changes in the Android framework. We found several common code change patterns that can induce configuration compatibility issues in the Android framework (Section~\ref{sec:3}). %\civi{Please use reference.}

Based on the findings, we further propose~\textsc{ConfDroid}, which is a static analyzer encoding the common code change patterns to automatically generate issue-detection rules.
Specifically, \textsc{ConfDroid} performs an intra-class level symbolic execution based on the insight that \textit{common configuration compatibility issues are induced by the code changes within a single class that can process XML attributes in the Android framework}.
In this way, we can greatly reduce the cost of conducting path-sensitive analysis while ensuring comparable accuracy when identifying detection rules.

We implemented \textsc{ConfDroid} based on Soot~\cite{lam2011soot} and ran it on the Android framework code among API levels 21-30 to extract rules for detecting configuration compatibility issues.
The experimental results show that \textsc{ConfDroid} achieved a precision of 91.9\% by successfully generating 282 valid detection rules, which can be reproducible when manifesting runtime inconsistencies in the Android emulators.
Besides, \textsc{ConfDroid} can generate 65 validated rules that are missed by three state-of-the-art baseline methods.
Furthermore, we evaluated the usefulness of \textsc{ConfDroid} for issue detection in the 316 real-world Android apps from F-Droid~\cite{fdroid} and AppBrain~\cite{appbrain}.
Among 65 valid rules that are uniquely returned by \textsc{ConfDroid}, 11 of them have been enabled to generate 107 warnings that can be reproduced to manifest configuration compatibility issues in 30 apps.
So far, 52 warnings have been confirmed, and 51 warnings have been fixed by the developers.
We released the empirical and experiment datasets as well as the \textsc{ConfDroid} artifact in our project website~\cite{confdroid}.

To summarize, our work makes three major contributions.
\begin{itemize}
	\item An empirical research on open-source apps to help understand the common root causes and patterns of configuration compatibility issues.
	\item A technique named \textsc{ConfDroid} that can generate rules to facilitate an automatic detection of configuration compatibility issues in Android apps.
	\item An empirical evaluation showing that (1) \textsc{ConfDroid} outperforms existing approaches on generating new detection rules with high precision, and (2) rules extracted by \textsc{ConfDroid} can be successfully applied for detecting previously-unknown configuration compatibility issues.
\end{itemize}


{\Huge }