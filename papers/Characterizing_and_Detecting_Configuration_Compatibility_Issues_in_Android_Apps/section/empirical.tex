\section{Empirical Study}
\label{sec:3}
To facilitate automated detection of configuration compatibility issues, we conducted an empirical study on the characteristics and symptoms of such issues in real-world Android apps.
The study aims at answering the following two research questions:
\begin{itemize}
	\item \textbf{RQ1 (Issue types and root causes):} What are the common types and the corresponding root causes of configuration compatibility issues?
	\item \textbf{RQ2 (Issue symptoms):} What are the common symptoms of configuration compatibility issues?
\end{itemize}

\subsection{Dataset Collection}
We collected bug-related code revisions from well-maintained open-source
Android apps as the empirical dataset.
To this end, we searched for suitable subjects on F-Droid
~\cite{fdroid}, which is a famous repository containing high-quality open-source Android apps.
Specifically, we selected subjects that meet the following criteria: (1)
maintaining a public issue tracking system, (2) receiving more than 500 stars on
GitHub~\cite{github} (popularity), and (3) pushing the latest git commit within 
the most recent three months (well-maintenance).
We chose these three criteria because the configuration compatibility issues
located in these selected subjects are likely to affect many users due to the popularity of the apps. 
As a result, 43 apps were returned. 

In order to locate the configuration compatibility issues affecting the 43
selected apps, we used the following two types of keywords to search for
related code revisions:
\sethlcolor{orange}
\begin{itemize}
	\item Keywords related to Android framework versions. In practice, developers often indicate the specific versions of the Android framework in which compatibility issues occur in the changelog.
	Specifically, we used two keywords, \texttt{API} (API level for short), and \texttt{Android [i]} where \texttt{[i]} stands for an integer, to search for Android system versions in changelogs.
	Besides, we also looked for code revisions that contain version-specific XML files, which are stored in the path that contains a version qualifier \texttt{v[L]}, where \texttt{[L]} represents the minimum API level applicable to the files.
	\item Keywords related to XML configuration files in Android apps.
	Specifically, we chose the following two keywords: \texttt{resource}, and \texttt{AndroidManifest}, so that they can effectively cover all types of XML configuration files supported in the Android framework.
\end{itemize}
In total, 2,376 unique code revisions were identified from the 43 apps after removing duplicates from the searching results.

{Next, we conducted manual analysis on the 2,376 code revisions to refine configuration compatibility issues. Specifically, we collected the code revisions in three steps. First, we screened out the code revisions unrelated to valid configuration compatibility issues because some irrelevant code revisions (e.g., introducing new app features) can be accidentally returned by our keyword-based search. Second, we collected the incompatibility-inducing attributes and XML elements from the revision-related commit logs, bug reports, or code diffs. 
To answer RQ1, the code changes related to the incompatibility-inducing attributes and XML elements should also be identified in the update history of the Android framework to investigate how these changes can cause issues. Third, to answer RQ2, we referred to the information of code revisions and online discussions of similar issues for the consequences when developers did not handle problematic XML elements or attributes well. Eventually, we collected 196 configuration compatibility issues from code revisions as the empirical dataset.}

\subsection{RQ1: Issue Root Causes}
\label{sec:RQ1}
\begin{table}[t]
	\caption{Common root causes of configuration compatibility issues}
	\begin{tabular}{lp{5cm}r}
		\toprule
		&\multicolumn{1}{c}{\textbf{Root Causes}}         & \textbf{Issue \#} \\ \hline
		Type 1&Unavailable configuration APIs & 116 (59.2\%)       \\
		Type 2&Inconsistent configuration APIs & 42 (21.4\%)       \\
		Type 3&Inconsistent Android internal XML configuration files & 19 (9.7\%)\\
		Type 4&Inconsistent attribute dependencies    & 9 (4.6\%)        \\
		Type 5&Inconsistent attribute usages             & 9 (4.6\%)    \\
		Type 6&Inconsistent attribute default values         & 1 (0.5\%)    \\
		\bottomrule
		\label{tab:issuecategorization}
	\end{tabular}
\end{table}

We elaborated on the six common types (or causes) identified from the 196 configuration compatibility issues as shown in Table~\ref{tab:issuecategorization}.

\begin{figure}[t]
	\centering
	\includegraphics[width=0.5\textwidth]{./img/layerdrawable.pdf}
	\caption{The Android framework code for loading the attribute value of \texttt{android:gravity} in the class \texttt{LayerDrawable}.}
	\label{fig:layerdrawable}
\end{figure}
\textbf{Unavailable configuration APIs.}
The Android framework loads attribute values by calling configuration APIs after parsing the XML tags in configuration files to \texttt{AttributeSet} or \texttt{TypedArray} objects.
Some statements invoking configuration APIs are introduced or removed as the Android framework evolves, resulting in an inability to load the associated configuration attribute values in a certain range of API levels.
In our empirical dataset, we found 116 (59.2\%) issues that were induced by such a type of code changes.
For example, the attribute value of \texttt{android:gravity} in Figure~\ref{fig:layerdrawable} is loaded by \texttt{LayerDrawable} to adjust the gravity for layer alignment starting from API level 23.
A navigation app OsmAnd~\cite{osmand} filed an issue in commit 1bbf578 that the attribute value of \texttt{android:gravity} is not loaded when running at an API level below 23, causing the incorrect display of graphic user interfaces. %There are 116 (59.2\%) issues falling into this issue type.

\textbf{Inconsistent configuration APIs.}~Configuration APIs in the Android framework are designed to load attribute values in specific data formats.
%The configuration APIs to load an attribute may vary across API levels.
Compatibility issues can happen when the configuration APIs to load an attribute vary across API levels.
The example in Figure~\ref{fig:configurationfile} between API levels 22 and 23 falls into this types.
Such an issue is caused by the style format attribute value of \texttt{android:color} not being loaded by the configuration API \texttt{getAttributeIntValue()} at API level 22.
The loading of \texttt{android:color} in an unsupported format can result in app crashes at API level 22. 
There are in total 42 (21.4\%) issues of this type.

\textbf{Inconsistent Android internal XML configuration files.}
The Android framework provides a set of internal XML configuration files that can be referenced by the developers as a part of their apps.
Compatibility issues can happen when there are changes in those internal XML configuration files as the Android framework evolves.
For example, QKSMS~\cite{qksms} commit 6b70a47 describes an issue caused by the internal XML configuration file \texttt{ic\_menu\_added.xml}, which was introduced at API level 23. There are 19 (9.7\%) issues falling into this issue type.

\begin{figure}[t]
	\centering
	\includegraphics[width=0.5\textwidth]{./img/datepicker.pdf}
	\caption{Code changes of loading \texttt{android:spinnersShown} in the class \texttt{DatePicker}.}
	\label{fig:datepicker}
\end{figure}
\textbf{Inconsistent attribute dependencies.}~
In the Android framework, there are dependencies across configuration attributes.
In other words, the runtime behaviors of one attribute depend on the value of other attributes.
We found nine compatibility issues that were induced by the inconsistent implementations on attribute dependencies among API levels. %of triggering conditions of invoking configuration APIs in the Android framework.
%Such changes can cause an attribute not to be loaded at specific API levels.
For example, open-keychain~\cite{openkeychain} reported an issue in commit be06c4c.
As Figure~\ref{fig:datepicker}(a) shows, developers have specified the value of \texttt{android:spinnersShown} as \texttt{true} to make the date picker widget be displayed in the spinner mode.
The attribute value cannot be loaded at API level 21 without specifying the value of \texttt{android:datePickerMode} as \texttt{spinner}, causing the date picker to be displayed in the calendar mode by default.
In this case, the app crashed at API level 21 due to a specific implementation of the date picker in the calendar mode.
The code changes in Figure~\ref{fig:datepicker}(b) show that starting from API level 21, the configuration API for \texttt{android:spinnersShown} is guarded by the conditional statement that checks whether the value of \texttt{android:datePickerMode} is \texttt{spinner}.
%Note that the existing path-insensitive analysis technique can fail to analyze the code changes falling into this type although the proportion of issues caused by such a pattern are small. 
%As the example in Figure~\ref{fig:frameworkprocess} shows, existing techniques cannot infer that \texttt{android:color} will be loaded in Line 5 without analyzing the conditional statement in Line 4. 
%Such a case can make the existing techniques derive spurious issue-inducing code changes as \texttt{android:color} is loaded in \texttt{ColorStateList} starting from API level 23.

\textbf{Inconsistent attribute usages.}
Compatibility issues can happen when there are inconsistent implementations on how the Android framework uses the attribute values after being loaded by configuration APIs.
As Figure~\ref{fig:frameworkprocess} shows, there is a change in processing the value of \texttt{android:color} (in Line 12 and 13) between API levels 21 and 22. The change avoids the \texttt{ArrayIndexOutOfBoundsException} when the Android framework parses the XML element in Figure~\ref{fig:configurationfile} at API level 22.
In total, there are nine (4.6\%) issues falling into this issue type.

\textbf{Inconsistent attribute default values.}
There is one (0.5\%) issue caused by inconsistent default values of the attribute \texttt{android:useLevel} in the XML tag \texttt{<shape>} between API levels 21 and 22, as reported in the commit a221442 of OsmAnd~\cite{osmand}.

\subsection{RQ2: Issue Symptoms}
We further analyzed the common issue symptoms as below.
Specifically, 89 (45.4\%) of the 196 issues in our empirical dataset can cause the apps to crash when triggering incompatibility-inducing XML configuration elements, as shown by the motivating example in Figure~\ref{fig:frameworkprocess}.
Another 88 issues (44.9\%) can induce an inconsistent look-and-feel across different API levels, affecting the apps' functionalities.
For example, the problem reported in the commit a221442 of OsmAnd~\cite{osmand} can force the progress bar to always show a full circle.
The remaining 19 issues (9.7\%) can cause inconsistent app behaviors beyond crashes and look-and-feel.
For example, the app Slide~\cite{slide} specified \texttt{android:requestLegacyExternalStorage} to make sure the app can still request for the external storage at an API level $\geq$ 29.
This shows that configuration compatibility issues can cause severe consequences to the app developers.
