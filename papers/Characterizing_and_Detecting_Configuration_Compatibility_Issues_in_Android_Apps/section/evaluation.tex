\section{Evaluation}
\label{sec:5}
We implemented \textsc{ConfDroid} based on Soot~\cite{lam2011soot}.
In our evaluation, we answer the following research questions:
\begin{itemize}
	\item \textbf{RQ3 (Effectiveness of detection rule extraction):} What is the effectiveness of \textsc{ConfDroid} on extracting detection rules compared with baseline methods?
	\item \textbf{RQ4 (Usefulness):} Can rules that are uniquely extracted by \textsc{ConfDroid} be useful for detecting issues in real-world Android apps?
\end{itemize}

\subsection{RQ3: Effectiveness of Detection Rule Extraction}
\label{sec:rq3}

To answer RQ3, we ran \textsc{ConfDroid} on the Android framework code among
API levels between 21 and 30 (inclusive). We evaluated its accuracy in
extracting the detection rules of configuration compatibility issues.
We compared the results with the following baseline methods.
\begin{itemize}
	\item \textbf{Baseline I:} \textsc{Lint}~\cite{lint}, the popular static analyzer officially released by Google.
	By modeling the API levels when attributes were introduced, \textsc{Lint} can detect configuration compatibility issues caused by an unavailable attribute $\mathcal{A}$ at API level $\mathcal{L}$.
	\item \textbf{Baseline II:}
	\textsc{ORPLocator}~\cite{dong2016orplocator}, the state-of-the-art
	technique for extracting attributes in a software system by conducting path-insensitive analysis on its configuration APIs.
	We ran \textsc{ORPLocator} at different API levels to extract detection
	rules for unavailable attribute $\mathcal{A}$ at API level $\mathcal{L}$.
	\item \textbf{Baseline III:}
	\textsc{cDep}~\cite{chen2020understanding}, the state-of-the-art software misconfiguration tool which focuses on analyzing attribute dependencies in the software system.
	We ran \textsc{cDep} among API levels to generate detection rules for the unavailable dependencies between two attributes $\mathcal{A}_1$ and $\mathcal{A}_2$ at API level $\mathcal{L}$.
\end{itemize}

Note that we did not choose \textsc{SCIC}~\cite{behrang2015users}, the state-of-the-art tool for detecting misconfigurations caused by software evolution, as the baseline since its artifact is not available. Besides,
\textsc{ORPLocator} (Baseline II) and \textsc{cDep} (Baseline III) are more recently published techniques.
The original algorithms of \textsc{ORPLocator} and \textsc{cDep} comprise
inter-class analyses. However, they cannot scale up to the Android framework
and fail to extract any rules. 
As such, we slightly adapted \textsc{ORPLocator} and \textsc{cDep} to
performing intra-class analyses in the construction of the two baselines.
%\scc{Please check if my rewriting is correct.}
%\huaxun{Yes.}

\textbf{Validating detection rules.}
We considered an issue-detection rule $r$ as valid if the XML configuration
files that satisfy the triggering conditions specified by $r$ can induce
configuration compatibility issues between two API levels.
%We followed the strategies conducted by an existing study\cite{wei2019pivot} to validate rules.
%Note that the issue-detection rules produced by different methods take different forms.
Specifically, for each issue-detection rule $r$ extracted by \textsc{ConfDroid} and other baselines,
we used the attribute name, the XML tag, and the incompatibility-inducing API levels
specified in $r$ as keywords to search for relevant discussions in Stack
Overflow~\cite{stackoverflow} and GitHub~\cite{github}.
For those rules $r$ with no relevant discussions found, we manually crafted an
Android app that includes XML configuration files satisfying the
issue-triggering condition in $r$ to manifest inconsistent behaviors
across API levels.
Furthermore, the manifested inconsistent behaviors should disappear after
removing the incompatibility-inducing attribute from the concerned XML configuration file.
A rule $r$ is considered valid if we can find a relevant online discussion or
successfully craft an app confirming the existence of configuration
compatibility issues.

\begin{table}[t]
	\caption{Execution Time for Rule Extraction and Validation Rate of
	Extracted Rules. Rules in Lint are hardcoded.}
	\centering
	\begin{center}
	\begin{tabular}{p{40pt}p{40pt}p{30pt}p{40pt}p{35pt}l}
		\toprule
		\textbf{Method} {\centering}   &  {\centering \textsc{ConfDroid}} &  {\centering \textsc{Lint}}  & \textsc{ORPLocator} {\centering} &{ \centering \textsc{cDep}} \\ \hline
				 		\textbf{Time}  &  2h58m36s  & - & 2m26s & 3m25s  \\\hline
		 \textbf{Validation Rate}  &   91.9\% (282/307)  & 100.0\% (218/218) &
		 85.6\% (131/153) & 12.7\% (28/220) \\

\bottomrule& & &
\label{tab:ruleextraction}	\end{tabular}
\end{center}
\end{table}

\begin{comment}
	\begin{figure}[t]
	\centering
	\includegraphics[width=0.28\textwidth]{./img/issue-distribution.pdf}
	\caption{Valid issue-detection rules generated by \textsc{ConfDroid} and the other two baselines. \civi{Make the font in the Figure larger}}
	\label{fig:issuedistribution}
	\end{figure}
\end{comment}

Table~\ref{tab:ruleextraction} shows the execution time as well as the number
of rules
extracted by \textsc{ConfDroid} and baseline methods.
The execution time of \textsc{Lint} is inapplicable as the detection rule is
hardcoded.
Although the execution time of \textsc{ConfDroid} is larger than
path-insensitive approaches \textsc{ORPLocator} and \textsc{cDep},
\textsc{ConfDroid} can extract more valid detection rules and complete
analyzing the Android frameworks within three hours.
In total, \textsc{ConfDroid} successfully extracts 282 valid detection rules,
achieving a precision of 91.9\%.
By performing backward symbolic execution on the configuration APIs in the
Android framework, 
the 65 unique valid detection rules (47 unavailable configuration
APIs and 18 inconsistent configuration APIs) extracted by \textsc{ConfDroid}
are not found in any of the three baselines.

On the other hand, the 25 detection rules extracted by \textsc{ConfDroid} cannot be validated due to the following two reasons.
First, the intra-class level symbolic execution and the inabilities of using Z3 to process configuration APIs can cause incomplete configuration constraint extraction.
Second, the Android framework has proposed an internal workaround at the stage of the Android app compilations,
which is not analyzed by \textsc{ConfDroid}.
Besides, 14.4\% of the rules extracted by \textsc{ORPLocator} are invalid due to
(1) the adaptation of intra-class level analysis to ensure scalability, and (2) the inaccuracy of its path-insensitive analysis on the Android framework.
While we have resolved the scalability issues of the original version of
\textsc{cDep}, it achieves the lowest rule validation rate because of
(1) extracting incomplete attribute dependencies by only performing intra-class
level data flow analysis, and (2) identifying minor code changes that will not
induce compatibility issues.

Still, we found 28 rules (28/218=12.8\%) that are uniquely included in \textsc{Lint} but not \textsc{ConfDroid}.
The false negative rules of \textsc{ConfDroid} concern the theme attributes (e.g., \texttt{android:attr/textColorSecondary} in Figure~\ref{fig:configurationfile}) whose value can only be referenced by another attributes in XML.
Such theme attributes are not loaded using the Android configuration APIs and
hence are missed by \textsc{ConfDroid}.
We have not witnessed any false negative rules in \textsc{ConfDroid} that were
extracted by \textsc{ORPLocator} because the valid rules of 
\textsc{ConfDroid} and \textsc{ORPLocator} fall into the pattern of \textit{unavailable
configuration APIs}.
We also did not witnessed any false negative rules that are extracted by
\textsc{cDep}. 
The \textsc{cDep} tool is designed based on the change pattern in attribute dependencies (Type 4 in Section~\ref{sec:RQ1}).
Such a pattern is not modelled by \textsc{ConfDroid} since it only accounts for 4.6\% of the issues in the empirical study dataset.
%Besides, more than 80\% of the compatibility issues were induced by a single attribute in our empirical study.
In fact, the validation rules generated by \textsc{cDep} are all caused by the introduction and removal of attributes, which are also dependent on other attributes in the Android framework.
Moreover, false negatives can also be induced by the inabilities of Z3 in
processing configuration APIs.
Although it is challenging to measure false negatives as we lack an authoritative
ground truth, we argue that the number of false negatives can be small since \textsc{ConfDroid} can
successfully process 98.1\% (1504/1533) of configuration APIs, which cover
79.0\% (1134/1435) of public attributes at the API level 30.

\subsection{RQ4: Usefulness}
\sethlcolor{green}
\begin{table}[t]
	\caption{{Number of warnings detected using the rules extracted by
	\textsc{ConfDroid} and baselines}}
	\centering
\begin{center}
		\begin{tabular}{lcccc}
\toprule
 &  {\centering \textsc{ConfDroid}} &  {\centering \textsc{Lint}}  & \textsc{ORPLocator} {\centering} & \textsc{cDep} \\ \hline
$D$&50,157&9,998&8,424&4,304\\
$F_{v}$ & 46,580  & 7,316 & 5,433 & 3,595 \\
$F_{lib}$ & 41,790 & 5,007 & 4,134 & 4,290 \\
$D-F_{v}-F_{lib}$ & 688 & 435 &  54& 140 \\
 \bottomrule
\label{tab:issuedetection}	\end{tabular}

\thefootnote{\leftline{$D$: Warnings generated by rules only.}}
\thefootnote{\leftline{$F_{v}$: False warnings identified by API level filter.}}
\thefootnote{\leftline{$F_{lib}$: False warnings identified by support library filter.}}
\thefootnote{\leftline{$D-F_{v}-F_{lib}$: The final output after filtering false warnings by $F_{v}$ and $F_{lib}$}}
\end{center}

\end{table}

To answer RQ4, we checked whether the rules of \textsc{ConfDroid} could be applied to detect real issues with the comparison of three baselines. Note that Android has introduced various protection mechanisms documented in Android Developers~\cite{androiddevelopers} to prevent configuration compatibility issues. The protection mechanisms include (1) checking the API level ranges in which the XML configuration file can be used and (2) using the Android support library (AndroidX) to parse XML configuration files. Leveraging detection rules alone without considering any protection mechanisms can induce a large number of false warnings. Therefore, we built two false warning filters $F_{v}$ and $F_{lib}$ to investigate the impact of protection mechanisms on issue detection. Specifically, $F_{v}$ checks whether the XML configuration file can be applied to incompatibility-inducing API levels. App developers can provide different copies of the XML configuration files for different API levels with the identifier \texttt{-v} (e.g., \texttt{-v24} in \texttt{layout-v24/a.xml}). For an issue reported by a rule $r$ with problematic API levels $[l_1, l_2]$, $F_{v}$ considers the API level identifier with the parameter value of \texttt{minSdkVersion} to check whether the XML file can be used at $l_1$. $F_{v}$ also checks whether there is another copy of the file for API level $l_2$. The reported issue is considered as invalid if the XML file cannot be used in both $l_1$ and $l_2$. $F_{lib}$ checks whether an XML configuration file can be parsed by the Android support library (AndroidX) instead. Here, we consider an XML configuration file to be free of compatibility issues if it is only referenced by the APIs or attributes defined in the library. We then built four different variants of issue detectors with two filters $F_{v}$ and $F_{lib}$ as Table~\ref{tab:issuedetection} shows.
	
We applied the above variants to detect issues in real-world Android apps.
We crawled 116 open-source Android apps in F-Droid~\cite{fdroid} with at least 50 stars in GitHub~\cite{github} and the last git commit made within the past six
months.
We further collected 200 closed-source apps marked as top-ranked in AppBrain~\cite{appbrain}.
None of these 316 apps overlaps with those used in our
empirical study (Section~\ref{sec:3}) to avoid the risk of overfitting.

\sethlcolor{green}{Table~\ref{tab:issuedetection} shows the number of warnings produced by each variant of issue detectors. We can see that the number of false warnings is significantly less when combining two false warning filters $F_{v}$ and $F_{lib}$, showing that the above two mechanisms are often adopted to handle compatibility issues. Several frequently-used attributes contribute to the number of warnings reported in Table~\ref{tab:issuedetection}. For example, developers intensively use an incompatibility-inducing attribute \texttt{android:strokeColor} in their apps. $F_{v}$ helps reduce 12,445 false warnings that developers have properly used such an attribute with API level identifiers. }

Considering the output of $D-F_{v}-F_{lib}$, the rules of \textsc{ConfDroid} generated 688 issue warnings, outperforming the other baseline methods.
Note that the number of warnings can be varied depending on the frequency of which incompatibility-inducing attributes are used in apps.
%Among them, 218 warnings were generated on 33 open-source apps in F-Droid, and 517 warnings were generated on 98 closed-source apps.
%Although 31 valid rules can be extracted by
%\textsc{Lint} and \textsc{cDep} but not \textsc{ConfDroid}, we only witnessed two warnings generated by them as the incompatibility-inducing attributes concerning those rules were not widely used by app developers.
As for the 65 rules that were uniquely identified by \textsc{ConfDroid}, 11 of them were activated to output 253 warnings from 74 apps, among which 90 warnings were from 20 open-source apps, while the remaining 163 warnings were from 56 closed-source apps.
The above results show that the rules uniquely generated by \textsc{ConfDroid} can be leveraged to generate issue reports that are unknown to other baselines.

We then conducted manual reproduction trying to manifest inconsistent runtime behaviors on the warnings that are uniquely reported by rules in \textsc{ConfDroid}.
Specifically, we first tried to build test cases to reach the \texttt{Activity} using the XML element in the issue reports.
Then, we tried to manifest the runtime behavior controlled by the incompatibility-inducing attributes in the XML element by reading the code logic of the \texttt{Activity}.
Since we have no access to the source code of the closed-source apps, we
analyzed the apps' code logic by reading the smali code decompressed from their
apk files.
An issue is considered reproducible if we can observe inconsistent runtime
behavior by building a test case based on the above two steps.
We successfully reproduced 107 warnings in 30 apps (67 warnings in 12 open-source apps and 40 warnings in 18 closed-source apps).
We failed to reproduce the remaining warnings for the following three reasons.
First, the XML elements where the incompatibility-inducing attributes are located were not reachable because of special triggering conditions (e.g., paying for unlocking hidden functionalities).
Second, false warnings were generated because it cannot recover the
workaround of configuration compatibility issues made by app developers.
Third, the app was obfuscated so that it is difficult for us to understand the app's runtime behavior through analyzing the smali code.
\sethlcolor{lightgray}
\begin{table}[t]
	\caption{{Issue Reports}}
	\centering
	\begin{tabular}{cccccl}
		\toprule
		 {\textbf{App}}   & \textbf{ID} {\centering} & \textbf{Warnings} & \textbf{Confirmed} & \textbf{Fixed} \\ \hline
		AndOTP~\cite{andotp}  & 539$^1$ & 42&42 & 42 \\
		Aria2App~\cite{arial2app} &42& 2& 0& 0\\
		Dash Wallet~\cite{dashwallet} &648& 3&0 &0 \\
		document-viewer~\cite{documentviewer} &328&1 &0 &0 \\
		FastScroll~\cite{androidfastscroll} &36$^1$& 1&1 & 0\\
		GoodTime~\cite{goodtime}  & 197$^1$ &1 & 1 & 1 \\
		Metro~\cite{metro} &55& 6& 0& 0\\
		Kontalk~\cite{kontalk} &1315& 1&0 &0 \\
		openWorkout~\cite{openworkout} &45&1 &0 &0 \\ 
		PersianCalendar~\cite{droidpersiancalendar}  & 621$^1$ &5 & 5 & 5 \\
		Twire~\cite{twire}  & 119$^1$ & 3& 3&3 \\
		UniPatcher~\cite{unipatcher} &52&1 &0 &0 \\ \hline
		\textbf{Total} &-&67 &52 &51 \\
		\bottomrule& 
		\label{tab:issuereport}& & 	\end{tabular}
	\footnote{*}{$^1$Issues that have received replies from app developers.}

\end{table}


We further reported the 67 reproducible warnings concerning 12 open-source apps
to the original app developers.
For each app, we only sent one issue report containing all warnings found in it.
Table~\ref{tab:issuereport} shows the issue reports that have been sent to the developers.
The distribution of reported issue warnings depends on the attribute usages in different apps. For example, our issue detector reports 42 warnings in the evaluation subject AndOTP since it frequently uses the attribute \texttt{android:fillColor} to define the colors in the app icons. The processing of \texttt{android:fillColor} in the Android framework has been changed between API levels 23 and 24, causing an inconsistent look-and-feel on the app's icons.
So far, 52 warnings concerning five apps have been confirmed, and 51 warnings in four apps have been fixed by developers.
We did not report issues for closed-source apps as they do not have a public issue tracker.
Instead, we released the reproduction results of both open-source and closed-source apps on our project website~\cite{confdroid}.
The above confirmed or fixed warnings demonstrate the usefulness of rules extracted by \textsc{ConfDroid} to facilitate configuration compatibility issue detection.
