\documentclass[conference,10pt]{IEEEtran}
\IEEEoverridecommandlockouts
% The preceding line is only needed to identify funding in the first footnote. If that is unneeded, please comment it out.
\newcommand{\tabincell}[2]{\begin{tabular}{@{}#1@{}}#2\end{tabular}}
\usepackage{amsfonts}
\usepackage{amsmath}
\usepackage{cite}
\let\Bbbk\relax
\usepackage{amssymb}
\usepackage{algorithmic}
\usepackage{graphicx}
\usepackage{textcomp}
\usepackage{xcolor}
\usepackage{amsthm}
\usepackage{verbatim}
\usepackage{balance}
\usepackage{multirow}
\usepackage{booktabs}
\usepackage{url}
\usepackage[switch]{lineno}
\usepackage{soul}
\usepackage[colorlinks, linkcolor=blue, anchorcolor=blue,citecolor=blue]{hyperref}
\soulregister{\cite}7 
\soulregister{\citep}7 
\soulregister{\citet}7 
\soulregister{\ref}7 
\soulregister{\pageref}7 
\soulregister{\texttt}7 
\def\BibTeX{{\rm B\kern-.05em{\sc i\kern-.025em b}\kern-.08em
    T\kern-.1667em\lower.7ex\hbox{E}\kern-.125emX}}
\usepackage[utf8]{inputenc}
\usepackage[ruled,linesnumbered,vlined]{algorithm2e}
\usepackage{threeparttable}
\SetKwInput{KwInput}{Input}
\SetKwInput{KwOutput}{Output}
\def\UrlBreaks{\do\A\do\B\do\C\do\D\do\E\do\F\do\G\do\H\do\I\do\J
	\do\K\do\L\do\M\do\N\do\O\do\P\do\Q\do\R\do\S\do\T\do\U\do\V
	\do\W\do\X\do\Y\do\Z\do\[\do\\\do\]\do\^\do\_\do\`\do\a\do\b
	\do\c\do\d\do\e\do\f\do\g\do\h\do\i\do\j\do\k\do\l\do\m\do\n
	\do\o\do\p\do\q\do\r\do\s\do\t\do\u\do\v\do\w\do\x\do\y\do\z
	\do\.\do\@\do\\\do\/\do\!\do\_\do\|\do\;\do\>\do\]\do\)\do\,
	\do\?\do\'\do+\do\=\do\#}

\newcommand{\todored}[1]{\todoc{red}  {[#1]}}
\newcommand{\todoblue}[1]{\todoc{blue}  {[#1]}}
\newcommand{\todoorange}[1]{\todoc{orange}  {[#1]}}
\newcommand{\todocyan}[1]{\todoc{cyan}  {[#1]}}
\newcommand{\todoviolet}[1]{\todoc{violet}  {[#1]}}
\newcommand{\todomagenta}[1]{\todoc{magenta}  {[#1]}}
\newcommand{\todoc}[2]{{\textcolor{#1} {#2}}}
\newcommand{\huaxun}[1]{\todoorange{Huaxun: #1}}
\newcommand{\yepang}[1]{\todoblue{Yepang: #1}}
\newcommand{\scc}[1]{\todoviolet{scc: #1}}
\newcommand{\lili}[1]{\todocyan{Lili: #1}}
\newcommand{\civi}[1]{\todomagenta{Ming: #1}}
\newtheorem{myDef}{Definition}
\newcommand{\hytt}[1]{\texttt{\hyphenchar\font=\defaulthyphenchar #1}}
\renewcommand{\thefootnote}{}
\usepackage{diagbox}

\begin{document}

\title{Characterizing and Detecting Configuration Compatibility Issues in Android Apps\\
{}
\thanks{\textsuperscript{$*$}Shing-Chi Cheung is the corresponding author of this paper.}
}
\author{\IEEEauthorblockN{Huaxun Huang\textsuperscript{$\dagger$}, Ming Wen\textsuperscript{$\S$}, Lili Wei\textsuperscript{$\dagger$}, Yepang Liu\textsuperscript{$\P$}, Shing-Chi Cheung\textsuperscript{$\dagger$*}}
	\IEEEauthorblockA{\textsuperscript{$\dagger$}\textit{Dept. of Computer Science and Engineering, The Hong Kong University of Science and Technology, Hong Kong, China}}
	\IEEEauthorblockA{\textsuperscript{$\S$}\textit{School of Cyber Science and Engineering, Huazhong University of Science and Technology, Wuhan, China}}
	\IEEEauthorblockA{\textsuperscript{$\P$}\textit{Dept. of Computer Science and Engineering, Southern University of Science and Technology, Shenzhen, China}}
	Emails: \{hhuangas@cse.ust.hk, mwenaa@hust.edu.cn, liliwei@cse.ust.hk, liuyp1@sustech.edu.cn, scc@cse.ust.hk\}
	}

\maketitle

\begin{abstract}

XML configuration files are widely used in Android to define an app's user interface and essential runtime information such as system permissions.
As Android evolves, it might introduce functional changes in the configuration environment, thus causing compatibility issues that manifest as inconsistent app behaviors at different API levels.
Such issues can often induce software crashes and inconsistent look-and-feel when running at specific Android versions.
Existing works incur plenty of false positive and false negative issue-detection rules by conducting trivial data-flow analysis while failing to model the XML tree hierarchies of the Android configuration files.
Besides, little is known about how the changes in an Android framework can induce such compatibility issues.
To bridge such gaps, we conducted a systematic study by analyzing 196 real-world issues collected from 43 popular apps.
We identified common patterns of Android framework code changes that induce such configuration compatibility issues.
Based on the findings, we propose \textsc{ConfDroid} that can automatically extract rules for detecting configuration compatibility issues.
The intuition is to perform symbolic execution based on a model learned from the common code change patterns.
Experiment results show that \textsc{ConfDroid} can successfully extract 282 valid issue-detection rules with a precision of 91.9\%.
Among them, 65 extracted rules can manifest issues that cannot be detected by the rules of state-of-the-art baselines.
More importantly, 11 out of them have led to the detection of 107 reproducible configuration compatibility issues that the baselines cannot detect in 30 out of 316 real-world Android apps.
\end{abstract}

\begin{IEEEkeywords}
XML configuration, Android, compatibility, static analysis
% \civi{remember to revise}
\end{IEEEkeywords}

\section{Introduction}  \label{sec:introduction}

\newcommand\inexpIntro[3]{#1?(#2,#3).}
\newcommand\rinexpIntro[3]{*#1?(#2,#3).}
\newcommand\outexpIntro[3]{#1!(#2,#3).}
\newcommand\outatomIntro[3]{#1!(#2,#3)}

We propose a fully automated method for proving termination of \(\pi\)-calculus processes.
Although there have been a lot of studies on termination analysis for the \(\pi\)-calculus
and related calculi~\cite{Deng06IC,Demangeon07,SangiorgiTermination,KobayashiHybrid,Yoshida04IC,DBLP:journals/jlp/DemangeonHS10,Venet98SAS}, most of them have been rather theoretical,
and there have been surprisingly little efforts in developing  fully automated termination
verification methods and tools based on them. To our knowledge,
Kobayashi's \typical{}~\cite{TyPiCal,KobayashiHybrid} is the only exception that
can prove termination of \(\pi\)-calculus processes (extended with natural numbers)
fully automatically, but its termination analysis is quite limited (see Section~\ref{sec:relatedwork}).

Our method is based on a reduction to termination analysis for sequential programs:
we translate a \(\pi\)-calculus process \(P\) to a sequential program \(S_P\), so that
if \(S_P\) is terminating, so is \(P\). The reduction allows us to use
powerful, mature methods and tools
for termination analysis of sequential programs~\cite{heizmann2016ultimate,freqterm,DBLP:conf/lics/PodelskiR04,Kuwahara2014Termination,DBLP:journals/cacm/CookPR11}.

The idea of the translation is to convert a chain of communications on replicated input
channels to a chain of recursive function calls of the target sequential program.
Let us consider the following Fibonacci process:
\begin{align*}
    & \rinexpIntro{\fib}{n}{r}
        \ifexp{n<2}{ \soutatom{r}{1} \\ &\quad}
                   { \nuexp{s_1} \nuexp{s_2} (\outatomIntro{\fib}{n-1}{s_1} \PAR \outatomIntro{\fib}{n-2}{s_2} \PAR \sinexp{s_1}{x}\sinexp{s_2}{y}\soutatom{r}{x+y}) \\}
    & \PAR \outatomIntro{\fib}{m}{r}
\end{align*}
Here, the process
$\rinexpIntro{\fib}{n}{r} \ldots$ is a function server that computes the \(n\)-th Fibonacci number
in parallel and returns the result to \(r\),
and $\outatom{\fib}{m}{r}$ sends a request for computing the \(m\)-th Fibonacci number;
those who are not familiar with the syntax of the \(\pi\)-calculus may wish to consult
Section~\ref{sec:targetlanguage} first.
To prove that the process above is terminating for any integer \(m\),
it suffices to show that there is no infinite chain of communications on $\fib$:
\[
    \fib(m,r) \to \fib(m_1,r_1) \to \fib(m_2,r_2) \to \cdots.
\]
We convert the process above to the following program:\footnote{The actual translation
  given later is a little more complex.}
\begin{verbatim}
 let rec fib(n) = if n<2 then () else (fib(n-1) [] fib(n-2)) in
 fib(m)
\end{verbatim}
Here, \texttt{[]} represents the non-deterministic choice.
Note that, although the calculation of Fibonacci numbers is not preserved,
for each chain of communications on \texttt{fib}, there is a corresponding
sequence of recursive calls:
\[
\mathtt{fib}(m) \to \mathtt{fib}(m_1) \to \mathtt{fib}(m_2) \to \cdots.
\]
Thus, the termination of the sequential program above implies the termination of
the original process.
As shown in the example above, (i) each communication on a replicated input channel
is converted to a function call, (ii) each communication on a non-replicated input
channel is just removed (or, in the actual translation, replaced by a call of
a trivial function defined by \(f(\seq{x})=(\,)\)), and (iii) parallel composition
is replaced by a non-deterministic choice.
We formalize the translation outlined above and prove its correctness.

The basic translation sketched above sometimes loses too much information.
For example, consider the following process:
\begin{align*}
    & \rinexpIntro{\pre}{n}{r} \soutatom{r}{n-1} \\
    & \PAR \rinexpIntro{f}{n}{r} \ifexp{n<0}{ \soutatom{r}{1} }
                                       { \nuexp{s} (\outatomIntro{\pre}{n}{s} \PAR \sinexp{s}{x}\outatomIntro{f}{x}{r}) } \\
    & \PAR \outatomIntro{f}{m}{r}
\end{align*}
The translation sketched above would yield:
\begin{verbatim}
  let pred(n) = n-1 in
  let rec f(n) = if n<0 then () else (pred(n) [] f(*)) in
  f(m)
\end{verbatim}
Here, \texttt{*} represents a non-deterministic integer: since we have removed
the input $\sinatom{s}{x}$, we do not have information about the value of \( x \).
As a result, the sequential program above is non-terminating, although the original
process is terminating.
To remedy this problem, we also refine the basic translation above by using a refinement
type system for the \(\pi\)-calculus. Using the refinement type system,
we can infer that the value of \(x\) in the original process is less than \(n\),
so that we can refine the definition of \texttt{f} to:
\begin{verbatim}
 let rec f(n) = ... else (pred(n) [] let x=* in assume(x<n);f(x))
\end{verbatim}
The target program is now terminating, from which
we can deduce that the original process is also terminating.
We have implemented an automated tool based on the refined translation above.

The contributions of this paper are summarized as follows.
\begin{itemize}
\item The formalization of the basic translation from the \(\pi\)-calculus
  (extended with integers) to sequential programs, and a proof of its correctness.
\item The formalization of a refined translation based on a refinement type system.
\item An implementation of the refined translation, including automated refinement type
  inference based on CHC solving, and experiments to evaluate the effectiveness of
  our method.
\end{itemize}

The rest of this paper is structured as follows.
Section~\ref{sec:targetlanguage} introduces the source and target languages
of our translation.
Section~\ref{sec:approach} 
formalizes the basic translation, and proves its correctness.
Section~\ref{sec:refinement} refines the basic translation by using a refinement type system.
Section~\ref{sec:implementation} reports an implementation and experiments.
Section~\ref{sec:relatedwork} discusses related work,
and Section~\ref{sec:conclusion} concludes the paper.


% Panoptic segmentation

% 3D segmentation

% Multi-object tracking

% Online 3D panoptic:

% PanopticFusion: (IROS 2019)
% https://arxiv.org/pdf/1903.01177.pdf
%
% - most similar to ours
% - PSPNet + M-RCNN + 2D fusion
% - volumetric mapping, 
% - greedy matching with IoU -> optimal only with 0.5 threshold
% - voxel & class weighting
% - CRF regularisation
%
% - good:
%
% - bad:
%  - CRF post-processing step
%  - greedy data-association
%    - can't be tuned for lower overlap ratios -> has to have high framerate, large changes in viewpoint could break this
%    - IoU: sensitive to 2D labels projecting over object borders (CRF and voxel weighting seem to alleviate this)

% Voxblox++: (Robotics & automation letters 2019)
% https://arxiv.org/pdf/1903.00268.pdf
% https://github.com/ethz-asl/voxblox-plusplus
%
% - M-RCNN + geometric segmentation + fusion 
% - data association of geometric segments with 3D overlap (no. points inside volume), fixed threshold for min number of points
% - instance label is assigned to a segment based on highest overlap
% - only one detected segment per reference label, as in PanopticFusion and Ours
% - TSDF Integration 
%
% good: 
% - because of geometric segmentation objects with no associated semantic class can also be segmented
% bad:
% - two different object segment types -> confusing, overly complicated ?
% - quite inaccurate (fixed below)

% Reconstructing Interactive 3D Scenes by Panoptic Mapping and CAD Model Alignments (ICRA 2021)
% https://arxiv.org/pdf/2103.16095.pdf
% https://github.com/hmz-15/Interactive-Scene-Reconstruction
%
% - based heavily on Voxblox++, much more accurate
% - Scene-graph ("contact graph") for mapping object relations
% - Search & replace voxels with CAD models, with geometrical and physical constraints
% - Object 6D pose
% - Format for robot interaction
%
% - Segmentation: bilateral fusion of geomatric and semantic segments -> reduce segmentation noise compared to Voxblox++
% - Fusion: triplet count improves consistency over Voxblox++ pairwise count strategy (take semantic label into account in addition to instance and geometry)
% - Fusion: instance labels are also combined if there is enough overlap with common geometric label for long enough time
%   - this means multiple detections can match the same reference unlike ours, voxblox++ and PanopticFusion ?
%

% Panoptic-MOPE: (ROBOTICS AND AUTOMATION LETTERS 2020)
% https://ieeexplore.ieee.org/stamp/stamp.jsp?tp=&arnumber=8977356
% https://github.com/hoangcuongbk80/Object-RPE/tree/panoptic-mope
%
% - novel RGB-D semantic segmentation model + M-RCNN
% - camera tracking based on "addaptively weighted optimization of geometric, appearance, and semantic cues"
% - surfel map: 
%   - how does it scale ? authors satate they tested on room-sized environments, but could be applied in larger scale as well ...
%     - could maybe be applied as VO in a SLAM algorithm ...
%   - demo only on a small pallet + surroundings, might not be applicable in large-scale SLAM

% US VS THEM:
%
% - based heavily on PanopticFusion, with modifications:
%   - instead of greedy data-association (which seems to be the case in others as well), we solve LAP (JPDA?)
%     - overlap threshold can be tuned, which renders the algorithm more flexible
%     - could be extended to dynamic tracking ?
%   - multiple options for association likelihood
%   - outlier rejection (either clustering or probabilistic)
%   - test different options for decreasing processing time
%   - no post-processing
%
% - model-agnostic:
%   - completely separated from segmentation
%   - does not care how point clouds are obtained -> applicable for LIDAR segmentation (e.g. EfficientLPS) as well
%
% - also agnostic to localisation method
%   - could, however, be utilised to find landmark locations / poses

% More compact version of this paragraph to introduction to save space?
%Panoptic segmentation -- proposed in \cite{panoptic_segmentation} -- aims to solve the unified task of semantic- and instance segmentation. Semantic classes are separated to \textit{stuff} -- amorphous, unquantifiable regions like sky, road or floor -- and \textit{things} -- quantifiable objects. The distinction between the two can vary depending on the application, but a semantic class can only belong to one or another. The article also proposes a unified panoptic evaluation metric, coined \textbf{Panoptic Quality} (PQ). Many 2D approaches to panoptic segmentation -- \textit{e.g.} \cite{panopticfpn,seamless,panoptic_deeplab,efficientps} -- have since been proposed. Deep neural networks for performing semantic- or instance segmentation directly on the 3D reconstruction -- \textit{e.g.} on \cite{scannet,s3dis,paris_lille_3d} -- have also been proposed, but since they require the reconstructed 3D scene, they are mostly offline approaches and therefore out of scope for this work. Some recent works also apply panoptic segmentation to point clouds -- \textit{e.g.} methods in the SemanticKITTI panoptic segmentation competition \cite{semantic_kitti} -- mostly aimed at segmenting LiDAR output. They are suitable for online processing, but similar to RGB-D images require a method for tracking object instances persistent in both time and space. In fact, our proposed method, as well as some others mentioned in this work, could use segmented LiDAR point clouds as an input similarly to RGB-D images.

PanopticFusion \cite{panopticfusion} is the first work to propose online integration of panoptic image segmentations to a 3D reconstruction. They integrate point clouds generated from segmented images to a TSDF voxel volume \cite{tsdf,voxblox} by greedily matching detected segments with the reconstruction and regulating each voxel's corresponding instance with a weighting function. Semantic labels are inferred in a bayesian manner based on confidence scores provided by the segmentation model. They also apply a Conditional Random Field (CRF) to regularise the reconstruction, improving results significantly. Voxblox++ \cite{voxblox++} -- introduced later the same year -- is a similar approach that also integrates segmented RGB-D images into a TSDF volume. It leverages geometric segmentation of depth images to improve instance segmentation accuracy. Both geometric and semantic segments are used to compute a pair-wise weight, which is used to greedily match them with segments in the reconstruction. Because of the geometric segmentation, the method allows segmentation of objects with no known semantic class in addition to objects recognised by the instance segmentation model. 

Recently, \cite{interactive_3d_scenes} built upon the idea of Voxblox++. They apply Voxblox++ for 3D instance integration, with two small but effective modifications: the pair-wise weight is replaced by a triplet weight that also takes semantic labels into account in the fusion, and -- in addition to geometric segments -- instance segments are fused if they overlap by a significant amount. The article introduces a method for searching and aligning CAD models to reconstructed objects based on geometry and semantic class, as well as geometrical and physical rules. With the CAD models, a contact graph and interactive virtual scene are reconstructed to allow a robot to simulate its interaction with the environment. SceneGraphFusion \cite{scenegraphfusion} is another approach that forms a scene graph online from a stream of RGB-D images, but unlike the above-mentioned approach, it generates the graph with a deep neural network, after which the panoptic labels for geometrically segmented portions of the 3D reconstruction are produced a side product.

Panoptic-MOPE \cite{panoptic_mope} is another recent approach, which integrates sequences of RGB-D images into a surfel reconstruction. Unlike other mentioned approaches -- which assume the camera pose either known or estimated elsewhere -- it also tracks camera movements based on geometric-, appearance- and semantic cues. The method also applies a novel RGB-D panoptic segmentation model. Although it is only tested on room-sized environments, the authors claim it could be scaled to larger environments as well.
\section{Empirical Study}
\label{sec:3}
To facilitate automated detection of configuration compatibility issues, we conducted an empirical study on the characteristics and symptoms of such issues in real-world Android apps.
The study aims at answering the following two research questions:
\begin{itemize}
	\item \textbf{RQ1 (Issue types and root causes):} What are the common types and the corresponding root causes of configuration compatibility issues?
	\item \textbf{RQ2 (Issue symptoms):} What are the common symptoms of configuration compatibility issues?
\end{itemize}

\subsection{Dataset Collection}
We collected bug-related code revisions from well-maintained open-source
Android apps as the empirical dataset.
To this end, we searched for suitable subjects on F-Droid
~\cite{fdroid}, which is a famous repository containing high-quality open-source Android apps.
Specifically, we selected subjects that meet the following criteria: (1)
maintaining a public issue tracking system, (2) receiving more than 500 stars on
GitHub~\cite{github} (popularity), and (3) pushing the latest git commit within 
the most recent three months (well-maintenance).
We chose these three criteria because the configuration compatibility issues
located in these selected subjects are likely to affect many users due to the popularity of the apps. 
As a result, 43 apps were returned. 

In order to locate the configuration compatibility issues affecting the 43
selected apps, we used the following two types of keywords to search for
related code revisions:
\sethlcolor{orange}
\begin{itemize}
	\item Keywords related to Android framework versions. In practice, developers often indicate the specific versions of the Android framework in which compatibility issues occur in the changelog.
	Specifically, we used two keywords, \texttt{API} (API level for short), and \texttt{Android [i]} where \texttt{[i]} stands for an integer, to search for Android system versions in changelogs.
	Besides, we also looked for code revisions that contain version-specific XML files, which are stored in the path that contains a version qualifier \texttt{v[L]}, where \texttt{[L]} represents the minimum API level applicable to the files.
	\item Keywords related to XML configuration files in Android apps.
	Specifically, we chose the following two keywords: \texttt{resource}, and \texttt{AndroidManifest}, so that they can effectively cover all types of XML configuration files supported in the Android framework.
\end{itemize}
In total, 2,376 unique code revisions were identified from the 43 apps after removing duplicates from the searching results.

{Next, we conducted manual analysis on the 2,376 code revisions to refine configuration compatibility issues. Specifically, we collected the code revisions in three steps. First, we screened out the code revisions unrelated to valid configuration compatibility issues because some irrelevant code revisions (e.g., introducing new app features) can be accidentally returned by our keyword-based search. Second, we collected the incompatibility-inducing attributes and XML elements from the revision-related commit logs, bug reports, or code diffs. 
To answer RQ1, the code changes related to the incompatibility-inducing attributes and XML elements should also be identified in the update history of the Android framework to investigate how these changes can cause issues. Third, to answer RQ2, we referred to the information of code revisions and online discussions of similar issues for the consequences when developers did not handle problematic XML elements or attributes well. Eventually, we collected 196 configuration compatibility issues from code revisions as the empirical dataset.}

\subsection{RQ1: Issue Root Causes}
\label{sec:RQ1}
\begin{table}[t]
	\caption{Common root causes of configuration compatibility issues}
	\begin{tabular}{lp{5cm}r}
		\toprule
		&\multicolumn{1}{c}{\textbf{Root Causes}}         & \textbf{Issue \#} \\ \hline
		Type 1&Unavailable configuration APIs & 116 (59.2\%)       \\
		Type 2&Inconsistent configuration APIs & 42 (21.4\%)       \\
		Type 3&Inconsistent Android internal XML configuration files & 19 (9.7\%)\\
		Type 4&Inconsistent attribute dependencies    & 9 (4.6\%)        \\
		Type 5&Inconsistent attribute usages             & 9 (4.6\%)    \\
		Type 6&Inconsistent attribute default values         & 1 (0.5\%)    \\
		\bottomrule
		\label{tab:issuecategorization}
	\end{tabular}
\end{table}

We elaborated on the six common types (or causes) identified from the 196 configuration compatibility issues as shown in Table~\ref{tab:issuecategorization}.

\begin{figure}[t]
	\centering
	\includegraphics[width=0.5\textwidth]{./img/layerdrawable.pdf}
	\caption{The Android framework code for loading the attribute value of \texttt{android:gravity} in the class \texttt{LayerDrawable}.}
	\label{fig:layerdrawable}
\end{figure}
\textbf{Unavailable configuration APIs.}
The Android framework loads attribute values by calling configuration APIs after parsing the XML tags in configuration files to \texttt{AttributeSet} or \texttt{TypedArray} objects.
Some statements invoking configuration APIs are introduced or removed as the Android framework evolves, resulting in an inability to load the associated configuration attribute values in a certain range of API levels.
In our empirical dataset, we found 116 (59.2\%) issues that were induced by such a type of code changes.
For example, the attribute value of \texttt{android:gravity} in Figure~\ref{fig:layerdrawable} is loaded by \texttt{LayerDrawable} to adjust the gravity for layer alignment starting from API level 23.
A navigation app OsmAnd~\cite{osmand} filed an issue in commit 1bbf578 that the attribute value of \texttt{android:gravity} is not loaded when running at an API level below 23, causing the incorrect display of graphic user interfaces. %There are 116 (59.2\%) issues falling into this issue type.

\textbf{Inconsistent configuration APIs.}~Configuration APIs in the Android framework are designed to load attribute values in specific data formats.
%The configuration APIs to load an attribute may vary across API levels.
Compatibility issues can happen when the configuration APIs to load an attribute vary across API levels.
The example in Figure~\ref{fig:configurationfile} between API levels 22 and 23 falls into this types.
Such an issue is caused by the style format attribute value of \texttt{android:color} not being loaded by the configuration API \texttt{getAttributeIntValue()} at API level 22.
The loading of \texttt{android:color} in an unsupported format can result in app crashes at API level 22. 
There are in total 42 (21.4\%) issues of this type.

\textbf{Inconsistent Android internal XML configuration files.}
The Android framework provides a set of internal XML configuration files that can be referenced by the developers as a part of their apps.
Compatibility issues can happen when there are changes in those internal XML configuration files as the Android framework evolves.
For example, QKSMS~\cite{qksms} commit 6b70a47 describes an issue caused by the internal XML configuration file \texttt{ic\_menu\_added.xml}, which was introduced at API level 23. There are 19 (9.7\%) issues falling into this issue type.

\begin{figure}[t]
	\centering
	\includegraphics[width=0.5\textwidth]{./img/datepicker.pdf}
	\caption{Code changes of loading \texttt{android:spinnersShown} in the class \texttt{DatePicker}.}
	\label{fig:datepicker}
\end{figure}
\textbf{Inconsistent attribute dependencies.}~
In the Android framework, there are dependencies across configuration attributes.
In other words, the runtime behaviors of one attribute depend on the value of other attributes.
We found nine compatibility issues that were induced by the inconsistent implementations on attribute dependencies among API levels. %of triggering conditions of invoking configuration APIs in the Android framework.
%Such changes can cause an attribute not to be loaded at specific API levels.
For example, open-keychain~\cite{openkeychain} reported an issue in commit be06c4c.
As Figure~\ref{fig:datepicker}(a) shows, developers have specified the value of \texttt{android:spinnersShown} as \texttt{true} to make the date picker widget be displayed in the spinner mode.
The attribute value cannot be loaded at API level 21 without specifying the value of \texttt{android:datePickerMode} as \texttt{spinner}, causing the date picker to be displayed in the calendar mode by default.
In this case, the app crashed at API level 21 due to a specific implementation of the date picker in the calendar mode.
The code changes in Figure~\ref{fig:datepicker}(b) show that starting from API level 21, the configuration API for \texttt{android:spinnersShown} is guarded by the conditional statement that checks whether the value of \texttt{android:datePickerMode} is \texttt{spinner}.
%Note that the existing path-insensitive analysis technique can fail to analyze the code changes falling into this type although the proportion of issues caused by such a pattern are small. 
%As the example in Figure~\ref{fig:frameworkprocess} shows, existing techniques cannot infer that \texttt{android:color} will be loaded in Line 5 without analyzing the conditional statement in Line 4. 
%Such a case can make the existing techniques derive spurious issue-inducing code changes as \texttt{android:color} is loaded in \texttt{ColorStateList} starting from API level 23.

\textbf{Inconsistent attribute usages.}
Compatibility issues can happen when there are inconsistent implementations on how the Android framework uses the attribute values after being loaded by configuration APIs.
As Figure~\ref{fig:frameworkprocess} shows, there is a change in processing the value of \texttt{android:color} (in Line 12 and 13) between API levels 21 and 22. The change avoids the \texttt{ArrayIndexOutOfBoundsException} when the Android framework parses the XML element in Figure~\ref{fig:configurationfile} at API level 22.
In total, there are nine (4.6\%) issues falling into this issue type.

\textbf{Inconsistent attribute default values.}
There is one (0.5\%) issue caused by inconsistent default values of the attribute \texttt{android:useLevel} in the XML tag \texttt{<shape>} between API levels 21 and 22, as reported in the commit a221442 of OsmAnd~\cite{osmand}.

\subsection{RQ2: Issue Symptoms}
We further analyzed the common issue symptoms as below.
Specifically, 89 (45.4\%) of the 196 issues in our empirical dataset can cause the apps to crash when triggering incompatibility-inducing XML configuration elements, as shown by the motivating example in Figure~\ref{fig:frameworkprocess}.
Another 88 issues (44.9\%) can induce an inconsistent look-and-feel across different API levels, affecting the apps' functionalities.
For example, the problem reported in the commit a221442 of OsmAnd~\cite{osmand} can force the progress bar to always show a full circle.
The remaining 19 issues (9.7\%) can cause inconsistent app behaviors beyond crashes and look-and-feel.
For example, the app Slide~\cite{slide} specified \texttt{android:requestLegacyExternalStorage} to make sure the app can still request for the external storage at an API level $\geq$ 29.
This shows that configuration compatibility issues can cause severe consequences to the app developers.

\section{The Technique}

Our approach to Alloy specification repair involves a series of tasks, for fault detection, fault localization, fix candidate generation, and fix candidate assessment. We describe these in more detail below. 

\subsection{Fault Detection and Fix Acceptance Criterion}

In general, given an Alloy specification, we may say that such specification is \emph{faulty} if at least one of the analysis commands in the specification has an outcome contrary to its corresponding expectation. This can be either a failing assertion (assertion with counterexamples), or a predicate that is unsatisfiable while the user expected it to be satisfiable, or vice versa. We may also allow for other flavors in commands, in particular Alloy test cases, in the spirit of AUnit \cite{Sullivan+2018}. The fault detection stage then resorts to SAT solving, the underlying analysis mechanism behind Alloy Analyzer, the tool for Alloy specification analysis \cite{Jackson2006}. Similarly, a fix candidate can be considered an acceptable patch when all the analysis commands in the specification have an outcome that coincides with the corresponding command's expectations. 

Our technique requires the user to identify the specification oracle, i.e., the assertions, predicates or tests that the technique will have to consider as fix acceptance criterion. The technique will then identify faults in the remainder of the specification (the oracle is left out of the analysis space for fault localization), and generate fix candidates for the faulty locations. Therefore, our repair approach cannot fix any faulty situation, but only those where the developer is certain about some part of it (the oracle), and wishes to alter the remainder of the specification to pass it. Looking for solutions that may modify the specification \emph{and} the criterion for acceptance would lead to fixes that may simply relax the acceptance criterion. Notice that, in this respect, we follow the same approach that ARepair and most test-based program repair techniques: the tests (the repair oracle) cannot be changed in the repair process. As described later on in this section, other trivial solutions such as changing a command's expectations or simply removing a command are prevented, due to how the fault localization is performed (which cannot be blamed on commands) and how fix candidates are generated (only by mutating the faulty locations). 

\subsection{Fault Localization}

Once a specification is deemed faulty, we need to identify the specific parts of the specification that are more likely to be blamed for the fault or faults. We do not deal with fault localization in this paper, and we assume an external technique/tool provides fault localization information. There exist techniques for fault localization that specifically target Alloy specifications, such as the spectrum-based fault localization mechanism behind ARepair \cite{Wang+2018}, and our fault localization technique presented in \cite{Zheng+2021}. While in principle any fault localization technique would fit our technique, as long as the employed fault localization can handle the oracles present in the faulty specification, it is worth to remark that the fault localization within ARepair inherently depends on having tests as oracles (acceptance criteria) for specifications \cite{Wang+2018}. Moreover, the fault localization in ARepair can dynamically change the identified faulty locations, as the specification is transformed during the repair process. Our technique, on the other hand, uses an \emph{offline} process for fault localization: the faulty program is fed to the fault localization tool, and a number of suspicious specification locations are returned. This is the input to our specification repair approach, and the space of \emph{all} possible patches for these locations, for a maximum depth in mutation application and a given set of mutation operators, will be considered. 

For our experiments in Section~IV, we use the FLACK fault localization technique \cite{Zheng+2021}. While we do not describe in detail the fault localization technique in this paper (we refer the reader to \cite{Zheng+2021}), let us remark a number of facts about FLACK: it supports arbitrary satisfiability checks and assertions, as well as tests, as specification oracles; it is based on the use of (partial) maximum satisfiability procedures, to process counterexamples of an Alloy model (witnessing the faulty status of the specification); and it can only identify faults within formulas and relational expressions, it cannot locate faults in data definitions, such as signature and field declarations, nor in commands (Alloy's runs and checks). 

\subsection{Generation of Fix Candidates}

Once the suspicious expressions are identified, syntactical variants of these expressions are produced. We consider an ample set of mutation operations, including the obvious logical and relational operator insertion, removal and replacement, quantification mutation (e.g., changing a quantifier), multiplicity constraint replacement, field/variable swap/replacement, etc., based on Alloy's grammar. Our tool processes the specification to obtain some typing information, so that some legal expressions that necessarily lead to empty relation/contradictory formulas are disregarded, as well as innocuous operation application (e.g., double transitive closure). Two elements are important to highlight here, namely the use of join to produce navigation chains, using fields, signatures, etc., and the possibility of \emph{combining} mutations, i.e., applying further mutations to an already mutated expression, akin the so called higher-order mutants \cite{DBLP:journals/infsof/JiaH09} in mutation testing.

Both the mutation operators and the maximum depth, i.e., the number of cumulative mutations (hence, the higher order nature of the generated mutants) that can be applied to a given faulty location, are configurable. These are bounded-exhaustively generated as the space of fix candidates is traversed (see below). In our experiments, we used 21 mutation operators in total, typically leading to roughly between 60 and 260 1-level mutants per location.  

\subsection{Fix Candidate Space Traversal}

Here we present our general repair approach. The two pruning techniques just introduced, are also described in more detail, and we argue about their soundness. The search space is organized as a search tree in a traditional search problem: the root is the original specification, with its faulty locations identified; and if a specification $s$ is in the tree and $s'$ can be obtained by applying a mutation to a faulty location, then $s'$ is also in the tree, with the same locations marked as faulty (so that the mutation process can be iterated). This in principle leads to an infinite fix candidate space, which we explore up to a maximum depth. While any search strategy may be applied, we explore the state space in a breadth-first fashion. 

\subsubsection{Partial repair checking}

Our first pruning strategy consists of identifying one of the suspicious locations for which a current repair candidate fails, as established by an analysis check that does not depend on the remainder of the faulty locations. We will describe it in more detail, assuming two faulty locations, without loss of generality. Let $\textit{Spec}$ be an Alloy specification, $\textit{Check}_1, \dots, \textit{Check}_k$ its analysis checks used as oracles, and $L_0, L_1$ the suspicious locations identified by the fault localization phase. Each analysis check $\textit{Check}_i$ refers to a specific part of $\textit{Spec}$, which can be determined by a straightforward syntactic analysis: $\textit{Check}_i$ refers to the formula it directly mentions (the body of the corresponding predicate or assertion), all the facts (axioms of the specification that are implicitly involved in every analysis check), and the symbols directly and indirectly referred syntactically to by these (predicates called, relations used, etc.). This syntactic analysis can determine then, for every $\textit{Check}_i$, which of the suspicious locations $L_0$ and $L_1$ it involves.

Most logics, and certainly Alloy's relational logic, have a sort of syntactic locality property, that guarantees that the validity/satisfiability of a formula depends only on the symbols it refers to. (In the case of Alloy, since validity/satisfiability is actually bounded validity/satisfiability, it can also depend on the scope, the bound, of analysis; but since the bound of analysis cannot be modified in the patch generation phase, we can disregard it). Moreover, the logic is monotonic, meaning that adding more assumptions to a formula can never reduce the conclusions drawn originally from it. These properties allow us to make the following observation. Let $m_0$ and $n_0$ constitute the modifications to locations $L_0$ and $L_1$, respectively, in the current fix candidate (i.e., be the expressions substituting the original expressions in locations $L_0$ and $L_1$ of $\textit{Spec}$). If a failing satisfiability check $\textit{Check}_i$ refers to only one of the suspicious locations, say $L_0$ and its current expression $m_0$, this means that the formula in $\textit{Check}_i$ is determined to be false independently of $n_0$. Then, for every alternative expression $n_i$ for location $L_1$, the corresponding fix candidate $(m_0, n_i)$ (the replacement expressions for locations $L_0$ and $L_1$) will still make $\textit{Check}_i$ to be false, due to the monotonicity of the logic. In other words, the specification cannot be repaired by modifying location $L_1$ if the current fix for location $L_0$ is maintained. We can therefore exclude (prune from the checking) all $(m_0, n_i)$ fix candidates as soon as we determine this situation, which in turn can be determined by a syntactic analysis of the specification, and the analysis outcome for fix candidate $(m_0, n_0)$. 

We refer to this analysis and the corresponding pruning it enables as \emph{partial repair checking}, due to the partiality of fix candidates when these do not involve all suspicious locations.

\subsubsection{Variabilization}

Our second pruning strategy is called \emph{variabilization}, due to the mechanism employed for prune checking, that requires introducing fresh variables to refer to fix candidates to specific locations, in a general way.

Let $\textit{Check}_i$ be a failing assertion (validity) check that refers to suspicious locations $L_0$ and $L_1$, and let $(m_0, n_0)$ be the current failing fix candidate. Notice that since $\textit{Check}_i$ is a failing validity check, we have a counterexample $\textit{CEX}_i$ as a result of the violation. That is, we have that:
\begin{displaymath}
    \textit{CEX}_i \not \models \textit{Spec}[m_0, n_0] \Rightarrow \textit{Check}_i,
\end{displaymath}
where $\textit{Spec}[m_0, n_0]$ denotes the fix candidate obtained by replacing $L_0$ and $L_1$ by $m_0$ and $n_0$, respectively, in $\textit{Spec}$. The purpose of variabilization is to check whether the current fix for $L_0$, i.e., $m_0$, \emph{may} work with \emph{some} candidate for $L_1$ (other than $n_0$, of course, which we already know it does not work). For technical reasons, we actually check whether some fix for $L_1$ may work in combination with $m_0$, for counterexample $\texttt{CEX}_i$. Let us describe the process for performing this check. 

Notice that fault locations can be subexpressions of a formula; let us refer by $F_1$ to the formula (predicate, fact, etc) containing $L_1$. Also, let $T$ be the most general type for $L_1$ in context $F_1$ (in Alloy, this most general type will depend on the arity required by $L_1$ in $F_1$, the context in which $L_1$ may depend upon, and will use the most general unary type, the universe \texttt{univ}). Let $\textit{Spec}_{L_1}$ be the specification obtained by replacing $F_1$ in $\textit{Spec}$ by 
\begin{displaymath}
\exists l_1: T \vert F_1[l_1/L_1] 
\end{displaymath}
i.e., we substitute $L_1$ by an existentially quantified variable of type $T$ (hence the name \emph{variabilization}). We now can check: 
\begin{displaymath}
    \textit{CEX}_i \models \textit{Spec}_{L_0}[m_0] \Rightarrow \textit{Check}_i,
\end{displaymath}
i.e., whether there exists a \emph{value} of type $T$ that can be put in place of location $L_1$, so that $\textit{CEX}_i$ ceases to be a counterexample. If this is the case, then local fix $m_0$ works as a fix for $L_0$, at least as far as $\textit{CEX}_i$ is concerned, and we may traverse the space of local candidates for $L_1$ to attempt to find a complete fix. But, on the other hand, if the above check fails, then there is no value that can be put in place of $L_1$ such that the local fix $m_0$ would work ($\textit{CEX}_i$ would still be a counterexample). Therefore, we can again exclude (prune from the checking) all $(m_0, n_i)$ fix candidates if the check fails. 

One may argue why not check the ``variabilized'' specification in the general case, instead of doing so \emph{only} for counterexample $\textit{CEX}_i$. The reason has to do with the type $T$ of location $L_1$. When this type is a relation of an arity greater than one, variabilization leads to a higher-order quantification, that Alloy cannot handle as a general analysis check, but it can do so for the specific counterexample $\textit{CEX}_i$. 

To clarify this variabilization process, and especially the reason why we typically have higher-order quantification, let us consider the example introduced in Section~II, where one local fix candidate is applied and the other was generalized with question marks. Assuming that assertion \texttt{ContainsCorrect} failed, a counterexample $\textit{CEX}$ was generated from this fix. To check whether variabilization pruning can be applied, we turn the question marks into existential quantifications. Intuitively, the corresponding variabilized specification would then be as follows (we are abusing the notation below, using Boolean for the type of the variabilized formula within \texttt{Sorted}):

%{\small
%\begin{verbatim}
\begin{lstlisting} []
pred Sorted[This: List] {
 some b: Boolean | all n: This.header.*link | b
}

pred Contains[This: List, x: Int, res: Boolean]{
 RepOk[This]
 (x !in This.header.*link.elem => res=False ) 
 && res = True
}
\end{lstlisting}
%{\end{verbatim}
%} }

\noindent
However, we need to take into account that the variabilization context, the place where the location being variabilized occurs, depends in this case both on \texttt{This} and \texttt{n}. Thus, the actual variabilization for the check is as follows (we are again abusing the notation for the sake of clarity):

%{\small
%\begin{verbatim}
\begin{lstlisting} []
pred Sorted[This: List] {
 some b: List -> Node -> Boolean | 
     all n: This.header.*link | b[This, n]
}

pred Contains[This: List, x: Int, res: Boolean]{
 RepOk[This]
 (x !in This.header.*link.elem => res=False ) 
 && res = True
 \end{lstlisting}
%\end{verbatim}
%} 

\noindent
We cannot check \texttt{ContainsCorrect} over this specification due to the higher-order quantification in \texttt{Sorted}; but we can check it for $\textit{CEX}$. 

It is worth remarking that the above check, if successful, will produce a \emph{relational value} for \texttt{b} that makes the variabilized specification work. It will \emph{not} produce an expression to put in place of the body of the quantification, as a local fix candidate. It would not even produce a relational value that can be ``hardwired'' as a local fix of the corresponding location, since it is in principle just a relational value that works for counterexample $\textit{CEX}$. But its existence is what enables us to decide that a local fix for $L_1$ (\texttt{Sorted}) may be possible, considering the current local fix for $L_0$ (the \texttt{\&\&} in \texttt{Contains}). Our check essentially corresponds to only checking for feasibility of a local fix with respect to other locations. 

An alternative to the above would be to attempt to turn the relational value bound to $b$ into a relational \emph{expression}, that can be considered a local fix candidate. Such a process would correspond to a \emph{synthesis procedure}, which would require a grammar for expressions, so that the solver can attempt to work out an instance (an actual expression) during the satisfiability process. While it is technically feasible, it is also significantly more costly than our simpler query for satisfiability, which we solely use for pruning. 

%Such queries are infeasible in the context of programs, but are realisable in specifications, especially in the context of Alloy and its automated analysis. It can actually be ``implemented'' in different ways. One approach, which is in a sense followed by ARepair \cite{XXX}, is to exploit a \emph{synthesis} mechanism, as \emph{sketching}. That is, not only query for satisfiability, but do so in a way that the satisfying valuation can be turned into a fix. The cost is however greater than solely querying for satisfiability, as one needs find a single expression that would make \emph{all} oracles to pass at the same time, and its feasibility needs to consider a limited set of expressions, to make it solvable by the underlying SAT solver. Unsatisfiability in this case means there is no expression, for the a given (limited) grammar, that would fix the specification. Thus, (un)satisfiability is relative both to the scope (as usual in the context of Alloy) and to the grammar used for candidate expressions. 

%When checking the oracles on this specification, satisfiability will \emph{not} produce an expression to put in place of the body of the quantification, but a \emph{relation}, i.e., an element of the semantics of the specification (not of Alloy's syntax), so we cannot use it, at least not directly, as a fix, but only as a generic check for repair candidate feasibility. 

%As we show in our evaluation section, these checks can have a significant impact in pruning, and lead to a technique for program repair that is complementary to that introduced by ARepair.

%checks that serve the purpose of checking whether a fix candidate is a feasible repair for the corresponding location. As no grammar for the expressions is involved, this check is bounded complete. We call our approach \emph{variabilisation}, as it is based on replacing the buggy location by a variable of the corresponding type. To make sound pruning, we need to make our variabilisations \emph{context sentitive}. That is, instead of turning the above specification into (we are abusing the notation below):  


%!TEX root = main.tex
\section{Evaluation}
\label{sec:eval}

In this section, we evaluate the performance of our unsupervised Ordered Word Mover's Distance metric and supervised Multi-scale Sentence Matching model with factorized sentences as input. We apply our algorithms to semantic textual similarity estimation tasks and sentence pair paraphrase identification tasks, based on four datasets: STSbenchmark, SICK, MSRP and MSRvid. 

\subsection{Experimental Setup}
\label{subsec:setup}


\begin{table}[tb]
  \caption{Description of evaluation datasets.}
  \label{tab:datasets}
  \begin{tabular}{lllll}
    \toprule
    Dataset & Task & Train & Dev & Test\\
    \midrule
    STSbenchmark & Similarity scoring & $5748$ & $1500$ & $1378$ \\
    SICK & Similarity scoring & $4500$ & $500$ & $4927$ \\
    MSRP & Paraphrase identification & $4076$ & - & $1725$ \\
    MSRvid & Similarity scoring & $750$ & - & $750$ \\
    \bottomrule
  \end{tabular}
  \vspace{-2mm}
\end{table}

We will start with a brief description for each dataset:
\begin{itemize}
\item \textbf{STSbenchmark}\cite{cer2017semeval}: it is a dataset for semantic textual similarity (STS) estimation. The task is to assign a similarity score to each sentence pair on a scale of 0.0 to 5.0, with 5.0 being the most similar.

\item \textbf{SICK}\cite{marelli2014sick}: it is another STS dataset from the SemEval 2014 task 1. It has the same scoring mechanism as STSbenchmark, where 0.0 represents the least amount of relatedness and 5.0 represents the most.

\item \textbf{MSRvid}: the Microsoft Research Video Description Corpus contains 1500 sentences that are concise summaries on the content of a short video. Each pair of sentences is also assigned a semantic similarity score between 0.0 and 5.0. 

\item \textbf{MSRP}\cite{quirk2004monolingual}: the Microsoft Research Paraphrase Corpus is a set of 5800 sentence pairs collected from news articles on the Internet. Each sentence pair is labeled 0 or 1, with 1 indicating that the two sentences are paraphrases of each other.
\end{itemize}

Table \ref{tab:datasets} shows a detailed breakdown of the datasets used in evaluation.
For STSbenchmark dataset we use the provided train/dev/test split.
The SICK dataset does not provide development set out of the box, so we extracted 500 instances from the training set as the development set.
For MSRP and MSRvid, since their sizes are relatively small to begin with, we did not create any development set for them.

One metric we used to evaluate the performance of our proposed models on the task of semantic textual similarity estimation is the Pearson Correlation coefficient, commonly denoted by $r$. Pearson Correlation is defined as:
\begin{equation}
\label{eq:pearson}
 r = cov(X,Y) /( \sigma_X \sigma_Y),
\end{equation}
where $cov(X,Y)$ is the co-variance between distributions X and Y, and $\sigma_X$, $\sigma_Y$ are the standard deviations of X and Y.
The Pearson Correlation coefficient can be thought as a measure of how well two distributions fit on a straight line. Its value has range [-1, 1], where a value of 1 indicates that data points from two distribution lie on the same line with a positive slope.
% Due to this unique property, we believe the Pearson Correlation coefficient is a strong indicator of the performance of our metric. 

Another metric we utilized is the Spearman's Rank Correlation coefficient. Commonly denoted by $r_s$, the Spearman's Rank Correlation coefficient shares a similar mathematical expression with the Pearson Correlation coefficient, but it is applied to ranked variables.
Formally it is defined as \cite{wiki:spearman}:
\begin{equation}
\label{eq:spearman}
 \rho = cov(rg_X, rg_Y) / (\sigma_{rg_X} \sigma_{rg_Y}),
\end{equation}
where $rg_X$, $rg_Y$ denotes the ranked variables derived from $X$ and $Y$. $cov(rg_X,rg_Y)$, $\sigma_{rg_X}$, $\sigma_{rg_Y}$ corresponds to the co-variance and standard deviations of the rank variables. The term ranked simply means that each instance in X is ranked higher or lower against every other instances in X and the same for Y. We then compare the rank values of X and Y with \ref{eq:spearman}. Like the Pearson Correlation coefficient, the Spearman's Rank Correlation coefficient has an output range of [-1, 1], and it measures the monotonic relationship between X and Y. A Spearman's Rank Correlation value of 1 implies that as X increases, Y is guaranteed to increase as well.
The Spearman's Rank Correlation is also less sensitive to noise created by outliers compared to the Pearson Correlation.

For the task of paraphrase identification, the classification accuracy of label $1$ and the F1 score are used as metrics. 

In the supervised learning portion, we conduct the experiments on the aforementioned four datasets. We use training sets to train the models, development set to tune the hyper-parameters and each test set is only used once in the final evaluation. For datasets without any development set, we will use cross-validation in the training process to prevent overfitting, that is, use $10\%$ of the training data for validation and the rest is used in training. For each model, we carry out training for 10 epochs. We then choose the model with the best validation performance to be evaluated on the test set.  


\subsection{Unsupervised Matching with OWMD}
\label{subsec:eval-owmd}

To evaluate the effectiveness of our Ordered Word Mover's Distance metric, we first take an unsupervised approach towards the similarity estimation task on the STSbenchmark, SICK and MSRvid datasets. Using the distance metrics listed in Table \ref{tab:compare-pearson} and \ref{tab:compare-spearman}, we first computed the distance between two sentences, then calculated the Pearson Correlation coefficients and the Spearman's Rank Correlation coefficients between all pair's distances and their labeled scores. We did not use the MSRP dataset since it is a binary classification problem.


In our proposed Ordered Word Mover's Distance metric, distance between two sentences is calculated using the order preserving Word Mover's Distance algorithm. For all three datasets, we performed hyper-parameter tuning using the training set and calculated the Pearson Correlation coefficients on the test and development set. We found that for the STSbenchmark dataset, setting $\lambda_1=10$, $\lambda_2=0.03$ produces the most optimal result. For the SICK dataset, a combination of $\lambda_1=3.5$, $\lambda_2=0.015$ works best. And for the MSRvid dataset, the highest Pearson Correlation is attained when $\lambda_1=0.01$, $\lambda_2=0.02$.
We maintain a max iteration of 20 since in our experiments we found that it is sufficient for the correlation result to converge.
During hyper-parameter tuning we discovered that using the Euclidean metric along with $\sigma=10$ produces better results, so all OWMD results summarized in Table \ref{tab:compare-pearson} and \ref{tab:compare-spearman} are acquired under these parameter settings. Finally, it is worth mentioning that our OWMD metric calculates the distances using factorized versions of sentences, while all other metrics use the original sentences. Sentence factorization is a necessary preprocessing step for the OWMD metric.


We compared the performance of Ordered Word Mover's Distance metric with the following methods:

\begin{itemize}
\item \textbf{Bag-of-Words (BoW)}: in the Bag-of-Words metric, distance between two sentences is computed as the cosine similarity between the word counts of the sentences.

\item \textbf{LexVec}~\cite{salle2016enhancing}: calculate the cosine similarity between the  averaged 300-dimensional LexVec word embedding of the two sentences. 

\item \textbf{GloVe}~\cite{pennington2014glove}: calculate the cosine similarity between the averaged 300-dimensional GloVe 6B word embedding of the two sentences. 

\item \textbf{Fastext}~\cite{joulin2016bag}: calculate the cosine similarity between the averaged 300-dimensional Fastext word embedding of the two sentences. 

\item \textbf{Word2vec}~\cite{mikolov2013efficient}: calculate the cosine similarity between the averaged 300-dimensional Word2vec word embedding of the two sentences.

\item \textbf{Word Mover's Distance (WMD)}~\cite{kusner2015word}: estimating the semantic distance between two sentences by WMD introduced in Sec.~\ref{sec:owmd}.
\end{itemize} 


\begin{table}[tb]
  \caption{Pearson Correlation results on different distance metrics.}
  \label{tab:compare-pearson}
  \begin{tabular}{c|cc|cc|c}
    \toprule
    \multirow{2}{*}{Algorithm} & \multicolumn{2}{c}{STSbenchmark} & \multicolumn{2}{c}{SICK} & MSRvid\\ 
     & Test & Dev & Test & Dev & Test\\
    \midrule
    BoW & $0.5705$ & $0.6561$ & $0.6114$ & $0.6087$ & $0.5044$ \\
    LexVec & $0.5759$ & $0.6852$ & $0.6948$ & $\mathbf{0.6811}$ & $0.7318$\\
    GloVe & $0.4064$ & $0.5207$ & $0.6297$ & $0.5892$  & $0.5481$ \\
    Fastext & $0.5079$ & $0.6247$ & $0.6517$ & $0.6421$  & $0.5517$  \\
    Word2vec & $0.5550$ & $0.6911$ & $\mathbf{0.7021}$ & $0.6730$  & $0.7209$  \\
    WMD & $0.4241$ & $0.5679$ & $0.5962$ & $0.5953$  & $0.3430$  \\
    OWMD & $\mathbf{0.6144}$ & $\mathbf{0.7240}$ & $0.6797$ & $0.6772$  & $\mathbf{0.7519}$  \\
    \bottomrule
  \end{tabular}
  \vspace{-4mm}
\end{table}

\begin{table}[tb]
  \caption{Spearman's Rank Correlation results on different distance metrics.}
  \label{tab:compare-spearman}
  \begin{tabular}{c|cc|cc|c}
    \toprule
    \multirow{2}{*}{Algorithm} & \multicolumn{2}{c}{STSbenchmark} & \multicolumn{2}{c}{SICK} & MSRvid\\ 
     & Test & Dev & Test & Dev & Test\\
    \midrule
    BoW & $0.5592$ & $0.6572$ & $0.5727$ & $0.5894$ & $0.5233$ \\
    LexVec & $0.5472$ & $0.7032$ & $0.5872$ & $0.5879$ & $0.7311$\\
    GloVe & $0.4268$ & $0.5862$ & $0.5505$ & $0.5490$  & $0.5828$ \\
    Fastext & $0.4874$ & $0.6424$ & $0.5739$ & $0.5941$  & $0.5634$  \\
    Word2vec & $0.5184$ & $0.7021$ & $0.6082$ & $0.6056$  & $0.7175$  \\
    WMD & $0.4270$ & $0.5781$ & $0.5488$ & $0.5612$  & $0.3699$  \\
    OWMD & $\mathbf{0.5855}$ & $\mathbf{0.7253}$ & $\mathbf{0.6133}$ & $\mathbf{0.6188}$  & $\mathbf{0.7543}$  \\
    \bottomrule
  \end{tabular}
  \vspace{-2mm}
\end{table}


Table \ref{tab:compare-pearson} and Table \ref{tab:compare-spearman} compare the performance of different metrics in terms of the Pearson Correlation coefficients and the Spearman's Rank Correlation coefficients.
We can see that the result of our OWMD metric achieves the best performance on all the datasets in terms of the Spearman's Rank Correlation coefficients.
It also produced the best Pearson Correlation results on the STSbenchmark and the MSRvid dataset, while the performance on SICK datasets are close to the best.
This can be attributed to the two characteristics of OWMD. First, the input sentence is re-organized into a predicate-argument structure using the sentence factorization tree. Therefore, corresponding semantic units in the two sentences will be aligned roughly in order. Second, our OWMD metric takes word positions into consideration and penalizes disordered matches. Therefore, it will produce less mismatches compared with the WMD metric.

% On the SICK dataset, although the result of our metric falls slightly behind Word2vec, LexVec on the test set and Word2vec on the development set, we still believe that it is a superior metric because it produced competitive results across multiple datasets. 

% Table \ref{tab:compare-spearman} presents the Spearman's Rank Correlation coefficients acquired with the same distance metrics. We can observe that our OWMD metric achieves the highest correlation scores on all three datasets. Which proves once again that OWMD is a better distance metric for the task of semantic similarity detection.

\subsection{Supervised Multi-scale Semantic Matching}
\label{subsec:eval-multilayer}

\begin{table*}[tb]
  \caption{A comparison among different supervised learning models in terms of accuracy, F1 score, Pearson's $r$ and Spearman's $\rho$ on various test sets.}
  \label{tab:sts}
  \begin{tabular}{c|cc|cc|cc|cc}
    \toprule
    \multirow{2}{*}{Model} & \multicolumn{2}{c}{MSRP} & \multicolumn{2}{c}{SICK} & \multicolumn{2}{c}{MSRvid} & \multicolumn{2}{c}{STSbenchmark}\\ 
     & Acc.(\%) & F1(\%) & $r$ & $\rho$ & $r$ & $\rho$ & $r$ & $\rho$ \\
    \midrule
    MaLSTM & $66.95$ & $73.95$ & $0.7824$ & $0.71843$ & $0.7325$ & $0.7193$ & $0.5739$ & $0.5558$\\
    Multi-scale MaLSTM & $\mathbf{74.09}$ & $\mathbf{82.18}$ & $\mathbf{0.8168}$ & $\mathbf{0.74226}$ & $\mathbf{0.8236}$ & $\mathbf{0.8188}$ & $\mathbf{0.6839}$ & $\mathbf{0.6575}$\\
    \midrule
    HCTI & $73.80$ & $80.85$ & $0.8408$ & $0.7698$ & $\mathbf{0.8848}$ & $\mathbf{0.8763}$  & $\mathbf{0.7697}$ & $\mathbf{0.7549}$ \\
    Multi-scale HCTI & $\mathbf{74.03}$ & $\mathbf{81.76}$ & $\mathbf{0.8437}$ & $\mathbf{0.7729}$ & $0.8763$ & $0.8686$  & $0.7269$ & $0.7033$  \\
    \bottomrule
  \end{tabular}
  \vspace{-2mm}
\end{table*}

The use of sentence factorization can improve both existing unsupervised metrics and existing supervised models. 
% We extend the normal Siamese model to Fig. \ref{fig:network} to take advantage of different level of information in the factorized sentence. 
To evaluate how the performance of existing Siamese neural networks can be improved by our sentence factorization technique and the multi-scale Siamese architecture, we implemented two types of Siamese sentence matching models, HCTI \cite{mueller2016siamese} and MaLSTM \cite{shao2017hcti}. HCTI is a Convolutional Neural Network (CNN) based Siamese model, which achieves the best Pearson Correlation coefficient on STSbenchmark dataset in SemEval2017 competition (compared with all the other neural network models). MaLSTM is a Siamese adaptation of the Long Short-Term Memory (LSTM) network for learning sentence similarity. As the source code of HCTI is not released in public, we implemented it according to \cite{shao2017hcti} by Keras \cite{chollet2015keras}. With the same parameter settings listed in paper \cite{shao2017hcti} and tried our best to optimize the model, we got a Pearson correlation of 0.7697 (0.7833 in paper \cite{shao2017hcti}) in STSbencmark test dataset.

We extended HCTI and MaLSTM to our proposed Siamese architecture in Fig. \ref{fig:network}, namely the Multi-scale MaLSTM and the Multi-scale HCTI. To evaluate the performance of our models, the experiment is conducted on two tasks: 1) semantic textual similarity estimation based on the STSbenchmark, MSRvid, and SICK2014 datasets; 2) paraphrase identification based on the MSRP dataset.

Table \ref{tab:sts} shows the results of HCTI, MaLSTM and our multi-scale models on different datasets. Compared with the original models, our models with multi-scale semantic units of the input sentences as network inputs significantly improved the performance on most datasets. 
Furthermore, the improvements on different tasks and datasets also proved the general applicability of our proposed architecture.

Compared with MaLSTM, our multi-scaled Siamese models with factorized sentences as input perform much better on each dataset. For MSRvid and STSbenmark dataset, both Pearson's $r$ and Spearman's $\rho$ increase about $10\%$ with Multi-scale MaLSTM. Moreover, the Multi-scale MaLSTM achieves the highest accuracy and F1 score on the MSRP dataset compared with other models listed in Table \ref{tab:sts}.

There are two reasons why our Multi-scale MaLSTM significantly outperforms MaLSTM model. First, for an input sentence pair, 
we explicitly model their semantic units with the factorization algorithm.
%we explicitly model the different scales of semantics of them with the semantic units produced by our sentence factorization algorithm. 
Second, our multi-scaled network architecture is 
specifically designed
%specially adapted to 
for multi-scaled sentences representations. Therefore, it is able to explicitly match a pair of sentences at different granularities.

We also report the results of HCTI and Multi-scale HCTI in Table \ref{tab:sts}. For the paraphrase identification task, our model shows better accuracy and F1 score on MSRP dataset. For the semantic textual similarity estimation task, the performance varies across datasets. On the SICK dataset, the performance of Multi-scale HCTI is close to HCTI with slightly better Pearson' $r$ and Spearman's $\rho$. However, the Multi-scale HCTI is not able to outperform HCTI on MSRvid and STSbenchmark. HCTI is still the best neural network model on the STSbenchmark dataset, and the MSRvid dataset is a subset of STSbenchmark.
Although HCTI has strong performance on these two datasets, it performs worse than our model on other datasets.
% Overall, the experimental results demonstrated the superior applicability and generalizability of our proposed models.
Overall, the experimental results demonstrated the general applicability of our proposed model architecture, which performs well on various semantic matching tasks.

% \begin{table}[tb]
%   \caption{Results of Accuracy and F1 score on MSRP test dataset.}
%   \label{tab:MSRP result}
%   \begin{tabular}{lllll}
%     \toprule
%     Model & Acc.(\%) & F1(\%)  \\
%     \midrule
%     MaLSTM & $66.95$ & $73.95$ \\
%     Factorized MaLSTM & $\mathbf{74.09}$ & $\mathbf{82.18}$ \\
%     HCTI & $73.80$ & $80.85$ \\
%     Factorized HCTI & $\mathbf{74.03}$ & $\mathbf{81.76}$ \\
%     \bottomrule
%   \end{tabular}
%   \vspace{0mm}
% \end{table}


% \begin{table}[tb]
%   \caption{Results of Pearson's $r$ and Spearman's $\rho$ on SICK test dataset.}
%   \label{tab:SICK result}
%   \begin{tabular}{lllll}
%     \toprule
%     Model & r & $\rho$ \\
%     \midrule
%     MaLSTM & $0.7824$ & $0.71843$ \\
%     Factorized MaLSTM & $\mathbf{0.8168}$ & $\mathbf{0.74226}$ \\
%     HCTI & $0.8408$ & $\mathbf{0.7698}$ \\
%     Factorized HCTI & $\mathbf{0.8429}$ & $0.7676$ \\
%     \bottomrule
%   \end{tabular}
%   \vspace{0mm}
% \end{table}


% \begin{table}[tb]
%   \caption{Results of Pearson's $r$ and Spearman's $\rho$ on MSRvid test dataset.}
%   \label{tab:MSRvid result}
%   \begin{tabular}{lll}
%     \toprule
%     Model & r & $\rho$  \\
%     \midrule
%     MaLSTM & $0.7325$ & $0.7193$ \\
%     Factorized MaLSTM & $\mathbf{0.8236}$ & $\mathbf{0.8188}$ \\
%     HCTI & $\mathbf{0.8848}$ & $\mathbf{0.8763}$ \\
%     Factorized HCTI & $0.8763$ & $0.8686$ \\
%     \bottomrule
%   \end{tabular}
%   \vspace{0mm}
% \end{table}



% \begin{table}[tb]
%   \caption{Results of Pearson's $r$ and Spearman's $\rho$ on STSbenchmark test dataset.}
%   \label{tab:STSbenchmark result}
%   \begin{tabular}{lllll}
%     \toprule
%     Model & r & $\rho$ \\
%     \midrule
%     MaLSTM & $0.5739$ & $0.5558$ \\
%     Factorized MaLSTM & $\mathbf{0.6839}$ & $\mathbf{0.6575}$ \\
%     HCTI & $\mathbf{0.7697}$ & $\mathbf{0.7549}$ \\
%     Factorized HCTI & $0.7269$ & $0.7033$ \\
%     \bottomrule
%   \end{tabular}
%   \vspace{0mm}
% \end{table}




The industry standard for pose edition is to create rigs, a collection of pieces of software designed to manipulate a character's skeleton. The rig describes the skeleton's bones, how they relate to each other, are constrained in their possible motion and are deformed. These rules are loosely specified and creating a good rig requires a detailed understanding of physics and anatomy, as well as technical and artistic skills. Rigging is thus a time consuming task even for experienced animators, and even more so in large scale productions which often require a different in-depth rig for each character in the cast.
Previous work has helped alleviate this difficulty by providing efficient tools to speed up/and or ease the rigging process, relying on inverse kinematics or data-driven methods.
\subsection{Character pose design}
\subsubsection{Inverse Kinematics (IK)}
IK solvers are a family of methods commonly used in robotics, engineering and computer graphics, in which the parameterization of a kinematic chain is determined from the position of its end effector.
They are a staple tool in pose design software, ensuring the respect of elementary constraints during pose edition. Their de-facto role is to guarantee the length of the limbs, and in some cases to enforce the orientation angle range of a joint.
Many IK solutions have been studied over the years \cite{aristidou_inverse_2018}; usually revolving around approximated linearizations or heuristics. 

Numerical methods require a set of iterations to achieve a satisfactory solution formulated by a cost function to be minimized.
IK solutions can generally be divided into three sub-categories: Jacobian \cite{Siciliano_Handbook_Robot_2007}, Newtonians \cite{cohen_ik_1996} and Heuristics. Most software implement heuristic methods such as Cyclic Coordinate Descent (CCD) \cite{wang_ccd_1991} or 
Forward-Backward Reaching IK (FABRIK) \cite{aristidou_fabrik:_2011} due to their simplicity and extensibility. 

The main drawback of 
these solvers is that they manipulate kinematic chains without taking into account many morphological aspects that make a pose more or less plausible. They offer a first level of help to users but are not sufficient to guarantee a realistic pose. Many joints constraints are dependent on each other and require subjective, human-made approximations.

\subsubsection{Data-driven pose edition}
Data-driven methods offer promising opportunities to solve these approximations. Using real-life data can help in modelling the complex inter-dependencies of skeletons and providing users with smarter edition tools.
While it is still an early field of research, some solutions have been studied. Wu \etal \cite{wu_posing_2009} propose a method for natural character posing from a large motion database. It employs adaptive KD-clustering to select a representative frame from a database and sparse approximations to accelerate training and posing. 
Huang \etal in \cite{Huang_IK_MGDM_2017} present a method based on the formulation of multi-variate Gaussian distribution models (MGDMs), which learn the joint constraints of a kinematic skeleton from motion capture data. 

Some work has also been dedicated to finding new editing interfaces. \modify{}{Instead of the usual setup manipulating joints directly, Guay \etal \cite{guay_line_2013} articulate a framework based on the conceptual "line of action" which describes the overall pose dynamics. They provide a mathematical definition of the line of action, and a interface in which the software modifies the pose to follow a user-provided line. In the same line of though} Garcia \etal \cite{garcia_sketching_2019} propose \modify{a method transforming doodle of trajectories (position and orientation over time) }{a virtual reality-based interface where the user's hands motion (position and orientation over time) are transformed} into sequences of actions and then into detailed character animations using a dataset of parametrized motion clips automatically fitted to the trajectory. 

% ==> DL et Latent Space. 
\subsection{Neural modelling of human motion}
Neural networks have received a great amount of attention over the last decade and shown impressive result in modelling complex data. Human motion has not been spared and deep learning methods have proven their capability of generating realistic motion in a number of difficult cases. 

The literature in neural-based animation include example in user-controlled character navigation \cite{Holden2017} and interactions with the environment \cite{starke_neural_2019}. 
Holden \etal \cite{Holden2020} also show that neural networks can be used to replace parts of existing data-driven methods, improving their scalability potential.
More recently, some work has also focused on improving smaller parts of the animation pipeline rather than replacing it completely. Berson et al. \cite{berson_intuitive_2020} leverage neural networks to provide an interactive system to edit facial animation. 

% Wrap up
Data-driven IK and pose editing can relieve animators from time-consuming, back-and-forth pose adjustments by applying constraints extracted from real-world data. Recently, neural-network-based approaches have demonstrated their ability to model the intricacies of human motion while scaling to large amount of data and retaining a fast inference time. In this paper we seek to take advantage of these properties to create an efficient posing tool, intuitively usable even by a inexperienced user.

%\balance
\bibliographystyle{IEEEtran}
\bibliography{sample-base}


\end{document}
