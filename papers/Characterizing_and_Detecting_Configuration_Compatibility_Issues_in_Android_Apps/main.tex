\documentclass[conference,10pt]{IEEEtran}
\IEEEoverridecommandlockouts
% The preceding line is only needed to identify funding in the first footnote. If that is unneeded, please comment it out.
\newcommand{\tabincell}[2]{\begin{tabular}{@{}#1@{}}#2\end{tabular}}
\usepackage{amsfonts}
\usepackage{amsmath}
\usepackage{cite}
\let\Bbbk\relax
\usepackage{amssymb}
\usepackage{algorithmic}
\usepackage{graphicx}
\usepackage{textcomp}
\usepackage{xcolor}
\usepackage{amsthm}
\usepackage{verbatim}
\usepackage{balance}
\usepackage{multirow}
\usepackage{booktabs}
\usepackage{url}
\usepackage[switch]{lineno}
\usepackage{soul}
\usepackage[colorlinks, linkcolor=blue, anchorcolor=blue,citecolor=blue]{hyperref}
\soulregister{\cite}7 
\soulregister{\citep}7 
\soulregister{\citet}7 
\soulregister{\ref}7 
\soulregister{\pageref}7 
\soulregister{\texttt}7 
\def\BibTeX{{\rm B\kern-.05em{\sc i\kern-.025em b}\kern-.08em
    T\kern-.1667em\lower.7ex\hbox{E}\kern-.125emX}}
\usepackage[utf8]{inputenc}
\usepackage[ruled,linesnumbered,vlined]{algorithm2e}
\usepackage{threeparttable}
\SetKwInput{KwInput}{Input}
\SetKwInput{KwOutput}{Output}
\def\UrlBreaks{\do\A\do\B\do\C\do\D\do\E\do\F\do\G\do\H\do\I\do\J
	\do\K\do\L\do\M\do\N\do\O\do\P\do\Q\do\R\do\S\do\T\do\U\do\V
	\do\W\do\X\do\Y\do\Z\do\[\do\\\do\]\do\^\do\_\do\`\do\a\do\b
	\do\c\do\d\do\e\do\f\do\g\do\h\do\i\do\j\do\k\do\l\do\m\do\n
	\do\o\do\p\do\q\do\r\do\s\do\t\do\u\do\v\do\w\do\x\do\y\do\z
	\do\.\do\@\do\\\do\/\do\!\do\_\do\|\do\;\do\>\do\]\do\)\do\,
	\do\?\do\'\do+\do\=\do\#}

\newcommand{\todored}[1]{\todoc{red}  {[#1]}}
\newcommand{\todoblue}[1]{\todoc{blue}  {[#1]}}
\newcommand{\todoorange}[1]{\todoc{orange}  {[#1]}}
\newcommand{\todocyan}[1]{\todoc{cyan}  {[#1]}}
\newcommand{\todoviolet}[1]{\todoc{violet}  {[#1]}}
\newcommand{\todomagenta}[1]{\todoc{magenta}  {[#1]}}
\newcommand{\todoc}[2]{{\textcolor{#1} {#2}}}
\newcommand{\huaxun}[1]{\todoorange{Huaxun: #1}}
\newcommand{\yepang}[1]{\todoblue{Yepang: #1}}
\newcommand{\scc}[1]{\todoviolet{scc: #1}}
\newcommand{\lili}[1]{\todocyan{Lili: #1}}
\newcommand{\civi}[1]{\todomagenta{Ming: #1}}
\newtheorem{myDef}{Definition}
\newcommand{\hytt}[1]{\texttt{\hyphenchar\font=\defaulthyphenchar #1}}
\renewcommand{\thefootnote}{}
\usepackage{diagbox}

\begin{document}

\title{Characterizing and Detecting Configuration Compatibility Issues in Android Apps\\
{}
\thanks{\textsuperscript{$*$}Shing-Chi Cheung is the corresponding author of this paper.}
}
\author{\IEEEauthorblockN{Huaxun Huang\textsuperscript{$\dagger$}, Ming Wen\textsuperscript{$\S$}, Lili Wei\textsuperscript{$\dagger$}, Yepang Liu\textsuperscript{$\P$}, Shing-Chi Cheung\textsuperscript{$\dagger$*}}
	\IEEEauthorblockA{\textsuperscript{$\dagger$}\textit{Dept. of Computer Science and Engineering, The Hong Kong University of Science and Technology, Hong Kong, China}}
	\IEEEauthorblockA{\textsuperscript{$\S$}\textit{School of Cyber Science and Engineering, Huazhong University of Science and Technology, Wuhan, China}}
	\IEEEauthorblockA{\textsuperscript{$\P$}\textit{Dept. of Computer Science and Engineering, Southern University of Science and Technology, Shenzhen, China}}
	Emails: \{hhuangas@cse.ust.hk, mwenaa@hust.edu.cn, liliwei@cse.ust.hk, liuyp1@sustech.edu.cn, scc@cse.ust.hk\}
	}

\maketitle

\begin{abstract}

XML configuration files are widely used in Android to define an app's user interface and essential runtime information such as system permissions.
As Android evolves, it might introduce functional changes in the configuration environment, thus causing compatibility issues that manifest as inconsistent app behaviors at different API levels.
Such issues can often induce software crashes and inconsistent look-and-feel when running at specific Android versions.
Existing works incur plenty of false positive and false negative issue-detection rules by conducting trivial data-flow analysis while failing to model the XML tree hierarchies of the Android configuration files.
Besides, little is known about how the changes in an Android framework can induce such compatibility issues.
To bridge such gaps, we conducted a systematic study by analyzing 196 real-world issues collected from 43 popular apps.
We identified common patterns of Android framework code changes that induce such configuration compatibility issues.
Based on the findings, we propose \textsc{ConfDroid} that can automatically extract rules for detecting configuration compatibility issues.
The intuition is to perform symbolic execution based on a model learned from the common code change patterns.
Experiment results show that \textsc{ConfDroid} can successfully extract 282 valid issue-detection rules with a precision of 91.9\%.
Among them, 65 extracted rules can manifest issues that cannot be detected by the rules of state-of-the-art baselines.
More importantly, 11 out of them have led to the detection of 107 reproducible configuration compatibility issues that the baselines cannot detect in 30 out of 316 real-world Android apps.
\end{abstract}

\begin{IEEEkeywords}
XML configuration, Android, compatibility, static analysis
% \civi{remember to revise}
\end{IEEEkeywords}

% \leavevmode
% \\
% \\
% \\
% \\
% \\
\section{Introduction}
\label{introduction}

AutoML is the process by which machine learning models are built automatically for a new dataset. Given a dataset, AutoML systems perform a search over valid data transformations and learners, along with hyper-parameter optimization for each learner~\cite{VolcanoML}. Choosing the transformations and learners over which to search is our focus.
A significant number of systems mine from prior runs of pipelines over a set of datasets to choose transformers and learners that are effective with different types of datasets (e.g. \cite{NEURIPS2018_b59a51a3}, \cite{10.14778/3415478.3415542}, \cite{autosklearn}). Thus, they build a database by actually running different pipelines with a diverse set of datasets to estimate the accuracy of potential pipelines. Hence, they can be used to effectively reduce the search space. A new dataset, based on a set of features (meta-features) is then matched to this database to find the most plausible candidates for both learner selection and hyper-parameter tuning. This process of choosing starting points in the search space is called meta-learning for the cold start problem.  

Other meta-learning approaches include mining existing data science code and their associated datasets to learn from human expertise. The AL~\cite{al} system mined existing Kaggle notebooks using dynamic analysis, i.e., actually running the scripts, and showed that such a system has promise.  However, this meta-learning approach does not scale because it is onerous to execute a large number of pipeline scripts on datasets, preprocessing datasets is never trivial, and older scripts cease to run at all as software evolves. It is not surprising that AL therefore performed dynamic analysis on just nine datasets.

Our system, {\sysname}, provides a scalable meta-learning approach to leverage human expertise, using static analysis to mine pipelines from large repositories of scripts. Static analysis has the advantage of scaling to thousands or millions of scripts \cite{graph4code} easily, but lacks the performance data gathered by dynamic analysis. The {\sysname} meta-learning approach guides the learning process by a scalable dataset similarity search, based on dataset embeddings, to find the most similar datasets and the semantics of ML pipelines applied on them.  Many existing systems, such as Auto-Sklearn \cite{autosklearn} and AL \cite{al}, compute a set of meta-features for each dataset. We developed a deep neural network model to generate embeddings at the granularity of a dataset, e.g., a table or CSV file, to capture similarity at the level of an entire dataset rather than relying on a set of meta-features.
 
Because we use static analysis to capture the semantics of the meta-learning process, we have no mechanism to choose the \textbf{best} pipeline from many seen pipelines, unlike the dynamic execution case where one can rely on runtime to choose the best performing pipeline.  Observing that pipelines are basically workflow graphs, we use graph generator neural models to succinctly capture the statically-observed pipelines for a single dataset. In {\sysname}, we formulate learner selection as a graph generation problem to predict optimized pipelines based on pipelines seen in actual notebooks.

%. This formulation enables {\sysname} for effective pruning of the AutoML search space to predict optimized pipelines based on pipelines seen in actual notebooks.}
%We note that increasingly, state-of-the-art performance in AutoML systems is being generated by more complex pipelines such as Directed Acyclic Graphs (DAGs) \cite{piper} rather than the linear pipelines used in earlier systems.  
 
{\sysname} does learner and transformation selection, and hence is a component of an AutoML systems. To evaluate this component, we integrated it into two existing AutoML systems, FLAML \cite{flaml} and Auto-Sklearn \cite{autosklearn}.  
% We evaluate each system with and without {\sysname}.  
We chose FLAML because it does not yet have any meta-learning component for the cold start problem and instead allows user selection of learners and transformers. The authors of FLAML explicitly pointed to the fact that FLAML might benefit from a meta-learning component and pointed to it as a possibility for future work. For FLAML, if mining historical pipelines provides an advantage, we should improve its performance. We also picked Auto-Sklearn as it does have a learner selection component based on meta-features, as described earlier~\cite{autosklearn2}. For Auto-Sklearn, we should at least match performance if our static mining of pipelines can match their extensive database. For context, we also compared {\sysname} with the recent VolcanoML~\cite{VolcanoML}, which provides an efficient decomposition and execution strategy for the AutoML search space. In contrast, {\sysname} prunes the search space using our meta-learning model to perform hyperparameter optimization only for the most promising candidates. 

The contributions of this paper are the following:
\begin{itemize}
    \item Section ~\ref{sec:mining} defines a scalable meta-learning approach based on representation learning of mined ML pipeline semantics and datasets for over 100 datasets and ~11K Python scripts.  
    \newline
    \item Sections~\ref{sec:kgpipGen} formulates AutoML pipeline generation as a graph generation problem. {\sysname} predicts efficiently an optimized ML pipeline for an unseen dataset based on our meta-learning model.  To the best of our knowledge, {\sysname} is the first approach to formulate  AutoML pipeline generation in such a way.
    \newline
    \item Section~\ref{sec:eval} presents a comprehensive evaluation using a large collection of 121 datasets from major AutoML benchmarks and Kaggle. Our experimental results show that {\sysname} outperforms all existing AutoML systems and achieves state-of-the-art results on the majority of these datasets. {\sysname} significantly improves the performance of both FLAML and Auto-Sklearn in classification and regression tasks. We also outperformed AL in 75 out of 77 datasets and VolcanoML in 75  out of 121 datasets, including 44 datasets used only by VolcanoML~\cite{VolcanoML}.  On average, {\sysname} achieves scores that are statistically better than the means of all other systems. 
\end{itemize}


%This approach does not need to apply cleaning or transformation methods to handle different variances among datasets. Moreover, we do not need to deal with complex analysis, such as dynamic code analysis. Thus, our approach proved to be scalable, as discussed in Sections~\ref{sec:mining}.
\section{Background and Motivation}

\subsection{IBM Streams}

IBM Streams is a general-purpose, distributed stream processing system. It
allows users to develop, deploy and manage long-running streaming applications
which require high-throughput and low-latency online processing.

The IBM Streams platform grew out of the research work on the Stream Processing
Core~\cite{spc-2006}.  While the platform has changed significantly since then,
that work established the general architecture that Streams still follows today:
job, resource and graph topology management in centralized services; processing
elements (PEs) which contain user code, distributed across all hosts,
communicating over typed input and output ports; brokers publish-subscribe
communication between jobs; and host controllers on each host which
launch PEs on behalf of the platform.

The modern Streams platform approaches general-purpose cluster management, as
shown in Figure~\ref{fig:streams_v4_v6}. The responsibilities of the platform
services include all job and PE life cycle management; domain name resolution
between the PEs; all metrics collection and reporting; host and resource
management; authentication and authorization; and all log collection. The
platform relies on ZooKeeper~\cite{zookeeper} for consistent, durable metadata
storage which it uses for fault tolerance.

Developers write Streams applications in SPL~\cite{spl-2017} which is a
programming language that presents streams, operators and tuples as
abstractions. Operators continuously consume and produce tuples over streams.
SPL allows programmers to write custom logic in their operators, and to invoke
operators from existing toolkits. Compiled SPL applications become archives that
contain: shared libraries for the operators; graph topology metadata which tells
both the platform and the SPL runtime how to connect those operators; and
external dependencies. At runtime, PEs contain one or more operators. Operators
inside of the same PE communicate through function calls or queues. Operators
that run in different PEs communicate over TCP connections that the PEs
establish at startup. PEs learn what operators they contain, and how to connect
to operators in other PEs, at startup from the graph topology metadata provided
by the platform.

We use ``legacy Streams'' to refer to the IBM Streams version 4 family. The
version 5 family is for Kubernetes, but is not cloud native. It uses the
lift-and-shift approach and creates a platform-within-a-platform: it deploys a
containerized version of the legacy Streams platform within Kubernetes.

\subsection{Kubernetes}

Borg~\cite{borg-2015} is a cluster management platform used internally at Google
to schedule, maintain and monitor the applications their internal infrastructure
and external applications depend on. Kubernetes~\cite{kube} is the open-source
successor to Borg that is an industry standard cloud orchestration platform.

From a user's perspective, Kubernetes abstracts running a distributed
application on a cluster of machines. Users package their applications into
containers and deploy those containers to Kubernetes, which runs those
containers in \emph{pods}. Kubernetes handles all life cycle management of pods,
including scheduling, restarting and migration in case of failures.

Internally, Kubernetes tracks all entities as \emph{objects}~\cite{kubeobjects}.
All objects have a name and a specification that describes its desired state.
Kubernetes stores objects in etcd~\cite{etcd}, making them persistent,
highly-available and reliably accessible across the cluster. Objects are exposed
to users through \emph{resources}. All resources can have
\emph{controllers}~\cite{kubecontrollers}, which react to changes in resources.
For example, when a user changes the number of replicas in a
\code{ReplicaSet}, it is the \code{ReplicaSet} controller which makes sure the
desired number of pods are running. Users can extend Kubernetes through
\emph{custom resource definitions} (CRDs)~\cite{kubecrd}. CRDs can contain
arbitrary content, and controllers for a CRD can take any kind of action.

Architecturally, a Kubernetes cluster consists of nodes. Each node runs a
\emph{kubelet} which receives pod creation requests and makes sure that the
requisite containers are running on that node. Nodes also run a
\emph{kube-proxy} which maintains the network rules for that node on behalf of
the pods. The \emph{kube-api-server} is the central point of contact: it
receives API requests, stores objects in etcd, asks the scheduler to schedule
pods, and talks to the kubelets and kube-proxies on each node. Finally,
\emph{namespaces} logically partition the cluster. Objects which should not know
about each other live in separate namespaces, which allows them to share the
same physical infrastructure without interference.

\subsection{Motivation}
\label{sec:motivation}

Systems like Kubernetes are commonly called ``container orchestration''
platforms. We find that characterization reductive to the point of being
misleading; no one would describe operating systems as ``binary executable
orchestration.'' We adopt the idea from Verma et al.~\cite{borg-2015} that
systems like Kubernetes are ``the kernel of a distributed system.'' Through CRDs
and their controllers, Kubernetes provides state-as-a-service in a distributed
system. Architectures like the one we propose are the result of taking that view 
seriously.

The Streams legacy platform has obvious parallels to the Kubernetes
architecture, and that is not a coincidence: they solve similar problems.
Both are designed to abstract running arbitrary user-code across a distributed
system.  We suspect that Streams is not unique, and that there are many
non-trivial platforms which have to provide similar levels of cluster
management.  The benefits to being cloud native and offloading the platform
to an existing cloud management system are: 
\begin{itemize}
    \item Significantly less platform code.
    \item Better scheduling and resource management, as all services on the cluster are 
        scheduled by one platform.
    \item Easier service integration.
    \item Standardized management, logging and metrics.
\end{itemize}
The rest of this paper presents the design of replacing the legacy Streams 
platform with Kubernetes itself.


\section{Applications of Matrix Balancing}
\label{app:related-works}
This section contains a brief survey on the applications of matrix balancing in a diverse range of disciplines.

\textbf{Traffic and Transportation Networks.} These applications are some of the earliest and most popular uses of matrix balancing. \citet{kruithof1937telefoonverkeersrekening} considered the problem of estimating new telephone traffic patterns among telephone exchanges given existing traffic volumes and marginal densities of departing and terminating traffic for each exchange when their subscribers are updated. A closely related problem in transportation networks is to use observed \emph{total} traffic flows out of each origin and into each destination to estimate \emph{detailed} traffics between origin-destination pairs \citep{carey1981method,nguyen1984estimating,sheffi1985urban,chang2021mobility}. The key idea is to find a traffic assignment satisfying the total flow constraints that is ``close'' to some known reference traffic pattern. The resulting (relative) entropy minimization principle, detailed in \eqref{eq:relative-entropy-minimization}, is an important optimization perspective on Sinkhorn's algorithm.

\textbf{Demography.} A problem similar to that in networks arises in demography. Given out-of-date inter-regional migration statistics and up-to-date net migrations from and into each region, the task is to estimate migration flows that are consistent with the marginal statistics \citep{plane1982information}.

\textbf{Economics.}
General equilibrium models in economics employ \emph{social accounting matrices}, which record the flow of funds between important (aggregate) agents in an economy at different points in time \citep{stone1962multiple,pyatt1985social}. Often accurate data is available on the total expenditure and receipts for each agent, but due to survey error or latency, detailed flows are not always consistent with these marginal statistics. Thus they need to be ``adjusted'' to satisfy consistency requirements.  Other important applications of the matrix balancing problem in economics include the estimation of gravity equations in inter-regional and international trade \citep{uribe1966information,wilson1969use,anderson2003gravity,silva2006log} and coefficient matrices in input-output models \citep{leontief1965structure,stone1971computable,bacharach1970biproportional}. In recent years, optimal transport \citep{villani2009optimal} has found great success in economics \citep{carlier2016vector,galichon2018optimal,galichon2021unreasonable,galichon2021matching}. As matrix balancing and Sinkhorn's algorithm are closely connected to optimal transport (\cref{sec:linear-convergence}), they are likely to have more applications in economics.

\textbf{Statistics.}
A contingency table encodes frequencies of subgroups of populations, where the rows and columns correspond to values of two categorical variables, such as gender and age. Similar to social accounting matrices, a common problem is to adjust out-of-date or inaccurate cell values of a table given accurate marginal frequencies. The problem is first studied by  \citet{deming1940least}, who proposed the classic iterative algorithm. \citet{ireland1968contingency} formalized its underlying entropy optimization principle, and \citet{fienberg1970iterative} analyzed its convergence. 

\textbf{Optimization and Machine Learning.} Matrix balancing plays a different but equally important role in optimization. Given a linear system $Ax=b$ with non-singular $A$, it is well-known that the convergence of first order solution methods depends on the \emph{condition number} of $A$, and an important problem is to find diagonal \emph{preconditioners} $D^1,D^0$ such that $D^1AD^0$ has smaller condition number. Although it is possible to find optimal diagonal preconditioners via semidefinite programming \citep{boyd1994linear,qu2022optimal}, matrix balancing methods remain very attractive heuristics due to their low computational costs, and continue to be an important component of modern workhorse optimization solvers \citep{ruiz2001scaling,bradley2010algorithms,knight2013fast,stellato2020osqp,gao2022hdsdp}.

In recent years, optimal transport distances have become an important tool in machine learning and optimization for measuring the similarity between probability distributions \citep{arjovsky2017wasserstein,peyre2019computational,blanchet2019robust,mohajerin2018data,kuhn2019wasserstein}. Besides appealing theoretical properties, efficient methods to approximate them in practice have also contributed to their wide adoption. This is achieved through an entropic regularization of the OT problem, which is precisely equivalent to the matrix balancing problem and solved via Sinkhorn's algorithm \citep{cuturi2013sinkhorn,altschuler2017near,dvurechensky2018computational}.   

\textbf{Political Representation.} The apportionment of representation seats based on election results has found unexpected solution in matrix balancing. A standard example consists of the matrix recording the votes each party received from different regions. The marginal constraints are that each party's total number of seats be proportional to the number of votes they receive, and similar for each region. A distinct feature of this problem is that the final apportionment matrix must have integer values, and variants of the standard algorithm that incorporate \emph{rounding} have been proposed \citep{balinski2006matrices,pukelsheim2006current,maier2010divisor}. More than just mathematical gadgets, they have found real-world implementations in Swiss cities such as Zurich \citep{pukelsheim2009iterative}.

\textbf{Markov Chains.}
Last but not least, Markov chains and related topics offer another rich set of applications for matrix balancing. \citet{schro1931uber} considered a continuous version of the following problem. Given a ``prior'' transition matrix $A$ of a Markov chain and \emph{observed} distributions $p^0,p^1$ before and after the transition, find the most probable transition matrix (or path) $\hat A$ that satisfies $\hat A p^0=p^1$. This is a variant of the matrix balancing problem and has been studied and generalized in a long line of works \citep{fortet1940resolution,beurling1960automorphism,ruschendorf1995convergence,gurvits2004classical,georgiou2015positive,friedland2017schrodinger}. Applications in marketing estimate customers' transition probabilities between different brands using market share data
\citep{theil1966quadratic}. Coming full circle back to  choice modeling, matrix balancing has also been used to rank nodes of a network. \citet{knight2008sinkhorn} explains how the inverses of left and right scalings of the adjacency matrix with uniform target marginals (stationary distributions) can be naturally interpreted as measures of their ability to attract and emit traffic. This approach is also related to the works of \citet{lamond1981bregman,kleinberg1999authoritative,tomlin2003new}.


\section{Numerical Experiments}
\label{sec:empirics}
We compare the empirical performance of Sinkhorn's algorithm with the iterative LSR (I-LSR) algorithm of \citet{maystre2015fast} on real choice datasets. Because the implementation of I-LSR by \citet{maystre2015fast}
only accommodates pairwise comparison data and partial ranking data, but does not easily generalize to multi-way choice data, we focus on data with pairwise comparisons. 

We use the
natural parameters $\log s_{j}$ (logits) instead of $s_{j}$ when computing and evaluating
the updates, as the probability of $j$ winning
over $k$ is proportional to the ratio $s_{j}/s_{k}$, so that $s_j$ are usually logarithmically spaced. To make sure that estimates
are normalized, we impose the normalization that $\sum_{j}s_{j}=m$,
the number of objects, at the end of each iteration, although due to
the logarithm scale of the convergence criterion, the choice of normalization
does not seem to significantly affect the performances of the algorithms.
%generated based on the proposal in \citet{agarwal2018accelerated}, and partial ranking data, generated with code from \citet{maystre2015fast}.

We evaluate the algorithms on five real-world datasets consisting
of partial ranking or pairwise comparison data. The NASCAR dataset
consists of ranking results of the 2002 season NASCAR races. The SUSHI
datasets consist of rankings of sushi items. The Youtube dataset consists
of pairwise comparisons between videos and which one was considered
more entertaining by users. The GIFGIF dataset similarly consists
of pairwise comparisons of GIFs that are rated based on which one
is closer to describing a specific sentiment, such as happiness and
anger. We downsampled the Youtube and GIFGIF datasets due to memory
constraints. 
\begin{table*}[t]
\begin{centering}
\hspace*{-3cm}%
\begin{tabular}{|c|c|c|c|c|c|c|c|c|}
\hline 
\multirow{2}{*}{dataset} & \multirow{2}{*}{data type} & \multirow{2}{*}{items} & \multirow{2}{*}{observations} & \multirow{2}{*}{$k$} & \multicolumn{2}{c|}{Sinkhorn} & \multicolumn{2}{c|}{I-LSR}\tabularnewline
\cline{6-9} \cline{7-9} \cline{8-9} \cline{9-9} 
 &  &  &  &  & iterations & time & iterations & time\tabularnewline
\hline 
\hline 
NASCAR & $k$-way ranking & 83 & 36 & 43 & 20 & \textbf{0.029} & 13 & 0.960\tabularnewline
\hline 
SUSHI-10 & $k$-way ranking & 10 & 5000 & 10 & 16 & \textbf{0.025} & 8 & 1.708\tabularnewline
\hline 
SUSHI-100 & $k$-way ranking & 100 & 5000 & 10 & 21 & \textbf{0.253} & 9 & 2.142\tabularnewline
\hline 
Youtube & pairwise comparison & 2156 & 28134 & 2 & 89 & \textbf{7.984} & 33 & 15.026\tabularnewline
\hline 
GIFGIF & pairwise comparison & 2503 & 6876 & 2 & 1656 & \textbf{30.97} & 315 & 138.63\tabularnewline
\hline 
\end{tabular}\hspace*{-3cm}
\par\end{centering}
\caption{Performance of iterative ML inference algorithms on five real datasets. Youtube and GIFGIF data were subsampled. 
Convergence is declared when the maximum entry-wise change of an update
is less than $10^{-8}$. At convergence, the ML estimates returned
by the algorithms have entry-wise difference of at most $10^{-10}$.}
\label{tab:empirical}
\end{table*}

In \cref{tab:empirical} we report the running time of the three algorithms on different
datasets. Convergence is declared when the maximum entry-wise change
of an update to the natural parameters $\log s_{i}$ is less than
$10^{-8}$. At convergence, the MLEs returned by the algorithms have entry-wise difference of at most $10^{-10}$. %Since
% we know that the sequence of MM algorithm must converge to a stationary
% point, and that the matrix scaling algorithm is equivalent to the
% MM algorithm for pairwise comparisons and partial rankings, we know
% that the final estimates must be the MLE. 
We see that Sinkhorn's algorithm consistently outperforms
the I-LSR algorithm in terms of convergence speed. It also has the additional advantage of being parallelized with
elementary matrix-vector operations, whereas the iterative I-LSR algorithm needs to repeatedly
compute the steady-state of a continuous-time Markov chain, which
is prone to problems of ill-conditioning. This also explains why Sinkhorn's algorithm may take more iterations but has better wall clock time, since each iteration is much less costly. On the other hand, we note
that for large datasets, particularly those with a large number of
observations or alternatives, the dimension of $A$ used
may become too large for the memory of a single machine. If this is still a problem after removing duplicate rows and columns according to \cref{sec:equivalence}, we can use distributed implementations of Sinkhorn's algorithm, which in view of its connections to message passing algorithms, is a standard procedure.
% decompose $A$ into blocks of sub-matrices, on which \cref{alg:scaling} allows for parallel computation. 
%See \cref{alg:scaling-distributed}.
\section{The Technique}

Our approach to Alloy specification repair involves a series of tasks, for fault detection, fault localization, fix candidate generation, and fix candidate assessment. We describe these in more detail below. 

\subsection{Fault Detection and Fix Acceptance Criterion}

In general, given an Alloy specification, we may say that such specification is \emph{faulty} if at least one of the analysis commands in the specification has an outcome contrary to its corresponding expectation. This can be either a failing assertion (assertion with counterexamples), or a predicate that is unsatisfiable while the user expected it to be satisfiable, or vice versa. We may also allow for other flavors in commands, in particular Alloy test cases, in the spirit of AUnit \cite{Sullivan+2018}. The fault detection stage then resorts to SAT solving, the underlying analysis mechanism behind Alloy Analyzer, the tool for Alloy specification analysis \cite{Jackson2006}. Similarly, a fix candidate can be considered an acceptable patch when all the analysis commands in the specification have an outcome that coincides with the corresponding command's expectations. 

Our technique requires the user to identify the specification oracle, i.e., the assertions, predicates or tests that the technique will have to consider as fix acceptance criterion. The technique will then identify faults in the remainder of the specification (the oracle is left out of the analysis space for fault localization), and generate fix candidates for the faulty locations. Therefore, our repair approach cannot fix any faulty situation, but only those where the developer is certain about some part of it (the oracle), and wishes to alter the remainder of the specification to pass it. Looking for solutions that may modify the specification \emph{and} the criterion for acceptance would lead to fixes that may simply relax the acceptance criterion. Notice that, in this respect, we follow the same approach that ARepair and most test-based program repair techniques: the tests (the repair oracle) cannot be changed in the repair process. As described later on in this section, other trivial solutions such as changing a command's expectations or simply removing a command are prevented, due to how the fault localization is performed (which cannot be blamed on commands) and how fix candidates are generated (only by mutating the faulty locations). 

\subsection{Fault Localization}

Once a specification is deemed faulty, we need to identify the specific parts of the specification that are more likely to be blamed for the fault or faults. We do not deal with fault localization in this paper, and we assume an external technique/tool provides fault localization information. There exist techniques for fault localization that specifically target Alloy specifications, such as the spectrum-based fault localization mechanism behind ARepair \cite{Wang+2018}, and our fault localization technique presented in \cite{Zheng+2021}. While in principle any fault localization technique would fit our technique, as long as the employed fault localization can handle the oracles present in the faulty specification, it is worth to remark that the fault localization within ARepair inherently depends on having tests as oracles (acceptance criteria) for specifications \cite{Wang+2018}. Moreover, the fault localization in ARepair can dynamically change the identified faulty locations, as the specification is transformed during the repair process. Our technique, on the other hand, uses an \emph{offline} process for fault localization: the faulty program is fed to the fault localization tool, and a number of suspicious specification locations are returned. This is the input to our specification repair approach, and the space of \emph{all} possible patches for these locations, for a maximum depth in mutation application and a given set of mutation operators, will be considered. 

For our experiments in Section~IV, we use the FLACK fault localization technique \cite{Zheng+2021}. While we do not describe in detail the fault localization technique in this paper (we refer the reader to \cite{Zheng+2021}), let us remark a number of facts about FLACK: it supports arbitrary satisfiability checks and assertions, as well as tests, as specification oracles; it is based on the use of (partial) maximum satisfiability procedures, to process counterexamples of an Alloy model (witnessing the faulty status of the specification); and it can only identify faults within formulas and relational expressions, it cannot locate faults in data definitions, such as signature and field declarations, nor in commands (Alloy's runs and checks). 

\subsection{Generation of Fix Candidates}

Once the suspicious expressions are identified, syntactical variants of these expressions are produced. We consider an ample set of mutation operations, including the obvious logical and relational operator insertion, removal and replacement, quantification mutation (e.g., changing a quantifier), multiplicity constraint replacement, field/variable swap/replacement, etc., based on Alloy's grammar. Our tool processes the specification to obtain some typing information, so that some legal expressions that necessarily lead to empty relation/contradictory formulas are disregarded, as well as innocuous operation application (e.g., double transitive closure). Two elements are important to highlight here, namely the use of join to produce navigation chains, using fields, signatures, etc., and the possibility of \emph{combining} mutations, i.e., applying further mutations to an already mutated expression, akin the so called higher-order mutants \cite{DBLP:journals/infsof/JiaH09} in mutation testing.

Both the mutation operators and the maximum depth, i.e., the number of cumulative mutations (hence, the higher order nature of the generated mutants) that can be applied to a given faulty location, are configurable. These are bounded-exhaustively generated as the space of fix candidates is traversed (see below). In our experiments, we used 21 mutation operators in total, typically leading to roughly between 60 and 260 1-level mutants per location.  

\subsection{Fix Candidate Space Traversal}

Here we present our general repair approach. The two pruning techniques just introduced, are also described in more detail, and we argue about their soundness. The search space is organized as a search tree in a traditional search problem: the root is the original specification, with its faulty locations identified; and if a specification $s$ is in the tree and $s'$ can be obtained by applying a mutation to a faulty location, then $s'$ is also in the tree, with the same locations marked as faulty (so that the mutation process can be iterated). This in principle leads to an infinite fix candidate space, which we explore up to a maximum depth. While any search strategy may be applied, we explore the state space in a breadth-first fashion. 

\subsubsection{Partial repair checking}

Our first pruning strategy consists of identifying one of the suspicious locations for which a current repair candidate fails, as established by an analysis check that does not depend on the remainder of the faulty locations. We will describe it in more detail, assuming two faulty locations, without loss of generality. Let $\textit{Spec}$ be an Alloy specification, $\textit{Check}_1, \dots, \textit{Check}_k$ its analysis checks used as oracles, and $L_0, L_1$ the suspicious locations identified by the fault localization phase. Each analysis check $\textit{Check}_i$ refers to a specific part of $\textit{Spec}$, which can be determined by a straightforward syntactic analysis: $\textit{Check}_i$ refers to the formula it directly mentions (the body of the corresponding predicate or assertion), all the facts (axioms of the specification that are implicitly involved in every analysis check), and the symbols directly and indirectly referred syntactically to by these (predicates called, relations used, etc.). This syntactic analysis can determine then, for every $\textit{Check}_i$, which of the suspicious locations $L_0$ and $L_1$ it involves.

Most logics, and certainly Alloy's relational logic, have a sort of syntactic locality property, that guarantees that the validity/satisfiability of a formula depends only on the symbols it refers to. (In the case of Alloy, since validity/satisfiability is actually bounded validity/satisfiability, it can also depend on the scope, the bound, of analysis; but since the bound of analysis cannot be modified in the patch generation phase, we can disregard it). Moreover, the logic is monotonic, meaning that adding more assumptions to a formula can never reduce the conclusions drawn originally from it. These properties allow us to make the following observation. Let $m_0$ and $n_0$ constitute the modifications to locations $L_0$ and $L_1$, respectively, in the current fix candidate (i.e., be the expressions substituting the original expressions in locations $L_0$ and $L_1$ of $\textit{Spec}$). If a failing satisfiability check $\textit{Check}_i$ refers to only one of the suspicious locations, say $L_0$ and its current expression $m_0$, this means that the formula in $\textit{Check}_i$ is determined to be false independently of $n_0$. Then, for every alternative expression $n_i$ for location $L_1$, the corresponding fix candidate $(m_0, n_i)$ (the replacement expressions for locations $L_0$ and $L_1$) will still make $\textit{Check}_i$ to be false, due to the monotonicity of the logic. In other words, the specification cannot be repaired by modifying location $L_1$ if the current fix for location $L_0$ is maintained. We can therefore exclude (prune from the checking) all $(m_0, n_i)$ fix candidates as soon as we determine this situation, which in turn can be determined by a syntactic analysis of the specification, and the analysis outcome for fix candidate $(m_0, n_0)$. 

We refer to this analysis and the corresponding pruning it enables as \emph{partial repair checking}, due to the partiality of fix candidates when these do not involve all suspicious locations.

\subsubsection{Variabilization}

Our second pruning strategy is called \emph{variabilization}, due to the mechanism employed for prune checking, that requires introducing fresh variables to refer to fix candidates to specific locations, in a general way.

Let $\textit{Check}_i$ be a failing assertion (validity) check that refers to suspicious locations $L_0$ and $L_1$, and let $(m_0, n_0)$ be the current failing fix candidate. Notice that since $\textit{Check}_i$ is a failing validity check, we have a counterexample $\textit{CEX}_i$ as a result of the violation. That is, we have that:
\begin{displaymath}
    \textit{CEX}_i \not \models \textit{Spec}[m_0, n_0] \Rightarrow \textit{Check}_i,
\end{displaymath}
where $\textit{Spec}[m_0, n_0]$ denotes the fix candidate obtained by replacing $L_0$ and $L_1$ by $m_0$ and $n_0$, respectively, in $\textit{Spec}$. The purpose of variabilization is to check whether the current fix for $L_0$, i.e., $m_0$, \emph{may} work with \emph{some} candidate for $L_1$ (other than $n_0$, of course, which we already know it does not work). For technical reasons, we actually check whether some fix for $L_1$ may work in combination with $m_0$, for counterexample $\texttt{CEX}_i$. Let us describe the process for performing this check. 

Notice that fault locations can be subexpressions of a formula; let us refer by $F_1$ to the formula (predicate, fact, etc) containing $L_1$. Also, let $T$ be the most general type for $L_1$ in context $F_1$ (in Alloy, this most general type will depend on the arity required by $L_1$ in $F_1$, the context in which $L_1$ may depend upon, and will use the most general unary type, the universe \texttt{univ}). Let $\textit{Spec}_{L_1}$ be the specification obtained by replacing $F_1$ in $\textit{Spec}$ by 
\begin{displaymath}
\exists l_1: T \vert F_1[l_1/L_1] 
\end{displaymath}
i.e., we substitute $L_1$ by an existentially quantified variable of type $T$ (hence the name \emph{variabilization}). We now can check: 
\begin{displaymath}
    \textit{CEX}_i \models \textit{Spec}_{L_0}[m_0] \Rightarrow \textit{Check}_i,
\end{displaymath}
i.e., whether there exists a \emph{value} of type $T$ that can be put in place of location $L_1$, so that $\textit{CEX}_i$ ceases to be a counterexample. If this is the case, then local fix $m_0$ works as a fix for $L_0$, at least as far as $\textit{CEX}_i$ is concerned, and we may traverse the space of local candidates for $L_1$ to attempt to find a complete fix. But, on the other hand, if the above check fails, then there is no value that can be put in place of $L_1$ such that the local fix $m_0$ would work ($\textit{CEX}_i$ would still be a counterexample). Therefore, we can again exclude (prune from the checking) all $(m_0, n_i)$ fix candidates if the check fails. 

One may argue why not check the ``variabilized'' specification in the general case, instead of doing so \emph{only} for counterexample $\textit{CEX}_i$. The reason has to do with the type $T$ of location $L_1$. When this type is a relation of an arity greater than one, variabilization leads to a higher-order quantification, that Alloy cannot handle as a general analysis check, but it can do so for the specific counterexample $\textit{CEX}_i$. 

To clarify this variabilization process, and especially the reason why we typically have higher-order quantification, let us consider the example introduced in Section~II, where one local fix candidate is applied and the other was generalized with question marks. Assuming that assertion \texttt{ContainsCorrect} failed, a counterexample $\textit{CEX}$ was generated from this fix. To check whether variabilization pruning can be applied, we turn the question marks into existential quantifications. Intuitively, the corresponding variabilized specification would then be as follows (we are abusing the notation below, using Boolean for the type of the variabilized formula within \texttt{Sorted}):

%{\small
%\begin{verbatim}
\begin{lstlisting} []
pred Sorted[This: List] {
 some b: Boolean | all n: This.header.*link | b
}

pred Contains[This: List, x: Int, res: Boolean]{
 RepOk[This]
 (x !in This.header.*link.elem => res=False ) 
 && res = True
}
\end{lstlisting}
%{\end{verbatim}
%} }

\noindent
However, we need to take into account that the variabilization context, the place where the location being variabilized occurs, depends in this case both on \texttt{This} and \texttt{n}. Thus, the actual variabilization for the check is as follows (we are again abusing the notation for the sake of clarity):

%{\small
%\begin{verbatim}
\begin{lstlisting} []
pred Sorted[This: List] {
 some b: List -> Node -> Boolean | 
     all n: This.header.*link | b[This, n]
}

pred Contains[This: List, x: Int, res: Boolean]{
 RepOk[This]
 (x !in This.header.*link.elem => res=False ) 
 && res = True
 \end{lstlisting}
%\end{verbatim}
%} 

\noindent
We cannot check \texttt{ContainsCorrect} over this specification due to the higher-order quantification in \texttt{Sorted}; but we can check it for $\textit{CEX}$. 

It is worth remarking that the above check, if successful, will produce a \emph{relational value} for \texttt{b} that makes the variabilized specification work. It will \emph{not} produce an expression to put in place of the body of the quantification, as a local fix candidate. It would not even produce a relational value that can be ``hardwired'' as a local fix of the corresponding location, since it is in principle just a relational value that works for counterexample $\textit{CEX}$. But its existence is what enables us to decide that a local fix for $L_1$ (\texttt{Sorted}) may be possible, considering the current local fix for $L_0$ (the \texttt{\&\&} in \texttt{Contains}). Our check essentially corresponds to only checking for feasibility of a local fix with respect to other locations. 

An alternative to the above would be to attempt to turn the relational value bound to $b$ into a relational \emph{expression}, that can be considered a local fix candidate. Such a process would correspond to a \emph{synthesis procedure}, which would require a grammar for expressions, so that the solver can attempt to work out an instance (an actual expression) during the satisfiability process. While it is technically feasible, it is also significantly more costly than our simpler query for satisfiability, which we solely use for pruning. 

%Such queries are infeasible in the context of programs, but are realisable in specifications, especially in the context of Alloy and its automated analysis. It can actually be ``implemented'' in different ways. One approach, which is in a sense followed by ARepair \cite{XXX}, is to exploit a \emph{synthesis} mechanism, as \emph{sketching}. That is, not only query for satisfiability, but do so in a way that the satisfying valuation can be turned into a fix. The cost is however greater than solely querying for satisfiability, as one needs find a single expression that would make \emph{all} oracles to pass at the same time, and its feasibility needs to consider a limited set of expressions, to make it solvable by the underlying SAT solver. Unsatisfiability in this case means there is no expression, for the a given (limited) grammar, that would fix the specification. Thus, (un)satisfiability is relative both to the scope (as usual in the context of Alloy) and to the grammar used for candidate expressions. 

%When checking the oracles on this specification, satisfiability will \emph{not} produce an expression to put in place of the body of the quantification, but a \emph{relation}, i.e., an element of the semantics of the specification (not of Alloy's syntax), so we cannot use it, at least not directly, as a fix, but only as a generic check for repair candidate feasibility. 

%As we show in our evaluation section, these checks can have a significant impact in pruning, and lead to a technique for program repair that is complementary to that introduced by ARepair.

%checks that serve the purpose of checking whether a fix candidate is a feasible repair for the corresponding location. As no grammar for the expressions is involved, this check is bounded complete. We call our approach \emph{variabilisation}, as it is based on replacing the buggy location by a variable of the corresponding type. To make sound pruning, we need to make our variabilisations \emph{context sentitive}. That is, instead of turning the above specification into (we are abusing the notation below):  


\section{Evaluation}
\label{sec:evaluation}
\begin{table*}[!t]
\begin{center}
%\small
\caption {Benchmarks and applications for the study of the application-level resilience}
\vspace{-5pt}
\label{tab:benchmark}
\tiny
\begin{tabular}{|p{1.7cm}|p{7.5cm}|p{4cm}|p{2.5cm}|}
\hline
\textbf{Name} 	& \textbf{Benchmark description} 		& \textbf{Execution phase for evaluation}  			& \textbf{Target data objects}             \\ \hline \hline
CG (NPB)             & Conjugate Gradient, irregular memory access (input class S)   & The routine conj\_grad in the main computation loop  & The arrays $r$ and $colidx$     \\\hline
MG (NPB)    	       & Multi-Grid on a sequence of meshes (input class S)             & The routine mg3P in the main computation loop & The arrays $u$ and $r$ 	\\ \hline
FT (NPB)             & Discrete 3D fast Fourier Transform (input class S)            & The routine fftXYZ in the main computation loop  & The arrays $plane$ and $exp1$    \\ \hline
BT (NPB)             & Block Tri-diagonal solver (input class S)         		& The routine x\_solve in the main computation loop & The arrays $grid\_points$ and $u$	\\ \hline
SP (NPB)             & Scalar Penta-diagonal solver (input class S)         		& The routine x\_solve in the main computation loop & The arrays $rhoi$ and $grid\_points$  \\ \hline
LU (NPB)            & Lower-Upper Gauss-Seidel solver (input class S)        	& The routine ssor 	& The arrays $u$ and $rsd$ \\ \hline \hline
LULESH~\cite{IPDPS13:LULESH} & Unstructured Lagrangian explicit shock hydrodynamics (input 5x5x5) & 
The routine CalcMonotonicQRegionForElems 
& The arrays $m\_elemBC$ and $m\_delv\_zeta$ \\ \hline
AMG2013~\cite{anm02:amg} & An algebraic multigrid solver for linear systems arising from problems on unstructured grids (we use  GMRES(10) with AMG preconditioner). We use a compact version from LLNL with input matrix $aniso$. & The routine hypre\_GMRESSolve & The arrays $ipiv$ and $A$   \\ \hline
%$hierarchy.levels[0].R.V$ \\ \hline
\end{tabular}
\end{center}
\vspace{-5pt}
\end{table*}

%We evaluate the effectiveness of ARAT, and 
%We use ARAT to study the application-level resilience.
%The goal is to demonstrate 
%that aDVF can be a very useful metric to quantify the resilience of data objects
%at the application level. 
We study 12 data objects from six benchmarks of the NAS parallel benchmark (NPB) suite (we use SNU\_NPB-1.0.3) and 4 data objects from two scientific applications. 
%which is a c version of NPB 3.3, but ARAT can work for Fortran.
Those data objects are chosen to be representative: they have various data access patterns and participate in various execution phases.  
%For the benchmarks, we use CLASS S as the input problems and use the default compiler options of NPB.
For those benchmarks and applications, we use their default compiler options, and use gcc 4.7.3 and LLVM 3.4.2 for trace generation.
To count the algorithm-level fault masking, we use the default convergence thresholds (or the fault tolerance levels) for those benchmarks.
Table~\ref{tab:benchmark} gives 
%for->on by anzheng
detailed information on the benchmarks and applications.
The maximum fault propagation path for aDVF analysis is set to 10 by default.
%the value shadowing threshold is set as 0.01 (except for BT, we use $1 \times 10^{-6}$).
%These value shadowing thresholds are chosen such that any error corruption
%that results in the operand's value variance less than 1\% (for the threshold 0.01) or 0.0001\% (for the threshold $1 \times 10^{-6}$) during the 
%trace analysis does not impact the outcome correctness of six benchmarks.
%LU: check the newton-iteration residuals against the tolerance levels
%SP: check the newton-iteration residuals against the tolerance levels
%BT: check the newton-iteration residuals against the tolerance levels

\subsection{Resilience Modeling Results}
%We use ARAT to calculate aDVF values of 16 data objects. 
Figure~\ref{fig:aDVF_3tiers_profiling}
shows the aDVF results and breaks them down into the three levels 
(i.e., the operation-level, fault propagation level, and algorithm-level).
Figure~\ref{fig:aDVF_3classes_profiling} shows the 
%for->of by anzheng
results for the analyses at the levels of the operation and fault propagation,
and further breaks down the results into 
the three classes (i.e., the value overwriting, logical and comparison operations,
and value shadowing). %based on the reasons of the fault masking.
We have multiple interesting findings from the results.

\begin{figure*}
	\centering
        \includegraphics[width=0.8\textwidth]{three_tiers_gray.pdf}
% * <azguolu@gmail.com> 2017-03-23T03:20:28.808Z:
%
% ^.
        \vspace{-5pt}
        \caption{The breakdown of aDVF results based on the three level analysis. The $x$ axis is the data object name.}
        \vspace{-8pt}
        \label{fig:aDVF_3tiers_profiling}
\end{figure*}


\begin{figure*}
	\centering
	\includegraphics[width=0.8\textwidth]{three_types_gray.pdf}
	\vspace{-5pt}
	\caption{The breakdown of aDVF results based on the three classes of fault masking. The $x$ axis is the data object name. \textit{zeta} and \textit{elemBC} in LULESH are \textit{m\_delv\_zeta} and \textit{m\_elemBC} respectively.} % Anzheng
	\vspace{-5pt}
	\label{fig:aDVF_3classes_profiling}
    %\vspace{-5pt}
\end{figure*}

(1) Fault masking is common across benchmarks and applications.
Several data objects (e.g., $r$ in CG, and $exp1$ and $plane$ in FT)
have aDVF values close to 1 in Figure~\ref{fig:aDVF_3tiers_profiling}, 
which indicates that most of operations working on these data objects
have fault masking.
However, a couple of data objects have much less intensive fault masking.
For example, the aDVF value of $colidx$ in CG is 0.28 (Figure~\ref{fig:aDVF_3tiers_profiling}). 
Further study reveals that $colidx$ is an array to store column indexes of sparse matrices, and there is few operation-level or fault propagation-level fault masking  (Figure~\ref{fig:aDVF_3classes_profiling}).
The corruption of it can easily cause segmentation fault caught by the
algorithm-level analysis. 
$grid\_points$ in SP and BT also have a relatively small aDVF value (0.14 and 0.38 for SP and BT respectively in Figure~\ref{fig:aDVF_3tiers_profiling}).
Further study reveals that $grid\_points$ defines input problems for SP and BT. 
A small corruption of $grid\_points$ 
%change->changes by anzheng
can easily cause major changes in computation
caught by the fault propagation analysis. 

The data object $u$ in BT also has a relatively small aDVF value (0.82 in Figure~\ref{fig:aDVF_3tiers_profiling}).
Further study reveals that $u$ is read-only in our target code region
for matrix factorization and Jacobian, neither of which is friendly
for fault masking.
Furthermore, the major fault masking for $u$ comes from value shadowing,
and value shadowing only happens in a couple of the least significant bits 
of the operands that reference $u$, which further reduces the value of aDVF.
%also reduces fault masking.

(2) The data type is strongly correlated with fault masking.
Figure~\ref{fig:aDVF_3tiers_profiling} reveals that the integer data objects ($colidx$ in CG, $grid\_points$ in BT and SP, $m\_elemBC$ in LULESH) appear to be 
more sensitive to faults than the floating point data objects 
($u$ and $r$ in MG, $exp1$ and $plane$ in FT, $u$ and $rsd$ in LU, $m\_delv\_zeta$ in LULESH, and $rhoi$ in SP).
In HPC applications, the integer data objects are commonly employed to
define input problems and bound computation boundaries (e.g., $colidx$ in CG and $grid\_points$ in BT), 
or track computation status (e.g., $m\_elemBC$ in LULESH). Their corruption 
%these integer data objects
is very detrimental to the application correctness. 

(3) Operation-level fault masking is very common.
For many data objects, the operation-level fault masking contributes 
more than 70\% of the aDVF values. For $r$ in CG, $exp1$ in FT, and $rhoi$ in SP,
the contribution of the operation-level fault masking is close to 99\% (Figure~\ref{fig:aDVF_3tiers_profiling}).

Furthermore, the value shadowing is a very common operation level fault masking,
especially for floating point data objects (e.g., $u$ and $r$ in BT, $m\_delv\_zeta$ in LULESH, and $rhoi$ in SP in Figure~\ref{fig:aDVF_3classes_profiling}).
This finding has a very important indication for studying the application resilience.
In particular, the values of a data object can be different across different input problems. If the values of the data object are different, 
then the number of fault masking events due to the value shadowing will be different. 
Hence, we deduce that the application resilience
can be correlated with the input problems,
because of the correlation between the value shadowing and input problems. 
We must consider the input problems when studying the application resilience.
This conclusion is consistent with a very recent work~\cite{sc16:guo}.

(4) The contribution of the algorithm-level fault masking to the application resilience can be nontrivial.
For example, the algorithm-level fault masking contributes 19\% of the aDVF value for $u$ in MG and 27\% for $plane$ in FT (Figure~\ref{fig:aDVF_3tiers_profiling}).
The large contribution of algorithm-level fault masking in MG is consistent with
the results of existing work~\cite{mg_ics12}. 
For FT (particularly 3D FFT), the large contribution of algorithm-level fault masking in $plane$ (Figure~\ref{fig:aDVF_3tiers_profiling})
comes from frequent transpose and 1D FFT computations that average out 
or overwrite the data corruption.
CG, as an iterative solver, is known to have the algorithm-level fault masking
because of the iterative nature~\cite{2-shantharam2011characterizing}.
Interestingly, the algorithm-level fault masking in CG contributes most to the resilience of $colidx$ which is a vulnerable integer data object (Figure~\ref{fig:aDVF_3tiers_profiling}).

%Our study reveals the algorithm-level fault masking of CG from
%two perspectives. First, $a$ in CG, which is an array for intermediate results,
%has few algorithm-level fault masking (0.008\%);
%Second, $x$ in CG, which is a result vector, has 5.4\% of the aDVF value coming from the algorithm-level fault masking.
%This result indicates that the effects of the algorithm-level fault masking
%are not uniform across data objects. 

(5) Fault masking at the fault propagation level is small.
For all data objects, the contribution of the fault masking at the level of fault propagation is less than 5\% (Figure~\ref{fig:aDVF_3tiers_profiling}).
For 6 data objects ($r$ and $colidx$ in CG, $grid\_points$ and $u$ in BT, and 
$grid\_points$ and $rhoi$ in SP),  there is no fault masking at the level of fault propagation.
In combination with the finding 4, we conclude that once the fault
is propagated, it is difficult to mask it because of the contamination of
more data objects after fault propagation, and only the algorithm semantics can tolerate  propagated faults well. 
%This finding is consistent with our sensitivity analysis. 

(6) Fault masking by logical and comparison operations is small,
%For all data objects, the fault masking contributions due to logical and comparison operations are very small, 
comparing with the contributions of value shadowing and overwriting (Figure~\ref{fig:aDVF_3classes_profiling}). 
Among all data objects, 
the logical and comparison operations in $grid\_points$ in BT contribute the most (25\% contribution in Figure~\ref{fig:aDVF_fine_profiling}), 
because of intensive ICmp operations (integer comparison). %logical OR and SHL (left shifting).


(7) The resilience varies across data objects. %within the same application.
This fact is especially pronounced in two data objects $colidx$ and $r$ in CG (Figure~\ref{fig:aDVF_3tiers_profiling}).
 $colidx$ has aDVF much smaller than $r$, which means $colidx$ is much less resilient than $r$ (see finding 1 for a detailed analysis on $colidx$). 
Furthermore, $colidx$ and $r$ have different algorithm-level
fault masking (see finding 4 for a detailed analysis).

\begin{comment}
\textbf{Finding 7: The resilience of the same data objects varies across different applications.}
This fact is especially pronounced in BT and SP.
BT and SP address the same numerical problem but with different algorithms.
BT and SP have the same data objects, $qs$ and $rhoi$, but
$qs$ manifests different resilience in BT and SP.
This result is interesting, because it indicates that by using
different algorithms, we have opportunities to
improve the resilience of data objects.
\end{comment}

To further investigate the reasons for fault masking, 
we break down the aDVF results at the granularity of LLVM instructions,
based on the analyses at the levels of operation and fault propagation.
The results are shown in Figure~\ref{fig:aDVF_fine_profiling}.
%Because of the space limitation, 
%we only show one data object per benchmark, but each selected data object has the most diverse fault masking events within the corresponding benchmark.
%Based on Figure~\ref{fig:aDVF_fine_profiling}, we have another interesting finding.

(8) Arithmetic operations make a lot of contributions to fault masking.
%For $r$ in CG, $r$ in MG, $exp1$ in FT, $u$ in BT, $qs$ in SP, and $u$ in LU,
%the arithmetic operations, FMul (100\%), Add (16\%), FMul (85\%), 
%FMul (94\%), FMul (28\%), and FAdd (50\%)
For $r$ in CG, $u$ in BT, $plane$ and $exp1$ in FT, $m\_elemBC$ in LULESH, 
arithmetic operations (addition, multiplication, and division) contribute to almost 100\% of the fault masking (Figure~\ref{fig:aDVF_fine_profiling}).  
%(at the operation level and the fault propagation level).
%For $qs$ in SP and $u$ in LU, the store operation also makes
%important contributions as the arithmetic operations because of value overwriting.

\begin{figure*}
	\centering
	\includegraphics[width=0.77\textheight, height=0.23\textheight]{pie_chart.pdf}
	\vspace{-10pt}
	\caption{Breakdown of the aDVF results based on the analyses at the levels of operation and fault propagation}
    \vspace{-10pt}
	\label{fig:aDVF_fine_profiling}
\end{figure*}


\subsection{Sensitivity Study}
\label{sec:eval_sen}
%\textbf{change the fault propagation threshold and study the sensitivity of analysis to the threshold}
ARAT uses 10 as the default fault propagation analysis threshold. 
The fault propagation analysis will not go beyond 10 operations. Instead,
we will use deterministic fault injection after 10 operations. 
In this section, we study the impact of this threshold on the modeling accuracy. We use a range of threshold values and examine how the aDVF value varies and whether
the identification of fault masking varies. 
Figure~\ref{fig:sensitivity_error_propagation} shows the results for 
%add , after BT by anzheng
multiple data objects in CG, BT, and SP.
We perform the sensitivity study for all 16 data objects.
%in six benchmarks and two applications.
Due to the page space limitation, we only show the results for three data objects,
but we summarize the sensitivity study results for all data objects in this section.
%but other data objects in all benchmarks have the same trend.

Our results reveal that the identification of fault masking by tracking fault propagation is not significantly 
affected by the fault propagation analysis threshold. Even if we use a rather large threshold (50), 
the variation of aDVF values is 4.48\% on average among all data objects,
and the variation at each of the three levels of analysis (the operation level, fault propagation level,  and algorithm level) is less than 5.2\% on average. 
In fact, using a threshold value of 5 is sufficiently accurate in most of the cases (14 out of 16 data objects).
This result is consistent with our finding 5 (i.e., fault masking at the fault propagation level is small). %in most benchmarks).
However, we do find a data object ($m\_elementBC$ in LULESH) %and $exp1$ in FT) 
showing relatively high-sensitive (up to 15\% variation) to the threshold. For this uncommon data object, using 50 as the fault propagation path is sufficient. 

%In other words, even though using a larger threshold value can identify more error masking by tracking error 
%propagation, the implicit error masking induced by the error propagation is very limited.

\begin{figure}
		\begin{center}
		\includegraphics[width=0.48\textwidth,height=0.11\textheight]{sensi_study_gray.pdf}
		\vspace{-15pt}
		\caption{Sensitivity study for fault propagation threshold}
		\label{fig:sensitivity_error_propagation}
		\end{center}
\vspace{-15pt}
\end{figure}


\begin{comment}
\subsection{Comparison with the Traditional Random Fault Injection}
%\textbf{compare with the traditional fault injection to verify accuracy}
To show the effectiveness of our resilience modeling, we compare traditional random fault injection
and our analytical modeling. Figure~\ref{fig:comparison_fi} and Table~\ref{tab:comparison} show the results.
The figure shows the success rate of all random fault injection. The ``success'' means the application
outcome is verified successfully by the benchmarks and the execution does not have any segfault. The success rate is used as a metric
to evaluate the application resilience.

We use a data-oriented approach to perform random fault injection.
In particular, given a data object, for each fault injection test we trigger a bit flip
in an operand of a random instruction, and this operand must be a reference to the
target data object. We develop a tool based on PIN~\cite{pintool} to implement the above fault injection functionality.
For each data object, we conduct five sets of random fault injection tests, 
and each set has 200 tests (in total 1000 tests per data object). 
We show the results for CG and FT in this section, but we find that
the conclusions we draw from CG and FT are also valid for the other four benchmarks.


%\begin{table*}
%\label{tab:success_rate}
%\begin{centering}
%\renewcommand\arraystretch{1.1}
%\begin{tabular}{|c|c|c|c|c|c|c|}
%\hline 
%Success Rate (Difference) & Test set 1 & Test set 2 & Test set 3 & Test set 4 & Test set 5 & Average\tabularnewline
%\hline 
%\hline 
%CG-a & 66.1\% (11.7\%) & 68.5\% (15.7\%) & 56.7\% (4.21\%) & 61.3\% (3.57\%) & 43.3\% (26.8\%) & 59.2\%\tabularnewline
%\hline 
%CG-x & 99.2\% (2.2\%) & 98.6\% (1.5\%) & 96.5\% (0.63\%) & 97.8\% (0.64\%) & 93.6\% (3.7\%) & 97.1\%\tabularnewline
%\hline 
%CG-colidx & 36.8\% (12.7\%) & 49.6\% (17.8\%) & 40.2\% (4.6\%) & 52.6\% (24.9\%) & 31.4\% (25.4\%) & 42.1\%\tabularnewline
%\hline 
%FT-exp1 & 52.7\% (1.4\%) & 22.6\% (56.5\%) & 78.5\% (51.0\%) & 60.7\% (16.7\%) & 45.4\% (12.7\%) & 51.9\%\tabularnewline
%\hline 
%FT-plane & 82.1\% (2.5\%) & 79.3\% (5.6\%) & 99.5\% (18.2\%) & 93.2\% (10.7\%) & 66.8\% (20.6\%) & 84.2\%\tabularnewline
%\hline 
%\end{tabular}
%\par\end{centering}
%\caption{XXXXX}
%\end{table*}


\begin{table*}
\begin{centering}
\caption{\small The results for random fault injection. The numbers in parentheses for each set of tests (200 tests per set) are the success rate difference from the average success rate of 1000 fault injection tests.}
\label{tab:comparison}
\renewcommand\arraystretch{1.1}
\begin{tabular}{|c|p{2.2cm}|p{2.2cm}|p{2.2cm}|p{2.2cm}|p{2.2cm}|p{1.8cm}|}
\hline 
       %& Test set 1 & Test set 2 & Test set 3 & Test set 4 & Test set 5 & Average\tabularnewline
       & \hspace{13pt} Test set 1 \hspace{1pt}/  & \hspace{13pt} Test set 2 \hspace{1pt}/ & \hspace{13pt} Test set 3 \hspace{1pt}/ & \hspace{13pt} Test set 4 \hspace{1pt}/ & \hspace{13pt} Test set 5 \hspace{1pt}/ & Ave. of all test / \\
       & success rate (diff.) & success rate (diff.) & success rate (diff.) & success rate (diff.) & success rate (diff.) & \hspace{5pt} success rate \\
\hline 
\hline 
CG-a & 66.1\% (6.9\%) & 68.5\% (9.3\%) & 56.7\% (-2.5\%) & 61.3\% (2.1\%) & 43.3\% (-15.9\%) & 59.2\%\tabularnewline
\hline 
CG-x & 99.2\% (2.1\%) & 98.6\% (1.5\%) & 96.5\% (-0.6\%) & 97.8\% (0.7\%) & 93.6\% (-3.5\%) & 97.1\%\tabularnewline
\hline 
CG-colidx & 36.8\% (-5.3\%) & 49.6\% (7.5\%) & 40.2\% (-2.0\%) & 52.6\% (10.5\%) & 31.4\% (-10.7\%) & 42.1\%\tabularnewline
\hline 
FT-exp1 & 52.7\% (0.8\%) & 22.6\% (-29.3\%) & 78.5\% (26.6\%) & 60.7\% (8.8\%) & 45.4\% (-6.5\%) & 51.9\%\tabularnewline
\hline 
FT-plane & 82.1\% (-2.1\%) & 79.3\% (-4.9\%) & 99.5\% (15.3\%) & 93.2\% (9.0\%) & 66.8\% (-17.4\%) & 84.2\%\tabularnewline
\hline 
\end{tabular}
\par\end{centering}
\vspace{-0.4cm}
\end{table*}

\begin{figure}
	\begin{center}
		\includegraphics[width=0.48\textwidth,keepaspectratio]{verifi-study.png}
		\caption{The traditional random fault injection vs. ARAT}
		\label{fig:comparison_fi}
	\end{center}
\vspace{-0.7cm}
\end{figure}


We first notice from Table~\ref{tab:comparison} that 
%across 5 sets of random fault injection tests, there are big variances (up to 55.9\% in $exp1$ of FT) in terms of the success rate. 
the results of 5 test sets can be quite different from each other and from 1000 random fault inject tests (up to 29.3\%).
1000 fault injection tests provide better statistical significance than 200 fault injection tests.
We expect 1000 fault injection tests potentially provide higher accuracy to quantify the application resilience.
The above result difference is clearly an indication to the randomness of fault injection, and there
is no guarantee on the random fault injection accuracy.

%In Figure~\ref{fig:comparison_fi}, 
We compare the success rate of 1000 fault inject tests with the aDVF value (Fig.~\ref{fig:comparison_fi}). 
We find that the order of the success rate of the three data objects in CG (i.e., $colidx < a < x$) and the two data objects in FT 
(i.e., $exp1 < plane$) is the same as the order of the aDVF values of these data objects. 
%In fact, 1000 fault injection tests
%account for \textcolor{blue}{\textbf{xxx\%}} of total memory references to the data object,
%and provide better resilience quantification than 200 fault injection tests.
The same order (or the same resilience trend)
%between our approach and the random fault injection based on a large number of tests 
is a demonstration of the effectiveness of our approach.
Note that the values of the aDVF and success rate %for a data object
cannot be exactly the same (even if we have sufficiently large numbers of random fault injection), 
because aDVF and random fault injection quantify
the resilience based on different metrics.
Also, the random fault injection can miss some fault masking events that can be captured by our approach.

\end{comment}
\section{Related Work}
%\mz{We lack a comparison to this paper: https://arxiv.org/abs/2305.14877}
%\anirudh{refine to be more on-topic?}
\iffalse
\paragraph{In-Context Learning} As language models have scaled, the ability to learn in-context, without any weight updates, has emerged. \cite{brown2020language}. While other families of large language models have emerged, in-context learning remains ubiquitous \cite{llama, bloom, gptneo, opt}. Although such as HELM \cite{helm} have arisen for systematic evaluation of \emph{models}, there is no systematic framework to our knowledge for evaluating \emph{prompting methods}, and validating prompt engineering heuristics. The test-suite we propose will ensure that progress in the field of prompt-engineering is structured and objectively evaluated. 

\paragraph{Prompt Engineering Methods} Researchers are interested in the automatic design of high performing instructions for downstream tasks. Some focus on simple heuristics, such as selecting instructions that have the lowest perplexity \cite{lowperplexityprompts}. Other methods try to use large language models to induce an instruction when provided with a few input-output pairs \cite{ape}. Researchers have also used RL objectives to create discrete token sequences that can serve as instructions \cite{rlprompt}. Since the datasets and models used in these works have very little intersection, it is impossible to compare these methods objectively and glean insights. In our work, we evaluate these three methods on a diverse set of tasks and models, and analyze their relative performance. Additionally, we recognize that there are many other interesting angles of prompting that are not covered by instruction engineering \cite{weichain, react, selfconsistency}, but we leave these to future work.

\paragraph{Analysis of Prompting Methods} While most prompt engineering methods focus on accuracy, there are many other interesting dimensions of performance as well. For instance, researchers have found that for most tasks, the selection of demonstrations plays a large role in few-shot accuracy \cite{whatmakesgoodicexamples, selectionmachinetranslation, knnprompting}. Additionally, many researchers have found that even permuting the ordering of a fixed set of demonstrations has a significant effect on downstream accuracy \cite{fantasticallyorderedprompts}. Prompts that are sensitive to the permutation of demonstrations have been shown to also have lower accuracies \cite{relationsensitivityaccuracy}. Especially in low-resource domains, which includes the large public usage of in-context learning, these large swings in accuracy make prompting less dependable. In our test-suite we include sensitivity metrics that go beyond accuracy and allow us to find methods that are not only performant but reliable.

\paragraph{Existing Benchmarks} We recognize that other holistic in-context learning benchmarks exist. BigBench is a large benchmark of 204 tasks that are beyond the capabilities of current LLMs. BigBench seeks to evaluate the few-shot abilities of state of the art large language models, focusing on performance metrics such as accuracy \cite{bigbench}. Similarly, HELM is another benchmark for language model in-context learning ability. Rather than only focusing on performance, HELM branches out and considers many other metrics such as robustness and bias \cite{helm}. Both BigBench and HELM focus on ranking different language model, while fix a generic instruction and prompt format. We instead choose to evaluate instruction induction / selection methods over a fixed set of models. We are the first ever evaluation script that compares different prompt-engineering methods head to head. 
\fi

\paragraph{In-Context Learning and Existing Benchmarks} As language models have scaled, in-context learning has emerged as a popular paradigm and remains ubiquitous among several autoregressive LLM families \cite{brown2020language, llama, bloom, gptneo, opt}. Benchmarks like BigBench \cite{bigbench} and HELM \cite{helm} have been created for the holistic evaluation of these models. BigBench focuses on few-shot abilities of state-of-the-art large language models, while HELM extends to consider metrics like robustness and bias. However, these benchmarks focus on evaluating and ranking \emph{language models}, and do not address the systematic evaluation of \emph{prompting methods}. Although contemporary work by \citet{yang2023improving} also aims to perform a similar systematic analysis of prompting methods, they focus on simple probability-based prompt selection while we evaluate a broader range of methods including trivial instruction baselines, curated manually selected instructions, and sophisticated automated instruction selection.

\paragraph{Automated Prompt Engineering Methods} There has been interest in performing automated prompt-engineering for target downstream tasks within ICL. This has led to the exploration of various prompting methods, ranging from simple heuristics such as selecting instructions with the lowest perplexity \cite{lowperplexityprompts}, inducing instructions from large language models using a few annotated input-output pairs \cite{ape}, to utilizing RL objectives to create discrete token sequences as prompts \cite{rlprompt}. However, these works restrict their evaluation to small sets of models and tasks with little intersection, hindering their objective comparison. %\mz{For paragraphs that only have one work in the last line, try to shorten the paragraph to squeeze in context.}

\paragraph{Understanding in-context learning} There has been much recent work attempting to understand the mechanisms that drive in-context learning. Studies have found that the selection of demonstrations included in prompts significantly impacts few-shot accuracy across most tasks \cite{whatmakesgoodicexamples, selectionmachinetranslation, knnprompting}. Works like \cite{fantasticallyorderedprompts} also show that altering the ordering of a fixed set of demonstrations can affect downstream accuracy. Prompts sensitive to demonstration permutation often exhibit lower accuracies \cite{relationsensitivityaccuracy}, making them less reliable, particularly in low-resource domains.

Our work aims to bridge these gaps by systematically evaluating the efficacy of popular instruction selection approaches over a diverse set of tasks and models, facilitating objective comparison. We evaluate these methods not only on accuracy metrics, but also on sensitivity metrics to glean additional insights. We recognize that other facets of prompting not covered by instruction engineering exist \cite{weichain, react, selfconsistency}, and defer these explorations to future work. 

%\balance
\bibliographystyle{IEEEtran}
\bibliography{sample-base}


\end{document}
