\documentclass[12pt, reqno,oneside]{amsart}
\usepackage[utf8]{inputenc}

\usepackage{amssymb, amsmath, amsthm, wasysym, mathrsfs}
\usepackage{hyperref}
\usepackage{cleveref}
\usepackage[alphabetic,lite]{amsrefs}
%\usepackage{amscd}   % for commutative diagrams
\usepackage{tikz-cd}
\usepackage{fullpage}
\usepackage{comment}
\usepackage{thmtools}
%\usepackage[all]{xy} % for complicated commutative diagrams
\usepackage{setspace}
\usepackage{mathtools}
\usepackage{enumitem}
\usepackage{thm-restate}
\DeclarePairedDelimiter{\floor}{\lfloor}{\rfloor}
\usepackage{caption}
\usepackage{subcaption}
\usepackage{float}

    \setstretch{1.10}






% Theorems

\newtheorem{lemma}{Lemma}[section]
\newtheorem{lemma*}{Lemma}
\newtheorem{theorem}[lemma]{Theorem}
\newtheorem{proposition}[lemma]{Proposition}
\newtheorem{prop}[lemma]{Proposition}
\newtheorem{cor}[lemma]{Corollary}
\newtheorem{conj}[lemma]{Conjecture}
\newtheorem{claim}[lemma]{Claim}
\newtheorem{problem}[lemma]{Problem}
\newtheorem{question}[lemma]{Question}
\newtheorem{claim*}{Claim}
%\newtheorem{rmk}[lemma]{Remark}
%\newtheorem{remark}[lemma]{Remark}
\newtheorem{thm}[lemma]{Theorem}
\newtheorem{defn}[lemma]{Definition}
\newtheorem{example}[lemma]{Example}

\theoremstyle{definition}
\newtheorem{remark}[lemma]{Remark}
\newtheorem*{lem}{Acknowledgements}
\newtheorem{rmk}[lemma]{Remark}
\theoremstyle{plain}


    \newtheoremstyle{TheoremNum}
        {\topsep}{\topsep}              %%% space between body and thm
        {\itshape}                      %%% Thm body font
        {}                              %%% Indent amount (empty = no indent)
        {\bfseries}                     %%% Thm head font
        {.}                             %%% Punctuation after thm head
        { }                             %%% Space after thm head
        {\thmname{#1}\thmnote{ \bfseries #3}}%%% Thm head spec
    \theoremstyle{TheoremNum}
    \newtheorem{thmn}{Theorem}

%Characters
\newcommand{\Aff}{{\mathbb A}}
\newcommand{\G}{{\mathbb G}}
\newcommand{\Hh}{{\mathbb H}}
\newcommand{\PP}{{\mathbb P}}
\newcommand{\C}{{\mathbb C}}
\newcommand{\F}{{\mathbb F}}
\newcommand{\E}{{\mathbb E}}
\newcommand{\Q}{{\mathbb Q}}
\newcommand{\R}{{\mathbb R}}
\newcommand{\Z}{{\mathbb Z}}

\newcommand{\eps}{\epsilon}

\newcommand{\BB}{{\mathbb B}}
\newcommand{\BBl}{{\mathbb B}_{\ell, \infty}}
\newcommand{\BBp}{{\mathbb B}_{p, \infty}}
\newcommand{\WW}{{\mathbb W}}
\newcommand{\EE}{{\mathbb E}}

\newcommand{\Xbar}{{\overline{X}}}
\newcommand{\Qbar}{{\overline{\Q}}}
\newcommand{\Zhat}{{\hat{\Z}}}
\newcommand{\Zbar}{{\overline{\Z}}}
\newcommand{\kbar}{{\overline{k}}}
\newcommand{\Kbar}{{\overline{K}}}
\newcommand{\ksep}{{k^{\operatorname{sep}}}}
\newcommand{\kperf}{{k^{\operatorname{perf}}}}
\newcommand{\Fbar}{{\overline{\F}}}
\newcommand{\Xsep}{{X^{\operatorname{sep}}}}
\newcommand{\Vbar}{{\overline{V}}}
\newcommand{\Ybar}{{\overline{Y}}}

\newcommand{\Adeles}{{\mathbb A}}
\newcommand{\kk}{{\mathbf k}}

\newcommand{\elltilde}{\widetilde{\ell}}

\newcommand{\phibar}{{\overline{\phi}}}
\newcommand{\alphabar}{{\overline{\alpha}}}
\newcommand{\betabar}{{\overline{\beta}}}

\newcommand{\Phibar}{{\overline{\Phi}}}
\newcommand{\Dtilde}{{\widetilde{D}}}
\newcommand{\Ktilde}{{\widetilde{K}}}
\newcommand{\Ftilde}{{\widetilde{F}}}
\newcommand{\Rtilde}{{\widetilde{R}}}
\newcommand{\Phitilde}{{\widetilde{\Phi}}}
\newcommand{\dtilde}{{\widetilde{d}}}
\newcommand{\wtilde}{{\widetilde{w}}}

% mathcal characters
\newcommand{\calA}{{\mathcal A}}
\newcommand{\calB}{{\mathcal B}}
\newcommand{\calC}{{\mathcal C}}
\newcommand{\calD}{{\mathcal D}}
\newcommand{\calE}{{\mathcal E}}
\newcommand{\calF}{{\mathcal F}}
\newcommand{\calG}{{\mathcal G}}
\newcommand{\calH}{{\mathcal H}}
\newcommand{\calI}{{\mathcal I}}
\newcommand{\calJ}{{\mathcal J}}
\newcommand{\calK}{{\mathcal K}}
\newcommand{\calL}{{\mathcal L}}
\newcommand{\calM}{{\mathcal M}}
\newcommand{\calN}{{\mathcal N}}
\newcommand{\calO}{{\mathcal O}}
\newcommand{\calP}{{\mathcal P}}
\newcommand{\calQ}{{\mathcal Q}}
\newcommand{\calR}{{\mathcal R}}
\newcommand{\calS}{{\mathcal S}}
\newcommand{\calT}{{\mathcal T}}
\newcommand{\calU}{{\mathcal U}}
\newcommand{\calV}{{\mathcal V}}
\newcommand{\calW}{{\mathcal W}}
\newcommand{\calX}{{\mathcal X}}
\newcommand{\calY}{{\mathcal Y}}
\newcommand{\calZ}{{\mathcal Z}}
\newcommand{\OO}{{\mathcal O}}

% mathfrak characters
\newcommand{\fraka}{{\mathfrak a}}
\newcommand{\frakb}{{\mathfrak b}}
\newcommand{\frakc}{{\mathfrak c}}
\newcommand{\frakd}{{\mathfrak d}}
\newcommand{\frake}{{\mathfrak e}}
\newcommand{\frakf}{{\mathfrak f}}
\newcommand{\frakg}{{\mathfrak g}}
\newcommand{\frakh}{{\mathfrak h}}
\newcommand{\fraki}{{\mathfrak i}}
\newcommand{\frakj}{{\mathfrak j}}
\newcommand{\frakk}{{\mathfrak k}}
\newcommand{\frakl}{{\mathfrak l}}
\newcommand{\frakm}{{\mathfrak m}}
\newcommand{\frakn}{{\mathfrak n}}
\newcommand{\frako}{{\mathfrak o}}
\newcommand{\frakp}{{\mathfrak p}}
\newcommand{\frakq}{{\mathfrak q}}
\newcommand{\frakr}{{\mathfrak r}}
\newcommand{\fraks}{{\mathfrak s}}
\newcommand{\frakt}{{\mathfrak t}}
\newcommand{\fraku}{{\mathfrak u}}
\newcommand{\frakv}{{\mathfrak v}}
\newcommand{\frakw}{{\mathfrak w}}
\newcommand{\frakx}{{\mathfrak x}}
\newcommand{\fraky}{{\mathfrak y}}
\newcommand{\frakz}{{\mathfrak z}}

\newcommand{\frakA}{{\mathfrak A}}
\newcommand{\frakB}{{\mathfrak B}}
\newcommand{\frakC}{{\mathfrak C}}
\newcommand{\frakD}{{\mathfrak D}}
\newcommand{\frakE}{{\mathfrak E}}
\newcommand{\frakF}{{\mathfrak F}}
\newcommand{\frakG}{{\mathfrak G}}
\newcommand{\frakH}{{\mathfrak H}}
\newcommand{\frakI}{{\mathfrak I}}
\newcommand{\frakJ}{{\mathfrak J}}
\newcommand{\frakK}{{\mathfrak K}}
\newcommand{\frakL}{{\mathfrak L}}
\newcommand{\frakM}{{\mathfrak M}}
\newcommand{\frakN}{{\mathfrak N}}
\newcommand{\frakO}{{\mathfrak O}}
\newcommand{\frakP}{{\mathfrak P}}
\newcommand{\frakQ}{{\mathfrak Q}}
\newcommand{\frakR}{{\mathfrak R}}
\newcommand{\frakS}{{\mathfrak S}}
\newcommand{\frakT}{{\mathfrak T}}
\newcommand{\frakU}{{\mathfrak U}}
\newcommand{\frakV}{{\mathfrak V}}
\newcommand{\frakW}{{\mathfrak W}}
\newcommand{\frakX}{{\mathfrak X}}
\newcommand{\frakY}{{\mathfrak Y}}
\newcommand{\frakZ}{{\mathfrak Z}}

\newcommand{\pp}{{\mathfrak p}}
\newcommand{\qq}{{\mathfrak q}}
\newcommand{\mm}{{\mathfrak m}}

\newcommand{\frakabar}{\overline{\mathfrak{a}}}
\newcommand{\frakbbar}{\overline{\mathfrak{b}}}
\newcommand{\frakcbar}{\overline{\mathfrak{c}}}
\newcommand{\frakqbar}{\overline{\mathfrak{q}}}
\newcommand{\frakRtilde}{\widetilde{\mathfrak{R}}}

\newcommand{\hideqed}{\renewcommand{\qed}{}}

% Math operators
\DeclareMathOperator{\HH}{H}
\DeclareMathOperator{\tr}{tr}
\DeclareMathOperator{\Tr}{Tr}
\DeclareMathOperator{\trdeg}{tr deg}
\DeclareMathOperator{\lcm}{lcm}
\DeclareMathOperator{\supp}{supp}
\DeclareMathOperator{\Frob}{Frob}
\DeclareMathOperator{\coker}{coker}
\DeclareMathOperator{\rk}{rk}
\DeclareMathOperator{\Char}{char}
\DeclareMathOperator{\inv}{inv}
\DeclareMathOperator{\nil}{nil}
\DeclareMathOperator{\im}{im}
\DeclareMathOperator{\re}{Re}
\DeclareMathOperator{\End}{End}
\DeclareMathOperator{\END}{\bf End}
\DeclareMathOperator{\Lie}{Lie}
\DeclareMathOperator{\Hom}{Hom}
\DeclareMathOperator{\Ext}{Ext}
\DeclareMathOperator{\HOM}{\bf Hom}
\DeclareMathOperator{\Aut}{Aut}
\DeclareMathOperator{\Gal}{Gal}
\DeclareMathOperator{\Ind}{Ind}
\DeclareMathOperator{\Cor}{Cor}
\DeclareMathOperator{\Res}{Res}
\DeclareMathOperator{\Norm}{Norm}
\DeclareMathOperator{\Br}{Br}
\DeclareMathOperator{\Gr}{Gr}
\DeclareMathOperator{\cd}{cd}
\DeclareMathOperator{\scd}{scd}
\DeclareMathOperator{\Sel}{Sel}
\DeclareMathOperator{\Cl}{Cl}
\DeclareMathOperator{\divv}{div}
\DeclareMathOperator{\ord}{ord}
\DeclareMathOperator{\Sym}{Sym}
\DeclareMathOperator{\Div}{Div}
\DeclareMathOperator{\Pic}{Pic}
\DeclareMathOperator{\Jac}{Jac}
\DeclareMathOperator{\Num}{Num}
\DeclareMathOperator{\PIC}{\bf Pic}
\DeclareMathOperator{\Spec}{Spec}
\DeclareMathOperator{\SPEC}{\bf Spec}
\DeclareMathOperator{\Proj}{Proj}
\DeclareMathOperator{\Frac}{Frac}
\DeclareMathOperator{\Az}{Az}
\DeclareMathOperator{\ev}{ev}
\DeclareMathOperator{\PGL}{PGL}
\DeclareMathOperator{\sep}{sep}
\DeclareMathOperator{\tors}{tors}
\DeclareMathOperator{\et}{et}
\DeclareMathOperator{\fppf}{fppf}
\DeclareMathOperator{\descent}{descent}
\DeclareMathOperator{\N}{N}
\DeclareMathOperator{\red}{red}
\DeclareMathOperator{\res}{res}
\DeclareMathOperator{\Princ}{Princ}
\DeclareMathOperator{\rank}{rank}
\DeclareMathOperator{\Mat}{M}
\DeclareMathOperator{\Disc}{Disc}
\DeclareMathOperator{\SL}{SL}
\DeclareMathOperator{\GL}{GL}
\DeclareMathOperator{\Sp}{Sp}
\DeclareMathOperator{\M}{M}
\DeclareMathOperator{\CM}{CM}
\DeclareMathOperator{\disc}{disc}
\DeclareMathOperator{\id}{id}
\DeclareMathOperator{\val}{val}
\DeclareMathOperator{\cond}{cond}
\DeclareMathOperator{\GCD}{GCD}
\DeclareMathOperator{\Sgn}{Sgn}
\DeclareMathOperator{\Diag}{Diag}
\DeclareMathOperator{\gen}{gen}
\DeclareMathOperator{\diag}{diag}
\DeclareMathOperator{\Id}{Id}
\DeclareMathOperator{\Vol}{Vol}
\DeclareMathOperator{\perf}{perf}

% Commands

\newcommand{\isom}{\simeq}
\newcommand{\la}{\langle}
\newcommand{\ra}{\rangle}
\newcommand{\lideal}{\langle}
\newcommand{\rideal}{\rangle}
\newcommand{\into}{\hookrightarrow}

\newcommand{\coeffset}{\mathfrak W}

\numberwithin{equation}{section}
\numberwithin{table}{section}
\setcounter{tocdepth}{1}
\newcommand{\defi}[1]{\textsf{#1}} % for defined terms


\title{Profinite rigidity and the congruence subgroup problem for mapping class groups}
\author{Tamunonye Cheetham-West}
\date{Fall 2023}
\address{Department of Mathematics \\ Yale University \\ New Haven, CT, 06511}
  \email{tamunonye.cheetham-west@yale.edu}

\begin{document}

\maketitle
\begin{abstract}
    We show that all hyperbolic surface bundles over the circle with fibers of genus zero, one, or two are distinguished by the finite quotients of their fundamental groups among all 3-manifold groups. In general, we show that a conjecture of Ivanov implies that all fibered hyperbolic 3-manifolds have fundamental groups that are profinitely rigid among 3-manifold groups. 
\end{abstract}
\bibliographystyle{alpha}


\section{Introduction}
\noindent For a compact, connected 3-manifold $M$, the profinite completion $\widehat{\pi_1(M)}$ is the inverse limit of the system of finite quotients of $\pi_1(M)$. When $\widehat{\pi_1(M)}\cong\widehat{\pi_1(N)}$ implies that $N$ is homeomorphic to $M$, we say that $\pi_1(M)$ is {\it profinitely rigid among 3-manifold groups}. 
\medbreak Profinite completions of 3-manifold groups are known to determine certain properties of the 3-manifolds that produce them. For example, if two compact 3-manifolds $M$ and $N$ have $\widehat{\pi_1(M)}\cong\widehat{\pi_1(N)}$, it is a consequence of work of Lott and L{\"u}ck \cite{Luck1995} \cite{LuckApprox} on the first $L^2-$Betti number of compact 3-manifolds that $M$ is irreducible if and only if $N$ is. Furthermore, for an irreducible manifold, Wilton and Zalesskii \cite{WZ1} prove that the profinite completion determines whether a manifold is finite-volume hyperbolic. Remarkably, Liu \cite{Y} proved that any set of finite-volume hyperbolic 3-manifolds whose fundamental groups have a fixed common profinite completion is always finite. 
\medbreak The main focus of this paper is the fundamental groups of hyperbolic 3-manifolds that fiber over the circle. In that regard, Jaikin-Zapirain \cite{JZ} proved that being fibered is a profinite invariant. Our main results shed light on when fibered hyperbolic 3-manifolds have fundamental groups that are profinitely rigid among 3-manifold groups. To state our results we establish some notation and terminology. 
\medbreak Throughout, $\Sigma_{g,n}$ will refer to a genus $g$ surface with $n$ punctures. The {\it Congruence Subgroup Problem} asks whether every finite quotient of the mapping class group $Mod(\Sigma_{g,n})$ factors through a reduction homomorphism $Mod(\Sigma_{g,n})\to Out(\pi_1(\Sigma_{g,n}))/K$ for $K<\pi_1(\Sigma_{g,n})$ a characteristic finite index subgroup. When this is true for some surface $\Sigma_{g,n}$, we say that $Mod(\Sigma_{g,n})$ has the {\it Congruence Subgroup Property (CSP)}. Bridson, Reid, and Wilton \cite{BRW} showed that whenever $Mod(\Sigma_{g,n})$ has CSP and is {\it omnipotent} (Definition 3.2 \cite{omnipotence}), the hyperbolic $\Sigma_{g,n}$ bundles over $S^1$ with first Betti number 1 are distinguished from one another by finite quotients. In this way, they proved that the fundamental groups of $\Sigma_{1,1}$ bundles are profinitely rigid among 3-manifold groups. The main theorem of this paper extends \cite{BRW} and in particular shows how the hypotheses of omnipotence and first Betti number 1 can be dropped. 
\begin{theorem}\label{csp+omni}
Let $M$ be a hyperbolic $\Sigma_{g,n}$ bundle over $S^1$. If $Mod(\Sigma_{g,n})$ has the Congruence Subgroup Property, $\pi_1(M)$ is profinitely rigid among 3-manifold groups. 
\end{theorem}
Using Theorem~\ref{csp+omni}, we prove the following:
\begin{theorem}\label{profiniterig}
Let $M$ be a hyperbolic $\Sigma_{g,n}$ bundle over $S^1$ with $g=0,1,2$. Then $\pi_1(M)$ is profinitely rigid among 3-manifold groups. 
\end{theorem}
Theorem~\ref{profiniterig} gives many new examples of closed and cusped hyperbolic 3-manifolds that are distinguished from every other compact 3-manifold by the profinite completions of their fundamental groups because they are fibered 3-manifolds with some fiber having genus $\leq 2$ (see Figure~\ref{fig:knotslinks} for some examples of fibered links with some fiber having genus $\leq 2$). 
\begin{figure}[h]
     \centering
     \begin{subfigure}[b]{0.3\textwidth}
         \centering
         \includegraphics[scale=0.09]{six_two.png}
         \caption{The knot $6_2$ \cite{knotinfo}}
         \label{fig:6_2}
     \end{subfigure}
     \hspace{0.5em}
     \begin{subfigure}[b]{0.3\textwidth}
         \centering
         \includegraphics[scale=0.09]{six_three.png}
         \caption{The knot $6_3$ \cite{knotinfo}}
         \label{fig:6_3}
    \end{subfigure}
     \hspace{0.5em}
     \begin{subfigure}[b]{0.3\textwidth}
         \centering
         \includegraphics[scale=0.60]{whitehead.png}
         \caption{The Whitehead link $5^2_1$ (Example 1 \S 2, p 114 \cite{Thurston1986ANF})}
         \label{fig:whitehead}
     \end{subfigure}
     \hspace{0.5em}
     \begin{subfigure}[b]{0.3\textwidth}
         \centering
         \includegraphics[scale=0.27]{magic.png}
         \caption{The magic manifold $6^3_1$ (e.g. Lemma 2.6(1) \cite{EikoKin})}
         \label{magic}
     \end{subfigure}
     \hspace{0.5em}
     \begin{subfigure}[b]{0.3\textwidth}
         \centering
         \includegraphics[scale=0.5]{borromean.png}
         \caption{The Borromean rings $6^3_2$ (Example 2 \S 2, p 114 \cite{Thurston1986ANF})}
         \label{fig:borromean}
     \end{subfigure}
     \caption{Some fibered, hyperbolic knots and links in $S^3$ with genus $\leq 2$ fiber determined by the profinite completions of their fundamental groups.}
     \label{fig:knotslinks}
\end{figure}
\medbreak Ivanov \cite{Ivanov} (Problem 2.10 \cite{KirbyProblemList}) has conjectured that for all surfaces $\Sigma_{g,n}$ with $\chi(\Sigma_{g,n})<0$, $Mod(\Sigma_{g,n})$ has the Congruence Subgroup Property (stated as Conjecture~\ref{Ivanov}). Combined with the Virtual Fibering Theorem of Agol and Wise \cite{AgolHaken}\cite{WiseHaken}, the methods of Theorem~\ref{csp+omni} are sufficient to prove
\begin{theorem}\label{assumingIvanov2}
Let $M$ and $N$ be finite-volume hyperbolic 3-manifolds with $\widehat{\pi_1(M)}\cong\widehat{\pi_1(N)}$. Assuming CSP, 
\begin{enumerate}
    \item $M\cong N$ if $M$ is fibered. 
    \item $M$ and $N$ are commensurable 3-manifolds with the same volume. 
    \item $M$ and $N$ cover a common hyperbolic 3-orbifold.
    \item The size of the profinite genus (see Definition~\ref{profgenus}) $|\mathcal{G}_3(M)|\leq C(M)$ where $C$ depends only on the geometry of a minimal degree, fibered, regular cover of $M$.
\end{enumerate}
\end{theorem}
We also have the following immediate corollary of Theorem~\ref{assumingIvanov2} for $L-$space knots using a theorem of Ni \cite{Ni}.
\begin{cor}
    Let $K\subset S^3$ be a hyperbolic $L-$space knot. Assuming CSP, the knot exterior $E(K)$ is distinguished up to homeomorphism from all other compact 3-manifolds by the profinite completion of $\pi_1(E(K))$. 
    \begin{proof}
    Since $K$ is an $L-$space knot, $E(K)$ fibers over the circle (Corollary 1.3, \cite{Ni}), and so we can apply Theorem~\ref{assumingIvanov2} to conclude that $\pi_1(E(K))$ is profinitely rigid among 3-manifold groups.
    \end{proof}
\end{cor}
Theorem~\ref{csp+omni} is proven in part by extending Theorem 1.2 \cite{Y} to show that the profinite completion of a fibered, hyperbolic 3-manifold group recognizes the topological type of a fiber. This is clear from \cite{Y} for closed surfaces. 
\begin{theorem}\label{toptype}
 Let $M$ and $N$ be finite-volume hyperbolic manifolds with $\widehat{\pi_1(M)}\cong\widehat{\pi_1(N)}$. Liu's Thurston-norm and fiber class preserving isomorphism $H^1(N,\Z)\to H^1(M,\Z)$ induced by an isomorphism $\Phi:\widehat{\pi_1(M)}\to\widehat{\pi_1(N)}$ sends fibered classes to fibered classes where the corresponding fiber surfaces have the same topological type.
\end{theorem}
 


\begin{lem}
    {\it The author thanks Autumn Kent, Chris Leininger, Rylee Lyman, Mark Pengitore, and Zhiyi Zhang for helpful conversations about this project. The author thanks Alan Reid, Ben McReynolds, and Ryan Spitler for helpful conversations and comments on earlier drafts of this paper. The author is also grateful for correction and comments from Martin Bridson and Biao Ma who pointed out mistakes in a previous version of this paper.} 
\end{lem}


    
\section{Preliminaries}\label{sec:prelim}
Let $\Sigma$ be a finite-type surface. Given an orientation-preserving homeomorphism $f:\Sigma\to \Sigma$, the mapping torus of this homomorphism, $M_f$ is a 3-manifold $(\Sigma\times [0,1])/(x,0)\sim (f(x),1)$. By projecting to the second factor of the product, the 3-manifold $M_f$ is a fibration with base space $S^1$ and fiber $\Sigma$. The mapping torus admits a hyperbolic metric exactly when the isotopy class of $f$ is {\it pseudo-Anosov} \cite{ThurstonFiber}. 
At the level of fundamental group, $\pi_1(M_f)$ is a semidirect product $\pi_1(\Sigma)\rtimes_{f_*}\Z$, where $f_*:\pi_1(\Sigma)\to \pi_1(\Sigma)$ is the automorphism induced on fundamental group by $f$. It is a theorem of Stallings \cite{StallingsFiber} that whenever $\pi_1(M)\cong A\rtimes\Z$ for $A$ a finitely generated normal subgroup, then $M$ fibers over the circle. 
\medbreak The {\it Thurston norm} \cite{Thurston1986ANF} on the homology of an orientable irreducible 3-manifold $M$ with empty or torus boundary is a seminorm $|\cdot|$ on $H_2(M,\partial M;\R)$. For $\beta\in H_2(M,\partial M;\Z)$ $$|\beta|=\min\{-\chi(S)\,|\,[S]=\beta\}$$ where $S$ is a system of surfaces $S=S_1\cup \dots\cup S_n$in $M$ representing the class $\beta$, and $-\chi(S)=\sum_{i=1}^n\max\{0,-\chi(S_i)\}$ where $\chi(S_i)$ is the Euler characteristic of $S_i$. By Poincar{\'e} duality, there is a seminorm on $H^1(M;\R)$, and any (integral) cohomology class that represents a fibration of $M$ lies in the cone over a top dimensional face of the unit norm ball called a {\it fibered face} of $M$. When $M$ is hyperbolic, the Thurston seminorm is a norm (Theorem 1 \cite{Thurston1986ANF}).
\medbreak We now consider the profinite completion $\widehat{\pi_1(M_f)}$ of the fundamental group of a fibered 3-manifold $\widehat{\pi_1(M_f)}$. One can check that the subspace topology induced on the fiber subgroup $\pi_1(\Sigma)$ coincides with the profinite topology on this subgroup (Lemma 2.2 \cite{BR}), and so there is an exact sequence
$$1\to\widehat{\pi_1(\Sigma)}\to \widehat{\pi_1(M_f)}\to\hat{\Z}\to 1$$
\section{The Congruence Subgroup Property for mapping class groups of genus at most 2}\label{modS}
\noindent Recall that the Mapping Class Group of a finite type surface $\Sigma$, here denoted as $Mod(\Sigma)$, is the group of orientation-preserving homeomorphisms of $\Sigma$ up to isotopy. 
\begin{defn}
    A principal congruence quotient of $Mod(\Sigma)$ is the image of $Mod(\Sigma)$ under the canonical map $$Mod(\Sigma)\to Out(\pi_1(\Sigma))\to Out(\pi_1(\Sigma)/K)$$ where $K<\pi_1(\Sigma)$ is a characteristic subgroup of finite index. A congruence quotient $$Mod(\Sigma)\to Q$$ is a finite quotient that factors through a principal congruence quotient of $Mod(\Sigma)$.
\end{defn}
\begin{defn}
    The group $Mod(\Sigma)$ has the Congruence Subgroup Property when every finite quotient of $Mod(\Sigma)$ is a congruence quotient.
\end{defn}
\begin{conj}[Ivanov \cite{Ivanov}, Problem 2.10 \cite{KirbyProblemList}]\label{Ivanov}
    For $\Sigma$ a finite-type surface with $\chi(\Sigma)<0$, $Mod(\Sigma)$ has the Congruence Subgroup Property.
\end{conj}
\begin{theorem}\label{Mod04}
The groups $Mod(\Sigma_{g,n})$ have the Congruence Subgroup Property when $g=0,1,2$ and $(g,n)\ne (1,0)$. 
\begin{proof}
By a theorem of Diaz-Donagi-Harbater \cite{CSPDDH}, the groups $Mod(\Sigma_{0,n})$ have the Congruence Subgroup Property (different proofs of this theorem are given in Section 4 \cite{Kent} and \cite{McreynoldsThurston}). In genus one (for $n>1$), a theorem of Asada \cite{Asada} establishes the Congruence Subgroup Property for $Mod(\Sigma_{1,n})$ (see also \cite{BuxErshRapinchuk}). Boggi (\cite{Boggi},\cite{Boggi2}) proved the genus 2 cases. 
\end{proof}
\end{theorem}
We can now prove Theorem~\ref{profiniterig} assuming Theorem~\ref{csp+omni}
\begin{proof}[Proof of Theorem~\ref{profiniterig}]
Let $M$ be a hyperbolic $\Sigma_{g,n}$-bundle over $S^1$ with $g=0,1,2$. By Theorem~\ref{Mod04}, $Mod(\Sigma_{g,n})$ has the Congruence Subgroup Property, and so by Theorem~\ref{csp+omni}, $M$ is distinguished from every other compact 3-manifold by the profinite completion of its fundamental group. 
\end{proof}

\begin{defn}\label{profgenus}
    The profinite genus $\mathcal{G}_3(M)$ of a 3-manifold $M$ is the set of 3-manifolds $N$ for which $\widehat{\pi_1(N)}\cong\widehat{\pi_1(M)}$.
\end{defn}
We recall that the main theorem of Liu \cite{Y} states that for $M$ a finite-volume hyperbolic 3-manifold, $|\mathcal{G}_3(M)|<\infty$. This finiteness result for finite-volume hyperbolic 3-manifolds is proven by first establishing that fibered hyperbolic 3-manifolds are determined up to finite ambiguity by the profinite completions of their fundamental groups. The Virtual Fibering Theorem of Agol and Wise \cite{AgolHaken}\cite{WiseHaken} is then used to promote this finite ambiguity to all finite-volume hyperbolic 3-manifolds. Proving profinite rigidity results for the fundamental group of a fibered, hyperbolic 3-manifold $M'$ allows us to describe the profinite genus of any manifold $M$ that $M'$ regularly covers.  
\begin{cor}\label{keycor}
    Let $M$ be a finite volume hyperbolic 3-manifolds that admits a degree $d<\infty$ regular cover $M'\to M$ with $M'$ a hyperbolic $\Sigma_{g,n}$ bundle $g=0,1,2$. Then
    \begin{enumerate}
        \item Any manifold $N$ with $\widehat{\pi_1(N)}\cong\widehat{\pi_1(M)}$ is commensurable to $M$ and has the same volume as $M$.
        \item If a manifold $N$ has $\widehat{\pi_1(N)}\cong\widehat{\pi_1(M)}$ , then $M$ and $N$ cover a common hyperbolic 3-orbifold. 
        \item The number $|\mathcal{G}_3(M)|\leq C(M')$ where $C(M')$ is the number of conjugacy classes of subgroups isomorphic to $\pi_1(M)/\pi_1(M')$ in $Isom(M')$. 
    \end{enumerate}
\begin{proof}
    \begin{enumerate}
        \item From the correspondence between finite-index subgroups of $\pi_1(N)$ and open subgroups of $\widehat{\pi_1(N)}\cong\widehat{\pi_1(M)}$ (Proposition 3.2.2 \cite{RZ}), it follows that $N$ has a degree $d$ finite-sheeted cover $N'$ with $\widehat{\pi_1(N')}\cong\widehat{\pi_1(M')}$ corresponding to $M'$. Applying \cite{WZ1},\cite{JZ} to $N'$, it follows that $N'$ is a hyperbolic 3-manifold that fibers over the circle with $\Sigma_{g,n}$ fiber ($g=0,1,2)$. By Theorem~\ref{csp+omni}, $N'\cong M'$, and therefore, $M$ and $N$ are commensurable. Moreover, since $M$ and $N$ have a common $d-$sheeted cover $M'$, $vol(M)=vol(M')/d=vol(N)$. 
        \item From (1), any finite-volume manifold $N$ with 
        $\widehat{\pi_1(N)}\cong\widehat{\pi_1(M)}$ has $M'$ as a $d-$sheeted regular cover. It follows that $\pi_1(M')<\pi_1(N)<Aut(\pi_1(M'))$ and $Aut(\pi_1(M'))$ is a lattice in $Isom(\Hh^3)$ by Mostow-Prasad rigidity \cite{Mostow}\cite{Prasad}. Since $\pi_1(M)$ and $\pi_1(N)$ are both subgroups of $Aut(\pi_1(M'))$, $M$ and $N$ both cover the hyperbolic 3-orbifold $\Hh^3/Aut(\pi_1(M'))$. 
        \item The image in $Out(\pi_1(M'))\cong Isom(M')$ of $\pi_1(N)<Aut(\pi_1(M'))$ (up to conjugation in $Aut(\pi_1(M'))$) is a subgroup isomorphic to $\pi_1(M)/\pi_1(M')$ (up to conjugacy). 
    \end{enumerate}
\end{proof}      
\end{cor}

We can now prove Theorem~\ref{assumingIvanov2} assuming Theorem~\ref{csp+omni}.
\smallbreak
\begin{thmn}[\ref{assumingIvanov2}]
    Let $M$ and $N$ be finite volume hyperbolic 3-manifolds with $\widehat{\pi_1(M)}\cong\widehat{\pi_1(N)}$. Assuming CSP,
    \begin{enumerate}
        \item $M\cong N$ if $M$ is fibered. 
        \item $M$ and $N$ are commensurable 3-manifolds with the same volume. 
        \item $M$ and $N$ cover a common hyperbolic 3-orbifold.
        \item For any finite-volume hyperbolic 3-manifold $M$, $|\mathcal{G}_3(M)|\leq C(M)$ where $C$ depends only on the geometry of a minimal degree regular cover of $M$ that fibers over $S^1$.
    \end{enumerate}
\begin{proof}
    \begin{enumerate}
        \item This is a direct application of Theorem~\ref{csp+omni}. 
        \item By the Virtual Fibering theorem of Agol and Wise \cite{AgolHaken}\cite{WiseHaken}, every finite-volume hyperbolic manifold $M$ has a finite-sheeted, fibered cover $M'$. By Part (1) above, $M'$ is determined by the profinite completion of its fundamental group. If $N$ and $M$ have $\widehat{\pi_1(N)}\cong\widehat{\pi_1(M)}$, then $N$ and $M$ are both finitely covered by $M'$ with the same covering degree. Thus, $M$ and $N$ are commensurable manifolds with the same volume. 
        \item The proof is the same as that of Corollary~\ref{keycor}(2). In particular, by passing to a finite regular cover $M'\to M$ and applying Part (1), we have that $M'\to N$, and that $\pi_1(N)<Aut(\pi_1(M'))$ which is a lattice in $Isom(\Hh^3)$. Thus, we can conclude that $M$ and $N$ cover the 3-orbifold $\Hh^3/Aut(\pi_1(M'))$.
        \item The proof is the same as that of Corollary~\ref{keycor}(3). We use the Virtual Fibering Theorem \cite{AgolHaken}\cite{WiseHaken} to obtain a finite-sheeted regular cover $M'\to M$ that fibers over the circle. Theorem~\ref{csp+omni} implies that $\pi_1(M')$ is profinitely rigid among 3-manifold groups, and therefore we can conclude as in the proof of Corollary~\ref{keycor}(2) that $|\mathcal{G}_3(M)|<C(M)$ where $C(M)$ is the number of conjugacy classes of subgroups isomorphic to $\pi_1(M)/\pi_1(M')$ in $Isom(M')\cong Out(\pi_1(M'))$.
    \end{enumerate}
\end{proof}
\end{thmn}
       


\section{Detecting topological type of fibers using profinite completions}
In this section, we will prove Theorem~\ref{toptype}. On the way to proving the finiteness theorem, Liu \cite{Y} Liu proves the following theorems that are crucial for this work
\begin{theorem}[\cite{Y}, Theorem 1.2, Theorem 1.3]\label{liuthm}
    Let $M, N$ be finite volume hyperbolic 3-manifolds and let $\Phi:\widehat{\pi_1(M)}\to\widehat{\pi_1(N)}$ be an isomorphism between the profinite completions of their fundamental groups. The isomorphism $\Phi$ induces an isomorphism $\Phi_*:\widehat{H}_1(M;\Z)\to \widehat{H}_1(N;\Z)$ with $\Phi_*=\hat{h}\circ \mu$ where $h:H_1(M;\Z)\to H_1(N;\Z)$ and $\mu$ denotes the scalar multiplication by a unit $\mu\in\hat{\Z}^\times$. The dual homomorphism $h^*:H^1(N;\Z)\to H^1(M;\Z)$ is Thurston-norm preserving and sends fibered classes of $N$ to fibered classes of $M$.
\end{theorem}
\begin{theorem}[\cite{Y}, Corollary 6.2, Corollary 6.3]\label{liuthm2}
    Let $M, N$ be finite volume hyperbolic 3-manifolds and let $\Phi:\widehat{\pi_1(M)}\to\widehat{\pi_1(N)}$ be an isomorphism between the profinite completions of their fundamental groups. For any connected fiber surface $\Sigma_M$ for $M$, there is a connected fiber surface $\Sigma_N$ for $N$ such that $\Phi(\widehat{\pi_1(\Sigma_M)})\cong\widehat{\pi_1(\Sigma_N)}$. Furthermore, $\chi(\Sigma_M)=\chi(\Sigma_N)$.
\end{theorem}
To obtain the correspondence of fibers with the same topological type we analyze surface automorphisms and their induced actions on characteristic quotients of the surface groups. Then let $\Sigma=\Sigma_{g,n}$ be a finite-type genus $g$ surface with $n$ punctures and $f:\pi_1(\Sigma)\to\pi_1(\Sigma)$ an automorphism. Assume $f$ is induced by a pseudo-Anosov. We can replace $f$ with $f^m$ for some integer $m$ to obtain a pseudo-Anosov that fixes each puncture of $\Sigma$ (that is, fixes the conjugacy classes in $\pi_1(\Sigma)$ of peripheral loops around each puncture of $\Sigma$). We define the following numbers:
\begin{align*}
    n_f &=\text{number of fixed primitive conjugacy classes of } f \text{ in }\pi_1(\Sigma)\\
    \hat{n}_f &=\text{number of fixed primitive conjugacy classes of }\hat{f} \text{ in } \widehat{\pi_1(\Sigma)}
\end{align*}
\begin{prop}\label{keyprop}
In the setup above, for a pseudo-Anosov $f$, $n=n_f=\hat{n}_f$.
\begin{proof}
Since a pseudo-Anosov is not reducible by definition, $n=n_f$. Upon completion, each fixed conjugacy class of $f$ in $\pi_1(S)$ will give a fixed conjugacy class of $\hat{f}$ in $\widehat{\pi_1(\Sigma)}$, hence $n_f\leq \hat{n}_f$. To see that $\hat{n}_f\leq n_f$, we observe that for any element $\alpha$ of a fixed conjugacy class of $\hat{f}$ in $\widehat{\pi_1(\Sigma)}$, we can choose $t\in\widehat{\pi_1(M_{f})}$, such that $t$ and $\alpha$ commute. The element $t$ can be chosen to be a pre-image of a unit $\xi\in\hat{\Z}$ (which is a topological generator for $\hat{\Z}$) under the fixed epimorphism $\widehat{\pi_1(M_{f})}\to\hat{\Z}$). Let $H=\overline{\langle\alpha,t\rangle}$ be the closed abelian group generated by $\alpha$ and $t$. 
\par We argue that this subgroup $H$ is not procyclic. First, a single topological generator $s$ for this group cannot live in the kernel $\widehat{\pi_1(\Sigma)}$ of the epimorphism $\widehat{\pi_1(M_{f})}\twoheadrightarrow \hat{\Z}$ because $t$ maps non-trivially under this epimorphism. If, on the other hand, $s$ maps non-trivially to $\hat{\Z}$ under the fixed epimorphism above, the subgroup topologically generated by $s$ also maps non-trivially to $\hat{\Z}$, and $\alpha$ cannot map non-trivially to $\hat{\Z}$ under this epimorphism. Thus, $H\leq \widehat{\pi_1(M_{f})}$ is not procyclic. Since $H$ is abelian and not procyclic, $H$ is not a closed subgroup of a free profinite group, and therefore, $H$ is not projective by Lemma 7.6.3 \cite{RZ}. By Theorem 9.3 \cite{WZ1}, $H$ is conjugate into the closure of a cusp subgroup $P<\pi_1(M_{f})$. The intersection $\overline{P}\cap\widehat{\pi_1(\Sigma)}$ is generated by a peripheral element $\beta\in\pi_1(\Sigma)$. Since $\langle\alpha\rangle=H\cap \widehat{\pi_1(\Sigma)}<\overline{P}\cap\widehat{\pi_1(\Sigma)}$, $\alpha$ is in the closure of a peripheral element. Thus, $\hat{n}_f\leq n_f$. 
\end{proof}
\end{prop}
We now restate and prove Theorem~\ref{toptype}.
\begin{thmn}[\ref{toptype}]
Let $M$ and $N$ be finite-volume hyperbolic manifolds with $\widehat{\pi_1(M)}\cong\widehat{\pi_1(N)}$. Liu's Thurston-norm and fiber class preserving isomorphism $H^1(N,\Z)\to H^1(M,\Z)$ induced by an isomorphism $\Phi:\widehat{\pi_1(M)}\to\widehat{\pi_1(N)}$ sends fibered classes to fibered classes where the corresponding fiber surfaces have the same topological type.
\begin{proof}
When $M$ (and $N$) are closed, Theorem~\ref{liuthm} shows that Theorem~\ref{toptype} holds. Thus, we assume that $M$ is cusped and fix a fibration of $M$ over $S^1$ with fiber a punctured surface $\Sigma_M$ and monodromy $\varphi$. By Theorem~\ref{liuthm2}, for a fixed isomorphism $\Phi:\widehat{\pi_1(M)}\to\widehat{\pi_1(N)}$, there is a punctured surface $\Sigma_N$ (with the same complexity as $\Sigma_M$) which is the fiber of a fibration of $N$ over $S^1$ (with monodromy $\psi$) and $\Phi(\widehat{\pi_1(\Sigma_M))}\cong\widehat{\pi_1(\Sigma_N)}$. The goal is to show that $\Sigma_M$ is homeomorphic to $\Sigma_N$. The surface $\Sigma_M$ is a genus $g_M$ surface with $n_M$ punctures and the surface $\Sigma_N$ is a genus $g_N$ surface with $n_N$ punctures. Since $\chi(\Sigma_M)=\chi(\Sigma_N)$ by Theorem~\ref{liuthm2}, it is sufficient to prove that $n_N=n_M$. To establish this we argue as follows: \medbreak By Theorem~\ref{liuthm} and Theorem~\ref{liuthm2}, we get the following diagram
 \[
   \begin{tikzcd}
  1 \arrow[r] &\pi_1(\Sigma_M)\arrow[r]\arrow[d] &\pi_1(M)\arrow[r]\arrow[d] &\Z\arrow[r]\arrow[d] & 1
    \\
    1 \arrow[r] &\widehat{\pi_1(\Sigma_M)}\arrow[r]\arrow[d, "\Phi|_{\widehat{\pi_1(\Sigma_M)}}"] &\widehat{\pi_1(M)}\arrow[r]\arrow[d, "\Phi"] &\widehat{\Z}\arrow[r]\arrow[d, "\times \mu"] & 1
    \\
    1 \arrow[r] &\widehat{\pi_1(\Sigma_N)}\arrow[r] &\widehat{\pi_1(N)}\arrow[r] &\widehat{\Z}\arrow[r] & 1
    \\
    1 \arrow[r] &\pi_1(\Sigma_N)\arrow[r]\arrow[u] &\pi_1(N)\arrow[r]\arrow[u] &\Z\arrow[r]\arrow[u] & 1
    \\
 \end{tikzcd}
\]
where the $\times\mu:\hat{\Z}\to\hat{\Z}$ map is multiplication by a profinite unit $\mu$ in $\hat{\Z}$. Replace $\varphi$ and $\psi$ (as needed) with an appropriate power (say $k!$ for a large enough natural number $k$) such that both $\varphi^{k!}$ and $\psi^{k!}$ are pure mapping classes. 
\medbreak We first observe that $\hat{\varphi}^{k!}$ and $\hat{\psi}^{k!}$ have an equal number of fixed conjugacy classes in $\widehat{\pi_1(\Sigma_M)}\cong\widehat{\pi_1(\Sigma_N)}$ i.e. $\hat{n}_{\varphi^{k!}}=\hat{n}_{\psi^{k!}}$. To see this, set $t\in\widehat{\pi_1(M)}$ to be in the preimage of a unit in $\hat{\Z}$, the conjugation action of $t$ on $\widehat{\pi_1(\Sigma_M)}$ is $\hat{\varphi}$. The conjugation action of $\Phi(t)$ on $\widehat{\pi_1(\Sigma_N)}$ is $\hat{\psi}$ by the diagram above. The conjugation action on $\widehat{\pi_1(\Sigma_M)}$ by $t^{k!}$ is $\hat{\varphi}^{k!}$ and the conjugation action on $\widehat{\pi_1(\Sigma_N)}$ by $\Phi(t^{k!})$ is $\hat{\psi}^{k!}$. For every conjugacy class $\alpha$ of elements in $\widehat{\pi_1(\Sigma_M)}$ fixed by $t^{k!}$-conjugation, there is a conjugacy class $\Phi(\alpha)$ of elements in $\widehat{\pi_1(\Sigma_N)}$ fixed by $\Phi(t^{k!})$-conjugation, and vice versa. Thus, $\hat{n}_{\varphi^{k!}}=\hat{n}_{\psi^{k!}}$. By Proposition~\ref{keyprop}, $n_M=n_{\varphi^{k!}}=\hat{n}_{\varphi^{k!}}$ and $n_N=n_{\psi^{k!}}=\hat{n}_{\psi^{k!}}$. Thus, $n_M=n_N$, $g_M=g_N$, and $\Sigma_M$ is homeomorphic to $\Sigma_N$ as claimed. 
\end{proof}

\end{thmn}
\begin{rmk}[M. Bridson]\label{bridsonrmk}
    The hypothesis of hyperbolicity is necessary for the proof of Theorem~\ref{toptype}. Without hyperbolicity, for instance, every mapping class $\phi\in Mod(\Sigma_{0,3})$ gives a group $F_2\rtimes_\phi\Z$ which is also the fundamental group of a $\Sigma_{1,1}-$bundle over $S^1$. 
\end{rmk}

\section{Proof of Theorem~\ref{csp+omni}}
The following key lemma makes use of the proofs of Theorem 2.4 and Lemma 2.5 \cite{BRW}. 
\begin{lemma}\label{LiuplusBRW}
Let $\Sigma_{g,n}$ be a surface for which $Mod(\Sigma_{g,n})$ has the Congruence Subgroup Property. For hyperbolic 3-manifolds $M$ and $N$ that fiber over $S^1$ with $\Sigma_{g,n}$ fiber and monodromies $\varphi$ and $\psi$ respectively, let $\Phi:\widehat{\pi_1(M)}\to\widehat{\pi_1(N)}$ be an isomorphism that identifies the closures of the fiber subgroups of $M$ and $N$. Then $\varphi=\psi^{\pm}$ in $Mod(\Sigma_{g,n})$, and therefore $M$ and $N$ are homeomorphic. 
\end{lemma}

We complete the proof of Theorem~\ref{csp+omni} assuming Lemma~\ref{LiuplusBRW}.
\begin{proof}[Proof of Theorem~\ref{csp+omni}]
Let $\Sigma_{g,n}\hookrightarrow M\to S^1$ be a fixed fibration. By \cite{WZ1},\cite{JZ}, and Theorem~\ref{liuthm2}, a 3-manifold $N$ with $\widehat{\pi_1(N)}\cong\widehat{\pi_1(M)}$ will be hyperbolic and will fiber over $S^1$ with fiber a surface $S$ with Euler characteristic $\chi(S)=\chi(\Sigma_{g,n})$. Fix an identification $\Phi:\widehat{\pi_1(M)}\to\widehat{\pi_1(N)}$. By Theorem~\ref{liuthm2}, there is a connected fiber surface $\Sigma\hookrightarrow N$ with $\Phi(\widehat{\pi_1(\Sigma_{g,n}))}\cong\widehat{\pi_1(\Sigma)}$. By Theorem~\ref{toptype}, $\Sigma$ is also homeomorphic to $\Sigma_{g,n}$, and therefore by Lemma~\ref{LiuplusBRW} $M\cong N$. 
\end{proof}
We now prove Lemma~\ref{LiuplusBRW}.
\begin{proof}[Proof of Lemma~\ref{LiuplusBRW}]
   We denote the monodromies of $M$ and $N$ by $\varphi$ and $\psi$ respectively, and we will refer to the outer automorphisms of $F_m$ (the free group on $m=2g+n-1$ generators) induced by $\varphi,\psi$ by the same names. Let $\Sigma,\Sigma'$ be the corresponding aligned fiber surfaces (both homeomorphic to $\Sigma_{g,n}$) for $M$ and $N$ respectively. We have the following exact sequences 
  \[
   \begin{tikzcd}
  1 \arrow[r] &\pi_1(\Sigma)\arrow[r]\arrow[d] &\pi_1(M)\arrow[r]\arrow[d] &\Z\arrow[r]\arrow[d] & 1
    \\
    1 \arrow[r] &\widehat{\pi_1(\Sigma)}\arrow[r]\arrow[d, "\Phi|_{\widehat{\pi_1(\Sigma)}}"] &\widehat{\pi_1(M)}\arrow[r]\arrow[d, "\Phi"] &\widehat{\Z}\arrow[r]\arrow[d, "\cong"] & 1
    \\
    1 \arrow[r] &\widehat{\pi_1(\Sigma')}\arrow[r] &\widehat{\pi_1(N)}\arrow[r] &\widehat{\Z}\arrow[r] & 1
    \\
    1 \arrow[r] &\pi_1(\Sigma')\arrow[r]\arrow[u] &\pi_1(N)\arrow[r]\arrow[u] &\Z\arrow[r]\arrow[u] & 1
    \\
 \end{tikzcd}
\]
where all the unlabelled vertical arrows are canonical inclusions of groups into their profinite completions. 
\medbreak The isomorphism $\Phi$ embeds $\pi_1(M)$ and $\pi_1(\Sigma)$ in $\widehat{\pi_1(N)}$ as dense subgroups of $\widehat{\pi_1(N)}$ and $\widehat{\pi_1(\Sigma')}$ respectively. We consider $\Phi(\pi_1(\Sigma))<\Phi(\pi_1(M))$. Let $K_i$ be the intersection of all subgroups of $\Phi(\pi_1(\Sigma))$ of index $\leq i$. The tower $\{K_i\,|\,i\in\mathbb{N}\}$ is a cofinal tower of characteristic subgroups of $\Phi(\pi_1(\Sigma))$. Set $L_i=\widehat{K_i}\cap\pi_1(\Sigma')$ to obtain a corresponding characteristic tower of $\pi_1(\Sigma')$. Conjugation by elements of $\Phi(\pi_1(M))$ on $\Phi(\pi_1(\Sigma))$ induces outer automorphisms in $Out(\Phi(\pi_1(\Sigma))/K_i)\cong Out(\widehat{\pi_1(\Sigma')}/\widehat{L_i})$ for all $i$. Similarly, conjugation by elements of $\pi_1(N))$ on $\pi_1(\Sigma')$ induces outer automorphisms in $Out(\pi_1(\Sigma')/L_i)\cong Out(\widehat{\pi_1(\Sigma')}/\widehat{L_i})$ for all $i$. 
\medbreak The monodromy $\varphi$ is induced for the extension $1\to \Phi(\pi_1(\Sigma))\to \Phi(\pi_1(M))\to \Z\to 1$ by the conjugation action of an element $t\in \Phi(\pi_1(M))$  where the rightmost $\Z$ is a dense subgroup of $\hat{\Z}$ (and $t$ maps to a generator of this $\hat{\Z}$ under the fixed profinite epimorphism $\widehat{\pi_1(N)}\twoheadrightarrow\hat{\Z}$). Likewise, the monodromy $\psi$ is induced by the conjugation action of an element $s\in\pi_1(N)<\widehat{\pi_1(N)}$ that maps to a generator of $\Z$. Since the images of $t$ and $s$ under the fixed $\widehat{\pi_1(N)}\twoheadrightarrow\hat{\Z}$ both topologically generate $\hat{\Z}$, the actions of $t$ and $s$ (and therefore the actions of $\varphi$ and $\psi$) induce outer automorphisms $\varphi_i\in Out(\Phi(\pi_1(\Sigma))/K_i)\cong Out(\widehat{\pi_1(\Sigma')}/\widehat{L_i})$ and $\psi_i\in Out(\pi_1(\Sigma')/L_i)\cong Out(\widehat{\pi_1(\Sigma')}/\widehat{L_i})$ such that $\varphi_i$ and $\psi_i$ generate the same cyclic subgroup $C_i< Out(\widehat{\pi_1(\Sigma')}/\widehat{L_i})$.
\medbreak We now claim that the cyclic subgroups $\psi$ is in the subgroup $<\varphi>$ in $Mod(\Sigma_{g,n})$. To see this, assume that $\psi\notin <\varphi>$. The cyclic subgroup $<\varphi>$ in $Mod(\Sigma_{g,n})$ is separable (Theorem 1.1 \cite{ML}), and so there is a finite quotient $q:Mod(\Sigma_{g,n})\twoheadrightarrow Q$ for which $q(\psi)\notin q(<\varphi>)$. From the previous paragraph, the images $\varphi_i$ and $\psi_i$ generate the same cyclic subgroup $C_i< Out(\widehat{\pi_1(\Sigma')}/\widehat{L_i})$ in every congruence quotient. It follows that $Q$ is not a congruence quotient of $Mod(\Sigma_{g,n})$. However, this contradicts the claim that $Mod(\Sigma_{g,n})$ has the Congruence Subgroup Property. Thus, $\psi$ is in the subgroup $<\varphi>$, and by reversing the roles of $\psi$ and $\varphi$ in the separability argument above, $<\varphi>=<\psi>$. Therefore $\varphi=\psi^{\pm}$ as claimed, and $M$ is homeomorphic to $N$.

\end{proof}





\begin{comment}
\begin{theorem}\label{pureprofiniterig}
Let $M$ be a hyperbolic 4-punctured $S^2$-bundle over $S^1$ with monodromy in $PMod(\Sigma_{g,n})$, then $\pi_1(M)$ is profinitely rigid among 3-manifold groups. 
\begin{proof}
For $N$ a 3-manifold with $\widehat{\pi_1(N)}\cong\widehat{\pi_1(M)}$, we can apply Theorem 1.3 \cite{Y} and Theorem 9.3 \cite{WZ1} to see that $N$ is a hyperbolic 3-manifold that fibers over $S^1$ with fiber a surface $S$ of complexity 1. In particular, $S$ is either the four-times punctured sphere or the twice-punctured torus. Since the pA monodromy of $M$ fixes every puncture of $\Sigma_{g,n}$, it follows that $M$ is a 4-cusped hyperbolic 3-manifold. By Theorem 9.3 \cite{WZ1} and Theorem 1.1 \cite{Chagas2016Hyperbolic3G}, $N$ is 4-cusped as well, and so $S$, a fiber surface for a fibration of $N$ over $S^1$, cannot be the twice-punctured torus. Thus, $S=\Sigma_{g,n}$, and we can use Lemma~\ref{LiuplusBRW} to conclude that $N$ is homeomorphic to $M$.
\end{proof}
\end{theorem}
\end{comment}


 






\begin{comment}
\begin{figure}
    \centering
    \includegraphics[width=0.4\textwidth]{six_two_ss.png}
    \caption{The knot $6_2$}
    \label{fig:enter-label}
\end{figure}

\begin{figure}
    \centering
    \includegraphics[width=0.4\textwidth]{six_three_ss.png}
    \caption{The knot $6_3$}
    \label{fig:enter-label}
\end{figure}

\begin{figure}
    \centering
    \includegraphics[width=0.35\textwidth]{magic.png}
    \caption{The magic manifold $6^3_1$}
    \label{fig:enter-label}
\end{figure}

\begin{figure}
    \centering
    \includegraphics[width=0.4\textwidth]{whitehead.png}
    \caption{The Whitehead link $5^2_1$}
    \label{The Whitehead link $5^2_1$}
\end{figure}

\begin{figure}
    \centering
    \includegraphics[width=0.3\textwidth]{borromean.png}
    \caption{The Borromean rings $6^3_2$}
    \label{The Borromean rings $6^3_2$}
\end{figure}
\end{comment}
%%%%%%%%%%%%%%%%%%%%%%%%%%%%%%%%%%%%%%%%%%%%%%%%%%%%
%%%%%%%%%%%%%%%%%%%%%%%%%%%%%%%%%%%%%%%%%%%%%%%%%%%%
%%%%%%%%%%%%%%%%%%                                           %%%%%%%%%%%%%%%%%
%%%%%%%%%%%%%%%%%%		Bibliography		%%%%%%%%%%%%%%%%%
%%%%%%%%%%%%%%%%%%                                           %%%%%%%%%%%%%%%%%
%%%%%%%%%%%%%%%%%%%%%%%%%%%%%%%%%%%%%%%%%%%%%%%%%%%%
%%%%%%%%%%%%%%%%%%%%%%%%%%%%%%%%%%%%%%%%%%%%%%%%%%%%


\bibliography{main}

\end{document}
