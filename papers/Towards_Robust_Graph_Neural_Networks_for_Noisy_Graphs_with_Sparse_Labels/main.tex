\documentclass[sigconf]{acmart}
\usepackage{graphicx}
\usepackage{xcolor}
\usepackage{multirow}
\usepackage{tabularx}
\usepackage{algorithm}
\usepackage{algorithmic}
\usepackage{subcaption}
\usepackage{amsthm}
\usepackage{enumitem}
\renewcommand{\algorithmicrequire}{\textbf{Input:}}
\renewcommand{\algorithmicensure}{\textbf{Output:}}
\newcolumntype{C}{>{\centering\arraybackslash}X} % centered version of 'X' col. type

\newcommand{\suhang}[1]{\textcolor{blue}{#1}}
\newcommand{\wei}[1]{\textcolor{orange}{#1}}

\copyrightyear{2022}
\acmYear{2022}
\setcopyright{acmcopyright}\acmConference[WSDM '22]{Proceedings of the Fifteenth
ACM International Conference on Web Search and Data Mining}{February 21--25,
2022}{Tempe, AZ, USA}
\acmBooktitle{Proceedings of the Fifteenth ACM International Conference on Web
Search and Data Mining (WSDM '22), February 21--25, 2022, Tempe, AZ, USA}
\acmPrice{15.00}
\acmDOI{10.1145/3488560.3498408}
\acmISBN{978-1-4503-9132-0/22/02}

\title{Towards Robust Graph Neural Networks for Noisy Graphs with Sparse Labels}

\author{Enyan Dai$^\dagger$, Wei Jin$^\ddagger$, Hui Liu$^\ddagger$, Suhang Wang$^\dagger$ }

\affiliation{$\dagger$ The Pennsylvania State University,
{${\ddagger}$} Michigan State University%\\
}

\email{{emd5759, szw494}@psu.edu, {jinwei2, liuhui7}@msu.edu}
\settopmatter{printacmref=True}


\begin{document}
\fancyhead{}
\begin{abstract}
Graph Neural Networks (GNNs) have shown their great ability in modeling graph structured data. However, real-world graphs usually contain structure noises and have limited labeled nodes. The performance of GNNs would drop significantly when trained on such graphs, which hinders the adoption of GNNs on many applications. Thus, it is important to develop noise-resistant GNNs with limited labeled nodes. However, the work on this is rather limited. Therefore, we study a novel problem of developing robust GNNs on noisy graphs with limited labeled nodes. Our analysis shows that both the noisy edges and limited labeled nodes could harm the message-passing mechanism of GNNs. To mitigate these issues, we propose a novel framework which adopts the noisy edges as supervision to learn a denoised and dense graph, which can down-weight or eliminate noisy edges and facilitate message passing of GNNs to alleviate the issue of limited labeled nodes. The generated edges are further used to 
regularize the predictions of unlabeled nodes with label smoothness to better train GNNs. Experimental results on real-world datasets demonstrate the robustness of the proposed framework on noisy graphs with limited labeled nodes. 




\end{abstract}

% \keywords{Robust Graph Neural Network; Noisy Graph}

\begin{CCSXML}
<ccs2012>
<concept>
<concept_id>10010147.10010257.10010282.10011305</concept_id>
<concept_desc>Computing methodologies~Semi-supervised learning settings</concept_desc>
<concept_significance>500</concept_significance>
</concept>
<concept>
<concept_id>10010147.10010257.10010293.10010294</concept_id>
<concept_desc>Computing methodologies~Neural networks</concept_desc>
<concept_significance>500</concept_significance>
</concept>
</ccs2012>
\end{CCSXML}

\ccsdesc[500]{Computing methodologies~Semi-supervised learning settings}
\ccsdesc[500]{Computing methodologies~Neural networks}

\keywords{Noisy Edges; Robustness; Graph Neural Networks}
% \vskip -4em
\maketitle

\section{Introduction}
Graph Neural Networks (GNNs)~\cite{kipf2016semi,hamilton2017inductive} have made remarkable achievements in modeling graphs from various domains such as social networks~\cite{hamilton2017inductive}, financial system~\cite{wang2019semi}, and recommendation system~\cite{wang2019knowledge}. The success of GNNs relies on the message-passing mechanism~\cite{kipf2016semi,hamilton2017inductive}, where node representations are updated by aggregating the information from neighbors. With this mechanism, the node representations capture node features, information of neighbors and local graph structure, which facilitate various tasks, especially semi-supervised node classification. 

Although GNNs have shown great ability in modeling graphs, their performance can degrade significantly when trained on graphs with \textit{noisy edges} and/or \textit{limited labeled nodes}.
\textit{First}, due to the message passing, GNNs are vulnerable to adversarial or noisy edges. For example, as shown in Fig.~\ref{fig:illustration}, poisoning attacks~\cite{zugner2019adversarial} add/delete carefully chosen edges to the graph. These adversarial edges (shown in red) usually connect nodes of different labels or features, thus contaminating the neighborhoods of nodes, propagating noises/errors to node representations. In addition, inherent edge noises also exist in real-world graphs. For instance, in social networks, bots tend to build connections with normal users to spread misinformation~\cite{ferrara2016rise}, which can also harm the performance of GNNs for bot detection. 
\begin{figure}
    \centering
    \includegraphics[width=0.95\linewidth]{figure/ill.pdf}
    \vskip -1.9em
    \caption{An illustration of down-weighting/removing noise edges and densifying the graph for better performance.}
    \label{fig:illustration}
    \vskip -2em
\end{figure}
\textit{Second}, for many applications, graphs are often sparsely labeled such as cell phone network for fraud detection~\cite{gallagher2008using}. 
Label sparsity can severely reduce the involvement of unlabeled nodes during message passing, leading to poor performance. Generally, in a $K$-layer GNN, a labeled node aggregates its $K$-hop neighborhood information, thus making many unlabeled nodes in $K$-hop neighborhood participate in the training, which is one major reason that GNNs can leverage unlabeled nodes for semi-supervised node classification. However,  as verified in our preliminary analysis in Fig.~\ref{fig:1_a} of Sec.~\ref{sec:3_3}, when the number of labeled nodes decreases, the amount of unlabeled nodes participating in training drops quickly, making message passing less effective.
These shortcomings of GNNs hinder the adoption of GNNs for many real-world applications. Thus, it is important to develop robust GNNs that can simultaneously handle noisy graphs with sparse labels.

However, developing robust GNNs for graphs with noisy edges and limited labeled nodes is challenging. \textit{First}, the training graph itself is noisy, i.e., noisy edges are mixed with the normal edges. Thus, we need supervision in down-weighting or eliminating noisy edges. \textit{Second}, alleviating the limited label issue requires more labels, while obtaining more labeled nodes is time-consuming and expensive. Hence, we need alternative approaches to more effectively utilize the limited labels. Some initial efforts~\cite{wu2019adversarial,jin2020graph,tang2020transferring,jin2020graph} have been taken to alleviate the effects of the adversarial edges such as pruning edges by using node similarity~\cite{wu2019adversarial}, and adopting Gaussian distribution as node representations to absorb noises~\cite{zhu2019robust}. To address the problem of sparsely labeled graphs, some methods~\cite{sun2019multi,li2018deeper,peng2020self} propose to obtain better representations by training GNNs with self-supervised learning tasks such as pseudo label prediction~\cite{sun2019multi,li2018deeper} and global context predictions~\cite{peng2020self}. 
However, little efforts are taken for robust GNNs that can simultaneously handle noisy edges and label sparsity.



Since both the noisy edges and limited labeled nodes harm the message passing of GNNs and message passing is directly related to the graph structure, we argue that learning a denoised and dense graph guided by the raw attributed graph is promising to facilitate message passing for robust GNNs. \textit{First}, for many graphs such as social networks, nodes with similar features and labels tend to be linked~\cite{liben2007link}, while noisy edges would link nodes of dissimilar features~\cite{wu2019adversarial}. 
Thus, we can use node attributes to predict the links. For existing links, the link predictor will assign small weights to links connecting nodes of dissimilar features while large weights to links connecting nodes of similar features, thus alleviating negative issue of noisy edges during message passing. \textit{Second}, 
real-world graphs are usually very sparse, containing many missing edges. With the link predictor, nodes that are potentially to be linked could be identified. Densifying the graph by linking similar nodes would induce more unlabeled nodes to become neighbors of labeled nodes with the same labels as shown in Fig.~\ref{fig:illustration}, which can alleviate the label sparsity issue. 
In addition, since adjacent nodes tend to have the same labels, the predicted new links can be used to further regularize the label predictions of unlabeled nodes. Though promising, the work on down-weighting noisy edges and densifying graph for robust GNN on noisy graphs with sparse labels are rather limited. 

Therefore, in this paper, we investigate a novel problem of developing robust noise-resistant GNNs with limited labeled nodes by learning a denoised and densified graph. In essence, we need to solve two challenges: (i) how to effectively learn a link predictor from the noisy graph which can eliminate noisy edges and densify the graph; and (ii) how to simultaneously use the learned graph to learn a structural noise-resistant GNNs with limited labeled nodes. To address these challenges, we propose a novel framework named 
robust structural noise-resistant GNN (RS-GNN)~\footnote{Codes are available at: https://github.com/EnyanDai/RSGNN}. RS-GNN adopts the node attributes and supervision from the noisy edges to
denoise and dense graph, which can alleviate the negative effects of noisy edges and facilitate the message passing between unlabeled nodes and labeled nodes. The learned graph is used as input for learning a GNN. RS-GNN also adopts the predicted edges to further explicitly regularize the predictions of unlabeled nodes to alleviate the label sparsity issue. In summary, our main contributions are:
\begin{itemize}[leftmargin=*]
    \item We study a new problem of learning robust noise-resistant GNNs with limited labeled nodes;
    \item We propose a novel framework RS-GNN, which can simultaneously learn a denoised and densified graph and a robust GNN on noisy graphs with limited labeled nodes; and
    \item We conduct extensive experiments on real-world datasets to demonstrate the robustness of RS-GNN on both noisy/clean graphs with limited labeled nodes. 
\end{itemize}

\section{Related work}
\label{sec:related_work}

Accessibility is an essential component of computing, which aims to make technology broadly accessible to as many users as possible, including those with differing sets of abilities. Improvements in usability and accessibility falls to the community, to better understand the needs of users with differing abilities, and to design technologies that play to this spectrum of abilities \citep{Wobbrock2011AbilityBasedDC}.
In computing, significant strides have been made to increase the accessibility of web content. For example, various versions of the Web Content Accessibility Guidelines (WCAG) \citep{Chisholm2001WebCA, Caldwell2008WebCA} and the in-progress working draft for WCAG 3.0,\footnote{\href{https://www.w3.org/TR/wcag-3.0/}{https://www.w3.org/TR/wcag-3.0/}} or standards such as ARIA from the W3C's Web Accessibility Initiative (WAI)\footnote{\href{https://www.w3.org/WAI/standards-guidelines/aria/}{https://www.w3.org/WAI/standards-guidelines/aria/}} have been released and used to guide web accessibility design and implementation. Similarly, positive steps have been made to improve the accessibility of user interfaces and user experience \citep{Peissner2012MyUIGA, Peissner2013UserCI, Thompson2014ImprovingTU, Bigham2014MakingTW}, as well as various types of media content \citep{Mirri2017TowardsAG, Nengroo2017AccessibleI, Gleason2020TwitterAA}. 

We take inspiration from accessibility design principles in our effort to make research publications more accessible to users who are blind and low vision. Blindness and low vision are some of the most common forms of disability, affecting an estimated 3--10\% of Americans depending on how visual impairment is defined \citep{CDCVisionLossBurden}. BLV researchers also make up a representative sample of researchers in the United States and worldwide. A recent Nature editorial pushes the scientific community to better support researchers with visual impairments \citep{NatureCareerColumn2020}, since existing tools and resources can be limited. There are many inherent accessibility challenges to performing research. In this paper, we engage with one of these challenges that affects all domains of study, accessing and reading the content of academic publications. 

BLV users interact with papers using screen readers, braille displays, text-to-speech, and other assistive tools. A WebAIM survey of screen reader users found that the vast majority (75.1\%) of respondents indicate that PDF documents are very or somewhat likely to pose significant accessibility issues.\footnote{\href{https://webaim.org/projects/screenreadersurvey8/}{https://webaim.org/projects/screenreadersurvey8/}} Most paper are published in PDF, which is inherently inaccessible, due in large part to its conflation of visual layout information with semantic content \citep{NielsenPDFStillUnfit, Bigham2016AnUT}. 
\citet{Bigham2016AnUT} describe the historical reasons we use PDF as the standard document format for scientific publications, as well as the barriers the format itself presents to accessibility. Prior work on scientific accessibility have made recommendations for how to make PDFs more accessible \cite{Rajkumar2020PDFAO, Darvishy2018PDFAT}, including greater awareness for what constitutes an accessible PDF and better tooling for generating accessible PDFs. Some work has focused on addressing components of paper accessibility, such as the correct way for screen readers to interpret and read mathematical equations \citep{Flores2010MathMLTA, Bates2010SpokenMU, Sorge2014TowardsMM, Mackowski2017MultimediaPF, Ahmetovic2018AxessibilityAL, Ferreira2004EnhancingTA, Sojka2013AccessibilityII}, describe charts and figures \citep{Elzer2008AccessibleBC, Engel2017TowardsAC, Engel2019SVGPlottAA}, automatically generate figure captions \citep{Chen2019NeuralCG, Qian2020AFS}, or automatically classify the content of figures \citep{Kim2018MultimodalDL}. Other work applicable to all types of PDF documents aims to improve automatic text and layout detection of scanned documents \cite{Nazemi2014PracticalSM} and extract table content \cite{Fan2015TableRD, Rastan2019TEXUSAU}. In this work, we focus on the issue of representing overall document structure, and navigation within that structure. Being able to quickly navigate the contents of a paper through skimming and scanning is an essential reading technique \citep{Maxwell1972SkimmingAS}, which is currently under-supported by PDF documents and PDF readers when reading these documents by screen reader. 

There also exists a variety of automatic and manual tools that assess and fix accessibility compliance issues in PDFs, including the Adobe Acrobat Pro Accessibility Checker\footnote{\href{https://www.adobe.com/accessibility/products/acrobat/using-acrobat-pro-accessibility-checker.html}{https://www.adobe.com/accessibility/products/acrobat/using-acrobat-pro-accessibility-checker.html}}, Common Look\footnote{\href{https://monsido.com/monsido-commonlook-partnership}{https://monsido.com/monsido-commonlook-partnership}}, ABBYY FineReader\footnote{\href{https://pdf.abbyy.com/}{https://pdf.abbyy.com/}}, PAVE\footnote{\href{https://pave-pdf.org/faq.html}{https://pave-pdf.org/faq.html}}, and PDFA Inspector\footnote{\href{https://github.com/pdfae/PDFAInspector}{https://github.com/pdfae/PDFAInspector}}. To our knowledge, PAVE and PDFA Inspector are the only non-proprietary, open-source tools for this purpose. Based on our experiences, however, all of these tools require some degree of human intervention to properly tag a scientific document, and tagging and fixing must be performed for each new version of a PDF, regardless of how minor the change may be.

Guidelines and policy changes have been introduced in the past decade to ameliorate some of the issues around scientific PDF accessibility. Some conferences, such as The ACM CHI Virtual Conference on Human Factors in Computing Systems (CHI) and The ACM SIGACCESS Conference on Computers and Accessibility (ASSETS), have released guidelines for creating accessible submissions.\footnote{See \href{http://chi2019.acm.org/authors/papers/guide-to-an-accessible-submission/}{http://chi2019.acm.org/authors/papers/guide-to-an-accessible-submission/} and \href{https://assets19.sigaccess.org/creating_accessible_pdfs.html}{https://assets19.sigaccess.org/creating\_accessible\_pdfs.html}} The ACM Digital Library\footnote{\href{https://dl.acm.org/}{https://dl.acm.org/}} provides some publications in HTML format, which is easier to make accessible than PDF~\cite{Graells2007EstudioDL}. \citet{Ribera2019PublishingAP} conducted a case study on DSAI 2016 (Software Development and Technologies for Enhancing Accessibility and Fighting Infoexclusion). The authors of DSAI were responsible for creating accessible proceedings and identified barriers to creating accessible proceedings, including lack of sufficient tooling and lack of awareness of accessibility. The authors recommended creating a new role in the organizing committee dedicated to accessible publishing. These policy changes have led to improvements in localized communities, but have not been widely adopted by all academic publishers and conference organizers.

Table~\ref{tab:prior_work} lists prior studies that have analyzed PDF accessibility of academic papers, and shows how our study compares. Prior work has primarily focused on papers published in Human-Computer Interaction and related fields, specific to certain publication venues, while our analysis tries to quantify paper accessibility more broadly.
\citet{Brady2015CreatingAP} quantified the accessibility of 1,811 papers from CHI 2010-2016, ASSETS 2014, and W4A, assessing the presence of document tags, headers, and language. They found that compliance improved over time as a response to conference organizers offering to make papers accessible as a service to any author upon request. \citet{Lazar2017MakingTF} conducted a study quantifying accessibility compliance at CHI from 2010 to 2016 as well as ASSETS 2015,
%\jb{Define acronyms in prev para}
confirming the results of \citet{Brady2015CreatingAP}. They found that across 5 accessibility criteria, the rate of compliance was less than 30\% for CHI papers in each of the 7 years that were studied. The study also analyzed papers from ASSETS 2015, an ACM conference explicitly focused on accessibility, and found that those papers had significantly higher rates of compliance, with over 90\% of the papers being tagged for correct reading order and no criteria having less than 50\% compliance. This finding indicates that community buy-in is an important contributor to paper accessibility.
\citet{Nganji2015ThePD} conducted a study of 200 PDFs of papers published in four disability studies journals, finding that accessibility compliance was between 15-30\% for the four journals analyzed, with some publishers having higher adherence than others. To date, no large scale analysis of scientific PDF accessibility has been conducted outside of disability studies and HCI, due in part to the challenge of scaling such an analysis. We believe such an analysis is useful for establishing a baseline and characterizing routes for future improvement. Consequently, as part of this work, we conduct an analysis of scientific PDF accessibility across various fields of study, and report our findings relative to prior work. 


\begin{table}[t!]
\small
    \centering
    \begin{tabularx}{\linewidth}{L{22mm}L{15mm}L{48mm}L{16mm}L{34mm}}
        \toprule
        \textbf{Prior work} & \textbf{PDFs analyzed} & \textbf{Venues} & \textbf{Year} & \textbf{Accessibility checker} \\
        \midrule
        \citet{Brady2015CreatingAP} & 1811 & CHI, ASSETS and W4A & 2011--2014 & PDFA Inspector \\ [0.5mm]
        \hline \\ [-2.5mm]
        \citet{Lazar2017MakingTF} & 465 + 32 & CHI and ASSETS & 2014--2015 & Adobe Acrobat Action Wizard \\ [0.5mm]
        \hline \\ [-2.5mm]
        \citet{Ribera2019PublishingAP} & 59 & DSAI & 2016 & Adobe PDF Accessibility Checker 2.0 \\ [0.5mm]
        \hline \\ [-2.5mm]
        \citet{Nganji2015ThePD} & 200 & \textit{Disability \& Society}, \textit{Journal of Developmental and Physical Disabilities}, \textit{Journal of Learning Disabilities}, and \textit{Research in Developmental Disabilities} & 2009--2013 & Adobe PDF Accessibility Checker 1.3 \\ [0.6mm]
        \hline \\ [-2.5mm]
        \textbf{\textit{Our analysis}} & \numpdfs & Venues across various fields of study & 2010--2019 & Adobe Acrobat Accessibility Plug-in Version 21.001.20145 \\
        \bottomrule
    \end{tabularx}
    \caption{Prior work has investigated PDF accessibility for papers published in specific venues such as CHI, ASSETS, W4A, DSAI, or various disability journals. Several of these works were conducted manually, and were limited to a small number of papers, while the more thorough analysis was conducted for CHI and ASSETS, two conference venues focused on accessibility and HCI. Our study expands on this prior work to investigate accessibility over \numpdfs PDFs sampled from across different fields of study.
    }
    % \Description{
    % Prior work, PDFs analyzed, Venues, Year, Accessibility checker 
    % Brady et al. [7], 1811, CHI, ASSETS and W4A, 2011--2014, PDFA Inspector 
    % Lazar et al. [23], 465 + 32, CHI and ASSETS, 2014--2015, Adobe Acrobat Action Wizard 
    % Ribera et al. [40], 59, DSAI, 2016, Adobe PDF Accessibility Checker 2.0 
    % Nganji [33], 200, Disability & Society, Journal of Developmental and Physical Disabilities, Journal of Learning Disabilities, and Research in Developmental Disabilities, 2009--2013, Adobe PDF Accessibility Checker 1.3
    % Our analysis, 11397, Venues across various fields of study, 2010--2019, Adobe Acrobat Accessibility Plug-in Version 21.001.20145 
    % }
    \label{tab:prior_work}
\end{table}
\section{Preliminary Analysis}
\label{Sec:pre_analysis}
In this section, we discuss the inner working of GNNs, conduct preliminary analysis to show the issues of GNN with sparse labels and verify that densifying graphs by connecting similar nodes can potentially alleviate the issue.

\subsection{Notations}
We use $\mathcal{G}=(\mathcal{V},\mathcal{E}, \mathbf{X})$ to denote an attributed graph, where $\mathcal{V}=\{v_1,...,v_N\}$ is the set of $N$ nodes, $\mathcal{E} \subseteq \mathcal{V} \times \mathcal{V}$ is the set of edges, and $\mathbf{X}=\{\mathbf{x}_1,...,\mathbf{x}_N\}$ is the set of attributes of $\mathcal{V}$. $\mathbf{A} \in \mathbb{R}^{N \times N}$ is the adjacency matrix of the graph $\mathcal{G}$, where $\mathbf{A}_{ij}=1$ if nodes ${v}_i$ and ${v}_j$ are connected, otherwise $\mathbf{A}_{ij}=0$. In 
our setting, only a limited number of nodes $\mathcal{V}_L=\{v_1,...,v_l\}$ are provided with labels $\mathcal{Y}=\{\mathbf{y}_1,...,\mathbf{y}_l\}$, where $\mathbf{y}_i \in \mathbb{R}^C$ is a one-hot vector of node $v_i$'s label for multi-class classification. Note that the topology of the graph $\mathcal{G}$ could be noisy such as poisoned by adversarial edges or containing inherent noises, which leads to poor performance.
\label{sec:3_1}

\subsection{Basic Design and Inner Working of GNNs}
In this subsection, we briefly introduce the common architecture of graph neural networks (GNNs). 
Generally, GNNs adopt message-passing mechanism to learn node representations, i.e., they update the representation of a node by aggregating the representations of the neighborhood nodes. The updating process of the $k$-th layer in GNNs could be written as:
\begin{equation}
\begin{aligned}
    \mathbf{a}^{(k)}_v & = \text{AGGREGATE}^{(k-1)}(\{\mathbf{h}^{(k-1)}_u: u \in \mathcal{N}(v)\}),
    \label{eq:GNN_a} \\
    \mathbf{h}^{(k)}_{v} & =\text{COMBINE}^{(k)}(\mathbf{h}^{(k-1)}_v, \mathbf{a}_v^{(k)}),
    % \label{eq:GNN_h}
\end{aligned}
\end{equation}
where $\mathbf{h}^{(k)}_v$ is the representation vector of node $v \in \mathcal{V}$ at the $k$-th layer and $\mathcal{N}(v)$ is the set of neighborhoods of $v$. 
During the training of node classification, the representations of labeled nodes are used to give prediction and obtain the training loss to minimize.
With the message-passing mechanism, after $K$-layers of GNN, the node representation of $v_i$ would capture the node features and structure information of the $K$-hop neighborhoods of $v_i$, and thus facilitating downstream tasks. 
In other words, in GNN, \textit{one labeled node would make the $K$-hop neighborhood participate in the training of GNN}, which is one reason that GNNs have great ability in leveraging unlabeled nodes for semi-supervised node classification. 

\subsection{Analysis of GNNs with Sparse Labels}

In this subsection, we conduct preliminary analysis on real-world graphs to show the issues of GNNs when limited labeled nodes are available for training, which paves us a way to design robust GNNs for alleviating the label sparsity issue. The analysis is based on three widely used datasets, i.e., Citeseer~\cite{sen2008collective}, Cora and Cora-ML~\cite{mccallum2000automating}.

Generally, GNNs, such as GCN and GAT, rely on the classification loss of the labeled nodes to learn the parameters,  which is effective when we have adequate labeled nodes. However, when the size of labeled node set $\mathcal{V}_L$ is small and the graph is sparse, only a small portion of nodes would be involved in the training. This may lead to poor performance of GNNs. More specifically, for a $K$-layer GNN, the nodes involved in the training phase include the labeled nodes and the unlabeled nodes within $K$-hop distance of labeled nodes. We usually set $K$ as 2 to 3 because deep GNNs have over-smoothing issue~\cite{li2018deeper}. Since real-world graphs are usually sparse, the $K$-hop neighbors of the labeled nodes would be limited as well. Thus, when $\mathcal{V}_L$ is small, only a small portion of nodes would be involved in training, making GNNs less effective in leveraging unlabeled nodes.

We analyze how the label rate affects the rates of uninvolved nodes of real-world datasets for a two layer GNN. We vary label rates from 0.01 to 0.25. The average uninvolved node rates and the standard deviations are shown in Fig. \ref{fig:1_a}. From the figure, we observe that (\textbf{i}) when the label rate is high, say above 0.1, most of the nodes are involved in training GNN. The benefit of further increasing label rate is marginal as the 2-hop neighbors of labeled nodes could overlap. This is one reason that GNNs have great ability for semi-supervised node classification with small but adequate amount of labeled nodes, and the increase of labeled nodes can marginally improve the performance; (\textbf{ii}) As the label rate decreases from 0.1, the uninvolved node rate increases significantly, i.e., the majority of nodes are not involved in the training. This indicates that GNNs would have difficulty in handling sparsely labeled graphs.

\begin{figure}[t]
\centering
\begin{subfigure}{0.49\columnwidth}
    \centering
    \includegraphics[width=0.9\linewidth]{figure/unlabeled_rate.png} 
    \vskip -0.5em
    \caption{Impacts of label rate}
    \label{fig:1_a}
\end{subfigure}
%\vspace{-1em}
\begin{subfigure}{0.49\columnwidth}
    \centering
    \includegraphics[width=0.9\linewidth]{figure/edge.png} 
    \vskip -0.5em
    \caption{Impacts of graph density}
    \label{fig:1_b}
\end{subfigure}
\vspace{-1.2em}
\caption{The impacts of label rate and density of graph to uninvolved node rate in the training phase. }
% \label{fig:abl}
\vskip -1.5em
\end{figure}

Although a higher label rate could help to reduce the uninvolved node rate, it can be expensive to obtain more labels~\cite{gallagher2008leveraging}. Thus, we need an alternative approach to effectively use the labels. From the analysis above, one potential solution is to make the graph denser so that one labeled node could have more neighbors to be involved in the training of GNN. To verify it, we randomly add different amount of edges to the three graphs. We denote the number of edges of the new graph as $|\mathcal{E}_A|$ and that of raw graph as $|\mathcal{E}|$. We fix label rate as 0.01. The impact of the graph density on the uninvolved node rate is presented in Fig.~\ref{fig:1_b}. From the figure, we observe that when $|\mathcal{E}_A|/|\mathcal{E}|$ increases from 1 to 3, i.e., we add two times the number of original edges, the uninvoled node rate drops significantly. For example, it drops from 0.8 to around 0.3 on Citeseer. 

As real-world graphs such as social networks have many pairs of nodes who are similar but not connected together, the analysis above shows that it is promising to predict links to densify the graph, which can help the message passing of GNNs to alleviate the issue of limited labeled nodes. In addition, these predicted edges can also be directly used to regularize the predicted labels of unlabeled nodes, i.e., if two nodes are more likely to have a link, they are more likely to have the same labels.

\label{sec:3_3}
\subsection{Problem Definition}
Our preliminary analysis shows that predicting links to densify the graph can potentially alleviate the label sparsity issue.  In addition, the link prediction can potentially down-weight or eliminate noisy edges as noisy edges usually connect nodes with low node attribute similarity.
Therefore, we aim to simultaneously eliminate noisy edges and densify the graph with a link predictor and train a robust GNN on the new graph. The problem is formally defined as:
\newtheorem{problem}{Problem}
\begin{problem}
Given an attributed graph $\mathcal{G}=(\mathcal{V},\mathcal{E}, \mathbf{X})$ with edge set $\mathcal{E}$ might contain a small amount of noisy edges, and a small set of labeled nodes $\mathcal{V}_L \in \mathcal{V}$ with the corresponding labels in $\mathcal{Y}$, simultaneously learn adjacency matrix $\mathbf{S} \in [0,1]^{N \times N}$ which down-weights/removes noisy edges and completes missing links by a link predictor $f_E:(v_i,v_j) \rightarrow \mathbf{S}_{ij}$, and a GNN on the learned graph for node classification, i.e., $f_{\mathcal{G}}:(\mathbf{S}, \mathbf{X}) \rightarrow \hat{\mathcal{Y}}$, where $\mathbf{S}_{ij}$ indicates the weight of edge linking $v_i$ and $v_j$ and $\hat{\mathcal{Y}}$ is the set of predictions for unlabeled nodes.
\end{problem}

\section{Proposed Framework -- RS-GNN}
\label{sec:methodology}
In this section, we present the details of the proposed RS-GNN. 
The main challenges are: (i) given the noisy graph, how can we learn a link predictor which can down-weight/eliminate noisy edges and densify the graph; and (ii) how to simultaneously use the  learned graph for node classification. As the graph topology is noisy, we cannot directly apply a GNN on $\mathcal{G}$ to predict edges because the message passing would magnify the negative effects of the noisy edges. Generally, nodes sharing similar features tend to connect to each other; while noisy edges tend to connect nodes of dissimilar nodes. Thus, we propose 
to learn a MLP-based link predictor which predicts links using node attributes. The more similar the node features of two nodes are, the larger weights the link predictor will assign. Thus, the link predictor is able to down-weight or eliminate noisy edges in the initial graph. Meanwhile, the edge predictor can predict missing links to alleviate label sparsity issue. We design a novel feature similarity weighted edge-reconstruction loss to train the link predictor so as to reduce the negative effects of noisy edges on the link predictor.
% we propose to use the clean node attributes to learn a dense and clean graph, and adopt $\mathbf{A}$ to supervise the learning process as most links in $\mathbf{A}$ are clean. 
An illustration of the framework is shown in Figure \ref{fig:framework}, which contains a link predictor $f_E$ and a GCN classifier $f_{\mathcal{G}}$. The link predictor $f_E$ takes node features as input to learn a dense adjacency matrix $\mathbf{S}$, aiming to remove adversarial edges and assign edges that benefit predictions. The GCN classifier $f_{\mathcal{G}}$ takes $\mathbf{S}$ and node features $\mathbf{X}$ to predict the node labels with the node features. Finally, label smoothness constraint based on the predicted edges will be added to the predictions of unlabeled nodes to further alleviate label sparsity issue. Next, we give the details of each component.


\subsection{Link Prediction}
As the given graph contains structural noises and has missing edges, we propose to learn a new graph that down-weights noisy edges to eliminate their negative effects and completes the missing links to facilitate  GNN in dealing noisy graphs with sparse labels. 

\noindent\textbf{Building Link Predictor.} Generally, noisy edges connect two nodes with dissimilar node features; while nodes of similar features are likely to have similar labels and should be connected. Therefore, we propose to predict edge weights and missing edges between nodes using nodes features.
Specifically, for node $v_i$, a MLP takes its node attributes $\mathbf{x}_i$ to learn its node representation as: $\mathbf{z}_i = MLP(\mathbf{x}_i)$.
% \end{equation}
With the node representations, we predict the weight $w(i,j)$ between $v_i \in \mathcal{V}$ and $v_j \in \mathcal{V}$ as:
\begin{equation}
    w(i,j) = f(\mathbf{z}_i^T \mathbf{z}_j),
    \label{eq:MLP}
\end{equation}
where $f$ is the activation function. For $f$, we use ReLU instead of sigmoid as we find that when the learned adjacency matrix is used as the input of GCN, the use of sigmoid function will lead to gradient vanishing, which is consistent with previous observations~\cite{he2017neural}. Note that we use MLP instead of a GNN as the link predictor because the graph structure is noisy and the message passing of GNN could magnify the negative effects.


\begin{figure}
    \centering
    \includegraphics[width=0.9\linewidth]{figure/Framework.png}
    \vskip -1em
    \caption{An illustration of the proposed RS-GNN.}
    \label{fig:framework}
    \vskip -1.8em
\end{figure}

\noindent\textbf{Learning Link Predictor.} Our goal is to learn a link predictor which can (i) assign small weights to two nodes of different features so as to eliminate noisy edges; and (ii) assign larger weights to two nodes of similar node features so as to densify the graph to facilitate message passing.
As for many real-world graphs, similar nodes tend to link together and linked nodes usually have high feature similarity. Thus, to learn a good link predictor $f_E$, we utilize the adjacency matrix reconstruction as the loss function. % Though the graph is noisy, it usually only contains a very small portion of noisy edges. For example, adversarial attacks on graph structures are carried out within a limited perturbation budget to make the perturbations unnoticeable. 
%The clean edges (1) and clean negative samples (0) will dominate the loss function in training the link predictor, resulting in a good link predictor.
Since the graph is sparse, the adjacency matrix $\mathbf{A}$ contains many zero entries. Directly adopting adjacency matrix reconstruction as the loss function would (i) result in poor performance as the link predictor will be biased on predicting missing links; and (ii) require large computational cost as we need to calculate $N^2$ edges. To address this problem, negative sampling~\cite{mikolov2013distributed} is adopted, i.e., for each $v_j \in \mathcal{N}(v_i)$, we randomly sample $Q$ nodes that's not connected to $v_i$ and use them as negative samples. 

However, a small portion of edges in $\mathbf{A}$ are noisy, which might have negative effects in training the predictor. To mitigate the negative effects of noisy edges and to learn a link predictor that can assign lower weights to edges that link dissimilar nodes, we propose to reweight the positive and negative samples based on the feature similarity of two nodes. Specifically, for node  $v_i$ and its positive sample $v_j \in \mathcal{N}(v_i)$, we minimize $\exp(-\frac{\|\mathbf{x}_i-\mathbf{x}_j\|^2}{\sigma^2}) (w(i,j)-1)^2$, where $\sigma$ is the hyperparameter to control the variance of the sample weights. Thus, if the node features of $v_i$ and $v_j$ are similar, $A_{ij}$ is likely to be a clean edge and $\exp(-\frac{\|\mathbf{x}_i-\mathbf{x}_j\|^2}{\sigma^2})$ would be large. Minimizing the loss will force $w(i,j)$ to be close to 1; while if the features are dissimilar, then $A_{ij}$ is likely to be a noisy edge and $\exp(-\frac{\|\mathbf{x}_i-\mathbf{x}_j\|^2}{\sigma^2})$ would be small, thus minimizing the loss will have little effect on $w(i,j)$. Similarly, for $v_i$ and its negative sample $v_n$, we minimize $\exp(\frac{\|\mathbf{x}_i-\mathbf{x}_n\|^2}{\sigma^2}) (w(i,n)-0)^2$. If the node features of $v_i$ and $v_n$ are dissimialr, then $\exp(\frac{\|\mathbf{x}_i-\mathbf{x}_n\|^2}{\sigma^2})$ is large, minimizing the loss would make $w(i,n)$ close to 0 as expected. With the weight defined in this way, the loss for training the link predictor is:
\begin{equation}
\small
\begin{aligned}
    \mathcal{L}_E = \sum_{v_i \in \mathcal{V}} & \sum_{v_j \in \mathcal{N}(v_i)} \Big[\exp(-\frac{\|\mathbf{x}_i-\mathbf{x}_j\|^2}{\sigma^2}) (w(i,j)-1)^2 \\
    & +   \sum_{n=1}^{Q} \cdot \mathbb{E}_{v_n \sim P_n(v_i)} \exp(\frac{\|\mathbf{x}_i-\mathbf{x}_n\|^2}{\sigma^2}) (w(i,n)-0)^2\big],
    \label{eq:edge}
\end{aligned}
\end{equation}
where $P_n(v_i)$ is the distribution of sampling negative nodes for $v_i$, which is a uniform distribution. With the loss function Eq.(\ref{eq:edge}), the link predictor would be able to downweight the noisy edges and densify the graph to facilitate the learning of robust GNN on noisy graph with limited labels. 

\noindent\textbf{Graph Denoising and Densification.} 
With the link predictor, we could apply the learned weights to the existing edges and drop edges whose predicted weights are small to eliminate the negative effects of noisy/adversarial edges. Moreover, to increase the involvement of unlabeled nodes to facilitate the message passing of GNNs, we also link nodes that have large weights predicted by the link predictor. However, if we predict weights of all pairs of nodes, the computation cost will be very large because we will train a link predictor and a GNN classifier end-to-end as shown in Sec.~\ref{sec:4_4}, which means we need to do prediction in each iteration. To save the computational cost, for each node $v_i$, we first construct a candidate subset $\mathcal{S}(v_i)$, which contains $K$ nodes having the largest cosine similarities with $v_i$ in the raw feature space $\mathbf{X}$. Note that this only needs to be done once. Since nodes not in $\mathcal{S}(v_i)$ are not likely to be connected with $v_i$, we only need to compute weights between $v_i$ and $\mathcal{S}(v_i)$. The whole process of obtaining a clean and dense adjacency matrix $\mathbf{S}$ could be formally stated as:
\begin{equation}
    \mathbf{S}_{ij} = \left\{ \begin{array}{ll}
         w(i,j) & \mbox{if $w(i,j) > T_l$ and $v_j \in \mathcal{N}(v_i) \cup \mathcal{S}(v_i)$} ;\\
        0 & \mbox{else},\end{array} \right.
        \label{eq:generate_graph}
\end{equation}
where $\mathcal{N}(v_i)$ are neighbors of $v_i$ in the noisy graph, and $T_l$ is a threshold to determine whether we should keep/add the edge.  With the above operation, those noisy edges would be assigned smaller weights or even dropped, which mitigate the negative effects of noisy edges. Meanwhile, more edges are introduced to facilitate the message passing of GNNs during training.

\subsection{GNN for Node Classification}
With the learned adjacency matrix $\mathbf{S}$, we can apply GNNs to learn the node representation as $\mathbf{H} = GNN(\mathbf{S}, \mathbf{X})$. 
% \begin{equation}
%     \mathbf{H} = GNN(\mathbf{S}, \mathbf{X})
% \end{equation}
Note that the proposed framework is a flexible framework which can facilitate various GNNs such as GAT~\cite{velivckovic2017graph} and GIN~\cite{xu2018powerful}. With the node representation, the label of node $v_i$ can be predicted as
$\hat{y}_i = softmax(\mathbf{h}_i)$, where $\mathbf{h}_i$ is the representation of node $v_i$. Then, the training loss is:
\begin{equation}
\small
    \mathcal{L}_{GNN} = \sum_{v_i \in \mathcal{V}_L} l(\mathbf{\hat{y}}_i, \mathbf{y}_i)
    \label{eq:GNN_dense}
\end{equation}
where $l(\mathbf{\hat{y}}_i, \mathbf{y}_i)$ is the cross entropy between $\hat{y}_i$ and $\mathbf{y}_i$. Since $\mathbf{S}$ is denser than the original graph, more unlabeled nodes are involved in the training even with limited amount of labeled nodes, thus making the propagation of information more efficient.

\subsection{Label Smoothness on Unlabeled Nodes}
%As it is discussed in Sec. \ref{sec:3_3}, purely relying on densifying the graphs could not ensure all the unlabeled nodes participate the training process when the available labels is extremely limited. 
Though the dense graph $\mathbf{S}$ can help to include more unlabeled nodes in the loss function, their information is propagated through the message-passing mechanism instead of being directly used in the training loss. To further alleviate the issue of limited labeled nodes, we  propose to adopt the predicted weighted edges for label smoothness regularization. The basic idea is the larger weights of an edge $S_{ij}$ is, the more likely that $v_i$ and $v_j$ have the same label~\cite{wang2019knowledge}.
% To ensure the unlabeled nodes are involved, we propose to further apply label smoothness regularization to unlabeled nodes and their neighbors to utilize the self-supervision from the generated graph adjacency matrix. 
%The label smoothness posits the adjacent items in the graph are likely to have similar labels~\cite{wang2019knowledge,zhang2007hyperparameter}. Since the graph might be perturbed to link nodes with different labels, the label smoothness is based on the generated graph instead of the provided graph.The nodes predicted to be likely linked by link predictor are forced to have similar predictions. 
Thus, for an unlabeled node $v_i$, if its edge weight with node $v_j$ is larger than a threshhold $T_h$, i.e., $S_{ij} > T_h$, we want their predicted labels to be similar with each other. This can be formally written as
%This self-supervision term can be written as:
\begin{equation}
    \mathcal{L}_u = \sum_{v_i \in \mathcal{V}_u}\sum_{v_j \in \mathcal{V}} \mathbf{T}_{ij} \|\mathbf{\hat{y}}_i-\mathbf{\hat{y}}_j\|^2,
\end{equation}
where $\mathcal{V}_u$ denotes the set of unlabeled nodes, $\mathbf{\hat{y}}_i$ and $\mathbf{\hat{y}}_j$ represent the predictions of node $v_i \in \mathcal{V}_u$ and $v_j \in \mathcal{V}$, respectively. $\mathbf{T}_{ij}=\mathbf{S}_{ij}$ if $\mathbf{S}_{ij}>T_h$; otherwise 0.
In this way, we explicitly smooth the predicted labels between unlabeled nodes and nodes that are similar to them. By including $\mathbf{T}_{ij}$ in $\mathcal{L}_u$, edge weights are also considered. 

\subsection{Final Objective Function of RS-GNN} \label{sec:4_4}
With the link predictor denoising and densifying the graph with the supervision from $\mathbf{A}$, the GNN adopting the learned graph for label prediction and the label smoothness regularization from the generated graph, the final loss function can be written as
\begin{equation}
    \mathop{\arg \min}_{\theta_E,\theta_{\mathcal{G}}} \mathcal{L}_{GNN} + \alpha \mathcal{L}_E + \beta \mathcal{L}_u,
    \label{eq:final}
\end{equation}
where $\theta_E$ and $\theta_{\mathcal{G}}$ are parameters of link predictor $f_E$ and GNN classifier $f_\mathcal{G}$, respectively. $\alpha$ and $\beta$ are hyperparameters to balance the contributions of reconstructing the adjacency matrix with $f_E$ and label smoothness regularization. The proposed framework is an end-to-end framework where we simultaneously learn the link predictor and utilize the predicted edges for training a robust GNN to alleviate the noisy graph and limited labeled nodes issues. The training algorithm is shown in the supplementary material.
% \suhang{The training algorithm of RS-GNN is presented in XXX. We need to mention the supplementary material}

% \subsection{A Training Algorithm of RS-GNN}
% The training algorithm of RS-GNN is presented in Algorithm \ref{alg:Framwork}. In line 1, link predictor $f_E$ and GCN classifier $f_{\mathcal{G}}$ are randomly initialized with Xavier initialization~\cite{glorot2010understanding}. In line 2, we generate the graph with $f_E$. 
% Then the link predictor and GCN classifier are jointly trained in an end-to-end manner by Eq. (\ref{eq:final}) in line 3. Adam optimizer~\cite{kingma2014adam} with learning rate set as 0.001 is applied to update the parameters of link predictor and GCN classifier.
% \label{sec:app_alg}
% \begin{algorithm}[t] 
% \caption{ Training Algorithm of RS-GNN.} 
% \label{alg:Framwork} 
% \begin{algorithmic}[1]
% \REQUIRE
% $\mathcal{G}=(\mathcal{V},\mathcal{E}, \mathbf{X})$, $\mathcal{Y}$, $K$, $Q$ $T_l$, $T_h$, $\sigma$, $\alpha$ and $\beta$.
% \ENSURE $f_{\mathcal{G}}$ and $f_E$
% \STATE Randomly initialize the parameters of $f_{\mathcal{G}}$ and $f_E$.
% \REPEAT 
% % \STATE Randomly select $Q$ negative samples for each node
% \STATE Obtain the generated graph $\mathbf{S}$ with $f_E$ by Eq.(\ref{eq:generate_graph}).
% \STATE Jointly optimize the GCN classifier parameters $\theta_{\mathcal{G}}$ and the link predictor parameters $\theta_E$ by Eq.(\ref{eq:final}). 

% \UNTIL convergence
% \RETURN $f_{\mathcal{G}}$ and $f_E$
% \end{algorithmic}
% \end{algorithm}




\section{Experiments}

\label{Sec:experiments}



\begin{table*}[t]
    \small
    \centering
    \caption{Node classification performance (Accuracy(\%)$\pm$Std) on various types of noisy graphs}
    \vskip -1.5em
    \begin{tabularx}{0.985\textwidth}{|p{0.05\textwidth}|p{0.14\textwidth}|CC>{\centering\arraybackslash}p{0.1\linewidth}C>{\centering\arraybackslash}p{0.1\linewidth}>{\centering\arraybackslash}p{0.08\linewidth}CC|}
    \hline
    Dataset & Graph & GCN & SuperGAT &Self-Training & RGCN & GCN-jaccard & GCN-SVD & Pro-GNN & Ours \\
    \hline
    
    \multirow{4}{*}{Cora}
        &Raw Graph            & 65.5 $\pm 0.5$& 69.0 $\pm 1.7$ & 67.9 $\pm 0.9$ & 63.0 $\pm 0.7$ &65.7 $\pm 0.6$ & 62.9 $\pm 1.1$  & 65.9 $\pm 1.3$ & \textbf{75.3} $\pm \textbf{0.6}$\\
        &Random Noise        & 59.2 $\pm 0.7$ & 58.8 $\pm 0.4$ & 63.1 $\pm 0.5$ &51.5 $\pm 0.7$ & 57.8 $\pm 1.4$ & 51.5 $\pm 0.7$ & 56.1 $\pm 3.0$ & \textbf{71.8} $\pm \textbf{1.5}$\\
        &Non-Targeted Attack  & 26.8 $\pm 2.5$ & 41.5 $\pm 1.6$ & 29.6 $\pm 0.4$ &30.4 $\pm 1.0$ & 48.3 $\pm2.0$ & 37.1 $\pm 1.4$ & 41.7 $\pm 5.7$& \textbf{70.8} $\pm \textbf{0.7}$  \\
        &Targeted Attack      & 45.3 $\pm 1.2$& 44.4 $\pm 1.3$ &46.7 $\pm 2.1$ &40.3 $\pm 1.0$ & 49.5 $\pm 1.0$ & 44.8 $\pm 0.7$ & 49.7 $\pm 0.9$ & \textbf{67.8} $\pm \textbf{1.2}$ \\

    \hline
    \multirow{4}{*}{Cora-ML}
        &Raw Graph         & 72.4 $\pm 0.8$ & 73.8 $\pm 1.4$ & 72.7 $\pm 1.4$ & 72.9 $\pm 0.7$ & 71.0 $\pm 1.2 $ & 71.1 $\pm 1.0$ & 62.0 $\pm 1.5$ & \textbf{75.6} $\pm \textbf{0.4}$\\
        &Random Noise       & 62.3 $\pm 0.6$ & 63.7 $\pm 0.9$ & 62.8 $\pm 1.3$ & 61.4 $\pm 1.1$ & 61.3 $\pm 0.5$ & 62.6 $\pm 0.6$ & 57.1 $\pm 2.1$ & \textbf{72.9} $\pm \textbf{0.7}$\\
        &Non-Targeted Attack & 13.2 $\pm 1.4$ & 18.6 $\pm 1.5$ & 15.0 $\pm 0.7$ & 11.0 $\pm 1.0$ & 48.9 $\pm 5.3$ & 16.3 $\pm 0.6$ & 18.2 $\pm 2.4$ & \textbf{73.2} $ \pm \textbf{1.2}$ \\
        &Targeted Attack     & 55.7 $\pm 0.7$ & 56.5 $\pm 1.7$ & 57.7 $\pm 1.2$ & 54.6 $\pm 0.6$ & 61.2 $\pm 0.9$ & 53.0 $\pm 0.8$ & 55.1 $\pm 1.6$ & \textbf{70.8} $\pm \textbf{0.7}$ \\

    \hline
    \multirow{4}{*}{Citeseer}
        &Raw Graph           & 64.8 $\pm 1.4$ & 64.2 $\pm 1.7$ & 65.7 $\pm 1.1$ & 56.6 $\pm 1.2$ & 62.2 $\pm 2.0$ & 61.3 $\pm 2.0$ & 60.6 $\pm 2.0 $  & \textbf{71.2} $\pm \textbf{1.4}$\\
        &Random Noise       & 57.0 $\pm 1.2$ & 54.6 $\pm 1.3$ & 58.7 $\pm 2.1$ & 48.2 $\pm 1.2$ & 61.1 $\pm 2.8$ & 48.3 $\pm 1.6$ & 54.4 $\pm 2.6$ & \textbf{68.8} $\pm \textbf{1.5}$\\
        &Non-Targeted Attack & 26.6 $\pm 2.5$ & 42.3 $\pm 2.6$ & 28.8 $\pm 2.7$ &26.6 $\pm 1.1$ & 57.9 $\pm 2.7$ & 41.7 $\pm 1.6$ & 41.6 $\pm 3.1$ & \textbf{68.0} $\pm \textbf{0.4}$ \\
        &Targeted Attack     & 43.9 $\pm 1.7$ & 42.9 $\pm 0.4$ & 47.6 $\pm 1.2$ &35.3 $\pm 1.5$ & 52.5 $\pm 2.3$ & 40.5 $\pm 0.7$ & 48.1 $\pm 1.6$ & \textbf{67.2} $\pm \textbf{1.3}$\\
    \hline
    \multirow{4}{*}{Pubmed}
    & Raw Graph          & 85.9 $\pm 0.1$ & 86.0 $\pm 1.2$ & 86.1 $\pm 0.2$ & 85.1 $\pm 0.1$ & 86.0 $\pm 0.1$  & 83.0 $\pm 0.1$ & 86.1 $\pm 0.1$ & \textbf{86.9} $\pm \textbf{0.1}$  \\
    & Random Noise & 80.5 $\pm 0.1$ & 79.8 $\pm 0.1$ & 81.2 $\pm 0.2$ & 79.7 $\pm 0.1$ &  83.0 $\pm 0.1$  & 82.0 $\pm 0.1$ &  85.1 $\pm 0.2$ &  \textbf{86.4} $\pm \textbf{0.1}$\\
    & Non-Targeted Attack & 73.7 $\pm 0.2$ & 73.8 $\pm 0.2$ & 73.5 $\pm 0.3$ & 73.8 $\pm 0.3$ & 84.4 $\pm 0.1$ & 83.0 $\pm 0.1$ & 86.0 $\pm$ 0.1 & \textbf{86.3} $\pm \textbf{0.1}$ \\
    & Targeted Attack & 76.5 $\pm 0.1$ & 75.6 $\pm 0.1$ & 76.8 $\pm 0.2$ & 76.2 $\pm 0.2$  & 82.7 $\pm 0.2$ &78.1 $\pm 1.3$ & 79.1 $\pm 0.1$ & \textbf{84.3} $\pm \textbf{0.2}$ \\
    \hline
    \end{tabularx}
    
    \label{tab:results}
    \vskip -1.2em
\end{table*}

In this section, we evaluate the proposed RS-GNN on noisy graphs with limited labels to answer the following research questions:
\begin{itemize}[leftmargin=*]
    \item \textbf{RQ1} How robust is the proposed framework on various types of noisy graphs with limited labeled nodes?
    \item \textbf{RQ2} How does the proposed framework perform under various label rates and graph sparsity levels?
    \item \textbf{RQ3} What are the contributions of link predictor and label smoothness regularization from predicted edges on RS-GNN?
\end{itemize}
\subsection{Experimental Settings}
\label{Sec:ex_settings}


\subsubsection{Datasets} 
\label{Sec:datasets}
For a fair comparison, we conduct experiments on four widely used benchmark datasets, i.e., Cora, Cora-ML, Citeseer and Pubmed~\cite{sen2008collective}.
The statistics of the datasets are presented in the Table \ref{tab:dataset} in Appendix. Note that the split of validation and testing on all datasets are the same as described in the cited papers to keep consistence. For the training set, we randomly sample 1\% of nodes as the labeled set for Cora, Cora-ML and Citeseer. For Pubmed, we randomly sample 10\% of nodes to compose the labeled set. The training node set doesn't overlap with the validation and test sets. 

\subsubsection{Noisy Graphs}
To show RS-GNN is robust to various structural noises, we evaluate RS-GNN on the following types of noises:
\begin{itemize}[leftmargin=*]
    \item \textbf{Raw Graphs}: They are the original graphs of the benchmark datasets which may contain inherent structural noise.
    \item \textbf{Random Noise}: We randomly inject fake edges and remove normal edges to add random noise to graphs.
    \item \textbf{Non-Targeted Attack}: 
    We adopt \textit{metattack}~\cite{zugner2019adversarial} to poison the graph structures by adding and removing edges, which aims to reduces the overall performance of GNNs on the whole graph. 
    \item \textbf{Targeted Attack}: It aims to lead the GNN to misclassify target nodes. Following~\cite{tang2020transferring}, we randomly select 15\% nodes as target nodes and apply \textit{nettack}~\cite{zugner2018adversarial} to perturb the graph structure. 

\end{itemize}




\subsubsection{Baselines} We compare RS-GNN with the representative and state-of-the-art GNNs, and robust GNNs against adversarial attacks:
\begin{itemize}[leftmargin=*]
    \item \textbf{GCN}~\cite{kipf2016semi}: GCN is a representative GNN which defines Graph convolution with spectral analysis.
     \item \textbf{SuperGAT}~\cite{kim2021find}: This extends GAT~\cite{velivckovic2017graph} with self-supervised learning. Edge prediction is deployed as the pretext task to guide the learning of attention to facilitate the message-passing.
    \item \textbf{Self-Training}~\cite{li2018deeper}: This is a self-supervised learning method. A GCN is firstly trained on given labels. Then, confident pseudo labels would be added to the label set to improve the GCN.
    \item \textbf{RGCN}~\cite{zhu2019robust}: It uses Gaussian distributions as representations to absorb the effects of adversarial edges. 
    \item \textbf{GCN-jaccard}~\cite{wu2019adversarial}: GCN-Jaccard eliminates edges that connect nodes with low Jaccard similarity, then apply GCN on the graph.
    \item \textbf{GCN-SVD}~\cite{entezari2020all}: This preprocessing method is based on low rank assumption. Low-rank approximation of the perturbed graph is used to train GNNs against adversarial attacks.
    \item \textbf{Pro-GNN}~\cite{jin2020graph}: It applies low-rank and sparsity constraints to learn a clean graph structure close to the noisy graph structure. 
\end{itemize}
For all the baselines, we use the implementation from the repository DeepRobust~\cite{li2020deeprobust}. All the hyperparameters of the baselines are tuned on the validation set to make a fair comparison with RS-GNN.

\subsubsection{Implementation Details}
\label{sec:implementation}
\textit{Each experiment is conducted 5 times} and average results with standard deviations are reported. The hyperparameters are tuned based on the performance of validation set. More specifically, for RS-GNN, we vary $\alpha$ as  \{0.003, 0.03, 0.3, 3, 30 \}, and $\beta$ as \{0.01, 0.03, 0.1, 0.3, 1\}. For all experiments, $T_l$, $T_h$, $\sigma$, and $Q$ are fixed as 0.1, 0.8, 100, and 50, respectively. $K$ is set as 100, 300, 400 and 10 for Cora, Cora-ML, Citeseer and Pubmed, respectively. More details about the hyperparameters sensitivity is discussed in Sec. \ref{Sec:para_analysis}.
A one-hidden layer MLP with 64 filters is applied as the link predictor. We use GCN as the backbone of RS-GNN. Various GNNs can be used in RS-GNN and we leave it as a future work. 


\begin{figure}[t]
\centering
\begin{subfigure}{0.49\columnwidth}
    \centering
    \includegraphics[width=0.85\linewidth]{figure/meta_ptb.pdf} 
    \vskip -0.5em
    \caption{Metattack}
    % \label{fig:1_a}
\end{subfigure}
%\vspace{-1em}
\begin{subfigure}{0.49\columnwidth}
    \centering
    \includegraphics[width=0.85\linewidth]{figure/random_ptb.pdf} 
    \vskip -0.5em
    \caption{Random Noise}
    % \label{fig:1_b}
\end{subfigure}
\vspace{-1.2em}
\caption{Robustness under different Ptb rates on Cora.  }
\label{fig:ptb}
\vskip -1.5em
\end{figure}

\subsection{Performance on Noisy Graphs}
To answer \textbf{RQ1}, we first compare RS-GNN with the baselines on various noisy graphs. We then evaluate the performance of RS-GNN on the graphs with different levels of structural noise.



\subsubsection{Comparisons with baselines}
We conduct experiments on four types of noisy graphs, i.e., raw graphs, graphs with random noise, non-targeted attack perturbed graphs and targeted attack perturbed graphs. The perturbation rate of non-targeted attack and targeted attack is 0.15. The perturbation rate of random noise is set as 0.3. Since we focus on noisy graph with sparse labels, we set the label rates as 0.01 for Cora, Cora-ML, Citeseer and 0.1 for Pubmed. The results are reported in Table \ref{tab:results}, where we can observe:

\begin{itemize}[leftmargin=*]
    \item With limited labeled nodes, GCN even hardly performs well on raw graph, which indicates the necessity of investigating method to address the challenge of sparsely labeled graphs. Though recent GNNs such as SuperGAT and Self-Training can improve the performance with self-supervised learning, our RS-GNN still outperforms them by a large margin. This shows the effectiveness of graph densification in dealing with sparsely labeled graphs.
    \item The structural noise further degrades the performance of GCN, but its impact to RS-GNN is negligible. RS-GNN achieves better results than the state-of-the-art robust GNNs. This indicates RS-GNN could eliminate the effects of the noisy edges.
    \item Compared with the preprocessing methods and Pro-GNN, RS-GNN achieves higher accuracy on the sparsely labeled graphs perturbed by attack methods. 
    This is because the baselines only focus on eliminating potential noisy edges, which will even result in less involvement of unlabeled nodes. 
    By contrast, RS-GNN can down-weights/removes the adversarial edges to defend the adversarial attacks and densify the graph to facilitate the message passing for predictions of unlabeled nodes.
    
\end{itemize}



\subsubsection{Robustness Under Different Ptb Rates } 
To show that RS-GNN is resistant to different levels of structural noise, we vary the perturbation rate as $\{0\%, 5\%, 10\%, \dots, 25\%\}$ and compare the performance of RS-GNN with the most effective baselines. The label rate is fixed as 0.01. Since we have similar observations on other datasets, we only report the average accuracy and standard deviation on Cora in Figure \ref{fig:ptb}. From the figure, we make following observations:
\begin{itemize}[leftmargin=*]
    \item As the perturbation rate increases, the performance of all the baselines drop significantly, which is as expected. Though the performance of RS-GNN also drops, it is much stable and consistently outperforms the baselines, which shows the robustness of RS-GNN against various levels of attacks and random noise; and %  the proposed framework RS-GNN  For the noisy graph perturbed by \textit{metattack}, the performance of GCN degrades significantly. Our proposed RS-GNN could achieve high performance under high perturbation rate and consistently outperforms the baselines, which demonstrates RS-GNN could well defend attacks under different perturbation rates.
    \item  Compared with GCN, RS-GNN uses GCN as backbone but significantly outperforms GCN, especially when the perturbation rate is large, which shows the effectiveness of eliminating the effects of noisy edges and densifying the graph to benefit the predictions given limited labels. % consistently achieve high performance on the graph containing various amounts of noisy edges. It demonstrates that RS-GNN is robust when applied to different levels of random structural noises. \suhang{TODO}
\end{itemize}

\begin{figure}[t]
\centering
\begin{subfigure}{0.49\columnwidth}
    \centering
    \includegraphics[width=0.85\linewidth]{figure/random_cora_2.pdf} 
    \vskip -0.8em
    \caption{Cora}
    % \label{fig:1_a}
\end{subfigure}
%\vspace{-1em}
\begin{subfigure}{0.49\columnwidth}
    \centering
    \includegraphics[width=0.85\linewidth]{figure/random_cora_ml_2.pdf} 
    \vskip -0.8em
    \caption{CoraML}
    % \label{fig:1_b}
\end{subfigure}
\vspace{-1.5em}
\caption{ Distributions of the weights of normal and noisy edges on the generated graph.}
\label{fig:weight}
\vskip -1em
\end{figure}

\subsection{Analysis of the Learned Graph} 
To demonstrate that RS-GNN could alleviate negative effects of noisy edges by downweighting the noisy edges, we investigate the distribution of the learned edge weights $\mathbf{S}_{ij}$ of normal and noisy edges in this subsection. The edge weight  distributions of graphs perturbed by random noise with 30\% perturbation rate on Cora and Cora-ML are shown in Fig.~\ref{fig:weight}.  From this figure, we observe: (\textbf{i}) The weights of noisy edges are significantly lower than the weights of normal edges, which indicates RS-GNN manages to reduce the effects of noisy edges for robust GNN; and (\textbf{ii}) Although most normal edges have higher weights, some of their weights are very low, which implies inherent noise exists in the graph and RS-GNN is able to get rid of such inherent structural noise. 

We also provide more details about the number of involved unlabeled nodes with the learned graph in Appendix~\ref{sec:app_graph}, which proves RS-GNN can enhance the involvement of unlabeled nodes.






\begin{figure}[t]
\centering
\begin{subfigure}{0.49\columnwidth}
    \centering
    \includegraphics[width=0.9\linewidth]{figure/clean_label_rate.pdf} 
    \vskip -0.5em
    \caption{Raw Graph}
    % \label{fig:1_a}
\end{subfigure}
%\vspace{-1em}
\begin{subfigure}{0.49\columnwidth}
    \centering
    \includegraphics[width=0.9\linewidth]{figure/ptb_label_rate.pdf} 
    \vskip -0.5em
    \caption{Metattack with 15\% Ptb}
    % \label{fig:1_b}
\end{subfigure}
\vspace{-1.2em}
\caption{Performance on Cora with different label rates.  }
\label{fig:label_rate}
\vskip -0.8em
\end{figure}


\subsection{Impacts of Label Rate and Graph Sparsity}

To answer \textbf{RQ2}, we study the impacts of the number of labeled nodes and sparsity of the graph by varying the label rate and edge rate of the graph. The hyperparameters are selected with the process described in Sec. \ref{sec:implementation}. Each experiment is conducted 5 times and average accuracy with standard deviation are reported.



\subsubsection{Impacts of Label Rate} We vary label rates as \{0.01, 0.02,\dots, 0.06\}. Experiments are conducted on raw graphs and graphs perturbed by \textit{mettack} to study the effectiveness of RS-GNN under various label rates. The results on Cora are shown in Fig.~\ref{fig:label_rate}. We have similar observations on other datasets. From Fig.~\ref{fig:label_rate}, we observe:
\begin{itemize}[leftmargin=*]
    \item Generally, as the increase of label rate, the performances of all the methods increase, which is as expected.
    \item For the raw graph, though RS-GNN consistently outperforms the baselines, as the label rate increases, the improvement of RS-GNN becomes marginal. This is because the raw graph doesn't contain much noise. Thus, as label rate increases to 6\%, there are already adequate labels. Since higher label rates would result in more unlabeled nodes involving in the training, the effects of densifying graphs and label smoothness become less significant; %when the label rate is high enough to involve most of the unlabeled nodes.
    \item For the metattack graph, as the label rate increases, RS-GNN still significantly outperforms baselines. That's because the training graph contains a lot of adversarial edges. Though we have enough training labels, the adversarial edges can still contaminate the message passing of GNNs. But RS-GNN can eliminate noisy edges and densify the graph, thus having better results.
\end{itemize}



\subsubsection{Impacts of Graph Sparsity} As RS-GNN can generate dense graphs, it should have the ability to handle sparse graphs. Thus, we randomly select $x\%$ edges from the raw graph to build graphs of different sparsity levels. We vary edge rate $x\%$ from 20\% to 100\% with a step of 40\%. Since we are interested in how the sparsity of the graph could affect RS-GNN in generating dense graphs, we only focus on the performance on raw graphs. 
The average results of 5 runs on Citeseer are reported in Table~\ref{tab:sparsity}. From the table, we have the following observations:
\begin{itemize}[leftmargin=*]
    \item As the edge rate decreases, the performance of all the methods decrease, which is because message-passing of GNNs becomes ineffective on very sparse graphs;
    \item RS-GNN consistently outperforms the baselines. In particular, when the graph becomes more sparse, the improvement of RS-GNN over the baselines becomes larger. For example, the improvement of RS-GNN over GCN on Citeseer is 6.4\% when Edge Rate is 100\%, and becomes 9.2\% when Edge Rate is 20\%, which shows the importance of generating edges for densifying the graph and smoothing predictions with the learned graph. 
\end{itemize}


\begin{table}[t]
    \small
    \centering
    \caption{Accuracy (\%) on Citeseer in different sparsity levels. }
    \vskip-1.5em
    \begin{tabularx}{0.96\linewidth}{>{\centering\arraybackslash}p{0.20\linewidth}CCC}
    \toprule
    Edge Rate (\%) & GCN & Pro-GNN & RS-GNN\\
    \midrule
    % \multirow{3}{*}{Cora} 
    % & 20 & 51.9 $\pm 2.0$ & 49.5 $\pm 0.8$ & \textbf{64.7} $\pm \textbf{1.7}$\\ 
    % % & 40 & 53.9 $\pm 1.2$ & 51.2 $\pm 0.7$ & \textbf{66.3} $\pm \textbf{1.2}$ \\
    % & 60 & 62.0 $\pm 0.3$ & 62.5 $\pm 0.5$ & \textbf{68.5} $\pm \textbf{1.7}$\\
    % % & 80 & 64.2 $\pm 0.5$ & 64.6 $\pm 0.2$ & \textbf{72.8} $\pm \textbf{1.4}$ \\
    % & 100 & 65.5 $\pm 0.5$ & 65.9 $\pm 1.1$ & \textbf{75.3} $\pm \textbf{0.6}$\\
    % \hline

    20 & 54.5 $\pm 1.2$ & 55.2 $\pm 1.6$ & \textbf{63.7} $\pm \textbf{2.2}$\\
    % & 40 & 56.8 $\pm 1.1$ & 58.6 $\pm 1.5$ & \textbf{67.3} $\pm \textbf{1.8}$\\
    60 & 58.7 $\pm 1.8$ & 58.3 $\pm 2.4$ &
    \textbf{69.8} $\pm \textbf{1.1}$\\
    % & 80 & 60.1 $\pm 2.2$ &60.2 $\pm 1.3$ & \textbf{70.4} $\pm \textbf{0.7}$\\
    100 & 64.8 $\pm 1.4$ & 60.6 $\pm 2.0$ &
    \textbf{71.2} $\pm \textbf{1.4}$\\
    \bottomrule
    \end{tabularx}
    \label{tab:sparsity}
    \vskip -1.em
\end{table}




%\vspace{-1em}

\begin{figure}[h]
\centering
\begin{subfigure}{0.49\columnwidth}
    \centering
    \includegraphics[width=0.85\linewidth]{figure/neta_abl.pdf} 
    \vskip -0.5em
    \caption{Nettack}
    \label{fig:abla_neta}
\end{subfigure}
\begin{subfigure}{0.49\columnwidth}
    \centering
    \includegraphics[width=0.85\linewidth]{figure/abla_meta.pdf} 
    \vskip -0.5em
    \caption{Metattack with 15\% Ptb}
    \label{fig:abla_meta}
\end{subfigure}
\vspace{-1.3em}
\caption{Ablation studies on Cora with different label rates.}
\label{fig:abl}
\vskip -1.8em
\end{figure}
\subsection{Ablation Study}
To answer \textbf{RQ3}, we conduct ablation studies to understand the effects of graph densification, graph purification and label smoothness regularization. In RS-GNN, the link predictor densify the graph to enhance the performance on unlabeled nodes. To demonstrate the effects of adding edges with the link predictor, we remove the process of adding edges and obtain RS-GNN$\backslash$A. 
To testify the effectiveness of the label smoothness regularization based on the generated graph, we eliminate the label smoothness regularization and get RS-GNN$\backslash$U. To show our link predictor can eliminate the effects of noisy edges, we compare a variant named as RS-GNN$\backslash$AU which only use the link predictor to denoise graphs. Graph desification and label smoothness are not applied in RS-GNN$\backslash$AU. 
We also implement a variant named as RS-GNN$_{GCN}$ which uses GCN as link predictor to show that the noisy edges would largely affects the GNNs for link prediction. Hyperparameters selection follows the process in Sec~\ref{sec:implementation}. We only show the results on the Cora graph perturbed with \textit{metattack} and random noise, because similar trends are observed on other datasets. Results are presented in Fig.~\ref{fig:abl}. From this figure, we observe that: 
\begin{itemize}[leftmargin=*]
    \item RS-GNN performs much better than RS-GNN$\backslash$A and RS-GNN$\backslash$U, which shows that densifying graphs and label smoothness with the learned graph can address the label sparsity issue;
    \item With the increase of label rate, the gap between RS-GNN and RS-GNN$\backslash$U will be narrowed. This is consistent with our analysis that higher label rates would involve more unlabeled nodes;
    \item RS-GNN$_{GCN}$ performs much worse than RS-GNN, which indicates adversarial edges would impair GCN and result in a poor link predictor for denoising and densification.
\end{itemize} 


\begin{figure}[t]
\centering
\begin{subfigure}{0.49\columnwidth}
    \centering
    \includegraphics[width=0.98\linewidth]{figure/cora_para_clean_2}
    \vskip -0.5em
    \caption{Raw Graph}
    \label{fig:para_raw}
\end{subfigure}
%\vspace{-1em}
\begin{subfigure}{0.49\columnwidth}
    \centering
    \includegraphics[width=0.98\linewidth]{figure/cora_para_ptb_2}
    \vskip -0.5em
    \caption{Metattack with 15\% Ptb}
    \label{fig:para_meta}
\end{subfigure}
\vspace{-1em}
\caption{Parameter sensitivity analysis on Cora.}
\label{fig:para}
\vskip -1.7em
\end{figure}


\subsection{Parameter Sensitivity Analysis}
\label{Sec:para_analysis}
In this subsection, we explore the sensitivity of the most crucial hyperparameters $\alpha$ and $\beta$ which are in the final objective function of RS-GNN. The analysis about other hyperparameters is presented in the supplementary material. $\alpha$ controls how well the link predictor reconstructs the noisy graph and $\beta$ controls the contribution of label smoothness. To investigate the effects of $\alpha$ and $\beta$, we vary the values of $\alpha$ as $\{0.003, 0.03, 0.3, 3, 30\}$ and $\beta$ as $\{0.01, 0.03, 0.1, 0.3, 1, 3\}$ on Cora. The results are shown in Fig~\ref{fig:para}. In the raw graph, when $\alpha$ is large, the accuracy is stable and high. But if the $\alpha$ is too large in the perturbed graph, the performance would decrease. This difference is due to the noise levels of the raw graph and the perturbed graph. The structural noise in the perturbed graph is severe, faithfully reconstructing the perturbed graph with high $\alpha$ would lead to a poor link predictor. As for the $\beta$, a value between 0.03 to 0.3 generally gives good performance, which eases the parameter selection.

 We propose a novel commonsense reasoning challenge, \textsc{RiddleSense}, which requires complex commonsense skills for reasoning about creative and counterfactual questions, coming with a large multiple-choice QA dataset.  
 We systematically evaluate recent commonsense reasoning methods over the proposed \textsc{RiddleSense} dataset, and find that the best model is still far behind human performance, suggesting that there is still much space for commonsense reasoning methods to improve.
 We hope \textsc{RiddleSense} can serve as a benchmark dataset for future research targeting complex commonsense reasoning and computational creativity.


\section*{Acknowledgements}
This research is supported in part by the Office of the Director of National Intelligence (ODNI), Intelligence Advanced Research Projects Activity (IARPA), via Contract No. 2019-19051600007, the DARPA MCS program under Contract No. N660011924033 with the United States Office Of Naval Research, the Defense Advanced Research Projects Agency with award W911NF-19-20271, and NSF SMA 18-29268. The views and conclusions contained herein are those of the authors and should not be interpreted as necessarily representing the official policies, either expressed or implied, of ODNI, IARPA, or the U.S. Government. We would like to thank all the collaborators in USC INK research lab and the reviewers for their constructive feedback on the work.



%% The file named.bst is a bibliography style file for BibTeX 0.99c
\bibliographystyle{ACM-Reference-Format}
\bibliography{ref}

\newpage
\newpage
\appendix
\section{Pricing equations}
\subsection{Credit default swap}
\label{CDS_pricing}
A credit default swap (CDS) is a contract designed to exchange credit risk of a Reference Name (RN) between a Protection Buyer (PB) and a Protection Seller (PS). PB makes periodic coupon payments to PS conditional on no default of RN, up to the nearest payment date, in the exchange for receiving from PS the loss given RN's default.

Consider a CDS contract written on the first bank (RN), denote its price $C_1(t, x)$.\footnote{For the CDS contracts written on the second bank, the similar expression could be provided by analogy.} We assume that the coupon is paid continuously and equals to $c$. Then, the value of a standard CDS contract can be given (\cite{BieleckiRutkowski}) by the solution of  (\ref{kolm_1})--(\ref{kolm_2})  with $\chi(t, x) = c$ and terminal condition
\begin{equation*}
	\psi(x) = 
	\begin{cases}
		1 - \min(R_1, \tilde{R}_1(1)), \quad (x_1, x_2) \in D_2, \\
		1 - \min(R_1, \tilde{R}_1(\omega_2)), \quad (x_1, x_2) \in D_{12}, \\		
	\end{cases}
\end{equation*}
where $\omega_2 = \omega_2(x)$ is defined in (\ref{term_cond}) and 
\begin{equation*}
	\tilde{R}_1(\omega_2) = \min \left[1, \frac{A_1(T) +  \omega_2 L_{2 1}(T)}{L_1(T) + \omega_2 L_{12}(T)}\right].
\end{equation*}
Thus, the pricing problem for CDS contract on the first bank is
\begin{equation}
\begin{aligned}
		& \frac{\partial}{\partial t} C_1(t, x) + \mathcal{L} C_1(t, x) = c, \\
		& C_1(t, 0, x_2) = 1 - R_1, \quad C_1(t, \infty, x_2) = -c(T-t), \\
		& C_1(t, x_1, 0) = \Xi(t, x_1) = 
		\begin{cases}
			c_{1,0}(t, x_1), & x_1 \ge \tilde{\mu}_1, \\
			1-R_1, & x_1 < \tilde{\mu}_i,
		\end{cases} \quad C_1(t, x_1, \infty) = c_{1,\infty}(t, x_1),\\
		& C_1(T, x) = \psi(x) = 
	\begin{cases}
		1 - \min(R_1, \tilde{R}_1(1)), \quad (x_1, x_2) \in D_2, \\
		1 - \min(R_1, \tilde{R}_1(\omega_2)), \quad (x_1, x_2) \in D_{12}, \\		
	\end{cases}
\end{aligned}
\end{equation}
where $c_{1,0}(t, x_1)$ is the solution of the following boundary value problem:
\begin{equation}
\begin{aligned}
		& \frac{\partial}{\partial t} c_{1, 0}(t, x_1) + \mathcal{L}_1 c_{1, 0}(t, x_1) = c, \\
		& c_{1, 0}(t, \tilde{\mu}_1^{<}) = 1 - R_1, \quad c_{1, 0}(t, \infty) = -c(T-t), \\
		& c_{1, 0}(T, x_1) = (1 - R_1) \mathbbm{1}_{\{\tilde{\mu}_1^{<} \le x_1 \le \tilde{\mu}_1^{=}\}}, 
\end{aligned}
\end{equation}
and $c_{1,\infty}(t, x_1)$ is the solution of the following boundary value problem
\begin{equation}
\begin{aligned}
		& \frac{\partial}{\partial t} c_{1, \infty}(t, x_1) + \mathcal{L}_1 c_{1, \infty}(t, x_1) = c, \\
		& c_{1, \infty}(t, 0) = 1 - R_1, \quad c_{1, \infty}(t, \infty) = -c(T-t), \\
		& c_{1, \infty}(T, x_1) = (1 - R_1) \mathbbm{1}_{\{x_1 \le \mu_1^{=}\}}.
\end{aligned}
\end{equation}

\subsection{First-to-default swap}
An FTD contract refers to a basket of reference names (RN). Similar to a regular CDS, the Protection Buyer (PB) pays a regular coupon payment $c$ to the Protection Seller (PS) up to the first default of any of the RN in the basket or maturity time $T$. In return, PS compensates PB the loss caused by the first default.

Consider the FTD contract referenced on $2$ banks, and denote its price $F(t, x)$. We assume that the coupon is paid continuously and equals to $c$. Then, the value of FTD contract can be given (\cite{LiptonItkin2015}) by the solution of  (\ref{kolm_1})--(\ref{kolm_2})  with $\chi(t, x) = c$ and terminal condition
\begin{equation*}
	\psi(x) = \beta_0  \mathbbm{1}_{\{x \in D_{12}\}} + \beta_1 \mathbbm{1}_{\{x \in D_{1}\}} + \beta_2 \mathbbm{1}_{\{x \in D_{2}\}},
\end{equation*}
where
\begin{equation*}
	\begin{aligned}
		\beta_0 = 1 - \min[\min(R_1, \tilde{R}_1(\omega_2), \min(R_2, \tilde{R}_2(\omega_1)], \\
		\beta_1 = 1 - \min(R_2, \tilde{R}_2(1)), \quad \beta_2 = 1 - \min(R_1, \tilde{R}_1(1)),
	\end{aligned}
\end{equation*}
and
\begin{equation*}
	\tilde{R}_1(\omega_2) = \min \left[1, \frac{A_1(T) +  \omega_2 L_{2 1}(T)}{L_1(T) + \omega_2 L_{12}(T)}\right], \quad \tilde{R}_2(\omega_1) = \min \left[1, \frac{A_2(T) +  \omega_1 L_{1 2}(T)}{L_2(T) + \omega_1 L_{21}(T)}\right].
\end{equation*}
with $\omega_1 = \omega_1(x)$ and $\omega_2 = \omega_2(x)$ defined in (\ref{term_cond}).

Thus, the pricing problem for a FTD contract is
\begin{equation}
\begin{aligned}
		& \frac{\partial}{\partial t} F(t, x) + \mathcal{L} F(t, x) = c, \\
		& F(t, x_1, 0) = 1 - R_2,  \quad F(t, 0, x_2) = 1 - R_1, \\
		& F(t, x_1, \infty) = f_{2,\infty}(t, x_1), \quad F(t, \infty, x_2) = f_{1,\infty}(t, x_2), \\
		& F(T, x) = \beta_0  \mathbbm{1}_{\{x \in D_{12}\}} + \beta_1 \mathbbm{1}_{\{x \in D_{1}\}} + \beta_2 \mathbbm{1}_{\{x \in D_{2}\}},
\end{aligned}
\end{equation}
where $f_{1,\infty}(t, x_1)$ and $f_{2,\infty}(t, x_2)$ are the solutions of the following boundary value problems
\begin{equation}
\begin{aligned}
		& \frac{\partial}{\partial t} f_{i, \infty}(t, x_i) + \mathcal{L}_i f_{i, \infty}(t, x_i) = c, \\
		& f_{i, \infty}(t, 0) = 1 - R_i, \quad f_{i, \infty}(t, \infty) = -c(T-t), \\
		& f_{1, \infty}(T, x_i) = (1 - R_i) \mathbbm{1}_{\{x_i \le \mu_i^{=}\}}.
\end{aligned}
\end{equation}

\subsection{Credit and Debt Value Adjustments for CDS}

Credit Value Adjustment and Debt Value Adjustment can be considered either unilateral or bilateral. For unilateral counterparty risk, we need to consider only two banks (RN, and PS for CVA and PB for DVA), and a two-dimensional problem can be formulated, while bilateral counterparty risk requires a three-dimensional problem, where Reference Name, Protection Buyer, and Protection Seller are all taken into account. We follow \cite{LiptonSav} for the pricing problem formulation but include jumps and mutual liabilities, which affects the boundary conditions.

\paragraph{Unilateral CVA and DVA}
The Credit Value Adjustment represents the additional price associated with the possibility of a counterparty's default. Then, CVA can be defined as
\begin{equation}
	V^{CVA} = (1- R_{PS}) \mathbb{E}[\mathbbm{1}_{\{\tau^{PS} < \min(T, \tau^{RN}) \}} (V_{\tau^{PS}}^{CDS})^{+} \, | \mathcal{F}_t],
\end{equation}
where $R_{PS}$ is the recovery rate of PS, $\tau^{PS}$ and $\tau^{RN}$ are the default times of PS and RN, and $V_t^{CDS}$ is the price of a CDS without counterparty credit risk.

We associate $x_1$ with the Protection Seller and $x_2$ with the Reference Name, then CVA can be given by the solution of  (\ref{kolm_1})--(\ref{kolm_2})  with $\chi(t, x) = 0$ and $\psi(x) = 0$. Thus,
\begin{equation}
\begin{aligned}
		& \frac{\partial}{\partial t} V^{CVA}+ \mathcal{L} V^{CVA} = 0, \\
		& V^{CVA}(t, 0, x_2) = (1 - R_{PS}) V^{CDS}(t, x_2)^{+}, \quad V^{CVA}(t, x_1, 0) = 0, \\
		& V^{CVA}(T, x_1, x_2) = 0.
\end{aligned}
\end{equation}

Similar, Debt Value Adjustment represents the additional price associated with the default and defined as
\begin{equation}
	V^{DVA} = (1- R_{PB}) \mathbb{E}[\mathbbm{1}_{\{\tau^{PB} < \min(T, \tau^{RN}) \}} (V_{\tau^{PB}}^{CDS})^{-} \, | \mathcal{F}_t],
\end{equation}
where $R_{PB}$ and $\tau^{PB}$ are the recovery rate and default time of the protection buyer.

Here, we associate $x_1$ with the Protection Buyer and $x_2$ with the Reference Name, then, similar to CVA,  DVA can be given by the solution of  (\ref{kolm_1})--(\ref{kolm_2}),
\begin{equation}
\begin{aligned}
		& \frac{\partial}{\partial t} V^{DVA}+ \mathcal{L} V^{DVA} = 0, \\
		& V^{DVA}(t, 0, x_2) = (1 - R_{PB}) V^{CDS}(t, x_2)^{-}, \quad V^{DVA}(t, x_1, 0) = 0, \\
		& V^{DVA}(T, x_1, x_2) = 0.
\end{aligned}
\end{equation}

\paragraph{Bilateral CVA and DVA}

When we defined unilateral CVA and DVA, we assumed that either protection  buyer, or protection seller are risk-free. Here we assume that they are both risky. Then, 
The Credit Value Adjustment represents the additional price associated with the possibility of counterparty's default and defined as
\begin{equation}
	V^{CVA} = (1 - R_{PS}) \mathbb{E}[\mathbbm{1}_{\{\tau^{PS} < \min(\tau^{PB}, \tau^{RN}, T)\}} (V^{CDS}_{\tau^{PS}})^{+} \, | \mathcal{F}_t],
\end{equation} 

Similar, for DVA
\begin{equation}
	V^{DVA} = (1 - R_{PB}) \mathbb{E}[\mathbbm{1}_{\{\tau^{PB} < \min(\tau^{PS}, \tau^{RN}, T)\}} (V^{CDS}_{\tau^{PB}})^{-} \, | \mathcal{F}_t],
\end{equation} 


We associate $x_1$ with protection seller, $x_2$ with protection buyer, and $x_3$ with reference name. Here, we have a three-dimensional process. Applying three-dimensional version of (\ref{kolm_1})--(\ref{kolm_2}) with $\psi(x) = 0, \chi(t, x) = 0$, we get
\begin{equation}
	\label{CVA_pde}
\begin{aligned}
		& \frac{\partial}{\partial t} V^{CVA} + \mathcal{L}_3 V^{CVA} = 0, \\
		& V^{CVA}(t, 0, x_2, x_3) = (1 - R_{PS}) V^{CDS}(t, x_3)^{+}, \\
		& V^{CVA}(t, x_1, 0, x_3 ) = 0, \quad V^{CVA}(t, x_1, x_2, 0)  = 0, \\
		& V^{CVA}(T, x_1, x_2, x_3) = 0,
\end{aligned}
\end{equation}
and
\begin{equation}
\label{DVA_pde}
\begin{aligned}
		& \frac{\partial}{\partial t} V^{DVA} + \mathcal{L}_3 V^{DVA} = 0, \\
		& V^{DVA}(t, 0, x_2, x_3) = (1 - R_{PB}) V^{CDS}(t, x_3)^{-}, \\
		& V^{DVA}(t, x_1, 0, x_3 ) = 0, \quad V^{DVA}(t, x_1, x_2, 0)  = 0, \\
		& V^{DVA}(T, x_1, x_2, x_3) = 0,
\end{aligned}
\end{equation}
where $\mathcal{L}_3 f$ is the three-dimensional infinitesimal generator.




\end{document}

