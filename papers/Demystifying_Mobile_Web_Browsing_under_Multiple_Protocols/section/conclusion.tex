\section{Conclusion}

In this paper, we want to analyze how Web pages performance vary over various parameter settings, including RTT, bandwidth, loss rate, number of objects on a page, and objects sizes, with different protocols. We conduct our measurement study of Web page performance with HTTP, HTTPS, SPDY, and HTTP/2. First, we compare these four protocols as transfer protocols with our own client and conduct a decision tree analysis to find out how those parameter settings affect these protocols` efficiency of transferring Web contents hosted in remote servers. We use Linux Traffic Control (TC) to emulate different network to ensure browsers load pages under consistent network conditions for each protocol. We find that HTTP/2 and SPDY perform worse when packet loss rate is high, but help with many objects when packet loss rate is low. Then, we use real browsers to load pages with these protocols to analyze how these protocols work in real browsers. We clone the landing pages of the Alexa top 200 websites that have corresponding mobile version into our local server and convert all external links to local links. Overall, we collect 400 Web pages, consisting of 200 desktop pages and 200 corresponding mobile pages. We find that mobile pages have better PLTs that respective desktop pages, and HTTP/2 helps under low packet loss rate, and performs better when loading mobile pages. Meanwhile, we conduct our previous controlled experiments on a older device to observer if different devices could affect Web page performance. We find similar performance for all four protocols under different network conditions. We infer that mobile browsers are not able to take full advantage of high-performance CPU when exceeding a threshold.