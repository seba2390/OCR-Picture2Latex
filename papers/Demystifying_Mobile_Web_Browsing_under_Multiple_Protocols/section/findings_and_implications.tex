\section{Findings and Implications}
In this paper, we want to analyze how Web pages performance vary over various parameter settings with different protocols. We conduct our measurement study of Web page performance with HTTP, HTTPS, SPDY, and HTTP/2. First, we compare these four protocols as transfer protocols with our own client and conduct a decision tree analysis to find out how those parameter settings affect these protocols` efficiency of transferring Web contents hosted in remote servers. Then, we use real browsers to load pages with these protocols to analyze how these protocols work in real browsers. We have findings as following :
 \begin{itemize}

    \item{HTTP/2 and SPDY perform worse with high packet loss. HTTP/2 and SPDY both have the feature of one TCP connection per origin. A single connection hurts under high packet loss because it aggressively reduces the congestion window compared to HTTPS which reduces the congestion window on only one of its parallel connections. When this single connection suffers from packet loss, all streams running over this unique TCP connection are negatively impacted.}
    
    \item{HTTP/2 and SPDY perform better with many objects under low loss. TCP implements congestion control by counting outstanding packets not bytes. Thus, sending a few small objects with HTTP will promptly use up the congestion window, though outstanding bytes are far below the window limit. In contrast, A single HTTP/2 connection can contain multiple concurrently-open streams, with either endpoint interleaving frames from multiple streams.}
    
    \item{The complicated dependencies and computation in real browsers may blur the bright spot of HTTP/2 and SPDY. Loading a Web page is not as simple as fetching all resources in parallel. Loading a Web page in modern browsers is a complex work, including parsing HTML, parsing JavaScript/CSS, and interpreting JavaScript/CSS etc.}
    
    \item{The mobile pages perform better than respective desktop pages. Mobile pages are dedicated for mobile devices with limited capabilities of computation and smaller screen, and may be optimized with less resources and smaller objects. Mobile pages under bad network conditions may even perform better than respective desktop pages under good network conditions.}
    
    \item{PLT on Galaxy Note 2 is not significantly different compared to the page load time on Nexus 6 for both desktop pages and mobile pages across all four protocols. We could infer that 1) mobile browsers are not able to take full advantage of high-performance CPU when exceeding a threshold; 2) network activities may dominate in the page load process against computation activities, and the improvement of computation just have marginal effect on whole page load process.}
        
 \end{itemize}
 
 In our experiments, we do find that the benefits from HTTP/2 when loading real Web pages using mobile browsers, especially under certain network conditions. However, HTTP/2 hurts under other network conditions, and the performance could be affected by the characteristics of Web pages. We give some implications as shown in Table~\ref{tab:findings and implications}. For website developers, they should adjust their expectations that it may not speed their Web pages. It may be more practical to optimize their websites to reduce Web object sizes, improve cache configures etc. Of course, we should not be too negative, since mobile devices are becoming more and more powerful and related technologies of HTTP/2 is advancing. If developers want to deploy HTTP/2, they should carefully consider where their customers will visit their Web pages. It is better to switch the protocols according the network conditions when customers visit the Web pages. For example, client developers can force the client to fetch resources using HTTP/2 when devices are connected to a good WiFi and using HTTP when   connected to 2G/3G.
 
 
 
 \begin{table*}[htbp]\normalsize
\newcommand{\tabincell}[2]{\begin{tabular}{@{}#1@{}}#2\end{tabular}}
  \centering
  \caption{Summary of Findings and Implications}\label{tab:findings and implications}
  \begin{tabularx}
{\linewidth}{|X|X|}
\hline
    \rowcolor{mygray}\bfseries Findings & \bfseries Implications \\
%\caption{Our major findings and implications}\label{tab:findings and implications}
%% Some packages, such as MDW tools, offer better commands for making tables
%% than the plain LaTeX2e tabular which is used here.
%\begin{tabular}{| p{8cm}| p{8cm}|}
%\hline
%\rowcolor{lightgray} {\centering} Findings & Implications\\
\hline
HTTP/2 and SPDY could both decrease and increase page load times under certain network conditions and page characteristics. For example, HTTP/2 and SPDY helps Web pages with many Web objects when packet loss rate is low.
&
Developers should consider the adoption of HTTP/2 and SPDY thoroughly. It will not speed their websites with few Web objects. It is necessary to characterize websites before turning to HTTP/2 and SPDY.\\
\hline
Different protocols perform variously under different parameter settings. For example, HTTP/2 and SPDY perform better with many objects under low loss, but perform worse when packet loss rate is high.
&
Developers should carefully consider where their customers will visit their Web pages. It is better to switch the protocols according the network conditions when customers visit the Web pages. For example, client developers can force the client to fetch resources using HTTP/2 when devices are connected to a good WiFi and using HTTP when   connected to 2G/3G.
\\
\hline
Mobile Web pages have much better performance than the respective desktop pages. Mobile pages under bad network conditions may even perform better than respective desktop pages under good network conditions.
&
More and more customers are used to visiting Web pages using mobile devices, which have smaller screen, limited capabilities of computation and power. Developers should optimize their Web pages to improve user experience on mobile devices. For customers, they should visit mobile page instead of desktop page when using mobile devices.
\\
\hline
More powerful devices may not speed the page load. Loading Web page in browser is a complicated process, consisting of network activities and computation activities. Powerful CPU may speed the computation activities, but may just gain marginal improvements as a whole.
&
Developers should not just focus on the optimization of computation activities in a Web page, but also the optimization of network activities. They could try HTTP/2 to speed page load. Customers should not rely on more powerful devices to speed Web pages visiting completely.
\\
\hline
\end{tabularx}
\end{table*} 