\section{Related Work}
Web service publishers aim to provide competitive services, and utilize all kinds of technologies to improve services' performance. In this paper, we focus on how HTTP/2 could reduce users perceived latency, which is a important factor of QoS. HTTP/2 derives from SPDY and aims to improve the performance of HTTP, so we focus on prior studies on improvement of HTTP and works about HTTP/2 and SPDY in this section.

\textbf{Quality of Service (QoS):} QoS is always attractive research topic of Web service. Ma et al.~\cite{Ma:SCC13} investigate how to measure Web service QoS precisely from both subjective and objective aspects. Nacer et al.~\cite{Ahmed-Nacer:SCC15} and Chen et al.~\cite{Chen:SCC15} present service selection and service recommendation, respectively. Elshater et al.~\cite{Elshater:SCC15} utilizes design pattern to improve performance in terms of average response time and throughput. Liu et al.~\cite{liuTSC2009}  alleviates the consumers from time-consuming discovery tasks. However, the performance of services could be easily affected by network conditions and complexity of services. It is important to speed the delivery of response content to services consumers more than services processing. 

\textbf{HTTP:} Radhakrishnan et al.~\cite{Radhakrishnan:CONEXT11} describes the TCP Fast Open protocol, a new mechanism that enables data exchange during TCP`s initial handshake. Flach et al.~\cite{Flach:SIGCOMM13} presents the design of novel loss recovery mechanisms for TCP that judiciously use redundant transmissions to minimize timeout-driven recovery.



\textbf{SPDY:} Erman et al.~\cite{Erman:CONEXT13} finds that SPDY performed poorly while interacting with radios due to a large body of unnecessary retransmissions. Wang et al.~\cite{Wang:NSDI14} swipe a complete parameter space including network parameters, TCP settings, and Web page characteristics to learn which factors affect the performance of SPDY. Thus, they also dive into the analysis of the effect of dependencies of network activities and computation activities in real browsers. However, we are more curious how HTTP/2 performs in mobile devices with limited capacities and poor network conditions. El-khatib et al.~\cite{El-khatibTW:IFIP14} also analyze the effect of network and infrastructural variables on SPDY's performance. Prior works on SPDY do encourage us to reveal the mysterious mask of HTTP/2. 

\textbf{HTTP/2:} Chowdhury et al.~\cite{Chowdhury:PEERJ15} focus on the energy efficiency of HTTP/2, and shows that HTTP/2 exhibits better performance as the RTT goes up. However, we are more curious if HTTP/2 could reduce page load time. Hugues et al.~\cite{Saxce:INFCOM15} evaluate the influence of HTTP/2` new features on the Web performance. They notice generally speedier page load times with HTTP/2, thanks to the multiplexing and compression features enabled by the protocol. The limitation is that they only test 15 websites. We test the Alexa Top 200 websites with both mobile version and desktop version in our experiment. Zarifis et al.~\cite{Zarifis:PAM16} develop a model to compare the PLT of HTTP/1.1 and HTTP/2 from the resource timing data for a single HTTP/1.1 page view. Varvello et al.~\cite{Varvello:PAM16} present a measurement platform to monitor both adoption and performance of HTTP/2. They find that developers do not reconstruct their Web pages when turning to HTTP/2 and 80\% of websites supporting HTTP/2 experience a decrease in page load time compared With HTTP/1.1 and the decrease grows in mobile networks. Although HTTP/2 announces to provide better performance, real adoptions do not gain much performance improvements and developers may still be in doubt. 