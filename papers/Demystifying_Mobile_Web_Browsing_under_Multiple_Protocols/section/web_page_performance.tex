\section{Web page performance under multiple protocols}

In this section, we analyze the impact of multiple protocols under different network conditions on Web page performance when loading pages in real browsers. We also conduct our experiments on Samsung Galaxy Note 2 to analyze the impact of different devices on Web page performance. We find that Web page performances differ marginally between these two devices.

\subsection{Characterizing Web pages}
\begin{figure}[htbp]
\centering
\subfigure[\# Object]{
\label{fig:object_num}
\includegraphics[width=0.32\textwidth]{images/object_num.png}}
\subfigure[Page Size]{
\label{fig:page_size}
\includegraphics[width=0.32\textwidth]{images/page_size.png}}
\subfigure[Overall Object Size]{
\label{fig:overall_object_size}
\includegraphics[width=0.32\textwidth]{images/overall_object_size.png}}
\caption{Characteristics of Web pages, including desktop pages and mobile pages}
\label{fig:characteristics_of_pages}
\end{figure}

First, we analyze the complexity of Web pages we collect. Especially, we focus on the differences between desktop pages and mobile pages. Original Web pages are developed for desktop browsers, and contains large number of objects to provide abundant contents. With the popularity of mobile devices, users are used to visiting Web pages using mobile browsers. However, mobile devices have smaller screen and limited resources including computing capacity, power supply, and data plan. It may not be suitable to present desktop pages in mobile browser. Many Web developers maintain a mobile version of pages to provide better user experience. 

We collect landing pages of Alexa Top 200 websites that have corresponding mobile version. We need address that we do not collect those pages with responsive Web design. A responsive Web page may present differently on mobile browser, but it does download a same collection of resources as desktop browser. Overall, we have 400 Web pages, consisting of 200 desktop pages and 200 corresponding mobile pages. 

Figure~\ref{fig:characteristics_of_pages} shows the cumulative distribution function (CDF) of  characteristics of Web pages including desktop version and mobile version. We could see that mobile pages does reduce the number of objects and overall page size in Figure~\ref{fig:object_num} and Figure~\ref{fig:page_size}. From Figure~\ref{fig:object_num}, We can see that the median object number of mobile pages is 90, but the median object number of desktop page is 178. Meanwhile, the median page size of mobile pages is 2.25MB, but the median page size of desktop page is 4.88MB. We can see that mobile pages are optimized with less objects and smaller page size and reduce nearly half of objects and page sizes compared to desktop pages in the median case. Figure~\ref{fig:overall_object_size} shows the distribution of object size across all Web pages we collect for desktop pages and mobile pages, respectively. We can find that the distributions of object size for desktop pages and mobile pages are similar.

\begin{figure}[htbp]
	\centering
    \includegraphics[width=0.6\textwidth]{images/pagetype_pageload_overall.png}
    \caption{Page load times for desktop pages and mobile pages. Results are across network conditions with four protocols.} \label{fig:pagetype_pageload_overall}
\end{figure}

Figure~\ref{fig:pagetype_pageload_overall} shows the page load times on mobile browser for desktop pages and mobile pages, respectively. The results are across all network conditions with four protocols as we present in last section. In the median case, page load times are 6.08s for mobile pages and 10.06s for desktop pages, and browsers could save about 40\% of time when loading mobile pages. 


\subsection{Web page performance with different protocols}

In this section, we analyze the impact of four protocols under different network conditions on Web page performance when loading pages in real browsers. We focus on how Web page performance may be affected for each protocol under different network conditions as shown in Figure~\ref{fig:plt_with_http}, Figure~\ref{fig:plt_with_https}, Figure~\ref{fig:plt_with_http2}, and Figure~\ref{fig:plt_with_spdy}, respectively.

\begin{figure*}[tbp]
\centering
\subfigure[PLT with HTTP under different RTTs]{
\label{fig:rtt_http}
\includegraphics[width=0.32\textwidth]{images/rtt_http.png}}
\subfigure[PLT with HTTP under different bandwidths]{
\label{fig:bw_http}
\includegraphics[width=0.32\textwidth]{images/bw_http.png}}
\subfigure[PLT with HTTP under different packet loss rate]{
\label{fig:loss_http}
\includegraphics[width=0.32\textwidth]{images/loss_http.png}}
\caption{PLT with HTTP under different network .}
\label{fig:plt_with_http}
\end{figure*}

\begin{figure*}[tbp]
\centering
\subfigure[PLT with HTTPS under different RTTs]{
\label{fig:rtt_https}
\includegraphics[width=0.32\textwidth]{images/rtt_https.png}}
\subfigure[PLT with HTTPS under different bandwidths]{
\label{fig:bw_https}
\includegraphics[width=0.32\textwidth]{images/bw_https.png}}
\subfigure[PLT with HTTPS under different packet loss rate]{
\label{fig:loss_https}
\includegraphics[width=0.32\textwidth]{images/loss_https.png}}
\caption{PLT with HTTPS under different network .}
\label{fig:plt_with_https}
\end{figure*}

Figure~\ref{fig:plt_with_http} shows the distribution of PLT with HTTP under different network conditions. Figure~\ref{fig:rtt_http}, Figure~\ref{fig:bw_http}, and Figure~\ref{fig:loss_http} show how PLTs vary under different RTTs, different bandwidths, and different packet loss rates for both mobile pages and desktop pages, respectively. In the median case, mobile pages can reduce 37.3\% $\sim$ 42.3\% PLT compared to desktop pages as RTTs arise, and mobile pages perform better with higher RTT. Figure~\ref{fig:bw_http} shows that bandwidths do not affect PLT for both mobile pages and desktop pages. Figure~\ref{fig:loss_http} shows that PLTs increase as packet loss rates arise. Performance of mobile pages with 2\% packet loss rate could even approach performance of desktop pages with 0\% packet loss rate. We recommend adaption of mobile pages for developers to provide better user experience even under bad network conditions.


Figure~\ref{fig:plt_with_https} shows the distribution of PLT with HTTPS under different network conditions. Figure~\ref{fig:rtt_https}, Figure~\ref{fig:bw_https}, and Figure~\ref{fig:loss_https} show how PLTs vary under different RTTs, different bandwidths, and different packet loss rates for both mobile pages and desktop pages, respectively. In the median case, Web pages load with HTTPS cost extra 6.5\% $\sim$ 11.5\% time since HTTPS costs extra two RTTs for establishing a new connection. Overall, PLTs with HTTPS have a similar trends as PLTs with HTTP. Thus, Bandwidths do not have significant affect on PLTs with HTTPS, too. Figure~\ref{fig:rtt_https} shows that in the median case, mobile pages can reduce 38.0\% $\sim$ 40.3\% PLT compared to desktop pages as RTTs arise.

\begin{figure}[htbp]
\centering
\subfigure[PLT with HTTP/2 under different RTTs]{
\label{fig:rtt_http2}
\includegraphics[width=0.32\textwidth]{images/rtt_http2.png}}
\subfigure[PLT with HTTP/2 under different bandwidths]{
\label{fig:bw_http2}
\includegraphics[width=0.32\textwidth]{images/bw_http2.png}}
\subfigure[PLT with HTTP/2 under different packet loss rate]{
\label{fig:loss_http2}
\includegraphics[width=0.32\textwidth]{images/loss_http2.png}}
\caption{PLT with HTTP/2 under different network .}
\label{fig:plt_with_http2}
\end{figure}

\begin{figure}[htbp]
\centering
\subfigure[PLT with SPDY under different RTTs]{
\label{fig:rtt_spdy}
\includegraphics[width=0.32\textwidth]{images/rtt_spdy.png}}
\subfigure[PLT with SPDY under different bandwidths]{
\label{fig:bw_spdy}
\includegraphics[width=0.32\textwidth]{images/bw_spdy.png}}
\subfigure[PLT with SPDY under different packet loss rate]{
\label{fig:loss_spdy}
\includegraphics[width=0.32\textwidth]{images/loss_spdy.png}}
\caption{PLT with SPDY under different network .}
\label{fig:plt_with_spdy}
\end{figure}

Figure~\ref{fig:plt_with_http2} shows the distribution of PLT with HTTP/2 under different network conditions. Figure~\ref{fig:rtt_http2}, Figure~\ref{fig:bw_http2}, and Figure~\ref{fig:loss_http2} show how PLTs vary under different RTTs, different bandwidths, and different packet loss rates for both mobile pages and desktop pages, respectively. Bandwidths do not have significant affect on PLTs with HTTP/2, too. However, we find that PLTs with HTTP/2 are more sensitive to other network parameters. As the RTT changes from 20ms to 100ms, PLTs double for both desktop pages and mobile pages. In the median case, mobile pages can reduce 45.9\% $\sim$ 50.6\% PLT compared to desktop pages as RTTs arise. In the median case, HTTP/2 reduces 12.5\% and 11.2\% PLTs against HTTP for mobile pages and desktop pages, respectively when packet loss rate is 0\%. However, HTTP/2 performs worse as packet loss rate arise. When packet loss rate arises to 2\%, PLT of HTTP/2 is almost twice as long as that of HTTP for both mobile pages and desktop pages. As we discuss in previous section, HTTP/2 and SPDY work worse when packet loss rate is high. HTTP/2 and SPDY have the feature of one TCP connection per origin. A single connection hurts under high packet loss because it aggressively reduces the congestion window compared to HTTPS which reduces the congestion window on only one of its parallel connections. When this single connection suffers from packet loss, all streams running over this unique TCP connection are negatively impacted.




Figure~\ref{fig:plt_with_spdy} shows the distribution of PLT with SPDY under different network conditions. Figure~\ref{fig:rtt_spdy}, Figure~\ref{fig:bw_spdy}, and Figure~\ref{fig:loss_spdy} show how PLTs vary under different RTTs, different bandwidths, and different packet loss rates for both mobile pages and desktop pages, respectively. In the median case, mobile pages can reduce 47.2\% $\sim$ 52.1\% PLT compared to desktop pages as RTTs arise. We find that SPDY has a similar performance trend as HTTP/2. We infer that HTTP/2 derived from SPDY and they share a similar implementation in the Nginx server on which we conduct our experiments.

\subsection{Web page performance with different devices}

We conduct our previous controlled experiments on Samsung Galaxy Note 2 to observer if different devices could affect Web page performance. The Nexus 6 phone is equipped with 3GB RAM and powered by a 2.7GHz Qualcomm Snapdragon 805 with quad-core CPU (APQ 8084-AB) running Android Lollipop. The Samsung Galaxy Note 2 phone is equipped with 2 GB RAM and powered by a 1.6GHz quad-core CPU running Android KitKat. We can see that Samsung Galaxy Note 2 has fewer RAM and weaker CPU.

\begin{figure}[htbp]
\centering
\subfigure[Page load time when loading pages in Nexus 6 phones and Galaxy Note 2 phones with HTTP]{
\label{fig:device_pageload_http}
\includegraphics[width=0.48\textwidth]{images/device_pageload_http.png}}
\subfigure[Page load time when loading pages in Nexus 6 phones and Galaxy Note 2 phones with HTTPS]{
\label{fig:device_pageload_https}
\includegraphics[width=0.48\textwidth]{images/device_pageload_https.png}}
\subfigure[Page load time when loading pages in Nexus 6 phones and Galaxy Note 2 phones with SPDY]{
\label{fig:device_pageload_spdy}
\includegraphics[width=0.48\textwidth]{images/device_pageload_spdy.png}}
\subfigure[Page load time when loading pages in Nexus 6 phones and Galaxy Note 2 phones with HTTP/2]{
\label{fig:device_pageload_http2}
\includegraphics[width=0.48\textwidth]{images/device_pageload_http2.png}}
\caption{Page load time when loading pages in Nexus 6 phones and Galaxy Note 2 phones. Results are tested under 20ms RTT.}
\label{fig:device_pageload}
\end{figure}


Figure~\ref{fig:device_pageload} depicts the distribution of page load time when loading pages using Nexus 6 and Galaxy Note 2. We only show results under 0\% packet loss rate and 1mbps bandwidth. Figure~\ref{fig:device_pageload} shows that page load time on Galaxy Note 2 is not significantly different compared to the page load time on Nexus 6 for both desktop pages and mobile pages across all four protocols.~\cite{Zhu:MICRO15} concludes that mobile CPU performance improvements yield marginal improvements in Web browsing performance with high clock frequency. We could infer that mobile browsers are not able to take full advantage of high-performance CPU when exceeding a threshold.
% TODO 怎么细粒度点分析?还是说是网络请求占主要,所以即使计算提升了,效果也不明显
