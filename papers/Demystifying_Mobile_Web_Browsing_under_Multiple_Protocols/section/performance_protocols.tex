\section{Performance as Transfer Protocols }
\begin{figure}[htbp]
    \includegraphics[width=1.0\textwidth]{images/decision.png}
    \caption{The decision tree that tells how HTTP/2, SPDY, and HTTP perform over parameter space.} \label{fig:decision_tree}
\end{figure}


In this section, we compare the performance when consider four protocols as transfer protocols to access Web contents hosted in remote servers. We need to eliminate the effect of dependencies and computation in real browsers, so we develop our own mobile client to load pages, which just downloads all resources of a page and records the timeline of resources downloading. We consider a wide range of parameter settings as showed in Table~\ref{tab:factors}. Our client will fetch all the resources from these synthetic pages with pre-specified object size and object number. We switch transfer protocol of Nginx proxy among HTTP, HTTPS, SPDY, and HTTP/2, so that our mobile client can work with these protocols, respectively. 

To understand how these four protocols perform under different conditions, we try to build a predictive model based on decision tree analysis. We have mentioned that HTTPS piggybacks HTTP entirely on top of TLS/SSL, which need one more expensive TLS/SSL handshake for each new connection with extra two Round-Trip Time (RTT). HTTPS do have worse performance that HTTP, and we just conduct the decision tree analysis among HTTP, SPDY, and HTTP/2.

In the analysis, each configuration is a combination of values for all factors listed in Table~\ref{tab:factors}. For each configuration, we add an additional variable \textit{s}, which means the performance of these protocols. We consider a protocol better than another if at least in the 70\% of the measurement cases performed at least 10\% faster than the other protocol’s average PLT. If two of the protocols are fulfilling this condition against the third one but not against each other we marked both of them. In case of this condition doesn’t stand between any of the three protocols we marked them as equal. We run the decision tree with ID3 algorithm~\cite{decisiontree} to predict the factor settings under which HTTP/2 performs better (or HTTP/SPDY performs better). The ID3 algorithm begins with the original set \textit{S} as the root node. On each iteration of the algorithm, it iterates through every unused factor (including RTT, bw, packet loss rate, object size, and object number in our data set) of the set \textit{S} and calculates information gain \textit{IG(A)} of that attribute. It then selects the attribute which has the largest information gain value. The set \textit{S} is then split by the selected attribute to produce subsets of the data. The algorithm continues to recurse on each subset, considering only attributes never selected before. 

Figure~\ref{fig:decision_tree} shows the decision tree that tells how HTTP/2, SPDY, and HTTP perform over parameter space. We could see that SPDY and HTTP/2 have similar performance across different settings. It is not surprising since HTTP/2 derived from SPDY and they share similar implementation. We have some conclusions as following:

\begin{itemize}
	\item{HTTP/2 and SPDY do not always gain performance improvement under different parameter settings. }
    \item{HTTP/2 and SPDY perform worse with high packet loss. HTTP/2 and SPDY both have the feature of one TCP connection per origin. A single connection hurts under high packet loss because it aggressively reduces the congestion window compared to HTTPS which reduces the congestion window on only one of its parallel connections. When this single connection suffers from packet loss, all streams running over this unique TCP connection are negatively impacted.}
    \item{HTTP/2 and SPDY perform better with many objects under low loss. TCP implements congestion control by counting outstanding packets not bytes. Thus, sending a few small objects with HTTP will promptly use up the congestion window, though outstanding bytes are far below the window limit. In contrast, A single HTTP/2 connection can contain multiple concurrently-open streams, with either endpoint interleaving frames from multiple streams.}
    \item{High RTT favors HTTP/2 and SPDY against HTTP due to multiplexing. HTTP/2 and SPDY benefits from having a single connection and stream multiplexing. One connection per origin significantly reduces the associated overhead: fewer sockets to manage along the connection path, smaller memory footprint, better connection throughput, less time in slow-start, faster congestion and loss recovery. As the RTT goes up, new established TCP connections cost more time.}
\end{itemize}


The decision tree also depicts the relative importance of contributing factors. Intuitively, factors close to the root of the decision tree affect HTTP/2‘ performance more than those near the leaves. This is because the decision tree places the important factors near the root to reduce the number of branches. We find that object number, object size, and packet loss rate are the most important factors in predicting HTTP/2‘ performance. However, RTT and bandwidth play a less important role as shown in Figure~\ref{fig:decision_tree}.





