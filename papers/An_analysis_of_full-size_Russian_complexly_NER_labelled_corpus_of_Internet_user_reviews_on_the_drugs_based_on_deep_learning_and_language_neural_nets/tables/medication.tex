\begin{tabular}{|p{0.15\textwidth}|p{0.8\textwidth}|}
\hline
\textbf{Drugname}       & Marks a mention of a drug. For example, in the sentence <<Препарат Aventis ``Трентал'' для улучшения мозгового кровообращения>> (The Aventis ``Trental'' drug to improve cerebral circulation), the word ``Trental'' (without quotation marks) is marked as a Drugname.
\\ \hline
\textbf{DrugBrand}      & A drug name is also marked as DrugBrand if it is a registered trademark. For example, in the sentence <<Противовирусный и иммунотропный препарат Экофарм ``Протефлазид''>> (The Ecopharm ``Proteflazid'' antiviral and immunotropic drug), the word ``Протефлазид'' (Proteflazid) is marked as DrugBrand.
\\ \hline
\textbf{Drugform}       & Dosage form of the drug (ointment, tablets, drops, etc.). For example, in the sentence <<Эти таблетки не плохие, если начать принимать с первых признаков застуды>> (These pills are not bad if you start taking them since the first signs of a cold), the word ``таблетки'' (pills) is marked as DrugForm.
\\ \hline
\textbf{Drugclass}      & Type of drug (sedative, antiviral agent, sleeping pill, etc.) For example, in the sentence <<Противовирусный и иммунотропный препарат Экофарм "Протефлазид”>> (The Ecopharm ``Proteflazid'' antiviral and immunotropic drug), two mentions marked as Drugclass: ``Противовирусный'' (Antiviral) and ``иммунотропный'' (immunotropic).
\\ \hline
\textbf{MedMaker}       & The drug manufacturer. This attribute has two values: Domestic and Foreign. For example, in the sentence <<Седативный препарат Материа медика ``Тенотен''>> (The Materia Medica ``Tenoten'' sedative) the word combination ``Материа медика'' (Materia Medica) is marked as MedMaker/Domestic.
\\ \hline
\textbf{MedFrom}        & This is an attribute of a Medication entity that takes one of the two values -- Domestic and Foreign, characterizing the manufacturer of the drug. For example, in the sentence <<Седативные таблетки Фармстандарт “Афобазол”>> (The Pharmstandard ``Afobazol'' sedative pills) the drug name ``Афобазол'' (Afobazol) has its MedFrom attribute equal to Domestic.
\\ \hline
\textbf{Frequency}      & The drug usage frequency. For example, in the sentence <<Неудобство было в том, что его приходилось наносить 2 раза в день>> (Its inconvenience was that it had to be applied two times a day), the phrase ``2 раза в день'' (two times a day) is marked as Frequency.
\\ \hline
\textbf{Dosage}         & The drug dosage (including units of measurement, if specified). For example, in the sentence <<Ректальные суппозитории ``Виферон'' 15000 МЕ -- эффекта ноль>> (Rectal suppositories “Viferon” 150000 IU have zero effect), the mention ``15000 МЕ'' (150000 IU) is marked as Dosage.
\\ \hline
\textbf{Duration}       & This entity specifies the duration of use. For example, in the sentence <<Время использования: 6 лет>> (Time of use: 6 years), ``6 лет'' (6 years) is marked as Duration.
\\ \hline
\textbf{Route}          & Application method (how to use the drug). For example, in the sentence <<удобно то, что можно готовить раствор небольшими порциями>> (it is convenient that one can prepare the solution in small portions), the mention ``можно готовить раствор небольшими порциями'' (can prepare a solution in small portions) is marked as a Route.
\\ \hline
\textbf{SourceInfodrug} & The source of information about the drug. For example, in the sentence <<Этот спрей мне посоветовали в аптеке в его состав входят такие составляющие вещества как мята>> (This spray was recommended to me at a pharmacy, it includes such ingredient as mint), the word combination ``посоветовали в аптеке'' (recommended to me at a pharmacy) is marked as SourceInfoDrug.
\\ \hline
\end{tabular}