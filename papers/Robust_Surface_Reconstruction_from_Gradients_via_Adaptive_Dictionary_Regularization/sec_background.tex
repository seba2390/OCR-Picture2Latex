%!TEX root = G2S_ICIP.tex

Let $n(x,y) \in \mathbb{R}^3$ denote the normal vector of a differentiable surface $z(x,y)$ at position $(x,y)$, and let $n_1(x,y)$, $n_2(x,y)$, and $n_3(x,y)$ denote the $x$, $y$, and $z$ components of this vector, respectively. Under this ideal model, one can relate the $x$ and $y$ derivatives of the surface $z$ to its normal vectors via the relation
\begin{equation} \label{eq:partials}
\frac{\partial z(x,y)}{\partial x} = -p(x,y), \ \ \ \frac{\partial z(x,y)}{\partial y} = q(x,y),
\end{equation}
where we have defined $p(x,y) := n_1(x,y)/n_3(x,y)$ and $q(x,y) := n_2(x,y)/n_3(x,y)$. In practice, the estimated (e.g., by photometric stereo) normal vectors of a surface and its gradient fields will not exactly satisfy \eqref{eq:partials}, so one must instead find a function $z(x,y)$ with derivatives \emph{close} to $p(x,y)$ and $q(x,y)$ in an appropriate sense, often by minimizing a variational problem of the form
\begin{equation} \label{eq:cont_cost}
\int \int_{\Omega} \left (\frac{\partial z(x,y)}{\partial x}  - p(x,y) \right )^2 + \left ( \frac{\partial z(x,y)}{\partial y} - q(x,y) \right )^2 \ dx \ dy.
\end{equation}
When our data is instead sampled on a discrete grid, we will not have access to a continuous normal map $n(x,y)$ but will instead have a matrix $N \in \mathbb{R}^{m \times n \times 3}$ containing the normal vectors of the object on the grid. Following \eqref{eq:partials}, we can compute matrices $P \in \mathbb{R}^{m \times n}$ and $Q \in \mathbb{R}^{m \times n}$ containing the measured gradients, and our goal then becomes to estimate the matrix $Z \in \mathbb{R}^{m \times n}$ containing the values of the surface $z(x,y)$ sampled on the grid. The discrete analogue of \eqref{eq:cont_cost} is commonly expressed \cite{simchony1990,frankot1988,harker2008} as a standard least squares problem of the form
\begin{equation} \label{eq:surf_ls}
z^* = \argmin_{z} \ \|A z - v \|_2^2,
\end{equation}
where $z = \vec(Z) \in \mathbb{R}^{mn}$ is the vectorized surface, $A$ is a numerical differentiation operator, and the vector $v$ is an appropriate function of the measured gradients, $P$ and $Q$. Solving this problem yields a representation of our surface that is optimal in the least squares sense.

Note that the specific forms of $A$ and $v$ can vary. One possible formulation is
\begin{equation}
A = \begin{bmatrix}
D_n \otimes I_m \\
I_n \otimes D_m
\end{bmatrix}, \ \ \ \ v = \begin{bmatrix}
\textbf{vec}(P) \\
\textbf{vec}(Q)
\end{bmatrix},
\end{equation}
where $D_n$ is the discrete first differences matrix, and $\otimes$ denotes the Kronecker product. However, multiple models are possible, each yielding reconstructed surfaces with different properties. Importantly, the dictionary learning based approach that we introduce in Section~\ref{sec:dl} can be coupled with any least squares model of the form \eqref{eq:surf_ls}, so our proposed approach is quite flexible.






