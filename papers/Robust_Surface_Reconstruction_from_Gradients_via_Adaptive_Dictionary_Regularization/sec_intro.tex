%!TEX root = G2S_ICIP.tex

Imaging techniques such as photometric stereo \cite{woodham1980} allow one to efficiently estimate the normal vector map of an object. The primary goal of such methods is to ultimately derive a three-dimensional representation of the object, which requires some flavor of numerical integration of the gradient fields defined by the normal vector map. Robust photometric stereo---the problem of accurately determining the normal map of a non-ideal surface or from noisy data---has attracted considerable attention in recent years \cite{wu2011,ikehata2012,ikehata2014}. In this work, we seek to develop a robust approach to the problem of reconstructing surfaces from gradient fields that can accurately estimate a 3D representation of an object in the presence of noise.

The problem of reconstructing a surface from estimates of its photometric stereo gradient fields has been investigated since the late 1980s. The seminal works of Simchony \textit{et al.} \cite{simchony1990} and Frankot and Chellappa \cite{frankot1988} seek to solve the problem through a least squares approach, utilizing efficient discrete Fourier transform (DFT) or discrete cosine transform (DCT) based solvers.
%---essentially attempting to project the surface onto Fourier basis functions or the DCT basis.
Harker and O'Leary \cite{harker2008} propose a modified ``global'' least squares problem and extend this method to incorporate regularization \cite{harker2015}, solving a Sylvester equation to obtain the solution. Recently, Qu\'{e}au and Durou \cite{queau2015} introduced a weighted-least squares formulation as well as formulations minimizing total-variation and incorporating the $\ell_1$ norm to promote sparsity. Further attempts at applying a regularization term while integrating the gradients have also been proposed at the expense of computation time \cite{agrawal2006,ng2010}. Additional approaches include line-integral based methods \cite{wu1988,robles2005} and reconstructions based on the calculus of variations \cite{horn1986,balzer2012,durou2007}. A range of other methods have also been proposed with mixed results \cite{horovitz2004,lee1993,karaccali2003,karaccali2004,kovesi2005,balzer2011}.

Our work builds on these previous works, specifically those that utilize a least squares-type formulation to relate the underlying surface and its gradient fields. In particular, we propose a novel adaptive dictionary learning framework that enables the robust estimation of surfaces from noisy gradients. Dictionary learning \cite{elad2006image,aharon2006rm,kreutz2003dictionary} has, in recent years, been successfully applied to many imaging applications, e.g.,  \cite{ravishankar2011mr,ravishankar2016lassi,ravishankar2016low}. In dictionary learning models, one typically seeks to learn sparse representations of local regions (patches) of the data. These models often induce a type of smoothness constraint on the underlying data that, in the case of surface reconstruction, we show leads to robust reconstructions with desirable noise rejection properties. Our framework is general and can be easily combined with any existing method that utilizes a least squares-type objective to estimate the underlying surface.
