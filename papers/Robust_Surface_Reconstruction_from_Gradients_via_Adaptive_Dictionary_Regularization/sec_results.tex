%!TEX root = G2S_ICIP.tex

\begin{figure*}
\centering
\begin{subfigure}[b]{0.23\textwidth}
  \includegraphics[width=\textwidth]{figures/frog_13_17_dls_view1.png}
  %\caption{DLS}
\end{subfigure}
~ 
\begin{subfigure}[b]{0.23\textwidth}
  \includegraphics[width=\textwidth]{figures/frog_13_17_glss_view1.png}
  %\caption{SR}
\end{subfigure}
~ %\quad
\begin{subfigure}[b]{0.23\textwidth}
  \includegraphics[width=\textwidth]{figures/frog_13_17_tv_view1.png}
  %\caption{TV}
\end{subfigure}
~
\begin{subfigure}[b]{0.23\textwidth}
  \includegraphics[width=\textwidth]{figures/frog_13_17_sim2_view1.png}
  %\caption{DCTLS}
\end{subfigure}
\end{figure*}

\setcounter{figure}{2}  

\begin{figure*}
\centering
\begin{subfigure}[b]{0.23\textwidth}
  \includegraphics[width=\textwidth]{figures/frog_13_17_dls_view1_zoom.png}
  \caption{\textbf{DLS}}
\end{subfigure}
~ 
\begin{subfigure}[b]{0.23\textwidth}
  \includegraphics[width=\textwidth]{figures/frog_13_17_glss_view1_zoom.png}
  \caption{SR}
\end{subfigure}
~ %\quad
\begin{subfigure}[b]{0.23\textwidth}
  \includegraphics[width=\textwidth]{figures/frog_13_17_tv_view1_zoom.png}
  \caption{TV}
\end{subfigure}
~
\begin{subfigure}[b]{0.23\textwidth}
  \includegraphics[width=\textwidth]{figures/frog_13_17_sim2_view1_zoom.png}
  \caption{DCTLS}
\end{subfigure}
\caption{Surface reconstructions of the photometric stereo Frog dataset \cite{xiong2015shading} with SNR = 17 dB.}
\label{fig:frog_zoomed}
\end{figure*}

In this section, we numerically evaluate our proposed DLS method on several datasets. In each case we compare our method to the spectral regularization method (SR) \cite{harker2015}, the isotropic total variation (TV) method \cite{queau2015}, and DCT based least squares (DCTLS) \cite{simchony1990}. For methods that include tunable parameters, we sweep over a wide range of values, reporting the best results obtained. Our proposed DLS method can incorporate any least squares based solver by simply defining $A$ and $v$ in \eqref{eq:dl_surf} accordingly. For all results given here, we use the least squares cost found in \cite{simchony1990}. To evaluate the robustness of each algorithm, we add Gaussian noise to the data.

For our proposed DLS method, we use dictionary atoms of size $8 \times 8$ pixels and a square $64 \times 64$ dictionary $D$, initialized with a DCT matrix. We extracted patches from $z$ using a spatial stride of two pixels in each direction, allowing adjacent patches to overlap. Finally, we initialized $z$ as the vectorized surface produced by solving the stand-alone least squares problem in \cite{simchony1990}, and initialized $B = 0$.

\subsection{Synthetic Surface Reconstructions}
To quantitatively evaluate our method, we first considered two synthetic datasets, which we call ``Tent'' and ``Vase'', for which we have analytic expressions for $z = f(x,y)$. Given $f(x,y)$, we can differentiate to obtain the gradients, $\partial f(x,y) / \partial x$ and $\partial f(x,y) / \partial y$, and sample on a discrete grid. After reconstructing the surface from these gradients subject to additive noise, we evaluate the integrity of the reconstructions against the ground truth, $f(x,y)$, via the SSIM metric \cite{wang2004}. In these experiments, we add noise directly to the gradient fields to achieve a prescribed signal-to-noise ratio (SNR).

Figures~\ref{fig:tent} and \ref{fig:vase} show the reconstructed surfaces produced by each algorithm. As these images illustrate, the proposed DLS method produces significantly smoother surfaces from noisy data compared to the existing methods. Intuitively, the locally sparse model imposed by the dictionary regularization denoises the surfaces, while the adaptive nature of the dictionary allows DLS to represent and reconstruct both sharp edges and smooth regions on a data-dependent basis.

The surfaces obtained by SR, TV, and DCTLS are much more sensitive to the noisy gradients. Indeed, while they retain the general shape of the surface, they exhibit significantly more corruption. In particular, the spectral regularization method seems to introduce a systematic ``rippling" into the reconstructions.
\vspace{2mm}

\begin{table}[ht!]
\centering
\begin{tabular}{|c|c|c|c|c|}
\cline{1-5}
SNR (dB) & \bf{DLS} & SR & TV & DCTLS \\
\cline{1-5}
1 & \textbf{0.969} & 0.944 & 0.918 & 0.924 \\ 
\cline{1-5}
5 & \textbf{0.971} & 0.950 & 0.938 & 0.944 \\ 
\cline{1-5}
10 & \textbf{0.976} & 0.956 & 0.957 & 0.962 \\ 
\cline{1-5}
20 & \textbf{0.988} & 0.969 & 0.979 & 0.983 \\ 
\cline{1-5}
30 & \textbf{0.995} & 0.978 & 0.989 & 0.992 \\ 
\cline{1-5}
40 & \textbf{0.997} & 0.985 & 0.994 & 0.996 \\ 
\cline{1-5}
50 & \textbf{0.998} & 0.988 & 0.996 & \textbf{0.998} \\ 
\cline{1-5}
60 & \textbf{0.999} & 0.989 & 0.997 & 0.998 \\ 
\cline{1-5}
\end{tabular}
\caption{Quality of Tent reconstructions in SSIM as a function of SNR.}
\label{tab:pyramid}
\end{table}

\begin{table}[ht!]
\vspace{-4mm}
\centering
\begin{tabular}{|c|c|c|c|c|}
\cline{1-5}
SNR (dB) & \bf{DLS} & SR & TV & DCTLS \\
\cline{1-5}
1 & \textbf{0.958} & 0.930 & 0.889 & 0.894 \\ 
\cline{1-5}
5 & \textbf{0.966} & 0.934 & 0.911 & 0.915 \\ 
\cline{1-5}
10 & \textbf{0.971} & 0.942 & 0.933 & 0.936 \\ 
\cline{1-5}
20 & \textbf{0.977} & 0.961 & 0.965 & 0.966 \\
\cline{1-5} 
30 & \textbf{0.982} & 0.975 & 0.981 & 0.981 \\ 
\cline{1-5}
40 & \textbf{0.990} & 0.982 & 0.990 & 0.989 \\ 
\cline{1-5}
50 & \textbf{0.993} & 0.984 & \textbf{0.993} & 0.992 \\ 
\cline{1-5}
60 & \textbf{0.995} & 0.985 & \textbf{0.995} & 0.993 \\ 
\cline{1-5}
\end{tabular}
\caption{Quality of Vase reconstructions in SSIM as a function of SNR.}
\label{tab:vase}
\end{table}

\vspace{-2mm}
Tables~\ref{tab:pyramid} and \ref{tab:vase} numerically corroborate the qualitative results from Figures~\ref{fig:tent} and \ref{fig:vase}. In the low SNR regime, DLS significantly outperforms the other approaches. As SNR increases, the gap decreases. When the data is essentially noiseless, DLS, TV, and DCTLS are all able to reconstruct the surfaces with comparably negligible error.


\subsection{Photometric Stereo}
We now return to the problem of reconstructing a 3D representation of an object from normal vectors obtained through photometric stereo. We consider a dataset containing 10 images, each taken under a unique, known lighting direction, and corrupt the images with Gaussian noise at a prescribed SNR. We compute the normal vectors from the noisy images using the standard least squares approach \cite{wu2011}. Given the normal vectors, we then compute gradient fields as discussed in Section 2, and from those we generate a 3D reconstruction of the object using each method. Figure~\ref{fig:frog_zoomed} illusrates the results of this procedure on the Frog dataset,\footnote{This dataset can be found at \url{http://vision.seas.harvard.edu/qsfs/Data.html}} which contains real images of a frog statue \cite{xiong2015shading}.

The reconstructions in Figure~\ref{fig:frog_zoomed} showcase the ability of the proposed DLS approach to produce a smoother surface from the noisy gradients compared to the existing methods. The denoising capability of DLS may prove valuable when running photometric stereo on real-world data, where noise and other non-idealities are inevitable.


