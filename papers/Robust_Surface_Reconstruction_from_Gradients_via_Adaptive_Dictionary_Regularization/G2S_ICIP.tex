\documentclass{article}

\usepackage{spconf}
\usepackage{amsmath}
\usepackage{graphicx}
\usepackage{enumerate}
\usepackage{color}
\usepackage{soul}
\usepackage{amsfonts}
\usepackage{amssymb}
\usepackage{caption}
\usepackage{subcaption}
\usepackage{float}
\usepackage{comment}
\usepackage{url}

%\captionsetup[table]{skip=3pt}
%\setlength{\belowcaptionskip}{-3pt}
%\usepackage[skip=0pt]{caption}

\DeclareMathOperator*{\argmin}{arg\,min}
\renewcommand{\vec}{\mathbf{vec}}

\title{ROBUST SURFACE RECONSTRUCTION FROM GRADIENTS VIA ADAPTIVE DICTIONARY REGULARIZATION}
\name{Andrew J. Wagenmaker, Brian E. Moore, and Raj Rao Nadakuditi
\thanks{This work was supported in part by the following grants: ONR grant N00014-15-1-2141, DARPA Young Faculty Award D14AP00086, and ARO MURI grants W911NF-11-1-0391 and 2015-05174-05.}}
\address{Department of EECS, University of Michigan, Ann Arbor, MI, USA}

\begin{document}
%\ninept

\maketitle

\begin{abstract}
This paper introduces a novel approach to robust surface reconstruction from photometric stereo normal vector maps that is particularly well-suited for reconstructing surfaces from noisy gradients. Specifically, we propose an adaptive dictionary learning based approach that attempts to simultaneously integrate the gradient fields while sparsely representing the spatial patches of the reconstructed surface in an adaptive dictionary domain. We show that our formulation learns the underlying structure of the surface, effectively acting as an adaptive regularizer that enforces a smoothness constraint on the reconstructed surface. Our method is general and may be coupled with many existing approaches in the literature to improve the integrity of the reconstructed surfaces. We demonstrate the performance of our method on synthetic data as well as real photometric stereo data and evaluate its robustness to noise.
\end{abstract}

\begin{keywords}
Dictionary learning, photometric stereo, sparse representations, inverse problems.
\end{keywords}

\section{Introduction} \label{sec:intro}
\section{Introduction}
\label{sec:intro}

Computer graphics has long been concerned with generating photorealistic images at high resolution that allow for direct control over semantic attributes. Until recently, the primary paradigm was to create carefully designed 3D models which are then rendered using realistic camera and illumination models. A parallel line of research approaches the problem from a data-centric perspective. 
In particular, probabilistic generative models ~\cite{Goodfellow2014NEURIPS,Oord2017NEURIPS,Song2021ICLR} have shifted the paradigm from designing assets to designing training procedures and datasets. Style-based GANs (StyleGANs) are a specific instance of these models, and they exhibit many desirable properties. They achieve high image fidelity~\cite{Karras2019CVPR, Karras2020CVPR}, fine-grained semantic control~\cite{Haerkoenen2020NEURIPS, WU2021CVPRa,Ling2021ARXIV}, and recently alias-free generation enabling realistic animation~\cite{Karras2021NEURIPS}. Moreover, they reach impressive photorealism on carefully curated datasets, especially of human faces. However, when trained on large and unstructured datasets like ImageNet~\cite{Deng2009CVPR}, StyleGANs do not achieve satisfactory results yet. One other problem plaguing data-centric methods, in general, is that they become prohibitively more expensive when scaling to higher resolutions as bigger models are required.

Initially, StyleGAN~\cite{Karras2019CVPR} was proposed to explicitly disentangle factors of variations, allowing for better control and interpolation quality. 
However, its architecture is more restrictive than a standard generator network~\cite{Radford2016ICLR, Karras2018ICLR} which seems to come at a price when training on complex and diverse datasets such as ImageNet. Previous attempts at scaling StyleGAN and StyleGAN2 to ImageNet led to sub-par results~\cite{Gwern2020MISC, Grigoryev2022ICLR},  giving reason to believe it might be fundamentally limited for highly diverse datasets~\cite{Gwern2020MISC}.

BigGAN~\cite{Brock2019ICLR} is the state-of-the-art GAN model for image synthesis on ImageNet. The main factors for BigGANs success are larger batch and model sizes.
However, BigGAN has not reached a similar standing as StyleGAN as its performance varies significantly between training runs~\cite{Karras2020NeurIPS} and as it does not employ an intermediate latent space which is essential for GAN-based image editing~\cite{Abdal2021TOG, Patashnik2021ICCV, Collins2020CVPR, WU2021CVPRa}. Recently, BigGAN has been superseded in performance by diffusion models~\cite{Dhariwal2021NEURIPS}. Diffusion models achieve more diverse image synthesis than GANs but are significantly slower during inference and prior work on GAN-based editing is not directly applicable. Following these arguments, successfully training StyleGAN on ImageNet has several advantages over existing methods.

The previously failed attempts at scaling StyleGAN raise the question of whether architectural constraints fundamentally limit style-based generators or if the missing piece is the right training strategy.
Recent work by \cite{Sauer2021NEURIPS} introduced \textit{Projected GANs} which project generated and real samples into a fixed, pretrained feature space. Rephrasing the GAN setup this way leads to significant improvements in training stability, training time, and data efficiency.
Leveraging the benefits of Projected GAN training might enable scaling StyleGAN to ImageNet.
However, as observed by~\cite{Sauer2021NEURIPS}, the advantages of Projected GANs only partially extend to StyleGAN on the unimodal datasets they investigated.
We study this issue and propose architectural changes to address it.
We then design a progressive growing strategy tailored to the latest StyleGAN3. 
These changes in conjunction with Projected GAN already allow surpassing prior attempts of training StyleGAN on ImageNet. 
To further improve results, we analyze the pretrained feature network used for Projected GANs and find that the two standard neural architectures for computer vision, CNNs and ViTs~\cite{Dosovitskiy2021ICLR}, significantly improve performance when used jointly. Lastly, we leverage \textit{classifier guidance}, a technique originally introduced for diffusion models to inject additional class-information~\cite{Dhariwal2021NEURIPS}.

Our contributions culminate in a new state-of-the-art on large-scale image synthesis, pushing the performance beyond existing GAN and diffusion models. 
We showcase inversion and editing for ImageNet classes and find that Pivotal Tuning Inversion (PTI)~\cite{Roich2021ARXIV}, a powerful new inversion paradigm, combines well with our model and even embeds out-of-domain images smoothly into our learned latent space. Our efficient training strategy allows us to triple the parameters of the standard StyleGAN3 while reaching prior state-of-the-art performance of diffusion models~\cite{Dhariwal2021NEURIPS} in a fraction of their training time. It further enables us to be the first to demonstrate image synthesis on ImageNet-scale at a resolution of $1024^2$ pixels. We will open-source our code and models upon publication. 


\section{Surface Reconstruction from Gradient Fields} \label{sec:background}
%!TEX root = G2S_ICIP.tex

Let $n(x,y) \in \mathbb{R}^3$ denote the normal vector of a differentiable surface $z(x,y)$ at position $(x,y)$, and let $n_1(x,y)$, $n_2(x,y)$, and $n_3(x,y)$ denote the $x$, $y$, and $z$ components of this vector, respectively. Under this ideal model, one can relate the $x$ and $y$ derivatives of the surface $z$ to its normal vectors via the relation
\begin{equation} \label{eq:partials}
\frac{\partial z(x,y)}{\partial x} = -p(x,y), \ \ \ \frac{\partial z(x,y)}{\partial y} = q(x,y),
\end{equation}
where we have defined $p(x,y) := n_1(x,y)/n_3(x,y)$ and $q(x,y) := n_2(x,y)/n_3(x,y)$. In practice, the estimated (e.g., by photometric stereo) normal vectors of a surface and its gradient fields will not exactly satisfy \eqref{eq:partials}, so one must instead find a function $z(x,y)$ with derivatives \emph{close} to $p(x,y)$ and $q(x,y)$ in an appropriate sense, often by minimizing a variational problem of the form
\begin{equation} \label{eq:cont_cost}
\int \int_{\Omega} \left (\frac{\partial z(x,y)}{\partial x}  - p(x,y) \right )^2 + \left ( \frac{\partial z(x,y)}{\partial y} - q(x,y) \right )^2 \ dx \ dy.
\end{equation}
When our data is instead sampled on a discrete grid, we will not have access to a continuous normal map $n(x,y)$ but will instead have a matrix $N \in \mathbb{R}^{m \times n \times 3}$ containing the normal vectors of the object on the grid. Following \eqref{eq:partials}, we can compute matrices $P \in \mathbb{R}^{m \times n}$ and $Q \in \mathbb{R}^{m \times n}$ containing the measured gradients, and our goal then becomes to estimate the matrix $Z \in \mathbb{R}^{m \times n}$ containing the values of the surface $z(x,y)$ sampled on the grid. The discrete analogue of \eqref{eq:cont_cost} is commonly expressed \cite{simchony1990,frankot1988,harker2008} as a standard least squares problem of the form
\begin{equation} \label{eq:surf_ls}
z^* = \argmin_{z} \ \|A z - v \|_2^2,
\end{equation}
where $z = \vec(Z) \in \mathbb{R}^{mn}$ is the vectorized surface, $A$ is a numerical differentiation operator, and the vector $v$ is an appropriate function of the measured gradients, $P$ and $Q$. Solving this problem yields a representation of our surface that is optimal in the least squares sense.

Note that the specific forms of $A$ and $v$ can vary. One possible formulation is
\begin{equation}
A = \begin{bmatrix}
D_n \otimes I_m \\
I_n \otimes D_m
\end{bmatrix}, \ \ \ \ v = \begin{bmatrix}
\textbf{vec}(P) \\
\textbf{vec}(Q)
\end{bmatrix},
\end{equation}
where $D_n$ is the discrete first differences matrix, and $\otimes$ denotes the Kronecker product. However, multiple models are possible, each yielding reconstructed surfaces with different properties. Importantly, the dictionary learning based approach that we introduce in Section~\ref{sec:dl} can be coupled with any least squares model of the form \eqref{eq:surf_ls}, so our proposed approach is quite flexible.








\section{Adaptive Dictionary Learning Regularization} \label{sec:dl}
%!TEX root = PSviaDL.tex

In this section, we propose two adaptive dictionary learning methods for estimating the normal vectors of a surface from (possibly) noisy images, $Y$. Intuitively, these models seek to learn a locally sparse representation of the data with respect to a collection of learned basis ``atoms'' that capture the underlying local structure of the data.

\subsection{Preprocessing Images through Dictionary Learning (DLPI)}
Our first approach applies dictionary learning to the data in a preprocessing step before estimating the normal vectors. This formulation represents the input image data $Y$ as locally sparse in an adaptive dictionary domain---thereby removing non-idealities that are not well-represented by the dictionary. Specifically, we propose to solve the problem
\begin{align} \label{eq:dlpi}
\min_{v,B,D} & ~ \frac{1}{2} \left \| y -  v \right \|_2^2 + \lambda \left(\textstyle\sum_{j=1}^c \left \| P_j v  - D b_j \right \|_2^2 +  \mu^2 \left \| B \right \|_0 \right) \nonumber \\
\text{s.t.} & ~ \left \| d_i \right \|_2 = 1, \ \ \left \| b_j \right \|_{\infty} \leq a, ~~ \forall i,j.
\end{align}
%\ \ \text{rank}(R(d_i)) \leq r,
Here, $y = \textbf{vec}(Y)$ and $P_j$ is a matrix that extracts a (vectorized) 3D patch of dimensions $c_x \times c_y \times c_z$ from $v$, where $c_x$ and $c_y$ are the dimensions of the patches extracted from each image and $c_z$ is the number of distinct images whose patches are combined to form the 3D patch. $D \in \mathbb{R}^{c_x c_y c_z \times K}$ is a dictionary matrix whose columns $d_i$ are the (learned) dictionary atoms, and $B \in \mathbb{R}^{K \times c}$ is a sparse coding matrix whose columns $b_j$ define (usually sparse) linear combinations of dictionary atoms used to represent each patch. Also, $\left \|\cdot\right \|_0$ is the familiar $\ell_0$ ``norm", and $\lambda,\mu > 0$ are parameters. 

We impose the constraint $\|b_j\|_{\infty} \leq a$, where $a$ is typically very large, since \eqref{eq:dlpi} is non-coercive with respect to $B$, but the constraint is typically inactive in practice \cite{sairajfes2}. Without loss of generality, we impose a unit-norm constraint on the dictionary atoms $d_i$ to avoid scaling ambiguity between $D$ and $B$ \cite{kar}.
We allow the possibility that patches from $c_z > 1$ input images can be combined into a 3D patch to allow the dictionary atoms to learn correlated features between images, but one can set $c_z = 1$ to work with 2D per-image patches.

Once we have solved \eqref{eq:dlpi}, we reshape $v$ (back) into an $m_1 m_2 \times d$ matrix whose columns are vectorized (now preprocessed) images, and then we estimate the associated normal vectors using the standard least squares model \eqref{eq:ls}. Henceforth, we refer to this approach as the Dictionary Learning with Preprocessed Imgaes (DLPI) method.

\subsection{Normal Vector Computation through Dictionary Learning (DLNV)}
We next propose modifying \eqref{eq:ls} by applying an adaptive dictionary regularization term to the normal vectors, $N$, under the Lambertian model \eqref{eq:ls}. Specifically, we propose to solve the problem
\begin{align} \label{eq:dlnv}
\min_{n,B,D} & ~ \frac{1}{2} \left \| y - A n \right \|_2^2 + \lambda \left ( \begin{matrix} \sum_{j=1}^w \left \| P_j n - D b_j \right \|_2^2 \end{matrix} + \mu^2 \left \| B \right \|_0 \right ) \nonumber \\
\text{s.t.} & ~ \left \| d_i \right \|_2 = 1, \ \ \left \| b_j \right \|_{\infty} \leq a, ~~ \forall i,j.
\end{align}
Here $y = \textbf{vec}(Y)$, $A = L^T \otimes I$---where $\otimes$ denotes the Kronecker product and $I$ is the $m_1 m_2 \times m_1 m_2$ identity matrix---and $n = \textbf{vec}(N)$. Also, $P_j$ denotes a patch extraction matrix that extracts (vectorized) patches of dimensions $w_x \times w_y \times w_z$ from $n$. All other terms are defined analogously to the corresponding terms in \eqref{eq:dlpi} with appropriate dimensions. 

The dictionary learning terms in \eqref{eq:dlnv} encourage the estimated normal vectors to be well-represented by sparse linear combinations of a few (learned) dictionary atoms. Intuitively,  this acts as an adaptive regularization that yields normal vectors that are more robust to noise and other non-idealities in the data.
Henceforth, we refer to this approach as the Dictionary Learning on Normal Vectors (DLNV) method.






\subsection{Algorithms for DLPI and DLNV} \label{sec:dl_sol}
We propose solving \eqref{eq:dlpi} and \eqref{eq:dlnv}, respectively, via block coordinate descent-type algorithms where we alternate between updating $n$ and $v$, respectively, with $(D,B)$ fixed and then updating $(D,B)$ with $n$ or $v$ held fixed. We omit the $(D,B)$ updates here due to space considerations, but the precise update expressions can be found in \cite{sairajfes2,dinokat2016}. 

\vspace{-2mm}
\subsubsection{$v$ update}
Solving \eqref{eq:dlpi} for $v$ with $D$ and $B$ fixed yields the problem
\begin{equation} \label{vupdate}
\min_{v} ~ \frac{1}{2} \left \| y -  v \right \|_2^2 + \begin{matrix} \lambda  \sum_{j=1}^c \left \| P_j v  - D b_j \right \|_2^2 \end{matrix}.
\end{equation}
Equation \eqref{vupdate} is a simple least squares problem whose solution $v$ satisfies the normal equation
\begin{equation}\label{eq:vnormal}
\left ( I + \begin{matrix} 2 \lambda \sum_{j=1}^c P_j^T P_j \end{matrix} \right ) v = y + \begin{matrix} 2 \lambda \sum_{j=1}^c P_j^T D b_j \end{matrix},
\end{equation}
where $I$ denotes the identity matrix. The matrix pre-multiplying $v$ in \eqref{eq:vnormal} is diagonal, so its inverse can be cheaply computed, allowing us to efficiently update $v$.

\vspace{-2mm}
\subsubsection{$n$ update}
On the other hand, solving \eqref{eq:dlnv} for $n$ with $D$ and $B$ fixed yields the problem
\begin{equation}\label{nupdate}
\min_{n} ~ \frac{1}{2} \left \| y - A n \right \|_2^2 + \begin{matrix} \lambda \sum_{j=1}^w  \left \| P_j n - D b_j \right \|_2^2 \end{matrix}.
\end{equation}
Note that while \eqref{nupdate} is also a least squares problem, its normal equation cannot be easily inverted as in \eqref{vupdate} due to the presence of the $A$ matrix. We therefore adopt a proximal gradient scheme \cite{parboyd}. The cost function in \eqref{nupdate} can be written in the form $f(n)+g(n)$ where $f(n) = \frac{1}{2} \left \| y - A n \right \|_2^2$ and $g(n) = \lambda \sum_{j=1}^w \left \| P_j n - D b_j \right \|_2^2$. The proximal updates thus become
\begin{equation} \label{nproxupdate}
n^{k+1} = \textbf{prox}_{\tau g}(n^{k} - \tau \nabla f(n^{k})),
\end{equation}
where
\begin{equation} \label{nprox}
\textbf{prox}_{\tau g} (z) := \argmin_{x} \ \frac{1}{2} \left \| z - x \right \|_2^2 + \tau g(x).
\end{equation}
Define $\tilde{n}^{k} := n^{k} - \tau \nabla f(n^{k})$. Then \eqref{nproxupdate} and \eqref{nprox} imply that $n^{k+1}$ satisfies the normal equation
\begin{equation}\label{eq:nnormal}
 \left(I +  \begin{matrix} 2 \tau \lambda \sum_{j=1}^w P_j^T P_j \end{matrix} \right ) n^{k+1} = \tilde{n}^{k} + \begin{matrix} 2 \tau \lambda \sum_{j=1}^w P_j^T D b_j  \end{matrix}.
\vspace{1mm}
\end{equation}
As in \eqref{eq:vnormal}, the matrix multiplying $n^{k+1}$ in \eqref{eq:nnormal} is diagonal and can be efficiently inverted, yielding $n^{k+1}$. Note that proximal gradient is one of a wealth of available iterative schemes for minimzing  the (quadratic) objective \eqref{nupdate}.


\section{Results} \label{sec:results}
\section{Results}
\label{sec:results}
In this section, we first compare StyleGAN-XL to the state-of-the-art approaches for image synthesis on ImageNet. We then evaluate the inversion and editing capabilities of StyleGAN-XL. As described above, we scale our model to a resolution of $1024^2$ pixels, which no prior work has attempted so far on ImageNet. The resolution of most images in ImageNet is lower. We therefore preprocess the data with a super-resolution network~\cite{Liang2021ICCV}, see supplementary.

\subsection{Image Synthesis}
Both our work and ~\cite{Dhariwal2021NEURIPS} use classifier networks to guide the generator. To ensure the models are not inadvertently optimizing for FID and IS, which also utilize a classifier network, we propose random-FID (rFID). For rFID, we calculate the Fr\'echet distance in the \texttt{pool\_3} layer of a randomly initialized inception network~\cite{Szegedy2015CVPR}.
The efficacy of random features for evaluating generative models has been demonstrated in~\cite{Naeem2020ICML}. Furthermore, we report sFID~\cite{Nash2021ICML} to assess spatial structure.
Lastly, sample fidelity and diversity are evaluated via precision and  recall~\cite{Kynknniemi2019NEURIPS}. 

In \tabref{tab:sotapr}, we compare StyleGAN-XL to the currently strongest GAN model (BigGAN-deep~\cite{Brock2019ICLR}) and diffusion models (CDM~\cite{Ho2022JMLR}, ADM~\cite{Dhariwal2021NEURIPS}) on ImageNet.
The values for ADM are calculated with and without additional methods (Upsampling \textbf{U} and Classifier Guidance \textbf{G}). For StyleGAN2, we report numbers by~\cite{Grigoryev2022ICLR}.
We find that StyleGAN-XL substantially outperforms all baselines across all resolutions in FID, sFID, rFID, and IS. 
An exception is recall, according
to which StyleGAN-XL’s sample diversity lies between BigGAN and
ADM, making progress in closing the gap between these model types.
BigGAN's sample quality is the best among all compared approaches, which comes at the price of significantly lower recall. 
StyleGAN-XL allows for the truncation trick to increase sample fidelity, i.e., we can interpolate a sampled style code $w$ with the class-wise mean style code $\bar{w}$.
We observe that for StyleGAN-XL, truncation does not increase precision, indicating that developing novel truncation methods for high-diversity GANs is an exciting research direction for future work.
Interestingly, StyleGAN-XL attains high diversity across all resolutions, which can be attributed to our progressive growing strategy. Furthermore, this strategy enables to scale to megapixel resolution successfully. Training at $1024^2$ for a single V100-day yields a noteworthy FID~of~$2.8$. At this resolution, we do not compare to baselines 
because of resource constraints as they are prohibitively expensive to train. 
visualizes generated samples at increasing resolutions.
\figref{fig:highres} visualizes generated samples at increasing resolutions. In the supplementary, we show additional  interpolations and qualitative comparisons to BigGAN and ADM.
\sotapr
\highres

\subsection{Inversion and Manipulation}
GAN-editing methods first \textit{invert} a given image into latent space, i.e., find a style code $w$ that reconstructs the image as faithful as possible when passed through $\bG_s$. Then, $w$ can be manipulated to achieve semantically meaningful edits~\cite{Goetschalckx2019ICCV,Shen2020TPAMI}.

\boldparagraph{Inversion.}
Standard approaches for inverting $\bG_s$ use either latent optimization~\cite{Abdal2019ICCV,Creswell2019NEURAL,Karras2020CVPR} or an encoder~\cite{Perarnau2016ARXIV,Alaluf2021ICCV,Tov2021TOG}. A common way to achieve low reconstruction error is to use an extended definition of the latent space: $\mathcal{W}+$. For $\mathcal{W}+$ a separate $\bw$ is chosen for each layer of $\bG_s$. However, as highlighted by~\cite{Zhu2020ECCV,Tov2021TOG}, this extended definition achieves higher reconstruction quality in exchange for lower editability. Therefore, ~\cite{Tov2021TOG} carefully design an encoder to maintain editability by mapping to regions of $\mathcal{W}+$ that are close to the original distribution of $\mathcal{W}$.
We follow~\cite{Karras2020CVPR} and use the original latent space $\mathcal{W}$.
We find that StyleGAN-XL already achieves satisfactory inversion results using basic latent optimization.
For inversion on the ImageNet validation set at $512^2$, StyleGAN-XL yields $\text{PSNR}=13.5$ on average, improving over BigGAN at $\text{PSNR}=10.8$. 
Besides better pixel-wise reconstruction, StyleGAN-XL's inversions are
semantically closer to the target images.
We measure the FID between reconstructions and targets, and StyleGAN-XL attains $\text{FID}=21.7$ while BigGAN reaches $\text{FID}=47.5$. 
For qualitative results, implementation details and additional metrics, we refer to the supplementary.

Given the results above, it is also possible to further refine the obtained reconstructions.~\cite{Roich2021ARXIV} recently introduced pivotal tuning inversion (PTI). PTI uses an initial inverted style code as a pivot point around which the generator is finetuned. Additional regularization prevents altering the generator output far from the pivot. Combining PTI with StyleGAN-XL allows us to invert both in-domain (ImageNet validation set) and out-of-domain images almost precisely. At the same time, the generator output remains perceptually smooth, see~\figref{fig:interpolations}.
\interpolations

\boldparagraph{Image Manipulation.}
Given the inverted images, we can leverage GAN-based editing methods~\cite{Voynov2020ICML,Haerkoenen2020NEURIPS,Shen2021CVPR,Kocasari2021WACV,Spingarn2021ICLR} to manipulate the style code $\bw$. In \figref{fig:editing}~(Left), we first invert a given source image via latent space optimization. 
We can then apply a manipulation directions obtained by, e.g., GANspace~\cite{Haerkoenen2020NEURIPS}.
Prior work~\cite{Jahanian2020ICLR} also investigates in-plane translation. This operation can be directly defined in the input grid of StyleGAN-XL. The input grid also allows performing extrapolation, see  \figref{fig:editing}~(Left).

An inherent property of StyleGAN is the ability of style mixing by supplying the style codes of two samples to different layers of $\bG_s$, generating a hybrid image. This hybrid takes on different semantic properties of both inputs. Style mixing is commonly employed for instances of a single domain, i.e., combining two human portraits. StyleGAN-XL inherits this ability and, to a certain extent, even generates out-of-domain combinations between different classes, akin to counterfactual images~\cite{Sauer2021ICLR}. This technique works best for aligned samples, similar to StyleGAN's originally favored setting, FFHQ. Curated examples are shown in \figref{fig:editing}~(Right).
\editing
\editingsupp


\section{Conclusion}
In this work, we explored the use of adaptive dictionary learning based regularization for the estimation of surfaces from their gradient fields. We showed that our proposed dictionary learning approach is able to effectively reject the addition of noise to gradient fields/images and produce more accurate and smooth representations of the underlying surfaces compared to existing methods. Our dictionary learning framework is very general and would be straightforward to combine with many existing algorithms. In future work, we plan to investigate these combinations and perform a more thorough study of the influence of the various parameters of our dictionary learning model on the reconstructed surfaces.

\bibliographystyle{IEEEbib}
\ninept
\bibliography{G2S_ICIP}

\end{document}
