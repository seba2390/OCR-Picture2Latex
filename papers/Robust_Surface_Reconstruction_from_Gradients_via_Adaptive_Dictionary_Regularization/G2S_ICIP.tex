\documentclass{article}

\usepackage{spconf}
\usepackage{amsmath}
\usepackage{graphicx}
\usepackage{enumerate}
\usepackage{color}
\usepackage{soul}
\usepackage{amsfonts}
\usepackage{amssymb}
\usepackage{caption}
\usepackage{subcaption}
\usepackage{float}
\usepackage{comment}
\usepackage{url}

%\captionsetup[table]{skip=3pt}
%\setlength{\belowcaptionskip}{-3pt}
%\usepackage[skip=0pt]{caption}

\DeclareMathOperator*{\argmin}{arg\,min}
\renewcommand{\vec}{\mathbf{vec}}

\title{ROBUST SURFACE RECONSTRUCTION FROM GRADIENTS VIA ADAPTIVE DICTIONARY REGULARIZATION}
\name{Andrew J. Wagenmaker, Brian E. Moore, and Raj Rao Nadakuditi
\thanks{This work was supported in part by the following grants: ONR grant N00014-15-1-2141, DARPA Young Faculty Award D14AP00086, and ARO MURI grants W911NF-11-1-0391 and 2015-05174-05.}}
\address{Department of EECS, University of Michigan, Ann Arbor, MI, USA}

\begin{document}
%\ninept

\maketitle

\begin{abstract}
This paper introduces a novel approach to robust surface reconstruction from photometric stereo normal vector maps that is particularly well-suited for reconstructing surfaces from noisy gradients. Specifically, we propose an adaptive dictionary learning based approach that attempts to simultaneously integrate the gradient fields while sparsely representing the spatial patches of the reconstructed surface in an adaptive dictionary domain. We show that our formulation learns the underlying structure of the surface, effectively acting as an adaptive regularizer that enforces a smoothness constraint on the reconstructed surface. Our method is general and may be coupled with many existing approaches in the literature to improve the integrity of the reconstructed surfaces. We demonstrate the performance of our method on synthetic data as well as real photometric stereo data and evaluate its robustness to noise.
\end{abstract}

\begin{keywords}
Dictionary learning, photometric stereo, sparse representations, inverse problems.
\end{keywords}

\section{Introduction} \label{sec:intro}
\subsection{The privacy policies and challenges in medical intelligence}
The privacy issue, while important in every domain, is enforced vigorously for medical data. Multiple level of regulations such as HIPAA~\cite{annas2003hipaa,centers2003hipaa,mercuri2004hipaa,gostin2009beyond} and the approval process for the Institutional Review Board (IRB)~\cite{bankert2006institutional} protect the patients' sensitive data from malicious copy or even tamper evidence of medical conditions~\cite{mirsky2019ct}. Like a double-edge sword, these regulations objectively cause insufficient collaborations in health records.
For instance, America, European Union and many other countries do not allow patient data leave their country~\cite{kerikmae2017challenges,seddon2013cloud}. As a result, many hospitals and research institutions are wary of cloud platforms and prefer to use their own server. Even if in the same country the medical data collaborate still face a big hurdle.


\subsection{The restriction of the medical data accessibility}
It's widely known that sufficient data volume is necessary for training a successful machine learning algorithm~\cite{domingos2012few} for medical image analysis. 
However, due to the policies and challenges mentioned above, it is hard to acquire enough medical scans for training a machine learning model. In 2016, there were approximately 38 million MRI scans and 79 million CT scans performed in the United States~\cite{papanicolas2018health}. Even so, the available datasets for machine learning research are still very limited: the largest set of medical image data available to public is 32 thousand~\cite{yan2018deeplesion} CT images, only 0.02\% of the annual acquired images in the United States.
In contrast, the ImageNet~\cite{deng2009imagenet} project, which is the large visual dataset designed for use in visual object recognition research, has more than 14 million images that have been annotated in more than 20,000 categories.

\subsection{Learning from synthetic images: a solution}
In this work, we design a framework using centralized generator and distributed discriminators to learn the generative distribution of target dataset. In the health entities learning context, our proposed framework can aggregate datasets from multiple hospitals to obtain a faithful estimation of the overall distribution. The specific task (e.g., segmentation and classification) can be accomplished locally by acquiring data from the generator. Learning from synthetic images has several advantages:

\textbf{Privacy mechanism}:
The central generator has limited information for the raw images in each hospital. When the generator communicates with discriminators in hospitals, only information about the synthetic image is transmitted. Such a mechanism prohibits the central generator's direct access to raw data thus secures privacy.

\textbf{Synthetic data sharing}: The natural of synthetic data allows the generator to share the synthetic images without restriction. Such aggregation and redistribution system can build a public accessible and faithful medical database. The inexhaustible database can benefit researchers, practitioners and boost the development of medical intelligence.
  
\textbf{Adaptivity to architecture updates}: The machine learning architecture evolves rapidly to achieve a better performance by novel loss functions~\cite{sudre2017generalised,hochberg1964depth}, network modules~\cite{hoffman2016fcn, ronneberger2015u,milletari2016v,qu2019improving} or optimizers~\cite{ruder2016overview, zeiler2012adadelta, mason2000boosting,zhang2019taming,zhanglocal}. 
%To accelerate this process, modern machine learning frameworks like Pytorch~\cite{ketkar2017introduction} and Tensorflow~\cite{abadi2016tensorflow} provide more efficient ways to build and update networks and training configurations.
We could reasonably infer that the recently well-trained model may be outdated or underperformed in the future as new architectures invented. Since the private-sensitive data may be not always accessible, even if we trained a model based on these datasets, we couldn't embrace new architectures to achieve higher performance. Instead of training a task-specific model, our proposed method trains a generator that learns from distributed discriminators. Specifically, we learn the distribution of private datasets by a generator to produce synthetic images for future use, without worrying about the lost of the proprietary datasets.

%Our proposed approach
To the best of our knowledge, we are the first to use GAN to address the medical privacy problem. Briefly, our contributions lie in three folds: (1) A distributed asynchronized discriminator GAN (AsynDGAN) is proposed to learn the real images' distribution without sharing patients' raw data from different datasets. (2) AsynDGAN achieves higher performance than models that learn from real images of only one dataset. (3) AsynDGAN achieves almost the same performance as the model that learns from real images of all datasets.



%regulation...
%
%
%Especially when Adversarial Generative Network(GAN) attract everyone's attention, the privacy of medical data face a more serious challenge. Though we could apply GAN to achieve many goals like artifact reduction[adversarial sparse-view CBCT], domain adaption for different disease or modality[task driven...], data augmentation[], the GAN, like a double-edged sword, could also hurt us by tampering the medical images, ie., add or remove critical medical findings.


%Since patient data in European countries is typically not allowed to leave Europe, many hospitals and research institutions are wary of cloud platforms and prefer to use their own servers.



\section{Surface Reconstruction from Gradient Fields} \label{sec:background}
%!TEX root = G2S_ICIP.tex

Let $n(x,y) \in \mathbb{R}^3$ denote the normal vector of a differentiable surface $z(x,y)$ at position $(x,y)$, and let $n_1(x,y)$, $n_2(x,y)$, and $n_3(x,y)$ denote the $x$, $y$, and $z$ components of this vector, respectively. Under this ideal model, one can relate the $x$ and $y$ derivatives of the surface $z$ to its normal vectors via the relation
\begin{equation} \label{eq:partials}
\frac{\partial z(x,y)}{\partial x} = -p(x,y), \ \ \ \frac{\partial z(x,y)}{\partial y} = q(x,y),
\end{equation}
where we have defined $p(x,y) := n_1(x,y)/n_3(x,y)$ and $q(x,y) := n_2(x,y)/n_3(x,y)$. In practice, the estimated (e.g., by photometric stereo) normal vectors of a surface and its gradient fields will not exactly satisfy \eqref{eq:partials}, so one must instead find a function $z(x,y)$ with derivatives \emph{close} to $p(x,y)$ and $q(x,y)$ in an appropriate sense, often by minimizing a variational problem of the form
\begin{equation} \label{eq:cont_cost}
\int \int_{\Omega} \left (\frac{\partial z(x,y)}{\partial x}  - p(x,y) \right )^2 + \left ( \frac{\partial z(x,y)}{\partial y} - q(x,y) \right )^2 \ dx \ dy.
\end{equation}
When our data is instead sampled on a discrete grid, we will not have access to a continuous normal map $n(x,y)$ but will instead have a matrix $N \in \mathbb{R}^{m \times n \times 3}$ containing the normal vectors of the object on the grid. Following \eqref{eq:partials}, we can compute matrices $P \in \mathbb{R}^{m \times n}$ and $Q \in \mathbb{R}^{m \times n}$ containing the measured gradients, and our goal then becomes to estimate the matrix $Z \in \mathbb{R}^{m \times n}$ containing the values of the surface $z(x,y)$ sampled on the grid. The discrete analogue of \eqref{eq:cont_cost} is commonly expressed \cite{simchony1990,frankot1988,harker2008} as a standard least squares problem of the form
\begin{equation} \label{eq:surf_ls}
z^* = \argmin_{z} \ \|A z - v \|_2^2,
\end{equation}
where $z = \vec(Z) \in \mathbb{R}^{mn}$ is the vectorized surface, $A$ is a numerical differentiation operator, and the vector $v$ is an appropriate function of the measured gradients, $P$ and $Q$. Solving this problem yields a representation of our surface that is optimal in the least squares sense.

Note that the specific forms of $A$ and $v$ can vary. One possible formulation is
\begin{equation}
A = \begin{bmatrix}
D_n \otimes I_m \\
I_n \otimes D_m
\end{bmatrix}, \ \ \ \ v = \begin{bmatrix}
\textbf{vec}(P) \\
\textbf{vec}(Q)
\end{bmatrix},
\end{equation}
where $D_n$ is the discrete first differences matrix, and $\otimes$ denotes the Kronecker product. However, multiple models are possible, each yielding reconstructed surfaces with different properties. Importantly, the dictionary learning based approach that we introduce in Section~\ref{sec:dl} can be coupled with any least squares model of the form \eqref{eq:surf_ls}, so our proposed approach is quite flexible.








\section{Adaptive Dictionary Learning Regularization} \label{sec:dl}
%!TEX root = PSviaDL.tex

In this section, we propose two adaptive dictionary learning methods for estimating the normal vectors of a surface from (possibly) noisy images, $Y$. Intuitively, these models seek to learn a locally sparse representation of the data with respect to a collection of learned basis ``atoms'' that capture the underlying local structure of the data.

\subsection{Preprocessing Images through Dictionary Learning (DLPI)}
Our first approach applies dictionary learning to the data in a preprocessing step before estimating the normal vectors. This formulation represents the input image data $Y$ as locally sparse in an adaptive dictionary domain---thereby removing non-idealities that are not well-represented by the dictionary. Specifically, we propose to solve the problem
\begin{align} \label{eq:dlpi}
\min_{v,B,D} & ~ \frac{1}{2} \left \| y -  v \right \|_2^2 + \lambda \left(\textstyle\sum_{j=1}^c \left \| P_j v  - D b_j \right \|_2^2 +  \mu^2 \left \| B \right \|_0 \right) \nonumber \\
\text{s.t.} & ~ \left \| d_i \right \|_2 = 1, \ \ \left \| b_j \right \|_{\infty} \leq a, ~~ \forall i,j.
\end{align}
%\ \ \text{rank}(R(d_i)) \leq r,
Here, $y = \textbf{vec}(Y)$ and $P_j$ is a matrix that extracts a (vectorized) 3D patch of dimensions $c_x \times c_y \times c_z$ from $v$, where $c_x$ and $c_y$ are the dimensions of the patches extracted from each image and $c_z$ is the number of distinct images whose patches are combined to form the 3D patch. $D \in \mathbb{R}^{c_x c_y c_z \times K}$ is a dictionary matrix whose columns $d_i$ are the (learned) dictionary atoms, and $B \in \mathbb{R}^{K \times c}$ is a sparse coding matrix whose columns $b_j$ define (usually sparse) linear combinations of dictionary atoms used to represent each patch. Also, $\left \|\cdot\right \|_0$ is the familiar $\ell_0$ ``norm", and $\lambda,\mu > 0$ are parameters. 

We impose the constraint $\|b_j\|_{\infty} \leq a$, where $a$ is typically very large, since \eqref{eq:dlpi} is non-coercive with respect to $B$, but the constraint is typically inactive in practice \cite{sairajfes2}. Without loss of generality, we impose a unit-norm constraint on the dictionary atoms $d_i$ to avoid scaling ambiguity between $D$ and $B$ \cite{kar}.
We allow the possibility that patches from $c_z > 1$ input images can be combined into a 3D patch to allow the dictionary atoms to learn correlated features between images, but one can set $c_z = 1$ to work with 2D per-image patches.

Once we have solved \eqref{eq:dlpi}, we reshape $v$ (back) into an $m_1 m_2 \times d$ matrix whose columns are vectorized (now preprocessed) images, and then we estimate the associated normal vectors using the standard least squares model \eqref{eq:ls}. Henceforth, we refer to this approach as the Dictionary Learning with Preprocessed Imgaes (DLPI) method.

\subsection{Normal Vector Computation through Dictionary Learning (DLNV)}
We next propose modifying \eqref{eq:ls} by applying an adaptive dictionary regularization term to the normal vectors, $N$, under the Lambertian model \eqref{eq:ls}. Specifically, we propose to solve the problem
\begin{align} \label{eq:dlnv}
\min_{n,B,D} & ~ \frac{1}{2} \left \| y - A n \right \|_2^2 + \lambda \left ( \begin{matrix} \sum_{j=1}^w \left \| P_j n - D b_j \right \|_2^2 \end{matrix} + \mu^2 \left \| B \right \|_0 \right ) \nonumber \\
\text{s.t.} & ~ \left \| d_i \right \|_2 = 1, \ \ \left \| b_j \right \|_{\infty} \leq a, ~~ \forall i,j.
\end{align}
Here $y = \textbf{vec}(Y)$, $A = L^T \otimes I$---where $\otimes$ denotes the Kronecker product and $I$ is the $m_1 m_2 \times m_1 m_2$ identity matrix---and $n = \textbf{vec}(N)$. Also, $P_j$ denotes a patch extraction matrix that extracts (vectorized) patches of dimensions $w_x \times w_y \times w_z$ from $n$. All other terms are defined analogously to the corresponding terms in \eqref{eq:dlpi} with appropriate dimensions. 

The dictionary learning terms in \eqref{eq:dlnv} encourage the estimated normal vectors to be well-represented by sparse linear combinations of a few (learned) dictionary atoms. Intuitively,  this acts as an adaptive regularization that yields normal vectors that are more robust to noise and other non-idealities in the data.
Henceforth, we refer to this approach as the Dictionary Learning on Normal Vectors (DLNV) method.






\subsection{Algorithms for DLPI and DLNV} \label{sec:dl_sol}
We propose solving \eqref{eq:dlpi} and \eqref{eq:dlnv}, respectively, via block coordinate descent-type algorithms where we alternate between updating $n$ and $v$, respectively, with $(D,B)$ fixed and then updating $(D,B)$ with $n$ or $v$ held fixed. We omit the $(D,B)$ updates here due to space considerations, but the precise update expressions can be found in \cite{sairajfes2,dinokat2016}. 

\vspace{-2mm}
\subsubsection{$v$ update}
Solving \eqref{eq:dlpi} for $v$ with $D$ and $B$ fixed yields the problem
\begin{equation} \label{vupdate}
\min_{v} ~ \frac{1}{2} \left \| y -  v \right \|_2^2 + \begin{matrix} \lambda  \sum_{j=1}^c \left \| P_j v  - D b_j \right \|_2^2 \end{matrix}.
\end{equation}
Equation \eqref{vupdate} is a simple least squares problem whose solution $v$ satisfies the normal equation
\begin{equation}\label{eq:vnormal}
\left ( I + \begin{matrix} 2 \lambda \sum_{j=1}^c P_j^T P_j \end{matrix} \right ) v = y + \begin{matrix} 2 \lambda \sum_{j=1}^c P_j^T D b_j \end{matrix},
\end{equation}
where $I$ denotes the identity matrix. The matrix pre-multiplying $v$ in \eqref{eq:vnormal} is diagonal, so its inverse can be cheaply computed, allowing us to efficiently update $v$.

\vspace{-2mm}
\subsubsection{$n$ update}
On the other hand, solving \eqref{eq:dlnv} for $n$ with $D$ and $B$ fixed yields the problem
\begin{equation}\label{nupdate}
\min_{n} ~ \frac{1}{2} \left \| y - A n \right \|_2^2 + \begin{matrix} \lambda \sum_{j=1}^w  \left \| P_j n - D b_j \right \|_2^2 \end{matrix}.
\end{equation}
Note that while \eqref{nupdate} is also a least squares problem, its normal equation cannot be easily inverted as in \eqref{vupdate} due to the presence of the $A$ matrix. We therefore adopt a proximal gradient scheme \cite{parboyd}. The cost function in \eqref{nupdate} can be written in the form $f(n)+g(n)$ where $f(n) = \frac{1}{2} \left \| y - A n \right \|_2^2$ and $g(n) = \lambda \sum_{j=1}^w \left \| P_j n - D b_j \right \|_2^2$. The proximal updates thus become
\begin{equation} \label{nproxupdate}
n^{k+1} = \textbf{prox}_{\tau g}(n^{k} - \tau \nabla f(n^{k})),
\end{equation}
where
\begin{equation} \label{nprox}
\textbf{prox}_{\tau g} (z) := \argmin_{x} \ \frac{1}{2} \left \| z - x \right \|_2^2 + \tau g(x).
\end{equation}
Define $\tilde{n}^{k} := n^{k} - \tau \nabla f(n^{k})$. Then \eqref{nproxupdate} and \eqref{nprox} imply that $n^{k+1}$ satisfies the normal equation
\begin{equation}\label{eq:nnormal}
 \left(I +  \begin{matrix} 2 \tau \lambda \sum_{j=1}^w P_j^T P_j \end{matrix} \right ) n^{k+1} = \tilde{n}^{k} + \begin{matrix} 2 \tau \lambda \sum_{j=1}^w P_j^T D b_j  \end{matrix}.
\vspace{1mm}
\end{equation}
As in \eqref{eq:vnormal}, the matrix multiplying $n^{k+1}$ in \eqref{eq:nnormal} is diagonal and can be efficiently inverted, yielding $n^{k+1}$. Note that proximal gradient is one of a wealth of available iterative schemes for minimzing  the (quadratic) objective \eqref{nupdate}.


\section{Results} \label{sec:results}
%!TEX root = G2S_ICIP.tex

\begin{figure*}
\centering
\begin{subfigure}[b]{0.23\textwidth}
  \includegraphics[width=\textwidth]{figures/frog_13_17_dls_view1.png}
  %\caption{DLS}
\end{subfigure}
~ 
\begin{subfigure}[b]{0.23\textwidth}
  \includegraphics[width=\textwidth]{figures/frog_13_17_glss_view1.png}
  %\caption{SR}
\end{subfigure}
~ %\quad
\begin{subfigure}[b]{0.23\textwidth}
  \includegraphics[width=\textwidth]{figures/frog_13_17_tv_view1.png}
  %\caption{TV}
\end{subfigure}
~
\begin{subfigure}[b]{0.23\textwidth}
  \includegraphics[width=\textwidth]{figures/frog_13_17_sim2_view1.png}
  %\caption{DCTLS}
\end{subfigure}
\end{figure*}

\setcounter{figure}{2}  

\begin{figure*}
\centering
\begin{subfigure}[b]{0.23\textwidth}
  \includegraphics[width=\textwidth]{figures/frog_13_17_dls_view1_zoom.png}
  \caption{\textbf{DLS}}
\end{subfigure}
~ 
\begin{subfigure}[b]{0.23\textwidth}
  \includegraphics[width=\textwidth]{figures/frog_13_17_glss_view1_zoom.png}
  \caption{SR}
\end{subfigure}
~ %\quad
\begin{subfigure}[b]{0.23\textwidth}
  \includegraphics[width=\textwidth]{figures/frog_13_17_tv_view1_zoom.png}
  \caption{TV}
\end{subfigure}
~
\begin{subfigure}[b]{0.23\textwidth}
  \includegraphics[width=\textwidth]{figures/frog_13_17_sim2_view1_zoom.png}
  \caption{DCTLS}
\end{subfigure}
\caption{Surface reconstructions of the photometric stereo Frog dataset \cite{xiong2015shading} with SNR = 17 dB.}
\label{fig:frog_zoomed}
\end{figure*}

In this section, we numerically evaluate our proposed DLS method on several datasets. In each case we compare our method to the spectral regularization method (SR) \cite{harker2015}, the isotropic total variation (TV) method \cite{queau2015}, and DCT based least squares (DCTLS) \cite{simchony1990}. For methods that include tunable parameters, we sweep over a wide range of values, reporting the best results obtained. Our proposed DLS method can incorporate any least squares based solver by simply defining $A$ and $v$ in \eqref{eq:dl_surf} accordingly. For all results given here, we use the least squares cost found in \cite{simchony1990}. To evaluate the robustness of each algorithm, we add Gaussian noise to the data.

For our proposed DLS method, we use dictionary atoms of size $8 \times 8$ pixels and a square $64 \times 64$ dictionary $D$, initialized with a DCT matrix. We extracted patches from $z$ using a spatial stride of two pixels in each direction, allowing adjacent patches to overlap. Finally, we initialized $z$ as the vectorized surface produced by solving the stand-alone least squares problem in \cite{simchony1990}, and initialized $B = 0$.

\subsection{Synthetic Surface Reconstructions}
To quantitatively evaluate our method, we first considered two synthetic datasets, which we call ``Tent'' and ``Vase'', for which we have analytic expressions for $z = f(x,y)$. Given $f(x,y)$, we can differentiate to obtain the gradients, $\partial f(x,y) / \partial x$ and $\partial f(x,y) / \partial y$, and sample on a discrete grid. After reconstructing the surface from these gradients subject to additive noise, we evaluate the integrity of the reconstructions against the ground truth, $f(x,y)$, via the SSIM metric \cite{wang2004}. In these experiments, we add noise directly to the gradient fields to achieve a prescribed signal-to-noise ratio (SNR).

Figures~\ref{fig:tent} and \ref{fig:vase} show the reconstructed surfaces produced by each algorithm. As these images illustrate, the proposed DLS method produces significantly smoother surfaces from noisy data compared to the existing methods. Intuitively, the locally sparse model imposed by the dictionary regularization denoises the surfaces, while the adaptive nature of the dictionary allows DLS to represent and reconstruct both sharp edges and smooth regions on a data-dependent basis.

The surfaces obtained by SR, TV, and DCTLS are much more sensitive to the noisy gradients. Indeed, while they retain the general shape of the surface, they exhibit significantly more corruption. In particular, the spectral regularization method seems to introduce a systematic ``rippling" into the reconstructions.
\vspace{2mm}

\begin{table}[ht!]
\centering
\begin{tabular}{|c|c|c|c|c|}
\cline{1-5}
SNR (dB) & \bf{DLS} & SR & TV & DCTLS \\
\cline{1-5}
1 & \textbf{0.969} & 0.944 & 0.918 & 0.924 \\ 
\cline{1-5}
5 & \textbf{0.971} & 0.950 & 0.938 & 0.944 \\ 
\cline{1-5}
10 & \textbf{0.976} & 0.956 & 0.957 & 0.962 \\ 
\cline{1-5}
20 & \textbf{0.988} & 0.969 & 0.979 & 0.983 \\ 
\cline{1-5}
30 & \textbf{0.995} & 0.978 & 0.989 & 0.992 \\ 
\cline{1-5}
40 & \textbf{0.997} & 0.985 & 0.994 & 0.996 \\ 
\cline{1-5}
50 & \textbf{0.998} & 0.988 & 0.996 & \textbf{0.998} \\ 
\cline{1-5}
60 & \textbf{0.999} & 0.989 & 0.997 & 0.998 \\ 
\cline{1-5}
\end{tabular}
\caption{Quality of Tent reconstructions in SSIM as a function of SNR.}
\label{tab:pyramid}
\end{table}

\begin{table}[ht!]
\vspace{-4mm}
\centering
\begin{tabular}{|c|c|c|c|c|}
\cline{1-5}
SNR (dB) & \bf{DLS} & SR & TV & DCTLS \\
\cline{1-5}
1 & \textbf{0.958} & 0.930 & 0.889 & 0.894 \\ 
\cline{1-5}
5 & \textbf{0.966} & 0.934 & 0.911 & 0.915 \\ 
\cline{1-5}
10 & \textbf{0.971} & 0.942 & 0.933 & 0.936 \\ 
\cline{1-5}
20 & \textbf{0.977} & 0.961 & 0.965 & 0.966 \\
\cline{1-5} 
30 & \textbf{0.982} & 0.975 & 0.981 & 0.981 \\ 
\cline{1-5}
40 & \textbf{0.990} & 0.982 & 0.990 & 0.989 \\ 
\cline{1-5}
50 & \textbf{0.993} & 0.984 & \textbf{0.993} & 0.992 \\ 
\cline{1-5}
60 & \textbf{0.995} & 0.985 & \textbf{0.995} & 0.993 \\ 
\cline{1-5}
\end{tabular}
\caption{Quality of Vase reconstructions in SSIM as a function of SNR.}
\label{tab:vase}
\end{table}

\vspace{-2mm}
Tables~\ref{tab:pyramid} and \ref{tab:vase} numerically corroborate the qualitative results from Figures~\ref{fig:tent} and \ref{fig:vase}. In the low SNR regime, DLS significantly outperforms the other approaches. As SNR increases, the gap decreases. When the data is essentially noiseless, DLS, TV, and DCTLS are all able to reconstruct the surfaces with comparably negligible error.


\subsection{Photometric Stereo}
We now return to the problem of reconstructing a 3D representation of an object from normal vectors obtained through photometric stereo. We consider a dataset containing 10 images, each taken under a unique, known lighting direction, and corrupt the images with Gaussian noise at a prescribed SNR. We compute the normal vectors from the noisy images using the standard least squares approach \cite{wu2011}. Given the normal vectors, we then compute gradient fields as discussed in Section 2, and from those we generate a 3D reconstruction of the object using each method. Figure~\ref{fig:frog_zoomed} illusrates the results of this procedure on the Frog dataset,\footnote{This dataset can be found at \url{http://vision.seas.harvard.edu/qsfs/Data.html}} which contains real images of a frog statue \cite{xiong2015shading}.

The reconstructions in Figure~\ref{fig:frog_zoomed} showcase the ability of the proposed DLS approach to produce a smoother surface from the noisy gradients compared to the existing methods. The denoising capability of DLS may prove valuable when running photometric stereo on real-world data, where noise and other non-idealities are inevitable.




\section{Conclusion}
In this work, we explored the use of adaptive dictionary learning based regularization for the estimation of surfaces from their gradient fields. We showed that our proposed dictionary learning approach is able to effectively reject the addition of noise to gradient fields/images and produce more accurate and smooth representations of the underlying surfaces compared to existing methods. Our dictionary learning framework is very general and would be straightforward to combine with many existing algorithms. In future work, we plan to investigate these combinations and perform a more thorough study of the influence of the various parameters of our dictionary learning model on the reconstructed surfaces.

\bibliographystyle{IEEEbib}
\ninept
\bibliography{G2S_ICIP}

\end{document}
