\section{Conclusion} \label{sec:conclusion}
\vspace{-1mm}
In this paper, we have investigated the functional split optimization problem i\thirdrev{n which the BS functions can be deployed at the CU or DUs.} We have formulated the problem mathematically and analyzed the complexity, which is shown to be combinatorial and NP-hard. 
%
Because finding the exact solution is computationally expensive and \thirdrev{precise modelling the actual vRAN system is highly non-trivial}, we have proposed CDRS \thirdrev{as a solution framework to optimize the functional splits of the BSs amidst minimal assumptions about the underlying system}.
\thirdrev{We have developed CDRS using} a chain rule-based stochastic policy to handle the interdependence between split decisions and the large action space. \thirdrev{We have applied} LSTM networks-based sequence-to-sequence model to approximate the policy. \thirdrev{Since this policy is limited to an unconstrained problem, and vRAN's constraint requirements bound each function placement decision}, we have leveraged a constrained policy gradient method to train the policy. We have \thirdrev{also provided} a search strategy \thirdrev{by greedy decoding or temperature sampling to improve the optimality performance at the test time}. 
%
\thirdrev{The performance of CDRS has been} extensively evaluated using synthetic and real network datasets. The results have shown that CDRS successfully learns the functional split decision with less than 0.05\% optimality gap, \thirdrev{attains considerable cost savings} compared to \thirdrev{C-RAN or D-RAN systems}, and has a faster computational time than the optimal baseline. 
%Finally, our proposed framework has effectively solved the functional split optimization amidst minimal assumptions about the underlying system, offering a near-optimal solution and fast computational time.