%\documentclass[journal,12pt, onecolumn, draftclsnofoot]{IEEEtran}
\documentclass[journal, final]{IEEEtran}

\usepackage{textpos}
\newcommand\copyrighttext{%\small  This work has been accepted in IEEE Transactions on Wireless Communications. \\
%	\small \textcopyright 2022 IEEE. Personal use of this material is permitted. Permission from IEEE must be obtained for all other uses, in any current or future media including reprinting/republishing this material for advertising or promotional purposes, creating new collective works, for resale or redistribution to servers or lists, or reuse of any copyrighted component of this work in other works. 
%\centering
\small This article has been accepted for publication in IEEE Transactions on Wireless Communications.  %\textcopyright 2022 IEEE.

}
\newcommand\copyrightnotice{%
	\begin{tikzpicture}[remember picture,overlay]
		\node[anchor=north,yshift=-15pt] at (current page.north)  {\copyrighttext}
		;
	\end{tikzpicture}%
}

%%% for comsoc format: remove asmthm, amssymb. Replace with netxmath
\usepackage{setspace}
%\usepackage{newtxmath}
%\usepackage{tcolorbox}
\usepackage{amsmath}
\usepackage{bm}
\usepackage{mdframed}
\usepackage{amsthm}
\newtheorem{thm}{Theorem}
\usepackage[utf8]{inputenc}
\usepackage{graphicx}
\usepackage[colorinlistoftodos]{todonotes}
\usepackage{amssymb}
\usepackage{longtable}
\usepackage[colorlinks=true, allcolors=blue]{hyperref}
\usepackage[noadjust]{cite}
\allowdisplaybreaks
\usepackage{lineno}
\usepackage{lipsum}
\usepackage{multicol}
\usepackage{subcaption}
\usepackage{enumitem}
\newtheorem{theorem}{Theorem}
\newtheorem{definition}{Definition}
\newtheorem{lemma}{Lemma}
\newtheorem{assumption}{Assumption}
\usepackage{multirow}
\usepackage{array}

\usepackage{xcolor}

\usepackage[vlined,linesnumbered,ruled,resetcount]{algorithm2e}


\newcommand{\fm}[1]{{\color{purple}\textbf{}{FM: #1}}}


\newcommand{\rev}[1]{{\color{black}#1}}
\newcommand{\secrev}[1]{{\color{black}#1}}
\newcommand{\thirdrev}[1]{{\color{black}#1}}

\SetKwInput{KwInput}{Input}                % Set the Input
\SetKwInput{KwOutput}{Output}              % set the Output
\SetKwInput{KwInitialize}{Initialize}




%%%%%%%%%%%TABLE PACKAGES
\usepackage{booktabs}
\usepackage{caption}
\usepackage{float}
%\usepackage{titlesec}
\usepackage{capt-of}
\usepackage{arydshln}
\usepackage{soul,xcolor}
\usepackage{tabularx}

%dashed line
\usepackage{arydshln}
\setlength\dashlinedash{0.2pt}
\setlength\dashlinegap{1.5pt}
\setlength\arrayrulewidth{0.3pt}

%Widows & Orphans & Penalties

\widowpenalty500
\clubpenalty500
\clubpenalty=9996
\exhyphenpenalty=50 %for line-breaking at an explicit hyphen
\brokenpenalty=4991
\predisplaypenalty=10000
\postdisplaypenalty=1549
\displaywidowpenalty=1602
\floatingpenalty = 20000
%%%%%%%%%%%TABLE

\title{Constrained Deep Reinforcement Based Functional Split Optimization in Virtualized RANs}


\author{\IEEEauthorblockN{Fahri Wisnu Murti, Samad Ali, and Matti Latva-aho}\\
%\vspace{2mm}
\IEEEauthorblockA{
	Centre for Wireless Communications, University of Oulu, Finland}
\thanks{A preliminary version of this work appears in IEEE ICC 2021 Workshop \cite{murti_iccw_cdrs_vran}. This research has been supported by the Academy of Finland, 6G Flagship program under Grant 346208.}%
}



\IEEEoverridecommandlockouts
\begin{document}
	
%\linespread{1.23}

\maketitle
%\thispagestyle{plain}
%\pagestyle{plain}

%\vspace{-1.cm}
\begin{abstract}
\thirdrev{In virtualized radio access network (vRAN), the base station (BS) functions are decomposed into virtualized components that can be hosted at the centralized unit or distributed units through functional splits. Such flexibility has many benefits; however, it also requires solving the problem of finding the optimal splits of functions of the BSs in such a way that minimizes the total network cost. The underlying vRAN system is complex and precise modelling of it is not trivial. Formulating the functional split problem to minimize the cost results in a combinatorial problem that is provably NP-hard, and solving it is computationally expensive. In this paper, a constrained deep reinforcement learning (RL) approach is proposed to solve the problem with minimal assumptions about the underlying system. Since in deep RL, the action selection is the outcome of inference of a neural network, it can be done in real-time while training to update the neural networks can be done in the background. However, since the problem is combinatorial, even for a small number of functions, the action space of the RL problem becomes large. Therefore, to deal with such a large action space, a chain rule-based stochastic policy is exploited in which a long short-term memory (LSTM) network-based sequence-to-sequence model is applied to estimate the policy that is selecting the functional split actions. However, the utilized policy is still limited to an unconstrained problem, and each split decision is bounded by vRAN’s constraint requirements. Hence, a constrained policy gradient method is leveraged to train and guide the policy toward constraint satisfaction. Further, a search strategy by greedy decoding or temperature sampling is utilized to improve the optimality performance at the test time. Simulations are performed to evaluate the performance of the proposed solution using synthetic and real network datasets. Our numerical results show that the proposed RL solution architecture successfully learns to make optimal functional split decisions with the accuracy of the solution is up to 0.05\% of the optimality gap. Moreover, our solution can achieve considerable cost savings compared to C-RAN or D-RAN systems and a faster computational time than the optimal baseline.}

\end{abstract}

\IEEEpeerreviewmaketitle

\copyrightnotice
%%%%%%%%%%%%%%%%%%%%%%%%%%%%%%%% ================== MAIN CONTENT ==================
% INTRODUCTION
\vspace{-2mm}
\section{Introduction}
\label{sec:Introduction}


The goal in top-$\size$ recommendation is to recommend to each
consumer a small set of $\size$ items from a large collection of
items~\cite{cremonesi2010performance}.  For example, Netflix may want
to recommend $\size$ appealing movies to each consumer.  Collaborative
Filtering (CF)~\cite{herlocker2002empirical,lee2012comparative} is a
common top-$\size$ recommendation method.  CF infers user interests by
analyzing partially observed user-item interaction data, such as user
ratings on movies or historical purchase
logs~\cite{kanagal2012supercharging}. The main assumption in CF is that
users with similar interaction patterns have similar interests.


Standard CF methods for top-$\size$ recommendation focus on making  suggestions  that accurately reflect the user's preference history. However, as  observed in previous work,  CF recommendations are generally biased toward  popular items, leading to a rich get richer effect~\cite{vargas2014improving,steck2011item}.  The major reasons for this are \textit{popularity bias} and \textit{sparsity} of CF interaction data (detailed in Section~\ref{sec:related-work}). In a nutshell, to maintain  accuracy, recommendations are generated from the dense regions of the data,  where the popular items lie.  

However,  accurately suggesting popular items, may not be satisfactory for the consumers. For example, in Netflix, an accuracy-focused movie recommender may recommend ``Star Wars: The Force Awakens'' to users who have seen ``Star Wars: Rogue One''.  But, those users are probably already aware of ``The Force Awakens''. Considering additional factors, such as novelty of recommendations,  can lead to more effective suggestions~\cite{cremonesi2010performance,Castells2015,zhang2008avoiding,ziegler2005improving,zhang2012auralist}. 
%Second, accuracy-focused models typically achieve a   overall item-space coverage across their recommendations,  whereas high item-space coverage helps providers of the items increase revenue
%, users satisfaction since they are  likely already aware of or can find these items on their own.  

Focusing on popular items also adversely affects the satisfaction of  the providers of the items. This is because  accuracy-focused models typically achieve a  low overall item space coverage across their recommendations, whereas   high item space coverage helps providers of the items increase their revenue~\cite{vargas2014improving,Castells2015,adomavicius2011maximizing,anderson2006thelongtail, yin2012challenging,adomavicius2012improving}.
%accuracy-focused models typically achieve a

In contrast to the relatively small number of popular items, there are copious  {\it long-tail\/} items that have fewer observations (e.g., ratings) available. More precisely,  using the Pareto  principle (i.e.,~the $80/20$ rule),  long-tail items can be defined as items that generate the lower $20\%$ of observations~\cite{yin2012challenging}. Experimentally we found that these items correspond to almost $85\%$ of the items in several datasets (Sections~\ref{sec:Notation} and \ref{sec:Experiments}). %Table~\ref{tab:DatasetStatsticsSmall})


As previously shown, one way to improve the novelty of top-$\size$ sets is to recommend interesting long-tail items~\cite{cremonesi2010performance,ge2010beyond}.  The intuition  is that since they have fewer observations available,  they are more likely to be unseen~\cite{Kaminskas:2016:DSN:3028254.2926720}.  
 %For example, in online commerce,  newly added items are long-tail items that are yet to be discovered.  
Moreover, long-tail item promotion also results in higher overall coverage of the item space%, which increases profits for providers of the items
~\cite{vargas2014improving,Castells2015,zhang2008avoiding,zhang2012auralist,adomavicius2011maximizing,anderson2006thelongtail,yin2012challenging,jambor2010optimizing}. Because long-tail promotion reduces accuracy~\cite{steck2011item}, there are trade-offs to be explored.


%original submitted to ICDE
%This work studies three aspects of top-$\size$ recommendation: accuracy, novelty, and item-space coverage, and examines their trade-offs. In most previous work, predictions of a base recommendation system are re-ranked to handle their trade-offs~\cite{adomavicius2012improving,jambor2010optimizing,zhang2013personalize,wang2009portfolio}. Due to performance considerations, however, these techniques are not customized per user. For example,  parameters that balance the trade-off between novelty and accuracy are cross-validated at a global level.  This can be detrimental since users have varying preferences for  objectives such as long-tail novelty. We explore how to  automatically infer  user  preference for long-tail novelty, and how to leverage  it to correct  the popularity bias in standard recommender models. Our work does not rely on any additional contextual data, although such data, if available, can help promote newly-added long-tail items~\cite{agarwal2009regression,Saveski:2014:ICR:2645710.2645751}.

This work studies three aspects of top-$\size$ recommendation: accuracy, novelty, and item space coverage, and examines their trade-offs. In most previous work, predictions of a base recommendation algorithm are \textit{re-ranked} to handle these trade-offs~\cite{adomavicius2012improving,jambor2010optimizing,zhang2013personalize,wang2009portfolio}. The re-ranking models are computationally efficient but suffer from two drawbacks. First, due to performance considerations,  parameters that balance the trade-off between novelty and accuracy  are not customized per user. Instead they are cross-validated at a global level.  This can be detrimental since users have varying preferences for  objectives such as long-tail novelty. Second,  the re-ranking methods are often limited to a specific base recommender  that may be sensitive to dataset density. 
As a result, the datasets are pruned and the problem is studied in dense settings~\cite{adomavicius2012improving,ho2014likes}; but real world  scenarios are often sparse~\cite{kanagal2012supercharging,liu2017experimental}.   
% Because  dataset density can impact the performance of most base recommenders (like R-SVD), which in turn affects the performance of the re-ranking model, 

\iffalse
We address these limitations by directly inferring  user  preference for long-tail novelty  from interaction data.  This  allows us to customize the re-ranking  per user, and design a \textit{generic} framework, which resolves the second problem. In particular, since the long-tail novelty preferences are estimated independently of any base  recommender model, we can  plug-in an appropriate base recommender w.r.t. the dataset sparsity.% including ones that are more suitable for sparse settings.  

Modelling  user  preference for  long-tail novelty using only item popularity statistics, e.g., the average popularity of rated items as in~\cite{jugovac2017efficient}, disregards additional information like whether the user found the item interesting and the long-tail preferences of other users  of the items. \iffalse To incorporate them, we introduce the notion of  \emph{item long-tail importance}. Both  user long-tail preferences and item long-tail importance are dependent:  a user has high preference for discovering long-tail items if she is interested in important long-tail items, and an item that is associated with many of these kinds of users is likely to be more important.  We propose a joint optimization framework to directly learn,  from interaction data, both the users' long-tail preferences and the  items' long-tail importance. \fi
We propose an optimization approach that  incorporates  this information and  directly learns,  from interaction data, the users' long-tail novelty preferences.

Next, we use these learned preferences  to design a  top-$\size$ recommendation framework thats is generic, and provides customized balance between accuracy, novelty, and coverage. We refer to it as framework as GANC.  Using GANC, we design a novel algorithm, {\it Ordered Sampling-based Locally Greedy (OSLG)\/}, that relies on the learned long-tail novelty preferences  to scalably correct for popularity bias. Our work does not rely on any additional contextual data, although such data, if available, can help promote newly-added long-tail items~\cite{agarwal2009regression,Saveski:2014:ICR:2645710.2645751}. In summary:
\fi

We address the first limitation by directly inferring  user  preference for long-tail novelty  from interaction data.   Estimating these  preferences  using only item popularity statistics, e.g., the average popularity of rated items as in~\cite{jugovac2017efficient}, disregards additional information, like whether the user found the item interesting or the long-tail preferences of other users  of the items. We propose an approach that  incorporates  this information and  learns the users' long-tail novelty preferences from interaction data.

This approach allows us to customize the re-ranking  per user, and  design a \textit{generic} re-ranking framework, which resolves the second limitation of prior work. In particular, since the long-tail novelty preferences are estimated independently of any base recommender, we can  plug-in an appropriate one w.r.t. different factors, such as the dataset sparsity.

Our top-$\size$ recommendation framework, \textbf{GANC}, is \textbf{G}eneric, and provides customized balance between \textbf{A}ccuracy, \textbf{N}ovelty, and \textbf{C}overage. % Moreover, based on the learned long-tail novelty preferences, we also design a novel algorithm, {\it Ordered Sampling-based Locally Greedy (OSLG)\/}, that relies on the learned long-tail novelty preferences  to scalably correct for popularity bias. 
Our work does not rely on any additional contextual data, although such data, if available, can help promote newly-added long-tail items~\cite{agarwal2009regression,Saveski:2014:ICR:2645710.2645751}. In summary:

%Consider  the following toy example:
\vspace{-0.2cm}
\begin{table}[htb]
\centering
\scriptsize
%\small
\begin{tabular}{ccccccc} 
%\toprule
%&\multirow{2}{*}{}&\multicolumn{7}{c}{Ratings}\\
& & \cellcolor{blue!35}$w_1$ &\cellcolor{blue!18} $w_2$ & $\dots$ &\cellcolor{blue!8} $w_{89}$  &\cellcolor{blue!8} $w_{99}$   
\\
&   &$i_1$&$i_2$&$\dots$&$i_{89}$&$i_{90}$\\ 
\cmidrule(r){3-7} 	 
%\midrule
\cellcolor{red!35}$\theta_1$  &$u_1 $   &5 &   & $\dots$ &  &   \\
\cellcolor{red!28}$\theta_2$  &$u_2$     &5 &    & $\dots$ &  &  \\
 $\theta_3=?$  &$\bf u_3$  &5 &  &   $\dots$ &  &  \\
\cellcolor{red!10}$\theta_4$ & $u_4$  &  &5   & $\dots$ & &\\ 
\cellcolor{red!10}$\theta_5$ & $u_5$  &  & 5  & $\dots$ & &\\ 
$\theta_6=?$  & $\bf u_6$ & &5  &      $\dots$& &  \\ 
 & & $\hdots$  &$\hdots$   &$\hdots$   &$\hdots$   &$\hdots$  \\
%\midrule 
\cmidrule(r){3-7} 	 
\multicolumn{2}{c}{item pop.}  & 3  & 3  & $\dots$ &50&60\\  
%\bottomrule
%$ f_i$    &3  &3  &1  &3  &1  &2  \\  \hline
\end{tabular}
%#.
\caption{Simplified user-item interaction data. The user long-tail novelty preference ($\theta_u$), item long-tail importance weight ($w_i$) are highlighted. Darker colors indicate larger values. } \label{tab:example}
\end{table} 
\vspace{-0.2cm}
\begin{example}  
In Table~\ref{tab:example}, we are interested in estimating $\theta_3$ and $\theta_6$,  the long-tail preference of users $u_3$ and $u_6$ who have each rated a single movie. Additional ratings for other users  are not included here.  Considering only rating information, we observe $i_1$ and $i_2$ are  equally popular $|\mathcal{U}_{i_1}^{\trainset}| = |\mathcal{U}_{i_2}^{\trainset}|=3$, and $r_{31}=5$ and $r_{62}=5$. Using Eq.~\ref{eq:tfidf-risk}  we have $\theta_3 = \theta_6$. However, if we were given the long-tail preferences of the each item's user set, specifically that $u_1$ and $u_2$ have high long-tail preference (darker red), while $u_4$ and $u_5$ have lower long-tail preference (lighter red), we could conclude $i_1$ is a more important long-tail item compared to $i_2$ (indicated by a darker blue shade for $w_1$), and we expect  $\theta_3 \geq \theta_6$.

% On the other hand, if we knew that $u_4$ and $u_5$ have lower long-tail preference, we could conclude $i_2$ is a  less significant long-tail item. Therefore, However, if we  consider the long-tail preferences of other users, we may reason differently.    We need another variable $w_i$ which captures this information. 
%we would conclude that $u_3$ has higher long-tail preference compared to $u_6$, since the users $i_1$ is a more prominent long-tail item. 

% Relying only  on item popularity information, we would  conclude   $u_3$ and $u_6$ have equal long-tail preference, since $i_1$ and $i_2$ are  equally popular. However, considering  the second column,  long-tail preference of users,  long-tail importance for each item,  which captures the long-tail preference of its users. Since  that  both users of $i_1$ have high long-tail preference while  the users of $i_2$ have lower preference,  we may conclude $i_1$ is a more important long-tail item compared to $i_2$. Therefore, $u_3$'s long-tail preference should be at least as large as $u_6$'s preference. Specifically, consider two  items $i_1$ and $i_2$, with the following rating data: $i_1=\{u_1:5, u_2:5, u_3:5 \}$, $i_2=\{u_4:5, u_5:5, u_6:5\}$.  

%Table~\ref{tab:example} shows  simplified rating data. We want an estimate of the long-tail preference of $u_3$ and $u_6$, who have each  rated a single movie.  Relying only  on movie popularity information, we would  conclude   $u_3$ and $u_6$ have similar long-tail preference, since $m_1$ and $m_2$ are  equally popular. However, considering the long-tail preferences of other users of those movies, we may reason differently: since $u_1$ and $u_2$ have high long-tail preference, and $u_4$ and $u_5$ have low long-tail preference, $m_1$ is a more prominent long-tail item compared to $m_2$. Therefore, it is likely that $u_3$ has higher long-tail preference compared to $u_6$.considering the long-tail preferences of other users of those movies, we may reason differently.  For example, 
\label{ex:running}
\end{example}



%------------------------------

\iffalse
\begin{example}
Table~\ref{tab:example} shows rating data for a simplified system. %Note the user-item interaction matrix is sparse.
For this example, we define popular movies as those that have received  three or more ratings; $\{m_1, m_2, m_4\}$ are popular and  $\{m_3, m_5, m_6\}$ are niche movies. We observe $u_1$ and $u_3$  have rated relatively popular movies (risk-averse) while $u_2$ and $u_4$ have rated niche movies (risk-loving). 
\label{ex:running}
\end{example}

\begin{table}[htb]
\centering
\scriptsize
\begin{tabular}{ccccccc} 
\toprule
			&$m_1$ &$m_2$   &$m_3$    &$m_4$   &$m_5$ &$m_6$  \\ \hline 
$u_1 $ &5  &4  & - &-  &-  &-   \\
$u_2$  &-  &-  &-  &-  &5  &5   \\
$u_3$  &-  &4  &-  &5  &-  &-   \\
$u_4$  &-  &-  &3  &-  &-  &4   \\ 
$u_5$  &5  &-  &-  &3  &-  &-   \\ 
$u_6$  &4  &2  &-  &4  &-  &-   \\ 
\bottomrule
%$ f_i$    &3  &3  &1  &3  &1  &2  \\  \hline
\end{tabular}
\caption{User-Movie rating data} \label{tab:example}
\end{table}

It is essential to consider consumer characteristics in designing recommender systems so that they promote long-tail items to the right group of users and spread demand evenly between hit and niche items.  

\fi





%------------------------------
\iffalse
\begin{table}[htb]
\centering
\scriptsize
\begin{tabular}{ccccccc} 
\toprule
			&$m_1$ &$m_2$   &$m_3$    &$m_4$   &$m_5$ &$m_6$  \\ \hline 
$u_1 $ &\textbf{5}  & \textbf{4}  &\textcolor{gray}{ 1.2} &-  &-  &-   \\
$u_2$  &-  &-  &-  &-  & \textbf{5}  &\textbf{5}   \\
$u_3$  &-  &\textbf{4}  &-  &\textbf{5}  &-  &-   \\
$u_4$  &-  &-  &\textbf{3}  &-  &-  &\textbf{4}   \\ 
$u_5$  &\textbf{5}  &-  &-  &\textbf{3}  &-  &-   \\ 
$u_6$  &\textbf{4}  &\textbf{2}  &-  &\textbf{4}  &-  &-   \\ 
\bottomrule
%$ f_i$    &3  &3  &1  &3  &1  &2  \\  \hline
\end{tabular}
\caption{User-Movie rating data} \label{tab:example}
\end{table}
% $\mathcal{P}^1= \{ \mathcal{P}_1^1 \{i_1,i_2,i_3\}, \mathcal{P}_2^1:\{i_2,i_3,i_5\}  \}$
 %$\mathcal{P}^2= \{ \mathcal{P}_1^2: \{i_1,i_2,i_3\}, \mathcal{P}_2^2:\{i_2,i_5,i_6\}  \}$
 %$\mathcal{P}^3= \{ \mathcal{P}_1^3: \{i_7,i_8,i_9\}, \mathcal{P}_2^3:\{i_{10},i_{11},i_{12}\}  \}$
\begin{table}[htb]
\centering
\tiny
\begin{tabular}{ccc} 
\toprule
		&$u_1$&$u_2$  \\ \hline 
$\mathcal{P}^1 $ & $\{i_1,i_2,i_3\}$ & $\{i_2,i_3,i_5\} $ \\
$\mathcal{P}^2$ & $\{i_1,i_2,i_3\}$ & $\{i_2,i_5,i_6\} $ \\
$\mathcal{P}^3$ & $\{i_7,i_8,i_9\}$ & $\{i_{10},i_{11},i_{12} \}$ \\
\bottomrule
%$ f_i$    &3  &3  &1  &3  &1  &2  \\  \hline
\end{tabular}
\caption{Top-$\size$ allocations to users.} \label{tab:paretoExamples}
\end{table}
\fi


\iffalse
When considering long-tail items, it is important to consider consumers' willingness  to explore niche or unpopular items and their propensity towards similar items. In particular, they can be characterized by their  {\it risk degree\/} and {\it focusing degree\/}, respectively.  We compute these estimates  based on historical rating information. The following example further describes these notions in the context of movie rating data. 

\begin{example}  
Table~\ref{tab:example} shows rating data for a simplified system with $6$ users, $6$ movies, and $3$ genres. $m_i^{j}$ implies that movie $m_i$ belongs to genre $j$. Note the user-item interaction matrix is sparse. 
  For this setting, we define popular movies as those that have received  three or more ratings; $\{m_1, m_2, m_4\}$ are popular and  $\{m_3, m_5, m_6\}$ are niche movies. We now profile the users according to their risk and focusing degree. E.g., $u_1$ has rated relatively popular movies belonging to the same genre (risk-averse, high focusing degree); $u_2$ has rated niches movies in the same genre (risk-loving, high focusing degree); $u_3$ has rated popular movies in two different genres (risk-averse, low focusing degree), and $u_4$ has rated niches movies in two different genres (risk-loving, low focusing degree). 
\label{ex:running}
\end{example}
\begin{table}[htb]
\centering
\tiny
\begin{tabular}{ccccccc} 
\toprule
			&$m_1^{1}$ &$m_2^{1}$   &$m_3^{2}$    &$m_4^{3}$   &$m_5^{3}$ &$m_6^{3}$  \\ \hline 
$u_1 $ &5  &4  &-  &-  &-  &-   \\
$u_2$  &-  &-  &-  &-  &5  &5   \\
$u_3$  &-  &4  &-  &5  &-  &-   \\
$u_4$  &-  &-  &3  &-  &-  &4   \\ 
$u_5$  &5  &-  &-  &3  &-  &-   \\ 
$u_6$  &4  &2  &-  &4  &-  &-   \\ 
\bottomrule
%$ f_i$    &3  &3  &1  &3  &1  &2  \\  \hline
\end{tabular}
\caption{User-Movie rating data} \label{tab:example}
\end{table}
It is essential to consider these consumer characteristics in designing recommender systems so that they promote long-tail items to the right group of users and spread demand evenly between the hit and niche items.  
\fi
\iffalse
\begin{center}
\begin{figure*}[tp]
%\scalebox{0.5}{%
\resizebox{1\textwidth}{!}{%
%\small%\addtolength{\tabcolsep}{5pt}% below sums to 8
\begin{tabularx}{1.5\textwidth}{>{\hsize=2.5\hsize}X>{\hsize=2.5\hsize}X>{\hsize=0.5\hsize}X>{\hsize=0.5\hsize}X>{\hsize=0.5\hsize}X>{\hsize=0.5\hsize}X>{\hsize=0.5\hsize}X>{\hsize=0.5\hsize}X}
    \multirow{12}{*}{\includegraphics[scale=0.3]{codeForExample/popularity-movie.png}} & \multirow{12}{*}{\includegraphics[scale=0.3]{codeForExample/scatterplot.png}} & & & & & & \\
%   & &               &       &       &       &       &       \\
    & &\multicolumn{1}{l|}{}               &$m_1^{g1}$   	&$m_2^{g1}$    	&$m_3^{g2}$    &$m_4^{g2}$      &$m_5^{g3}$    \\ \cline{3-8}%\hline
    & &\multicolumn{1}{l|}{u1}          &5  &5  &-  &-   &-  \\
    & &\multicolumn{1}{l|}{u2}    		&-  &-  &4  &4  &5  \\
    & &\multicolumn{1}{l|}{u3}   			&1  &2  &1  &-  &-   \\
    & &\multicolumn{1}{l|}{u4}     		&1  &-  &-  &-  &-  \\
    & &               &       &       &       &       &       \\
    & &               &       &       &       &       &       \\
    & &               &       &       &       &       &       \\
    & &               &       &       &       &       &	\\
    \\
\end{tabularx}}
\caption{User-Movie interaction data a) Popularity-Movie histogram b)Movie genres/clusters c) User-Movie rating data} \label{fig:example}
\end{figure*}
\end{center}
\fi



%We propose a novel approach that allows us to  promote long-tail items in a targeted manner, thereby improving the novelty of top-$\size$ sets, the overall item-space coverage across recommendations, while maintaining reasonable levels of accuracy.

%Next, we integrate these learned preferences  in a generic  top-$\size$ recommendation framework to provide customized balance between accuracy and coverage.

%sequentially make recommendations, while adjusting its parameters with regard to the set of top-$\size$ recommendations made so far. However, since  sequential parameter updates  cause  scalability issues, we propose a sampling based algorithm. This variant of our framework, called {\it Ordered Sampling-based Locally Greedy (OSLG)\/},  allows us to  correct for the popularity bias in recommendations with regard to individual user long-tail preferences. 

%ICDE submission
%Our framework differs with  prior work in the following aspects:  unlike~\cite{adomavicius2011maximizing,adomavicius2012improving,zhang2013personalize,ho2014likes},  the long-tail preference personalization in our framework is learned rather than optimized using cross-validation or parameter tuning. In other words, our personalization method is independent of the underlying base  recommendation models.  Moreover, our framework is  generic. This enables us to  plug-in several base recommenders, and evaluate their  effectiveness without requiring  extensive tuning for the accuracy and coverage trade-off. 


%\vspace{-2.8pt}
\begin{itemize}

\item  We examine various measures for estimating user long-tail novelty preference in Section~\ref{sec:lt-pref} and formulate an optimization problem  to directly learn users' preferences for long-tail  items from interaction data in Section~\ref{sec:learning-lt-pref}. %In addition, we introduce several heuristics for measuring the user preference for less common items from historical rating data.% 

\item  We integrate the user preference estimates into GANC %, a generic re-ranking framework that provides customized balance between accuracy, novelty, and coverage 
(Section~\ref{sec:RiskbasedReranking}), and  introduce {\it Ordered Sampling-based Locally Greedy (OSLG)\/}, a scalable algorithm that relies  on user long-tail preferences to correct the popularity bias (Section~\ref{sec:optimizationAlgorithm}).
%We introduce OSLG, a scalable algorithm that relies  on user long-tail preferences to  maximize item space coverage \textcolor{red}{while maintaining acceptable levels of accuracy} (Section~\ref{sec:optimizationAlgorithm}).

\item   We conduct an extensive empirical study and evaluate performance from  accuracy, novelty, and coverage perspectives (Section~\ref{sec:Experiments}).  We use five  datasets with varying density and difficulty levels. %:  Netflix, MovieTweetings, and MovieLens (100K, 1M, 10M). 
  In contrast to most related work,  our evaluation considers realistic settings that include a large number of infrequent  items and users. %This enables us to study the impact of  data density on the performance trade-offs of several  state of the art top-$\size$ recommendation algorithms. %   %,  and use the all-items ranking protocol~\cite{steck2013evaluation,vargas2014improving}, where performance is measured using all items with train data. to evaluate the performance of several  state of the art top-$\size$ recommendation algorithms 
 
\item Our empirical results confirm that the performance of re-ranking models is impacted by the underlying   base recommender and the dataset density. Our generic approach enables us to easily incorporate a suitable base recommender to devise an effective solution for both dense and sparse settings. In dense settings, we use the same base recommender as existing re-ranking approaches, and we outperform them in accuracy and coverage metrics. For sparse settings, we plug-in a more suitable base recommender, and devise an effective solution that is competitive with existing top-$\size$ recommendation methods in accuracy and novelty. 

%Directly estimating the long-tail novelty preferences allows us to customize re-ranking per user, and  devise a generic framework.   
 
\end{itemize}

Section~\ref{sec:related-work} describes related work. Section~\ref{sec:conclusion} concludes.


% RELATED WORK
%\vspace{-1mm}
%\section{Related Work}


\subsection{Sampling-based Motion Planning Method}

Plenty of modifications are proposed to enhance the performance of the RRT algorithm \cite{lavalle1998rapidly} such as the RRT* algorithm \cite{karaman2011sampling}.
The rewiring stage of the RRT* only rewires locally, which means the global optimization of the current tree is ignored. 
The RRT\# \cite{arslan2013use} proposes to find the global optimality in each rewiring stage with dynamic programming.
Dynamic programming is also used in the Fast Marching Tree (FMT*) method \cite{janson2015fast} to grow the searching tree.
% And the FMT* introduces the thought of batch sampling into the robot motion planning field.
The Informed sampling strategy \cite{gammell2014informed, gammell2018informed} is proposed to overcome the drawback of uniform sampling.
% It can accelerate the convergence speed significantly with very little computation consumption. 
% The Informed sampling strategy uses a direct sampling method to generate samples in the $L_2$-Informed set.
% An advanced version of the Informed sampling strategy is proposed in \cite{gammell2018informed}, which includes the graph pruning stage to keep a relatively constricted tree. 
% A tree with fewer vertices means that the cost reduction in finding the nearest tree vertex. 
Using the neural network to reinforce the sampling stage to enhance the sampling efficiency \cite{wang2020neural, li2021efficient, qureshi2019motion} is proved as a promising technique.

% In the human-robot coexisting environment, the planning problem become more complex than that in the static environment \cite{wang2020eb}.
% The objective of optimization needs to consider safety, efficiency, and human feelings.


% The Informed sampling strategy can constraint the whole planning procedure in a subset of the whole state space, the $L_2$-Informed set, and the Lebesgue measure of the $L_2$-Informed set decreases as the solution improves.
% \\ \textcolor{red}{*SWIRRT*: maybe not including the SWIRRT is better}




\subsection{Batch Sampling Technique}

% A batch sampling method is described in the FMT* method \cite{janson2015fast}.
The FMT* \cite{janson2015fast} introduces the thought of batch sampling into the robot motion planning field.
The FMT* samples a batch of points and constructs the searching tree according to this batch of samples.
% The asymptotic optimal of the FMT* is guaranteed when the size of the batch goes to infinity. 
The Batch Informed Trees (BIT*) \cite{gammell2015batch, gammell2020batch} method is developed based on the Informed RRT*, besides, the BIT* absorbs the thoughts in the FMT* method \cite{janson2015fast} and the Lifelong Planning A* (LPA*) algorithm \cite{koenig2004lifelong}.
The Regionally Accelerated Batch Informed Trees (RABIT*) \cite{choudhury2016regionally} aims to solve the difficult-to-sample planning problem, like the narrow passage problem.
% The RABIT* uses the Covariant Hamiltonian Optimization for Motion Planning (CHOMP) method as its local optimizer, and the local optimizer will exploit the local information.
The Fast-BIT* \cite{holston2017fast} modifies the edge queue and searches the initial solution more aggressively. 
The Greedy BIT* \cite{chen2021greedy} uses the greedy searching method to generate the initial solution faster and accelerate the convergence speed.
But these greedy-based methods often fail to assist the searching procedure without an accurate heuristic estimation method. 
The Adaptively Informed Trees (AIT*) \cite{strub2020adaptively} and the Advanced BIT* (ABIT*) \cite{strub2020advanced} proposed by Strub and Gammell are developed based on the BIT* as well. 
The AIT* calculate a relatively accurate heuristic estimation with a lazy reverse-searching tree.
The ABIT* proposes to utilize inflation and truncation to balance the exploitation and exploration in the increasingly complex Random Geometric Graph (RGG) \cite{penrose2003random}.
Though the AIT* and the ABIT* achieve significant improvements, their sampling regions are not compact enough, and the sampling efficiency will be critically low in the complex environment.


\subsection{Relevant Region Sampling Strategy}

The concept of 'relevant' is first proposed in the searching-based robot path planning method like the A* \cite{hart1968formal}. 
In the A* algorithm, the set of expanded vertices is relevant to the query, such that the A* algorithm could expand a smaller set of vertices than the Dijkstra's algorithm \cite{dijkstra1959note}.  
% And it is also not a new idea in the sampling-based planning field. 
% The Relevant Region is formally defined in \cite{arslan2013use}. 
% The Relevant Region related vertices are the vertices of which the sum of the optimal cost-to-come and the heuristic is less than the cost of the current optimal solution.
% Since the Relevant Region is the most promising region that could help to improve the solution, so a straightforward modification is to reduce the chance of sampling outside the  Relevant Region.
The concept of the Relevant Region is formally introduced in \cite{arslan2013use}, whose sum of the optimal cost-to-come and cost-to-go heuristic is less than the cost of the current optimal solution. 
Since the Relevant Region is the most promising area for improving the solution, a straightforward modification would be reducing the likelihood of sampling outside of it.
Three different metrics are used to achieve this in \cite{arslan2015dynamic}, the modified versions achieve better performance in the convergence speed than the RRT\# method.
The methods described in \cite{arslan2013use} and \cite{arslan2015dynamic} use the rejection method for sampling, which is not efficient since the Relevant Region is a small subset of the whole state space in most scenarios.
The direct sampling method is illustrated to overcome this drawback, and the details are described in \cite{joshi2020relevant}.
However, they all use the cumulative cost along the direct connection between the current state and the goal state as the cost-to-go. 
This approach results in inaccurate estimated cost-to-go in most scenarios.
% Their ordered priority queues are also far from the ground truth.

% The optimal cost-to-come value can be defined as the vertex which has the lowest optimal cost-to-come value in the destination region. 


\subsection{Bi-directional Searching Method}

The RRT and RRT* methods may not always discover a solution within the allotted time, particularly when dealing with narrow passages
The RRT-Connect \cite{kuffner2000rrt} is proposed to find the initial solution faster. 
% It grows two trees from the source point and the goal region simultaneously.
% It is proved that the RRT-Connect can achieve better performance than the RRT.
However, the approach described in \cite{kuffner2000rrt} is not asymptotically optimal. Therefore, its successor, RRT-Connect, is also not asymptotically optimal. 
To overcome this, an enhanced version of the bidirectional searching RRT is introduced in \cite{klemm2015rrt} to guarantee the asymptotical optimality.
To take advantage of the benefits of bi-directional search, the kinematic constraints are taken into consideration in the bi-directional search method to generate executable trajectories efficiently \cite{wang2021kinematic}.
% The method described in \cite{klemm2015rrt} is an asymptotically optimal single-query version of the RRT-Connect, called the RRT*-Connect. 
% The RRT*-Connect provides asymptotically optimal guarantee like the RRT*, and its efficiency and robustness are proofed in real-world experiments.
% In addition, the bi-directional searching method can be used to combine with the kinematic constraints , which is essential in generating executable trajectory.

% In the RRT-Connect, one tree is extended in each iteration and tries to connect itself to the other tree; this manner will attempt to grow the trees towards each other.

One drawback of the Informed RRT* \cite{gammell2014informed} \cite{gammell2018informed} is that it uses the RRT* to search the whole state space before finding the initial solution.
Therefore, the Informed RRT* often fails to find the solution in the required period, same as the RRT*.
By combining the advantages of both the Informed and the RRT*-Connect, the Informed RRT*-Connect \cite{2020Informed} proposes to use the RRT*-Connect to generate the initial solution and use the Informed sampling strategy to constrain the sampling region after the initial solution is found.
% It combines the advantages of both the Informed RRT* and the RRT*-Connect.
% The Informed RRT*-Connect can achieve a much higher success rate in its simulations than the Informed RRT*.
Besides, the AIT* \cite{strub2020adaptively} can also be viewed as a bi-directional searching method.





% SYSTEM MODE
\vspace{-1mm}
\section{System Model} \label{sec:model}
%\vspace{-1mm}

\begin{figure}[t!]
	\centering
	\includegraphics[width=0.45 \textwidth]{images/model.pdf}   
	\caption{\small vRAN over integrated fronthaul/midhaul (xHaul). It has many degrees of design freedom by possibly hosting BS functions at the CU or DUs.}
	\label{fig:vran}
	\vspace{-3mm}
\end{figure}

\textbf{Background.} In C-RAN, all BS functions are centralized at the Base Band Unit (BBU) except RF layers at the RU.  \thirdrev{In vRANs}, the BBU is decoupled into the CU and DU \cite{split_3gpp_rel16}. Hence, functions of a BS can be deployed at the CU, DU and RU. Fig \ref{fig:vran} illustrates that a CU is typically executed at a bigger and more centralized \thirdrev{server} (e.g., edge server), while a DU is at a smaller server (e.g., far-edge server) and located near (or co-located) with an RU.

Our model refers to the standardization of 3GPP \cite{split_3gpp,split_3gpp_rel16} and seminal white paper \cite{smallcell}, where each split has a different performance gain \cite{vran_murti2,function_split_survey}. \thirdrev{Although 3GPP has defined eight options for the splits, several are still hardly implemented. Therefore, we consider four splits that have been experimentally validated in a prototype \cite{costdu_nikaein,adaptive_alba}. }
 \textbf{Split 0}: All functions are at the DU, except the RF layers at the RU. It is a typical D-RAN setup. \textbf{Split 1} (PDCP-RLC): RRC, PDCP, and upper layers are hosted at the CU, while RLC, MAC, and PHY are at the DU. \secrev{This split enables a separate user plane \thirdrev{and control plane with} centralized RRC.} \textbf{Split 2} (MAC-PHY): MAC and upper layers are at the CU, while PHY is at the DU. It allows improvement for CoMP by centralized HARQ. \textbf{Split 3} (PHY-RF): All functions are at the CU, except RF layers. It is a fully centralized version of vRANs. It gains power-saving and improved joint reception CoMP with uplink PHY level combining. Going from \secrev{Split 0 to Split 3}, more functions are hosted at the CU. In addition to increasing network performance, a higher centralization level can lead to more computing cost savings \cite{vran_murti2}. However, centralizing more functions increases the data load to be transferred to the CU, going from $\lambda$ in \secrev{Split 1 to 2.5 Gbps in Split 3} for each BS, and has a stricter delay requirement. Table \ref{table:splits} summarizes vRAN split options and their requirements\footnote{\secrev{The requirements are tailored from \cite{smallcell,vranmec_andres} by following settings: 1 user per TTI, 20MHz channel bandwidth, 1 carrier component, UE IP MTU 1500 bytes, $2 \times 2$ MIMO.} }.  

%
%\begin{table}[t!] \centering
%	%\ra{1.3}
%	\begin{small}
%		\begin{tabular}{@{}lcccc@{}}\toprule
%			\textbf{}& \textbf{Traditional RAN} & \textbf{Cloud RAN}  & \textbf{Open RAN} & \textbf{Open vRAN}
%			\\ \midrule
%			\textbf{RU} &      Locked   & Locked  & Open        & Open
%			\\ \hdashline
%			{\textbf{Interface}} &  Locked & Locked & Open & Open
%			\\ \hdashline
%			{\textbf{CU/DU SW}} &  Locked   & Locked, Virtualized & Open & Open, Virtualized
%			\\ \hdashline
%			{\textbf{CU/DU HW}} &   Locked   & Open & Open & Open \\ %\midrule
%			\bottomrule
%		\end{tabular}
%	\end{small}
%	\caption{\small\textbf{RAN transformation}. \textit{Locked} means that is a solely property of a single vendor. \textit{Open} means that it allows interoperable to flexibly work with different vendors. }
%	\label{table:ran_transform}
%	\vspace{-5mm}
%\end{table}

\begin{table}[t] \centering
	%\ra{1.3}
	\begin{small}
		\begin{tabular}{@{}lll@{}}\toprule
			\textbf{}& \textbf{Flow (Mbps)} & \textbf{Delay Req. (ms)}  
			\\ \midrule
			{Split 0 \secrev{(S0)} } &      $\lambda$   & $30$          
			\\ \hdashline
			{Split 1 \secrev{(S1)} } &  {$\lambda$} & $30$ 
			\\ \hdashline
			{Split 2 \secrev{(S2)} } &  {$1.02\lambda+1.5$}   & $2$
			\\ \hdashline
			{Split 3 \secrev{(S3)} } &   {$2500$}   & $0.25$ \\ %\midrule
			\bottomrule
		\end{tabular}
	\end{small}
	\caption{\small Data and delay requirements of vRAN split when the traffic load is $\lambda$ Mbps \cite{smallcell, vranmec_andres}.}
	\label{table:splits}
	\vspace{-3mm}
\end{table}


\textbf{RAN}. We model a vRAN architecture with a graph $G = (\mathcal{I}, \mathcal{E})$ where $\mathcal{I}$ has a subsets $\mathcal{N}$ of the $N=|\mathcal{N}|$ DUs, \secrev{$\mathcal{L}$ of the $L=|\mathcal{L}|$ routers} and a CU (index $0$). Each node is connected through a link of $(i,j)$ with a set $\mathcal{E}$ of links and has capacity $c_{ij}$ (Mbps) each. The DU-$n$ is connected to $\{0\}$ with a single path (e.g., shortest path) $p_{n0}$; hence, we define $r_{p_{n0}}$ as the amount of data flow (Mbps) to be transferred and routed through a path $p_{n0} \! := \!  \{(n,i_1), ..., (i_k,0) \! : \! (i,j) \! \in \! \mathcal{E} \}$. The BS functions are deployed in servers using virtual machines (VMs)\footnote{\secrev{Each BS
	function can operate as a virtual network function (VNF), and the VNFs can be executed on top of a single VM or multiple VMs}}. Each server has a processing capacity, i.e., $H_n$ for DU-$n$ and $H_0$ for CU. Naturally, a central server has a higher computing performance and capacity, \secrev{hence} $H_0 \! \geq \! H_n$. \secrev{We define} $\rho_{o}^c $ and $ \rho_{o}^d$ as the incurred computational load (cycle/Mb/s) in results of deploying the split configuration $o \! \in \! \{0,1,2,3\}$ at each CU and DU, respectively. %\secrev{The processing load at the CU is generally lower than at the DU while serving the same traffic load \cite{costdu_nikaein}.}


\textbf{Demand $\&$ Cost}. We focus on the uplink transmission where $\lambda_{n} \geq 0$ (Mbps) is the aggregate data flow of DU-$n$ to serve the users traffic;
hence, there are $N$ different flows in the network. We denote $ \bm{\alpha} = (\alpha_n, n \in \mathcal{N})$ and $\bm{\beta} \!=\! (\beta_n, n \in \mathcal{N})$ as the VM instantiation cost (monetary units) and the computing cost (monetary units/cycle) at the DUs, respectively, while $\alpha_0$ and $\beta_0$ are the respective cost for the CU. We also have a routing cost $\zeta_{p_{n0}}$ (monetary units/Mbps) for each path $p_{n0}$. This cost arises from the network links being leased from third parties or maintaining the links. 

\textbf{Problem Statement.} We have four choices of the \secrev{splits} for each BS in vRANs. What is the best-deployed split for each BS that minimizes the total network cost? The decision leads to interesting problems. Each \secrev{split} generates a different DU-CU data flow and has a distinct delay requirement. Executing more functions at the CU is more efficient in computing cost; however, it produces a higher load for xHaul links. \secrev{The BSs share the same capacitated servers and network links, where each split decision is interdependent. Moreover, the behaviour of the vRAN system (e.g., resources, performance) is complex and highly non-trivial, which makes complete assumptions of the model can be unfeasible or inaccurate. The goal is to design a framework to solve this problem by taking minimal assumptions about the model of the system.}







% DESIGN
\vspace{-2mm}
\section{Formalization of vRAN Split Problem} \label{sec:problem}
\vspace{-1mm}
The BS functions can be deployed at the DUs or CU \secrev{depending on} the splits, as seen in Table \ref{table:splits}. \secrev{Each split} must respect to the \emph{chain of functions} $f_0 \!\rightarrow\! f_1 \!\rightarrow\! f_2 \!\rightarrow\! f_3$\footnote{\secrev{$f_0$ is a function that encapsulates RF layers. Then, $f_1, f_2$ and $f_3$ are the functions for Layer 1 (PHY), Layer 2 (MAC, RLC) and Layer 3 (PDCP, RRC and the upper layers), respectively. }}. Thus, we define $x_{on} \in \{0,1\}$ as the decision for deploying split $o \in \{0,1,2,3\}$ at DU-$n$. For instance, $x_{0n} = 1$ is for deploying $f_0, f_1, f_2, f_3$ (Split 0); $x_{1n} = 1$ for $f_0, f_1, f_2$ (Split 1); $x_{2n}= 1$ for $f_0, f_1$ (Split 2); or $x_{3n}= 1$ for $f_0$ (Split 3) at DU-$n$. 
%
%
We only deploy a single split configuration for each BS. Therefore, \secrev{a} set of eligible \secrev{splits} is:
%
%\vspace{-1mm}
\begin{align} \label{eq:setx}
\mathcal{X} =  &\Bigl\{  \bm{x}_n  \in \{0,1\} \Big| \sum_{o=0}^3 x_{on}=1 , 
\ \ \forall n \in \mathcal{N}  \Bigr\} , 
\end{align}
%
where $\bm{x}_n = ( x_{on}, \forall o  )$ and $\bm{x} = (\bm{x}_n, \forall n)$. The BS functions,  $f_1, f_2$ and $f_3$, are deployed \secrev{using} VMs at each server. We have computational processing at the CU and DU-$n$ that must respect \secrev{to} its capacity as:
%
%\vspace{-1mm}
\begin{align} \label{eq:computing1}
\sum_{n \in \mathcal{N}} \lambda_{n} \sum_{o =0}^3    x_{on} \rho^{c}_o \leq \ H_0, 
%
\end{align}
%\vspace{-1mm}
%
%
%\vspace{-1mm}
\begin{align} \label{eq:computing2} \lambda_{n} \sum_{o =0}^3   x_{on} \rho^{d}_o \leq H_n, \ \forall n \in \mathcal{N}. 
%
\end{align}
%
%\textit{We consider the realistic scenario of CU processing where the parameters of the servers change over time in unknown fashion.} Each CU is shared to many BS functions (resource pooling); so, it has an \textit{average capacity} affecting the processing performance, i.e., when the load exceed average capacity ($\bar{H}_0$), the processing delay will increase and lead to the system non-responsive. This condition is generated by random process $\{ H_0^t\}_{t=1}^{\infty}$ with $ \lim_{T \to \infty} \frac{1}{T} \sum_{t=1}^{T}H_0^t = \bar{H}_0, \forall t.$ However, our decision only has information on the instantaneous value in each time slot. In here, we only need this pertubation is bounded at each time slot ($ H_0^t \leq H^{\text{max}}_0$) and the average converges to some finite value ($\bar{H}_0$)

\textbf{Data Flow $\&$ Delay.}  Let define $r_{p_{n0}}$ (Mbps) as the amount of data flow (Mbps) to be transferred through a path $p_{n0}$. Hence, the flow must respect \secrev{capacity of each link}:
%
%\vspace{-1mm}
\begin{align} \label{eq:route1}
\sum_{n \in \mathcal{N}} r_{p_{n0}} I^{ij}_{p_{n0}} \leq c_{ij}, \ \ \forall (i,j) \in \mathcal{E},
\end{align}
%
where $I^{ij}_{p_{n0}} \in \{ 0,1 \}$ indicating whether the link $(i,j)$ is used by path $p_{n0}$. Assuming a single path (e.g., shortest path), the amount of data flow depending on each split configuration is \cite{fluidran_andres}:
%
%\vspace{-1mm}
\begin{align} \label{eq:route2}
r_{p_{n0}} \!=\! \lambda_{n} (x_{0n} + x_{1n}) + x_{2n} (1.02 \lambda_{n} + 1.5) + 2500 x_{3n}.
\end{align}
%
We let $d_{p_{n0}}$ denote the incurred delay for routing through path $p_{n0}$ from DU-$n$ to the CU. Each split has to satisfy the respective delay requirement (Table \ref{table:splits}):
%
%\vspace{-1mm}
\begin{equation} \label{eq:delay}
x_{on} d_{p_{n0}} \leq d_o^{\text{max}}, \ \ \forall o, \forall n \in \mathcal{N}.
\end{equation}
%
\subsection{Objective Function}
We aim to minimize the total network cost consisting of the computational \secrev{costs at the DUs and CU} and the routing cost\footnote{In this case, we follow the linear objective cost function similar to the previous studies \cite{fluidran_andres,vranmec_andres}. However, our solution approach does not restrict only to the linear objective. Our approach \secrev{relies on} the \secrev{scalar} reward and penalty as feedback; hence, it can also be tailored to a non-linear objective.}. The needs of computing cost \secrev{for each BS-$n$} at DU-$n$ is:
%
\begin{align} \label{eq:cost-du}
V_n(\bm{x}_n) = \alpha_n + \beta_n \lambda_{n} \sum_{o=0}^{3}\rho_{o}^{d} x_{on}.
\end{align} 
%
We also have \secrev{the required} computing cost \secrev{of BS-$n$} at the CU:
%
\begin{align} \label{eq:cost-cu}
\secrev{V_{n0}(\bm{x}_n)} = \sum_{o=0}^3 x_{on} (\alpha_0 + \lambda_{n} \beta_0 \rho_{o}^{c} ).
\end{align} 
\secrev{The first terms in \eqref{eq:cost-du} and $\eqref{eq:cost-cu}$ represent the required instantiating cost at each DU and CU for BS-$n$. The last terms in \eqref{eq:cost-du} and \eqref{eq:cost-cu} are the required data processing cost by each DU and CU to serve BS-$n$ load.}
%
Next, we have the cost to route the data flow from DU-$n$ to the CU:
%
\begin{align} \label{eq:cost-route}
U_{n0}(\bm{x}_n) =  \zeta_{p_{n0}} r_n (\bm{x}).
\end{align} 
%
Finally, we have the total vRAN cost as:
%
\begin{align} \label{eq: total-cost}
J(\bm{x}) &= \sum_{n \in \mathcal{N}} \Big(  V_n(\bm{x}_n) + U_{n0}(\bm{x}_n) +\secrev{ V_{n0}(\bm{x}_n}) \Big), \end{align}
which leads to the following problem:
%
\begin{align}
\mathbb{P}: \,\,\,\, & \underset{\bm{x} \in \mathcal{X}}{\text{minimize}} \   J(\bm{x}), \ \ \ \notag  \text{s.t} \ \  
 (\ref{eq:computing1}) - (\ref{eq:delay}). \notag
\end{align}
%
$\mathbb{P}$ is a combinatorial problem to decide the function placement $\bm{x}$ for all the BSs and serve the traffic load $\bm{\lambda}$ with DU-CU path $p_{n0}$ for each BS-$n$ in the network graph $G = (\mathcal{I}, \mathcal{E})$. Next, we discuss the complexity of $\mathbb{P}$.

\subsection{Complexity Analysis}

The complexity of $\mathbb{P}$ can be identified from the polynomial reduction of \textit{multiple-choice multidimensional knapsack problem (MMKP).} 

\textbf{MMKP.} Let suppose there are $N$ items with values ${v_1, v_2, ..., v_N}$. We also have $r_1, r_2, ..., r_N$ correspond to the required resources to pick the items. In the 0-1 knapsack problem (KP), the aim is to pick the items $x_i \in \{0, 1\}, \secrev{\forall i}$ that maximize the total value $\sum_{i=1}^N  x_i v_i$, subject to constraint \secrev{$\sum_{i=1}^N x_i r_i \leq R$}. This is a well-known NP hard problem and there is a pseudo-polynomial algorithm using a dynamic programming concept that has complexity $\mathcal{O}(NR)$ \cite{mmkp_convexhull}.  MKKP is a variant of 0-1 KP where there are $M$ groups of items, e.g., group $i$ has $l_i$ items. Each item has a specific value $v_{ij}$ corresponds to $j$-th item of $i$-th group and needs $K$ resources. Hence, each item in a group has a resource vector $\bm{r}_{ij} = (r_{ij1}, ..., r_{ijK} )$ and $\bm{R} = (R_1, ..., R_K)$ is the resource bound of the knapsack. The aim is to exactly pick one item from each group, e.g., $ \sum_{j=1}^{l_i} x_{ij} = 1, x_{ij} \in \{0,1\}$ that maximizes the total value: $\sum_{i=1}^{M} \sum_{j=1}^{l_i} x_{ij} v_{ij}$, subject to the resource constraint: $\sum_{i=1}^{M} \sum_{j=1}^{l_i} x_{ij} r_{ijk} \leq R_k, k = 1,...,K$. 

Finding an exact solution for MMKP is also NP-hard \cite{mmkp_convexhull}. It is also worth noting that the search space for solution in MMKP is smaller than other KP variants; hence, exact solution is not implementable in many practical problems as there is more limitation of picking items from a group in MMKP instance \cite{mmkp_convexhull}. Next, We prove that $\mathbb{P}$ is harder than MMKP. 

\noindent
\begin{theorem} \label{theo:mmkp}
	\textit{MKKP can be reduced to $\mathbb{P}$ in polynomial time, e.g., MMKP $\leq_P \mathbb{P}$ }.
\end{theorem}  

\noindent
\textit{Proof.} Let suppose we have unlimited link capacity, no routing cost and no delay requirement. Hence, all paths of \secrev{the} DU-CU pair are eligible and \eqref{eq:route1}-\eqref{eq:delay} are always satisfied. This problem then can be mapped to MMKP by setting: 1) $M$ groups to $N$ BSs, 2) each $i$-th group with $l_i$ items to each BS-$n$ with $|o|=4 $ of split options, 3) $j$-th item of $i$-th group to the split $o_n$ of BS-$n$, 4) $r_{ij}$ to the \secrev{incurred} computing loads, e.g., $\lambda_{n}\rho_{i}^c$ and $\lambda_{n}\rho_{i}^d$, and 5) the knapsack constraints to computing constraints $H_n$ and $H_0$. The value $v_{ij}$ of item-$j$ in group-$i$ also can be mapped with the costs (e.g., computing and routing) of deploying split-$o$ of BS-$n$, where the MMKP is a maximization problem and $\mathbb{P}$ is a minimization problem.  We can see the reduction is of polynomial time: we select the functional split for every BS correspond to that we activate an item that we pick to a knapsack in each group. Therefore, we can conclude that if we can solve $\mathbb{P}$ in polynomial time we also can solve any MMKP problem. 

%
%\begin{table}[t]
%%\begin{tabular}{cp{5cm}} \hline
%\begin{small}
%\begin{tabular}{cp{5cm}} \hline
%\textbf{Notation} & \textbf{Definition} \\ \hline
%$\mathcal{N}, N, n$ & Set of DUs, number of DUs, DU index \\ 
%$\mathcal{M}, M, m$ & Set of CUs, number of CUs, CU index \\ 
%$\mathcal{P}_{nm}$ & Set of paths between DU $n$ and CU $m$\\ 
%$\mathcal{P}_{m}$ & Set of paths between all DUs and CU $m$\\ 
%$c_{ij}$ & Bandwidth in Mbps of the link between nodes $i$ and $j$ \\ 
%$d_{p_k}$ & End-to-end delay of the path $p_k$ in seconds \\ 
%$\lambda_{n}$ & Traffic flow from DU-$n$ in Mbps \\ 
%$H_{n}, H_{m}$ & Processing capacity (cycles/sec) of DU-$n$ and CU-$m$\\ 
%$\rho_1, \rho_2, \rho_3$ & Processing load (cycles) per Mbps of $f_1$, $f_2$, $f_3$\\ 
%$a_{m}, b_{m}$ &  Cost of VM instantiation and computing at CU $m$\\ 
%$\alpha_{n}, \beta_{n}$ &  Cost of instantiation and computing at DU $n$\\ 
%$\zeta_k$ & Routing cost of path $k$ (monetary units per Mbps) \\
%$c_d$ & Routing cost per kilometer per Mbps \\ \hline
%\end{tabular}
%\end{small}
%\centering
%
%\bigskip
%\caption{Notation table}
%\label{tab:notation_table}
%\end{table}

% SOLUTION ALGORITHM
\vspace{-2mm}
\section{Constrained Deep Reinforcement based Functional Split Optimization Framework} \label{sec:solution}
%\vspace{-1mm}

\secrev{We leverage a constrained deep RL paradigm to solve our vRAN splitting problem by treating the vRAN system as a black-box environment, which makes minimal assumptions about the underlying system. Consequently, our RL agent does not need to know the information about the formulation in \eqref{eq:setx}-\eqref{eq: total-cost} to decide the splits. Our agent relies on the scalar reward and penalization returned from the environment to assess the quality of the solutions.}
%
%Our vRAN splitting problem also comprises combinatorially large discrete action spaces as the split configuration should simultaneously be deployed to multiple BSs. Therefore, we propose a CDRS framework to follow NCO with a deep RL paradigm to handle this challenge \cite{neural_bello,solozabal_constrained,vnf_drl_solozabal}
%
%
\secrev{At each episode, our agent observes a \textit{state} of incoming a sequence of all BS functions drawn from the \textit{environment} of vRANs, takes an \textit{action} to decide the splits for all the BSs, and expects to receive feedback signals of the \textit{reward} (total network cost) and \textit{penalization} (for violating the constraints). Our state comprises of a sequence information of BS functions: $\mathcal{F} = \{ \mathcal{F}_n \}_{n=1}^{N}$,  where $\mathcal{F}_n$ is a set of BS-$n$ functions. Given the input state, our agent assigns $\mathcal{O} \!\!=\!\! \{o_n \! \in \! \{0,1,2,3\}, \forall n \in \mathcal{N} \}$ as a set of selected splits for all the BSs, which decides the placement of BS functions at the CU or DUs. Our objective is to minimize the total network cost while enforcing the constraint requirements. Given the selected splits, our agent expects to receive scalar values from the environment consisting of: \textit{i)} $J(\mathcal{O}|\mathcal{F})$, the total induced cost and \textit{ii)} $\xi(\mathcal{O}|\mathcal{F})$, the weighted sum of penalization. Further, we consider a particular RL algorithm using one-step constrained policy optimization and neural network architecture, where the interactions are narrowed to a single time step at every episode, and our agent learns iteratively over episodes.}

%CDRS follows the deep RL paradigm based on the constrained policy optimization to solve \secrev{the split problem in vRANs through} a Policy Gradient \cite{rl_sutton} with Lagrangian relaxation method \cite{Bertsekas}. In addition to the reward (cost) signal, we also give penalization for every constraint violation that guides the policy to constraint satisfaction. In this sense, the penalty coefficient (Lagrangian multiplier) setting is a multi-objective problem where there is a different optimum solution for each configuration. To this end, we follow two penalty coefficient updates: 1) CDRS-Fixed that uses a fixed penalty coefficient (reward shaping) \cite{vnf_drl_solozabal, cmpd_solo} and 2) CDRS-Ada that applies an adaptive penalty coefficient (updated in the ascent direction)  \cite{pdo_risk}.


\secrev{Our goal is to design a stochastic policy $\pi_\theta(\mathcal{O}|\mathcal{F})$ parameterized by a neural network with weights $\theta$ to predict the splits for all the BSs to minimize the total cost while satisfying constraint requirements. However, we have the $N$ BSs that need to deploy the splits together, where each has four possible split options. Each split decision is also interdependent as the BSs share the same network links and computing servers. Consequently, our problem has a combinatorially large discrete action space with a total of $4^N$ possible actions. Such a curse dimensionality in high dimensional spaces can be avoided by modelling complicated joint probability distributions using the chain rule decomposition. Therefore, we design our policy based on a chain rule by factorizing the output probability, parameterized by a neural network with weights $\theta$ as:
	%
	\begin{align}
		\pi_\theta(\mathcal{O}| \mathcal{F}) = \prod_{n=1}^{N} \pi_\theta(o_n| o_{(<n)}, \mathcal{F}_n). 
	\end{align}
	%
	%
	This policy strategy assigns a higher probability to the splits for having a lower cost and vice versa for every BS, which also can be represented by individual softmax modules (e.g., at the output layer). Motivated by \cite{seq2seq, attention_bahdanau} that uses neural networks \thirdrev{to estimate} the same factorization \thirdrev{of our stochastic policy} for machine translation, we design our policy network using an encoder-decoder sequence-to-sequence model based on LSTM networks. Our policy network architecture, which also utilizes an attention mechanism, captures the dependency and correlation between split decisions. This architecture allows our policy to read input information from all BS functions, then maps them into split selections for all the BSs.
	%
	%We provide the summarize and illustration of CDRS training operation in Algorithm 1 and Fig xxx, respectively. 
	In the training, we use a batch of $B$ i.i.d samples on the stochastic policy to select the splits and generate several pretraining models. In the test, we perform an inference through a search strategy by greedy decoding or temperature sampling.%The detail of CDRS operation is discussed as follows. 
	} 

%
%In supervised learning, the performance of the model depends on the quality of the supervised labels, where finding high-quality labelled data (e.g., optimal label) is expensive and may not be possible. Therefore, we follow NCO framework with deep RL paradigm \cite{neural_bello,solozabal_constrained,vnf_drl_solozabal}. We define \textit{environment} as a vRAN system consisting of CU, DUs and network links. The \textit{states} is defined as a sequence of all BS functions. Our RL \textit{agent} generates \textit{action} that corresponds to a set of decisions to choose the functional split configuration for every BS. This action decides which functions are deployed at the DU and CU for every BS. The environment will evaluate for every action taken and give a feedback signal from the environment. This feedback consists of \textit{reward} (the total network cost) and \textit{penalization} (constraint violation). As opposed to supervised learning, RL can deliver an appropriate paradigm for training and updating the neural network parameters to improve the performance of the model for the functional split problem. It simply relies on the reward and penalty feedbacks (interaction with the environment) instead of the high-quality labelled data.  Since our problem is combinatorial that has highly dimensional action space, we leverage model-free policy-based RL that optimizes the weight parameter $\theta$ that infers a policy strategy to deploy the split configuration. It is worth noting that we have a constrained environment; hence, we also consider a constraint relaxation technique in the cost function of our policy-based method to deal with constraint dissatisfaction. 
%%Fig xxx depics the general RL scheme for our problem.
%
%Our approach follows the deep RL paradigm based on the constrained policy optimization and neural network architecture to solve the functional split problem in the vRAN. We utilize Policy Gradient \cite{rl_sutton} with Lagrangian relaxation method \cite{Bertsekas}. In addition to the reward (cost) signal, we also give penalization for every constraint violation that guides the policy to constraint satisfaction. In this sense, the penalty coefficient (Lagrangian multiplier) setting is a multi-objective problem where there is a different optimum solution for each configuration. To this end, we follow two penalty coefficient updates: 1) CDRS-Fixed that uses a fixed penalty coefficient (reward shaping) \cite{vnf_drl_solozabal, cmpd_solo} and 2) CDRS-Ada that applies an adaptive penalty coefficient (updated in the ascent direction)  \cite{pdo_risk}. A neural network architecture formed by an encoder-decoder sequence-to-sequence model \cite{seq2seq,attention_bahdanau} based on stacked LSTM is also utilized that approximate the stochastic policy over the solution. 
%
%Our agent receives input of set of BS functions $\mathcal{F} = \{ \mathcal{F}_n \}_{n=1}^{N} \!\!$,  where $\mathcal{F}_n \!=\! \{f_0,f_1,f_2,f_3 \}$ is a set of functions for BS-$n$. In the output, we expect to receive $\mathcal{O} \!\!=\!\! \{o_n \}_{n=1}^{N}$ as a set of selected configuration for all BSs. It addresses the split configuration of BS-$n$ with $o_n \! \in \! \{0,1,2,3\}$. 
%%In relation to Eq. \eqref{eq:setx}, $o_n$ activates the respected split configuration variables for BS-$n$ following indicator function: $x_{0n} \!=\! \mathbbm{1}_{(o_n \!= 0)}, x_{1n} \!=\! \mathbbm{1}_{(o_n \!= 1)}, x_{2n} \!=\! \mathbbm{1}_{(o_n \!= 2)},$ and $ x_{3n} \!=\! \mathbbm{1}_{(o_n \!= 3)} $.  
%We also use the neural network with weight parameter $\theta$ that infers a policy strategy $\pi_\theta(\mathcal{O}| \mathcal{F}, \theta)$ to deploy the split configuration. 
%
%
%In practice, our approach does not have to know the defined problem in Section \ref{sec:problem}. Our agent interacts with the environment expecting to receive a reward (network cost) and penalization (constraints violation); then learn from this interaction to find the optimal solution. At the test time, we perform an inference process through search strategies by a greedy decoding or temperature sampling method. 


  
\vspace{-2mm}
\subsection{\secrev{Policy} Network Architecture}
%\vspace{-1mm}
%Our system is bounded under computational and link capacity, where each BS has distinct network parameters. Hence, the BS input sequence (to which BS needs to decide first) affects the solution. To this end, 
Our policy network infers a strategy to deploy the splits for all the BSs, given a sequence information of BS functions as an input $\mathcal{F} = \{\mathcal{F}_1, ...., \mathcal{F}_N \}$. It is constructed from an encoder decoder sequence-to-sequence model with an attention mechanism based on LSTM networks \cite{seq2seq,attention_bahdanau}.
\secrev{We also consider a batch training by drawing} a batch of $B$ i.i.d samples with different sequence order \secrev{to encourage the exploration further}. 

\secrev{\textbf{LSTM structure.} We leverage LSTM networks, a particular RNN architecture \cite{lstm},  to construct  our sequence-to-sequence model that maps the input BS functions into split decisions for all the BSs.
	An LSTM cell has three main structures comprising of: \textit{(i)} a forget gate that receives the cell state input and learns how long should memorize or forget from the past; \textit{(ii)} an input gate that aggregates the current input and the output of past steps, then feeds them to the activation function; and \textit{(iii)} an output gate that provides the LSTM output from the combination of current cell state and the output of input gate. The relationship of these blocks can be expressed as:
	\begin{align}
		&\bm{\hat{f}}_n = \sigma \big( W_f \big[\bm{h}_{n-1}^T, \bm{s}_{n}^T  \big]^T + \bm{b}_f \big), \\
		%
		&\bm{\hat{r}}_n = \sigma \big( W_r \big[\bm{h}_{n-1}^T, \bm{s}_{n}^T   \big]^T + \bm{b}_r \big), \\
		%
		&\bm{\tilde{c}}_n = \tanh \big( W_c \big[\bm{h}_{n-1}^T, \bm{s}_{n}^T   \big]^T + \bm{b}_c  \big), \\
		%
		&\bm{\hat{c}}_n = \bm{\hat{f}}_n *  \bm{\hat{c}}_{n-1} + \bm{\hat{r}}_n  * \bm{\tilde{c}}_n, \\
		%
		%
		&\bm{\hat{o}}_n = \sigma \big( W_o \big[\bm{h}_{n-1}^T, \bm{s}_{n}^T   \big]^T + \bm{b}_o  \big), \\
		& \bm{h}_n = \bm{\hat{o}}_n * \tanh(\bm{\hat{c}}_n),
	\end{align}
	where function $\sigma(x) \triangleq \frac{1}{1 + \exp(-x)} $ is the sigmoid function and symbol $*$ is element-wise multiplication. The weight and bias matrices for the respective forget, input and output gates of the LSTM cell are represented by $W_f, W_r, W_c, W_o$  and $\bm{b}_f, \bm{b}_r, \bm{b}_c, \bm{b}_o$.
	%
	%
	Multiple LSTM layers can be further stacked one on top of another (a stacked LSTM) to create a deeper model, which may obtain more accurate prediction. Each LSTM cell reads an input of embedding vector representation $\bm{s}_n \in [ -1,1]^E$ translated from each input $\mathcal{F}_n$, where $E$ is the embedding size.
	The structure of an LSTM cell is illustrated in Fig \ref{fig:lstm} and utilized to construct our sequence-to-sequence model.}
	%
	\begin{figure}[t!] 
		\centering
		\includegraphics[width=0.45 \textwidth]{images/lstm.pdf}   
		\caption{\small \secrev{A generic architecture of an LSTM cell.}  } 
		\label{fig:lstm}
		\vspace{-3mm}
	\end{figure}
	%
%Fig. \ref{fig:neural} illustrates the overall architecture of our policy network.
%
%Additionally, \secrev{the attention mechanism in our policy network allows to capture the dependency and correlation between each split decision, such as the shared network link and computing resources and the routing and computational weighting costs.}
%The architecture is depicted in Fig. \ref{fig:nn}. \fm{need more rephrasing}

\secrev{\textbf{Policy Network.}} Our policy network is built from an encoder-decoder sequence-to-sequence model based on LSTM networks. \secrev{One main drawback of vanilla sequence model is generally unable to learn accurately long sequence. Therefore, the vanilla model may not be able learn our problem with large number of BSs.
An attention mechanism comes to address this issue as it considers all the hidden state from all input sequences.} The encoder read the entire input sequence to a fixed-length vector. The decoder decides \secrev{the deployed split of each BS} at each step from an output function based on its own previous state combined with an attention over the encoder hidden states \cite{attention_bahdanau}. The decoder network hidden state is defined with a function: $\bm{h}_t = f(\bm{h}_{t-1}, \bm{\bar{h}}_{t-1}, \bm{c}_t)$, \secrev{where $\bm{c}_t$ and $\bar{\bm{h}}_{t}$ are the context vector and the source hidden state at time step $t$}. 
Our model derives the context vector $\bm{c}_t$ that captures relevant source information that helps to predict the splits. The main idea is to use \secrev{an attention mechanism}, where the context vector $\bm{c}_t$ takes consideration of all the hidden states of the encoder and the alignment vector $\bm{a}_{t}$: 
\begin{align}
	\bm{c}_t = \sum_{k \in \mathcal{N}} \bm{a}_{tk} \bm{\bar{h}}_k.
\end{align}
%
% 
Note that the alignment vector has an equal size to the number of steps in the source side, which can be calculated by comparing the current target hidden state of decoder $\bm{h}_t$ with each source hidden state $\bm{\bar{h}}_k$ as:
\begin{align} \label{eq:softmax}
	 \bm{a}_{tk} = \frac{\exp(\text{score}(\bm{h}_t,\bm{\bar{h}}_k))}{\sum_{k'=1}^{N} \exp(\text{score}(\bm{h}_t,\bm{\bar{h}}_k')))}
\end{align}
This \secrev{alignment} model gives a score $\bm{a}_{tk}$ which describes how well the pair of input at position $k$ and the output at position $t$. The \secrev{alignment} score is parameterized by a feed-forward network where the network is trained jointly with the other models \cite{attention_bahdanau}. The score function is defined by a non-linear activation function following Bahdanau's additive style:
\begin{align}\label{eq:score}
	\text{score}(\bm{h}_t,\bm{\bar{h}}_k) = \bm{v}_a^{\top} (\tanh(\bm{w}_1 \bm{h}_t +  \bm{w}_2 \bm{\bar{h}}_k )),
\end{align}
%
%
where $\bm{v}_a^{\top} \in \mathbb{R}^{n}, \bm{w}_1 \!\in\! \mathbb{R}^{n \times n} $ and $\bm{w}_2 \!\in\! \mathbb{R}^{n \times n}$ are \secrev{defined as the weight matrices to be learned in the alignment model, and $n$ is the size of hidden layers}. The overall architecture of our policy network is illustrated in Fig. \ref{fig:neural}.
%
\begin{figure}[t!] 
	\centering
	\includegraphics[width=0.49 \textwidth]{images/NN-v2.pdf}   
	\caption{\small \secrev{\textbf{Policy Network.} CDRS utilizes a neural network architecture to approximate the stochastic policy over the solution. It is constructed by an encoder-decoder sequence-to-sequence model with attention mechanism based on LSTM networks.} }
	\label{fig:neural}
	\vspace{-3mm}
\end{figure}
%
%
\vspace{-2mm}
\subsection{Constrained Policy Gradient with Baseline}
%\vspace{-1mm}
%
%
%
%
%
%
\secrev{We train the above neural network model using a constrained policy gradient method with a self competing baseline.}
We define the objective of $\mathbb{P}$ as an expected reward that is obtained for every vector of weights $\theta$. Hence, the expected cost $J$ in associated with the selected split $o_n$ given BS-$n$ functions \secrev{is denoted as}:
%
\begin{align}
	J^\pi(\theta|\mathcal{F}_n) = \underset{o_n \sim \pi(.|\mathcal{F}_n) }{\mathbb{E}} [ J(o_n) ],
\end{align}
%
and we have the expected of total cost from all BSs:
%
\begin{align} \label{eq:total_cost_theta}
J^\pi(\theta) = \underset{o_n \sim \mathcal{O} }{\mathbb{E}} [ J(\theta|\mathcal{O}) ].
\end{align}
%
The vRAN system has constraints of delay requirement and computational and link capacity. 
%
%we put these to the constraint disatisfication that associated with our policy with: 
%\begin{align} \label{eq:disatisfication}
%J_C^\pi(\theta) = \underset{o_n \sim \mathcal{O} }{\mathbb{E}} [ J(\theta|\mathcal{O}) ]. 
%\end{align}
%
Therefore, our original problem turns to a primal problem as:
%
\begin{align} 
\mathbb{P}_{1\text{P}}: \ \underset{\pi \sim \Pi }{\text{min}} \  J^\pi(\theta); \ \ \text{s.t.} \ \secrev{J_{C_i}^\pi(\theta) \leq 0, \forall i}, \notag
\end{align}
%
where \secrev{we define $J_{C}^\pi(\theta) = \big(J_{C_{i}}^\pi(\theta), \forall i \big)$} as a function of constraint dissatisfaction to capture the penalization that the environment returns for violating each $i$ constraint requirement, e.g., computing, link, delay. 
In this problem, we consider parametrized stochastic policy using a neural network. In order to ensure the convergence of our policy to constraint \secrev{satisfaction}, we follow \cite{reward_constraint} and make assumptions:
%
\begin{assumption} \label{assumption:cost}
	$J^\pi$ is bounded for all policies $\pi \in \Pi$.
\end{assumption}
\begin{assumption} \label{assumption:localminima}
	Each local minima of $J_{C}^\pi(\theta)$  is a feasible solution.
\end{assumption}
\noindent
Assumption \ref{assumption:localminima} describes that any local minima $\pi_\theta$ satisfies all constraints, e.g., $J_{C_i}^\pi(\theta) \leq 0, \forall i$. It is the minimal requirement that guarantees the convergence of a gradient algorithm to a feasible solution. The stricter assumptions, e.g., convexity, may guarantee the optimal solution.
%

%%%%%%%% DUAL FUNCTION START HERE %%%%%%%%%%%%%%%
Next, we reformulate $\mathbb{P}_{1\text{P}}$ to unconstraied problem with Lagrange relaxation method \cite{Bertsekas}. The penalty signal is also included aside \secrev{from the} original objective for infeasibility, which leads to a sub-optimality for infeasible solutions. Given $\mathbb{P}_{\text{1P}}$, we have the dual function:
%
\begin{align} 
	\label{eq:dual_function}
	g({\mu}) = \underset{\theta }{\text{min}} \  J_L^\pi({\mu},\theta) &= \underset{\theta }{\text{min}} \  J^\pi(\theta) + \sum_{i} \mu_i J_{C_{i}}^\pi(\theta) \notag \\
	&=\underset{\theta }{\text{min}} \  J^\pi(\theta) + J_\zeta^\pi(\xi),
\end{align}
where $\mu \!=\! (\mu_i, \forall i), J_L^\pi({\mu},\theta)$ and $J_\zeta^\pi(\xi)$ are the  penalty coefficients (Lagrange multipliers), Lagrange objective function and the expected penalization, respectively. Then, we define the dual problem:
%
\begin{align}
\mathbb{P}_{1\text{D}}: \ \underset{{\mu} }{\text{max}} \  g({\mu}). \notag
\end{align}
%
%
$\mathbb{P}_{1\text{D}}$ aims to find a local optima or a saddle point $(\theta({\mu}^*), {\mu}^*)$, which is a feasible solution. The feasible solution is a solution that satisfies:  $J_{C_{i}}^\pi(\theta) \leq 0, \forall i$. 
%
% 
%
To compute the weights $\theta$ that optimize the objective, we use Monte-Carlo policy gradient and stochastic gradient descent by the following update:
\begin{align}\label{eq:update1}
	\theta_{k+1} = \theta_{k} - \eta_a(k) 	 \nabla_\theta J_L^\pi({\mu},\theta), 
\end{align}
%
where $\eta_a(k) $ is the step-size. The gradient $\nabla_\theta J_L^\pi({\mu},\theta)$  with regards to weights $\theta$ can be calculated using a log-likelihood method as:
%
%
\begin{align}
\nabla_\theta J_L^\pi(\theta) = \underset{\mathcal{O} \sim \pi_\theta(.|\mathcal{F}) }{\mathbb{E}} [ L(\mathcal{O}|\mathcal{F}) \ \nabla_\theta \log \pi_\theta(\mathcal{O}|\mathcal{F}) ].
\end{align} 
%
%
$L(\mathcal{O}|\mathcal{F})$ represents the total cost with penalization obtained from: 
	 $L(\mathcal{O}|\mathcal{F}) = J(\mathcal{O}|\mathcal{F}) + \xi (\mathcal{O}|\mathcal{F}) $, where
$J(\mathcal{O}|\mathcal{F})$ is the total network cost in each iteration and $\xi (\mathcal{O}|\mathcal{F}) = {\mu} C(\mathcal{O}|\mathcal{F})$ is the weighted sum of constraint dissatisfaction of $C(\mathcal{O}|\mathcal{F})$. 

The penalty coefficient ${\mu}$ is set manually \cite{vnf_drl_solozabal,cmpd_solo} for CDRS-Fixed within a range $[0, \mu_{\text{max}} ]$\footnote{If Assumption \ref{assumption:localminima} is satisfied, $\mu_{\text{max}}$ can be set to $\infty$ \cite{reward_constraint}.}. In this case, the selection of ${\mu}$ can be set following intuition approach in \cite{vnf_drl_solozabal} (Appendix C), i.e., agent will not pay attention to penalty if $\mu = 0$, and it will only converge to penalization if $\mu = \infty$. Hence, selecting the appropriate penalty coefficient is important in this case. Otherwise, we can follow a less intuitive approach by adaptively updating the penalty coefficient (CDRS-Ada). CDRS-Ada is updated based on the primal-dual optimization (PDO) method inspired from \cite{pdo_risk}. Hence, we update the penalty coefficient in the ascent direction as:
%
\begin{align} \label{eq:update2}
{\mu}_{k+1} &= {\mu}_{k} + \eta_d(k) \nabla_\mu J_L^\pi({\mu},\theta) \\
& = {\mu}_{k} + \eta_d(k) (  J_{C}^\pi(\theta))_+, 
\end{align} 
where $\eta_d(k)$ is the step-size (Dual) and \secrev{$\nabla_\mu J_L^\pi({\mu},\theta) = \mathbb{E}_{\mathcal{O} \sim \pi_\theta(.|\mathcal{F})} [ C(\mathcal{O}|\mathcal{F}) ]$ is the gradient with respect to $\mu$}. The penalty coefficient ${\mu}_{k}$ is updated for every $k$-th iteration and will converge to a fixed value once the constraints are satisfied \cite{reward_constraint,pdo_risk}. 
%
%
Then, Monte-Carlo sampling can be applied to approximate \secrev{$J_L^\pi(\theta)$} by drawing  $B$ i.i.d samples $ \!\mathcal{F}^1,...,\mathcal{F}^B \!\sim\! \mathcal{F}$, which can be written:
\begin{align} \label{eq:lag_grads}
\!	\nabla_{\!\theta} J_L^\pi(\theta) \! \approx \! \frac{1}{B} \! \sum_{i=1}^{B} \! \! \Big(\! L(\mathcal{O}^i | \mathcal{F}^i) \! - \! b_{\theta_v}(\mathcal{F}^i)\! \Big) \! \nabla_{\!\theta} \! \log \! \pi_\theta(\mathcal{O}^i | \mathcal{F}^i), \!\!
\end{align} 
\secrev{where $b_{\theta_v}(\mathcal{F}^i)$ is the baseline estimation given the state input of $i$-th batch, parameterized by a neural network structure with weights $\theta_v$.}

\textbf{Baseline estimator.} The baseline choice can be from an exponential moving average of the reward over time that captures the improving policy in training. Although it succeeds in the Christofides algorithm, it does not perform well because it can not differentiate between inputs \cite{neural_bello}. To this end, we use a parametric baseline $b_{\theta_v}$ to estimate the expected total cost with penalization that typically improves the learning performance. \secrev{We estimate the baseline through} an auxiliary network built from an LSTM encoder connected to a multilayer perceptron output layer. \secrev{The auxiliary network (parameterized by $\theta_v$) that approximates the expected cost with penalization from input $\mathcal{F}$ is trained with stochastic gradient descent.} It employs a mean squared error (MSE) objective, calculated from the prediction of $b_{\theta_v}$ and the total cost with penalization $L(\mathcal{O}^i | \mathcal{F}^i)$, and sampled by the most recent policy (obtained from the environment). We formulate \secrev{the auxiliary network goal is to minimize the below loss function:}
%$\mathbb{E}_{\mathcal{O} \sim \pi(.|\mathcal{F})} L(\mathcal{O}|\mathcal{F})$
\begin{align} \label{eq:aux_mse}
	\mathcal{L}(\theta_v) = \frac{1}{B} \sum_{i=1}^{B} \left\| b_{\theta_v}(\mathcal{F}^i) - L(\mathcal{O}^i | \mathcal{F}^i) \right\|_2^2.
\end{align}
%
Fig. \ref{fig:baseline} illustrates the architecture of the auxiliary network for estimating the baseline. 
%
%
%
\begin{figure}[t!] 
	\centering
	\includegraphics[width=0.24 \textwidth]{images/baseline.pdf}   
	\caption{\small\secrev{\textbf{Baseline Estimator.} The self-competing baseline of CDRS is estimated using an auxiliary network constructed from an LSTM encoder connected to a multilayer perceptron output linear layer.}  } 
	\label{fig:baseline}
	%\vspace{-1mm}
\end{figure}
%%
%
%
%  
%
%
\begin{figure}[t!] 
	\centering
	\includegraphics[width=0.49 \textwidth]{images/RL_diagram.pdf}   
	\caption{\small\textbf{CDRS Diagram.} CDRS is trained using a single time step Monte-Carlo policy gradient algorithm, where at every epoch, the interactions with the environment are narrowed to a single time step. Our agent learns the policy iteratively over epochs.} 
	\label{fig:rl_diagram}
	\vspace{-3mm}
\end{figure}
%
%
\begin{algorithm}[t!]  \caption{CDRS Training}
	\label{algo:cdrs}
	%\myproc{TRAIN(Learning Set $\mathcal{F}$, batch size $B$)}
	\SetAlgoLined
	\DontPrintSemicolon
	\KwInput{$K$ (Num of epoch), $B$ (Batch size), $\mathcal{F}$ (Learning set)}
	\KwInitialize{ assign agent and critic (baseline) networks with random weights $\theta$ and $\theta_v. \;$} 
	%
	%	 
	\For{ $ k=1, ..., K$}  
	{
		$d\theta$ $\leftarrow$ 0 \% Reset gradient \\
		$\mathcal{F}^i \sim $ \text{SampleInput} $(\mathcal{F})$ for $i \in \{1,...,B \}$. \;
		$\mathcal{O}^i \sim $ SampleSolution $(\pi_\theta(.|\mathcal{F}))$ for $i \in \{1,...,B \}$. \;
		$b^i \leftarrow b_{\theta_v} (\mathcal{F}^i)$ for $i \in \{1,...,B \}$. \;
		Compute $L(\mathcal{O}^i)$ for $i \in \{1,...,B \}$. \;
		$g_\theta \leftarrow \frac{1}{B} \! \sum_{i=1}^{B} \! \! \Big(\! L(\mathcal{O}^i) \! - \! b^{i}\! \Big) \! \nabla_{\!\theta} \! \log \! \pi_\theta(\mathcal{O}^i | \mathcal{F}^i)$ from \eqref{eq:lag_grads}. \;
		$\theta \leftarrow$ Adam($\theta, g_\theta$) \%Run Adam algorithm \;
		$\mathcal{L}_v \leftarrow \frac{1}{B} \sum_{i=1}^{B} \left\| b^{i} - L(\mathcal{O}^i) \right\|_2^2 $ from \eqref{eq:aux_mse}. \;
		$\theta_v \leftarrow$ Adam($\theta_v, \mathcal{L}_v$) \%Run Adam algorithm \;
		{\color{black} Update ${\mu}$ from \eqref{eq:update2} \%CDRS-Ada\\ }
		{\color{black} Set ${\mu} = \max(0,{\mu})$ \%CDRS-Ada}
	}
	\Return $\theta, \theta_v, \mu$
	\;
\end{algorithm}
%
%

\secrev{To sum up, our training procedures are summarized in Algorithm \ref{algo:cdrs} and illustrated in Fig. \ref{fig:rl_diagram}, which run iteratively by $K$ episodes (epochs) based on a single time-step Monte-Carlo policy gradient with a baseline estimator.}
The sequence of policy updates will converge to a locally optimal policy and the penalty coefficient updates (e.g., CDRS-Ada) will converge to a fixed value when all constraints are satisfied; see also \cite{pdo_risk,reward_constraint}. 


%

%for the standard convergence proof of stochastic approximation algorithm with constraints. 

%\subsection{Training Algorithm}
%Algorithm 1 summarizes our training procedure of single time-step Monte-Carlo Policy Gradient with baseline estimator, which runs until $T$ epochs. We also include the penalty coefficient update in this procedure. Our training firstly requires the training set from a set of all BS functions $\mathcal{F}$, the number of minibatch $B$, and the number of epochs $T$. For initalization, we randomly give the weight values for our agent $\theta$ and baseline $\theta_v$. Algorithm 1 runs iteratively until $T$ epochs (Step 1). It resets the gradient ($d\theta $) by assigning zero value (Step 2). Then, it randomly generates i.i.d samples from the training set, e.g., $\mathcal{F}^1,...,\mathcal{F}^B \sim \mathcal{F}$ (Step 3).

%Next, we discuss the convergence of Algorithm 1. 

%It almost surely converges to a fixed values, which is a feasible solution (local optimal)
%\begin{theorem} \label{theo:penalty}
%	The penalty coefficient updates of CDRS-Ada in Algorithm 1 will converge to a fixed value once all constrains are satisfied.
%\end{theorem}
%\noindent
%\textit{Proof.}
%
%
%\begin{theorem} \label{theo:algo1}
%	The policy updates in Algorithm 1 almost surely converges to a locally optimal policy and hold in our case.
%\end{theorem}
%\noindent
%\textit{Proof.} We can proof it following standard procedure for stochastic approximation algorithm...


% Next, we prove that our approach will converge to a fixed values, which is a feasible solution (local minima) for our problem.
%%
%%
%\begin{theorem}
%	Algorithm 1 converges to a feasible solution (local optimality) 
%\end{theorem}
%%
%\textit{Proof.} We prove the convergence of Algorithm 1 for our case following  Theorem xxx of \cite{}. 1) The dual function is always convex despite the primal problem is non-convex \cite{}. Hence, our dual function in \eqref{eq:dual_function}  is also a convex function, so it is also Lipschitz continuous. 

\vspace{-1mm}
\subsection{Searching Strategy}
%\vspace{-1mm}
At the test time, evaluating the total network cost is inexpensive \secrev{as it only requires a forward pass from the policy network to decide the splits}. Our agent can add a search procedure during the inference process by considering solution candidates from multiple \secrev{pretraining} models to select the splits. It can help to reduce the inferred policy suffering from a severe suboptimality. 
%In this part, we employ two different search strategies: greedy decoding and \secrev{temperature sampling} \cite{neural_bello}. 
We employ two different search strategies by greedy decoding and \secrev{temperature sampling} \cite{neural_bello}.

\textbf{Greedy decoding.} It is the simplest search strategy. The idea is to \secrev{greedily} select the splits with the \secrev{highest} probability for having the lowest cost \secrev{from multiple pretraining models during the inference time}. %At the inference time, the greedy output from each model is evaluated to \secrev{choose} the best one \cite{vnf_drl_solozabal}. 
Then, we can extend CDRS to CDRS-Fixed-G, which uses a fixed penalty coefficient with greedy decoding and CDRS-Ada-G that uses an adaptive penalty coefficient with greedy decoding. 

\textbf{Temperature sampling.} This method samples through stochastic policy for \secrev{each pretraining model to generate several candidate solutions} then \secrev{decides} the splits with the lowest total cost \secrev{among them} \cite{neural_bello,vnf_drl_solozabal}. As opposed to the heuristic solvers, it does not sample the different split options. Instead, \secrev{it samples through the stochastic policy} and controls the sparsity of the output distribution \secrev{by} a temperature hyperparameter $T$. The softmax function in \eqref{eq:softmax} is modified to $\bm{a}_{tk} = \frac{\exp\big( \text{score}(\bm{h}_t,\bm{\bar{h}}_k)/T \big)}{\sum_{k'=1}^{N} \exp\big(\text{score}(\bm{h}_t,\bm{\bar{h}}_k')/T ) \big)}$ (softmax temperature). \secrev{In the training}, the temperature hyperparameter $T$ is \secrev{set} to 1. \secrev{Meanwhile, we modify to $T>1$ during the test}, hence the output distribution becomes less step, \secrev{which} prevents the model from being overconfident. With this method, we can extend CDRS to CDRS-Fixed-T (fixed penalty coefficient, temperature sampling) and CDRS-Ada-T (adaptive penalty coefficient, temperature sampling). \secrev{Note that this method requires additional time, which depends on the number of samples.}


 %It considers multiple candidate solutions, then infers the best solution. The approach is to sample candidate solutions from stochastic policy, then select the split configuration with the lowest total cost. A temperature hyperparameter controls the diversity of the sampling to attain an improvement in finding the best solution. The detailed algorithm is described in \cite{neural_bello}.









% RESULT
\vspace{-2mm}


\section{Results and Discussion} \label{sec:results}
\vspace{-1mm}
In this section, we conduct several experiments to evaluate our approach using synthetic and real network datasets. We aim to examine our approach in regards to: \textit{(i)} the behaviour during the training process, \textit{(ii)} the accuracy and solution distributions to the optimality with different penalty coefficient and search strategy settings, \textit{(iii)} the impact of routing costs and traffic loads on the optimality performance and total network cost, and \textit{(iv)} the computational time.




\vspace{-2mm}
\subsection{Environment \& Experiment Setup}
%\vspace{-1mm}
We use synthetic (R1) and real (R2) network datasets to evaluate our approach. We generate R1 \thirdrev{with} stricter constraints and a larger scale environment than R2. R1 is generated using the Waxman algorithm \cite{waxman} with parameters such as link probability ($\alpha$) and edge length control ($\beta$). These respective parameters $(\alpha,\beta)$ are set to $(0.5, 0.1)$. R1 has 1 CU and 99 DUs. In the case of R2, we utilize a real network dataset from \cite{network_sndb}, which has 1 CU and 63 DUs. \secrev{We assume that the routers are co-located with the DUs.} R1 and R2 differ in parameters, e.g., location, link capacity, weighted link, delay. We use  a standard store-and-forward model to calculate the delay. It is from $12000/c_{ij}$, $4 \mu\text{secs}$/Km and $5 \mu\text{secs}$ for transmission, propagation and processing delay, respectively; see \cite{vranmec_andres}. The link capacity varies to $100$ Gbps (R1) and $252$ Gbps (R2). The path delay reaches to $3658 \ \mu s$ (R1) and $42 \ \mu s$ (R2). In R1, the routing cost per path is calculated from the total cost per link (randomly generated) which belongs to the selected path. A link with a routing cost of 1 monetary unit per Mbps means having the same cost as a DU computing cost. We consider the routing cost within a range of $0.001 - 0.01$ times of DU computing cost (for the same network load) for each link in R1. In R2, we calculate the distance between nodes based on its geolocation dataset from \cite{network_sndb} and charge the cost of $0.01$ monetary units per Mbps/km. Fig. \ref{fig:ran_params} depicts the parameter distributions of our RANs with eCDF.

%%%%%%%%%%%%%%%%%%%%%%%%%%%%%%%%%%%%%%%%%%%%%%%%%%%%%%%%%%

\begin{figure}[t]
	\centering
	\begin{subfigure}[t]{.23\textwidth}
		\centering
		\includegraphics[width=\textwidth]{./images/cdf_weight1.pdf}
		%\vspace{-3mm}	
		\small\caption{\small}
	\end{subfigure}
	%
	\begin{subfigure}[t]{.235\textwidth}
		\centering
		\includegraphics[width=\textwidth]{./images/cdf_bw1.pdf}
		%\vspace{-3mm}	
		\small\caption{\small }
	\end{subfigure}
	\begin{subfigure}[t]{.235\textwidth}
		\centering
		\includegraphics[width=\textwidth]{./images/cdf_lat1.pdf}
		%\vspace{-3mm}	
		\small\caption{\small }
	\end{subfigure}
	\caption{\small \textbf{RANs dist.} eCDF of (a) per-path routing cost, (b) per-link capacity, (c) per-path latency for R1 and R2.}
	\label{fig:ran}	
	\label{fig:ran_params}
	%\vspace{-3mm}	
	\vspace{-3mm}	
\end{figure}

%%%%%%%%%%%%%%%%%%%%%%%%%%%%%%%%%%%%%%%%%%%%%%


In this experiment, all system parameters correspond to testbed measurements of previous studies \cite{crancomplexity, vranmec_andres, vran_murti2,cost_vm}. We assume a high load scenario $\lambda_{n} = 150$ Mbps for every DU. This setting is based on 1 user/TTI, $2 \times 2$ MIMO, 20 Mhz (100 PRB), 2 TBs of 75376 bits/subframe and IP MTU 1500B. We use an Intel Haswell i7-4770 3.40GHz CPU as the \textit{reference core}, and set the maximum computing capacity to 75 RCs for CU and 7.5 RCs for each DU. Each split $o \in \{ 0,1,2,3 \}$ inccurs computational load $\rho_{o}^{{d}} = \{ 0.05, 0.04, 0.00325, 0\}$ RCs per Mbps at each DU and $\rho_{o}^{{c}} = \{0, 0.001, 0.00175, 0.05 \} $ RCs per Mbps at the CU. The VM instantiation cost at the CU is half of the DU $(\alpha_0 = \alpha_n/2)$ and the processing cost is set to $\beta_0 = 0.017 \beta_n$. 

Our learning rate is initially set to $\eta_a = 0.0001$ (Agent) and $\eta_b = 0.005$ (Baseline) with the batch size: 128. Our neural network has the number of layers, hidden dimension and embedding size with $1, 32$ and $ 32$, respectively. The temperature hyperparameter is set to $T=1$ by default, so the model computes the softmax function directly. We scale all the original values of weighted paths and traffic loads randomly with uniform distribution $[0,1]$ as in \cite{neural_bello}. Then, we generate three models (RL-pretaining) as outputs of our training with 50000 (in R1) and 15000 (in R2) epochs each. CDRS-Fixed uses a fixed penalty coefficient with $\mu_i =1, \forall i$ for all epochs while CDRS-Ada is set with initial penalty coefficient $\mu_i (0) =1, \forall i$ and step-size $\eta_d = 0.001$. The training is performed with Tensorflow 1.15.3 and Python 3.7.4. In the test, the temperature sampling method uses $16$ samples and $T = 15 $ (softmax temperature). 
%Finally, we summarize our default parameters in this experiment in Table xxx. 

\vspace{-2mm}
\subsection{Training Analysis}
%\vspace{-1mm}

\begin{figure*}[t] 
	\centering
	\begin{subfigure}[t]{.49\textwidth} %\label{fig:res_traina}
		\centering
		\includegraphics[width=\textwidth]{./images/train_r1.pdf}
		\small\caption{\small R1}
	\end{subfigure}
	%
	\begin{subfigure}[t]{.49\textwidth} %\label{fig:res_trainb}
		\centering
		\includegraphics[width=\textwidth]{./images/train_r2.pdf}
		\small\caption{\small R2}
	\end{subfigure}		
	\caption{\small \textbf{Training results of CDRS in (a) R1 and (b) R2.} CDRS-Fixed uses a fixed value of penalty coefficient (reward shaping) with $\mu_i = 1, \forall i$. CDRS-Ada utilizes an adaptive update of penalty coefficient.} 
	\label{fig:res_train} 
	\vspace{-3mm}
\end{figure*}

We aim to examine the behaviour of CDRS-Fixed and CDRS-Ada during the training process in R1 and R2. We focus on the mini-batch loss, reward (total network cost), Lagrangian cost and penalization.  

Fig. \ref{fig:res_train} visualizes the training of CDRS-Fixed and CDRS-Ada in R1 and R2. We found additional costs because of penalization at the beginning of the training for both settings in R1 and R2. It occurs because CDRS-Fixed and CDRS-Ada try to find the solution, but violate the constraint sets (e.g., latency, bandwidth, computation). Fig. \ref{fig:res_train} also shows a significant difference in the cost of penalization in R1 compared to R2. The main reason is that R1 has stricter constraints, e.g., larger path delays, smaller link capacity than R2. We can also see that CDRS-Fixed and CDRS-Ada improve their policy by focusing on constraint satisfaction and then correcting the weights via stochastic gradient descent. It is proven from our agent's behaviour in R1 and R2, where each penalization cost keeps decreasing and turns to zero as soon as the training goes. CDRS-Ada sets the penalty coefficient increasing in the ascent direction, causing a higher penalization value than CDRS-Fixed. However, it can help speed up the policy toward constraint satisfaction, i.e., CDRS-Ada penalization downs faster than CDRS-Fixed.


We also found that the policy of CDRS-Ada converges faster than CDRS-Fixed from the behaviour of mini-batch loss in R1. Despite the mini-batch loss decreases to near zero after several epochs, the mini-batch loss of CDRS-Ada diminishes faster than CDRS-Fixed. However, CDRS-Ada suffers from more severe sub-optimality. It is shown by the total vRAN cost of CDRS-Ada that converges to a fixed value but has a higher cost compared to CDRS-Fixed.  Then, we have the Lagrangian cost from the sum of vRAN cost and penalization cost. It describes how our agent tries to minimize the primal problem $\mathbb{P}_{\text{1P}}$ through the dual problem $\mathbb{P}_{\text{1D}}$.  When our agent finally dismisses the penalization cost, it means that all constraints are satisfied. As a result, the Lagrangian cost becomes equal to the vRAN cost, and the penalty coefficient of CDRS-Ada converges to a fixed value.  Although having different behaviours, CDRS-Ada and CDRS-Fixed can learn the solution and converge to the local minima or saddle point in R1 and R2.

\textbf{Findings:} 1) R1 has stricter constraint requirements than R2; hence, it produces a higher additional cost for penalization to CDRS-Fixed and CDRS-Ada. 2) CDRS-Fixed and CDRS-Ada improve the policy by focusing on the penalization; then, it adjusts the weights as the training goes. 3) CDRS-Ada receives higher penalization compared to CDRS-Fixed as a result of increasing the penalty coefficient in the ascent direction; however, it also helps speed up the policy to constraint satisfaction. 4) CDRS-Ada converges faster but has a higher cost than CDRS-Fixed in R1. 5) When all constraints are satisfied,  the Lagrangian cost becomes equal to the total vRAN cost, and the penalty coefficient of CDRS-Ada converges to a fixed value.

%

\vspace{-2mm}
\subsection{Accuracy of Solutions}
%\vspace{-1mm}
%
%
\begin{figure*}[t]
	\centering
	\begin{subfigure}[t]{.47\textwidth}
		\centering
		\includegraphics[width=\textwidth]{./images/acc_r1.pdf}
		\small\caption{\small R1}
	\end{subfigure}
	%
	\begin{subfigure}[t]{.47\textwidth}
		\centering
		\includegraphics[width=\textwidth]{./images/acc_r2.pdf}
		\small\caption{\small R2}
	\end{subfigure}	
	\caption{\small \textbf{Histogram of CDRS accuracy in (a) R1 and (b) R2.} The accuracy is calculated over 128 tests. CDRS-Ada-T and CDRS-Fixed-T are set with $T=15$ and $16$ samples.} 	\label{fig:_accmain}
	\vspace{-3mm}	
\end{figure*}


In this part, we study the accuracy of CDRS over different penalty coefficient and search strategy settings: CDRS-Fixed-G, CDRS-Fixed-T, CDRS-Ada-G and CDRS-Ada-T. We conduct 128 tests with a distinct sequence order of the BSs in R1 and R2 to assess how accurate these four CDRS settings find the solution of the vRAN split problem. We utilize three pretraining models \secrev{from} our CDRS training.

Fig. \ref{fig:_accmain} shows the distribution of \secrev{the solutions from} CDRS-Fixed-G, CDRS-Fixed-T, CDRS-Ada-G and CDRS-Ada-T in R1 and R2. \secrev{Each bar counts the number of offered solutions resulting in some suboptimality, represented using the optimality gap (error). It shows that the distribution varies between four settings, especially in a stricter environment (R1). Still,} all of these settings can guarantee less than $0.6 \%$ (R1) and $0.1 \%$ (R2) of the optimality gap. In R1, CDRS-Fixed-G and CDRS-Fixed-T perform better by offering lower solution errors ($ \leq 0.05 \%$ and $\leq 0.05 \%$ of optimality gap) than CDRS-Ada-G and CDRS-Ada-T ($\leq 0.6 \%$). It means that a fixed penalty coefficient setting can lead to a better optimality performance during the test than the adaptive one. However, CDRS-Fixed-G, CDRS-Ada-G and CDRS-Ada-T have a similar performance in R2. Regardless of R1 or R2, using a sampling method with a temperature hyperparameter can improve (or at least at same) the optimality performance than greedy decoding. It is shown from the higher total number of solutions (counts) for a sampling method that having a lower error. The combination of a fixed penalty in the training and temperature sampling method (CDRS-Fixed-T) can improve the solution performance significantly both in R1 and R2. It can achieve an optimal value (R2) and less than $0.05 \%$ of error for a more complex environment (R1). It is also shown that CDRS-Fixed-T is less affected to the stricter environment than any other settings where all of the distribution solutions are in less than $0.05 \%$.


\textbf{Findings:} 1) CDRS-Fixed-G, CDRS-Fixed-T, CDRS-Ada-G and CDRS-Ada-T can guarantee the solution with very close to the optimal value offering less than $0.6 \%$ (R1) and $0.1 \%$ (R2) of the optimality gap over 128 tests. 2) CDRS-Fixed-T can significantly improve the optimality performance (offers $\leq 0.05 \%$ of optimality gap) and outperforms the other settings.  



\vspace{-2mm}
\subsection{Impact of Routing Cost}
\vspace{-1mm}


This part studies the impact of altering the routing cost to CDRS-Fixed-G, CDRS-Fixed-T, CDRS-Ada-G and CDRS-Ada-T. We aim to examine how the routing cost affects optimality performance and the total network cost. Hence, the default routing cost is changed within a range of $\gamma=0.1$ to $\gamma=1$. This change can arise due to increasing or decreasing the leasing agreement's price, maintenance, etc. The traffic load is fixed with $\lambda_{n}=150$ Mbps. We utilize three \secrev{pretraining} models, conduct 128 tests for each routing cost scale, and analyze the offered solutions' distribution. We also consider benchmarking with two extremes of RAN setups, fully D-RAN and C-RAN\footnote{We practically can not implement C-RAN because our RANs do not meet the constraint requirements of delay, bandwidth and CU capacity to deploy C-RAN. The presented C-RAN in this experiment is just for benchmarking; hence we also do not consider the penalization cost (constrains violation) for this case.} to assess how significant the routing cost affects the total network cost over various RAN setups.


\begin{figure*}[t] 
	\centering
	\begin{subfigure}[t]{.99\textwidth}
		\centering
		\includegraphics[width=\textwidth]{./images/routing_acc_r1.pdf}
		\small\caption{\small R1}
	\end{subfigure}
	%
	\begin{subfigure}[t]{.99\textwidth}
		\centering
		\includegraphics[width=\textwidth]{./images/routing_acc_r2.pdf}
		\small\caption{\small R2}
	\end{subfigure}		
	\caption{\small \textbf{Impact of the routing cost to the accuracy in (a) R1 and (b) R2.} Study of altering the routing cost to the optimality performance with $\lambda_{n}=150$ Mbps, $\forall n \in \mathcal{N}$. There are 128 tests for each routing scale $[0.1,1]$.} \label{fig:routing_acc}
	\vspace{-3mm}
\end{figure*}


\begin{figure*}[t] 
	\centering
	\begin{subfigure}[t]{.375\textwidth}
		\centering
		\includegraphics[width=\textwidth]{./images/routing_cost_r1.pdf}
		\small\caption{\small R1}
	\end{subfigure}
	%
	\begin{subfigure}[t]{.375\textwidth}
		\centering
		\includegraphics[width=\textwidth]{./images/routing_cost_r2.pdf}
		\small\caption{\small R2}
	\end{subfigure}		
	\caption{\small \textbf{Impact of routing cost to the total cost in (a) R1 and (b) R2.} We also compare our approach (e.g., CDRS-Fixed-T) to two extreme cases: fully D-RAN and C-RAN, and the optimal value with the routing cost scaling from 0.1 to 1 of default R1 and R2. The presented cost above is normalized toward fully C-RAN cost.} \label{fig:routing_cost}
	\vspace{-3mm}
\end{figure*}


Fig. \ref{fig:routing_acc} depicts how the routing cost affects the optimality performance of CDRS-Fixed-G, CDRS-Fixed-T, CDRS-Ada-G and CDRS-Ada-T. It shows that the \secrev{overall} optimality gap (error) diminishes as the routing cost increases; then, it converges to a specific value. In R1, we see a performance improvement as the errors decrease for CDRS-Fixed-G ($\approx75\%$), CDRS-Fixed-T ($\approx75\%$), CDRS-Ada-G ($\approx78\%$) and CDRS-Ada-T ($\approx75\%$) \secrev{by the increase of routing cost}. It also shows that CDRS-Ada-G gets the most impact while CDRS-Fixed-T is the least affected. In R2, all CDRS settings \secrev{also} have a similar trend in terms of error \secrev{performance}. Although we have changed the routing cost from the default parameter, we found that \secrev{altering the} routing cost gives relatively less effect to these settings where the errors are maintained under $1.8 \%$. CDRS-Fixed-T even can guarantee the solution under $0.08 \%$ ($\gamma  = 0.1$) of the optimality gap.



Fig. \ref{fig:routing_cost} shows the routing cost's effect on the total network cost of CDRS-Fixed-T and D-RAN, normalized to the C-RAN cost in R1 and R2. \secrev{It shows that CDRS-Fixed-T can obtain a larger cost-saving than the D-RAN cost at a cheaper routing cost by up to $59.06\%$ of cost-saving at $\gamma = 0.1$ while only $25.49\%$ of cost-saving at $\gamma = 1$ in R1. Compared to C-RAN, CDRS-Fixed-T can save the cost by up to $92\%$ at $\gamma = 1$ in R1. However, this gain diminishes as the routing cost decreases and eventually CDRS-Fixed-T will reach near the C-RAN cost if all constraint requirements are eligible. A similar trend also appears for R2. Moreover, CDRS-Fixed-T can offer the solution extremely close to the optimal solution by $\leq 0.09\%$ (R1) and $\leq 0.5\%$  (R2). }
%

%It shows that the increase of routing cost gives CDRS-Fixed-T and D-RAN cost relatively decrease (around $320\%$ and  $420\%$ from $\gamma = 0.1$ to $\gamma = 1$ in R1) to the C-RAN cost. Hence, we can conclude that the routing cost gives more impact to C-RAN than other setups. Additionally, CDRS-Fixed-T is the most cost-efficient with around $500\%$ and $200\%$ cost-saving of C-RAN and D-RAN at low routing cost ($ \gamma = 0.1$). It also can save to 20 times and two-fold compared to the respective RAN setups at high routing cost ($ \gamma = 1$) in R1. In R2, CDRS-Fixed-T can save to around five times and two times of C-RAN and D-RAN cost at low routing cost ($ \gamma = 0.1$). It has cost-saving to 20 times and two-fold compared to the respective RAN setups at high routing cost ($ \gamma = 1$). CDRS-Fixed-T offers the solution very close to the optimal solution ($\leq 0.09\%$ in R1 and $\leq 0.5\%$ in R2) and efficiently adapts to the change of the routing cost. 



\textbf{Findings:} 1) The increase of routing cost \secrev{reduces} the optimality gap \secrev{(error)}; then, \secrev{it} converges to a fixed value. 2) CDRS-Fixed-T is the least affected by the routing cost changes, while CDRS-Ada-G is the most affected. 3) \secrev{Scaling} the routing cost from $\gamma = 0.1$ to $\gamma = 1$ does not significantly degrade the optimality performance. 4) CDRS-Fixed-T has the lowest optimality gap  \secrev{than other CDRS settings}, and becomes the most cost-effective setup in R1 and R2. \secrev{5) CDRS-Fixed-T can reach near the D-RAN cost at a high routing cost, while it can be near the C-RAN cost at a cheap routing cost if all constraint requirements are eligible.}
%

\vspace{-2mm}
\subsection{Impact of Traffic Load}
\vspace{-1mm}

\begin{figure*}[t] 
	\centering
	\begin{subfigure}[t]{.99\textwidth}
		\centering
		\includegraphics[width=\textwidth]{./images/traffic_acc_r1.pdf}
		\small\caption{\small R1}
	\end{subfigure}
	%
	\begin{subfigure}[t]{.99\textwidth}
		\centering
		\includegraphics[width=\textwidth]{./images/traffic_acc_r2.pdf}
		\small\caption{\small R2}
	\end{subfigure}		
	\caption{\small \textbf{Impact of the traffic load to the accuracy in (a) R1 and (b) R2.} Study of traffic load to the optimality performance. There are 128 tests for each traffic load. } \label{fig:traffic_acc}
	\vspace{-3mm}
\end{figure*}


\begin{figure*}[t] 
	\centering
	\begin{subfigure}[t]{.375\textwidth}
		\centering
		\includegraphics[width=\textwidth]{./images/traffic_cost_r1.pdf}
		\small\caption{\small R1}
	\end{subfigure}
	%
	\begin{subfigure}[t]{.375\textwidth}
		\centering
		\includegraphics[width=\textwidth]{./images/traffic_cost_r2.pdf}
		\small\caption{\small R2}
	\end{subfigure}		
	\caption{\small \textbf{Impact of traffic load to total vRAN cost in (a) R1 and (b) R2.} On the comparison of our approach (e.g., CDRS-Fixed-T) to fully D-RAN. The presented cost above is normalized toward fully C-RAN cost.} \label{fig:traffic_cost}
	\vspace{-3mm}
\end{figure*}

In this part, we assess how the traffic load affects the optimality performance and the total network cost. We change the traffic load from 10 Mbps to 150 Mbps. This evaluation is conducted using three \secrev{pretraining} models and examined over 128 tests.   

Fig \ref{fig:traffic_acc} shows the impact of altering the traffic load to the optimality performance of CDRS-Fixed-G, CDRS-Fixed-T, CDRS-Ada-G and CDRS-Ada-T. In R1, it shows that the increase of traffic load in line with the rise of the error to CDRS-Ada-G and CDRS-Ada-T, but it then diminishes to a fixed value, i.e., around $0.4 \%$ (CDRS-Ada-G) and $0.18\%$ (CDRS-Ada-T). However, the traffic load does not significantly affect CDRS-Fixed-G and CDRS-Fixed-T, where they stay at around $0.04\%$ and $0.02\%$ of errors, respectively, in R1. In R2, CDRS-Fixed-G, CDRS-Fixed-T, CDRS-Ada-G and CDRS-Ada-T have the same trend where the optimality gap increases with the traffic load; then, it diminishes at around  $0.05\%$. We also found that CDRS-Fixed-T \secrev{has a} better optimality performance and a more stable solution. 


Fig \ref{fig:traffic_cost} examines the impact of traffic load on CDRS-Fixed-T and D-RAN cost normalized to the C-RAN cost. Despite an increase in CDRS-Fixed-T cost as the traffic load rises, it shows that CDRS-Fixed-T is still the most cost-effective compared to D-RAN and C-RAN \secrev{in} R1 and R2. \secrev{CDRS-Fixed-T almost has the same cost as D-RAN at the low traffic load with only $12.33\%$ cost-saving. This cost-saving then increases for the higher traffic load settings by up to $25.5\%$ at 150 Mbps in R1. This trend also happens in R2. Compared to C-RAN, CDRS-Fixed-T significantly outperforms at the low traffic load, but this gain then diminishes as the increase of the load. CDRS-Fixed-T can reach near the C-RAN cost when all constraint requirements are satisfied, and the traffic load is high, but the routing cost is significantly low.}

%
%In R1, CDRS-Fixed-T can save $114\%$ at low traffic load (10 Mbps) and $134\%$ at high traffic load (150 Mbps) of D-RAN cost, while $124\%$ and $177\%$ for the respective load in R2. The cost-saving gap is also more prominent with the increase of traffic load. We also found that D-RAN is the most affected by the increase in traffic load.

\textbf{Findings:} 1) CDRS-Fixed-T can offer to better optimality performance and more stable solution \secrev{than other CDRS settings}. 2) In R2, all CDRS settings have similar trends where the increase of traffic load can also increase the optimality gap, but it then diminishes and stays at around $0.05\%$ for CDRS-Fixed-T and $0.06\%$ for the others. 3)  CDRS-Fixed-T is the most cost-efficient compared to C-RAN and D-RAN. 4) \secrev{CDRS-Fixed-T can eventually almost have the same C-RAN cost when all constraint requirements are satisfied, and the traffic load is high, but the routing cost is significantly low.}

\vspace{-2mm}
\subsection{Computational Time}
\vspace{-1mm}

Finally, we examine the computational time to solve a single instance of the vRAN split problem. We use a small laptop with an Intel Core i5-7300U CPU@2.60GHz and 8GB memory. The computational time for each CDRS setting is a result of averaging 128 executions with a trained model. We report this evaluation in Table \ref{table:computationaltime}.  Overall, our proposed CDRS settings: CDRS-Fixed-G, CDRS-Fixed-T, CDRS-Ada-G and CDRS-Ada-T, have a faster computational time than the MIP solver. CDRS-Ada-G is the fastest with $0.0120$ secs and $0.0077$ secs in R1 and R2 reaching to $22.82$ times faster than the MIP solver. We also found that any CDRS settings with greedy decoding for the inference process, e.g., CDRS-Fixed-G, CDRS-Ada-G, is more time-efficient than a temperature sampling method with around 10-20 times faster. It is also shown that CDRS-Ada-G/T has a slightly faster computational time than CDRS-Fixed-G/T. Finally, we can sort from the fastest computational time as 1) CDRS-Ada-G, 2) CDRS-Fixed-G, 3) CDRS-Ada-T, 4) CDRS-Fixed-T, 5) the MIP solver.
%Besides, an adaptive penalty coefficient can speed up the policy to find the solution than a fixed penalty coefficient, especially in the highly constrained environment (R1).

\begin{table*}[t!] \centering
	%\ra{1.3}
	\begin{small}
		\begin{tabular}{@{}cccccc@{}}\toprule
			\textbf{Topology}& \textbf{MIP solver} &\textbf{CDRS-Fixed-T} & \textbf{CDRS-Fixed-G}  & \textbf{CDRS-Ada-T} & \textbf{CDRS-Ada-G}
			\\ \midrule
			\textbf{R1} &      0.2527   & 0.2026 & 0.0155& 0.1985 & 0.0120          
			\\ \hdashline
			{\textbf{R2}} &  0.1756 &  0.1240 & 0.0098 & 0.1207 &0.0077
			\\ \hdashline
			\bottomrule
		\end{tabular}
	\end{small}
	\caption{\small \textbf{Computational time.} Study of computational time for solving a single problem instance in seconds. The presented computational time is a result of averaging 128 executions.}
	\label{table:computationaltime}
	\vspace{-3mm}
\end{table*}

\textbf{Findings:} 1) CDRS-Ada-G, CDRS-Fixed-G, CDRS-Ada-T, and CDRS-Fixed-T can reach up to $22.82, 17.99, 1.45$ and $1.41$ times faster than the MIP solver. 2) Greedy decoding is more time-efficient than a temperature sampling method for the inference process.


%CONCLUSION
\vspace{-2mm}
\vspace{-1em}
\section{Conclusions}
\vspace{-0.6em}
\label{SEC:CONC}
In this paper, we investigated the impact of workload dependent parameters on the failure ratio of the SSDs under power outage. To this end, we presented a fault injection and failure detection platform which injects the realistic power faults to the under test SSDs. During power failure, SSDs experience the exact voltage drop behavior that occurs during power failures in data centers. The results of our experiments reveal that the failure ratio in SSDs due to power outage is significantly affected by the parameters of the running workloads in the application layer. In addition, we show that failures in SSDs are not only due to volatile DRAM cache but also we observe similar failures in SSDs with disabled internal cache.







%%%%%%%%%%%%%%%%%%%%%%%%%%%%%%%%%%%%%%%% ============================================= %%%%%%%%%%%%%%%%%

\bibliographystyle{IEEEtran}
\bibliography{IEEEabrv,ref-vran-01}



%
%\begin{IEEEbiography}
%	[{\includegraphics[width=1in,height=1.25in,clip,keepaspectratio]{./biography/fahri.jpg}}]{Fahri Wisnu Murti} is currently pursuing his doctoral degree at Centre for Wireless Communication (CWC), University of Oulu, Finland. His current research interests lie in the development of machine learning and optimization techniques for intelligent wireless networks.
%	%
%	Prior to his doctoral study, he worked as a research assistant at Dept. Computer Science, Trinity College Dublin, Ireland. He received his B.S. from Telkom University, Indonesia and completed his master's degree from WENS Lab., Dept. IT Convergence Eng., Kumoh National Institute of Technology, South Korea. 
%\end{IEEEbiography}
%
%%\vspace*{-2\baselineskip}
%
%\begin{IEEEbiography}[{\includegraphics[width=1.1 in,height=1.25in,clip,keepaspectratio]{./biography/samad.jpg}}]{Samad Ali} received the B.S. degree in electrical engineering from the University of Tabriz, Tabriz, Iran, and the M.S. and Ph.D. degrees in wireless communications engineering from the University of Oulu, Oulu, Finland. He is currently a Postdoctoral Researcher with the University of Oulu and Senior Research Specialist at Nokia Standards. His research interests include machine learning in wireless communications, machine type communications and RIS. 
%\end{IEEEbiography}
%
%%\vspace*{-2\baselineskip}
%
%\begin{IEEEbiography}[{\includegraphics[width=1in,height=1.25in,clip,keepaspectratio]{./biography/matti.jpg}}]{Matti Latva-aho} received the M.Sc., Lic.Tech. and Dr. Tech (Hons.) degrees in Electrical Engineering from the University of Oulu, Finland in 1992, 1996 and 1998, respectively. From 1992 to 1993, he was a Research Engineer at Nokia Mobile Phones, Oulu, Finland after which he joined Centre for Wireless Communications (CWC) at the University of Oulu. Prof. Latva-aho was Director of CWC during the years 1998-2006 and Head of Department for Communication Engineering until August 2014. Currently he serves as Academy of Finland Professor in 2017 – 2022 and is Director for National 6G Flagship Programme for 2018 - 2026. His research interests are related to mobile broadband communication systems and currently his group focuses on beyond 5G systems research. Prof. Latva-aho has published close to 500 conference or journal papers in the field of wireless communications. He received Nokia Foundation Award in 2015 for his achievements in mobile communications research.
%\end{IEEEbiography}



\end{document}
