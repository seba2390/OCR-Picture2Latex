%\documentclass[journal,12pt, onecolumn, draftclsnofoot]{IEEEtran}
\documentclass[journal, final]{IEEEtran}

\usepackage{textpos}
\newcommand\copyrighttext{%\small  This work has been accepted in IEEE Transactions on Wireless Communications. \\
%	\small \textcopyright 2022 IEEE. Personal use of this material is permitted. Permission from IEEE must be obtained for all other uses, in any current or future media including reprinting/republishing this material for advertising or promotional purposes, creating new collective works, for resale or redistribution to servers or lists, or reuse of any copyrighted component of this work in other works. 
%\centering
\small This article has been accepted for publication in IEEE Transactions on Wireless Communications.  %\textcopyright 2022 IEEE.

}
\newcommand\copyrightnotice{%
	\begin{tikzpicture}[remember picture,overlay]
		\node[anchor=north,yshift=-15pt] at (current page.north)  {\copyrighttext}
		;
	\end{tikzpicture}%
}

%%% for comsoc format: remove asmthm, amssymb. Replace with netxmath
\usepackage{setspace}
%\usepackage{newtxmath}
%\usepackage{tcolorbox}
\usepackage{amsmath}
\usepackage{bm}
\usepackage{mdframed}
\usepackage{amsthm}
\newtheorem{thm}{Theorem}
\usepackage[utf8]{inputenc}
\usepackage{graphicx}
\usepackage[colorinlistoftodos]{todonotes}
\usepackage{amssymb}
\usepackage{longtable}
\usepackage[colorlinks=true, allcolors=blue]{hyperref}
\usepackage[noadjust]{cite}
\allowdisplaybreaks
\usepackage{lineno}
\usepackage{lipsum}
\usepackage{multicol}
\usepackage{subcaption}
\usepackage{enumitem}
\newtheorem{theorem}{Theorem}
\newtheorem{definition}{Definition}
\newtheorem{lemma}{Lemma}
\newtheorem{assumption}{Assumption}
\usepackage{multirow}
\usepackage{array}

\usepackage{xcolor}

\usepackage[vlined,linesnumbered,ruled,resetcount]{algorithm2e}


\newcommand{\fm}[1]{{\color{purple}\textbf{}{FM: #1}}}


\newcommand{\rev}[1]{{\color{black}#1}}
\newcommand{\secrev}[1]{{\color{black}#1}}
\newcommand{\thirdrev}[1]{{\color{black}#1}}

\SetKwInput{KwInput}{Input}                % Set the Input
\SetKwInput{KwOutput}{Output}              % set the Output
\SetKwInput{KwInitialize}{Initialize}




%%%%%%%%%%%TABLE PACKAGES
\usepackage{booktabs}
\usepackage{caption}
\usepackage{float}
%\usepackage{titlesec}
\usepackage{capt-of}
\usepackage{arydshln}
\usepackage{soul,xcolor}
\usepackage{tabularx}

%dashed line
\usepackage{arydshln}
\setlength\dashlinedash{0.2pt}
\setlength\dashlinegap{1.5pt}
\setlength\arrayrulewidth{0.3pt}

%Widows & Orphans & Penalties

\widowpenalty500
\clubpenalty500
\clubpenalty=9996
\exhyphenpenalty=50 %for line-breaking at an explicit hyphen
\brokenpenalty=4991
\predisplaypenalty=10000
\postdisplaypenalty=1549
\displaywidowpenalty=1602
\floatingpenalty = 20000
%%%%%%%%%%%TABLE

\title{Constrained Deep Reinforcement Based Functional Split Optimization in Virtualized RANs}


\author{\IEEEauthorblockN{Fahri Wisnu Murti, Samad Ali, and Matti Latva-aho}\\
%\vspace{2mm}
\IEEEauthorblockA{
	Centre for Wireless Communications, University of Oulu, Finland}
\thanks{A preliminary version of this work appears in IEEE ICC 2021 Workshop \cite{murti_iccw_cdrs_vran}. This research has been supported by the Academy of Finland, 6G Flagship program under Grant 346208.}%
}



\IEEEoverridecommandlockouts
\begin{document}
	
%\linespread{1.23}

\maketitle
%\thispagestyle{plain}
%\pagestyle{plain}

%\vspace{-1.cm}
\begin{abstract}
\thirdrev{In virtualized radio access network (vRAN), the base station (BS) functions are decomposed into virtualized components that can be hosted at the centralized unit or distributed units through functional splits. Such flexibility has many benefits; however, it also requires solving the problem of finding the optimal splits of functions of the BSs in such a way that minimizes the total network cost. The underlying vRAN system is complex and precise modelling of it is not trivial. Formulating the functional split problem to minimize the cost results in a combinatorial problem that is provably NP-hard, and solving it is computationally expensive. In this paper, a constrained deep reinforcement learning (RL) approach is proposed to solve the problem with minimal assumptions about the underlying system. Since in deep RL, the action selection is the outcome of inference of a neural network, it can be done in real-time while training to update the neural networks can be done in the background. However, since the problem is combinatorial, even for a small number of functions, the action space of the RL problem becomes large. Therefore, to deal with such a large action space, a chain rule-based stochastic policy is exploited in which a long short-term memory (LSTM) network-based sequence-to-sequence model is applied to estimate the policy that is selecting the functional split actions. However, the utilized policy is still limited to an unconstrained problem, and each split decision is bounded by vRAN’s constraint requirements. Hence, a constrained policy gradient method is leveraged to train and guide the policy toward constraint satisfaction. Further, a search strategy by greedy decoding or temperature sampling is utilized to improve the optimality performance at the test time. Simulations are performed to evaluate the performance of the proposed solution using synthetic and real network datasets. Our numerical results show that the proposed RL solution architecture successfully learns to make optimal functional split decisions with the accuracy of the solution is up to 0.05\% of the optimality gap. Moreover, our solution can achieve considerable cost savings compared to C-RAN or D-RAN systems and a faster computational time than the optimal baseline.}

\end{abstract}

\IEEEpeerreviewmaketitle

\copyrightnotice
%%%%%%%%%%%%%%%%%%%%%%%%%%%%%%%% ================== MAIN CONTENT ==================
% INTRODUCTION
\vspace{-2mm}
\IEEEraisesectionheading{\section{Introduction}}

\IEEEPARstart{V}{ision} system is studied in orthogonal disciplines spanning from neurophysiology and psychophysics to computer science all with uniform objective: understand the vision system and develop it into an integrated theory of vision. In general, vision or visual perception is the ability of information acquisition from environment, and it's interpretation. According to Gestalt theory, visual elements are perceived as patterns of wholes rather than the sum of constituent parts~\cite{koffka2013principles}. The Gestalt theory through \textit{emergence}, \textit{invariance}, \textit{multistability}, and \textit{reification} properties (aka Gestalt principles), describes how vision recognizes an object as a \textit{whole} from constituent parts. There is an increasing interested to model the cognitive aptitude of visual perception; however, the process is challenging. In the following, a challenge (as an example) per object and motion perception is discussed. 



\subsection{Why do things look as they do?}
In addition to Gestalt principles, an object is characterized with its spatial parameters and material properties. Despite of the novel approaches proposed for material recognition (e.g.,~\cite{sharan2013recognizing}), objects tend to get the attention. Leveraging on an object's spatial properties, material, illumination, and background; the mapping from real world 3D patterns (distal stimulus) to 2D patterns onto retina (proximal stimulus) is many-to-one non-uniquely-invertible mapping~\cite{dicarlo2007untangling,horn1986robot}. There have been novel biology-driven studies for constructing computational models to emulate anatomy and physiology of the brain for real world object recognition (e.g.,~\cite{lowe2004distinctive,serre2007robust,zhang2006svm}), and some studies lead to impressive accuracy. For instance, testing such computational models on gold standard controlled shape sets such as Caltech101 and Caltech256, some methods resulted $<$60\% true-positives~\cite{zhang2006svm,lazebnik2006beyond,mutch2006multiclass,wang2006using}. However, Pinto et al.~\cite{pinto2008real} raised a caution against the pervasiveness of such shape sets by highlighting the unsystematic variations in objects features such as spatial aspects, both between and within object categories. For instance, using a V1-like model (a neuroscientist's null model) with two categories of systematically variant objects, a rapid derogate of performance to 50\% (chance level) is observed~\cite{zhang2006svm}. This observation accentuates the challenges that the infinite number of 2D shapes casted on retina from 3D objects introduces to object recognition. 

Material recognition of an object requires in-depth features to be determined. A mineralogist may describe the luster (i.e., optical quality of the surface) with a vocabulary like greasy, pearly, vitreous, resinous or submetallic; he may describe rocks and minerals with their typical forms such as acicular, dendritic, porous, nodular, or oolitic. We perceive materials from early age even though many of us lack such a rich visual vocabulary as formalized as the mineralogists~\cite{adelson2001seeing}. However, methodizing material perception can be far from trivial. For instance, consider a chrome sphere with every pixel having a correspondence in the environment; hence, the material of the sphere is hidden and shall be inferred implicitly~\cite{shafer2000color,adelson2001seeing}. Therefore, considering object material, object recognition requires surface reflectance, various light sources, and observer's point-of-view to be taken into consideration.


\subsection{What went where?}
Motion is an important aspect in interpreting the interaction with subjects, making the visual perception of movement a critical cognitive ability that helps us with complex tasks such as discriminating moving objects from background, or depth perception by motion parallax. Cognitive susceptibility enables the inference of 2D/3D motion from a sequence of 2D shapes (e.g., movies~\cite{niyogi1994analyzing,little1998recognizing,hayfron2003automatic}), or from a single image frame (e.g., the pose of an athlete runner~\cite{wang2013learning,ramanan2006learning}). However, its challenging to model the susceptibility because of many-to-one relation between distal and proximal stimulus, which makes the local measurements of proximal stimulus inadequate to reason the proper global interpretation. One of the various challenges is called \textit{motion correspondence problem}~\cite{attneave1974apparent,ullman1979interpretation,ramachandran1986perception,dawson1991and}, which refers to recognition of any individual component of proximal stimulus in frame-1 and another component in frame-2 as constituting different glimpses of the same moving component. If one-to-one mapping is intended, $n!$ correspondence matches between $n$ components of two frames exist, which is increased to $2^n$  for one-to-any mappings. To address the challenge, Ullman~\cite{ullman1979interpretation} proposed a method based on nearest neighbor principle, and Dawson~\cite{dawson1991and} introduced an auto associative network model. Dawson's network model~\cite{dawson1991and} iteratively modifies the activation pattern of local measurements to achieve a stable global interpretation. In general, his model applies three constraints as it follows
\begin{inlinelist}
	\item \textit{nearest neighbor principle} (shorter motion correspondence matches are assigned lower costs)
	\item \textit{relative velocity principle} (differences between two motion correspondence matches)
	\item \textit{element integrity principle} (physical coherence of surfaces)
\end{inlinelist}.
According to experimental evaluations (e.g.,~\cite{ullman1979interpretation,ramachandran1986perception,cutting1982minimum}), these three constraints are the aspects of how human visual system solves the motion correspondence problem. Eom et al.~\cite{eom2012heuristic} tackled the motion correspondence problem by considering the relative velocity and the element integrity principles. They studied one-to-any mapping between elements of corresponding fuzzy clusters of two consecutive frames. They have obtained a ranked list of all possible mappings by performing a state-space search. 



\subsection{How a stimuli is recognized in the environment?}

Human subjects are often able to recognize a 3D object from its 2D projections in different orientations~\cite{bartoshuk1960mental}. A common hypothesis for this \textit{spatial ability} is that, an object is represented in memory in its canonical orientation, and a \textit{mental rotation} transformation is applied on the input image, and the transformed image is compared with the object in its canonical orientation~\cite{bartoshuk1960mental}. The time to determine whether two projections portray the same 3D object
\begin{inlinelist}
	\item increase linearly with respect to the angular disparity~\cite{bartoshuk1960mental,cooperau1973time,cooper1976demonstration}
	\item is independent from the complexity of the 3D object~\cite{cooper1973chronometric}
\end{inlinelist}.
Shepard and Metzler~\cite{shepard1971mental} interpreted this finding as it follows: \textit{human subjects mentally rotate one portray at a constant speed until it is aligned with the other portray.}



\subsection{State of the Art}

The linear mapping transformation determination between two objects is generalized as determining optimal linear transformation matrix for a set of observed vectors, which is first proposed by Grace Wahba in 1965~\cite{wahba1965least} as it follows. 
\textit{Given two sets of $n$ points $\{v_1, v_2, \dots v_n\}$, and $\{v_1^*, v_2^* \dots v_n^*\}$, where $n \geq 2$, find the rotation matrix $M$ (i.e., the orthogonal matrix with determinant +1) which brings the first set into the best least squares coincidence with the second. That is, find $M$ matrix which minimizes}
\begin{equation}
	\sum_{j=1}^{n} \vert v_j^* - Mv_j \vert^2
\end{equation}

Multiple solutions for the \textit{Wahba's problem} have been published, such as Paul Davenport's q-method. Some notable algorithms after Davenport's q-method were published; of that QUaternion ESTimator (QU\-EST)~\cite{shuster2012three}, Fast Optimal Attitude Matrix \-(FOAM)~\cite{markley1993attitude} and Slower Optimal Matrix Algorithm (SOMA)~\cite{markley1993attitude}, and singular value decomposition (SVD) based algorithms, such as Markley’s SVD-based method~\cite{markley1988attitude}. 

In statistical shape analysis, the linear mapping transformation determination challenge is studied as Procrustes problem. Procrustes analysis finds a transformation matrix that maps two input shapes closest possible on each other. Solutions for Procrustes problem are reviewed in~\cite{gower2004procrustes,viklands2006algorithms}. For orthogonal Procrustes problem, Wolfgang Kabsch proposed a SVD-based method~\cite{kabsch1976solution} by minimizing the root mean squared deviation of two input sets when the determinant of rotation matrix is $1$. In addition to Kabsch’s partial Procrustes superimposition (covers translation and rotation), other full Procrustes superimpositions (covers translation, uniform scaling, rotation/reflection) have been proposed~\cite{gower2004procrustes,viklands2006algorithms}. The determination of optimal linear mapping transformation matrix using different approaches of Procrustes analysis has wide range of applications, spanning from forging human hand mimics in anthropomorphic robotic hand~\cite{xu2012design}, to the assessment of two-dimensional perimeter spread models such as fire~\cite{duff2012procrustes}, and the analysis of MRI scans in brain morphology studies~\cite{martin2013correlation}.

\subsection{Our Contribution}

The present study methodizes the aforementioned mentioned cognitive susceptibilities into a cognitive-driven linear mapping transformation determination algorithm. The method leverages on mental rotation cognitive stages~\cite{johnson1990speed} which are defined as it follows
\begin{inlinelist}
	\item a mental image of the object is created
	\item object is mentally rotated until a comparison is made
	\item objects are assessed whether they are the same
	\item the decision is reported
\end{inlinelist}.
Accordingly, the proposed method creates hierarchical abstractions of shapes~\cite{greene2009briefest} with increasing level of details~\cite{konkle2010scene}. The abstractions are presented in a vector space. A graph of linear transformations is created by circular-shift permutations (i.e., rotation superimposition) of vectors. The graph is then hierarchically traversed for closest mapping linear transformation determination. 

Despite of numerous novel algorithms to calculate linear mapping transformation, such as those proposed for Procrustes analysis, the novelty of the presented method is being a cognitive-driven approach. This method augments promising discoveries on motion/object perception into a linear mapping transformation determination algorithm.




% RELATED WORK
%\vspace{-1mm}
%\section{Related Work}


\subsection{Sampling-based Motion Planning Method}

Plenty of modifications are proposed to enhance the performance of the RRT algorithm \cite{lavalle1998rapidly} such as the RRT* algorithm \cite{karaman2011sampling}.
The rewiring stage of the RRT* only rewires locally, which means the global optimization of the current tree is ignored. 
The RRT\# \cite{arslan2013use} proposes to find the global optimality in each rewiring stage with dynamic programming.
Dynamic programming is also used in the Fast Marching Tree (FMT*) method \cite{janson2015fast} to grow the searching tree.
% And the FMT* introduces the thought of batch sampling into the robot motion planning field.
The Informed sampling strategy \cite{gammell2014informed, gammell2018informed} is proposed to overcome the drawback of uniform sampling.
% It can accelerate the convergence speed significantly with very little computation consumption. 
% The Informed sampling strategy uses a direct sampling method to generate samples in the $L_2$-Informed set.
% An advanced version of the Informed sampling strategy is proposed in \cite{gammell2018informed}, which includes the graph pruning stage to keep a relatively constricted tree. 
% A tree with fewer vertices means that the cost reduction in finding the nearest tree vertex. 
Using the neural network to reinforce the sampling stage to enhance the sampling efficiency \cite{wang2020neural, li2021efficient, qureshi2019motion} is proved as a promising technique.

% In the human-robot coexisting environment, the planning problem become more complex than that in the static environment \cite{wang2020eb}.
% The objective of optimization needs to consider safety, efficiency, and human feelings.


% The Informed sampling strategy can constraint the whole planning procedure in a subset of the whole state space, the $L_2$-Informed set, and the Lebesgue measure of the $L_2$-Informed set decreases as the solution improves.
% \\ \textcolor{red}{*SWIRRT*: maybe not including the SWIRRT is better}




\subsection{Batch Sampling Technique}

% A batch sampling method is described in the FMT* method \cite{janson2015fast}.
The FMT* \cite{janson2015fast} introduces the thought of batch sampling into the robot motion planning field.
The FMT* samples a batch of points and constructs the searching tree according to this batch of samples.
% The asymptotic optimal of the FMT* is guaranteed when the size of the batch goes to infinity. 
The Batch Informed Trees (BIT*) \cite{gammell2015batch, gammell2020batch} method is developed based on the Informed RRT*, besides, the BIT* absorbs the thoughts in the FMT* method \cite{janson2015fast} and the Lifelong Planning A* (LPA*) algorithm \cite{koenig2004lifelong}.
The Regionally Accelerated Batch Informed Trees (RABIT*) \cite{choudhury2016regionally} aims to solve the difficult-to-sample planning problem, like the narrow passage problem.
% The RABIT* uses the Covariant Hamiltonian Optimization for Motion Planning (CHOMP) method as its local optimizer, and the local optimizer will exploit the local information.
The Fast-BIT* \cite{holston2017fast} modifies the edge queue and searches the initial solution more aggressively. 
The Greedy BIT* \cite{chen2021greedy} uses the greedy searching method to generate the initial solution faster and accelerate the convergence speed.
But these greedy-based methods often fail to assist the searching procedure without an accurate heuristic estimation method. 
The Adaptively Informed Trees (AIT*) \cite{strub2020adaptively} and the Advanced BIT* (ABIT*) \cite{strub2020advanced} proposed by Strub and Gammell are developed based on the BIT* as well. 
The AIT* calculate a relatively accurate heuristic estimation with a lazy reverse-searching tree.
The ABIT* proposes to utilize inflation and truncation to balance the exploitation and exploration in the increasingly complex Random Geometric Graph (RGG) \cite{penrose2003random}.
Though the AIT* and the ABIT* achieve significant improvements, their sampling regions are not compact enough, and the sampling efficiency will be critically low in the complex environment.


\subsection{Relevant Region Sampling Strategy}

The concept of 'relevant' is first proposed in the searching-based robot path planning method like the A* \cite{hart1968formal}. 
In the A* algorithm, the set of expanded vertices is relevant to the query, such that the A* algorithm could expand a smaller set of vertices than the Dijkstra's algorithm \cite{dijkstra1959note}.  
% And it is also not a new idea in the sampling-based planning field. 
% The Relevant Region is formally defined in \cite{arslan2013use}. 
% The Relevant Region related vertices are the vertices of which the sum of the optimal cost-to-come and the heuristic is less than the cost of the current optimal solution.
% Since the Relevant Region is the most promising region that could help to improve the solution, so a straightforward modification is to reduce the chance of sampling outside the  Relevant Region.
The concept of the Relevant Region is formally introduced in \cite{arslan2013use}, whose sum of the optimal cost-to-come and cost-to-go heuristic is less than the cost of the current optimal solution. 
Since the Relevant Region is the most promising area for improving the solution, a straightforward modification would be reducing the likelihood of sampling outside of it.
Three different metrics are used to achieve this in \cite{arslan2015dynamic}, the modified versions achieve better performance in the convergence speed than the RRT\# method.
The methods described in \cite{arslan2013use} and \cite{arslan2015dynamic} use the rejection method for sampling, which is not efficient since the Relevant Region is a small subset of the whole state space in most scenarios.
The direct sampling method is illustrated to overcome this drawback, and the details are described in \cite{joshi2020relevant}.
However, they all use the cumulative cost along the direct connection between the current state and the goal state as the cost-to-go. 
This approach results in inaccurate estimated cost-to-go in most scenarios.
% Their ordered priority queues are also far from the ground truth.

% The optimal cost-to-come value can be defined as the vertex which has the lowest optimal cost-to-come value in the destination region. 


\subsection{Bi-directional Searching Method}

The RRT and RRT* methods may not always discover a solution within the allotted time, particularly when dealing with narrow passages
The RRT-Connect \cite{kuffner2000rrt} is proposed to find the initial solution faster. 
% It grows two trees from the source point and the goal region simultaneously.
% It is proved that the RRT-Connect can achieve better performance than the RRT.
However, the approach described in \cite{kuffner2000rrt} is not asymptotically optimal. Therefore, its successor, RRT-Connect, is also not asymptotically optimal. 
To overcome this, an enhanced version of the bidirectional searching RRT is introduced in \cite{klemm2015rrt} to guarantee the asymptotical optimality.
To take advantage of the benefits of bi-directional search, the kinematic constraints are taken into consideration in the bi-directional search method to generate executable trajectories efficiently \cite{wang2021kinematic}.
% The method described in \cite{klemm2015rrt} is an asymptotically optimal single-query version of the RRT-Connect, called the RRT*-Connect. 
% The RRT*-Connect provides asymptotically optimal guarantee like the RRT*, and its efficiency and robustness are proofed in real-world experiments.
% In addition, the bi-directional searching method can be used to combine with the kinematic constraints , which is essential in generating executable trajectory.

% In the RRT-Connect, one tree is extended in each iteration and tries to connect itself to the other tree; this manner will attempt to grow the trees towards each other.

One drawback of the Informed RRT* \cite{gammell2014informed} \cite{gammell2018informed} is that it uses the RRT* to search the whole state space before finding the initial solution.
Therefore, the Informed RRT* often fails to find the solution in the required period, same as the RRT*.
By combining the advantages of both the Informed and the RRT*-Connect, the Informed RRT*-Connect \cite{2020Informed} proposes to use the RRT*-Connect to generate the initial solution and use the Informed sampling strategy to constrain the sampling region after the initial solution is found.
% It combines the advantages of both the Informed RRT* and the RRT*-Connect.
% The Informed RRT*-Connect can achieve a much higher success rate in its simulations than the Informed RRT*.
Besides, the AIT* \cite{strub2020adaptively} can also be viewed as a bi-directional searching method.





% SYSTEM MODE
\vspace{-1mm}
\section{System Model} \label{sec:model}
%\vspace{-1mm}

\begin{figure}[t!]
	\centering
	\includegraphics[width=0.45 \textwidth]{images/model.pdf}   
	\caption{\small vRAN over integrated fronthaul/midhaul (xHaul). It has many degrees of design freedom by possibly hosting BS functions at the CU or DUs.}
	\label{fig:vran}
	\vspace{-3mm}
\end{figure}

\textbf{Background.} In C-RAN, all BS functions are centralized at the Base Band Unit (BBU) except RF layers at the RU.  \thirdrev{In vRANs}, the BBU is decoupled into the CU and DU \cite{split_3gpp_rel16}. Hence, functions of a BS can be deployed at the CU, DU and RU. Fig \ref{fig:vran} illustrates that a CU is typically executed at a bigger and more centralized \thirdrev{server} (e.g., edge server), while a DU is at a smaller server (e.g., far-edge server) and located near (or co-located) with an RU.

Our model refers to the standardization of 3GPP \cite{split_3gpp,split_3gpp_rel16} and seminal white paper \cite{smallcell}, where each split has a different performance gain \cite{vran_murti2,function_split_survey}. \thirdrev{Although 3GPP has defined eight options for the splits, several are still hardly implemented. Therefore, we consider four splits that have been experimentally validated in a prototype \cite{costdu_nikaein,adaptive_alba}. }
 \textbf{Split 0}: All functions are at the DU, except the RF layers at the RU. It is a typical D-RAN setup. \textbf{Split 1} (PDCP-RLC): RRC, PDCP, and upper layers are hosted at the CU, while RLC, MAC, and PHY are at the DU. \secrev{This split enables a separate user plane \thirdrev{and control plane with} centralized RRC.} \textbf{Split 2} (MAC-PHY): MAC and upper layers are at the CU, while PHY is at the DU. It allows improvement for CoMP by centralized HARQ. \textbf{Split 3} (PHY-RF): All functions are at the CU, except RF layers. It is a fully centralized version of vRANs. It gains power-saving and improved joint reception CoMP with uplink PHY level combining. Going from \secrev{Split 0 to Split 3}, more functions are hosted at the CU. In addition to increasing network performance, a higher centralization level can lead to more computing cost savings \cite{vran_murti2}. However, centralizing more functions increases the data load to be transferred to the CU, going from $\lambda$ in \secrev{Split 1 to 2.5 Gbps in Split 3} for each BS, and has a stricter delay requirement. Table \ref{table:splits} summarizes vRAN split options and their requirements\footnote{\secrev{The requirements are tailored from \cite{smallcell,vranmec_andres} by following settings: 1 user per TTI, 20MHz channel bandwidth, 1 carrier component, UE IP MTU 1500 bytes, $2 \times 2$ MIMO.} }.  

%
%\begin{table}[t!] \centering
%	%\ra{1.3}
%	\begin{small}
%		\begin{tabular}{@{}lcccc@{}}\toprule
%			\textbf{}& \textbf{Traditional RAN} & \textbf{Cloud RAN}  & \textbf{Open RAN} & \textbf{Open vRAN}
%			\\ \midrule
%			\textbf{RU} &      Locked   & Locked  & Open        & Open
%			\\ \hdashline
%			{\textbf{Interface}} &  Locked & Locked & Open & Open
%			\\ \hdashline
%			{\textbf{CU/DU SW}} &  Locked   & Locked, Virtualized & Open & Open, Virtualized
%			\\ \hdashline
%			{\textbf{CU/DU HW}} &   Locked   & Open & Open & Open \\ %\midrule
%			\bottomrule
%		\end{tabular}
%	\end{small}
%	\caption{\small\textbf{RAN transformation}. \textit{Locked} means that is a solely property of a single vendor. \textit{Open} means that it allows interoperable to flexibly work with different vendors. }
%	\label{table:ran_transform}
%	\vspace{-5mm}
%\end{table}

\begin{table}[t] \centering
	%\ra{1.3}
	\begin{small}
		\begin{tabular}{@{}lll@{}}\toprule
			\textbf{}& \textbf{Flow (Mbps)} & \textbf{Delay Req. (ms)}  
			\\ \midrule
			{Split 0 \secrev{(S0)} } &      $\lambda$   & $30$          
			\\ \hdashline
			{Split 1 \secrev{(S1)} } &  {$\lambda$} & $30$ 
			\\ \hdashline
			{Split 2 \secrev{(S2)} } &  {$1.02\lambda+1.5$}   & $2$
			\\ \hdashline
			{Split 3 \secrev{(S3)} } &   {$2500$}   & $0.25$ \\ %\midrule
			\bottomrule
		\end{tabular}
	\end{small}
	\caption{\small Data and delay requirements of vRAN split when the traffic load is $\lambda$ Mbps \cite{smallcell, vranmec_andres}.}
	\label{table:splits}
	\vspace{-3mm}
\end{table}


\textbf{RAN}. We model a vRAN architecture with a graph $G = (\mathcal{I}, \mathcal{E})$ where $\mathcal{I}$ has a subsets $\mathcal{N}$ of the $N=|\mathcal{N}|$ DUs, \secrev{$\mathcal{L}$ of the $L=|\mathcal{L}|$ routers} and a CU (index $0$). Each node is connected through a link of $(i,j)$ with a set $\mathcal{E}$ of links and has capacity $c_{ij}$ (Mbps) each. The DU-$n$ is connected to $\{0\}$ with a single path (e.g., shortest path) $p_{n0}$; hence, we define $r_{p_{n0}}$ as the amount of data flow (Mbps) to be transferred and routed through a path $p_{n0} \! := \!  \{(n,i_1), ..., (i_k,0) \! : \! (i,j) \! \in \! \mathcal{E} \}$. The BS functions are deployed in servers using virtual machines (VMs)\footnote{\secrev{Each BS
	function can operate as a virtual network function (VNF), and the VNFs can be executed on top of a single VM or multiple VMs}}. Each server has a processing capacity, i.e., $H_n$ for DU-$n$ and $H_0$ for CU. Naturally, a central server has a higher computing performance and capacity, \secrev{hence} $H_0 \! \geq \! H_n$. \secrev{We define} $\rho_{o}^c $ and $ \rho_{o}^d$ as the incurred computational load (cycle/Mb/s) in results of deploying the split configuration $o \! \in \! \{0,1,2,3\}$ at each CU and DU, respectively. %\secrev{The processing load at the CU is generally lower than at the DU while serving the same traffic load \cite{costdu_nikaein}.}


\textbf{Demand $\&$ Cost}. We focus on the uplink transmission where $\lambda_{n} \geq 0$ (Mbps) is the aggregate data flow of DU-$n$ to serve the users traffic;
hence, there are $N$ different flows in the network. We denote $ \bm{\alpha} = (\alpha_n, n \in \mathcal{N})$ and $\bm{\beta} \!=\! (\beta_n, n \in \mathcal{N})$ as the VM instantiation cost (monetary units) and the computing cost (monetary units/cycle) at the DUs, respectively, while $\alpha_0$ and $\beta_0$ are the respective cost for the CU. We also have a routing cost $\zeta_{p_{n0}}$ (monetary units/Mbps) for each path $p_{n0}$. This cost arises from the network links being leased from third parties or maintaining the links. 

\textbf{Problem Statement.} We have four choices of the \secrev{splits} for each BS in vRANs. What is the best-deployed split for each BS that minimizes the total network cost? The decision leads to interesting problems. Each \secrev{split} generates a different DU-CU data flow and has a distinct delay requirement. Executing more functions at the CU is more efficient in computing cost; however, it produces a higher load for xHaul links. \secrev{The BSs share the same capacitated servers and network links, where each split decision is interdependent. Moreover, the behaviour of the vRAN system (e.g., resources, performance) is complex and highly non-trivial, which makes complete assumptions of the model can be unfeasible or inaccurate. The goal is to design a framework to solve this problem by taking minimal assumptions about the model of the system.}







% DESIGN
\vspace{-2mm}
\section{Formalization of vRAN Split Problem} \label{sec:problem}
\vspace{-1mm}
The BS functions can be deployed at the DUs or CU \secrev{depending on} the splits, as seen in Table \ref{table:splits}. \secrev{Each split} must respect to the \emph{chain of functions} $f_0 \!\rightarrow\! f_1 \!\rightarrow\! f_2 \!\rightarrow\! f_3$\footnote{\secrev{$f_0$ is a function that encapsulates RF layers. Then, $f_1, f_2$ and $f_3$ are the functions for Layer 1 (PHY), Layer 2 (MAC, RLC) and Layer 3 (PDCP, RRC and the upper layers), respectively. }}. Thus, we define $x_{on} \in \{0,1\}$ as the decision for deploying split $o \in \{0,1,2,3\}$ at DU-$n$. For instance, $x_{0n} = 1$ is for deploying $f_0, f_1, f_2, f_3$ (Split 0); $x_{1n} = 1$ for $f_0, f_1, f_2$ (Split 1); $x_{2n}= 1$ for $f_0, f_1$ (Split 2); or $x_{3n}= 1$ for $f_0$ (Split 3) at DU-$n$. 
%
%
We only deploy a single split configuration for each BS. Therefore, \secrev{a} set of eligible \secrev{splits} is:
%
%\vspace{-1mm}
\begin{align} \label{eq:setx}
\mathcal{X} =  &\Bigl\{  \bm{x}_n  \in \{0,1\} \Big| \sum_{o=0}^3 x_{on}=1 , 
\ \ \forall n \in \mathcal{N}  \Bigr\} , 
\end{align}
%
where $\bm{x}_n = ( x_{on}, \forall o  )$ and $\bm{x} = (\bm{x}_n, \forall n)$. The BS functions,  $f_1, f_2$ and $f_3$, are deployed \secrev{using} VMs at each server. We have computational processing at the CU and DU-$n$ that must respect \secrev{to} its capacity as:
%
%\vspace{-1mm}
\begin{align} \label{eq:computing1}
\sum_{n \in \mathcal{N}} \lambda_{n} \sum_{o =0}^3    x_{on} \rho^{c}_o \leq \ H_0, 
%
\end{align}
%\vspace{-1mm}
%
%
%\vspace{-1mm}
\begin{align} \label{eq:computing2} \lambda_{n} \sum_{o =0}^3   x_{on} \rho^{d}_o \leq H_n, \ \forall n \in \mathcal{N}. 
%
\end{align}
%
%\textit{We consider the realistic scenario of CU processing where the parameters of the servers change over time in unknown fashion.} Each CU is shared to many BS functions (resource pooling); so, it has an \textit{average capacity} affecting the processing performance, i.e., when the load exceed average capacity ($\bar{H}_0$), the processing delay will increase and lead to the system non-responsive. This condition is generated by random process $\{ H_0^t\}_{t=1}^{\infty}$ with $ \lim_{T \to \infty} \frac{1}{T} \sum_{t=1}^{T}H_0^t = \bar{H}_0, \forall t.$ However, our decision only has information on the instantaneous value in each time slot. In here, we only need this pertubation is bounded at each time slot ($ H_0^t \leq H^{\text{max}}_0$) and the average converges to some finite value ($\bar{H}_0$)

\textbf{Data Flow $\&$ Delay.}  Let define $r_{p_{n0}}$ (Mbps) as the amount of data flow (Mbps) to be transferred through a path $p_{n0}$. Hence, the flow must respect \secrev{capacity of each link}:
%
%\vspace{-1mm}
\begin{align} \label{eq:route1}
\sum_{n \in \mathcal{N}} r_{p_{n0}} I^{ij}_{p_{n0}} \leq c_{ij}, \ \ \forall (i,j) \in \mathcal{E},
\end{align}
%
where $I^{ij}_{p_{n0}} \in \{ 0,1 \}$ indicating whether the link $(i,j)$ is used by path $p_{n0}$. Assuming a single path (e.g., shortest path), the amount of data flow depending on each split configuration is \cite{fluidran_andres}:
%
%\vspace{-1mm}
\begin{align} \label{eq:route2}
r_{p_{n0}} \!=\! \lambda_{n} (x_{0n} + x_{1n}) + x_{2n} (1.02 \lambda_{n} + 1.5) + 2500 x_{3n}.
\end{align}
%
We let $d_{p_{n0}}$ denote the incurred delay for routing through path $p_{n0}$ from DU-$n$ to the CU. Each split has to satisfy the respective delay requirement (Table \ref{table:splits}):
%
%\vspace{-1mm}
\begin{equation} \label{eq:delay}
x_{on} d_{p_{n0}} \leq d_o^{\text{max}}, \ \ \forall o, \forall n \in \mathcal{N}.
\end{equation}
%
\subsection{Objective Function}
We aim to minimize the total network cost consisting of the computational \secrev{costs at the DUs and CU} and the routing cost\footnote{In this case, we follow the linear objective cost function similar to the previous studies \cite{fluidran_andres,vranmec_andres}. However, our solution approach does not restrict only to the linear objective. Our approach \secrev{relies on} the \secrev{scalar} reward and penalty as feedback; hence, it can also be tailored to a non-linear objective.}. The needs of computing cost \secrev{for each BS-$n$} at DU-$n$ is:
%
\begin{align} \label{eq:cost-du}
V_n(\bm{x}_n) = \alpha_n + \beta_n \lambda_{n} \sum_{o=0}^{3}\rho_{o}^{d} x_{on}.
\end{align} 
%
We also have \secrev{the required} computing cost \secrev{of BS-$n$} at the CU:
%
\begin{align} \label{eq:cost-cu}
\secrev{V_{n0}(\bm{x}_n)} = \sum_{o=0}^3 x_{on} (\alpha_0 + \lambda_{n} \beta_0 \rho_{o}^{c} ).
\end{align} 
\secrev{The first terms in \eqref{eq:cost-du} and $\eqref{eq:cost-cu}$ represent the required instantiating cost at each DU and CU for BS-$n$. The last terms in \eqref{eq:cost-du} and \eqref{eq:cost-cu} are the required data processing cost by each DU and CU to serve BS-$n$ load.}
%
Next, we have the cost to route the data flow from DU-$n$ to the CU:
%
\begin{align} \label{eq:cost-route}
U_{n0}(\bm{x}_n) =  \zeta_{p_{n0}} r_n (\bm{x}).
\end{align} 
%
Finally, we have the total vRAN cost as:
%
\begin{align} \label{eq: total-cost}
J(\bm{x}) &= \sum_{n \in \mathcal{N}} \Big(  V_n(\bm{x}_n) + U_{n0}(\bm{x}_n) +\secrev{ V_{n0}(\bm{x}_n}) \Big), \end{align}
which leads to the following problem:
%
\begin{align}
\mathbb{P}: \,\,\,\, & \underset{\bm{x} \in \mathcal{X}}{\text{minimize}} \   J(\bm{x}), \ \ \ \notag  \text{s.t} \ \  
 (\ref{eq:computing1}) - (\ref{eq:delay}). \notag
\end{align}
%
$\mathbb{P}$ is a combinatorial problem to decide the function placement $\bm{x}$ for all the BSs and serve the traffic load $\bm{\lambda}$ with DU-CU path $p_{n0}$ for each BS-$n$ in the network graph $G = (\mathcal{I}, \mathcal{E})$. Next, we discuss the complexity of $\mathbb{P}$.

\subsection{Complexity Analysis}

The complexity of $\mathbb{P}$ can be identified from the polynomial reduction of \textit{multiple-choice multidimensional knapsack problem (MMKP).} 

\textbf{MMKP.} Let suppose there are $N$ items with values ${v_1, v_2, ..., v_N}$. We also have $r_1, r_2, ..., r_N$ correspond to the required resources to pick the items. In the 0-1 knapsack problem (KP), the aim is to pick the items $x_i \in \{0, 1\}, \secrev{\forall i}$ that maximize the total value $\sum_{i=1}^N  x_i v_i$, subject to constraint \secrev{$\sum_{i=1}^N x_i r_i \leq R$}. This is a well-known NP hard problem and there is a pseudo-polynomial algorithm using a dynamic programming concept that has complexity $\mathcal{O}(NR)$ \cite{mmkp_convexhull}.  MKKP is a variant of 0-1 KP where there are $M$ groups of items, e.g., group $i$ has $l_i$ items. Each item has a specific value $v_{ij}$ corresponds to $j$-th item of $i$-th group and needs $K$ resources. Hence, each item in a group has a resource vector $\bm{r}_{ij} = (r_{ij1}, ..., r_{ijK} )$ and $\bm{R} = (R_1, ..., R_K)$ is the resource bound of the knapsack. The aim is to exactly pick one item from each group, e.g., $ \sum_{j=1}^{l_i} x_{ij} = 1, x_{ij} \in \{0,1\}$ that maximizes the total value: $\sum_{i=1}^{M} \sum_{j=1}^{l_i} x_{ij} v_{ij}$, subject to the resource constraint: $\sum_{i=1}^{M} \sum_{j=1}^{l_i} x_{ij} r_{ijk} \leq R_k, k = 1,...,K$. 

Finding an exact solution for MMKP is also NP-hard \cite{mmkp_convexhull}. It is also worth noting that the search space for solution in MMKP is smaller than other KP variants; hence, exact solution is not implementable in many practical problems as there is more limitation of picking items from a group in MMKP instance \cite{mmkp_convexhull}. Next, We prove that $\mathbb{P}$ is harder than MMKP. 

\noindent
\begin{theorem} \label{theo:mmkp}
	\textit{MKKP can be reduced to $\mathbb{P}$ in polynomial time, e.g., MMKP $\leq_P \mathbb{P}$ }.
\end{theorem}  

\noindent
\textit{Proof.} Let suppose we have unlimited link capacity, no routing cost and no delay requirement. Hence, all paths of \secrev{the} DU-CU pair are eligible and \eqref{eq:route1}-\eqref{eq:delay} are always satisfied. This problem then can be mapped to MMKP by setting: 1) $M$ groups to $N$ BSs, 2) each $i$-th group with $l_i$ items to each BS-$n$ with $|o|=4 $ of split options, 3) $j$-th item of $i$-th group to the split $o_n$ of BS-$n$, 4) $r_{ij}$ to the \secrev{incurred} computing loads, e.g., $\lambda_{n}\rho_{i}^c$ and $\lambda_{n}\rho_{i}^d$, and 5) the knapsack constraints to computing constraints $H_n$ and $H_0$. The value $v_{ij}$ of item-$j$ in group-$i$ also can be mapped with the costs (e.g., computing and routing) of deploying split-$o$ of BS-$n$, where the MMKP is a maximization problem and $\mathbb{P}$ is a minimization problem.  We can see the reduction is of polynomial time: we select the functional split for every BS correspond to that we activate an item that we pick to a knapsack in each group. Therefore, we can conclude that if we can solve $\mathbb{P}$ in polynomial time we also can solve any MMKP problem. 

%
%\begin{table}[t]
%%\begin{tabular}{cp{5cm}} \hline
%\begin{small}
%\begin{tabular}{cp{5cm}} \hline
%\textbf{Notation} & \textbf{Definition} \\ \hline
%$\mathcal{N}, N, n$ & Set of DUs, number of DUs, DU index \\ 
%$\mathcal{M}, M, m$ & Set of CUs, number of CUs, CU index \\ 
%$\mathcal{P}_{nm}$ & Set of paths between DU $n$ and CU $m$\\ 
%$\mathcal{P}_{m}$ & Set of paths between all DUs and CU $m$\\ 
%$c_{ij}$ & Bandwidth in Mbps of the link between nodes $i$ and $j$ \\ 
%$d_{p_k}$ & End-to-end delay of the path $p_k$ in seconds \\ 
%$\lambda_{n}$ & Traffic flow from DU-$n$ in Mbps \\ 
%$H_{n}, H_{m}$ & Processing capacity (cycles/sec) of DU-$n$ and CU-$m$\\ 
%$\rho_1, \rho_2, \rho_3$ & Processing load (cycles) per Mbps of $f_1$, $f_2$, $f_3$\\ 
%$a_{m}, b_{m}$ &  Cost of VM instantiation and computing at CU $m$\\ 
%$\alpha_{n}, \beta_{n}$ &  Cost of instantiation and computing at DU $n$\\ 
%$\zeta_k$ & Routing cost of path $k$ (monetary units per Mbps) \\
%$c_d$ & Routing cost per kilometer per Mbps \\ \hline
%\end{tabular}
%\end{small}
%\centering
%
%\bigskip
%\caption{Notation table}
%\label{tab:notation_table}
%\end{table}

% SOLUTION ALGORITHM
\vspace{-2mm}
\section{Constrained Deep Reinforcement based Functional Split Optimization Framework} \label{sec:solution}
%\vspace{-1mm}

\secrev{We leverage a constrained deep RL paradigm to solve our vRAN splitting problem by treating the vRAN system as a black-box environment, which makes minimal assumptions about the underlying system. Consequently, our RL agent does not need to know the information about the formulation in \eqref{eq:setx}-\eqref{eq: total-cost} to decide the splits. Our agent relies on the scalar reward and penalization returned from the environment to assess the quality of the solutions.}
%
%Our vRAN splitting problem also comprises combinatorially large discrete action spaces as the split configuration should simultaneously be deployed to multiple BSs. Therefore, we propose a CDRS framework to follow NCO with a deep RL paradigm to handle this challenge \cite{neural_bello,solozabal_constrained,vnf_drl_solozabal}
%
%
\secrev{At each episode, our agent observes a \textit{state} of incoming a sequence of all BS functions drawn from the \textit{environment} of vRANs, takes an \textit{action} to decide the splits for all the BSs, and expects to receive feedback signals of the \textit{reward} (total network cost) and \textit{penalization} (for violating the constraints). Our state comprises of a sequence information of BS functions: $\mathcal{F} = \{ \mathcal{F}_n \}_{n=1}^{N}$,  where $\mathcal{F}_n$ is a set of BS-$n$ functions. Given the input state, our agent assigns $\mathcal{O} \!\!=\!\! \{o_n \! \in \! \{0,1,2,3\}, \forall n \in \mathcal{N} \}$ as a set of selected splits for all the BSs, which decides the placement of BS functions at the CU or DUs. Our objective is to minimize the total network cost while enforcing the constraint requirements. Given the selected splits, our agent expects to receive scalar values from the environment consisting of: \textit{i)} $J(\mathcal{O}|\mathcal{F})$, the total induced cost and \textit{ii)} $\xi(\mathcal{O}|\mathcal{F})$, the weighted sum of penalization. Further, we consider a particular RL algorithm using one-step constrained policy optimization and neural network architecture, where the interactions are narrowed to a single time step at every episode, and our agent learns iteratively over episodes.}

%CDRS follows the deep RL paradigm based on the constrained policy optimization to solve \secrev{the split problem in vRANs through} a Policy Gradient \cite{rl_sutton} with Lagrangian relaxation method \cite{Bertsekas}. In addition to the reward (cost) signal, we also give penalization for every constraint violation that guides the policy to constraint satisfaction. In this sense, the penalty coefficient (Lagrangian multiplier) setting is a multi-objective problem where there is a different optimum solution for each configuration. To this end, we follow two penalty coefficient updates: 1) CDRS-Fixed that uses a fixed penalty coefficient (reward shaping) \cite{vnf_drl_solozabal, cmpd_solo} and 2) CDRS-Ada that applies an adaptive penalty coefficient (updated in the ascent direction)  \cite{pdo_risk}.


\secrev{Our goal is to design a stochastic policy $\pi_\theta(\mathcal{O}|\mathcal{F})$ parameterized by a neural network with weights $\theta$ to predict the splits for all the BSs to minimize the total cost while satisfying constraint requirements. However, we have the $N$ BSs that need to deploy the splits together, where each has four possible split options. Each split decision is also interdependent as the BSs share the same network links and computing servers. Consequently, our problem has a combinatorially large discrete action space with a total of $4^N$ possible actions. Such a curse dimensionality in high dimensional spaces can be avoided by modelling complicated joint probability distributions using the chain rule decomposition. Therefore, we design our policy based on a chain rule by factorizing the output probability, parameterized by a neural network with weights $\theta$ as:
	%
	\begin{align}
		\pi_\theta(\mathcal{O}| \mathcal{F}) = \prod_{n=1}^{N} \pi_\theta(o_n| o_{(<n)}, \mathcal{F}_n). 
	\end{align}
	%
	%
	This policy strategy assigns a higher probability to the splits for having a lower cost and vice versa for every BS, which also can be represented by individual softmax modules (e.g., at the output layer). Motivated by \cite{seq2seq, attention_bahdanau} that uses neural networks \thirdrev{to estimate} the same factorization \thirdrev{of our stochastic policy} for machine translation, we design our policy network using an encoder-decoder sequence-to-sequence model based on LSTM networks. Our policy network architecture, which also utilizes an attention mechanism, captures the dependency and correlation between split decisions. This architecture allows our policy to read input information from all BS functions, then maps them into split selections for all the BSs.
	%
	%We provide the summarize and illustration of CDRS training operation in Algorithm 1 and Fig xxx, respectively. 
	In the training, we use a batch of $B$ i.i.d samples on the stochastic policy to select the splits and generate several pretraining models. In the test, we perform an inference through a search strategy by greedy decoding or temperature sampling.%The detail of CDRS operation is discussed as follows. 
	} 

%
%In supervised learning, the performance of the model depends on the quality of the supervised labels, where finding high-quality labelled data (e.g., optimal label) is expensive and may not be possible. Therefore, we follow NCO framework with deep RL paradigm \cite{neural_bello,solozabal_constrained,vnf_drl_solozabal}. We define \textit{environment} as a vRAN system consisting of CU, DUs and network links. The \textit{states} is defined as a sequence of all BS functions. Our RL \textit{agent} generates \textit{action} that corresponds to a set of decisions to choose the functional split configuration for every BS. This action decides which functions are deployed at the DU and CU for every BS. The environment will evaluate for every action taken and give a feedback signal from the environment. This feedback consists of \textit{reward} (the total network cost) and \textit{penalization} (constraint violation). As opposed to supervised learning, RL can deliver an appropriate paradigm for training and updating the neural network parameters to improve the performance of the model for the functional split problem. It simply relies on the reward and penalty feedbacks (interaction with the environment) instead of the high-quality labelled data.  Since our problem is combinatorial that has highly dimensional action space, we leverage model-free policy-based RL that optimizes the weight parameter $\theta$ that infers a policy strategy to deploy the split configuration. It is worth noting that we have a constrained environment; hence, we also consider a constraint relaxation technique in the cost function of our policy-based method to deal with constraint dissatisfaction. 
%%Fig xxx depics the general RL scheme for our problem.
%
%Our approach follows the deep RL paradigm based on the constrained policy optimization and neural network architecture to solve the functional split problem in the vRAN. We utilize Policy Gradient \cite{rl_sutton} with Lagrangian relaxation method \cite{Bertsekas}. In addition to the reward (cost) signal, we also give penalization for every constraint violation that guides the policy to constraint satisfaction. In this sense, the penalty coefficient (Lagrangian multiplier) setting is a multi-objective problem where there is a different optimum solution for each configuration. To this end, we follow two penalty coefficient updates: 1) CDRS-Fixed that uses a fixed penalty coefficient (reward shaping) \cite{vnf_drl_solozabal, cmpd_solo} and 2) CDRS-Ada that applies an adaptive penalty coefficient (updated in the ascent direction)  \cite{pdo_risk}. A neural network architecture formed by an encoder-decoder sequence-to-sequence model \cite{seq2seq,attention_bahdanau} based on stacked LSTM is also utilized that approximate the stochastic policy over the solution. 
%
%Our agent receives input of set of BS functions $\mathcal{F} = \{ \mathcal{F}_n \}_{n=1}^{N} \!\!$,  where $\mathcal{F}_n \!=\! \{f_0,f_1,f_2,f_3 \}$ is a set of functions for BS-$n$. In the output, we expect to receive $\mathcal{O} \!\!=\!\! \{o_n \}_{n=1}^{N}$ as a set of selected configuration for all BSs. It addresses the split configuration of BS-$n$ with $o_n \! \in \! \{0,1,2,3\}$. 
%%In relation to Eq. \eqref{eq:setx}, $o_n$ activates the respected split configuration variables for BS-$n$ following indicator function: $x_{0n} \!=\! \mathbbm{1}_{(o_n \!= 0)}, x_{1n} \!=\! \mathbbm{1}_{(o_n \!= 1)}, x_{2n} \!=\! \mathbbm{1}_{(o_n \!= 2)},$ and $ x_{3n} \!=\! \mathbbm{1}_{(o_n \!= 3)} $.  
%We also use the neural network with weight parameter $\theta$ that infers a policy strategy $\pi_\theta(\mathcal{O}| \mathcal{F}, \theta)$ to deploy the split configuration. 
%
%
%In practice, our approach does not have to know the defined problem in Section \ref{sec:problem}. Our agent interacts with the environment expecting to receive a reward (network cost) and penalization (constraints violation); then learn from this interaction to find the optimal solution. At the test time, we perform an inference process through search strategies by a greedy decoding or temperature sampling method. 


  
\vspace{-2mm}
\subsection{\secrev{Policy} Network Architecture}
%\vspace{-1mm}
%Our system is bounded under computational and link capacity, where each BS has distinct network parameters. Hence, the BS input sequence (to which BS needs to decide first) affects the solution. To this end, 
Our policy network infers a strategy to deploy the splits for all the BSs, given a sequence information of BS functions as an input $\mathcal{F} = \{\mathcal{F}_1, ...., \mathcal{F}_N \}$. It is constructed from an encoder decoder sequence-to-sequence model with an attention mechanism based on LSTM networks \cite{seq2seq,attention_bahdanau}.
\secrev{We also consider a batch training by drawing} a batch of $B$ i.i.d samples with different sequence order \secrev{to encourage the exploration further}. 

\secrev{\textbf{LSTM structure.} We leverage LSTM networks, a particular RNN architecture \cite{lstm},  to construct  our sequence-to-sequence model that maps the input BS functions into split decisions for all the BSs.
	An LSTM cell has three main structures comprising of: \textit{(i)} a forget gate that receives the cell state input and learns how long should memorize or forget from the past; \textit{(ii)} an input gate that aggregates the current input and the output of past steps, then feeds them to the activation function; and \textit{(iii)} an output gate that provides the LSTM output from the combination of current cell state and the output of input gate. The relationship of these blocks can be expressed as:
	\begin{align}
		&\bm{\hat{f}}_n = \sigma \big( W_f \big[\bm{h}_{n-1}^T, \bm{s}_{n}^T  \big]^T + \bm{b}_f \big), \\
		%
		&\bm{\hat{r}}_n = \sigma \big( W_r \big[\bm{h}_{n-1}^T, \bm{s}_{n}^T   \big]^T + \bm{b}_r \big), \\
		%
		&\bm{\tilde{c}}_n = \tanh \big( W_c \big[\bm{h}_{n-1}^T, \bm{s}_{n}^T   \big]^T + \bm{b}_c  \big), \\
		%
		&\bm{\hat{c}}_n = \bm{\hat{f}}_n *  \bm{\hat{c}}_{n-1} + \bm{\hat{r}}_n  * \bm{\tilde{c}}_n, \\
		%
		%
		&\bm{\hat{o}}_n = \sigma \big( W_o \big[\bm{h}_{n-1}^T, \bm{s}_{n}^T   \big]^T + \bm{b}_o  \big), \\
		& \bm{h}_n = \bm{\hat{o}}_n * \tanh(\bm{\hat{c}}_n),
	\end{align}
	where function $\sigma(x) \triangleq \frac{1}{1 + \exp(-x)} $ is the sigmoid function and symbol $*$ is element-wise multiplication. The weight and bias matrices for the respective forget, input and output gates of the LSTM cell are represented by $W_f, W_r, W_c, W_o$  and $\bm{b}_f, \bm{b}_r, \bm{b}_c, \bm{b}_o$.
	%
	%
	Multiple LSTM layers can be further stacked one on top of another (a stacked LSTM) to create a deeper model, which may obtain more accurate prediction. Each LSTM cell reads an input of embedding vector representation $\bm{s}_n \in [ -1,1]^E$ translated from each input $\mathcal{F}_n$, where $E$ is the embedding size.
	The structure of an LSTM cell is illustrated in Fig \ref{fig:lstm} and utilized to construct our sequence-to-sequence model.}
	%
	\begin{figure}[t!] 
		\centering
		\includegraphics[width=0.45 \textwidth]{images/lstm.pdf}   
		\caption{\small \secrev{A generic architecture of an LSTM cell.}  } 
		\label{fig:lstm}
		\vspace{-3mm}
	\end{figure}
	%
%Fig. \ref{fig:neural} illustrates the overall architecture of our policy network.
%
%Additionally, \secrev{the attention mechanism in our policy network allows to capture the dependency and correlation between each split decision, such as the shared network link and computing resources and the routing and computational weighting costs.}
%The architecture is depicted in Fig. \ref{fig:nn}. \fm{need more rephrasing}

\secrev{\textbf{Policy Network.}} Our policy network is built from an encoder-decoder sequence-to-sequence model based on LSTM networks. \secrev{One main drawback of vanilla sequence model is generally unable to learn accurately long sequence. Therefore, the vanilla model may not be able learn our problem with large number of BSs.
An attention mechanism comes to address this issue as it considers all the hidden state from all input sequences.} The encoder read the entire input sequence to a fixed-length vector. The decoder decides \secrev{the deployed split of each BS} at each step from an output function based on its own previous state combined with an attention over the encoder hidden states \cite{attention_bahdanau}. The decoder network hidden state is defined with a function: $\bm{h}_t = f(\bm{h}_{t-1}, \bm{\bar{h}}_{t-1}, \bm{c}_t)$, \secrev{where $\bm{c}_t$ and $\bar{\bm{h}}_{t}$ are the context vector and the source hidden state at time step $t$}. 
Our model derives the context vector $\bm{c}_t$ that captures relevant source information that helps to predict the splits. The main idea is to use \secrev{an attention mechanism}, where the context vector $\bm{c}_t$ takes consideration of all the hidden states of the encoder and the alignment vector $\bm{a}_{t}$: 
\begin{align}
	\bm{c}_t = \sum_{k \in \mathcal{N}} \bm{a}_{tk} \bm{\bar{h}}_k.
\end{align}
%
% 
Note that the alignment vector has an equal size to the number of steps in the source side, which can be calculated by comparing the current target hidden state of decoder $\bm{h}_t$ with each source hidden state $\bm{\bar{h}}_k$ as:
\begin{align} \label{eq:softmax}
	 \bm{a}_{tk} = \frac{\exp(\text{score}(\bm{h}_t,\bm{\bar{h}}_k))}{\sum_{k'=1}^{N} \exp(\text{score}(\bm{h}_t,\bm{\bar{h}}_k')))}
\end{align}
This \secrev{alignment} model gives a score $\bm{a}_{tk}$ which describes how well the pair of input at position $k$ and the output at position $t$. The \secrev{alignment} score is parameterized by a feed-forward network where the network is trained jointly with the other models \cite{attention_bahdanau}. The score function is defined by a non-linear activation function following Bahdanau's additive style:
\begin{align}\label{eq:score}
	\text{score}(\bm{h}_t,\bm{\bar{h}}_k) = \bm{v}_a^{\top} (\tanh(\bm{w}_1 \bm{h}_t +  \bm{w}_2 \bm{\bar{h}}_k )),
\end{align}
%
%
where $\bm{v}_a^{\top} \in \mathbb{R}^{n}, \bm{w}_1 \!\in\! \mathbb{R}^{n \times n} $ and $\bm{w}_2 \!\in\! \mathbb{R}^{n \times n}$ are \secrev{defined as the weight matrices to be learned in the alignment model, and $n$ is the size of hidden layers}. The overall architecture of our policy network is illustrated in Fig. \ref{fig:neural}.
%
\begin{figure}[t!] 
	\centering
	\includegraphics[width=0.49 \textwidth]{images/NN-v2.pdf}   
	\caption{\small \secrev{\textbf{Policy Network.} CDRS utilizes a neural network architecture to approximate the stochastic policy over the solution. It is constructed by an encoder-decoder sequence-to-sequence model with attention mechanism based on LSTM networks.} }
	\label{fig:neural}
	\vspace{-3mm}
\end{figure}
%
%
\vspace{-2mm}
\subsection{Constrained Policy Gradient with Baseline}
%\vspace{-1mm}
%
%
%
%
%
%
\secrev{We train the above neural network model using a constrained policy gradient method with a self competing baseline.}
We define the objective of $\mathbb{P}$ as an expected reward that is obtained for every vector of weights $\theta$. Hence, the expected cost $J$ in associated with the selected split $o_n$ given BS-$n$ functions \secrev{is denoted as}:
%
\begin{align}
	J^\pi(\theta|\mathcal{F}_n) = \underset{o_n \sim \pi(.|\mathcal{F}_n) }{\mathbb{E}} [ J(o_n) ],
\end{align}
%
and we have the expected of total cost from all BSs:
%
\begin{align} \label{eq:total_cost_theta}
J^\pi(\theta) = \underset{o_n \sim \mathcal{O} }{\mathbb{E}} [ J(\theta|\mathcal{O}) ].
\end{align}
%
The vRAN system has constraints of delay requirement and computational and link capacity. 
%
%we put these to the constraint disatisfication that associated with our policy with: 
%\begin{align} \label{eq:disatisfication}
%J_C^\pi(\theta) = \underset{o_n \sim \mathcal{O} }{\mathbb{E}} [ J(\theta|\mathcal{O}) ]. 
%\end{align}
%
Therefore, our original problem turns to a primal problem as:
%
\begin{align} 
\mathbb{P}_{1\text{P}}: \ \underset{\pi \sim \Pi }{\text{min}} \  J^\pi(\theta); \ \ \text{s.t.} \ \secrev{J_{C_i}^\pi(\theta) \leq 0, \forall i}, \notag
\end{align}
%
where \secrev{we define $J_{C}^\pi(\theta) = \big(J_{C_{i}}^\pi(\theta), \forall i \big)$} as a function of constraint dissatisfaction to capture the penalization that the environment returns for violating each $i$ constraint requirement, e.g., computing, link, delay. 
In this problem, we consider parametrized stochastic policy using a neural network. In order to ensure the convergence of our policy to constraint \secrev{satisfaction}, we follow \cite{reward_constraint} and make assumptions:
%
\begin{assumption} \label{assumption:cost}
	$J^\pi$ is bounded for all policies $\pi \in \Pi$.
\end{assumption}
\begin{assumption} \label{assumption:localminima}
	Each local minima of $J_{C}^\pi(\theta)$  is a feasible solution.
\end{assumption}
\noindent
Assumption \ref{assumption:localminima} describes that any local minima $\pi_\theta$ satisfies all constraints, e.g., $J_{C_i}^\pi(\theta) \leq 0, \forall i$. It is the minimal requirement that guarantees the convergence of a gradient algorithm to a feasible solution. The stricter assumptions, e.g., convexity, may guarantee the optimal solution.
%

%%%%%%%% DUAL FUNCTION START HERE %%%%%%%%%%%%%%%
Next, we reformulate $\mathbb{P}_{1\text{P}}$ to unconstraied problem with Lagrange relaxation method \cite{Bertsekas}. The penalty signal is also included aside \secrev{from the} original objective for infeasibility, which leads to a sub-optimality for infeasible solutions. Given $\mathbb{P}_{\text{1P}}$, we have the dual function:
%
\begin{align} 
	\label{eq:dual_function}
	g({\mu}) = \underset{\theta }{\text{min}} \  J_L^\pi({\mu},\theta) &= \underset{\theta }{\text{min}} \  J^\pi(\theta) + \sum_{i} \mu_i J_{C_{i}}^\pi(\theta) \notag \\
	&=\underset{\theta }{\text{min}} \  J^\pi(\theta) + J_\zeta^\pi(\xi),
\end{align}
where $\mu \!=\! (\mu_i, \forall i), J_L^\pi({\mu},\theta)$ and $J_\zeta^\pi(\xi)$ are the  penalty coefficients (Lagrange multipliers), Lagrange objective function and the expected penalization, respectively. Then, we define the dual problem:
%
\begin{align}
\mathbb{P}_{1\text{D}}: \ \underset{{\mu} }{\text{max}} \  g({\mu}). \notag
\end{align}
%
%
$\mathbb{P}_{1\text{D}}$ aims to find a local optima or a saddle point $(\theta({\mu}^*), {\mu}^*)$, which is a feasible solution. The feasible solution is a solution that satisfies:  $J_{C_{i}}^\pi(\theta) \leq 0, \forall i$. 
%
% 
%
To compute the weights $\theta$ that optimize the objective, we use Monte-Carlo policy gradient and stochastic gradient descent by the following update:
\begin{align}\label{eq:update1}
	\theta_{k+1} = \theta_{k} - \eta_a(k) 	 \nabla_\theta J_L^\pi({\mu},\theta), 
\end{align}
%
where $\eta_a(k) $ is the step-size. The gradient $\nabla_\theta J_L^\pi({\mu},\theta)$  with regards to weights $\theta$ can be calculated using a log-likelihood method as:
%
%
\begin{align}
\nabla_\theta J_L^\pi(\theta) = \underset{\mathcal{O} \sim \pi_\theta(.|\mathcal{F}) }{\mathbb{E}} [ L(\mathcal{O}|\mathcal{F}) \ \nabla_\theta \log \pi_\theta(\mathcal{O}|\mathcal{F}) ].
\end{align} 
%
%
$L(\mathcal{O}|\mathcal{F})$ represents the total cost with penalization obtained from: 
	 $L(\mathcal{O}|\mathcal{F}) = J(\mathcal{O}|\mathcal{F}) + \xi (\mathcal{O}|\mathcal{F}) $, where
$J(\mathcal{O}|\mathcal{F})$ is the total network cost in each iteration and $\xi (\mathcal{O}|\mathcal{F}) = {\mu} C(\mathcal{O}|\mathcal{F})$ is the weighted sum of constraint dissatisfaction of $C(\mathcal{O}|\mathcal{F})$. 

The penalty coefficient ${\mu}$ is set manually \cite{vnf_drl_solozabal,cmpd_solo} for CDRS-Fixed within a range $[0, \mu_{\text{max}} ]$\footnote{If Assumption \ref{assumption:localminima} is satisfied, $\mu_{\text{max}}$ can be set to $\infty$ \cite{reward_constraint}.}. In this case, the selection of ${\mu}$ can be set following intuition approach in \cite{vnf_drl_solozabal} (Appendix C), i.e., agent will not pay attention to penalty if $\mu = 0$, and it will only converge to penalization if $\mu = \infty$. Hence, selecting the appropriate penalty coefficient is important in this case. Otherwise, we can follow a less intuitive approach by adaptively updating the penalty coefficient (CDRS-Ada). CDRS-Ada is updated based on the primal-dual optimization (PDO) method inspired from \cite{pdo_risk}. Hence, we update the penalty coefficient in the ascent direction as:
%
\begin{align} \label{eq:update2}
{\mu}_{k+1} &= {\mu}_{k} + \eta_d(k) \nabla_\mu J_L^\pi({\mu},\theta) \\
& = {\mu}_{k} + \eta_d(k) (  J_{C}^\pi(\theta))_+, 
\end{align} 
where $\eta_d(k)$ is the step-size (Dual) and \secrev{$\nabla_\mu J_L^\pi({\mu},\theta) = \mathbb{E}_{\mathcal{O} \sim \pi_\theta(.|\mathcal{F})} [ C(\mathcal{O}|\mathcal{F}) ]$ is the gradient with respect to $\mu$}. The penalty coefficient ${\mu}_{k}$ is updated for every $k$-th iteration and will converge to a fixed value once the constraints are satisfied \cite{reward_constraint,pdo_risk}. 
%
%
Then, Monte-Carlo sampling can be applied to approximate \secrev{$J_L^\pi(\theta)$} by drawing  $B$ i.i.d samples $ \!\mathcal{F}^1,...,\mathcal{F}^B \!\sim\! \mathcal{F}$, which can be written:
\begin{align} \label{eq:lag_grads}
\!	\nabla_{\!\theta} J_L^\pi(\theta) \! \approx \! \frac{1}{B} \! \sum_{i=1}^{B} \! \! \Big(\! L(\mathcal{O}^i | \mathcal{F}^i) \! - \! b_{\theta_v}(\mathcal{F}^i)\! \Big) \! \nabla_{\!\theta} \! \log \! \pi_\theta(\mathcal{O}^i | \mathcal{F}^i), \!\!
\end{align} 
\secrev{where $b_{\theta_v}(\mathcal{F}^i)$ is the baseline estimation given the state input of $i$-th batch, parameterized by a neural network structure with weights $\theta_v$.}

\textbf{Baseline estimator.} The baseline choice can be from an exponential moving average of the reward over time that captures the improving policy in training. Although it succeeds in the Christofides algorithm, it does not perform well because it can not differentiate between inputs \cite{neural_bello}. To this end, we use a parametric baseline $b_{\theta_v}$ to estimate the expected total cost with penalization that typically improves the learning performance. \secrev{We estimate the baseline through} an auxiliary network built from an LSTM encoder connected to a multilayer perceptron output layer. \secrev{The auxiliary network (parameterized by $\theta_v$) that approximates the expected cost with penalization from input $\mathcal{F}$ is trained with stochastic gradient descent.} It employs a mean squared error (MSE) objective, calculated from the prediction of $b_{\theta_v}$ and the total cost with penalization $L(\mathcal{O}^i | \mathcal{F}^i)$, and sampled by the most recent policy (obtained from the environment). We formulate \secrev{the auxiliary network goal is to minimize the below loss function:}
%$\mathbb{E}_{\mathcal{O} \sim \pi(.|\mathcal{F})} L(\mathcal{O}|\mathcal{F})$
\begin{align} \label{eq:aux_mse}
	\mathcal{L}(\theta_v) = \frac{1}{B} \sum_{i=1}^{B} \left\| b_{\theta_v}(\mathcal{F}^i) - L(\mathcal{O}^i | \mathcal{F}^i) \right\|_2^2.
\end{align}
%
Fig. \ref{fig:baseline} illustrates the architecture of the auxiliary network for estimating the baseline. 
%
%
%
\begin{figure}[t!] 
	\centering
	\includegraphics[width=0.24 \textwidth]{images/baseline.pdf}   
	\caption{\small\secrev{\textbf{Baseline Estimator.} The self-competing baseline of CDRS is estimated using an auxiliary network constructed from an LSTM encoder connected to a multilayer perceptron output linear layer.}  } 
	\label{fig:baseline}
	%\vspace{-1mm}
\end{figure}
%%
%
%
%  
%
%
\begin{figure}[t!] 
	\centering
	\includegraphics[width=0.49 \textwidth]{images/RL_diagram.pdf}   
	\caption{\small\textbf{CDRS Diagram.} CDRS is trained using a single time step Monte-Carlo policy gradient algorithm, where at every epoch, the interactions with the environment are narrowed to a single time step. Our agent learns the policy iteratively over epochs.} 
	\label{fig:rl_diagram}
	\vspace{-3mm}
\end{figure}
%
%
\begin{algorithm}[t!]  \caption{CDRS Training}
	\label{algo:cdrs}
	%\myproc{TRAIN(Learning Set $\mathcal{F}$, batch size $B$)}
	\SetAlgoLined
	\DontPrintSemicolon
	\KwInput{$K$ (Num of epoch), $B$ (Batch size), $\mathcal{F}$ (Learning set)}
	\KwInitialize{ assign agent and critic (baseline) networks with random weights $\theta$ and $\theta_v. \;$} 
	%
	%	 
	\For{ $ k=1, ..., K$}  
	{
		$d\theta$ $\leftarrow$ 0 \% Reset gradient \\
		$\mathcal{F}^i \sim $ \text{SampleInput} $(\mathcal{F})$ for $i \in \{1,...,B \}$. \;
		$\mathcal{O}^i \sim $ SampleSolution $(\pi_\theta(.|\mathcal{F}))$ for $i \in \{1,...,B \}$. \;
		$b^i \leftarrow b_{\theta_v} (\mathcal{F}^i)$ for $i \in \{1,...,B \}$. \;
		Compute $L(\mathcal{O}^i)$ for $i \in \{1,...,B \}$. \;
		$g_\theta \leftarrow \frac{1}{B} \! \sum_{i=1}^{B} \! \! \Big(\! L(\mathcal{O}^i) \! - \! b^{i}\! \Big) \! \nabla_{\!\theta} \! \log \! \pi_\theta(\mathcal{O}^i | \mathcal{F}^i)$ from \eqref{eq:lag_grads}. \;
		$\theta \leftarrow$ Adam($\theta, g_\theta$) \%Run Adam algorithm \;
		$\mathcal{L}_v \leftarrow \frac{1}{B} \sum_{i=1}^{B} \left\| b^{i} - L(\mathcal{O}^i) \right\|_2^2 $ from \eqref{eq:aux_mse}. \;
		$\theta_v \leftarrow$ Adam($\theta_v, \mathcal{L}_v$) \%Run Adam algorithm \;
		{\color{black} Update ${\mu}$ from \eqref{eq:update2} \%CDRS-Ada\\ }
		{\color{black} Set ${\mu} = \max(0,{\mu})$ \%CDRS-Ada}
	}
	\Return $\theta, \theta_v, \mu$
	\;
\end{algorithm}
%
%

\secrev{To sum up, our training procedures are summarized in Algorithm \ref{algo:cdrs} and illustrated in Fig. \ref{fig:rl_diagram}, which run iteratively by $K$ episodes (epochs) based on a single time-step Monte-Carlo policy gradient with a baseline estimator.}
The sequence of policy updates will converge to a locally optimal policy and the penalty coefficient updates (e.g., CDRS-Ada) will converge to a fixed value when all constraints are satisfied; see also \cite{pdo_risk,reward_constraint}. 


%

%for the standard convergence proof of stochastic approximation algorithm with constraints. 

%\subsection{Training Algorithm}
%Algorithm 1 summarizes our training procedure of single time-step Monte-Carlo Policy Gradient with baseline estimator, which runs until $T$ epochs. We also include the penalty coefficient update in this procedure. Our training firstly requires the training set from a set of all BS functions $\mathcal{F}$, the number of minibatch $B$, and the number of epochs $T$. For initalization, we randomly give the weight values for our agent $\theta$ and baseline $\theta_v$. Algorithm 1 runs iteratively until $T$ epochs (Step 1). It resets the gradient ($d\theta $) by assigning zero value (Step 2). Then, it randomly generates i.i.d samples from the training set, e.g., $\mathcal{F}^1,...,\mathcal{F}^B \sim \mathcal{F}$ (Step 3).

%Next, we discuss the convergence of Algorithm 1. 

%It almost surely converges to a fixed values, which is a feasible solution (local optimal)
%\begin{theorem} \label{theo:penalty}
%	The penalty coefficient updates of CDRS-Ada in Algorithm 1 will converge to a fixed value once all constrains are satisfied.
%\end{theorem}
%\noindent
%\textit{Proof.}
%
%
%\begin{theorem} \label{theo:algo1}
%	The policy updates in Algorithm 1 almost surely converges to a locally optimal policy and hold in our case.
%\end{theorem}
%\noindent
%\textit{Proof.} We can proof it following standard procedure for stochastic approximation algorithm...


% Next, we prove that our approach will converge to a fixed values, which is a feasible solution (local minima) for our problem.
%%
%%
%\begin{theorem}
%	Algorithm 1 converges to a feasible solution (local optimality) 
%\end{theorem}
%%
%\textit{Proof.} We prove the convergence of Algorithm 1 for our case following  Theorem xxx of \cite{}. 1) The dual function is always convex despite the primal problem is non-convex \cite{}. Hence, our dual function in \eqref{eq:dual_function}  is also a convex function, so it is also Lipschitz continuous. 

\vspace{-1mm}
\subsection{Searching Strategy}
%\vspace{-1mm}
At the test time, evaluating the total network cost is inexpensive \secrev{as it only requires a forward pass from the policy network to decide the splits}. Our agent can add a search procedure during the inference process by considering solution candidates from multiple \secrev{pretraining} models to select the splits. It can help to reduce the inferred policy suffering from a severe suboptimality. 
%In this part, we employ two different search strategies: greedy decoding and \secrev{temperature sampling} \cite{neural_bello}. 
We employ two different search strategies by greedy decoding and \secrev{temperature sampling} \cite{neural_bello}.

\textbf{Greedy decoding.} It is the simplest search strategy. The idea is to \secrev{greedily} select the splits with the \secrev{highest} probability for having the lowest cost \secrev{from multiple pretraining models during the inference time}. %At the inference time, the greedy output from each model is evaluated to \secrev{choose} the best one \cite{vnf_drl_solozabal}. 
Then, we can extend CDRS to CDRS-Fixed-G, which uses a fixed penalty coefficient with greedy decoding and CDRS-Ada-G that uses an adaptive penalty coefficient with greedy decoding. 

\textbf{Temperature sampling.} This method samples through stochastic policy for \secrev{each pretraining model to generate several candidate solutions} then \secrev{decides} the splits with the lowest total cost \secrev{among them} \cite{neural_bello,vnf_drl_solozabal}. As opposed to the heuristic solvers, it does not sample the different split options. Instead, \secrev{it samples through the stochastic policy} and controls the sparsity of the output distribution \secrev{by} a temperature hyperparameter $T$. The softmax function in \eqref{eq:softmax} is modified to $\bm{a}_{tk} = \frac{\exp\big( \text{score}(\bm{h}_t,\bm{\bar{h}}_k)/T \big)}{\sum_{k'=1}^{N} \exp\big(\text{score}(\bm{h}_t,\bm{\bar{h}}_k')/T ) \big)}$ (softmax temperature). \secrev{In the training}, the temperature hyperparameter $T$ is \secrev{set} to 1. \secrev{Meanwhile, we modify to $T>1$ during the test}, hence the output distribution becomes less step, \secrev{which} prevents the model from being overconfident. With this method, we can extend CDRS to CDRS-Fixed-T (fixed penalty coefficient, temperature sampling) and CDRS-Ada-T (adaptive penalty coefficient, temperature sampling). \secrev{Note that this method requires additional time, which depends on the number of samples.}


 %It considers multiple candidate solutions, then infers the best solution. The approach is to sample candidate solutions from stochastic policy, then select the split configuration with the lowest total cost. A temperature hyperparameter controls the diversity of the sampling to attain an improvement in finding the best solution. The detailed algorithm is described in \cite{neural_bello}.









% RESULT
\vspace{-2mm}
\section{Simulations}



The simulations are all carried through the benchmark platform of the OMPL \cite{sucan2012open} \cite{moll2015benchmarking}.
To validate the generalization ability of our method, we solve the planning problem in both the $SE(2)$ and $SE(3)$ state spaces with our method and several state-of-art algorithms.
In simulation environments, the state spaces are continuous. 
The algorithms that have been tested are all based on random sampling and take samples from these continuous state spaces without discretizing the space.
The robot is represented by a collection of convex polyhedrons and occupies a certain volume.


\subsection{Qualitative Analysis}

\begin{figure*}[t]
    \centering
    \begin{minipage}[t]{1\linewidth}
        \subfigure[]{
            \begin{minipage}[t]{1\linewidth}
                \centering
                \includegraphics[width=1.0\textwidth]{./img/BenchmarkImgs/3D_Apartment_Result.png}
            \end{minipage}%
        }
    \end{minipage}%

    \centering
    \begin{minipage}[t]{1\linewidth}
        \subfigure[]{
            \begin{minipage}[t]{1\linewidth}
                \centering
                \includegraphics[width=1.0\textwidth]{./img/BenchmarkImgs/3D_CustomEasy_Result.png}
            \end{minipage}%
        }
    \end{minipage}%
    \caption{(a) and (b) show the 2D simulation result in `BugTrap', the `Maze', and the `RandomPolygons' environments, where the left pictures are the time each planner spent to meet the optimization objective 
    and the right pictures are the cost variations over time.
    Planners try to meet the optimization objective, dashed lines in the right pictures show the cost value of the optimization objective.
    }
\label{SimulationResults_3D}
\end{figure*}

\begin{figure}[t]
    \centering
    \includegraphics[width=0.46\textwidth]{./img/envs_3D.png}
    \caption{The 3D simulation environments.}
    \label{SimulationEnvironments_3D}
\end{figure}


% To provide a further explanation of our method, we use the RRT\# \cite{arslan2013use}, the AIT* \cite{strub2020adaptively}, and our method to solve the motion planning problem in an OMPL benchmark environment called the `BugTrap'.
To give more detail explanation about our approach, we employ the RRT\# \cite{arslan2013use}, AIT* \cite{strub2020adaptively}, and our own method to address the path planning challenge within the `BugTrap' OMPL benchmark environment.
The state space of the `BugTrap' environment is the $SE(2)$ state space, which is composed of the position $x$, $y$ and the orientation $w$.
The planning procedures of the RRT\# \cite{arslan2013use}, the AIT* \cite{strub2020adaptively}, and our method are illustrated in Fig. \ref{PlanningProcedure}, where obstacles, the free space, the start state, the goal region, and vertices are indicated with black, ivory white, pale blue, wine, and orange color, respectively.
We use the dark green lines and violet lines to show the forward tree and the current optimal solution, respectively.
The reverse trees are shown as the grey lines in the figure of the AIT* \cite{strub2020adaptively} and our method.


In the simulation shown in Fig. \ref{PlanningProcedure}, the planning problem contains two optimal solutions, one is to pass through the region upper the obstacle, and the other one is to pass through the lower part.
Fig. \ref{PlanningProcedure} shows that all the methods in Fig. \ref{PlanningProcedure} can acquire the global asymptotical optimality.
Both the AIT* and our method use the lazy reverse-searching tree to guide the sampling and have the graph pruning method to constraint the samples and the trees.
From the (j)-(m) in Fig. \ref{PlanningProcedure}, it can find that both the forward and reverse trees of our method are optimal under current state space abstraction.
In addition, our method concentrates on taking samples in the region with a higher potential to improve the current solution, which can be seen in (m) of Fig. \ref{PlanningProcedure}, our method pays more attention to the turning corners with our direct sampling method.


\subsection{Simulations in $SE(2)$ State Space}



We choose the $SE(2)$ environments shown in Fig. \ref{SimulationEnvironments_2D} to verify our method, and they are called the `BugTrap', the `Maze', and the `RandomPolygons' in the OMPL benchmark platform. 
To give the reader an intuitional understanding of our 2D planning simulations, we show the trajectories found by our method in Fig. \ref{SimulationPath_2D}.
The trajectories are interpolated in terms of time.


In our 2D simulations, the state space definition contains the position $x, y$ and orientation $w$.
To manifest the superiority of our method, we compared with seven different state-of-art algorithms, they are the RRT*, the BIT*, the AIT*, the ABIT*, the Informed RRT*, the RRT\#, and the Informed $+$ Relevant sampling method proposed in \cite{joshi2020relevant}.
In these simulations, we use the trajectory length as the cost metric.
The optimization objective is set as $\beta \times c_{opt}$, where the $c_{opt}$ is the cost of the optimal solution and $\beta$ is a number close to $100\%$.
% The $c_{opt}$ is the solution cost of the RRT* method after $300$ seconds' execution, which is nearly the optimal solution cost, and we choose to use this number to represent the optimal cost.
The $c_{opt}$ is the solution cost of the RRT* method after $300$ seconds of execution, which is nearly optimal. 
We choose to use this number to represent the optimal cost.
To reduce the randomness, each planner runs $100$ times in each environment.



The simulation results in the $SE(2)$ state spaces are shown in Fig. \ref{SimulationResults_2D}.
On the left side of the pictures, the charts shows the amount of time each planner took to generate the required path.
On the right side, the cost distribution is presented in terms of time. We begin plotting the line charts once 50\% of all runs have found a solution, and stop once 95\% have completed the problem-solving process. 
Hence, the speed of obtaining the initial solution can also be displayed in the same chart.
Additionally, error bars are provided for all bar charts and line charts.


The 2D simulation results show that our method acquired significant improvements and achieved better performance.
Both the initial solution quality and the convergence rate of our method are better than the others.
The only drawback of our method is we generate the initial solution slower than the others, but we acquired the best initial solution quality.
And our initial solution is better than the others' optimized solutions at the same time point.


\subsection{Simulations in $SE(3)$ State Space}



Besides the 2D simulation introduced previously, we also carried on the simulation in the $SE(3)$ state space. 
In the 3D simulations, we include the `3D\_Apartment' planning problem from the OMPL benchmark platform \cite{moll2015benchmarking}, which is a `piano movers' problem, as the left environment in Fig. \ref{SimulationEnvironments_3D} shows.
The other simulation is set as a planning problem in 3D narrow passage environment.
In the 3D simulation, planners and their parameter sets are the same as the planners we choose in 2D simulations.
Each planner will solve each planning problem 100 times to avoid the randomness.
The 3D simulation results are shown in Fig. \ref{SimulationResults_3D}.




%CONCLUSION
\vspace{-2mm}
\section{Experimental Results And Discussion}
\label{sec:results}
The results presented in this section test the performance of the Autoencoder model. We evaluate our model using the performance metrics: accuracy, precision, recall, and F1 score, defined as follow: 

\vspace{-5mm}
\begin{align*}
    Accuracy &= \frac{TP+TN}{TP+TN+FP+FN}
\end{align*}
\vspace{-3mm}
\begin{align*}
    Precision &= \frac{TP}{TP+FP}
\end{align*}
\vspace{-3mm}
\begin{align*}
    Recall &= \frac{TP}{TP+FN}
\end{align*}
\vspace{-3mm}
\begin{align*}
    F1 ~Score &= 2 \times \frac{Precision \times Recall}{Precision + Recall}
\end{align*}

In our experiments, a \textit{positive} outcome means an abnormal activity was detected, whereas a negative outcome means a normal activity was detected.
True Positive (TP) refers to an abnormal activity that was correctly classified as abnormal. 
True Negative (TN) refers to a normal activity that was correctly classified as normal.
False Positive (FP) refers to a normal activity that was misclassified as abnormal.
and False Negative (FN) refers to an abnormal activity that was misclassified as normal.

The success of our model is based on measuring the reconstruction error that is produced by any given data point. Figure \ref{fig:recon} shows an example of reconstructed data overlaid the original data that was inserted into the model. %To be clear, the values shown in this graph are not measurements of the reconstruction loss that are shown in Figure \ref{fig:thresh}. This figure only shows the normalized temperature measurement from each data point, that is why they are not being represented in degrees.
In this figure, extremely severe dips in temperature denoted by the blue line (representing our original data) can be noticed. The data reconstructed by the model, represented by the red line, does not dip as much as the original data. This is because our model was not able to reconstruct these points accurately due to the fact that they are anomalies. The reconstruction loss (i.e. different between the original and the reconstructed data), where the model recognizes normal or abnormal behavior, is shown in Figure \ref{fig:thresh}. The figure shows a visualization of the mean-squared-error (MSE) generated by the model after it was given each data point within the test data set. The dotted red line denotes the threshold determined as mentioned in Section \ref{sec:ml-model}. Each data point's actual label is represented either by blue color to denote a normal behavior or red color to denote an anomaly and every data point that lies above the threshold was classified as anomalous. This figure illustrates our model's capability to detect the majority of anomalies by measuring the MSE produced by each data point.

Overall, as shown in Figure \ref{fig:aeresults}, our model was able to attain high performance with over $90\%$ in all metrics. The precision is lower than the recall metric which shows that the model produced slightly more false positives than false negatives. In a smart farming environment, a higher rate of false positives would not have a dramatic affect on the productivity of day to day operations and would ensure a higher number of anomalous situations are detected. A rather problematic situation would be if there were more false negatives than positives. A user would much prefer receiving an alert when nothing was wrong than not receiving an alert and enabling potential harm to occur to the crops and hardware. In the future, we hope to further decrease the number of false positives and negatives in order to fine-tune an overall more accurate model. This can be done by using more training samples.


% \begin{table}[!t]
%     \caption{Results}
%     \centering
%     \begin{tabular}{| c | c | c | c |}
%     \hline
%     Accuracy & Precision & Recall & F1\\ [0.5ex] % inserts table %heading
%     \hline
    
%     98.98\% & 90\% & 92.95\% & 91.45\% \\
    
%     \hline
%     \end{tabular}
%     \label{table:results}
% \end{table}

\begin{figure}[t!]
    \centering
    \includegraphics[width=8cm]{figures/aeresults-v2.png}
    \caption{Performance metrics for Autoencoder Model}
    \label{fig:aeresults}
\end{figure}


\section{Conclusion and Future Work}
\label{sec:conclusion}
Our approach has shown that smart farming anomaly detection can be done at an extremely accurate level by using an Autoencoder. Our approach would allow vast scalability by only requiring non-anomalous data for training. Greenhouses provide controlled environments that create consistent conditions for crops and data collection. Environments such as this are a perfect use case for our approach since the performance of an Autoencoder can drastically improve when provided with large amounts of non-anomalous data. Our approach shows that it may not be entirely necessary for machine learning professionals that are working on anomaly detection within smart farming to be highly concerned with developing models that are trained using labeled data that contains both normal and anomalous data. 

In the future, we will explore more anomaly detection models in order to optimize the system's performance. Once the best model has been selected, the architecture could be brought online to be used and tested with the added interactions of Internet connectivity. By bringing the system online we will have the ability to alert users of potential threats or anomalous behavior. These alerts could be coupled with actuators such as fertilization, watering, video monitoring, etc. The introduction of cameras can be ``used to calculate biomass development and fertilization status of crops" \cite{Walter6148}. They can also be used to allow the system-user to monitor their property from afar. We plan to introduce photo and video monitoring as one of our next steps to improve security and broaden our scope.

\section{Acknowledgements}
\label{sec:ack}
We thank TTU Shipley Farms for allowing to use greenhouse, and setup smart farm testbed. Dr. Brian Leckie and his group were instrumental in our system and early stages of data collection. We are thankful to Ms. Deepti Gupta to provide helpful guidance on dealing with time-series, correlated data and gave input on our model selection. This research is partially supported by the NSF Grant 2025682 at TTU.





%%%%%%%%%%%%%%%%%%%%%%%%%%%%%%%%%%%%%%%% ============================================= %%%%%%%%%%%%%%%%%

\bibliographystyle{IEEEtran}
\bibliography{IEEEabrv,ref-vran-01}



%
%\begin{IEEEbiography}
%	[{\includegraphics[width=1in,height=1.25in,clip,keepaspectratio]{./biography/fahri.jpg}}]{Fahri Wisnu Murti} is currently pursuing his doctoral degree at Centre for Wireless Communication (CWC), University of Oulu, Finland. His current research interests lie in the development of machine learning and optimization techniques for intelligent wireless networks.
%	%
%	Prior to his doctoral study, he worked as a research assistant at Dept. Computer Science, Trinity College Dublin, Ireland. He received his B.S. from Telkom University, Indonesia and completed his master's degree from WENS Lab., Dept. IT Convergence Eng., Kumoh National Institute of Technology, South Korea. 
%\end{IEEEbiography}
%
%%\vspace*{-2\baselineskip}
%
%\begin{IEEEbiography}[{\includegraphics[width=1.1 in,height=1.25in,clip,keepaspectratio]{./biography/samad.jpg}}]{Samad Ali} received the B.S. degree in electrical engineering from the University of Tabriz, Tabriz, Iran, and the M.S. and Ph.D. degrees in wireless communications engineering from the University of Oulu, Oulu, Finland. He is currently a Postdoctoral Researcher with the University of Oulu and Senior Research Specialist at Nokia Standards. His research interests include machine learning in wireless communications, machine type communications and RIS. 
%\end{IEEEbiography}
%
%%\vspace*{-2\baselineskip}
%
%\begin{IEEEbiography}[{\includegraphics[width=1in,height=1.25in,clip,keepaspectratio]{./biography/matti.jpg}}]{Matti Latva-aho} received the M.Sc., Lic.Tech. and Dr. Tech (Hons.) degrees in Electrical Engineering from the University of Oulu, Finland in 1992, 1996 and 1998, respectively. From 1992 to 1993, he was a Research Engineer at Nokia Mobile Phones, Oulu, Finland after which he joined Centre for Wireless Communications (CWC) at the University of Oulu. Prof. Latva-aho was Director of CWC during the years 1998-2006 and Head of Department for Communication Engineering until August 2014. Currently he serves as Academy of Finland Professor in 2017 – 2022 and is Director for National 6G Flagship Programme for 2018 - 2026. His research interests are related to mobile broadband communication systems and currently his group focuses on beyond 5G systems research. Prof. Latva-aho has published close to 500 conference or journal papers in the field of wireless communications. He received Nokia Foundation Award in 2015 for his achievements in mobile communications research.
%\end{IEEEbiography}



\end{document}
