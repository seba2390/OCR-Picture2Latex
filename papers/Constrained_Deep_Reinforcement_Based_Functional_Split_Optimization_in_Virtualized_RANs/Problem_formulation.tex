\section{Formalization of vRAN Split Problem} \label{sec:problem}
\vspace{-1mm}
The BS functions can be deployed at the DUs or CU \secrev{depending on} the splits, as seen in Table \ref{table:splits}. \secrev{Each split} must respect to the \emph{chain of functions} $f_0 \!\rightarrow\! f_1 \!\rightarrow\! f_2 \!\rightarrow\! f_3$\footnote{\secrev{$f_0$ is a function that encapsulates RF layers. Then, $f_1, f_2$ and $f_3$ are the functions for Layer 1 (PHY), Layer 2 (MAC, RLC) and Layer 3 (PDCP, RRC and the upper layers), respectively. }}. Thus, we define $x_{on} \in \{0,1\}$ as the decision for deploying split $o \in \{0,1,2,3\}$ at DU-$n$. For instance, $x_{0n} = 1$ is for deploying $f_0, f_1, f_2, f_3$ (Split 0); $x_{1n} = 1$ for $f_0, f_1, f_2$ (Split 1); $x_{2n}= 1$ for $f_0, f_1$ (Split 2); or $x_{3n}= 1$ for $f_0$ (Split 3) at DU-$n$. 
%
%
We only deploy a single split configuration for each BS. Therefore, \secrev{a} set of eligible \secrev{splits} is:
%
%\vspace{-1mm}
\begin{align} \label{eq:setx}
\mathcal{X} =  &\Bigl\{  \bm{x}_n  \in \{0,1\} \Big| \sum_{o=0}^3 x_{on}=1 , 
\ \ \forall n \in \mathcal{N}  \Bigr\} , 
\end{align}
%
where $\bm{x}_n = ( x_{on}, \forall o  )$ and $\bm{x} = (\bm{x}_n, \forall n)$. The BS functions,  $f_1, f_2$ and $f_3$, are deployed \secrev{using} VMs at each server. We have computational processing at the CU and DU-$n$ that must respect \secrev{to} its capacity as:
%
%\vspace{-1mm}
\begin{align} \label{eq:computing1}
\sum_{n \in \mathcal{N}} \lambda_{n} \sum_{o =0}^3    x_{on} \rho^{c}_o \leq \ H_0, 
%
\end{align}
%\vspace{-1mm}
%
%
%\vspace{-1mm}
\begin{align} \label{eq:computing2} \lambda_{n} \sum_{o =0}^3   x_{on} \rho^{d}_o \leq H_n, \ \forall n \in \mathcal{N}. 
%
\end{align}
%
%\textit{We consider the realistic scenario of CU processing where the parameters of the servers change over time in unknown fashion.} Each CU is shared to many BS functions (resource pooling); so, it has an \textit{average capacity} affecting the processing performance, i.e., when the load exceed average capacity ($\bar{H}_0$), the processing delay will increase and lead to the system non-responsive. This condition is generated by random process $\{ H_0^t\}_{t=1}^{\infty}$ with $ \lim_{T \to \infty} \frac{1}{T} \sum_{t=1}^{T}H_0^t = \bar{H}_0, \forall t.$ However, our decision only has information on the instantaneous value in each time slot. In here, we only need this pertubation is bounded at each time slot ($ H_0^t \leq H^{\text{max}}_0$) and the average converges to some finite value ($\bar{H}_0$)

\textbf{Data Flow $\&$ Delay.}  Let define $r_{p_{n0}}$ (Mbps) as the amount of data flow (Mbps) to be transferred through a path $p_{n0}$. Hence, the flow must respect \secrev{capacity of each link}:
%
%\vspace{-1mm}
\begin{align} \label{eq:route1}
\sum_{n \in \mathcal{N}} r_{p_{n0}} I^{ij}_{p_{n0}} \leq c_{ij}, \ \ \forall (i,j) \in \mathcal{E},
\end{align}
%
where $I^{ij}_{p_{n0}} \in \{ 0,1 \}$ indicating whether the link $(i,j)$ is used by path $p_{n0}$. Assuming a single path (e.g., shortest path), the amount of data flow depending on each split configuration is \cite{fluidran_andres}:
%
%\vspace{-1mm}
\begin{align} \label{eq:route2}
r_{p_{n0}} \!=\! \lambda_{n} (x_{0n} + x_{1n}) + x_{2n} (1.02 \lambda_{n} + 1.5) + 2500 x_{3n}.
\end{align}
%
We let $d_{p_{n0}}$ denote the incurred delay for routing through path $p_{n0}$ from DU-$n$ to the CU. Each split has to satisfy the respective delay requirement (Table \ref{table:splits}):
%
%\vspace{-1mm}
\begin{equation} \label{eq:delay}
x_{on} d_{p_{n0}} \leq d_o^{\text{max}}, \ \ \forall o, \forall n \in \mathcal{N}.
\end{equation}
%
\subsection{Objective Function}
We aim to minimize the total network cost consisting of the computational \secrev{costs at the DUs and CU} and the routing cost\footnote{In this case, we follow the linear objective cost function similar to the previous studies \cite{fluidran_andres,vranmec_andres}. However, our solution approach does not restrict only to the linear objective. Our approach \secrev{relies on} the \secrev{scalar} reward and penalty as feedback; hence, it can also be tailored to a non-linear objective.}. The needs of computing cost \secrev{for each BS-$n$} at DU-$n$ is:
%
\begin{align} \label{eq:cost-du}
V_n(\bm{x}_n) = \alpha_n + \beta_n \lambda_{n} \sum_{o=0}^{3}\rho_{o}^{d} x_{on}.
\end{align} 
%
We also have \secrev{the required} computing cost \secrev{of BS-$n$} at the CU:
%
\begin{align} \label{eq:cost-cu}
\secrev{V_{n0}(\bm{x}_n)} = \sum_{o=0}^3 x_{on} (\alpha_0 + \lambda_{n} \beta_0 \rho_{o}^{c} ).
\end{align} 
\secrev{The first terms in \eqref{eq:cost-du} and $\eqref{eq:cost-cu}$ represent the required instantiating cost at each DU and CU for BS-$n$. The last terms in \eqref{eq:cost-du} and \eqref{eq:cost-cu} are the required data processing cost by each DU and CU to serve BS-$n$ load.}
%
Next, we have the cost to route the data flow from DU-$n$ to the CU:
%
\begin{align} \label{eq:cost-route}
U_{n0}(\bm{x}_n) =  \zeta_{p_{n0}} r_n (\bm{x}).
\end{align} 
%
Finally, we have the total vRAN cost as:
%
\begin{align} \label{eq: total-cost}
J(\bm{x}) &= \sum_{n \in \mathcal{N}} \Big(  V_n(\bm{x}_n) + U_{n0}(\bm{x}_n) +\secrev{ V_{n0}(\bm{x}_n}) \Big), \end{align}
which leads to the following problem:
%
\begin{align}
\mathbb{P}: \,\,\,\, & \underset{\bm{x} \in \mathcal{X}}{\text{minimize}} \   J(\bm{x}), \ \ \ \notag  \text{s.t} \ \  
 (\ref{eq:computing1}) - (\ref{eq:delay}). \notag
\end{align}
%
$\mathbb{P}$ is a combinatorial problem to decide the function placement $\bm{x}$ for all the BSs and serve the traffic load $\bm{\lambda}$ with DU-CU path $p_{n0}$ for each BS-$n$ in the network graph $G = (\mathcal{I}, \mathcal{E})$. Next, we discuss the complexity of $\mathbb{P}$.

\subsection{Complexity Analysis}

The complexity of $\mathbb{P}$ can be identified from the polynomial reduction of \textit{multiple-choice multidimensional knapsack problem (MMKP).} 

\textbf{MMKP.} Let suppose there are $N$ items with values ${v_1, v_2, ..., v_N}$. We also have $r_1, r_2, ..., r_N$ correspond to the required resources to pick the items. In the 0-1 knapsack problem (KP), the aim is to pick the items $x_i \in \{0, 1\}, \secrev{\forall i}$ that maximize the total value $\sum_{i=1}^N  x_i v_i$, subject to constraint \secrev{$\sum_{i=1}^N x_i r_i \leq R$}. This is a well-known NP hard problem and there is a pseudo-polynomial algorithm using a dynamic programming concept that has complexity $\mathcal{O}(NR)$ \cite{mmkp_convexhull}.  MKKP is a variant of 0-1 KP where there are $M$ groups of items, e.g., group $i$ has $l_i$ items. Each item has a specific value $v_{ij}$ corresponds to $j$-th item of $i$-th group and needs $K$ resources. Hence, each item in a group has a resource vector $\bm{r}_{ij} = (r_{ij1}, ..., r_{ijK} )$ and $\bm{R} = (R_1, ..., R_K)$ is the resource bound of the knapsack. The aim is to exactly pick one item from each group, e.g., $ \sum_{j=1}^{l_i} x_{ij} = 1, x_{ij} \in \{0,1\}$ that maximizes the total value: $\sum_{i=1}^{M} \sum_{j=1}^{l_i} x_{ij} v_{ij}$, subject to the resource constraint: $\sum_{i=1}^{M} \sum_{j=1}^{l_i} x_{ij} r_{ijk} \leq R_k, k = 1,...,K$. 

Finding an exact solution for MMKP is also NP-hard \cite{mmkp_convexhull}. It is also worth noting that the search space for solution in MMKP is smaller than other KP variants; hence, exact solution is not implementable in many practical problems as there is more limitation of picking items from a group in MMKP instance \cite{mmkp_convexhull}. Next, We prove that $\mathbb{P}$ is harder than MMKP. 

\noindent
\begin{theorem} \label{theo:mmkp}
	\textit{MKKP can be reduced to $\mathbb{P}$ in polynomial time, e.g., MMKP $\leq_P \mathbb{P}$ }.
\end{theorem}  

\noindent
\textit{Proof.} Let suppose we have unlimited link capacity, no routing cost and no delay requirement. Hence, all paths of \secrev{the} DU-CU pair are eligible and \eqref{eq:route1}-\eqref{eq:delay} are always satisfied. This problem then can be mapped to MMKP by setting: 1) $M$ groups to $N$ BSs, 2) each $i$-th group with $l_i$ items to each BS-$n$ with $|o|=4 $ of split options, 3) $j$-th item of $i$-th group to the split $o_n$ of BS-$n$, 4) $r_{ij}$ to the \secrev{incurred} computing loads, e.g., $\lambda_{n}\rho_{i}^c$ and $\lambda_{n}\rho_{i}^d$, and 5) the knapsack constraints to computing constraints $H_n$ and $H_0$. The value $v_{ij}$ of item-$j$ in group-$i$ also can be mapped with the costs (e.g., computing and routing) of deploying split-$o$ of BS-$n$, where the MMKP is a maximization problem and $\mathbb{P}$ is a minimization problem.  We can see the reduction is of polynomial time: we select the functional split for every BS correspond to that we activate an item that we pick to a knapsack in each group. Therefore, we can conclude that if we can solve $\mathbb{P}$ in polynomial time we also can solve any MMKP problem. 

%
%\begin{table}[t]
%%\begin{tabular}{cp{5cm}} \hline
%\begin{small}
%\begin{tabular}{cp{5cm}} \hline
%\textbf{Notation} & \textbf{Definition} \\ \hline
%$\mathcal{N}, N, n$ & Set of DUs, number of DUs, DU index \\ 
%$\mathcal{M}, M, m$ & Set of CUs, number of CUs, CU index \\ 
%$\mathcal{P}_{nm}$ & Set of paths between DU $n$ and CU $m$\\ 
%$\mathcal{P}_{m}$ & Set of paths between all DUs and CU $m$\\ 
%$c_{ij}$ & Bandwidth in Mbps of the link between nodes $i$ and $j$ \\ 
%$d_{p_k}$ & End-to-end delay of the path $p_k$ in seconds \\ 
%$\lambda_{n}$ & Traffic flow from DU-$n$ in Mbps \\ 
%$H_{n}, H_{m}$ & Processing capacity (cycles/sec) of DU-$n$ and CU-$m$\\ 
%$\rho_1, \rho_2, \rho_3$ & Processing load (cycles) per Mbps of $f_1$, $f_2$, $f_3$\\ 
%$a_{m}, b_{m}$ &  Cost of VM instantiation and computing at CU $m$\\ 
%$\alpha_{n}, \beta_{n}$ &  Cost of instantiation and computing at DU $n$\\ 
%$\zeta_k$ & Routing cost of path $k$ (monetary units per Mbps) \\
%$c_d$ & Routing cost per kilometer per Mbps \\ \hline
%\end{tabular}
%\end{small}
%\centering
%
%\bigskip
%\caption{Notation table}
%\label{tab:notation_table}
%\end{table}