\section{System Model} \label{sec:model}
%\vspace{-1mm}

\begin{figure}[t!]
	\centering
	\includegraphics[width=0.45 \textwidth]{images/model.pdf}   
	\caption{\small vRAN over integrated fronthaul/midhaul (xHaul). It has many degrees of design freedom by possibly hosting BS functions at the CU or DUs.}
	\label{fig:vran}
	\vspace{-3mm}
\end{figure}

\textbf{Background.} In C-RAN, all BS functions are centralized at the Base Band Unit (BBU) except RF layers at the RU.  \thirdrev{In vRANs}, the BBU is decoupled into the CU and DU \cite{split_3gpp_rel16}. Hence, functions of a BS can be deployed at the CU, DU and RU. Fig \ref{fig:vran} illustrates that a CU is typically executed at a bigger and more centralized \thirdrev{server} (e.g., edge server), while a DU is at a smaller server (e.g., far-edge server) and located near (or co-located) with an RU.

Our model refers to the standardization of 3GPP \cite{split_3gpp,split_3gpp_rel16} and seminal white paper \cite{smallcell}, where each split has a different performance gain \cite{vran_murti2,function_split_survey}. \thirdrev{Although 3GPP has defined eight options for the splits, several are still hardly implemented. Therefore, we consider four splits that have been experimentally validated in a prototype \cite{costdu_nikaein,adaptive_alba}. }
 \textbf{Split 0}: All functions are at the DU, except the RF layers at the RU. It is a typical D-RAN setup. \textbf{Split 1} (PDCP-RLC): RRC, PDCP, and upper layers are hosted at the CU, while RLC, MAC, and PHY are at the DU. \secrev{This split enables a separate user plane \thirdrev{and control plane with} centralized RRC.} \textbf{Split 2} (MAC-PHY): MAC and upper layers are at the CU, while PHY is at the DU. It allows improvement for CoMP by centralized HARQ. \textbf{Split 3} (PHY-RF): All functions are at the CU, except RF layers. It is a fully centralized version of vRANs. It gains power-saving and improved joint reception CoMP with uplink PHY level combining. Going from \secrev{Split 0 to Split 3}, more functions are hosted at the CU. In addition to increasing network performance, a higher centralization level can lead to more computing cost savings \cite{vran_murti2}. However, centralizing more functions increases the data load to be transferred to the CU, going from $\lambda$ in \secrev{Split 1 to 2.5 Gbps in Split 3} for each BS, and has a stricter delay requirement. Table \ref{table:splits} summarizes vRAN split options and their requirements\footnote{\secrev{The requirements are tailored from \cite{smallcell,vranmec_andres} by following settings: 1 user per TTI, 20MHz channel bandwidth, 1 carrier component, UE IP MTU 1500 bytes, $2 \times 2$ MIMO.} }.  

%
%\begin{table}[t!] \centering
%	%\ra{1.3}
%	\begin{small}
%		\begin{tabular}{@{}lcccc@{}}\toprule
%			\textbf{}& \textbf{Traditional RAN} & \textbf{Cloud RAN}  & \textbf{Open RAN} & \textbf{Open vRAN}
%			\\ \midrule
%			\textbf{RU} &      Locked   & Locked  & Open        & Open
%			\\ \hdashline
%			{\textbf{Interface}} &  Locked & Locked & Open & Open
%			\\ \hdashline
%			{\textbf{CU/DU SW}} &  Locked   & Locked, Virtualized & Open & Open, Virtualized
%			\\ \hdashline
%			{\textbf{CU/DU HW}} &   Locked   & Open & Open & Open \\ %\midrule
%			\bottomrule
%		\end{tabular}
%	\end{small}
%	\caption{\small\textbf{RAN transformation}. \textit{Locked} means that is a solely property of a single vendor. \textit{Open} means that it allows interoperable to flexibly work with different vendors. }
%	\label{table:ran_transform}
%	\vspace{-5mm}
%\end{table}

\begin{table}[t] \centering
	%\ra{1.3}
	\begin{small}
		\begin{tabular}{@{}lll@{}}\toprule
			\textbf{}& \textbf{Flow (Mbps)} & \textbf{Delay Req. (ms)}  
			\\ \midrule
			{Split 0 \secrev{(S0)} } &      $\lambda$   & $30$          
			\\ \hdashline
			{Split 1 \secrev{(S1)} } &  {$\lambda$} & $30$ 
			\\ \hdashline
			{Split 2 \secrev{(S2)} } &  {$1.02\lambda+1.5$}   & $2$
			\\ \hdashline
			{Split 3 \secrev{(S3)} } &   {$2500$}   & $0.25$ \\ %\midrule
			\bottomrule
		\end{tabular}
	\end{small}
	\caption{\small Data and delay requirements of vRAN split when the traffic load is $\lambda$ Mbps \cite{smallcell, vranmec_andres}.}
	\label{table:splits}
	\vspace{-3mm}
\end{table}


\textbf{RAN}. We model a vRAN architecture with a graph $G = (\mathcal{I}, \mathcal{E})$ where $\mathcal{I}$ has a subsets $\mathcal{N}$ of the $N=|\mathcal{N}|$ DUs, \secrev{$\mathcal{L}$ of the $L=|\mathcal{L}|$ routers} and a CU (index $0$). Each node is connected through a link of $(i,j)$ with a set $\mathcal{E}$ of links and has capacity $c_{ij}$ (Mbps) each. The DU-$n$ is connected to $\{0\}$ with a single path (e.g., shortest path) $p_{n0}$; hence, we define $r_{p_{n0}}$ as the amount of data flow (Mbps) to be transferred and routed through a path $p_{n0} \! := \!  \{(n,i_1), ..., (i_k,0) \! : \! (i,j) \! \in \! \mathcal{E} \}$. The BS functions are deployed in servers using virtual machines (VMs)\footnote{\secrev{Each BS
	function can operate as a virtual network function (VNF), and the VNFs can be executed on top of a single VM or multiple VMs}}. Each server has a processing capacity, i.e., $H_n$ for DU-$n$ and $H_0$ for CU. Naturally, a central server has a higher computing performance and capacity, \secrev{hence} $H_0 \! \geq \! H_n$. \secrev{We define} $\rho_{o}^c $ and $ \rho_{o}^d$ as the incurred computational load (cycle/Mb/s) in results of deploying the split configuration $o \! \in \! \{0,1,2,3\}$ at each CU and DU, respectively. %\secrev{The processing load at the CU is generally lower than at the DU while serving the same traffic load \cite{costdu_nikaein}.}


\textbf{Demand $\&$ Cost}. We focus on the uplink transmission where $\lambda_{n} \geq 0$ (Mbps) is the aggregate data flow of DU-$n$ to serve the users traffic;
hence, there are $N$ different flows in the network. We denote $ \bm{\alpha} = (\alpha_n, n \in \mathcal{N})$ and $\bm{\beta} \!=\! (\beta_n, n \in \mathcal{N})$ as the VM instantiation cost (monetary units) and the computing cost (monetary units/cycle) at the DUs, respectively, while $\alpha_0$ and $\beta_0$ are the respective cost for the CU. We also have a routing cost $\zeta_{p_{n0}}$ (monetary units/Mbps) for each path $p_{n0}$. This cost arises from the network links being leased from third parties or maintaining the links. 

\textbf{Problem Statement.} We have four choices of the \secrev{splits} for each BS in vRANs. What is the best-deployed split for each BS that minimizes the total network cost? The decision leads to interesting problems. Each \secrev{split} generates a different DU-CU data flow and has a distinct delay requirement. Executing more functions at the CU is more efficient in computing cost; however, it produces a higher load for xHaul links. \secrev{The BSs share the same capacitated servers and network links, where each split decision is interdependent. Moreover, the behaviour of the vRAN system (e.g., resources, performance) is complex and highly non-trivial, which makes complete assumptions of the model can be unfeasible or inaccurate. The goal is to design a framework to solve this problem by taking minimal assumptions about the model of the system.}




