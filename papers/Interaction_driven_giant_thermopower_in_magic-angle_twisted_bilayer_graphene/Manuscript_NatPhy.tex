%\documentclass[11pt,english,article,amsmath,amssymb]{article}
\documentclass{nature}
\linespread{1.2}
%\usepackage{epsfig}
%\usepackage{amsmath,amssymb, amstext}
\usepackage{amsmath}
\usepackage[T1]{fontenc}
\usepackage[latin1]{inputenc}
\usepackage{graphicx}
\usepackage{amssymb}
\usepackage{dcolumn}
\usepackage{color}
\usepackage{bm}
\usepackage[normalem]{ulem}
\bibliographystyle{naturemag}
\usepackage{url}
%\usepackage{babel}
\makeatletter
\makeatother
\usepackage{pdfpages}
\usepackage{pgffor}
\begin{document}

\title{Interaction driven giant thermopower in magic-angle twisted bilayer graphene}
\author{Arup Kumar Paul$^{1}$\footnote{equally contributed}, Ayan Ghosh$^{1}$\footnote{equally contributed}, Souvik Chakraborty$^{1}$\footnote{equally contributed}, Ujjal Roy$^{1}$, Ranit Dutta$^{1}$, K. Watanabe$^{2}$,T. Taniguchi$^{2}$, Animesh Panda$^1$, Adhip Agarwala$^{3}$, Subroto Mukerjee$^{1}$, Sumilan Banerjee$^{1}$ and Anindya Das$^{1}$\footnote{anindya@iisc.ac.in}}

\maketitle

\begin{affiliations}

\item Department of Physics,Indian Institute of Science, Bangalore, 560012, India.
\item National Institute for Materials Science, 1-1 Namiki, Tsukuba 305-0044, Japan.
\item Max Planck Institute for the Physics of Complex Systems,  N{\"o}thnitzer Stra\ss e 38, 01187 Dresden, Germany.

\end{affiliations}


\noindent\textbf{Magic-angle twisted bilayer graphene (MtBLG) has proven to be an extremely promising new platform to realize and study a host of emergent quantum phases arising from the strong correlations in its narrow bandwidth flat band. In this regard, thermal transport phenomena like thermopower, in addition to being coveted technologically, is also sensitive to the particle-hole (PH) asymmetry, making it a crucial tool to probe the underlying electronic structure of this material. We have carried out thermopower measurements of MtBLG as a function of carrier density, temperature and magnetic field, and report the observation of an unusually large thermopower reaching up to a value as high as $\sim \bf{100\mu V/K}$ at a low temperature of 1K. Surprisingly, our observed thermopower exhibiting peak-like features in close correspondence to the resistance peaks around the integer Moire fillings, including the Dirac Point, violates the Mott formula. %Surprisingly, our observed thermopower exhibits peak-like features in close correspondence to the resistance peaks around the integer Moire fillings, including the Dirac Point, which completely violates the Mott formula. 
We show that the large thermopower peaks and their %non-monotonic dependence with temperature and magnetic field 
associated behaviour arise from the emergent highly PH asymmetric electronic structure due to the cascade of Dirac revivals. Furthermore, the thermopower shows an anomalous peak around the superconducting transition on the hole side and points towards the possible role of enhanced superconducting fluctuations in MtBLG.}


%Surprisingly, our observed thermopower exhibits peak-like features closely following the resistance peaks around the integer Moire fillings and at the Dirac Point, thereby completely violating the Mott formula.



\noindent\textbf{Introduction.}\
Interactions in many body %condensed matter 
systems lead to various complex emergent quantum phenomena like superconductivity, magnetism and correlated insulating phases. Understanding these many body quantum phenomena and utilizing their various applicability remains a key focus of condensed matter research. For this reason, MtBLG %magic-angle twisted bilayer graphene 
is a promising material with its flat band~\cite{Bistritzer12233,cao2018correlated,cao2018unconventional} induced plethora of exotic states like correlated insulator~\cite{cao2018correlated,yankowitz2019tuning,wu2021chern,saito2020independent,lu2019superconductors}, superconductivity~\cite{cao2018unconventional,yankowitz2019tuning,lu2019superconductors,arora2020superconductivity,saito2020independent,park_tunable_2021}, ferromagnetism~\cite{Sharpe605}, Chern insulator~\cite{wu2021chern,das_symmetry-broken_2021,nuckolls_strongly_2020,choi_correlation-driven_2021,stepanov2020competing,Andrew2021}, quantum anomalous Hall effect~\cite{Serlin900}, nematicity~\cite{jiang2019charge,cao_nematicity_2021} and Pomeranchuk effect~\cite{rozen2020entropic,saito2021isospin}. The discovery of these emergent quantum phases together with its easy tunability using a variety of experimental knobs makes MtBLG an unprecedented platform to probe the role of interactions in its unique electronic band-structure and further the search of novel electronic properties with technological applicability. In this direction, primarily electrical transport and local spectroscopic measurements have been utilised to probe and study the nature of the various symmetry-breaking electronic states~\cite{choi2019electronic,xie2019spectroscopic,kerelsky2019maximized,wong2020cascade,zondiner2020cascade}. Notably, recent measurements of local compressibility~\cite{zondiner2020cascade} and scanning tunneling microscopy~\cite{wong2020cascade} have revealed that the Fermi surface of MtBLG is highly malleable and undergoes interaction-driven quantum phase transitions at integer fillings of Moire lattice. The key finding is the resetting of the Fermi surface with strongly PH asymmetric density of states (DOS) around the integer fillings via a cascade of Dirac revival transitions~\cite{wong2020cascade,zondiner2020cascade}. However, their unambiguous signatures in global transport measurement are still lacking. Some signature %of Fermi-surface resetting 
is observed in Hall measurement~\cite{wu2021chern,park_tunable_2021}, where the Hall carrier density suddenly resets from finite value to zero without changing its sign at the integer fillings, but, the nature and degree of PH asymmetry of the electronic structure at the transition points remain unexplored.


%%% figure one is here 
\begin{figure*}
\centerline{\includegraphics[width=1\textwidth]{Fig1.pdf}}
\caption{\textbf{Thermopower measurement set-up and device response.}  \textbf{a}, Set-up of devices. Passing a current, $I_h$ through the heater creates the temperature gradient across the device, where the colder end was directly bonded to the cold ground (c.g). The gate voltage, $V_{bg}$ controls the carrier density ($n$) of the device. The relay switches between the low-frequency and high-frequency measurement schemes. Low frequency, $2\omega$ method was used to measure the thermoelectric voltage ($V_{2\omega}$) at $\sim 13$ Hz using standard lock-in technique. High frequency ($\sim 720kHz$) thermal noise ($S_{V}$) measurement consisting LC resonant tank circuit and cryo-amplifier (CA) was used to measure the temperature difference ($\Delta T$) across the  device as $S_{V} = 2k_{B}\Delta T R$, where $R$ is the resistance of the device. \textbf{b}, Resistance versus filling fraction ($n/n_s$) as a function of increasing temperatures, where %$n$ is the carrier density induced by the gate voltage and 
$n_{s}$ is the carrier density required to fill the flat band. The top axis shows in terms of numbers of electrons ($\nu$) per Moire lattice. The resistance peaks at the positive integer fillings ($\nu$) are visible at lower temperatures. At the hole side no such peaks are observed except at full filling, and bellow $\sim500mK$ resistance drops to $\sim 1.8k\Omega$, within $n/n_s -0.5$ to $-0.75$, which shows emergence of superconductivity. %The finite resistance is due to the two-probe measurement scheme.
\textbf{c}, Measured $V_{2\omega}$ (upper panel) and $\Delta T$ (lower panel) as function of $I_h$ near the Dirac point at $1K$. \textbf{d} $V_{2\omega}$ as function of $\Delta T$ at $1K$, for different $n/n_s$, showing linear response regime. %of thermopower measurement. 
The slope of each curve gives the value of $S$. \textbf{e}, Differential resistance ($dV/dI$) versus bias current ($I_{sd}$) as function of temperature, at $n/n_s \sim -0.67$, where critical current ($I_c$) is maximum. The dark region corresponds to the superconducting region. The green solid line shows the theoretically generated $I_c$ using the BCS theory, $I_{c}(T) = I_{c}(0)(1-T/T_{c})^{2}$, where the $I_{c}(0)$ is the experimentally measured value at $T = 20mK$. %and $I_{c}(T)$ at a given temperature was tried to match to the corresponding experimentally measured $I_{c}$ in order to find the $T_c$ from the extrapolation of the curve. This was repeated for other carrier densities and shown in the SI. 
\textbf{f}, $dV/dI$ with increasing perpendicular magnetic field at $n/n_s \sim -0.67$ and at $\sim 20mK$. The dark black region corresponds to the superconducting region, which is killed at $B_{\perp} = 0.1T$. %The Fraunhofer-like pattern can be explained by the interference between percolating superconducting paths separated by the normal islands, which is generic feature in MtBLG due to twist-angle in-homogeneity~\cite{zondiner2020cascade,uri_mapping_2020,park_tunable_2021}.
}
\label{Figure1}
\end{figure*}
%%%%%%%%%%%% figure ends


In this context, thermopower or the Seebeck effect is a unique %ly sensitive 
tool to probe the PH asymmetry of the electronic structure of MtBLG. %Moreover, 
Compared to %conventional 
electrical transport, %measurements
it is relatively non-invasive as an open circuit voltage ($\Delta V$) is measured across the sample in the presence of a small temperature gradient ($\Delta T$) relative to the sample temperature. In the linear regime, using semi-classical Boltzmann transport theory and assuming energy independent scattering time, the Seebeck coefficient ($S = -\Delta V/\Delta T$) can be written as $S=-(k_\mathrm{B}/Te)[\int (\epsilon-\mu)g(\varepsilon)(-df/d\varepsilon)d\varepsilon]/[\int g(\varepsilon)(-df/d\varepsilon)d\varepsilon]$, where $e$, $T$, $\mu$, $g(\varepsilon)$ and $-df/d\varepsilon$ are respectively the electronic charge, temperature, chemical potential, DOS and derivative of Fermi function. It can be seen that the numerator is an odd function due to the ($\epsilon-\mu$) term, and thus, the sign and magnitude of $S$ depend on nature and extent of asymmetry of the DOS around the chemical potential. %only the asymmetric part of the DOS ($g(\varepsilon)$) around the chemical potential gives non-zero $S$. In addition, the sign and magnitude of $S$ depend on the nature and the extent of the asymmetry in the DOS. 
As a result, $S$ is a highly sensitive probe to study the electronic structure around the %singular points
transition points of MtBLG. %of the Fermi surface. and Fermi-surface reconstructions like at Lifshitz transitions. 
Moreover, MtBLG with superconducting dome around half filling analogous %reminiscent 
of high-$T_c$ cuprate superconductors is an ideal playground to study the %thermoelectric transport 
thermopower response %around the superconductor transition 
as it has been employed to study the superconducting fluctuations in cuprates\cite{xu_vortex-like_2000}.


Motivated by these, we have extensively explored the thermopower response of MtBLG and non magic-angle tBLG devices. Unlike previous works involving graphene and tBLG~\cite{zuev2009thermoelectric,PhysRevB.80.081413,nam2010thermoelectric,wang2011enhanced,duan2016high,ghahari2016enhanced,mahapatra2020misorientation,ghawri2020excess}, we have utilized Johnson noise thermometry~\cite{Srivastaveaaw5798,fong2012ultrasensitive,crossno2016observation,betz2013supercollision} to directly measure the temperature gradient across the MtBLG device and accurately determine $S$ across a temperature ranging from $100mK$ to $10K$. Our measurements reveal intricate dependence of $S$ on carrier density ($\nu$), temperature ($T$), and magnetic field ($B$). Our key observations are following: i) The measured thermopower at low temperatures deviates completely from the expected zero-crossings following the semi-classical Mott formula~\cite{PhysRev.181.1336}. Instead, the thermopower exhibits peak-like features at all positive integer fillings including the Dirac point. ii) We observe a non-monotonic temperature dependence of the thermopower. The thermopower reaches a record high value of $\sim 100\mu V/K$ at $1K$ for half filling of the conduction band. iii) We also observe unusually large peaks in $S$ $\sim - (10-15)\mu V/K$ at sub-Kelvin temperatures around the superconducting transition tracing the superconducting dome in the hole side. We explain the first two results qualitatively using a simple model within self-consistent Hartree-Fock (HF) approximations showing emergent highly PH asymmetric DOS %due to the Dirac revivals 
at integer fillings. Furthermore, we attribute the anomalous peaks around the $T_c$ to enhanced superconducting fluctuations due to inherent PH asymmetry in MtBLG. Our work highlights the ability of thermopower to independently provide unique insights into the novel quantum phenomena observed in MtBLG and opens a new route to achieve high thermoelectric cooling devices and generators at cryogenic temperatures.


\noindent\textbf{Set-up and device response.}\
Figure 1a shows the schematic of the device and the measurement setup for thermopower measurement. The devices consist of hBN encapsulated twisted bilayer graphene (tBLG) on a $Si/SiO_2$ substrate. %The typical length and width of the devices are $\sim 6\mu m$ and $\sim 2\mu m$, respectively. The usual `tear and stack' technique\cite{cao2018correlated,cao2018unconventional} is used to fabricate the device and is described in detail in the supplementary information (SI-1). 
The details are described in method and supplementary information (SI-1). %For the resistance measurement, we employ the low-frequency ($\sim 13 Hz$) lock-in technique (SI-3).
For the thermopower measurement, an isolated gold heater line, as shown in Fig. 1a, is placed parallel to one side of the tBLG. %at a separation of $\sim 3\mu m$. Passing a current ($I_{\omega}$) through the heater creates a temperature gradient across the length of the tBLG as depicted by the color gradient (red to blue) in Fig. 1a. As a result, the contact near to the heater will be hotter ($T_h$) compared to the far contact ($T_c$). The temperature of the far contact ($T_c$) is maintained at the bath temperature of the cryo-free dilution fridge by directly anchoring it to the cold finger attached to the mixing chamber plate, which we call a cold ground (c.g). 
To determine the thermopower or Seebeck coefficient ($S$), one needs to measure the generated thermoelectric voltage and the temperature difference ($\Delta T = T_h - T_c$). %between the contacts. 
We have utilized well established $2\omega$ lock-in technique~\cite{zuev2009thermoelectric,PhysRevB.80.081413,nam2010thermoelectric,wang2011enhanced,duan2016high,ghahari2016enhanced,mahapatra2020misorientation} for measuring the thermoelectric voltage ($V_{2\omega}$) at $\omega \sim 13 Hz$. %(SI-3). 
To measure $\Delta T$, we have utilized Johnson noise thermometry~\cite{Srivastaveaaw5798,fong2012ultrasensitive,crossno2016observation,betz2013supercollision}. %As shown in Fig. 1a, the thermometry circuit consists of a LC resonant ($f_r\sim 720kHz$) tank circuit, followed by a cryogenic amplifier (ca). The relay sitting at the mixing chamber plate (Fig. 1a) is used to switch between the thermoelectric voltage and temperature measurement. 
The details of the noise thermometry setup can be found in our earlier works~\cite{Srivastaveaaw5798,PhysRevLett.126.216803} and shown in SI-3. The excess thermal noise, $S_V=2k_B\Delta TR$ measured across the sample is used to determine the $\Delta T$ (see SI-6 and SI-7), where $k_B$ is the Boltzmann constant and $R$ is the resistance of the device. Fig. 1c shows the measured $V_{2\omega}$ and the $\Delta T$ as a function of the heater current at a bath temperature ($T$) of 1K. In Fig. 1d, we plot $V_{2\omega}$ with $\Delta T$ for MtBLG %of $\sim 1.05^0$
at different carrier densities ($n$). The linearity of the plots in Fig. 1d suggests that we are in the linear regime, and the slope of each curve gives the $S$ for a given $n$. We have measured the $S$ from $100mK$ to $10K$ in the linear regime by adjusting the heater current such that the $\Delta T$ always remains much smaller than $T$ (SI-7). We have used three devices with twist angles of $\sim 0.26^0$, $1.05^0$ $1.86^0$.

The gate-dependent resistance ($R$) of MtBLG for different temperatures is shown in Fig. 1b. Here the gate voltage is replaced with an equivalent Moire filling factor, $\nu=4n/n_s$, where $n$ is the carrier density induced by the gate voltage and $n_s$ is the carrier density required to full-fill the flat band (4 electrons/holes per Moire unit cell). As can be seen from the $R$ versus $\nu$ response%at T=20 mK
, multiple resistance peaks appear at positive integer fillings, including the Dirac point, and these peak features survive up to $\sim$50K and above. On the contrary, for negative filling, we see the prominent resistance peak at $\nu=-4$, and between $\nu=-2$ and $\nu=-3$, %with the value of 
the resistance drops %very fast 
below $600mK$ and saturates like a plateau %with resistance minima 
at $\sim 1.8 k\Omega$ %at $20mK$ 
showing the emergence of superconductivity. %These observations are consistent with earlier reports for MtBLG~\cite{cao2018correlated,yankowitz2019tuning,wu2021chern,saito2020independent,lu2019superconductors,cao2018unconventional,arora2020superconductivity,park_tunable_2021}. %with a strong correlation effect.
The resistance value at the full-filling ($\nu=\pm 4$) continuously decreases with increasing temperature. %up to much higher T $\sim 100K$. From its activated nature, we extract the gap between the flat and dispersive band and is found to be of the order of $\Delta \sim 10 meV$. 
On the other hand, resistance value at $\nu=0$ and $2$ decreases with increasing temperature up to $\sim 10 K$ and then increases linearly, showing metallic nature (SI-10 and SI-11). %Furthermore, it can be seen (SI-10) that there is a crossover from metallic nature to insulating one at a higher temperature due to interband excitation of the carriers between the flat and dispersive bands~\cite{polshyn_large_2019}. These observations are consistent with earlier reports for MtBLG~\cite{cao2018correlated,yankowitz2019tuning,wu2021chern,saito2020independent,lu2019superconductors,cao2018unconventional,arora2020superconductivity,park_tunable_2021}. %with a strong correlation effect. %From the crossover temperature, $\sim 120K$ at the Dirac point, one can estimate the bandwidth ($W$) and found to be of the order of $\sim 10 meV$ for our device (SI-11). 
Fig. 1e and 1f plot the evolution of differential resistance ($\frac{dV}{dI}$) versus bias current ($I_{sd}$) response with temperature and perpendicular magnetic field, respectively, at $\nu \sim -2.5$, and confirms the existence of the superconductivity though the resistance is measured in two-probe geometry (Method and SI-12). %Comparing the experimental data with the theoretically generated critical current ($I_c$) versus $T$ using BCS theory (sky blue line in Fig. 1e) we extract the critical temperature ($T_c$) and found to be $\sim 500mK$ at $\nu \sim -2.5$ (SI-12 for details). The measured value of $T_c$ and critical field ($B_{c} \sim 100mT$) of our device matches reasonably well with the available data for MtBLG~\cite{cao2018unconventional,yankowitz2019tuning,lu2019superconductors,arora2020superconductivity,saito2020independent,park_tunable_2021}. %Note that the Fraunhofer-like patterns in Fig. 1f can be explained by the interference between percolating superconducting paths separated by the normal islands, which is usually unavoidable in MtBLG due to twist-angle in-homogeneity~\cite{zondiner2020cascade,uri_mapping_2020,park_tunable_2021} 

%%%%%%% figure begins
\begin{figure*}
\centerline{\includegraphics[width=1\textwidth]{Fig2.pdf}}
\caption{\textbf{Thermopower response at integer Moire fillings.} \textbf{a}, Measured thermopower of MtBLG with carrier fillings at $10K$. The dashed horizontal line corresponds to zero thermopower and vertical dashed lines correspond to the integer number of electron fillings of the Moire lattice. \textbf{b}, Measured thermopower at several temperatures from $0.2K$ to $7K$. For the clarity the data are shifted by $20\mu V/K$ along the y-axis. The dashed horizontal lines are the zero thermopower for the corresponding temperatures. \textbf{c}, Temperature dependence of $S$ for the flat band in log-log scale at $n/n_{s} = 0, 0.25, 0.5, 0.75$, and \textbf{d}, for dispersive bands at hole-side ($n/n_{s} = -1.31$) and electron-side ($n/n_{s} = 1.47$). \textbf{e}, $S$ with $B_{\parallel}$ for a different thermal cycle at $2K$. \textbf{f}, Comparison of $S$ at half-filling with temperature between zero and $B_{\perp}$ = $3.5T$. \textbf{g}, Derivative of resistance of MtBLG with carrier density at $0.2K$ according to Mott formula. The dashed horizontal line correspond to zero derivative line. The clear sign changes at $\nu = 0, 1, 2$ are seen but absent in the $S$ data in Fig. 2b. \textbf{h}, Measured $S$ and derivative of resistance for $0.26^{0}$ non-magic angle tBLG at $1K$. The qualitative agreement of sign changes between them can be seen.}
\label{Figure2}
\end{figure*} 
%%%%%%% figure ends

\noindent\textbf{Band reconstruction of MtBLG probed by thermopower.}\ 
The Fig. 2a and 2b show the measured thermopower versus $\nu$ at several temperatures, from $200mK$ to $10 K$ for MtBLG. %For the clarity the data has been shifted by a constant amount ($20\mu V/K$) in Fig. 2b. 
At $10K$ (Fig. 2a), thermopower has approximate mirror symmetry for both conduction and valence band albeit with opposite signs. It can be seen that the thermopower changes its sign at the Dirac point, at flat band full-filling ($\nu \sim \pm 4$) and around $\nu \sim \pm 1$. The sign of the thermopower depends on the type of the carriers, positive for hole-like and negative for electron-like carriers, and its magnitude goes to zero at the symmetric points of the electronic structure as described by the semi-classical equation. Like, at the Dirac point, the density of states (DOS) goes to zero symmetrically from both the conduction and valence band. Similarly, at the band full-filling with the energy gap between the flat and higher energy-dispersive bands, the $S$ is expected to change the sign. One more sign change is expected at the middle of the conduction or valence band as the single-particle DOS of the flat band %gradually increases and 
reaches a maximum (van-hove singularity - VHS) around $\nu=\pm 2$. %and followed by decreases gradually towards the full-filling. 
If the DOS is symmetric around the maxima, one would expect a sign change in $S$ exactly at $\nu=\pm 2$. However, the inherent asymmetry of the DOS, which is complex for MtBLG, of the conduction band or valence band can give rise to the sign change shifted from $\nu=\pm 2$. %at finite temperatures. 

As we decrease the temperature below $10K$, the apparent asymmetry of the $S$ (Fig. 2b) between the conduction and valence band grows similar to the asymmetry observed in the resistance data in Fig. 1b. Most importantly, the thermopower exhibits a positive peak around $\nu \sim 2$, and its magnitude increases rapidly with decreasing temperature and reaches a maximum value of $\sim 95-100\mu V/K$ at $\sim 1K$, followed by a decrement of the magnitude with a further reduction of the temperature. Similar, positive peaks are also seen around $\nu \sim 1,3$ at $\sim 2K$ and at the Dirac point below $1K$. %the $S$ shows positive peaks around the other positive integer fillings $\nu \sim 1,3$ at $\sim 2K$, including around the Dirac point below $1K$. 
The observed positive peak in thermopower at the positive integer fillings, including the Dirac point, is quite striking. Any energy gap $\gtrsim k_\mathrm{B}T$ either from the single-particle band structure or induced by electronic interactions will give a sign change of $S$. In particular, one would expect $S$ to go to zero at the resistance maxima, i.e., at the integer fillings as the Mott formula~\cite{PhysRev.181.1336} $S=(\pi^2 k_B T/3e)(d\ln(R)/dn)g(\epsilon)$ %$dln(R)/dn$ 
gives zero at those points and shown in Fig. 2g for $0.2K$. %according to the Mott formula~\cite{PhysRev.181.1336,zuev2009thermoelectric,PhysRevB.80.081413,nam2010thermoelectric,wang2011enhanced,duan2016high,mahapatra2020misorientation} as shown in Fig. 2g for $0.2K$. %In Fig. 2b, we have shown the $\frac{dln(R)}{dn}$ for the device-1 at $1K$ and 
Thus, one can see a complete violation of the Mott formula for MtBLG. The violation persists even up to $10K$, as shown in SI-13. On the contrary, for non magic-angle tBLG devices, the measured sign of $S$ and the Mott formula matches well, as shown in Fig. 2h for $\sim 0.26^0$ (SI-Fig. 15b for $1.86^0$).


The recurring thermopower peaks (Fig. 2b) at integer fillings with a positive sign (which usually occurs for hole-like carriers) suggest, at least within an effective single-particle picture, repeated restructuring of the Fermi surface at integer fillings such that overall hole-like carriers are dominant. The pliable Fermi surfaces due to interactions around the integer fillings have been reported in MtBLG and Stoner like transitions~\cite{wong2020cascade,zondiner2020cascade,wu2021chern,park_tunable_2021} have been observed experimentally. The key features of these transitions are -- a Lifshitz transition followed by a Dirac revival, which essentially gives rise to large asymmetric DOS around the transition point such that, for $\nu>0$, from one side (left side of the transition), the DOS rapidly drops whereas other side (right side of the transition) the DOS increases gradually, similar to a sawtooth. Such asymmetric DOS can give rise to peak in $S$ with large value around the transition point as discussed in the theoretical section. It can be seen in Fig. 2h and SI-13 that for non magic-angle %twisted angles with $\sim 0.35^0$ and $1.66^0$, 
tBLG devices, we do not observe any thermopower peaks, and the measured $S$ is around $\sim 1\mu V/K$ at $\sim 1K$ as expected for graphene-based devices at such low temperatures~\cite{zuev2009thermoelectric,PhysRevB.80.081413,nam2010thermoelectric}.

The temperature dependence of $S$ for different integer fillings, including Dirac point, is shown in Fig. 2c. The common key feature is the non-monotonic temperature dependence of $S$ with a maxima at a certain temperature, which depends on the fillings. For example, at $\nu \sim 2$ and $\sim 3$ the peak appears around $\sim 1K$ whereas it is $\sim 0.3K$ for $\nu \sim 1$ and the Dirac point. The deviation from the linear $T$ dependence of $S$ again suggests the strong violation of Mott's formula~\cite{PhysRev.181.1336} for the flat band of MtBLG. However, it can be seen that for the dispersive bands ($n/n_{s} \sim 1.47$ and $n/n_{s} \sim -1.31$) the $S$ increases almost linearly with $T$ (Fig. 2d), consistent with the Mott formula. %and its magnitude qualitatively agrees with the Mott formula (see SI).
Furthermore, the response of $S$ with in-plane magnetic field ($B_{\parallel}$) underlies the nature of the ground states at different integer fillings. As can be seen in Fig. 2e (for a different thermal cycle as shown in SI-14) the thermopower peaks increase with $B_{\parallel}$ at $\nu \sim 1$ %and $3$
, but decreases at $\nu \sim 2$. %and remain unchanged at the Dirac point. 
These observations are consistent with the cascade of Dirac revival picture in Ref~\cite{zondiner2020cascade}, where the emergence of flavored symmetry breaking in MtBLG with %spin 
polarized ground state at $\nu \sim 1$ %and $3$ 
strengthens the transition, and thus make it more PH asymmetric DOS resulting in higher $S$. At $\nu \sim 2$ the value of $S$ decreases with both $B_{\parallel}$ and $B_{\perp}$, as shown as a function of $T$ in Fig. 2f. It can be noticed that the peak position of the $S$ shifted to lower temperature $\sim 0.6K$ at $B_{\perp} = 3.5T$ with a value of $\sim 70 \mu V/K$, thus emphasizing its tunability for use as thermoelectric cooling devices at cryogenic temperatures. It should be noted that the $S$ of the dispersive band and non magic-angle tBLG devices remain insensitive to $B_{\parallel}$ (see SI-14).

%%%%% figures begin
\begin{figure*}
\centerline{\includegraphics[width=1\textwidth]{Fig3.pdf}}
\caption{\textbf{Thermopower across the superconducting transition.} \textbf{a}, 2D colourmap of resistance as a function of temperature and carrier filling for the hole side flat band of MtBLG. The dark blue portion corresponds to the superconducting region around the weaker Mott peak at $n/n{s} \sim -0.55$. The open circles are the $T_{c}$ at different carrier densities as determined from the differential resistance versus critical current plot as a function of temperature as shown in Fig. 1e. \textbf{b}, 2D colourmap of measured thermopower as a function of temperature and carrier filling. The blue dark portions are the negative thermopower peak as shown as the cut lines in Fig. 3d for the carrier densities marked by the white vertical arrows. The regions $I$ and $II$ correspond to two different dome like portions, where region $II$ matches well with the superconducting dome seen in Fig. 3a as shown by the white dashed line, which is the trace of $T_{c}$ as shown in Fig.  3a. The other white dashed line enclosing the regions $I$ is the guiding line. \textbf{c}, The measured thermopower at a $B_{\perp} = 0.1T$ without peak like features in $S$. \textbf{d}, Open circles with the connected lines in different panels are the cut lines of measured $S$ as a function of temperature from Fig. 3b at $n/n_{s} = -0.56, -0.65, -0.74$ and $-0.8$. The solid lines are corresponding cut lines of $dR/dT$ from Fig. 3a. \textbf{e}, Thermopower cut lines for $B_{\perp} = 0.1T$ at $n/n_{s} = -0.65.$ and $-0.8$.}
\label{Figure3}
\end{figure*} 
%%%%% figures end

\noindent\textbf{Anomalous thermopower response around the superconducting dome.}\ 
As shown in Fig. 2b, there are no thermopower peaks for the valence flat band of the MtBLG device in the temperature range of $2-10K$. This is consistent with the weaker cascaded transitions observed for the hole side in Ref~\cite{zondiner2020cascade}. It was shown that a larger twist angle $\sim 1.1^0$ is required to observe the sawtooth behavior in the DOS for the hole side. However, for the MtBLG device, some peak-like features are developed within $\nu \sim -2$ to $-3.5$ below $2K$. The number of peaks and their positions in filling changes with decreasing temperature; one weak negative peak is seen at $2K$, whereas three prominent negative peaks are observed at $0.2K$. Fig. 3a shows the two-dimensional (2D) color map of the resistance with temperature and fillings. The darker blue region corresponds to the superconducting phase seen close to a weak resistance peak at $n/n_{s} \sim -0.55$ (details in SI-12). The open circles denote superconducting transition temperature $T_{c}$ as a function of filling, obtained from critical current measurement shown in Fig. 1e and SI-12.

Fig. 3b shows the corresponding 2D color map of $S$ at zero magnetic fields. Here the darker blue shaded ribbons enclosing dome-like structures correspond to negative thermopower peaks. A closer view shows two inter-penetrating dome-like structures marked by the dashed white lines and labeled as $I$ and $II$ in Fig.3b. The dashed white line enclosing the region $II$ is the trace of $T_{c}$ as shown in Fig.3a, whereas the white dashed line enclosing the region $I$ is a guide to the eye to follow the locus of the broad negative thermopower peaks. As evident from Fig. 3a and 3b, there are clear resemblances between the peak position in $S$ enclosing region $II$ and $T_{c}(n)$ dome. This can be further ascertained from Fig. 3d, where the $S$ from Fig. 3b and $dR/dT$ for Fig. 3a are plotted as a function of $T$ for fixed fillings, $n/n_{s} = -0.56$ and $-0.65$. Both $S$ and $dR/dT$ exhibits prominent peaks around $T_c$. However, peak in $S$ enclosing region $I$ in Fig. 3b has hardly any direct correspondence in the resistance data in Fig. 3a, as can be seen in Fig. 3d, where $S(T)$ exhibits broad peak at $n/n_{s} = -0.74$ and $-0.80$, but $dR/dT$ does not show any such feature around the same temperature. %For filling below $n/n_{s} \sim -0.7$, there are no peaks in $S$ around $T_c$, where $dR/dT$ still, of course, exhibits peaks demarcating the superconducting region (Fig.3a). 

Apart from the correlation between locus of the thermopower peak in the $n-T$ plane and $T_c(n)$ over a large part of the superconducting dome, the most important clue for the possible origin of the unusual thermopower peak is obtained by applying a $B_{\perp}$. As shown in Fig. 3c, the peak in $S$ completely disappears with the application of tiny $B_\perp=0.1$T. This is demonstrated in Fig. 3e by plotting $S(T)$ for $n/n_{s} = -0.65$ and $-0.8$. %at $B_\perp=0.1$T. 
These observations suggest that anomalous peak in $S$ in Fig. 3b, particularly enclosing the region $II$, directly relates to the superconductivity of MtBLG. One cannot help but to notice the superficial resemblance of the trace of the broad thermopower peak enclosing region $I$ with the putative pseudogap temperature line in the doping-temperature plane of cuprates~\cite{timusk_pseudogap_1999}. However, the origin of the boundary enclosing region $I$ remains unclear at this moment and will be an interesting direction for future studies, like the possible role of nematicity and competing orders in MtBLG~\cite{cao_nematicity_2021}. Nonetheless, even the observed anomalous thermopower peak around $T_{c}$ is quite striking. Anomalously large thermopower response around $T_{c}$, and even extending far above $T_{c}$, have been reported in transverse thermoelectric coefficient, i.e. the Nernst coefficient~\cite{xu_vortex-like_2000}, for high $T_{c}$ cuprate superconductors. However, experimental observations of peaks in the longitudinal component of $S$ are scarce \cite{howson_anomalous_1989} and remain controversial. Such thermopower peaks have been theoretically predicted~\cite{howson_anomalous_1989,lu_fluctuation_1995} to exist above, albeit close to, $T_c$ from superconducting fluctuations under certain situations, like for a superconductor in the dirty limit~\cite{howson_anomalous_1989,lu_fluctuation_1995}. %In SI-10, we rule out any experimental artifact in our observations of anomalously large $S$ around the superconducting dome.
The observed thermopower in our MtBLG device might have similar origin, as disorder due to twist-angle inhomogeneity~\cite{zondiner2020cascade,uri_mapping_2020} is naturally present in the system. Effects of superconducting fluctuations in the thermopower~\cite{howson_anomalous_1989,lu_fluctuation_1995} are expected to be much more enhanced in MtBLG near half filling due to inherent PH asymmetry from the low-energy VHS of the flat band as well as the emergent PH asymmetry due to the proximate weak cascade transition in the hole side.

%\textcolor{blue} {Peaks have been theoretically predicted in the Peltier conductance, i.e. the quantity $S/R$~\cite{howson_anomalous_1989,lu_fluctuation_1995} to exist above, albeit close to, $T_c$ from superconducting fluctuations in certain situations, like for a superconductor in the dirty limit~\cite{howson_anomalous_1989,lu_fluctuation_1995}. However, it is not immediately clear whether this also results in a peak in the thermopower owing to the the non-universal dependence of the resistance close to $T_c$. Nevertheless, a peak in the thermopower near $T_C$ is strongly indicative of particle-hole asymmetry in superconducting fluctuations and presumably the parent superconducting state itself.}


%%%%%%%% Figures begin
\begin{figure*}
\centerline{\includegraphics[width=1.0\textwidth]{Fig4.pdf}}
\caption{\textbf{Cascade of Dirac revivals and thermopower peaks around integer fillings.} \textbf{a}, The occupation ($n_\alpha,~\alpha=1,\dots,4$) of individual flavors as a function  filling $n/n_s$, obtained from Hartree-Fock (HF) calculations for $T=0.005W$ and a local inter-flavor interaction $U=1.2W$. At zero filling, the Dirac cones corresponding to the four spin-valley degrees of freedom are degenerate. A cascade of Stoner-like transitions close to the integer fillings lead to complete filling of one, two and three of the flavors successively while the filling of the remaining flavor(s) resets to Dirac point. \textbf{b}, The resultant HF DOS $g(\epsilon)$ at the chemical potential at $T=0.005W$ exhibits the sawtooth feature. The effective single-particle DOS $g(\epsilon)$ changes drastically at each integer fillings and shows strong low-energy particle-hole asymmetry (SI-Fig. 20). \textbf{c}, The calculated thermopower $S$ in HF approximation (red circles with line) shows peak-like features around the integer fillings due to the Dirac revivals (Fig. {\bf a}). The top to bottom panels are in the order of decreasing temperatures $T=0.27W,0.14W,0.08W,0.005W$. The thermopower $S_0$ obtained using 'rigid' non-interacting single particle DOS (Fig. {\bf c}, top panel, inset) is shown by the solid blue lines. The non-interacting $S_0$ exhibits one sign change around the half filling for higher temperature $T\gtrsim 0.08W$, and multiple sign changes across the two VHSs %van Hove singularities 
of the non-interacting DOS at very low temperature (the bottom panel), but these peaks depend on the details of non-interacting DOS and are not necessarily tied to integer the fillings, unlike $S(\nu)$ in the interacting cases. \textbf{d}, Non-monotonic temperature dependence of thermopower at integer fillings from the cascaded transitions. The Dirac revived symmetry broken state at $n/n_s=0.50$ only gets stabilized at finite temperature for the particular non-interacting DOS, as indicated by a sign change in $S$ around $T\sim 0.1W$.}
\label{Figure4}
\end{figure*} 
%%%%% figures end

\noindent\textbf{Emergent low-energy particle-hole asymmetry and giant thermopower peaks.}\ 
%In this section we discuss the possible origin of the stark violation of Mott relation around the integer fillings ($0<\nu<4$), namely the presence of giant thermopower peaks concomitant with the resistance peaks. 
As already mentioned, the thermopower peaks suggest strong emergent low-energy %particle-hole 
PH asymmetry of the putative correlated states at integer fillings, at least, within the effective single-particle or Hartree-Fock descriptions of various possible symmetry broken states \cite{PhysRevX.8.031089,PhysRevX.10.031034,shavit2021theory}. As discussed in method and SI-16, we use a simple minimal model \cite{zondiner2020cascade} with four fermionic flavors, corresponding to the spin and valley degrees of freedom, each described in terms of a single-particle DOS \cite{zondiner2020cascade,Bistritzer12233,PhysRevX.8.031087}, and interacting via a local Coulomb interaction. We treat the latter via self-consistent Hartree-Fock (HF) approximation and use the resulting HF DOS to calculate the resistivity and thermopower as a function of filling and temperature, using Kubo formulae (SI-16). We have used different non-interacting DOSs, obtained from both effective continuum models \cite{Bistritzer12233,PhysRevX.8.031087}, with and without lattice relaxation effects \cite{PhysRevX.8.031087}, as well as tight-binding model \cite{PhysRevB.85.195458} (SI-16). The main results are summarized in Fig. 4, where the peak value of $S$ reaches $\sim 50-100~\mu\mathrm{V/K}$ for $\nu\simeq 2,3$ at $T\sim 0.1 W$, consistent with our experimental observations (Fig. 2b). The temperature range $T\simeq 0.005W-0.27W$ corresponds to $\sim 200~\mathrm{mK}-13~\mathrm{K}$, for a bandwidth $2W\simeq 10~\mathrm{meV}$. For comparison, in Fig. 4c (solid blue lines), we have shown the $S_0$ for the non-interacting case (see Fig. 4c caption for details). We find the thermopower peak around an integer filling to be a robust feature whenever the Dirac revival is stabilized within the HF approximation, and support the simultaneous presence of thermopower (Fig. 2b) and resistance (Fig. 1b) peaks, as well as the non-monotonic temperature dependence of $S$ (Fig. 2c) in our experiment.

%the robust peak position of $S$ being independent of a large temperature range in our experiment (Fig. 2b).
%We find that the simultaneous presence of thermopower (Fig. 2b) and resistance (Fig. 1b) peaks, as well as the non-monotonic temperature dependence of $S$ (Fig. 2c), can be understood in terms of cascade of Dirac revivals proposed in Ref.\cite{zondiner2020cascade} for describing local electronic compressibility measurements. 
 

\noindent\textbf{Discussion.}\ 
Our theory qualitatively captures the thermopower peaks, but the $S(n)$ follows an overall `background' profile dictated by the non-interacting $S_0(n)$ (Fig. 4c) and its sign change around the half filling. There could be several reasons behind the deviation of $S$ obtained from HF approximation compared to the experimental one, e.g., effects of more complex and realistic single-particle DOS for MtBLG than the used continuum model~\cite{zondiner2020cascade,Bistritzer12233,PhysRevX.8.031087}, twist angle inhomogeneity~\cite{zondiner2020cascade,uri_mapping_2020} and strong correlations in the strange metal state~\cite{PhysRevLett.124.076801} (see SI-16 for a detailed discussion). Moreover, we should note that there are theoretical models~\cite{PhysRevX.8.031089,PhysRevX.10.031034,shavit2021theory} which lead to a small gap ($\Delta$) at the Dirac revivals. This will be consistent with the simultaneous presence of thermopower, and resistance peaks at integer fillings provided $k_\mathrm{B}T\gtrsim \Delta$. At very lower temperatures, $S$ is expected to change sign across the position of resistance peak. Thus, our thermopower results put a tighter upper bound, $\Delta \sim 0.1-0.2$ meV (activation gap in SI-11), on the correlation-induced gap at integer fillings. In the SI-16, we also discuss the expected thermopower from various other kinds of ground states %at the integer fillings of MtBLG and 
and their possible signatures in our measurements.

%Our theory qualitatively captures the thermopower peaks, but the $S(n)$ follows an overall `background' profile dictated by the non-interacting $S_0(n)$ (Fig.4c) and its sign change around half filling. There could be several reasons behind the deviation of theoretically calculated $S$ from the experimental one; the single-particle DOS is much more complex for MtBLG than the used continuum model~\cite{zondiner2020cascade,Bistritzer12233,PhysRevX.8.031087}, the effect of twist angle inhomogeneity~\cite{zondiner2020cascade,uri_mapping_2020}, effect of strongly correlated metal, e.g., a strange metal~\cite{PhysRevLett.124.076801,Jaoui2021} (detailed discussion in SI). Moreover, we should note that there are theoretical models~\cite{PhysRevX.8.031089,PhysRevX.10.031034,shavit2021theory} which lead to a small gap ($\Delta$) at the Dirac revivals. This will be consistent with the simultaneous presence of thermopower, and resistance peaks at integer fillings provided $k_\mathrm{B}T\gtrsim \Delta$. At very lower temperatures, $S$ is expected to change sign across the position of resistance peak. Thus, our thermopower results put a tighter upper bound, $\Delta \sim 0.1-0.2$ meV (activation gap in SI-11), on the correlation-induced gap at integer fillings. In the SI, we also discuss the expected thermopower from various other kinds of ground states %at the integer fillings of MtBLG and its role in our measurements. 

\noindent\textbf{Conclusion.}\ 
In summary, our experiments reveal unusual low-temperature thermopower response and anomalously large peaks in Seebeck coefficient originating from emergent highly particle-hole asymmetry in the flat band of MtBLG. %electron-doped flat band of MtBLG. Remarkably, these thermopower peaks appear in concurrence with the resistance peaks. %associated with correlated insulating/semi-metallic ($d\rho/dT<0$ $\nu=0,2$) and metallic ($d\rho/dT>0$, $\nu=1$) states at integer fillings. 
In the hole side of the MtBLG flat band, close to half-filling, our measurements also reveal two interpenetrating dome-like structures, traced by peaks in the thermopower on the doping-temperature plane. The boundary of the dome at lower temperature coincides with the line of superconducting transitions, presumably providing a rare glimpse of the enhanced superconducting fluctuations in MtBLG. %However, the origin of the mysterious dome at higher temperature remain completely unclear at present, and deciphering its nature would be a challenging problem for future experimental and theoretical studies. 
%We show that a simple minimal theoretical model of cascaded Dirac revival transitions and the resulting emergent low-energy particle-hole asymmetry qualitatively explain our results. %the thermopower peaks and their temperature dependence at integer fillings. %However, the non-standard sign (positive) of the thermopower even away from the integer fillings in the electron side indicates correlated metallic states emerging out of the parent Dirac revived states at lower temperature. Further studies on thermopower over larger temperature range will provide important insights into the putative `strange metallic' states in MtBLG and will be complementary to other local probes and global transport measurements. 
%Finally, our experiments on thermoelectric response of MtBLG showing large magnitude of the order of $\sim k_{B}/e$ pave the way to design future high-performing solid-state thermoelectric cooling devices at cryogenic temperatures using twisted bilayer graphene and their hybrids.


\newpage

\bibliography{ref}

\pagebreak
\section{Methods}

\subsection{Device fabrication and measurement scheme:}
The devices consist of hBN encapsulated twisted bilayer graphene (tBLG) on a $Si/SiO_2$ substrate. The typical length and width of the devices are $\sim 6\mu m$ and $\sim 2\mu m$, respectively. The usual `tear and stack' technique~\cite{cao2018correlated,cao2018unconventional} is used to fabricate the device and is described in detail in the supplementary information (SI-1). For the resistance measurement, we employ the low-frequency ($\sim 13 Hz$) lock-in technique (SI-3). For the thermopower measurement, an isolated gold line, as shown in Fig.1a, is placed parallel to one side of the tBLG at a separation of $\sim 3\mu m$. Passing a current ($I_{\omega}$) through the heater creates a temperature gradient across the length of the tBLG as depicted by the color gradient (red to blue) in Fig. 1a. As a result, the contact near to the heater will be hotter ($T_h$) compared to the far contact ($T_c$). The temperature of the far contact ($T_c$) is maintained at the bath temperature of the cryo-free dilution fridge by directly anchoring it to the cold finger attached to the mixing chamber plate %(see SI-3)
, which we call a cold ground (c.g). To measure $\Delta T$, we have utilized Johnson noise thermometry. As shown in Fig. 1a, the thermometry circuit consists of a LC resonant ($f_r\sim 720kHz$) tank circuit, followed by a cryogenic amplifier (ca). The relay sitting at the mixing chamber plate (Fig. 1a) is used to switch between the thermoelectric voltage and temperature measurement. 


\subsection{Activation gaps, band-width and superconducting transition temperature of MtBLG:}
The value of the resistance at the full filling ($\nu=\pm 4$) continuously decreases with increasing temperature up to much higher T $\sim 100K$. %From its activated nature, we extract the gap between the flat and dispersive band and is found to be of the order of $\Delta \sim 10 meV$. 
On the other hand, the value of the resistance at $\nu=0$ and $2$ decreases with increasing temperature up to $\sim 10 K$ and then increases linearly, showing metallic nature (SI-10). These observations are consistent with earlier reports for MtBLG~\cite{cao2018correlated,yankowitz2019tuning,wu2021chern,saito2020independent,lu2019superconductors,cao2018unconventional,arora2020superconductivity,park_tunable_2021}. %with a strong correlation effect. 
The gap ($\Delta$) determined from the activated plot for  $\nu=0$, $\nu=2$ and $\pm4$ are, respectively, $\sim$ $0.05meV$, $0.25meV$, $11.5 meV$ and  $9.25meV$ as shown in the SI-10 and SI-11. Furthermore, it can be seen (SI-10) that there is a crossover from metallic nature to insulating one at a higher temperature due to interband excitation of the carriers between the flat and dispersive bands. From the crossover temperature, $\sim 150K$ around the Dirac point, one can estimate the bandwidth ($2W$) and found to be of the order of $\sim 10 meV$ for the MtBLG. In Fig. 1e and 1f, we have shown the differential resistance versus bias current with temperature and perpendicular magnetic field around the superconducting dome. In order to extract the transition temperature at a giving filling, we compare the experimental data with the theoretically generated critical current versus temperature using BCS theory, $I_{c}(T) = I_{c}(0)(1-T/T_{c})^{2}$, where the $I_{c}(0)$ is the experimentally measured critical current at $T = 20mK$, and vary the $T_c$ such that the theoretically generated $I_{c}(T)$ traces the experimentally measured $I_{c}$ in Fig. 1e. This was repeated for other carrier densities and shown in the SI-12. The extracted the critical temperature was found to be $\sim 500mK$ at $\nu \sim -2.5$ (SI-12). The measured value of $T_c$ and critical field ($B_{c} \sim 100mT$) of our device matches reasonably well with the available data for MtBLG~\cite{cao2018unconventional,yankowitz2019tuning,lu2019superconductors,arora2020superconductivity,saito2020independent,park_tunable_2021}. 
Note that the Fraunhofer-like pattern in Fig. 1f can be explained by the interference between percolating superconducting paths separated by the normal islands, which is generic feature in MtBLG due to twist-angle in-homogeneity~\cite{zondiner2020cascade,uri_mapping_2020,park_tunable_2021}. These patterns further establish the existence of the superconductivity in our device though the measurement was carried out in two-probe geometry.

\subsection{Theory:} 
As discussed in detail in SI-16, we compute the thermopower and resistivity as a function of filling and temperature for the model of Ref.\cite{zondiner2020cascade} using Hartree-Fock (HF) approximation. The model consists of four spin-valley flavors, interacting with local Coulomb interaction $U$. For the results reported in the main text, we have taken $U=1.2 W$, where $W$ is the band width of the conduction (valence) band. The HF self-consistency equations depend on the non-interacting DOS of the moire' flat bands. We use various non-interacting DOSs, e.g. DOSs obtained from the continuum Bistritzer-MacDonal model~\cite{Bistritzer12233} in Ref.~\cite{zondiner2020cascade} and the DOS generated from the continuum model of Ref.~\cite{PhysRevX.8.031087}, which includes lattice relaxation effects. The self-consistent HF DOS is then used to compute thermopower and resistivity via Kubo formulae neglecting vertex corrections. We assume a constant band velocity and use a small impurity scattering rate $\Gamma_0=0.001W$ in the Kubo formulae.

\section{Acknowledgements}
A.D. thanks the Department of Science and Technology (DST), India for financial support (DSTO-2051), the MHRD, Government of India under STARS research funding (STARS/APR2019/PS/156/FS), and also acknowledges the Swarnajayanti Fellowship of the DST/SJF/PSA-03/2018-19. K.W. and T.T. acknowledge support from the Elemental Strategy Initiative conducted by the MEXT, Japan and the CREST (JPMJCR15F3), JST.


\section{Author contributions}
S.C., A.K.P. and U.R. contributed to device fabrication. A.G. and A.K.P. contributed to data acquisition and analysis. R.D. contributed in initial measurements. A.D. contributed in conceiving the idea and designing the experiment, data interpretation and analysis. K.W and T.T synthesized the hBN single crystals. A.P., A.A., S.M. and S.B. contributed in development of theory, data interpretation, and all the authors contributed in writing the manuscript.


\thispagestyle{empty}
\mbox{}


\includepdf[pages=-]{MtBLG_Supplementary_Information_page1.pdf}
\includepdf[pages=-]{MtBLG_Supplementary_Information-pages-2-5.pdf}
\includepdf[pages=-]{MtBLG_Supplementary_Information-pages-6-11.pdf}
\includepdf[pages=-]{MtBLG_Supplementary_Information-pages-12-18.pdf}
\includepdf[pages=-]{MtBLG_Supplementary_Information-pages-19-25.pdf}
\end{document}
