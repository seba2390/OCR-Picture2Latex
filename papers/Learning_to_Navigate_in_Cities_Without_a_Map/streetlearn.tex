\documentclass{article}

% if you need to pass options to natbib, use, e.g.:
%     \PassOptionsToPackage{numbers, compress}{natbib}
% before loading neurips_2018

% ready for submission
% \usepackage{neurips_2018}

% to compile a preprint version, e.g., for submission to arXiv, add add the
% [preprint] option:
%     \usepackage[preprint]{neurips_2018}

% to compile a camera-ready version, add the [final] option, e.g.:
%     \usepackage[final]{neurips_2018}
     \usepackage[final,nonatbib]{neurips_2018}

% to avoid loading the natbib package, add option nonatbib:
%     \usepackage[nonatbib]{neurips_2018}

\usepackage[utf8]{inputenc} % allow utf-8 input
\usepackage[T1]{fontenc}    % use 8-bit T1 fonts
\usepackage{graphicx}
\usepackage{amsmath,amssymb} % define this before the line numbering.
\usepackage{hyperref}       % hyperlinks
\usepackage{url}            % simple URL typesetting
\usepackage{booktabs}       % professional-quality tables
\usepackage{amsfonts}       % blackboard math symbols
\usepackage{nicefrac}       % compact symbols for 1/2, etc.
\usepackage{microtype}      % microtypography
\usepackage{color}
\usepackage{xspace}
\usepackage{mathabx}
%\usepackage{gensymb}
\usepackage{caption}
\usepackage{subcaption}
\usepackage{floatrow}
\usepackage{soul}
\newfloatcommand{capbtabbox}{table}[][\FBwidth]

\title{Learning to Navigate in Cities Without a Map}

\author{
  Piotr Mirowski, Matthew Koichi Grimes, Mateusz Malinowski, Karl Moritz Hermann, \\
  \bf{Keith Anderson, Denis Teplyashin, Karen Simonyan, Koray Kavukcuoglu,} \\
  \bf{Andrew Zisserman, Raia Hadsell}\\
  DeepMind\\
  London, United Kingdom\\
  \texttt{\{piotrmirowski, mkg, mateuszm, kmh, keithanderson, \}@google.com} \\
  \texttt{\{teplyashin, simonyan, korayk, zisserman, raia\}@google.com} \\
}

\usepackage{tikz}
\usepackage{amsmath}

\usepackage[utf8]{inputenc}

%\usepackage{amssymb}
\usepackage{booktabs} % For formal tables
\usepackage{graphicx}
\usepackage{amsfonts}
\usepackage{xspace}
\usepackage{color,soul}
% \usepackage{algpseudocode}
\usepackage[ruled, vlined, linesnumbered]{algorithm2e}

\usepackage{mdsymbol}
\usepackage{amsmath}
\usepackage{mathtools}
\usepackage{listings}

\usepackage{threeparttable}
\usepackage{tabularx}
%\usepackage{minted}
\usepackage{pifont}
\usepackage{multirow}
\let\labelindent\relax
\usepackage{enumitem}
\usepackage{cancel}
\usepackage[normalem]{ulem}
\usepackage{xcolor,colortbl}
\usepackage{arydshln}

\usepackage[normalem]{ulem}

%\usepackage[noadjust]{cite}
%\renewcommand{\citepunct}{,\penalty\citepunctpenalty\,}
%\renewcommand{\citedash}{--}
\usepackage{hyperref,xcolor}% http://ctan.org/pkg/{hyperref,xcolor}
\definecolor{winered}{rgb}{0.7,0,0}

\definecolor{gray}{gray}{0.7}
%\setlength{\textfloatsep}{15pt plus 1.0pt minus 2.0pt}
%\setlength{\intextsep}{10pt plus 1.0pt minus 2.0pt} %(CJ) I added this for formatting

\setlength\dashlinedash{0.5pt}
\setlength\dashlinegap{2pt}
\setlength\arrayrulewidth{1pt}

\newcommand{\cross}{\ding{55}}

\hypersetup
{
colorlinks=true,
linkcolor=winered,
urlcolor={winered},
filecolor={winered},
citecolor={winered},
allcolors={winered}
}


\newcommand{\gcell}{\cellcolor{gray!35}}
\newcommand{\gcelll}{\cellcolor{gray!15}}

\newcommand\Tstrut{\rule{0pt}{2.6ex}}         % = `top' strut
\newcommand\Bstrut{\rule[-0.9ex]{0pt}{0pt}}   % = `bottom' strut

%\usepackage[scale=0.95877]{tgpagella} % use "palatino" as main font, ...
\renewcommand{\ttdefault}{lmtt}       % and use lmtt for teletype family

\usepackage{tikz}
\usepackage{comment}
\usepackage[framemethod=tikz]{mdframed}

\usepackage{flushend}

\definecolor{darkpastelgreen}{rgb}{0.01, 0.75, 0.24}
\definecolor{cadmiumgreen}{rgb}{0.0, 0.42, 0.24}
\definecolor{brickred}{rgb}{0.8, 0.25, 0.33}
\definecolor{cornellred}{rgb}{0.7, 0.11, 0.11}
\definecolor{burgundy}{rgb}{0.5, 0.0, 0.13}
\definecolor{frenchblue}{rgb}{0.0, 0.45, 0.73}
\definecolor{light-gray}{gray}{0.92}
\definecolor{lightlight-gray}{gray}{0.97}
\definecolor{codegray}{gray}{0.90}
\definecolor{inputgray}{gray}{0.90}

%\newcommand{\rtbox}[1]{\vspace{0.25em}\noindent\fbox{\parbox{0.97\columnwidth}{#1}}}


\global\mdfdefinestyle{rtboxstyle}{%
linecolor=black,%
leftmargin=0cm,rightmargin=0cm,linewidth=0.5pt,
roundcorner=3,
skipbelow=0pt,backgroundcolor=lightlight-gray
}

\newcommand{\rtbox}[1]{\begin{mdframed}[style=rtboxstyle]{{#1}}\end{mdframed}}

\usepackage[most]{tcolorbox}
\newcommand{\hlbox}[1]{
\begin{tcolorbox}[enhanced,
  breakable,
  colback=white,
  boxrule=0pt,
  frame hidden,
  top=0mm,left=1.5mm,right=0mm,bottom=0mm,arc=0mm,
  borderline west={0.75mm}{0mm}{gray}]
#1
\end{tcolorbox}
}


\newcommand{\code}[1]{\texttt{#1}}
\newcommand{\greencode}[1]{{\color{cadmiumgreen}\texttt{#1}}}
\newcommand{\redcode}[1]{{\color{red}\texttt{#1}}}
\newcommand{\graycode}[1]{{\color{darkgray}\texttt{#1}}}
%\newcommand{\code}[1]{
%    \begingroup\setlength{\fboxsep}{1pt}%
%    \colorbox{codegray}{\small\texttt{\hspace*{2pt}\vphantom{Ay}#1\hspace*{2pt}}}%
%    \endgroup    
%}
\newcommand{\inputtext}[1]{
    \begingroup\setlength{\fboxsep}{1pt}%
    \colorbox{inputgray}{\small\texttt{\hspace*{2pt}\vphantom{Ay}#1\hspace*{2pt}}}%
    \endgroup    
}




\mdfdefinestyle{MyFrame}{%
%     linecolor=blue,
%     outerlinewidth=2pt,
     roundcorner=1pt,
     innertopmargin=6pt,
     innerbottommargin=6pt,
     innerrightmargin=6pt,
     innerleftmargin=6pt,
%     backgroundcolor=gray!50!white
    }

\newcommand*\Let[2]{#1 $\gets$ #2}
% \algrenewcommand\algorithmicindent{1.0em}%
% \algrenewcommand\algorithmicrequire{\textbf{Input:}}
% \algrenewcommand\algorithmicensure{\textbf{Output:}}

% \algdef{SE}[DOWHILE]{Do}{doWhile}{\algorithmicdo}[1]{\algorithmicwhile\ #1}%

% \algnewcommand\algorithmicswitch{\textbf{switch}}
% \algnewcommand\algorithmiccase{\textbf{case}}
% \algnewcommand\algorithmicassert{\texttt{assert}}

% \algdef{SE}[SWITCH]{Switch}{EndSwitch}[1]{\algorithmicswitch\ #1\ \algorithmicdo}{\algorithmicend\ \algorithmicswitch}%
% \algdef{SE}[CASE]{Case}{EndCase}[1]{\algorithmiccase\ #1}{\algorithmicend\ \algorithmiccase}%
% \algtext*{EndSwitch}%
% \algtext*{EndCase}%
\newcommand*\blkcc[1]{\tikz[baseline=(char.base)]{
            \node[shape=circle,fill,inner sep=1pt] (char) {\textcolor{white}{#1}};}}

\newcommand*\redcc[1]{\tikz[baseline=(char.base)]{
            \node[shape=circle,fill=black!30!red,inner sep=1pt] (char) {\textcolor{white}{#1}};}}

\newcommand*\bredcc[1]{\tikz[baseline=(char.base)]{
                \node[shape=circle,fill=red,inner sep=1pt] (char) {\textcolor{white}{#1}};}}
\newcommand*\graycc[1]{\tikz[baseline=(char.base)]{
                \node[shape=circle,fill=black!50,inner sep=1pt] (char) {\textcolor{white}{#1}};}}
    
\newcommand*\greencc[1]{\tikz[baseline=(char.base)]{
            \node[shape=circle,fill=black!50!green,inner sep=1pt] (char) {\textcolor{white}{#1}};}}

\newcommand{\mt}[1]{\mathit{#1}}
\newcommand{\mn}[1]{\mathrm{#1}}
\newcommand{\mset}{\mathbb{P}}
\newcommand{\mb}[1]{\mathbb{#1}}
\newcommand{\X}{\times}
\newcommand{\italicsubsec}[1]{\smallskip \noindent {\it {#1}: }}

\newcommand{\rsrcmap}[1]{\textsc{Rc}($#1$)\xspace}
\newcommand{\rsrcmapsym}{\textsc{Rc}\xspace}
\newcommand{\sysname}{{{\textsc{Spinner}}}\xspace}
\newcommand{\secref}[1]{\S~\ref{#1}}
\newcommand{\cmt}[1]{}
\newcommand{\sysbegin}{$S_\mt{BEGIN}$\xspace}
\newcommand{\sysend}{$S_\mt{END}$\xspace}
\newcommand{\Erec}{$E_\mathit{rec}$\xspace}
\newcommand{\Erep}{$E_\mathit{rep}$\xspace}
\newcommand{\Efrep}{$E_\mathit{faithful\_rep}$\xspace}
\newcommand{\Errep}{$E_\mathit{RoR}$\xspace}
\newcommand{\ALT}{~\mid~}
\newcommand{\tsup}[1]{\textsuperscript{#1}}
\newcommand{\cct}[1]{\raisebox{.5pt}{\textcircled{\raisebox{-.9pt} {#1}}}}
\newcommand{\homedir}{\raise.17ex\hbox{$\scriptstyle\sim$}}
\newcommand{\V}{$ \checkmark $}
\newcommand{\N}{$ \text{\ding{55}} $}
\newcommand{\xmark}{$ \text{\ding{53}} $}

\newcommand{\TRI}{$ \triangle $}
\newcommand{\progname}[1]{{\tt #1}\xspace}


\newcommand{\optmap}[2]{`\code{#1}' $\mapsto$ `\code{\color{red} #2}'}

\newcommand{\YK}[1]{\textcolor{blue}{[{\bf YK:} #1]}}
\newcommand{\mw}[1]{\textcolor{orange}{[{\bf mw:} #1]}}
\newcommand{\MA}[1]{\textcolor{red}{[{\bf MA:} #1]}}
\newcommand{\CJ}[1]{\textcolor{violet}{[{\bf CJ:} #1]}}
\definecolor{darkgreen}{RGB}{40,125,40}
\newcommand{\fixme}[1]{\textcolor{red}{[{\bf FIXME:} #1]}}
\newcommand{\fix}[1]{\textcolor{red}{\hl{#1}}}
\newcommand{\corr}[2] {\sout{#1} \hl{#2}}

\newcommand*\ruleline[1]{\par\noindent\raisebox{.8ex}{\makebox[\linewidth]{\hrulefill\hspace{1ex}\raisebox{-.8ex}{#1}\hspace{1ex}\hrulefill}}}

%\newcommand{\updated}[2]{{\color{red}\sout{#1}{#2}}}
%\newcommand{\updated}[2]{{\color{red}{#2}}}
\newcommand{\updated}[2]{{{#2}}}
%\newcommand{\revised}[1]{{\color{red}{#1}}}
\newcommand{\revised}[1]{#1}
\newcommand{\avgtotal}[2]{#1 & (#2)}

\begin{document}

\maketitle

% Proposed edits
\newcommand{\ed}[1]{\textcolor{cyan}{\bf \small #1}}
% Piotr
\newcommand{\piotr}[1]{\textcolor{cyan}{\bf \small [#1 --PM]}}
% Raia
\newcommand{\rh}[1]{\textcolor{blue}{\bf \small [#1 --RH]}}
% Mateusz
\newcommand{\mati}[1]{\textcolor{green}{\bf \small [#1 --MM]}}
% Karen
\newcommand{\karen}[1]{\textcolor{black}{\bf \small [#1 --Karen]}}
% Andrew Zisserman
\newcommand{\az}[1]{\textcolor{red}{\bf \small [#1 --AZ]}}
% Karl Moritz
\definecolor{darkblue}{RGB}{0, 0, 83}
\newcommand{\kmh}[1]{\textcolor{darkblue}{\bf\small [#1 --KMH]}}


\begin{abstract}
Navigating through unstructured environments is a basic capability of intelligent creatures, and thus is of fundamental interest in the study and development of artificial intelligence. Long-range navigation is a complex cognitive task that relies on developing an internal representation of space, grounded by recognisable landmarks and robust visual processing, that can simultaneously support continuous self-localisation (``I am \emph{here}'') and a representation of the goal (``I am going \emph{there}''). Building upon recent research that applies deep reinforcement learning to maze navigation problems, we present an end-to-end deep reinforcement learning approach that can be applied on a city scale. Recognising that successful navigation relies on integration of general policies with locale-specific knowledge, we propose a dual pathway architecture that allows locale-specific features to be encapsulated, while still enabling transfer to multiple cities. A key contribution of this paper is an interactive navigation environment that uses Google Street View for its photographic content and worldwide coverage. Our baselines demonstrate that deep reinforcement learning agents can learn to navigate in multiple cities and to traverse to target destinations that may be kilometres away. The project webpage \url{http://streetlearn.cc} contains a video summarizing our research and showing the trained agent in diverse city environments and on the transfer task, the form to request the StreetLearn dataset and links to further resources. The StreetLearn environment code is available at \url{https://github.com/deepmind/streetlearn}.

\end{abstract}

\section{Introduction}
\label{sec:intro}
% \leavevmode
% \\
% \\
% \\
% \\
% \\
\section{Introduction}
\label{introduction}

AutoML is the process by which machine learning models are built automatically for a new dataset. Given a dataset, AutoML systems perform a search over valid data transformations and learners, along with hyper-parameter optimization for each learner~\cite{VolcanoML}. Choosing the transformations and learners over which to search is our focus.
A significant number of systems mine from prior runs of pipelines over a set of datasets to choose transformers and learners that are effective with different types of datasets (e.g. \cite{NEURIPS2018_b59a51a3}, \cite{10.14778/3415478.3415542}, \cite{autosklearn}). Thus, they build a database by actually running different pipelines with a diverse set of datasets to estimate the accuracy of potential pipelines. Hence, they can be used to effectively reduce the search space. A new dataset, based on a set of features (meta-features) is then matched to this database to find the most plausible candidates for both learner selection and hyper-parameter tuning. This process of choosing starting points in the search space is called meta-learning for the cold start problem.  

Other meta-learning approaches include mining existing data science code and their associated datasets to learn from human expertise. The AL~\cite{al} system mined existing Kaggle notebooks using dynamic analysis, i.e., actually running the scripts, and showed that such a system has promise.  However, this meta-learning approach does not scale because it is onerous to execute a large number of pipeline scripts on datasets, preprocessing datasets is never trivial, and older scripts cease to run at all as software evolves. It is not surprising that AL therefore performed dynamic analysis on just nine datasets.

Our system, {\sysname}, provides a scalable meta-learning approach to leverage human expertise, using static analysis to mine pipelines from large repositories of scripts. Static analysis has the advantage of scaling to thousands or millions of scripts \cite{graph4code} easily, but lacks the performance data gathered by dynamic analysis. The {\sysname} meta-learning approach guides the learning process by a scalable dataset similarity search, based on dataset embeddings, to find the most similar datasets and the semantics of ML pipelines applied on them.  Many existing systems, such as Auto-Sklearn \cite{autosklearn} and AL \cite{al}, compute a set of meta-features for each dataset. We developed a deep neural network model to generate embeddings at the granularity of a dataset, e.g., a table or CSV file, to capture similarity at the level of an entire dataset rather than relying on a set of meta-features.
 
Because we use static analysis to capture the semantics of the meta-learning process, we have no mechanism to choose the \textbf{best} pipeline from many seen pipelines, unlike the dynamic execution case where one can rely on runtime to choose the best performing pipeline.  Observing that pipelines are basically workflow graphs, we use graph generator neural models to succinctly capture the statically-observed pipelines for a single dataset. In {\sysname}, we formulate learner selection as a graph generation problem to predict optimized pipelines based on pipelines seen in actual notebooks.

%. This formulation enables {\sysname} for effective pruning of the AutoML search space to predict optimized pipelines based on pipelines seen in actual notebooks.}
%We note that increasingly, state-of-the-art performance in AutoML systems is being generated by more complex pipelines such as Directed Acyclic Graphs (DAGs) \cite{piper} rather than the linear pipelines used in earlier systems.  
 
{\sysname} does learner and transformation selection, and hence is a component of an AutoML systems. To evaluate this component, we integrated it into two existing AutoML systems, FLAML \cite{flaml} and Auto-Sklearn \cite{autosklearn}.  
% We evaluate each system with and without {\sysname}.  
We chose FLAML because it does not yet have any meta-learning component for the cold start problem and instead allows user selection of learners and transformers. The authors of FLAML explicitly pointed to the fact that FLAML might benefit from a meta-learning component and pointed to it as a possibility for future work. For FLAML, if mining historical pipelines provides an advantage, we should improve its performance. We also picked Auto-Sklearn as it does have a learner selection component based on meta-features, as described earlier~\cite{autosklearn2}. For Auto-Sklearn, we should at least match performance if our static mining of pipelines can match their extensive database. For context, we also compared {\sysname} with the recent VolcanoML~\cite{VolcanoML}, which provides an efficient decomposition and execution strategy for the AutoML search space. In contrast, {\sysname} prunes the search space using our meta-learning model to perform hyperparameter optimization only for the most promising candidates. 

The contributions of this paper are the following:
\begin{itemize}
    \item Section ~\ref{sec:mining} defines a scalable meta-learning approach based on representation learning of mined ML pipeline semantics and datasets for over 100 datasets and ~11K Python scripts.  
    \newline
    \item Sections~\ref{sec:kgpipGen} formulates AutoML pipeline generation as a graph generation problem. {\sysname} predicts efficiently an optimized ML pipeline for an unseen dataset based on our meta-learning model.  To the best of our knowledge, {\sysname} is the first approach to formulate  AutoML pipeline generation in such a way.
    \newline
    \item Section~\ref{sec:eval} presents a comprehensive evaluation using a large collection of 121 datasets from major AutoML benchmarks and Kaggle. Our experimental results show that {\sysname} outperforms all existing AutoML systems and achieves state-of-the-art results on the majority of these datasets. {\sysname} significantly improves the performance of both FLAML and Auto-Sklearn in classification and regression tasks. We also outperformed AL in 75 out of 77 datasets and VolcanoML in 75  out of 121 datasets, including 44 datasets used only by VolcanoML~\cite{VolcanoML}.  On average, {\sysname} achieves scores that are statistically better than the means of all other systems. 
\end{itemize}


%This approach does not need to apply cleaning or transformation methods to handle different variances among datasets. Moreover, we do not need to deal with complex analysis, such as dynamic code analysis. Thus, our approach proved to be scalable, as discussed in Sections~\ref{sec:mining}.

\section{Related Work}
\label{sec:related}
\section{Related Work}
%\mz{We lack a comparison to this paper: https://arxiv.org/abs/2305.14877}
%\anirudh{refine to be more on-topic?}
\iffalse
\paragraph{In-Context Learning} As language models have scaled, the ability to learn in-context, without any weight updates, has emerged. \cite{brown2020language}. While other families of large language models have emerged, in-context learning remains ubiquitous \cite{llama, bloom, gptneo, opt}. Although such as HELM \cite{helm} have arisen for systematic evaluation of \emph{models}, there is no systematic framework to our knowledge for evaluating \emph{prompting methods}, and validating prompt engineering heuristics. The test-suite we propose will ensure that progress in the field of prompt-engineering is structured and objectively evaluated. 

\paragraph{Prompt Engineering Methods} Researchers are interested in the automatic design of high performing instructions for downstream tasks. Some focus on simple heuristics, such as selecting instructions that have the lowest perplexity \cite{lowperplexityprompts}. Other methods try to use large language models to induce an instruction when provided with a few input-output pairs \cite{ape}. Researchers have also used RL objectives to create discrete token sequences that can serve as instructions \cite{rlprompt}. Since the datasets and models used in these works have very little intersection, it is impossible to compare these methods objectively and glean insights. In our work, we evaluate these three methods on a diverse set of tasks and models, and analyze their relative performance. Additionally, we recognize that there are many other interesting angles of prompting that are not covered by instruction engineering \cite{weichain, react, selfconsistency}, but we leave these to future work.

\paragraph{Analysis of Prompting Methods} While most prompt engineering methods focus on accuracy, there are many other interesting dimensions of performance as well. For instance, researchers have found that for most tasks, the selection of demonstrations plays a large role in few-shot accuracy \cite{whatmakesgoodicexamples, selectionmachinetranslation, knnprompting}. Additionally, many researchers have found that even permuting the ordering of a fixed set of demonstrations has a significant effect on downstream accuracy \cite{fantasticallyorderedprompts}. Prompts that are sensitive to the permutation of demonstrations have been shown to also have lower accuracies \cite{relationsensitivityaccuracy}. Especially in low-resource domains, which includes the large public usage of in-context learning, these large swings in accuracy make prompting less dependable. In our test-suite we include sensitivity metrics that go beyond accuracy and allow us to find methods that are not only performant but reliable.

\paragraph{Existing Benchmarks} We recognize that other holistic in-context learning benchmarks exist. BigBench is a large benchmark of 204 tasks that are beyond the capabilities of current LLMs. BigBench seeks to evaluate the few-shot abilities of state of the art large language models, focusing on performance metrics such as accuracy \cite{bigbench}. Similarly, HELM is another benchmark for language model in-context learning ability. Rather than only focusing on performance, HELM branches out and considers many other metrics such as robustness and bias \cite{helm}. Both BigBench and HELM focus on ranking different language model, while fix a generic instruction and prompt format. We instead choose to evaluate instruction induction / selection methods over a fixed set of models. We are the first ever evaluation script that compares different prompt-engineering methods head to head. 
\fi

\paragraph{In-Context Learning and Existing Benchmarks} As language models have scaled, in-context learning has emerged as a popular paradigm and remains ubiquitous among several autoregressive LLM families \cite{brown2020language, llama, bloom, gptneo, opt}. Benchmarks like BigBench \cite{bigbench} and HELM \cite{helm} have been created for the holistic evaluation of these models. BigBench focuses on few-shot abilities of state-of-the-art large language models, while HELM extends to consider metrics like robustness and bias. However, these benchmarks focus on evaluating and ranking \emph{language models}, and do not address the systematic evaluation of \emph{prompting methods}. Although contemporary work by \citet{yang2023improving} also aims to perform a similar systematic analysis of prompting methods, they focus on simple probability-based prompt selection while we evaluate a broader range of methods including trivial instruction baselines, curated manually selected instructions, and sophisticated automated instruction selection.

\paragraph{Automated Prompt Engineering Methods} There has been interest in performing automated prompt-engineering for target downstream tasks within ICL. This has led to the exploration of various prompting methods, ranging from simple heuristics such as selecting instructions with the lowest perplexity \cite{lowperplexityprompts}, inducing instructions from large language models using a few annotated input-output pairs \cite{ape}, to utilizing RL objectives to create discrete token sequences as prompts \cite{rlprompt}. However, these works restrict their evaluation to small sets of models and tasks with little intersection, hindering their objective comparison. %\mz{For paragraphs that only have one work in the last line, try to shorten the paragraph to squeeze in context.}

\paragraph{Understanding in-context learning} There has been much recent work attempting to understand the mechanisms that drive in-context learning. Studies have found that the selection of demonstrations included in prompts significantly impacts few-shot accuracy across most tasks \cite{whatmakesgoodicexamples, selectionmachinetranslation, knnprompting}. Works like \cite{fantasticallyorderedprompts} also show that altering the ordering of a fixed set of demonstrations can affect downstream accuracy. Prompts sensitive to demonstration permutation often exhibit lower accuracies \cite{relationsensitivityaccuracy}, making them less reliable, particularly in low-resource domains.

Our work aims to bridge these gaps by systematically evaluating the efficacy of popular instruction selection approaches over a diverse set of tasks and models, facilitating objective comparison. We evaluate these methods not only on accuracy metrics, but also on sensitivity metrics to glean additional insights. We recognize that other facets of prompting not covered by instruction engineering exist \cite{weichain, react, selfconsistency}, and defer these explorations to future work. 

\section{Environment}
\label{sec:environment}

This section presents an interactive environment, named \emph{StreetLearn}, constructed using Google Street View, which provides a public API\footnote{\url{https://developers.google.com/maps/documentation/streetview/}}. 
Street View provides a set of geolocated $360^{\degree}$ panoramic images which form the nodes of an undirected graph. We selected a number of large regions in New York City, Paris and London that contain between 7,000 and 65,500 nodes (and between 7,200 and 128,600 edges, respectively), have a mean node spacing of 10m, and cover a range of up to 5km (see Fig.~\ref{fig:nyc}).
We do not simplify the underlying connectivity, thus there are congested areas with complex occluded intersections, tunnels and footpaths, and other ephemera. Although the graph is used to construct the environment, the agent only sees the raw RGB images (see Fig.~\ref{fig:streetview}).


\subsection{Agent Interface and the Courier Task}
\label{sec:agent}


\begin{figure}[t]
%\vskip -0.1in
\begin{center}
\begin{subfigure}{.45\textwidth}
  \centering
  \includegraphics[width=.9\linewidth]{figures/goal_description.pdf}
  \caption{Goal description using landmarks.}
  \label{fig:goals}
\end{subfigure}%
\begin{subfigure}{.55\textwidth}
  \centering
  \includegraphics[width=.99\linewidth]{figures/streetlearn_architectures_v2_small_labels.pdf}
  \caption{Comparison of architectures.}
  \label{fig:architecture}
\end{subfigure}
\caption{
\textbf{(a)} In the illustration of the goal description, we show a set of 5 nearby landmarks and 4 distant ones; the code $g_i$ is a vector with a softmax-normalised distance to each landmark.
\textbf{(b)} \emph{Left:} GoalNav is a convolutional encoder plus policy LSTM  with goal description input. \emph{Middle:} CityNav is a single-city navigation architecture with a separate goal LSTM and optional auxiliary heading ($\theta$). \emph{Right:} MultiCityNav is a multi-city architecture with individual goal LSTM pathways for each city.
}
\label{fig:goals_architecture}
\end{center}
% \vskip -0.4in
\vskip -0.25in
\end{figure}

An RL environment needs to specify the start space, observations, and action space of the agent as well as the task reward. The agent has two inputs: the image  ${\bf x_t}$, which is a cropped, $60^\degree$ square, RGB image that is scaled to $84\times84$ pixels (i.e.\ not the entire panorama), and the goal description $g_t$. The action space is composed of five discrete actions: ``slow'' rotate left or right ($\pm 22.5^\degree$), ``fast'' rotate left or right ($\pm 67.5^\degree$), or move forward---this action becomes a \texttt{noop} if there is not an edge in view from the current agent pose. If there are multiple edges in the view cone of the agent, then the most central one is chosen. 

There are many options for how to specify the goal to the agent, from images to agent-relative directions, to text descriptions or addresses. We choose to represent the current goal in terms of its proximity to a set $\mathcal{L}$ of fixed landmarks: $\mathcal{L}=\{(Lat_k,Long_k)\}_k$, specified using the Lat/Long (latitude and longitude) coordinate system. To represent a goal at $(Lat^g_t,Long^g_t)$ we take a softmax over the distances to the $k$ landmarks (see Fig.~\ref{fig:goals}), thus for distances $\{d^g_{t,k}\}_k$ the goal vector contains $g_{t,i} = \exp(-\alpha d^g_{t,i}) / \sum_k \exp(-\alpha d^g_{t,k})$
for the $ith$ landmark with $\alpha =0.002$ (which we chose through cross-validation). This forms a goal description with certain desirable qualities: it is a scalable representation that extends easily to new regions, it does not rely on any arbitrary scaling of map coordinates, and it has intuitive meaning---humans and animals also navigate with respect to fixed landmarks. Note that landmarks are fixed per map and we used the same list of landmarks across all experiments; $g_t$ is computed using the distance to all landmarks, but by feeding these distances through a non-linearity, the contribution of distant landmarks is reduced to zero.
In the Supplementary material, we show that the locally-continuous landmark-based representation of the goal performs as well as the linear scalar representation $(Lat^g_t,Long^g_t)$. Since the landmark-based representation performs well while being independent of the coordinate system and thus more scalable, we use this representation as canonical.
Note that the goal description is not relative to the agent's position and only changes when a new goal is sampled. Locations of the 644 manually defined landmarks in New York, London and Paris are given in the Supplementary material, where we also show that the density of landmarks does not impact the agent performance.


In the \emph{courier} task, which we define as the problem of navigating to a series of random locations in a city, the agent starts each episode from a randomly sampled position and orientation. If the agent gets within 100m of the goal (approximately one city block), the next goal is randomly chosen and input to the agent. 
Each episode ends after 1000 agent steps.  The reward that the agent gets upon reaching a goal is proportional to the shortest path between the goal and the agent's position when the goal is first assigned; much like a delivery service, the agent receives a higher reward for longer journeys. Note that we do not reward agents for taking detours, but rather that the reward in a given level is a function of the optimal distance from start to goal location. As the goals get more distant during the training curriculum, per-episode reward statistics should ideally reach and stay at a plateau performance level if the agent can equally reach closer and further goals.


\section{Methods}
\label{sec:methods}
\section{SYSTEM OVERVIEW}
\begin{figure}
\centering

\def\picScale{0.08}    % define variable for scaling all pictures evenly
\def\colWidth{0.5\linewidth}

\begin{tikzpicture}
\matrix [row sep=0.25cm, column sep=0cm, style={align=center}] (my matrix) at (0,0) %(2,1)
{
\node[style={anchor=center}] (FREEhand) {\includegraphics[width=0.85\linewidth]{figures/FREEhand.pdf}}; %\fill[blue] (0,0) circle (2pt);
\\
\node[style={anchor=center}] (rigid_v_soft) {\includegraphics[width=0.75\linewidth]{figures/FREE_vs_rigid-v8.pdf}}; %\fill[blue] (0,0) circle (2pt);
\\
};
\node[above] (FREEhand) at ($ (FREEhand.south west)  !0.05! (FREEhand.south east) + (0, 0.1)$) {(a)};
\node[below] (a) at ($ (rigid_v_soft.south west) !0.20! (rigid_v_soft.south east) $) {(b)};
\node[below] (b) at ($ (rigid_v_soft.south west) !0.75! (rigid_v_soft.south east) $) {(c)};
\end{tikzpicture}


% \begin{tikzpicture} %[every node/.style={draw=black}]
% % \draw[help lines] (0,0) grid (4,2);
% \matrix [row sep=0cm, column sep=0cm, style={align=center}] (my matrix) at (0,0) %(2,1)
% {
% \node[style={anchor=center}] {\includegraphics[width=\colWidth]{figures/photos/labFREEs3.jpg}}; %\fill[blue] (0,0) circle (2pt)
% &
% \node[style={anchor=center}] {\includegraphics[width=\colWidth, height=160pt]{figures/stewartRender.png}}; %\fill[blue] (0,0) circle (2pt);
% \\
% };

% %\node[style={anchor=center}] at (0,-5) (FREEstate) {\includegraphics[width=0.7\linewidth]{figures/FREEstate_noLabels2.pdf}};

% \end{tikzpicture}

\caption{\revcomment{2.3}{(a) A fiber-reinforced elastomerc enclosure (FREE) is a soft fluid-driven actuator composed of an elastomer tube with fibers wound around it to impose specific deformations under an increase in volume, such as extension and torsion. (b) A linear actuator and motor combined in \emph{series} has the ability to generate 2 dimensional forces at the end effector (shown in red), but is constrained to motions only in the directions of these forces. (b) Three FREEs combined in \emph{parallel} can generate the same 2 dimensional forces at the end effector (shown in red), without imposing kinematic constraints that prohibit motion in other directions (shown in blue).}}

% \caption{A fiber-reinforced elastomeric enclosure (FREE) (top) is a soft fluid-driven actuator composed of an elastomer tube with fibers wound around it to impose deformation in specific directions upon pressurization, such as extension and torsion. \revcomment{2.3}{In this paper we explore the potential of combining multiple FREEs in parallel to generate fully controllable multi-dimensional spacial forces}, such as in a parallel arrangement around a flexible spine element (bottom-left), or a Stewart Platform arrangement (bottom-right).}

\label{fig:overview}
\end{figure}


We now give an overview of our learning framework as illustrated in Figure~\ref{fig:overview}. Our framework splits athletic jumps into two phases: a run-up phase and a jump phase. The {\em take-off state} marks the transition between these two phases, and consists of a time instant midway through the last support phase before becoming airborne. The take-off state is key to our exploration strategy, as it is a strong determinant of the resulting jump strategy. We characterize the take-off state by a feature vector that captures key aspects of the state, such as the net angular velocity and body orientation. This defines a low-dimensional take-off feature space that we can sample in order to explore and evaluate a variety of motion strategies. While random sampling of take-off state features is straightforward, it is computationally impractical as evaluating one sample involves an expensive DRL learning process that takes hours even on modern machines. Therefore, we introduce a sample-efficient Bayesian Diversity Search (BDS) algorithm as a key part of our Stage~1 optimization process.

Given a specific sampled take-off state, we then need to produce an optimized run-up controller and a jump controller that result in the best possible corresponding jumps. This process has several steps. We first train a {\em }run-up controller, using deep reinforcement learning, that imitates a single generic run-up motion capture clip while also targeting the desired take-off state. For simplicity, the run-up controller and its training are not shown in Figure~\ref{fig:overview}. These are discussed in Section~\ref{sec:Experiments-Runup}. The main challenge lies with the synthesis of the actual jump controller which governs the remainder of the motion, and for which we wish to discover strategies without any recourse to known solutions.

The jump controller begins from the take-off state and needs to control the body during take-off, over the bar, and to prepare for landing. This poses a challenging learning problem because of the demanding nature of the task, the sparse fail/success rewards, and the difficulty of also achieving natural human-like movement. We apply two key insights to make this task learnable using deep reinforcement learning. First, we employ an action space defined by a subspace of natural human poses as modeled with a Pose Variational Autoencoder (P-VAE). Given an action parameterized as a target body pose, individual joint torques are then realized using PD-controllers. We additionally allow for regularized {\em offset} PD-targets that are added to the P-VAE targets to enable strong takeoff forces. Second, we employ a curriculum that progressively increases the task difficulty, i.e., the height of the bar, based on current performance.

A diverse set of strategies can already emerge after the Stage 1 BDS optimization. To achieve further strategy variations, we reuse the take-off states of the existing discovered strategies for another stage of optimization. The diversity is explicitly incentivized during this Stage 2 optimization via a novelty reward, which is focused specifically on features of the body pose at the peak height of the jump. As shown in Figure~\ref{fig:overview}, Stage~2 makes use of the same overall DRL learning procedure as in Stage~1, albeit with a slightly different reward structure.




\section{Results}
\label{sec:experiments}
In this section we conduct comprehensive experiments to emphasise the effectiveness of DIAL, including evaluations under white-box and black-box settings, robustness to unforeseen adversaries, robustness to unforeseen corruptions, transfer learning, and ablation studies. Finally, we present a new measurement to test the balance between robustness and natural accuracy, which we named $F_1$-robust score. 


\subsection{A case study on SVHN and CIFAR-100}
In the first part of our analysis, we conduct a case study experiment on two benchmark datasets: SVHN \citep{netzer2011reading} and CIFAR-100 \cite{krizhevsky2009learning}. We follow common experiment settings as in \cite{rice2020overfitting, wu2020adversarial}. We used the PreAct ResNet-18 \citep{he2016identity} architecture on which we integrate a domain classification layer. The adversarial training is done using 10-step PGD adversary with perturbation size of 0.031 and a step size of 0.003 for SVHN and 0.007 for CIFAR-100. The batch size is 128, weight decay is $7e^{-4}$ and the model is trained for 100 epochs. For SVHN, the initial learinnig rate is set to 0.01 and decays by a factor of 10 after 55, 75 and 90 iteration. For CIFAR-100, the initial learning rate is set to 0.1 and decays by a factor of 10 after 75 and 90 iterations. 
%We compared DIAL to \cite{madry2017towards} and TRADES \citep{zhang2019theoretically}. 
%The evaluation is done using Auto-Attack~\citep{croce2020reliable}, which is an ensemble of three white-box and one black-box parameter-free attacks, and various $\ell_{\infty}$ adversaries: PGD$^{20}$, PGD$^{100}$, PGD$^{1000}$ and CW$_{\infty}$ with step size of 0.003. 
Results are averaged over 3 restarts while omitting one standard deviation (which is smaller than 0.2\% in all experiments). As can be seen by the results in Tables~\ref{black-and_white-svhn} and \ref{black-and_white-cifar100}, DIAL presents consistent improvement in robustness (e.g., 5.75\% improved robustness on SVHN against AA) compared to the standard AT 
%under variety of attacks 
while also improving the natural accuracy. More results are presented in Appendix \ref{cifar100-svhn-appendix}.


\begin{table}[!ht]
  \caption{Robustness against white-box, black-box attacks and Auto-Attack (AA) on SVHN. Black-box attacks are generated using naturally trained surrogate model. Natural represents the naturally trained (non-adversarial) model.
  %and applied to the best performing robust models.
  }
  \vskip 0.1in
  \label{black-and_white-svhn}
  \centering
  \small
  \begin{tabular}{l@{\hspace{1\tabcolsep}}c@{\hspace{1\tabcolsep}}c@{\hspace{1\tabcolsep}}c@{\hspace{1\tabcolsep}}c@{\hspace{1\tabcolsep}}c@{\hspace{1\tabcolsep}}c@{\hspace{1\tabcolsep}}c@{\hspace{1\tabcolsep}}c@{\hspace{1\tabcolsep}}c@{\hspace{1\tabcolsep}}c}
    \toprule
    & & \multicolumn{4}{c}{White-box} & \multicolumn{4}{c}{Black-Box}  \\
    \cmidrule(r){3-6} 
    \cmidrule(r){7-10}
    Defense Model & Natural & PGD$^{20}$ & PGD$^{100}$  & PGD$^{1000}$  & CW$^{\infty}$ & PGD$^{20}$ & PGD$^{100}$ & PGD$^{1000}$  & CW$^{\infty}$ & AA \\
    \midrule
    NATURAL & 96.85 & 0 & 0 & 0 & 0 & 0 & 0 & 0 & 0 & 0 \\
    \midrule
    AT & 89.90 & 53.23 & 49.45 & 49.23 & 48.25 & 86.44 & 86.28 & 86.18 & 86.42 & 45.25 \\
    % TRADES & 90.35 & 57.10 & 54.13 & 54.08 & 52.19 & 86.89 & 86.73 & 86.57 & 86.70 &  49.50 \\
    $\DIAL_{\kl}$ (Ours) & 90.66 & \textbf{58.91} & \textbf{55.30} & \textbf{55.11} & \textbf{53.67} & 87.62 & 87.52 & 87.41 & 87.63 & \textbf{51.00} \\
    $\DIAL_{\ce}$ (Ours) & \textbf{92.88} & 55.26  & 50.82 & 50.54 & 49.66 & \textbf{89.12} & \textbf{89.01} & \textbf{88.74} & \textbf{89.10} &  46.52  \\
    \bottomrule
  \end{tabular}
\end{table}


\begin{table}[!ht]
  \caption{Robustness against white-box, black-box attacks and Auto-Attack (AA) on CIFAR100. Black-box attacks are generated using naturally trained surrogate model. Natural represents the naturally trained (non-adversarial) model.
  %and applied to the best performing robust models.
  }
  \vskip 0.1in
  \label{black-and_white-cifar100}
  \centering
  \small
  \begin{tabular}{l@{\hspace{1\tabcolsep}}c@{\hspace{1\tabcolsep}}c@{\hspace{1\tabcolsep}}c@{\hspace{1\tabcolsep}}c@{\hspace{1\tabcolsep}}c@{\hspace{1\tabcolsep}}c@{\hspace{1\tabcolsep}}c@{\hspace{1\tabcolsep}}c@{\hspace{1\tabcolsep}}c@{\hspace{1\tabcolsep}}c}
    \toprule
    & & \multicolumn{4}{c}{White-box} & \multicolumn{4}{c}{Black-Box}  \\
    \cmidrule(r){3-6} 
    \cmidrule(r){7-10}
    Defense Model & Natural & PGD$^{20}$ & PGD$^{100}$  & PGD$^{1000}$  & CW$^{\infty}$ & PGD$^{20}$ & PGD$^{100}$ & PGD$^{1000}$  & CW$^{\infty}$ & AA \\
    \midrule
    NATURAL & 79.30 & 0 & 0 & 0 & 0 & 0 & 0 & 0 & 0 & 0 \\
    \midrule
    AT & 56.73 & 29.57 & 28.45 & 28.39 & 26.6 & 55.52 & 55.29 & 55.26 & 55.40 & 24.12 \\
    % TRADES & 58.24 & 30.10 & 29.66 & 29.64 & 25.97 & 57.05 & 56.71 & 56.67 & 56.77 & 24.92 \\
    $\DIAL_{\kl}$ (Ours) & 58.47 & \textbf{31.19} & \textbf{30.50} & \textbf{30.42} & \textbf{26.91} & 57.16 & 56.81 & 56.80 & 57.00 & \textbf{25.87} \\
    $\DIAL_{\ce}$ (Ours) & \textbf{60.77} & 27.87 & 26.66 & 26.61 & 25.98 & \textbf{59.48} & \textbf{59.06} & \textbf{58.96} & \textbf{59.20} & 23.51  \\
    \bottomrule
  \end{tabular}
\end{table}


% \begin{table}[!ht]
%   \caption{Robustness comparison of DIAL to Madry et al. and TRADES defense models on the SVHN dataset under different PGD white-box attacks and the ensemble Auto-Attack (AA).}
%   \label{svhn}
%   \centering
%   \begin{tabular}{llllll|l}
%     \toprule
%     \cmidrule(r){1-5}
%     Defense Model & Natural & PGD$^{20}$ & PGD$^{100}$ & PGD$^{1000}$ & CW$_{\infty}$ & AA\\
%     \midrule
%     $\DIAL_{\kl}$ (Ours) & $\mathbf{90.66}$ & $\mathbf{58.91}$ & $\mathbf{55.30}$ & $\mathbf{55.12}$ & $\mathbf{53.67}$  & $\mathbf{51.00}$  \\
%     Madry et al. & 89.90 & 53.23 & 49.45 & 49.23 & 48.25 & 45.25  \\
%     TRADES & 90.35 & 57.10 & 54.13 & 54.08 & 52.19 & 49.50 \\
%     \bottomrule
%   \end{tabular}
% \end{table}


\subsection{Performance comparison on CIFAR-10} \label{defence-settings}
In this part, we evaluate the performance of DIAL compared to other well-known methods on CIFAR-10. 
%To be consistent with other methods, 
We follow the same experiment setups as in~\cite{madry2017towards, wang2019improving, zhang2019theoretically}. When experiment settings are not identical between tested methods, we choose the most commonly used settings, and apply it to all experiments. This way, we keep the comparison as fair as possible and avoid reporting changes in results which are caused by inconsistent experiment settings \citep{pang2020bag}. To show that our results are not caused because of what is referred to as \textit{obfuscated gradients}~\citep{athalye2018obfuscated}, we evaluate our method with same setup as in our defense model, under strong attacks (e.g., PGD$^{1000}$) in both white-box, black-box settings, Auto-Attack ~\citep{croce2020reliable}, unforeseen "natural" corruptions~\citep{hendrycks2018benchmarking}, and unforeseen adversaries. To make sure that the reported improvements are not caused by \textit{adversarial overfitting}~\citep{rice2020overfitting}, we report best robust results for each method on average of 3 restarts, while omitting one standard deviation (which is smaller than 0.2\% in all experiments). Additional results for CIFAR-10 as well as comprehensive evaluation on MNIST can be found in Appendix \ref{mnist-results} and \ref{additional_res}.
%To further keep the comparison consistent, we followed the same attack settings for all methods.


\begin{table}[ht]
  \caption{Robustness against white-box, black-box attacks and Auto-Attack (AA) on CIFAR-10. Black-box attacks are generated using naturally trained surrogate model. Natural represents the naturally trained (non-adversarial) model.
  %and applied to the best performing robust models.
  }
  \vskip 0.1in
  \label{black-and_white-cifar}
  \centering
  \small
  \begin{tabular}{cccccccc@{\hspace{1\tabcolsep}}c}
    \toprule
    & & \multicolumn{3}{c}{White-box} & \multicolumn{3}{c}{Black-Box} \\
    \cmidrule(r){3-5} 
    \cmidrule(r){6-8}
    Defense Model & Natural & PGD$^{20}$ & PGD$^{100}$ & CW$^{\infty}$ & PGD$^{20}$ & PGD$^{100}$ & CW$^{\infty}$ & AA \\
    \midrule
    NATURAL & 95.43 & 0 & 0 & 0 & 0 & 0 & 0 &  0 \\
    \midrule
    TRADES & 84.92 & 56.60 & 55.56 & 54.20 & 84.08 & 83.89 & 83.91 &  53.08 \\
    MART & 83.62 & 58.12 & 56.48 & 53.09 & 82.82 & 82.52 & 82.80 & 51.10 \\
    AT & 85.10 & 56.28 & 54.46 & 53.99 & 84.22 & 84.14 & 83.92 & 51.52 \\
    ATDA & 76.91 & 43.27 & 41.13 & 41.01 & 75.59 & 75.37 & 75.35 & 40.08\\
    $\DIAL_{\kl}$ (Ours) & 85.25 & $\mathbf{58.43}$ & $\mathbf{56.80}$ & $\mathbf{55.00}$ & 84.30 & 84.18 & 84.05 & \textbf{53.75} \\
    $\DIAL_{\ce}$ (Ours)  & $\mathbf{89.59}$ & 54.31 & 51.67 & 52.04 &$ \mathbf{88.60}$ & $\mathbf{88.39}$ & $\mathbf{88.44}$ & 49.85 \\
    \midrule
    $\DIAL_{\awp}$ (Ours) & $\mathbf{85.91}$ & $\mathbf{61.10}$ & $\mathbf{59.86}$ & $\mathbf{57.67}$ & $\mathbf{85.13}$ & $\mathbf{84.93}$ & $\mathbf{85.03}$  & \textbf{56.78} \\
    $\TRADES_{\awp}$ & 85.36 & 59.27 & 59.12 & 57.07 & 84.58 & 84.58 & 84.59 & 56.17 \\
    \bottomrule
  \end{tabular}
\end{table}



\paragraph{CIFAR-10 setup.} We use the wide residual network (WRN-34-10)~\citep{zagoruyko2016wide} architecture. %used in the experiments of~\cite{madry2017towards, wang2019improving, zhang2019theoretically}. 
Sidelong this architecture, we integrate a domain classification layer. To generate the adversarial domain dataset, we use a perturbation size of $\epsilon=0.031$. We apply 10 of inner maximization iterations with perturbation step size of 0.007. Batch size is set to 128, weight decay is set to $7e^{-4}$, and the model is trained for 100 epochs. Similar to the other methods, the initial learning rate was set to 0.1, and decays by a factor of 10 at iterations 75 and 90. 
%For being consistent with other methods, the natural images are padded with 4-pixel padding with 32-random crop and random horizontal flip. Furthermore, all methods are trained using SGD with momentum 0.9. For $\DIAL_{\kl}$, we balance the robust loss with $\lambda=6$ and the domains loss with $r=4$. For $\DIAL_{\ce}$, we balance the robust loss with $\lambda=1$ and the domains loss with $r=2$. 
%We also introduce a version of our method that incorporates the AWP double-perturbation mechanism, named DIAL-AWP.
%which is trained using the same learning rate schedule used in ~\cite{wu2020adversarial}, where the initial 0.1 learning rate decays by a factor of 10 after 100 and 150 iterations. 
See Appendix \ref{cifar10-additional-setup} for additional details.

\begin{table}[ht]
  \caption{Black-box attack using the adversarially trained surrogate models on CIFAR-10.}
  \vskip 0.1in
  \label{black-box-cifar-adv}
  \centering
  \small
  \begin{tabular}{ll|c}
    \toprule
    \cmidrule(r){1-2}
    Surrogate (source) model & Target model & robustness \% \\
    % \midrule
    \midrule
    TRADES & $\DIAL_{\ce}$ & $\mathbf{67.77}$ \\
    $\DIAL_{\ce}$ & TRADES & 65.75 \\
    \midrule
    MART & $\DIAL_{\ce}$ & $\mathbf{70.30}$ \\
    $\DIAL_{\ce}$ & MART & 64.91 \\
    \midrule
    AT & $\DIAL_{\ce}$ & $\mathbf{65.32}$ \\
    $\DIAL_{\ce}$ & AT  & 63.54 \\
    \midrule
    ATDA & $\DIAL_{\ce}$ & $\mathbf{66.77}$ \\
    $\DIAL_{\ce}$ & ATDA & 52.56 \\
    \bottomrule
  \end{tabular}
\end{table}

\paragraph{White-box/Black-box robustness.} 
%We evaluate all defense models using Auto-Attack, PGD$^{20}$, PGD$^{100}$, PGD$^{1000}$ and CW$_{\infty}$ with step size 0.003. We constrain all attacks by the same perturbation $\epsilon=0.031$. 
As reported in Table~\ref{black-and_white-cifar} and Appendix~\ref{additional_res}, our method achieves better robustness compared to the other methods. Specifically, in the white-box settings, our method improves robustness over~\citet{madry2017towards} and TRADES by 2\% 
%using the common PGD$^{20}$ attack 
while keeping higher natural accuracy. We also observe better natural accuracy of 1.65\% over MART while also achieving better robustness over all attacks. Moreover, our method presents significant improvement of up to 15\% compared to the the domain invariant method suggested by~\citet{song2018improving} (ATDA).
%in both natural and robust accuracy. 
When incorporating 
%the double-perturbation mechanism of 
AWP, our method improves the results of $\TRADES_{\awp}$ by almost 2\%.
%and reaches state-of-the-art results for robust models with no additional data. 
% Additional results are available in Appendix~\ref{additional_res}.
When tested on black-box settings, $\DIAL_{\ce}$ presents a significant improvement of more than 4.4\% over the second-best performing method, and up to 13\%. In Table~\ref{black-box-cifar-adv}, we also present the black-box results when the source model is taken from one of the adversarially trained models. %Then, we compare our model to each one of them both as the source model and target model. 
In addition to the improvement in black-box robustness, $\DIAL_{\ce}$ also manages to achieve better clean accuracy of more than 4.5\% over the second-best performing method.
% Moreover, based on the auto-attack leader-board \footnote{\url{https://github.com/fra31/auto-attack}}, our method achieves the 1st place among models without additional data using the WRN-34-10 architecture.

% \begin{table}
%   \caption{White-box robustness on CIFAR-10 using WRN-34-10}
%   \label{white-box-cifar-10}
%   \centering
%   \begin{tabular}{lllll}
%     \toprule
%     \cmidrule(r){1-2}
%     Defense Model & Natural & PGD$^{20}$ & PGD$^{100}$ & PGD$^{1000}$ \\
%     \midrule
%     TRADES ~\cite{zhang2019theoretically} & 84.92  & 56.6 & 55.56 & 56.43  \\
%     MART ~\cite{wang2019improving} & 83.62  & 58.12 & 56.48 & 56.55  \\
%     Madry et al. ~\cite{madry2017towards} & 85.1  & 56.28 & 54.46 & 54.4  \\
%     Song et al. ~\cite{song2018improving} & 76.91 & 43.27 & 41.13 & 41.02  \\
%     $\DIAL_{\ce}$ (Ours) & $ \mathbf{90}$  & 52.12 & 48.88 & 48.78  \\
%     $\DIAL_{\kl}$ (Ours) & 85.25 & $\mathbf{58.43}$ & $\mathbf{56.8}$ & $\mathbf{56.73}$ \\
%     \midrule
%     $\DIAL_{\kl}$+AWP (Ours) & $\mathbf{85.91}$ & $\mathbf{61.1}$ & - & -  \\
%     TRADES+AWP \cite{wu2020adversarial} & 85.36 & 59.27 & 59.12 & -  \\
%     % MART + AWP & 84.43 & 60.68 & 59.32 & -  \\
%     \bottomrule
%   \end{tabular}
% \end{table}


% \begin{table}
%   \caption{White-box robustness on MNIST}
%   \label{white-box-mnist}
%   \centering
%   \begin{tabular}{llllll}
%     \toprule
%     \cmidrule(r){1-2}
%     Defense Model & Natural & PGD$^{40}$ & PGD$^{100}$ & PGD$^{1000}$ \\
%     \midrule
%     TRADES ~\cite{zhang2019theoretically} & 99.48 & 96.07 & 95.52 & 95.22 \\
%     MART ~\cite{wang2019improving} & 99.38  & 96.99 & 96.11 & 95.74  \\
%     Madry et al. ~\cite{madry2017towards} & 99.41  & 96.01 & 95.49 & 95.36 \\
%     Song et al. ~\cite{song2018improving}  & 98.72 & 96.82 & 96.26 & 96.2  \\
%     $\DIAL_{\kl}$ (Ours) & 99.46 & 97.05 & 96.06 & 95.99  \\
%     $\DIAL_{\ce}$ (Ours) & $\mathbf{99.49}$  & $\mathbf{97.38}$ & $\mathbf{96.45}$ & $\mathbf{96.33}$ \\
%     \bottomrule
%   \end{tabular}
% \end{table}


% \paragraph{Attacking MNIST.} For consistency, we use the same perturbation and step sizes. For MNIST, we use $\epsilon=0.3$ and step size of $0.01$. The natural accuracy of our surrogate (source) model is 99.51\%. Attacks results are reported in Table~\ref{black-and_white-mnist}. It is worth noting that the improvement margin is not conclusive on MNIST as it is on CIFAR-10, which is a more complex task.

% \begin{table}
%   \caption{Black-box robustness on MNIST and CIFAR-10 using naturally trained surrogate model and best performing robust models}
%   \label{black-box-mnist-cifar}
%   \centering
%   \begin{tabular}{lllllll}
%     \toprule
%     & \multicolumn{3}{c}{MNIST} & \multicolumn{3}{c}{CIFAR-10} \\
%     \cmidrule(r){2-4} 
%     \cmidrule(r){5-7}  
%     Defense Model & PGD$^{40}$ & PGD$^{100}$ & PGD$^{1000}$ & PGD$^{20}$ & PGD$^{100}$ & PGD$^{1000}$ \\
%     \midrule
%     TRADES ~\cite{zhang2019theoretically} & 98.12 & 97.86 & 97.81 & 84.08 & 83.89 & 83.8 \\
%     MART ~\cite{wang2019improving} & 98.16 & 97.96 & 97.89  & 82.82 & 82.52 & 82.47 \\
%     Madry et al. ~\cite{madry2017towards}  & 98.05 & 97.73 & 97.78 & 84.22 & 84.14 & 83.96 \\
%     Song et al. ~\cite{song2018improving} & 97.74 & 97.28 & 97.34 & 75.59 & 75.37 & 75.11 \\
%     $\DIAL_{\kl}$ (Ours) & 98.14 & 97.83 & 97.87  & 84.3 & 84.18 & 84.0 \\
%     $\DIAL_{\ce}$ (Ours)  & $\mathbf{98.37}$ & $\mathbf{98.12}$ & $\mathbf{98.05}$  & $\mathbf{89.13}$ & $\mathbf{88.89}$ & $\mathbf{88.78}$ \\
%     \bottomrule
%   \end{tabular}
% \end{table}



% \subsubsection{Ensemble attack} In addition to the white-box and black-box settings, we evaluate our method on the Auto-Attack ~\citep{croce2020reliable} using $\ell_{\infty}$ threat model with perturbation $\epsilon=0.031$. Auto-Attack is an ensemble of parameter-free attacks. It consists of three white-box attacks: APGD-CE which is a step size-free version of PGD on the cross-entropy ~\citep{croce2020reliable}. APGD-DLR which is a step size-free version of PGD on the DLR loss ~\citep{croce2020reliable} and FAB which  minimizes the norm of the adversarial perturbations, and one black-box attack: square attack which is a query-efficient black-box attack ~\citep{andriushchenko2020square}. Results are presented in Table~\ref{auto-attack}. Based on the auto-attack leader-board \footnote{\url{https://github.com/fra31/auto-attack}}, our method achieves the 1st place among models without additional data using the WRN-34-10 architecture.

%Additional results can be found in Appendix ~\ref{additional_res}.

% \begin{table}
%   \caption{Auto-Attack (AA) on CIFAR-10 with perturbation size $\epsilon=0.031$ with $\ell_{\infty}$ threat model}
%   \label{auto-attack}
%   \centering
%   \begin{tabular}{lll}
%     \toprule
%     \cmidrule(r){1-2}
%     Defense Model & AA \\
%     \midrule
%     TRADES ~\cite{zhang2019theoretically} & 53.08  \\
%     MART ~\cite{wang2019improving} & 51.1  \\
%     Madry et al. ~\cite{madry2017towards} & 51.52    \\
%     Song et al.   ~\cite{song2018improving} & 40.18 \\
%     $\DIAL_{\ce}$ (Ours) & 47.33  \\
%     $\DIAL_{\kl}$ (Ours) & $\mathbf{53.75}$ \\
%     \midrule
%     DIAL-AWP (Ours) & $\mathbf{56.78}$ \\
%     TRADES-AWP \cite{wu2020adversarial} & 56.17 \\
%     \bottomrule
%   \end{tabular}
% \end{table}


% \begin{table}[!ht]
%   \caption{Auto-Attack (AA) Robustness (\%) on CIFAR-10 with $\epsilon=0.031$ using an $\ell_{\infty}$ threat model}
%   \label{auto-attack}
%   \centering
%   \begin{tabular}{cccccc|cc}
%     \toprule
%     % \multicolumn{8}{c}{Defence Model}  \\
%     % \cmidrule(r){1-8} 
%     TRADES & MART & Madry & Song & $\DIAL_{\ce}$ & $\DIAL_{\kl}$ & DIAL-AWP  & TRADES-AWP\\
%     \midrule
%     53.08 & 51.10 & 51.52 &  40.08 & 47.33  & $\mathbf{53.75}$ & $\mathbf{56.78}$ & 56.17 \\

%     \bottomrule
%   \end{tabular}
% \end{table}

% \begin{table}[!ht]
% \caption{$F_1$-robust measurement using PGD$^{20}$ attack in white-box and black-box settings on CIFAR-10}
%   \label{f1-robust}
%   \centering
%   \begin{tabular}{ccccccc|cc}
%     \toprule
%     % \multicolumn{8}{c}{Defence Model}  \\
%     % \cmidrule(r){1-8} 
%     Defense Model & TRADES & MART & Madry & Song & $\DIAL_{\kl}$ & $\DIAL_{\ce}$ & DIAL-AWP  & TRADES-AWP\\
%     \midrule
%     White-box & 0.659 & 0.666 & 0.657 & 0.518 & $\mathbf{0.675}$  & 0.643 & $\mathbf{0.698}$ & 0.682 \\
%     Black-box & 0.844 & 0.831 & 0.846 & 0.761 & 0.847 & $\mathbf{0.895}$ & $\mathbf{0.854}$ &  0.849 \\
%     \bottomrule
%   \end{tabular}
% \end{table}

\subsubsection{Robustness to Unforeseen Attacks and Corruptions}
\paragraph{Unforeseen Adversaries.} To further demonstrate the effectiveness of our approach, we test our method against various adversaries that were not used during the training process. We attack the model under the white-box settings with $\ell_{2}$-PGD, $\ell_{1}$-PGD, $\ell_{\infty}$-DeepFool and $\ell_{2}$-DeepFool \citep{moosavi2016deepfool} adversaries using Foolbox \citep{rauber2017foolbox}. We applied commonly used attack budget 
%(perturbation for PGD adversaries and overshot for DeepFool adversaries) 
with 20 and 50 iterations for PGD and DeepFool, respectively.
Results are presented in Table \ref{unseen-attacks}. As can be seen, our approach  gains an improvement of up to 4.73\% over the second best method under the various attack types and an average improvement of 3.7\% over all threat models.


\begin{table}[ht]
  \caption{Robustness on CIFAR-10 against unseen adversaries under white-box settings.}
  \vskip 0.1in
  \label{unseen-attacks}
  \centering
%   \small
  \begin{tabular}{c@{\hspace{1.5\tabcolsep}}c@{\hspace{1.5\tabcolsep}}c@{\hspace{1.5\tabcolsep}}c@{\hspace{1.5\tabcolsep}}c@{\hspace{1.5\tabcolsep}}c@{\hspace{1.5\tabcolsep}}c@{\hspace{1.5\tabcolsep}}c}
    \toprule
    Threat Model & Attack Constraints & $\DIAL_{\kl}$ & $\DIAL_{\ce}$ & AT & TRADES & MART & ATDA \\
    \midrule
    \multirow{2}{*}{$\ell_{2}$-PGD} & $\epsilon=0.5$ & 76.05 & \textbf{80.51} & 76.82 & 76.57 & 75.07 & 66.25 \\
    & $\epsilon=0.25$ & 80.98 & \textbf{85.38} & 81.41 & 81.10 & 80.04 & 71.87 \\\midrule
    \multirow{2}{*}{$\ell_{1}$-PGD} & $\epsilon=12$ & 74.84 & \textbf{80.00} & 76.17 & 75.52 & 75.95 & 65.76 \\
    & $\epsilon=7.84$ & 78.69 & \textbf{83.62} & 79.86 & 79.16 & 78.55 & 69.97 \\
    \midrule
    $\ell_{2}$-DeepFool & overshoot=0.02 & 84.53 & \textbf{88.88} & 84.15 & 84.23 & 82.96 & 76.08 \\\midrule
    $\ell_{\infty}$-DeepFool & overshoot=0.02 & 68.43 & \textbf{69.50} & 67.29 & 67.60 & 66.40 & 57.35 \\
    \bottomrule
  \end{tabular}
\end{table}


%%%%%%%%%%%%%%%%%%%%%%%%% conference version %%%%%%%%%%%%%%%%%%%%%%%%%%%%%%%%%%%%%
\paragraph{Unforeseen Corruptions.}
We further demonstrate that our method consistently holds against unforeseen ``natural'' corruptions, consists of 18 unforeseen diverse corruption types proposed by \citet{hendrycks2018benchmarking} on CIFAR-10, which we refer to as CIFAR10-C. The CIFAR10-C benchmark covers noise, blur, weather, and digital categories. As can be shown in Figure \ref{corruption}, our method gains a significant and consistent improvement over all the other methods. Our method leads to an average improvement of 4.7\% with minimum improvement of 3.5\% and maximum improvement of 5.9\% compared to the second best method over all unforeseen attacks. See Appendix \ref{corruptions-apendix} for the full experiment results.


\begin{figure}[h]
 \centering
  \includegraphics[width=0.4\textwidth]{figures/spider_full.png}
%   \caption{Summary of accuracy over all unforeseen corruptions compared to the second and third best performing methods.}
  \caption{Accuracy comparison over all unforeseen corruptions.}
  \label{corruption}
\end{figure}


%%%%%%%%%%%%%%%%%%%%%%%%% conference version %%%%%%%%%%%%%%%%%%%%%%%%%%%%%%%%%%%%%

%%%%%%%%%%%%%%%%%%%%%%%%% Arxiv version %%%%%%%%%%%%%%%%%%%%%%%%%%%%%%%%%%%%%
% \newpage
% \paragraph{Unforeseen Corruptions.}
% We further demonstrate that our method consistently holds against unforeseen "natural" corruptions, consists of 18 unforeseen diverse corruption types proposed by \cite{hendrycks2018benchmarking} on CIFAR-10, which we refer to as CIFAR10-C. The CIFAR10-C benchmark covers noise, blur, weather, and digital categories. As can be shown in Figure  \ref{spider-full-graph}, our method gains a significant and consistent improvement over all the other methods. Our approach leads to an average improvement of 4.7\% with minimum improvement of 3.5\% and maximum improvement of 5.9\% compared to the second best method over all unforeseen attacks. Full accuracy results against unforeseen corruptions are presented in Tables \ref{corruption-table1} and \ref{corruption-table2}. 

% \begin{table}[!ht]
%   \caption{Accuracy (\%) against unforeseen corruptions.}
%   \label{corruption-table1}
%   \centering
%   \tiny
%   \begin{tabular}{lcccccccccccccccccc}
%     \toprule
%     Defense Model & brightness & defocus blur & fog & glass blur & jpeg compression & motion blur & saturate & snow & speckle noise  \\
%     \midrule
%     TRADES & 82.63 & 80.04 & 60.19 & 78.00 & 82.81 & 76.49 & 81.53 & 80.68 & 80.14 \\
%     MART & 80.76 & 78.62 & 56.78 & 76.60 & 81.26 & 74.58 & 80.74 & 78.22 & 79.42 \\
%     AT &  83.30 & 80.42 & 60.22 & 77.90 & 82.73 & 76.64 & 82.31 & 80.37 & 80.74 \\
%     ATDA & 72.67 & 69.36 & 45.52 & 64.88 & 73.22 & 63.47 & 72.07 & 68.76 & 72.27 \\
%     DIAL (Ours)  & \textbf{87.14} & \textbf{84.84} & \textbf{66.08} & \textbf{81.82} & \textbf{87.07} & \textbf{81.20} & \textbf{86.45} & \textbf{84.18} & \textbf{84.94} \\
%     \bottomrule
%   \end{tabular}
% \end{table}


% \begin{table}[!ht]
%   \caption{Accuracy (\%) against unforeseen corruptions.}
%   \label{corruption-table2}
%   \centering
%   \tiny
%   \begin{tabular}{lcccccccccccccccccc}
%     \toprule
%     Defense Model & contrast & elastic transform & frost & gaussian noise & impulse noise & pixelate & shot noise & spatter & zoom blur \\
%     \midrule
%     TRADES & 43.11 & 79.11 & 76.45 & 79.21 & 73.72 & 82.73 & 80.42 & 80.72 & 78.97 \\
%     MART & 41.22 & 77.77 & 73.07 & 78.30 & 74.97 & 81.31 & 79.53 & 79.28 & 77.8 \\
%     AT & 43.30 & 79.58 & 77.53 & 79.47 & 73.76 & 82.78 & 80.86 & 80.49 & 79.58 \\
%     ATDA & 36.06 & 67.06 & 62.56 & 70.33 & 64.63 & 73.46 & 72.28 & 70.50 & 67.31 \\
%     DIAL (Ours) & \textbf{48.84} & \textbf{84.13} & \textbf{81.76} & \textbf{83.76} & \textbf{78.26} & \textbf{87.24} & \textbf{85.13} & \textbf{84.84} & \textbf{83.93}  \\
%     \bottomrule
%   \end{tabular}
% \end{table}


% \begin{figure}[!ht]
%   \centering
%   \includegraphics[width=9cm]{figures/spider_full.png}
%   \caption{Accuracy comparison with all tested methods over unforeseen corruptions.}
%   \label{spider-full-graph}
% \end{figure}
% %%%%%%%%%%%%%%%%%%%%%%%%% Arxiv version %%%%%%%%%%%%%%%%%%%%%%%%%%%%%%%%%%%%%
%%%%%%%%%%%%%%%%%%%%%%%%% Arxiv version %%%%%%%%%%%%%%%%%%%%%%%%%%%%%%%%%%%%%

\subsubsection{Transfer Learning}
Recent works \citep{salman2020adversarially,utrera2020adversarially} suggested that robust models transfer better on standard downstream classification tasks. In Table \ref{transfer-res} we demonstrate the advantage of our method when applied for transfer learning across CIFAR10 and CIFAR100 using the common linear evaluation protocol. see Appendix \ref{transfer-learning-settings} for detailed settings.

\begin{table}[H]
  \caption{Transfer learning results comparison.}
  \vskip 0.1in
  \label{transfer-res}
  \centering
  \small
\begin{tabular}{c|c|c|c}
\toprule

\multicolumn{2}{l}{} & \multicolumn{2}{c}{Target} \\
\cmidrule(r){3-4}
Source & Defence Model & CIFAR10 & CIFAR100 \\
\midrule
\multirow{3}{*}{CIFAR10} & DIAL & \multirow{3}{*}{-} & \textbf{28.57} \\
 & AT &  & 26.95  \\
 & TRADES &  & 25.40  \\
 \midrule
\multirow{3}{*}{CIFAR100} & DIAL & \textbf{73.68} & \multirow{3}{*}{-} \\
 & AT & 71.41 & \\
 & TRADES & 71.42 &  \\
%  \midrule
% \multirow{3}{}{SVHN} & DIAL &  &  & \multirow{3}{}{-} \\
%  & Madry et al. &  &  &  \\
%  & TRADES &  &  &  \\ 
\bottomrule
\end{tabular}
\end{table}


\subsubsection{Modularity and Ablation Studies}

We note that the domain classifier is a modular component that can be integrated into existing models for further improvements. Removing the domain head and related loss components from the different DIAL formulations results in some common adversarial training techniques. For $\DIAL_{\kl}$, removing the domain and related loss components results in the formulation of TRADES. For $\DIAL_{\ce}$, removing the domain and related loss components results in the original formulation of the standard adversarial training, and for $\DIAL_{\awp}$ the removal results in $\TRADES_{\awp}$. Therefore, the ablation studies will demonstrate the effectiveness of combining DIAL on top of different adversarial training methods. 

We investigate the contribution of the additional domain head component introduced in our method. Experiment configuration are as in \ref{defence-settings}, and robust accuracy is based on white-box PGD$^{20}$ on CIFAR-10 dataset. We remove the domain head from both $\DIAL_{\kl}$, $\DIAL_{\awp}$, and $\DIAL_{\ce}$ (equivalent to $r=0$) and report the natural and robust accuracy. We perform 3 random restarts and omit one standard deviation from the results. Results are presented in Figure \ref{ablation}. All DIAL variants exhibits stable improvements on both natural accuracy and robust accuracy. $\DIAL_{\ce}$, $\DIAL_{\kl}$, and $\DIAL_{\awp}$ present an improvement of 1.82\%, 0.33\%, and 0.55\% on natural accuracy and an improvement of 2.5\%, 1.87\%, and 0.83\% on robust accuracy, respectively. This evaluation empirically demonstrates the benefits of incorporating DIAL on top of different adversarial training techniques.
% \paragraph{semi-supervised extensions.} Since the domain classifier does not require the class labels, we argue that additional unlabeled data can be leveraged in future work.
%for improved results. 

\begin{figure}[ht]
  \centering
  \includegraphics[width=0.35\textwidth]{figures/ablation_graphs3.png}
  \caption{Ablation studies for $\DIAL_{\kl}$, $\DIAL_{\ce}$, and $\DIAL_{\awp}$ on CIFAR-10. Circle represent the robust-natural accuracy without using DIAL, and square represent the robust-natural accuracy when incorporating DIAL.
  %to further investigate the impact of the domain head and loss on natural and robust accuracy.
  }
  \label{ablation}
\end{figure}

\subsubsection{Visualizing DIAL}
To further illustrate the superiority of our method, we visualize the model outputs from the different methods on both natural and adversarial test data.
% adversarial test data generated using PGD$^{20}$ white-box attack with step size 0.003 and $\epsilon=0.031$ on CIFAR-10. 
Figure~\ref{tsne1} shows the embedding received after applying t-SNE ~\citep{van2008visualizing} with two components on the model output for our method and for TRADES. DIAL seems to preserve strong separation between classes on both natural test data and adversarial test data. Additional illustrations for the other methods are attached in Appendix~\ref{additional_viz}. 

\begin{figure}[h]
\centering
  \subfigure[\textbf{DIAL} on natural logits]{\includegraphics[width=0.21\textwidth]{figures/domain_ce_test.png}}
  \hspace{0.03\textwidth}
  \subfigure[\textbf{DIAL} on adversarial logits]{\includegraphics[width=0.21\textwidth]{figures/domain_ce_adversarial.png}}
  \hspace{0.03\textwidth}
    \subfigure[\textbf{TRADES} on natural logits]{\includegraphics[width=0.21\textwidth]{figures/trades_test.png}}
    \hspace{0.03\textwidth}
    \subfigure[\textbf{TRADES} on adversarial logits]{\includegraphics[width=0.21\textwidth]{figures/trades_adversarial.png}}
  \caption{t-SNE embedding of model output (logits) into two-dimensional space for DIAL and TRADES using the CIFAR-10 natural test data and the corresponding PGD$^{20}$ generated adversarial examples.}
  \label{tsne1}
\end{figure}


% \begin{figure}[ht]
% \centering
%   \begin{subfigure}{4cm}
%     \centering\includegraphics[width=3.3cm]{figures/domain_ce_test.png}
%     \caption{\textbf{DIAL} on nat. examples}
%   \end{subfigure}
%   \begin{subfigure}{4cm}
%     \centering\includegraphics[width=3.3cm]{figures/domain_ce_adversarial.png}
%     \caption{\textbf{DIAL} on adv. examples}
%   \end{subfigure}
  
%   \begin{subfigure}{4cm}
%     \centering\includegraphics[width=3.3cm]{figures/trades_test.png}
%     \caption{\textbf{TRADES} on nat. examples}
%   \end{subfigure}
%   \begin{subfigure}{4cm}
%     \centering\includegraphics[width=3.3cm]{figures/trades_adversarial.png}
%     \caption{\textbf{TRADES} on adv. examples}
%   \end{subfigure}
%   \caption{t-SNE embedding of model output (logits) into two-dimensional space for DIAL and TRADES using the CIFAR-10 natural test data and the corresponding adversarial examples.}
%   \label{tsne1}
% \end{figure}



% \begin{figure}[ht]
% \centering
%   \begin{subfigure}{6cm}
%     \centering\includegraphics[width=5cm]{figures/domain_ce_test.png}
%     \caption{\textbf{DIAL} on nat. examples}
%   \end{subfigure}
%   \begin{subfigure}{6cm}
%     \centering\includegraphics[width=5cm]{figures/domain_ce_adversarial.png}
%     \caption{\textbf{DIAL} on adv. examples}
%   \end{subfigure}
  
%   \begin{subfigure}{6cm}
%     \centering\includegraphics[width=5cm]{figures/trades_test.png}
%     \caption{\textbf{TRADES} on nat. examples}
%   \end{subfigure}
%   \begin{subfigure}{6cm}
%     \centering\includegraphics[width=5cm]{figures/trades_adversarial.png}
%     \caption{\textbf{TRADES} on adv. examples}
%   \end{subfigure}
%   \caption{t-SNE embedding of model output (logits) into two-dimensional space for DIAL and TRADES using the CIFAR-10 natural test data and the corresponding adversarial examples.}
%   \label{tsne1}
% \end{figure}



\subsection{Balanced measurement for robust-natural accuracy}
One of the goals of our method is to better balance between robust and natural accuracy under a given model. For a balanced metric, we adopt the idea of $F_1$-score, which is the harmonic mean between the precision and recall. However, rather than using precision and recall, we measure the $F_1$-score between robustness and natural accuracy,
using a measure we call
%We named it
the
\textbf{$\mathbf{F_1}$-robust} score.
\begin{equation}
F_1\text{-robust} = \dfrac{\text{true\_robust}}
{\text{true\_robust}+\frac{1}{2}
%\cdot
(\text{false\_{robust}}+
\text{false\_natural})},
\end{equation}
where $\text{true\_robust}$ are the adversarial examples that were correctly classified, $\text{false\_{robust}}$ are the adversarial examples that were miss-classified, and $\text{false\_natural}$ are the natural examples that were miss-classified.
%We tested the proposed $F_1$-robust score using PGD$^{20}$ on CIFAR-10 dataset in white-box and black-box settings. 
Results are presented in Table~\ref{f1-robust} and demonstrate that our method achieves the best $F_1$-robust score in both settings, which supports our findings from previous sections.

% \begin{table}[!ht]
%   \caption{$F_1$-robust measurement using PGD$^{20}$ attack in white and black box settings on CIFAR-10}
%   \label{f1-robust}
%   \centering
%   \begin{tabular}{lll}
%     \toprule
%     \cmidrule(r){1-2}
%     Defense Model & White-box & Black-box \\
%     \midrule
%     TRADES & 0.65937  & 0.84435 \\
%     MART & 0.66613  & 0.83153  \\
%     Madry et al. & 0.65755 & 0.84574   \\
%     Song et al. & 0.51823 & 0.76092  \\
%     $\DIAL_{\ce}$ (Ours) & 0.65318   & $\mathbf{0.88806}$  \\
%     $\DIAL_{\kl}$ (Ours) & $\mathbf{0.67479}$ & 0.84702 \\
%     \midrule
%     \midrule
%     DIAL-AWP (Ours) & $\mathbf{0.69753}$  & $\mathbf{0.85406}$  \\
%     TRADES-AWP & 0.68162 & 0.84917 \\
%     \bottomrule
%   \end{tabular}
% \end{table}

\begin{table}[ht]
\small
  \caption{$F_1$-robust measurement using PGD$^{20}$ attack in white and black box settings on CIFAR-10.}
  \vskip 0.1in
  \label{f1-robust}
  \centering
%   \small
  \begin{tabular}{c
  @{\hspace{1.5\tabcolsep}}c @{\hspace{1.5\tabcolsep}}c @{\hspace{1.5\tabcolsep}}c @{\hspace{1.5\tabcolsep}}c
  @{\hspace{1.5\tabcolsep}}c @{\hspace{1.5\tabcolsep}}c @{\hspace{1.5\tabcolsep}}|
  @{\hspace{1.5\tabcolsep}}c
  @{\hspace{1.5\tabcolsep}}c}
    \toprule
    % \cmidrule(r){8-9}
     & TRADES & MART & AT & ATDA & $\DIAL_{\ce}$ & $\DIAL_{\kl}$ & $\DIAL_{\awp}$ & $\TRADES_{\awp}$ \\
    \midrule
    White-box & 0.659 & 0.666 & 0.657 & 0.518 & 0.660 & \textbf{0.675} & \textbf{0.698} & 0.682 \\
    Black-box & 0.844 & 0.831 & 0.845 & 0.761 & \textbf{0.890} & 0.847 & \textbf{0.854} & 0.849 \\ 
    \bottomrule
  \end{tabular}
\end{table}


\section{Conclusion}
\label{sec:conclusion}
% \vspace{-0.5em}
\section{Conclusion}
% \vspace{-0.5em}
Recent advances in multimodal single-cell technology have enabled the simultaneous profiling of the transcriptome alongside other cellular modalities, leading to an increase in the availability of multimodal single-cell data. In this paper, we present \method{}, a multimodal transformer model for single-cell surface protein abundance from gene expression measurements. We combined the data with prior biological interaction knowledge from the STRING database into a richly connected heterogeneous graph and leveraged the transformer architectures to learn an accurate mapping between gene expression and surface protein abundance. Remarkably, \method{} achieves superior and more stable performance than other baselines on both 2021 and 2022 NeurIPS single-cell datasets.

\noindent\textbf{Future Work.}
% Our work is an extension of the model we implemented in the NeurIPS 2022 competition. 
Our framework of multimodal transformers with the cross-modality heterogeneous graph goes far beyond the specific downstream task of modality prediction, and there are lots of potentials to be further explored. Our graph contains three types of nodes. While the cell embeddings are used for predictions, the remaining protein embeddings and gene embeddings may be further interpreted for other tasks. The similarities between proteins may show data-specific protein-protein relationships, while the attention matrix of the gene transformer may help to identify marker genes of each cell type. Additionally, we may achieve gene interaction prediction using the attention mechanism.
% under adequate regulations. 
% We expect \method{} to be capable of much more than just modality prediction. Note that currently, we fuse information from different transformers with message-passing GNNs. 
To extend more on transformers, a potential next step is implementing cross-attention cross-modalities. Ideally, all three types of nodes, namely genes, proteins, and cells, would be jointly modeled using a large transformer that includes specific regulations for each modality. 

% insight of protein and gene embedding (diff task)

% all in one transformer

% \noindent\textbf{Limitations and future work}
% Despite the noticeable performance improvement by utilizing transformers with the cross-modality heterogeneous graph, there are still bottlenecks in the current settings. To begin with, we noticed that the performance variations of all methods are consistently higher in the ``CITE'' dataset compared to the ``GEX2ADT'' dataset. We hypothesized that the increased variability in ``CITE'' was due to both less number of training samples (43k vs. 66k cells) and a significantly more number of testing samples used (28k vs. 1k cells). One straightforward solution to alleviate the high variation issue is to include more training samples, which is not always possible given the training data availability. Nevertheless, publicly available single-cell datasets have been accumulated over the past decades and are still being collected on an ever-increasing scale. Taking advantage of these large-scale atlases is the key to a more stable and well-performing model, as some of the intra-cell variations could be common across different datasets. For example, reference-based methods are commonly used to identify the cell identity of a single cell, or cell-type compositions of a mixture of cells. (other examples for pretrained, e.g., scbert)


%\noindent\textbf{Future work.}
% Our work is an extension of the model we implemented in the NeurIPS 2022 competition. Now our framework of multimodal transformers with the cross-modality heterogeneous graph goes far beyond the specific downstream task of modality prediction, and there are lots of potentials to be further explored. Our graph contains three types of nodes. while the cell embeddings are used for predictions, the remaining protein embeddings and gene embeddings may be further interpreted for other tasks. The similarities between proteins may show data-specific protein-protein relationships, while the attention matrix of the gene transformer may help to identify marker genes of each cell type. Additionally, we may achieve gene interaction prediction using the attention mechanism under adequate regulations. We expect \method{} to be capable of much more than just modality prediction. Note that currently, we fuse information from different transformers with message-passing GNNs. To extend more on transformers, a potential next step is implementing cross-attention cross-modalities. Ideally, all three types of nodes, namely genes, proteins, and cells, would be jointly modeled using a large transformer that includes specific regulations for each modality. The self-attention within each modality would reconstruct the prior interaction network, while the cross-attention between modalities would be supervised by the data observations. Then, The attention matrix will provide insights into all the internal interactions and cross-relationships. With the linearized transformer, this idea would be both practical and versatile.

% \begin{acks}
% This research is supported by the National Science Foundation (NSF) and Johnson \& Johnson.
% \end{acks}

% % Acknowledgements should only appear in the accepted version.
\section*{Acknowledgements}

The authors wish to acknowledge Andras Banki-Horvath for open-sourcing the StreetLearn environment, Lasse Espeholt and Hubert Soyer for technical help with the IMPALA algorithm, Razvan Pascanu, Ross Goroshin, Pushmeet Kohli and Nando de Freitas for their feedback, Chloe Hillier, Razia Ahamed and Vishal Maini for help with the project, and the Google Street View team (Tilman Reinhardt, Wenfeng Li, Ben Mears, Karen Guo, Oliver Metzger, Jayanth Nayak) as well as Richard Ives and Ashwin Kakarla for their support in accessing the data.

\bibliography{streetlearn}
\bibliographystyle{plain}

\newpage
\appendix{}

\section{Video of the Agent Trajectories and Observations}
\label{supp:video}
The video available at \url{http://streetlearn.cc} and \url{https://youtu.be/2yjWDNXYh5s} shows the performance of trained \emph{CityNav} agents in the Paris Rive Gauche and Central London environments, as well as of the \emph{MultiCityNav} agents trained jointly on 4 environments (Greenwich Village, Midtown, Central Park and Harlem) and then transferred to a fifth environment (Lower Manhattan). The video shows the high-resolution StreetView images (actual inputs to the network are $84\times84$ RGB observations), overlaid with the map of the environment indicating its position and the location of the goal.

\section{Further Analysis}
\label{supp:analysis}

\section{Case Studies}
\label{sec:case_studies}
In this section, we present a case study of Facebook posts from an Australian public page.
The page shifts between early 2020 (\emph{2019-2020 Australian bushfire season}) and late 2020 (\emph{COVID-19 crises}) from being a moderate-right group for discussion around climate change to a far-right extremist group for conspiracy theories.


\begin{figure*}[!tbp]
	\begin{subfigure}{0.21\textwidth}
		\includegraphics[width=\textwidth]{images/facebook1.png}
		\caption{}
		\label{subfig:first-posting}
		\includegraphics[width=0.9\textwidth]{images/facebook3.jpg}
		\caption{}
		\label{subfig:comment-post-1}
	\end{subfigure}
    \begin{subfigure}{0.28\textwidth}
		\includegraphics[width=\textwidth]{images/facebook2.jpg}
		\caption{}
		\label{subfig:second-posting}
	\end{subfigure}
    \begin{subfigure}{0.23\textwidth}
		\includegraphics[width=\textwidth]{images/facebook4.jpg}
		\caption{}
		\label{subfig:comment-post-2a}
	\end{subfigure}
    \begin{subfigure}{0.23\textwidth}
		\includegraphics[width=\textwidth]{images/facebook5.jpg}
		\caption{}
		\label{subfig:comment-post-2b}
	\end{subfigure}
	\caption{
		Examples of postings and comment threads from a public Facebook page from two periods of time early 2020 (a) and late 2020 (b)-(e), which show a shift from climate change debates to extremist and far-right messaging.
	}
	\label{fig:facebook}
\end{figure*}

We focus on a sample of 2 postings and commenting threads from one Australian Facebook page we classified as ``far-right'' based on the content on the page. 
We have anonymized the users in \Cref{fig:facebook} to avoid re-identification.
The first posting and comment thread (see \Cref{subfig:first-posting}) was collected on Jan 10, 2020, and responded to the Australian bushfire crisis that began in late 2019 and was still ongoing in January 2020. It contains an ambivalent text-based provocation that references disputes in the community regarding the validity of climate change and climate science. 

The second posting and comment thread (see \Cref{subfig:second-posting}) was collected from the same page in September 2020, months after the bushfire crisis had abated.
At that time, a new crisis was energizing and connecting the far-right groups in our dataset --- i.e., the COVID-19 pandemic and the government interventions to curb the spread of the virus. 
The post is different in style compared to the first.
It is image-based instead of text-based and highly emotive, with a photo collage bringing together images of prison inmates with iron masks on their faces (top row) juxtaposed to people wearing face masks during COVID-19 (bottom row). 
The image references the public health orders issued during Melbourne's second lockdown and suggests that being ordered to wear masks is an infringement of citizen rights and freedoms, similar to dehumanizing restraints used on prisoners.

To analyze reactions to the posts, two researchers used a deductive analytical approach to separately code and to analyze the commenting threads --- see \Cref{subfig:comment-post-1} for comments of the first posting, and \Cref{subfig:comment-post-2a,subfig:comment-post-2b} for comments on the second posting. 
Conversations were also inductively coded for emerging themes. 
During the analysis, we observed qualitative differences in the types of content users posted, interactions between commenters, tone and language of debate, linked media shared in the commenting thread, and the opinions expressed.
The rest of this section further details these differences.
To ensure this was not a random occurrence, we tested the exemplar threads against field notes collected on the group during the entire study.
We also used Facebook's search function within pages to find a sample of posts from the same period and which dealt with similar topics. 
After this analysis, we can confidently say that key changes occurred in the group between the bushfire crisis and COVID-19, that we detail next.

\subsubsection*{Exemplar 1 --- climate change skepticism.}
To explore this transformation in more depth, we analyzed comments scraped on the first posting --- \cref{sub@subfig:comment-post-1} shows a small sample of these comments.
The language used was similar to comments that we observed on numerous far-right nationalist pages at the time of the bushfires.
These comments are usually text-based, employing emojis to denote emotions, and sometimes being mocking or provocative in tone. 
Noteworthy for this commenting thread is the 50/50 split in the number of members posting in favor of action on climate change (on one side) and those who posted anti-Greens and anti-climate change science posts and memes (on the other side).
The two sides aligned strongly with political partisanship --- either with Liberal/National coalition (climate change deniers) or Labor/Green (climate change believers) parties. 
This is rather unusual for pages classified as far-right. 

We observed trolling practices between the climate change deniers and believers, which often descend into \emph{flame wars} --- i.e., online ``firefights that take place between disembodied combatants on electronic bulletin boards''~\citep{bukatman1994flame}.
The result is a boosted engagement on the post but also the frustration and confusion of community members and lurkers who came to the discussions to become informed or debate rationally on key differences between the two positions.
They often even become targeted, victimized, and baited by trolls on both sides of the partisan divide. 
The opinions expressed by deniers in commenting sections range from skepticism regarding climate change science to plain denial.
Deniers also regard a range of targets as embroiled in a climate change conspiracy to deceive the public, such as The Greens and their environmental policy, in some cases the government, the United Nations, and climate change celebrities like David Attenborough and Greta Thunberg. 
These figures are blamed for either exaggerating risks of climate change or creating a climate change hoax to increase the influence of the UN on domestic governments or to increase domestic governments' social control over citizens. 

Both coders noted that flame wars between these opposing personas contained very few links to external media. 
Where links were added, they often seemed disconnected from the rest of the conversation and were from users whose profiles suggested they believed in more radical conspiracy theories.
One such example is ``geo-engineering'' (see \cref{sub@subfig:comment-post-1}).
Its adherents believe that solar geo-engineering programs designed to combat climate change are secretly used by a global elite to depopulate the world through sterilization or to control and weaponize the weather.

Nonetheless, apart from the random comments that hijack the thread, redirecting users to external ``alternative'' news sites and Twitter, and the trolls who seem to delight in victimizing unsuspecting victims, the discussion was pretty healthy.
There are many questions, rational inquiries, and debates between users of different political persuasion and views on climate change.
This, however, changes in the span of only a couple of months.

\subsubsection*{Exemplar 2 --- posting and commenting thread.}
We observe a shift in the comment section of the post collected during the second wave of the COVID pandemic (\Cref{sub@subfig:second-posting}) --- which coincided with government laws mandating the public to wear masks and stay at home in Victoria, Australia.
There emerges much more extreme far-right content that converges with anti-vaccination opinions and content.
We also note a much higher prevalence of conspiracy theories often implicating racialized targets.
This is exemplified in the comments on the second post (\Cref{sub@subfig:comment-post-2a,sub@subfig:comment-post-2b}) where Islamophobia and antisemitism are confidently asserted alongside anti-mask rhetoric.
These comments consider face masks similar to the religious head coverings worn by some Muslim women, which users describe as ``oppressive'' and ``silencing''. 
In this way, anti-maskers cast women as a distinct, sympathetic marginalized demographic.
However, this is enacted alongside the racialization and demonization of Islam as an oppressive religion. 

Given the extreme racialization of anti-mask rhetoric, some commenters contest these positions, arguing that the page is becoming less an anti-Scott Morrison page (Australia's Prime Minister at the time) and changing into a page that harbors ``far-right dickheads''.
This questioning is actively challenged by far-right commenters and conspiracy theorists on the page, who regarded pro-mask users and the Scott Morrison government as ``puppets'' being manipulated by higher forces (see \Cref{sub@subfig:comment-post-2b}). 

This indicates a significant change on the page's membership towards the extreme-right, who employs more extreme forms of racialized imagery, with more extreme opinion being shared.
Conspiracy theorists become more active and vocal, and they consistently challenge the opinions of both center conservative and left-leaning users. 
This is evident in the final two comments in \Cref{subfig:comment-post-2b}, which reflect QAnon style conspiracy theories and language.
Public health orders to wear masks are being connected to a conspiracy that all of these decisions are directed by a secret network of global Jewish elites, who manipulate the pandemic to increase their power and control. 
This rhetoric intersects with the contemporary ``QAnon'' conspiracy theory, which evolved from the ``Pizzagate'' conspiracy theory.
They also heavily draw on well-established antisemitic blood libel conspiracy theories, which foster beliefs that a powerful global elite is controlling the decisions of organizations such as WHO and are responsible for the vaccine rollout and public health orders related to the pandemic.
The QAnon conspiracy is also influenced by Bill Gates' Microchips conspiracy theory, i.e., the theory that the WHO and the Bill Gates Foundation global vaccine programs are used to inject tracking microchips into people.

These conspiracy theories have, since COVID-19, connected formerly separate communities and discourses, uniting existing anti-vaxxer communities, older demographics who are mistrustful of technology, far-right communities suspicious of global and national left-wing agendas, communities protesting against 5G mobile networks (for fear that they will brainwash, control, or harm people), as well as generating its own followers out of those anxious during the 2020 onset of the COVID-19 pandemic.
We detect and describe some of these opinion dynamics in the next section.


\section{Implementation Details}
\label{supp:implementation}
\section{Implementation: Ring Abstraction}
\label{sec:implement}
\subsection{Distributed \mbox{$G_t$} in QMC Solver}
\label{distributedG4}
Before introducing the communication phase of the ring abstraction layer,
it is important to understand how the authors distributed the large device array $G_t$ across MPI ranks.
%
Original $G_t$ was compared, and $G^d_t$ versions were distributed
(Figure~\ref{fig:compare_original_distributed_g4}). 


In the original $G_t$ implementation, the measurements---which were computed by matrix-matrix multiplication---are distributed statically and independently over the MPI ranks to avoid
inter-node communications. Each MPI rank keeps its partial copy of $G_{t,i}$ to accumulate 
measurements within a rank, where $i$ is the rank index. 
After all the measurements are finished, a reduction step is 
taken to accumulate $G_{t,i}$ across all MPI ranks into a final and complete
$G_t$ in the root MPI rank. The size of the $G_{t,i}$ in each rank is 
the same size as the final and complete $G_t$. 

With the distributed $G^d_t$ implementation, this large device array 
$G_t$ was evenly partitioned across all MPI ranks; each portion of it is local to each MPI rank.
%
Instead of keeping its partial copy of $G_t$, 
each rank now keeps an instance of $G^d_{t,i}$ to accumulate measurements 
of a portion or sub-slice of the final and complete $G_t$, where the notation
$d$ in $G^d_t$  refers to the distributed version, and $i$ means the $i$-th rank.
%
The $G^d_{t,i}$ size in each rank is 
reduced to $1/p$ of the size of the final and complete $G_t$, comparing the same configuration 
in original $G_t$ implementation, where $p$ is the number of MPI ranks used. 
%
For example, in Figure~\ref{fig:distributed_g4}, there are four ranks, and rank $i$
now only keeps $G^d_{t,i}$, which is one-fourth the size of the original $G_t$ array size.
% and contains values indexing from range of $[0, ..., N/4)$ in original $G_t$ array where $N$ is the 
% number of entries in  $G_t$  when viewed as a one-dimensional array.

To compute the final and complete $G^d_{t,i}$ for the distributed $G^d_t$ implementation, 
each rank must see every $G_{\sigma,i}$ from all ranks. 
%
In other words, each rank must broadcast the
locally generated $G_{\sigma,i}$ to the remainder of the other ranks at every measurement step. 
%
To efficiently perform this ``all-to-all'' broadcast, a ring abstraction layer was built (Section. \ref{section:ring_algorithm}), which circulates
all $G_{\sigma,i}$ across all ranks.

% over high-speed GPUs interconnect (GPUDirect RDMA) to facilitate the communication phase.

% \begin{figure}
% \centering
% \subfloat[Original $G_t$ implementation.]
%     {\includegraphics[width=\columnwidth]{original_g4.pdf}}\label{fig:original_g4}

% \subfloat[Distributed $G_t$ implementation.]
%     {\includegraphics[width=0.99\columnwidth]{distributed_g4.pdf} \label{fig:distributed_g4}}

% \caption{Comparison of the original $G_t$ vs. the distributed $G^d_t$ implementation. Each rank contains one GPU resource.}
% \label{fig:compare_original_distributed_g4} 
% \end{figure} 

\begin{figure}
\centering
     \begin{subfigure}[b]{\columnwidth}
         \centering
         \includegraphics[width=\textwidth]{images/original_g4.pdf}
         \caption{Original $G_t$ implementation.}
         \label{fig:original_g4}
     \end{subfigure}
     
    \begin{subfigure}[b]{\columnwidth}
         \centering
         \includegraphics[width=\textwidth]{images/distributed_g4.pdf}
         \caption{Distributed $G_t$ implementation.}
         \label{fig:distributed_g4}
     \end{subfigure}
     
\caption{Comparison of the original $G_t$ vs. the distributed $G^d_t$ implementation. Each rank contains one GPU resource.}
\label{fig:compare_original_distributed_g4}
\end{figure}

\subsection{Pipeline Ring Algorithm}
\label{section:ring_algorithm}
A pipeline ring algorithm was implemented that broadcasts the $G_{\sigma}$ 
array circularly during every measurement. 
%
The algorithm (Algorithm \ref{alg:ring_algorithm_code}) is 
visualized in Figure~\ref{fig:ring_algorithm_figure}.

\begin{algorithm}
\SetAlgoLined
    generateGSigma(gSigmaBuf)\; \label{lst:line:generateG2}
    updateG4(gSigmaBuf)\;       \label{lst:line:updateG4}
    %\texttt{\\}
    {$i\leftarrow 0$}\;         \label{lst:line:initStart}
    {$myRank \leftarrow worldRank$}\;          \label{lst:line:initRankId}
    {$ringSize \leftarrow mpiWorldSize$}\;      \label{lst:line:initRingSize}
    {$leftRank \leftarrow (myRank - 1 + ringSize) \: \% \: ringSize $}\;
    {$rightRank \leftarrow (myRank + 1 + ringSize) \: \% \: ringSize $}\;
    sendBuf.swap(gSigmaBuf)\;           \label{lst:line:initEnd}
    \While{$i< ringSize$}{
        MPI\_Irecv(recvBuf, source=leftRank, tag = recvTag, recvRequest)\; \label{lst:line:Irecv}
        MPI\_Isend(sendBuf, source=rightRank, tag = sendTag, sendRequest)\; \label{lst:line:Isend}
        MPI\_Wait(recvRequest)\;        \label{lst:line:recevBuffWait}
        
        updateG4(recvBuf)\;             \label{lst:line:updateG4_loop}
        
        MPI\_Wait(sendRequest)\;        \label{lst:line:sendBuffWait}
        
        sendBuf.swap(recvBuf)\;         \label{lst:line:swapBuff}
        i++\;
    }
\caption{Pipeline ring algorithm}
\label{alg:ring_algorithm_code}
\end{algorithm}

\begin{figure}
	\centering
	\includegraphics[width=\columnwidth, trim=0 5cm 0 0, clip]{images/ring_algorithm.pdf}
	\caption{Workflow of ring algorithm per iteration. }
	\label{fig:ring_algorithm_figure}
\end{figure}

At the start of every new measurement, a single-particle Green's function $G_{\sigma}$
 (Line~\ref{lst:line:generateG2}) is generated 
and then used to update $G^d_{t,i}$ (Line~\ref{lst:line:updateG4})
via the formula in Eq.~(\ref{eq:G4}).
%
% Different from original method that performs 
% full matrix-matrix multiplication (Equation~(\ref{eq:G4})), the current ring algorithm only performs partial
% matrix-matrix multiplication that contributes to $G^d_{t,i}$, a subslice of the final and complete $G_t$.
%
Between Lines \ref{lst:line:initStart} to \ref{lst:line:initEnd}, the algorithm 
initializes the indices
of left and right neighbors and prepares the sending message buffer from the
previously generated $G_{\sigma}$ buffer. 
%
The processes are organized as a ring so that the first and last rank are considered to be neighbors to each other. 
%
A \textit{swap} operation is used to avoid unnecessary memory copies for \textit{sendBuf} preparation.
%
A walker-accumulator thread allocates an additional \textit{recvBuf} buffer of the same size 
as \textit{gSigmaBuf} to hold incoming 
\textit{gSigmaBuf} buffer from \textit{leftRank}. 

The \textit{while} loop is the core part of the pipeline ring algorithm. 
%
For every iteration, each thread in a rank 
receives a $G_{\sigma}$ buffer from its left neighbor rank and sends a $G_{\sigma}$ buffer to its right neighbor rank. 
A synchronization step (Line~\ref{lst:line:recevBuffWait}) is performed
afterward to ensure that each rank receives a new buffer to update the 
local $G^d_{t,i}$ (Line~\ref{lst:line:updateG4_loop}). 
%
Another synchronization step
follows to ensure that all send requests are finalized 
(Line~\ref{lst:line:sendBuffWait}). Lastly, another \textit{swap} operation is used to exchange
content pointers between \textit{sendBuf} and \textit{recvBuf} to avoid unnecessary memory copy and prepare
for the next iteration of communication.
%
In the multi-threaded version (Section~\ref{subsec:multi-thread}), the thread of index, \textit{i}, only communicates with
	threads of index, \textit{i}, in neighbor ranks, and each thread allocates two buffers: \textit{sendBuff} and \textit{recvBuff}.

The \textit{while} loop will be terminated after $\mbox{\textit{ringSize}} - 1$ steps. By that time, 
each locally generated $G_{\sigma,i}$ will have traveled across all MPI ranks and
updated $G^d_{t,i}$ in all ranks. Eventually, each $G_{\sigma,i}$ reaches
to the left neighbor of its birth rank. For example, $G_{\sigma,0}$ generated from rank $0$ will end 
in last rank in the ring communicator.

Additionally, if the $G_t$ is too large to be stored in one node, 
it is optional to accumulate all $G^d_{t,i}$
at the end of all measurements. 
%
Instead, a parallel write into the file system could be taken.

\subsubsection{Sub-Ring Optimization.}

A sub-ring optimization strategy is further proposed to reduce message communication
times if the large device array $G_t$ can fit in fewer devices. 
%
The sub-ring algorithm is visualized in Figure~\ref{fig:subring_algorithm_figure}.

For the ring algorithm (Section~\ref{section:ring_algorithm}), the size of the ring communicator
(\textit{mpiWorldSize}) is set to the same size of the global \mbox{\texttt{MPI\_COMM\_WORLD}}, and thus the size of $G_t$ is equally 
distributed across all MPI ranks.

However, to complete the update to $G^d_{t,i}$ in one measurement, 
one $G_{\sigma,i}$
must travel \textit{mpiWorldSize} ranks. In total, 
there are \textit{mpiWorldSize} numbers of $G_{\sigma,i}$
being sent and received concurrently in one measurement 
in the global
\mbox{\texttt{MPI\_COMM\_WORLD}} 
communicator. If the size of $G^d_{t,i}$ is relatively small per rank, then this will cause high communication overhead.

If $G_t$ can be distributed and fitted in fewer devices, then a shorter travel distance is required 
for $G_{\sigma,i}$, thus reducing the communication overhead. One reduction
step was performed at the end of all measurements to accumulate $G^d_{t,s_i}$, 
where $s_i$ means $i$-th rank on the $s$-th sub-ring.

At the beginning of MPI initialization, the global \mbox{\texttt{MPI\_COMM\_WORLD}} was partitioned  into several new sub-ring communicators by using \mbox{\texttt{MPI\_Comm\_split}}. 
% where each new communicator represents conceptually a subring. 
The new
communicator information was passed to the DCA++ concurrency class by substituting the original global 
\mbox{\texttt{MPI\_COMM\_WORLD}} with this new communicator. Now, only a few minor modifications
are needed to transform the ring algorithm (Algorithm~\ref{alg:ring_algorithm_code})
to sub-ring Algorithm~\ref{alg:sub_ring_algorithm}. In Line~\ref{lst:line:initRankId}, \textit{myRank} is 
initialized to \textit{subRingRank} instead of \textit{worldRank}, where 
\textit{subRingRank} is the rank index in the local sub-ring communicator. 
%
In Line~\ref{lst:line:initRingSize}, \textit{ringSize} is initialized to \textit{subRingSize}
instead of \textit{mpiWorldSize}, where \textit{subRingSize} is the
size of the new communicator.
%
The general ring algorithm is a special case for the sub-ring algorithm because the
\textit{subRingSize} of the general ring algorithm is equal to \textit{mpiWorldSize}, and
there is only one sub-ring group throughout all MPI ranks.


\LinesNumberedHidden
\begin{algorithm}
    {$\mbox{\textit{myRank}} \leftarrow \mbox{\textit{subRingRank}}$}\;         
    {$\mbox{\textit{ringSize}} \leftarrow \mbox{\textit{subRingSize}}$}\;      
\caption{Modified ring algorithm to support sub-ring communication}
\label{alg:sub_ring_algorithm}
\end{algorithm}


\begin{figure}
	\centering
	\includegraphics[width=\columnwidth, trim=0 5cm 0 0, clip]{images/subring_alg.pdf}
	\caption{Workflow of sub-ring algorithm per iteration. Every consecutive $S$ rank forms a sub-ring communicator, 
	and no communication occurs between sub-ring communicators until all measurements are finished. Here, $S$ is the number of ranks in a sub-ring.}
	\label{fig:subring_algorithm_figure}
\end{figure}

\subsubsection{Multi-Threaded Ring Communication.}
\label{subsec:multi-thread}
To take advantage of the multi-threaded QMC model already in DCA++, 
multi-threaded ring communication support was further implemented in the ring algorithm.
%
Figure~\ref{fig:dca_overview} shows that in the original DCA++ method,
each walker-accumulator
thread in a rank is independent of each other, and all the threads in a 
rank synchronize only after all rank-local measurements are finished.
%
Moreover, during every measurement, each walker-accumulator thread
generates its own thread-private $G_{\sigma, i}$ to update $G_t$. 
%

The multi-threaded ring algorithm now allows concurrent message exchange so that threads of same rank-local thread index exchange their thread-private $G_{\sigma, i}$. 
%
Conceptually, there are $k$ parallel and independent rings, where $k$ 
is number of threads per rank, because threads of the same local thread ID
form a closed ring. 
%
For example, a thread of index $0$ in rank $0$ will send its $G_\sigma$ to 
the thread of index $0$ in rank $1$ and receive another $G_\sigma$ from thread index of $0$ 
from last rank in the ring algorithm.
%

The only changes in the ring algorithm are offsetting the tag values 
(\texttt{recvTag} and \texttt{sendTag}) by the thread index value. For example,
Lines~\ref{lst:line:Irecv} and ~\ref{lst:line:Isend} from 
Algorithm~\ref{alg:ring_algorithm_code} are modified to Algorithm~\ref{alg:multi_threaded_ring}.

\LinesNumberedHidden
\begin{algorithm}
        MPI\_Irecv(recvBuf, source=leftRank, tag = recvTag + threadId, recvRequest)\; 
        MPI\_Isend(sendBuf, source=rightRank, tag = sendTag + threadId, sendRequest)\;
\caption{Modified ring algorithm to support multi-threaded ring}
\label{alg:multi_threaded_ring}
\end{algorithm}

To efficiently send and receive $G_\sigma$, each thread
will allocate one additional \textit{recvBuff} to hold incoming 
\textit{gSigmaBuf} buffer from \textit{leftRank} and perform send/receive efficiently.
%
In the original DCA++ method, there are $k$ numbers of buffers of $G_\sigma$ 
size per rank, and in the multi-threaded ring method, there are $2k$
numbers of buffers of $G_\sigma$ size per rank, where $k$ is number of 
threads per rank.


\section{Environment}
\label{supp:environment}
For the experiments on data from Manhattan, New York, we relied on sub-areas of a larger StreetView graph that contains 256961 nodes and 266040 edges. We defined 5 areas by selecting a starting point at a given coordinate and collecting panoramas in a panorama adjacency graph using breadth-first-search, until a given depth of the search tree. We defined areas as following:

\begin{itemize}
    \item Wall Street / Lower Manhattan: 6917 nodes and 7191 edges, 200-deep search tree starting at (40.705510, -74.013589).
    \item NYU / Greenwich Village: 17227 nodes and 17987 edges, 200-deep search tree starting at (40.731342, -73.996903).
    \item Midtown: 16185 nodes and 16723 edges, 200-deep search tree starting at (40.756889, -73.986147).
    \item Central Park: 10557 nodes and 10896 edges, 200-deep search tree starting at (40.773863, -73.971984).
    \item Harlem: 14589 nodes and 15099 edges, 220-deep search tree starting at (40.806379, -73.950124).
\end{itemize}

The Central London StreetView environment contains 24428 nodes and 25352 edges, and is defined by a bounding box between the following Lat/Long coordinates: (51.500567, -0.139157) and (51.526175, -0.080043). The Paris Rive Gauche environment contains 34026 nodes and 35475 edges, and is defined by a bounding box between Lat/Long coordinates: (48.839413, 2.2829247) and (48.866578, 2.3653221).

We provide, in a text file\footnote{Available at \url{http://streetlearn.cc}}, the locations of the 644 landmarks used throughout the study.


\end{document}
