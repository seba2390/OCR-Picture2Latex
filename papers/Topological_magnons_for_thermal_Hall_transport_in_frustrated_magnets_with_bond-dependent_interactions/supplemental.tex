\documentclass[reprint,amsmath,amssymb,aps,prb,superscriptaddress,longbibliography]{revtex4-1}
\usepackage{array}
\usepackage{bm}
\usepackage{color}
\usepackage{float}
\usepackage{graphicx}
\usepackage{hyperref}
\hypersetup{colorlinks=true,allcolors=blue}
\usepackage{natbib}
\usepackage{newtxtext}
\usepackage{newtxmath}

\usepackage[section]{placeins}
\usepackage[caption=false]{subfig}



%%%%%%%%%%%%%%%%%%%%%%%%%%%%%% User specified LaTeX commands.
\usepackage{braket}

\renewcommand\[{\begin{equation}}
\renewcommand\]{\end{equation}}

\makeatother

\usepackage{babel}
\usepackage{xcolor}
\newcommand\myworries[1]{\textcolor{red}{#1}}


%%%%%%%%%%%%%%%%%%%%%%%%%%%%%% LyX specific LaTeX commands.
%% Because html converters don't know tabularnewline
\providecommand{\tabularnewline}{\\}

\newlength{\lyxlabelwidth}      % auxiliary length 

\providecommand{\comment}[1]{\textbf{[#1]}}

\usepackage{hyperref}
\makeatother


\begin{document}

\title{Supplemental Materials}

\author{Emily Zhang}
\affiliation{Department of Physics, University of Toronto, Toronto, Ontario M5S 1A7, Canada}

\author{Li Ern Chern}
\affiliation{Department of Physics, University of Toronto, Toronto, Ontario M5S 1A7, Canada}

\author{Yong Baek Kim}
\affiliation{Department of Physics, University of Toronto, Toronto, Ontario M5S 1A7, Canada}

\maketitle

\setcounter{equation}{0}
\setcounter{figure}{0}
\setcounter{table}{0}
%\setcounter{page}{1}

\renewcommand{\thesection}{S\arabic{section}}
\renewcommand{\theequation}{S\arabic{equation}}
\renewcommand{\thefigure}{S\arabic{figure}}
\renewcommand{\thetable}{S\arabic{table}}


\onecolumngrid

\section{Examples of the Magnetic Orders}

We provide the spin configurations for each phase observed on the phase diagram in the main text in Figs. \ref{fig:zza}-\ref{fig:10c}. Each spin configuration was obtained using classical simulated annealing. The examples shown in each figure used external fields under an in-plane field $\mathbf{h}=h\cos\theta\left[11\bar{2}\right]+h\sin\theta\left[\bar{1}10\right]$. 

\begin{figure*}
\begin{centering}
\includegraphics[width=0.6\columnwidth]{zza}
\par\end{centering}
\caption{(a) The sublattice structure for the ZZa unit cell with the lattice
vectors shown in black. (b) The spin configuration of the ZZa order
shown on the honeycomb lattice. The parameters used were $K=-1,$
$\Gamma=0.5$, $\Gamma'=-0.02$, $h/S|K|=0.15$, and $\theta=150^{\circ}$.
\label{fig:zza}}

\end{figure*}

\begin{figure*}
\begin{centering}
\includegraphics[width=0.6\columnwidth]{zzb}
\par\end{centering}
\caption{(a) The sublattice structure for the ZZb unit cell with the lattice
vectors shown in black. (b) The spin configuration of the ZZb order
shown on the honeycomb lattice. The parameters used were $K=-1,$
$\Gamma=0.5$, $\Gamma'=-0.02$, $h/S|K|=0.15$, and $\theta=30^{\circ}$.
\label{fig:zzb}}
\end{figure*}

\begin{figure*}
\begin{centering}
\includegraphics[width=0.6\columnwidth]{zzc}
\par\end{centering}
\caption{(a) The sublattice structure for the ZZc unit cell with the lattice
vectors shown in black. (b) The spin configuration of the ZZc order
shown on the honeycomb lattice. The parameters used were $K=-1,$
$\Gamma=0.5$, $\Gamma'=-0.02$, $h/S|K|=0.15$, and $\theta=90^{\circ}$.
\label{fig:zzc}}
\end{figure*}

\begin{figure*}
\begin{centering}
\includegraphics[width=0.6\columnwidth]{6a}
\par\end{centering}
\caption{(a) The sublattice structure for the 6a unit cell with the lattice
vectors shown in black. (b) The spin configuration of the 6a order
shown on the honeycomb lattice. The parameters used were $K=-1,$
$\Gamma=0.5$, $\Gamma'=-0.02$, $h/S|K|=0.15$, and $\theta=0^{\circ}$.
\label{fig:6a}}
\end{figure*}

\begin{figure*}
\begin{centering}
\includegraphics[width=0.6\columnwidth]{6b}
\par\end{centering}
\caption{(a) The sublattice structure for the 6a unit cell with the lattice
vectors shown in black. (b) The spin configuration of the 6a order
shown on the honeycomb lattice. The parameters used are the same as
the 6a configuration, since the 6a and 6b are degenerate at $\theta=0^{\circ}$.
\label{fig:6b}}
\end{figure*}

\begin{figure*}
\begin{centering}
\includegraphics[width=0.6\columnwidth]{6c}
\par\end{centering}
\caption{(a) The sublattice structure for the 6c unit cell with the lattice
vectors shown in black. (b) The spin configuration of the 6c order
shown on the honeycomb lattice. The parameters used were $K=-1,$
$\Gamma=0.5$, $\Gamma'=-0.02$, $h/S|K|=0.05$, and $\theta=0^{\circ}$.
\label{fig:6c}}
\end{figure*}

\begin{figure*}
\begin{centering}
\includegraphics[width=0.6\columnwidth]{10a}
\par\end{centering}
\caption{(a) The sublattice structure for the 10a unit cell with the lattice
vectors shown in black. (b) The spin configuration of the 10a order
shown on the honeycomb lattice. The parameters used were $K=-1,$
$\Gamma=0.5$, $\Gamma'=-0.02$, $h/S|K|=0.25$, and $\theta=150^{\circ}$.
\label{fig:10a}}
\end{figure*}

\begin{figure*}
\begin{centering}
\includegraphics[width=0.6\columnwidth]{10b}
\par\end{centering}
\caption{(a) The sublattice structure for the 10b unit cell with the lattice
vectors shown in black. (b) The spin configuration of the 10b order
shown on the honeycomb lattice. The parameters used were $K=-1,$
$\Gamma=0.5$, $\Gamma'=-0.02$, $h/S|K|=0.25$, and $\theta=30^{\circ}$.
\label{fig:10b}}
\end{figure*}

\begin{figure*}
\begin{centering}
\includegraphics[width=0.6\columnwidth]{10c}
\par\end{centering}
\caption{(a) The sublattice structure for the 10c unit cell with the lattice
vectors shown in black. (b) The spin configuration of the 10c order
shown on the honeycomb lattice. The parameters used were $K=-1,$
$\Gamma=0.5$, $\Gamma'=-0.02$, $h/S|K|=0.25$, and $\theta=90^{\circ}$.
\label{fig:10c}}
\end{figure*}

\end{document}