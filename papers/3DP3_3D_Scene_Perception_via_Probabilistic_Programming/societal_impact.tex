\section{Broader Impact}

While the goal of robust scene parsing and pose estimation is challenging, and
the present work is an early step with much more work lying ahead, it is
important to consider potential societal impacts of this work, both positive
and negative.  Robust pose estimation will be instrumental in improving the
reliability of a wide variety of applications---including assistive
technologies for people with limited mobility, improved fault detection in
manufacturing plants, and safer autopilot for autonomous vehicles.  On the
other hand, these same technologies, if used toward the wrong ends, could have
negative societal impacts as well, such as unjust or inequitable surveillance,
or weapon guidance systems that fall into the wrong hands.  Even applications
that are largely beneficial must be implemented thoughtfully to avoid negative
side effects.  For example, in the present work, the choice of prior
distribution on contact structures implies an inductive bias that, if chosen
incorrectly, could lead to technologies that are less reliable when the scene
being parsed contains a person in a wheelchair.  As a scientific community, it
is important that we place continued emphasis on developing technical
safeguards against both overt misuse and unintended consequences like the
above.  Furthermore, we must remember that technical safeguards on their own
are not sufficient: we must communicate to broader society not just the
benefits, but also the risks of this technology, so that users can be informed
participants and apply this technology towards a better world.

