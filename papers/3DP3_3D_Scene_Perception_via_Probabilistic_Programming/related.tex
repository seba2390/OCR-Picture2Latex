\section{Related Work}


\textbf{Analysis-by-synthesis approaches to computer vision}
A long line of work has interpreted computer vision as the inverse problem to computer graphics \citep{knill1996introduction, yuille2006vision, lee2003hierarchical, kersten2004object}.
This `analysis-by-synthesis' approach has been used for various tasks including character recognition, CAPTCHA-breaking, lane detection, object pose estimation, and human pose estimation
\citep{zhu1996region, tu2002image, mansinghka2013approximate, moreno2016overcoming, george2017generative, romaszko2017vision}.
To our knowledge, our work is the first to use an analysis-by-synthesis approach to infer a hierarchical 3D object-based representation of real multi-object scenes while exploiting inductive biases about the contacts between objects.

\textbf{Hierarchical latent 3D scene representations} We use a scene
graph representation~\cite{zinberg2019sg} that is closely related to
hierarchical scene graph representations in computer
graphics~\citep{clark1976}.  Unlike in graphics, we address the
\emph{inverse problem} of inferring hierarchical scene graphs from
observed image data.  Inferring hierarchical 3D scene graphs from RGB
or depth images in a probabilistic framework is relatively unexplored.
One concurrent\footnote{An early version of our
  work~\citep{zinberg2019sg} is concurrent with an early version of
  GSGN~\citep{deng2019rich} } and independent work, Generative Scene
Graph Networks (GSGN~\citep{deng2020generative}), proposes a
variational autoencoder architecture for decomposing images into
objects and parts using a tree-structured latent scene graph that is
similar to our scene graph representation.  However, GSGN learns RGB
appearance models of objects and their parts, uses an inference
network instead of a hybrid of data-driven and model-based inference,
was not evaluated on real images or scenes, and uses more restricted
scene graphs that cannot represent objects with independent 6DoF pose.
GSGN builds on an earlier deep generative
model~\citep{eslami2016attend} that generates multi-object scenes but
does not model dependencies between object poses and was not
quantitatively evaluated on real 3D scenes.  Incorporating a learned
inference network for jointly proposing scene graphs into our
framework is an interesting area for future work.  The term `scene
graph' has also been used in computer vision to refer to various less
related graph representations of
scenes~\citep{armeni20193d,chen2019holistic++,rosinol20203d}.



\textbf{Probabilistic programming for computer vision} Prior work has
used probabilistic programs to represent generative models of images
and implemented inference in these models using probabilistic
programming systems~\citep{mansinghka2013approximate,
  kulkarni2015picture}.  Unlike these prior works, which relied on
manually specified and/or semi-parametric shape models, 3DP3 learns
object shapes non-parametrically. 3DP3 also models occlusion of one 3D
object by another; uses a novel hierarchical scene graph prior that
allows for dependencies between object poses in the prior; uses a
novel involutive MCMC~\citep{cusumano2020automating} kernel for
inferring scene graph structure; and uses a novel pseudo-marginal
approach for handling uncertainty about object shape during
inference. We also present a proof of concept that our system can
infer the presence and pose of fully occluded objects.

\textbf{6DoF object pose estimation}
We use 6DoF estimation of object pose from RGBD images as an example application.
Registration of point clouds~\citep{besl1992method} can be used to estimate the 6DoF pose of objects
with known 3D geometry from depth images.
Many recent 6DoF object pose estimators use deep learning \citep{xiang2017posecnn, tremblay2018deep}
and many also take depth images~\citep{wang2019densefusion, tian2020robust}.
Some pose estimation methods model scene structure, contact relationships, stability, or other semantic information \citep{huang2018holistic, chen2019holistic++, kulkarni20193d, du2018stability, bao2011semantic}, and
some use probabilistic inference \citep{desingh2019efficient, chen2019grip, glover2012monte,deng2021poserbpf}.
To our knowledge, we present the first 6DoF pose estimator that uses
Bayesian inference about the structure of hierarchical 3D scene graphs.

\textbf{Learning models of novel 3D objects} Classic algorithms for structure-from-motion infer a 3D model of a scene from multiple images \citep{schonberger2016structure,agarwal2010bundle}.
Our approach for learning the shape of novel 3D objects produces coarse-grained probabilistic voxel models of objects
that can represent uncertainty about the occupancy of self-occluded volumes.
Integrating other representations of object shape and object appearance~\citep{mildenhall2020nerf} with our scene graph representation is a promising area of future work.
