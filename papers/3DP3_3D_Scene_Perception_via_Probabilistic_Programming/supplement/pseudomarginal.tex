\section{Pseudomarginal shape inference}
\label{sec:pseudomarginal}

We use MH and involutive MCMC kernels that are stationary with respect to a target distribution on an extended state space that includes auxiliary variables $\mathbf{O}^{(i)}$ for $i = 1, \ldots, R$ ($R$ copies of all object shape models) by replacing the likelihood $p(\mathbf{Y} | N, \mathbf{c}, G', \bm{\theta}')$ for each proposed state in the acceptance probability with the following unbiased estimate obtained by sampling object models from the prior:
\begin{equation}
\frac{1}{R} \sum_{i=1}^R p(\mathbf{Y} | \mathbf{O}_1^{(i)}, \ldots, \mathbf{O}_M^{(i)}, N, \mathbf{c}, G, \bm\theta) \mbox{ where } \mathbf{O}_c^{(i)} \stackrel{\mathrm{i.i.d.}}{\sim} p(\cdot)
\end{equation}
and replacing the likelihood in the denominator of the acceptance ratios with the unbiased estimate computed by the last accepted proposal.
This is an instance of the pseudomarginal MCMC~\citep{andrieu2009pseudo} framework.
For example, for the scene graph involutive MCMC kernel described in Section~\ref{sec:involutive_move}, each factor of the form:
\begin{equation}
\frac{p(\mathbf{Y} | N, \mathbf{c}, G', \bm\theta')}{p(\mathbf{Y} | N, \mathbf{c}, G, \bm\theta)}
\end{equation}
is replaced with a factor:
\begin{equation}
\frac{
\frac{1}{R} \sum_{i=1}^R p(\mathbf{Y} | {\mathbf{O}_1^{(i)}}', \ldots, {\mathbf{O}_M^{(i)}}', N, \mathbf{c}, \mathcal{G})
}{
\frac{1}{R} \sum_{i=1}^R p(\mathbf{Y} | \mathbf{O}_1^{(i)}, \ldots, \mathbf{O}_M^{(i)}, N, \mathbf{c}, \mathcal{G})
}
\end{equation}
where the $\mathbf{O}_j^{(i)}$ are the shape models that were sampled when proposing the last state that was accepted, and where the ${\mathbf{O}_j^{(i)}}'$ are the shape models that are sampled during the current proposal.
Note that the old sampled shape models $\mathbf{O}_j^{(i)}$ themselves do not need to be persisted across steps of the Markov chain---only the denominator in the expression above needs to be stored.
The resulting moves can be interpreted as MH (or involutive MCMC) moves on an extended state space
that includes additional auxiliary random variables $\mathbf{O}_{1:M}^{(1)}, \ldots, \mathbf{O}_{1:M}^{(R)}$.
The moves are stationary with respect to the following target distribution on the extended state space:
\begin{equation}
p(G, \bm\theta | N, \mathbf{c}, \mathbf{Y}) \frac{1}{R} \sum_{i=1}^R p(\mathbf{O}_{1:M}^{(i)} | N, \mathbf{c}, G, \bm\theta, \mathbf{Y}) \prod_{j \ne i} p(\mathbf{O}_{1:M}^{(j)})
\end{equation}
of which the marginal on $(G, \bm\theta)$ is the original target distribution $p(G, \bm\theta | N, \mathbf{c}, \mathbf{Y})$.
