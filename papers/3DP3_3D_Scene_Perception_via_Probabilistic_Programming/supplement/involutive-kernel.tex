\section{Involutive MCMC kernel on scene graph structure and parameters} \label{sec:involutive_move}

This section gives details of the involutive MCMC kernel on scene graphs introduced in Section 5.

\subsection{Notation for coordinate projections}\label{sec:proj-notation}
In several places below, we define sets of tuples using set-builder notation such as
\[ X := \{(x, y, z)\ |\ \text{some condition on $x, y, z$}\}. \]
In such a case, we may also define coordinate projections that get their names from the formal variables (``$x$,'' ``$y$,'' ``$z$'') used in the set-builder expression.
Our convention is to denote these coordinate projections by the name $\proj_\bullet$, where $\bullet$ is either a variable name, e.g.,
\[ \proj_x(x, y, z) := x \]
or a comma-separated list of variable names, e.g.,
\[ \proj_{y,z}(x, y, z) := (y, z). \]
This definition depends not just on the set $X$, but on the notation used to define it; thus, in the exposition below, we explicitly declare each time we define a function $\proj_\bullet$ using the above convention.
Note that the name $\proj_x$ is to be taken as a single unit, i.e., $x$ does not have independent meaning; in particular, if there is also a variable $x$ in the scope of discourse, the name $\proj_x$ does not have anything to do with that variable's value.

\subsection{The number of scene graph structures on a fixed set of objects}

\begin{proposition}
For a given set of $N$ objects, the number of possible scene graph structures is $(N+1)^{N-1}$.  (Here the root node $r$ is not considered an object.)
\end{proposition}
\begin{proof}
Let $V'$ be a set of $N$ objects, and let $V := V' \cup \{r\}$.  For a given {\em undirected} tree $\widetilde{G}$ on vertices $V$, there is a unique way to assign edge directions to $\widetilde{G}$ to turn it into a directed tree rooted at $r$.
This gives a one-to-one correspondence
\[
    \text{directed trees on $V$ rooted at $r$} \quad\longleftrightarrow\quad \text{undirected trees on $V$}.
\]
By Cayley~\citep{cayley_2009}, the number of undirected trees on $V$ is $(N+1)^{N-1}$.
\end{proof}

\subsection{An involutive MCMC kernel on scene graph structure only}

For some set $V$ of vertices and a root vertex $r \in V$, let $\mathbf{G}(V)$ denote the set of directed trees over vertices $V$ rooted at $r$.
For each $G = (V, E) \in \mathbf{G}(V)$ and each $v \in V \setminus \{r\}$, let $S(G, v) \subset V$ denote the vertices of the subtree rooted at $v$, i.e., the set containing $v$ and its descendants.
Let
\begin{equation}
T(G) := \left\{
    (v, u) \subset V \times V : v \neq r, u \notin S(G, v)
\right\}.
\end{equation}
That is, $T(G)$ contains a every pair of vertices $(v, u)$ such that $v$ is not the root note, and $u$ is not a descendant of $v$.
(Intuitively, we can think of $T(G)$ as the set of pairs of vertices $(v, u)$ such that it is possible to sever the subtree rooted at $v$ and re-attach that subtree as a child of $u$.)
Next, let
\begin{equation}
U(V) := \left\{
    (G, v, u) : G \in \mathbf{G}(V), (v, u) \in T(G)
\right\}
\end{equation}
and equip $U(V)$ with coordinate projections $\proj_G$, $\proj_v$, etc.{} as in Section~\ref{sec:proj-notation}.
(Intuitively, we can think of the triples $(G, v, u) \in U(V)$ as denoting a graph $G$, a choice of vertex $v$ at which to sever, and a choice of vertex $u$ at which to graft.)
Finally, define the function
\[ g: U(V) \to U(V) \]
by $g(G, v, u) = (G', v, u')$, where
(i) $u'$ is the parent of $v$ in $G$, and
(ii) $G'$ is the graph obtained from $G$ by removing the edge $(u', v)$ and adding the edge $(u, v)$.
(By Lemma~\ref{lem:tree_add_remove}, $G'$ is a tree, so $(G', v, u') \in U(V)$.)

\begin{proposition}
The function $g: U(V) \to U(V)$ is an involution.
\end{proposition}
\begin{proof}
Let $(G', v', u') := g(G, v, u)$; let $(G'', v'', u'') := g(G', v', u')$; and let $E$, $E'$ and $E''$ be the edge sets of $G$, $G'$ and $G''$ respectively.
Then, by the definition of $g$, we have $v'' = v' = v$.
Also, because $G'$ is a tree that contains the edge $(u, v)$, it follows that $u$ is the parent of $v$ in $G'$.
Thus $u'' = u$, since $u''$ is also (by definition) the parent of $v$ in $G'$.
Next, by the definition of $g$, we have $E' = (E \setminus \{(u', v)\}) \cup \{(u, v)\}$ and
\[
    E'' = (E' \setminus \{(u, v)\}) \cup \{(u, v)\} = (E \setminus \{(u', v)\}) \cup \{(u', v)\} = E
\]
(here we are using the fact that $(u', v) \in E$, which holds by the definition of $g$).
Thus $G'' = G$, so $g(g(G, v, u)) = (G, v, u)$.
\end{proof}


\begin{proposition}
Let $G,G' \in \mathbf{G}(V)$.
Then there exists a sequence of triples
\[ (G_0, v_0, u_0), \ldots, (G_k, v_k, u_k) \]
satisfying all of the following:
\begin{enumerate}[nosep,label=(\roman*)]
    \item\label{itm:g-iter-first-condition} $(G_i, v_i, u_i) \in U(V)$ for all\footnote{
            This proposition doesn't say anything about $u_k$ and $v_k$; we leave them in just to simplify the exposition and notation.
        } $i = 0,\ldots,k-1$
    \item $G_0 = G$
    \item $G_k = G'$
    \item\label{itm:g-iter-recursion} for each $i < k$ we have $G_{i+1} = \proj_G(g(G_i, v_i, u_i))$.
\end{enumerate}
\label{prop:g-iter-is-irreducible}
\end{proposition}
\begin{proof}
Let $G^\circ$ denote the scene structure that has no relations between objects; that is,
\[ G^\circ := (V, \{(r, v) : v \in V \setminus \{r\}\}). \]
We first prove the result in the case where $G = G^\circ$; then we extend to the general case.

For the case where $G = G^\circ$, the intuition is to work backwards from $G'$ to $G^\circ$ by grafting subtrees onto $r$ until there are no non-singleton subtrees left.
Let $G' = (V, E')$, and let $\widetilde{E} := \{(u,v) \in E : v \neq r\}$.
We then take\footnote{
    Except in the degenerate case where $\widetilde{E}$ is empty.
    In that case we have $G' = G^\circ$, so we simply take $k := 0$, $G_0 := G^\circ$, and $u_0$ and $v_0$ to be any vertices.
} $k := |\widetilde{E}|-1$.
We define $v_i$ (from the proposition statement) and $u'_i$ (a variable we're now introducing) by arbitrarily choosing an ordered enumeration of $\widetilde{E}$ and defining the sequence of pairs $(u'_0, v_0), \ldots, (u'_k, v_k)$ to equal that enumeration.
Thus,
\[ \widetilde{E} = \{(u'_0, v_0), \ldots, (u'_k, v_k)\}. \]
Next, we define $G_i$ and $u_i$ simultaneously by backward recursion: we take $G_k := G'$ (and choose $u_k$ arbitrarily; the proposition doesn't actually say anything about $u_k$); and for $0 \leq i < k$, we define $G_i$ and $u_i$ by the equation
\begin{equation}
g(G_{i+1}, v_i, r) = (G_i, v_i, u_i).
\label{eq:g-iter-in-reverse}
\end{equation}
(To justify the left-hand side being well-defined, we must show $(G_{i+1}, v_i, r) \in U(V)$ for all $i$.  By construction, $v_i \neq r$ for all $i$, so $r \notin S(G_{i+1}, v)$; hence $(G_{i+1}, v_i, r) \in U(V)$.)
By applying $g$ to both sides of \eqref{eq:g-iter-in-reverse}, we get \ref{itm:g-iter-recursion}.
Finally, note that by induction, $G_i$ has exactly $i$ edges whose source node is not $r$.
Thus $G_0 = G^\circ$.
This completes the proof of the case where $G = G^\circ$.

We now move to the general case, where $G \in \mathbf{G}(V)$ is arbitrary.
Using the above special case, let $(G_0 = G^\circ, v_0, u_0), (G_1, v_1, u_1), \ldots, (G_k, v_k, u_k)$ be a sequence satisfying \ref{itm:g-iter-first-condition}--\ref{itm:g-iter-recursion} and \eqref{eq:g-iter-in-reverse} with $G_0 = G^\circ$.
Let $(\overline{G}_0, \overline{v}_0, \overline{u}_0), (\overline{G}_1, \overline{v}_1, \overline{u}_1), \ldots, (\overline{G}_k, \overline{v}_k, \overline{u}_{\overline{k}})$ be a sequence satisfying \ref{itm:g-iter-first-condition}--\ref{itm:g-iter-recursion} and \eqref{eq:g-iter-in-reverse} but with $\overline{G}_0 = G^\circ$ and $\overline{G}_{\overline{k}} = G$.
Let $\underline{G}_j := \overline{G}_{\overline{k}-j}$ for $0 \leq j < \overline{k}$ and $\underline{G}_{\overline{k}} := G^\circ$.
Substituting $j := \overline{k} - (i+1)$ into \eqref{eq:g-iter-in-reverse} gives
\[
    g\left(
        \underline{G}_j,\ \overline{v}_{\overline{k}-j-1},\ r
    \right)
    =
    \left(
        \underline{G}_{j+1},\ \overline{v}_{\overline{k}-j-1},\ \overline{u}_{\overline{k}-j-1}
    \right)
\]
for each $j = 0,\ldots,\overline{k}-1$.
Thus, letting $\underline{k} := \overline{k}$ and $\underline{v}_j := \overline{v}_{\overline{k}-j-1}$ and $\underline{u}_j := r$, the sequence
\[
    (\underline{G}_0, \underline{v}_0, \underline{u}_0), \ldots, (\underline{G}_{\underline{k}}, \underline{u}_{\underline{k}}, \underline{v}_{\underline{k}})
\]
satisfies \ref{itm:g-iter-first-condition}--\ref{itm:g-iter-recursion} but with $\underline{G}_0 = G$ and $\underline{G}_k = G^\circ$.
Then, the sequence
\[
    (\underline{G}_0, \underline{v}_0, \underline{u}_0), \ldots, (\underline{G}_{\underline{k}-1}, \underline{u}_{\underline{k}-1}, \underline{v}_{\underline{k}-1}),
    \ (\underline{G}_{\underline{k}} = G_0 = G^\circ, v_0, u_0),
    \ (G_1, v_1, u_1), \ldots, (G_k, v_k, u_k)
\]
satisfies \ref{itm:g-iter-first-condition}--\ref{itm:g-iter-recursion}: the only piece of this claim that wasn't already proved above is \ref{itm:g-iter-recursion} at the concatenation boundary where $G^\circ$ appears; i.e., it remains only to check that
\[
    G^\circ = \proj_G(g(\underline{G}_{\underline{k}-1}, \underline{v}_{\underline{k}-1}, \underline{u}_{\underline{k}-1})).
\]
Unpacking the definitions, this condition is
\[
    G^\circ = \proj_G(g(\overline{G}_1, \overline{v}_0, r)),
\]
and indeed the condition is satisfied, by \eqref{eq:g-iter-in-reverse}.
\end{proof}

For some fixed $V$ with $r \in V$ and $N := |V|-1$, suppose we have in hand (i) a prior probability distribution $p(G)$ on $G \in \mathbf{G}(V)$; (ii) some data $\mathcal{D}$; and (iii) a likelihood function $p(\mathcal{D} | G)$.
From these, we can construct an involutive MCMC~\citep{cusumano2020automating} kernel on the space of graphs ($\mathbf{G}(V)$) by combining the involution $g$ defined above with
a family of auxiliary probability distributions $q(v, u; G)$ on $V \times V$ such that $q(v, u; G) > 0$ if and only if $(v, u) \in T(G)$.
The kernel takes as input a graph $G$, samples $(v, u) \sim q(\cdot; G)$, computes $(G', v', u') := g(G, v, u)$, and then returns the new graph $G'$ with probability
\begin{equation}
\min\left\{
1, \frac{p(G') p(\mathcal{D} | G') q(v', u'; G') }{p(G) p(\mathcal{D} | G) q(v, u; G)}
\right\}
\end{equation}
and otherwise returns the previous graph $G$ (i.e. rejects).
Because this kernel satisfies the requirements of involutive MCMC, it is stationary with respect to the following target distribution on $\mathbf{G}(V)$:
\begin{equation}
p(G | \mathcal{D}) = \frac{p(G) p(\mathcal{D} | G)}{\sum_{G' \in \mathbf{G}(V)} p(G') p(\mathcal{D} | G')}
\end{equation}
Furthermore, if $p(G) > 0$ and $p(\mathcal{D} | G) > 0$ for all $G \in \mathbf{G}(V)$, then the Markov chain generated by repeated application of this kernel converges to the target distribution as the number of steps goes to infinity
(the chain is irreducible by Prop.~\ref{prop:g-iter-is-irreducible} and aperiodic since the proposal has a positive probability of choosing $u$ to be the parent of $v$, and in that case $G' = G$).

\subsection{Transforming between two alternative 6DoF pose parametrizations} \label{sec:two_orientation_descr}
Before defining our full involutive MCMC kernel on scene graphs,
we define a transformation between continuous parameter spaces that
will be used as a building block of the full kernel.

Objects $v$ that are children of the root vertex have their 6DoF pose relative to the world coordinate frame parametrized directly via $\theta_v \in SE(3)$.
Recall that the pose of an object $v$ that is child of another object $u'$
is parametrized via the relative pose between a face of $u'$ and a face of $v$.
We choose a parametrization for the pose of one face relative to another that makes it natural to express a prior distribution in which:
(i) the two faces are nearly in flush contact with high probability, and
(ii) we know little about the relative in-plane offset of the two faces and their relative in-plane orientation.
A natural parametrization for this prior uses:
\begin{itemize}
\item Two dimensions for the in-plane offset
($a \in \mathbb{R}$ and $b \in \mathbb{R}$)
with relatively broad priors
\item One dimension for the perpendicular offset
($z \in \mathbb{R}$)
with a concentrated prior
\item An (outward) face normal vector
($\nu \in S^2$)
with a prior that concentrates on anti-parallel face normals
\item One dimension of in-plane angular rotation
($\varphi \in S^1$)
with a uniform prior.
\end{itemize}
We now define a bijection
\begin{multline}
\xi :
\mathbb{R}^3 \times SO(3)
\supseteq \R^3 \times \{\omega \in SO(3) : \omega(0,0,1)^\top \neq (0,0,-1)\} \\
\to \mathbb{R} \times \mathbb{R} \times \mathbb{R} \times (S^2 \setminus \{(0, 0, -1)^\top\}) \times S^1
\subseteq \R \times \R \times \R \times S^2 \times S^1.
\label{eq:xi-ae-bijective}
\end{multline}
Note that $\xi$ is a.e.{} bijective on the supersets as well, in the sense that in \eqref{eq:xi-ae-bijective}, $\xi$ is a bijection between the subsets, and each of the subsets has a complement of measure zero in its superset.
In $\xi$, the first three coordinates are copied directly and the orientations are transformed as follows.


In the first parameter space, $SO(3)$, orientations are represented ``directly,'' as the linear transformation that carries the parent coordinate frame to the child coordinate frame (translated to have the same origin).  The base measure on $SO(3)$ is the Haar measure.

In the second parameter space, $S^2 \times S^1$, orientations are represented in Hopf coordinates \cite{yershova2010hopf}:
intuitively, these coordinates characterize a rotation by where it carries the north pole $(0, 0, 1)^\top$ (we call this $\eta \in S^2$) and how much planar rotation it does after carrying the north pole to $\eta$ (we call this $\varphi \in S^1$).
However, there is no globally consistent (i.e., jointly continuous in $\eta$ and $\varphi$) way to choose where the rotation corresponding to $\varphi$ ``starts'' (i.e. which orientations have $\varphi = 0$)---formally, the fiber bundle induced by the Hopf fibration is not a trivial bundle.
But, it is possible to make these choices consistently on an open subset of $S^2 \times S^1$ whose complement has measure zero, as we do in Section~\ref{sec:hopf_explicit_coords}.
In particular, provided that $\eta$ is not the south pole $(0, 0, -1)$, there is a unique unit quaternion $(w, x, y, z) \in S^3$ that satisfies $\spin(w, x, y, z) (0, 0, 1)^\top = \eta$ and has minimal geodesic distance to the identity (where $\spin: S^3 \to SO(3)$ is the usual covering map, described in Section~\ref{sec:RN_existence}).
Then, $\spin(w, x, y, z) \in SO(3)$ is the rotation we take to correspond to $\eta=\eta$, $\varphi=0$.
The resulting map $(S^2 \setminus \{ (0, 0, -1)^\top \}) \times S^1 \to SO(3)$ is smooth; in Section~\ref{sec:hopf_explicit_coords} we compute the mapping explicitly in terms of coordinates.

% TODO do experiments into the alternatie, where we don't use the slack
% variables but instead propose the angular slack and offset slack during the
% transition

\subsection{Full involutive MCMC kernel on scene graphs}

Our full involutive MCMC kernel on scene graphs (including their parameters $\bm\theta$) is based on an extension of the involution $g$ defined above.
We first define the latent space of pairs $(G, \bm\theta)$, as:
\begin{equation}
X := \bigsqcup_{G \in \mathbf{G}(V)} \left(
\left(
    \bigtimes_{\substack{v \in V\setminus \{r\}\\(r, v) \in E}} SE(3)
\right)
\times
\left(
\bigtimes_{\substack{v \in V \setminus \{r\}\\(r, v) \not \in E}} (F \times F \times \mathbb{R} \times \mathbb{R} \times \mathbb{R} \times S^2 \times S^1)
\right)
\right)
\end{equation}
($\sqcup$ denotes disjoint union), and we endow this set with
a reference measure $\mu_P$ formed by sums (over the disjoint union)
of product measures that are composed from: the Lebesgue measure on $\R$,
the Haar measure on $SO(3)$, the uniform (spherical) measures on $S^2$ and $S^1$,
and the counting measure on $F \times F$.
Then, we define the space of auxiliary variables as:
\begin{equation}
Y := 
\left\{
    (v, r) : v \in V \setminus\{r\}
\right\}
\sqcup
\left\{
(v, u, f, f') : v,u \in V \setminus\{r\} \text{ and } f, f' \in F
\right\}
\end{equation}
with the reference measure $\mu_Q$ being the counting measure.
We define an auxiliary probability distribution $q(y; G, \bm\theta)$ such that
\begin{equation}
\begin{array}{lll}
q(v, r; G, \bm\theta) &> 0 &\mbox{ for all } v \in V \setminus \{r\}\\
q(v, u, f, f'; G, \bm\theta) &> 0 &\mbox { for all } (v, u) \in T(G) \mbox{ and all } f, f' \in F\\
q(y; G, \bm\theta) &= 0 &\mbox{ otherwise.}
\end{array}
\end{equation}

The extended state space of the involutive MCMC kernel is then
\begin{equation}
Z := \{(x, y) \in X \times Y : p(x) q(y; x) > 0\}.
\end{equation}
We construct an involution $h$ on the space $Z$ using the graph involution $g$ defined above as a building block.
In particular, $Z$ consists of tuples of four forms, and we define the involution $h$ piecewise depending on which of these four forms the input has: we take
\[
    Z = Z_1 \sqcup Z_2 \sqcup Z_3 \sqcup Z_4
\]
where the definitions of $Z_1, Z_2, Z_3, Z_4$, and the parametric form that $\theta_v$ takes on each component, are as follows:
\begin{align*}
    Z_1 &:= \left\{ (G, v, r, \bm{\theta}) : (r, v) \in E \right\} && \theta_v \in SE(3) \\
    Z_2 &:= \left\{ (G, v, r, \bm{\theta}) : (r, v) \notin E \right\} && \theta_v \in F \times F \times \mathbb{R} \times \mathbb{R} \times \mathbb{R} \times S^2 \times S^1 \\
    Z_3 &:= \left\{ (G, v, u, \bm{\theta}, f, f') : u \neq r, (r, v) \in E \right\} && \theta_v \in SE(3) \\
    Z_4 &:= \left\{ (G, v, u, \bm{\theta}, f, f') : u \neq r, (r, v) \notin E \right\} && \theta_v \in F \times F \times \mathbb{R} \times \mathbb{R} \times \mathbb{R} \times S^2 \times S^1.
\end{align*}
We define coordinate projections $\proj_\bullet$ on $Z_1$ and $Z_2$ (for the free parameters $G, v, \bm\theta$), and on $Z_3$ and $Z_4$ (for the free parameters $G, u, \bm\theta, f, f'$), as in Section~\ref{sec:proj-notation}.
Also, for notational convenience below, we extend $\proj_u$ to $Z_1$ and $Z_2$ by defining $\proj_u(G,v,r,\bm\theta) := r$; i.e., $\proj_u$ is constant on $Z_1 \sqcup Z_2$ with value $r$.

Then, $h$ is defined piecewise as follows:
\begin{equation}
h(z) = \begin{cases}
    h_{\mathrm{f} \to \mathrm{f}}(z) &\quad\text{ if $z \in Z_1$ \quad (floating to floating)} \\
    h_{\mathrm{c} \to \mathrm{f}}(z) &\quad\text{ if $z \in Z_2$ \quad (contact to floating)} \\
    h_{\mathrm{f} \to \mathrm{c}}(z) &\quad\text{ if $z \in Z_3$ \quad (floating to contact)} \\
    h_{\mathrm{c} \to \mathrm{c}}(z) &\quad\text{ if $z \in Z_4$ \quad (contact to contact).}
\end{cases}
\end{equation}

\begin{figure}[t]
  \centering
  \includegraphics[width=1.0\textwidth]{imgs/scene-graphs-supplement.pdf}
  \caption{
A subset of the possible transition types for our involutive MCMC kernel on scene graphs.
Left: `contact to floating` (forwards) and `floating to contact' (backwards).
Right: `contact to contact` (forwards) and `contact to contact' (backwards).
The vertex $v$ is the chosen `sever' vertex, and its subtree $S(G, v)$ is shaded.
Parameters $\theta_v \in SE(3)$, which are the independent 6DoF pose of an object relative to the world coordinate frame, are shown in blue;
and parameters $\theta_v \in F \times F \times \mathbb{R} \times \mathbb{R} \times \mathbb{R} \times S^2 \times S^1$, which parametrize the pose of an object relative to another object (specifically the relative pose between two faces of the two objects) are shown in red.
}
  \label{fig:scene-graph-transition-sup}
\end{figure}

We next define the function $h_{\bullet \to \bullet}$ corresponding to each of these four components, and give the acceptance probability in each case (the acceptance probabilities will be derived later in this section).

\paragraph{Floating to floating}
This transition makes no change to the structure $G$ or the parameters $\bm\theta$:
\begin{equation}
h_{\mathrm{f} \to \mathrm{f}}(G, v, r, \bm\theta) := (G, v, r, \bm\theta) \quad \mbox{ (no change) }
\end{equation}

\paragraph{Contact to floating}
This transition severs the edge from the parent of $v$ in $G$ (another object)
and replaces it with a new edge from the root $r$ to $v$ in $G'$.
The parameters of all vertices other than $v$ are unchanged.
The parameters $\theta_v$ are set to the absolute pose (relative to $r$) of $v$ in $(G, \bm\theta)$:
\[ h_{\mathrm{c} \to \mathrm{f}}(G, v, r, \bm\theta) := (G', v, u', \bm\theta', f, f'), \]
where
\begin{align*}
(G', v, u') &= g(G, v, r) \quad\mbox{}\\
(f, f') &= \proj_{f,f'}(\theta_v) \\
\theta_w' &= \theta_w \mbox{ for } w \ne v \\
\theta_v' &= \mathbf{x}_v(G, \bm\theta) \quad \mbox{(pose of $v$ with respect to $r$ in $(G, \bm\theta)$)}
\end{align*}

\paragraph{Floating to contact}
This transition severs the edge from the parent of $v$ in $G$ (which is $r$)
and replaces it with a new edge from another (object) vertex $u$ to $v$ in $G'$.
The parameters of all vertices other than $v$ are unchanged.
The parameters $\theta_v$ are computed by (i) computing the relative pose $\Delta \mathbf{x}_{(u,f') \to (v, f)}(G, \bm\theta) \in SE(3)$ between the face $f$ of object $v$ (oriented according to its outward normal) and face $f'$ of object $u$ in $(G, \bm\theta)$, and then (ii) transforming this pose into
an element of $\mathbb{R} \times \mathbb{R} \times \mathbb{R} \times S^2 \times S^1$ via the function $\xi$ defined in Section~\ref{sec:two_orientation_descr}:
\[ h_{\mathrm{f} \to \mathrm{c}}(G, v, u, \bm\theta, f, f') := (G', v, r, \bm\theta'), \]
where
\begin{align*}
(G', v, r) &= g(G, v, u) \\
\theta_w' &= \theta_w \mbox{ for } w \ne v \\
\theta_v' &= (f, f', a, b, z, \eta, \varphi) \\
(a, b, z, \eta, \varphi) &= \xi(\Delta \mathbf{x}_{(u, f') \to (v, f)}(G, \bm\theta))
\end{align*}

\paragraph{Contact to contact}
This transition severs the edge from the parent of $v$ in $G$ (which is some object $u'$)
and replaces it with a new edge from another object $u$ to $v$ in $G'$.
The parameters of all vertices other than $v$ are unchanged.
The parameters $\theta_v$ are again computed by first computing the relative pose
computing the relative pose $\Delta \mathbf{x}_{(u,f') \to (v, f)}(G, \bm\theta) \in SE(3)$,
then applying $\xi$:
\[ h_{\mathrm{c} \to \mathrm{c}}(G, v, u, \bm\theta, f, f') := (G', v, u', \bm\theta', f_2, f_2') \]
where
\begin{align*}
(G', v, u') &= g(G, v, u) \\
\theta_w' &= \theta_w \mbox{ for } w \ne v \\
\theta_v' &= (f, f', a, b, z, \eta, \varphi) \\
(f_2, f_2') &= \proj_{f,f'}(\theta_v) \\
(a, b, z, \eta, \varphi) &= \xi(\Delta \mathbf{x}_{(u, f') \to (v, f)}(G, \bm\theta))
\end{align*}

\begin{proposition}
The function $h$ is an involution.
\end{proposition}
\begin{proof}
First note that in each of the four cases ($z \in Z_i$ for $i = 1,2,3,4$), we have
\[ \proj_{G,v,u}(h(z)) = g(\proj_{G,v,u}(z)). \]
Therefore, since $g$ is an involution, we have
\[ \proj_{G,v,u}(h(h(z))) = g(\proj_{G,v,u}(h(z))) = g(g(\proj_{G,v,u}(z))) = \proj_{G,v,u}(z).  \]
Next, note that if $z \in Z_2 \sqcup Z_4$, then $\proj_{f,f'}(h(h(z))) = \proj_{f,f'}(z)$ simply by unraveling the definitions.

It remains to show that $\proj_{\bm\theta}(h(h(z))) = \proj_{\bm\theta}(z)$.
This clearly holds when $z \in Z_1$, as $h_{\mathrm{f} \to \mathrm{f}}$ is the identity.

For the case $z \in Z_2$, let $z = (G, v, r, \bm\theta)$, and let $f, f' := \proj_{f,f'}(\theta_v)$, and let $G$ and $u$ be such that $g(G, v, r) = (G', v, u')$.
Then, unraveling the definitions, we have
\[
    \proj_{\bm\theta}(h(z)) = \left\{\begin{aligned}
        w &\mapsto \theta_w \quad\text{for $w \neq v$} \\
        v &\mapsto \mathbf{x}_v(G, \theta)
    \end{aligned}\right\}.
\]
Unraveling the definitions one step further, we have
\[
    \proj_{\bm\theta}(h(h(z))) = \left\{\begin{aligned}
        w &\mapsto \theta_w \quad\text{for $w \neq v$} \\
        v &\mapsto (f, f', \underbrace{\xi\left(
                \Delta\mathbf{x}_{(u', f') \to (v, f)}\left(
                    G',
                    \left\{\begin{aligned}
                        w &\mapsto \theta_w \quad\text{for $w \neq v$} \\
                        v &\mapsto \mathbf{x}_v(G, \bm\theta)
                    \end{aligned}\right\}
                \right)
        \right)}_{:=\ \theta''_v\ :=\ (a', b', z', \eta', \varphi')}
        )
    \end{aligned}\right\}.
\]
So we need to show $\theta''_v = \theta_v$.
By construction, $\theta''_v$ is the contact-parametrized relative pose for $v$ (face $f$) relative to $u'$ (the parent of $v$ in $G$, face $f'$) that, when converted to an absolute (relative to $r$) pose in the scene graph $(G, \bm\theta)$, gives $\mathbf{x}_v(G, \theta)$.
In other words, indeed $\theta''_v = \theta_v$.

For the case $z \in Z_3$, let $z = (G, v, u, \bm\theta, f, f')$ and $G' := \proj_G(g(G, v, u))$.
Unraveling two layers of definitions similarly to above, we have
\[
    \proj_{\bm\theta}(h(h(z))) = \left\{\begin{aligned}
        w &\mapsto \theta_w \quad\text{for $w \neq v$} \\
        v &\mapsto \underbrace{
            \mathbf{x}_v\left(
                G',
                \left\{\begin{aligned}
                        w &\mapsto \theta_w \quad\text{for $w \neq v$} \\
                        v &\mapsto \Delta\mathbf{x}_{(u,f) \to (v,f')}(G, \bm\theta)
                \end{aligned}\right\}
            \right)
        }_{:=\ \theta''_v}
    \end{aligned}\right\}.
\]
It again suffices to show $\theta''_v = \theta_v$.
By construction, $\theta''_v$ is the pose in world frame (relative to $r$) of object $v$ in $G'$; and the contact-parametrized relative pose of $v$ (face $f$) relative to $u$ (face $f'$) in $G'$ by construction has the property that, when converted to an absolute pose in $(G, \bm\theta)$, the result is $\theta_v$.
Thus $\theta''_v = \theta_v$.

For the case $z \in Z_4$, let $z = (G, v, u, \bm\theta, f, f')$, and let $(f_2, f_2') := \proj_{f,f'}(\theta_v)$, and let $G'$ and $u'$ be such that $(G', v, u') = g(G, v, u)$.
Unraveling two layers of definitions again, we have
\[
    \proj_{\bm\theta}(h(h(z))) = \left\{\begin{aligned}
        w &\mapsto \theta_w \quad\text{for $w \neq v$} \\
        v &\mapsto \underbrace{(f, f', \xi\left(
                \Delta\mathbf{x}_{(u', f_2') \to (v, f_2)}\left(
                    G',
                    \left\{\begin{aligned}
                        w &\mapsto \theta_w \quad\text{for $w \neq v$} \\
                        v &\mapsto (f, f', \xi\left(
                            \Delta\mathbf{x}_{(u, f') \to (v, f)}(G, \bm\theta)
                        \right),
                        f, f')
                    \end{aligned}\right\}
                \right)
            \right)
        )}_{:=\ \theta''_v\ :=\ (a', b', z', \eta', \varphi')}
    \end{aligned}\right\}.
\]
It again suffices to show $\theta''_v = \theta_v$.  Similarly to the above, $\theta''_v$ is a contact-parametrized relative pose for $v$ (face $f$) relative to $u'$ (the parent of $v$ in $G$, face $f'$) defined by the property that it produces the same absolute pose for $v$ as a second contact-parametrized relative pose.
This second relative pose is for $v$ (face $f_2$) relative to $u'$ (face $f_2'$) that by construction produces the same absolute pose as $v$ has in $(G, \bm\theta)$.
It again follows that $\theta''_v = \theta_v$, and this completes the proof.
\end{proof}

The automated involutive MCMC implementation in Gen~\citep{cusumano2019gen}
includes an optional dynamic check that applies the involution twice to check that it is indeed an involution.
We applied this check during testing of the algorithm to gain confidence in our implementation.
% TODO argue for this...

\subsection{The Radon--Nikodym derivative}


The acceptance ratio for involutive MCMC~\citep{cusumano2020automating} includes a ``generalized Jacobian correction'' term, equal to the Radon--Nikodym derivative of a pushforward measure $\mu_*$ with respect to a base measure $\mu$ defined on the state space $Z$
($\mu$ is constructed the product measure of $\mu_P$ and $\mu_Q$~\citep{cusumano2020automating}).
Next, $\mu_* := \mu \circ h^{-1}$ is the pushforward of $\mu$ by the involution $h: Z \to Z$ described above.
To justify the validity of this involutive MCMC kernel, we must show that $\mu_*$ is absolutely continuous with respect to $\mu$, i.e., that the Radon--Nikodym derivative $\frac{d\mu_*}{d\mu}$ exists.
Because all the discrete choices in the model (graph structure, contact faces, etc.) are assigned positive probability mass in both the model and the proposal, it suffices to show absolute continuity for the continuous part of the involution: the mapping (call it $\ell^\circ$) that, for given contact faces $f, f' \in F$, converts between a 6DoF pose $\theta_1 \in SE(3)$ and a contact-parameterized relative pose $\theta_2 = (a, b, z, \eta, \varphi) \in \R \times \R \times \R \times S^2 \times S^1$
(note that in this section we use $\theta_2$ to denote only the continuous part of the contact-parameterized relative pose).

Note that $\ell^\circ$ depends on not just $\theta_1$, but also on the scene graph, objects and faces $(G, \bm\theta, v, u', f, f')$.
Specifically, the absolute pose $\theta_1$ is gotten by pre- and post-composing $\Delta\mathbf{x}_{(u',f') \to (v, f)}(G, \bm\theta)$ with rigid motions that depend on the absolute poses of face $f'$ of $u$ and face $f$ of $v$ in scene graph $(G, \bm\theta)$, but these rigid motions do not depend on $\theta_1$ or $\theta_2$ themselves.
Thus, in the sections below, rather than $\ell^\circ$ itself, we analyze $\ell$, the variant of $\ell^\circ$ which operates on 6DoF relative poses $\Delta\mathbf{x}$ where $\ell^\circ$ operates on 6DoF absolute poses $\mathbf{x}$.
Because rigid transformations are diffeomorphisms and their Radon--Nikodym derivatives (Jacobian determinants) are identically $1$, the results in the sections below, which show that $\ell$ has a Radon--Nikodym derivative that is piecewise constant on $A \sqcup B$ (defined below), apply equally well to the map $\ell^\circ$ which parameterizes $\theta_1$ relative to the world coordinate frame in some particular scene graph.

We can denote a rigid motion by the pair $(\mathbf{t}, \omega)$, where $\mathbf{t} \in \R^3$ is the translation component and $\omega \in SO(3)$ is the rotation component.
(In algebraic terms, we are identifying $SE_3$ with the semidirect product $\R^3 \rtimes SO(3)$.)
Accordingly, define projection functions $\proj_{\mathbf{t}}$ and $\proj_{\omega}$ on $SE(3)$ as in Section~\ref{sec:proj-notation}.

Let $Z := A \sqcup B$ where $A := \R^3 \times SO(3)$ and $B := \R \times \R \times \R \times S^2 \times S^1$, and let $\nu$ denote the base measure on $Z$.\footnote{
    That is, the sum of (i) the product of Lebesgue measure on $\R^3$ and Haar measure on $SO(3)$, and (ii) the product of Lebesgue measure on $\R \times \R \times \R$, spherical uniform measure on $S^2$, and uniform spherical measure on $S^1$.
}
In the sections below, we discuss the pushforward $\nu_* := \nu \circ \ell^{-1}$ and its Radon--Nikodym derivative $\frac{d\nu_*}{d\nu}$.

\subsubsection{Existence of the Radon--Nikodym derivative}\label{sec:RN_existence}

In this section, we prove $\nu_* \ll \nu$.  Because $\nu_* = \nu \circ \ell^{-1}$, the proof proceeds by analyzing the involution $\ell$.
First we show that $\ell$ is defined almost everywhere, so that $\nu_*$ is well-defined.
Then we show that there exists a subset $Z'' \subseteq Z$ whose complement has measure zero, such that the restriction of $\ell$ to $Z''$ is a diffeomorphism.
It follows that $\nu_* \ll \nu$ by \cite[Prop 6.5]{lee2003smooth}, since $\ell^{-1}$ is a smooth map.

First, to show that $\nu_*$ is well-defined, we show that $\ell$ is defined almost everywhere on $Z$.
Indeed, the domain of $\ell$ is $A' \sqcup B'$, where
   $A' := \R^3 \times \domain(\xi)$ and
   $B' := \R \times \R \times \R \times \domain(\xi^{-1})$
(where $\xi$ is as defined in Section~\ref{sec:two_orientation_descr}).
Now, $\domain(\xi) = \{\omega \in SO(3) : \omega (0, 0, 1)^\top \neq (0, 0, -1)^\top \}$ has a complement of measure zero in $SO(3)$, and $\domain(\xi^{-1}) = (S^2 \setminus \{(0, 0, -1)^\top\}) \times S^1$ has a complement of measure zero in $S^2 \times S^1$, so indeed $\domain(\ell) = A' \sqcup B'$ has a complement of measure zero in $Z$.

Next, we show that there exist subsets $A'' \subseteq A'$, $B'' \subseteq B'$, whose complements also have measure zero, such that $\ell$ fixes $A \sqcup B$ setwise and the restriction of $\ell$ to $A'' \sqcup B''$ is a diffeomorphism.
First, note that the coordinates $\mathbf{t}$ in $A$ and the coordinates $a, b, z$ in $B$ represent the same translation in a different coordinate frame.
Thus, for any fixed $\omega \in SO(3)$, the function $\mathbf{t} \mapsto \proj_{a,b,z}(\ell(\mathbf{t}, \omega))$ is a rigid motion, hence a diffeomorphism.
Furthermore, the rotation component of $\ell(\theta)$ (regardless of whether $\theta \in A$ or $\theta \in B$) depends only on the rotation component of $\theta$, not at all on the translation component. 
Thus, $\ell$ is a diffeomorphism from $A''$ to $B''$ if and only if $\ell^\star$ is a diffeomorphism from $\proj_\omega(A'')$ to $\proj_{\eta,\varphi}(B'')$, where $\ell^\star(\omega) := \proj_{\eta,\varphi}(\ell(0, \omega))$.
Taking $A'' := \R^3 \times A^\star$ and $B'' := \R \times \R \times \R \times B^\star$, we see that it suffices to find subsets $A^\star \subseteq SO(3)$ and $B^\star \subseteq S^2 \times S^1$ whose complements have measure zero, such that the restriction of $\ell^\star$ to $A^\star$ is a diffeomorphism onto $B^\star$.

Denote elements of $SO(3)$ as $\omega = \spin(w, x, y, z)$, where $\spin: S^3 \to SO(3)$ is the 2-to-1 smooth covering map that carries a unit quaternion $w + x\mathbf{i} + y\mathbf{j} + z\mathbf{k}$ to its corresponding rotation.
Then, we take
\begin{align*}
    A^\star &= \left\{\spin(w, x, y, z) \mvert (w,x,y,z) \in S^3;\ z \neq 0 \right\} \\
    B^\star &= \left\{((a, b, c), \varphi) \in S^2 \times S^1 \mvert c \neq \pm 1 \right\}
\end{align*}
To show that $\ell$ is a diffeomorphism from $A^\star$ to $B^\star$, we give an explicit formula for $\ell^\star$ in terms of coordinates below (Section~\ref{sec:hopf_explicit_coords}).


\subsubsection{Formula for the mapping in coordinates}\label{sec:hopf_explicit_coords}

In this section we give an explicit formula for the map $\ell^\star$ defined in Section\ref{sec:RN_existence} in terms of coordinates.
For elements $\omega = \spin(w, x, y, z) \in A^\star$, the $S^2$ component of $\ell^\star(\omega)$ is the image of $(0, 0, 1)^\top$ under the rotation, and is given by~\cite[\S8.2]{gallier2001quat}:
\[ \eta
    = \spin(w, x, y, z) \begin{pmatrix} 0 \\ 0 \\ 1 \end{pmatrix}
    = \begin{pmatrix}
        2(xz + wy) \\
        2(yz - wx) \\
        1 - 2(x^2 + y^2)
      \end{pmatrix}.
\]
Even though $(w,x,y,z)$ is not uniquely determined by $\spin(w,x,y,z)$, the above expression is well-defined because both possible choices of quaternion---$(w,x,y,z)$ and $(-w,-x,-y,-z)$---give the same value for the right-hand side.

To compute the $S^1$ component $\varphi$, note that the set of all rotations that carry $(0, 0, 1)^\top$ to $\eta$ is
\[ \{ \spin(w, x, y, z) \circ R_{(0,0,1)}(-\varphi') : 0 \leq \varphi' < 2\pi\}, \]
where $R_{(0,0,1)}(\varphi')$ is a rotation about the axis $(0, 0, 1)^\top$ by angle $\varphi'$.
Because the action of $S^3$ on itself by quaternion multiplication is a geometric rotation of $S^3$ (in particular, an isometry) \cite[\S8.3]{gallier2001quat}, minimizing geodesic distance among the above family of rotations is equivalent to minimizing (over $\varphi'$) geodesic distance from $R_{(0,0,1)}(\varphi')$ to $(w,x,y,z)$.
Explicitly, $R_{(0,0,1)}(\varphi')$ corresponds to the unit quaternions
\[ \pm \left( \cos(\varphi'/2),\ 0,\ 0,\ \sin(\varphi'/2) \right). \]

Note that minimizing geodesic distance on the sphere $S^3$ is equivalent to minimizing the cosine between the corresponding vectors in $S^3 \subseteq \R^4$.
By the cosine double angle formula, the cosine between $R_{(0,0,1)}(\varphi')$ and $(w,x,y,z)$ in this sense is
\[ 2 \left(\Big. \pm(\cos(\varphi'/2),\ 0,\ 0,\ \sin(\varphi'/2)) \cdot (w, x, y, z)\right)^2 - 1, \]
where $\cdot$ denotes the dot product in $\R^4$.
This quantity is maximized precisely when the doct product is either maximized or minimized, so we can drop the $\pm$.
We can then compute the minima and maxima of the dot product by setting the derivative equal to zero: we have
\[ 
    (\cos(\varphi'/2),\ 0,\ 0,\ \sin(\varphi'/2)) \cdot (w, x, y, z) = w \cos(\varphi'/2) + z \sin(\varphi'/2)
\]
and the above expression is minimized or maximized when
\[
    \varphi'/2 = \arctan(z/w) + \pi n \quad\text{for some $n \in \mathbb{Z}$}
\]
or equivalently, $\varphi' = 2\arctan(z/w) + 2\pi n$.
Thus, the $S^1$ component of $\ell^\star(\spin(w,x,y,z))$ that we set out to compute is
\[
    \varphi = 2\arctan(z/w),
\]
where the branch cut in $\arctan$ is chosen so that the output lies in the interval $[0, \pi)$, and we allow $z/w$ to lie on the extended real line, with $\arctan(\pm \infty) = \pi/2$.

Since $z \neq 0$ in $A^\star$, $\varphi$ never lands on the branch cut.
Thus, $\ell^\star(\spin(w,x,y,z))$ is a smooth function of the quaternion $(w,x,y,z)$.
Because $\spin$ is a smooth covering map \cite[ch.~4]{lee2003smooth}, it follows\footnote{
    This can be seen by pre-composing $h''$ with a lifting to one of the sheets in an evenly covered neighborhood of $\omega$.
} that $\spin$ is also a smooth function of the element $\omega = \spin(w,x,y,z) \in SO(3)$.

The above definition expresses in coordinates the geometry of Hopf fibration and choice of branch cut described in Section~\ref{sec:two_orientation_descr}.
We now need only show that $\ell^\star$ has a smooth inverse.
For elements $(\eta, \varphi) \in B^\star$, we take~\cite[\S7]{baldwin2017hopf}
\begin{multline*}
    (\ell^\star)^{-1}\left((a, b, c), \varphi\right) = \\
    \spin\left(
        \tfrac{1}{\sqrt{2(1 + c)}} \left(
            (1 + c)\cos(\varphi),\ 
            a\sin(\varphi) - b\cos(\varphi),\ 
            a\cos(\varphi) + b\sin(\varphi),\ 
            (1 + c)\sin(\varphi)
        \right)
    \right).
\end{multline*}
This function is clearly smooth on $B^\star$, and direct computation shows that $(\ell^\star)^{-1} \circ \ell^\star$ is the identity.


\subsubsection{Value of the Radon--Nikodym derivative}

In the preceding sections we showed that $\nu_* := \nu \circ \ell^{-1}$ has a density with respect to $\nu$, the Radon--Nikodym derivative $\rho := \frac{d\nu_*}{d\nu}$.
In this section we argue that $\rho$ is (a.e.) constant on each of the connected components $A$ and $B$.
Because $h$ acts as isometries on the translation components, and acts on rotation components in a way that doesn't depend on the translation components, we need only look at orientation components.
That is, the Radon--Nikodym derivative $\rho$ is equal to the Radon--Nikodym derivative of the pushforward $\nu^\star_*$ of the base measure\footnote{
    In this case, sum of the Haar measure on $A^\star \subseteq SO(3)$ and the product of uniform measures on $B^\star \subseteq S^2 \times S^1$.
} $\nu^\star$ on $A^\star \sqcup B^\star$ by $\ell^\star$:
\[ \textstyle
    \frac{d\nu_*}{d\nu}(\mathbf{t}, \omega) = \frac{d\nu^\star_*}{d\nu^\star}(\omega)
    \quad\text{and}\quad
    \frac{d\nu_*}{d\nu}(a,b,z,\eta,\varphi) = \frac{d\nu^\star_*}{d\nu^\star}(\eta,\varphi).
\]

Note the following chain of equivalences: $\frac{d\nu^\star_*}{d\nu^\star}$ is a.e. constant on $A^\star$ $\iff$ $\nu^\star_*$ is a scalar multiple of the Haar measure on $A^\star$ $\iff$ $\nu^\star_*$ is invariant under the action of $SO(3)$ on itself by multiplication.
We can see that the latter statement holds by noting three things:
First, the Haar measure on $SO(3)$ equals the pushforward by $\spin$ of the Haar measure on unit quaternions.
Next, the action of $S^3$ on itself by group multiplication is an action by isometries \cite[\S8.3]{gallier2001quat}.
Finally, by \cite{yershova2010hopf}, the volume element on $S^3$ equals the product of the volume elements on $S^2$ and $S^1$ (here we are using the fact that the Haar measure on $S^3$ coincides with the Borel measure when it is viewed as a Riemannian manifold).
Thus the action of $SO(3)$ on itself by multiplication, when pushed through $(\ell^\star)^{-1}$, becomes an action by local isometries on $S^2 \times S^1$, and is thus invariant under the base measure $\nu^\star$ (which on $B^\star$ is the product of uniform measures).
Therefore indeed $\frac{d\nu^\star_*}{d\nu^\star}$ is a.e. constant on $A^\star$.
Since $\ell^\star$ is an involution and $\ell^\star(A^\star) = B^\star$, it follows that $\frac{d\nu^\star_*}{d\nu^\star}$ is a.e. constant on $B^\star$, and the values of the Radon--Nikodym derivative on $A^\star$ and $B^\star$ are reciprocals of each other.


\subsubsection{Acceptance probability}

For our choice of scaling constants, if we assign total measures $SO(3) \mapsto \pi^2$, $S^2 \mapsto 4 \pi$, and $S^1 \mapsto 2\pi$,
then the Radon--Nikodym derivative corrections in the involutive MCMC acceptance probability~\citep{cusumano2020automating} are:
$1$ (for `floating to floating' and `contact to contact' moves),
$(4 \pi \cdot 2 \pi)/\pi^2 = 8$ (for a `floating to contact' move), and
$\pi^2/(4 \pi \cdot 2 \pi) = 1/8$ (for a `contact to floating' move).
This gives the 
following acceptance probabilities for each of the four possible proposed moves from $(G, \bm\theta)$ to $(G', \bm\theta')$ that can be proposed within our kernel:

\paragraph{Floating to floating}
When $u = r$ and $u' = r$, the state is unchanged, and the move always accepts:
\begin{equation}
\alpha = \min\left\{
1, 
\frac{p(\mathbf{Y} | N, \mathbf{c}, G, \bm\theta)}{p(\mathbf{Y} | N, \mathbf{c}, G, \bm\theta)}
\right\} = 1
\end{equation}

\paragraph{Floating to contact}
When $u \ne r$ and $u' = r$
(we are severing $v$ from the root and grafting $v$ onto another object),
the acceptance probability is:
\begin{equation}
\alpha = \min\left\{
1, 
\frac{p(\mathbf{Y} | N, \mathbf{c}, G', \bm\theta')}{p(\mathbf{Y} | N, \mathbf{c}, G, \bm\theta)}
\frac{p(\bm\theta' | N, \mathbf{c}, G')}{p(\bm\theta | N, \mathbf{c}, G)}
\frac{q(v', u'; G')}{q(v, u, f_1, f_1'; G)}
\cdot
8
\right\}
\end{equation}
where $(f_1, f_1') = \proj_{f,f'}(\theta'_v)$.

\paragraph{Contact to floating}
When $u = r$ and $u' \ne r$
(we are severing $v$ from an object and grafting $v$ onto the root),
the acceptance probability is:
\begin{equation}
\alpha = \min\left\{
1, 
\frac{p(\mathbf{Y} | N, \mathbf{c}, G', \bm\theta')}{p(\mathbf{Y} | N, \mathbf{c}, G, \bm\theta)}
\frac{p(\bm\theta' | N, \mathbf{c}, G')}{p(\bm\theta | N, \mathbf{c}, G)}
\frac{q(v', u', f_2, f_2'; G')}{q(v, u; G)}
\cdot
\frac{1}{8}
\right\}
\end{equation}
where $(f_2, f_2') = \proj_{f,f'}(\theta_v)$.

\paragraph{Contact to contact}
When $u \ne r$ and $u' \ne r$
(we are severing $v$ from an object and grafting $v$ onto another object),
the acceptance probability is:
\begin{equation}
\alpha = \min\left\{
1, 
\frac{p(\mathbf{Y} | N, \mathbf{c}, G', \bm\theta')}{p(\mathbf{Y} | N, \mathbf{c}, G, \bm\theta)}
\frac{p(\bm\theta' | N, \mathbf{c}, G')}{p(\bm\theta | N, \mathbf{c}, G)}
\frac{q(v', u', f_2, f_2'; G')}{q(v, u, f_1, f_1'; G)}
\right\}
\end{equation}
where $(f_1, f_1') = \proj_{f,f'}(\theta'_v)$ and $(f_2, f_2') = \proj_{f,f'}(\theta_v)$.

\subsection{Lemmas}

\begin{lemma}
Let $G = (V, E)$ be a directed tree rooted at $r \in V$, and suppose $u, u', v \in V$ are such that the following conditions hold:
\begin{enumerate}[nosep,label=(\roman*)]
    \item $u \neq v$
    \item $u$ is not a descendant of $v$
    \item $(u', v) \in E$.
\end{enumerate}
Let $G'$ be the directed graph obtained from $G$ by deleting the edge $(u', v)$ and adding the edge $(u, v)$, that is, $G' = (V, (E \setminus \{(u', v)\}) \cup \{(u, v)\})$.
Then $G'$ is a directed tree rooted at $r$.
\label{lem:tree_add_remove}
\end{lemma}
\begin{proof}
We show that for any vertex $w$, there is a unique path in $G'$ from $r$ to $w$.

First, suppose $w$ is not a descendant of $v$ in $G'$.
Then $w$ is not a descendant of $v$ in $G$, since the set of descendants of $v$ is the same in $G$ and $G'$.
Thus no path from $r$ to $w$ in either $G$ or $G'$ passes through $v$; consequently, no path from $r$ to $w$ in either $G$ or $G'$ contains either of the edges $(u' v)$ or $(u, v)$.
Thus, a sequence of vertices $r = x_0,\ x_1,\ \ldots,\ x_n = w$ is a path in $G'$ if and only if it is a path in $G$.
Since there is a unique path from $r$ to $w$ in $G$, it follows that there is a unique path from $r$ to $w$ in $G'$.

Next, suppose $w$ is a descendant of $v$ in $G'$ (and hence also in $G$).
Because $u'$ is the only in-neighbor of $v$ in $G$, it follows that $u$ is the only in-neighbor of $v$ in $G'$.
Thus, every path from $r$ to $w$ in $G'$ is the concatenation of a path from $r$ to $u$ with a path from $v$ to $w$.
But since $u$ is not a descendant of $v$ in $G$ (hence neither in $G'$), there is a unique path from $r$ to $u$ in $G'$, by the above paragraph.
And of course, there is a unique path from $v$ to $w$ in $G$, hence also in $G'$ since the subtrees rooted at $v$ are the same.
Therefore there is a unique path from $v$ to $w$ in $G'$.
\end{proof}

