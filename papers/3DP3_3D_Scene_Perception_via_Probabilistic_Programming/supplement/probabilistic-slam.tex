\section{Probabilistic SLAM using Sequential Monte Carlo}

To infer the camera poses $\mathbf{x}_{1:T}$ corresponding to the sequence of $T$ depth images, we implemented a probabilistic SLAM using Sequential Monte Carlo. We assume the depth images contain background, which can be mapped and used for localization between frames. (In our data, the object is placed in the center of a rectangular room with a floor, ceiling, and four walls.) We also assume that the camera is the same distance away from the object in all $T$ images. Finally, in order to ensure a reference frame match between the learned object model and ground truth model (such that at test time, object pose estimates of our system can be compared to the ground truth object poses), we provide the initial pose of the object in the camera frame.

To perform SLAM, we initialize a set of $K$ particles with the observation, camera pose, and implied map at $t=1$. Then, we enumerate over the position and viewing angle at $t=2$, given the map at $t=1$, observation at $t=2$, and a prior on the camera pose conditioned on the pose at $t=1$, and compute scores for each pose. We then construct a Gaussian mixture proposal where each component is centered on a different pose and has weight corresponding to the normalized score computed by the enumeration. We step the particles forward to $t=2$ with this proposal distribution. Then, for each of the particles, we update the map given the observations and inferred poses. We repeat this for all $T$ timesteps, at which point we have $K$ particles with inferred camera poses for each of the $T$ timesteps.

The depth image likelihood $p'(\mathbf{I}_t | \mathbf{O}, \mathbf{M}, \mathbf{x}_t) = \delta_{d(\mathbf{O}, \mathbf{M}, \mathbf{x}_t)}(\mathbf{I}_t)$ where $d$ is a depth rendering function.