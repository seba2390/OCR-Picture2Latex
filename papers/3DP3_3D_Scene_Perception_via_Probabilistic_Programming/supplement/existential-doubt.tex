\section{Parsing scenes with fully occluded objects and number uncertainty} \label{sec:existential-doubt}

Consider the setting when the number of objects in the scene ($N$) is unknown a-priori.
Possibilities for prior probability distributions on $N$ ($p(N)$) include
(i) an a-priori known number of objects $N_0$ (used in the experiments in Section 6),
(ii) $\mathrm{Binomial}(N_0, p_{\mathrm{present}})$, which is induced by a prior belief that each of $N_0$ objects is present with probability $p_{\mathrm{present}}$ (used in experiments described in Supplement~\ref{sec:existential-doubt}),
and
(iii) $\mathrm{Poisson}(\lambda)$, which places no a-priori upper bound on the number of objects.

In this section, we apply our framework to do probabilistic inference about the 6DoF pose and presence or absence of a fully occluded object, and investigate the dynamics of these inferences as we vary the fraction of the volume in the scene that is occluded.

Suppose a robot is tasked with assembling a piece of furniture, performing maintenance on a vehicle, or retrieving something from the kitchen.
In each of these cases, the robot has a strong prior expectation that some object (e.g. a tool, component, or kitchen item) is present in the environment.
However, in complex cluttered real-world environments the target object is likely to be fully occluded from the robot's view.
That target object may even even be absent, especially in human-robot interactive task situations (e.g. the component or item is missing or misplaced).
To perform rationally in such situations, the robot will need to generate possible poses of the object that are concordant with its \emph{absence} from its visual field.
Also, the robot must consider the possibility that the object is indeed not present, by weighing the lack of observed presence of the object against the prior expectations.



\begin{figure}[p]
\begin{subfigure}[t]{1\textwidth}
    \centering
    \includegraphics[width=0.7\textwidth]{imgs/existential-doubt/fig-scenario.png}
    \caption{The scenario (left). A depth camera is viewing a scene that may or may not contain a cracker box (middle) and a mug (right). We perform Bayesian inference on the existence and contingently, their 6DoF poses within the scene, of both objects. Only the cracker box is visible in the observed depth images (see below).}
    \label{fig:scenario}
\end{subfigure}
\begin{subfigure}[t]{\textwidth}
    \centering
    \includegraphics[width=\textwidth]{imgs/existential-doubt/results-1.png}
    \caption{Observed depth image and posterior samples where the existence of both the box and mug are assumed.}
    \label{fig:results-1}
\end{subfigure}
\begin{subfigure}[t]{\textwidth}
    \centering
    \includegraphics[width=\textwidth]{imgs/existential-doubt/results-2.png}
    \caption{Observed depth image and posterior samples where the existences of each object have prior probability 0.90.
The posterior probabilities of existence for the mug and box are 0.37 and 1.0, respectively.}
    \label{fig:results-2}
\end{subfigure}
\begin{subfigure}[t]{\textwidth}
    \centering
    \includegraphics[width=\textwidth]{imgs/existential-doubt/results-3.png}
    \caption{Observed depth image and posterior samples where the existences of each object have prior probability 0.90.
Note that the observed image has the box angled so that it occupies less of the field of view than in (c).
The posterior probabilities of existence for the mug and box are 0.33 and 1.0, respectively.}
    \label{fig:results-3}
\end{subfigure}
\begin{subfigure}[t]{\textwidth}
    \centering
    \includegraphics[width=\textwidth]{imgs/existential-doubt/results-4.png}
    \caption{Observed depth image and posterior samples where the existences of each object have prior probability 0.90.
Note that the observed image has the box closer to the camera, so that it occupies more of the field of view than in (c). 
The posterior probabilities of existence for the mug and box are 0.63 and 1.0, respectively.}
    \label{fig:results-4}
\end{subfigure}%
    \caption{
Inferring the 6DoF pose and existence of multiple objects from depth images using MCMC in a generative model.
Several scenarios are shown, with five approximate posterior samples from each.
To estimate posterior probabilities of object existence, 20 posterior samples were used.
The \emph{lack of percept} of the mug in the visual field (i) reduces the posterior probability of its presence, but also (ii) informs the distribution on its 6DoF pose, if it is present.
Note that as the fraction of volume in the scene that is occluded by the box decreases, the posterior probability that mug is present decreases.
}
    \label{fig:results}
\end{figure}

\paragraph{Prior}
Consider a scenario where there are $N_0 = M$ unique objects that may or may not be present in a scene
($N_0$ denotes the total number of object instances, and $M$ denote the number of object types).
Suppose that the prior probability that the object of type $m$ is present with probability $p_{\mathrm{pres}}^{(m)}$ for $m \in \{1, \ldots, M\}$.
Then, the prior $p(N, \mathbf{c})$ is:
\begin{equation}
p(N, \mathbf{c}) =
\left\{
\begin{array}{ll}
\frac{1}{N!} \prod_{m=1}^M {p_{\mathrm{pres}}^{(m)}}^{\mathbf{1}[m \in \mathbf{c}]} {(1 - p_{\mathrm{pres}}^{(m)})}^{\mathbf{1}[m \not \in \mathbf{c}]} &
\mbox{if } |\mathbf{c}| = N \mbox{ and } \sum_{i=1}^N \mathbf{1}[c_i = m] \le 1\, \forall m\\
0 & \mbox{otherwise}
\end{array}
\right.
\end{equation}
In the special case when $p_{\mathrm{pres}}^{(m)}$ is the same for all $m$ (this is the case in our experiments below), we can write the marginal distribution $p(N)$ and conditional distribution $p(\mathbf{c} | N)$ as:
\begin{equation}
N \sim \mathrm{Binomial}(M = N_0, p_{\mathrm{pres}})
\end{equation}
and
\begin{equation}
p(\mathbf{c} | N) =
\left\{
\begin{array}{ll}
\frac{(M - N)!}{M!} &
\mbox{if } |\mathbf{c}| = N \mbox{ and } \sum_{i=1}^N \mathbf{1}[c_i = m] \le 1\, \forall m\\
0 &
\mbox{otherwise}
\end{array}
\right.
\end{equation}

For each possible $(N, \mathbf{c})$, we fix the scene graph $G$ to be the graph $G_0(N)$ on $N$ vertices that has with no object-object edges:
\begin{equation}
p(G | N) = \left\{
\begin{array}{ll}
1 & \mbox{if } G = G_0(N)\\
0 & \mbox{otherwise}
\end{array}
\right.
\end{equation}
That is, each object $v$ has an independent 6DoF pose $\mathbf{x}_v$.
The prior distribution on the orientation component used the uniform distribution on an Euler angle parametrization,
and the prior on the translation component (i.e. the location of the object) the uniform distribution on a cuboid volume representing the extent of the scene.

\paragraph{Likelihood}
Instead of the likelihood on point clouds used in Section~3.3, here we use an alternative likelihood based on
(i) rendering a depth image $\tilde{\mathbf{I}}(\mathbf{O}_{1:M}, \mathbf{c}, G, \bm\theta)$ and then (ii) adding noise to generate an observed depth image $\mathbf{I}$.
The likelihood is a per-pixel mixture between a uniform distribution on the range of possible depth values, and a normal distribution with fixed variance $\sigma^2$:
\[
    p(\mathbf{I} | x) = \prod_i \left( 0.1 \cdot \frac{1}{D} + 0.9 \cdot \mathcal{N}(I_{i}; \tilde{I}_i, \sigma)) \right)
\]
where $i$ indexes pixels of the depth image.
Pixels whose ray does not intersect an object are assigned the maximum depth value $D$.
A similar likelihood function on depth images was used in~\citep{moreno2016overcoming}.

\paragraph{MCMC inference algorithm}
We use a Markov chain Monte Carlo (MCMC) inference algorithm that cycles through each object type $m \in \{1, \ldots, M\}$, and applies several types of MCMC moves for each type, based on the following proposals:
(i) involutive MCMC kernels that switch an object type from being absent to being present and vice versa (proposing its pose from the prior) ,
(ii) Metropolis--Hastings kernels that propose the translational components of the pose $\mathbf{x}_v$ for each object from the prior,
(iii) Metropolis--Hastings kernels that propose the rotational component of the pose for each object from the prior, and
(iv) coordinate-wise random-walk proposals to each of the 6 dimensions of the pose of each object (the 3 coordinates of its location and its Euler angles).
We initialize the Markov chain with a sample from the prior distribution.

\paragraph{Inferring the 6DoF pose of a fully occluded object}
We first investigated inference about the 6DoF pose of an object (the mug from the YCB object set~\citep{calli2015benchmarking}) that is assumed to be in a volume in front of the camera, but that is not visible.
This scenario arises when searching for a component or tool that is expected to be in the environment.
Narrowing down where the object could be, based on observing where it is not, is important for efficiently planning and acting to obtain more information and retrieve the object.
In order for inference to be coherent, the lack of the object's visible presence must be explained away by the occluding presence of another object.
Therefore, we also assume the presence of another object (the cracker box).
The results (Figure~\ref{fig:results}(b)) show that the algorithm
successfully infer a variety of 6DoF poses of the mug in which it lies behind the cracker box.

\paragraph{Jointly inferring the existence and poses of multiple objects}
If an object is not visible, we may conclude that it is not present in the scene.
The degree of belief in the presence of an object that is not visible depends on the degree of prior belief in its presence and the volume of possible states in which the object is fully occluded.
If there are no other objects in the scene, then intuitively, there is nowhere the object could be hiding in the scene, so it must be absent.
If there are other objects in the scene, then the pose of these other objects interacts with the object's existence, in our beliefs.
To investigate the interaction between the beliefs about the poses and existence of multiple objects, we generated synthetic depth data for several scenarios (Figure~\ref{fig:results}(c-e)).
In all scenarios, the prior probabilities that the box and mug are present are both 0.9.
In the first scenario (Figure~\ref{fig:results}(c)) the box is oriented so a wide face is facing the camera.
We correctly infer the existence of the box with high confidence (1.0 posterior probability) and we assign 0.37 posterior probability to the existence of the mug.
In the second scenario (Figure~\ref{fig:results}(d)) the box is rotated so that it occupies less of the camera's field of view.
As expected, this causes the posterior probability of the mug's existence to decrease to 0.31.
In the third scenario (Figure~\ref{fig:results}(e)), we move the box closer to the camera so that it occupies more of the field of view.
The posterior probability of the mug's presence then increases to 0.63.
These experiments illustrate the dependence between one object's pose and another object's existence, which is a consequence of occlusion.
