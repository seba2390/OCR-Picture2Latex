%\documentclass[aps,pre,preprint]{revtex4}
\documentclass[aps,reprint,twocolumn]{revtex4-1}
%\documentclass[preprint,...]{revtex4-1} 
%\documentclass[reprint,...]{revtex4-1}
%%
\usepackage{amsmath}
\usepackage{graphicx,color}
\usepackage{booktabs}
\usepackage{datetime}
\definecolor{navyblue}{rgb}{0.0,0.0,1}
\usepackage{xcolor}
\usepackage{epstopdf}
\usepackage{wrapfig}
\usepackage{ragged2e}
\usepackage{epsfig}
\usepackage{slashed}
\usepackage{caption}
\usepackage{subcaption}
%\usepackage{siunitx}
%\usepackage[noadjust]{cite}
%\usepackage[natbib]{biblatex}
%\usepackage{citep}
%\DeclareOldFontCommand{\rm}{\normalfont\rmfamily}{\mathrm}
%\DeclareOldFontCommand{\sf}{\normalfont\sffamily}{\mathsf}
%\DeclareOldFontCommand{\tt}{\normalfont\ttfamily}{\mathtt}
%\DeclareOldFontCommand{\bf}{\normalfont\bfseries}{\mathbf}
%\DeclareOldFontCommand{\it}{\normalfont\itshape}{\mathit}
%\DeclareOldFontCommand{\sl}{\normalfont\slshape}{\@nomath\sl}
%\DeclareOldFontCommand{\sc}{\normalfont\scshape}{\@nomath\sc}
%\DeclareRobustCommand*\cal{\@fontswitch\relax\mathcal}
%\DeclareRobustCommand*\mit{\@fontswitch\relax\mathnormal}
%\setcounter[page]{321}


% COLOR %%%%%%%%%%%%%%%%%%%%%%%%%%%%%%%%%%%%%%%%%%%%%%%%%%%%%%%%%%
\def\NOTE#1{{\textcolor{red}{\bf #1}}}   % note
%\def\DEL#1{{\textcolor{red}{[#1]}}}      % suggested deletions
%\def\ADDA#1{{\textcolor{blue}{#1}}}        % addition (Referee 1)
%\def\ADDB#1{{\textcolor{green}{#1}}}       % addition (Referee 2)
%\def\CH#1{{\textcolor{blue}{#1}}}       % blue change
%\def\CH#1{{{#1}}}                      % remove blue
\def\CHG#1{{\textcolor{blue}{#1}}}
%%%%%%%%%%%%%%%%%%%%%%%%%%%%%%%%%%%%%%%%%%%%%%%%%%%%%%%%%%%%%%%%%%
\begin{document}
\title{Wake effects of a  stationary charged grain in streaming magnetized ions}
%\title{Interplay of grain size and separation on inter-grain interaction in streaming non-Maxwellian plasma}


%\author{a b c}

\author{Sita Sundar}
%, Hanno K{\"a}hlert, J.-P. Joost, Patrick Ludwig, and Michael Bonitz}
%\author{SS, MKV, AGC, and AA}

\affiliation{Department of Aerospace Engineering, Indian Institute of Technology Madras, Chennai - 600036, India}
%Institut f{\"u}r Theoretische Physik und Astrophysik, Christian-Albrechts-Universit{\"a}t zu Kiel, Leibnizstra{\ss}e 15, Kiel 24098, Germany} 
%\affiliation
\begin{abstract}
A systematic numerical study of wake potential and ion density distribution of a single grain in streaming ions 
%  modification
%wake downstream grain 
under the influence of the magnetic field applied along flow is presented. 
%Wake size and strength manifests itself in a variety of forms for weakly and strongly magnetized plasma.  
Strong magnetic field introduces ion focus depletion behind grain facilitating the entrance of  electrons far away in the downstream towards the grain. 
It is shown that the magnetic field suppresses the amplitude of wake potential and modifies the ion density distribution substantially.
The wake peak potential and  position characteristics, and 
density distribution of plasma constituents  in the 
presence  of magnetic field and charge-exchange collisions  for the subsonic, sonic, and supersonic regime is also delineated.   In the subsonic regime, simulations demonstrate the accumulation of ions near  dust grain in the transverse direction while complete suppression of oscillations in the transverse direction  takes place for sonic and subsonic regime.
% for grain in magnetized ion flow.
%By applying a magnetic field, both the ion density distribution and wake potential changed greatly, demonstrating that grain in plasma system can be controlled by applying a magnetic field.
%present the induced density profile with and without magntic field and discuss
%The impact and interplay of parallel magnetic field with 
%%collisionality and 
%streaming speed on the wake potential and ion density distribution in the presence of . 
%and creates an ion depletion region downstream grain. 
\end{abstract}
\maketitle
%upto here 
\section{Introduction}
Dusty plasma is ubiquitous in nature and laboratory plasmas i.e. spokes of Saturn rings, interplanetary dust, charged ice particles near moon, 
noctilucent clouds in Earth's atmosphere, Fusion devices etc. are various manifestations of dusty plasmas. Due to their heavy mass and the ability
to acquire high charge,  presence of dust particles in plasma makes them responsible for a variety of new novel phenomena which has been widely reported~\cite{Morfill:RMP2009,Melzer:WVV2008, Bonitz:Book2010}.
%~\cite{Morfill:RMP2009, Bonitz:Book2010, melzer2008fundamentals,bonitz_2010_complex}.
 Recent advances in scientific tools and technology has led to a surge in the interest of scientific community towards phenomena in dusty plasma regime which were hidden hitherto.
One among these is the study of impact of magnetic field on the dynamics of grain in streaming magnetised plasma. Magnetic field influences the behavior of charged plasma particles i.e. ions, electrons, and hence affects the overall dynamics of the system. It is well known that the influence of magnetic field introduces anisotropy in plasmas.  Dust particulates are heavy and it takes comparatively higher strength of magnetic field to make the grain magnetized.   It is pertinent to ask about the role of magnetic field on dusty plasma phenomena especially the exciting wake field features reported for the case of grains in streaming ions in the sheath region. Recently some experiments are being conducted and a few theoretical/numerical work in this regard has been reported~\cite{Edward:IEEE2013,Edward:POP2016,Miloch:JPP2014,Joost:PPCF2015}. 
%These studies mostly dealt with ....

The study of grain in magnetized ion flow began with the work by Nambu et.al.~\cite{Nambu:PRE2001}. They  provided the impact of magnetic field on the wake features for grain in streaming ions. In their paper, using analytical methods, they described that the role of magnetic field is to damp the strength of wakefield due to the reduction in ion overshielding around grain. In another work, Shukla et. al.~\cite{Shukla:PLA2001} presented the effect of ion polarization drift on  dynamical potentials and shielding  for magnetized plasmas.
Samsonov et.al.~\cite{Samsonov:NJP2003} discussed the impact of magnetic field on complex plasma from a different viewpoint. They demonstrated the levitation and agglomeration of magnetic grains in a complex plasma, and also envisaged the possibility of magnetically induced plasma crystal formation. Interest of wider scientific community in magnetized dusty plasma stemmed from these works.
%With this interest in magnetized dusty plasma sprang up.
%This led to the increase in interest of wider scientific community to look for the impact of magnetc field on dusty plasmas.

Around the same time, Yaroshenko et.al.~\cite{Yaroshenko:NJP2003} described the fine details of mutual interactions of magnetized particles in complex plasmas. They presented that the dipole short-range force is the reason behind  the formation of field-aligned individual particle containing chains often observed in experiments.
% Such chains may incorporate a few tens of individual particles, as frequently observed in experiments.
Carstensen et.al.~\cite{Carstensen:PRL2012} presented the description of the inter-particle forces mediated by ion wakes in the presence of a strong magnetic field aligned along the ion flow.  Their observation was a decay in the interaction force with increasing magnetic field strength. They provided the reasoning for the decay at a critical parameter range where the ion cyclotron frequency is higher than the ion plasma frequency.


Recent preliminary results from MDPX experiments by Thomas et.al.~\cite{Edward:IEEE2013, Edward:POP2016} has revived the interest and aim of dusty plasma physicists towards the study of grain in magnetic field. They discussed the formation of ordered structures, properties of dust density waves, and filament generation with and without magnetic field.  The work relevant to the physics of wake formation in weakly and strongly magnetized plasmas is yet to be explored. 
One numerical work regarding wake formation for grain in streaming magnetized ions has been presented very recently by Miloch et.al.~\cite{Miloch:JPP2014} where they demonstrated that the wake size and strength can be significantly affected by the presence of the magnetic field for both stationary and streaming ions.

Studies by Joost et.al.~\cite{Joost:PPCF2015} were done for grain in magnetized ions using Linear Response (LR) formalism and they reported damping of wake-field with increasing magnetization.  
%One of the  limitations of the LR framework is symmetry breaking due to applied  magnetic field. 
Note that  they studied the wake-field for a moving grain in stationary ions. This is not equivalent to the case of stationary grain in streaming ions because here we have magnetic field which introduces anisotropy. The velocity components are not isotropic in all the three directions and one should not substitute the dielectric function for stationary grain in streaming magnetized ions with streaming grain in stationary ions.
Nevertheless, the study served as a fruitful attempt to explore the effect of a magnetic  field on the screening potential.
% within a kinetic framework.
% and the system no longer remains equivalent to the . 
In a recent work,  study for grain wakefield and induced charge density and other parametric dependence has been done~\cite{Ludwig:EPJD2017}  wherein they compared LR results with Particle-in-Cell (PIC) Simulations and have shown that the results have qualitatively similar characteristics. 

In the present work, we are going to study the impact of magnetic field on the wake-field formed downstream a single grain due to streaming ions, and their eventual impact on
% induced 
density distribution around grain. A schematic of the system considered is presented in Fig.~\ref{fig:Figure1}. Here, we depict the streaming of magnetized ions past grain and focusing downstream grain. The force due to magnetic field alters the configuration substantially by introducing gyration and crossed magnetic drifts.
%%In the absence of magnetic field\\
%\begin{figure}
% \includegraphics[scale=0.45, trim = 8cm 6cm 1cm 5cm, clip =false, angle=0]{ion_streaming_mag} 
%\caption{Schematic  depicting the ion streaming past grain and focusing downstream eventually leading to wake formation  for non-zero magnetic field applied parallel to the ion flow.}
%\label{fig:figure1}
%\end{figure}
\begin{figure}
 \includegraphics[scale=0.45, trim = 9.0cm 6cm 1cm 5cm, clip =false, angle=0]{Figure1} 
\caption{Schematic  depicting the ion streaming past grain and focusing downstream eventually leading to wake formation  for non-zero magnetic field applied parallel to the ion flow.}
\label{fig:Figure1}
\end{figure}
%
%\begin{figure}
% \includegraphics[scale=0.45, trim = 8cm 6cm 1cm 5cm, clip =false, angle=0]{test_mag6} 
%\caption{Schematic  depicting the ion streaming past grain and focusing downstream eventually leading to wake formation  for non-zero magnetic field applied parallel to the ion flow.}
%\label{fig:figure1}
%\end{figure}
We know that ion focusing takes place downstream grain for the case of a stationary grain in streaming ions. 
%But this scenario gets modified if one applies magnetic field along the flow direction. 
But if one applies magnetic field along the flow direction, the streaming ions get magnetized (and dust also depending on the strength of the magnetic field applied) and their behavior along parallel and transverse directions are altogether different. This symmetry breaking can play a crucial role in the ion focusing downstream grain and  lead to significant modifications in  the grain-plasma dynamics as well as wake effects. 
 

 In the presence of magnetic field, the ions passing far away from the grain (i.e. with large impact parameter) stream the way they used to in the absence of magnetic field nevertheless with gyration. Some of these streaming ions move past grain  eventually leading to ion focusing and wake effects in the downstream.  For ions streaming nearby grain, due to the influence of magnetic field and field due to negatively charged grain and wake potential, one comes across three different motions.  One among them is flow to due to  $dc$  field in the sheath which leads to accelerating flow, $v_E$. The second one is $v_c$, the gyrating motion around magnetic field. The third important force flow is $v_d$ resulting due to cross drift of  $E_{grain}$ or wake field and $B$ field (since external $E$ and $B$ applied are along the flow, they don't lead to any $E\times B$ drift). This $v_d$ shifts the ion focus behind grain and creates an ion depletion region through which electrons far away in the downstream region stream towards the grain. It has been observed in the present work that this ion depletion region has a strong dependence on the amplitude of magnetic field as well as streaming ion speed.
 
Here, the fact which also needs to be  emphasized is the role played by collisions. In the absence of collisions, particles do not diffuse in the perpendicular direction, rather they keep gyrating about the same line of force. The possibility of drift on these particles across $\vec{B}$ could be incurred due to electric fields or gradients in $\vec{B}$.  However, collisions assist the  particle diffusion  across $\vec{B}$ by a  random-walk process.  %\textit{
Whenever an ion suffers a collision with a neutral atom, it undergoes a change in direction.
%ion leaves the collision travelling in a different direction.
 It keeps gyrating about the magnetic field in the same direction, but with  change in phase of gyration and the guiding center shifts its position.
 % in a collision. 
%and undergoes a random walk. 
 The particles can then diffuse in the direction opposite to the density gradient. 
 %} 
The magnitude of Larmor radius $r_L$ is the governing factor in determining the scale length of the random walk. This supplies us with the knowledge that one needs to maneuver the strength of $\vec{B}$ to control the diffusion across $\vec{B}$, and hence, other manifesting magnetized dusty plasma phenomena like coherent structure and wake formation.
% The step length in the random walk is no longer $\lambda_m$, as in the magnetic-field-free diffusion but has instead the magnitude of the Larmor radius $r_L$.
%  Diffusion across $\vec{B}$ can therefore be slowed down by decreasing $r_L$ i.e. by increasing $\vec{B}$. }


The outline of the paper is as follows. In  Sec.II, we introduce the simulation scheme utilized and present  the description of methodology. %distribution functions chosen and its physical interpretation.
 In Sec.III, we present the systematic results regarding the impact of magnetic field on the grain in streaming ions.
 % and discuss their implications. 
Finally, we 
%discuss the relevance of the present study to experiments and 
present a summary and conclusion in section IV followed by acknowledgments in section V.

\section{Numerical Details}

The equation to delineate the ion dynamics in six-dimensional phase space in the presence of the self-consistent electric field $-\nabla \phi$,  an optional external force {\bf D}~\cite{Hutch:POP2013} and an external magnetic field is given by
\begin{equation}
 {m_i} \frac{d \mathbf{v}}{dt} = e \left[- \nabla {\phi} + \mathbf{v}  \times \mathbf{B} \right]+ \mathbf{D} .
\end{equation}
Here, $\bf{B}$ denotes the applied magnetic field and is aligned along the direction of ion flow in our simulations.
For  the {\it shifted Maxwellian distribution}, this extra force $\bf D$ is zero for most of our simulations, and ions are driven solely by a flow of neutrals. 
Simulation is performed with three-dimensional Cartesian mesh, oblique boundary, particles and thermals in cell (COPTIC) code~\cite{Hutch:POP2011}. COPTIC is a hybrid PIC code in a sense that electron dynamics are governed by the Boltzmann description, $n_e = n_{e \infty}\exp(e\phi/T_e)$, whereas ion dynamics are considered in six-dimensional phase space in the presence of the self-consistent electric field and optional external fields.  

The simulation set-up is similar to the one considered in our recent paper~\cite{Sundar:POP2017} except that we have an extra magnetic field aligned along the flow to take care of.
%incorporated magnetic field here. 
Further numerical detail and fundamentals can be gleaned from the paper with descriptions about COPTIC~\cite{Hutch:POP2011, Hutch:POP2013}.
%On the other hand, for the {\it drift-driven distribution} (non-Maxwellian distribution), the neutrals are stationary, and the non-zero force field {\bf D} is responsible for the ion drift. The force {\bf D} thus represents the effect of the background electric field. However, it is not taken into account in the calculation of the electron Boltzmann factor. This preferred choice leads to a uniform electron density in the absence of the dust grain, see also the discussion in Ref.~\cite{Hutch:POP2013}.

%While the simulations are performed in 3D Cartesian coordinates, the grain potential $\phi(r,z)$ has a cylindrical symmetry around the flow, which is here along the positive $\hat{z}$ direction. In order to improve statistics, the full three-dimensional potential from the simulation is averaged over the cylindrical angle. The radial coordinate $r=\sqrt{x^2+y^2}$ is limited by the box size and given by $\sqrt{2}\,l_b$, where $l_b$ is the box half-width. For $r\gtrsim l_b$, the grain potential is increasingly affected by the finite box size.~\cite{Hutch:PRE2012}
%In this work, 
% the component of potential gradient along $\hat{z}$ direction is set  to be zero on the computational boundary.
%For the other two directions, the potential gradient is set to be zero in the $M\hat{z}+\hat{r}$ direction,   where $\hat{r}$ is the radial direction~\citep{Hutch:PRE2012}.

To perform simulations, we chose a cell grid  of $64\times64\times128$ with more than 60 million ions and grid side length of $8\times8\times20$ Debye lengths. We performed few simulations with grids of even higher resolution and non-uniform mesh spacing to resolve the dynamics in the near neighborhood of the grain~\cite{Hutch:POP2011}.
% as well as more than 80 million ions. 
Normalizations for the  length scale, velocity  and other physical variables  follow the standard normalization described in the  paper by Hutchinson et al.~\cite{Hutch:POP2011}, i.e., the space coordinate is normalized as $r\rightarrow r/r_0$,  velocity as $v \rightarrow v/c_s $, and potential as $\phi \rightarrow \phi/(   T_e/e)$, where $r_0=( \lambda_{De}/5)$ is the normalizing scale length and $c_s$ is unity in normalized units. The collision frequency $\nu$ is normalized in the time units as $\nu/(c_s/r_0) \sim 0.2(\nu/\omega_{pi})$, where $\omega_{pi}$ is the ion plasma frequency and the Debye length is fixed at $5 r_0$.  We  define ion Mach number $M$  in terms of the thermal Mach number $M_\text{th}$ as $M=v_d/c_s=\sqrt{T_i/T_e}\, M_\text{th}$, where $c_s=\sqrt{T_e/m_i}$ is the ion sound speed and $M_\text{th}=v_d/v_\text{th}$ is the thermal Mach number. Time-advancement of the simulation run till it reaches steady-state which is usually 1000 time-steps for the cases considered herein.  The analytical radius for point-charge sphere is chosen as $r_a=0.1 \lambda_{De}$. All the simulation parameters are summarized in Table~\ref{table:TABLE I}.
%
%In this work, we have considered only point-charge grains using the particle-particle-particle-mesh (PPPM) scheme~\cite{Hockney:Book1988} implemented in COPTIC~\citep{Hutch:POP2011}. Here, the grain potential is treated analytically in a sphere of a predefined radius $r_a$ around the grain shielded by opposite charge cloud~\cite{Hutch:PPCF2013}.  The field due to the analytic potential is zero outside this predefined radius. 
%The remaining potential is represented on the grid by solving the discrete Poisson equation. The advantage of this method is that the mesh does not have to resolve the grain potential near the grain where it is represented in analytic form, thus allowing for a coarser mesh without sacrificing accuracy~\citep{Hutch:POP2011, Hutch:PPCF2013}. 

Incorporation of collisions are carried out in accordance to Poisson statistical distribution with fixed velocity-independent collision frequency i.e. similar to the BGK-type collisions. We have considered mainly the charged-exchange collision which is the dominant one in the system considered here, and is incorporated in the code by mutual interchange of velocity of the colliding ion with that of the velocity of neutral chosen randomly from the neutral velocity distribution. Magnetic field considerably modifies the dynamics by changing the wakefield potential amplitude, number of wake peaks behind grain and pattern of oscillations behind the grain, and  is discussed in sec.~\ref{sec:Res}.

%Collisions are incorporated according to a 
%% the same assumption as is used in the derivation of Eq.~\eqref{eqn:drift} 
%%(BGK-type collisions). Charge-exchange collisions of ions with neutrals are performed by exchanging the velocity of the colliding ion with the velocity of a neutral chosen randomly from the (uniform) neutral velocity distribution~\cite{Hutch:POP2011}. 
% As we will see in sec.~\ref{sec:Res}, magnetic field substantially alters the dynamics by changing the wakefield  potential amplitude as well as the number and pattern of oscillations behind the grain.

\begin{table}[h]
\caption{Detailed list of the simulation parameters.}. 
\label{param_table}
\setlength{\tabcolsep}{7 pt}
\hspace*{-1cm}
\begin{tabular}{l l  }
\toprule[0.80pt]
\hline
\hline
 Magnetization, $\beta$ & 0.0-2.0\\
 Temperature ratio, $T_e/T_i$ & 100 \\
  Mach Number, $M$ & 0.5 - 1.5 \\
  Collision frequency, $\nu/\omega_{pi}$  & 0.002  \\
  Electron Debye length, $\lambda_{De}$  & 5 \\
  grid size, & $64 \times 64 \times 128$ \\
  number of particles, & $60 \times 10^6$\\
  total number of time steps, & 1000 \\
  Grain potential, $\phi_a$  & 0.05-0.2 \\
  Time-step, $dt$ & 0.1 \\
  Normalized grain charge, $\bar{Q}_d$ & 0.01\\ 
 
\hline
\hline
\bottomrule[0.80pt]
\end{tabular}
\label{table:TABLE I}
\end{table}
\section{Results}\label{sec:Res}
We have performed a systematic study of the wake potential as a function of both, the Mach number and the magnetization in the presence of 
 collisions. The electron-ion temperature ratio was fixed at $T_e/T_i =100$.  To delineate the differences in the wake potential for the unmagnetized and magnetized ion cases,  we performed simulations for parameters in the range $M=v_d/c_s=0.5-1.5$ and $\beta$ = 0.0 - {2.0} for $\nu=0.002$, where $\beta$ represents the magnetization parameter defined as the ratio of ion cyclotron frequency and ion plasma frequency, $\beta=\omega_{ci}/\omega_{pi}$.
% Thermal Mach Number, $M_\text{th}=v_d/v_\text{th}$ the thermal Mach number, where $v_d=q E_0/(m\nu_\text{in})$ is the drift velocity (ion charge $q$) .
\begin{figure*}
%    \centering
%    \begin{subfigure}[b]{0.3\textwidth}
%        \includegraphics[scale=0.35, trim = 0cm 7cm 1cm 6cm, clip =true, angle=270]{plot0002}
%        \caption{}
%        \label{fig:a}
%    \end{subfigure}
%    ~ %add desired spacing between images, e. g. ~, \quad, \qquad, \hfill etc. 
%      %(or a blank line to force the subfigure onto a new line)
%    \begin{subfigure}[b]{0.3\textwidth}
%        \includegraphics[scale=0.35, trim = 0cm 7cm 1cm 6cm, clip =true, angle=270]{plot0004}
%        \caption{}
%        \label{fig:b}
%    \end{subfigure}
%    ~ %add desired spacing between images, e. g. ~, \quad, \qquad, \hfill etc. 
%    %(or a blank line to force the subfigure onto a new line)
%    \begin{subfigure}[b]{0.3\textwidth}
%        \includegraphics[scale=0.35, trim = 0cm 7cm 1cm 6cm, clip =true, angle=270]{plot0008}
%        \caption{}
%        \label{fig:c}
%    \end{subfigure}
%    \begin{subfigure}[b]{0.3\textwidth}
%        \includegraphics[scale=0.35, trim = 0cm 7cm 1cm 6cm, clip =true, angle=270]{plot0012}
%        \caption{}
%        \label{fig:d}
%    \end{subfigure}
%    ~ %add desired spacing between images, e. g. ~, \quad, \qquad, \hfill etc. 
%      %(or a blank line to force the subfigure onto a new line)
%    \begin{subfigure}[b]{0.3\textwidth}
%        \includegraphics[scale=0.35, trim = 0cm 7cm 1cm 6cm, clip =true, angle=270]{plot0014}
%        \caption{}
%        \label{fig:e}
%    \end{subfigure}
%    ~ %add desired spacing between images, e. g. ~, \quad, \qquad, \hfill etc. 
%    %(or a blank line to force the subfigure onto a new line)
%    \begin{subfigure}[b]{0.3\textwidth}
%        \includegraphics[scale=0.35, trim = 0cm 7cm 1cm 6cm, clip =true, angle=270]{plot0018}
%        \caption{}
%        \label{fig:f}
%    \end{subfigure}
\includegraphics[height=18cm,width=15cm, trim = 6.9cm 6.5cm 1.9cm 5.9cm, clip =true]{Figure2_new}
    \caption{Wake potential contours $e \phi / T_e$, averaged over the azimuthal angle, for various strengths of magnetic field :
    (a) $\beta=0.05$; (b) $\beta=0.1$, and (c) $\beta=1.0$ (from left to right) with streaming velocity $M=0.5$ (top row), $M=1.0$ (middle row), and $M=1.5$ (bottom row) for the shifted Maxwellian case.
    }\label{fig:Figure2}
\end{figure*}
\subsection{Interplay of magnetization and ion streaming speed on wakefield potential}
Impact of ion streaming speed and ion-neutral charge-exchange collision has been widely reported in earlier works~\cite{Hutch:POP2013, Sundar:POP2017, Ludwig:NJP2012}.  
At first, we investigate in detail the impact of magnetzation on wakefield produced downstream grain due to ion focusing or depletion.  In Fig.~\ref{fig:Figure2}, we present wake potential contours, for various magnetizing stregths, for the shifted Maxwellian distribution with moderate to high streaming speeds. Similar to the observations made in linear response calculations~\cite{Joost:PPCF2015}, here also, we  observe that the role of magnetization is to reduce the amplitude of wakefield oscillations behind the grain.
%, for moderate to high Mach numbers. 
%In Fig.~\ref{fig:Figure2}, 
%With nine subplots we show here the impact of magnetization on wake potential by varying the strength of the magnetic field.
%In Fig.~\ref{fig:Figure2}, we present the wake potential contours for varying strengths of magnetic field with ions streaming in the subsonic, sonic, and supersonic regimes.
We start with the regime of very small magnetization, $\beta=0.05$, Fig.~\ref{fig:Figure2} (left column). The magnetization strength is meagre and the difference from unmagnetizaed case, cf. Fig.~\ref{fig:Figure5}~\cite{Sundar:POP2017}, is inconspicuous and its wake potential profile is similar to that of the unmagnetized case for all the Mach numbers considered. However, due to the presence of magnetic field one can notice  a very slight modification  appearing in the wake potential at the far end of the grain  in the downstream.
%  can be noticed.
% In subplot (a), the strength of the magnetic field is meagre, and its wake potential profile is similar to that of the unmagnetized case. 
% In subplot (b), we see the small modification at the far end of the wake due to magentic field. 

 As we further increase the magnetic field strength, we see the growing impact of magnetization on the wake potential profile. For moderate magnetization, $\beta=0.1$, Fig.~\ref{fig:Figure2} (middle column), streaming of electrons in the upstream starts at the far end of the grain in the downstream and traverses through the ion focus. The impact of magnetization is stronger for higher ion streaming speeds. It can be observed that the role of streaming speed is to elongate the wake oscillation wavelength,  and that of the magnetic field is to suppress the wake amplitude.
 For smaller value of applied magnetic field, the depletion of ion focus starts from the far end of the wake oscillations downstream grain and this depletion region moves slowly towards the grain with increasing magnetic field strength eventually facilitating the entrance for faraway electrons in the downstream to move upstream towards the grain.

 At even higher magnetization strengths $\beta=1.0$, Fig.~\ref{fig:Figure2} (right column), the change in wake feature is prominent even in subsonic regime. Here, electrons are able to propagate upstream with less obstruction through the ion focus region and reach the grain. For subsonic regime, $M=0.5$ (see subplot(c) Fig.~\ref{fig:Figure2}), we see the potential is ``bent" towards the grain and is qualitatively similar to the  results observed in LR calculations~\cite{Joost:PPCF2015}. However, for higher ion streaming speeds, i.e. $M=1.0$, we observe that the potential is somehow compressed onto the streaming axis and the wake potential contour along the direction of flow is bell shaped (see subplot(h)  Fig.~\ref{fig:Figure2}) with very small density of ions in the transverse direction near the grain.
In the supersonic regime i.e. $M=1.5$ increasing the strength of magnetization, cf.  Fig.~\ref{fig:Figure2} subplot (i),  leads to better penetration of ion focus region by electrons eventually leading to stronger streaming of electrons upstream towards grain. The potential contour is no longer  limited to bell shape rather like a flattened rod completely wiping out the ion focus in the vicinity of the grain. 
 
To understand the interplay of streaming speed with magnetization strength on the wakefield profile, let us revisit Fig.~\ref{fig:Figure2} from the perspective of streaming speed. Even a small amount of magnetization initiates the motion of electrons upstream towards the grain and ions are pushed in the transverse direction, at higher Mach numbers. In the unmagnetized case, upstream propagation of any disturbance is suppressed for the supersonic ion speed. However, for the strongly magnetized case with ions flowing at supersonic speed, we see the strong upstream propagation of electrons.
 % which one intuitively expects not to go upstream in the supersonic flow. 
 %On the contrary, t
 %
\begin{figure}
\includegraphics[scale=1,trim = 0cm 0cm 0cm 0cm, clip =true]{Figure3}
\caption{Wake potential along the streaming axis  for $M=1$ ($M_{th}=10$) and $B_z=0.05$ (blue dashed line), $B_z=0.1$ (red dotted line), and $B_z=1.0$ (green solid line). 
}
\label{fig:Figure3}
\end{figure}

In Fig.~\ref{fig:Figure3}, the wake potential variation along magnetization (and streaming direction) is shown for three different values of magnetization strength, $\beta=0.1, 0.5$, and $1.0$.  The  grain is at the origin and the normalized grain charge is $\bar{Q}_d= 0.01$. Here, one can observe clearly that the increasing strength of magnetic field decreases the amplitude of wake oscillations downstream grain. At higher streaming speed, when magnetization strength is such that the ion gyro-frequency is equal to or greater than the ion plasma frequency, the wake oscillations positive peak is completely subdued. Note that, it has been envisaged in LR simulations~\cite{Ludwig:NJP2012}, and is not a numerical artifact. The new features that we see now due to higher resolution and contribution of nonlinear features, which has not been observed in LR simulations, is that the electrons are more streamlined and move upstream at higher magnetic field strengths which was forbidden in the case without magnetic field.

\begin{figure}
\includegraphics[scale=1,trim = 0cm 0cm 0cm 0cm, clip =true]{Figure4}
\caption{Wake potential transverse to the streaming axis  for $M=0.5$ ($M_{th}=5$) and $B_z=0.05$ (blue dashed line), $B_z=0.1$ (red dotted line), and $B_z=1.0$ (green solid line). }
\label{fig:Figure4}
\end{figure}
The wake potential variation in the direction transverse to the magnetization (and ion flow) is shown in Fig.~\ref{fig:Figure4}  for three different values of magnetization strength, $\beta=0.1$, $0.5$, and $1.0$ for ions streaming in subsonic regime, $M=0.5$. Transverse to the streaming direction, we see that with increasing magnetization, oscillation amplitude increases in the subsonic regime. This is due to the fact that the ions streaming in closer proximity of the grain are scattered and an ion depletion is created behind the grain.  Some of these scattered ions assemble in the vicinity of the grain in the transverse direction manifesting transverse oscillations.

\begin{figure}
\includegraphics[scale=1,trim = 0cm 0cm 0cm 0cm, clip =true]{Figure5}
\caption{Wake potential  transverse to the streaming axis  for $M=1$ ($M_{th}=10$) and $B_z=0.05$ (blue dashed line), $B_z=0.1$ (red dotted line), and $B_z=1.0$ (green solid line).  }
\label{fig:Figure5}
\end{figure}
In Fig.~\ref{fig:Figure5}, the wake potential variation in the direction transverse to the magnetization (and streaming direction) is shown for magnetization strengths, $\beta=0.1, 0.5$, and $1.0$ with ions in the sonic regime, $M=1.0$.
At higher streaming speeds, there is no more accumulation of ions in the transverse direction. Streaming ion flow is strong enough to make the ions overcome the electrostatic potential barrier due to grain and wake potential, and hence is able to sweep them completely from the grain proximity.

 \begin{figure}
 \includegraphics[scale=0.85, trim = 0cm 0.0cm 0cm 0.0cm, clip =true, angle=0]{Figure6} 
\caption{Variation of the maximum of the  peak position (left subplot) and peak amplitudes (right subplot) of the wake potential as a function
of magnetic field strength in normalized units for the shifted Maxwellian distribution at $\nu/\omega_{pi}=0.002$.}
\label{fig:Figure6}
 \end{figure}
 
We present the variations of the maximum peak position and the peak height with magnetization, for various streaming speeds, $M=0.5, 1, 1.5$ in Fig.~\ref{fig:Figure6}.
In accordance with previous reported results by Joost et. al.~\cite{Joost:PPCF2015}, here also, we see a decline in peak position and amplitude of peak height with increasing strength of magnetization. For subsonic regime, ion focusing occurs very near to the grain due to its longer interaction time with grain and smaller kinetic energy. As the streaming speed increases, the ions have higher kinetic energy to move past the grain to a farther location and hence the peak position for higher streaming speeds is farther than the subsonic ion flow. The trend of the peak position with streaming speeds is such that it increases with increasing Mach number.
%In contrast to previous results by Joost et. al.~\cite{Joost:PPCF2015}, here
We observe that the wake potential amplitude for subsonic regime is higher than that for supersonic regime for the case of magnetized ion flow. 
%The wake potential, in fact, exhibits a decreasing pattern with increase in streaming speed. 
Increasing magnetization strength doesn't change the wake peak position significantly for ions streaming at subsonic speeds. However, we notice that the wake peak position shifts closer to grain with increasing magnetization strength for supersonic ion flows.
 

\subsection{Impact of magnetization strength on the ion density distribution}
\begin{figure*}
 \includegraphics[scale=0.68, trim = 0cm 0cm 0cm 0cm, clip =true, angle=0]{Figure7} 
\caption{Spatial profiles of the ion density (normalized to the distant unperturbed ion density), averaged over the azimuthal angle, for various strengths of magnetic field : (a) $\beta=0.05$, (b) $\beta=0.1$,  (c) $\beta=0.15$,  and (d) $\beta=1.0$ with streaming velocity $M=0.5$.
% for the shifted Maxwellian case. 
%Density contours $e n_i/ T_e$, averaged over the azimuthal angle, for various strengths of magnetic field : (a) $\beta=0.1$; (b) $\beta=0.15$, (c) $\beta=1$, and (d) $\beta=2$ with streaming velocity $M=0.5$ for the shifted Maxwellian case.  Here,  the grain is at the origin and the normalized grain charge is $\bar{Q}_d= 0.01$.
%%Variation of the maximum of the  peak position (left subplot) and peak amplitudes (right subplot) of the wake potential as a function
%%of magnetic field strength in normalized units for the shifted Maxwellian distribution at $M=1$ ($M_{th}=10$) and $\nu/\omega_{pi}=0.002$.
}
\label{fig:Figure7}
 \end{figure*} 
In Fig.~\ref{fig:Figure7}, we present the density contour for various magnetizing strengths with streaming speed $M=0.5$. For very small magnetic field strength, it exhibits the pattern as one observed in the unmagnetized ion case~\cite{Sundar:POP2017, Ludwig:NJP2012}. As   the strength of the magnetic field increases, the density downstream grain develops as candle flame structure similar to the induced density density distribution reported by Zhandos et.al.~\cite{Zhandos:arxiv2017}. Further increase in the strength of the field leads to elongation and broadening  of the candle flame shaped ion density distribution downstream grain. Basically, what we see as  elongated structure is the ion depletion created behind grain due to the strong magnetic field applied along flow.
%downstream grain develops as candle flame structure similar to that reported by Zhandos et.al.~\cite{Zhandos:POP2017}. Further increase in the strength of the field leads to elogation and broadening  of the candle flame structure downstream grain. 

 \begin{figure*}
 \includegraphics[scale=0.68, trim = 0cm 0cm 0cm 0cm, clip =true, angle=0]{Figure8} 
\caption{
%Spatial profiles of the ion density (normalized to the distant unperturbed ion density), for a Mach number M = 0.8, for a shifted Maxwellian (top row) and a non-Maxwellian drift distribution (bottom row), for a collision frequency of ν/ωpi = 0.02 (left column) and ν/ωpi = 0.2 (right column). Spatial scales r and z axes are in normalized units (λDe/5).
%Density contours $e n_i / T_e$, 
Spatial profiles of the ion density (normalized to the distant unperturbed ion density), averaged over the azimuthal angle, for various strengths of magnetic field : (a) $\beta=0.05$,  (b) $\beta=0.1$,  (c) $\beta=0.15$,  and (d) $\beta=1.0$ with streaming velocity $M=1.0$.
% for the shifted Maxwellian case.  %Here,  the grain is at the origin and the normalized grain charge is $\bar{Q}_d= 0.01$.
%Variation of the maximum of the  peak position (left subplot) and peak amplitudes (right subplot) of the wake potential as a function
%of magnetic field strength in normalized units for the shifted Maxwellian distribution at $M=1$ ($M_{th}=10$) and $\nu/\omega_{pi}=0.002?$.
}
\label{fig:Figure8}
 \end{figure*}
We present the density contour for various magnetizing strengths in the sonic regime $M=1.0$ in Fig.~\ref{fig:Figure8}. For very small magnetic field strength, it exhibits the pattern as one observed in the unmagnetized ion case as usual. However, as one increases the strength of the magnetic field, the density contour exhibits an altogether different pattern. It evolves as low density candle flame structure moving from far end of the wake towards the grain with increasing magnetization strength. Elongation of ion depletion region is stronger in this case. This can be understood with our wake potential  profile explanation wherein we see the streaming of electrons from the far end propagating upstream towards the grain.



%\section{Relevance to Experiments}
%\section{Discussion}


\section{Conclusion}

%We have presented here a comparative study of grain charge and potential dependence on magnetic field using shifted Maxwellian distribution. 
%% Under typical experimental situations, the interaction potential  is generally substantially anisotropic and also nonreciprocal due to the presence of plasma flow [2,4,24].  
In the present work, we have investigated the electrostatic potential distribution around a point-like charged grain in a streaming plasma in the presence of magnetic field applied along the ion streaming direction for a shifted-Maxwellian ion distribution function.
%, for a wide range of
%%% ion-neutral collision frequencies and 
%Mach numbers. 
%We have presented a 
%%comparative 
%detailed study  using accurate 3D particle-in-cell simulations.
%The simulations presented here are based on certain assumptions, i.e., we have ignored the role of plasma  inhomogeneity, photoelectric emission, secondary electron emission, etc., which could be of importance for experimental systems. Nevertheless,
 We provide here an advanced numerical work by including the effect of an external electric and magnetic field with ion-neutral charge-exchange collisions on the wake potential and ion density distribution. The electric field imitates  the sheath region of electric discharges where the electric field makes the the ions accelerate towards the electrodes. 
% Its effect has been neglected in many previous works, e.g., Refs.~\cite{Lampe:POP2000, Hutch:POP2011,Ludwig:NJP2012}, but has received increasing attention in recent years, e.g.,~\cite{Lampe:POP2012, Hutch:POP2013,Hanno:POP2015, Kompaneets:PRL2016, KompaneetsPRE}. Compared to a situation where the ion flow is caused by a flow of neutrals, which gives rise to a shifted Maxwellian ion distribution, the inclusion of an external electric field yields a non-Maxwellian drift distribution. 
 The presence of magnetic field suppresses the wake amplitude formed downstream grain. Impact of streaming ion speed on these magnetized ion wake manifests in the wake flown away farther from the grain corresponding to the applied streaming speed strength. In the presence of magnetic field, even the  low-velocity ions alter  ion focusing behind the dust grain eventually modifying the the wake potential substantially. 
       
Using LR formalism, the study of the impact of magnetization on wake was attempted by Joost et. al.~\cite{Ludwig:NJP2012}. They discussed the limitations of the LR formalism regarding non-Maxwellian distribution and apparent symmetry breaking in the presence of neutrals. For small streaming speeds i.e. upto $M\sim 0.5$ our result is qualitatively similar to the results presented in the paper by Joost et. al.~\cite{Ludwig:NJP2012}. Our results also indicate that the potential is somewhat ``bent" towards the grain at high beta (Fig. 3). For $M>1$ the potential is somehow compressed onto the streaming axis. This resembles closely to the observations made by Joost et. al.~\cite{Ludwig:NJP2012}. 
However, there are few striking differences in the characteristic profile of wake potential which has its origin in the difference in the basic system configuration adopted in the two cases. As mentioned in the Section I, we have,  a stationary grain in streaming magnetized ions. On the other hand, Joost et. al.~\cite{Joost:PPCF2015} considered the case for grain moving in stationary plasma.

																	An important observation is damping of the wake potential with the magnetic field strength for the entire flow range and many novel features in the wake and density downstream grain.  The location of the wake peak maximum exhibited strong dependence on the Mach number as well as magnetic field strength. 
 % structure. 
 %For the entire flow range considered in this paper, we find damping of the wake potential for the shifted-Maxwellian distribution.
  For the smaller streaming speeds, as the magnetic field strength increases, we see a depletion in the ion charge density at the far end downstream grain which eventually facilitates the electrons to stream through the depleted positive space charge region towards the grain.  Upon further increase in  the streaming speed,  electron upstream flow is more pronounced and manifest in the form of bell-shape negative potential around grain wiping out the ion focus from the dust vicinity. 
  % i.e. for ions streaming past grain at supersonic speeds. 
The electrons begin to streamline for comparatively smaller value of magnetization. Nevertheless, the upstream propagation of electrons is observed to be significantly predominant for supersonic ion flows in stronger magnetic field. We have described  in detail a fairly qualitative physics for the grain-wakefield phenomena for streaming magnetized ions which will contribute to the understanding of inter-grain interaction for magnetized dusty plasmas and render assistance in revealing underlying physics in future.
%higher magnetization strength.

%The impact of magnetic field on the supersonic  ion flow is significantly greater than that for ions flowing in subsonic and sonic regime. 
%The electron streamline formation takes place .
%Particle attraction caused by ion focusing has been reported in many experiments, e.g.,~\cite{Hebner:IEEE2005,Kroll:POP2010,Melzer:PRE2014,Jung:POP2015}. In order to verify the collision-induced amplification of the wake potential experimentally, the ion-neutral collision rate must be sufficiently small. Our simulations show that only in this regime, an increase of the collisionality leads to amplification of the wake. The constraints on the ion Mach number are less severe since the effect occurs over a wide range. Nevertheless, it should be most pronounced for small $M$. Thus, the particle would ideally be located in a region with subsonic ion flow, i.e., in the presheath.

\section{Acknowledgments}
S. Sundar would like to thank I. H. Hutchinson for support in using the COPTIC code and acknowledge support of CAU Kiel. This work was supported by the DFG via SFB-TR24, project A9. Our numerical simulations were performed
at the HPC cluster of Christian-Albrechts-Universit{\"a}t zu Kiel.
S. Sundar would also like to thank M. Bonitz, H. K{\"a}hlert,  P. Ludwig, J.-P. Joost, and  Dr. Zh. Moldabekov for their help in scientific discussion.
%Our  numerical  simulations  were  performed at HPC rzcluster of Christian-Albrechts-Universitaet(CAU) zu Kiel.
%This work was supported
%by the DFG via SFB-TR24, Project A9.

%\newpage

%\bibliography{dusty_manus}

%\begin{thebi
%merlin.mbs apsrev4-1.bst 2010-07-25 4.21a (PWD, AO, DPC) hacked
%Control: key (0)
%Control: author (8) initials jnrlst
%Control: editor formatted (1) identically to author
%Control: production of article title (-1) disabled
%Control: page (0) single
%Control: year (1) truncated
%Control: production of eprint (0) enabled
\begin{thebibliography}{18}%
\makeatletter
\providecommand \@ifxundefined [1]{%
 \@ifx{#1\undefined}
}%
\providecommand \@ifnum [1]{%
 \ifnum #1\expandafter \@firstoftwo
 \else \expandafter \@secondoftwo
 \fi
}%
\providecommand \@ifx [1]{%
 \ifx #1\expandafter \@firstoftwo
 \else \expandafter \@secondoftwo
 \fi
}%
\providecommand \natexlab [1]{#1}%
\providecommand \enquote  [1]{``#1''}%
\providecommand \bibnamefont  [1]{#1}%
\providecommand \bibfnamefont [1]{#1}%
\providecommand \citenamefont [1]{#1}%
\providecommand \href@noop [0]{\@secondoftwo}%
\providecommand \href [0]{\begingroup \@sanitize@url \@href}%
\providecommand \@href[1]{\@@startlink{#1}\@@href}%
\providecommand \@@href[1]{\endgroup#1\@@endlink}%
\providecommand \@sanitize@url [0]{\catcode `\\12\catcode `\$12\catcode
  `\&12\catcode `\#12\catcode `\^12\catcode `\_12\catcode `\%12\relax}%
\providecommand \@@startlink[1]{}%
\providecommand \@@endlink[0]{}%
\providecommand \url  [0]{\begingroup\@sanitize@url \@url }%
\providecommand \@url [1]{\endgroup\@href {#1}{\urlprefix }}%
\providecommand \urlprefix  [0]{URL }%
\providecommand \Eprint [0]{\href }%
\providecommand \doibase [0]{http://dx.doi.org/}%
\providecommand \selectlanguage [0]{\@gobble}%
\providecommand \bibinfo  [0]{\@secondoftwo}%
\providecommand \bibfield  [0]{\@secondoftwo}%
\providecommand \translation [1]{[#1]}%
\providecommand \BibitemOpen [0]{}%
\providecommand \bibitemStop [0]{}%
\providecommand \bibitemNoStop [0]{.\EOS\space}%
\providecommand \EOS [0]{\spacefactor3000\relax}%
\providecommand \BibitemShut  [1]{\csname bibitem#1\endcsname}%
\let\auto@bib@innerbib\@empty
%</preamble>
\bibitem [{\citenamefont {Morfill}\ and\ \citenamefont
  {Ivlev}(2009)}]{Morfill:RMP2009}%
  \BibitemOpen
  \bibfield  {author} {\bibinfo {author} {\bibfnamefont {G.~E.}\ \bibnamefont
  {Morfill}}\ and\ \bibinfo {author} {\bibfnamefont {A.~V.}\ \bibnamefont
  {Ivlev}},\ }\href@noop {} {\bibfield  {journal} {\bibinfo  {journal} {Rev.
  Mod. Phys.}\ }\textbf {\bibinfo {volume} {81}},\ \bibinfo {pages} {1353}
  (\bibinfo {year} {2009})}\BibitemShut {NoStop}%
\bibitem [{\citenamefont {Melzer}\ and\ \citenamefont
  {Goree}(2008)}]{Melzer:WVV2008}%
  \BibitemOpen
  \bibfield  {author} {\bibinfo {author} {\bibfnamefont {A.}~\bibnamefont
  {Melzer}}\ and\ \bibinfo {author} {\bibfnamefont {J.}~\bibnamefont {Goree}},\
  }\href@noop {} {\enquote {\bibinfo {title} {6 fundamentals of dusty
  plasmas},}\ } (\bibinfo {year} {2008})\BibitemShut {NoStop}%
\bibitem [{\citenamefont {Bonitz}\ \emph {et~al.}(2010)\citenamefont {Bonitz},
  \citenamefont {Horing},\ and\ \citenamefont {(Eds.)}}]{Bonitz:Book2010}%
  \BibitemOpen
  \bibfield  {author} {\bibinfo {author} {\bibfnamefont {M.}~\bibnamefont
  {Bonitz}}, \bibinfo {author} {\bibfnamefont {N.}~\bibnamefont {Horing}}, \
  and\ \bibinfo {author} {\bibfnamefont {P.~L.}\ \bibnamefont {(Eds.)}},\
  }\href@noop {} {\bibfield  {journal} {\bibinfo  {journal} {Springer Series,
  Berlin}\ }\textbf {\bibinfo {volume} {59}} (\bibinfo {year}
  {2010})}\BibitemShut {NoStop}%
\bibitem [{\citenamefont {Edward~Thomas}\ and\ \citenamefont
  {Rosenberg}(2013)}]{Edward:IEEE2013}%
  \BibitemOpen
  \bibfield  {author} {\bibinfo {author} {\bibfnamefont {R.~L.~M.}\
  \bibnamefont {Edward~Thomas}, \bibfnamefont {Jr.}}\ and\ \bibinfo {author}
  {\bibfnamefont {M.}~\bibnamefont {Rosenberg}},\ }\href@noop {} {\bibfield
  {journal} {\bibinfo  {journal} {IEEE Trans. Plasma Sci.}\ }\textbf {\bibinfo
  {volume} {41}} (\bibinfo {year} {2013})}\BibitemShut {NoStop}%
\bibitem [{\citenamefont {Jr.}\ \emph {et~al.}(2016)\citenamefont {Jr.},
  \citenamefont {Konopka}, \citenamefont {Merlino},\ and\ \citenamefont
  {Rosenberg}}]{Edward:POP2016}%
  \BibitemOpen
  \bibfield  {author} {\bibinfo {author} {\bibfnamefont {E.~T.}\ \bibnamefont
  {Jr.}}, \bibinfo {author} {\bibfnamefont {U.}~\bibnamefont {Konopka}},
  \bibinfo {author} {\bibfnamefont {R.~L.}\ \bibnamefont {Merlino}}, \ and\
  \bibinfo {author} {\bibfnamefont {M.}~\bibnamefont {Rosenberg}},\ }\href
  {\doibase 10.1063/1.4943112} {\bibfield  {journal} {\bibinfo  {journal}
  {Physics of Plasmas}\ }\textbf {\bibinfo {volume} {23}},\ \bibinfo {pages}
  {055701} (\bibinfo {year} {2016})}\BibitemShut {NoStop}%
\bibitem [{\citenamefont {Miloch}(2014)}]{Miloch:JPP2014}%
  \BibitemOpen
  \bibfield  {author} {\bibinfo {author} {\bibfnamefont {W.~J.}\ \bibnamefont
  {Miloch}},\ }\href {\doibase 10.1017/S0022377814000300} {\bibfield  {journal}
  {\bibinfo  {journal} {Journal of Plasma Physics}\ }\textbf {\bibinfo {volume}
  {80}},\ \bibinfo {pages} {795?801} (\bibinfo {year} {2014})}\BibitemShut
  {NoStop}%
\bibitem [{\citenamefont {Joost}\ \emph {et~al.}(2015)\citenamefont {Joost},
  \citenamefont {Ludwig}, \citenamefont {K{\"a}hlert}, \citenamefont {Arran},\
  and\ \citenamefont {Bonitz}}]{Joost:PPCF2015}%
  \BibitemOpen
  \bibfield  {author} {\bibinfo {author} {\bibfnamefont {J.-P.}\ \bibnamefont
  {Joost}}, \bibinfo {author} {\bibfnamefont {P.}~\bibnamefont {Ludwig}},
  \bibinfo {author} {\bibfnamefont {H.}~\bibnamefont {K{\"a}hlert}}, \bibinfo
  {author} {\bibfnamefont {C.}~\bibnamefont {Arran}}, \ and\ \bibinfo {author}
  {\bibfnamefont {M.}~\bibnamefont {Bonitz}},\ }\href@noop {} {\bibfield
  {journal} {\bibinfo  {journal} {Plasma Physics and Controlled Fusion}\
  }\textbf {\bibinfo {volume} {57}},\ \bibinfo {pages} {025004} (\bibinfo
  {year} {2015})}\BibitemShut {NoStop}%
\bibitem [{\citenamefont {Nambu}\ \emph {et~al.}(2001)\citenamefont {Nambu},
  \citenamefont {Salimullah},\ and\ \citenamefont {Bingham}}]{Nambu:PRE2001}%
  \BibitemOpen
  \bibfield  {author} {\bibinfo {author} {\bibfnamefont {M.}~\bibnamefont
  {Nambu}}, \bibinfo {author} {\bibfnamefont {M.}~\bibnamefont {Salimullah}}, \
  and\ \bibinfo {author} {\bibfnamefont {R.}~\bibnamefont {Bingham}},\ }\href
  {\doibase 10.1103/PhysRevE.63.056403} {\bibfield  {journal} {\bibinfo
  {journal} {Phys. Rev. E}\ }\textbf {\bibinfo {volume} {63}},\ \bibinfo
  {pages} {056403} (\bibinfo {year} {2001})}\BibitemShut {NoStop}%
\bibitem [{\citenamefont {Shukla}\ \emph {et~al.}(2001)\citenamefont {Shukla},
  \citenamefont {Nambu},\ and\ \citenamefont {Salimullah}}]{Shukla:PLA2001}%
  \BibitemOpen
  \bibfield  {author} {\bibinfo {author} {\bibfnamefont {P.}~\bibnamefont
  {Shukla}}, \bibinfo {author} {\bibfnamefont {M.}~\bibnamefont {Nambu}}, \
  and\ \bibinfo {author} {\bibfnamefont {M.}~\bibnamefont {Salimullah}},\
  }\href {\doibase https://doi.org/10.1016/S0375-9601(01)00762-9} {\bibfield
  {journal} {\bibinfo  {journal} {Physics Letters A}\ }\textbf {\bibinfo
  {volume} {291}},\ \bibinfo {pages} {413 } (\bibinfo {year}
  {2001})}\BibitemShut {NoStop}%
\bibitem [{\citenamefont {Samsonov}\ \emph {et~al.}(2003)\citenamefont
  {Samsonov}, \citenamefont {Zhdanov}, \citenamefont {Morfill},\ and\
  \citenamefont {Steinberg}}]{Samsonov:NJP2003}%
  \BibitemOpen
  \bibfield  {author} {\bibinfo {author} {\bibfnamefont {D.}~\bibnamefont
  {Samsonov}}, \bibinfo {author} {\bibfnamefont {S.}~\bibnamefont {Zhdanov}},
  \bibinfo {author} {\bibfnamefont {G.}~\bibnamefont {Morfill}}, \ and\
  \bibinfo {author} {\bibfnamefont {V.}~\bibnamefont {Steinberg}},\ }\href
  {http://stacks.iop.org/1367-2630/5/i=1/a=324} {\bibfield  {journal} {\bibinfo
   {journal} {New Journal of Physics}\ }\textbf {\bibinfo {volume} {5}},\
  \bibinfo {pages} {24} (\bibinfo {year} {2003})}\BibitemShut {NoStop}%
\bibitem [{\citenamefont {Yaroshenko}\ \emph {et~al.}(2003)\citenamefont
  {Yaroshenko}, \citenamefont {Morfill}, \citenamefont {Samsonov},\ and\
  \citenamefont {Vladimirov}}]{Yaroshenko:NJP2003}%
  \BibitemOpen
  \bibfield  {author} {\bibinfo {author} {\bibfnamefont {V.~V.}\ \bibnamefont
  {Yaroshenko}}, \bibinfo {author} {\bibfnamefont {G.~E.}\ \bibnamefont
  {Morfill}}, \bibinfo {author} {\bibfnamefont {D.}~\bibnamefont {Samsonov}}, \
  and\ \bibinfo {author} {\bibfnamefont {S.~V.}\ \bibnamefont {Vladimirov}},\
  }\href {http://stacks.iop.org/1367-2630/5/i=1/a=318} {\bibfield  {journal}
  {\bibinfo  {journal} {New Journal of Physics}\ }\textbf {\bibinfo {volume}
  {5}},\ \bibinfo {pages} {18} (\bibinfo {year} {2003})}\BibitemShut {NoStop}%
\bibitem [{\citenamefont {Carstensen}\ \emph {et~al.}(2012)\citenamefont
  {Carstensen}, \citenamefont {Greiner},\ and\ \citenamefont
  {Piel}}]{Carstensen:PRL2012}%
  \BibitemOpen
  \bibfield  {author} {\bibinfo {author} {\bibfnamefont {J.}~\bibnamefont
  {Carstensen}}, \bibinfo {author} {\bibfnamefont {F.}~\bibnamefont {Greiner}},
  \ and\ \bibinfo {author} {\bibfnamefont {A.}~\bibnamefont {Piel}},\ }\href
  {\doibase 10.1103/PhysRevLett.109.135001} {\bibfield  {journal} {\bibinfo
  {journal} {Phys. Rev. Lett.}\ }\textbf {\bibinfo {volume} {109}},\ \bibinfo
  {pages} {135001} (\bibinfo {year} {2012})}\BibitemShut {NoStop}%
\bibitem [{\citenamefont {Ludwig}\ \emph {et~al.}(2017)\citenamefont {Ludwig},
  \citenamefont {Jung}, \citenamefont {K{\"a}hlert}, \citenamefont {Joost},
  \citenamefont {Greiner}, \citenamefont {Moldabekov}, \citenamefont
  {Carstensen}, \citenamefont {Sundar}, \citenamefont {Bonitz},\ and\
  \citenamefont {Piel}}]{Ludwig:EPJD2017}%
  \BibitemOpen
  \bibfield  {author} {\bibinfo {author} {\bibfnamefont {P.}~\bibnamefont
  {Ludwig}}, \bibinfo {author} {\bibfnamefont {H.}~\bibnamefont {Jung}},
  \bibinfo {author} {\bibfnamefont {H.}~\bibnamefont {K{\"a}hlert}}, \bibinfo
  {author} {\bibfnamefont {J.-P.}\ \bibnamefont {Joost}}, \bibinfo {author}
  {\bibfnamefont {F.}~\bibnamefont {Greiner}}, \bibinfo {author} {\bibfnamefont
  {Z.~A.}\ \bibnamefont {Moldabekov}}, \bibinfo {author} {\bibfnamefont
  {J.}~\bibnamefont {Carstensen}}, \bibinfo {author} {\bibfnamefont
  {S.}~\bibnamefont {Sundar}}, \bibinfo {author} {\bibfnamefont
  {M.}~\bibnamefont {Bonitz}}, \ and\ \bibinfo {author} {\bibfnamefont
  {A.}~\bibnamefont {Piel}},\ }\href@noop {} {\bibfield  {journal} {\bibinfo
  {journal} {EPJ D In Press}\ }\textbf {\bibinfo {volume} {}},\ \bibinfo {pages}
  {} (\bibinfo {} {2018})}\BibitemShut {NoStop}%
\bibitem [{\citenamefont {Hutchinson}\ and\ \citenamefont
  {Haakonsen}(2013)}]{Hutch:POP2013}%
  \BibitemOpen
  \bibfield  {author} {\bibinfo {author} {\bibfnamefont {I.~H.}\ \bibnamefont
  {Hutchinson}}\ and\ \bibinfo {author} {\bibfnamefont {C.~B.}\ \bibnamefont
  {Haakonsen}},\ }\href@noop {} {\bibfield  {journal} {\bibinfo  {journal}
  {Physics of Plasmas}\ }\textbf {\bibinfo {volume} {20}},\ \bibinfo {eid}
  {083701} (\bibinfo {year} {2013})}\BibitemShut {NoStop}%
\bibitem [{\citenamefont {Hutchinson}(2011)}]{Hutch:POP2011}%
  \BibitemOpen
  \bibfield  {author} {\bibinfo {author} {\bibfnamefont {I.~H.}\ \bibnamefont
  {Hutchinson}},\ }\href@noop {} {\bibfield  {journal} {\bibinfo  {journal}
  {Physics of Plasmas}\ }\textbf {\bibinfo {volume} {18}},\ \bibinfo {pages}
  {032111} (\bibinfo {year} {2011})}\BibitemShut {NoStop}%
\bibitem [{\citenamefont {Sundar}\ \emph {et~al.}(2017)\citenamefont {Sundar},
  \citenamefont {K{\"a}hlert}, \citenamefont {Joost}, \citenamefont {Ludwig},\
  and\ \citenamefont {Bonitz}}]{Sundar:POP2017}%
  \BibitemOpen
  \bibfield  {author} {\bibinfo {author} {\bibfnamefont {S.}~\bibnamefont
  {Sundar}}, \bibinfo {author} {\bibfnamefont {H.}~\bibnamefont {K{\"a}hlert}},
  \bibinfo {author} {\bibfnamefont {J.-P.}\ \bibnamefont {Joost}}, \bibinfo
  {author} {\bibfnamefont {P.}~\bibnamefont {Ludwig}}, \ and\ \bibinfo {author}
  {\bibfnamefont {M.}~\bibnamefont {Bonitz}},\ }\href {\doibase
  10.1063/1.5008898} {\bibfield  {journal} {\bibinfo  {journal} {Physics of
  Plasmas}\ }\textbf {\bibinfo {volume} {24}},\ \bibinfo {pages} {102130}
  (\bibinfo {year} {2017})}\BibitemShut {NoStop}%
\bibitem [{\citenamefont {Ludwig}\ \emph {et~al.}(2012)\citenamefont {Ludwig},
  \citenamefont {Miloch}, \citenamefont {K{\"a}hlert},\ and\ \citenamefont
  {Bonitz}}]{Ludwig:NJP2012}%
  \BibitemOpen
  \bibfield  {author} {\bibinfo {author} {\bibfnamefont {P.}~\bibnamefont
  {Ludwig}}, \bibinfo {author} {\bibfnamefont {W.~J.}\ \bibnamefont {Miloch}},
  \bibinfo {author} {\bibfnamefont {H.}~\bibnamefont {K{\"a}hlert}}, \ and\
  \bibinfo {author} {\bibfnamefont {M.}~\bibnamefont {Bonitz}},\ }\href@noop {}
  {\bibfield  {journal} {\bibinfo  {journal} {New Journal of Physics}\ }\textbf
  {\bibinfo {volume} {14}},\ \bibinfo {pages} {053016} (\bibinfo {year}
  {2012})}\BibitemShut {NoStop}%
\bibitem [{\citenamefont {Moldabekov}\ \emph {et~al.}(2017)\citenamefont
  {Moldabekov}, \citenamefont {Ludwig}, \citenamefont {Joost}, \citenamefont
  {Bonitz},\ and\ \citenamefont {Ramazanov}}]{Zhandos:arxiv2017}%
  \BibitemOpen
  \bibfield  {author} {\bibinfo {author} {\bibfnamefont {Z.~A.}\ \bibnamefont
  {Moldabekov}}, \bibinfo {author} {\bibfnamefont {P.}~\bibnamefont {Ludwig}},
  \bibinfo {author} {\bibfnamefont {J.-P.}\ \bibnamefont {Joost}}, \bibinfo
  {author} {\bibfnamefont {M.}~\bibnamefont {Bonitz}}, \ and\ \bibinfo {author}
  {\bibfnamefont {T.~S.}\ \bibnamefont {Ramazanov}},\ }\href {arXiv:1709.09531}
  {\bibfield  {journal} {\bibinfo  {journal} {arXiv}\ }\textbf {\bibinfo
  {volume} {52}},\ \bibinfo {pages} {124004} (\bibinfo {year}
  {2017})}\BibitemShut {NoStop}%
\end{thebibliography}%



\end{document}















