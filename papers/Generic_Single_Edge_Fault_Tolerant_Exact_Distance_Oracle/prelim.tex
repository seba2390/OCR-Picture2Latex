% !TEX root = paper.tex
\iflong
\else
\vspace{-2mm}
\fi

\section{Preliminaries}

We use the following notation throughout the paper:

\begin{itemize}[noitemsep,nolistsep]
\item $xy$ :\ Given
two vertices $x$ and $y$, let $xy$ denote a path between
$x$ and $y$. Normally this path will be the shortest path
from $x$ to $y$ in $G$. However, in some places in the paper, the
use of $xy$ will be clear from the context.

\item $|xy|$ :\ It denotes the number of edges in the path
$xy$.

\item $(\cdot \conc \cdot)$ :  Given two paths
$sx$ and $xt$, $sx \conc xt$ denotes the concatenation of
paths $sx$ and $xt$.

\item {\em after or below/before or above x} : We will assume that the $st$ path (for $s\in S$ and $t \in V$)  is drawn from top to bottom. Assume that $x \in st$. The term {\em after or below $x$} on $st$ path refers to the path $xt$. Similarly {\em before or above x} on $st$ path refers to the path $sx$.

\item {\em replacement path}: The shortest path that avoids any given edge is called a
{\em replacement} path.
\end{itemize}
