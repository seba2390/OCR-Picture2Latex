% !TEX root = paper.tex
\iflong
\else
\vspace{-2mm}
\fi
\section{From Single Source to Multiple Sources}

Unfortunately, the analysis for the single source case is not easily extendible to multiple source case. We identify the exact problem here. Consider the case described in Section \ref{subsec:singlecasetwo}. In this case, we show that if $|P'| > |P|$ and $P$ intersects with $P'$, then there is a path available for $P$ (that is $ac \conc
cb' \conc b't$).  We can use this path because it also avoids $e $ (the edge avoided by path $P$). First, we show that the above assertion is not true when we move to multiple source case. Consider the following example (See Figure \ref{fig:multiple}).
Here, $P$ avoids $e$ on $st$ path and $P'$ avoids $e'$ on $s't$ path. $\DET(P)$ starts at $a$ and its intersects $P'$ at $c$. $\DET(P')$ starts at $a'$ and it hits $st$ path at $b'$ and then   passes through $e$. Note that the full path $P$ from $s$ to $t$ is not shown in Figure \ref{fig:multiple}. The reader can check that the  path $ac \conc
cb' \conc b't$ is not an alternate path for $P$ as it passes through $e$. We say that such a path is a bad path because it breaks the easy analysis of single source case (we will formally define bad paths in Section \ref{subsec:pintersects}).
However, we are able to show that the total number of {\em good} paths (paths which are not bad)
is $\ge$ the number of bad paths. Good paths exhibit properties similar to the set
$\RR$ in Section \ref{sec:avoids}. This will help us in bounding them (and thus bad paths too).
%However, we will show that the above bad case does not %happen too often. This completes a short overview of our %approach.
\label{sec:problem}
\iflong
\subsection{Multiple Spallation}
\label{sec:multispa}
Since neutrons in a hadronic shower are a good indicator of
the production of spallation nuclei, then the observation of the decay of a spallation nucleus is likewise an indicator for a hadronic shower. We therefore apply
(in addition to the neutron cloud cut), a preselection removing clusters of isotopes produced by the same muon. Instead of pairing candidate events with possible parent muons, this multiple spallation cut identifies clusters of low energy events observed within a few tens of seconds and a few meters of each other. Here, we consider a sample composed of all SK-IV events passing the first reduction and quality cuts defined for the solar analysis, as discussed in Sec.~\ref{sec:solaranalysis}. Since we need to take all spallation isotope decays into account we apply neither the old spallation cut nor the pattern likelihood cut, as the latter targets $\mathrm{^{16}N}$ $\beta\gamma$ decays. We find that the optimal cut removes candidates found within $4$~m and $60$~s of any event from this sample. This cut allows to remove $45\%$ of spallation background events with a deadtime of $1.3\%$. The solar angle distribution of the rejected and the remaining events is shown in Fig.\ref{fig:multispa}. The absence of a peak around $\cos\theta_{sun} = 1$ for the rejected events confirms the low deadtime for this cut.  

\begin{figure}
        \includegraphics[width=\linewidth]{./multiple/figures/hmulti.eps} 
    \caption{Comparison of the events removed~(dashed) and remaining~(solid) in SK-IV solar sample using multiple spallation cut above 5.99~MeV . The sample above uses the final sample criteria from \cite{skivsolar}.}
    \label{fig:multispa}
\end{figure}

\iffalse
\begin{figure}
    \centering
    \begin{subfigure}{0.5\textwidth}
    \includegraphics[width=0.5\textwidth{multiple/figures/multicut.png}
    \end{subfigure}
    \begin{subfigure}{0.5\textwidth}
    \includegraphics{multiple/figures/nomulti.png}
    \end{subfigure}
    \caption{Comparison of the events removed~(left) and remaining~(right) in SK-IV solar sample using multiple spallation cut above 5.5~MeV. The sample above uses the final sample criteria from~\cite{skivsolar}.}
    \label{fig:multispa}
\end{figure}
\fi


\fi
Once again we will fix a vertex $t$ and show that  the number of replacement paths from
$s \in S$ to $t$ that also avoids $t_s$ is $ O(\sqrt {\sigma n})$.
Let $\BFS(t)$ denote  the  union of all  shortest paths from $t$ to $s \in S$.
The reader can check that the union of these paths does not admit a cycle, so we
can assume that its a tree rooted at $t$. Since $\BFS(t)$ has at most $\sigma$
leaves, the number of vertices with degree $> 2$ in $\BFS(t)$ is $O(\sigma)$.
We now contract all the vertices of degree 2 (except $t$ and $s \in S$) in $\BFS(t)$
to get a tree that only contains
leaves of $\BFS(t)$, the root $t$, all the sources and all other vertices
with degree $> 2$ in $\BFS(t)$.
\begin{definition} (\SBFS($t$))
\SBFS($t$) is a tree obtained by contracting all the vertices with degree exactly 2 in $\BFS(t)$
except $t$ and source $s \in S$.
\end{definition}

\begin{definition} (Intersection vertex and segment in \SBFS($t$))

\noindent Each node  \SBFS$(t)$  is called an intersection vertex.
An edge $xy \in$ \SBFS($t$) denotes a path between two vertices in $\BFS(t)$.
We call such an edge in \SBFS\  a segment. We use this term in order to differentiate between  edges in $\BFS(t)$ and  \SBFS($t$).
Also, we will use the following convention: if $xy$ is a segment,
then $y$ is closer to $t$ than $x$.
\end{definition}
\noindent \SBFS($t$) has at most $\sigma$
vertices with degree $\le 2$. This implies that there are at most $O(\sigma)$
intersection vertices and segments in \SBFS($t$).
%Similarly, the total number of
%segments in \SBFS($t$) is $O(\sigma)$.

As in the single source case, we would like to find the preferred path for each avoided edge on the $st$
path where $s \in S$. However, we don't have enough space to store all these paths.
Also storing all  paths seems wasteful.
Consider two preferred replacement paths $P$ and $P'$ that start from $s$ and $s'$ respectively.
These two paths meet at an intersection vertex $x$ after which they are same, that is, they take
the same detour to reach $t$. Storing both $P$ and $P'$ seems wasteful as they are essentially
the same path once they hit $x$.  To this end, we only store preferred path
corresponding to each segment in \SBFS($t$). We now describe our approach in detail.

Let $xy$ be a segment in \SBFS($t$). We divide replacement paths whose detour start in $xy$ into
two types:
\begin{itemize}[noitemsep,nolistsep]
  % \item[($\TZE$)] Replacement path whose failed vertex %is  an intersection vertex in \SBFS(t).

%   Thus for the next two sets, the failed vertex is not %an intersection vertex in \SBFS(t).
   \item[] $\TON(xy)$: Preferred replacement paths from $x$ to $t$ whose detour starts in $xy$
    but the avoided edge lies in $yt_x$.

   \item[] $\TTW(xy)$: Preferred replacement paths from $x$ to $t$ whose
   start of detour and avoided edge both lie strictly inside segment $xy$ (that is, detour cannot
   start from $x$ or $y$).
\end{itemize}


\noindent Let $\TON := \cup_{xy \in \text{\SBFS($t$)}} \TON(xy)$ and
$\TTW := \cup_{xy \in \text{\SBFS($t$)}} \TTW(xy)$. The set  $\TON$ helps us to weed out
simple preferred replacement paths.
We will show that we can store  preferred replacement paths in  $\TON$ in
$O(\sigma)$ space --  one per segment in \SBFS($t$).
The hardest case for us in $\TTW$, which contains bad paths.
%$\TON$ helps us weed out simple replacement paths so that we don't have
%to bother about them when we are analyzing the harder case,
%that is paths in $\TTW$.
%For $\TTW$, we have to take care of the bad paths.
Let $\BP$ denote the set of bad paths in $\TTW$. We will show  that
$|\BP| \le |\TTW \setminus \BP|$ (the number of bad paths is $\le$ number of good paths in $\TTW$)
and $|\TTW \setminus \BP| =  O(\sqrt{n\sigma})$ (the number of good path is $ O(\sqrt{n\sigma}$)).
This implies that $|\TTW| =  O(\sqrt{n \sigma})$. %This would imply that
%the total number of path from $s \in S$ to $t$ that don't %pass through
%$t_s$ is $\tilde O(\sqrt{n \sigma})$.

Since $\TON$ and $\TTW$ are of size $O(\sqrt{n\sigma})$, we can make a data-structure
of size $O(\sqrt{n\sigma})$.
In this data-structure, we have stored a preferred path for each segment.
However, we have to answer queries of type $\textsc{Q}(s,t,e)$
where $s$ is a source.
%It is not clear how to use our data-structure
%to answer this query correctly.
In Section \ref{sec:data}, we will see
how to use preferred paths of segments to
answer queries in $\tilde O(1)$ time.
