% !TEX root = paper.tex
\begin{abstract}
 Given an undirected  unweighted graph $G$ and a source set $S$ of $|S| = \sigma
$ sources, we want to build a data structure which can process
the following query {\sc Q}$(s,t,e):$ find the shortest distance from $s$ to $t$ avoiding an edge $e$,  where $s \in S$ and
$t \in V$. When $\sigma=n$, Demetrescu, Thorup, Chowdhury and Ramachandran (SIAM Journal of Computing, 2008) designed an algorithm
%\footnote{Their algorithm finds the shortest path avoiding a vertex and it can be adapted for our problem too.}
with $\tilde O(n^2)$ space\footnote{$\tilde
O(\cdot)$ hides poly $\log n$ factor.} and $O(1)$ query time. A natural open question is to
generalize this result to any number of sources. Recently,
Bil{\`o} et. al. (STACS 2018) designed a data-structure of size $\tilde O(\sigma^{1/2}n^{3/2})$ with the query time of $O(\sqrt{n\sigma})$ for the above problem.  We improve their result by designing
a data-structure of size $\tilde O(\sigma^{1/2} n^{3/2})$ that can answer queries in $\tilde O(1)$ time.

\begin{comment}
 A related question is to find a subgraph of $G$ such that BFS tree from $s \in S$ is
 preserved in the subgraph after any edge deletion. Parter and Peleg (ESA\ 2013) showed
 that a subgraph of size $O(\sigma^{1/2} n^{3/2})$ is both necessary and sufficient to
 solve the above problem. However, if we want to answer any query {\sc Q}$(s,t,e)$,
 then we have to do a BFS in this graph after deleting $e$. Thus, the query time of
 trivial algorithm is $ O(\sigma^{1/2} n^{3/2})$. Surprisingly, we show that there
 exists a data-structure of nearly  same size,
 that is $\tilde O(\sigma^{1/2} n^{3/2})$, that can answer queries in $\tilde O(1)$ time.
\end{comment}

In a related problem of finding fault tolerant subgraph, Parter and Peleg (ESA 2013) showed that
if  detours of  {\em replacement} paths ending at a vertex $t$  are disjoint,
then the number of such paths is $O(\sqrt{n\sigma})$.
This eventually gives a bound of $O( n \sqrt{n \sigma}) = O(\sigma^{1/2}n^{3/2})$
for their problem. {\em Disjointness of detours} is a very crucial property used
in the above result. We show a similar result for a subset of replacement path which
\textbf{may not} be disjoint. This result is the crux of our paper and may be
of independent interest.

\end{abstract}
