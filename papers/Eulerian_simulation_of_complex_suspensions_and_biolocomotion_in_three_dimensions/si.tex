\documentclass[times, 10pt]{article}
% set dimensions of columns, gap between columns, and paragraph indent
\setlength{\textheight}{9in}
\setlength{\textwidth}{7in}
\setlength{\columnsep}{0.3125in}
\setlength{\topmargin}{0in}
\setlength{\headheight}{0in}
\setlength{\headsep}{0in}
\setlength{\parindent}{1pc}
\setlength{\oddsidemargin}{-.304in}
\setlength{\evensidemargin}{-.304in}

\usepackage{times}
\usepackage{graphicx}
\usepackage{placeins}
\usepackage[utf8]{inputenc}
\usepackage[T1]{fontenc}
\usepackage{amsmath,amsfonts,amssymb}
\usepackage{url}

\usepackage[dvipsnames]{xcolor}
\usepackage{hyperref}
\usepackage{cleveref}

\newcommand\myshade{85}
\colorlet{mylinkcolor}{purple}
\colorlet{mycitecolor}{YellowOrange}
\colorlet{myurlcolor}{NavyBlue}

\hypersetup{
  linkcolor  = mylinkcolor!\myshade!black,
  citecolor  = mycitecolor!\myshade!black,
  urlcolor   = myurlcolor!\myshade!black,
  colorlinks = true,
}
\renewcommand{\thefigure}{S\arabic{figure}}
\renewcommand{\thetable}{S\arabic{table}}
\renewcommand{\thepage}{S\arabic{page}}
\renewcommand{\theequation}{S\arabic{page}}

% Method section math symbols
\renewcommand{\vec}[1]{\mathbf{#1}}
\newcommand{\p}{\partial}
\newcommand{\vu}{\vec{u}}
\newcommand{\vb}{\vec{b}}
\newcommand{\vq}{\vec{q}}
\newcommand{\nor}{\vec{n}}
\newcommand{\vx}{\vec{x}}
\newcommand{\vX}{\vec{X}}
\newcommand{\vE}{\vec{E}}
\newcommand{\vF}{\vec{F}}
\newcommand{\vC}{\vec{C}}
\newcommand{\Ident}{\mathbf{1}}
\newcommand{\vxi}{\boldsymbol\xi}
\newcommand{\vsigma}{\boldsymbol\sigma}
\newcommand{\vbeta}{\boldsymbol\beta}
\newcommand{\vtau}{\boldsymbol\tau}
\newcommand{\vchi}{\boldsymbol\chi}
\newcommand{\Rey}{\textit{Re}}
\newcommand{\trans}{\mathsf{T}}
\DeclareMathOperator{\Tr}{tr}

% Swimming section math symbols
\newcommand\xhat{\vec{d}_X}
\newcommand\yhat{\vec{d}_Y}
\newcommand\zhat{\vec{d}_Z}
\newcommand\Xhat{\vec{e}_X}
\newcommand\Yhat{\vec{e}_Y}
\newcommand\Zhat{\vec{e}_Z}
\newcommand\va{\vec a}
\newcommand\vI{\vec 1}
\newcommand\vB{\vec B}
\newcommand\vD{\vec D}

% Author info
\usepackage{authblk}

\makeatletter
\@fpsep\textheight
\makeatother

\title{Eulerian simulation of complex suspensions and biolocomotion in three dimensions\\
\huge Supplementary Information }
\author[a]{Yuexia L. Lin}
\author[a]{Nicholas J. Derr}
\author[a,b]{Chris H. Rycroft}

\affil[a]{John A. Paulson School of Engineering and Applied Sciences, Harvard University, 29 Oxford Street, Cambridge, MA 02138}
\affil[b]{Mathematics Group, Lawrence Berkeley National Laboratory, 1 Cyclotron Road, Berkeley, CA 94720}

\date{}

\begin{document}
\maketitle


%--------------------------
% SEC 1
%--------------------------
\section{Details of numerical algorithms on Cartesian grid}
The numerical algorithms that we use for modeling fluid--structure interactions (FSI) are built upon a modern implementation for simulating incompressible fluid mechanics
that is based upon Chorin's projection method~\cite{chorin67, chorin68}.
Since its introduction in 1968, a variety of extensions have been explored in the literature to improve accuracy and stability~\cite{brown01}.
We make use of a mature implementation that incorporates many such advancements,
developed by Almgren, Bell, and coworkers~\cite{bell89, almgren96, almgren98, colella90, sussman99}.
We refer the reader to the papers by Yu \textit{et al.}~\cite{yu03, yu07} that use this implementation to simulate an inkjet printer nozzle;
these papers provide a comprehensive description of the numerical approaches.

In our algorithms, we keep the fluid simulation component the same as in this existing body of work.
We then build the reference map technique (RMT) on top of the same framework for handling solid objects, using similar principles in the numerical discretization.
The numerical methods and approach are similar to the two-dimensional implementation of the RMT that we previously developed~\cite{rycroft20}.
We use a second-order accurate discretization in space and first-order accurate explicit scheme in time for both the fluid and the solid phases.
Due to error contributions from the interfacial coupling procedure, the overall FSI method is approximately first-order accurate in both space and time.
Convergence rates are reported for various test cases in sections below.

As described in the main text, the treatment of the reference map field and extrapolation is improved and simplified,
and removes the need for specialized level-set methods and reinitialization techniques that have been required in previous implementations~\cite{kamrin12,valkov15,rycroft20}.
A major benefit of the new implementation is that it is more amenable for parallelization in the distributed memory paradigm, a necessity in 3D simulations.

%--------------------------
% Variable
%--------------------------
\subsection{Overview}
\label{sub:overall}

\subsubsection*{Governing equations}
We first provide a broad sketch of the overall numerical algorithm and introduce the staggered arrangement of variables on a Cartesian grid.
We discretize the following governing equations for the coupled fluid--structure system,
%--------------------------------------------------
\begin{align}
    \rho \left (\frac{\p \vu}{\p t}
    + ( \vu \cdot \nabla ) \vu \right) &= - \nabla p +\nabla \cdot \vtau + \vb, \label{eq:full_mmt}\\
    %
    \frac{\p  \vxi}{\p t} + (\vu \cdot \nabla)  \vxi &= \vec{0}, \label{eq:xi_adv}\\
    %
    \nabla \cdot \vu &= 0,
    \label{eq:incomp}
\end{align}
where $\rho$ is the mass density, $\vu$ is the global velocity,
$\vxi$ is the reference map variable in the solid bodies,
$p$ is the global hydrostatic pressure,
$\vec{b}$ is a body force density if applicable,
and $ \vtau$ is the deviatoric stress tensor.
The deviatoric stress tensor $ \vtau$ is the fluid stress tensor $\vtau_f$ in the fully fluid phase, and it is the solid stress tensor $\vtau_s$ in the fully solid phase.
Near the fluid--solid interface, $\vtau$ is a mixed quantity using a smooth transition function over a blur zone of characteristic width $2\epsilon$,
%--------------------------------------------------
\begin{equation}
  \vtau = H_\epsilon(\phi)  \vtau_f   + (1-H_\epsilon(\phi)) (\vtau_s +  \vtau_a)
  \label{eq:mixed_stresses}
\end{equation}
where $\vtau_a$ denotes the artificial viscous stress tensor added to the solid.
$H_\epsilon(\phi)$ is the smoothed Heaviside function, which is defined in the main text and has been used in other works~\cite{sussman94,yu03,rycroft20}.
To simplify notation, we refer to $\phi_0$, a signed-distance defined in the reference configuration of the solid object, as $\phi$ hereinafter.
The artificial viscous stress is defined as
%--------------------------------------------------
\begin{equation}
\vtau_a =  \mu_a( 1 + \gamma_t \epsilon H'_\epsilon(\phi)) (\nabla \vu + \nabla \vu^\trans)
\end{equation}
where $\mu_a$ is the solid artificial viscosity and $\gamma_t$ is a dimensionless multiplier to amplify the artificial viscosity in the blur zone.
The contribution of $\nabla \vu^\trans$ term to the divergence of viscous stress is negligible due to the incompressibility constraint $\nabla \cdot \vu = 0$.
Therefore, in the actual computation, we do not compute $\nabla \vu^\trans$ even though it is present in the fluid stress and the artificial stress tensors.

\subsubsection*{Variable arrangement}

The placement of variables on the computational grid is illustrated in Fig.~\ref{fig:var_arr}.
A domain $[a_x, b_x] \times [a_y, b_y] \times [a_z, b_z]$ is divided in $M, N, O$ number of grid cells,
yielding grid spacings $\Delta x, \Delta y, \Delta z$, respectively.
To integrate in time, we take steps of size $\Delta t$.
We denote the $n^\text{th}$ timestep $t_n$, and the total number of timesteps $U$.
Thus $t_n = n \Delta t$, $n=1,2,\ldots,U$.
The discretized velocity solution is denoted $\vu_{i,j,k}^n$, where the subscripts $i, j, k$ indicate cell-centered position on the grid,
and the superscript $n$ indicates the number of timesteps.
Other quantities on the grid are indexed in a similar fashion unless specified otherwise.
When applicable, $1/2$ in the indices indicate either cell faces between to grid cells or the midpoint between two timesteps.
When spatial indices are omitted, we refer to the solutions in the entire computation domain.
Primary variables are mass density $\rho_{i,j,k}^n$, global velocity $\vu_{i,j,k}^n$, and reference map variables in solids and blur zones $\vxi_{i,j,k}^n$, which are all placed at the cell centers.
In addition, there is also a global pressure variable \smash{$p_{i,j,k}^{n-1/2}$} placed at the nodes. Note that subscripts $i,j,k$ denote nodal quantities for the pressure field only.

\subsubsection*{Spatial and temporal discretizations}

We discretize Eqs.~\eqref{eq:full_mmt} \& \eqref{eq:xi_adv} by
%--------------------------------------------------
\begin{align}
\frac{\vu^{n+1} - \vu^n}{\Delta t} + \left[ (\vu \cdot \nabla ) \vu \right]^{n+1/2} &=
\frac{1}{\rho(\phi^{n+1/2})}  \bigg[
- \nabla p^{n+1/2}
+ \nabla \cdot \left ((1- H_\epsilon (\phi^{n+1/2}) )(\vtau_s^{n+1/2} + \vtau_a^{\tilde{n}}) + H_\epsilon (\phi^{n+1/2})\vtau_f^n \right)
+ \vb^{n+1/2} \bigg], \label{eq:dis_full_mmt} \\
\frac{\vxi^{n+1} - \vxi^n} {\Delta t} + \left[ (\vu \cdot \nabla ) \vxi \right]^{n+1/2} &= \vec{0}. \label{eq:dis_xi_adv}
\end{align}
Note that $\phi^{n+1/2}$ refers to $\phi_0(\vxi^{n+1/2})$, not the reinitialized level-set function.
A number of terms on the right hand side (RHS) of the above equations are calculated at the half-timestep $n+\tfrac{1}{2}$ for improved accuracy.
If applicable, a body force density can be prescribed at cell centers at mid timestep, $\mathbf b_{i,j,k}^{n+1/2}$.
Since this is an explicit scheme it is necessary to compute $\vtau_f^n$ and $\vtau_a^{\tilde{n}}$ using information at timestep $n$,
which results in $\vtau_a^{\tilde{n}}$ using quantities at two different times, i.e.,
%--------------------------------------------------
\begin{equation}
  \vtau_a^{\tilde{n}} = \mu_a( 1 + \gamma_t \epsilon H'_\epsilon(\phi^{n+1/2})) (\nabla \vu^n + (\nabla \vu^n)^\trans).
\end{equation}
Better accuracy could be achieved by using a Crank--Nicolson-type update formula~\cite{crank47}, but since this would depend on $\vu^{n+1}$ it would result in an implicit numerical scheme, which is outside the scope of this work.

To handle the advective terms $ \left[ (\vu \cdot \nabla ) \vu \right]^{n+1/2}$ and $\left[ (\vu \cdot \nabla ) \vxi \right]^{n+1/2}$, we require at cell faces
%--------------------------------------------------
\begin{enumerate}
  \item intermediate velocities $\vu^{n+1/2}_{i\pm1/2, j,k}, \vu^{n+1/2}_{i,j\pm1/2,k}, \vu^{n+1/2}_{i,j,k\pm1/2}$,
  \item intermediate reference maps $\vxi^{n+1/2}_{i\pm1/2,j,k}, \vxi^{n+1/2}_{i,j\pm1/2,k}, \vxi^{n+1/2}_{i,j,k\pm1/2}$.
\end{enumerate}
We use a Godunov-type upwinding scheme to compute these quantities, which will be discussed in subsection \ref{sub:adv}.
%
\subsubsection*{MAC projection}
Using the intermediate face velocities, we can compute the discrete divergence in the cell centers using centered finite differences.
However, due to discretization error, this discrete divergence field may not evaluate to precisely zero as expected by Eq.~\eqref{eq:incomp}.
We employ an intermediate marker-and-cell (MAC) projection to correct the face velocities to satisfy the discrete divergence-free property.
This has been shown to improve accuracy and volume conservation in previous work~\cite{rycroft20}.
To do this we first solve the equation
%--------------------------------------------------
\begin{align}
    \nabla \cdot \left ( \frac{1}{\rho} \nabla q^{n+1/2} \right) = \frac{ \nabla \cdot \vu^{n+1/2}}{\Delta t/2}
    \label{eq:mac_proj}
\end{align}
for an intermediate solution $q^{n+1/2}$ at the cell centers.
The RHS of the equation above is evaluated using the intermediate face velocities.
We subtract $\frac{\Delta t}{2 \rho} \nabla q$ from the intermediate velocities, thus ensuring the discrete divergences are zero to machine precision.
In Eq.~\eqref{eq:mac_proj} we have dropped the subscripts that indicate cell faces, and we will continue to do so for brevity.

\subsubsection*{Approximate pressure-Poisson projection}
Using update equations \eqref{eq:dis_full_mmt}  \& \eqref{eq:dis_xi_adv}, an intermediate velocity update $\vu^{*}$ at $t=t_n$ can be computed.
Once this is computed, we apply the projection step making use of the approximate finite-element projection introduced
by Almgren \textit{et al.}~\cite{almgren96}. This involves solving the Poisson equation
\begin{equation}
  \nabla \cdot \left ( \frac{1}{\rho} \nabla \psi \right) = \frac{\nabla \cdot \left( \vu^{*} - \vu^n \right) }{\Delta t}
\end{equation}
for the pressure correction $\psi$, after which the pressure is updated using $p^{n+1/2} = p^{n-1/2} +  \psi$.
The finite-element projection is implemented using trilinear basis functions centered on each cell corner.
At a particular pressure point $p_{i,j,k}^n$ located at a cell corner, the corresponding basis function
is non-zero over the $2\times 2\times 2$ block of adjacent grid cells.

\subsubsection*{Stress calculations}
The fluid deviatoric stress $\vtau_f^n$ at time $t_n$ is computed at the cell faces by using centered finite differences on the cell centered velocities $\vu^n$.
To compute solid deviatoric stress $\vtau_s^n$, deformation gradient $\vF^n$ is first computed at the cell faces by using centered finite differences on the reference maps, $\vF = (\nabla_x \vxi)^{-1}$.
Then $\vtau_s^n$ is computed at the cell faces directly using the incompressible neo-Hookean constitutive relation
%--------------------------------------------------
\begin{equation}
  \vtau_s^n = G\left ( \mathbf B^n - \tfrac13\Tr(\mathbf B^n)\right)
\end{equation}
where $\mathbf B^n = \vF^n (\vF^n)^\trans$.
If blur zones of $S$ solid objects overlap, $S>1$,
computations of solid stress and artificial viscous stress remain the same for each object.
To avoid notation confusion, we drop the spatial indices, and use subscript $m$ to represent different solid objects.
But it is assumed that the discussion below only applies at cell faces where it is needed.
In each solid object, we separately compute
%--------------------------------------------------
\begin{align}
&\vtau_{s, m}^n = G_m\left ( \mathbf B_m^n - \tfrac13\Tr(\mathbf B_m^n)\right),\\
&\vtau_{a, m}^{\tilde{n}} = \mu_a( 1 + \gamma_t \epsilon H'_\epsilon(\phi_m^{n+1/2})) (\nabla \vu^n + (\nabla \vu^n)^\trans),
\end{align}
where $\phi_m^{n+1/2}$ is the specific level-set function defining the fluid--solid interface of solid $m$, evaluated on the reference map variables of this solid object at time $n+1/2$.
In the overlapping region, the overall stress tensors $\vtau_s$ and $\vtau_a$ are combined using the respective stress tensors from each of the overlapping solid objects, 
%--------------------------------------------------
\begin{equation}
\vtau_s^n =
\begin{cases}
  \sum_{m=1}^S (1-H_\epsilon(\phi_m^n)) \vtau_s^n & \qquad \text{if $\sum_{m=1}^S (1-H_\epsilon(\phi_m^n)) \le 1$,} \\ \\
\frac{\sum_{m=1}^S (1-H_\epsilon(\phi_m^n)) \vtau_s^n } { \sum_{m=1}^S (1-H_\epsilon(\phi_m^n)) } & \qquad \text{otherwise}.
\end{cases}
\end{equation}
The artificial viscous stress tensor  $\vtau_a^n$ is computed similarly.
To blend multiple solid blur zones with the fluid, we modify the volume fraction used in Eq.~\eqref{eq:mixed_stresses} to
%--------------------------------------------------
\begin{equation}
H_\epsilon(\phi^n) = \max \left[ 1 - \sum_{s=1}^S (1-H_\epsilon(\phi_s^n)), 0 \right].
\label{eq:mod_heaviside}
\end{equation}


%--------------------------------------------------
% Section:
%--------------------------------------------------
\subsection{Evaluating the advective term}
\label{sub:adv}
\subsubsection*{Advective derivatives}
To evaluate advective derivatives on the left hand side of Eqs.~\eqref{eq:dis_full_mmt},~\eqref{eq:dis_xi_adv}, we use the formulae
%--------------------------------------------------
\begin{align}
\label{eq:adv_terms}
 \left[ (\vu \cdot \nabla ) \vu \right]^{n+1/2} &= \frac{u^{n+1/2}_{i+1/2} + u^{n+1/2}_{i-1/2}}{2} \frac{\vu^{n+1/2}_{i+1/2} - \vu^{n+1/2}_{i-1/2}}{\Delta x} \nonumber \\
 &\phantom{=}+ \frac{v^{n+1/2}_{j+1/2} + v^{n+1/2}_{j-1/2}}{2} \frac{\vu^{n+1/2}_{j+1/2} - \vu^{n+1/2}_{j-1/2}}{\Delta y} \nonumber \\
 &\phantom{=}+  \frac{w^{n+1/2}_{k+1/2} + w^{n+1/2}_{k-1/2}}{2} \frac{\vu^{n+1/2}_{k+1/2} - \vu^{n+1/2}_{k-1/2}}{\Delta z}, \\
 %
  \left[ (\vu \cdot \nabla ) \vxi \right]^{n+1/2}  &= \frac{u^{n+1/2}_{i+1/2} + u^{n+1/2}_{i-1/2}}{2} \frac{\vxi^{n+1/2}_{i+1/2} - \vxi^{n+1/2}_{i-1/2}}{\Delta x} \nonumber \\
  &\phantom{=}+ \frac{v^{n+1/2}_{j+1/2} + v^{n+1/2}_{j-1/2}}{2} \frac{\vxi^{n+1/2}_{j+1/2} - \vxi^{n+1/2}_{j-1/2}}{\Delta y} \nonumber \\
  &\phantom{=}+  \frac{w^{n+1/2}_{k+1/2} + w^{n+1/2}_{k-1/2}}{2} \frac{\vxi^{n+1/2}_{k+1/2} - \vxi^{n+1/2}_{k-1/2}}{\Delta z},
\end{align}
where $\vu=(u,v,w)$. For brevity, only the indices that differ from $i,j,k$ are shown in the subscripts. To obtain $\vu^{n+1/2}$ and $\vxi^{n+1/2}$ at a cell face, we use a Taylor expansion using values straddling the cell face, keeping terms up to first order in time and space. Without loss of generality, we consider a face in $x$-direction, indexed by ${i+1/2, j, k}$. Two Taylor expansions are constructed, centered at values at cell $i,j,k$ and cell ${i+1,j,k}$.
%
Expanding from the left side of the face, we have
%--------------------------------------------------
\begin{align}
\vu^{n+1/2,\text{L}}_{i+1/2,j,k} = \vu^{n}_{i,j,k} + \frac{\Delta x}{2} \frac{\p \vu}{\p x} \bigg |^n_{i,j,k} + \frac{\Delta t}{2} \frac{\p \vu}{\p t} \bigg |^n_{i,j,k}.
\label{eq:taylor_left}
\end{align}
From the momentum balance equation (Eq.~\eqref{eq:full_mmt}) we obtain an expression for $\p \vu/\p t$ as
%--------------------------------------------------
\begin{align}
\frac{\p \vu}{\p t}\bigg |^n_{i,j,k}  = - \left ( u \frac{\p \vu^n}{\p x}
+ \overline{\left(v\frac{\p \vu^n}{\p y}\right)}
+ \overline{\left(w\frac{\p \vu^n}{\p z}\right)} \right )_{i,j,k}
+ \frac{1}{\rho(\phi^n)_{i,j,k}} \left (- \nabla p^{n-1/2} + \nabla \cdot \vtau^n(\vu^n, \vxi^n, \phi^n) + \vb^n \right)_{i,j,k}.
\label {eq:time_partial}
\end{align}
The terms on the RHS of Eq.~\eqref{eq:time_partial} are evaluated at step $n$, except for
the pressure, which is evaluated at step $n-1/2$.
Since we are expanding in the $x$-direction, the barred terms in the bracket
are the tangential derivatives (also called transverse derivatives) and they
are treated differently; details will be discussed later in this subsection.
Substituting Eq.~\eqref{eq:time_partial} into Eq.~\eqref{eq:taylor_left}, we have
%--------------------------------------------------
\begin{align}
  \vu^{n+1/2,\text{L}}_{i+1/2,j,k} &= \vu^{n}_{i,j,k} + \left( \frac{\Delta x}{2} - \frac{\Delta t}{2} u^n_{i,j,k} \right) \frac{\p \vu}{\p x} \bigg |^n_{i,j,k}
                                   - \frac{\Delta t}{2} \left (\overline{\left(v\frac{\p \vu^n}{\p y}\right)} + \overline{\left(w\frac{\p \vu^n}{\p z}\right)} \right )_{i,j,k} \nonumber \\
                                   &\phantom{=} +  \frac{\Delta t }{2 \rho(\phi^n)_{i,j,k}} \left (- \nabla p^{n-1/2} + \nabla \cdot \vtau^n(\vu^n, \vxi^n, \phi^n) + \vb^n \right)_{i,j,k}.
\label{eq:left_final}
\end{align}
Similarly, Taylor expanding from the right side of the cell face, we have
%--------------------------------------------------
\begin{align}
  \vu^{n+1/2,\text{R}}_{i+1/2,j,k} &= \vu^{n}_{i+1,j,k} - \left( \frac{\Delta x}{2} + \frac{\Delta t}{2} u^n_{i+1,j,k} \right) \frac{\p \vu}{\p x} \bigg |^n_{i+1,j,k} 
                                   - \frac{\Delta t}{2}\left (\overline{\left(v\frac{\p \vu^n}{\p y}\right)} + \overline{\left(w\frac{\p \vu^n}{\p z}\right)} \right )_{i+1,j,k} \nonumber \\
  & \phantom{=}+ \frac{\Delta t}{2\rho(\phi^n)_{i+1,j,k}} \left (- \nabla p^{n-1/2} + \nabla \cdot \vtau^n(\vu^n, \vxi^n, \phi^n) + \vb^n \right)_{i+1,j,k}.
\label{eq:right_final}
\end{align}
The Taylor expansion procedure to create values of $\vxi^{n+1/2}_{i+1/2, j, k}$ makes use of Eq.~\eqref{eq:xi_adv}, which has a much simpler form. Expanding from the left and right sides respectively gives
%--------------------------------------------------
\begin{align}
\vxi^{n+1/2,\text{L}}_{i+1/2,j,k} =& \vxi^{n}_{i,j,k} + \left( \frac{\Delta x}{2} - \frac{\Delta t}{2} u^n_{i,j,k} \right) \frac{\p \vxi}{\p x} \bigg |^n_{i,j,k}
- \frac{\Delta t}{2} \left (\overline{\left(v\frac{\p \vxi^n}{\p y}\right)} + \overline{\left(w\frac{\p \vxi^n}{\p z}\right)} \right )_{i,j,k},  \\
%
\vxi^{n+1/2,\text{R}}_{i+1/2,j,k} =& \vxi^{n}_{i+1,j,k} - \left( \frac{\Delta x}{2} + \frac{\Delta t}{2} u^n_{i+1,j,k} \right) \frac{\p \vxi}{\p x} \bigg |^n_{i+1,j,k}
- \frac{\Delta t}{2} \left (\overline{\left(v\frac{\p \vxi^n}{\p y}\right)} + \overline{\left(w\frac{\p \vxi^n}{\p z}\right)} \right )_{i+1,j,k}.
\label{eq:xi_taylor_final}
\end{align}

\subsubsection*{Godunov-type upwinding scheme}
\label{sub:godunov}
Now there are two choices for velocity and two for reference map on the cell face indexed by ${i+1/2,j,k}$, we perform Godunov upwinding to select one.
We define the normal velocity as $u_{\nor}^{n+1/2, \text{L}}$ and $u_{\nor}^{n+1/2, \text{R}}$, expanded from the left and the right cells, respectively.
We use $q^{n+1/2, \text{L}}$ and $q^{n+1/2, \text{R}}$ to represent other components of the velocity or the reference map, expanded from the left and the right, respectively.
To choose the normal velocity, we follow
%--------------------------------------------------
\begin{align}
u_{\nor}^{n+1/2} =
\begin{cases}
u_{\nor}^{n+1/2, \text{L}} & \text{if~} u_{\nor}^{n+1/2, \text{L}}  > 0 \text{~and~} u_{\nor}^{n+1/2, \text{L}}  + u_{\nor}^{n+1/2, \text{R}}  > 0, \\
u_{\nor}^{n+1/2, \text{R}} & \text{if~} u_{\nor}^{n+1/2, \text{R}} < 0 \text{~and~} u_{\nor}^{n+1/2, \text{L}}  + u_{\nor}^{n+1/2, \text{R}}  < 0, \\
0  & \text{otherwise.}
\end{cases}
\end{align}
After this, we select a value for the other quantities at the cell face based on $u_{\nor}^{n+1/2}$ according to
%--------------------------------------------------
\begin{align}
q^{n+1/2} =
\begin{cases}
q^{n+1/2, \text{L}} & \text{if~} u_{\nor}^{n+1/2}  > 0, \\
\frac{q^{n+1/2, \text{L}}  + q^{n+1/2, \text{R}} }{2}  & \text{if~} u_{\nor}^{n+1/2} = 0, \\
q^{n+1/2, \text{R}} & \text{if~} u_{\nor}^{n+1/2} < 0.
\end{cases}
\end{align}
Here we have dropped all spatial index subscripts since all terms are evaluated at $i+1/2, j, k$.

\subsubsection*{Normal derivatives}
\label{sub:norm}

To compute the normal derivatives (in the example given here, $\p \vu/\p x$ and  $\p \vxi/\p x$), we employ the fourth-order monotonicity-limited scheme of Collela~\cite{colella90}, which is described in detail by Yu \textit{et al.}~\cite{yu03} and Rycroft \textit{et al.}~\cite{rycroft20}.

\subsubsection*{Tangential derivatives}
\label{sub:tang}

To ensure stability, especially in intermediate to high Reynolds number regime, we compute the tangential derivatives (shown as the barred terms in Eqs.~\eqref{eq:left_final}, \eqref{eq:right_final}, \& \eqref{eq:xi_taylor_final}) using an upwinding scheme which is commonly used for solving hyperbolic conservative laws ~\cite{bell89,almgren96,sussman99,yu03}.
Our approach to construct an upwinding scheme in 3D is only one of the many possibilities, and in selecting the following scheme, we prioritize algorithmic simplicity and the ease implementation.

Without loss of generality, we consider the quantities on the transverse faces indexed by ${i,j+1/2,k}$, required in Eq.~\eqref{eq:left_final},~\eqref{eq:right_final},~\eqref{eq:xi_taylor_final} when applied to faces indexed by ${i+1/2,j,k}$.
The procedure to compute tangential derivative terms along the $z$-direction is similar.
We start by performing a Taylor expansion again to construct the velocities and reference maps on the transverse cell faces, but neglect contributions from pressure, stress, body forces, and tangential derivatives to obtain
%--------------------------------------------------
\begin{align}
\bar \vu^{n+1/2,\text{D}}_{i,j+1/2,k} =& \vu^{n}_{i,j,k} + \left( \frac{\Delta y}{2} - \frac{\Delta t}{2} v^n_{i,j,k} \right) \frac{\p \vu}{\p y} \bigg |^n_{i,j,k}, \\
\bar \vu^{n+1/2,\text{U}}_{i,j+1/2,k} =& \vu^{n}_{i,j+1,k} - \left( \frac{\Delta y}{2} + \frac{\Delta t}{2} v^n_{i,j+1,k} \right) \frac{\p \vu}{\p y} \bigg |^n_{i,j+1,k}, \\
%
\bar \vxi^{n+1/2,\text{D}}_{i,j+1/2,k} =& \vxi^{n}_{i,j,k} + \left( \frac{\Delta y}{2} - \frac{\Delta t}{2} v^n_{i,j,k} \right) \frac{\p \vxi}{\p y} \bigg |^n_{i,j,k}, \\
\bar \vxi^{n+1/2,\text{U}}_{i,j+1/2,k} =& \vxi^{n}_{i,j+1,k} - \left( \frac{\Delta y}{2} + \frac{\Delta t}{2} v^n_{i,j+1,k} \right) \frac{\p \vxi}{\p y} \bigg |^n_{i,j+1,k},
\end{align}
where the superscripts D and U denote down and up, respectively.
We perform a similar Godunov upwinding procedure to select a normal advective velocity at each face, so that
%--------------------------------------------------
\begin{align}
\bar v_{\text{adv}}^{n+1/2} =
\begin{cases}
\bar v^{n+1/2, \text{D}} & \text{if~} \bar v^{n+1/2, \text{D}} > 0 \text{~and~}  \bar v^{n+1/2, \text{D}} +  \bar v^{n+1/2, \text{U}} > 0, \\
\bar v^{n+1/2, \text{U}} & \text{if~} \bar v^{n+1/2, \text{U}} < 0 \text{~and~}  \bar v^{n+1/2, \text{D}} +  \bar v^{n+1/2, \text{U}} < 0, \\
0  & \text{otherwise,}
\end{cases}
\end{align}
where we have dropped the subscripts since all terms are evaluated at $i,j+1/2,k$.
Next, we select between Taylor expansions $\bar \vu^{n+1/2,\text{D}}$ and $\bar \vu^{n+1/2,\text{U}}$, as well as between expansions $\bar \vxi^{n+1/2,\text{D}}$ and $\bar \vxi^{n+1/2,\text{U}}$, based on $\bar v_{\text{adv}}^{n+1/2}$.
For two generic vector quantities at the face denoted by $\bar \vq^{n+1/2,\text{D}}$ and $\bar \vq^{n+1/2,\text{U}}$, we define
%--------------------------------------------------
\begin{align}
\bar \vq^{n+1/2} =
\begin{cases}
\bar \vq^{n+1/2, \text{D}} & \text{if~} \bar v_\text{adv}^{n+1/2} > 0, \\
\left ( \bar \vq^{n+1/2, \text{D}} + \bar \vq^{n+1/2, \text{U}} \right) / 2 & \text{if~} \bar v_\text{adv}^{n+1/2} = 0, \\
\bar \vq^{n+1/2, \text{U}} & \text{if~} \bar v_\text{adv}^{n+1/2} < 0.
\end{cases}
\end{align}
Finally, we compute the tangential derivative terms,
%--------------------------------------------------
\begin{align}
\overline {\left( v\frac{\p \vq}{\p y} \right) }\bigg |^{n+1/2} _{i,j,k}
= \frac{\bar v^{n+1/2}_{i,j-1/2, k,\text{adv}} + \bar v^{n+1/2}_{i,j+1/2,k, \text{adv}}}{2} \frac{\bar \vq^{n+1/2}_{i,j+1/2,k} - \bar \vq^{n+1/2}_{i,j-1/2,k}}{\Delta y}.
\end{align}



%--------------------------------------------------
% Section:
%--------------------------------------------------
\subsection{Contact with a wall}
We use subscript $\alpha$ to represent walls of the rectangular domain, with $\alpha=1,2,3,4,5,6$ referring to walls with normal vectors $\vec{e}_x, -\vec{e}_x, \vec{e}_y, -\vec{e}_y, \vec{e}_z, -\vec{e}_z$, respectively.
The inward unit normal at a wall is denoted by $\nor_\alpha$.
When a part of the solid body is within a threshold distance from a wall with $\nor_\alpha$, we impose a repulsive acceleration, $\mathbf a_{\text{rep},\alpha}$, to the grid cells in the solid body that have breached the threshold, acting at the cell centers.
This acceleration is multiplied by a transition function in a small transition zone of width $\Delta h_\alpha$, where $\Delta h_\alpha$ is the grid spacing in the Cartesian direction aligned with $\nor_\alpha$.

For those solid grid cells that experience wall acceleration,
%--------------------------------------------------
\begin{equation}
  \mathbf a_{\text{rep}, \alpha} = \left\{
  \begin{array}{ll}
    A_\text{rep}  \nor_\alpha & \qquad \text{if $\phi\le-\Delta h_\alpha$,} \\
    \left ( \frac{1}{2}- \frac{\phi}{2\Delta h_\alpha} \right) A_\text{rep}  \nor_\alpha & \qquad \text{if $|\phi|<\Delta h_\alpha$,} \\
    0 & \qquad \text{if $\phi\ge\Delta h_\alpha$}.
  \end{array}
  \right.
  \label{eq:wall_acc}
\end{equation}
If a solid body is in close approach to more than one wall, the repulsive accelerations are added together.
As the solid approaches the wall and deforms to form a contact area, a small region of cells experiences the repulsive acceleration described by Eq.~\eqref{eq:wall_acc}.

We use the following approach to choose an appropriate value for $A_\text{rep}$.
First, given the velocity of an object at its center of mass, the repulsive acceleration should be sufficient to stop the object from going through any physical boundary.
In a rigid body, this means that the center of mass velocity is zero before or when the edge of the solid reaches the boundary.
Though the solid objects will deform in our simulations, the rigid body case offers good guiding intuition.
Second, we impose the condition that the repulsive acceleration, at a minimum, should be able to support an object resting against a wall.

We first define a critical threshold, $d_w$. When the distance between any cell belonging to the solid object and the wall falls below $d_w$, the wall repulsive acceleration comes into effect.
In practice, we set $d_w = \Delta h_\alpha$ so that the solid objects are sufficiently close to the wall and experience effects due to boundaries,
however, $d_w$ can be increased so that the repulsive acceleration can be applied to more grid cells.
%\chr{So it sounds that in the current procedure the wall contact forces are limited to just a single layer of grid points? In the 2D code the wall contact forces are applied over several grid points, although it's easier in 2D since the resolution is higher.}

Let us consider an object of volume $V$ approaches the wall with a normal velocity component $U$, which has a momentum along the normal direction, $\rho_s V U$.
Suppose the volume of the object that actually experiences repulsive acceleration is $V_\text{rep}$,
the effective acceleration at the center of mass is $\frac{V_\text{rep} A_\text{rep} }{ V}$.
Following the reasoning of the first case above,
%--------------------------------------------------
\begin{equation}
\begin{aligned}
&U  - \frac{V_\text{rep} A_\text{rep} }{ V} t_s = 0, \\
&U t_s - \frac12 \frac{V_\text{rep} A_\text{rep} }{ V} t_s^2 = d_w.
\end{aligned}
\end{equation}
Solving these equations yields $A_\text{rep} = \frac{V U^2}{2d_wV_\text{rep}}$ and $t_s = 2d_w/U$.

We use the smallest length scale of the solid, $L_\text{min}$,  to obtain a conservative estimate of the contact area between the solid and the wall, and use $d_w$ as the thickness of the region of cells that breach this threshold.
Thus a conservative estimate of $V_\text{rep}$ is $V_\text{rep} = d_w L^2_\text{min}$, which yields
%--------------------------------------------------
\begin{equation}
A_\text{rep} =\frac{V U^2}{2d_w^2 L^2_\text{min}}.
\label{eq:rep1}
\end{equation}
In the second case described above, when the object is at rest at a wall, we need to make sure that the forces experienced by the region of cells that have breached the $d_w$ threshold is large enough to support the entire object, so that $|\rho_s - \rho_f| Vg = \rho_s V_\text{rep} A_\text{rep}$ and hence
\begin{equation}
	A_\text{rep} = \frac{|\rho_s - \rho_f| Vg}{ \rho_s V_\text{rep}}.
	\label{eq:rep2}
\end{equation}
In practice, we also find that it is helpful to set a minimum value of $A_\text{re} = U^2_\text{safe}/d_w$, $U_\text{safe} \approx 4.5$, as an additional safety measure to prevent solid objects from moving through physical boundaries. Given this lower limit and Eqs.~\eqref{eq:rep1},~\eqref{eq:rep2}, we choose the maximum value among the three to be the value of $A_\text{rep}$,
%--------------------------------------------------
\begin{equation}
 A_\text{rep} = \max \left( U^2_\text{safe}/d_w, \frac{V U^2}{2d_w^2 L^2_\text{min}}, \frac{|\rho_s - \rho_f| Vg}{ \rho_s V_\text{rep}} \right).
\end{equation}
For simulations with multiple objects, $A_\text{rep}$ is computed for each object separately.

%\iffalse
%Thus, we replace $A_\text{rep}$ with $A'_\text{rep}$.  \chr{I think using the language of ``replacing'' $A_\text{rep}$ with $A'_\text{rep}$ is a bit hard to follow. I think it would be better keep the definitions distinct, e.g. start with $A_\text{full}$ to denote the acceleration applied to the full body, and then $A_\text{rep}$ to denote the repulsion applied in the contact area. With that, you don't have to switch the meaning of a quantity halfway through.} In practice, as the solid body approaches the wall, it deforms to form a larger ``contact'' area, so the estimate of $d_w^3$ as the volume affected by the wall acceleration leads to a larger $A_\text{rep}$ than what may be actually required. Since we do not know the ``contact'' area a priori, we use a parameter to moderate the size of $A'_\text{rep}$. If the acceleration $A'_\text{rep} $ is too large, we will set it to a maximum value that depends on the simulation time step size.
%
%If a rigid body settles toward a wall and reaches terminal velocity before reach the wall, from the drag equation for a rigid solid body with reference area $A$, density $\rho_s$, settling in fluid with density $\rho_f$, subject to gravitational acceleration $g$,
%the terminal velocity squared, along the settling direction, (in which limit)
%\begin{equation}
%	U^2 = \frac{1}{2}\frac{(\rho_s - \rho_f) gV} {C_dA \rho_f}
%\end{equation}
%where $C_d$ is the drag coefficient, though it is well-characterized for solid bodies such as a rigid sphere, in this calculation we assume it to be order 1.
%The reference area cannot be known a priori since the solid bodies are deformable and often undergo rotations, therefore, we estimate it with the smallest cross-area of the object.
%
%In general, we can compute the normal velocity at the center of mass of an object. So instead of an acceleration, we use an acceleration multiplier,
%$$
%	A_\text{mul} =  \beta \frac{V}{2d_w^4}.
%$$
%In our simulations, we have chosen $\beta = 0.01$.
%
%When the object settles at a wall and reaches equilibrium, we ensure the forces experienced by the ``contact'' region is large enough to support the buoyancy force experience by the object's center of mass
%\begin{equation}
%	|\rho_s - \rho_f| Vg = \rho_s \beta d_w^3 A^s_\text{rep},
%\end{equation}
%where $\beta$ is a constant that account for the characteristic width of the contact area. Therefore,
%\begin{equation}
%	A^s_\text{rep} = \frac{|\rho_s - \rho_f| Vg}{\beta \rho_s d_w^3}.
%\end{equation}
%Index each object in the simulation by $i$, $i=0, 1, \ldots, S-1$, where $S$ is the total number of objects. At each timestep, compute the velocities of each object's center of mass in the normal direction, $U_i$.
%We choose our wall acceleration constant by
%\begin{equation}
%  \max_{i\in\{0, \ldots, S-1\}} \left( A_{\text{mul}, i} U_i, A^s_{\text{rep}, i}\right).
%\end{equation}
%\fi
%
%--------------------------------------------------
% Section:
%--------------------------------------------------
\subsection{Weighting scheme for weighted least squares-based extrapolation}
In the absence of the reinitialized level-set function, we use an exponential decay kernel centered at the extrapolated cell to weigh data points in the linear squares regression.
Near the fluid--solid interface we incorporate approximate geometric information via $\phi_0(\vxi)$.

Consider a cell $(p,q,r)$ which provides data $\vxi_d$ at position $\vx_d=(x_p, x_q, x_r)$ in the weighted least squares linear regression to extrapolate
to $\vxi_e$ at cell $(i,j,k)$ with position $\vx_e = (x_i, x_j, x_k)$.
First, we compute the normalized gradient vector $\nor_e$ of $\phi_0(\vxi^n)$ at cell $(i,j,k)$
as well as a physical vector $\vx' = \vx_e - \vx_d$.
Then, depending on which layer extrapolation $\vxi_e$ resides in, the weight of data provided by cell $(p,q,r)$ is defined as
%--------------------------------------------------
\begin{equation}
\label{eq:weight}
\omega=
\left \{
  \begin{array}{ll}
   \max\left( 0, \frac{\vx' \cdot \nor_e}{|\vx'|} {2 ^{- (r_i + r_j + r_k)}} \right) & \quad \text{if $l_{i, j, k}\le2$, } \\
    {2 ^{- (r_i + r_j + r_k)}} &  \quad \text{if $l_{i, j, k}>2$,}
  \end{array}
  \right.
\end{equation}
where $r_i = |i-p|, r_j=|j-q|, r_k = |k-r|$ and $l_{p,q,r}$ is the index of the extrapolation layer as defined in the main text.
The exponential kernel accounts for locality on the Cartesian grid,
while $\nor_e$ approximates a surface normal and accounts for geometric information near the interface.
However, we only make use of $\nor_e$ in the extrapolating to the first two layers,
since $\phi (\vxi^n)$ approximates a signed-distance function more poorly farther away from the interface in the presence of deformation.

%--------------------------------------------------
% Section:
%--------------------------------------------------
\subsection{Configuring a simulation}
\label{sub:setup}
Before commencing a simulation, we need expressions of level-set functions that represent the fluid--solid interface.
Each solid body is associated with such as level-set function.
In addition, we need to configure the simulation domain, and prescribe material properties, boundary conditions, and initial conditions.
Besides physics-related parameters, we  also need to specify the type of data to collect and the frequency of outputting data files and checkpoint files.
Main parameters that specify the physics in a simulation are documented in Table~\ref{table:pars}.

\subsubsection*{Initialization}
\begin{enumerate}
    \item Set initial conditions for velocity $\vu_{i,j,k}^0$, and for reference map $\vxi_{i,j,k}^0$ for all $i, j, k$ such that $\phi_0(\vxi_{i,j,k}^0) \le 0$.
    We refer to these reference map variables as primary reference maps (PRMs).
    Using the undeformed state as the reference state of a solid body, the primary reference map variable is set to be the physical coordinate on the grid, $\vxi_{i,j,k}^0 = \vx_{i,j,k}$, where $\vx=(x_i, y_j, z_k)$.

    \item If a test case has known pressure, also set $p_{i,j,k}^{-1/2}$, otherwise, we may set pressure to 0 everywhere.
    However, since pressure is computed as an auxiliary field rather than from an equation of state, the initial pressure profile is not known in general.
    Therefore, $p_{i,j,k}^{-1/2}=0$ is generally not compatible with the initial velocity.
    The incompatibility can degrade the accuracy of the solution.

    \qquad To address this, we perform initial iterations to estimate $p_{i,j,k}^{-1/2}$. We run the timestepping algorithm, described in subsection \ref{sub:timestepping}, with an initial guess of $p_{i,j,k}^{-1/2}=0$.
    After an iteration, we keep the pressure estimate $\tilde p_{i,j,k}^{1/2}$ but discard changes to all other fields.
    In the next iteration, let $p_{i,j,k}^{-1/2} = \tilde p_{i,j,k}^{1/2}$, and repeat until a tolerance on the divergence of $\vu_{i,j,k}^0$ is reached, or until a maximum number of iterations is reached.
    Note that in order to carry out initial iterations, the variant of approximate projection method that computes the pressure increments should be used \cite{almgren00}.

    \item If applicable, initialize $\mathbf b_{i,j,k}^0$.
    \item Extrapolate reference map variables to the blur zones.
    Extrapolated reference maps (ERMs) and the PRMs (together denoted as  $\vxi_{i,j,k}^0$) are both required to begin the simulation.
    \item Evaluate $\phi_0(\vxi^0)$ everywhere reference map variables  $\vxi_{i,j,k}^0$ exist.
    \item Initialize $\rho_{i,j,k}^0$ by blending $\rho_f$ and $\rho_s$ using using $H_\epsilon(\phi_0(  \vxi_{i,j,k}^0 ))$ (Eq.~\eqref{eq:mod_heaviside}).
\end{enumerate}

%
%--------------------------------------------------
% Section:
%--------------------------------------------------
\subsection{Timestepping algorithm}
\label{sub:timestepping}

After initializing the simulation, we perform the following steps for timestep $n=0, 1, 2, \ldots, U$ of size $\Delta t$.
This procedure is also used in the iterations to create an initial pressure profile.
\begin{enumerate}
	\item Compute $\vtau_f^n$, $\vtau_s^n$, and $\vtau_a^n$, and mix them to get $\vtau^n$, as described in subsection \ref{sub:overall}.

    \item Using the Godunov-type upwinding scheme described in subsection \ref{sub:godunov}, we compute $\vu_{i\pm1/2,j,k}^{n+1/2}, \vu_{i,j\pm1/2,k}^{n+1/2}, \vu_{i,j,k\pm1/2}^{n+1/2}$ everywhere, and $\vxi_{i\pm1/2,j,k}^{n+1/2}, \vxi_{i,j\pm1/2,k}^{n+1/2}, \vxi_{i,j,k\pm1/2}^{n+1/2}$ only within solids.

	\item Perform the MAC projection, then update the normal face velocities.
	Any boundary conditions should be enforced before the projection step and after the velocity update.

	\item Compute the advective term $ [(\vu \cdot \nabla) \vu ]_{i,j,k}^{n+1/2}$ and $ [(\vu \cdot \nabla) \vxi ]_{i,j,k}^{n+1/2}$ according to Eqs.~\eqref{eq:adv_terms}.

	\item Compute the predictor values of PRMs, i.e. $\vxi_{i,j,k}^{n+1}$ within solids (Eq.~\eqref{eq:dis_xi_adv}).

	\item Extrapolate PRMs to yield ERMs to cover the blur zones, these are the predictor values of ERMs.

	\item Apply corrector to all available reference maps (PRMs and ERMs), $\vxi_{i,j,k}^{n+1/2} = \frac12 \left(\vxi_{i,j,k}^{n} + \vxi_{i,j,k}^{n+1}\right) $.

	\item Compute solid stress  $\vtau_s^{n+1/2}$ and artificial viscous stress $\vtau_a^{\tilde{n}}$, then mix with $\vtau_f^n$ to create $\vtau^{n+1/2}$, as described in subsection \ref{sub:overall}.

	\item Update density to $\rho_{i,j,k}^{n+1/2}$ using $H_\epsilon(\phi_0(  \vxi_{i,j,k}^{n+1/2} ))$ (Eq.~\eqref{eq:mod_heaviside}).

	\item Compute the intermediate velocity for approximate projection, $\vu_{i,j,k}^{*}$, by
	%--------------------------------------------------
	\begin{equation} \vu_{i,j,k}^{*} = \vu_{i,j,k}^n - \Delta t [(\vu \cdot \nabla) \vu ]_{i,j,k}^{n+1/2} - \frac{1}{\rho_{i,j,k}^{n+1/2}} \left( \nabla \cdot  \vtau^{n+1/2} - \nabla p_{i,j,k}^{n-1/2} + \mathbf b_{i,j,k}^{n+1/2}\right)
	\end{equation}

	\item Perform the approximate projection, then update cell center velocities to $\vu_{i,j,k}^{n+1}$ and nodal pressure to $p_{i,j,k}^{n+1/2}$. Any boundary conditions should be enforced before the projection step and after updating velocity and pressure.

	\item Restore all PRMs and ERMs to their predictor values, $\vxi_{i,j,k}^{n+1}$.

	\item Update the density to $\rho_{i,j,k}^{n+1}$ using $H_\epsilon(\phi_0(  \vxi_{i,j,k}^{n+1/2} ))$ (Eq.~\eqref{eq:mod_heaviside}).
\end{enumerate}
%We omit discussions on the finite difference schemes to compute stress tensors $\vtau_f$, $\vtau_s$, and $\vtau_a$, or the procedure of the MAC projection and the approximate projection methods, since they are an extension of the two-dimensional implementations detailed by Rycroft \textit{et al.} \cite{rycroft20}.

%--------------------------------------------------
% Section:
%--------------------------------------------------
\section{Test cases}
    In this section, we first present simulations of a pre-stretched, immersed sphere to demonstrate that
    the convergence of our method, as well as the scaling behavior of the simulated FSI problem, are as expected.
    We also provide additional details and results of examples mentioned in the main text.

    %--------------------------------------------------
    %	 Subsection
    %--------------------------------------------------
	\subsection{Pre-stretched sphere}  We simulate a viscoelastic neo-Hookean solid sphere, subject to initial strain, relaxing to equilibrium in a cubic periodic domain.
	Parameters reported here are dimensionless.


	\begin{enumerate}
		\item The sphere has a radius $R=0.2$ and a shear modulus $G=1$, positioned at the center of a periodic box with unit length, fluid viscosity $\mu = 0.01$.
		The sphere has the same density as the fluid, and is pre-strained at $T=0$ in the $x$-direction by a stretch $\lambda=1.21$.
		We use isotropic grid spacing $\Delta x = \Delta y = \Delta z = h$.
		The solution with the smallest $h$ is used as the reference solution in the convergence plot.

		\qquad Elastic energy and volume of the sphere over time are reported in Fig.~\ref{fig:str_conv}(a),
		and the convergence in surface area, volume, and $\|J-1\|_2$ is reported in Fig.~\ref{fig:str_conv}(b), where
		$J$ is the determinant of the deformation gradient.
		The pressure-Poisson projection method we use to enforce incompressibility is an approximate one,
		which means locally, it is possible to have $J \ne 1$.
        To account for the local violation of incompressibility, we compute strain energy density with a logarithmic correction factor \cite{flory53},
        %\chr{Is there a citation for this logarithmic correction factor?} (a similar term is in the strain energy density function of a compressible neo-Hookean material),
		\begin{align}
		\label{eq:elas}
		\Phi_\text{elas} = \frac12 G\big (\Tr(\vC) - 3 - 2\ln(J) \big)
		\end{align}
		where $\vC=\vF^\trans \vF$ is the right Cauchy--Green tensor.
		This strain energy density equation is used throughout our analysis.

		\qquad Since we use a reference solution rather than the exact solution,
		only part of the error can be captured by a Richardson error model \cite{richardson11, hairer93, heath02, rycroft20}.
		For other sources of error, e.g.~reference map extrapolations, Godunov-type upwinding procedures, Richardson model is not a good fit.
		Thus, we adopt the 3-parameter error model proposed by Rycroft \textit{et al.}~\cite{rycroft20}
		\begin{align}
		\label{eq:3params}
		E(h) = B(h^s - \alpha h_\text{ref}^s) + \mathcal{O}(h^{s+1})
		\end{align}
		where $\alpha$ is the Richardson correction factor which indicates the proportion in the error that can be captured by a Richardson error model, and $s$ is the overall convergence rate.
		We report the fitted convergence parameters in the caption of Fig.~\ref{fig:str_conv}.

		\item We demonstrate scaling of elastic energy of the immersed sphere in various combinations of $G$, $\mu$, $\rho_s/\rho_f$, and $R$ (Fig.~\ref{fig:str_scaling}). First, we find the pertinent parameter in our test case by dimensional analysis.
		Consider a standalone, ideal, incompressible neo-Hookean elastic body with density $\rho_s=\rho$, shear modulus $G$, and a characteristic length scale $R$.
		In the absence of viscous damping, uniaxial tension or compression induces an oscillatory response due to elastic restoring forces in the solid body.
		The oscillation time scale is $\tau_\text{osc} = R\sqrt{\rho/G}$.
		In addition to being the characteristic oscillation period, this time scale can also be interpreted as the time it takes for elastic waves to traverse the elastic body.
		%
		Now, suppose the elastic body is immersed in a viscous fluid, with viscosity $\mu$ and density $\rho_f = \rho$.
		In our method, the solid material is made viscoelastic by using a  constant artificial viscosity $\mu_a$.
		We impose $\mu_a = \mu$ in this case.
		Out of the parameters relevant to the fluid $\rho, \mu, R$, we can build another time scale, $\tau_\text{diss} = \rho R^2 / \mu$.
		We can interpret $\tau_\text{diss}$ as the time scale of the fluid dissipating the kinetic energy generated by the oscillating sphere as it returns to equilibrium.

		\qquad The ratio between these two time scales is
		$$\gamma = \tau_\text{diss} / \tau_\text{osc} = {R \sqrt{\rho G}}/{\mu}.$$
		%
		In the simple system we have considered here, having assumed solid and fluid have identical densities $\rho$, and viscosity $\mu_a = \mu$, we have four dimensional parameters $\mu, G, \rho, R$.
		By the Buckingham $\Pi$ Theorem, $\gamma$ is the only dimensionless parameter in this system.
		It can be interpreted as the number of oscillations a viscoelastic sphere undergoes before returning to the equilibrium.
		Incidentally, we can also consider the relaxation time scale of the viscoelastic solid, $\tau_\text{rel} = \mu / G$.
		Taking the ratio $\tau_\text{osc} / \tau_\text{rel}$ also yield the dimensionless parameter $\gamma$.

		\qquad We carry out a series of simulations (parameters documented in Table~\ref{table:str_params}) to demonstrate the scaling behavior in this system.
		In Fig.~\ref{fig:str_scaling}(a), we show solid elastic energy from various simulations normalized by its initial value $E_0$.
		Values of $G$, $R$, and $\mu$ are varied across simulations, but dimensionless parameter $\gamma$ is kept the same.
		The number of visible oscillations before energy is fully dissipated by the fluid is the same across these simulations.
		After rescaling time by $\tau_\text{diss}$, we also find that peaks and valleys of the elastic energy fall at similar places on the rescaled time axis.
		Furthermore, given the same $\gamma$, if the sphere radius is kept the same, which implies $\sqrt{G} / \mu$ is constant, elastic energy curves collapse onto the same one (simulation $A_1$ and $A_2$).

		\qquad In Fig.~\ref{fig:str_scaling}(b), we show the number of visible oscillations is qualitatively linearly proportional to $\gamma$.
		Simulation $B_1$, which has $\gamma = 10$, is compared to simulation $B_2$, $A_1$, and $B_3$, which have have $\gamma = 15.8, 20, 30$, respectively.
		The orange dashed line indicates approximately one oscillation period in simulation $B_1$ after initial release.
		Within the same rescaled time (in units of respective $\tau_\text{diss}$), we observe about 1.5, 2, and 3 times more periods in simulation  $B_2$, $A_1$, and $B_3$, respectively. The increase in the number of periods follows the increase in $\gamma$ approximately linearly.

		\qquad If the solid has a different density from fluid, the matter becomes more complex.
		Though the oscillation time scale $\tau_\text{osc}$ should only depend on $\rho_s$, the energy dissipation, influenced by dynamics in both the solid and the fluid phases, now depends on both $\rho_s$ and $\rho_f$.
		We demonstrate this effect in Fig.~\ref{fig:str_scaling}(c).
		In simulation $C_1$ and $C_2$, we have chosen to use $\rho_s$ in calculating $\gamma$ and $\tau_\text{diss}$.
		While the number of visible oscillations remains similar among simulations $A_1$, $C_1$, and $C_2$, which have identical $\gamma$,
		rescaling time with $\tau_\text{diss}$ no longer lead to the alignment of energy peaks and valleys on the rescaled time axis across simulations.
	\end{enumerate}

    %--------------------------------------------------
    %	 Subsection
    %--------------------------------------------------
	\subsection{Settling}
		We provide \texttt{si\textunderscore movie\textunderscore 1.mov} {Settling of 150 ellipsoids, described in main text Fig.~3.}
	
    %--------------------------------------------------
    %	 Subsection
    %--------------------------------------------------
	\subsection{Lid-driven Cavity} Here we provide results in comparing lid-driven cavity flow simulated with RMT3D without any solid objects with 3D benchmarks as a validation test for our Navier--Stokes solver code.
	In addition, we present convergence tests and centroid positions data of an immersed sphere in cubic lid-driven cavity flow.
	Convergence rates are computed using Eq.~\eqref{eq:3params} and reported in the figure caption when applicable.
	Finally, we provide additional simulation parameters used in Fig.~4 in the main text, and include movies mentioned there.
	\begin{enumerate}
		%--------------------------------------------------
		\item Comparison of normal velocities and pressure profiles in a cubic lid-driven cavity with Reynolds number $\Rey=1000$ against benchmark results is shown in Fig.~\ref{fig:aspect1_profiles_re1000}; convergence of normal velocity extrema and steady state kinetic energy is shown in Fig.~\ref{fig:aspect1_conv}.

		%--------------------------------------------------
		\item Comparison of normal velocities and pressure profiles in a cubic lid-driven cavity with Reynolds number $\Rey=100, 400$ against benchmark results is shown in Fig.~\ref{fig:aspect1_profiles_re400}.

		%--------------------------------------------------
        \item Comparisons of normal velocities and pressure profiles in a lid-driven cavity of various aspect ratios with $\Rey=1000$ against benchmark results are shown in Figs.~\ref{fig:aspect23_profiles} \& \ref{fig:aspectz2_profiles}.

		%--------------------------------------------------
		\item Convergence in surface area, volume, elastic energy,  and $\|J-1\|_2$, where $J$ is the determinant of the deformation gradient, for a sphere with $G=0.1$ in a cubic lid-driven cavity with $\Rey=100$ is shown in Fig.~\ref{fig:object_conv}.

		%--------------------------------------------------
		\item Additional parameters for simulations presented in Fig.~4 in the main text are documented in Table~\ref{table:lid_params}.

		%--------------------------------------------------
		\item \texttt{trajectories.txt} {trajectories.txt}{Trajectories of the centroid of a sphere with $G=0.1, 0.25, 0.5$ in a cubic lid-driven cavity with $Re=100$. Isotropic grid spacing $h=1/160$.}


		\item \texttt{si\textunderscore movie\textunderscore 2.mov} {A sphere of radius $0.2$ in unit cubic lid-driven cavity with $\Rey=100$ and solid shear modulus $G=0.03$, described in main text Fig.~4(a).}

		\item \texttt{si\textunderscore movie\textunderscore 3.mov}  {A sphere of radius $0.2$ in unit cubic lid-driven cavity with $\Rey=100$ and solid shear modulus $G=0.1$, described in main text Fig.~4(b)\&(c).}

		\item \texttt{si\textunderscore movie\textunderscore 4.mov} {A sphere of radius $0.2$ in unit cubic lid-driven cavity with $\Rey=100$ and solid shear modulus $G=0.25$, described in main text Fig.~4(d)\&(e).}

		\item \texttt{si\textunderscore movie\textunderscore 5.mov} {A sphere of radius $0.2$ in unit cubic lid-driven cavity with $\Rey=100$ and solid shear modulus $G=0.5$, described in main text Fig.~4(f)\&(g).}
	\end{enumerate}

    %--------------------------------------------------
    %	 Subsection
    %--------------------------------------------------
    \subsection{Swimming}
        In this section, we expand on some of the contexts and results from the section in the main text on simulating swimming with the RMT.
        
        \subsubsection*{Amplitude parameter} The amplitude parameter $W = B/3GIk^2$ is derived from scaling arguments and linear bending theory.
        Here, we explicitly list the relationships and scalings between bending moment amplitude $B$, displacement scaling constant $W$, and the vertical linear stress density $\sigma_0$.
        
        In Euler beam theory, the stress density is related to the bending moment across a cross-section as $\sigma_0 \sim B/I$, where $I$ is the area moment of inertia of the cross-section.
        There is a similar relationship to the beam curvature $\kappa \sim B/3GI$, where $3GI$ is the beam bending modulus for a shear modulus $G$.
        Given a vertical displacement scale $W$ and length scale $k^{-1}$, the curvature also satisfies $\kappa \sim Wk^2$ for $Wk\ll 1$. Thus, $W \sim B/3GIk^2$ and $\sigma_0 \sim 3WGk^2$, motivating the definitions in the main text.
    
        \subsubsection*{Swimming statistics} For each simulation, we calculate a swim speed $U$, active power $P$, drag coefficient $C$, and swimming efficiency $e$. Here, we detail the form of each and describe its calculation.
        
        We let $\left<\psi\right>$ describe the time-average and $\hat{\psi}$ the oscillation magnitude of a time-dependent value $\psi(t)$. To calculate these quantities, time traces $\psi(t)$ are fit to a function
        \begin{equation}
            f(t) = c_0 + c_1 t + c_2 t^2 + \sum_{j=1}^2 \left[A_j \cos(2\pi j t) + B_j \sin(2 \pi j t)\right].
        \end{equation}
        Then $\left<\psi\right> = c_0$, $\hat{\psi} = \sqrt{A_1^2+B_1^2}.$
        
        \begin{itemize}
            \item The swim speed $U = \left<u^{(c)}_x\right>$, where $\vu^{(c)}$ is centroid velocity and the subscript $x$ denotes the direction of swimming
            \item The power $P = \left<-\int_{\Omega_s} \vsigma^{(a)}:\nabla \vu dV\right>$
            where $\vsigma^{(a)}$ is the active part of the solid stress
            \item The drag coefficient $C$ is computed for each combination of swimmer body radius $R$ and length $L$.
            A simulation is conducted for each body shape with no active stress subject to a density ratio $\rho_s / \rho_f = 2$ and gravitational acceleration $g = 1/10$ in the swimming direction.
            The centroid velocity $u_x^{(c)}(t)$ is fit to the function
            \begin{equation}
                g(t) = U_f + (U_0-U_f) e^{-\lambda t},
            \end{equation}
            for an initial and final velocity $U_0$ and $U_f$ and a rate $\lambda$ representing the speed with which the object approaches terminal velocity. The drag coefficient is then calculated as $C = (\rho_s - \rho_f) V_s g / U_f$, where $V_s$ is the swimmer volume.
            \item The swimming efficiency is typically defined $e = F U/P$, where $F$ is the force required to tow the swimmer at velocity $U$.
            In the main text, we substitute an approximate tow force $\bar F = CU$ representing a linear estimate.
        \end{itemize}
    
        \subsubsection*{Reynolds numbers} We describe two Reynolds numbers: one corresponding to the steady flow induced by time-averaged motion of the object through the fluid, and another describing the oscillatory flow driven by the object's cyclic deformation. Here we describe the form of both and report the calculated values.
        
        \begin{itemize}
            \item The steady Reynolds number $\Rey_s = \rho U R_h/\mu$ uses the calculated swim speed as a velocity scaling and the swimmer head radius as a length scale.
            \item The oscillatory Reynolds number $\Rey_o = \rho \hat{u}^{(c)}_z R / \mu$ uses $\hat{u}^{(c)}_z$---the magnitude of the oscillating vertical velocity---as a velocity scale and the swimmer radius as a length scale.
        \end{itemize}
        
        For each simulation reported in Fig.~5 in the main text, the values of the two Reynolds numbers are calculated and documented in Fig.~\ref{fig:renums}.

          \subsubsection*{Movie} Finally, we provide \texttt{si\textunderscore movie\textunderscore 6.mov}
        {Swimmers with various body geometries, actuated by active stress, described in main text Fig.~5.}


\newpage
%------------------------------------
% ALL FIGURES, TABLES, ETC.
%------------------------------------

            %------------------------------------
            % FIGURE 1 VAR ARRANGEMENT
            %------------------------------------
            \begin{figure}
              \centering
               \includegraphics[width=0.75\textwidth]{si-schematic}
               \caption{
              Arrangement of variables in the three-dimensional reference map simulation. Depending on the finite difference scheme we use, variable can reside at cell centers, faces, or nodes.
              \label{fig:var_arr}}
            \end{figure}

            %--------------------------
            % TABLE OF SIM PARAMS
            %--------------------------
            \begin{table}\centering
                \caption{Main dimensionless simulation parameters in the RMT.
                In addition, Dirichlet and Neumann boundary conditions can be specified for each face of the rectangular simulation domain.}
                \label{table:pars}

                \begin{tabular} { c || l }
                    \hline
                    $\rho_f$ 	& fluid density \\\hline
                    $\mu$       	& fluid viscosity \\\hline
                    %
                    $\rho_s$ 	& solid density \\\hline
                    $G$      		& solid shear modulus \\\hline
                    $\mu_a$ 	& solid artificial viscosity \\\hline
                    $\gamma_t$  & blur zone viscosity constant \\\hline
                    $\epsilon$ 	& half blur zone width\\\hline
                    $L_x, L_y, L_z$ & simulation domain side lengths \\\hline
                \end{tabular}
            \end{table}

            %--------------------------
            % FIGURE STRETCHED SPH 1
            %--------------------------
            \begin{figure}
            \centering
                \includegraphics[width=\textwidth]{si-stre1}
                \caption{ (a) Elastic energy and volume of a sphere with radius $0.2$ and shear modulus $G=1$ in a periodic unit box.
                The incompressible sphere is stretched in $x$-direction by a factor of $1.44$, and compressed in the other directions accordingly.
                 Fluid viscosity $\mu=0.01$, and solid sphere has the same density as the fluid, $\rho_s=\rho_f=1$.
                 An incompressible neo-Hookean solid is modeled.
                 However,  $J \ne 1$ in our numerical scheme, especially near the fluid--solid interface. Therefore, a correction factor is used in the strain energy density (Eq.~\eqref{eq:elas}).
                 The correction approaches zero as the grid is refined, since $J$ converges to 1.
                (b) Convergence of errors in surface area, volume, elastic energy, and $\|J-1\|_2$, where $J$ is the determinant of the deformation gradient.
                Errors are measured at $T=0.2$.
                The solid lines are fitted curves using Eq.~\eqref{eq:3params}.
                Convergence rates (with Richardson correction factor $\alpha$ in brackets) are $1.34 (0.97), 1.44 (1.0), \text{~and~} 1.04 (1.0)$, for area, volume, and $\|J-1\|_2$, respectively.}
                \label{fig:str_conv}
            \end{figure}


            %----------------------------------------------------
            % TABLE FOR PARAMS IN STRETCHED SPH
            %----------------------------------------------------
   	        \begin{table}\centering
                \caption{Dimensionless simulation parameters for pre-stretched sphere scaling test. Fluid density $\rho_f$ is always kept at unity. Solid artificial viscosity $\mu_a$ is identical to fluid viscosity $\mu$.}
                \label{table:str_params}
                \begin{tabular} { c || c | c | c | c || c | c}
                    \hline
                    Case       	& $\mu$ & $\rho_s/\rho_f$ & $R$ & $G$  & $\gamma$  & $\tau_\text{diss}$ \\ \hline
                    %
                    $A_1$	& 0.005     &	1	& 0.2   & 0.25  & 20.0	& 8 \\ \hline
                    $A_2$	& 0.01      &	1	& 0.2   & 1     & 20.0 	& 4 \\ \hline
                    $A_3$	& 0.005     &	1	& 0.1   &  1    & 20.0 	& 2 \\ \hline
                    $A_4$	& 0.0075    &	1	& 0.3   & 0.25  & 20.0	& 12 \\ \hline
                    \hline
                    %
                    $B_1$	& 0.005 &   1   & 0.1   & 0.25 & 10.0  & 2 \\ \hline
                    $B_2$	& 0.02  &	1	& 0.2   & 2.5  & 15.8 	& 2 \\ \hline
                    $B_3$	& 0.005 &	1	& 0.3   & 0.25 & 30.0 	& 18 \\ \hline
                    \hline
                    %
                    $C_1$	& 0.005 &   0.5625  & $\tfrac{4}{15}$   & 0.25  & 20.0  & 8 \\ \hline
                    $C_2$	& 0.005  &	2.25    & $\tfrac{2}{15}$   & 0.25 	& 20.0 	& 8 \\ \hline
                 \end{tabular}
            \end{table}

	        %--------------------------
            % FIGURE STRETCHED SPH 2
            %--------------------------
            \begin{figure}
            \centering
                \includegraphics[width=\textwidth]{si-stre2}
                \caption{ Elastic energy (rescaled by respective initial value $E_0$) over time (rescaled by respective characteristic time $\tau_\text{diss}$) of pre-stretched spheres.
                 Simulation parameters are documented in Table~\ref{table:str_params}.
                 In (b), normalized elastic energies are further reduced by a factor of $100$, $10^4$, $10^6$ for simulation $B_2$, $A_1$, $B_3$
                 so that the curves are clearly separated.
                 In (c), similar reduction factors of $100$ and $10^4$ are applied to curves from simulation $C_1$ and $C_2$ for clarity.
                }
                \label{fig:str_scaling}
            \end{figure}

            %----------------------------------------------------
            % FIGURE LID DRIVEN CAVITY 1
            %----------------------------------------------------
            \begin{figure}
            \centering
                \includegraphics[width=\textwidth]{si-lid1}
                \caption{
                Comparing normal velocities and pressure profiles against benchmark results by Albensoeder and Kuhlmann \cite{albensoeder05} for cubic lid-driven cavity with Reynolds number $\Rey=1000$.
                Spatial coordinates are shifted by $0.5$ in both plots such that the domain is $[-0.5,0.5]\times[-0.5,0.5]$, and rotated to be aligned with figures in the work by Albensoeder and Kuhlmann \cite{albensoeder05}.
                (a) Normal velocities along the center lines in the central $xz$-plane are rescaled and plotted for five simulation
                resolutions, $N=32, 48, 64, 96, 128$. (b) Pressure profiles along the center lines in the central $xz$-plane are plotted. Pressure
                values are shifted by a constant such that it is kept at zero at position $(0.0, 0.0)$.}
                \label{fig:aspect1_profiles_re1000}
            \end{figure}


            %--------------------------
            % FIGURE LID DRIVEN CAVITY 2
            %--------------------------
            \begin{figure}
            \centering
                \includegraphics[width=\textwidth]{si-lid2}
                \caption{Error convergence in a cubic lid-driven cavity flow with $\Rey=1000$, using $N=128$ solution of as the reference solution. (a) Error in the extrema of normal velocities, $\min(u)$, $\min(w)$, and $\max(w)$, along center lines in the central $xz$-plane. (b) Error in the total kinetic energy after steady state has been reached.}\label{fig:aspect1_conv}
            \end{figure}

            %--------------------------
            % FIGURE LID DRIVEN CAVITY 3
            %--------------------------
            \begin{figure}
            \centering
                \includegraphics[width=0.75\textwidth]{si-lid3}
                \caption{
                Comparing normal velocities against benchmark results by Albensoeder and Kuhlmann \cite{albensoeder05} for cubic lid-driven cavity with Reynolds number $\Rey=100, 400$.
                Spatial coordinates are shifted by $0.5$ in both plots such that the domain is $[-0.5,0.5]\times[-0.5,0.5]$, and rotated to be aligned with figures in the work by Albensoeder and Kuhlmann \cite{albensoeder05}.
                Normal velocities along the center lines in the central $xz$-plane are rescaled and plotted for resolution $N=96$.}
                \label{fig:aspect1_profiles_re400}
            \end{figure}



            %--------------------------
            % FIGURE LID DRIVEN CAVITY 4
            %--------------------------
            \begin{figure}
            \centering
                \includegraphics[width=\textwidth]{si-lid4}
                \caption{Comparing normal velocities and pressure profiles against benchmark results by Albensoeder and Kuhlmann \cite{albensoeder05} for lid-driven cavities with aspect ratios $1:2:1$ and $1:3:1$, with Reynolds number $\Rey=1000$.
                Spatial coordinates are shifted by $0.5$ in both plots such that the domain is $[-0.5,0.5]\times[-0.5,0.5]$, and rotated to be aligned with figures in the work by Albensoeder and Kuhlmann \cite{albensoeder05}.
                (a) Normal velocities along the center lines in the central $xz$-plane are rescaled and plotted for five simulation
                resolutions $N=96$ per unit simulation length. (b) Pressure profiles along the center lines in the central $xz$-plane are plotted. Pressure
                values are shifted by a constant such that it is kept at zero at position $(0.0, 0.0)$. }\label{fig:aspect23_profiles}
            \end{figure}

            %--------------------------
            % FIGURE LID DRIVEN CAVITY 5
            %--------------------------
            \begin{figure}
            \centering
                \includegraphics[width=\textwidth]{si-lid5}
                \caption{Comparing normal velocities and pressure profiles against benchmark results by Albensoeder and Kuhlmann \cite{albensoeder05} for lid-driven cavities with aspect ratio $1:1:2$, with Reynolds number $\Rey=1000$.
                Spatial coordinates are shifted in both plots such that the domain is $[-0.5,0.5]\times[-1.0,1.0]$, and rotated to be aligned with figures in the work by Albensoeder and Kuhlmann \cite{albensoeder05}.
                (a) Normal velocities along the center lines in the central $xz$-plane are rescaled and plotted for five simulation
                resolutions $N=96$ per unit simulation length. (b) Pressure profiles along the center lines in the central $xz$-plane are plotted. Pressure
                values are shifted by a constant such that it is kept at zero at position $(0.0, 0.0)$. }\label{fig:aspectz2_profiles}
            \end{figure}



            %--------------------------
            % FIGURE
            %--------------------------
            \begin{figure}
            \centering
                \includegraphics[width=\textwidth]{si-lid6}
                \caption{ (a) Sphere elastic energy and volume over time for a sphere with radius $0.2$ and shear modulus $G=0.1$ in a cubic lid-driven cavity.
                Reynolds number $\Rey=100$.
                 An incompressible neo-Hookean solid is modeled.
                 However,  $J \ne 1$ in our numerical scheme, especially near the fluid--solid interface.
                 Therefore, a correction factor is used in the strain energy density (Eq.~\eqref{eq:elas}).
                 The correction approaches zero as the grid is refined, since $J$ converges to 1.
                (b) Convergence of errors in surface area, volume, elastic energy, and $\|J-1\|_2$, where $J$ is the determinant of the deformation gradient.
                Errors are measured at $T=0.5$.
                The solid lines are fitted curves using Eq.~\eqref{eq:3params}.
                Convergence rates (with Richardson correction factor $\alpha$ in brackets) are $2.00 (0.98), 2.08 (1.0), 1.51 (0.92), \text{~and~} 1.16 (1.0)$ for surface area, volume, elastic energy, and $\|J-1\|_2$, respectively.
                }
                \label{fig:object_conv}
            \end{figure}


	    	%----------------------------------------------------
            	% TABLE FOR PARAMS IN STRETCHED SPH
            	%----------------------------------------------------
   	        \begin{table}\centering
                \caption{Additional parameters simulations presented in Fig.~4 in the main text. Number of grid cells $N$ and blur zone viscosity multiplier $\gamma_t$ are reported for each case in subfigure Fig.~4(a), (b), (d), (f), (h), and (j). Isotropic grid spacing $h=1/N$.}
                \label{table:lid_params}
                \begin{tabular} { c || c | c }
                    \hline
                    Case       	& $N$ & $\gamma_t$ \\ \hline
                    %
                    $(a)$	& 128 &	0	\\ \hline
                    \hline
                    $(b)$	& 48        &	2	\\ \hline
                    $(b)$	& 64        &	4	\\ \hline
                    $(b)$	& 96        &	2	 \\ \hline
                    $(b)$	& 128       &	0	 \\ \hline
                    $(b)$	& 160       &	0	 \\ \hline
                    \hline
                    $(d)$	& 48        &	0	\\ \hline
                    $(d)$	& 64        &	1	\\ \hline
                    $(d)$	& 96        &	0	 \\ \hline
                    $(d)$	& 128       &	0	 \\ \hline
                    $(d)$	& 160       &	0	 \\ \hline
                    \hline
                    $(f)$	& 48        &	0	\\ \hline
                    $(f)$	& 64        &	0	\\ \hline
                    $(f)$	& 96        &	0	 \\ \hline
                    $(f)$	& 128       &	0	 \\ \hline
                    $(f)$	& 160       &	0	 \\ \hline
                    \hline
                    $(h)$	& 128       &	0	 \\ \hline
                    \hline
                    $(j)$	& 64        &	0	 \\ \hline
                    \hline
                 \end{tabular}
            \end{table}


            %--------------------------
            % FIGURE
            %--------------------------
            \begin{figure}
                \centering
                \includegraphics[width=0.7\textwidth]{si-renums.pdf}
                \caption{Reynolds numbers describing the oscillatory ($\Rey_o$, left column) and steady ($\Rey_s$, right column) flows about the swimmer.
                Top row: Swimmer body length is kept at $L=1.5$ while its body radius $R$ varies.
                Bottom row: Swimmer body radius is kept at $R=0.15$ while its body length $L$ varies.
		%
                Other simulation parameters are reported in Fig.~5 caption in the main text.
                }
                \label{fig:renums}
            \end{figure}

%%% Add this line AFTER all your figures and tables
\FloatBarrier
\renewcommand\refname{Supplmentary Information References}
\bibliographystyle{unsrt}
\bibliography{si-references}
\end{document}
