\documentclass{article}
\usepackage[utf8]{inputenc} % allow utf-8 input
\usepackage[T1]{fontenc}    % use 8-bit T1 fonts
\usepackage{fullpage}
\usepackage[numbers,compress]{natbib}


% NeurIPS packages
% \usepackage{hyperref}       % hyperlinks
\usepackage{url}            % simple URL typesetting
\usepackage{booktabs}       % professional-quality tables
\usepackage{amsfonts}       % blackboard math symbols
\usepackage{microtype}      % microtypography
\usepackage{wrapfig}
\usepackage{caption}
\usepackage{nicefrac}       % compact symbols for 1/2, etc.

%%% Custom packages

\usepackage{times}
\usepackage{algorithm}
\usepackage{adjustbox}
\usepackage[noend]{algpseudocode}
\usepackage{amsmath}
\usepackage{amssymb}
\usepackage{amsthm}
\usepackage{bm}
% \usepackage{bbm}
% \usepackage{booktabs}
\usepackage{graphicx}
\usepackage{latexsym}
\usepackage{mathtools}
\usepackage{multirow}
\usepackage{paralist}
% \usepackage{titletoc} % for appendix toc
\usepackage{minitoc} % for appendix toc
\renewcommand \thepart{} % 1/2: minitoc Make the "Part I" text invisible
\renewcommand \partname{} % 1/2: minitoc Make the "Part I" text invisible
% \usepackage{xcolor}
\usepackage{xspace}
\usepackage{enumitem}
\usepackage{dsfont}
% \usepackage{tcolorbox} % Load after xcolor 

\usepackage[colorlinks,citecolor=bluegray,linkcolor=darkbrown,urlcolor=blue,breaklinks]{hyperref}
\usepackage{cleveref}


% Theorems
\newtheorem{theorem}{Theorem}
\newtheorem{lemma}[theorem]{Lemma}
\newtheorem{proposition}[theorem]{Proposition}
\newtheorem{corollary}[theorem]{Corollary}
\newtheorem{remark}{Remark}
\theoremstyle{definition}
\newtheorem{definition}{Definition}
\newtheorem{assumption}{Assumption}
\newtheorem{example}{Example}
\newtheorem{note}{Note}

\newtheorem{innercustomasmp}{Assumption}
\newenvironment{customasmp}[1]
  {\renewcommand\theinnercustomasmp{#1}\innercustomasmp}
  {\endinnercustomasmp}
  
\newtheorem{innercustomthm}{Theorem}
\newenvironment{customthm}[1]
  {\renewcommand\theinnercustomthm{#1}\innercustomthm}
  {\endinnercustomthm}
  
\newtheorem{innercustomprop}{Proposition}
\newenvironment{customprop}[1]
  {\renewcommand\theinnercustomprop{#1}\innercustomprop}
  {\endinnercustomprop}


% Paragraphs
% \newcommand{\myparagraph}[1]{\paragraph{#1.}\hspace{-0.8em}}  % For Arxiv
\newcommand{\myparagraph}[1]{\par\noindent\textbf{{#1}.}} % For conference formats
\newcommand{\myparagraphsmall}[1]{\par\noindent\textit{{#1}.}}


% Tikz related
\usepackage[dvipsnames]{xcolor} % color names 
\usepackage{tikz}
\usepackage{pgfplots}
\usetikzlibrary{shapes}
\usetikzlibrary{positioning}
\usetikzlibrary{plotmarks}
\usetikzlibrary{patterns}
\usetikzlibrary{intersections,shapes.arrows}
\usetikzlibrary{pgfplots.fillbetween}
\definecolor{darkpink}{rgb}{0.91, 0.33, 0.5}

% Nice colors
\definecolor{puorange}{rgb}{0.80,0.20,0}
\definecolor{bluegray}{rgb}{0.04,0,0.7}
\definecolor{greengray}{rgb}{0.05,0.50,0.15}
\definecolor{darkbrown}{rgb}{0.40,0.2,0.05}
\definecolor{darkcyan}{rgb}{0,0.4,1}
\definecolor{black}{rgb}{0,0,0}
\definecolor{grey}{rgb}{0.93,0.93,0.93}
\definecolor{royalazure}{rgb}{0.0, 0.22, 0.66}

\newcommand{\tabemph}[1]{\cellcolor{grey!30}\textcolor{black!50!royalazure}{#1}}%

\crefname{section}{Sec.}{Sections}
% \Crefname{section}{Section}{Sections}
\crefname{appendix}{Appx.}{Appxs}

\crefname{theorem}{Thm.}{Thms.}
\crefname{innercustomthm}{Thm.}{Thms.}
\crefname{lemma}{Lem.}{Lems.}
\crefname{corollary}{Cor.}{Cors.}
\crefname{proposition}{Prop.}{Props.}
\crefname{innercustomprop}{Prop.}{Props.}
\crefname{assumption}{Asm.}{Asms.}
\crefname{innercustomasmp}{Asmp.}{Asmps.}
\Crefname{example}{Ex.}{Exs.}

\crefname{algorithm}{Alg.}{Algs.}
\Crefname{algorithm}{Algorithm}{Algorithms}
\crefname{figure}{Fig.}{Figs.}
\crefname{table}{Tab.}{Tabs.}
 %% all packages
\newcommand{\score}{\ell}
\newcommand{\grad}{S}
\newcommand{\risk}{L}
\newcommand{\rao}{T_{\text{Rao}}}
\newcommand{\lr}{T_{\text{LR}}}
\newcommand{\wald}{T_{\text{Wald}}}
\newcommand{\bbar}[1]{\bar{\bar{#1}}}  %% useful macros
% Sets
\newcommand{\reals}{{\mathbb R}}
\newcommand{\spacex}{(\mathsf{X}, \mathcal{X})}
\newcommand{\spacey}{(\mathsf{Y}, \mathcal{Y})}
\newcommand{\spacez}{(\mathsf{Z}, \mathcal{Z})}
\newcommand{\dom}{\operatorname{dom}}

% lin alg stuff
\newcommand{\abs}[1]{\left| #1 \right|}
\newcommand{\norm}[1]{\left\lVert #1 \right\rVert}
\newcommand{\anorm}[1]{\left| #1 \right|}
\newcommand{\ones}{\operatorname{\mathbf 1}}
\newcommand{\id}{\operatorname{I}}
\newcommand{\Null}{\operatorname{Null}}
\newcommand{\Ker}{\operatorname{Ker}}
\newcommand{\Rank}{\operatorname{\bf Rank}}
\newcommand{\Tr}{\operatorname{\bf Tr}}
\newcommand{\diag}{\operatorname{\bf diag}}
\newcommand{\Vect}{\operatorname{Vec}}
\newcommand{\Span}{\operatorname{Span}}
\newcommand{\Proj}{\operatorname{Proj}}
\newcommand{\ip}[1]{{\langle #1 \rangle}}
\newcommand{\lone}{\mathbf{L}^1}
\newcommand{\ltwo}{\mathbf{L}^2}
\newcommand{\linf}{\mathbf{L}^\infty}

% probability stuff
\newcommand{\Expect}{\operatorname{\mathbb E}}
\newcommand{\Var}{\operatorname{\mathbb{V}ar}}
\newcommand{\Std}{\operatorname{\mathbb{S}td}}
\newcommand{\Cov}{\operatorname{\mathbb{C}ov}}
\newcommand{\Prob}{\operatorname{\mathbb P}}
\newcommand{\Ent}{\operatorname{Ent}}
\newcommand{\Supp}[1]{\text{Supp}(#1)}
\newcommand{\wass}{\mathsf{W}}
\newcommand{\pspace}{\mathcal{M}_1}
\newcommand \D {\mathrm{d}}
\newcommand{\kl}{\operatorname{KL}}

% hypothesis testing
\newcommand{\hnull}{\mathbf{H}_0}
\newcommand{\halt}{\mathbf{H}_1}

% convexity & optimization stuff
\DeclareMathOperator*{\argmax}{arg\,max}
\DeclareMathOperator*{\argmin}{arg\,min}
\newcommand{\prox}{\operatorname{prox}}
\newcommand{\bigO}{\mathcal{O}}

% kernel and Hilbert space
\newcommand{\rkhs}{{\mathcal{H}}}
\newcommand{\fmap}{\phi}
\newcommand{\hnorm}[1]{\left\lVert #1 \right\rVert_{\mathcal{H}}}
\newcommand{\hip}[1]{{\langle #1 \rangle_{\mathcal{H}}}}
\newcommand{\kmap}[1]{{k(#1, \cdot)}}

% special font letters
\newcommand{\calA}{\mathcal{A}}
\newcommand{\calB}{\mathcal{B}}
\newcommand{\calC}{\mathcal{C}}
\newcommand{\calD}{\mathcal{D}}
\newcommand{\calE}{\mathcal{E}}
\newcommand{\calF}{\mathcal{F}}
\newcommand{\calG}{\mathcal{G}}
\newcommand{\calH}{\mathcal{H}}
\newcommand{\calI}{\mathcal{I}}
\newcommand{\calL}{\mathcal{L}}
\newcommand{\calM}{\mathcal{M}}
\newcommand{\calN}{\mathcal{N}}
\newcommand{\calP}{\mathcal{P}}
\newcommand{\calS}{\mathcal{S}}
\newcommand{\calV}{\mathcal{V}}
\newcommand{\calW}{\mathcal{W}}
\newcommand{\calX}{\mathcal{X}}
\newcommand{\calY}{\mathcal{Y}}
\newcommand{\calZ}{\mathcal{Z}}

\newcommand{\sfX}{\mathsf{X}}
\newcommand{\sfY}{\mathsf{Y}}
\newcommand{\sfZ}{\mathsf{Z}}

\newcommand{\bbC}{\mathbb{C}}
\newcommand{\bbF}{\mathbb{F}}
\newcommand{\bbH}{\mathbb{H}}
\newcommand{\bbN}{{\mathbb N}}
\newcommand{\bbP}{\mathbb{P}}
\newcommand{\bbW}{\mathbb{W}}
\newcommand{\bbX}{\mathbb{X}}
\newcommand{\bbY}{\mathbb{Y}}
\newcommand{\bbZ}{\mathbb{Z}}

\newcommand{\ind}{\mathds{1}}

% Latin abbreviations etc.
\newcommand{\ie}{{\em i.e.,~}}
\newcommand{\wlg}{{\em w.l.o.g.,~}}
\newcommand{\eg}{{\em e.g.,~}}
\newcommand{\etal}{{\em et al.~}}
\newcommand{\wrt}{{\em w.r.t.~}}
\newcommand{\st}{{s.t.~}}
\newcommand{\iid}{{\em i.i.d.~}}


% highlight colors
\newcommand{\blue}[1]{\textcolor{blue}{#1}}
\newcommand{\bluegray}[1]{\textcolor{bluegray}{#1}}
\newcommand{\red}[1]{\textcolor{red}{#1}}

% equations
\newcommand{\txtover}[2]{\overset{\mbox{\scriptsize #1}}{#2}}
\newcommand\numberthis{\addtocounter{equation}{1}\tag{\theequation}}

\title{Confidence Sets under Generalized Self-Concordance}

\author{Lang Liu \qquad Zaid Harchaoui \\
Department of Statistics, University of Washington}
\date{}

\begin{document}

\maketitle
\doparttoc % Tell to minitoc to generate a toc for the parts
\faketableofcontents % Run a fake tableofcontents command for the partocs

\begin{abstract}
    This paper revisits a fundamental problem in statistical inference from a non-asymptotic theoretical viewpoint---the construction of confidence sets. We establish a finite-sample bound for the estimator, characterizing its asymptotic behavior in a non-asymptotic fashion. An important feature of our bound is that its dimension dependency is captured by the effective dimension---the trace of the limiting sandwich covariance---which can be much smaller than the parameter dimension in some regimes. We then illustrate how the bound can be used to obtain a confidence set whose shape is adapted to the optimization landscape induced by the loss function. Unlike previous works that rely heavily on the strong convexity of the loss function, we only assume the Hessian is lower bounded at optimum and allow it to gradually becomes degenerate. This property is formalized by the notion of generalized self-concordance which originated from convex optimization. Moreover, we demonstrate how the effective dimension can be estimated from data and characterize its estimation accuracy. We apply our results to maximum likelihood estimation with generalized linear models, score matching with exponential families, and hypothesis testing with Rao's score test.
\end{abstract}

\section{INTRODUCTION}
\label{sec:intro}
% !TEX root = ../arxiv.tex

Unsupervised domain adaptation (UDA) is a variant of semi-supervised learning \cite{blum1998combining}, where the available unlabelled data comes from a different distribution than the annotated dataset \cite{Ben-DavidBCP06}.
A case in point is to exploit synthetic data, where annotation is more accessible compared to the costly labelling of real-world images \cite{RichterVRK16,RosSMVL16}.
Along with some success in addressing UDA for semantic segmentation \cite{TsaiHSS0C18,VuJBCP19,0001S20,ZouYKW18}, the developed methods are growing increasingly sophisticated and often combine style transfer networks, adversarial training or network ensembles \cite{KimB20a,LiYV19,TsaiSSC19,Yang_2020_ECCV}.
This increase in model complexity impedes reproducibility, potentially slowing further progress.

In this work, we propose a UDA framework reaching state-of-the-art segmentation accuracy (measured by the Intersection-over-Union, IoU) without incurring substantial training efforts.
Toward this goal, we adopt a simple semi-supervised approach, \emph{self-training} \cite{ChenWB11,lee2013pseudo,ZouYKW18}, used in recent works only in conjunction with adversarial training or network ensembles \cite{ChoiKK19,KimB20a,Mei_2020_ECCV,Wang_2020_ECCV,0001S20,Zheng_2020_IJCV,ZhengY20}.
By contrast, we use self-training \emph{standalone}.
Compared to previous self-training methods \cite{ChenLCCCZAS20,Li_2020_ECCV,subhani2020learning,ZouYKW18,ZouYLKW19}, our approach also sidesteps the inconvenience of multiple training rounds, as they often require expert intervention between consecutive rounds.
We train our model using co-evolving pseudo labels end-to-end without such need.

\begin{figure}[t]%
    \centering
    \def\svgwidth{\linewidth}
    \input{figures/preview/bars.pdf_tex}
    \caption{\textbf{Results preview.} Unlike much recent work that combines multiple training paradigms, such as adversarial training and style transfer, our approach retains the modest single-round training complexity of self-training, yet improves the state of the art for adapting semantic segmentation by a significant margin.}
    \label{fig:preview}
\end{figure}

Our method leverages the ubiquitous \emph{data augmentation} techniques from fully supervised learning \cite{deeplabv3plus2018,ZhaoSQWJ17}: photometric jitter, flipping and multi-scale cropping.
We enforce \emph{consistency} of the semantic maps produced by the model across these image perturbations.
The following assumption formalises the key premise:

\myparagraph{Assumption 1.}
Let $f: \mathcal{I} \rightarrow \mathcal{M}$ represent a pixelwise mapping from images $\mathcal{I}$ to semantic output $\mathcal{M}$.
Denote $\rho_{\bm{\epsilon}}: \mathcal{I} \rightarrow \mathcal{I}$ a photometric image transform and, similarly, $\tau_{\bm{\epsilon}'}: \mathcal{I} \rightarrow \mathcal{I}$ a spatial similarity transformation, where $\bm{\epsilon},\bm{\epsilon}'\sim p(\cdot)$ are control variables following some pre-defined density (\eg, $p \equiv \mathcal{N}(0, 1)$).
Then, for any image $I \in \mathcal{I}$, $f$ is \emph{invariant} under $\rho_{\bm{\epsilon}}$ and \emph{equivariant} under $\tau_{\bm{\epsilon}'}$, \ie~$f(\rho_{\bm{\epsilon}}(I)) = f(I)$ and $f(\tau_{\bm{\epsilon}'}(I)) = \tau_{\bm{\epsilon}'}(f(I))$.

\smallskip
\noindent Next, we introduce a training framework using a \emph{momentum network} -- a slowly advancing copy of the original model.
The momentum network provides stable, yet recent targets for model updates, as opposed to the fixed supervision in model distillation \cite{Chen0G18,Zheng_2020_IJCV,ZhengY20}.
We also re-visit the problem of long-tail recognition in the context of generating pseudo labels for self-supervision.
In particular, we maintain an \emph{exponentially moving class prior} used to discount the confidence thresholds for those classes with few samples and increase their relative contribution to the training loss.
Our framework is simple to train, adds moderate computational overhead compared to a fully supervised setup, yet sets a new state of the art on established benchmarks (\cf \cref{fig:preview}).



\section{PROBLEM FORMULATION}
\label{sec:problem}
We briefly recall the framework of statistical inference via empirical risk minimization.
Let $(\bbZ, \calZ)$ be a measurable space.
Let $Z \in \bbZ$ be a random element following some unknown distribution $\Prob$.
Consider a parametric family of distributions $\calP_\Theta := \{P_\theta: \theta \in \Theta \subset \reals^d\}$ which may or may not contain $\Prob$.
We are interested in finding the parameter $\theta_\star$ so that the model $P_{\theta_\star}$ best approximates the underlying distribution $\Prob$.
For this purpose, we choose a \emph{loss function} $\score$ and minimize the \emph{population risk} $\risk(\theta) := \Expect_{Z \sim \Prob}[\score(\theta; Z)]$.
Throughout this paper, we assume that
\begin{align*}
     \theta_\star = \argmin_{\theta \in \Theta} L(\theta)
\end{align*}
uniquely exists and satisfies $\theta_\star \in \text{int}(\Theta)$, $\nabla_\theta L(\theta_\star) = 0$, and $\nabla_\theta^2 L(\theta_\star) \succ 0$.

\myparagraph{Consistent loss function}
We focus on loss functions that are consistent in the following sense.

\begin{customasmp}{0}\label{asmp:proper_loss}
    When the model is \emph{well-specified}, i.e., there exists $\theta_0 \in \Theta$ such that $\Prob = P_{\theta_0}$, it holds that $\theta_0 = \theta_\star$.
    We say such a loss function is \emph{consistent}.
\end{customasmp}

In the statistics literature, such loss functions are known as proper scoring rules \citep{dawid2016scoring}.
We give below two popular choices of consistent loss functions.

\begin{example}[Maximum likelihood estimation]
    A widely used loss function in statistical machine learning is the negative log-likelihood $\score(\theta; z) := -\log{p_\theta(z)}$ where $p_\theta$ is the probability mass/density function for the discrete/continuous case.
    When $\Prob = P_{\theta_0}$ for some $\theta_0 \in \Theta$,
    we have $L(\theta) = \Expect[-\log{p_\theta(Z)}] = \kl(p_{\theta_0} \Vert p_\theta) - \Expect[\log{p_{\theta_0}(Z)}]$ where $\kl$ is the Kullback-Leibler divergence.
    As a result, $\theta_0 \in \argmin_{\theta \in \Theta} \kl(p_{\theta_0} \Vert p_\theta) = \argmin_{\theta \in \Theta} L(\theta)$.
    Moreover, if there is no $\theta$ such that $p_\theta \txtover{a.s.}{=} p_{\theta_0}$, then $\theta_0$ is the unique minimizer of $L$.
    We give in \Cref{tab:glms} a few examples from the class of generalized linear models (GLMs) proposed by \citet{nelder1972generalized}.
\end{example}

\begin{example}[Score matching estimation]
    Another important example appears in \emph{score matching} \citep{hyvarinen2005estimation}.
    Let $\bbZ = \reals^\tau$.
    Assume that $\Prob$ and $P_\theta$ have densities $p$ and $p_\theta$ w.r.t the Lebesgue measure, respectively.
    Let $p_\theta(z) = q_\theta(z) / \Lambda(\theta)$ where $\Lambda(\theta)$ is an unknown normalizing constant. We can choose the loss
    \begin{align*}
        \score(\theta; z) := \Delta_z \log{q_\theta(z)} + \frac12 \norm{\nabla_z \log{q_\theta(z)}}^2 + \text{const}.
    \end{align*}
    Here $\Delta_z := \sum_{k=1}^p \partial^2/\partial z_k^2$ is the Laplace operator.
    Since \cite[Thm.~1]{hyvarinen2005estimation}
    \begin{align*}
        L(\theta) = \frac12 \Expect\left[ \norm{\nabla_z q_\theta(z) - \nabla_z p(z)}^2 \right],
    \end{align*}
    we have, when $p = p_{\theta_0}$, that $\theta_0 \in \argmin_{\theta \in \Theta} L(\theta)$.
    In fact, when $q_\theta > 0$ and there is no $\theta$ such that $p_\theta \txtover{a.s.}{=} p_{\theta_0}$, the true parameter $\theta_0$ is the unique minimizer of $L$ \cite[Thm.~2]{hyvarinen2005estimation}.
\end{example}

\myparagraph{Empirical risk minimization}
Assume now that we have an i.i.d.~sample $\{Z_i\}_{i=1}^n$ from $\Prob$.
To learn the parameter $\theta_\star$ from the data, we minimize the empirical risk to obtain the \emph{empirical risk minimizer}
\begin{align*}
    \theta_n \in \argmin_{\theta \in \Theta} \left[ L_n(\theta) := \frac1n \sum_{i=1}^n \score(\theta; Z_i) \right].
\end{align*}
This applies to both maximum likelihood estimation and score matching estimation. 
In \Cref{sec:main_results}, we will prove that, with high probability, the estimator $\theta_n$ exists and is unique under a generalized self-concordance assumption.

\begin{figure}
    \centering
    \includegraphics[width=0.45\textwidth]{graphs/logistic-dikin} %0.4
    \caption{Dikin ellipsoid and Euclidean ball.}
    \label{fig:logistic_dikin}
\end{figure}

\myparagraph{Confidence set}
In statistical inference, it is of great interest to quantify the uncertainty in the estimator $\theta_n$.
In classical asymptotic theory, this is achieved by constructing an asymptotic confidence set.
We review here two commonly used ones, assuming the model is well-specified.
We start with the \emph{Wald confidence set}.
It holds that $n(\theta_n - \theta_\star)^\top H_n(\theta_n) (\theta_n - \theta_\star) \rightarrow_d \chi_d^2$, where $H_n(\theta) := \nabla^2 L_n(\theta)$.
Hence, one may consider a confidence set $\{\theta: n(\theta_n - \theta)^\top H_n(\theta_n) (\theta_n - \theta) \le q_{\chi_d^2}(\delta) \}$ where $q_{\chi_d^2}(\delta)$ is the upper $\delta$-quantile of $\chi_d^2$.
The other is the \emph{likelihood-ratio (LR) confidence set} constructed from the limit $2n [L_n(\theta_\star) - L_n(\theta_n)] \rightarrow_d \chi_d^2$, which is known as the Wilks' theorem \citep{wilks1938large}.
These confidence sets enjoy two merits: 1) their shapes are an ellipsoid (known as the \emph{Dikin ellipsoid}) which is adapted to the optimization landscape induced by the population risk; 2) they are asymptotically valid, i.e., their coverages are exactly $1 - \delta$ as $n \rightarrow \infty$.
However, due to their asymptotic nature, it is unclear how large $n$ should be in order for it to be valid.

Non-asymptotic theory usually focuses on developing finite-sample bounds for the \emph{excess risk}, i.e., $\Prob(L(\theta_n) - L(\theta_\star) \le C_n(\delta)) \ge 1 - \delta$.
To obtain a confidence set, one may assume that the population risk is twice continuously differentiable and $\lambda$-strongly convex.
Consequently, we have $\lambda \norm{\theta_n - \theta_\star}_2^2 / 2 \le L(\theta_n) - L(\theta_\star)$ and thus we can consider the confidence set $\calC_{\text{finite}, n}(\delta) := \{\theta: \norm{\theta_n - \theta}_2^2 \le 2C_n(\delta)/\lambda\}$.
Since it originates from a finite-sample bound, it is valid for fixed $n$, i.e., $\Prob(\theta_\star \in \calC_{\text{finite}, n}(\delta)) \ge 1 - \delta$ for all $n$; however, it is usually conservative, meaning that the coverage is strictly larger than $1 - \delta$.
Another drawback is that its shape is a Euclidean ball which remains the same no matter which loss function is chosen.
We illustrate this phenomenon in \Cref{fig:logistic_dikin}.
Note that a similar observation has also been made in the bandit literature \citep{faury2020improved}.

We are interested in developing finite-sample confidence sets.
However, instead of using excess risk bounds and strong convexity, we construct in \Cref{sec:main_results} the Wald and LR confidence sets in a non-asymptotic fashion, under a generalized self-concordance condition.
These confidence sets have the same shape as their asymptotic counterparts while maintaining validity for fixed $n$.
These new results are achieved by characterizing the critical sample size enough to enter the asymptotic regime.



\section{MAIN RESULTS}
\label{sec:main_results}
%\paragraph{Theoretical Analysis}
%
A significant benefit of using \trnn{} is that we can theoretically characterize its expressiveness  for approximating the underlying dynamics.   The main idea is to analyze a class of functions that satisfies certain regularity conditions. For such functions, tensor-train representations preserve the weak differentiability and yield a compact representation.


The following theorem characterizes the representation power of \trnn{}, viewed as a one-layer hidden neural network, in terms of 1) the regularity of the target function $f$, 2) the dimension of the input space, 3) the tensor train rank and 4) the order of the tensor:
%
\begin{theorem}
Let the target function $f\in \mathcal{H}^k_\mu$ be a H\"older continuous function defined on a input  domain $\mathcal{I} =I_1\times \cdots \times I_d$, with  bounded derivatives up to order $k$ and finite Fourier magnitude distribution $C_f$. A single layer \trnn{} with $h$ hidden units, $\hat{f}$ can approximate $f$ with approximation error $\epsilon$ at most:
%
\eq{
%
\epsilon \leq \fr{1}{h} \brck{C_f^2 \frac{d-1}{(k-1)(r+1)^{k-1}} + C(k)p^{-k} }
%
}
%
where $C_f = \int |\omega|_1 |\hat{f}(\omega) d \omega|$, $d$ is the dimension of the function, i.e., the size of the state space, $r$ is the tensor-train rank, $p$ is the degree of the higher-order polynomials i.e., the order of the tensor, and $C(k)$ is the coefficient of the spectral expansion of $f$.
\label{eqn:thm}
\end{theorem}

\textbf{Remarks}: The result above shows that the number of weights required to approximate the target function $f$ is dictated by its regularity (i.e., its H\"older-continuity order $k$). The expressiveness of \trnn{} is driven by the selection of the rank $r$ and the polynomial degree $p$; moreover, it improves for functions with increasing regularity.  Compared with ``first-order'' regular RNNs, \trnn{}s are  exponentially more powerful for large rank: if the order  $p$ increases, we require fewer hidden units $h$. 
% 
% The results also provide intuitions for choosing the hidden size and the rank for optimal storage. \ryedit{don't know what storage means}
%
% as long as we are given a state transitions $(\V{x}_t, \V{s}_t) \mapsto \V{s}_{t+1}$ (e.g. the state transition function learned by the  encoder).

\ti{Proof sketch}: 
For the full proof, see the Appendix. 
% 
We design \trnn{} to approximate the underlying system dynamics. The target function $f(\V{x})$ represents the state transition function, as in \refn{eqn:tensor_rnn}.
%
We first show that if $f$ preserves weak derivatives, then it has a compact tensor-train representation. Formally, let us assume that $f$ is a Sobolev function: $f\in\mathcal{H}^k_\mu$, defined on the input space $\T{I}= I_1 \times I_2\times \cdots I_d $, where each $I_i$ is a set of vectors. The space $\mathcal{H}^k_\mu$ is defined as the functions that have bounded derivatives up to some order $k$ and are $L_\mu$-integrable.
%
\begin{eqnarray}
\mathcal{H}^k_\mu =  \left\{  f  \in L_\mu(\T{I}):\sum_{i\leq k}\|D^{(i)}f\|^2   < +\infty \right\},
\end{eqnarray}
%
where $D^{(i)}f$ is the $i$-th weak derivative of $f$ and $\mu \geq 0$.\footnote{A weak derivative generalizes the derivative concept for (non)-differentiable functions and is implicitly defined as: e.g. $v\in L^1([a,b])$ is a weak derivative of $u\in L^1([a,b])$ if for all smooth $\varphi$ with $\varphi(a) = \varphi(b) = 0$: $\int_a^bu(t)\varphi'(t) = -\int_a^bv(t)\varphi(t)$.}
% 
% The space $\mathcal{H}^k_\mu$ is equipped with a norm $\|f\|^2_{k,\mu} =\sum_{|i|\leq k}\|D^{(i)}f\|^2$ and a semi-norm $|f|^2_{k,\mu} =\sum_{|i|=k}\|D^{(i)}f\|^2 $. For notation simplicity, we muted the subscript $\mu$ and used $\|\cdot\| $ for $\|\cdot \|_{L_{\mu}}$.
% f
It is known that any Sobolev function admits a Schmidt decomposition: $f(\cdot) = \sum_{i =0}^\infty \sqrt{\lambda_i } \gamma (\cdot)_i \otimes \phi (\cdot)_i $, where $\{\lambda \}$ are the eigenvalues and $\{\gamma\}, \{ \phi\}$ are the associated eigenfunctions.
%
Hence, for $\V{x}\in\calI$, we can represent the target function $f(\V{x}) $ as an infinite summation of products of a set of basis functions:
%
\begin{align}
&f(\V{x}) = \sum_{\alpha_0,\cdots,\alpha_d=1}^\infty
%
% \T{A}^1_{\alpha_0, x_1, \alpha_1}
\T{A}^1 (x_1)_{\alpha_0 \alpha_1}
%
\cdots
%
\T{A}^d(x_d)_{\alpha_{d-1}	 \alpha_d},
\label{eqn:ftt}
\end{align}
%
where  $ \{ \T{A}^j(x_j)_{\alpha_{j-1} \alpha_j} \}$ are basis functions over each input dimension.
% = \sqrt{\lambda^{d-1}_{\alpha_{d-1}}} \phi (x_d)_{\alpha_{d-1}}
These basis functions satisfy $\langle \T{A}^j(\cdot)_{im}, \T{A}^j (\cdot)_{in} \rangle = \delta_{mn}$ for all $j$.
% 
If we truncate \eqref{eqn:ftt} to a low dimensional subspace ($\V{r}<\infty$), we obtain a functional approximation of the state transition function $f(\V{x})$.  
% 
This approximation is also known as the \ti{functional tensor-train} (FTT):
%
\begin{align}
&f_{FTT}(\V{x}) = \sum_{\alpha_0,\cdots,\alpha_d}^\mathbf{r}
%
\T{A}^1(x_1)_{\alpha_0\alpha_1}
%
\cdots
%
\T{A}^d(x_d)_{\alpha_{d-1}\alpha_d},
\end{align}

In practice, \trnn{} implements a polynomial expansion of the states using $[\V{s}, \V{s}^{\otimes 2}, \cdots, \V{s}^{\otimes P}]$, where $P$ is the degree of the polynomial. The final function represented by \trnn{} is a polynomial approximation of the functional tensor-train function $f_{FTT}$. 

Given a target  function $f(\V{x}) = f(\V{s}\otimes \dots \otimes \V{s})$, we can express it using FTT and the polynomial expansion of the states $\V{s}$. This allows us to characterize \trnn{} using a family of functions that it can represent.  Combined with the classic neural network approximation theory \cite{barron1993universal}, we can bound the approximation error for \trnn{} with one hidden layer. The above results applies to the full family of \trnn{}s, including those using  vanilla RNN or LSTM as the recurrent cell.

One can think of the universal approximation result in Theorem \ref{eqn:thm} bounds the estimation bias of the model: $f-\mathbb{E} [\hat{f}]$, where the expectation is taken over training sets. While a large neural network can approximate any function, training as a large neural network will be hard given a finite data set, demonstrating bias-variance trade-off. In the next section, we provide bounds for the estimation variance. 
 
%
% Given a target function $f$, and a neural network with one hidden layer and sigmoid activation function, the following lemma describes the classic result of  describing the error between  $f$ and the single hidden-layer neural network that approximates it best:
%
% \begin{lemma}[NN Approximation \cite{barron1993universal}]
% 	Given a function $f$ with finite Fourier magnitude distribution $C_f$, there exists a  neural network of $n$ hidden units $f_n$, such that
% 	\begin{eqnarray}
% 		\| f - f_n\| \leq \frac{C_f}{\sqrt{n}}
% 	\end{eqnarray}
% 	where $C_f = \int |\omega|_1  | \hat{f}(\omega) | d \omega$ with Fourier representation $f(x)=\int e^{i\omega x}\hat{f}(\omega) d\omega$.
% 	\label{lemma:nn}
% \end{lemma}
% 
% We can  generalize Barron's approximation result in  lemma \ref{lemma:nn} to \trnn{}.  



% As the rank of the tensor-train and the polynomial order increase, the required size of the hidden units become smaller, up to a constant that depends on the regularity of the underlying dynamics $f$.
%
% This theorem applies to  the entire family of \trnn{}s, including those using vanilla RNN or LSTM as the recurrent cell, as long as we are given a state transition function $(\V{x}_t, \V{s}_t) \mapsto \V{s}_{t+1}$. (e.g. the state transition function learned by the  encoder).
%
% In this case, Theorem \ref{eqn:thm} bounds the estimation error when approximating a state-transition function $f:\V{s}_t \mapsto \V{x}_{t+1}$ using a tensor-train layer.
%
% However, the general case where the state-encoder is not known.
% For instance, when using $\V{x} = [\V{s} \V{s} \V{s}]$, we recover the.tensor transition function in
%
% $N$ refers to number of state transitions (or examples) seen during training. The fewer transitions we observe, the less information we get about the true dynamics, and we need to memorize more information for the same approximation error.
% \stzedit{If we assume that the Does this apply to LSTMs as well?}



\section{EXAMPLES AND APPLICATIONS}
\label{sec:application}
\section{Activation During Perception of Noisy Speech}\label{sec:Apps}
The dataset, provided as  {\tt data6} in the AFNI tutorial~\citet{cox96},
is originally from an fMRI study~\citet{nathandbeauchamp11} where
\begin{figure}[h]
\subfloat[]{\includegraphics[width=0.5\columnwidth]{figs/AR-FAST-001-Visual-crop}}
\subfloat[]{\includegraphics[width=0.5\columnwidth]{figs/AR-FAST-001-Audio-crop}}
%\subfloat[]{\includegraphics[width=0.33\textwidth]{figs/AM-FAST-005-diff}}
\caption{ AR-FAST-identified activation regions on SPMs obtained by
  fitting ~\ref{eq:lm} with   AR($\hat{p}$) to AFNI's {\tt data6} for
  (a) visual-reliable stimulus and (b) audio-reliable
  stimulus.}
\label{fig:AMSmoothingAFNI}
\end{figure}
\begin{comment}
\begin{figure*}[h]
\subfloat[]{\includegraphics[width=0.25\textwidth]{figures/Visual_AM-crop}}
\subfloat[]{\includegraphics[width=0.25\textwidth]{figures/Audio_AM-crop}}
\subfloat[]{\includegraphics[width=0.25\textwidth]{figures/Visual_Audio_AM-crop}}
\caption{Activation areas obtained using AR-FAST with in {\it AFNI data6} on the SPM obtained
  after fitting AR($\hat{p}$) of the (a) Visual-reliable, (b) Audio-reliable and (c) the difference contrast between Visual-reliable and Audio-reliable.}
\label{fig:AMSmoothingAFNI}
\end{figure*}
\end{comment}
a subject heard and saw a female volunteer speak words, separately, in
two different formats. The audio-reliable setting had the subject
clearly hear the spoken word but see a degraded image of the speaker
while the visual-reliable case had the subject clearly see the speaker
vocalize the word but the audio was of reduced quality.  There were
three experimental runs, each 
consisting of a randomized design of 10 blocks, equally divided into blocks of
audio-reliable and visual-reliable stimuli. %An echo-planar imaging
% sequence (TR=2s) was used to obtain
$\mbox{T}_2^*$-weighted images with volumes of $80 \times 80 \times
33$ (with voxels of dimension $2.75 \times  2.75 \times 3.0\  mm^3$)
from  echo-planar sequences (TR=2s) 
were obtained  over $152$ time-points. Our interest was in determining 
activation corresponding to the audio
($H_0:\beta_{a}=0$) and visual
($H_0:\beta_{v}=0$) tasks.
%, as well as their contrast  ($H_0:\beta_{v} - \beta_{a}=0$). The
%first two cases have one-sided alternatives while the contrast in
%activation corresponds to a two-sided alternative.
At each voxel, we fitted AR models for 
$p=0,1,2,3,4,5$ and chose $p$ with the highest BIC. 
Figure \ref{fig:AMSmoothingAFNI} uses AFNI and Surface Mapping (SUMA)
to display activated regions obtained using AR-FAST on the SPM:
see  Figure~\ref{fig:Visual-Audio} for  maps drawn from ALL-FAST, AS,
AWS and CT. We used $\alpha = 0.01$ because of the high (greater than
4) upper percentile of the voxel-wise estimated CNRs. Most of the activation 
occurs in Brodmann areas 18 and 19 (BA18 and BA19)
which comprise the
occipital cortex %in the human brain,  accounting for the bulk of the
                 %volume of the occipital lobe. Both areas form part
                 %of the visual association area while BA 19, also the occipital lobe cortex as well. Along with BA18, it comprises
and the extrastriate (or peristriate) cortex. In humans with normal
sight,  this area is for visual association where 
feature-extraction, shape recognition, attentional and multimodal
integrating functions occur. We also see increased activation in
the STS, which recent
studies~\citep{grossman2001brain} have related to  distinguishing
voices from environmental sounds, 
stories versus nonsensical speech, moving faces versus moving objects,
biological motion and so on. ALL-FAST performs similarly as AR-FAST,
while the other methods also identify the same regions but they identify
a lot more activated 
voxels, some of which appear to be false positives. Although a 
detailed analysis of the results of this study is beyond the purview
of this paper, we note that AR-FAST 
finds interpretable results even when applied to a single
subject high-level cognition experiment. 
\begin{comment}
\begin{table}[h!]
\centering
\caption{Coordinates of the maximum $t$-statistic and its corresponding value}\label{tab:maxt}
\begin{tabular}{c|c}
Task & $(x,y,z) mm$  \\
\hline
Visual-Audio & (-30.162,86.221,6.349) \\
Audio & (-27.412,75.221,-5.651)  \\
 Visual & (-30.162, 80.721, 15.349) \\
\hline
\end{tabular}
\end{table}
\end{comment}



\section{NUMERICAL STUDIES}
\label{sec:experiments}
In this section we conduct comprehensive experiments to emphasise the effectiveness of DIAL, including evaluations under white-box and black-box settings, robustness to unforeseen adversaries, robustness to unforeseen corruptions, transfer learning, and ablation studies. Finally, we present a new measurement to test the balance between robustness and natural accuracy, which we named $F_1$-robust score. 


\subsection{A case study on SVHN and CIFAR-100}
In the first part of our analysis, we conduct a case study experiment on two benchmark datasets: SVHN \citep{netzer2011reading} and CIFAR-100 \cite{krizhevsky2009learning}. We follow common experiment settings as in \cite{rice2020overfitting, wu2020adversarial}. We used the PreAct ResNet-18 \citep{he2016identity} architecture on which we integrate a domain classification layer. The adversarial training is done using 10-step PGD adversary with perturbation size of 0.031 and a step size of 0.003 for SVHN and 0.007 for CIFAR-100. The batch size is 128, weight decay is $7e^{-4}$ and the model is trained for 100 epochs. For SVHN, the initial learinnig rate is set to 0.01 and decays by a factor of 10 after 55, 75 and 90 iteration. For CIFAR-100, the initial learning rate is set to 0.1 and decays by a factor of 10 after 75 and 90 iterations. 
%We compared DIAL to \cite{madry2017towards} and TRADES \citep{zhang2019theoretically}. 
%The evaluation is done using Auto-Attack~\citep{croce2020reliable}, which is an ensemble of three white-box and one black-box parameter-free attacks, and various $\ell_{\infty}$ adversaries: PGD$^{20}$, PGD$^{100}$, PGD$^{1000}$ and CW$_{\infty}$ with step size of 0.003. 
Results are averaged over 3 restarts while omitting one standard deviation (which is smaller than 0.2\% in all experiments). As can be seen by the results in Tables~\ref{black-and_white-svhn} and \ref{black-and_white-cifar100}, DIAL presents consistent improvement in robustness (e.g., 5.75\% improved robustness on SVHN against AA) compared to the standard AT 
%under variety of attacks 
while also improving the natural accuracy. More results are presented in Appendix \ref{cifar100-svhn-appendix}.


\begin{table}[!ht]
  \caption{Robustness against white-box, black-box attacks and Auto-Attack (AA) on SVHN. Black-box attacks are generated using naturally trained surrogate model. Natural represents the naturally trained (non-adversarial) model.
  %and applied to the best performing robust models.
  }
  \vskip 0.1in
  \label{black-and_white-svhn}
  \centering
  \small
  \begin{tabular}{l@{\hspace{1\tabcolsep}}c@{\hspace{1\tabcolsep}}c@{\hspace{1\tabcolsep}}c@{\hspace{1\tabcolsep}}c@{\hspace{1\tabcolsep}}c@{\hspace{1\tabcolsep}}c@{\hspace{1\tabcolsep}}c@{\hspace{1\tabcolsep}}c@{\hspace{1\tabcolsep}}c@{\hspace{1\tabcolsep}}c}
    \toprule
    & & \multicolumn{4}{c}{White-box} & \multicolumn{4}{c}{Black-Box}  \\
    \cmidrule(r){3-6} 
    \cmidrule(r){7-10}
    Defense Model & Natural & PGD$^{20}$ & PGD$^{100}$  & PGD$^{1000}$  & CW$^{\infty}$ & PGD$^{20}$ & PGD$^{100}$ & PGD$^{1000}$  & CW$^{\infty}$ & AA \\
    \midrule
    NATURAL & 96.85 & 0 & 0 & 0 & 0 & 0 & 0 & 0 & 0 & 0 \\
    \midrule
    AT & 89.90 & 53.23 & 49.45 & 49.23 & 48.25 & 86.44 & 86.28 & 86.18 & 86.42 & 45.25 \\
    % TRADES & 90.35 & 57.10 & 54.13 & 54.08 & 52.19 & 86.89 & 86.73 & 86.57 & 86.70 &  49.50 \\
    $\DIAL_{\kl}$ (Ours) & 90.66 & \textbf{58.91} & \textbf{55.30} & \textbf{55.11} & \textbf{53.67} & 87.62 & 87.52 & 87.41 & 87.63 & \textbf{51.00} \\
    $\DIAL_{\ce}$ (Ours) & \textbf{92.88} & 55.26  & 50.82 & 50.54 & 49.66 & \textbf{89.12} & \textbf{89.01} & \textbf{88.74} & \textbf{89.10} &  46.52  \\
    \bottomrule
  \end{tabular}
\end{table}


\begin{table}[!ht]
  \caption{Robustness against white-box, black-box attacks and Auto-Attack (AA) on CIFAR100. Black-box attacks are generated using naturally trained surrogate model. Natural represents the naturally trained (non-adversarial) model.
  %and applied to the best performing robust models.
  }
  \vskip 0.1in
  \label{black-and_white-cifar100}
  \centering
  \small
  \begin{tabular}{l@{\hspace{1\tabcolsep}}c@{\hspace{1\tabcolsep}}c@{\hspace{1\tabcolsep}}c@{\hspace{1\tabcolsep}}c@{\hspace{1\tabcolsep}}c@{\hspace{1\tabcolsep}}c@{\hspace{1\tabcolsep}}c@{\hspace{1\tabcolsep}}c@{\hspace{1\tabcolsep}}c@{\hspace{1\tabcolsep}}c}
    \toprule
    & & \multicolumn{4}{c}{White-box} & \multicolumn{4}{c}{Black-Box}  \\
    \cmidrule(r){3-6} 
    \cmidrule(r){7-10}
    Defense Model & Natural & PGD$^{20}$ & PGD$^{100}$  & PGD$^{1000}$  & CW$^{\infty}$ & PGD$^{20}$ & PGD$^{100}$ & PGD$^{1000}$  & CW$^{\infty}$ & AA \\
    \midrule
    NATURAL & 79.30 & 0 & 0 & 0 & 0 & 0 & 0 & 0 & 0 & 0 \\
    \midrule
    AT & 56.73 & 29.57 & 28.45 & 28.39 & 26.6 & 55.52 & 55.29 & 55.26 & 55.40 & 24.12 \\
    % TRADES & 58.24 & 30.10 & 29.66 & 29.64 & 25.97 & 57.05 & 56.71 & 56.67 & 56.77 & 24.92 \\
    $\DIAL_{\kl}$ (Ours) & 58.47 & \textbf{31.19} & \textbf{30.50} & \textbf{30.42} & \textbf{26.91} & 57.16 & 56.81 & 56.80 & 57.00 & \textbf{25.87} \\
    $\DIAL_{\ce}$ (Ours) & \textbf{60.77} & 27.87 & 26.66 & 26.61 & 25.98 & \textbf{59.48} & \textbf{59.06} & \textbf{58.96} & \textbf{59.20} & 23.51  \\
    \bottomrule
  \end{tabular}
\end{table}


% \begin{table}[!ht]
%   \caption{Robustness comparison of DIAL to Madry et al. and TRADES defense models on the SVHN dataset under different PGD white-box attacks and the ensemble Auto-Attack (AA).}
%   \label{svhn}
%   \centering
%   \begin{tabular}{llllll|l}
%     \toprule
%     \cmidrule(r){1-5}
%     Defense Model & Natural & PGD$^{20}$ & PGD$^{100}$ & PGD$^{1000}$ & CW$_{\infty}$ & AA\\
%     \midrule
%     $\DIAL_{\kl}$ (Ours) & $\mathbf{90.66}$ & $\mathbf{58.91}$ & $\mathbf{55.30}$ & $\mathbf{55.12}$ & $\mathbf{53.67}$  & $\mathbf{51.00}$  \\
%     Madry et al. & 89.90 & 53.23 & 49.45 & 49.23 & 48.25 & 45.25  \\
%     TRADES & 90.35 & 57.10 & 54.13 & 54.08 & 52.19 & 49.50 \\
%     \bottomrule
%   \end{tabular}
% \end{table}


\subsection{Performance comparison on CIFAR-10} \label{defence-settings}
In this part, we evaluate the performance of DIAL compared to other well-known methods on CIFAR-10. 
%To be consistent with other methods, 
We follow the same experiment setups as in~\cite{madry2017towards, wang2019improving, zhang2019theoretically}. When experiment settings are not identical between tested methods, we choose the most commonly used settings, and apply it to all experiments. This way, we keep the comparison as fair as possible and avoid reporting changes in results which are caused by inconsistent experiment settings \citep{pang2020bag}. To show that our results are not caused because of what is referred to as \textit{obfuscated gradients}~\citep{athalye2018obfuscated}, we evaluate our method with same setup as in our defense model, under strong attacks (e.g., PGD$^{1000}$) in both white-box, black-box settings, Auto-Attack ~\citep{croce2020reliable}, unforeseen "natural" corruptions~\citep{hendrycks2018benchmarking}, and unforeseen adversaries. To make sure that the reported improvements are not caused by \textit{adversarial overfitting}~\citep{rice2020overfitting}, we report best robust results for each method on average of 3 restarts, while omitting one standard deviation (which is smaller than 0.2\% in all experiments). Additional results for CIFAR-10 as well as comprehensive evaluation on MNIST can be found in Appendix \ref{mnist-results} and \ref{additional_res}.
%To further keep the comparison consistent, we followed the same attack settings for all methods.


\begin{table}[ht]
  \caption{Robustness against white-box, black-box attacks and Auto-Attack (AA) on CIFAR-10. Black-box attacks are generated using naturally trained surrogate model. Natural represents the naturally trained (non-adversarial) model.
  %and applied to the best performing robust models.
  }
  \vskip 0.1in
  \label{black-and_white-cifar}
  \centering
  \small
  \begin{tabular}{cccccccc@{\hspace{1\tabcolsep}}c}
    \toprule
    & & \multicolumn{3}{c}{White-box} & \multicolumn{3}{c}{Black-Box} \\
    \cmidrule(r){3-5} 
    \cmidrule(r){6-8}
    Defense Model & Natural & PGD$^{20}$ & PGD$^{100}$ & CW$^{\infty}$ & PGD$^{20}$ & PGD$^{100}$ & CW$^{\infty}$ & AA \\
    \midrule
    NATURAL & 95.43 & 0 & 0 & 0 & 0 & 0 & 0 &  0 \\
    \midrule
    TRADES & 84.92 & 56.60 & 55.56 & 54.20 & 84.08 & 83.89 & 83.91 &  53.08 \\
    MART & 83.62 & 58.12 & 56.48 & 53.09 & 82.82 & 82.52 & 82.80 & 51.10 \\
    AT & 85.10 & 56.28 & 54.46 & 53.99 & 84.22 & 84.14 & 83.92 & 51.52 \\
    ATDA & 76.91 & 43.27 & 41.13 & 41.01 & 75.59 & 75.37 & 75.35 & 40.08\\
    $\DIAL_{\kl}$ (Ours) & 85.25 & $\mathbf{58.43}$ & $\mathbf{56.80}$ & $\mathbf{55.00}$ & 84.30 & 84.18 & 84.05 & \textbf{53.75} \\
    $\DIAL_{\ce}$ (Ours)  & $\mathbf{89.59}$ & 54.31 & 51.67 & 52.04 &$ \mathbf{88.60}$ & $\mathbf{88.39}$ & $\mathbf{88.44}$ & 49.85 \\
    \midrule
    $\DIAL_{\awp}$ (Ours) & $\mathbf{85.91}$ & $\mathbf{61.10}$ & $\mathbf{59.86}$ & $\mathbf{57.67}$ & $\mathbf{85.13}$ & $\mathbf{84.93}$ & $\mathbf{85.03}$  & \textbf{56.78} \\
    $\TRADES_{\awp}$ & 85.36 & 59.27 & 59.12 & 57.07 & 84.58 & 84.58 & 84.59 & 56.17 \\
    \bottomrule
  \end{tabular}
\end{table}



\paragraph{CIFAR-10 setup.} We use the wide residual network (WRN-34-10)~\citep{zagoruyko2016wide} architecture. %used in the experiments of~\cite{madry2017towards, wang2019improving, zhang2019theoretically}. 
Sidelong this architecture, we integrate a domain classification layer. To generate the adversarial domain dataset, we use a perturbation size of $\epsilon=0.031$. We apply 10 of inner maximization iterations with perturbation step size of 0.007. Batch size is set to 128, weight decay is set to $7e^{-4}$, and the model is trained for 100 epochs. Similar to the other methods, the initial learning rate was set to 0.1, and decays by a factor of 10 at iterations 75 and 90. 
%For being consistent with other methods, the natural images are padded with 4-pixel padding with 32-random crop and random horizontal flip. Furthermore, all methods are trained using SGD with momentum 0.9. For $\DIAL_{\kl}$, we balance the robust loss with $\lambda=6$ and the domains loss with $r=4$. For $\DIAL_{\ce}$, we balance the robust loss with $\lambda=1$ and the domains loss with $r=2$. 
%We also introduce a version of our method that incorporates the AWP double-perturbation mechanism, named DIAL-AWP.
%which is trained using the same learning rate schedule used in ~\cite{wu2020adversarial}, where the initial 0.1 learning rate decays by a factor of 10 after 100 and 150 iterations. 
See Appendix \ref{cifar10-additional-setup} for additional details.

\begin{table}[ht]
  \caption{Black-box attack using the adversarially trained surrogate models on CIFAR-10.}
  \vskip 0.1in
  \label{black-box-cifar-adv}
  \centering
  \small
  \begin{tabular}{ll|c}
    \toprule
    \cmidrule(r){1-2}
    Surrogate (source) model & Target model & robustness \% \\
    % \midrule
    \midrule
    TRADES & $\DIAL_{\ce}$ & $\mathbf{67.77}$ \\
    $\DIAL_{\ce}$ & TRADES & 65.75 \\
    \midrule
    MART & $\DIAL_{\ce}$ & $\mathbf{70.30}$ \\
    $\DIAL_{\ce}$ & MART & 64.91 \\
    \midrule
    AT & $\DIAL_{\ce}$ & $\mathbf{65.32}$ \\
    $\DIAL_{\ce}$ & AT  & 63.54 \\
    \midrule
    ATDA & $\DIAL_{\ce}$ & $\mathbf{66.77}$ \\
    $\DIAL_{\ce}$ & ATDA & 52.56 \\
    \bottomrule
  \end{tabular}
\end{table}

\paragraph{White-box/Black-box robustness.} 
%We evaluate all defense models using Auto-Attack, PGD$^{20}$, PGD$^{100}$, PGD$^{1000}$ and CW$_{\infty}$ with step size 0.003. We constrain all attacks by the same perturbation $\epsilon=0.031$. 
As reported in Table~\ref{black-and_white-cifar} and Appendix~\ref{additional_res}, our method achieves better robustness compared to the other methods. Specifically, in the white-box settings, our method improves robustness over~\citet{madry2017towards} and TRADES by 2\% 
%using the common PGD$^{20}$ attack 
while keeping higher natural accuracy. We also observe better natural accuracy of 1.65\% over MART while also achieving better robustness over all attacks. Moreover, our method presents significant improvement of up to 15\% compared to the the domain invariant method suggested by~\citet{song2018improving} (ATDA).
%in both natural and robust accuracy. 
When incorporating 
%the double-perturbation mechanism of 
AWP, our method improves the results of $\TRADES_{\awp}$ by almost 2\%.
%and reaches state-of-the-art results for robust models with no additional data. 
% Additional results are available in Appendix~\ref{additional_res}.
When tested on black-box settings, $\DIAL_{\ce}$ presents a significant improvement of more than 4.4\% over the second-best performing method, and up to 13\%. In Table~\ref{black-box-cifar-adv}, we also present the black-box results when the source model is taken from one of the adversarially trained models. %Then, we compare our model to each one of them both as the source model and target model. 
In addition to the improvement in black-box robustness, $\DIAL_{\ce}$ also manages to achieve better clean accuracy of more than 4.5\% over the second-best performing method.
% Moreover, based on the auto-attack leader-board \footnote{\url{https://github.com/fra31/auto-attack}}, our method achieves the 1st place among models without additional data using the WRN-34-10 architecture.

% \begin{table}
%   \caption{White-box robustness on CIFAR-10 using WRN-34-10}
%   \label{white-box-cifar-10}
%   \centering
%   \begin{tabular}{lllll}
%     \toprule
%     \cmidrule(r){1-2}
%     Defense Model & Natural & PGD$^{20}$ & PGD$^{100}$ & PGD$^{1000}$ \\
%     \midrule
%     TRADES ~\cite{zhang2019theoretically} & 84.92  & 56.6 & 55.56 & 56.43  \\
%     MART ~\cite{wang2019improving} & 83.62  & 58.12 & 56.48 & 56.55  \\
%     Madry et al. ~\cite{madry2017towards} & 85.1  & 56.28 & 54.46 & 54.4  \\
%     Song et al. ~\cite{song2018improving} & 76.91 & 43.27 & 41.13 & 41.02  \\
%     $\DIAL_{\ce}$ (Ours) & $ \mathbf{90}$  & 52.12 & 48.88 & 48.78  \\
%     $\DIAL_{\kl}$ (Ours) & 85.25 & $\mathbf{58.43}$ & $\mathbf{56.8}$ & $\mathbf{56.73}$ \\
%     \midrule
%     $\DIAL_{\kl}$+AWP (Ours) & $\mathbf{85.91}$ & $\mathbf{61.1}$ & - & -  \\
%     TRADES+AWP \cite{wu2020adversarial} & 85.36 & 59.27 & 59.12 & -  \\
%     % MART + AWP & 84.43 & 60.68 & 59.32 & -  \\
%     \bottomrule
%   \end{tabular}
% \end{table}


% \begin{table}
%   \caption{White-box robustness on MNIST}
%   \label{white-box-mnist}
%   \centering
%   \begin{tabular}{llllll}
%     \toprule
%     \cmidrule(r){1-2}
%     Defense Model & Natural & PGD$^{40}$ & PGD$^{100}$ & PGD$^{1000}$ \\
%     \midrule
%     TRADES ~\cite{zhang2019theoretically} & 99.48 & 96.07 & 95.52 & 95.22 \\
%     MART ~\cite{wang2019improving} & 99.38  & 96.99 & 96.11 & 95.74  \\
%     Madry et al. ~\cite{madry2017towards} & 99.41  & 96.01 & 95.49 & 95.36 \\
%     Song et al. ~\cite{song2018improving}  & 98.72 & 96.82 & 96.26 & 96.2  \\
%     $\DIAL_{\kl}$ (Ours) & 99.46 & 97.05 & 96.06 & 95.99  \\
%     $\DIAL_{\ce}$ (Ours) & $\mathbf{99.49}$  & $\mathbf{97.38}$ & $\mathbf{96.45}$ & $\mathbf{96.33}$ \\
%     \bottomrule
%   \end{tabular}
% \end{table}


% \paragraph{Attacking MNIST.} For consistency, we use the same perturbation and step sizes. For MNIST, we use $\epsilon=0.3$ and step size of $0.01$. The natural accuracy of our surrogate (source) model is 99.51\%. Attacks results are reported in Table~\ref{black-and_white-mnist}. It is worth noting that the improvement margin is not conclusive on MNIST as it is on CIFAR-10, which is a more complex task.

% \begin{table}
%   \caption{Black-box robustness on MNIST and CIFAR-10 using naturally trained surrogate model and best performing robust models}
%   \label{black-box-mnist-cifar}
%   \centering
%   \begin{tabular}{lllllll}
%     \toprule
%     & \multicolumn{3}{c}{MNIST} & \multicolumn{3}{c}{CIFAR-10} \\
%     \cmidrule(r){2-4} 
%     \cmidrule(r){5-7}  
%     Defense Model & PGD$^{40}$ & PGD$^{100}$ & PGD$^{1000}$ & PGD$^{20}$ & PGD$^{100}$ & PGD$^{1000}$ \\
%     \midrule
%     TRADES ~\cite{zhang2019theoretically} & 98.12 & 97.86 & 97.81 & 84.08 & 83.89 & 83.8 \\
%     MART ~\cite{wang2019improving} & 98.16 & 97.96 & 97.89  & 82.82 & 82.52 & 82.47 \\
%     Madry et al. ~\cite{madry2017towards}  & 98.05 & 97.73 & 97.78 & 84.22 & 84.14 & 83.96 \\
%     Song et al. ~\cite{song2018improving} & 97.74 & 97.28 & 97.34 & 75.59 & 75.37 & 75.11 \\
%     $\DIAL_{\kl}$ (Ours) & 98.14 & 97.83 & 97.87  & 84.3 & 84.18 & 84.0 \\
%     $\DIAL_{\ce}$ (Ours)  & $\mathbf{98.37}$ & $\mathbf{98.12}$ & $\mathbf{98.05}$  & $\mathbf{89.13}$ & $\mathbf{88.89}$ & $\mathbf{88.78}$ \\
%     \bottomrule
%   \end{tabular}
% \end{table}



% \subsubsection{Ensemble attack} In addition to the white-box and black-box settings, we evaluate our method on the Auto-Attack ~\citep{croce2020reliable} using $\ell_{\infty}$ threat model with perturbation $\epsilon=0.031$. Auto-Attack is an ensemble of parameter-free attacks. It consists of three white-box attacks: APGD-CE which is a step size-free version of PGD on the cross-entropy ~\citep{croce2020reliable}. APGD-DLR which is a step size-free version of PGD on the DLR loss ~\citep{croce2020reliable} and FAB which  minimizes the norm of the adversarial perturbations, and one black-box attack: square attack which is a query-efficient black-box attack ~\citep{andriushchenko2020square}. Results are presented in Table~\ref{auto-attack}. Based on the auto-attack leader-board \footnote{\url{https://github.com/fra31/auto-attack}}, our method achieves the 1st place among models without additional data using the WRN-34-10 architecture.

%Additional results can be found in Appendix ~\ref{additional_res}.

% \begin{table}
%   \caption{Auto-Attack (AA) on CIFAR-10 with perturbation size $\epsilon=0.031$ with $\ell_{\infty}$ threat model}
%   \label{auto-attack}
%   \centering
%   \begin{tabular}{lll}
%     \toprule
%     \cmidrule(r){1-2}
%     Defense Model & AA \\
%     \midrule
%     TRADES ~\cite{zhang2019theoretically} & 53.08  \\
%     MART ~\cite{wang2019improving} & 51.1  \\
%     Madry et al. ~\cite{madry2017towards} & 51.52    \\
%     Song et al.   ~\cite{song2018improving} & 40.18 \\
%     $\DIAL_{\ce}$ (Ours) & 47.33  \\
%     $\DIAL_{\kl}$ (Ours) & $\mathbf{53.75}$ \\
%     \midrule
%     DIAL-AWP (Ours) & $\mathbf{56.78}$ \\
%     TRADES-AWP \cite{wu2020adversarial} & 56.17 \\
%     \bottomrule
%   \end{tabular}
% \end{table}


% \begin{table}[!ht]
%   \caption{Auto-Attack (AA) Robustness (\%) on CIFAR-10 with $\epsilon=0.031$ using an $\ell_{\infty}$ threat model}
%   \label{auto-attack}
%   \centering
%   \begin{tabular}{cccccc|cc}
%     \toprule
%     % \multicolumn{8}{c}{Defence Model}  \\
%     % \cmidrule(r){1-8} 
%     TRADES & MART & Madry & Song & $\DIAL_{\ce}$ & $\DIAL_{\kl}$ & DIAL-AWP  & TRADES-AWP\\
%     \midrule
%     53.08 & 51.10 & 51.52 &  40.08 & 47.33  & $\mathbf{53.75}$ & $\mathbf{56.78}$ & 56.17 \\

%     \bottomrule
%   \end{tabular}
% \end{table}

% \begin{table}[!ht]
% \caption{$F_1$-robust measurement using PGD$^{20}$ attack in white-box and black-box settings on CIFAR-10}
%   \label{f1-robust}
%   \centering
%   \begin{tabular}{ccccccc|cc}
%     \toprule
%     % \multicolumn{8}{c}{Defence Model}  \\
%     % \cmidrule(r){1-8} 
%     Defense Model & TRADES & MART & Madry & Song & $\DIAL_{\kl}$ & $\DIAL_{\ce}$ & DIAL-AWP  & TRADES-AWP\\
%     \midrule
%     White-box & 0.659 & 0.666 & 0.657 & 0.518 & $\mathbf{0.675}$  & 0.643 & $\mathbf{0.698}$ & 0.682 \\
%     Black-box & 0.844 & 0.831 & 0.846 & 0.761 & 0.847 & $\mathbf{0.895}$ & $\mathbf{0.854}$ &  0.849 \\
%     \bottomrule
%   \end{tabular}
% \end{table}

\subsubsection{Robustness to Unforeseen Attacks and Corruptions}
\paragraph{Unforeseen Adversaries.} To further demonstrate the effectiveness of our approach, we test our method against various adversaries that were not used during the training process. We attack the model under the white-box settings with $\ell_{2}$-PGD, $\ell_{1}$-PGD, $\ell_{\infty}$-DeepFool and $\ell_{2}$-DeepFool \citep{moosavi2016deepfool} adversaries using Foolbox \citep{rauber2017foolbox}. We applied commonly used attack budget 
%(perturbation for PGD adversaries and overshot for DeepFool adversaries) 
with 20 and 50 iterations for PGD and DeepFool, respectively.
Results are presented in Table \ref{unseen-attacks}. As can be seen, our approach  gains an improvement of up to 4.73\% over the second best method under the various attack types and an average improvement of 3.7\% over all threat models.


\begin{table}[ht]
  \caption{Robustness on CIFAR-10 against unseen adversaries under white-box settings.}
  \vskip 0.1in
  \label{unseen-attacks}
  \centering
%   \small
  \begin{tabular}{c@{\hspace{1.5\tabcolsep}}c@{\hspace{1.5\tabcolsep}}c@{\hspace{1.5\tabcolsep}}c@{\hspace{1.5\tabcolsep}}c@{\hspace{1.5\tabcolsep}}c@{\hspace{1.5\tabcolsep}}c@{\hspace{1.5\tabcolsep}}c}
    \toprule
    Threat Model & Attack Constraints & $\DIAL_{\kl}$ & $\DIAL_{\ce}$ & AT & TRADES & MART & ATDA \\
    \midrule
    \multirow{2}{*}{$\ell_{2}$-PGD} & $\epsilon=0.5$ & 76.05 & \textbf{80.51} & 76.82 & 76.57 & 75.07 & 66.25 \\
    & $\epsilon=0.25$ & 80.98 & \textbf{85.38} & 81.41 & 81.10 & 80.04 & 71.87 \\\midrule
    \multirow{2}{*}{$\ell_{1}$-PGD} & $\epsilon=12$ & 74.84 & \textbf{80.00} & 76.17 & 75.52 & 75.95 & 65.76 \\
    & $\epsilon=7.84$ & 78.69 & \textbf{83.62} & 79.86 & 79.16 & 78.55 & 69.97 \\
    \midrule
    $\ell_{2}$-DeepFool & overshoot=0.02 & 84.53 & \textbf{88.88} & 84.15 & 84.23 & 82.96 & 76.08 \\\midrule
    $\ell_{\infty}$-DeepFool & overshoot=0.02 & 68.43 & \textbf{69.50} & 67.29 & 67.60 & 66.40 & 57.35 \\
    \bottomrule
  \end{tabular}
\end{table}


%%%%%%%%%%%%%%%%%%%%%%%%% conference version %%%%%%%%%%%%%%%%%%%%%%%%%%%%%%%%%%%%%
\paragraph{Unforeseen Corruptions.}
We further demonstrate that our method consistently holds against unforeseen ``natural'' corruptions, consists of 18 unforeseen diverse corruption types proposed by \citet{hendrycks2018benchmarking} on CIFAR-10, which we refer to as CIFAR10-C. The CIFAR10-C benchmark covers noise, blur, weather, and digital categories. As can be shown in Figure \ref{corruption}, our method gains a significant and consistent improvement over all the other methods. Our method leads to an average improvement of 4.7\% with minimum improvement of 3.5\% and maximum improvement of 5.9\% compared to the second best method over all unforeseen attacks. See Appendix \ref{corruptions-apendix} for the full experiment results.


\begin{figure}[h]
 \centering
  \includegraphics[width=0.4\textwidth]{figures/spider_full.png}
%   \caption{Summary of accuracy over all unforeseen corruptions compared to the second and third best performing methods.}
  \caption{Accuracy comparison over all unforeseen corruptions.}
  \label{corruption}
\end{figure}


%%%%%%%%%%%%%%%%%%%%%%%%% conference version %%%%%%%%%%%%%%%%%%%%%%%%%%%%%%%%%%%%%

%%%%%%%%%%%%%%%%%%%%%%%%% Arxiv version %%%%%%%%%%%%%%%%%%%%%%%%%%%%%%%%%%%%%
% \newpage
% \paragraph{Unforeseen Corruptions.}
% We further demonstrate that our method consistently holds against unforeseen "natural" corruptions, consists of 18 unforeseen diverse corruption types proposed by \cite{hendrycks2018benchmarking} on CIFAR-10, which we refer to as CIFAR10-C. The CIFAR10-C benchmark covers noise, blur, weather, and digital categories. As can be shown in Figure  \ref{spider-full-graph}, our method gains a significant and consistent improvement over all the other methods. Our approach leads to an average improvement of 4.7\% with minimum improvement of 3.5\% and maximum improvement of 5.9\% compared to the second best method over all unforeseen attacks. Full accuracy results against unforeseen corruptions are presented in Tables \ref{corruption-table1} and \ref{corruption-table2}. 

% \begin{table}[!ht]
%   \caption{Accuracy (\%) against unforeseen corruptions.}
%   \label{corruption-table1}
%   \centering
%   \tiny
%   \begin{tabular}{lcccccccccccccccccc}
%     \toprule
%     Defense Model & brightness & defocus blur & fog & glass blur & jpeg compression & motion blur & saturate & snow & speckle noise  \\
%     \midrule
%     TRADES & 82.63 & 80.04 & 60.19 & 78.00 & 82.81 & 76.49 & 81.53 & 80.68 & 80.14 \\
%     MART & 80.76 & 78.62 & 56.78 & 76.60 & 81.26 & 74.58 & 80.74 & 78.22 & 79.42 \\
%     AT &  83.30 & 80.42 & 60.22 & 77.90 & 82.73 & 76.64 & 82.31 & 80.37 & 80.74 \\
%     ATDA & 72.67 & 69.36 & 45.52 & 64.88 & 73.22 & 63.47 & 72.07 & 68.76 & 72.27 \\
%     DIAL (Ours)  & \textbf{87.14} & \textbf{84.84} & \textbf{66.08} & \textbf{81.82} & \textbf{87.07} & \textbf{81.20} & \textbf{86.45} & \textbf{84.18} & \textbf{84.94} \\
%     \bottomrule
%   \end{tabular}
% \end{table}


% \begin{table}[!ht]
%   \caption{Accuracy (\%) against unforeseen corruptions.}
%   \label{corruption-table2}
%   \centering
%   \tiny
%   \begin{tabular}{lcccccccccccccccccc}
%     \toprule
%     Defense Model & contrast & elastic transform & frost & gaussian noise & impulse noise & pixelate & shot noise & spatter & zoom blur \\
%     \midrule
%     TRADES & 43.11 & 79.11 & 76.45 & 79.21 & 73.72 & 82.73 & 80.42 & 80.72 & 78.97 \\
%     MART & 41.22 & 77.77 & 73.07 & 78.30 & 74.97 & 81.31 & 79.53 & 79.28 & 77.8 \\
%     AT & 43.30 & 79.58 & 77.53 & 79.47 & 73.76 & 82.78 & 80.86 & 80.49 & 79.58 \\
%     ATDA & 36.06 & 67.06 & 62.56 & 70.33 & 64.63 & 73.46 & 72.28 & 70.50 & 67.31 \\
%     DIAL (Ours) & \textbf{48.84} & \textbf{84.13} & \textbf{81.76} & \textbf{83.76} & \textbf{78.26} & \textbf{87.24} & \textbf{85.13} & \textbf{84.84} & \textbf{83.93}  \\
%     \bottomrule
%   \end{tabular}
% \end{table}


% \begin{figure}[!ht]
%   \centering
%   \includegraphics[width=9cm]{figures/spider_full.png}
%   \caption{Accuracy comparison with all tested methods over unforeseen corruptions.}
%   \label{spider-full-graph}
% \end{figure}
% %%%%%%%%%%%%%%%%%%%%%%%%% Arxiv version %%%%%%%%%%%%%%%%%%%%%%%%%%%%%%%%%%%%%
%%%%%%%%%%%%%%%%%%%%%%%%% Arxiv version %%%%%%%%%%%%%%%%%%%%%%%%%%%%%%%%%%%%%

\subsubsection{Transfer Learning}
Recent works \citep{salman2020adversarially,utrera2020adversarially} suggested that robust models transfer better on standard downstream classification tasks. In Table \ref{transfer-res} we demonstrate the advantage of our method when applied for transfer learning across CIFAR10 and CIFAR100 using the common linear evaluation protocol. see Appendix \ref{transfer-learning-settings} for detailed settings.

\begin{table}[H]
  \caption{Transfer learning results comparison.}
  \vskip 0.1in
  \label{transfer-res}
  \centering
  \small
\begin{tabular}{c|c|c|c}
\toprule

\multicolumn{2}{l}{} & \multicolumn{2}{c}{Target} \\
\cmidrule(r){3-4}
Source & Defence Model & CIFAR10 & CIFAR100 \\
\midrule
\multirow{3}{*}{CIFAR10} & DIAL & \multirow{3}{*}{-} & \textbf{28.57} \\
 & AT &  & 26.95  \\
 & TRADES &  & 25.40  \\
 \midrule
\multirow{3}{*}{CIFAR100} & DIAL & \textbf{73.68} & \multirow{3}{*}{-} \\
 & AT & 71.41 & \\
 & TRADES & 71.42 &  \\
%  \midrule
% \multirow{3}{}{SVHN} & DIAL &  &  & \multirow{3}{}{-} \\
%  & Madry et al. &  &  &  \\
%  & TRADES &  &  &  \\ 
\bottomrule
\end{tabular}
\end{table}


\subsubsection{Modularity and Ablation Studies}

We note that the domain classifier is a modular component that can be integrated into existing models for further improvements. Removing the domain head and related loss components from the different DIAL formulations results in some common adversarial training techniques. For $\DIAL_{\kl}$, removing the domain and related loss components results in the formulation of TRADES. For $\DIAL_{\ce}$, removing the domain and related loss components results in the original formulation of the standard adversarial training, and for $\DIAL_{\awp}$ the removal results in $\TRADES_{\awp}$. Therefore, the ablation studies will demonstrate the effectiveness of combining DIAL on top of different adversarial training methods. 

We investigate the contribution of the additional domain head component introduced in our method. Experiment configuration are as in \ref{defence-settings}, and robust accuracy is based on white-box PGD$^{20}$ on CIFAR-10 dataset. We remove the domain head from both $\DIAL_{\kl}$, $\DIAL_{\awp}$, and $\DIAL_{\ce}$ (equivalent to $r=0$) and report the natural and robust accuracy. We perform 3 random restarts and omit one standard deviation from the results. Results are presented in Figure \ref{ablation}. All DIAL variants exhibits stable improvements on both natural accuracy and robust accuracy. $\DIAL_{\ce}$, $\DIAL_{\kl}$, and $\DIAL_{\awp}$ present an improvement of 1.82\%, 0.33\%, and 0.55\% on natural accuracy and an improvement of 2.5\%, 1.87\%, and 0.83\% on robust accuracy, respectively. This evaluation empirically demonstrates the benefits of incorporating DIAL on top of different adversarial training techniques.
% \paragraph{semi-supervised extensions.} Since the domain classifier does not require the class labels, we argue that additional unlabeled data can be leveraged in future work.
%for improved results. 

\begin{figure}[ht]
  \centering
  \includegraphics[width=0.35\textwidth]{figures/ablation_graphs3.png}
  \caption{Ablation studies for $\DIAL_{\kl}$, $\DIAL_{\ce}$, and $\DIAL_{\awp}$ on CIFAR-10. Circle represent the robust-natural accuracy without using DIAL, and square represent the robust-natural accuracy when incorporating DIAL.
  %to further investigate the impact of the domain head and loss on natural and robust accuracy.
  }
  \label{ablation}
\end{figure}

\subsubsection{Visualizing DIAL}
To further illustrate the superiority of our method, we visualize the model outputs from the different methods on both natural and adversarial test data.
% adversarial test data generated using PGD$^{20}$ white-box attack with step size 0.003 and $\epsilon=0.031$ on CIFAR-10. 
Figure~\ref{tsne1} shows the embedding received after applying t-SNE ~\citep{van2008visualizing} with two components on the model output for our method and for TRADES. DIAL seems to preserve strong separation between classes on both natural test data and adversarial test data. Additional illustrations for the other methods are attached in Appendix~\ref{additional_viz}. 

\begin{figure}[h]
\centering
  \subfigure[\textbf{DIAL} on natural logits]{\includegraphics[width=0.21\textwidth]{figures/domain_ce_test.png}}
  \hspace{0.03\textwidth}
  \subfigure[\textbf{DIAL} on adversarial logits]{\includegraphics[width=0.21\textwidth]{figures/domain_ce_adversarial.png}}
  \hspace{0.03\textwidth}
    \subfigure[\textbf{TRADES} on natural logits]{\includegraphics[width=0.21\textwidth]{figures/trades_test.png}}
    \hspace{0.03\textwidth}
    \subfigure[\textbf{TRADES} on adversarial logits]{\includegraphics[width=0.21\textwidth]{figures/trades_adversarial.png}}
  \caption{t-SNE embedding of model output (logits) into two-dimensional space for DIAL and TRADES using the CIFAR-10 natural test data and the corresponding PGD$^{20}$ generated adversarial examples.}
  \label{tsne1}
\end{figure}


% \begin{figure}[ht]
% \centering
%   \begin{subfigure}{4cm}
%     \centering\includegraphics[width=3.3cm]{figures/domain_ce_test.png}
%     \caption{\textbf{DIAL} on nat. examples}
%   \end{subfigure}
%   \begin{subfigure}{4cm}
%     \centering\includegraphics[width=3.3cm]{figures/domain_ce_adversarial.png}
%     \caption{\textbf{DIAL} on adv. examples}
%   \end{subfigure}
  
%   \begin{subfigure}{4cm}
%     \centering\includegraphics[width=3.3cm]{figures/trades_test.png}
%     \caption{\textbf{TRADES} on nat. examples}
%   \end{subfigure}
%   \begin{subfigure}{4cm}
%     \centering\includegraphics[width=3.3cm]{figures/trades_adversarial.png}
%     \caption{\textbf{TRADES} on adv. examples}
%   \end{subfigure}
%   \caption{t-SNE embedding of model output (logits) into two-dimensional space for DIAL and TRADES using the CIFAR-10 natural test data and the corresponding adversarial examples.}
%   \label{tsne1}
% \end{figure}



% \begin{figure}[ht]
% \centering
%   \begin{subfigure}{6cm}
%     \centering\includegraphics[width=5cm]{figures/domain_ce_test.png}
%     \caption{\textbf{DIAL} on nat. examples}
%   \end{subfigure}
%   \begin{subfigure}{6cm}
%     \centering\includegraphics[width=5cm]{figures/domain_ce_adversarial.png}
%     \caption{\textbf{DIAL} on adv. examples}
%   \end{subfigure}
  
%   \begin{subfigure}{6cm}
%     \centering\includegraphics[width=5cm]{figures/trades_test.png}
%     \caption{\textbf{TRADES} on nat. examples}
%   \end{subfigure}
%   \begin{subfigure}{6cm}
%     \centering\includegraphics[width=5cm]{figures/trades_adversarial.png}
%     \caption{\textbf{TRADES} on adv. examples}
%   \end{subfigure}
%   \caption{t-SNE embedding of model output (logits) into two-dimensional space for DIAL and TRADES using the CIFAR-10 natural test data and the corresponding adversarial examples.}
%   \label{tsne1}
% \end{figure}



\subsection{Balanced measurement for robust-natural accuracy}
One of the goals of our method is to better balance between robust and natural accuracy under a given model. For a balanced metric, we adopt the idea of $F_1$-score, which is the harmonic mean between the precision and recall. However, rather than using precision and recall, we measure the $F_1$-score between robustness and natural accuracy,
using a measure we call
%We named it
the
\textbf{$\mathbf{F_1}$-robust} score.
\begin{equation}
F_1\text{-robust} = \dfrac{\text{true\_robust}}
{\text{true\_robust}+\frac{1}{2}
%\cdot
(\text{false\_{robust}}+
\text{false\_natural})},
\end{equation}
where $\text{true\_robust}$ are the adversarial examples that were correctly classified, $\text{false\_{robust}}$ are the adversarial examples that were miss-classified, and $\text{false\_natural}$ are the natural examples that were miss-classified.
%We tested the proposed $F_1$-robust score using PGD$^{20}$ on CIFAR-10 dataset in white-box and black-box settings. 
Results are presented in Table~\ref{f1-robust} and demonstrate that our method achieves the best $F_1$-robust score in both settings, which supports our findings from previous sections.

% \begin{table}[!ht]
%   \caption{$F_1$-robust measurement using PGD$^{20}$ attack in white and black box settings on CIFAR-10}
%   \label{f1-robust}
%   \centering
%   \begin{tabular}{lll}
%     \toprule
%     \cmidrule(r){1-2}
%     Defense Model & White-box & Black-box \\
%     \midrule
%     TRADES & 0.65937  & 0.84435 \\
%     MART & 0.66613  & 0.83153  \\
%     Madry et al. & 0.65755 & 0.84574   \\
%     Song et al. & 0.51823 & 0.76092  \\
%     $\DIAL_{\ce}$ (Ours) & 0.65318   & $\mathbf{0.88806}$  \\
%     $\DIAL_{\kl}$ (Ours) & $\mathbf{0.67479}$ & 0.84702 \\
%     \midrule
%     \midrule
%     DIAL-AWP (Ours) & $\mathbf{0.69753}$  & $\mathbf{0.85406}$  \\
%     TRADES-AWP & 0.68162 & 0.84917 \\
%     \bottomrule
%   \end{tabular}
% \end{table}

\begin{table}[ht]
\small
  \caption{$F_1$-robust measurement using PGD$^{20}$ attack in white and black box settings on CIFAR-10.}
  \vskip 0.1in
  \label{f1-robust}
  \centering
%   \small
  \begin{tabular}{c
  @{\hspace{1.5\tabcolsep}}c @{\hspace{1.5\tabcolsep}}c @{\hspace{1.5\tabcolsep}}c @{\hspace{1.5\tabcolsep}}c
  @{\hspace{1.5\tabcolsep}}c @{\hspace{1.5\tabcolsep}}c @{\hspace{1.5\tabcolsep}}|
  @{\hspace{1.5\tabcolsep}}c
  @{\hspace{1.5\tabcolsep}}c}
    \toprule
    % \cmidrule(r){8-9}
     & TRADES & MART & AT & ATDA & $\DIAL_{\ce}$ & $\DIAL_{\kl}$ & $\DIAL_{\awp}$ & $\TRADES_{\awp}$ \\
    \midrule
    White-box & 0.659 & 0.666 & 0.657 & 0.518 & 0.660 & \textbf{0.675} & \textbf{0.698} & 0.682 \\
    Black-box & 0.844 & 0.831 & 0.845 & 0.761 & \textbf{0.890} & 0.847 & \textbf{0.854} & 0.849 \\ 
    \bottomrule
  \end{tabular}
\end{table}



\subsubsection*{Acknowledgements}
The authors would like to thank K.~Jamieson, L.~Jain, and V. Roulet for fruitful discussions.
L.~Liu is supported by NSF CCF-2019844 and NSF DMS-2023166 and NSF DMS-2133244.
Z.~Harchaoui is supported by NSF CCF-2019844, NSF DMS-2134012, NSF DMS-2023166, CIFAR-LMB, and faculty research awards.
Part of this work was done while Z.~Harchaoui was visiting the Simons Institute for the Theory of Computing.

\clearpage

\bibliographystyle{abbrvnat}
\bibliography{biblio}

\clearpage
\appendix

% Make the appendix single col
\begingroup
\let\clearpage\relax 
\onecolumn 
\endgroup
% End: appendix single col

\addcontentsline{toc}{section}{Appendix} 
\part{Appendix} 
\parttoc
\clearpage


\section{Proof of main results}
\label{sec:proofs}
\section{Proofs from \secref{sec:qkd}}
\label{app:proofs}

In \secref{sec:qkd} we show how to define the security of QKD in a
composable framework and relate this to the trace distance security
criterion introduced in \textcite{Ren05}. This composable treatment of
the security of QKD follows the literature \cite{BHLMO05,MR09}, and
the results presented in \secref{sec:qkd} may be found in
\textcite{BHLMO05,MR09} as well. The formulation of the statements
differs however from those works, since we use here the Abstract
Cryptography framework of \textcite{MR11}. So for completeness, we
provide here proofs of the main results from \secref{sec:qkd}.

\begin{proof}[Proof of \thmref{thm:qkd}]
  Recall that in \secref{sec:security.simulator} we fixed the
  simulator and show that to satisfy \eqnref{eq:qkd.security} it is
  sufficient for \eqnref{eq:qkd.security.2} to hold. Here, we will
  break \eqnref{eq:qkd.security.2} into security [\eqnref{eq:qkd.cor}]
  and correctness [\eqnref{eq:qkd.sec}], thus proving the theorem.

  Let us define $\gamma_{ABE}$ to be a state obtained from
  $\rho^{\top}_{ABE}$ [\eqnref{eq:qkd.security.tmp}] by throwing away
  the $B$ system and replacing it with a copy of $A$, i.e., \[
  \gamma_{ABE} = \frac{1}{1-p^\bot} \sum_{k_A,k_B \in \cK} p_{k_A,k_B}
  \proj{k_A,k_A} \otimes \rho^{k_A,k_B}_E.\] From the triangle
  inequality we get \begin{multline*} D(\rho^\top_{ABE},\tau_{AB} \otimes
  \rho^\top_{E}) \leq \\ D(\rho^\top_{ABE},\gamma_{ABE}) +
  D(\gamma_{ABE},\tau_{AB} \otimes \rho^\top_{E}) .\end{multline*}

Since in the states $\gamma_{ABE}$ and
$\tau_{AB} \otimes \rho^\top_{E}$ the $B$ system is a copy of the $A$
system, it does not modify the distance. Furthermore,
$\trace[B]{\gamma_{ABE}} =
\trace[B]{\rho^{\top}_{ABE}}$. Hence
\[D(\gamma_{ABE},\tau_{AB} \otimes \rho^\top_{E}) =
  D(\gamma_{AE},\tau_{A} \otimes \rho^\top_{E}) =
  D(\rho^\top_{AE},\tau_{A} \otimes \rho^\top_{E}).\]

For the other term note that
\begin{align*}
  & D(\rho^\top_{ABE},\gamma_{ABE}) \\
  & \qquad \leq \sum_{k_A,k_B} \frac{p_{k_A,k_B}}{1-p^{\bot}}
    D\left(\proj{k_A,k_B} \otimes \rho^{k_A,k_B}_E,\right. \\
  & \qquad \qquad \qquad \qquad \qquad \qquad \left.\proj{k_A,k_A} \otimes \rho^{k_A,k_B}_E \right)\\
  & \qquad = \sum_{k_A \neq k_B} \frac{p_{k_A,k_B}}{1-p^{\bot}} = \frac{1}{1-p^{\bot}}\Pr
  \left[ K_A \neq K_B \right].
\end{align*}
Putting the above together with \eqnref{eq:qkd.security.2}, we get
\begin{align*} & D(\rho_{ABE},\tilde{\rho}_{ABE}) \\
  & \qquad = (1-p^\bot)
  D(\rho^\top_{ABE},\tau_{AB} \otimes \rho^\bot_{E}) \\ & \qquad \leq \Pr
  \left[ K_A \neq K_B \right] + (1-p^\bot) D(\rho^\top_{AE},\tau_{A}
  \otimes \rho^\top_{E}). \qedhere \end{align*}
\end{proof}

\begin{proof}[Proof of \lemref{lem:robustness}]
  By construction, $\aK_\delta$ aborts with exactly the same
  probability as the real system. And because $\sigma^{\qkd}_E$
  simulates the real protocols, if we plug a converter $\pi_E$ in
  $\aK\sigma^{\qkd}_E$ which emulates the noisy channel $\aQ_q$ and
  blogs the output of the simulated authentic channel, then
  $\aK_\delta = \aK\sigma^{\qkd}_E\pi_E$. Also note that by
  construction we have
  $\aQ_q \| \aA' = \left(\aQ \| \aA\right) \pi_E$. Thus
  \begin{multline*} d\left( \pi_A^{\qkd}\pi_B^{\qkd}(\aQ_q \| \aA')
      ,\aK_\delta\right) \\ = d\left( \pi_A^{\qkd}\pi_B^{\qkd}\left(\aQ
        \| \aA\right) \pi_E , \aK\sigma^{\qkd}_E\pi_E\right). \end{multline*}

  Finally, because the converter $\pi_E$ on both the real and ideal
  systems can only decrease their distance (see
  \secref{sec:ac.systems}), the result follows.
\end{proof}


%%% Local Variables:
%%% TeX-master: "main.tex"
%%% End:


\section{Examples and applications}
\label{sec:example}
\section{Discussion on Approximation \textit{vs} Stability and Recovery}\label{sec:approx-stability}


In the world of approximation algorithms, for a maximization problem for which an algorithm outputs $S$ and the optimum is $S^*$, what we typically try to prove is that
$w(S)\ge w(S^*)/\alpha$, even in the worst case; this \textit{approximation inequality} means that the algorithm at hand is an $\alpha$-approximation, so it is a \textit{good} algorithm. Though one might be quick to say that recovery of $\alpha$-stable instances immediately follows from the approximation inequality, this is not true because of the intersection $S\cap S^*$; if we have no intersection, then recovery indeed follows. 

What the research on stability and exact recovery suggests, is that we should try to understand if some of our already known approximation algorithms have the stronger property $w(S\setminus S^*)\ge w(S^*\setminus S)/\alpha$ or at least if they have it on stable instances. We refer to the latter as the \textit{recovery inequality}. This would directly imply an exact recovery result for $\alpha$-stable instances because we could $\alpha$-perturb only the $S\setminus S^*$ part of the input and get: 
\[
\noindent w(S\setminus S^*)\ge w(S^*\setminus S)/\alpha \implies \alpha\cdot w(S\setminus S^*) +w(S\cap S^*) \ge w(S^*\setminus S) +w(S\cap S^*) = w(S^*)
\] thus violating the fact we were given an $\alpha$-stable instance, unless $S\setminus S^* = \emptyset$.

This would mean that the algorithm successfully retrieved $S^*$ and could potentially explain why many approximation algorithms behave far better in practice than in theory. Furthermore, from a theory perspective, it would mean that many results from the well-studied area of approximation algorithms could be translated in terms of stability and recovery.

As a concluding remark, we want to point out that even though one might think that an $\alpha$-approximation algorithm needs at least $\alpha$-stability for recovery, this is not true as the somewhat counterintuitive result from \cite{balcan2015k} tells us: asymmetric $k$-center cannot be approximated to any constant factor, but can be solved optimally on 2-stable instances. This was the
first problem that is hard to approximate to any constant factor in the worst case, yet can be optimally
solved in polynomial time for 2-stable instances. The other direction (having an $\alpha$-approximation algorithm that cannot recover arbitrarily stable instances) is also true. These findings suggest that there are interesting connections between stability, exact recovery and approximation.


\section{Technical tools}
\label{sec:tools}
\section{Tools}
\label{sec:tool}

%In this section we present our algorithm \ebdjoin.  
Before presenting our algorithm, we would like to introduce a few tools that we shall use in \ebdjoin+, including the CGK-embedding, the LSH for the Hamming distance, and an algorithm for exact edit distance computation. We list in Table~\ref{tab:notation} a set of notations that will be used in the presentation.


%\begin{adjustwidth}{-1cm}{-1cm}
\begin{table}[t]
\centering
\begin{tabular}{|p{.12\textwidth}| p{.85\textwidth}| m{.01\textwidth}|} 
\hline
Notation & Definition\\ 
\hline
$[n]$ & $[n] = \{1, 2, \ldots, n\}$ \\
\hline
$K$ & Edit distance threshold\\ 
\hline
$\S$ & The set of input strings \\ 
\hline
$s_i$ & The $i$-th string in $\S$ \\ 
\hline
$\abs{x}$ & Length of string $x$ \\ 
\hline
$n$ & Number of input strings, i.e.,  $n = \abs{\S}$\\ 
\hline
$N$ & Maximum length of strings in $\S$\\ 
\hline
$\Sigma$ & Alphabet of strings in $\S$ \\ 
\hline
$r$ & Number of CGK-embeddings for each input string\\ 
\hline
$t_i^{\ell}$ & The output string generated by the $\ell$-th CGK-embedding of $s_i$ \\ 
\hline
$z$ & Number of hash functions used in LSH for each string generated by CGK-embedding\\ 
\hline
$m$ & Length of the LSH signature\\ 
\hline
$f_j^{\ell} $ &   $f_j^{\ell} : \Sigma^N \to \Sigma^m$, the $j$-th ($j\in[z]$)  LSH function for each string generated by the $\ell$-th CGK-embedding
\\ 
\hline
$\mathcal{D}_j^{\ell}$ & The hash table corresponding to the LSH function $f_j^{\ell}$ \\ 
\hline
$\Delta$ & A parameter for dealing with shifts  \\ 
\hline
$s_{i,k}$ & The $k$-th substring of $s_i$ starting at the $((k-1)\Delta+1)$-th character\\ 
\hline
$t_{i,k}^\ell$ & The output string generated by the $\ell$-th CGK-embedding of $s_{i,k}$\\ 
\hline
$T$ & The threshold of the number of matched hash signatures for a pair of substrings\\ 
\hline
\end{tabular}
\caption{Summary of Notations}
\label{tab:notation}
\end{table}
%\end{adjustwidth}


\subsection{The CGK-Embedding}
\label{sec:CGK}

We describe the CGK-embedding in Algorithm~\ref{alg:CGK}. Below we  illustrate the main idea behind the CGK-embedding, which we believe is useful and important to understand the intuition of \ebdjoin+.  We note that the original algorithm in \cite{CGK16} was only described for binary strings, and it was mentioned that we can encode an alphabet $\Sigma$ into binary codes using $\log\abs{\Sigma}$ bits for each character.  In our rewrite (Algorithm~\ref{alg:CGK}) we choose to use the alphabet $\Sigma$ directly without the encoding.  This may give some performance gain when the size of the alphabet is small. 


\begin{algorithm}[t]
\begin{algorithmic}[1]
\Require A string $x \in \Sigma^\eta$ for some $\eta \le N$, and a random string $R \in \{0,1\}^{ 3 N |\Sigma|}$
\Ensure A string $x' \in \Sigma^{3N}$
\smallskip

%\State Pad $x$ with ``$\perp$'' in the tail to make it of length $N$; denote the new string by $\tilde{x}$

\State Interpret $R$ as a set of functions 

$\pi_1,\dots,\pi_{3N}:  \Sigma \rightarrow \{0,1\}$; for the $k$-th char $\sigma_k$ in $\Sigma$, 

$\pi_j(\sigma_k) = R[(j-1) \cdot \abs{\Sigma} + k]$
\State $i \leftarrow 1$
\State $x' \leftarrow \emptyset$ 
\For{$j \in [3N]$} 
	\If{$i\le \abs{x}$}
		\State $x' \leftarrow x' \odot x[i]$ 
		\Comment the ``$\odot$'' denotes concatenation
		\State $i \leftarrow i+\pi_j(x[i])$
	\Else
		\State $x' \leftarrow x' \odot \perp$
		\Comment ``$\perp$'' can be an arbitrary character outside $\Sigma$
	\EndIf
\EndFor 
\end{algorithmic}
\caption{CGK-Embedding($s$, $R$) \cite{CGK16}}
\label{alg:CGK}
\end{algorithm}


Let $N$ be the maximum length of all input strings in $\S$. The CGK-embedding maps a string $x \in \S$ to an output string $x' \in \Sigma^{3N}$ using a random bit string $R \in \{0,1\}^{3N\abs{\Sigma}}$.  We maintain a counter $i \in [1 .. \abs{x}]$ pointing to the input string $x$, initialized to be $1$.  The embedding proceeds by steps $j = 1, \ldots, 3N$. At the $j$-th step, we first copy $x[i]$ to $x'[j]$. Next, with probability $1/2$, we increase $i$ by $1$, and with the rest of the probability we keep $i$ to be the same.  At the point when $i > \abs{x}$, if $j$ is still no more than $3N$, we simply pad an arbitrary character outside the dictionary $\Sigma$ (denoted by ``$\perp$'' in Algorithm~\ref{alg:CGK}) to make the length of $x'$ to be $3N$.  In practice this may introduce quite some overhead for short strings in the case that the string lengths vary significantly. We will discuss in Section~\ref{sec:speedup} how to efficiently deal with input strings of very different lengths.

Now consider two input stings $x$ and $y$.  We use $i_0$ and $i_1$ as two counters pointing to $x$ and $y$ respectively.   At the $j$-th step, we first copy $x[i_0]$ to $x'[j]$, and $y[i_1]$ to $y'[j]$, and then decide whether to increment $i_0$ and $i_1$ using the random bit string $R$.  There are four possibilities: (1) only $i_0$ increments; (2) only $i_1$ increments; (3) both $i_0$ and $i_1$ increment; and (4) neither $i_0$ nor $i_1$ increments.  Let $d = i_0 - i_1$ be the position shift of the two counters/pointers on the two strings.  Note that if $x[i_0] = y[i_1]$, then only the cases (3) and (4) can happen, so that $d$ will remain the same.  Otherwise if $x[i_0] \neq y[i_1]$, then each case can happen with probability $1/4$ -- whether $i_0$ or $i_1$ will increment depends on the two random hash values $\pi_j(x[i_0])$ and $\pi_j(y[i_1])$. Thus with probability $1/4$, $1/2$ and $1/4$, the value $d$ will increment, remain the same, or decrement, respectively.  Ignoring the case when the value $d$ remains the same, we can view $d$ as a (different) {\em simple random walk} on the integer line with $0$ as the origin.  

We now try to illustrate the high level idea of why CGK-embedding gives an $O(K)$ distortion.  Let $u = \abs{x}$ and $v = \abs{y}$.  Suppose that at some step $j$, letting $p = i_0(j)$ (the value of $i_0$ at step $j$) and $q = i_1(j)$, we have two tails $x[p .. u] = \alpha \circ \tau$ and $y = y[q .. v] = \tau$ where $\alpha, \tau$ are two substrings and $\abs{\alpha} = k \le K$. That is, we have $k$ consecutive deletions in the optimal alignment of the two tails.  Now if after a few random walk steps, at step $j' > j$, we have $p' = i_0(j') \ge p + k$, $q' = i_1(j') \ge q$ and $p' - q' = (p - q) + k$, then the two tails $x[p'..u]$ and $y[q'..v]$ can be perfectly aligned, and consequently the pairs of characters in the output strings $x', y'$ will always be the same; in other words, they will {\em not} contribute to the Hamming distance from step $j'$. 

Now observe that since the value of $d$ changes according to a simple random walk, by the theory of random walk, with probability $0.999$ it takes at most $O(k^2)$ steps for $d$ to go from $(p - q)$ to $(p' - q')$ where $\abs{(p - q) - (p' - q')} = k$.  Therefore the number of steps $j$ where $x'[j] \neq y'[j]$ is bounded by $O(k^2)$.  
%In general, every time $k$ edits occur when ``walking'' on $x$ and $y$, with probability $0.999$ it will take at most $O(k^2)$ steps to ``resolve'' the differences and go back to the right alignment track.  
This is roughly why $\text{Ham}(x', y')$ can be bounded by $O(K^2)$ if $\text{ED}(x, y) \le K$, and consequently the distortion can be bounded by $O(K)$.

%On the other hand, it is easy to show that $\text{Ham}(x', y') \ge \text{ED}(x, y) / 2$ holds with probability $1 - o(1)$. Combining the two we conclude that the embedding has an $O(K)$ distortion.

\paragraph{Small Distortion is Good for Edit Similarity Join}  We now explain why the distortion of the embedding matters. If we have an embedding $f$ such that for any pair of input strings $(x, y)$, the distortion of the embedding is upper bounded by $D$, then the set $\{(x, y)\ |\ \text{Ham}(f(x), f(y)) \le D \cdot K\}$ will include all pairs $(x, y)$ such that $\text{ED}(x, y) \le K$.  Therefore a small $D$ can help to reduce the number of false positives, and consequently reduce the verification time which typically dominates the total running time.

\medskip

\noindent{\bf Why CGK-embedding Does Better in Practice?}
Although the worst-case distortion of CGK-embedding can be large when $\text{ED}(x, y)$ is large, we have observed that its practical performance on the datasets that we have tested is much better.  While it is difficult to fully understand this phenomenon without a thorough investigation of the actual properties of the datasets, we can think of the following reasons.

First, if a set of $z$ edits fall into an interval of length $O(z)$, {\em and} the difference between the numbers of insertions and deletions among the $z$ edits is at most $O(\sqrt{z})$ (substitutions do not matter), then with probability $0.999$ after $O(z)$ walk steps the random walk will re-synchronize.  In other words, the distortion of the embedding is $O(1)$ with probability $0.999$ on this cluster of edits.  We have observed that in our protein/genome datasets (Section~\ref{sec:setup}) the edits are often clustered into small intervals; in each cluster most edits are substitutions, and consequently the difference between the numbers of insertions and deletions is small. 

Second, in the task of differentiating similar pairs of strings and dissimilar pairs of strings, as long as the distance gap between strings is preserved after the embedding, the distortion of CGK-embedding will not affect the performance by much. In particular, when the distortion of CGK-embedding is $\Theta(k)$ (which is very likely when edits are well separated), the embedding actually {\em amplifies} the distance gap between similar and dissimilar pairs, which makes the next LSH step easier.

%We believe that our datasets have (some of) these properties.  First, we have observed that in the protein/DNA datasets the edits are often clustered into small intervals; in each cluster most edits are substitutions, and consequently the difference between the numbers of insertions and deletions is small.  Second, in our datasets, in most cases, the distance gaps between similar pairs and dissimilar pairs are large so that the distortion of CGK is not a big issue.
 
To further improve the effectiveness of the CGK-embedding, we run the embedding multiple times and then take the one with the minimum Hamming distance. That is, we choose the run with the best distortion.  This is just a heuristic, and cannot improve the distortion by much in theory, but we have observed that for the real-world datasets that we have tested, repeating and then taking the minimum does help to reduce the distortion. In Figure~\ref{fig:CGK-distortion} we depicted the best distortions under different numbers of runs of the CGK-embedding on a real-world genome dataset.
%\ (see Section~\ref{sec:setup} for a detailed description of our datasets).  

\begin{figure}[t]
\centering
\includegraphics[height = 1.6in]{distortion.eps}
\caption{The CDF of the best distortions of 1000 random pairs strings from the \genoa\ dataset, under different numbers of CGK-embeddings (value $r$)}
\label{fig:CGK-distortion}
\end{figure}


\subsection{LSH for the Hamming Distance}
\label{sec:LSH}

Our second tool is the LSH for the Hamming distance, introduced in~\cite{IM98,GIM99} for solving nearest neighbor problems.  We first give the definition of LSH.  By $h \in_r \H$ we mean sampling a hash function $h$ randomly from a hash family $\H$. 

\begin{definition}(Locality Sensitive Hashing \cite{GIM99})
Let $U$ be the item universe, and $d(\cdot,\cdot)$ be a distance function. We say a hash family $\mathcal{H}$ is $(l, u, p_1,p_2)$-sensitive if for any $x, y \in U$
\begin{itemize}
\item if $d(x, y)\le l$, then $\Pr_{h\in_r \H}[h(x) = h(y)] \ge p_1$,
\item if $d(x, y)\ge u$, then $\Pr_{h\in_r \H}[h(x) = h(y)] \le p_2$.
\end{itemize}
\end{definition}

We will make use of the following vanilla version of LSH for the Hamming distance.
\begin{theorem}(Bit-sampling LSH for Hamming  \cite{GIM99})
For the Hamming distance over vectors in $\Sigma^N$, for any $d > 0, c > 1$, the family $$\mathcal{H}_N = \{v_i: v_i(b_1, \dots, b_N) = b_i\ |\ i \in [N]\}$$ is  $(d, cd, 1-{d}/{N}, 1-{cd}/{N})$-sensitive.
\end{theorem}

We can use the standard AND-OR amplification method\footnote{See, for example, \url{https://en.wikipedia.org/wiki/Locality-sensitive_hashing}.} to amplify the gap between $p_1$ and $p_2$.
We first concatenate $m$ ($m$ is a parameter) hash functions, and define 
\begin{eqnarray*}
f = h_1 \circ h_2 \circ \ldots \circ h_m \text{ where } \forall i \in [m], h_i \in_r \H,
\end{eqnarray*}
such that for $x \in U$, $f(x) = (h_1(x), h_2(x), \ldots, h_m(x))$ is a vector of $m$ bits.  Let $\F(m)$ be the set of all such hash functions $f$.
We then define (for a parameter $z$)
\begin{eqnarray*}
g = f_1 \vee f_2 \vee \ldots \vee f_z, \text{ where } \forall j \in [z], f_j \in_r \F(m),
\end{eqnarray*}
such that for $x, y \in U$ $g(x) = g(y)$ if and only if there is at least one $j \in [z]$ for which $f_j(x) = f_j(y)$.  Easy calculation shows that $g$ is
\begin{eqnarray*}
\left(d, cd, 1 - \left(1 - \left({d}/{N}\right)^m\right)^z, 1 - \left(1 - \left( {cd}/{N} \right)^m \right)^z \right) \text{-sensitive.}
\end{eqnarray*}
 By appropriately choosing the parameters $m$ and $z$,  we can amplify the gap between $p_1$ and $p_2$ so as to reduce the numbers of false positives/negatives.  
%Note that the total number of primitive hash functions $h \in \H$ we have used is $\beta = m \cdot z$, which will contribute to the running time.  

%We comment that this vanilla version of LSH is enough for our applications, and its good time performance fits our needs well.

\subsection{Exact Edit Distance Computation for Verification}
\label{sec:exact-ED}

We will use the classic algorithm by Ukkonen~\cite{Ukkonen85} for computing threshold edit distance as our verification algorithm.  In the high level, defining the diagonal $d$ of a matrix $D$ to be the set of all entries $D_{i, i+d}$,
the algorithm tries to fill {\em a subset of} the entries in the $2K+1$ diagonals $\{-K, \ldots, K\}$ in the $N \times N$ dynamic programming matrix, which are sufficient to give the final output.  The worst-case running time of this algorithm is $O(N K)$. But if one of the strings is a random string, then the algorithm only uses $O(N + K^2)$ time in expectation \cite{Myers86}.   In \cite{Myers86}, Myers also proposed another algorithm using suffix-tree whose worse-case running time is $O(N+K^2)$. However, we found that suffix-tree is computational expensive in practice and has no advantage over a ``brute force'' table filing \cite{Ukkonen85}. 

We also note that Belazzougui and Zhang~\cite{BZ16} (and independently, Chakraborty et al.~\cite{CGK16b}) showed that the $O(N + K^2)$ running time is also achievable in the simultaneous streaming model where we can only scan each string once in the coordinated fashion.  However, the algorithms in \cite{BZ16,CGK16b} still needs to use suffix-tree.  Chakraborty et al.~\cite{CGK16b} also proposed an algorithm with $N + O(K^3)$ running time in the simultaneous streaming model {\em without} using suffix-tree, but this bound would be large when the distance threshold $K$ is large, say, $20\%$ of the string length $N$.

In an earlier version of this paper~\cite{ZZ17} we used the algorithm in~\cite{LDW11} for computing edit distance in the verification step.  We later found that it is more efficient to use Ukkonen's algorithm.





\end{document}
