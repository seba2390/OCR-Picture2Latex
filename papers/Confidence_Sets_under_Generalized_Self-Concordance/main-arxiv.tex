\documentclass{article}
\usepackage[utf8]{inputenc} % allow utf-8 input
\usepackage[T1]{fontenc}    % use 8-bit T1 fonts
\usepackage{fullpage}
\usepackage[numbers,compress]{natbib}


% NeurIPS packages
% \usepackage{hyperref}       % hyperlinks
\usepackage{url}            % simple URL typesetting
\usepackage{booktabs}       % professional-quality tables
\usepackage{amsfonts}       % blackboard math symbols
\usepackage{microtype}      % microtypography
\usepackage{wrapfig}
\usepackage{caption}
\usepackage{nicefrac}       % compact symbols for 1/2, etc.

%%% Custom packages

\usepackage{times}
\usepackage{algorithm}
\usepackage{adjustbox}
\usepackage[noend]{algpseudocode}
\usepackage{amsmath}
\usepackage{amssymb}
\usepackage{amsthm}
\usepackage{bm}
% \usepackage{bbm}
% \usepackage{booktabs}
\usepackage{graphicx}
\usepackage{latexsym}
\usepackage{mathtools}
\usepackage{multirow}
\usepackage{paralist}
% \usepackage{titletoc} % for appendix toc
\usepackage{minitoc} % for appendix toc
\renewcommand \thepart{} % 1/2: minitoc Make the "Part I" text invisible
\renewcommand \partname{} % 1/2: minitoc Make the "Part I" text invisible
% \usepackage{xcolor}
\usepackage{xspace}
\usepackage{enumitem}
\usepackage{dsfont}
% \usepackage{tcolorbox} % Load after xcolor 

\usepackage[colorlinks,citecolor=bluegray,linkcolor=darkbrown,urlcolor=blue,breaklinks]{hyperref}
\usepackage{cleveref}


% Theorems
\newtheorem{theorem}{Theorem}
\newtheorem{lemma}[theorem]{Lemma}
\newtheorem{proposition}[theorem]{Proposition}
\newtheorem{corollary}[theorem]{Corollary}
\newtheorem{remark}{Remark}
\theoremstyle{definition}
\newtheorem{definition}{Definition}
\newtheorem{assumption}{Assumption}
\newtheorem{example}{Example}
\newtheorem{note}{Note}

\newtheorem{innercustomasmp}{Assumption}
\newenvironment{customasmp}[1]
  {\renewcommand\theinnercustomasmp{#1}\innercustomasmp}
  {\endinnercustomasmp}
  
\newtheorem{innercustomthm}{Theorem}
\newenvironment{customthm}[1]
  {\renewcommand\theinnercustomthm{#1}\innercustomthm}
  {\endinnercustomthm}
  
\newtheorem{innercustomprop}{Proposition}
\newenvironment{customprop}[1]
  {\renewcommand\theinnercustomprop{#1}\innercustomprop}
  {\endinnercustomprop}


% Paragraphs
% \newcommand{\myparagraph}[1]{\paragraph{#1.}\hspace{-0.8em}}  % For Arxiv
\newcommand{\myparagraph}[1]{\par\noindent\textbf{{#1}.}} % For conference formats
\newcommand{\myparagraphsmall}[1]{\par\noindent\textit{{#1}.}}


% Tikz related
\usepackage[dvipsnames]{xcolor} % color names 
\usepackage{tikz}
\usepackage{pgfplots}
\usetikzlibrary{shapes}
\usetikzlibrary{positioning}
\usetikzlibrary{plotmarks}
\usetikzlibrary{patterns}
\usetikzlibrary{intersections,shapes.arrows}
\usetikzlibrary{pgfplots.fillbetween}
\definecolor{darkpink}{rgb}{0.91, 0.33, 0.5}

% Nice colors
\definecolor{puorange}{rgb}{0.80,0.20,0}
\definecolor{bluegray}{rgb}{0.04,0,0.7}
\definecolor{greengray}{rgb}{0.05,0.50,0.15}
\definecolor{darkbrown}{rgb}{0.40,0.2,0.05}
\definecolor{darkcyan}{rgb}{0,0.4,1}
\definecolor{black}{rgb}{0,0,0}
\definecolor{grey}{rgb}{0.93,0.93,0.93}
\definecolor{royalazure}{rgb}{0.0, 0.22, 0.66}

\newcommand{\tabemph}[1]{\cellcolor{grey!30}\textcolor{black!50!royalazure}{#1}}%

\crefname{section}{Sec.}{Sections}
% \Crefname{section}{Section}{Sections}
\crefname{appendix}{Appx.}{Appxs}

\crefname{theorem}{Thm.}{Thms.}
\crefname{innercustomthm}{Thm.}{Thms.}
\crefname{lemma}{Lem.}{Lems.}
\crefname{corollary}{Cor.}{Cors.}
\crefname{proposition}{Prop.}{Props.}
\crefname{innercustomprop}{Prop.}{Props.}
\crefname{assumption}{Asm.}{Asms.}
\crefname{innercustomasmp}{Asmp.}{Asmps.}
\Crefname{example}{Ex.}{Exs.}

\crefname{algorithm}{Alg.}{Algs.}
\Crefname{algorithm}{Algorithm}{Algorithms}
\crefname{figure}{Fig.}{Figs.}
\crefname{table}{Tab.}{Tabs.}
 %% all packages
\newcommand{\score}{\ell}
\newcommand{\grad}{S}
\newcommand{\risk}{L}
\newcommand{\rao}{T_{\text{Rao}}}
\newcommand{\lr}{T_{\text{LR}}}
\newcommand{\wald}{T_{\text{Wald}}}
\newcommand{\bbar}[1]{\bar{\bar{#1}}}  %% useful macros
% Sets
\newcommand{\reals}{{\mathbb R}}
\newcommand{\spacex}{(\mathsf{X}, \mathcal{X})}
\newcommand{\spacey}{(\mathsf{Y}, \mathcal{Y})}
\newcommand{\spacez}{(\mathsf{Z}, \mathcal{Z})}
\newcommand{\dom}{\operatorname{dom}}

% lin alg stuff
\newcommand{\abs}[1]{\left| #1 \right|}
\newcommand{\norm}[1]{\left\lVert #1 \right\rVert}
\newcommand{\anorm}[1]{\left| #1 \right|}
\newcommand{\ones}{\operatorname{\mathbf 1}}
\newcommand{\id}{\operatorname{I}}
\newcommand{\Null}{\operatorname{Null}}
\newcommand{\Ker}{\operatorname{Ker}}
\newcommand{\Rank}{\operatorname{\bf Rank}}
\newcommand{\Tr}{\operatorname{\bf Tr}}
\newcommand{\diag}{\operatorname{\bf diag}}
\newcommand{\Vect}{\operatorname{Vec}}
\newcommand{\Span}{\operatorname{Span}}
\newcommand{\Proj}{\operatorname{Proj}}
\newcommand{\ip}[1]{{\langle #1 \rangle}}
\newcommand{\lone}{\mathbf{L}^1}
\newcommand{\ltwo}{\mathbf{L}^2}
\newcommand{\linf}{\mathbf{L}^\infty}

% probability stuff
\newcommand{\Expect}{\operatorname{\mathbb E}}
\newcommand{\Var}{\operatorname{\mathbb{V}ar}}
\newcommand{\Std}{\operatorname{\mathbb{S}td}}
\newcommand{\Cov}{\operatorname{\mathbb{C}ov}}
\newcommand{\Prob}{\operatorname{\mathbb P}}
\newcommand{\Ent}{\operatorname{Ent}}
\newcommand{\Supp}[1]{\text{Supp}(#1)}
\newcommand{\wass}{\mathsf{W}}
\newcommand{\pspace}{\mathcal{M}_1}
\newcommand \D {\mathrm{d}}
\newcommand{\kl}{\operatorname{KL}}

% hypothesis testing
\newcommand{\hnull}{\mathbf{H}_0}
\newcommand{\halt}{\mathbf{H}_1}

% convexity & optimization stuff
\DeclareMathOperator*{\argmax}{arg\,max}
\DeclareMathOperator*{\argmin}{arg\,min}
\newcommand{\prox}{\operatorname{prox}}
\newcommand{\bigO}{\mathcal{O}}

% kernel and Hilbert space
\newcommand{\rkhs}{{\mathcal{H}}}
\newcommand{\fmap}{\phi}
\newcommand{\hnorm}[1]{\left\lVert #1 \right\rVert_{\mathcal{H}}}
\newcommand{\hip}[1]{{\langle #1 \rangle_{\mathcal{H}}}}
\newcommand{\kmap}[1]{{k(#1, \cdot)}}

% special font letters
\newcommand{\calA}{\mathcal{A}}
\newcommand{\calB}{\mathcal{B}}
\newcommand{\calC}{\mathcal{C}}
\newcommand{\calD}{\mathcal{D}}
\newcommand{\calE}{\mathcal{E}}
\newcommand{\calF}{\mathcal{F}}
\newcommand{\calG}{\mathcal{G}}
\newcommand{\calH}{\mathcal{H}}
\newcommand{\calI}{\mathcal{I}}
\newcommand{\calL}{\mathcal{L}}
\newcommand{\calM}{\mathcal{M}}
\newcommand{\calN}{\mathcal{N}}
\newcommand{\calP}{\mathcal{P}}
\newcommand{\calS}{\mathcal{S}}
\newcommand{\calV}{\mathcal{V}}
\newcommand{\calW}{\mathcal{W}}
\newcommand{\calX}{\mathcal{X}}
\newcommand{\calY}{\mathcal{Y}}
\newcommand{\calZ}{\mathcal{Z}}

\newcommand{\sfX}{\mathsf{X}}
\newcommand{\sfY}{\mathsf{Y}}
\newcommand{\sfZ}{\mathsf{Z}}

\newcommand{\bbC}{\mathbb{C}}
\newcommand{\bbF}{\mathbb{F}}
\newcommand{\bbH}{\mathbb{H}}
\newcommand{\bbN}{{\mathbb N}}
\newcommand{\bbP}{\mathbb{P}}
\newcommand{\bbW}{\mathbb{W}}
\newcommand{\bbX}{\mathbb{X}}
\newcommand{\bbY}{\mathbb{Y}}
\newcommand{\bbZ}{\mathbb{Z}}

\newcommand{\ind}{\mathds{1}}

% Latin abbreviations etc.
\newcommand{\ie}{{\em i.e.,~}}
\newcommand{\wlg}{{\em w.l.o.g.,~}}
\newcommand{\eg}{{\em e.g.,~}}
\newcommand{\etal}{{\em et al.~}}
\newcommand{\wrt}{{\em w.r.t.~}}
\newcommand{\st}{{s.t.~}}
\newcommand{\iid}{{\em i.i.d.~}}


% highlight colors
\newcommand{\blue}[1]{\textcolor{blue}{#1}}
\newcommand{\bluegray}[1]{\textcolor{bluegray}{#1}}
\newcommand{\red}[1]{\textcolor{red}{#1}}

% equations
\newcommand{\txtover}[2]{\overset{\mbox{\scriptsize #1}}{#2}}
\newcommand\numberthis{\addtocounter{equation}{1}\tag{\theequation}}

\title{Confidence Sets under Generalized Self-Concordance}

\author{Lang Liu \qquad Zaid Harchaoui \\
Department of Statistics, University of Washington}
\date{}

\begin{document}

\maketitle
\doparttoc % Tell to minitoc to generate a toc for the parts
\faketableofcontents % Run a fake tableofcontents command for the partocs

\begin{abstract}
    This paper revisits a fundamental problem in statistical inference from a non-asymptotic theoretical viewpoint---the construction of confidence sets. We establish a finite-sample bound for the estimator, characterizing its asymptotic behavior in a non-asymptotic fashion. An important feature of our bound is that its dimension dependency is captured by the effective dimension---the trace of the limiting sandwich covariance---which can be much smaller than the parameter dimension in some regimes. We then illustrate how the bound can be used to obtain a confidence set whose shape is adapted to the optimization landscape induced by the loss function. Unlike previous works that rely heavily on the strong convexity of the loss function, we only assume the Hessian is lower bounded at optimum and allow it to gradually becomes degenerate. This property is formalized by the notion of generalized self-concordance which originated from convex optimization. Moreover, we demonstrate how the effective dimension can be estimated from data and characterize its estimation accuracy. We apply our results to maximum likelihood estimation with generalized linear models, score matching with exponential families, and hypothesis testing with Rao's score test.
\end{abstract}

\section{INTRODUCTION}
\label{sec:intro}
Reinforcement learning has achieved great success in areas such as Game-playing \citep{silver2018general,vinyals2019grandmaster}, robotics \cite{kober2013reinforcement}, large language models \citep{ouyang2022training}, etc.
However, due to safety concerns or physical limitations, in some real-world reinforcement learning problems, we must consider additional constraints that may influence the optimal policy and the learning process \citep{garcia2015comprehensive}.
% For example, a robotic arm must not take actions that may cause harm to itself or the environments.
A standard framework to handle such cases is the constrained Markov Decision Process (CMDP) \citep{altman1999constrained}.
Within the CMDP framework, the agent has to maximize
the expected cumulative reward while
obeying a finite number of constraints, which are usually in the form of expected cumulative cost criteria.

However, we are sometimes concerned with the problem with a continuum of constraints.
For example,
the constraints we meet might be time-evolving or subject to uncertain parameters, which
cannot be formulated as an ordinary CMDP
(see Examples \ref{Example_Time_Evolving} and  \ref{Example_Uncertain}).
In this paper we would study a generalized CMDP  
to address the above problem.  Because the constraints are not only infinite-number but also lie
in a continuous set,
the generalization is not trivial. Fortunately, we find that we can borrow the idea behind semi-infinite programming (SIP) \citep{remez1934determination, hettich1993semi} to deal with the semi-infinite constraints.
Accordingly, we propose \emph{semi-infinitely constrained Markov decision processes} (SICMDPs)
as a novel complement to the ordinary CMDP framework.
%More specifically,  an SICMDP model %, we consider 
%contains a continuum of constraints whereas an ordinary CMDP contains a finite number of constraints. 

%This generalization is natural but not trivial. However, we can brows the idea  
%The idea is quite natural and can be backtracked
%to the practice of extending linear programming to linear semi-infinite programming (LSIP) %\cite{remez1934determination, GobernaLSIO1998}.
%In addition, 
%As a complementary approach to the ordinary CMDP framework, 
%SICMDP can be used to model these problems  which cannot be described by a finite number of constraints
%that are not covered by .
%For example,
%the restrictions we consider can be time-evolving or subject to uncertain parameters
%, thus
%cannot be described by a finite number of constraints but a continuum of constraints 
%(see Examples \ref{Example_Time_Evolving} and  \ref{Example_Uncertain}).

We also present two reinforcement learning algorithms to solve SICMDPs called SI-CRL and SI-CPO, respectively.
SI-CRL is a model-based reinforcement learning algorithm designed for tabular cases, and SI-CPO is a policy optimization algorithm for non-tabular cases.
% and analyze its performance both theoretically and empirically.
The main challenge is that we need to deal with a continuum of constraints, thus reinforcement learning algorithms for ordinary CMDPs do not work anymore.
In SI-CRL, we tackle this difficulty by first transforming the reinforcement learning problem to an equivalent LSIP problem, which can then be solved using methods in the LSIP literature like the dual exchange methods \citep{Hu1990,reemtsen1998numerical}.
In SI-CPO, we resort to the idea of cooperative stochastic approximation developed in \cite{lan2020algorithms, wei2020comirror}.
As far as we know, we are the first to introduce tools from semi-infinitely programming (SIP) into the reinforcement learning community for solving constrained reinforcement learning problems.

% To the best of our knowledge, we are the first to apply tools from semi-infinitely programming (SIP) to solve reinforcement learning problems.
Furthermore, we give theoretical analysis for both SI-CRL and SI-CPO.
We decompose the error of SI-CRL into two parts: the statistical error from approximating the true SICMDP with an offline dataset and the optimization error due to the fact that the solution of the LSIP problem obtained by the dual exchange method is inexact.
On the optimization side, we show that the iteration complexity of SI-CRL is $O\left(\left\{\mathrm{diam}(Y)L\sqrt{|\gS|^2|\gA|m}/\left[(1-\gamma)\epsilon\right]\right\}^m\right)$.
On the statistical side, we show that the sample complexity of SI-CRL is $\widetilde O\left(\frac{|S|^2|A|^2}{\epsilon^2(1-\gamma)^3}\right)$ if the offline dataset is generated by a generative model, and $\widetilde O\left(\frac{|S||A|}{\nu_{\min} \epsilon^2(1-\gamma)^3}\right)$ if the dataset is generated by a probability measure $\nu$ as considered in \cite{chen2019information}.
Here $\widetilde O$ means that all logarithm terms are discarded.
For SI-CPO, things become a little more complicated because other than the statistical error and the optimization error, we also need to consider the function approximation error, which comes from imperfect policy parametrizations.
It is shown if the function approximation error can be controlled to $O(\epsilon)$ order, the iteration complexity of SI-CPO is $\widetilde{O}\left(\frac{1}{\epsilon^2(1-\gamma)^6}\right)$ and the sample complexity of SI-CPO is $\widetilde{O}(\frac{1}{\epsilon^4(1-\gamma)^{10}})$.
Here our iteration complexity bound is equivalent to a typical $\widetilde O(1/\sqrt{T})$ global convergence rate.

We perform a set of numerical experiments to illustrate the SICMDP model and validate our proposed algorithms.
Specifically, we examine two numerical examples, namely the discharge of sewage and ship route planning.
Through the discharge of sewage example, we show the advantage of the SICMDP framework over the CMDP baseline obtained by naive discretization in modeling realistic sequential decision-making problems.
Moreover, we demonstrate the effectiveness of the SI-CRL and SI-CPO algorithms in such tabular environments. 
In the ship route planning example, we illustrate the benefits of the SICMDP framework and the ability of the SI-CPO algorithm to address complex continuous control tasks involving continuous state spaces with modern deep reinforcement learning techniques.

% In summary, our contributions are listed as follows.
% First, we present the SICMDP model, which can be viewed as a generalization of the ordinary CMDP model.
% Second, we propose an algorithm to perform reinforcement learning for SICMDPs, which is called SI-CRL, and we believe that we are the first to apply tools from SIP
% to solve reinforcement learning problems.
% Third, we give a theoretical analysis of SI-CRL and identify both its sample complexity and iteration complexity.
% In addition, we perform numerical experiments to illustrate the SICMDP model and validate the SI-CRL algorithm.
% \{This paragraph can be removed!!! \}







\section{PROBLEM FORMULATION}
\label{sec:problem}
%!TEX root = main.tex
\section{Problem Definition and Notations}
\label{sec:problem}







% In this section, we will first describe key concepts and notations used in this paper, and formally define our problem. Then we will use a case study to make our idea of story tree more concrete.

% \subsection{Problem Definition and Notations}
% \label{subsec:problem-define}

We first present some definitions of key concepts in the top-down hierarchy: \textit{topic} $\rightarrow$ \textit{story} $\rightarrow$ \textit{event} to be used in this paper.

\begin{definition}
  \textit{Event}: an event $\mathcal{E}$ is a set of one or several documents that contain highly similar information.
\end{definition}

\begin{definition}
  \textit{Story}: a story $\mathcal{S}$ is a tree of events that revolve around a group of specific persons and happen at certain places during specific times. A directed edge from event $\mathcal{E}_1$ to $\mathcal{E}_2$ indicates a temporal evolution or a logical connection from $\mathcal{E}_1$ to $\mathcal{E}_2$.
\end{definition}

\begin{definition}
  \textit{Topic}: a topic consists of a set of stories that are highly correlated or similar to each other.
  \vspace{-1mm}
\end{definition}


Each topic may contain multiple story trees, and each story tree consists of multiple logically connected events.
In our work, events (instead of news documents) are the smallest atomic units. Each event is also assumed to belong to a single story and contains partial information about that story.
For instance, considering the topic \textit{American presidential election}, \textit{2016 U.S. presidential election} is a story within this topic, and  \textit{Trump and Hilary's first television debate} is an event within this story.


We now introduce some notations and describe our problem formally. Given a news document stream $D = \{ \mathcal{D}_1, \mathcal{D}_2, \ldots, \mathcal{D}_t,\ldots \}$, where $\mathcal{D}_t$ is the set of news documents collected on time period $t$, our objective is to: a) cluster all news documents $D$ into a set of events $E = \{ \mathcal{E}_1, \ldots, \mathcal{E}_{|E|} \}$, and b) connect the extracted events to form a set of stories $S = \{ \mathcal{S}_1, ..., \mathcal{S}_{|S|} \}$. Each story $\mathcal{S} = (E, L)$ contains a set of events $E$ and a set of links $L$, where $L_{i,j} := <\mathcal{E}_i, \mathcal{E}_j>$ denotes a directed link from event $\mathcal{E}_i$ to $\mathcal{E}_j$, which indicates a temporal evolution or logical connection relationship.

%We now illustrate our problem with an example. (A example Fig) Fig... shows ...
Furthermore, we require the events and story trees to be extracted in an online or incremental manner. That is, we extract events from each $\mathcal D_t$ individually when the news corpus $\mathcal D_t$ arrives in time period $t$, and \emph{merge} the discovered events into the existing story trees that were found at time $t-1$. This is a unique strength of our scheme as compared to prior work, since we do not need to repeatedly process older documents and can deliver  a set of evolving yet logically consistent story trees to users.  

% \subsection{Case Study}
% \label{subsec:case-study}

\begin{figure}
\includegraphics[width=3.4in]{figure/StoryStructures}
\caption{Different structures to characterize a story.}
\vspace{-2mm}
\label{fig:storyStructures}
\vspace{-2mm}
\end{figure}

For example, Fig.~\ref{fig:CaseStudy} illustrates the story tree of ``2016 U.S. presidential election''. The story contains $20$ nodes, where each node indicates an event in 2016 U.S. election, and each link indicates a temporal evolution or a logical connection between two events. %For example, event $19$ says America votes to elect new president, and event $20$ says Donald Trump is elected president. 
The index number on each node represents the event sequence over the timeline. There are $6$ paths within this story tree, where the path $1 \rightarrow 20$ indicates the whole presidential election process, branch $3 \rightarrow 6$ is about Hilary's health conditions, branch $7 \rightarrow 13$ talks about television debates, $14 \rightarrow 18$ depicts the investigation into Hilary's ``mail door'', etc. As we can see, by modeling the evolutionary and logical structure of a story into a story tree, users can easily grasp the logic of news stories and learn the main information quickly. 


Let us represent each story by an empty root node $s$ from which the story is originated, and denote each event by an event node $e$. The events in a story can be organized in one of the following four structures shown in Fig. \ref{fig:storyStructures}: a) a flat structure that does not include dependencies between events; b) a timeline structure that organizes events by their timestamps; c) a graph structure that checks the connection between all pairs of events and maintains a subset of most strong connections; d) a tree structure, which represents a story's evolving structure by a tree.  

Compared with a tree structure, sorting events by timestamps omits the logical connection between events, while using directed acyclic graphs to model event dependencies without considering the evolving consistency of the whole story can leads to unnecessary connections between events.
Through extensive user experience studies in Sec.~\ref{sec:eval}, we show that tree structures are the most effective way to represent breaking news stories as compared to other structures, including the more complex graph structures. 



\section{MAIN RESULTS}
\label{sec:main_results}
\begin{figure}
    \centering
    \includegraphics[width=0.6\textwidth]{graphs/convex-concordance} %0.45
    \caption{Strong convexity v.s.~self-concordance. Black curve: population risk; colored dot: reference point; colored dashed curve: quadratic approximation at the corresponding reference point.}
    \label{fig:convex_concordance}
\end{figure}

\subsection{Preliminaries}
\label{sub:preliminary}

\myparagraph{Notation}
We denote by $\grad(\theta; z) := \nabla_\theta \score(\theta; z)$ the gradient of the loss at $z$ and $H(\theta; z) := \nabla_\theta^2 \score(\theta; z)$ the Hessian at $z$.
Their population versions are $\grad(\theta) := \Expect[\grad(\theta; Z)]$ and $H(\theta) := \Expect[H(\theta; Z)]$, respectively.
We assume standard regularity assumptions so that $\grad(\theta) = \nabla_\theta L(\theta)$ and $H(\theta) = \nabla_\theta^2 L(\theta)$.
We write $H_\star := H(\theta_\star)$.
Note that the two optimality conditions then read $\grad(\theta_\star) = 0$ and $H_\star \succ 0$.
It follows that $\lambda_\star := \lambda_{\min}(H_\star) > 0$ and $\lambda^\star := \lambda_{\max}(H_\star) > 0$.
Furthermore, we let $G(\theta; z) := S(\theta; z) S(\theta; z)^\top$ and $G(\theta) := \Expect[\grad(\theta; Z)\grad(\theta; Z)^\top]$ be the autocorrelation matrices of the gradient.
We write $G_\star := G(\theta_\star)$.
We define their empirical quantities as $L_n(\theta) := n^{-1} \sum_{i=1}^n \score(\theta; Z_i)$, $\grad_n(\theta) := n^{-1} \sum_{i=1}^n \grad(\theta; Z_i)$, $H_n(\theta) := n^{-1} \sum_{i=1}^n H(\theta; Z_i)$, and $G_n(\theta) := n^{-1} \sum_{i=1}^n G(\theta; Z_i)$.
The first step of our analysis is to localize the estimator to a \emph{Dikin ellipsoid} at $\theta_\star$ of radius $r$, i.e.,
\begin{align*}
    \Theta_r(\theta_\star) := \left\{\theta \in \Theta: \norm{\theta - \theta_\star}_{H_\star} < r \right\},
\end{align*}
where, given a positive semi-definite matrix $J$, we let $\norm{x}_J := \norm{J^{1/2} x}_2 = \sqrt{x^\top J x}$.

\myparagraph{Effective dimension}
A quantity that plays a central role in our analysis is the \emph{effective dimension}.
\begin{definition}
\label{def:effective_dim}
    We define the effective dimension to be
    \begin{align}
        d_\star := \Tr( H_\star^{-1/2} G_\star H_\star^{-1/2} ).
    \end{align}
\end{definition}
The effective dimension appears recently in non-asymptotic analyses of (penalized) M-estimation; see, e.g., \citep{spokoiny2017penalized,ostrovskii2021finite}.
It better characterizes the complexity of the parameter space $\Theta$ than the parameter dimension $d$.
When the model is well-specified, it can be shown that $H_\star = G_\star$ and thus $d_\star = d$.
When the model is misspecified, it can be much smaller than $d$ depending on the spectra of $H_\star$ and $G_\star$.
Moreover, it is closely connected to classical asymptotic theory of M-estimation under model misspecification---it is the trace of the limiting covariance matrix of $\sqrt{n}H_n(\theta_n)^{1/2}(\theta_n - \theta_\star)$;
see \Cref{sub:discussion} for a thorough discussion.

\myparagraph{Generalized self-concordance}
We will use the notion of \emph{self-concordance} from convex optimization in our analysis.
Self-concordance originated from the analysis of the interior-point and Newton-type convex optimization methods \citep{yurii1994interior}.
It was later modified by \citet{bach2010self}, which we call the \emph{pseudo self-concordance}, to derive finite-sample bounds for the generalization properties of the logistic regression.
Recently, \citet{sun2019generalized} proposed the \emph{generalized self-concordance} which unifies these two notions.
For a function $f: \reals^d \to \reals$, we define $D_x f(x)[u] := \frac{\D}{\D t} f(x + tu) |_{t = 0}$, $D_x^2 f(x)[u, v] := D_x (D_x f(x)[u])[v]$ for $x, u, v \in \reals^d$, and $D_x^3 f(x)[u, v, w]$ similarly.
\begin{definition}[Generalized self-concordance]
\label{def:general_self_concordance}
    Let $\calX \subset \reals^d$ be open and $f: \calX \rightarrow \reals$ be a closed convex function.
    For $R > 0$ and $\nu > 0$, we say $f$ is $(R, \nu)$-generalized self-concordant on $\calX$ if
    \begin{align*}
        \abs{D_x^3 f(x) [u, u, v]} \le R \norm{u}_{\nabla^2 f(x)}^2 \norm{v}_{\nabla^2 f(x)}^{\nu-2} \norm{v}_2^{3-\nu}
    \end{align*}
    with the convention $0/0 = 0$ for the case $\nu < 2$ and $\nu > 3$.
    Recall that $\norm{u}_{\nabla^2 f(x)}^2 := u^\top \nabla^2 f(x) u$.
\end{definition}

\myparagraph{Remark}
When $\nu = 2$ and $\nu = 3$, this definition recovers the pseudo self-concordance and the standard self-concordance, respectively.

In contrast to strong convexity which imposes a gross lower bound on the Hessian, generalized self-concordance specifies the rate at which the Hessian can vary, leading to a finer control on the Hessian.
Concretely, it allows us to bound the Hessian in a neighborhood of $\theta_\star$ with the Hessian at $\theta_\star$, which is key to controlling $H_n(\theta_n)$.
We illustrate the difference between them in \Cref{fig:convex_concordance}.
As we will see in \Cref{sub:main_results}, thanks to the generalized self-concordance, we are able to remove the direct dependency on $\lambda_\star$ in our confidence set.
To the best of our knowledge, this is the first work extending classical results for M-estimation to generalized self-concordant losses.

\myparagraph{Concentration of Hessian}
One key result towards deriving our bounds is the concentration of empirical Hessian, i.e., $(1 - c_n(\delta))H(\theta) \preceq H_n(\theta) \preceq (1 + c_n(\delta)) H(\theta)$ with probability at least $1 - \delta$.
When the loss function is of the form $\ell(\theta; z) := \ell(y, \theta^\top x)$ (e.g., GLMs), the empirical Hessian reads $H_n(\theta) = n^{-1} \sum_{i=1}^n \ell''(Y_i, \theta^\top X_i) X_i X_i^\top$ where $\ell''(y, \bar y) := \D^2 \ell(y, \bar y) / \D \bar y^2$, which is of the form of a sample covariance.
Assuming $X$ to be sub-Gaussian, \citet{ostrovskii2021finite} obtained a concentration bound for $H_n(\theta_\star)$ with $c_n(\delta) = O(\sqrt{(d + \log{(1/\delta)})/n})$ via the concentration bound for sample covariance \citep[Thm.~5.39]{vershynin2010introduction}.
For general loss functions, such a special structure cannot be exploited.
We overcame this challenge by the matrix Bernstein inequality \citep[Thm.~6.17]{wainwright2019high}, obtaining a sharper concentration bound with $c_n(\delta) := O(\sqrt{\log{(d/\delta)}/n})$.
Note that the matrix Bernstein inequality has been used to control the empirical Hessian of kernel ridge regression with random features \citep[Prop.~6]{rudi2017generalization} and later extended to regularized empirical risk minimization \citep[Lem.~30]{marteau2019beyond}.
However, their results require the regularization parameter to be strictly positive (otherwise the bounds are vacuous) and the sample Hessian to be bounded.
On the contrary, our technique allows for zero regularization and unbounded Hessian as long as the Hessian satisfies a matrix Bernstein condition.
Moreover, combining generalized self-concordance with matrix Bernstein, we are able to show the concentration of $H_n(\theta_n)$ around $H_\star$ for general losses, which is itself a novel result.

\subsection{Assumptions}
\label{sub:assumption}

Our key assumption is the generalized self-concordance of the loss function.
\begin{assumption}[Generalized self-concordance]
\label{asmp:self_concordance}
    For any $z \in \calZ$, the scoring rule $\score(\cdot; z)$ is $(R, \nu)$-generalized self-concordant for some $R > 0$ and $\nu \ge 2$.
    Moreover, $\risk(\cdot)$ is also $(R, \nu)$-generalized self-concordant.
\end{assumption}

\myparagraph{Remark}
If $\score(\cdot; z)$ is generalized self-concordant with $\nu = 2$, so is $\risk(\cdot)$.

Many loss functions in statistical machine learning satisfy this assumption.
We give in \Cref{sub:examples} examples from generalized linear models and score matching.


In order to control the empirical gradient $\grad_n(\theta)$, we assume that the normalized gradient at $\theta_\star$ is sub-Gaussian.
\begin{assumption}[Sub-Gaussian gradient]
\label{asmp:sub_gaussian}
    There exists a constant $K_1 > 0$ such that the normalized gradient at $\theta_\star$ is sub-Gaussian with parameter $K_1$, i.e., $\lVert G_\star^{-1/2} \grad(\theta_\star; Z) \rVert_{\psi_2} \le K_1$.
    Here $\norm{\cdot}_{\psi_2}$ is the sub-Gaussian norm whose definition is recalled in \Cref{sec:tools}.
\end{assumption}

When the loss function is of the form $\ell(\theta; z) = \ell(y, \theta^\top x)$, we have $S(\theta; Z) = \ell'(Y, \theta^\top X) X$.
As a result, \Cref{asmp:sub_gaussian} holds true if (i) $\ell'(Y, \theta_\star^\top X)$ is sub-Gaussian and $X$ is bounded or (ii) $\ell'(Y, \theta_\star^\top X)$ is bounded and $X$ is sub-Gaussian.
For least squares with $\ell(y, \theta^\top x) = \frac12 (y - \theta^\top x)^2$, the derivative $\ell'(Y, \theta_\star^\top X) = \theta_\star^\top X - Y$ is the negative residual.
\Cref{asmp:sub_gaussian} is guaranteed if the residual is sub-Gaussian and $X$ is bounded.
For logistic regression with $\ell(y, \theta^\top x) = -\log{\sigma(y\cdot \theta^\top x)}$ where $\sigma(u) = (1 + e^{-u})^{-1}$, the derivative $\ell'(Y, \theta_\star^\top X) = [\sigma(Y \cdot \theta_\star^\top X) - 1]Y \in [-1, 1]$ is bounded.
Thus, \Cref{asmp:sub_gaussian} is guaranteed if $X$ is sub-Gaussian.

In order to control the empirical Hessian, we assume that the Hessian of the loss function satisfies the matrix Bernstein condition in a neighborhood of $\theta_\star$.

\begin{assumption}[Matrix Bernstein of Hessian]
\label{asmp:bernstein}
    There exist constants $K_2, r > 0$ such that, for any $\theta \in \Theta_{r}(\theta_\star)$, the standardized Hessian
    \begin{align*}
        H(\theta)^{-1/2} H(\theta; Z) H(\theta)^{-1/2} - I_d
    \end{align*}
    satisfies a Bernstein condition (defined in \Cref{sec:tools}) with parameter $K_2$. Moreover,
    \begin{align*}
        \sigma_H^2 := \sup_{\theta \in \Theta_{r}(\theta_\star)} \norm{\Var\left( H(\theta)^{-\frac12}H(\theta; Z)H(\theta)^{-\frac12} \right)}_2 < \infty,
    \end{align*}
    where $\norm{\cdot}_2$ is the spectral norm and $\Var(J) := \Expect[JJ^\top] - \Expect[J] \Expect[J]^\top$.
    By convention, we let $\Theta_0(\theta_\star) = \{\theta_\star\}$.
\end{assumption}

\subsection{Main Results}
\label{sub:main_results}

We now give simplified versions of our main theorems.
We use $C_\nu$ to represent a constant depending only on $\nu$ that may change from line to line; and $C_{K_1, \nu}$ similarly.
We use $\lesssim$ and $\gtrsim$ to hide constants depending only on $K_1, K_2, \sigma_H, \nu$.
The precise versions can be found in \Cref{sec:proofs}.
Recall that $\lambda_\star := \lambda_{\min}(H_\star)$ and $\lambda^\star := \lambda_{\max}(H_\star)$.
\begin{theorem}\label{thm:risk_bound_generalized}
    Let $\nu \in [2, 3)$.
    Under \Cref{asmp:self_concordance,asmp:sub_gaussian,asmp:bernstein} with $r = 0$, it holds that,
    whenever
    \begin{align*}
        n \gtrsim \log{(2d/\delta)} + \lambda_\star^{-1} \left[ R^2 d_\star \log{(e/\delta)} \right]^{1/(3-\nu)},
    \end{align*}
    the empirical risk minimizer $\theta_n$ uniquely exists and satisfies, with probability at least $1 - \delta$,
    \begin{align}\label{eq:conf_bound}
        \norm{\theta_n - \theta_\star}^2_{H_\star} \lesssim \log{(e/\delta)} \frac{d_\star}{n}.
    \end{align}
\end{theorem}

With a local matrix Bernstein condition, we can replace $H_\star$ by $H_n(\theta_n)$ in \eqref{eq:conf_bound} and obtain a finite-sample version of the Wald confidence set.
\begin{theorem}\label{thm:conf_set}
    Let $\nu \in [2, 3)$.
    Suppose the same assumptions in \Cref{thm:risk_bound_generalized} hold true.
    Furthermore, suppose that \Cref{asmp:bernstein} holds with $r = C_\nu \lambda_\star^{(3-\nu)/2} / R$.
    Let $\calC_{\text{Wald}, n}(\delta)$ be
    \begin{align}\label{eq:my_conf_set}
        \left\{\theta \in \Theta: \norm{\theta - \theta_n}_{H_n(\theta_n)}^2 \le C_{K_1,\nu} \frac{d_\star}{n} \log{\frac{e}{\delta}} \right\}.
    \end{align}
    Then we have $\Prob(\theta_\star \in \calC_{\text{Wald}, n}(\delta)) \ge 1 - \delta$ whenever
    \begin{align}\label{eq:n_large_enough}
        n \gtrsim \log{\frac{2d}{\delta}} + d\log{n} + \lambda_\star^{-1}\left[ R^2 d_\star \log{\frac{e}{\delta}} \right]^{\frac1{3-\nu}}.
    \end{align}
\end{theorem}

\myparagraph{Remark}
In the precise versions of \Cref{thm:risk_bound_generalized,thm:conf_set}, the term $d_\star \log{(e/\delta)}$ in the bounds \eqref{eq:conf_bound} and \eqref{eq:my_conf_set} should be replaced by $d_\star + \log{(e/\delta)} \lVert G_\star^{1/2} H_\star^{-1} G_\star^{1/2} \rVert_2$, which almost match the misspecified Cram\'er-Rao lower bound \citep[e.g.,][Thm.~1]{fortunati2016misspecified} up to a constant factor.

\Cref{thm:conf_set} suggests that the tail probability of $\norm{\theta_n - \theta_\star}_{H_n(\theta_n)}^2$ is governed by a $\chi^2$ distribution with $d_\star$ degrees of freedom, which coincides with the asymptotic result.
In fact, according to \citet{huber1967under}, under suitable regularity assumptions, it holds that $\sqrt{n} H_n(\theta_n)^{1/2}(\theta_n - \theta_\star) \rightarrow_d W \sim \mathcal{N}(0, H_\star^{-1/2} G_\star H_\star^{-1/2})$ which implies that
\begin{align*}
    n(\theta_n - \theta_\star)^\top H_n(\theta_n) (\theta_n - \theta_\star) \rightarrow_d W^\top W.
\end{align*}
This induces an asymptotic confidence set with a similar form of \eqref{eq:my_conf_set} and radius $O(\Expect[W^\top W] / n) = O(d_\star / n)$.
Our result characterizes the \emph{critical sample size} enough to enter the asymptotic regime.

From \Cref{thm:conf_set} we can also derive a finite-sample version of the LR confidence set.
\begin{corollary}\label{cor:lr_conf_set}
    Let $\nu \in [2, 3)$.
    Suppose the same assumptions in \Cref{thm:conf_set} hold true.
    Let $\calC_{\text{LR}, n}(\delta)$ be
    \begin{align}\label{eq:lr_conf_set}
        \left\{\theta \in \Theta: 2[L_n(\theta) - L_n(\theta_n)] \le C_{K_1,\nu} \frac{d_\star}{n} \log{\frac{e}{\delta}} \right\}.
    \end{align}
    Then we have $\Prob(\theta_\star \in \calC_{\text{LR}, n}(\delta)) \ge 1 - \delta$ whenever
    \begin{align*}
        n \gtrsim \log{\frac{2d}{\delta}} + d\log{n} + \lambda_\star^{-1}\left[ R^2 d_\star \log{\frac{e}{\delta}} \right]^{\frac1{3-\nu}}.
    \end{align*}
\end{corollary}

We give the proof sketches of \Cref{thm:risk_bound_generalized}, \Cref{thm:conf_set}, and \Cref{cor:lr_conf_set} here and defer their full proofs to \Cref{sec:proofs}.
We discuss in~\Cref{sub:discussion} 
how our proof techniques and theoretical results complement and improve on previous works.

We start by showing the existence and uniqueness of $\theta_n$.
The next result shows that $\theta_n$ exists and is unique whenever the quadratic form $\grad_n(\theta_\star)^\top H_n^{-1}(\theta_\star) \grad_n(\theta_\star)$ is small.
Note that this quantity is also known as Rao's score statistic for goodness-of-fit testing.
This result also localizes $\theta_n$ to a neighborhood of the target parameter $\theta_\star$.
\begin{proposition}\label{prop:localization}
    Under \Cref{asmp:self_concordance},
    if $\norm{\grad_n(\theta_\star)}_{H_n^{-1}(\theta_\star)} \le C_{\nu} [\lambda_{\min}(H_n(\theta_\star))]^{(3-\nu)/2} / (R n^{\nu/2-1})$,
    then the estimator $\theta_n$ uniquely exists and satisfies
    \begin{align*}
        \norm{\theta_n - \theta_\star}_{H_n(\theta_\star)} \le 4 \norm{\grad_n(\theta_\star)}_{H_n^{-1}(\theta_\star)}.
    \end{align*}
\end{proposition}

The main tool used in the proof of \Cref{prop:localization} is a strong convexity type result for generalized self-concordant functions recalled in \Cref{sec:tools}.
In order to apply \Cref{prop:localization}, we need to control $\norm{\grad_n(\theta_\star)}_{H_n^{-1}(\theta_\star)}$.
This result is summarized in the following proposition.

\begin{proposition}\label{prop:score}
    Under \Cref{asmp:sub_gaussian,asmp:bernstein} with $r = 0$, it holds that, with probability at least $1 - \delta$,
    \begin{align*}
        \norm{S_n(\theta_\star)}_{H_n^{-1}(\theta_\star)}^2 \lesssim \frac{d_\star}n \log{(e/\delta)}
    \end{align*}
    whenever $n \gtrsim \log{(2d/\delta)}$.
\end{proposition}

The proof of \Cref{prop:score} consists of two steps: (a) lower bound $H_n(\theta_\star)$ by $H_\star$ up to a constant using the Bernstein inequality and (b) upper bound $\norm{\grad_n(\theta_\star)}_{H^{-1}(\theta_\star)}$ using a concentration inequality for isotropic random vectors, where the tools are recalled in \Cref{sec:tools}.
Combining them implies that $\norm{\grad_n(\theta_\star)}_{H^{-1}(\theta_\star)}$ can be arbitrarily small and thus satisfies the requirement in \Cref{prop:localization} for sufficiently large $n$.
This not only proves the existence and uniqueness of the empirical risk minimizer $\theta_n$ but also provides an upper bound for $\norm{\theta_n - \theta_\star}_{H_n(\theta_\star)}$ through $\norm{\grad_n(\theta_\star)}_{H_n^{-1}(\theta_\star)}$.

In order to prove \Cref{thm:conf_set}, it remains to upper bound $H_n(\theta_n)$ by $H_\star$ up to a constant factor.
This can be achieved by the following result.
\begin{proposition}\label{prop:emp_hess_est}
    Under \Cref{asmp:self_concordance,asmp:bernstein} with $r = C_\nu \lambda_\star^{(\nu-3)/2} / R$, it holds that, with probability at least $1 - \delta$,
    \begin{align*}
        \frac1{2C_\nu} H_\star \preceq H_n(\theta) \preceq \frac32 C_\nu H_\star, \;\mbox{for all } \theta \in \Theta_{r}(\theta_\star),
    \end{align*}
    whenever $n \gtrsim \left\{ \log{(2d/\delta)} + d (\nu/2-1) \log{n}\right\}$.
\end{proposition}

Finally, \Cref{cor:lr_conf_set} follows from \Cref{thm:conf_set} and the Taylor expansion: there exists $\bar \theta_n \in \mbox{Conv}\{\theta_n, \theta_\star\}$ such that
\begin{align*}
    2[L_n(\theta_\star) - L_n(\theta_n)] = \norm{\theta_n - \theta_\star}_{H_n(\bar \theta_n)},
\end{align*}
where we have used $\nabla L_n(\theta_n) = 0$.

\subsection{Approximating the effective dimension}

One downside of \Cref{thm:conf_set,cor:lr_conf_set} is that $d_\star$ depends on the unknown data distribution.
Alternatively, we use the following empirical counterpart
\begin{align*}
    d_n := \Tr\left(H_n(\theta_n)^{-1/2} G_n(\theta_n) H_n(\theta_n)^{-1/2} \right).
\end{align*}
The next result implies that we do not lose much if we replace $d_\star$ by $d_n$.
This result is novel and of independent interest since one also needs to estimate $d_\star$ in order to construct asymptotic confidence sets under model misspecification.

\begin{customasmp}{2'}\label{asmp:subG_local}
    There exist constants $r, K_1 > 0$ such that, for any $\theta \in \Theta_r(\theta_\star)$, we have $\norm{G(\theta)^{-1/2} S(\theta; Z)}_{\psi_2} \le K_1$.
\end{customasmp}

\begin{assumption}\label{asmp:lip}
    There exists $r > 0$ such that $M := \Expect[M(Z)] < \infty$, where $M(z)$ is defined as
    \begin{align*}
        \sup_{\theta_1 \neq \theta_2 \in \Theta_r(\theta_\star)} \frac{\norm{G_\star^{-1/2} [G(\theta_1; z) - G(\theta_2; z)] G_\star^{-1/2}}_2}{\norm{\theta_1 - \theta_2}_{H_\star}}.
    \end{align*}
\end{assumption}

\myparagraph{Remark}
\Cref{asmp:lip} is a Lipschitz-type condition for $G(\theta; z)$. This assumption was previously used by \citep[Assumption 3]{mei2018landscape} to analyze non-convex risk landscapes. 

\begin{proposition}\label{prop:d_n}
    Let $\nu \in [2, 3)$.
    Under Asms.~\ref{asmp:self_concordance}, \ref{asmp:subG_local}, \ref{asmp:bernstein} and \ref{asmp:lip} with $r = C_\nu \lambda_\star^{(3-\nu)/2}/R$, it holds that
    \begin{align*}
        \frac1{C_\nu} d_\star \le d_n \le C_\nu d_\star,
    \end{align*}
    with probability at least $1 - \delta$,
    whenever $n$ is large enough (see \Cref{sub:appendix:consist_dn} for the precise condition).
\end{proposition}

\myparagraph{Remark}
The precise version of $\Cref{prop:d_n}$ in \Cref{sub:appendix:consist_dn} implies that $d_n$ is a consistent estimator of $d$.

With \Cref{prop:d_n} at hand, we can obtain finite-sample confidence sets involving $d_n$, which can be computed from data.
We illustrate it with the Wald confidence set.
\begin{corollary}\label{cor:wald_conf_set}
    Suppose the same assumptions in \Cref{prop:d_n} hold true.
    Let $\calC_{\text{Wald}, n}'(\delta)$ be
    \begin{align*}
        \left\{ \theta \in \Theta: \norm{\theta - \theta_\star}_{H_n(\theta_n)}^2 \le C_{K_1, \nu} \log{(e/\delta)} \frac{d_n}{n} \right\}.
    \end{align*}
    Then we have $\Prob(\theta_\star \in \calC_{\text{Wald}, n}'(\delta)) \ge 1 - \delta$ whenever $n$ satisfies the same condition as in \Cref{prop:d_n}.
\end{corollary}

\subsection{Discussion}
\label{sub:discussion}

\myparagraph{Fisher information and model misspecification}
When the model is well-specified, the autocorrelation matrix $G(\theta)$ coincides with the well-known Fisher information $\mathcal{I}(\theta) := \Expect_{Z \sim P_\theta}[S(\theta; Z)S(\theta; Z)^\top]$ at $\theta_\star$.
The Fisher information plays a central role in mathematical statistics and, in particular, M-estimation; see \citep{pennington2018spectrum,kunstner2019limitations,ash2021gone,soen2021variance} for recent developments in this line of research.
It quantifies the amount of information a random variable carries about the model parameter.
Under a well-specified model, it also coincides with the Hessian matrix $H(\theta)$ at the optimum which captures the local curvature of the population risk.
When the model is misspecified, the Fisher information deviates from the Hessian matrix.
In the asymptotic regime, this discrepancy is reflected in the limiting covariance of the weighted M-estimator which admits a sandwich form $H_\star^{-1/2} G_\star H_\star^{-1/2}$; see, e.g., \cite[Sec.~4]{huber1967under}.

\myparagraph{Effective dimension}
The counterpart of the sandwich covariance in the non-asymptotic regime is the effective dimension $d_\star$; see, e.g., \citep{spokoiny2017penalized,ostrovskii2021finite}.
Our bounds also enjoy the same merit---its dimension dependency is via the effective dimension.
When the model is well-specified, the effective dimension reduces to $d$, recovering the same rate of convergence $O(d/n)$ as in classical linear regression; see, e.g., \cite[Prop.~3.5]{bach2021learning}.
When the model is misspecified, the effective dimension provides a characterization of the problem complexity which is adapted to both the data distribution and the loss function via the matrix $H_\star^{-1/2} G_\star H_\star^{-1/2}$.
To gain a better understanding of the effective dimension $d_\star$, we summarize it in \Cref{tab:decay} in \Cref{sec:proofs} under different regimes of eigendecay, assuming that $G_\star$ and $H_\star$ share the same eigenvectors.
It is clear that, when the spectrum of $G_\star$ decays faster than the one of $H_\star$, the dimension dependency can be better than $O(d)$.
In fact, it can be as good as $O(1)$ when the spectrum of $G_\star$ and $H_\star$ decay exponentially and polynomially, respectively.

\myparagraph{Comparison to classical asymptotic theory}
Classical asymptotic theory of M-estimation is usually based on two assumptions: (a) the model is well-specified and (b) the sample size $n$ is much larger than the parameter dimension $d$.
These assumptions prevent it from being applicable to many real applications where the parametric family is only an approximation to the unknown data distribution and the data is of high dimension involving a large number of parameters.
On the contrary, our results do not require a well-specified model, and the dimension dependency is replaced by the effective dimension $d_\star$ which captures the complexity of the parameter space.
Moreover, they are of non-asymptotic nature---they hold true for any $n$ as long as it exceeds some constant factor of $d_\star$.
This allows the number of parameters to potentially grow with the same size.

\myparagraph{Comparison to recent non-asymptotic theory}
Recently, \citet{spokoiny2012parametric} achieved a breakthrough in finite-sample analysis of parametric M-estimation.
Although fully general, their results require strong global assumptions on the deviation of the empirical risk process and are built upon advanced tools from empirical process theory.
Restricting ourselves to generalized self-concordant losses, we are able to provide a more transparent analysis with neater assumptions only in a neighborhood of the optimum parameter $\theta_\star$.
Moreover, our results maintain some generality, covering several interesting examples in statistical machine learning as provided in \Cref{sub:examples}.

\citet{ostrovskii2021finite} also considered self-concordant losses for M-estimation.
However, their results are limited to generalized linear models whose loss is (pseudo) self-concordant and admits the form $\ell(\theta; Z) := \ell(Y, \theta^\top X)$.
While sharing the same rate $O(d_\star / n)$, our results are more general than theirs in two aspects.
First, the loss need not be of the form $\ell(Y, \theta^\top X)$, encompassing the score matching loss in \Cref{ex:score_matching} below.
Second, we go beyond pseudo self-concordance via the notion of generalized self-concordance.
Moreover, they focus on bounding the excess risk rather than providing confidence sets, and they do not study the estimation of $d_\star$.

Pseudo self-concordant losses have been considered for semi-parametric models \citep{liu2022orthogonal}.
However, they focus on bounding excess risk and require a localization assumption on $\theta_n$. Here we prove the localization result in \Cref{prop:localization} and we focus on confidence sets.

\myparagraph{Regularization}
Our results can also be applied to regularized empirical risk minimization by including the regularization term in the loss function.
Let $\theta_{n}^\lambda$ and $\theta_{\star}^\lambda$ be the minimizers of the \emph{regularized} empirical and population risk, respectively.
Let $d_\star^\lambda := \Tr\big((H_\star^\lambda)^{-1/2} G_\star^{\lambda} (H_\star^\lambda)^{-1/2}\big)$ where $H_\star^{\lambda}$ and $G_\star^{\lambda}$ are the regularized Hessian and the autocorrelation matrix of the regularized gradient at $\theta_\star^\lambda$, respectively.
Then our results characterize the concentration of $\theta_{n}^\lambda$ around $\theta_{\star}^\lambda$:
\begin{align*}
    \norm{\theta_n^{\lambda} - \theta_\star^\lambda}_{H_\star^\lambda}^2 \le O(d_\star^\lambda / n).
\end{align*}
This result coincides with \citet[Thm.~2.1]{spokoiny2017penalized}.
If the goal is to estimate the unregularized population risk minimizer $\theta_\star$, then we need to pay an additional error $\norm{\theta_\star^\lambda - \theta_\star}_{H_\star^\lambda}^2$ which is referred to as the modeling bias \citep[Sec.~2.5]{spokoiny2017penalized}.
One can invoke a so-called \emph{source condition} to bound the modeling bias and a \emph{capacity condition} to bound $d_\star^\lambda$.
An optimal value of $\lambda$ can be obtained by balancing between these two terms \cite[see, e.g.,][]{marteau2019beyond}.

For instance, let $Z := (X, Y)$ where $X \in \reals^d$ with $\Expect[XX^\top] = I_d$ and $Y \in \reals$.
Consider the regularized squared loss
$ \score^\lambda(\theta; z) := 1/2\, (y - \theta^\top x)^2 + 1/2\, \theta^\top U \theta$
where $U = \diag\{\mu_1, \dots, \mu_d\}$.
The regularized effective dimension is then~\citep[Sec.~2.1]{spokoiny2017penalized} of order 
$ O\big( \sum_{k=1}^d 1/(1 + \mu_k) \big)$
which can be much smaller than $d$ if $\{\mu_k\}$ is increasing.



\section{EXAMPLES AND APPLICATIONS}
\label{sec:application}
\section{Activation During Perception of Noisy Speech}\label{sec:Apps}
The dataset, provided as  {\tt data6} in the AFNI tutorial~\citet{cox96},
is originally from an fMRI study~\citet{nathandbeauchamp11} where
\begin{figure}[h]
\subfloat[]{\includegraphics[width=0.5\columnwidth]{figs/AR-FAST-001-Visual-crop}}
\subfloat[]{\includegraphics[width=0.5\columnwidth]{figs/AR-FAST-001-Audio-crop}}
%\subfloat[]{\includegraphics[width=0.33\textwidth]{figs/AM-FAST-005-diff}}
\caption{ AR-FAST-identified activation regions on SPMs obtained by
  fitting ~\ref{eq:lm} with   AR($\hat{p}$) to AFNI's {\tt data6} for
  (a) visual-reliable stimulus and (b) audio-reliable
  stimulus.}
\label{fig:AMSmoothingAFNI}
\end{figure}
\begin{comment}
\begin{figure*}[h]
\subfloat[]{\includegraphics[width=0.25\textwidth]{figures/Visual_AM-crop}}
\subfloat[]{\includegraphics[width=0.25\textwidth]{figures/Audio_AM-crop}}
\subfloat[]{\includegraphics[width=0.25\textwidth]{figures/Visual_Audio_AM-crop}}
\caption{Activation areas obtained using AR-FAST with in {\it AFNI data6} on the SPM obtained
  after fitting AR($\hat{p}$) of the (a) Visual-reliable, (b) Audio-reliable and (c) the difference contrast between Visual-reliable and Audio-reliable.}
\label{fig:AMSmoothingAFNI}
\end{figure*}
\end{comment}
a subject heard and saw a female volunteer speak words, separately, in
two different formats. The audio-reliable setting had the subject
clearly hear the spoken word but see a degraded image of the speaker
while the visual-reliable case had the subject clearly see the speaker
vocalize the word but the audio was of reduced quality.  There were
three experimental runs, each 
consisting of a randomized design of 10 blocks, equally divided into blocks of
audio-reliable and visual-reliable stimuli. %An echo-planar imaging
% sequence (TR=2s) was used to obtain
$\mbox{T}_2^*$-weighted images with volumes of $80 \times 80 \times
33$ (with voxels of dimension $2.75 \times  2.75 \times 3.0\  mm^3$)
from  echo-planar sequences (TR=2s) 
were obtained  over $152$ time-points. Our interest was in determining 
activation corresponding to the audio
($H_0:\beta_{a}=0$) and visual
($H_0:\beta_{v}=0$) tasks.
%, as well as their contrast  ($H_0:\beta_{v} - \beta_{a}=0$). The
%first two cases have one-sided alternatives while the contrast in
%activation corresponds to a two-sided alternative.
At each voxel, we fitted AR models for 
$p=0,1,2,3,4,5$ and chose $p$ with the highest BIC. 
Figure \ref{fig:AMSmoothingAFNI} uses AFNI and Surface Mapping (SUMA)
to display activated regions obtained using AR-FAST on the SPM:
see  Figure~\ref{fig:Visual-Audio} for  maps drawn from ALL-FAST, AS,
AWS and CT. We used $\alpha = 0.01$ because of the high (greater than
4) upper percentile of the voxel-wise estimated CNRs. Most of the activation 
occurs in Brodmann areas 18 and 19 (BA18 and BA19)
which comprise the
occipital cortex %in the human brain,  accounting for the bulk of the
                 %volume of the occipital lobe. Both areas form part
                 %of the visual association area while BA 19, also the occipital lobe cortex as well. Along with BA18, it comprises
and the extrastriate (or peristriate) cortex. In humans with normal
sight,  this area is for visual association where 
feature-extraction, shape recognition, attentional and multimodal
integrating functions occur. We also see increased activation in
the STS, which recent
studies~\citep{grossman2001brain} have related to  distinguishing
voices from environmental sounds, 
stories versus nonsensical speech, moving faces versus moving objects,
biological motion and so on. ALL-FAST performs similarly as AR-FAST,
while the other methods also identify the same regions but they identify
a lot more activated 
voxels, some of which appear to be false positives. Although a 
detailed analysis of the results of this study is beyond the purview
of this paper, we note that AR-FAST 
finds interpretable results even when applied to a single
subject high-level cognition experiment. 
\begin{comment}
\begin{table}[h!]
\centering
\caption{Coordinates of the maximum $t$-statistic and its corresponding value}\label{tab:maxt}
\begin{tabular}{c|c}
Task & $(x,y,z) mm$  \\
\hline
Visual-Audio & (-30.162,86.221,6.349) \\
Audio & (-27.412,75.221,-5.651)  \\
 Visual & (-30.162, 80.721, 15.349) \\
\hline
\end{tabular}
\end{table}
\end{comment}



\section{NUMERICAL STUDIES}
\label{sec:experiments}

\section{Experiments}\label{sec:experiments}
We validate our approach using multiple datasets containing real-life data from the fields of criminal risk assessment, credit, lending, and college admissions. In each of the datasets we select a binary feature and treat it as the protected attribute (e.g., race or gender), which is the feature we require our trained classifier to behave fairly upon. Our proposed method performs well on all of these datasets, succeeding in removing unfairness almost entirely, at a very modest price in terms of accuracy.


\begin{table*}[h]
\centering
\resizebox{\textwidth}{!}{
\def\arraystretch{1.2}

\begin{tabular}{c c c | c | c | c || c | c | c || c | c | c |}

\cline{4-12}
&&&
\multicolumn{9}{ c| }{\textbf{COMPAS Dataset}}
\\ \cline{4-12}
&&&
\multicolumn{3}{ c|| }{\textbf{FPR Considerations}}&
\multicolumn{3}{ c|| }{\textbf{FNR Considerations}}&
\multicolumn{3}{ c| }{\textbf{Both Considerations}}
\\ \cline{4-12}
&&&
 $\mathbf{Acc.}$ &  $\mathbf{D_{FPR}}$ &  $\mathbf{D_{FNR}}$ &  $\mathbf{Acc.}$ &  $\mathbf{D_{FPR}}$ &  $\mathbf{D_{FNR}}$ &  $\mathbf{Acc.}$ &  $\mathbf{D_{FPR}}$ &  $\mathbf{D_{FNR}}$
\\  \cline{4-12}
\vspace*{-0.5ex}
\\ \cline{1-2} \cline{4-12}
\multicolumn{1}{ |c  }{} &
\multicolumn{1}{ c|  }{  \textbf{Our Method (AVD Penalizers)}}  &&
$\mathbf{0.660}$    &  $\mathbf{0.01}$  &  $0.04$ &
$\mathbf{0.653}$    &  $0.02$   &  $\mathbf{0.04}$ &
$\mathbf{0.654}$    &  $\mathbf{0.02}$  &  $\mathbf{0.04}$
\\ \cline{1-2} \cline{4-12}
\multicolumn{1}{ |c  }{} &
\multicolumn{1}{ c|  }{  \textbf{Our Method (SD Penalizers)}}  &&
$\mathbf{0.664}$    &  $\mathbf{0.02}$  &  $0.09$ &
$\mathbf{0.661}$    &  $0.05$   &  $\mathbf{0.03}$ &
$\mathbf{0.661}$    &  $\mathbf{0.02}$  &  $\mathbf{0.03}$
\\ \cline{1-2} \cline{4-12}
\multicolumn{1}{ |c  }{} &
\multicolumn{1}{ c|  }{  Zafar et al.~(\citeyear{disparatemistreatment})}  &&
$0.660$    &   $0.06$    &   $0.14$  &
$0.662$    &   $0.03$    &   $0.10$  &
$0.661$    &   $0.03$    &   $0.11$
\\ \cline{1-2} \cline{4-12}
\multicolumn{1}{ |c  }{} &
\multicolumn{1}{ c|  }{  Zafar et al. Baseline~(\citeyear{disparatemistreatment})}  &&
$0.643$    &   $0.03$    &   $0.11$  &
$0.660$    &   $0.00$    &   $0.07$  &
$0.660$    &   $0.01$    &   $0.09$
\\ \cline{1-2} \cline{4-12}
\multicolumn{1}{ |c  }{} &
\multicolumn{1}{ c|  }{  Hardt et al.~(\citeyear{hardt})}  &&
$0.659$    &  $0.02$    &   $0.08$  &
$0.653$    &  $0.06$   &    $0.01$  &
$0.645$    &  $0.01$   &    $0.01$
\\ \cline{1-2} \cline{4-12}
\multicolumn{1}{ |c  }{} &
\multicolumn{1}{ c|  }{  \textbf{Vanilla Regularized Logistic Regression}}  &&
$\mathbf{0.672}$    &   $\mathbf{0.20}$    &   $\mathbf{0.30}$  &
$\mathbf{0.672}$    &   $\mathbf{0.20}$    &   $\mathbf{0.30}$  &
$\mathbf{0.672}$    &   $\mathbf{0.20}$    &   $\mathbf{0.30}$
\\ \cline{1-2} \cline{4-12}
\end{tabular}
}
\vspace{3mm}
\caption{Performance comparison on the COMPAS dataset. For the approaches in bold -- Accuracy, FPR difference and FNR difference are evaluated on the test set, averaging over five runs and using a 70-30 training/test split. The performance of the remaining three approaches is stated as reported in Zafar et al.~(\citeyear{disparatemistreatment}).} \label{table:comparison_results}
\end{table*}



\begin{figure*}[b]
  \includegraphics[scale=0.6]{compas0-400.png}
  \caption{COMPAS Dataset. Accuracy, FPR difference ($\mathbf{D_{FPR}}$), and FNR difference ($\mathbf{D_{FNR}}$) (all evaluated on the test set) of the learned classifier, as a function of the weight $c=c_1 = c_2 \geq 0$ placed on the fairness penalizer terms. On the left we use the Absolute Value Difference (AVD) penalizer, and the Squared Difference (SD) penalizer on the right, both as presented in Section~\ref{regularization}. ``Relaxed FPR/FNR Diff.'' plots the value of the relevant penalization term.} %In this particular run, parameters chosen for the absolute value relaxation were: $c=80, q_c=60$, and for the squared relaxation: $c=220, q_c=30$.}
  \label{fig:compas}
\end{figure*}


\subsection{Implementation}
\textbf{Our method} 
%We instantiate our method in the following way: Given dataset $Q$, we split it randomly into a training set $S$ (which we will use for learning) and a test set $T$ (which we will only use for reporting performance). 
For the purpose of comparison with  Zafar et al.~(\citeyear{disparatemistreatment}) and Hardt et al.~\cite{hardt} on the COMPAS data, we use a parameter $c$ to induce three possible combinations of weights on the FPR and FNR penalization terms: $c = c_1$ and $c_2 = 0$; $c_1 = 0$ and $c = c_2$; and $c = c_1 = c_2$. For the other three datasets, we consider only $c = c_1 = c_2$.\footnote{The reason for varying the values of $c$ in the training phase is since we shifted to a proxy problem, in which we rely on the distance from the decision boundary rather the actual classifications. 
%Our hope is that there is no need for a worst-case cross validation between all of the combinations of $c_1, c_2, c_3$, and that the training scheme we propose is sufficient. 
It is possible, of course, that even better results are attainable using our scheme with other combinations of $c_1, c_2$, and $q$.} To explore the accuracy/fairness trade-off curve for the relaxed optimization problem~(\ref{eq:2}), we train for different values of $c$, starting at $c=0$ (which is just standard logistic regression), and growing gradually.



Given a dataset $Q$ and fixing a $d_1, d_2 \in \{0, 1\}$ of interest, we use the following training scheme:
\begin{enumerate}
\item Split $Q$ at random into training set $S$ and test set $T$.
\item For each $c$, perform cross-validation on $S$ to select the corresponding best value $q_c$ for the regularization parameter.
\item For each $(c,q_c)$, let $\theta_c = \argmin\limits_{\theta} \text{Proxy}(\theta;S,c,c,q_c)$.
\item Select $\theta^* \in \argmin\limits_{\theta_c} \text{Objective}(\theta_c;S,d_1,d_2)$.
\item Evaluate performance using $\theta^*$ on test set $T$.
\end{enumerate}
We report the average of five such runs, each with a fresh training-test split.




%We instantiate our method by solving the relaxed optimization problem~(\ref{eq:2}), in place of the original, non-convex problem~(\ref{eq:1}).  
%We test our approach with three different combinations of weights on the penalization terms:
%\katrina{What are the $d$, and how are they related to the $c$s?}
%\begin{enumerate}
%\item FPR considerations only: $d_1 = 1, d_2 = 0$.
%\item FNR considerations only: $d_1 = 0, d_2 = 1$.
%\item Both FPR, FNR considerations, assigned similar significance: $d_1 = 1, d_2 = 1$.
%\end{enumerate}
%One could, of course, pick any other combination of the FPR and FNR penalty weights.

%\katrina{I don't understand how the below is distinct from the list above}
%Learning is done by training the parameters of a logistic regressor to solve~\ref{eq:2}, while picking the value of $c_1, %c_2$ as the following:
%\begin{enumerate}
%\item FPR considerations only: $c_1 = c \geq 0$, $c_2 = 0$.
%\item FNR considerations only: $c_1 = 0$, $c_2 = c \geq 0$.
%\item Both FPR, FNR considerations, assigned similar significance: $c_1 = c_2 = c \geq 0$
%\end{enumerate}



% We then cross-validate to pick the best $c_3$ (the weight on the standard $\ell_2$-regularization term) given $c$.\footnote{The reason for varying the values of $c$ in the training phase is since we shifted to a proxy problem, in which we rely on the distance from the decision boundary rather the actual classifications. 
%Our hope is that there is no need for a worst-case cross validation between all of the combinations of $c_1, c_2, c_3$, and that the training scheme we propose is sufficient. 
%It is possible, of course, that even better results are attainable using our scheme with other combinations of $c_1, c_2, c_3$.} For each such combination, we report results as the averages of multiple \katrina{how many?} different runs, each time splitting data randomly into training and test sets.
%\yahav{We need to shorten this description.}

We solve the relaxed convex optimization problem using the CVXPY solver. Due to stability issues with large training sets, we use a train/test split of 30-70 on the larger datasets, rather than 70-30 as on the COMPAS dataset\footnote{The code implementing our method can be found at https://github.com/jjgold012/lab-project-fairness}.

%
%
%We then report the results (as evaluated on the test set) attained by a regressor $\theta \in \mathbb{R}^d$ that minimizes (on the training set $S$) a weighted combination of the $0$-$1$ loss and the differences in FPR and FNR across populations:
%\begin{equation*}
%\begin{aligned}
%&\underset{\theta}{\text{argmin}}
%& & L_{S}^{0\text{-}1}(\theta) \\
%&&& + d_1|FPR_{A=0}(\theta;S)-FPR_{A=1}(\theta;S)| \\
%&&& + d_2|FNR_{A=0}(\theta;S)-FNR_{A=1}(\theta;S)|
%\end{aligned}
%\end{equation*}
%
%\katrina{What is $d_1$ vs. $c_1$ etc.?}



%For classification, we decided use a standard cut-off threshold of $c=0.5$. There are of course, further possible interactions between the FPR, FNR considerations, and picking a certain cut-off level. These are not straightforward, since  these interactions are data-specific. 



%allows for flexibility in picking the values of $c_1, c_2$, which reflect the significance we wish to place on the objectives of achieving accuracy, equal FPR, and equal FNR. As for $c_3$, we will want to find the value of it that achieves the best results, for any combined objective of accuracy and fairness defined by a specific selection of $c_1,c_2$. Therefore, given a specific selection of $c_1, c_2$, we apply cross-validation to select the value of $c_3$. 




We briefly describe the other algorithmic approaches to which we compare:\\
\textbf{Zafar et al.}~(\citeyear{disparatemistreatment}) performs optimization by considering a proxy for the bias: the covariance between the samples' sensitive attributes and the signed distance between the feature vectors of misclassified users and the classifier decision boundary.\\
\textbf{Zafar et al. Baseline}~(\citeyear{disparatemistreatment}) tries to enforce equal FP/FN rates on the different groups by introducing different penalties for misclassified data points with different sensitive attribute values during the training phase.\\
\textbf{Hardt et al.}~(\citeyear{hardt}) performs post-processing on a standard trained (unfair) logistic regressor, picking different decision thresholds for different groups, and possibly adding randomization.


\subsection{Experimental Results}

In what follows, we use the following notation, given a trained classifier $\hat{Y}$:
\begin{align*}
\mathbf{D_{FPR}}&=\left|FPR_{A=0}(\hat{Y})-FPR_{A=1}(\hat{Y})\right| \\ 
\mathbf{D_{FNR}}&=\left|FNR_{A=0}(\hat{Y})-FNR_{A=1}(\hat{Y})\right|
\end{align*}
The values $FPR_{A=0}(\hat{Y})$, $FPR_{A=1}(\hat{Y})$, $FNR_{A=0}(\hat{Y})$, $FNR_{A=1}(\hat{Y})$ are reported as evaluated on the test set.

\paragraph{The COMPAS Dataset\footnote{https://github.com/propublica/compas-analysis}} The Correctional Offender Management Profiling for Alternative Sanctions (COMPAS) records from Broward County, Florida 2013-2014, made available online by ProPublica, are perhaps the best-studied data in the context of fairness.  The goal in this scenario is to successfully predict recidivism within two years, based on features such as age, gender, race, number of prior offenses, and charge degree. The dataset contains 5,278 samples. The protected attribute in this scenario is race, where $A$ indicates black or white. We filtered the dataset using the same features as Zafar et al.~(\citeyear{disparatemistreatment}), to allow for comparison.

%\begin{table}[h]
%\centering
%\begin{tabularx}{\columnwidth}{c|c|c|c}
%\hline
%  &  Recid. ($y = 1$)        & No Recid.  ($y = 0$)       & Total \\ \hline
%Black &  $ 1661   $ & $ 1514 $ &  $ 3175 $ \\ \hline
%White &  $ 822   $  & $1281  $ &  $ 2103 $ \\ \hline
%Total &  $ 2483  $  & $2795 $ &  $ 5278 $ \\\hline
%\end{tabularx}
%\caption{Statistics of the ProPublica COMPAS data.} \label{table:compas-stats}
%\label{tab:stats}
%\end{table}
%\vspace{-1em}

%\begin{table}[h]
%\centering
%\begin{tabularx}{\columnwidth}{c|c}
%\hline
%Feature  &  Description \\ \hline
%Age Category &  $<25$, between $25$ and $45$, $>45$ \\
%Gender &  Male or Female \\
%Race &  White or Black \\
%Priors Count &  0--37 \\
%Charge Degree &  Misconduct or Felony \\
%\hline
%2-year-recid. & Whether or not the  \\
%(target feature)  & defendant recidivated within two years
%\end{tabularx}
%\caption{Description of features used from ProPublica COMPAS data.} \label{table:compas-features}
%\label{tab:features}
%\end{table}




\begin{table*}[t]
\centering
\caption{A description of the datasets used, along with parameters of the training procedure used for each.}
\label{table:datasets_description}
\begin{adjustbox}{max width=\textwidth}
\begin{tabular}{|l|l|l|l|l|l|l|l|}
\hline
\textbf{Dataset} & \textbf{No. Samples} & \textbf{No. Features} & \textbf{Train/Test Split} & \textbf{No. Repetitions} & \textbf{No. Folds in CV} & \textbf{Protected Feature} & \textbf{Target Variable} \\ \hline
COMPAS           & 5,278                     & 5                          & 70-30                     & 5                        & 5                                 & Race                       & 2-Year-Recidivism        \\ \hline
Adult            & 30,162                    & 10                         & 30-70                     & 5                        & 5                                 & Gender                     & Income Over/Under 50K    \\ \hline
Default          & 30,000                    & 23                         & 30-70                     & 5                        & 3                                 & Gender                     & Defaulting On Payments   \\ \hline
Admissions       & 20,839                    & 17                         & 30-70                     & 5                        & 3                                 & Race                       & Passing Bar Exam         \\ \hline
\end{tabular}
\end{adjustbox}
\end{table*}


\begin{table*}[t]
\centering
\resizebox{\textwidth}{!}{
\def\arraystretch{1.2}

\begin{tabular}{c c c | c | c | c || c | c | c || c | c | c |}

\cline{4-12}
&&&
\multicolumn{3}{ c|| }{\textbf{Adult Dataset}}&
\multicolumn{3}{ c|| }{\textbf{Default Dataset}}&
\multicolumn{3}{ c| }{\textbf{Admissions Dataset}}
\\ \cline{4-12}
%&&&
%\multicolumn{3}{ c|| }{\textbf{Both Considerations}}&
%\multicolumn{3}{ c|| }{\textbf{Both Considerations}}&
%\multicolumn{3}{ c| }{\textbf{Both Considerations}}
%\\ \cline{4-12}
&&&
 $\mathbf{Acc.}$ &  $\mathbf{D_{FPR}}$ &  $\mathbf{D_{FNR}}$ &  $\mathbf{Acc.}$ &  $\mathbf{D_{FPR}}$ &  $\mathbf{D_{FNR}}$ &  $\mathbf{Acc.}$ &  $\mathbf{D_{FPR}}$ &  $\mathbf{D_{FNR}}$
\\  \cline{4-12}
\vspace*{-0.5ex}
\\ \cline{1-2} \cline{4-12}
\multicolumn{1}{ |c  }{} &
\multicolumn{1}{ c|  }{  \textbf{Our Method (AVD Penalizers)}}  &&
$\mathbf{0.776}$    &  $\mathbf{0.00}$  &  $\mathbf{0.04}$ &
$\mathbf{0.807}$    &  $\mathbf{0.00}$   &  $\mathbf{0.01}$ &
$\mathbf{0.950}$    &  $\mathbf{0.01}$  &  $\mathbf{0.00}$
\\ \cline{1-2} \cline{4-12}
\multicolumn{1}{ |c  }{} &
\multicolumn{1}{ c|  }{  \textbf{Our Method (SD Penalizers)}}  &&
$\mathbf{0.783}$    &  $\mathbf{0.00}$  &  $\mathbf{0.09}$ &
$\mathbf{0.806}$    &  $\mathbf{0.01}$   &  $\mathbf{0.02}$ &
$\mathbf{0.950}$    &  $\mathbf{0.00}$  &  $\mathbf{0.00}$
\\ \cline{1-2} \cline{4-12}
\multicolumn{1}{ |c  }{} &
\multicolumn{1}{ c|  }{  \textbf{Vanilla Regularized Logistic Regression}}  &&
$\mathbf{0.800}$    &   $\mathbf{0.08}$    &   $\mathbf{0.39}$  &
$\mathbf{0.807}$    &   $\mathbf{0.01}$    &   $\mathbf{0.05}$  &
$\mathbf{0.951}$    &   $\mathbf{0.16}$    &   $\mathbf{0.02}$
\\ \cline{1-2} \cline{4-12}
\end{tabular}
}
\vspace{3mm}
\caption{Performance on the Adult, Loan Default, and Admissions datasets, penalizing for both FPR and FNR difference. Accuracy, FPR difference and FNR difference are evaluated on the test set, averaging over five runs and using a 30-70 training/test split.} \label{table:comparison_results_rest}
\end{table*}


In Table~\ref{table:comparison_results}, we compare the performance of our approach with that of three other techniques from the literature. Each method was trained based on logistic regression.  As a basis for comparison, we also present the performance of vanilla logistic regression, absent fairness considerations, with the regularization parameter selected via cross-validation.\footnote{Zafar et al.~(\citeyear{disparatemistreatment}) do not incorporate regularization in any of the approaches they report.}
%Results are reported as the averages of 5 different runs \katrina{Is that still correct?}, each time splitting data evenly and randomly into training and test sets. 
Results for Zafar et al., Zafar et al. baseline, and Hardt et al. appear here as reported in Zafar et al.~(\citeyear{disparatemistreatment}).\footnote{Our method selects the classifier based on the training set only and reports its performance over the test set. Results for the three other approaches, reported by Zafar et al.~(\citeyear{disparatemistreatment}), are based on tuning parameters after seeing the trade-off curve over the test set, and reporting according to the best selection of these parameters.}
%\katrina{Perhaps here is the right place for a footnote about the discrepancy with the Zafar baseline}

We find that the vanilla logistic regressor (absent fairness considerations) results in significant unfairness, as $\mathbf{D_{FPR}}=0.20$, and $\mathbf{D_{FNR}}=0.30$. The overall accuracy of this classifier measured on the test set was $0.672$.\footnote{Zafar et al.~(\citeyear{disparatemistreatment}) report a slightly different baseline of: Accuracy = 0.668, $\mathbf{D_{FPR}}=0.18$, $\mathbf{D_{FNR}}=0.30$.} Our SD penalization approach empirically achieves approximately the same accuracy as the Zafar et al.~(\citeyear{disparatemistreatment}) approach, with significantly better fairness. It is difficult to compare fairness-accuracy tradeoffs with the Hardt et al.~(\citeyear{hardt}) approach, since their accuracy is significantly lower than ours. A more direct comparison is possible by noting that our learned classifier can be post-processed to improve its fairness at a direct cost to accuracy. Hence, we can achieve accuracy of $0.659$ with $\mathbf{D_{FPR}} = \mathbf{D_{FNR}} = 0.01$, which compares very favorably with the Hardt et al. accuracy rate of 0.645 given the same FPR and FNR rates.\footnote{For completeness, we note that using a 50-50 training-test split (again not using the test set for parameter selection), our method (SD, both considerations) produces a classifier that provides: Accuracy = 0.659, $\mathbf{D_{FPR}} = 0.01, \mathbf{D_{FNR}} = 0.05$. This classifier can be post-processed to achieve rates of: Accuracy = 0.655, $\mathbf{D_{FPR}} = \mathbf{D_{FNR}} = 0.01$.}

Figure \ref{fig:compas} illustrates the accuracy/fairness trade-offs achievable using our scheme. Increasing the weight $c$ on the proxy fairness penalizers results in reducing their magnitude. The figure also illustrates how our relaxed penalizers succeed in tracking the real FPR and FNR differences. 
%
%
%\katrina{Must rewrite the following paragraph}
%We observe that our method succeeds in eliminating unfairness almost completely on the COMPAS dataset, while retaining most of the accuracy, when compared to the vanilla logistic regression. We achieve very low difference rates when penalizing for achieving each of the FPR and FNR criteria individually, and also for both. We achieve preferable results comparing to Zafar et al. and Zafar et al. baseline in all 3 scenarios, and also comparing to Hardt et al. in the settings of false positive/false negative considerations only. In the setting of both considerations - The Hardt et al. method removes a larger portion of the unfairness, however it results in major accuracy loss as it achieves accuracy rate of 0.645 in comparison to our method which results in accuracy of 0.665, retaining most of the original accuracy rate while removing most of the unfairness.




%The Hardt et al.~\cite{hardt} approach as reported removes a smaller portion of the bias in the different scenarios, however for FP/FN constraints alone, it provides higher accuracy rates. The Zafar et al.~(\citeyear{disparatemistreatment}) approach as reported retains significant bias (in most cases), but in some cases  achieves slightly superior accuracy rates to the methods above. 

%These performance comparisons are incomplete in the sense that each of the compared techniques has the potential to trade off between accuracy and fairness, using some degree of parameter tuning; what we report here is only one point on the achievable trade-off frontier for each algorithm. The ``correct'' trade-off, and, in particular, the best manner in which to weigh unfairness in the FPR against unfairness in the FNR, are matters of opinion. We have chosen to report our method's performance under parameters designed to very aggressively mitigate unfairness, at some cost to the accuracy.

%It would certainly be desirable to evaluate these and other approaches to fair learning on other datasets and on different tasks, particularly on larger datasets, which might afford both greater accuracy and better bias-reduction. The present empirical evaluations, however, suggest that our regularization-based approach provides a new tool worthy of consideration---we succeed in almost entirely eliminating bias on the hold-out set, at a modest price in terms of accuracy.

%Due to the fact that our true objective includes the original non-convex penalization terms, our approach does not carry any formal guarantees. However, the ease of implementation, generality, and empirical results are encouraging. Figure~\ref{fig:test1} illustrates the rate of convergence to a fair, accurate classifier on this dataset.
%In terms of computation costs, given that at each iteration we must calculate the gradient according to the FPR and FNR regularizers, we are required to predict the labels for the entire training set at each step. 
%However, this does not pose a computational burden, as it is already required by the (classic) gradient descent algorithm in our logistic regressor fitting scheme. Furthermore, when given a sufficiently large dataset (one or two orders of magnitude larger than the one currently available for the COMPAS scores data), this could be relaxed to sampling only a mini-batch of samples from the training data set at each iteration (much as is done in stochastic gradient descent).






\subsection{Additional Datasets}


Table~\ref{table:datasets_description} provides summary statistics on each of the datasets on which we tested our approach. We also briefly describe the datasets below. 


{\bf The Adult Dataset}\footnote{http://archive.ics.uci.edu/ml/datasets/Adult} is based on 1994 US Census data. The task we consider is to predict whether the income of each individual is over or under 50K dollars per year, based on features such as occupation, marital status, and education. The protected attribute selected in this task is gender. 

{\bf The Loan Default Dataset}\footnote{{\scriptsize https://archive.ics.uci.edu/ml/datasets/default+of+credit+card+clients}}
contains data regrading Taiwanese credit card users. The task we consider is to predict whether an individual will default on payments, based on features such as history of past payments, age, and the amount of given credit. The protected attribute is gender.

{\bf The Admissions Dataset}\footnote{http://www2.law.ucla.edu/sander/Systemic/Data.htm}
contains records of law school students who went on to take the bar exam. The task we consider is to predict whether a student will pass the exam based on features such as LSAT score, undergraduate GPA, and family income. The protected attribute is set to race.

Table~\ref{table:comparison_results_rest} describes the performance of our approach on these datasets, and Figures~\ref{fig:adult},~\ref{fig:default}, and~\ref{fig:lawschool} illustrate the fairness-accuracy trade-offs we achieve in each context. Overall, we see that unfairness is nearly eliminated while accuracy remains quite high. The dataset on which accuracy suffers most under our approach is the Adult dataset, which is also the dataset on which the vanilla regression is the most unfair.


\begin{figure*}[]
  \includegraphics[scale=0.6]{adult0-800.png}
  \caption{Adult Dataset. Fairness-Accuracy tradeoffs, as in Figure~\ref{fig:compas}.}
  \label{fig:adult}  
\end{figure*}



\begin{figure*}[]
  \includegraphics[scale=0.6]{default0-50.png}
  \caption{Loan Default Dataset. Fairness-Accuracy tradeoffs, as in Figure~\ref{fig:compas}.}
  \label{fig:default}
\end{figure*}



\begin{figure*}[]
  \includegraphics[scale=0.6]{admissions0-400.png}
  \caption{Admissions Dataset. Fairness-Accuracy tradeoffs, as in Figure~\ref{fig:compas}.}
  \label{fig:lawschool}
\end{figure*}





\subsubsection*{Acknowledgements}
The authors would like to thank K.~Jamieson, L.~Jain, and V. Roulet for fruitful discussions.
L.~Liu is supported by NSF CCF-2019844 and NSF DMS-2023166 and NSF DMS-2133244.
Z.~Harchaoui is supported by NSF CCF-2019844, NSF DMS-2134012, NSF DMS-2023166, CIFAR-LMB, and faculty research awards.
Part of this work was done while Z.~Harchaoui was visiting the Simons Institute for the Theory of Computing.

\clearpage

\bibliographystyle{abbrvnat}
\bibliography{biblio}

\clearpage
\appendix

% Make the appendix single col
\begingroup
\let\clearpage\relax 
\onecolumn 
\endgroup
% End: appendix single col

\addcontentsline{toc}{section}{Appendix} 
\part{Appendix} 
\parttoc
\clearpage


\section{Proof of main results}
\label{sec:proofs}
\section{Proofs}
\subsection{Proof of Theorem \ref{th:inexactLS1}}
We start from a similar argument as in \cite[proof of Therorem~2]{Blumen}. 
%proof of \cite[Theorem~2]{Blumen}. 
Set $g := 2\nabla f(x^{k-1})=2A^T(Ax^{k-1}-y)$ and $\g:=2\nablaa^{\nug} f(x^{k-1})= g+2\eg^k$ for some vector $\eg^k$ which by definition~\eqref{eq:grad} is bounded $\norm{\eg^k}\leq \nug^k$. It follows that
\ifCLASSOPTIONtwocolumn
\begin{align*} 
&\norm{y-Ax^k}^2-\norm{y-Ax^{k-1}}^2	\\
&= \langle x^k-x^{k-1},g \rangle +\norm{A(x^k-x^{k-1})}^2 \\
&\leq \langle x^k-x^{k-1},g \rangle + \MM \norm{x^k-x^{k-1}}^2, 
\end{align*}
\else
\begin{align*} 
\norm{y-Ax^k}^2-\norm{y-Ax^{k-1}}^2	&= \langle x^k-x^{k-1},g \rangle +\norm{A(x^k-x^{k-1})}^2 \\
&\leq \langle x^k-x^{k-1},g \rangle + \MM \norm{x^k-x^{k-1}}^2, 
\end{align*}
\fi
where the last inequality follows from the ULE property in Definition \ref{def:Lip}. Assuming $\MM \leq 1/\mu$, we have
\ifCLASSOPTIONtwocolumn
\begin{align*}
&\langle x^k-x^{k-1},g \rangle + \MM \norm{x^k-x^{k-1}}^2 \\
& \leq \langle x^k-x^{k-1},g \rangle + \frac{1}{\mu} \norm{x^k-x^{k-1}}^2\\
&= \langle x^k-x^{k-1},\g \rangle + \frac{1}{\mu} \norm{x^k-x^{k-1}}^2 - \langle x^k-x^{k-1},2\eg^k \rangle\\
& = \frac{1}{\mu} \norm{x^k-x^{k-1}+\frac{\mu}{2} \g }^2 - \frac{\mu}{4} \norm{\g}^2 - \langle x^k-x^{k-1},2\eg^k \rangle.
\end{align*}
\else
\begin{align*}
\langle x^k-x^{k-1},g \rangle + \MM \norm{x^k-x^{k-1}}^2 
& \leq \langle x^k-x^{k-1},g \rangle + \frac{1}{\mu} \norm{x^k-x^{k-1}}^2\\
&= \langle x^k-x^{k-1},\g \rangle + \frac{1}{\mu} \norm{x^k-x^{k-1}}^2 - \langle x^k-x^{k-1},2\eg^k \rangle\\
& = \frac{1}{\mu} \norm{x^k-x^{k-1}+\frac{\mu}{2} \g }^2 - \frac{\mu}{4} \norm{\g}^2 - \langle x^k-x^{k-1},2\eg^k \rangle.
\end{align*}
\fi
Due to the update rule of Algorithm \eqref{eq:inIP} and the inexact (fixed-precision) projection step, we have
\ifCLASSOPTIONtwocolumn
\begin{align*}
	&\norm{x^k-x^{k-1}+\frac{\mu}{2} \g }^2 \\
	&\leq  \norm{\pp_{\Cc}(x^{k-1}-\frac{\mu}{2} \g)-x^{k-1}+\frac{\mu}{2} \g }^2 +(\nup^k)^2\\
	&\leq \norm{x^\gt-x^{k-1}+\frac{\mu}{2} \g }^2 +(\nup^k)^2.
\end{align*}
\else
\begin{align*}
\norm{x^k-x^{k-1}+\frac{\mu}{2} \g }^2 
&\leq  \norm{\pp_{\Cc}(x^{k-1}-\frac{\mu}{2} \g)-x^{k-1}+\frac{\mu}{2} \g }^2 +(\nup^k)^2\\
&\leq \norm{x^\gt-x^{k-1}+\frac{\mu}{2} \g }^2 +(\nup^k)^2.
\end{align*}
\fi
The last inequality holds for any member of $\Cc$ and thus here for $x^\gt$. Therefore we can write
\ifCLASSOPTIONtwocolumn
\begin{align}
&\norm{y-Ax^k}^2-\norm{y-Ax^{k-1}}^2 \nonumber	\\
%&=\langle x^{t+1}-x^t,g \rangle + \frac{1}{\mu} \norm{x^{t+1}-x^t}^2 \nonumber\\
&\leq \frac{1}{\mu} \norm{x^\gt-x^{k-1}+\frac{\mu}{2} \g }^2 - \frac{\mu}{4} \norm{\g}^2 \nonumber\\
&\qquad - \langle x^k-x^{k-1},2\eg^k \rangle +(\frac{\nup^k}{\sqrt\mu})^2 \nonumber \\
&= \langle x^\gt-x^{k-1},\g \rangle + \frac{1}{\mu} \norm{x^\gt-x^{k-1}}^2 \nonumber\\
&\qquad - \langle x^k-x^{k-1},2\eg^k \rangle 
 +(\frac{\nup^k}{\sqrt\mu})^2\nonumber\\
%&= \langle x^\gt-x^{k-1},g \rangle + \frac{1}{\mu} \norm{x^\gt-x^{k-1}}^2 - \langle x^k-x^*,2\eg^k \rangle +\frac{\nup^k}{\mu}\nonumber\\
&\leq \langle x^\gt-x^{k-1},g \rangle + \frac{1}{\mu} \norm{x^\gt-x^{k-1}}^2 \nonumber\\
&\qquad+2\nug^k\norm{x^k-x^*} +(\frac{\nup^k}{\sqrt\mu})^2. \label{eq:p1b2}
\end{align}
\else
\begin{align}
\norm{y-Ax^k}^2-\norm{y-Ax^{k-1}}^2 \nonumber	
%&=\langle x^{t+1}-x^t,g \rangle + \frac{1}{\mu} \norm{x^{t+1}-x^t}^2 \nonumber\\
&\leq \frac{1}{\mu} \norm{x^\gt-x^{k-1}+\frac{\mu}{2} \g }^2 - \frac{\mu}{4} \norm{\g}^2 
 - \langle x^k-x^{k-1},2\eg^k \rangle +(\frac{\nup^k}{\sqrt\mu})^2 \nonumber \\
&= \langle x^\gt-x^{k-1},\g \rangle + \frac{1}{\mu} \norm{x^\gt-x^{k-1}}^2 
 - \langle x^k-x^{k-1},2\eg^k \rangle 
+(\frac{\nup^k}{\sqrt\mu})^2\nonumber\\
%&= \langle x^\gt-x^{k-1},g \rangle + \frac{1}{\mu} \norm{x^\gt-x^{k-1}}^2 - \langle x^k-x^*,2\eg^k \rangle +\frac{\nup^k}{\mu}\nonumber\\
&\leq \langle x^\gt-x^{k-1},g \rangle + \frac{1}{\mu} \norm{x^\gt-x^{k-1}}^2 
+2\nug^k\norm{x^k-x^*} +(\frac{\nup^k}{\sqrt\mu})^2. \label{eq:p1b2}
\end{align}
\fi
The last line replaces $\g= g+2\eg^k$ and uses the Cauchy-Schwartz inequality. 


Similarly we use the LLE property in Definition \ref{def:Lip} to obtain an upper bound on $ \langle x^\gt-x^{k-1},g \rangle$:
\begin{align*} 
\langle x^\gt-x^{k-1},g \rangle 	&= w^2-\norm{y-Ax^{k-1}}^2 +\norm{A(x_0-x^{k-1})}^2 \\
&\leq w^2 -\norm{y-Ax^{k-1}}^2 +\mmx\norm{x^\gt-x^{k-1}}^2,
\end{align*}
where $w=\norm{ y-Ax^\gt}$. Replacing this bound in \eqref{eq:p1b2} and cancelling $-\norm{y-Ax^{k-1}}^2$ from both sides of the inequality yields
\ifCLASSOPTIONtwocolumn
\begin{align}
&\norm{y-Ax^k}^2- 2\nug^k\norm{x^k-x^\gt}\nonumber \\ 
&\leq \left(\frac{1}{\mu}-\mmx \right)\norm{x^{k-1}-x^\gt}^2 + (\frac{\nup^k}{\sqrt\mu})^2+w^2. \label{eq:p1b3}
\end{align}
\else
\begin{align}
\norm{y-Ax^k}^2- 2\nug^k\norm{x^k-x^\gt}
\leq \left(\frac{1}{\mu}-\mmx \right)\norm{x^{k-1}-x^\gt}^2 + (\frac{\nup^k}{\sqrt\mu})^2+w^2. \label{eq:p1b3}
\end{align}
\fi
We continue to lower bound the left-hand side of this inequality:
\ifCLASSOPTIONtwocolumn
\begin{align*}
&\norm{y-Ax^k}^2- 2\nug^k\norm{x^k-x^\gt}\\
&= \norm{A(x^k-x^\gt)}^2+w^2-2\langle y-Ax^\gt, A(x^k-x^\gt)\rangle\\
&- 2\nug^k\norm{x^k-x^\gt} \\
&\geq \norm{A(x^k-x^\gt)}^2+w^2-2w \norm{A(x^k-x^\gt)}\\
&- 2\nug^k\norm{x^k-x^\gt} \\
& \geq \mmx\norm{x^k-x^\gt}^2+w^2-2(w \sqrt{\MM}+\nug^k)\norm{x^k-x^\gt}\\
&= \left(\sqrt{\mmx}\norm{x^k-x^\gt}-\frac{\nug^k}{\sqrt{\mmx}}- \sqrt{\frac{\MM}{\mmx}}w\right)^2 \\
&- (\frac{\nug^k}{\sqrt{\mmx}})^2 -(\frac{\MM}{\mmx}-1)w^2.
\end{align*}
\else
\begin{align*}
\norm{y-Ax^k}^2- 2\nug^k\norm{x^k-x^\gt}
&= \norm{A(x^k-x^\gt)}^2+w^2-2\langle y-Ax^\gt, A(x^k-x^\gt)\rangle- 2\nug^k\norm{x^k-x^\gt} \\
&\geq \norm{A(x^k-x^\gt)}^2+w^2-2w \norm{A(x^k-x^\gt)}
- 2\nug^k\norm{x^k-x^\gt} \\
& \geq \mmx\norm{x^k-x^\gt}^2+w^2-2(w \sqrt{\MM}+\nug^k)\norm{x^k-x^\gt}\\
&= \left(\sqrt{\mmx}\norm{x^k-x^\gt}-\frac{\nug^k}{\sqrt{\mmx}}- \sqrt{\frac{\MM}{\mmx}}w\right)^2 - (\frac{\nug^k}{\sqrt{\mmx}})^2 -(\frac{\MM}{\mmx}-1)w^2.
\end{align*}
\fi
The first inequality uses the Cauchy-Schwartz's and the second inequality follows from the ULE and LLE properties. Using this bound together with \eqref{eq:p1b3} we get
\ifCLASSOPTIONtwocolumn
\begin{align*}
&\left(\sqrt{\mmx}\norm{x^k-x^\gt}-\frac{\nug^k}{\sqrt{\mmx}}- \sqrt{\frac{\MM}{\mmx}}w\right)^2\\
&\leq \left(\frac{1}{\mu}-\mmx \right)\norm{x^{k-1}-x^\gt}^2 + (\frac{\nug^k}{\sqrt{\mmx}})^2+ (\frac{\nup^k}{\sqrt\mu})^2+\frac{\MM}{\mmx}w^2 \\
&\leq \left(\sqrt{\frac{1}{\mu}-\mmx} \norm{x^{k-1}-x^\gt} + \frac{\nug^k}{\sqrt{\mmx}}+ \frac{\nup^k}{\sqrt\mu}+\sqrt{\frac{\MM}{\mmx}}w \right)^2.
\end{align*}
\else
\begin{align*}
\left(\sqrt{\mmx}\norm{x^k-x^\gt}-\frac{\nug^k}{\sqrt{\mmx}}- \sqrt{\frac{\MM}{\mmx}}w\right)^2
&\leq \left(\frac{1}{\mu}-\mmx \right)\norm{x^{k-1}-x^\gt}^2 + (\frac{\nug^k}{\sqrt{\mmx}})^2+ (\frac{\nup^k}{\sqrt\mu})^2+\frac{\MM}{\mmx}w^2 \\
&\leq \left(\sqrt{\frac{1}{\mu}-\mmx} \norm{x^{k-1}-x^\gt} + \frac{\nug^k}{\sqrt{\mmx}}+ \frac{\nup^k}{\sqrt\mu}+\sqrt{\frac{\MM}{\mmx}}w \right)^2.
\end{align*}
\fi
The last inequality assumes $\mu\leq \mmx^{-1}$ which holds since we previously assumed $\mu\leq \MM^{-1}$. As a result we deduce that
\begin{align}
\norm{x^k-x^\gt}\leq \rho \norm{x^{k-1}-x^\gt} + \nut^k + 2\frac{\sqrt{\MM}}{\mmx}w \label{eq:p1b4}
\end{align}
for $\rho$ and $\nut^k$ defined in Theorem \ref{th:inexactLS1}. Applying this bound recursively (and setting $x^0=0$) completes the proof:
\begin{align*}
\norm{x^k-x^\gt}\leq \rho^k \norm{x^\gt} + \sum_{i=1}^k \rho^{k-i} \nut^i + \frac{2\sqrt{\MM}}{\mmx(1-\rho)}w.
\end{align*} 
Note that for convergence we require $\rho<1$ and therefore, a lower bound on the step size which is $\mu> (2\mmx)^{-1}$. 

\subsection{Proof of Corollary~\ref{cor:decay}}
Following the error bound \eqref{eq:errbound} derived in   Theorem~\ref{th:inexactLS1} and by setting $\nut^k\leq C r^k$ we obtain:
		\eq{
			\norm{x^{k}-x^\gt}\leq  \rho^k \left(\norm{x^\gt}+C\sum_{i=1}^k (r/\rho)^{i}  \right)+ \frac{2\sqrt{\MM}}{\mmx(1-\rho)}w,			
		}
which for $r<\rho$ it implies 		
		\eq{
			\norm{x^{k}-x^\gt}\leq 
			 \rho^k \left(\norm{x^\gt}+\frac{C}{1-r/\rho}\right)+ \frac{2\sqrt{\MM}}{\mmx(1-\rho)}w,
		}
and for $r>\rho$ implies %and following \eqref{eq:errbound} we get		
\begin{align*}
\norm{x^{k}-x^\gt}&\leq  \rho^k \norm{x^\gt}+C r^k \sum_{i=1}^k (\rho/r)^{k-i}  + \frac{2\sqrt{\MM}}{\mmx(1-\rho)}w\\
&\leq r^k \left(\norm{x^\gt}+\frac{C}{1-\rho/r}\right)+ \frac{2\sqrt{\MM}}{\mmx(1-\rho)}w,	
\end{align*}
and for $r=\rho$ we immediately get
\eq{
\norm{x^{k}-x^\gt}\leq  \rho^k \norm{x^\gt}+C k \rho^k + \frac{2\sqrt{\MM}}{\mmx(1-\rho)}w.	
}
Note that there exists a constant $c$ such that for an arbitrary small $\xi>0$ it holds $k\rho^k\leq c(\rho+\xi)^k$. Therefore we also achieve a linear convergence for the case $r=\rho$.
\subsection{Proof of Theorem \ref{th:inexactLS2}}
As before set $g= 2A^T(Ax^{k-1}-y)$ and $\g= g+2\eg^k$ for some bounded gradient error vector $\eg^k$ i.e. $\norm{\eg^k}\leq \nug^k$. Note that 
here the update rule of Algorithm \eqref{eq:inIP2} uses the  $(1+\epsilon)$-approximate projection  which by definition \eqref{eq:eproj} implies
\ifCLASSOPTIONtwocolumn
\begin{align*}
&\norm{x^k-x^{k-1}+\frac{\mu}{2} \g }^2 =  \norm{\pp^{\epsilon}_{\Cc}(x^{k-1}-\frac{\mu}{2} \g)-x^{k-1}+\frac{\mu}{2} \g }^2\\
&\leq  (1+\epsilon)^2\norm{\pp_{\Cc}(x^{k-1}-\frac{\mu}{2} \g)-x^{k-1}+\frac{\mu}{2} \g }^2\\
&\leq \norm{x^\gt-x^{k-1}+\frac{\mu}{2} \g }^2 + \phi(\epsilon)^2\frac{\mu^2}{4}\norm{\g}^2
\end{align*}
\else
\begin{align*}
\norm{x^k-x^{k-1}+\frac{\mu}{2} \g }^2 &=  \norm{\pp^{\epsilon}_{\Cc}(x^{k-1}-\frac{\mu}{2} \g)-x^{k-1}+\frac{\mu}{2} \g }^2\\
&\leq  (1+\epsilon)^2\norm{\pp_{\Cc}(x^{k-1}-\frac{\mu}{2} \g)-x^{k-1}+\frac{\mu}{2} \g }^2\\
&\leq \norm{x^\gt-x^{k-1}+\frac{\mu}{2} \g }^2 + \phi(\epsilon)^2\frac{\mu^2}{4}\norm{\g}^2
\end{align*}
\fi
where $\phi(\epsilon):=\sqrt{2\epsilon+\epsilon^2}$. For the last inequality we replace $\pp_{\Cc}(x^{k-1}-\frac{\mu}{2} \g)$ with two feasible points $x^\gt,x^{k-1}\in \Cc$. 

As a result by only replacing $\nug^k$ with $\mu\phi(\epsilon)\norm{\g}/2$, we can follow identical steps as for the proof of Theorem \ref{th:inexactLS1} up to \eqref{eq:p1b4}, revise the definition of $\nut^k:={2\nug^k}/{\mmx} + {\sqrt{\mu}\phi(\epsilon)\norm{\g}}/(2\sqrt{{\mmx}})$ and write
\ifCLASSOPTIONtwocolumn
\begin{align*}
\norm{x^k-x^\gt}\leq& \sqrt{\frac{1}{\mu\mmx}-1} \norm{x^{k-1}-x^\gt} \\
&+ \frac{2\nug^k}{\mmx} +\frac{\phi(\epsilon)}{2} \sqrt{\frac{\mu}{\mmx}}\norm{\g} + 2\frac{\sqrt{\MM}}{\mmx}w. 
\end{align*}
\else
\begin{align*}
\norm{x^k-x^\gt}\leq \sqrt{\frac{1}{\mu\mmx}-1} \norm{x^{k-1}-x^\gt} 
+ \frac{2\nug^k}{\mmx} +\frac{\phi(\epsilon)}{2} \sqrt{\frac{\mu}{\mmx}}\norm{\g} + 2\frac{\sqrt{\MM}}{\mmx}w. 
\end{align*}
\fi
Note that so far we only assumed $\mu\leq \MM^{-1}$. 

On the other hand by triangle inequality we have
\begin{align*}
	\norm{\g}&\leq \norm{g}+2\nug^k\\
	&\leq 2\norm{A^TA(x^{k-1}-x^\gt)}+2\norm{A^T(y-Ax^\gt)}+2\nug^k \\
	&\leq 2\sqrt \MM\vertiii{A}\norm{(x^{k-1}-x^\gt)}+2\vertiii{A}w+2\nug^k\\
	&\leq 2\sqrt{ 1/\mu}\vertiii{A}\norm{(x^{k-1}-x^\gt)}+2\vertiii{A}w+2\nug^k.
\end{align*}
The third inequality uses the ULE property and the last one holds since $\mu\leq \MM^{-1}$.
Therefore, we get
\ifCLASSOPTIONtwocolumn
\begin{align*}
&\norm{x^k-x^\gt}\leq
\left(\sqrt{\frac{1}{\mu\mmx}-1}+ \phi(\epsilon)\frac{\vertiii{A}}{\sqrt{\mmx}}\right) \norm{x^{k-1}-x^\gt} \\
&+ \left( \frac{2}{\mmx} +\phi(\epsilon){\sqrt{\frac{\mu}{\mmx}}}\right) \nug^k 
+ \left( 2\frac{\sqrt{\MM}}{\mmx}+ \phi(\epsilon)\sqrt{\frac{\mu}{\mmx}} \vertiii{A} \right)w. 
\end{align*}
\else
\begin{align*}
\norm{x^k-x^\gt}\leq&
\left(\sqrt{\frac{1}{\mu\mmx}-1}+ \phi(\epsilon)\frac{\vertiii{A}}{\sqrt{\mmx}}\right) \norm{x^{k-1}-x^\gt} \\
&+ \left( \frac{2}{\mmx} +\phi(\epsilon){\sqrt{\frac{\mu}{\mmx}}}\right) \nug^k 
+ \left( 2\frac{\sqrt{\MM}}{\mmx}+ \phi(\epsilon)\sqrt{\frac{\mu}{\mmx}} \vertiii{A} \right)w. 
\end{align*}
\fi
Based on assumption $\phi(\epsilon)\frac{\vertiii{A}}{\sqrt{\mmx}}\leq \delta$ of the theorem  we can deduce
\ifCLASSOPTIONtwocolumn
\begin{align*}
\norm{x^k-x^\gt}\leq&
\rho \norm{x^{k-1}-x^\gt} + \left( \frac{2}{\mmx} +\frac{\sqrt \mu}{\vertiii{A}} \delta\right) \nug^k \\
&
+ \left( 2\frac{\sqrt{\MM}}{\mmx}+\sqrt{\mu}\delta \right)w 
\end{align*}
\else
\begin{align*}
\norm{x^k-x^\gt}\leq
\rho \norm{x^{k-1}-x^\gt} + \left( \frac{2}{\mmx} +\frac{\sqrt \mu}{\vertiii{A}} \delta\right) \nug^k 
+ \left( 2\frac{\sqrt{\MM}}{\mmx}+\sqrt{\mu}\delta \right)w 
\end{align*}
\fi
where $\rho=\sqrt{\frac{1}{\mu\mmx}-1}+\delta$.

Applying this bound recursively (and setting $x^0=0$) completes the proof:
\eq{
\norm{x^{k}-x^\gt}\leq  \rho^k \norm{x^\gt}+\kappa_g \sum_{i=1}^k \rho^{k-i} \nug^i+ \frac{\kappa_w}{1-\rho}w
}
for $\kappa_g, \kappa_w$ defined in Theorem \ref{th:inexactLS2}. The condition for convergence is $\rho<1$ which implies $\delta<1$ and a lower bound on the step size which is $\mu> (\mmx+(1-\delta)^2\mmx)^{-1}$. 
		

\section{Examples and applications}
\label{sec:example}
\section{SIMULATION RESULTS}
\label{sec:examples}
This section presents simulation results of the proposed method implemented on the unicycle model example.
Each semidefinite program was prepared using a custom software toolbox and the modeling tool YALMIP \cite{lofberg2004yalmip}.
The programs are run with commercial solver MOSEK on a machine with $1$ TB availabe memory. 

\subsection{FRS Computation}
We computed the FRS for a 3$^\text{rd}$ order Taylor-expanded Dubins car as the low-fidelity model $f_s$.
Trajectories produced by this model were tracked by the unicycle model from Equation \eqref{eq:big_dyn} as the high-fidelity model $f$.
The vehicle's representation as an initial distribution $X_0 \subset X_s$, was a rectangle of length $0.2$ [m] in $x$ and width $0.1$ [m] in $y$, at $0^\circ$ initial heading, and centered at $x=-0.75$ and $y=0$.
This is the same vehicle representation shown in all previous figures.

% The error function $g$, illustrated in Figure \ref{fig:error_dynamics}, was given by:
% \begin{equation}
% \label{eq:g_definition}
% g(t,x_s) = \begin{bmatrix}
% v_\text{err}\cdot(1 - \frac{1}{2}\theta^2)  \\
% v_\text{err}\cdot(\theta - \frac{1}{6}\theta^3) \\
% \dot{\theta}_\text{err}
% \end{bmatrix}
% \end{equation}
% where $v_\text{err} = (t-1)^2$ and $\dot{\theta}_\text{err} = (t-1)^4$.
We chose $\tau_\text{stop} = \tau_\text{plan} = 0.5$ [s], so $T = 1$ [s].
The stopping time can be seen in Figure \ref{fig:error_dynamics}. 
The FRS computation took 79 hours and used a maximum of 150 GB of memory 
%on a server with 1 TB of available memory and 18 processors each running at 1.2 GHz.

\subsection{Set Intersection and Trajectory Planning}

We used the precomputed FRS for safe trajectory planning in $1000$ simulated trials in MATLAB on the aforementioned machine.
For each trial, the vehicle began at the same initial location and heading, surrounded by $1-10$ randomized obstacles and a randomly-located goal to reach.
%If the planning time took more than $\tau_\text{plan}$, the simulation paused until the computation was complete. 
%In practice, if $\tau_\text{plan}$ was exceeded the vehicle could begin braking to ensure safety.
The vehicle's initial speed, and the desired speed to maintain for the duration of the trial, were randomly chosen between $0.25$ and $0.75$ [m/s].
% The trials ran in 12.7 hours.
% Prior to running these trials, several example trials were run on a laptop with a 2.3 GHz processor and 16 GB of RAM.
% The trials run on the server were individually no faster than running on the laptop, because the set intersection optimization is a single-core process that uses very little memory. 
% Therefore, the server did not provide any significant decrease in the implemented planning time.


Obstacles were represented as line segments between $0.1$ and $0.2$[m] in length, with random location and orientation.
The obstacles were always placed between the vehicle and the goal.
We checked for crashes conservatively for each trial, by inspecting if any obstacle was within a circle circumscribing the rectangular vehicle at any point of the vehicle's trajectory. 
Using this method, \emph{no crashes were detected in any trial}.
Out of all the trials, $82\%$ reached the goal, and $15\%$ performed an emergency braking maneuver (by setting $v_\text{des} = 0$). 
The remaining 3\% hit a simulation iteration limit.
Examples of the vehicle's path from a randomly-generated trial and from two constructed emergency braking cases are shown in Figure \ref{fig:example_trial}.


\begin{figure}
\centering
\includegraphics[width=1\columnwidth]{running_examples.pdf}
\caption{The top subplot shows an example result out of the $1000$ trials.
This trial used eight randomly-generated obstacles.
The vehicle begins on the left at $x = -0.75$ and reaches a randomly-generated goal near $(2.5, 0.5)$, plotted as a blue circle.
Every $\tau_\text{plan} = 0.5$[s], the vehicle replans its trajectory, shown by an asterisk plotted on the global trajectory in blue.
The bounding box of the vehicle at each planning step is shown as a grey rectangle. In the bottom-left subplot, an obstacle was constructed between the vehicle and the goal, forcing an emergency braking maneuver. In the bottom-right subplot, an obstacle was constructed with a hole that would allow the vehicle to pass, but the set intersection result is overly conservative, resulting in a braking maneuver.}
\label{fig:example_trial}
\end{figure}

Currently, our implementation cannot consistently achieve $\tau_\text{plan} = 0.5$ [s].
Consequently, instead of replanning and driving simultaneously, we pause time every 0.5 [s] of the simulation to guarantee that the vehicle can finish replanning.
In a physical implementation, if $\tau_\text{plan}$ is exceeded, then the vehicle must emergency brake; recall that a safe braking trajectory is always available.
As shown in Figure \ref{fig:planning_time_vs_Nobs}, $\tau_\text{plan}$ scales linearly with the number of obstacles.
%Methods for reducing the set intersection to meet $\tau_\text{plan}$ will be presented in future work.

\begin{figure}
\centering
\includegraphics[scale=0.45,trim={1cm 6cm 1cm 7cm},clip]{planning_time_vs_Nobs.pdf}
\caption{The mean set intersection time (top) and trajectory optimization time (bottom) versus the number of obstacles. Over the $1000$ trials, each number of obstacles from $1$ to $10$ was used for $100$ trials. Notice that set intersection takes up to $3$[s], and scales with the number of obstacles. On the other hand, the trajectory optimization takes around $80$ [ms] and has low correlation with number of obstacles.}
\label{fig:planning_time_vs_Nobs}
\end{figure}

% \begin{figure}
% \centering
% \includegraphics[scale=0.5,trim={1cm 8cm 1cm 8cm},clip]{example_trial_bluecar.pdf}
% \caption{An example result out of the 1000 trials.
% This trial used eight randomly-generated obstacles.
% The vehicle begins on the left at $x = -0.75$ and reaches a randomly-generated goal near $(2.5, 0.5)$, plotted as a blue circle.
% Every $\tau_\text{plan} = 0.5$ [s], the vehicle replans its trajectory, shown by an asterisk plotted on the global trajectory in blue.
% The bounding box of the vehicle at each planning step is shown as a grey rectangle.}
% \label{fig:example_trial}
% \end{figure}

% \begin{figure}
% \centering
% \includegraphics[scale=0.4,trim={1cm 7cm 1cm 7cm},clip]{example_emergency_brake.pdf}
% \caption{An example of a forced emergency braking situation. The vehicle cannot find a path to the desired location (plotted as a blue circle), so it brakes.}
% \label{fig:example_emergency_brake}
% \end{figure}

% \begin{figure}
% \centering
% \includegraphics[scale=0.4,trim={1cm 7cm 1cm 7cm},clip]{example_overly_conservative.pdf}
% \caption{An example of an unnecessary emergency braking situation. The vehicle cannot find a path to the desired location despite an obviously-safe path existing, because the FRS is overly conservative.}
% \label{fig:example_overly_conservative}
% \end{figure}


\section{Technical tools}
\label{sec:tools}
\section{Tools}
\label{sec:tool}

%In this section we present our algorithm \ebdjoin.  
Before presenting our algorithm, we would like to introduce a few tools that we shall use in \ebdjoin+, including the CGK-embedding, the LSH for the Hamming distance, and an algorithm for exact edit distance computation. We list in Table~\ref{tab:notation} a set of notations that will be used in the presentation.


%\begin{adjustwidth}{-1cm}{-1cm}
\begin{table}[t]
\centering
\begin{tabular}{|p{.12\textwidth}| p{.85\textwidth}| m{.01\textwidth}|} 
\hline
Notation & Definition\\ 
\hline
$[n]$ & $[n] = \{1, 2, \ldots, n\}$ \\
\hline
$K$ & Edit distance threshold\\ 
\hline
$\S$ & The set of input strings \\ 
\hline
$s_i$ & The $i$-th string in $\S$ \\ 
\hline
$\abs{x}$ & Length of string $x$ \\ 
\hline
$n$ & Number of input strings, i.e.,  $n = \abs{\S}$\\ 
\hline
$N$ & Maximum length of strings in $\S$\\ 
\hline
$\Sigma$ & Alphabet of strings in $\S$ \\ 
\hline
$r$ & Number of CGK-embeddings for each input string\\ 
\hline
$t_i^{\ell}$ & The output string generated by the $\ell$-th CGK-embedding of $s_i$ \\ 
\hline
$z$ & Number of hash functions used in LSH for each string generated by CGK-embedding\\ 
\hline
$m$ & Length of the LSH signature\\ 
\hline
$f_j^{\ell} $ &   $f_j^{\ell} : \Sigma^N \to \Sigma^m$, the $j$-th ($j\in[z]$)  LSH function for each string generated by the $\ell$-th CGK-embedding
\\ 
\hline
$\mathcal{D}_j^{\ell}$ & The hash table corresponding to the LSH function $f_j^{\ell}$ \\ 
\hline
$\Delta$ & A parameter for dealing with shifts  \\ 
\hline
$s_{i,k}$ & The $k$-th substring of $s_i$ starting at the $((k-1)\Delta+1)$-th character\\ 
\hline
$t_{i,k}^\ell$ & The output string generated by the $\ell$-th CGK-embedding of $s_{i,k}$\\ 
\hline
$T$ & The threshold of the number of matched hash signatures for a pair of substrings\\ 
\hline
\end{tabular}
\caption{Summary of Notations}
\label{tab:notation}
\end{table}
%\end{adjustwidth}


\subsection{The CGK-Embedding}
\label{sec:CGK}

We describe the CGK-embedding in Algorithm~\ref{alg:CGK}. Below we  illustrate the main idea behind the CGK-embedding, which we believe is useful and important to understand the intuition of \ebdjoin+.  We note that the original algorithm in \cite{CGK16} was only described for binary strings, and it was mentioned that we can encode an alphabet $\Sigma$ into binary codes using $\log\abs{\Sigma}$ bits for each character.  In our rewrite (Algorithm~\ref{alg:CGK}) we choose to use the alphabet $\Sigma$ directly without the encoding.  This may give some performance gain when the size of the alphabet is small. 


\begin{algorithm}[t]
\begin{algorithmic}[1]
\Require A string $x \in \Sigma^\eta$ for some $\eta \le N$, and a random string $R \in \{0,1\}^{ 3 N |\Sigma|}$
\Ensure A string $x' \in \Sigma^{3N}$
\smallskip

%\State Pad $x$ with ``$\perp$'' in the tail to make it of length $N$; denote the new string by $\tilde{x}$

\State Interpret $R$ as a set of functions 

$\pi_1,\dots,\pi_{3N}:  \Sigma \rightarrow \{0,1\}$; for the $k$-th char $\sigma_k$ in $\Sigma$, 

$\pi_j(\sigma_k) = R[(j-1) \cdot \abs{\Sigma} + k]$
\State $i \leftarrow 1$
\State $x' \leftarrow \emptyset$ 
\For{$j \in [3N]$} 
	\If{$i\le \abs{x}$}
		\State $x' \leftarrow x' \odot x[i]$ 
		\Comment the ``$\odot$'' denotes concatenation
		\State $i \leftarrow i+\pi_j(x[i])$
	\Else
		\State $x' \leftarrow x' \odot \perp$
		\Comment ``$\perp$'' can be an arbitrary character outside $\Sigma$
	\EndIf
\EndFor 
\end{algorithmic}
\caption{CGK-Embedding($s$, $R$) \cite{CGK16}}
\label{alg:CGK}
\end{algorithm}


Let $N$ be the maximum length of all input strings in $\S$. The CGK-embedding maps a string $x \in \S$ to an output string $x' \in \Sigma^{3N}$ using a random bit string $R \in \{0,1\}^{3N\abs{\Sigma}}$.  We maintain a counter $i \in [1 .. \abs{x}]$ pointing to the input string $x$, initialized to be $1$.  The embedding proceeds by steps $j = 1, \ldots, 3N$. At the $j$-th step, we first copy $x[i]$ to $x'[j]$. Next, with probability $1/2$, we increase $i$ by $1$, and with the rest of the probability we keep $i$ to be the same.  At the point when $i > \abs{x}$, if $j$ is still no more than $3N$, we simply pad an arbitrary character outside the dictionary $\Sigma$ (denoted by ``$\perp$'' in Algorithm~\ref{alg:CGK}) to make the length of $x'$ to be $3N$.  In practice this may introduce quite some overhead for short strings in the case that the string lengths vary significantly. We will discuss in Section~\ref{sec:speedup} how to efficiently deal with input strings of very different lengths.

Now consider two input stings $x$ and $y$.  We use $i_0$ and $i_1$ as two counters pointing to $x$ and $y$ respectively.   At the $j$-th step, we first copy $x[i_0]$ to $x'[j]$, and $y[i_1]$ to $y'[j]$, and then decide whether to increment $i_0$ and $i_1$ using the random bit string $R$.  There are four possibilities: (1) only $i_0$ increments; (2) only $i_1$ increments; (3) both $i_0$ and $i_1$ increment; and (4) neither $i_0$ nor $i_1$ increments.  Let $d = i_0 - i_1$ be the position shift of the two counters/pointers on the two strings.  Note that if $x[i_0] = y[i_1]$, then only the cases (3) and (4) can happen, so that $d$ will remain the same.  Otherwise if $x[i_0] \neq y[i_1]$, then each case can happen with probability $1/4$ -- whether $i_0$ or $i_1$ will increment depends on the two random hash values $\pi_j(x[i_0])$ and $\pi_j(y[i_1])$. Thus with probability $1/4$, $1/2$ and $1/4$, the value $d$ will increment, remain the same, or decrement, respectively.  Ignoring the case when the value $d$ remains the same, we can view $d$ as a (different) {\em simple random walk} on the integer line with $0$ as the origin.  

We now try to illustrate the high level idea of why CGK-embedding gives an $O(K)$ distortion.  Let $u = \abs{x}$ and $v = \abs{y}$.  Suppose that at some step $j$, letting $p = i_0(j)$ (the value of $i_0$ at step $j$) and $q = i_1(j)$, we have two tails $x[p .. u] = \alpha \circ \tau$ and $y = y[q .. v] = \tau$ where $\alpha, \tau$ are two substrings and $\abs{\alpha} = k \le K$. That is, we have $k$ consecutive deletions in the optimal alignment of the two tails.  Now if after a few random walk steps, at step $j' > j$, we have $p' = i_0(j') \ge p + k$, $q' = i_1(j') \ge q$ and $p' - q' = (p - q) + k$, then the two tails $x[p'..u]$ and $y[q'..v]$ can be perfectly aligned, and consequently the pairs of characters in the output strings $x', y'$ will always be the same; in other words, they will {\em not} contribute to the Hamming distance from step $j'$. 

Now observe that since the value of $d$ changes according to a simple random walk, by the theory of random walk, with probability $0.999$ it takes at most $O(k^2)$ steps for $d$ to go from $(p - q)$ to $(p' - q')$ where $\abs{(p - q) - (p' - q')} = k$.  Therefore the number of steps $j$ where $x'[j] \neq y'[j]$ is bounded by $O(k^2)$.  
%In general, every time $k$ edits occur when ``walking'' on $x$ and $y$, with probability $0.999$ it will take at most $O(k^2)$ steps to ``resolve'' the differences and go back to the right alignment track.  
This is roughly why $\text{Ham}(x', y')$ can be bounded by $O(K^2)$ if $\text{ED}(x, y) \le K$, and consequently the distortion can be bounded by $O(K)$.

%On the other hand, it is easy to show that $\text{Ham}(x', y') \ge \text{ED}(x, y) / 2$ holds with probability $1 - o(1)$. Combining the two we conclude that the embedding has an $O(K)$ distortion.

\paragraph{Small Distortion is Good for Edit Similarity Join}  We now explain why the distortion of the embedding matters. If we have an embedding $f$ such that for any pair of input strings $(x, y)$, the distortion of the embedding is upper bounded by $D$, then the set $\{(x, y)\ |\ \text{Ham}(f(x), f(y)) \le D \cdot K\}$ will include all pairs $(x, y)$ such that $\text{ED}(x, y) \le K$.  Therefore a small $D$ can help to reduce the number of false positives, and consequently reduce the verification time which typically dominates the total running time.

\medskip

\noindent{\bf Why CGK-embedding Does Better in Practice?}
Although the worst-case distortion of CGK-embedding can be large when $\text{ED}(x, y)$ is large, we have observed that its practical performance on the datasets that we have tested is much better.  While it is difficult to fully understand this phenomenon without a thorough investigation of the actual properties of the datasets, we can think of the following reasons.

First, if a set of $z$ edits fall into an interval of length $O(z)$, {\em and} the difference between the numbers of insertions and deletions among the $z$ edits is at most $O(\sqrt{z})$ (substitutions do not matter), then with probability $0.999$ after $O(z)$ walk steps the random walk will re-synchronize.  In other words, the distortion of the embedding is $O(1)$ with probability $0.999$ on this cluster of edits.  We have observed that in our protein/genome datasets (Section~\ref{sec:setup}) the edits are often clustered into small intervals; in each cluster most edits are substitutions, and consequently the difference between the numbers of insertions and deletions is small. 

Second, in the task of differentiating similar pairs of strings and dissimilar pairs of strings, as long as the distance gap between strings is preserved after the embedding, the distortion of CGK-embedding will not affect the performance by much. In particular, when the distortion of CGK-embedding is $\Theta(k)$ (which is very likely when edits are well separated), the embedding actually {\em amplifies} the distance gap between similar and dissimilar pairs, which makes the next LSH step easier.

%We believe that our datasets have (some of) these properties.  First, we have observed that in the protein/DNA datasets the edits are often clustered into small intervals; in each cluster most edits are substitutions, and consequently the difference between the numbers of insertions and deletions is small.  Second, in our datasets, in most cases, the distance gaps between similar pairs and dissimilar pairs are large so that the distortion of CGK is not a big issue.
 
To further improve the effectiveness of the CGK-embedding, we run the embedding multiple times and then take the one with the minimum Hamming distance. That is, we choose the run with the best distortion.  This is just a heuristic, and cannot improve the distortion by much in theory, but we have observed that for the real-world datasets that we have tested, repeating and then taking the minimum does help to reduce the distortion. In Figure~\ref{fig:CGK-distortion} we depicted the best distortions under different numbers of runs of the CGK-embedding on a real-world genome dataset.
%\ (see Section~\ref{sec:setup} for a detailed description of our datasets).  

\begin{figure}[t]
\centering
\includegraphics[height = 1.6in]{distortion.eps}
\caption{The CDF of the best distortions of 1000 random pairs strings from the \genoa\ dataset, under different numbers of CGK-embeddings (value $r$)}
\label{fig:CGK-distortion}
\end{figure}


\subsection{LSH for the Hamming Distance}
\label{sec:LSH}

Our second tool is the LSH for the Hamming distance, introduced in~\cite{IM98,GIM99} for solving nearest neighbor problems.  We first give the definition of LSH.  By $h \in_r \H$ we mean sampling a hash function $h$ randomly from a hash family $\H$. 

\begin{definition}(Locality Sensitive Hashing \cite{GIM99})
Let $U$ be the item universe, and $d(\cdot,\cdot)$ be a distance function. We say a hash family $\mathcal{H}$ is $(l, u, p_1,p_2)$-sensitive if for any $x, y \in U$
\begin{itemize}
\item if $d(x, y)\le l$, then $\Pr_{h\in_r \H}[h(x) = h(y)] \ge p_1$,
\item if $d(x, y)\ge u$, then $\Pr_{h\in_r \H}[h(x) = h(y)] \le p_2$.
\end{itemize}
\end{definition}

We will make use of the following vanilla version of LSH for the Hamming distance.
\begin{theorem}(Bit-sampling LSH for Hamming  \cite{GIM99})
For the Hamming distance over vectors in $\Sigma^N$, for any $d > 0, c > 1$, the family $$\mathcal{H}_N = \{v_i: v_i(b_1, \dots, b_N) = b_i\ |\ i \in [N]\}$$ is  $(d, cd, 1-{d}/{N}, 1-{cd}/{N})$-sensitive.
\end{theorem}

We can use the standard AND-OR amplification method\footnote{See, for example, \url{https://en.wikipedia.org/wiki/Locality-sensitive_hashing}.} to amplify the gap between $p_1$ and $p_2$.
We first concatenate $m$ ($m$ is a parameter) hash functions, and define 
\begin{eqnarray*}
f = h_1 \circ h_2 \circ \ldots \circ h_m \text{ where } \forall i \in [m], h_i \in_r \H,
\end{eqnarray*}
such that for $x \in U$, $f(x) = (h_1(x), h_2(x), \ldots, h_m(x))$ is a vector of $m$ bits.  Let $\F(m)$ be the set of all such hash functions $f$.
We then define (for a parameter $z$)
\begin{eqnarray*}
g = f_1 \vee f_2 \vee \ldots \vee f_z, \text{ where } \forall j \in [z], f_j \in_r \F(m),
\end{eqnarray*}
such that for $x, y \in U$ $g(x) = g(y)$ if and only if there is at least one $j \in [z]$ for which $f_j(x) = f_j(y)$.  Easy calculation shows that $g$ is
\begin{eqnarray*}
\left(d, cd, 1 - \left(1 - \left({d}/{N}\right)^m\right)^z, 1 - \left(1 - \left( {cd}/{N} \right)^m \right)^z \right) \text{-sensitive.}
\end{eqnarray*}
 By appropriately choosing the parameters $m$ and $z$,  we can amplify the gap between $p_1$ and $p_2$ so as to reduce the numbers of false positives/negatives.  
%Note that the total number of primitive hash functions $h \in \H$ we have used is $\beta = m \cdot z$, which will contribute to the running time.  

%We comment that this vanilla version of LSH is enough for our applications, and its good time performance fits our needs well.

\subsection{Exact Edit Distance Computation for Verification}
\label{sec:exact-ED}

We will use the classic algorithm by Ukkonen~\cite{Ukkonen85} for computing threshold edit distance as our verification algorithm.  In the high level, defining the diagonal $d$ of a matrix $D$ to be the set of all entries $D_{i, i+d}$,
the algorithm tries to fill {\em a subset of} the entries in the $2K+1$ diagonals $\{-K, \ldots, K\}$ in the $N \times N$ dynamic programming matrix, which are sufficient to give the final output.  The worst-case running time of this algorithm is $O(N K)$. But if one of the strings is a random string, then the algorithm only uses $O(N + K^2)$ time in expectation \cite{Myers86}.   In \cite{Myers86}, Myers also proposed another algorithm using suffix-tree whose worse-case running time is $O(N+K^2)$. However, we found that suffix-tree is computational expensive in practice and has no advantage over a ``brute force'' table filing \cite{Ukkonen85}. 

We also note that Belazzougui and Zhang~\cite{BZ16} (and independently, Chakraborty et al.~\cite{CGK16b}) showed that the $O(N + K^2)$ running time is also achievable in the simultaneous streaming model where we can only scan each string once in the coordinated fashion.  However, the algorithms in \cite{BZ16,CGK16b} still needs to use suffix-tree.  Chakraborty et al.~\cite{CGK16b} also proposed an algorithm with $N + O(K^3)$ running time in the simultaneous streaming model {\em without} using suffix-tree, but this bound would be large when the distance threshold $K$ is large, say, $20\%$ of the string length $N$.

In an earlier version of this paper~\cite{ZZ17} we used the algorithm in~\cite{LDW11} for computing edit distance in the verification step.  We later found that it is more efficient to use Ukkonen's algorithm.





\end{document}
