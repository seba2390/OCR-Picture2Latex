In this section we will introduce the network structure we use for our ranking scheme as well as our approach. 

\begin{lemma}[{\tt League Network} definition]
The league network $\net=(\nodes,\edges)$, is a weighted directed network where an edge $\edge_{i,j} \in \edges$ points from node $i\in \nodes$ to node $j\in \nodes$ iff team $j$ has won against team $i$.  The edge weight $\weight_{\edge_{i,j}}$ is equal to the point differential of the corresponding game.
\label{def:net}
\end{lemma}  

The league network $\net$ captures ``who-wins-against-whom'' and by what margin.  
With $\bm{\beta}$ being a vector whose $i$-th element captures external (i.e., irrelevant from the network structure) factors affecting the strength of team $i$ (e.g., total budget, number of pro-ball players in the roster, total offense, total defense etc.), the PageRank of $\net$ is given by: 

\begin{equation}
\bm{\pi} = D(D-\alpha A)^{-1}\bm{\beta}
\label{eq:pr}
\end{equation}

\noindent where $A$ is the adjacency matrix of $\net$, $\alpha$ is a parameter (a typical value of which is 0.85) and $D$ is a diagonal matrix where $d_{ii} = \max(1,k_{i,out})$, with $k_{i,out}$ being the out-degree of node $i$.  
While in most practical cases PageRank considers only the network structure, Eq. (\ref{eq:pr}) is able to take into consideration - if needed - not only the network structure but external information that affect the ``power'' of team $i$ through vector $\bm{\beta}$.  
Figure \ref{fig:net} depicts the network obtained using all the games from the last NFL regular season.  
The edge width is proportional to the score differential in the corresponding game, while the vertex size is proportional to the node's PageRank score. 

\begin{figure}[t]
\begin{center}
\includegraphics[scale=0.32,angle=270]{plots/net.pdf}%\vspacecap
 \caption{The directed NFL League Network for the regular season 2015-2016.  More ``powerful'' nodes are represented with larger vertex size.}
 \label{fig:net}
\end{center}
\vspace{-0.2in}
\end{figure}


The rationale behind using PageRank is that compared to simple team standings that capture the fraction of wins for a team, PageRank further indirectly considers the strength of the opponent. 
In other words, it does not only matter to win but also to win an important team in order to improve in the rankings.  
In what follows we evaluate and delve into the details of the performance of {\method}.