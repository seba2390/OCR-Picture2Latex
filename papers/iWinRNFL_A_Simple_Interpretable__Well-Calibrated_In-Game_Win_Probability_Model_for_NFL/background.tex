\section{Background}
\label{sec:background}

Predicting the outcome of a sports game - and in particular American football in our case - has been of interest for several decades now. 
For example, Stern \cite{stern91} used data from the 1981, 1983 and 1984 NFL seasons and found that the distribution of the win margin follows a normal distribution with mean equal to the pregame point spread and standard deviation a little less than 14 points.  
He then used this observation to estimate the probability distribution of the number of games won by a team.  
The most influential work in the space, Win Probability Added, has been developed by Burke \cite{wpa} who uses various variables such as field position, down etc. to predict the win probability added after every play.  
This work forms the basis for ESPN's prediction engine, which uses an ensemble of machine learning models.   
Inspired by the early work and observations from Stern Winston developed an in-game win-probability model \cite{winston2012mathletics}, which was further adjusted from Pro-Football Reference to form their P-F-R win probability model using the notion of expected points \cite{pfrmodel}.  
More recently, Lock and Nettleton \cite{lock2014using} provided a random forest model for in-game win probability, while Shneider \cite{gambletron} created a system that uses real-time data from betting markets to estimate win probabilities. 
Finally, similar in-game win probability models exist for other sports (e.g., \cite{buttrey2011estimating,stern1994brownian}). 

One of the main reasons we introduce {\method} despite the presence of several in-game win probability models is the fact that the majority of them are hard to reproduce either because they are proprietary or because they are not described in great detail.  
Furthermore an older model, might not be applicable anymore ``as-is'' given the changes the game undergoes over the years (e.g., offenses rely much more on the passing game today as compared to a decade ago). 
Our goal with this work is twofold; (i) to provide an {\em open}\footnote{Source code and data will be made publicly available.}, simple, yet well-calibrated, in-game win probability model for the NFL, and (ii) to emphasize on the appropriate ways to evaluate such models. 
%The latter has been largely ignored in the current literature.  