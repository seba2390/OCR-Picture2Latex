In this section we evaluate {\method} through its ability to predict upcoming match-ups.  
In particular, let us assume that we want to predict the results of week $\tau$.  
Using the games up to that week, i.e., between week 1 and week $\tau-1$, we build the league network $\net_{\tau-1}$.  
This network will provide us with a ranking $\bm{\pi}(\tau-1)$. 
The prediction algorithm simply goes over this ranking and predicts as the winner of the match-up $i$-VS-$j$ team $i$, if $\pi_i(\tau-1) > \pi_j(\tau-1)$.  

Table \ref{tab:results} depicts the accuracy results obtain on the NFL dataset.  
We also present the accuracy of a similar baseline algorithm where the ranking is based on the winning percentage of the team.  
As we can see {\method} outperforms the baseline that uses simply the winning percentages.  
The overall accuracy of {\method} is 59\% with a standard error of 1.2\%, while that of the winning percentages is only 52\% with a standard error of 1.2\%.  
We would like to emphasize here that a 59\% accuracy is very close to the current state-of-the-art.  
In particular, during the last two years that Microsoft's Cortana \cite{cortana} is used to predict NFL game outcomes, the corresponding accuracy is not statistically significant different compared to the {\method}.  
More specifically for the season 2014-2015, Cortana exhibited a 66.4\% accuracy with a standard error of 2.9\% (compared with PageRank's 65.8\% with a standard error of 3\%), while for 2015-2016 Cortana's performance was 62.9\% with standard error 3\% (contrast with {\method}'s 59\% within 3\% standard error).  
Finally, the NFL league network is disconnected for the two first weeks of the season, and hence, {\method} can be confidently used for ranking teams after week 3.  
The results presented in Table \ref{tab:results} correspond to $\bm{\beta} = \bm{1}$.  
We have performed experiments where $\beta_i$ is set equal to the winning percentage of team $i$ and the results were indistinguishable.  

\begin{table}[ht!]
\centering
\begin{tabular}{c||c|c}
Year & {\method} & Winning Percentage\\
\hline
\hline
%05/30-05/31& 2& Art \& Craft Festival\\
% \hline
2009 & 0.64 & 0.57 \\ %0.033 \\ % 0.57 0.034
\hline
2010 & 0.56 & 0.5 \\ %0.034 \\ % 0.5 0.035
\hline 
2011 & 0.64 & 0.58 \\ %0.033\\  % 0.58 0.034
\hline
2012 & 0.65 & 0.56\\ %0.033 \\ % 0.56 0.035
\hline 
2013 & 0.56 & 0.5 \\ %0.034 \\ % 0.5 0.035
\hline
2014 & 0.66 & 0.56\\ %0.033 \\ % 0.56 0.034
\hline
2015 & 0.59 & 0.53 \\ %0.034\\ % 0.53 0.035
 \hline
\end{tabular}
\caption{Prediction accuracy with the {\method} is better as compared to that  of the simply winning percentages for each of the last 7 NFL seasons.}
\label{tab:results}
\end{table}

For the case of the NBA league network for the season 2014-15, {\method} provided an accuracy of 67\%, while the baseline method came close to it with a 66\% accuracy.  
Compared to the NFL case, NBA teams play many more games, and hence, despite a few ``surprising'' results, overall even the simple winning percentage provides a fairly good power ranking.  
In order to further explore this we compute the accuracy of {\method} and the baseline method in 4 different parts of the season.  
In particular, we divide the season into 4 parts, each of which covers approximately a 40-day period, and we calculate the corresponding accuracy.  
The results are presented in Figure \ref{fig:nba}.  
As we can see the performance gap is much larger during the beginning of the season, when the winning percentage of the teams has not converge to the real one.  
Later in the season the gap is reduced and the two methods provide essentially indistinguishable performance.  
With respect to the absolute performance, during the last quarter of the season the prediction accuracy is the highest, possibly due to the fact that the ranking (either through {\method} or through the winning percentage) has converged to one that reflects the actual power of the teams.  

\begin{figure}[t]
\begin{center}
\includegraphics[scale=0.35]{plots/nba.pdf}%\vspacecap
 \caption{Even though overall {\method} provides only marginal benefits over the winning percentage ranking, it is able to provide a good ranking early in the season.}
 \label{fig:nba}
\end{center}
\vspace{-0.2in}
\end{figure}

One of the problematic structures for PageRank are cycles in the network.  
In our case this corresponds to scenarios where team A has beaten team B, B has beaten team C and C has beaten team A. 
An acyclic network would provide a total order ranking, which in theory would have the highest accuracy.  
However, this is not the case in reality and cycles exists.  
Actually, given the fact that an NFL season consists of much fewer games than an NBA season (and not every teams plays with all the other teams), such cycles are expected to be less common.  
Hence, in what follows we examine the impact of the cycles on the performance of {\method}.  

In particular, we first calculate the minimum number of edges that one needs to remove from the network in order for it to be acyclic.  
This corresponds to the minimum feedback arc set problem \cite{eades1993fast}.  
Consequently, we compute the correlation between the fraction of edges in the minimum arc set and the difference in the performance between {\method} and the baseline approach.  
In particular, we compute the minimum arc set for the network after every game week for NFL and after every game day for NBA.  
Obtaining the correlation between the minimum arc set and the difference in the accuracy of {\method} and the baseline gives as {\bf -0.19 (p-value = 0.015) for NBA} and {\bf -0.21 (p-value = 0.039) for NFL}.  
The negative (and significant) correlation essentially means that {\method} is expected to outperform the winning percentage ranking as the network includes fewer directed cycles.  

Summarizing our results, we can say that {\method} is able to capture the interactions between teams better than the simple winning percentage ranking used in determining the teams that advance to the playoffs. 
In particular, in the case of NFL {\method} outperforms the baseline by approximately 7\%.  
In the case of NBA, the two approaches perform similar to each other, however, {\method} is able to converge to the appropriate ranking earlier in the season.  

% NBA -0.19 (0.01462)
% NFL -0.21 (0.03882)

