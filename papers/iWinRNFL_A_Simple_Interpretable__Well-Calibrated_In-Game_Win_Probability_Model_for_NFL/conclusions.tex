\section{Conclusions}
\label{sec:conclusions}

In this paper, motivated by several recent comebacks that seemed improbable at the moment, we designed {\method}, a simple generalized linear model for in-game win probabilities in NFL. 
{\method} uses 10 independent variables to assess the win probability of a team for any given state of the game. 
Our evaluations indicate that the probabilities provided by our model are consistent and well-calibrated. 
One crucial point is that the developed models need to be re-evaluated frequently.  
The game changes rapidly both due to changes in the rules but also because of changes in players skills or even due to analytics (e.g., see the explosion of three-point shots in basketball, or the number of NFL teams that are now run a pass-first offense).  
Hence, the details of the model can also change due to these changes.  

Finally, win probability models, while currently are mainly used from media in order to enhance sports storytelling, can form a central component for the evaluation of NFL players. 
In particular, the added win probability from each player can form the dependent variable in a (adjusted) plus-minus type of regression. 
In the future we plan to explore similar applications of {\method} related with personnel decision, in-game adjustments etc.