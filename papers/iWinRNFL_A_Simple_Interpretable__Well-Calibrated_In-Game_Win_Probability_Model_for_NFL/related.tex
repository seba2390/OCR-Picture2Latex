Various approaches for ranking teams have appeared through the years.  
The simplest one - that is actually used in professional sports as the basis for advancing to playoffs - is the winning percentage, i.e., simply the fraction of games won by a team.  
Other ranking techniques consider the strength of the opponents and are based on the assumption that not all the wins are equal (e.g., \cite{rpi1}).  
One of the most well-known sports ranking methods is Elo ranking \cite{elo1978rating}, which assigns teams with an initial number of ``rating credits'' and based on the outcome of the games (compared to the expected one) these credits are exchanged between the teams.  
While the dominant network-based approach for ranking teams are based on PageRank, Park and Newman \cite{park2005network} proposed an interesting approach for cases where not every team plays against every other team in the league.  
They introduce the idea of ``indirect wins'', i.e., directed paths of length $k$, and apply a method that resembles Katz centrality to rank the teams.  
A comparison of various ranking systems can be found in \cite{barrow13,chartier11}, while a comprehensive list of the corresponding literature is compiled by Wilson \cite{wilson}.

At a tangential line of research hybrid voting-ranking systems, such as the Borda count and the Condorcet method \cite{stahl06}, have been utilized in the sports industry.  
Variations of these methods are used for handing the Heisman Trophy as well as the NBA's MVP. % Most Valuable Player.  
The Associated Press is also using a similar voting/ranking scheme to obtain the college teams ranking \cite{ap-ncaa}.  
Finally, probabilistic models for modeling the intransitivity in game data for predicting the outcome of match-up have also been recently proposed \cite{chen2016modeling,chen2016kdd}.