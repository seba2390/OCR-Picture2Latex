\section{Introduction}
\label{sec:intro}

In-game win probability models provide the likelihood that a certain team will win a game given the current state of the game.  
Such models have become very popular during the last few years, mainly because they can provide the backbone for in-game decisions but also because they can potentially improve the viewing experience of the fans.  
Among other events this year's super bowl sparked a lot of discussion around the validity and accuracy of these models \cite{ringer17} since at some point late in the third quarter the probability given by some of these models to New England Patriots to win was less than 0.1\% or in other words a 1 in 1,000 Super Bowls comeback \cite{statsbylopez}.  
Clearly the result of the game was not the one projected at that point and hence, critiques of these models appeared.  

In this paper, we present the design of an in-game win probability model for NFL ({\method}) focusing on its appropriate evaluation.  
While {\method} is simple - at its core is based on a generalized linear model - our evaluations show that it is well-calibrated over the whole range of probabilities.  
Our model was trained using NFL play-by-play data for the past 8 seasons.  
The fact that New England Patriots won the Super Bowl even though they were given just 0.1\% probability to do so at the end of the third quarter, is not a {\em failure of the math}.  
In fact, this number tells us that roughly for every 1,000 times that a team is in the same position there will be 1 instance where the trailing team will win.  
Of course, the order with which we observe this instance is arbitrary - it can be the first time we ever observe this game setting (i.e., a team trailing with 25 points by the end of the third quarter) - which intensifies the opinion that {\em math failed}. 
So the real accuracy question is how well the predicted probabilities capture the actual winning chances of a team at a given point in the game.  
In other words, what fraction of the in-game instances where a team was given x\% probability of winning actually ended up with this team winning?  
Ideally, we would like this fraction to be also x\%, and we show that this is the case for {\method} over the whole range of probabilities. 

The rest of the paper is organized as follows: 
Section \ref{sec:background} briefly presents background on the win probability models for NFL.  
Section \ref{sec:model} presents our model, while Section \ref{sec:evaluation} presents the evaluation of {\method}.  
Finally, Section \ref{sec:conclusions} concludes our work and presents our future directions. 
