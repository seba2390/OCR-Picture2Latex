\begin{figure}
\small
    \centering
    \includegraphics[width=0.41\textwidth]{figures/images/factorize.pdf} 
    \vspace{-0.05in}
    \caption{\small{\textbf{Illustration of Kernel Factorization \& Reconstruction}} (1) we multiply two factorized vectors $\textbf{u}$ and $\textbf{v}$ to obtain kernel matrix. (2) we add the sparse bias matrix $\mu$ to complement non-linearity. (3) we reshape the matrix into the original kernel shape. }
    \label{fig:factorize_detail}
    \vspace{-0.3in}
\end{figure}

% \begin{figure*}[t]
% \small
% \centering
% \vspace{-0.05in}
% \begin{tabular}{c c c}
%     \small    
%     \hspace{-0.15in}
%     \includegraphics[width=0.39\textwidth]{figures/images/factorize.pdf} &
%     \hspace{-0.2in} 
%     \includegraphics[width=0.3\textwidth]{figures/images/permuted_uv.png} & 
%     \hspace{-0.15in}
%     \includegraphics[width=0.3\textwidth]{figures/images/domain_uv.png}
%     \\
%     \hspace{-0.1in} (a) Illustration of Kernel Factorization &
%     \hspace{-0.1in} (b) Label Heterogeneous Scenario &
%     \hspace{-0.1in} (c) Domain Heterogeneous Scenario

% \end{tabular}
% \vspace{-0.1in}
% \caption{\small{\textbf{Analysis of our Kernel Factorization method}} (a) describes our kernel factorization process. Following two plots show $L_2$ distance of the gradient updates of factorized parameters $\textbf{u}$ and $\textbf{v}$ in (b) label heterogeneous and (c) domain heterogeneous scenarios. }
% \label{fig:factorize}
% \vspace{-0.1in}
% \end{figure*}

