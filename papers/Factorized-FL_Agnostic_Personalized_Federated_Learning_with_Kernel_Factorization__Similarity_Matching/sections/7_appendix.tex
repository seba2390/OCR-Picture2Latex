\clearpage
\appendix
\onecolumn

\paragraph{Organization} We provide in-depth descriptions for our algorithms, experimental setups, i.e. dataset configurations, implementation \& training details, and additional experimental results \& analysis that are not covered in the main document, as organized as follows:

\begin{itemize}
    \item \textbf{Section~\ref{appdx:algorithm}}: We provide our pseudo-code algorithms for \texttt{Factorized-FL} and \texttt{Factorized-FL $\beta$}. 
    \item \textbf{Section~\ref{appdx:dataset}}: We describe dataset configurations for label- and domain-heterogenous scenario.
    \item \textbf{Section~\ref{appdx:details}} - We elaborate on detailed implementation and training details for our methods and baselines.
    \item \textbf{Section~\ref{appdx:experiment}} - We provide additional experimental results and analysis. 
\end{itemize}

\section{\texttt{Factorized-FL} Algorithms}
\label{appdx:algorithm}

In this section, we describe our pseudo-code algorithms for \texttt{Factorized-FL} and \texttt{Factorized-FL $\beta$} in Algorithm~\ref{algo:factorized_fl} and~\ref{algo:factorized_fl_beta}. Our \texttt{Factorized-FL} has strength for not only reducing the dimensionality of model parameters by factorizing them into rank 1 vector spaces and the additional highly-sparse matrices, but also effectively learning client-general and task-specific knowledge. Particularly, \texttt{Factorized-FL} transmits a small portion of the models which are a set of $\textbf{u}$ ($\mathcal{U}$) and a single vector $\textbf{v}^{L-1}$ form the second last layer of neural networks, which significantly reduces communication costs while showing strong performance in label- and domain-heterogeneous scenarios, as shown in Section~\ref{sec:exp} in the main document.


\section{Efficiently Recognizing {\em Cyclic Hyper Degrees}}
This section consists of two parts. In the first part will culminate with Theorem~\ref{thm:shiftrange} which provides an efficiently computable closed form formula for the range of values taken by contiguous sum of $N$ elements in a list $c_{i,n}$, for any $i$. In the second part we will show how to use Theorem~\ref{thm:shiftrange} to decide if a given degree sequence is a {\em cyclic hyper degree}.

The elements in the columns of $T_n$ do not change their relative position after application of a cyclic permutation when seen as a cyclic list. We shall use this property to efficiently search for possible bit subsets which may sum up to a given input degree sequence.

\subsection{Contiguous Sum of Bit Lists}

\begin{definition}[Contiguous Sum]
\label{def:csum}
 Given a list $L$ of length $m$, the contiguous sum of $N$ elements in $L$ starting at the index $i\in [m]$ is defined to be
 $$\mathcal{S}(L,i,N):=\sum_{j=0}^{N-1} L(1+ ((i+j-1)\mod m)).$$
\end{definition}

The summation above treats the list $L$ as a cyclic list. Next, we prove that the contiguous sum function is a `continuous' function, this property will allow us to specify the range of sum by stating the minimum and the maximum value taken by it.
 Note that if $L$ is a $0$-$1$ list, for any index $\ell \in [m]$, 
 we have $\vert \mathcal{S}(L,\ell,N)-\mathcal{S}(L,\ell+1,N) \vert \in \{0,1\}$. This fact gives us the following property.

\begin{observation}
\label{obs:continuity}
 Let $L$ be a size $m$ list having $0$-$1$ entries and $N\in \mathbb{Z}_+$.
 If $v_i=\mathcal{S}(L,i,N)$ and $v_j=\mathcal{S}(L,j,N)$, for some $i,j\in[m]$, then 
 for every $v\in \mathbb{Z}_+$ contained between  $v_i$ and $v_j$
 there exists a $k\in [m]$  such that $\mathcal{S}(L,k,N)=v$.
\end{observation}

As the lists $c_{i,n}$ are over $0$-$1$ we get an easy relation between the maximum and minimum values taken by the contiguous sum as follows.

\begin{lemma}
\label{lem:sumcomplement}
  Let  $j\in\{0,\dots,n\}$, $i \in [n]$ and $N\in[2^n]$. The minimum of the sum of $N$ contiguous bits in a bit list $c_{i,n}$ is $m$ if and only if its maximum is $N-m$.
\end{lemma}
\begin{proof}
 Let $\overline{c}_{i,n}$ be the bit list obtained from the list $c_{i,n}$ by flipping each zero to one and vice versa. Let $\sigma_{2^{i-1}}$ be an order $2^{i-1}$ cyclic permutation, observe that $\overline{c}_{i,n}$ is equal to $\sigma_{2^{i-1}}(c_{i,n})$.
 If the minimum value is obtained at the contiguous segment which starts at the index $j$ in $c_{i,n}$, then the value $N-m$ can be obtained by the contiguous sum starting at index $\sigma_{2^{i-1}}(j)$. Finally, note that $m$ is the minimum value if and only if $N-m$ is the maximum value.
\end{proof}

Combining Observation~\ref{obs:continuity} and Lemma~\ref{lem:sumcomplement} we get the following.

\begin{lemma}
\label{lem:minmax}
 Let $N\in[2^n]$ and $m=\min_{j \in [2^n]} \mathcal{S}(c_{i,n}, j, N)$. For every value $v$ in the range $\{m,\dots, N-m\}$ there exists a $j\in [2^n]$ such that $\mathcal{S}(c_{i,n}, j, N)=v$.
\end{lemma}

The lemma above allows us to find the range of values taken by the contiguous sum by just finding the minimum value taken by it.
Next we prove a simpler lemma about the range of values taken. Using that, in Theorem~\ref{thm:shiftrange}, we will find the range of values taken by the contiguous sum of $N$ elements in any list $c_{i,n}$.

\begin{lemma}
\label{lem:shiftpower}
 For  $j\in\{0,\dots,n\}$ and $i \in [n]$, the sum of $2^j$ contiguous bits in a bit list $c_{i,n}$ takes the following values.
\begin{enumerate}
\item
\label{enum:shiftpower-one}
If $j \leq (i-1)$, then the range is $\{0, \dots,2^j \}$, and
\item
\label{enum:shiftpower-two}
If $j\geq i$, then the sum is exactly $2^{j-1}$.
\end{enumerate}
\end{lemma}
\begin{proof}
 By Lemma~\ref{lem:minmax}, it suffices to find the minimum value of contiguous sum function. Notice that we have, $c_{i,n}= (0_{\times 2^{i-1}} \cdot 1_{\times 2^{i-1}})_{\times 2^{n-i}}$, by Observation~\ref{obs:zero-one-power}.
 \begin{enumerate}
 \item
 When $j\leq (i-1)$, we can pick a block of $2^j$ zeros giving a total of zero, which is the minimum possible value.
 \item
 When $j\geq i$, let $L_k$ be a list of $2^j$ contiguous bits of $c_{i,n}$ starting at the index $k$ in $c_{i,n}$. To prove that $\sum L_k = \sum L_{k+1} $, it suffices to show that $c_{i,n}(k)=c_{i,n}(k+2^j)$. Rewriting
 $c_{i,n}=((0_{\times 2^{i-1}}\cdot 1_{\times 2^{i-1}})_{\times 2^{j-i}})_{\times 2^{n-j}}$ shows that any two indices with difference equal to $2^j$ store the same value. As the choice of $k$ was arbitrary, the contiguous sum is equal to $2^{j-1}$.
 \end{enumerate} 
\end{proof}

\begin{theorem}
\label{thm:shiftrange}
 For $i \in [n]$, $N\in[2^{n}]$ and $p=2^i$, the sum of $N$ contiguous bits in a bit list $c_{i,n}$ takes values in the range, ${\rm range}(i,N) \triangleq$
 $$ 
 \left\{
	\floor[\Big]{\frac{N}{p}} \frac{p}{2} + \max \left( (N\mod p) - \frac{p}{2}, 0 \right),
  	\cdots,
   	\floor[\Big]{\frac{N}{p}} \frac{p}{2} + \min \left( N\mod p, \frac{p}{2} \right)
 \right\}.
 $$
\end{theorem}
\begin{proof}
 For a fixed $i\in[n]$ consider the list $c_{i,n}$. Assuming that the minimum value of the range is as claimed, by Lemma~\ref{lem:minmax}, the maximum value is
\begin{align*}
   \max_{j \in [2^n]} \mathcal{S}(c_{i,n}, j, N)
   &= N-\min_{j \in [2^n]} \mathcal{S}(c_{i,n}, j, N)\\
   &=N- \left( \floor[\Big]{\frac{N}{p}} \frac{p}{2} + \max \left( (N\mod p) - \frac{p}{2}, 0 \right)\right)\\
 &= \floor[\Big]{\frac{N}{p}}p + (N\mod p) - 
 \left( 
 \floor [\Big] {\frac{N}{p}} \frac{p}{2} + \max \left( (N\mod p) - \frac{p}{2}, 0 \right)
 \right)	\\
 &= \floor[\Big]{\frac{N}{p}} \frac{p}{2} +(N\mod p) - 
 \max	\left(		(N \mod p) - \frac{p}{2}, 	0 	\right)	\\
 &= \floor[\Big]{\frac{N}{p}} \frac{p}{2} + \min \left( (N\mod p)-(N \mod p) + \frac{p}{2}, N\mod p\right)\\
 &=  \floor[\Big]{\frac{N}{p}} \frac{p}{2} + \min \left(\frac{p}{2}, N\mod p\right).
\end{align*}
 
 As proved in case~\ref{enum:shiftpower-two} of Lemma~\ref{lem:shiftpower}, the sum of $\floor{\frac{N}{p}}p$ contiguous bits is equal to $\floor{\frac{N}{p}}\frac{p}{2}$  irrespective of the starting index.
 Therefore, it suffices to find the minimum sum of $R=(N\mod 2^i)$ contiguous bits. Let $L_k$ be a list of $R$ bits occurring contiguously in $c_{i,n}$ starting at index $k$. If the first bit of $L_k$ is $1$, then $\sum L_{k+1}\leq \sum L_k$. Therefore, we can keep on increasing the value of $k$ until the first bit is zero,  without increasing the value of the contiguous sum. On the other hand, if $c_{i,n}(k-1)=0$, then $\sum L_{k-1}\leq \sum L_k$. Therefore, we can keep on decreasing the value of $k$ one at a time until $c_{i,n}(k-1)=1$, without increasing the value of the contiguous sum. Thus the minimum value of the contiguous sum is achieved when the index $k$ points to the start of any block $0_{\times 2^{i-1}}$ contained in $c_{i,n}$. The value of minimum is  $\max(R-\frac{p}{2},0)$ as the ones start appearing after $\frac{p}{2}$ indices from the start of a list $0_{\times 2^{i-1}}\cdot 1_{\times 2^{i-1}}$.  Adding it to $\floor{\frac{N}{p}}\frac{p}{2}$ gives the required minimum value.
\end{proof}

\subsection{Algorithm}
We next state a theorem which gives an equivalent definition of {\em cyclic hyper degrees}.
\begin{theorem}
\label{thm:chd}
A list $w=\{w_1,\dots,w_n\}\in \mathbb{Z}_+^n$ is a {\em cyclic hyper degree} if and only if there exist $N\in[2^n]$ and a permutation $\pi$, such that for each $i\in [n]$, $w_{\pi(i)} \in {\rm range}(i,N)$.% (see Theorem~\ref{thm:shiftrange}).
\end{theorem}

\begin{proof}
 Forward direction is a direct consequence of the definition.
 
 Using Definitions~\ref{def:chd} and~\ref{def:csum}, we get that there exist numbers $s_1,\dots,s_n \in [2^n]$ such that for each $i\in [n]$, we have $w_{\pi(i)}=\mathcal{S}(c_{i,n},s_i,N)$. Let $\Pi^{-1}=(\sigma_{s_1}^{-1},\dots,\sigma_{s_n}^{-1})$ be the list of cyclic permutations, where for each $i\in [n]$, $\sigma_{s_i}^{-1}$ is the inverse of the cyclic permutation of order $s_i$. Consider the table $\Pi^{-1}(T_n)$, by
 Theorem~\ref{thm:rotatebits}, all its rows are distinct. In particular, the first $N$ rows are distinct and their sum is $\pi(w)$. Finally, $w\in H_n$ if and only if $\pi(w)\in H_n$.
\end{proof}

Theorem~\ref{thm:shiftrange} gives us a way to efficiently find the number of bits in a contiguous sum of $N$ bits. If we know the number of distinct bit sequences that can sum up to a given vector $w\in \mathbb{Z}_+^n$, then using Theorem~\ref{thm:shiftrange} we can generate all the possible ranges of values which can be taken by each coordinate of the sum. Finally, we need to check if each coordinate of $w$ is contained in different ranges, this corresponds to finding the permutation $\pi$ in Theorem~\ref{thm:chd}. In the next lemma, we will find the number of possible  distinct bit-sequences which can sum up to a given $w$ using cyclic shifts, this corresponds to finding $N$ in Theorem~\ref{thm:chd}.
\begin{lemma}
\label{lem:set-size}
If $w=\{w_1,\dots,w_n\}\in \mathbb{Z}_+^n$ is a {\em cyclic hyper degree}, then the number of bit sequences which sum up to $w$ is an element of the set
$$\mathcal{N}_w\triangleq \{2w_i+j~:~i\in[n], j\in \{-1,0,1\}\}.$$
\end{lemma}
\begin{proof}
 As one of the coordinates of $w$, say $w_k$, is the contiguous sum of $c_{1,n}$, we need to find the number of bits which sum up to $w_k$. From the structure of $c_{1,n}$, it is easily seen that there are just three values viz. $2w_k-1,2w_k,2w_k+1$ which contain $w_k$ in their range of sums. Conversely, for any number $x$ not contained in $\{2w_i+j~:~i\in[n], j\in \{-1,0,1\}\}$, the sum of $x$ contiguous bits $c_{1,n}$ will not contain any of $w_i$, for $i\in [n]$.
\end{proof}

\begin{lemma}
\label{lem:embed}
 Given $w\in\mathbb{Z}_+^n$ and a list of integer intervals $R_1,\dots,R_n \subset \mathbb{Z}_+^2$. There exists an algorithm running in time polynomial in $n$ which correctly answers if there exists a permutation $\pi$ such that  for each $i\in [n]$, $w_{\pi(i)}\in R_i$. 
\end{lemma}
\begin{proof}
 Construct a bipartite graph $G=(A,B,E)$ on $2n$ vertices. Let $A=B=[n]$ and $(i,j)\in E$ if and only if $w_i \in R_j$. Using a polynomial time algorithm one can find if there exists a perfect matching in $G$. If there is a perfect matching then the answer is \YES, otherwise it is \NO.
\end{proof}

\begin{theorem}
 \label{thm:chd-poly}
 There is a polynomial time algorithm in $n$ which decides if a given $w\in\mathbb{Z}_+^n$ is a {\em cyclic hyper degree}.
\end{theorem}

\begin{proof}
 For each $N\in\mathcal{N}_w$, given by Lemma~\ref{lem:set-size}, and $i\in[n]$ compute ${\rm range}(i,N)$ as given by Theorem~\ref{thm:shiftrange}. Now, use Lemma~\ref{lem:embed} on these ranges of numbers and decide if $w$ is a {\em cyclic hyper degree}, if it is not then try the next number from the set $\mathcal{N}_w$. If it succeeds for at least one element of $\mathcal{N}_w$, we answer \YES, otherwise we answer \NO. Finally, note that $\vert N_w \vert\leq 3n$ and all the other steps can be performed in time which is a polynomial function of $n$. 
\end{proof}

\section{Dataset Configurations}
\label{appdx:dataset}
In this section, we describe detailed configurations for datasets that we used in label- and domain-heterogeneous scenarios.

\subsection{Label Heterogeneous Scenario}

We use CIFAR-10 and SVHN for the label-heterogeneous scenario. We first split each dataset into train, validation, and test sets for CIFAR-10 ($48,000$/$6,000$/$6,000$) and SVHN ($79,431$/$9,929$/$9,929$). We then split the train set into $K$ local partitions $\mathcal{P}_{1:20}$ ($K$=$20$) for iid partitions (all instances in each class are evenly distributed to all clients) or for the non-iid partitions (instances in each class are sampled from Dirichlet distribution with $\alpha$=$0.5$). We further permute the labels for each class per local partition $\mathcal{P}_k$ for permuted iid and permuted non-iid scenarios. We use different random seed per client, i.e. $\textit{fixed global seed} + \textit{client id}$, for example, $1234 + 0$ for Client 1 and $1234 + 19$ for Client 20. We provide permutations of labels that we used for each dataset in Table~\ref{tbl:dataset_1}.

\begin{table*}[h]
    \caption{\textbf{Label permutations for label-heterogeneous scenario} We provide permutations of labels for each dataset. These permutations are randomly generated based on different seeds, calculated by $\textit{fixed global seed} + \textit{client id}$. }
    \vspace{-0.1in}
    \resizebox{\linewidth}{!}{
    \begin{tabular}{c c c c cccccccccccccccccccc}
        \toprule 
        \midrule
        \multirow{2}{*}{Dataset} & \multirow{2}{*}{Class} & Original & \multicolumn{20}{c}{ Client No. } \\
        & & Labels  & 1 & 2 & 3 & 4 & 5 & 6 & 7 & 8 & 9 & 10 & 11 & 12 & 13 & 14 & 15 & 16 & 17 & 18 & 19 & 20 \\
        \midrule
        \midrule
        \multirow{10}{*}{CIFAR-10} & Airplane & 0 & 2 & 5 & 3 & 0 & 8 & 2 & 4 & 4 & 2 & 2 & 0 & 6 & 1 & 8 & 4 & 0 & 0 & 6 & 9 & 7 \\
        & Automobile & 1 & 8 & 4 & 1 & 5 & 1 & 8 & 1 & 0 & 9 & 7 & 7 & 3 & 6 & 2 & 0 & 4 & 1 & 8 & 1 & 4 \\
        & Bird & 2 & 3 & 0 & 5 & 3 & 4 & 9 & 9 & 5 & 4 & 1 & 5 & 5 & 3 & 0 & 1 & 6 & 3 & 7 & 6 & 5 \\
        & Cat & 3 &  5 & 9 & 0 & 7 & 9 & 3 & 8 & 8 & 7 & 6 & 2 & 7 & 8 & 7 & 6 & 7 & 4 & 4 & 3 & 8 \\
        & Deer & 4 & 6 & 2 & 6 & 9 & 6 & 5 & 2 & 1 & 0 & 9 & 6 & 1 & 7 & 3 & 7 & 5 & 5 & 3 & 8 & 9 \\
        & Dog & 5 & 4 & 1 & 4 & 8 & 5 & 6 & 6 & 6 & 3 & 4 & 9 & 4 & 4 & 4 & 5 & 9 & 8 & 1 & 2 & 0 \\
        & Frog & 6 & 9 & 3 & 2 & 1 & 2 & 0 & 3 & 2 & 6 & 3 & 3 & 0 & 5 & 6 & 2 & 1 & 7 & 2 & 5 & 1 \\
        & Horse & 7 & 0 & 7 & 9 & 4 & 3 & 7 & 0 & 3 & 5 & 0 & 1 & 2 & 9 & 5 & 3 & 2 & 9 & 5 & 0 & 6 \\
        & Ship & 8 & 1 & 8 & 7 & 6 & 7 & 4 & 5 & 7 & 8 & 5 & 8 & 9 & 0 & 9 & 8 & 3 & 6 & 0 & 7 & 2 \\
        & Truck & 9 & 7 & 6 & 8 & 2 & 0 & 1 & 7 & 9 & 1 & 8 & 4 & 8 & 2 & 1 & 9 & 8 & 2 & 9 & 4 & 3 \\
        \midrule
        \multirow{10}{*}{SVHN} & Digit 0 & 10 & 2 & 5 & 3 & 0 & 8 & 2 & 4 & 4 & 2 & 2 & 0 & 6 & 1 & 8 & 4 & 0 & 0 & 6 & 9 & 7 \\
        & Digit 1 & 1 & 8 & 4 & 1 & 5 & 1 & 8 & 1 & 0 & 9 & 7 & 7 & 3 & 6 & 2 & 0 & 4 & 1 & 8 & 1 & 4 \\
        & Digit 2 & 2 & 3 & 0 & 5 & 3 & 4 & 9 & 9 & 5 & 4 & 1 & 5 & 5 & 3 & 0 & 1 & 6 & 3 & 7 & 6 & 5 \\
        & Digit 3 & 3 &  5 & 9 & 0 & 7 & 9 & 3 & 8 & 8 & 7 & 6 & 2 & 7 & 8 & 7 & 6 & 7 & 4 & 4 & 3 & 8 \\
        & Digit 4 & 4 & 6 & 2 & 6 & 9 & 6 & 5 & 2 & 1 & 0 & 9 & 6 & 1 & 7 & 3 & 7 & 5 & 5 & 3 & 8 & 9 \\
        & Digit 5 & 5 & 4 & 1 & 4 & 8 & 5 & 6 & 6 & 6 & 3 & 4 & 9 & 4 & 4 & 4 & 5 & 9 & 8 & 1 & 2 & 0 \\
        & Digit 6 & 6 & 9 & 3 & 2 & 1 & 2 & 0 & 3 & 2 & 6 & 3 & 3 & 0 & 5 & 6 & 2 & 1 & 7 & 2 & 5 & 1 \\
        & Digit 7 & 7 & 0 & 7 & 9 & 4 & 3 & 7 & 0 & 3 & 5 & 0 & 1 & 2 & 9 & 5 & 3 & 2 & 9 & 5 & 0 & 6 \\
        & Digit 8 & 8 & 1 & 8 & 7 & 6 & 7 & 4 & 5 & 7 & 8 & 5 & 8 & 9 & 0 & 9 & 8 & 3 & 6 & 0 & 7 & 2 \\
        & Digit 9 & 9 & 7 & 6 & 8 & 2 & 0 & 1 & 7 & 9 & 1 & 8 & 4 & 8 & 2 & 1 & 9 & 8 & 2 & 9 & 4 & 3 \\
        \midrule
        \bottomrule 
    \label{tbl:dataset_1}
    \end{tabular}}
\end{table*}

% \resizebox{\linewidth}{!}{
%     \begin{tabular}{c c cccccccccc}
%         \toprule 
%         \midrule
%         \multirow{2}{*}{Dataset} &  Client & \multicolumn{10}{c}{ Original Class No.} \\
%         & No. & 1 & 2 & 3 & 4 & 5 & 6 & 7 & 8 & 9 & 10  \\
%         \midrule
%         \multirow{20}{*}{\shortstack{CIFAR-10 \\ \& SVHN}} 
%         &  1 & 3 & 9 & 4 & 6 & 7 & 5 & 10 & 1 & 2 & 8  \\
%         &  2 & 6 & 5 & 1 & 10 & 3 & 2 & 4 & 8 & 9 & 7  \\
%         &  3 & 4 & 2 & 6 & 1 & 7 & 5 & 3 & 10 & 8 & 9  \\
%         &  4 & 1 & 6 & 4 & 8 & 10 & 9 & 2 & 5 & 7 & 3  \\
%         &  5 & 9 & 2 & 5 & 10 & 7 & 6 & 3 & 4 & 8 & 1  \\
%         &  6 & 3 & 9 & 10 & 4 & 6 & 7 & 1 & 8 & 5 & 2  \\
%         &  7 & 5 & 2 & 10 & 9 & 3 & 7 & 4 & 1 & 6 & 8  \\
%         &  8 & 5 & 1 & 6 & 9 & 2 & 7 & 3 & 4 & 8 & 10  \\
%         &  9 & 3 & 10 & 5 & 8 & 1 & 4 & 7 & 6 & 9 & 2  \\
%         &  10 & 3 & 8 & 2 & 7 & 10 & 5 & 4 & 1 & 6 & 9  \\
%         &  11 & 1 & 8 & 6 & 3 & 7 & 10 & 4 & 2 & 9 & 5  \\
%         &  12 & 7 & 4 & 6 & 8 & 2 & 5 & 1 & 3 & 10 & 9  \\
%         &  13 & 2 & 7 & 4 & 9 & 8 & 5 & 6 & 10 & 1 & 3  \\
%         &  14 & 9 & 3 & 1 & 8 & 4 & 5 & 7 & 6 & 10 & 2  \\
%         &  15 & 5 & 1 & 2 & 7 & 8 & 6 & 3 & 4 & 9 & 1  \\
%         &  16 & 1 & 5 & 7 & 8 & 6 & 10 & 2 & 3 & 4 & 9  \\
%         &  17 & 1 & 2 & 4 & 5 & 6 & 9 & 8 & 10 & 7 & 3  \\
%         &  18 & 7 & 9 & 8 & 5 & 4 & 2 & 3 & 6 & 1 & 10  \\
%         &  19 & 10 & 2 & 7 & 4 & 9 & 3 & 6 & 1 & 8 & 5  \\
%         &  20 & 8 & 5 & 6 & 9 & 10 & 1 & 2 & 7 & 3 & 4  \\
%         \midrule
%         \bottomrule 
%     \end{tabular}
% }\label{tbl:dataset_1}

\vspace{-0.3in}
\subsection{Domain Heterogeneous Scenario}

We use CIFAR-100 datasets ($60,000$) and create five sub-datasets grouped by $10$ similar classes, such as Fruits\&Foods ($6,000$), Transport ($6,000$), Household Objects ($6,000$), Animals ($6,000$), Trees\&Flowers ($6,000$). We then split train ($4,800$), test ($600$), validation ($600$) sets for each sub-datset. To have $20$ clients in total, we assign four clients per subdataset, and split each train set into $4$ partitions, making a single partition contains $1,200$ instances. Additionally, we further permute the labels for those $20$ partitions to simulate more realistic scenarios where labeling schemes are not synchronized across all clients even in the same domain (sub-dataset). We provide class division and label permutation information in Table~\ref{tbl:dataset_2}. 


\section{Implementation \& Training Details} 
\label{appdx:details}
In this section, we provide detailed implementation and training details that are not described in the main document.

\subsection{ResNet-9 Architecture}

\begin{wraptable}{r}{8.25cm}
    \vspace{-0.4in}
    \caption{\textbf{Detailed ResNet-9 Architecture}}
    \vspace{-0.1in}
    \resizebox{\linewidth}{!}{
    \begin{tabular}{cccccccc}
        \toprule
        \midrule
        {\textbf{Layer}} & Input & Output & Filter Size & Stride & Dimension of $\textbf{W}^l$ \\ 
        \midrule 
        Conv 1 & 3 & 64 & 3 & 1 & $64\times3\times3\times3$ \\
        Conv 2 & 64 & 128 & 5 & 2 & $128\times64\times5\times5$ \\
        Conv 3 & 128 & 128 & 3 & 1 & $128\times128\times3\times3$ \\
        Conv 4 & 128 & 128 & 3 & 1 & $128\times128\times3\times3$ \\
        Conv 5 & 128 & 256 & 3 & 1 & $256\times128\times3\times3$ \\
        Conv 6 & 256 & 256 & 3 & 1 & $256\times256\times3\times3$ \\
        Conv 7 & 256 & 256 & 3 & 1 & $256\times256\times3\times3$ \\
        Conv 8 & 256 & 256 & 3 & 1 & $256\times256\times3\times3$ \\
        FC 1 & 256 & $C$ & - & - & $256 \times C$ \\
        \midrule \bottomrule 
    \end{tabular}}
    \label{tbl:resnet9}
\end{wraptable}

We use ResNet-9 architecture consisting of eight convolutional layers and one fully connected layer as a classifier, as described in Table~\ref{tbl:resnet9}. We use max pooling with size $2$ after Conv $5$ and an adaptive max pooling after Conv $8$ to make output width 1 for the following FC layer. The total number of parameters of the model is $2.57 M$. As we use \texttt{PyTorch} framework for implementation and the default data type of tensor of the framework is 32-bits floating point, the model size can be calculated as $2.57 \times 4 = 10.28$ Mbytes. 


\begin{table*}[t!]
    \caption{\textbf{Class division and label permutation information for domain-heterogeneous scenario} We provide class division information and label permutation details for each domain. These permutations are randomly generated based on the same method used in label-heterogeneous scenario using different seeds, i.e. $\textit{fixed global seed} + \textit{client id}$.}
    \resizebox{\linewidth}{!}{
    \begin{tabular}{c c c c cccccccccccccccccccc}
        \toprule 
        \midrule
        \multirow{2}{*}{Domain} & \multirow{2}{*}{Class} & Original & \multicolumn{20}{c}{ Client No. } \\
        & & Labels  & 1 & 2 & 3 & 4 & 5 & 6 & 7 & 8 & 9 & 10 & 11 & 12 & 13 & 14 & 15 & 16 & 17 & 18 & 19 & 20 \\
        \midrule
        \midrule
        \multirow{10}{*}{\shortstack{Household \\ Objects } } 
        & Bed & 5 & 2 & 5 & 3 & 0 & - & - & - & - & - & - & - & - & - & - & - & - & - & - & - & - \\
        & Chair & 20 & 8 & 4 & 1 & 5 & - & - & - & - & - & - & - & - & - & - & - & - & - & - & - & - \\
        & Couch & 22 & 3 & 0 & 5 & 3 & - & - & - & - & - & - & - & - & - & - & - & - & - & - & - & - \\
        & Table & 25 & 5 & 9 & 0 & 7 & - & - & - & - & - & - & - & - & - & - & - & - & - & - & - & - \\
        & Wardrobe & 39 & 6 & 2 & 6 & 9 & - & - & - & - & - & - & - & - & - & - & - & - & - & - & - & - \\
        & Clock & 40 & 4 & 1 & 4 & 8 & - & - & - & - & - & - & - & - & - & - & - & - & - & - & - & - \\
        & Keyboard & 84 & 9 & 3 & 2 & 1 & - & - & - & - & - & - & - & - & - & - & - & - & - & - & - & - \\
        & Lamp & 86 & 0 & 7 & 9 & 4 & - & - & - & - & - & - & - & - & - & - & - & - & - & - & - & - \\
        & Telephone & 87 & 1 & 8 & 7 & 6 & - & - & - & - & - & - & - & - & - & - & - & - & - & - & - & - \\
        & Television & 94 & 7 & 6 & 8 & 2 & - & - & - & - & - & - & - & - & - & - & - & - & - & - & - & - \\
        \midrule
        \multirow{10}{*}{\shortstack{Fruits \\ \& Foods } }
        & Apple & 0 & - & - & - & - & 8 & 2 & 4 & 4 & - & - & - & - & - & - & - & - & - & - & - \\
        & Mushroom & 9 & - & - & - & - & 1 & 8 & 1 & 0 & - & - & - & - & - & - & - & - & - & - & - \\
        & Orange & 10 & - & - & - & - & 4 & 9 & 9 & 5 & - & - & - & - & - & - & - & - & - & - & - & - \\
        & Pear & 16 &  - & - & - & - & 9 & 3 & 8 & 8 & - & - & - & - & - & - & - & - & - & - & - & - \\
        & Sweet Pepper & 28 & - & - & - & - & 6 & 5 & 2 & 1 & - & - & - & - & - & - & - & - & - & - & - & - \\
        & Bottle & 51 & - & - & - & - & 5 & 6 & 6 & 6 & - & - & - & - & - & - & - & - & - & - & - & - \\
        & Bowl & 53 & - & - & - & - & 2 & 0 & 3 & 2 & - & - & - & - & - & - & - & - & - & - & - & - \\
        & Can & 57 & - & - & - & - & 3 & 7 & 0 & 3 & - & - & - & - & - & - & - & - & - & - & - & - \\
        & Cup & 61 & - & - & - & - & 7 & 4 & 5 & 7 & - & - & - & - & - & - & - & - & - & - & - & - \\
        & Plate & 83 & - & - & - & - & 0 & 1 & 7 & 9 & - & - & - & - & - & - & - & - & - & - & - & - \\
        \midrule
        \multirow{10}{*}{\shortstack{Trees \&  \\ Flowers } }
        & Orchid & 47  & - & - & - & - & - & - & - & - & 2 & 2 & 0 & 6 & - & - & - & - & - & - & - & - \\
        & Poppy & 52  & - & - & - & - & - & - & - & - & 9 & 7 & 7 & 3 & - & - & - & - & - & - & - \\
        & Rose & 54  & - & - & - & - & - & - & - & - & 4 & 1 & 5 & 5 & - & - & - & - & - & - & - & - \\
        & Sunflower & 56  &  - & - & - & - & - & - & - & - & 7 & 6 & 2 & 7 & - & - & - & - & - & - & - & - \\
        & Tulip & 59  & - & - & - & - & - & - & - & - & 0 & 9 & 6 & 1 & - & - & - & - & - & - & - & - \\
        & Maple Tree & 62  & - & - & - & - & - & - & - & - & 3 & 4 & 9 & 4 & - & - & - & - & - & - & - & - \\
        & Oak Tree & 70  & - & - & - & - & - & - & - & - & 6 & 3 & 3 & 0 & - & - & - & - & - & - & - & - \\
        & Palm Tree & 82  & - & - & - & - & - & - & - & - & 5 & 0 & 1 & 2 & - & - & - & - & - & - & - & - \\
        & Pine Tree & 92  & - & - & - & - & - & - & - & - & 8 & 5 & 8 & 9 & - & - & - & - & - & - & - & - \\
        & Willow Tree & 96 & - & - & - & - & - & - & - & - & 1 & 8 & 4 & 8 & - & - & - & - & - & - & - & - \\
        \midrule
        \multirow{10}{*}{\shortstack{Transport } }
        & Lawn Mower & 8 & - & - & - & - & - & - & - & - & - & - & - & - & 1 & 8 & 4 & 0 & - & - & - & - \\
        & Rocket & 13 & - & - & - & - & - & - & - & - & - & - & - & - & 6 & 2 & 0 & 4 & - & - & - & - \\
        & Streetcar & 41 & - & - & - & - & - & - & - & - & - & - & - & - & 3 & 0 & 1 & 6 & - & - & - & - \\
        & Tank & 48 &  - & - & - & - & - & - & - & - & - & - & - & - & 8 & 7 & 6 & 7 & - & - & - & - \\
        & Tractor & 58 & - & - & - & - & - & - & - & - & - & - & - & - & 7 & 3 & 7 & 5 & - & - & - & - \\
        & Bicycle & 69 & - & - & - & - & - & - & - & - & - & - & - & - & 4 & 4 & 5 & 9 & - & - & - & - \\
        & Bus & 81 & - & - & - & - & - & - & - & - & - & - & - & - & 5 & 6 & 2 & 1 & - & - & - & - \\
        & Motorcycle & 85 & - & - & - & - & - & - & - & - & - & - & - & - & 9 & 5 & 3 & 2 & - & - & - & - \\
        & Pickup Truck & 89 & - & - & - & - & - & - & - & - & - & - & - & - & 0 & 9 & 8 & 3 & - & - & - & - \\
        & Train & 90 & - & - & - & - & - & - & - & - & - & - & - & - & 2 & 1 & 9 & 8 & - & - & - & - \\
        \midrule
        \multirow{10}{*}{\shortstack{Animals} }
        & Fox & 3 & - & - & - & - & - & - & - & - & - & - & - & - & - & - & - & - & 0 & 6 & 9 & 7 \\
        & Porcupine & 34 & - & - & - & - & - & - & - & - & - & - & - & - & - & - & - & - & 1 & 8 & 1 & 4 \\
        & Possum & 42 & - & - & - & - & - & - & - & - & - & - & - & - & - & - & - & - & 3 & 7 & 6 & 5 \\
        & Raccoon & 43 &  - & - & - & - & - & - & - & - & - & - & - & - & - & - & - & - & 4 & 4 & 3 & 8 \\
        & Skunk & 63 & - & - & - & - & - & - & - & - & - & - & - & - & - & - & - & - & 5 & 3 & 8 & 9 \\
        & Bear & 64 & - & - & - & - & - & - & - & - & - & - & - & - & - & - & - & - & 8 & 1 & 2 & 0 \\
        & Leopard & 66 & - & - & - & - & - & - & - & - & - & - & - & - & - & - & - & - & 7 & 2 & 5 & 1 \\
        & Lion & 75 & - & - & - & - & - & - & - & - & - & - & - & - & - & - & - & - & 9 & 5 & 0 & 6 \\
        & Tiger & 88 & - & - & - & - & - & - & - & - & - & - & - & - & - & - & - & - & 6 & 0 & 7 & 2 \\
        & Wolf & 97 & - & - & - & - & - & - & - & - & - & - & - & - & - & - & - & - & 2 & 9 & 4 & 3 \\
        \midrule
        \bottomrule 
    \label{tbl:dataset_2}
    \end{tabular}}
\end{table*}

\subsection{Calculation of Communication Cost}
We measure the communication cost by $\{(P_{S2C}+P_{C2S}) \times 4 \}_{byte} \times K \times R$, where $P_{S2C}$ is number of server-to-client transmitted parameters and $P_{C2S}$ is number of client-to-server transmitted parameters. Depending on the FL algorithms, $P_{S2C}$ and  $P_{C2S}$ are differently calculated. For example, \texttt{FedFOMO} downloads few random models from server ($10$ as default, reported in the paper) but sends only single local model to server. Our \texttt{Factorized-FL} only sends the  small portion of model parameters, $\mathcal{U}$ and $\textbf{v}^{L-1}$, to server, while receiving a single set of $\mathcal{U}$ from server. 

\subsection{Training Details}

As default, all training configurations are equally set across all models, unless otherwise stated to ensure stricter fairness. We use ResNet-9 architecture as local backbone networks and train them on $32\times32$ sized images with $256$ for batch size. We apply data augmentations, i.e. cropping, flipping, jittering, etc, during training. Optimizer that we used is Stochastic Gradient Descent (SGD). We set $1$e-$3$ for learning rate, $1$e-$6$ for weight decay, and  $0.9$ for momentum. For baseline models, we use the reported hyper-parameters as default, or we adjust hyper-parameters so that they show the best performance for fairness. For ours and \texttt{pFedPara}, the model capacity is adjusted to around $90\%$ - $99\%$ of the original size, as we fairly compare with other methods that use full capacity ($2.57M$ number of parameters). For ours, we use [$5$e-$4$, $1$e-$3$] for $\lambda_{\text{sparsity}}$, [$0$-$0.75$] for $\tau$, [$1$, $20$] for $\epsilon$. 

\section{Additional Experimental Results}
\label{appdx:experiment}

% \subsection{Comparison with the Hadamard Product Factorization}

% \begin{wrapfigure}[8]{r}{0.45\textwidth}
\centering
\vspace{-0.35in}
\begin{tabular}{cc}
     \includegraphics[width=0.49\linewidth]{figures/images/pfedpara_perf.pdf} &
     \hspace{-0.2in} \includegraphics[width=0.49\linewidth]{figures/images/pfedpara_size.pdf} \\
     (a) Performance & (b) Model Capacity
\end{tabular}
\vspace{-0.1in}
\caption{\textbf{The effect of factorization methods}}
\label{fig:hadamard}
\end{wrapfigure}

% \texttt{pFedPara}~\citep{anonymous2022fedpara} adopts the Hadamard product that uses two sets of low rank matrices, i.e. $(\textbf{u}_1\times\textbf{v}_1^\intercal)\odot(\textbf{u}_2\times\textbf{v}_2^\intercal)$, and share one of the sets $(\textbf{u}_1\times\textbf{v}_1^\intercal)$ globally for personalized federated learning scenario. Unlike the prior work, our approach utilizes rank-1 vectors to perform aggregation in the lowest subspace possible for compatibility, while effectively yet efficiently enhancing expressiveness with sparse bias matrices. 

% In this subsection, we analyze To see the effectiveness of factorization techniques, we 

\subsection{Sparsity Analysis on FL Scenarios}
\label{appdx:sparisty}

In the main document, we show the effect of model size and sparsity controlled by  $\lambda_{\text{sparsity}}$ for a single model. In this section, we analyze it under federated learning scenario. In Figure~\ref{fig:model_size} (a), we show the performance over model size in domain heterogeneous scenario. As shown, our method show superior performance even with around $65\%$ of the model size over the baseline model that achieves the best performance (\texttt{Per-FedAvg}) amongst other baseline models. With $50\%$ sparsity, ours still shows competitive performance compared to~\texttt{Clustered-FL} and \texttt{FedAvg}, while it starts being significantly degenerated when sparsity becomes over $50\%$. 

\begin{figure*}[h!]
\begin{tabular}{cc}
     \includegraphics[width=0.5\linewidth]{figures/images/appdx_model_size.pdf} & 
     \hspace{-0.1in} \includegraphics[width=0.5\linewidth]{figures/images/appdx_cost.pdf}\\
     (a) Model size ratio controlled by $\lambda_{\text{sparsity}}$ &
     (b) Communication Costs 
\end{tabular}
\caption{\textbf{Model size and communication costs comparison} (a) we plot accuracy over model size on domain heterogeneous scenario. (b) we plot accuracy over transmission costs on domain heterogeneous scenario.}
\label{fig:model_size}
\end{figure*}

In Figure~\ref{fig:model_size} (b), we show accuracy over communication costs. Note that, in our method, the model size is not really related to the communication costs since we send very small portion of model parameters. For example, even though we use almost full model size ($\lambda_{\text{sparsity}}$=$3e$-$4$), our communication cost is significantly lesser than the other baseline models, as shown in the figure.


\subsection{Additional Results}

For label-heterogeneous FL scenario (Table~\ref{tbl:permuted} (Top), we provide test accuracy curves over communication rounds and transmission costs for results of CIFAR-10 and SVHN with stardard iid/non-iid and permuted iid/non-iid partitions in Figure~\ref{fig:all}. For domain-heterogeneous FL scenario (Table~\ref{tbl:permuted} (Bottom)), we provide performance of $20$ clients In Figure~~\ref{fig:bar}.

\begin{figure*}
    \centering
    
    \begin{tabular}{c c }
         
         \includegraphics[width=0.38\textwidth]{figures/images/conven_iid_c10.pdf} & \includegraphics[width=0.38\textwidth]{figures/images/permuted_iid_c10.pdf} \\
         (a) Standard IID (CIFAR-10) & (b) Permuted IID (CIFAR-10) \\
         
         \includegraphics[width=0.38\textwidth]{figures/images/conven_non_iid_c10.pdf} & \includegraphics[width=0.38\textwidth]{figures/images/permuted_non_iid_c10.pdf} \\
         (c) Standard NonIID (CIFAR-10) & (d) Permuted NonIID (CIFAR-10) \\
         
         \includegraphics[width=0.38\textwidth]{figures/images/conven_iid_svhn.pdf} & \includegraphics[width=0.38\textwidth]{figures/images/permuted_iid_svhn.pdf} \\
         (e) Standard IID (SVHN) & (f) Permuted IID (SVHN) \\
         
         \includegraphics[width=0.38\textwidth]{figures/images/conven_non_iid_svhn.pdf} &  \includegraphics[width=0.38\textwidth]{figures/images/permuted_non_iid_svhn.pdf} \\
         (g) Standard NonIID (SVHN) & (h) Permuted NonIID (SVHN) \\
         
    \end{tabular}
    \caption{\textbf{Test accuracy curves over communication round for standard federated learning and label-heterogeneous FL scenario}: We provide test accuracy curves on CIFAR-10 and SVHN in standard iid/non-iid and permuted iid/non-iid partitions ($E$=$5$,$R$=$50$). }
    \label{fig:all}
\end{figure*}
\begin{figure*}
    \centering
    
    \begin{tabular}{c c }
         
         \includegraphics[width=0.38\textwidth]{figures/images/conven_iid_c10_cost.pdf} & \includegraphics[width=0.38\textwidth]{figures/images/permuted_iid_c10_cost.pdf} \\
         (a) Standard IID (CIFAR-10) & (b) Permuted IID (CIFAR-10) \\
         
         \includegraphics[width=0.38\textwidth]{figures/images/conven_non_iid_c10_cost.pdf} & \includegraphics[width=0.38\textwidth]{figures/images/permuted_non_iid_c10_cost.pdf} \\
         (c) Standard NonIID (CIFAR-10) & (d) Permuted NonIID (CIFAR-10) \\
         
         \includegraphics[width=0.38\textwidth]{figures/images/conven_iid_svhn_cost.pdf} & \includegraphics[width=0.38\textwidth]{figures/images/permuted_iid_svhn_cost.pdf} \\
         (e) Standard IID (SVHN) & (f) Permuted IID (SVHN) \\
         
         \includegraphics[width=0.38\textwidth]{figures/images/conven_non_iid_svhn_cost.pdf} &  \includegraphics[width=0.38\textwidth]{figures/images/permuted_non_iid_svhn_cost.pdf} \\
         (g) Standard NonIID (SVHN) & (h) Permuted NonIID (SVHN) \\
         
    \end{tabular}
    \caption{\textbf{Test accuracy over communication costs for standard federated learning and label-heterogeneous FL scenario}: We provide test accuracy curves on CIFAR-10 and SVHN in standard iid/non-iid and permuted iid/non-iid partitions ($E$=$5$,$R$=$50$). }
    \label{fig:all}
\end{figure*}
\begin{figure}
    \centering
    \includegraphics[angle=270, origin=c, width=0.25\textwidth]{figures/images/fig_bar.pdf}
    \caption{\textbf{Performance of all $20$ clients in domain heterogeneous scenario}: We plot performance of $20$ clients in domain-heterogeneous scenario, of which results are corresponding to Table~\ref{tbl:permuted} (Bottom). }
    \label{fig:bar}
\end{figure}