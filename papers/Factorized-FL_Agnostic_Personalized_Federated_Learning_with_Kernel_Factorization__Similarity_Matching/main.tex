\pdfoutput=1

%%%%%%%% ICML 2022 EXAMPLE LATEX SUBMISSION FILE %%%%%%%%%%%%%%%%%

\documentclass[nohyperref]{article}

% Recommended, but optional, packages for figures and better typesetting:
\usepackage{microtype}
\usepackage{graphicx}
\usepackage{subfigure}
\usepackage{booktabs} % for professional tables

% hyperref makes hyperlinks in the resulting PDF.
% If your build breaks (sometimes temporarily if a hyperlink spans a page)
% please comment out the following usepackage line and replace
% \usepackage{icml2022} with \usepackage[nohyperref]{icml2022} above.
\usepackage{hyperref}


% Attempt to make hyperref and algorithmic work together better:
\newcommand{\theHalgorithm}{\arabic{algorithm}}

% Use the following line for the initial blind version submitted for review:
% \usepackage{icml2022}

% If accepted, instead use the following line for the camera-ready submission:
\usepackage[accepted]{icml2022}

% For theorems and such
\usepackage{amsmath}
\usepackage{amssymb}
\usepackage{mathtools}
\usepackage{amsthm}

% Custom
\usepackage[normalem]{ulem}
\usepackage{arydshln}
\usepackage{graphicx}
\usepackage{wrapfig}
\usepackage[font=small]{caption}
\usepackage{lipsum}
\usepackage{pifont}
\newcommand{\cmark}{\ding{51}}%
\newcommand{\xmark}{\ding{55}}%
\definecolor{navyblue}{rgb}{0.0, 0.0, 0.5}
\definecolor{darkblue}{rgb}{0.0, 0.0, 0.55}
\hypersetup{
    colorlinks = true,
    citecolor=darkblue,
    linkcolor=red,
    urlcolor  = blue,
    anchorcolor = blue}
\usepackage{enumitem}
\usepackage{multirow}
% \usepackage{subcaption}
\newcommand{\vio}[1]{{\color{violet}{#1}}}
\newcommand{\cy}[1]{{\color{cyan}{#1}}}
\newcommand{\red}[1]{{\color{red}{#1}}}
\newcommand{\ora}[1]{{\color{orange}{#1}}}
\newcommand{\pu}[1]{{\color{purple}{#1}}}
\newcommand{\blu}[1]{{\color{blue}{#1}}}

% Custom


% \definecolor{Red}{rgb}{1,0,0}
\definecolor{Blue}{rgb}{0,0,0.8}
\definecolor{Green}{rgb}{0,0.4,0.7}
\definecolor{airforceblue}{rgb}{0.36, 0.54, 0.66}
\definecolor{ao(english)}{rgb}{0.0, 0.5, 0.0}
\definecolor{azure(colorwheel)}{rgb}{0.0, 0.5, 1.0}
\definecolor{crimson}{rgb}{0.86, 0.08, 0.24}
\definecolor{darkcerulean}{rgb}{0.03, 0.27, 0.49}
\definecolor{cobalt}{rgb}{0.0, 0.28, 0.67}
\definecolor{rosegold}{rgb}{0.72, 0.43, 0.47}
\definecolor{orange-red}{rgb}{1.0, 0.27, 0.0}
\definecolor{mountainmeadow}{rgb}{0.19, 0.73, 0.56}
\definecolor{malachite}{rgb}{0.04, 0.85, 0.32}
\definecolor{darkblue}{rgb}{0.0, 0.0, 0.55}

\definecolor{customblue}{rgb}{0.2, 0.35, 0.8}

\usepackage{hyperref}
\hypersetup{colorlinks=true}
\hypersetup{linktoc=all}
\hypersetup{citecolor=darkblue}
\hypersetup{linkcolor=crimson}
\hypersetup{urlcolor=darkblue}
\usepackage[all]{hypcap}
\usepackage[percent]{overpic}
\usepackage[usestackEOL]{stackengine}

\usepackage[nameinlink]{cleveref}
\creflabelformat{equation}{#2\textup{#1}#3}  
\crefname{assumption}{assumption}{assumptions}
%\usepackage[pagebackref,breaklinks,colorlinks]{hyperref}

\definecolor{gg}{gray}{0.9}

\newcommand{\bsy}{\boldsymbol}
\newcommand{\highlight}[1]{{\color{crimson}{#1}}}
\newcommand{\hibf}[1]{{\color{crimson}{\textbf{#1}}}}
\newcommand{\modify}[1]{{\color{orange-red}{#1}}} %ao(english)
\newcommand{\modified}[1]{{\color{azure(colorwheel)}{#1}}} %airforceblue
\newcommand{\TBD}[1]{{\color{Red}{#1}}}
\creflabelformat{equation}{#2\textup{#1}#3}  

%\newcommand{\plhi}[1]{{\color{Red}{\textbf{#1}}}}
%\newcommand{\mihi}[1]{{\color{Blue}{\textbf{#1}}}}
% \definecolor{plRed}{rgb}{0.8,0,0}
% \definecolor{miBlue}{rgb}{0,0,0.8}
% \definecolor{miGreen}{rgb}{0.0, 0.5, 0.0}
% \newcommand{\red}[1]{{\color{plRed}{#1}}}
% \newcommand{\blu}[1]{{\color{miBlue}{#1}}}
% \newcommand{\gre}[1]{{\color{miGreen}{#1}}}

% if you use cleveref..
% \usepackage[capitalize,noabbrev]{cleveref}

%%%%%%%%%%%%%%%%%%%%%%%%%%%%%%%%
% THEOREMS
%%%%%%%%%%%%%%%%%%%%%%%%%%%%%%%%
\theoremstyle{plain}
\newtheorem{theorem}{Theorem}[section]
\newtheorem{proposition}[theorem]{Proposition}
\newtheorem{lemma}[theorem]{Lemma}
\newtheorem{corollary}[theorem]{Corollary}
\theoremstyle{definition}
\newtheorem{definition}[theorem]{Definition}
\newtheorem{assumption}[theorem]{Assumption}
\theoremstyle{remark}
\newtheorem{remark}[theorem]{Remark}

% Todonotes is useful during development; simply uncomment the next line
%    and comment out the line below the next line to turn off comments
%\usepackage[disable,textsize=tiny]{todonotes}
\usepackage[textsize=tiny]{todonotes}


% The \icmltitle you define below is probably too long as a header.
% Therefore, a short form for the running title is supplied here:
\icmltitlerunning{Factorized-FL: Agnostic Personalized Federated Learning with Kernel Factorization \& Similarity Matching}

\begin{document}

\twocolumn[
\icmltitle{Factorized-FL: Agnostic Personalized Federated Learning \\ with Kernel Factorization \& Similarity Matching}

% It is OKAY to include author information, even for blind
% submissions: the style file will automatically remove it for you
% unless you've provided the [accepted] option to the icml2022
% package.

% List of affiliations: The first argument should be a (short)
% identifier you will use later to specify author affiliations
% Academic affiliations should list Department, University, City, Region, Country
% Industry affiliations should list Company, City, Region, Country

% You can specify symbols, otherwise they are numbered in order.
% Ideally, you should not use this facility. Affiliations will be numbered
% in order of appearance and this is the preferred way.
\icmlsetsymbol{equal}{*}

% \begin{icmlauthorlist}
% \icmlauthor{Wonyong Jeong}{equal,yyy}
% \icmlauthor{Firstname2 Lastname2}{equal,yyy,comp}
% \icmlauthor{Firstname3 Lastname3}{comp}
% \icmlauthor{Firstname4 Lastname4}{sch}
% \icmlauthor{Firstname5 Lastname5}{yyy}
% \icmlauthor{Firstname6 Lastname6}{sch,yyy,comp}
% \icmlauthor{Firstname7 Lastname7}{comp}
% %\icmlauthor{}{sch}
% \icmlauthor{Firstname8 Lastname8}{sch}
% \icmlauthor{Firstname8 Lastname8}{yyy,comp}
% %\icmlauthor{}{sch}
% %\icmlauthor{}{sch}
% \end{icmlauthorlist}

% \icmlaffiliation{yyy}{Department of XXX, University of YYY, Location, Country}
% \icmlaffiliation{comp}{Company Name, Location, Country}
% \icmlaffiliation{sch}{School of ZZZ, Institute of WWW, Location, Country}

% \icmlcorrespondingauthor{Firstname1 Lastname1}{first1.last1@xxx.edu}
% \icmlcorrespondingauthor{Firstname2 Lastname2}{first2.last2@www.uk}

\begin{icmlauthorlist}
\icmlauthor{Wonyong Jeong}{to,goo}
\icmlauthor{Sung Ju Hwang}{to,goo}
\end{icmlauthorlist}

\icmlaffiliation{to}{Graduate School of Artificial Intelligence, KAIST, Seoul, South Korea}
\icmlaffiliation{goo}{AITRICS, Seoul, South Korea}
%\icmlaffiliation{ed}{School of Computation, University of Edenborrow, Edenborrow, United Kingdom}

\icmlcorrespondingauthor{Wonyong Jeong}{wyjeong@kaist.ac.kr}
\icmlcorrespondingauthor{Sung Ju Hwang}{sjhwang82@kaist.ac.kr}


% You may provide any keywords that you
% find helpful for describing your paper; these are used to populate
% the "keywords" metadata in the PDF but will not be shown in the document
\icmlkeywords{Machine Learning, ICML}

\vskip 0.3in
]

% this must go after the closing bracket ] following \twocolumn[ ...

% This command actually creates the footnote in the first column
% listing the affiliations and the copyright notice.
% The command takes one argument, which is text to display at the start of the footnote.
% The \icmlEqualContribution command is standard text for equal contribution.
% Remove it (just {}) if you do not need this facility.

\printAffiliationsAndNotice{}  % leave blank if no need to mention equal contribution
% \printAffiliationsAndNotice{\icmlEqualContribution} % otherwise use the standard text.

\begin{abstract}
%\medskip
%\centering \textcolor{red}{Write the abstract last}
Silicon-compatible short- and mid-wave infrared emitters are highly sought-after for on-chip monolithic integration of electronic and photonic circuits to serve a myriad of applications in sensing and communication. To address this longstanding challenge, GeSn semiconductors have been proposed as versatile building blocks for silicon-integrated optoelectronic devices. In this regard, this work demonstrates light-emitting diodes (LEDs) consisting of a vertical PIN double heterostructure  p-Ge$_{0.94}$Sn$_{0.06}$/i-Ge$_{0.91}$Sn$_{0.09}$/n-Ge$_{0.95}$Sn$_{0.05}$ grown epitaxially on a silicon wafer using germanium interlayer and multiple GeSn buffer layers. The emission from these GeSn LEDs at variable diameters in the 40-120 $\mu$m range is investigated under both DC and AC operation modes. The fabricated LEDs exhibit a room temperature emission in the extended short-wave range centered around 2.5 $\mu$m under an injected current density as low as 45 A/cm$^2$.  By comparing the photoluminescence and electroluminescence signals, it is demonstrated that the LED emission wavelength is not affected by the device fabrication process or heating during the LED operation. Moreover, the measured optical power was found to increase monotonically as the duty cycle increases indicating that the DC operation yields the highest achievable optical power. The LED emission profile and bandwidth are also presented and discussed. 
\end{abstract}
\section{Introduction}

Scientific literature is most commonly available in the form of PDFs, which pose challenges for accessibility \citep{NielsenPDFStillUnfit, Bigham2016AnUT}. When researchers, students, and other individuals who are blind or low vision (BLV) interact with scientific PDFs through screen readers, the availability of document structure tags, labeled reading order, labeled headers, and image alt-text are necessary to facilitate these interactions. However, these features must be painstakingly added by authors using proprietary software tools, and as a result, are often missing from papers. Low vision or dyslexic readers who interact with PDFs through screen magnification or text-to-speech may also find the complexity of certain academic paper PDF formats challenging, e.g., non-linear layout can interrupt the flow of text in a magnifying tool. Inaccessible paper PDFs can lead to high cognitive overload, frustration, and abandonment of reading for BLV readers. 

Unfortunately, we find that the majority of scientific PDFs lack basic accessibility features. We estimate based on a sample of \numpdfs PDFs from multiple fields of study that only around \percaccessible of paper PDFs released in the last decade satisfy all of the aforementioned accessibility requirements. 
Accessibility challenges for academic PDFs are largely due to three factors: (1) the complexity of the PDF file format, which make it less amenable to certain accessibility features, (2) the dearth of tools, especially non-proprietary tools, for creating accessible PDFs, and (3) the dependency on volunteerism from the community with minimal support or enforcement \citep{Bigham2016AnUT}. The intent of the PDF file format is to support faithful visual representation of a document for printing, a goal that is inherently divergent from that of document representation for the purposes of accessibility. Though some professional organizations like the Association for Computing Machinery (ACM) have encouraged PDF accessibility through standards and writing guidelines,\footnote{\href{https://www.acm.org/publications/authors/submissions}{https://www.acm.org/publications/authors/submissions}} uptake among academic publishers and disciplines more broadly has been limited. 

While policy changes help, the fact remains that most academic PDFs produced today, and historically, are inaccessible, yet remain as the dominant way to read those papers. A long-range solution will necessitate buy-in from multiple stakeholders---publishers, authors, readers, technologists, granting agencies, and the like. But in the interim, there are technological solutions that can be offered as a sort of ``band-aid'' to the problem. We use this paper to offer an in-depth qualitative and quantitative description of the problem as it stands, and to introduce one such technological solution: the \scially system that automatically extracts semantic information from paper PDFs and re-renders this content in the form of an accessible HTML document. Though the process is imperfect and can introduce errors, we demonstrate the ability of the rendered HTMLs to reduce cognitive load and facilitate in-paper navigation and interactions for BLV users. 

The goals and contributions of this paper are three-fold:

\begin{enumerate}
    \item We characterize the state of academic-paper PDF accessibility by estimating the degree of adherence to accessibility criteria for papers published in the last decade (2010--2019), and describe correlations between year, field of study, PDF typesetting software, and PDF accessibility.
    \item We propose an automated approach for extracting the content of academic PDFs and displaying this content in a more accessible HTML document format. We build a prototype that re-renders 12 million PDFs in HTML, and describe the design decisions, features, and quality of the renders (assessed as faithfulness to the source PDF). We perform expert grading of the rendered HTML and report an error analysis. A demo of our system is available at \href{https://scia11y.org/}{scia11y.org}, which makes available 1.5M HTML renders of open access PDFs.
    \item We conduct an exploratory user study with \numusers BLV scholars to better understand the challenges they experience when reading academic papers and how our proposed tool might augment their current workflow. During the study, we ask users to interact with the prototype and offer feedback for its improvement. We perform open coding of interviews to identify existing reading challenges, coping mechanisms, as well as positive and negative responses to prototype features. We summarize the findings of this user study into a set of design recommendations.
\end{enumerate}

Our analysis reveals that PDF accessibility adherence is low across all fields of study. Of the five accessibility criteria we assess, only \percaccessible of the PDFs we assess demonstrate full compliance. Though compliance for several criteria seems to be increasing over time, author awareness and contribution to accessibility remains low, as Alt-text has the lowest compliance of the five criteria at between 5--10\% (Alt-text is the only criterion of the five that \textit{requires} author intervention in all cases using current tools). We also find that typesetting software is strongly associated with accessibility compliance, with LaTeX and publishing software like Arbortext APP producing low compliance PDFs, while Microsoft Word is generally associated with higher compliance.


\begin{figure}[t!]
    \centering
    \includegraphics[width=\textwidth]{figures/pipeline.png}
    \caption{A schematic for creating the \scially HTML render from a paper PDF. Starting with the raw two-column PDF on the left, S2ORC \citep{lo-wang-2020-s2orc} is used to extract title, authors, abstract, section headers, body text, and references. S2ORC also identifies links between inline citations and references to figures and table objects. DeepFigures \citep{Siegel2018ExtractingSF} is used to extract figures and tables, along with their captions. The output of these two models are merged with metadata from the Semantic Scholar API. Heuristics are used to construct a table of contents, to insert figures and tables in the appropriate places in the text, and to repair broken URLs. We add HTML headers as illustrated (header tags for sections, paragraph tags for body text, and figure tags for figures and tables); highlighted components (table of contents and links in references) are not in the PDF and novel navigational features that we introduce to the HTML render. An example HTML render of parts of a paper document is show to the right (actual render is single column, which is split here for presentation).}
    \label{fig:pipeline}
    \Description{A schematic diagram showing the components of the SciA11y pipeline. An image of a paper PDF is on the left. Red boxes on the PDF image highlight the text components from the paper, with an arrow pointing to a box that says "S2ORC extracts: title, authors, abstract, section headers, body paragraphs, and references." A blue box on the PDF image highlights a figure, with an arrow pointing to a box that says "DeepFigures extracts: figures, figure captions, tables, and table titles/captions." A box below "S2ORC extracts" and "DeepFigures extracts" says "Additional content: metadata from Semantic Scholar API, table of contents, figures and tables inserted at first mention, and links between references and text." Arrows from all three boxes point into a larger box that describes the SciA11y prototype, where HTML tags are inserted around various blocks of text extracted from the PDF. On the right of all this is a screen capture of an example HTML render, showing how the semantic content from the PDF is represented as a single-column HTML page for easy reading.}
\end{figure}

To offset the reading challenges of inaccessible papers for BLV researchers, we propose and test the \scially system for rendering academic PDFs into accessible HTML documents. As shown in Figure~\ref{fig:pipeline}, our prototype integrates several machine learning text and vision models to extract the structure and semantic content of papers. The content is represented as an HTML document with headings and links for navigation, figures and tables, as well as other novel features to assist in document structure understanding. Our evaluation of the \scially system identifies common classes of extraction problems, and finds that though many papers exhibit some extraction errors, the majority (55\%) have no major problems that impact readability, and another 32\% have only some problems that impact readability.

Through our user study, we identify numerous challenges faced by BLV users when reading paper PDFs, including some that affect the whole document or limit navigation, and many that affect the ability of the reader to understand text or various elements of a paper like math content or tables. Responses to \scially were positive; participants especially liked navigation features such as headings, the table of contents, and bidirectional links between inline citations and references. Of the extraction errors in \scially, missed or incorrectly extracted headings were the most problematic, as these impact the user's ability to navigate between sections and fully trust the system. All users reported being likely to use the system in the future. When asked how the system might be integrated into their workflow, one participant replied ``I think it would become the workflow.'' Another participant said, ``for unaccessible PDFs, this is life-changing.'' We condense these findings into a set of recommendations for designing and engineering accessible reading systems (Section~\ref{sec:designrecs}). Most importantly, documents should be structured to match a reader's mental model, objects should be properly tagged, and care should be taken to reduce the reader's cognitive load and increase trust in the system. Features that emulate the external memory that visual layout provides to sighted users can be especially beneficial.

This paper is organized as follows. Following a description of related work in Section \ref{sec:related_work}, we first provide a meta-scientific analysis of the current state of academic PDF accessibility in Section \ref{sec:sos}. In Section \ref{sec:pdf2html}, we document our pipeline for converting PDF to HTML and describe the \scially prototype for rendering papers. An evaluation of HTML render quality and faithfulness is provided in Section \ref{sec:evaluation}. Section \ref{sec:user_study} describes our user study and findings. 
We recognize that no PDF extraction system is perfect, and many open research challenges remain in improving these systems. However, based on our findings, we believe \scially can dramatically improve screen reader navigation of most papers compared to PDFs, and is well-positioned to assist BLV researchers with many of their most common reading use cases. Our hope is that a system such as \scially can improve BLV researcher access to the content of academic papers, and that these design recommendations can be leveraged by others to create better, more faithful, and ultimately more usable tools and systems for scholars in the BLV community.

\section{Related Work}
\label{sec:related_work}
% In this section, we review the related work, which includes graph neural networks, and robust graph neural networks. 

\subsection{Graph Neural Networks}
Graph Neural Networks (GNNs) have shown their great power in modeling graph structured data for various applications~\cite{wang2019semi,wang2018cross,zhao2020semi,dai2021say,zhao2021graphsmote}.
To generalize neural networks for graphs, two categories of GNNs are proposed, i.e., spectral-based~\cite{bruna2013spectral,henaff2015deep,kipf2016semi,levie2018cayleynets} and spatial-based~\cite{velivckovic2017graph,hamilton2017inductive,chen2018fastgcn,chiang2019cluster}. \citeauthor{bruna2013spectral} \cite{bruna2013spectral} first propose spectral-based GNNs by defining graph convolution with spectral graph theory. For instance, GCN~\cite{kipf2016semi} simplifies the convolutional operation by using the first order approximation. Spatial-based graph convolution is defined in spatial domain, which updates node representation by aggregating its neighbors' representations \cite{niepert2016learning,gilmer2017neural,hamilton2017inductive}. 
For example, self-attention of neighbor nodes is leveraged in graph attention network (GAT) \cite{velivckovic2017graph}. Moreover, various spatial methods are proposed to solve the scalability issue~\cite{chen2018fastgcn,chiang2019cluster} and learn deeper GNNs~\cite{chen2020simple}.  Recently, to alleviate the problem of lacking labeled nodes, many efforts are taken to explore GNNs using self-supervision, which aims to learn better node representations with pretext tasks~\cite{sun2019multi,li2018deeper,kim2021find,zhu2020self,jin2020self,dai2021towards}. For instance, superGAT~\cite{kim2021find} deploys edge prediction in GAT to guide the learning of attention for better representations. SE-GNN~\cite{dai2021towards} deploys contrastive learning to benefit the similarity modeling for self-explainable GNN.

Inspired by the great success of GNNs, methods that construct graphs and adopt GNNs for data without explicit relational structure are also explored~\cite{henaff2015deep,chen2019multi,jiang2019semi,dai2021nrgnn}. Generally, a graph would be built based on certain rules~\cite{henaff2015deep,chen2019multi} or be learned in an end-to-end model~\cite{jiang2019semi,dai2021nrgnn}. Our RS-GNN is inherently different from these methods as we eliminate/down-weight the noisy edges and predict the missing edges for robust GNNs on noisy graphs with limited labels. 

\subsection{Robust GNNs}
Although GNNs have obtained great achievements, they are vulnerable to adversarial attacks~\cite{wu2019adversarial,dai2018adversarial,zugner2018adversarial,zugner2019adversarial}. Based on the objective, the adversarial attacks on GNNs can be split into two categories, i.e., targeted attack~\cite{dai2018adversarial,zugner2018adversarial} and non-targeted attack~\cite{zugner2019adversarial}. Targeted attack methods aim to degrade the performance of the GNNs on target nodes. 
For instance, \textit{nettack}~\cite{zugner2018adversarial} adds adversarial perturbations to a graph to attack targeted nodes. Non-targeted attack aims to reduce the overall performance of GNNs. For example, \textit{metattack}~\cite{zugner2019adversarial} poisons the graph globally to achieve non-targeted attack with meta-learning. To defend against adversarial attacks, many efforts are taken recently~\cite{zhu2019robust,wu2019adversarial,entezari2020all,jin2020graph,tang2020transferring,zhang2020gnnguard}. \cite{wu2019adversarial} prune the perturbed edges based on Jaccard similarity of node features. Another preprocessing method by low-rank approximation of adjacent matrix is investigated~\cite{entezari2020all}. Pro-GNN~\cite{jin2020graph} is the most similar work to ours, which learns a clean graph structure by low-rank constraint. However, they only tackle the adversarial edges and their computational cost is very large due to the direct learning of the graph and the sparse low-rank constraint.
This work is inherently different from these methods as: (i) we study a novel problem of developing robust GNN for both noisy graphs and label sparsity issues; and (ii) the proposed RS-GNN simultaneously tackles the two issues by learning an link predictor to 
down-weight noisy edges and connecting nodes with high similarity to facilitate message-passing; 
and (iii) RS-GNN uses link predictor instead of direct graph learning to save computational cost. 




\section{Problem Definition}
\label{sec:def}
We begin with the formal definition of the conventional federated learning scenario, and then introduce our novel Agnostic Personalized Federated Learning (APFL) problem. 

%We then describe the two scenarios for the APFL problem, namely Label and Domain Heterogeneous FL. 

\subsection{Preliminaries}
\label{subsec:fl}
Our main task is solving a given multi-class classification problem in an FL framework. Let $f_g$ be a global model (neural network) at the global server and $\mathcal{F}=\{f_k\}^{K}_{k=1}$ be a set of $K$ local neural networks, where $K$ is the number of local clients.  $\mathcal{D}=\{\textbf{x}_{i}, y_{i}\}^{N}_{i=1}$ be a given dataset, where $N$ is the number of instances, $\textbf{x}_i \in \mathbb{R}^{W \times H \times D}$ is the $i_{th}$ examples in a size of width $W$, height $H$, and depth $D$, with a corresponding target label $y_i \in \{1,\dots,C\}$ for the $C$-way multi-class classification problem. The given dataset $\mathcal{D}$ is then disjointly split into $K$ sub-partitions $\mathcal{P}_{k}=\{\textbf{x}_{k,i}, y_{k,i}\}^{N_{k}}_{i=1}$ s.t. $\mathcal{D} = \bigcup_{k=1}^{K} \mathcal{P}_{k}$, which are distributed to the corresponding local model $f_k$. Let $R$ be the total number of the communication rounds and $r$ denote the index of the $r_{th}$ communication round. At the first round $r$=$1$, the global model $f_g$ initialize the global weights $\theta^{(1)}_{f_g}$ and broadcasts $\theta^{(1)}_{f_g}$ to an arbitrary subset of local models that are available for training at round $r$, such that $\mathcal{F}^{(r)}\subset\mathcal{F}$, $|\mathcal{F}^{(r)}|=K^{(r)}$, and $K^{(r)} \leq K$, where $K^{(r)}$ is the number of available local models at round $r$. Then the active local models $f_{k}\in\mathcal{F}^{(r)}$ perform local training to minimize loss $\mathcal{L}( \theta^{(r)}_{k})$ on the corresponding sub-partition $\mathcal{P}_{k}$ and update their local weights $\theta^{(r+1)}_{k} \leftarrow \theta^{(r)}_{k}-\eta\nabla\mathcal{L}(\theta^{(r)}_{k})$, where $\theta^{(r)}_{k}$ is the set of weights for the local model $f_k$ at round $r$ and $\mathcal{L}(\cdot)$ is the loss function. When the local training is done, the global model $F$ collects and aggregates the learned weights $\theta^{(r+1)}_{f_g} \leftarrow \frac{N_{k}}{N}\sum_{i=1}^{K^{(r)}} \theta_{k}^{(r)}$ and then broadcasts newly updated weights to the local models available at the next round $r+1$. These learning procedures are repeated until the final round $R$. This is the standard setting for centralized federated learning, which aims to find a single global model that works well across all local data. On the other hand, Personalized Federated Learning aims to adapt the individual local models $f_{1:K}$ to their local data distribution $\mathcal{P}_{1:K}$, to obtain specialized solution for each task at the local client, while utilizing the knowledge from other clients. Thus merging the local knowledge for personalized FL is not necessarily done in the form of $\theta^{(r+1)}_{f_g} \leftarrow \frac{N_{k}}{N}\sum_{i=1}^{K^{(r)}} \theta_{k}^{(r)}$, and the specific ways to utilized the knowledge from others depends on the specific algorithm, i.e. $\theta^{(r+1)}_k\leftarrow \theta^{(r)}_k + \sum_{i\neq k}^{K^{(r)}} \omega_i(\theta_{k}^{(r)}-\theta_{i}^{(r)})$, wher $\omega(\cdot)$ is weighing function ~\citep{zhang2021personalized}.


\begin{figure*}[t]
% \vspace{-0.3in}
\small
    \centering
    \includegraphics[width=\textwidth]{figures/images/fig_factorize.pdf} 
    \vspace{-0.25in}
    \caption{\small{\textbf{Illustration of Parameter Factorization Methods:} Left shows conventional matrix factorization with two low rank matrices with rank $\gamma$. Middle represents the method utilizing Hadamard product of low rank matrix for federated learning~\citep{anonymous2022fedpara}. Right illustrates our factorization method for agnostic personalized federated learning, which utilizes rank 1 vectors and highly sparse bias.} }
    \label{fig:factorize}
    \vspace{-0.15in}
\end{figure*}

\begin{figure}[t]
\small
\centering
\vspace{-0.05in}
\begin{tabular}{c c}
    \small    
    
    \hspace{-0.15in} \includegraphics[width=0.24\textwidth]{figures/images/permuted_uv.pdf} & 
    \hspace{-0.225in} \includegraphics[width=0.24\textwidth]{figures/images/domain_uv.pdf}
    \\
    \hspace{-0.1in} (a) MNIST Part. 2 (Permuted) &
    \hspace{-0.125in} (b) CIFAR-10 (Hetero-Domain)
\end{tabular}
\vspace{-0.1in}
\caption{\small{\textbf{Analysis of $\textbf{u}$ and $\textbf{v}$:}} We plot normalized $L_2$ distance of the gradient updates of factorized parameters $\textbf{u}$ and $\textbf{v}$ while learning on (a) MNIST Partition 2 and (b) CIFAR-10 compared to learning on MNIST Partition 1. }
\label{fig:f_analysis}
\vspace{-0.25in}
\end{figure}



\subsection{Agnostic Personalized Federated Learning}
\label{subsec:apfl}

Agnostic Personalized Federated Learning (APFL) is a scenario where any local participants from diverse domains with their own personalized labeling schemes can collaboratively learn, benefiting each other. There exist two critical challenges that need to be tackled to achieve this objective: (1) Label Heterogeneity and (2) Domain Heterogeneity.

\vspace{-0.125in}
\paragraph{Label Heterogeneity } This scenario assumes that the labeling schemes are not perfectly synchronized across all clients, as described in Section~\ref{sec:intro} and Figure~\ref{fig:overview} Left. Most underlying setting for this scenario is the same as the conventional single-domain setting with synchronized labels that is described in Section~\ref{subsec:fl}, except that labels are arbitrarily permuted amongst clients. The local data $\mathcal{P}_{k}$ for the local model $f_k$ is now defined as $\mathcal{P}_{k}=\{\textbf{x}_{k,i}, \varphi_{k}(y_{k,i})\}^{N_{k}}_{i=1}$, where $\varphi_{k}(\cdot)$ is a mapping function for the local model $f_k$ which maps a given class $y_{k,i}$ with a randomly permuted label $p_{k,i}$=$\varphi_{k}(y_{k,i})$. Let the $j$th layer out of $L$ layers in the neural networks of local model $f_k$ be $\ell_{k}^{j}$ and the last layer $\ell_{k}^L$ be the classifier layer. Since each client has differently permuted labels, the personalized classifiers $\ell_{1:K}^{L}$ are no longer compatible to each other. While we can merge the layers below the classifier in this setting, training with heterogeneous labels could still lead to large disparity in the local gradients even in the initial communication round, as described in Figure \ref{fig:concept}.


\vspace{-0.125in}
\paragraph{Domain Heterogeneity} This scenario presumes that local clients learn on their own dataset $\mathcal{D}$, that are completely different from the datasets that are used at other clients, as described in Section~\ref{sec:intro} and Figure~\ref{fig:overview} Right. In this setting, $K$ disjoint datasets $\mathcal{D}_{1:K}$ are assigned to the $K$ local clients $f_{1:K}$, where $\mathcal{D}_k=\{\textbf{x}_{k,i}, y_{k,i}\}^{N_k}_{i=1}$ is the dataset assigned to the local model $f_k$. The number of target classes may differ across clients, such that $y_{k,i} \in \{1,\dots,C_k\}$. We assume complete disjointness across clients, such that there is no instance-wise and class-wise overlap across the datasets: $\varnothing =\bigcap_{k=1}^K \mathcal{D}_k$. Similarly to the label-heterogeneous scenario described above, the personalized classifiers $\ell_{1:K}^{L}$ are no longer compatible to each other due to the heterogeneity in the data and the labels. Hence, the aggregation is done for the layers before the classifier, but they will be also incompatible as the learned model weights will be largely different across domains.


%$\varnothing =\bigcap_{k=1}^K \mathcal{P}_k$



\section{Factorized Federated Learning}
\label{sec:method}

We now provide detailed descriptions of our novel algorithm \texttt{Factorized-FL}. 

\subsection{Kernel Factorization}
\label{subsec:facto}

\citet{Wang2020Federated} discussed that the conventional knowledge aggregation, that is often performed in a coordinate-wise manner, may have severe detrimental effects on the averaged model. This is because the deep neural networks have extremely high-dimensional parameters and thus meaningful element-wise neural matching is not guaranteed when aggregating the weights across different models trained under diverse settings. 

One naive solution to this problem is to factorize model parameters into lower dimensional space, i.e. low rank matrices, as shown in Figure~\ref{fig:factorize} (Left). Conventional approaches, such as SVD, Tucker, or Canonical Polyadic decomposition, however, factorize model parameters after training~\citep{lebedev2014speeding,phan2020stable} is done. Thus, the dimensionality at the time of knowledge aggregation will remain the same as the unfactorized model. ~\citet{konevcny2016federated,anonymous2022fedpara} pre-decompose model parameters to low rank matrices for FL scenarios. While~\citet{konevcny2016federated} use naive low rank matrices, ~\citet{anonymous2022fedpara} uses two sets of low rank matrices to improve expressiveness and utilize them as global and local weights (Figure~\ref{fig:factorize} (Middle)). Unlike prior works, our approach utilizes rank-1 vectors to perform aggregation in the lowest subspace possible for compatibility, while effectively yet efficiently enhancing expressiveness with sparse bias matrices, as shown in Figure~\ref{fig:factorize} (Right) and Figure~\ref{fig:factorize_detail}. Another crucial difference of our method from the previous factorization methods is that, our rank-1 vectors have distinct roles. Our factorization will separate the common knowledge from the task- or domain-specific knowledge, since $\textbf{u}$ could be thought as the bases  (the common knowledge across clients) and $\textbf{v}$ could be thought as the coefficients (client-specific information). 
%Specifically, $\textbf{u}$ captures , and $\textbf{v}$ as describing . 

In Figure~\ref{fig:f_analysis} (a) and (b), which shows the experimental results with the factorized model, we observe that $\textbf{u}$ trained on two datasets becomes closer to that of another dataset (MNIST Partition 1) while $\textbf{v}$ (personalized filter coefficient) remain largely different as federated learning goes on. With this observation, we further aggregate $\textbf{u}$ while allowing $\textbf{v}$ to be different across clients, to allow personalized FL. Further, we use the client specific $\textbf{v}$ for similarity matching, to identify relevant local models from other clients. In following paragraphs, we describe our factorization method in detail, for both fully-connected and convolutional layers. \vspace{-0.15in}


\begin{figure}
\small
    \centering
    \includegraphics[width=0.41\textwidth]{figures/images/factorize.pdf} 
    \vspace{-0.05in}
    \caption{\small{\textbf{Illustration of Kernel Factorization \& Reconstruction}} (1) we multiply two factorized vectors $\textbf{u}$ and $\textbf{v}$ to obtain kernel matrix. (2) we add the sparse bias matrix $\mu$ to complement non-linearity. (3) we reshape the matrix into the original kernel shape. }
    \label{fig:factorize_detail}
    \vspace{-0.3in}
\end{figure}

% \begin{figure*}[t]
% \small
% \centering
% \vspace{-0.05in}
% \begin{tabular}{c c c}
%     \small    
%     \hspace{-0.15in}
%     \includegraphics[width=0.39\textwidth]{figures/images/factorize.pdf} &
%     \hspace{-0.2in} 
%     \includegraphics[width=0.3\textwidth]{figures/images/permuted_uv.png} & 
%     \hspace{-0.15in}
%     \includegraphics[width=0.3\textwidth]{figures/images/domain_uv.png}
%     \\
%     \hspace{-0.1in} (a) Illustration of Kernel Factorization &
%     \hspace{-0.1in} (b) Label Heterogeneous Scenario &
%     \hspace{-0.1in} (c) Domain Heterogeneous Scenario

% \end{tabular}
% \vspace{-0.1in}
% \caption{\small{\textbf{Analysis of our Kernel Factorization method}} (a) describes our kernel factorization process. Following two plots show $L_2$ distance of the gradient updates of factorized parameters $\textbf{u}$ and $\textbf{v}$ in (b) label heterogeneous and (c) domain heterogeneous scenarios. }
% \label{fig:factorize}
% \vspace{-0.1in}
% \end{figure*}




\paragraph{Factorization of Fully-Connected Layers} We assume that each local model $f_k$ has a set of local weights $\theta_k$ across all layers; that is, $\theta_k=\{\textbf{W}^{i}_k\}^{L}_{i=1}$. The dimensionality of the dense weight $\textbf{W}_k^{i}$ for each fully connected layer is $\textbf{W}_k^{i} \in \mathbb{R}^{I \times O}$, where $I$ and $O$ indicate respective input and output dimensions. We can reduce the $I \times O$ complexity by factorizing the high order matrix into the outer product of two vectors as follows:
\begin{equation}
\begin{split}
\textbf{W}_k^{i} = \textbf{u}_k^{i} \times \textbf{v}_k^{i \intercal}, \text{where } \textbf{u}_k^{i} \in \mathbb{R}^{I}, \textbf{v}_k^{i} \in \mathbb{R}^{O}
\end{split}
\end{equation}
 However, such extreme factorization of the weight matrices may result in the loss of expressiveness in the parameter space. Thus, we additionally introduce a highly sparse bias matrix $\mu$ to further capture the information not captured by the outer product of the two vectors as follows:
\begin{equation}
\begin{split}
\textbf{W}_k^{i} = \textbf{u}_k^{i} \times \textbf{v}_k^{i \intercal} \oplus {\mu}_k^{i}, 
\text{where } \\ \textbf{u}_k^{i} \in \mathbb{R}^{I},  
\textbf{v}_k^{i} \in \mathbb{R}^{O}, {\mu}_k^{i} \in \mathbb{R}^{I \times O} 
\end{split}
\end{equation}
We initialize $\mu$ with zeros so that it can gradually capture the additional expressiveness that are not captured by $\textbf{u}$ and $\textbf{v}$ during training. We can control its sparsity by the hyper-parameter for the sparsity regularizer described in~\ref{subsec:algo}.

\vspace{-0.05in}
\paragraph{Factorization of Convolutional Layers} The difference between the fully-connected and convolutional layers is that the convolutional layers have multiple kernels (or filters) such that $\textbf{W}_k^{i} \in \mathbb{R}^{F \times F \times I \times O}$, where $F$ is a size of filters (we assume the filter size is equally paired for the simplicity). To induce $\textbf{u}$ to capture base filter knowledge and $\textbf{v}$ to learn filter coefficient, it is essential to design $\textbf{u}\in \mathbb{R}^{F \cdot F}$ and $\textbf{v}\in \mathbb{R}^{I \cdot O}$, but not in arbitrary ways, such as $\textbf{u}\in \mathbb{R}^{I \cdot F}$ and $\textbf{v}\in \mathbb{R}^{O \cdot F}$ or $\textbf{u}\in \mathbb{R}^{O}$ and $\textbf{v}\in \mathbb{R}^{I \cdot F \cdot F}$. We observe that performance is degenerated when the parameters are ambiguously factorized  (Figure~\ref{fig:analysis} (h)). Our proposed factorization method for convolutional layers are as follows:
\vspace{-0.1in}
\begin{equation}
\begin{split}
\textbf{W}_k^{i} = \pi(\textbf{u}_k^{i} \times \textbf{v}_k^{i \intercal} \oplus {\mu}_k^{i}), \text{where }  \textbf{u}_k^{i} \in \mathbb{R}^{F \cdot F }, \textbf{v}_k^{i} \in \mathbb{R}^{I \cdot O }, \\
{\mu}_k^{i} \in \mathbb{R}^{F \cdot F \times I \cdot O}, \pi(\cdot): \mathbb{R}^{F \cdot F \times I \cdot O} \rightarrow \mathbb{R}^{F \times F \times I \times O},
\end{split}
\end{equation}
% \vspace{-0.05in}
$\pi(\cdot)$ is the weight reshaping function. Note that we reparameterize our model \textit{at initialization time}. Then we reconstruct and train full weights of each layer $\textbf{W}_k^{1:L}$, while optimizing $\textbf{u}_k^{1:L}$, $\textbf{v}^{1:L}_k$, and ${\mu}^{1:L}_k$, respectively, during training phase. 






\subsection{Similarity Matching}
\label{subsec:sim}


% \begin{figure*}[t]
% % \vspace{-0.3in}
% \small
%     \centering
%     \includegraphics[width=\textwidth]{figures/images/framework.pdf} 
%     \vspace{-0.2in}
%     \caption{\small{\textbf{Illustration of Factorized-FL Framework:}} We match relevant clients based on the model representations obtained by the criteria input. Then we reflect the relevant knowledge based on their similarity. As our kernel weights are factorized, knowledge collapse can be effectively reduced when performing personalized knowledge reflection.}
%     \label{fig:framework}
%     \vspace{-0.15in}
% \end{figure*}
\begin{figure}[t]
% \vspace{-0.3in}
\small
    \centering
    % \includegraphics[width=0.47\textwidth]{figures/images/factorize.pdf}\\ 
    \includegraphics[width=0.49\textwidth]{figures/images/framework_2.pdf} 
    \vspace{-0.3in}
    \caption{\small{\textbf{Illustration of Similarity Matching:}} We match relevant clients utilizing the factorized vector $\textbf{v}$ that captures client-specific knowledge. Then we aggregate $\textbf{u}$ based on the similarity. }
    \label{fig:framework}
    \vspace{-0.2in}
\end{figure}



Since we assume task- and domain-heterogeneous FL scenarios, aggregating the parameter bases across all clients may not be optimal, since some of them could be highly irrelevant. \citet{yoon2021federated} and \citet{zhang2021personalized} also demonstrated that avoiding aggregation of irrelevant models from other clients improves local model performance. \citet{yoon2021federated} achieve this goal by taking the weighted combination of task-specific weights from other clients, and \citet{zhang2021personalized} suggest downloading the models from other clients and evaluating their performance on a local validation set, at each client. However, since they require additional communication and computing cost at the local clients, we provide a more efficient yet effective approach to find and match models that are beneficial to each other. 

\vspace{-0.1in}
\paragraph{Efficient similarity matching} Our method utilizes factorized vector $\textbf{v}_k$ for measuring similarity across different models, at the central server. Since $\textbf{v}$ are devised to learn personalized coefficient, we assume that clients trained on similar task or domain will have similar $\textbf{v}$. Specifically, we only use $\textbf{v}^{L-1}_{k}$ of the second last layer (before classifier layer) for similarity matching. The similarity matching function $\Omega(\cdot)$, is defined as the cosine similarity between target client $f_k$ and the other clients $\{f_i\}_{i\neq k}^K$, as follows:
\begin{equation}
\label{eq:sim}
\begin{split}
\Omega(\textbf{v}_{f_k}^{L-1}, \textbf{v}_{f_{i\neq k:K}}^{L-1}) = \{\sigma_i |  
\sigma_i = \frac{\textbf{v}_{f_k} \cdot \textbf{v}_{f_i}}{\|\textbf{v}_{f_k}\|\|\textbf{v}_{f_i}\|}
, \sigma_i \geq \tau \}_{i \neq k}^{K}
\end{split}
\end{equation}
The similarity scores for those with the cosine similarity scores lower than the given threshold $\tau$, are set to zero.
Our method is significantly more efficient than similarity matching approaches which use full gradient updates for clustering clients~\citep{sattler2019clustered,duan2021fedgroup}.

\vspace{-0.1in}
\paragraph{Personalized weighted averaging} We allow each local model to perform weighted aggregation of the model weights from other clients, utilizing their similarity scores:
\begin{equation}
\begin{split}
\textbf{u}_{k}^l \leftarrow \frac{\text{exp}({\epsilon \cdot \sigma_i})}{\sum_{i=1}^K \text{exp}(\epsilon \cdot \sigma_i)} \sum_{i=1}^{K} \textbf{u}_i^l, s.t. \forall  l \in \{1,2,\dots,L\}
\end{split}
\end{equation}
where $\epsilon$ is a hyperparameter for scaling the similarity score $\sigma_i$. We always set $\sigma_k$, the similarity score for itself, as $1.0$. 

\subsection{Learning Objective}
\label{subsec:algo}

Now we describe our final learning objective. Instead of utilizing the single term $\theta_k$ for local weights of neural network $f_k$, now let $\mathcal{U}_k$, $\mathcal{V}_k$, and $\mathcal{M}_k$ be sets of $\textbf{u}_k$, $\textbf{v}_k$, and $\mu_k$ of all layers in $f_k$, s.t. $\mathcal{U}_k=\{\textbf{u}^{i}_k\}^{L}_{i=1}$, $\mathcal{V}_k=\{\textbf{v}^{i}_k\}^{L}_{i=1}$, and $\mathcal{M}_k=\{{\mu}^{i}_k\}^{L}_{i=1}$, then our local objective function is,
\begin{equation}
\begin{split}
\min_{\mathcal{U}_k,\mathcal{V}_k,\mathcal{M}_k} \sum_{\mathcal{B} \in \mathcal{D}_k} \mathcal{L}(\mathcal{B}; \mathcal{U}_k, \mathcal{V}_k, \mathcal{M}_k) + \lambda_{\text{sparsity}} ||\mathcal{M}_k||_1,
\end{split}
\label{eq:loss}
\end{equation}
where $\mathcal{L}$ is the standard cross-entropy loss performed on all minibatch $\mathcal{B} \in \mathcal{D}_k$. We add the $L_1$ sparsity inducing regularization term to make the bias parameters highly sparse, controlling its effect with a hyperparameter $\lambda_{\text{sparsity}}$. Please see our pseudo-coded algorithm Algorithm 1 in Appendix~\ref{appdx:algorithm}.














%We also study such detrimental effects in Section~\ref{sec:intro} and Figure~\ref{fig:concept}. Even with the same initialization, models become extremely heterogeneous and incompatible to each other by label and domain heterogeneity in federated learning scenarios (Figure~\ref{fig:concept} (c)). 

%Reducing the dimensionality of model parameters may alleviate such issues caused by the coordinate-wise knowledge aggregation in the high dimensional space. 


%We also evolve each factorized vectors into one capturing the common knowledge and the other capturing the client-specific knowledge for personalized FL, where the latter is also used to measure the inter-model similarity, which is another crucial difference from previous factorization methods (Figure~\ref{fig:factorize} (Right)). 


% The prior work recommends to set proper rank $\gamma$ values for expressiveness, which increases dimensionality, our method simply uses rank 1 vectors, such as $\textbf{u}$ and $\textbf{v}$ (Figure~\ref{fig:factorize} Right). Our method can separately learn client-general (filter base) and client-specific (filter coefficient) knowledge and utilize them individually. 


%. In Figure~\ref{fig:f_analysis} (a) and (b), We train three (factorized) ResNet-9 networks on CIFAR-10 dataset and show their performance. Even with only $77\%$ of model capacity, our method show superior performance over the Hadamard product approach and even the full kernel model. 

%This results show that our factorization method can separate general and specific knowledge from the extremely high dimensional space while significantly reducing the dimensionality of parameter space, which has great applicability to extremely heterogeneous federated learning scenarios that we want to tackle.

%Such naive aggregation could be less dangerous if there exist some homogeneity across different clients, for example, if the private local data that the each model trains on have the same distribution, and if all models have the same initializations. However, when the local models learn on highly heterogeneous tasks from diverse domains, such simple aggregation scheme may lead to inter-client interference~ \citep{yoon2021federated} where the aggregated model achieves lower performance than those obtained by the local models without aggregation.

% To overcome such issues with simple parameter averaging, we aim to learn a low-dimensional subspace on which the projects of the parameters across different clients become more compatible to each other. To this end, we factorize the parameters of the local models into lower-rank vector components. 



%This efficient similarity matching is another clear advantage of our factorized-FL framework.

%Without additional training or evaluation process at each local client~\citep{yoon2021federated,zhang2021personalized}, 

%\paragraph{Effectiveness} Since $\textbf{v}$ are devised to learn personalized coefficient, we assume that clients trained on similar task or domain will have similar $\textbf{v}$. Thus, we can easily identify which clients are relevant to each other by measuring distance of $\textbf{v}$. We demonstrate its effectiveness by showing that clients are successfully clustered by simply matching $\textbf{v}_{f_k}^{L-1}$ in domain and label heterogeneous scenarios, in Figure~\ref{} (detailed explanation is described in Section~\ref{sec:exp}).



%We can further avoid aggregating irrelevant clients whose similarity scores are lower than the given threshold $\tau$.  

%Note that we omit the layer-wise notation for simplicity.
%where $\Theta$ is a set of $K$ local weights maintained at the server and $\Omega(\cdot)$ is a similarity function that returns a set of pairs of the similarity score $\sigma_i$ and the corresponding local weights $\theta_i$ for all $i \in \{1,2,\dots,K\}$, while satisfying the similarity scores are greater or equal to the given threshold $\tau$. 

% We propose a scheme to aggregate the parameters of only the most relevant local models, based on their task-level similarity. 
%\vspace{-0.05in}
% \input{algorithms/algo_factorized_fl}

% \begin{figure*}[t]
\small
\vspace{-0.1in}
    \small
    \centering
    \begin{tabular}{c c c c c c}
        \small    
        \hspace{-0.15in}
        \includegraphics[width=0.155\textwidth]{figures/images/plain_l1_1.png} &
        \hspace{-0.15in} \includegraphics[width=0.155\textwidth]{figures/images/plain_l1_100.png} &
        \hspace{-0.15in} \includegraphics[width=0.155\textwidth]{figures/images/u_l1_1.png} &
        \hspace{-0.15in} \includegraphics[width=0.155\textwidth]{figures/images/u_l1_100.png} &
        \hspace{-0.15in} \includegraphics[width=0.155\textwidth]{figures/images/v_l1_1.png} &
        \hspace{-0.15in} \includegraphics[width=0.155\textwidth]{figures/images/v_l1_100.png} \\
        
        \vspace{-0.05in}
        \hspace{-0.15in}
        
        % \includegraphics[width=0.155\textwidth]{figures/images/plain_l4_1.png} &
        % \hspace{-0.15in} \includegraphics[width=0.155\textwidth]{figures/images/plain_l4_100.png} &
        % \hspace{-0.15in} \includegraphics[width=0.155\textwidth]{figures/images/u_l4_1.png} &
        % \hspace{-0.15in} \includegraphics[width=0.155\textwidth]{figures/images/u_l4_100.png} &
        % \hspace{-0.15in} \includegraphics[width=0.155\textwidth]{figures/images/v_l4_1.png} &
        % \hspace{-0.15in} \includegraphics[width=0.155\textwidth]{figures/images/v_l4_100.png} \\
        
        \includegraphics[width=0.155\textwidth]{figures/images/plain_l8_1.png} &
        \hspace{-0.15in} \includegraphics[width=0.155\textwidth]{figures/images/plain_l8_100.png} &
        \hspace{-0.15in} \includegraphics[width=0.155\textwidth]{figures/images/u_l8_1.png} &
        \hspace{-0.15in} \includegraphics[width=0.155\textwidth]{figures/images/u_l8_100.png} &
        \hspace{-0.15in} \includegraphics[width=0.155\textwidth]{figures/images/v_l8_1.png} &
        \hspace{-0.15in} \includegraphics[width=0.155\textwidth]{figures/images/v_l8_100.png} \\
        
        
        \hspace{-0.1in} $E=1$ &
        \hspace{-0.1in} $E=100$ &
        \hspace{-0.05in} $E=1$ &
        \hspace{-0.05in} $E=100$ &
        \hspace{-0.05in} $E=1$ &
        \hspace{-0.05in} $E=100$ \\
        
        \multicolumn{2}{c}{(a) Regular Kernel } &
        \multicolumn{2}{c}{(b) $\textbf{u}$ of Factorized Kernel}  &
        \multicolumn{2}{c}{(c) $\textbf{v}$ of Factorized Kernel}
        
    \end{tabular}
    \vspace{-0.1in}
    \caption{\small{\textbf{Euclidean Distance of Parameters in Label Heterogeneous Scenario}} Top row indicates the first conv layer and bottom row represents the last conv layer. The brighter color represents the closer the parameters to each other (see Appendix C for full comparison). }
    \label{fig:distance}
    \vspace{-0.1in}
\end{figure*}



% \paragraph{Personalized Knowledge Reflection} We avoid averaging the local knowledge directly. Instead, we individually reflect only difference between target local model and the other clients to preserve local reliability, inspired by~\cite{zhang2021personalized}. First, given local model $f_k$, we select the other beneficial knowledge that are relevant to $f_k$ via $\Omega(\cdot)$ described in Eq.~\ref{eq:sim}, returning a set of $J$ pairs of $(\sigma_j, \zeta_j,\xi_j,\phi_j)$. Second, we separately update each factorized parameters $\zeta_j,\xi_j,$ and $\phi_j$ while minimizing the collapse of information as follows:
% \begin{equation}
% \begin{split}
% \zeta_{k} \leftarrow \zeta_{k} + \frac{exp(\alpha \cdot \sigma_j)}{\sum_{1:J} exp(\alpha \cdot \sigma_j)} \cdot \sum_{j=1}^{J} (\zeta_{j}-\zeta_{k}),
% \end{split}
% \end{equation}
% where $\alpha$ is scalar value for scaling the reflection ratio of similarity score $\sigma_j$. This personalized knowledge reflection is also equally applied to $\xi_j$ and $\phi_j$, respectively (please see Algorithm~\ref{algo:simfed}).




% Our idea is to view the individual local models as respective encoders that can effectively transform their locally learned knowledge to their own personalized vector embeddings. If we equally inject the same identical input to all encoders, then they will interpret the criteria inputs depending on what they have learned in their own local environments and produce their own personalized transformations, which we can efficiently utilize them as representations of local knowledge to measure the relevance. Formally, the neural networks $f_{k}$ can be represented as:
% \begin{equation}
% \begin{split}
% f_{k}(\hat{\textbf{y}}|\textbf{x}) = P_k(\hat{\textbf{y}}|E_k(\textbf{z}|\textbf{x})),
% \end{split}
% \end{equation}
% where $f_{k}$ is a combination of an encoder  $E_{k}(\textbf{z}|\textbf{x})$ and a predictor $P_{k}(\hat{\textbf{y}}|\textbf{z})$, where $\textbf{z} \in \mathbb{R}^d$ is $d$-length transformed latent vector of input data \textbf{x} and $\hat{\textbf{y}} \in \mathbb{R}^C$ is a logit for the final prediction. We generate the unbiased criteria input $\textbf{x}_{\mathcal{N}} \in \mathbb{R}^{W \times H \times D}$ from the Gaussian normal distribution, such that ${x}_{i} \sim \mathcal{N}(0,1)$ where $x_i$ is its element, and assume that it is located at the server. Since we have $K$ local models $f_{1:K}$, we equally feed the single criteria input $\textbf{x}_{\mathcal{N}}$ to $K$ local encoders so that we can generate a set of $K$ multiple personalized transformations $\mathcal{Z}_\mathcal{N}$ as follows:
% \begin{equation}
% \begin{split}
% \mathcal{Z}_\mathcal{N} = \{\textbf{z}_{\mathcal{N},k} |  \textbf{z}_{\mathcal{N},k}=E_{k}(\textbf{z}|\textbf{x}_{\mathcal{N}})  \}^{K}_{k=1}
% \end{split}
% \end{equation}


% Now our remaining task is to compute the relevancy between the transformations $\mathcal{Z}_\mathcal{N}$ which each local knowledge is efficiently encoded. We use the cosine similarity to measure the relevance between target model $f_k$ with the rest of models in $\mathcal{F}$, as follows:
% \begin{equation}
% \label{eq:sim}
% \begin{split}
% \Omega(k, \tau, \mathcal{Z}_\mathcal{N}, \Theta) = \{(\sigma_i, \theta_i) | \sigma_i = \frac{\textbf{z}_k \cdot \textbf{z}_i}{\|\textbf{z}_k\|\|\textbf{z}_i\|}, \\ \sigma_i \geq \tau,\textbf{z}_i\in\mathcal{Z},\theta_i\in\Theta \}_{i=1}^{K},
% \end{split}
% \end{equation}
% where $\Theta$ is a set of $K$ local weights maintained at the server and $\Omega(\cdot)$ is a similarity function that returns a set of pairs of the similarity score $\sigma_i$ and the corresponding local weights $\theta_i$ for all $i \in \{1,2,\dots,K\}$, while satisfying the similarity scores are greater or equal to the given threshold $\tau$. 
\section{Experiments}

\label{Sec:experiments}



\begin{table*}[t]
    \small
    \centering
    \caption{Node classification performance (Accuracy(\%)$\pm$Std) on various types of noisy graphs}
    \vskip -1.5em
    \begin{tabularx}{0.985\textwidth}{|p{0.05\textwidth}|p{0.14\textwidth}|CC>{\centering\arraybackslash}p{0.1\linewidth}C>{\centering\arraybackslash}p{0.1\linewidth}>{\centering\arraybackslash}p{0.08\linewidth}CC|}
    \hline
    Dataset & Graph & GCN & SuperGAT &Self-Training & RGCN & GCN-jaccard & GCN-SVD & Pro-GNN & Ours \\
    \hline
    
    \multirow{4}{*}{Cora}
        &Raw Graph            & 65.5 $\pm 0.5$& 69.0 $\pm 1.7$ & 67.9 $\pm 0.9$ & 63.0 $\pm 0.7$ &65.7 $\pm 0.6$ & 62.9 $\pm 1.1$  & 65.9 $\pm 1.3$ & \textbf{75.3} $\pm \textbf{0.6}$\\
        &Random Noise        & 59.2 $\pm 0.7$ & 58.8 $\pm 0.4$ & 63.1 $\pm 0.5$ &51.5 $\pm 0.7$ & 57.8 $\pm 1.4$ & 51.5 $\pm 0.7$ & 56.1 $\pm 3.0$ & \textbf{71.8} $\pm \textbf{1.5}$\\
        &Non-Targeted Attack  & 26.8 $\pm 2.5$ & 41.5 $\pm 1.6$ & 29.6 $\pm 0.4$ &30.4 $\pm 1.0$ & 48.3 $\pm2.0$ & 37.1 $\pm 1.4$ & 41.7 $\pm 5.7$& \textbf{70.8} $\pm \textbf{0.7}$  \\
        &Targeted Attack      & 45.3 $\pm 1.2$& 44.4 $\pm 1.3$ &46.7 $\pm 2.1$ &40.3 $\pm 1.0$ & 49.5 $\pm 1.0$ & 44.8 $\pm 0.7$ & 49.7 $\pm 0.9$ & \textbf{67.8} $\pm \textbf{1.2}$ \\

    \hline
    \multirow{4}{*}{Cora-ML}
        &Raw Graph         & 72.4 $\pm 0.8$ & 73.8 $\pm 1.4$ & 72.7 $\pm 1.4$ & 72.9 $\pm 0.7$ & 71.0 $\pm 1.2 $ & 71.1 $\pm 1.0$ & 62.0 $\pm 1.5$ & \textbf{75.6} $\pm \textbf{0.4}$\\
        &Random Noise       & 62.3 $\pm 0.6$ & 63.7 $\pm 0.9$ & 62.8 $\pm 1.3$ & 61.4 $\pm 1.1$ & 61.3 $\pm 0.5$ & 62.6 $\pm 0.6$ & 57.1 $\pm 2.1$ & \textbf{72.9} $\pm \textbf{0.7}$\\
        &Non-Targeted Attack & 13.2 $\pm 1.4$ & 18.6 $\pm 1.5$ & 15.0 $\pm 0.7$ & 11.0 $\pm 1.0$ & 48.9 $\pm 5.3$ & 16.3 $\pm 0.6$ & 18.2 $\pm 2.4$ & \textbf{73.2} $ \pm \textbf{1.2}$ \\
        &Targeted Attack     & 55.7 $\pm 0.7$ & 56.5 $\pm 1.7$ & 57.7 $\pm 1.2$ & 54.6 $\pm 0.6$ & 61.2 $\pm 0.9$ & 53.0 $\pm 0.8$ & 55.1 $\pm 1.6$ & \textbf{70.8} $\pm \textbf{0.7}$ \\

    \hline
    \multirow{4}{*}{Citeseer}
        &Raw Graph           & 64.8 $\pm 1.4$ & 64.2 $\pm 1.7$ & 65.7 $\pm 1.1$ & 56.6 $\pm 1.2$ & 62.2 $\pm 2.0$ & 61.3 $\pm 2.0$ & 60.6 $\pm 2.0 $  & \textbf{71.2} $\pm \textbf{1.4}$\\
        &Random Noise       & 57.0 $\pm 1.2$ & 54.6 $\pm 1.3$ & 58.7 $\pm 2.1$ & 48.2 $\pm 1.2$ & 61.1 $\pm 2.8$ & 48.3 $\pm 1.6$ & 54.4 $\pm 2.6$ & \textbf{68.8} $\pm \textbf{1.5}$\\
        &Non-Targeted Attack & 26.6 $\pm 2.5$ & 42.3 $\pm 2.6$ & 28.8 $\pm 2.7$ &26.6 $\pm 1.1$ & 57.9 $\pm 2.7$ & 41.7 $\pm 1.6$ & 41.6 $\pm 3.1$ & \textbf{68.0} $\pm \textbf{0.4}$ \\
        &Targeted Attack     & 43.9 $\pm 1.7$ & 42.9 $\pm 0.4$ & 47.6 $\pm 1.2$ &35.3 $\pm 1.5$ & 52.5 $\pm 2.3$ & 40.5 $\pm 0.7$ & 48.1 $\pm 1.6$ & \textbf{67.2} $\pm \textbf{1.3}$\\
    \hline
    \multirow{4}{*}{Pubmed}
    & Raw Graph          & 85.9 $\pm 0.1$ & 86.0 $\pm 1.2$ & 86.1 $\pm 0.2$ & 85.1 $\pm 0.1$ & 86.0 $\pm 0.1$  & 83.0 $\pm 0.1$ & 86.1 $\pm 0.1$ & \textbf{86.9} $\pm \textbf{0.1}$  \\
    & Random Noise & 80.5 $\pm 0.1$ & 79.8 $\pm 0.1$ & 81.2 $\pm 0.2$ & 79.7 $\pm 0.1$ &  83.0 $\pm 0.1$  & 82.0 $\pm 0.1$ &  85.1 $\pm 0.2$ &  \textbf{86.4} $\pm \textbf{0.1}$\\
    & Non-Targeted Attack & 73.7 $\pm 0.2$ & 73.8 $\pm 0.2$ & 73.5 $\pm 0.3$ & 73.8 $\pm 0.3$ & 84.4 $\pm 0.1$ & 83.0 $\pm 0.1$ & 86.0 $\pm$ 0.1 & \textbf{86.3} $\pm \textbf{0.1}$ \\
    & Targeted Attack & 76.5 $\pm 0.1$ & 75.6 $\pm 0.1$ & 76.8 $\pm 0.2$ & 76.2 $\pm 0.2$  & 82.7 $\pm 0.2$ &78.1 $\pm 1.3$ & 79.1 $\pm 0.1$ & \textbf{84.3} $\pm \textbf{0.2}$ \\
    \hline
    \end{tabularx}
    
    \label{tab:results}
    \vskip -1.2em
\end{table*}

In this section, we evaluate the proposed RS-GNN on noisy graphs with limited labels to answer the following research questions:
\begin{itemize}[leftmargin=*]
    \item \textbf{RQ1} How robust is the proposed framework on various types of noisy graphs with limited labeled nodes?
    \item \textbf{RQ2} How does the proposed framework perform under various label rates and graph sparsity levels?
    \item \textbf{RQ3} What are the contributions of link predictor and label smoothness regularization from predicted edges on RS-GNN?
\end{itemize}
\subsection{Experimental Settings}
\label{Sec:ex_settings}


\subsubsection{Datasets} 
\label{Sec:datasets}
For a fair comparison, we conduct experiments on four widely used benchmark datasets, i.e., Cora, Cora-ML, Citeseer and Pubmed~\cite{sen2008collective}.
The statistics of the datasets are presented in the Table \ref{tab:dataset} in Appendix. Note that the split of validation and testing on all datasets are the same as described in the cited papers to keep consistence. For the training set, we randomly sample 1\% of nodes as the labeled set for Cora, Cora-ML and Citeseer. For Pubmed, we randomly sample 10\% of nodes to compose the labeled set. The training node set doesn't overlap with the validation and test sets. 

\subsubsection{Noisy Graphs}
To show RS-GNN is robust to various structural noises, we evaluate RS-GNN on the following types of noises:
\begin{itemize}[leftmargin=*]
    \item \textbf{Raw Graphs}: They are the original graphs of the benchmark datasets which may contain inherent structural noise.
    \item \textbf{Random Noise}: We randomly inject fake edges and remove normal edges to add random noise to graphs.
    \item \textbf{Non-Targeted Attack}: 
    We adopt \textit{metattack}~\cite{zugner2019adversarial} to poison the graph structures by adding and removing edges, which aims to reduces the overall performance of GNNs on the whole graph. 
    \item \textbf{Targeted Attack}: It aims to lead the GNN to misclassify target nodes. Following~\cite{tang2020transferring}, we randomly select 15\% nodes as target nodes and apply \textit{nettack}~\cite{zugner2018adversarial} to perturb the graph structure. 

\end{itemize}




\subsubsection{Baselines} We compare RS-GNN with the representative and state-of-the-art GNNs, and robust GNNs against adversarial attacks:
\begin{itemize}[leftmargin=*]
    \item \textbf{GCN}~\cite{kipf2016semi}: GCN is a representative GNN which defines Graph convolution with spectral analysis.
     \item \textbf{SuperGAT}~\cite{kim2021find}: This extends GAT~\cite{velivckovic2017graph} with self-supervised learning. Edge prediction is deployed as the pretext task to guide the learning of attention to facilitate the message-passing.
    \item \textbf{Self-Training}~\cite{li2018deeper}: This is a self-supervised learning method. A GCN is firstly trained on given labels. Then, confident pseudo labels would be added to the label set to improve the GCN.
    \item \textbf{RGCN}~\cite{zhu2019robust}: It uses Gaussian distributions as representations to absorb the effects of adversarial edges. 
    \item \textbf{GCN-jaccard}~\cite{wu2019adversarial}: GCN-Jaccard eliminates edges that connect nodes with low Jaccard similarity, then apply GCN on the graph.
    \item \textbf{GCN-SVD}~\cite{entezari2020all}: This preprocessing method is based on low rank assumption. Low-rank approximation of the perturbed graph is used to train GNNs against adversarial attacks.
    \item \textbf{Pro-GNN}~\cite{jin2020graph}: It applies low-rank and sparsity constraints to learn a clean graph structure close to the noisy graph structure. 
\end{itemize}
For all the baselines, we use the implementation from the repository DeepRobust~\cite{li2020deeprobust}. All the hyperparameters of the baselines are tuned on the validation set to make a fair comparison with RS-GNN.

\subsubsection{Implementation Details}
\label{sec:implementation}
\textit{Each experiment is conducted 5 times} and average results with standard deviations are reported. The hyperparameters are tuned based on the performance of validation set. More specifically, for RS-GNN, we vary $\alpha$ as  \{0.003, 0.03, 0.3, 3, 30 \}, and $\beta$ as \{0.01, 0.03, 0.1, 0.3, 1\}. For all experiments, $T_l$, $T_h$, $\sigma$, and $Q$ are fixed as 0.1, 0.8, 100, and 50, respectively. $K$ is set as 100, 300, 400 and 10 for Cora, Cora-ML, Citeseer and Pubmed, respectively. More details about the hyperparameters sensitivity is discussed in Sec. \ref{Sec:para_analysis}.
A one-hidden layer MLP with 64 filters is applied as the link predictor. We use GCN as the backbone of RS-GNN. Various GNNs can be used in RS-GNN and we leave it as a future work. 


\begin{figure}[t]
\centering
\begin{subfigure}{0.49\columnwidth}
    \centering
    \includegraphics[width=0.85\linewidth]{figure/meta_ptb.pdf} 
    \vskip -0.5em
    \caption{Metattack}
    % \label{fig:1_a}
\end{subfigure}
%\vspace{-1em}
\begin{subfigure}{0.49\columnwidth}
    \centering
    \includegraphics[width=0.85\linewidth]{figure/random_ptb.pdf} 
    \vskip -0.5em
    \caption{Random Noise}
    % \label{fig:1_b}
\end{subfigure}
\vspace{-1.2em}
\caption{Robustness under different Ptb rates on Cora.  }
\label{fig:ptb}
\vskip -1.5em
\end{figure}

\subsection{Performance on Noisy Graphs}
To answer \textbf{RQ1}, we first compare RS-GNN with the baselines on various noisy graphs. We then evaluate the performance of RS-GNN on the graphs with different levels of structural noise.



\subsubsection{Comparisons with baselines}
We conduct experiments on four types of noisy graphs, i.e., raw graphs, graphs with random noise, non-targeted attack perturbed graphs and targeted attack perturbed graphs. The perturbation rate of non-targeted attack and targeted attack is 0.15. The perturbation rate of random noise is set as 0.3. Since we focus on noisy graph with sparse labels, we set the label rates as 0.01 for Cora, Cora-ML, Citeseer and 0.1 for Pubmed. The results are reported in Table \ref{tab:results}, where we can observe:

\begin{itemize}[leftmargin=*]
    \item With limited labeled nodes, GCN even hardly performs well on raw graph, which indicates the necessity of investigating method to address the challenge of sparsely labeled graphs. Though recent GNNs such as SuperGAT and Self-Training can improve the performance with self-supervised learning, our RS-GNN still outperforms them by a large margin. This shows the effectiveness of graph densification in dealing with sparsely labeled graphs.
    \item The structural noise further degrades the performance of GCN, but its impact to RS-GNN is negligible. RS-GNN achieves better results than the state-of-the-art robust GNNs. This indicates RS-GNN could eliminate the effects of the noisy edges.
    \item Compared with the preprocessing methods and Pro-GNN, RS-GNN achieves higher accuracy on the sparsely labeled graphs perturbed by attack methods. 
    This is because the baselines only focus on eliminating potential noisy edges, which will even result in less involvement of unlabeled nodes. 
    By contrast, RS-GNN can down-weights/removes the adversarial edges to defend the adversarial attacks and densify the graph to facilitate the message passing for predictions of unlabeled nodes.
    
\end{itemize}



\subsubsection{Robustness Under Different Ptb Rates } 
To show that RS-GNN is resistant to different levels of structural noise, we vary the perturbation rate as $\{0\%, 5\%, 10\%, \dots, 25\%\}$ and compare the performance of RS-GNN with the most effective baselines. The label rate is fixed as 0.01. Since we have similar observations on other datasets, we only report the average accuracy and standard deviation on Cora in Figure \ref{fig:ptb}. From the figure, we make following observations:
\begin{itemize}[leftmargin=*]
    \item As the perturbation rate increases, the performance of all the baselines drop significantly, which is as expected. Though the performance of RS-GNN also drops, it is much stable and consistently outperforms the baselines, which shows the robustness of RS-GNN against various levels of attacks and random noise; and %  the proposed framework RS-GNN  For the noisy graph perturbed by \textit{metattack}, the performance of GCN degrades significantly. Our proposed RS-GNN could achieve high performance under high perturbation rate and consistently outperforms the baselines, which demonstrates RS-GNN could well defend attacks under different perturbation rates.
    \item  Compared with GCN, RS-GNN uses GCN as backbone but significantly outperforms GCN, especially when the perturbation rate is large, which shows the effectiveness of eliminating the effects of noisy edges and densifying the graph to benefit the predictions given limited labels. % consistently achieve high performance on the graph containing various amounts of noisy edges. It demonstrates that RS-GNN is robust when applied to different levels of random structural noises. \suhang{TODO}
\end{itemize}

\begin{figure}[t]
\centering
\begin{subfigure}{0.49\columnwidth}
    \centering
    \includegraphics[width=0.85\linewidth]{figure/random_cora_2.pdf} 
    \vskip -0.8em
    \caption{Cora}
    % \label{fig:1_a}
\end{subfigure}
%\vspace{-1em}
\begin{subfigure}{0.49\columnwidth}
    \centering
    \includegraphics[width=0.85\linewidth]{figure/random_cora_ml_2.pdf} 
    \vskip -0.8em
    \caption{CoraML}
    % \label{fig:1_b}
\end{subfigure}
\vspace{-1.5em}
\caption{ Distributions of the weights of normal and noisy edges on the generated graph.}
\label{fig:weight}
\vskip -1em
\end{figure}

\subsection{Analysis of the Learned Graph} 
To demonstrate that RS-GNN could alleviate negative effects of noisy edges by downweighting the noisy edges, we investigate the distribution of the learned edge weights $\mathbf{S}_{ij}$ of normal and noisy edges in this subsection. The edge weight  distributions of graphs perturbed by random noise with 30\% perturbation rate on Cora and Cora-ML are shown in Fig.~\ref{fig:weight}.  From this figure, we observe: (\textbf{i}) The weights of noisy edges are significantly lower than the weights of normal edges, which indicates RS-GNN manages to reduce the effects of noisy edges for robust GNN; and (\textbf{ii}) Although most normal edges have higher weights, some of their weights are very low, which implies inherent noise exists in the graph and RS-GNN is able to get rid of such inherent structural noise. 

We also provide more details about the number of involved unlabeled nodes with the learned graph in Appendix~\ref{sec:app_graph}, which proves RS-GNN can enhance the involvement of unlabeled nodes.






\begin{figure}[t]
\centering
\begin{subfigure}{0.49\columnwidth}
    \centering
    \includegraphics[width=0.9\linewidth]{figure/clean_label_rate.pdf} 
    \vskip -0.5em
    \caption{Raw Graph}
    % \label{fig:1_a}
\end{subfigure}
%\vspace{-1em}
\begin{subfigure}{0.49\columnwidth}
    \centering
    \includegraphics[width=0.9\linewidth]{figure/ptb_label_rate.pdf} 
    \vskip -0.5em
    \caption{Metattack with 15\% Ptb}
    % \label{fig:1_b}
\end{subfigure}
\vspace{-1.2em}
\caption{Performance on Cora with different label rates.  }
\label{fig:label_rate}
\vskip -0.8em
\end{figure}


\subsection{Impacts of Label Rate and Graph Sparsity}

To answer \textbf{RQ2}, we study the impacts of the number of labeled nodes and sparsity of the graph by varying the label rate and edge rate of the graph. The hyperparameters are selected with the process described in Sec. \ref{sec:implementation}. Each experiment is conducted 5 times and average accuracy with standard deviation are reported.



\subsubsection{Impacts of Label Rate} We vary label rates as \{0.01, 0.02,\dots, 0.06\}. Experiments are conducted on raw graphs and graphs perturbed by \textit{mettack} to study the effectiveness of RS-GNN under various label rates. The results on Cora are shown in Fig.~\ref{fig:label_rate}. We have similar observations on other datasets. From Fig.~\ref{fig:label_rate}, we observe:
\begin{itemize}[leftmargin=*]
    \item Generally, as the increase of label rate, the performances of all the methods increase, which is as expected.
    \item For the raw graph, though RS-GNN consistently outperforms the baselines, as the label rate increases, the improvement of RS-GNN becomes marginal. This is because the raw graph doesn't contain much noise. Thus, as label rate increases to 6\%, there are already adequate labels. Since higher label rates would result in more unlabeled nodes involving in the training, the effects of densifying graphs and label smoothness become less significant; %when the label rate is high enough to involve most of the unlabeled nodes.
    \item For the metattack graph, as the label rate increases, RS-GNN still significantly outperforms baselines. That's because the training graph contains a lot of adversarial edges. Though we have enough training labels, the adversarial edges can still contaminate the message passing of GNNs. But RS-GNN can eliminate noisy edges and densify the graph, thus having better results.
\end{itemize}



\subsubsection{Impacts of Graph Sparsity} As RS-GNN can generate dense graphs, it should have the ability to handle sparse graphs. Thus, we randomly select $x\%$ edges from the raw graph to build graphs of different sparsity levels. We vary edge rate $x\%$ from 20\% to 100\% with a step of 40\%. Since we are interested in how the sparsity of the graph could affect RS-GNN in generating dense graphs, we only focus on the performance on raw graphs. 
The average results of 5 runs on Citeseer are reported in Table~\ref{tab:sparsity}. From the table, we have the following observations:
\begin{itemize}[leftmargin=*]
    \item As the edge rate decreases, the performance of all the methods decrease, which is because message-passing of GNNs becomes ineffective on very sparse graphs;
    \item RS-GNN consistently outperforms the baselines. In particular, when the graph becomes more sparse, the improvement of RS-GNN over the baselines becomes larger. For example, the improvement of RS-GNN over GCN on Citeseer is 6.4\% when Edge Rate is 100\%, and becomes 9.2\% when Edge Rate is 20\%, which shows the importance of generating edges for densifying the graph and smoothing predictions with the learned graph. 
\end{itemize}


\begin{table}[t]
    \small
    \centering
    \caption{Accuracy (\%) on Citeseer in different sparsity levels. }
    \vskip-1.5em
    \begin{tabularx}{0.96\linewidth}{>{\centering\arraybackslash}p{0.20\linewidth}CCC}
    \toprule
    Edge Rate (\%) & GCN & Pro-GNN & RS-GNN\\
    \midrule
    % \multirow{3}{*}{Cora} 
    % & 20 & 51.9 $\pm 2.0$ & 49.5 $\pm 0.8$ & \textbf{64.7} $\pm \textbf{1.7}$\\ 
    % % & 40 & 53.9 $\pm 1.2$ & 51.2 $\pm 0.7$ & \textbf{66.3} $\pm \textbf{1.2}$ \\
    % & 60 & 62.0 $\pm 0.3$ & 62.5 $\pm 0.5$ & \textbf{68.5} $\pm \textbf{1.7}$\\
    % % & 80 & 64.2 $\pm 0.5$ & 64.6 $\pm 0.2$ & \textbf{72.8} $\pm \textbf{1.4}$ \\
    % & 100 & 65.5 $\pm 0.5$ & 65.9 $\pm 1.1$ & \textbf{75.3} $\pm \textbf{0.6}$\\
    % \hline

    20 & 54.5 $\pm 1.2$ & 55.2 $\pm 1.6$ & \textbf{63.7} $\pm \textbf{2.2}$\\
    % & 40 & 56.8 $\pm 1.1$ & 58.6 $\pm 1.5$ & \textbf{67.3} $\pm \textbf{1.8}$\\
    60 & 58.7 $\pm 1.8$ & 58.3 $\pm 2.4$ &
    \textbf{69.8} $\pm \textbf{1.1}$\\
    % & 80 & 60.1 $\pm 2.2$ &60.2 $\pm 1.3$ & \textbf{70.4} $\pm \textbf{0.7}$\\
    100 & 64.8 $\pm 1.4$ & 60.6 $\pm 2.0$ &
    \textbf{71.2} $\pm \textbf{1.4}$\\
    \bottomrule
    \end{tabularx}
    \label{tab:sparsity}
    \vskip -1.em
\end{table}




%\vspace{-1em}

\begin{figure}[h]
\centering
\begin{subfigure}{0.49\columnwidth}
    \centering
    \includegraphics[width=0.85\linewidth]{figure/neta_abl.pdf} 
    \vskip -0.5em
    \caption{Nettack}
    \label{fig:abla_neta}
\end{subfigure}
\begin{subfigure}{0.49\columnwidth}
    \centering
    \includegraphics[width=0.85\linewidth]{figure/abla_meta.pdf} 
    \vskip -0.5em
    \caption{Metattack with 15\% Ptb}
    \label{fig:abla_meta}
\end{subfigure}
\vspace{-1.3em}
\caption{Ablation studies on Cora with different label rates.}
\label{fig:abl}
\vskip -1.8em
\end{figure}
\subsection{Ablation Study}
To answer \textbf{RQ3}, we conduct ablation studies to understand the effects of graph densification, graph purification and label smoothness regularization. In RS-GNN, the link predictor densify the graph to enhance the performance on unlabeled nodes. To demonstrate the effects of adding edges with the link predictor, we remove the process of adding edges and obtain RS-GNN$\backslash$A. 
To testify the effectiveness of the label smoothness regularization based on the generated graph, we eliminate the label smoothness regularization and get RS-GNN$\backslash$U. To show our link predictor can eliminate the effects of noisy edges, we compare a variant named as RS-GNN$\backslash$AU which only use the link predictor to denoise graphs. Graph desification and label smoothness are not applied in RS-GNN$\backslash$AU. 
We also implement a variant named as RS-GNN$_{GCN}$ which uses GCN as link predictor to show that the noisy edges would largely affects the GNNs for link prediction. Hyperparameters selection follows the process in Sec~\ref{sec:implementation}. We only show the results on the Cora graph perturbed with \textit{metattack} and random noise, because similar trends are observed on other datasets. Results are presented in Fig.~\ref{fig:abl}. From this figure, we observe that: 
\begin{itemize}[leftmargin=*]
    \item RS-GNN performs much better than RS-GNN$\backslash$A and RS-GNN$\backslash$U, which shows that densifying graphs and label smoothness with the learned graph can address the label sparsity issue;
    \item With the increase of label rate, the gap between RS-GNN and RS-GNN$\backslash$U will be narrowed. This is consistent with our analysis that higher label rates would involve more unlabeled nodes;
    \item RS-GNN$_{GCN}$ performs much worse than RS-GNN, which indicates adversarial edges would impair GCN and result in a poor link predictor for denoising and densification.
\end{itemize} 


\begin{figure}[t]
\centering
\begin{subfigure}{0.49\columnwidth}
    \centering
    \includegraphics[width=0.98\linewidth]{figure/cora_para_clean_2}
    \vskip -0.5em
    \caption{Raw Graph}
    \label{fig:para_raw}
\end{subfigure}
%\vspace{-1em}
\begin{subfigure}{0.49\columnwidth}
    \centering
    \includegraphics[width=0.98\linewidth]{figure/cora_para_ptb_2}
    \vskip -0.5em
    \caption{Metattack with 15\% Ptb}
    \label{fig:para_meta}
\end{subfigure}
\vspace{-1em}
\caption{Parameter sensitivity analysis on Cora.}
\label{fig:para}
\vskip -1.7em
\end{figure}


\subsection{Parameter Sensitivity Analysis}
\label{Sec:para_analysis}
In this subsection, we explore the sensitivity of the most crucial hyperparameters $\alpha$ and $\beta$ which are in the final objective function of RS-GNN. The analysis about other hyperparameters is presented in the supplementary material. $\alpha$ controls how well the link predictor reconstructs the noisy graph and $\beta$ controls the contribution of label smoothness. To investigate the effects of $\alpha$ and $\beta$, we vary the values of $\alpha$ as $\{0.003, 0.03, 0.3, 3, 30\}$ and $\beta$ as $\{0.01, 0.03, 0.1, 0.3, 1, 3\}$ on Cora. The results are shown in Fig~\ref{fig:para}. In the raw graph, when $\alpha$ is large, the accuracy is stable and high. But if the $\alpha$ is too large in the perturbed graph, the performance would decrease. This difference is due to the noise levels of the raw graph and the perturbed graph. The structural noise in the perturbed graph is severe, faithfully reconstructing the perturbed graph with high $\alpha$ would lead to a poor link predictor. As for the $\beta$, a value between 0.03 to 0.3 generally gives good performance, which eases the parameter selection.
 We propose a novel commonsense reasoning challenge, \textsc{RiddleSense}, which requires complex commonsense skills for reasoning about creative and counterfactual questions, coming with a large multiple-choice QA dataset.  
 We systematically evaluate recent commonsense reasoning methods over the proposed \textsc{RiddleSense} dataset, and find that the best model is still far behind human performance, suggesting that there is still much space for commonsense reasoning methods to improve.
 We hope \textsc{RiddleSense} can serve as a benchmark dataset for future research targeting complex commonsense reasoning and computational creativity.


\section*{Acknowledgements}
This research is supported in part by the Office of the Director of National Intelligence (ODNI), Intelligence Advanced Research Projects Activity (IARPA), via Contract No. 2019-19051600007, the DARPA MCS program under Contract No. N660011924033 with the United States Office Of Naval Research, the Defense Advanced Research Projects Agency with award W911NF-19-20271, and NSF SMA 18-29268. The views and conclusions contained herein are those of the authors and should not be interpreted as necessarily representing the official policies, either expressed or implied, of ODNI, IARPA, or the U.S. Government. We would like to thank all the collaborators in USC INK research lab and the reviewers for their constructive feedback on the work.

\bibliography{citation}
\bibliographystyle{icml2022}
%%%%%%%%%%%%%%%%%%%%%%%%%%%%%%%%%%%%%%%%%%%%%%%%%%%%%%%%%%%%%%%%%%%%%%%%%%%%%%%
%%%%%%%%%%%%%%%%%%%%%%%%%%%%%%%%%%%%%%%%%%%%%%%%%%%%%%%%%%%%%%%%%%%%%%%%%%%%%%%
% APPENDIX
%%%%%%%%%%%%%%%%%%%%%%%%%%%%%%%%%%%%%%%%%%%%%%%%%%%%%%%%%%%%%%%%%%%%%%%%%%%%%%%
%%%%%%%%%%%%%%%%%%%%%%%%%%%%%%%%%%%%%%%%%%%%%%%%%%%%%%%%%%%%%%%%%%%%%%%%%%%%%%%
\newpage
\newpage

\section*{Appendix}
\label{sec:appendix}

For the sake of completeness, we here report the same analyses depicted in Figure~\ref{fig:tranco_tp} and Figure~\ref{fig:ca_perf_tp} showing the number of Trackers instead of the number of Third-Parties. The two pictures lead to similar conclusions.

\begin{figure}[!h]
    \centering
    \includegraphics[width=0.5\columnwidth]{figures/cookieaccept_tranco_rank_eu_tracker_nb.pdf}
    \caption{Average number of Trackers per website (Tranco list).}
    \label{fig:tranco_trackers}
\end{figure}


\begin{figure}[!h]
    \centering
    \includegraphics[width=0.5\textwidth]{figures/cookieaccept_tracker_nb_tranco.pdf}
    \caption{Number of Trackers (Tranco list). Notice the log scales.}
    \label{fig:ca_perf_tracker}
\end{figure}


%%%%%%%%%%%%%%%%%%%%%%%%%%%%%%%%%%%%%%%%%%%%%%%%%%%%%%%%%%%%%%%%%%%%%%%%%%%%%%%
%%%%%%%%%%%%%%%%%%%%%%%%%%%%%%%%%%%%%%%%%%%%%%%%%%%%%%%%%%%%%%%%%%%%%%%%%%%%%%%


\end{document}


% This document was modified from the file originally made available by
% Pat Langley and Andrea Danyluk for ICML-2K. This version was created
% by Iain Murray in 2018, and modified by Alexandre Bouchard in
% 2019 and 2021 and by Csaba Szepesvari, Gang Niu and Sivan Sabato in 2022. 
% Previous contributors include Dan Roy, Lise Getoor and Tobias
% Scheffer, which was slightly modified from the 2010 version by
% Thorsten Joachims & Johannes Fuernkranz, slightly modified from the
% 2009 version by Kiri Wagstaff and Sam Roweis's 2008 version, which is
% slightly modified from Prasad Tadepalli's 2007 version which is a
% lightly changed version of the previous year's version by Andrew
% Moore, which was in turn edited from those of Kristian Kersting and
% Codrina Lauth. Alex Smola contributed to the algorithmic style files.
