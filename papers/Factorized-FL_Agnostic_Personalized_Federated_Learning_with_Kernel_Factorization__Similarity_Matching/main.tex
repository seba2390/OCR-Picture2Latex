\pdfoutput=1

%%%%%%%% ICML 2022 EXAMPLE LATEX SUBMISSION FILE %%%%%%%%%%%%%%%%%

\documentclass[nohyperref]{article}

% Recommended, but optional, packages for figures and better typesetting:
\usepackage{microtype}
\usepackage{graphicx}
\usepackage{subfigure}
\usepackage{booktabs} % for professional tables

% hyperref makes hyperlinks in the resulting PDF.
% If your build breaks (sometimes temporarily if a hyperlink spans a page)
% please comment out the following usepackage line and replace
% \usepackage{icml2022} with \usepackage[nohyperref]{icml2022} above.
\usepackage{hyperref}


% Attempt to make hyperref and algorithmic work together better:
\newcommand{\theHalgorithm}{\arabic{algorithm}}

% Use the following line for the initial blind version submitted for review:
% \usepackage{icml2022}

% If accepted, instead use the following line for the camera-ready submission:
\usepackage[accepted]{icml2022}

% For theorems and such
\usepackage{amsmath}
\usepackage{amssymb}
\usepackage{mathtools}
\usepackage{amsthm}

% Custom
\usepackage[normalem]{ulem}
\usepackage{arydshln}
\usepackage{graphicx}
\usepackage{wrapfig}
\usepackage[font=small]{caption}
\usepackage{lipsum}
\usepackage{pifont}
\newcommand{\cmark}{\ding{51}}%
\newcommand{\xmark}{\ding{55}}%
\definecolor{navyblue}{rgb}{0.0, 0.0, 0.5}
\definecolor{darkblue}{rgb}{0.0, 0.0, 0.55}
\hypersetup{
    colorlinks = true,
    citecolor=darkblue,
    linkcolor=red,
    urlcolor  = blue,
    anchorcolor = blue}
\usepackage{enumitem}
\usepackage{multirow}
% \usepackage{subcaption}
\newcommand{\vio}[1]{{\color{violet}{#1}}}
\newcommand{\cy}[1]{{\color{cyan}{#1}}}
\newcommand{\red}[1]{{\color{red}{#1}}}
\newcommand{\ora}[1]{{\color{orange}{#1}}}
\newcommand{\pu}[1]{{\color{purple}{#1}}}
\newcommand{\blu}[1]{{\color{blue}{#1}}}

% Custom


% \definecolor{Red}{rgb}{1,0,0}
\definecolor{Blue}{rgb}{0,0,0.8}
\definecolor{Green}{rgb}{0,0.4,0.7}
\definecolor{airforceblue}{rgb}{0.36, 0.54, 0.66}
\definecolor{ao(english)}{rgb}{0.0, 0.5, 0.0}
\definecolor{azure(colorwheel)}{rgb}{0.0, 0.5, 1.0}
\definecolor{crimson}{rgb}{0.86, 0.08, 0.24}
\definecolor{darkcerulean}{rgb}{0.03, 0.27, 0.49}
\definecolor{cobalt}{rgb}{0.0, 0.28, 0.67}
\definecolor{rosegold}{rgb}{0.72, 0.43, 0.47}
\definecolor{orange-red}{rgb}{1.0, 0.27, 0.0}
\definecolor{mountainmeadow}{rgb}{0.19, 0.73, 0.56}
\definecolor{malachite}{rgb}{0.04, 0.85, 0.32}
\definecolor{darkblue}{rgb}{0.0, 0.0, 0.55}

\definecolor{customblue}{rgb}{0.2, 0.35, 0.8}

\usepackage{hyperref}
\hypersetup{colorlinks=true}
\hypersetup{linktoc=all}
\hypersetup{citecolor=darkblue}
\hypersetup{linkcolor=crimson}
\hypersetup{urlcolor=darkblue}
\usepackage[all]{hypcap}
\usepackage[percent]{overpic}
\usepackage[usestackEOL]{stackengine}

\usepackage[nameinlink]{cleveref}
\creflabelformat{equation}{#2\textup{#1}#3}  
\crefname{assumption}{assumption}{assumptions}
%\usepackage[pagebackref,breaklinks,colorlinks]{hyperref}

\definecolor{gg}{gray}{0.9}

\newcommand{\bsy}{\boldsymbol}
\newcommand{\highlight}[1]{{\color{crimson}{#1}}}
\newcommand{\hibf}[1]{{\color{crimson}{\textbf{#1}}}}
\newcommand{\modify}[1]{{\color{orange-red}{#1}}} %ao(english)
\newcommand{\modified}[1]{{\color{azure(colorwheel)}{#1}}} %airforceblue
\newcommand{\TBD}[1]{{\color{Red}{#1}}}
\creflabelformat{equation}{#2\textup{#1}#3}  

%\newcommand{\plhi}[1]{{\color{Red}{\textbf{#1}}}}
%\newcommand{\mihi}[1]{{\color{Blue}{\textbf{#1}}}}
% \definecolor{plRed}{rgb}{0.8,0,0}
% \definecolor{miBlue}{rgb}{0,0,0.8}
% \definecolor{miGreen}{rgb}{0.0, 0.5, 0.0}
% \newcommand{\red}[1]{{\color{plRed}{#1}}}
% \newcommand{\blu}[1]{{\color{miBlue}{#1}}}
% \newcommand{\gre}[1]{{\color{miGreen}{#1}}}

% if you use cleveref..
% \usepackage[capitalize,noabbrev]{cleveref}

%%%%%%%%%%%%%%%%%%%%%%%%%%%%%%%%
% THEOREMS
%%%%%%%%%%%%%%%%%%%%%%%%%%%%%%%%
\theoremstyle{plain}
\newtheorem{theorem}{Theorem}[section]
\newtheorem{proposition}[theorem]{Proposition}
\newtheorem{lemma}[theorem]{Lemma}
\newtheorem{corollary}[theorem]{Corollary}
\theoremstyle{definition}
\newtheorem{definition}[theorem]{Definition}
\newtheorem{assumption}[theorem]{Assumption}
\theoremstyle{remark}
\newtheorem{remark}[theorem]{Remark}

% Todonotes is useful during development; simply uncomment the next line
%    and comment out the line below the next line to turn off comments
%\usepackage[disable,textsize=tiny]{todonotes}
\usepackage[textsize=tiny]{todonotes}


% The \icmltitle you define below is probably too long as a header.
% Therefore, a short form for the running title is supplied here:
\icmltitlerunning{Factorized-FL: Agnostic Personalized Federated Learning with Kernel Factorization \& Similarity Matching}

\begin{document}

\twocolumn[
\icmltitle{Factorized-FL: Agnostic Personalized Federated Learning \\ with Kernel Factorization \& Similarity Matching}

% It is OKAY to include author information, even for blind
% submissions: the style file will automatically remove it for you
% unless you've provided the [accepted] option to the icml2022
% package.

% List of affiliations: The first argument should be a (short)
% identifier you will use later to specify author affiliations
% Academic affiliations should list Department, University, City, Region, Country
% Industry affiliations should list Company, City, Region, Country

% You can specify symbols, otherwise they are numbered in order.
% Ideally, you should not use this facility. Affiliations will be numbered
% in order of appearance and this is the preferred way.
\icmlsetsymbol{equal}{*}

% \begin{icmlauthorlist}
% \icmlauthor{Wonyong Jeong}{equal,yyy}
% \icmlauthor{Firstname2 Lastname2}{equal,yyy,comp}
% \icmlauthor{Firstname3 Lastname3}{comp}
% \icmlauthor{Firstname4 Lastname4}{sch}
% \icmlauthor{Firstname5 Lastname5}{yyy}
% \icmlauthor{Firstname6 Lastname6}{sch,yyy,comp}
% \icmlauthor{Firstname7 Lastname7}{comp}
% %\icmlauthor{}{sch}
% \icmlauthor{Firstname8 Lastname8}{sch}
% \icmlauthor{Firstname8 Lastname8}{yyy,comp}
% %\icmlauthor{}{sch}
% %\icmlauthor{}{sch}
% \end{icmlauthorlist}

% \icmlaffiliation{yyy}{Department of XXX, University of YYY, Location, Country}
% \icmlaffiliation{comp}{Company Name, Location, Country}
% \icmlaffiliation{sch}{School of ZZZ, Institute of WWW, Location, Country}

% \icmlcorrespondingauthor{Firstname1 Lastname1}{first1.last1@xxx.edu}
% \icmlcorrespondingauthor{Firstname2 Lastname2}{first2.last2@www.uk}

\begin{icmlauthorlist}
\icmlauthor{Wonyong Jeong}{to,goo}
\icmlauthor{Sung Ju Hwang}{to,goo}
\end{icmlauthorlist}

\icmlaffiliation{to}{Graduate School of Artificial Intelligence, KAIST, Seoul, South Korea}
\icmlaffiliation{goo}{AITRICS, Seoul, South Korea}
%\icmlaffiliation{ed}{School of Computation, University of Edenborrow, Edenborrow, United Kingdom}

\icmlcorrespondingauthor{Wonyong Jeong}{wyjeong@kaist.ac.kr}
\icmlcorrespondingauthor{Sung Ju Hwang}{sjhwang82@kaist.ac.kr}


% You may provide any keywords that you
% find helpful for describing your paper; these are used to populate
% the "keywords" metadata in the PDF but will not be shown in the document
\icmlkeywords{Machine Learning, ICML}

\vskip 0.3in
]

% this must go after the closing bracket ] following \twocolumn[ ...

% This command actually creates the footnote in the first column
% listing the affiliations and the copyright notice.
% The command takes one argument, which is text to display at the start of the footnote.
% The \icmlEqualContribution command is standard text for equal contribution.
% Remove it (just {}) if you do not need this facility.

\printAffiliationsAndNotice{}  % leave blank if no need to mention equal contribution
% \printAffiliationsAndNotice{\icmlEqualContribution} % otherwise use the standard text.

\begin{abstract}

Visual perception tasks often require vast amounts of labelled data, including 3D poses and image space segmentation masks. The process of creating such training data sets can prove difficult or time-intensive to scale up to efficacy for general use. Consider the task of pose estimation for rigid objects. Deep neural network based approaches have shown good performance when trained on large, public datasets. However, adapting these networks for other novel objects, or fine-tuning existing models for different environments, requires significant time investment to generate newly labelled instances. Towards this end, we propose ProgressLabeller as a method for more efficiently generating large amounts of 6D pose training data from color images sequences for custom scenes in a scalable manner. ProgressLabeller is intended to also support transparent or translucent objects, for which the previous methods based on depth dense reconstruction will fail.
We demonstrate the effectiveness of ProgressLabeller by rapidly create a dataset of over 1M samples with which we fine-tune a state-of-the-art pose estimation network in order to markedly improve the downstream robotic grasp success rates. Progresslabeller is open-source at \href{https://github.com/huijieZH/ProgressLabeller}{https://github.com/huijieZH/ProgressLabeller}

\end{abstract}
\begin{figure}[t]
\begin{center}
   \includegraphics[width=1.0\linewidth]{figures/nas_comp_v3}
\end{center}
   \vspace{-4mm}
   \caption{The comparison between NetAdaptV2 and related works. The number above a marker is the corresponding total search time measured on NVIDIA V100 GPUs.}
\label{fig:nas_comparison}
\end{figure}

\section{Introduction}
\label{sec:introduction}

Neural architecture search (NAS) applies machine learning to automatically discover deep neural networks (DNNs) with better performance (e.g., better accuracy-latency trade-offs) by sampling the search space, which is the union of all discoverable DNNs. The search time is one key metric for NAS algorithms, which accounts for three steps: 1) training a \emph{super-network}, whose weights are shared by all the DNNs in the search space and trained by minimizing the loss across them, 2) training and evaluating sampled DNNs (referred to as \emph{samples}), and 3) training the discovered DNN. Another important metric for NAS is whether it supports non-differentiable search metrics such as hardware metrics (e.g., latency and energy). Incorporating hardware metrics into NAS is the key to improving the performance of the discovered DNNs~\cite{eccv2018-netadapt, Tan2018MnasNetPN, cai2018proxylessnas, Chen2020MnasFPNLL, chamnet}.


There is usually a trade-off between the time spent for the three steps and the support of non-differentiable search metrics. For example, early reinforcement-learning-based NAS methods~\cite{zoph2017nasreinforcement, zoph2018nasnet, Tan2018MnasNetPN} suffer from the long time for training and evaluating samples. Using a super-network~\cite{yu2018slimmable, Yu_2019_ICCV, autoslim_arxiv, cai2020once, yu2020bignas, Bender2018UnderstandingAS, enas, tunas, Guo2020SPOS} solves this problem, but super-network training is typically time-consuming and becomes the new time bottleneck. The gradient-based methods~\cite{gordon2018morphnet, liu2018darts, wu2018fbnet, fbnetv2, cai2018proxylessnas, stamoulis2019singlepath, stamoulis2019singlepathautoml, Mei2020AtomNAS, Xu2020PC-DARTS} reduce the time for training a super-network and training and evaluating samples at the cost of sacrificing the support of non-differentiable search metrics. In summary, many existing works either have an unbalanced reduction in the time spent per step (i.e., optimizing some steps at the cost of a significant increase in the time for other steps), which still leads to a long \emph{total} search time, or are unable to support non-differentiable search metrics, which limits the performance of the discovered DNNs.

In this paper, we propose an efficient NAS algorithm, NetAdaptV2, to significantly reduce the \emph{total} search time by introducing three innovations to \emph{better balance} the reduction in the time spent per step while supporting non-differentiable search metrics:

\textbf{Channel-level bypass connections (mainly reduce the time for training and evaluating samples, Sec.~\ref{subsec:channel_level_bypass_connections})}: Early NAS works only search for DNNs with different numbers of filters (referred to as \emph{layer widths}). To improve the performance of the discovered DNN, more recent works search for DNNs with different numbers of layers (referred to as \emph{network depths}) in addition to different layer widths at the cost of training and evaluating more samples because network depths and layer widths are usually considered independently. In NetAdaptV2, we propose \emph{channel-level bypass connections} to merge network depth and layer width into a single search dimension, which requires only searching for layer width and hence reduces the number of samples.

\textbf{Ordered dropout (mainly reduces the time for training a super-network, Sec.~\ref{subsec:ordered_droput})}: We adopt the idea of super-network to reduce the time for training and evaluating samples. In previous works, \emph{each} DNN in the search space requires one forward-backward pass to train. As a result, training multiple DNNs in the search space requires multiple forward-backward passes, which results in a long training time. To address the problem, we propose \emph{ordered dropout} to jointly train multiple DNNs in a \emph{single} forward-backward pass, which decreases the required number of forward-backward passes for a given number of DNNs and hence the time for training a super-network.

\textbf{Multi-layer coordinate descent optimizer (mainly reduces the time for training and evaluating samples and supports non-differentiable search metrics, Sec.~\ref{subsec:optimizer}):} NetAdaptV1~\cite{eccv2018-netadapt} and MobileNetV3~\cite{Howard_2019_ICCV}, which utilizes NetAdaptV1, have demonstrated the effectiveness of the single-layer coordinate descent (SCD) optimizer~\cite{book2020sze} in discovering high-performance DNN architectures. The SCD optimizer supports both differentiable and non-differentiable search metrics and has only a few interpretable hyper-parameters that need to be tuned, such as the per-iteration resource reduction. However, there are two shortcomings of the SCD optimizer. First, it only considers one layer per optimization iteration. Failing to consider the joint effect of multiple layers may lead to a worse decision and hence sub-optimal performance. Second, the per-iteration resource reduction (e.g., latency reduction) is limited by the layer with the smallest resource consumption (e.g., latency). It may take a large number of iterations to search for a very deep network because the per-iteration resource reduction is relatively small compared with the network resource consumption. To address these shortcomings,  we propose the \emph{multi-layer coordinate descent (MCD) optimizer} that considers multiple layers per optimization iteration to improve performance while reducing search time and preserving the support of non-differentiable search metrics.

Fig.~\ref{fig:nas_comparison} (and Table~\ref{tab:nas_result}) compares NetAdaptV2 with related works. NetAdaptV2 can reduce the search time by up to $5.8\times$ and $2.4\times$ on ImageNet~\cite{imagenet_cvpr09} and NYU Depth V2~\cite{nyudepth} respectively and discover DNNs with better performance than state-of-the-art NAS works. Moreover, compared to NAS-discovered MobileNetV3~\cite{Howard_2019_ICCV}, the discovered DNN has $1.8\%$ higher accuracy with the same latency.


\section{Related work}
\label{sec:related_work}

Accessibility is an essential component of computing, which aims to make technology broadly accessible to as many users as possible, including those with differing sets of abilities. Improvements in usability and accessibility falls to the community, to better understand the needs of users with differing abilities, and to design technologies that play to this spectrum of abilities \citep{Wobbrock2011AbilityBasedDC}.
In computing, significant strides have been made to increase the accessibility of web content. For example, various versions of the Web Content Accessibility Guidelines (WCAG) \citep{Chisholm2001WebCA, Caldwell2008WebCA} and the in-progress working draft for WCAG 3.0,\footnote{\href{https://www.w3.org/TR/wcag-3.0/}{https://www.w3.org/TR/wcag-3.0/}} or standards such as ARIA from the W3C's Web Accessibility Initiative (WAI)\footnote{\href{https://www.w3.org/WAI/standards-guidelines/aria/}{https://www.w3.org/WAI/standards-guidelines/aria/}} have been released and used to guide web accessibility design and implementation. Similarly, positive steps have been made to improve the accessibility of user interfaces and user experience \citep{Peissner2012MyUIGA, Peissner2013UserCI, Thompson2014ImprovingTU, Bigham2014MakingTW}, as well as various types of media content \citep{Mirri2017TowardsAG, Nengroo2017AccessibleI, Gleason2020TwitterAA}. 

We take inspiration from accessibility design principles in our effort to make research publications more accessible to users who are blind and low vision. Blindness and low vision are some of the most common forms of disability, affecting an estimated 3--10\% of Americans depending on how visual impairment is defined \citep{CDCVisionLossBurden}. BLV researchers also make up a representative sample of researchers in the United States and worldwide. A recent Nature editorial pushes the scientific community to better support researchers with visual impairments \citep{NatureCareerColumn2020}, since existing tools and resources can be limited. There are many inherent accessibility challenges to performing research. In this paper, we engage with one of these challenges that affects all domains of study, accessing and reading the content of academic publications. 

BLV users interact with papers using screen readers, braille displays, text-to-speech, and other assistive tools. A WebAIM survey of screen reader users found that the vast majority (75.1\%) of respondents indicate that PDF documents are very or somewhat likely to pose significant accessibility issues.\footnote{\href{https://webaim.org/projects/screenreadersurvey8/}{https://webaim.org/projects/screenreadersurvey8/}} Most paper are published in PDF, which is inherently inaccessible, due in large part to its conflation of visual layout information with semantic content \citep{NielsenPDFStillUnfit, Bigham2016AnUT}. 
\citet{Bigham2016AnUT} describe the historical reasons we use PDF as the standard document format for scientific publications, as well as the barriers the format itself presents to accessibility. Prior work on scientific accessibility have made recommendations for how to make PDFs more accessible \cite{Rajkumar2020PDFAO, Darvishy2018PDFAT}, including greater awareness for what constitutes an accessible PDF and better tooling for generating accessible PDFs. Some work has focused on addressing components of paper accessibility, such as the correct way for screen readers to interpret and read mathematical equations \citep{Flores2010MathMLTA, Bates2010SpokenMU, Sorge2014TowardsMM, Mackowski2017MultimediaPF, Ahmetovic2018AxessibilityAL, Ferreira2004EnhancingTA, Sojka2013AccessibilityII}, describe charts and figures \citep{Elzer2008AccessibleBC, Engel2017TowardsAC, Engel2019SVGPlottAA}, automatically generate figure captions \citep{Chen2019NeuralCG, Qian2020AFS}, or automatically classify the content of figures \citep{Kim2018MultimodalDL}. Other work applicable to all types of PDF documents aims to improve automatic text and layout detection of scanned documents \cite{Nazemi2014PracticalSM} and extract table content \cite{Fan2015TableRD, Rastan2019TEXUSAU}. In this work, we focus on the issue of representing overall document structure, and navigation within that structure. Being able to quickly navigate the contents of a paper through skimming and scanning is an essential reading technique \citep{Maxwell1972SkimmingAS}, which is currently under-supported by PDF documents and PDF readers when reading these documents by screen reader. 

There also exists a variety of automatic and manual tools that assess and fix accessibility compliance issues in PDFs, including the Adobe Acrobat Pro Accessibility Checker\footnote{\href{https://www.adobe.com/accessibility/products/acrobat/using-acrobat-pro-accessibility-checker.html}{https://www.adobe.com/accessibility/products/acrobat/using-acrobat-pro-accessibility-checker.html}}, Common Look\footnote{\href{https://monsido.com/monsido-commonlook-partnership}{https://monsido.com/monsido-commonlook-partnership}}, ABBYY FineReader\footnote{\href{https://pdf.abbyy.com/}{https://pdf.abbyy.com/}}, PAVE\footnote{\href{https://pave-pdf.org/faq.html}{https://pave-pdf.org/faq.html}}, and PDFA Inspector\footnote{\href{https://github.com/pdfae/PDFAInspector}{https://github.com/pdfae/PDFAInspector}}. To our knowledge, PAVE and PDFA Inspector are the only non-proprietary, open-source tools for this purpose. Based on our experiences, however, all of these tools require some degree of human intervention to properly tag a scientific document, and tagging and fixing must be performed for each new version of a PDF, regardless of how minor the change may be.

Guidelines and policy changes have been introduced in the past decade to ameliorate some of the issues around scientific PDF accessibility. Some conferences, such as The ACM CHI Virtual Conference on Human Factors in Computing Systems (CHI) and The ACM SIGACCESS Conference on Computers and Accessibility (ASSETS), have released guidelines for creating accessible submissions.\footnote{See \href{http://chi2019.acm.org/authors/papers/guide-to-an-accessible-submission/}{http://chi2019.acm.org/authors/papers/guide-to-an-accessible-submission/} and \href{https://assets19.sigaccess.org/creating_accessible_pdfs.html}{https://assets19.sigaccess.org/creating\_accessible\_pdfs.html}} The ACM Digital Library\footnote{\href{https://dl.acm.org/}{https://dl.acm.org/}} provides some publications in HTML format, which is easier to make accessible than PDF~\cite{Graells2007EstudioDL}. \citet{Ribera2019PublishingAP} conducted a case study on DSAI 2016 (Software Development and Technologies for Enhancing Accessibility and Fighting Infoexclusion). The authors of DSAI were responsible for creating accessible proceedings and identified barriers to creating accessible proceedings, including lack of sufficient tooling and lack of awareness of accessibility. The authors recommended creating a new role in the organizing committee dedicated to accessible publishing. These policy changes have led to improvements in localized communities, but have not been widely adopted by all academic publishers and conference organizers.

Table~\ref{tab:prior_work} lists prior studies that have analyzed PDF accessibility of academic papers, and shows how our study compares. Prior work has primarily focused on papers published in Human-Computer Interaction and related fields, specific to certain publication venues, while our analysis tries to quantify paper accessibility more broadly.
\citet{Brady2015CreatingAP} quantified the accessibility of 1,811 papers from CHI 2010-2016, ASSETS 2014, and W4A, assessing the presence of document tags, headers, and language. They found that compliance improved over time as a response to conference organizers offering to make papers accessible as a service to any author upon request. \citet{Lazar2017MakingTF} conducted a study quantifying accessibility compliance at CHI from 2010 to 2016 as well as ASSETS 2015,
%\jb{Define acronyms in prev para}
confirming the results of \citet{Brady2015CreatingAP}. They found that across 5 accessibility criteria, the rate of compliance was less than 30\% for CHI papers in each of the 7 years that were studied. The study also analyzed papers from ASSETS 2015, an ACM conference explicitly focused on accessibility, and found that those papers had significantly higher rates of compliance, with over 90\% of the papers being tagged for correct reading order and no criteria having less than 50\% compliance. This finding indicates that community buy-in is an important contributor to paper accessibility.
\citet{Nganji2015ThePD} conducted a study of 200 PDFs of papers published in four disability studies journals, finding that accessibility compliance was between 15-30\% for the four journals analyzed, with some publishers having higher adherence than others. To date, no large scale analysis of scientific PDF accessibility has been conducted outside of disability studies and HCI, due in part to the challenge of scaling such an analysis. We believe such an analysis is useful for establishing a baseline and characterizing routes for future improvement. Consequently, as part of this work, we conduct an analysis of scientific PDF accessibility across various fields of study, and report our findings relative to prior work. 


\begin{table}[t!]
\small
    \centering
    \begin{tabularx}{\linewidth}{L{22mm}L{15mm}L{48mm}L{16mm}L{34mm}}
        \toprule
        \textbf{Prior work} & \textbf{PDFs analyzed} & \textbf{Venues} & \textbf{Year} & \textbf{Accessibility checker} \\
        \midrule
        \citet{Brady2015CreatingAP} & 1811 & CHI, ASSETS and W4A & 2011--2014 & PDFA Inspector \\ [0.5mm]
        \hline \\ [-2.5mm]
        \citet{Lazar2017MakingTF} & 465 + 32 & CHI and ASSETS & 2014--2015 & Adobe Acrobat Action Wizard \\ [0.5mm]
        \hline \\ [-2.5mm]
        \citet{Ribera2019PublishingAP} & 59 & DSAI & 2016 & Adobe PDF Accessibility Checker 2.0 \\ [0.5mm]
        \hline \\ [-2.5mm]
        \citet{Nganji2015ThePD} & 200 & \textit{Disability \& Society}, \textit{Journal of Developmental and Physical Disabilities}, \textit{Journal of Learning Disabilities}, and \textit{Research in Developmental Disabilities} & 2009--2013 & Adobe PDF Accessibility Checker 1.3 \\ [0.6mm]
        \hline \\ [-2.5mm]
        \textbf{\textit{Our analysis}} & \numpdfs & Venues across various fields of study & 2010--2019 & Adobe Acrobat Accessibility Plug-in Version 21.001.20145 \\
        \bottomrule
    \end{tabularx}
    \caption{Prior work has investigated PDF accessibility for papers published in specific venues such as CHI, ASSETS, W4A, DSAI, or various disability journals. Several of these works were conducted manually, and were limited to a small number of papers, while the more thorough analysis was conducted for CHI and ASSETS, two conference venues focused on accessibility and HCI. Our study expands on this prior work to investigate accessibility over \numpdfs PDFs sampled from across different fields of study.
    }
    % \Description{
    % Prior work, PDFs analyzed, Venues, Year, Accessibility checker 
    % Brady et al. [7], 1811, CHI, ASSETS and W4A, 2011--2014, PDFA Inspector 
    % Lazar et al. [23], 465 + 32, CHI and ASSETS, 2014--2015, Adobe Acrobat Action Wizard 
    % Ribera et al. [40], 59, DSAI, 2016, Adobe PDF Accessibility Checker 2.0 
    % Nganji [33], 200, Disability & Society, Journal of Developmental and Physical Disabilities, Journal of Learning Disabilities, and Research in Developmental Disabilities, 2009--2013, Adobe PDF Accessibility Checker 1.3
    % Our analysis, 11397, Venues across various fields of study, 2010--2019, Adobe Acrobat Accessibility Plug-in Version 21.001.20145 
    % }
    \label{tab:prior_work}
\end{table}




\section{Problem Definition}
\label{sec:def}
We begin with the formal definition of the conventional federated learning scenario, and then introduce our novel Agnostic Personalized Federated Learning (APFL) problem. 

%We then describe the two scenarios for the APFL problem, namely Label and Domain Heterogeneous FL. 

\subsection{Preliminaries}
\label{subsec:fl}
Our main task is solving a given multi-class classification problem in an FL framework. Let $f_g$ be a global model (neural network) at the global server and $\mathcal{F}=\{f_k\}^{K}_{k=1}$ be a set of $K$ local neural networks, where $K$ is the number of local clients.  $\mathcal{D}=\{\textbf{x}_{i}, y_{i}\}^{N}_{i=1}$ be a given dataset, where $N$ is the number of instances, $\textbf{x}_i \in \mathbb{R}^{W \times H \times D}$ is the $i_{th}$ examples in a size of width $W$, height $H$, and depth $D$, with a corresponding target label $y_i \in \{1,\dots,C\}$ for the $C$-way multi-class classification problem. The given dataset $\mathcal{D}$ is then disjointly split into $K$ sub-partitions $\mathcal{P}_{k}=\{\textbf{x}_{k,i}, y_{k,i}\}^{N_{k}}_{i=1}$ s.t. $\mathcal{D} = \bigcup_{k=1}^{K} \mathcal{P}_{k}$, which are distributed to the corresponding local model $f_k$. Let $R$ be the total number of the communication rounds and $r$ denote the index of the $r_{th}$ communication round. At the first round $r$=$1$, the global model $f_g$ initialize the global weights $\theta^{(1)}_{f_g}$ and broadcasts $\theta^{(1)}_{f_g}$ to an arbitrary subset of local models that are available for training at round $r$, such that $\mathcal{F}^{(r)}\subset\mathcal{F}$, $|\mathcal{F}^{(r)}|=K^{(r)}$, and $K^{(r)} \leq K$, where $K^{(r)}$ is the number of available local models at round $r$. Then the active local models $f_{k}\in\mathcal{F}^{(r)}$ perform local training to minimize loss $\mathcal{L}( \theta^{(r)}_{k})$ on the corresponding sub-partition $\mathcal{P}_{k}$ and update their local weights $\theta^{(r+1)}_{k} \leftarrow \theta^{(r)}_{k}-\eta\nabla\mathcal{L}(\theta^{(r)}_{k})$, where $\theta^{(r)}_{k}$ is the set of weights for the local model $f_k$ at round $r$ and $\mathcal{L}(\cdot)$ is the loss function. When the local training is done, the global model $F$ collects and aggregates the learned weights $\theta^{(r+1)}_{f_g} \leftarrow \frac{N_{k}}{N}\sum_{i=1}^{K^{(r)}} \theta_{k}^{(r)}$ and then broadcasts newly updated weights to the local models available at the next round $r+1$. These learning procedures are repeated until the final round $R$. This is the standard setting for centralized federated learning, which aims to find a single global model that works well across all local data. On the other hand, Personalized Federated Learning aims to adapt the individual local models $f_{1:K}$ to their local data distribution $\mathcal{P}_{1:K}$, to obtain specialized solution for each task at the local client, while utilizing the knowledge from other clients. Thus merging the local knowledge for personalized FL is not necessarily done in the form of $\theta^{(r+1)}_{f_g} \leftarrow \frac{N_{k}}{N}\sum_{i=1}^{K^{(r)}} \theta_{k}^{(r)}$, and the specific ways to utilized the knowledge from others depends on the specific algorithm, i.e. $\theta^{(r+1)}_k\leftarrow \theta^{(r)}_k + \sum_{i\neq k}^{K^{(r)}} \omega_i(\theta_{k}^{(r)}-\theta_{i}^{(r)})$, wher $\omega(\cdot)$ is weighing function ~\citep{zhang2021personalized}.


\begin{figure*}[t]
% \vspace{-0.3in}
\small
    \centering
    \includegraphics[width=\textwidth]{figures/images/fig_factorize.pdf} 
    \vspace{-0.25in}
    \caption{\small{\textbf{Illustration of Parameter Factorization Methods:} Left shows conventional matrix factorization with two low rank matrices with rank $\gamma$. Middle represents the method utilizing Hadamard product of low rank matrix for federated learning~\citep{anonymous2022fedpara}. Right illustrates our factorization method for agnostic personalized federated learning, which utilizes rank 1 vectors and highly sparse bias.} }
    \label{fig:factorize}
    \vspace{-0.15in}
\end{figure*}

\begin{figure}[t]
\small
\centering
\vspace{-0.05in}
\begin{tabular}{c c}
    \small    
    
    \hspace{-0.15in} \includegraphics[width=0.24\textwidth]{figures/images/permuted_uv.pdf} & 
    \hspace{-0.225in} \includegraphics[width=0.24\textwidth]{figures/images/domain_uv.pdf}
    \\
    \hspace{-0.1in} (a) MNIST Part. 2 (Permuted) &
    \hspace{-0.125in} (b) CIFAR-10 (Hetero-Domain)
\end{tabular}
\vspace{-0.1in}
\caption{\small{\textbf{Analysis of $\textbf{u}$ and $\textbf{v}$:}} We plot normalized $L_2$ distance of the gradient updates of factorized parameters $\textbf{u}$ and $\textbf{v}$ while learning on (a) MNIST Partition 2 and (b) CIFAR-10 compared to learning on MNIST Partition 1. }
\label{fig:f_analysis}
\vspace{-0.25in}
\end{figure}



\subsection{Agnostic Personalized Federated Learning}
\label{subsec:apfl}

Agnostic Personalized Federated Learning (APFL) is a scenario where any local participants from diverse domains with their own personalized labeling schemes can collaboratively learn, benefiting each other. There exist two critical challenges that need to be tackled to achieve this objective: (1) Label Heterogeneity and (2) Domain Heterogeneity.

\vspace{-0.125in}
\paragraph{Label Heterogeneity } This scenario assumes that the labeling schemes are not perfectly synchronized across all clients, as described in Section~\ref{sec:intro} and Figure~\ref{fig:overview} Left. Most underlying setting for this scenario is the same as the conventional single-domain setting with synchronized labels that is described in Section~\ref{subsec:fl}, except that labels are arbitrarily permuted amongst clients. The local data $\mathcal{P}_{k}$ for the local model $f_k$ is now defined as $\mathcal{P}_{k}=\{\textbf{x}_{k,i}, \varphi_{k}(y_{k,i})\}^{N_{k}}_{i=1}$, where $\varphi_{k}(\cdot)$ is a mapping function for the local model $f_k$ which maps a given class $y_{k,i}$ with a randomly permuted label $p_{k,i}$=$\varphi_{k}(y_{k,i})$. Let the $j$th layer out of $L$ layers in the neural networks of local model $f_k$ be $\ell_{k}^{j}$ and the last layer $\ell_{k}^L$ be the classifier layer. Since each client has differently permuted labels, the personalized classifiers $\ell_{1:K}^{L}$ are no longer compatible to each other. While we can merge the layers below the classifier in this setting, training with heterogeneous labels could still lead to large disparity in the local gradients even in the initial communication round, as described in Figure \ref{fig:concept}.


\vspace{-0.125in}
\paragraph{Domain Heterogeneity} This scenario presumes that local clients learn on their own dataset $\mathcal{D}$, that are completely different from the datasets that are used at other clients, as described in Section~\ref{sec:intro} and Figure~\ref{fig:overview} Right. In this setting, $K$ disjoint datasets $\mathcal{D}_{1:K}$ are assigned to the $K$ local clients $f_{1:K}$, where $\mathcal{D}_k=\{\textbf{x}_{k,i}, y_{k,i}\}^{N_k}_{i=1}$ is the dataset assigned to the local model $f_k$. The number of target classes may differ across clients, such that $y_{k,i} \in \{1,\dots,C_k\}$. We assume complete disjointness across clients, such that there is no instance-wise and class-wise overlap across the datasets: $\varnothing =\bigcap_{k=1}^K \mathcal{D}_k$. Similarly to the label-heterogeneous scenario described above, the personalized classifiers $\ell_{1:K}^{L}$ are no longer compatible to each other due to the heterogeneity in the data and the labels. Hence, the aggregation is done for the layers before the classifier, but they will be also incompatible as the learned model weights will be largely different across domains.


%$\varnothing =\bigcap_{k=1}^K \mathcal{P}_k$



\section{Factorized Federated Learning}
\label{sec:method}

We now provide detailed descriptions of our novel algorithm \texttt{Factorized-FL}. 

\subsection{Kernel Factorization}
\label{subsec:facto}

\citet{Wang2020Federated} discussed that the conventional knowledge aggregation, that is often performed in a coordinate-wise manner, may have severe detrimental effects on the averaged model. This is because the deep neural networks have extremely high-dimensional parameters and thus meaningful element-wise neural matching is not guaranteed when aggregating the weights across different models trained under diverse settings. 

One naive solution to this problem is to factorize model parameters into lower dimensional space, i.e. low rank matrices, as shown in Figure~\ref{fig:factorize} (Left). Conventional approaches, such as SVD, Tucker, or Canonical Polyadic decomposition, however, factorize model parameters after training~\citep{lebedev2014speeding,phan2020stable} is done. Thus, the dimensionality at the time of knowledge aggregation will remain the same as the unfactorized model. ~\citet{konevcny2016federated,anonymous2022fedpara} pre-decompose model parameters to low rank matrices for FL scenarios. While~\citet{konevcny2016federated} use naive low rank matrices, ~\citet{anonymous2022fedpara} uses two sets of low rank matrices to improve expressiveness and utilize them as global and local weights (Figure~\ref{fig:factorize} (Middle)). Unlike prior works, our approach utilizes rank-1 vectors to perform aggregation in the lowest subspace possible for compatibility, while effectively yet efficiently enhancing expressiveness with sparse bias matrices, as shown in Figure~\ref{fig:factorize} (Right) and Figure~\ref{fig:factorize_detail}. Another crucial difference of our method from the previous factorization methods is that, our rank-1 vectors have distinct roles. Our factorization will separate the common knowledge from the task- or domain-specific knowledge, since $\textbf{u}$ could be thought as the bases  (the common knowledge across clients) and $\textbf{v}$ could be thought as the coefficients (client-specific information). 
%Specifically, $\textbf{u}$ captures , and $\textbf{v}$ as describing . 

In Figure~\ref{fig:f_analysis} (a) and (b), which shows the experimental results with the factorized model, we observe that $\textbf{u}$ trained on two datasets becomes closer to that of another dataset (MNIST Partition 1) while $\textbf{v}$ (personalized filter coefficient) remain largely different as federated learning goes on. With this observation, we further aggregate $\textbf{u}$ while allowing $\textbf{v}$ to be different across clients, to allow personalized FL. Further, we use the client specific $\textbf{v}$ for similarity matching, to identify relevant local models from other clients. In following paragraphs, we describe our factorization method in detail, for both fully-connected and convolutional layers. \vspace{-0.15in}


\begin{figure}
\small
    \centering
    \includegraphics[width=0.41\textwidth]{figures/images/factorize.pdf} 
    \vspace{-0.05in}
    \caption{\small{\textbf{Illustration of Kernel Factorization \& Reconstruction}} (1) we multiply two factorized vectors $\textbf{u}$ and $\textbf{v}$ to obtain kernel matrix. (2) we add the sparse bias matrix $\mu$ to complement non-linearity. (3) we reshape the matrix into the original kernel shape. }
    \label{fig:factorize_detail}
    \vspace{-0.3in}
\end{figure}

% \begin{figure*}[t]
% \small
% \centering
% \vspace{-0.05in}
% \begin{tabular}{c c c}
%     \small    
%     \hspace{-0.15in}
%     \includegraphics[width=0.39\textwidth]{figures/images/factorize.pdf} &
%     \hspace{-0.2in} 
%     \includegraphics[width=0.3\textwidth]{figures/images/permuted_uv.png} & 
%     \hspace{-0.15in}
%     \includegraphics[width=0.3\textwidth]{figures/images/domain_uv.png}
%     \\
%     \hspace{-0.1in} (a) Illustration of Kernel Factorization &
%     \hspace{-0.1in} (b) Label Heterogeneous Scenario &
%     \hspace{-0.1in} (c) Domain Heterogeneous Scenario

% \end{tabular}
% \vspace{-0.1in}
% \caption{\small{\textbf{Analysis of our Kernel Factorization method}} (a) describes our kernel factorization process. Following two plots show $L_2$ distance of the gradient updates of factorized parameters $\textbf{u}$ and $\textbf{v}$ in (b) label heterogeneous and (c) domain heterogeneous scenarios. }
% \label{fig:factorize}
% \vspace{-0.1in}
% \end{figure*}




\paragraph{Factorization of Fully-Connected Layers} We assume that each local model $f_k$ has a set of local weights $\theta_k$ across all layers; that is, $\theta_k=\{\textbf{W}^{i}_k\}^{L}_{i=1}$. The dimensionality of the dense weight $\textbf{W}_k^{i}$ for each fully connected layer is $\textbf{W}_k^{i} \in \mathbb{R}^{I \times O}$, where $I$ and $O$ indicate respective input and output dimensions. We can reduce the $I \times O$ complexity by factorizing the high order matrix into the outer product of two vectors as follows:
\begin{equation}
\begin{split}
\textbf{W}_k^{i} = \textbf{u}_k^{i} \times \textbf{v}_k^{i \intercal}, \text{where } \textbf{u}_k^{i} \in \mathbb{R}^{I}, \textbf{v}_k^{i} \in \mathbb{R}^{O}
\end{split}
\end{equation}
 However, such extreme factorization of the weight matrices may result in the loss of expressiveness in the parameter space. Thus, we additionally introduce a highly sparse bias matrix $\mu$ to further capture the information not captured by the outer product of the two vectors as follows:
\begin{equation}
\begin{split}
\textbf{W}_k^{i} = \textbf{u}_k^{i} \times \textbf{v}_k^{i \intercal} \oplus {\mu}_k^{i}, 
\text{where } \\ \textbf{u}_k^{i} \in \mathbb{R}^{I},  
\textbf{v}_k^{i} \in \mathbb{R}^{O}, {\mu}_k^{i} \in \mathbb{R}^{I \times O} 
\end{split}
\end{equation}
We initialize $\mu$ with zeros so that it can gradually capture the additional expressiveness that are not captured by $\textbf{u}$ and $\textbf{v}$ during training. We can control its sparsity by the hyper-parameter for the sparsity regularizer described in~\ref{subsec:algo}.

\vspace{-0.05in}
\paragraph{Factorization of Convolutional Layers} The difference between the fully-connected and convolutional layers is that the convolutional layers have multiple kernels (or filters) such that $\textbf{W}_k^{i} \in \mathbb{R}^{F \times F \times I \times O}$, where $F$ is a size of filters (we assume the filter size is equally paired for the simplicity). To induce $\textbf{u}$ to capture base filter knowledge and $\textbf{v}$ to learn filter coefficient, it is essential to design $\textbf{u}\in \mathbb{R}^{F \cdot F}$ and $\textbf{v}\in \mathbb{R}^{I \cdot O}$, but not in arbitrary ways, such as $\textbf{u}\in \mathbb{R}^{I \cdot F}$ and $\textbf{v}\in \mathbb{R}^{O \cdot F}$ or $\textbf{u}\in \mathbb{R}^{O}$ and $\textbf{v}\in \mathbb{R}^{I \cdot F \cdot F}$. We observe that performance is degenerated when the parameters are ambiguously factorized  (Figure~\ref{fig:analysis} (h)). Our proposed factorization method for convolutional layers are as follows:
\vspace{-0.1in}
\begin{equation}
\begin{split}
\textbf{W}_k^{i} = \pi(\textbf{u}_k^{i} \times \textbf{v}_k^{i \intercal} \oplus {\mu}_k^{i}), \text{where }  \textbf{u}_k^{i} \in \mathbb{R}^{F \cdot F }, \textbf{v}_k^{i} \in \mathbb{R}^{I \cdot O }, \\
{\mu}_k^{i} \in \mathbb{R}^{F \cdot F \times I \cdot O}, \pi(\cdot): \mathbb{R}^{F \cdot F \times I \cdot O} \rightarrow \mathbb{R}^{F \times F \times I \times O},
\end{split}
\end{equation}
% \vspace{-0.05in}
$\pi(\cdot)$ is the weight reshaping function. Note that we reparameterize our model \textit{at initialization time}. Then we reconstruct and train full weights of each layer $\textbf{W}_k^{1:L}$, while optimizing $\textbf{u}_k^{1:L}$, $\textbf{v}^{1:L}_k$, and ${\mu}^{1:L}_k$, respectively, during training phase. 






\subsection{Similarity Matching}
\label{subsec:sim}


% \begin{figure*}[t]
% % \vspace{-0.3in}
% \small
%     \centering
%     \includegraphics[width=\textwidth]{figures/images/framework.pdf} 
%     \vspace{-0.2in}
%     \caption{\small{\textbf{Illustration of Factorized-FL Framework:}} We match relevant clients based on the model representations obtained by the criteria input. Then we reflect the relevant knowledge based on their similarity. As our kernel weights are factorized, knowledge collapse can be effectively reduced when performing personalized knowledge reflection.}
%     \label{fig:framework}
%     \vspace{-0.15in}
% \end{figure*}
\begin{figure}[t]
% \vspace{-0.3in}
\small
    \centering
    % \includegraphics[width=0.47\textwidth]{figures/images/factorize.pdf}\\ 
    \includegraphics[width=0.49\textwidth]{figures/images/framework_2.pdf} 
    \vspace{-0.3in}
    \caption{\small{\textbf{Illustration of Similarity Matching:}} We match relevant clients utilizing the factorized vector $\textbf{v}$ that captures client-specific knowledge. Then we aggregate $\textbf{u}$ based on the similarity. }
    \label{fig:framework}
    \vspace{-0.2in}
\end{figure}



Since we assume task- and domain-heterogeneous FL scenarios, aggregating the parameter bases across all clients may not be optimal, since some of them could be highly irrelevant. \citet{yoon2021federated} and \citet{zhang2021personalized} also demonstrated that avoiding aggregation of irrelevant models from other clients improves local model performance. \citet{yoon2021federated} achieve this goal by taking the weighted combination of task-specific weights from other clients, and \citet{zhang2021personalized} suggest downloading the models from other clients and evaluating their performance on a local validation set, at each client. However, since they require additional communication and computing cost at the local clients, we provide a more efficient yet effective approach to find and match models that are beneficial to each other. 

\vspace{-0.1in}
\paragraph{Efficient similarity matching} Our method utilizes factorized vector $\textbf{v}_k$ for measuring similarity across different models, at the central server. Since $\textbf{v}$ are devised to learn personalized coefficient, we assume that clients trained on similar task or domain will have similar $\textbf{v}$. Specifically, we only use $\textbf{v}^{L-1}_{k}$ of the second last layer (before classifier layer) for similarity matching. The similarity matching function $\Omega(\cdot)$, is defined as the cosine similarity between target client $f_k$ and the other clients $\{f_i\}_{i\neq k}^K$, as follows:
\begin{equation}
\label{eq:sim}
\begin{split}
\Omega(\textbf{v}_{f_k}^{L-1}, \textbf{v}_{f_{i\neq k:K}}^{L-1}) = \{\sigma_i |  
\sigma_i = \frac{\textbf{v}_{f_k} \cdot \textbf{v}_{f_i}}{\|\textbf{v}_{f_k}\|\|\textbf{v}_{f_i}\|}
, \sigma_i \geq \tau \}_{i \neq k}^{K}
\end{split}
\end{equation}
The similarity scores for those with the cosine similarity scores lower than the given threshold $\tau$, are set to zero.
Our method is significantly more efficient than similarity matching approaches which use full gradient updates for clustering clients~\citep{sattler2019clustered,duan2021fedgroup}.

\vspace{-0.1in}
\paragraph{Personalized weighted averaging} We allow each local model to perform weighted aggregation of the model weights from other clients, utilizing their similarity scores:
\begin{equation}
\begin{split}
\textbf{u}_{k}^l \leftarrow \frac{\text{exp}({\epsilon \cdot \sigma_i})}{\sum_{i=1}^K \text{exp}(\epsilon \cdot \sigma_i)} \sum_{i=1}^{K} \textbf{u}_i^l, s.t. \forall  l \in \{1,2,\dots,L\}
\end{split}
\end{equation}
where $\epsilon$ is a hyperparameter for scaling the similarity score $\sigma_i$. We always set $\sigma_k$, the similarity score for itself, as $1.0$. 

\subsection{Learning Objective}
\label{subsec:algo}

Now we describe our final learning objective. Instead of utilizing the single term $\theta_k$ for local weights of neural network $f_k$, now let $\mathcal{U}_k$, $\mathcal{V}_k$, and $\mathcal{M}_k$ be sets of $\textbf{u}_k$, $\textbf{v}_k$, and $\mu_k$ of all layers in $f_k$, s.t. $\mathcal{U}_k=\{\textbf{u}^{i}_k\}^{L}_{i=1}$, $\mathcal{V}_k=\{\textbf{v}^{i}_k\}^{L}_{i=1}$, and $\mathcal{M}_k=\{{\mu}^{i}_k\}^{L}_{i=1}$, then our local objective function is,
\begin{equation}
\begin{split}
\min_{\mathcal{U}_k,\mathcal{V}_k,\mathcal{M}_k} \sum_{\mathcal{B} \in \mathcal{D}_k} \mathcal{L}(\mathcal{B}; \mathcal{U}_k, \mathcal{V}_k, \mathcal{M}_k) + \lambda_{\text{sparsity}} ||\mathcal{M}_k||_1,
\end{split}
\label{eq:loss}
\end{equation}
where $\mathcal{L}$ is the standard cross-entropy loss performed on all minibatch $\mathcal{B} \in \mathcal{D}_k$. We add the $L_1$ sparsity inducing regularization term to make the bias parameters highly sparse, controlling its effect with a hyperparameter $\lambda_{\text{sparsity}}$. Please see our pseudo-coded algorithm Algorithm 1 in Appendix~\ref{appdx:algorithm}.














%We also study such detrimental effects in Section~\ref{sec:intro} and Figure~\ref{fig:concept}. Even with the same initialization, models become extremely heterogeneous and incompatible to each other by label and domain heterogeneity in federated learning scenarios (Figure~\ref{fig:concept} (c)). 

%Reducing the dimensionality of model parameters may alleviate such issues caused by the coordinate-wise knowledge aggregation in the high dimensional space. 


%We also evolve each factorized vectors into one capturing the common knowledge and the other capturing the client-specific knowledge for personalized FL, where the latter is also used to measure the inter-model similarity, which is another crucial difference from previous factorization methods (Figure~\ref{fig:factorize} (Right)). 


% The prior work recommends to set proper rank $\gamma$ values for expressiveness, which increases dimensionality, our method simply uses rank 1 vectors, such as $\textbf{u}$ and $\textbf{v}$ (Figure~\ref{fig:factorize} Right). Our method can separately learn client-general (filter base) and client-specific (filter coefficient) knowledge and utilize them individually. 


%. In Figure~\ref{fig:f_analysis} (a) and (b), We train three (factorized) ResNet-9 networks on CIFAR-10 dataset and show their performance. Even with only $77\%$ of model capacity, our method show superior performance over the Hadamard product approach and even the full kernel model. 

%This results show that our factorization method can separate general and specific knowledge from the extremely high dimensional space while significantly reducing the dimensionality of parameter space, which has great applicability to extremely heterogeneous federated learning scenarios that we want to tackle.

%Such naive aggregation could be less dangerous if there exist some homogeneity across different clients, for example, if the private local data that the each model trains on have the same distribution, and if all models have the same initializations. However, when the local models learn on highly heterogeneous tasks from diverse domains, such simple aggregation scheme may lead to inter-client interference~ \citep{yoon2021federated} where the aggregated model achieves lower performance than those obtained by the local models without aggregation.

% To overcome such issues with simple parameter averaging, we aim to learn a low-dimensional subspace on which the projects of the parameters across different clients become more compatible to each other. To this end, we factorize the parameters of the local models into lower-rank vector components. 



%This efficient similarity matching is another clear advantage of our factorized-FL framework.

%Without additional training or evaluation process at each local client~\citep{yoon2021federated,zhang2021personalized}, 

%\paragraph{Effectiveness} Since $\textbf{v}$ are devised to learn personalized coefficient, we assume that clients trained on similar task or domain will have similar $\textbf{v}$. Thus, we can easily identify which clients are relevant to each other by measuring distance of $\textbf{v}$. We demonstrate its effectiveness by showing that clients are successfully clustered by simply matching $\textbf{v}_{f_k}^{L-1}$ in domain and label heterogeneous scenarios, in Figure~\ref{} (detailed explanation is described in Section~\ref{sec:exp}).



%We can further avoid aggregating irrelevant clients whose similarity scores are lower than the given threshold $\tau$.  

%Note that we omit the layer-wise notation for simplicity.
%where $\Theta$ is a set of $K$ local weights maintained at the server and $\Omega(\cdot)$ is a similarity function that returns a set of pairs of the similarity score $\sigma_i$ and the corresponding local weights $\theta_i$ for all $i \in \{1,2,\dots,K\}$, while satisfying the similarity scores are greater or equal to the given threshold $\tau$. 

% We propose a scheme to aggregate the parameters of only the most relevant local models, based on their task-level similarity. 
%\vspace{-0.05in}
% \input{algorithms/algo_factorized_fl}

% \begin{figure*}[t]
\small
\vspace{-0.1in}
    \small
    \centering
    \begin{tabular}{c c c c c c}
        \small    
        \hspace{-0.15in}
        \includegraphics[width=0.155\textwidth]{figures/images/plain_l1_1.png} &
        \hspace{-0.15in} \includegraphics[width=0.155\textwidth]{figures/images/plain_l1_100.png} &
        \hspace{-0.15in} \includegraphics[width=0.155\textwidth]{figures/images/u_l1_1.png} &
        \hspace{-0.15in} \includegraphics[width=0.155\textwidth]{figures/images/u_l1_100.png} &
        \hspace{-0.15in} \includegraphics[width=0.155\textwidth]{figures/images/v_l1_1.png} &
        \hspace{-0.15in} \includegraphics[width=0.155\textwidth]{figures/images/v_l1_100.png} \\
        
        \vspace{-0.05in}
        \hspace{-0.15in}
        
        % \includegraphics[width=0.155\textwidth]{figures/images/plain_l4_1.png} &
        % \hspace{-0.15in} \includegraphics[width=0.155\textwidth]{figures/images/plain_l4_100.png} &
        % \hspace{-0.15in} \includegraphics[width=0.155\textwidth]{figures/images/u_l4_1.png} &
        % \hspace{-0.15in} \includegraphics[width=0.155\textwidth]{figures/images/u_l4_100.png} &
        % \hspace{-0.15in} \includegraphics[width=0.155\textwidth]{figures/images/v_l4_1.png} &
        % \hspace{-0.15in} \includegraphics[width=0.155\textwidth]{figures/images/v_l4_100.png} \\
        
        \includegraphics[width=0.155\textwidth]{figures/images/plain_l8_1.png} &
        \hspace{-0.15in} \includegraphics[width=0.155\textwidth]{figures/images/plain_l8_100.png} &
        \hspace{-0.15in} \includegraphics[width=0.155\textwidth]{figures/images/u_l8_1.png} &
        \hspace{-0.15in} \includegraphics[width=0.155\textwidth]{figures/images/u_l8_100.png} &
        \hspace{-0.15in} \includegraphics[width=0.155\textwidth]{figures/images/v_l8_1.png} &
        \hspace{-0.15in} \includegraphics[width=0.155\textwidth]{figures/images/v_l8_100.png} \\
        
        
        \hspace{-0.1in} $E=1$ &
        \hspace{-0.1in} $E=100$ &
        \hspace{-0.05in} $E=1$ &
        \hspace{-0.05in} $E=100$ &
        \hspace{-0.05in} $E=1$ &
        \hspace{-0.05in} $E=100$ \\
        
        \multicolumn{2}{c}{(a) Regular Kernel } &
        \multicolumn{2}{c}{(b) $\textbf{u}$ of Factorized Kernel}  &
        \multicolumn{2}{c}{(c) $\textbf{v}$ of Factorized Kernel}
        
    \end{tabular}
    \vspace{-0.1in}
    \caption{\small{\textbf{Euclidean Distance of Parameters in Label Heterogeneous Scenario}} Top row indicates the first conv layer and bottom row represents the last conv layer. The brighter color represents the closer the parameters to each other (see Appendix C for full comparison). }
    \label{fig:distance}
    \vspace{-0.1in}
\end{figure*}



% \paragraph{Personalized Knowledge Reflection} We avoid averaging the local knowledge directly. Instead, we individually reflect only difference between target local model and the other clients to preserve local reliability, inspired by~\cite{zhang2021personalized}. First, given local model $f_k$, we select the other beneficial knowledge that are relevant to $f_k$ via $\Omega(\cdot)$ described in Eq.~\ref{eq:sim}, returning a set of $J$ pairs of $(\sigma_j, \zeta_j,\xi_j,\phi_j)$. Second, we separately update each factorized parameters $\zeta_j,\xi_j,$ and $\phi_j$ while minimizing the collapse of information as follows:
% \begin{equation}
% \begin{split}
% \zeta_{k} \leftarrow \zeta_{k} + \frac{exp(\alpha \cdot \sigma_j)}{\sum_{1:J} exp(\alpha \cdot \sigma_j)} \cdot \sum_{j=1}^{J} (\zeta_{j}-\zeta_{k}),
% \end{split}
% \end{equation}
% where $\alpha$ is scalar value for scaling the reflection ratio of similarity score $\sigma_j$. This personalized knowledge reflection is also equally applied to $\xi_j$ and $\phi_j$, respectively (please see Algorithm~\ref{algo:simfed}).




% Our idea is to view the individual local models as respective encoders that can effectively transform their locally learned knowledge to their own personalized vector embeddings. If we equally inject the same identical input to all encoders, then they will interpret the criteria inputs depending on what they have learned in their own local environments and produce their own personalized transformations, which we can efficiently utilize them as representations of local knowledge to measure the relevance. Formally, the neural networks $f_{k}$ can be represented as:
% \begin{equation}
% \begin{split}
% f_{k}(\hat{\textbf{y}}|\textbf{x}) = P_k(\hat{\textbf{y}}|E_k(\textbf{z}|\textbf{x})),
% \end{split}
% \end{equation}
% where $f_{k}$ is a combination of an encoder  $E_{k}(\textbf{z}|\textbf{x})$ and a predictor $P_{k}(\hat{\textbf{y}}|\textbf{z})$, where $\textbf{z} \in \mathbb{R}^d$ is $d$-length transformed latent vector of input data \textbf{x} and $\hat{\textbf{y}} \in \mathbb{R}^C$ is a logit for the final prediction. We generate the unbiased criteria input $\textbf{x}_{\mathcal{N}} \in \mathbb{R}^{W \times H \times D}$ from the Gaussian normal distribution, such that ${x}_{i} \sim \mathcal{N}(0,1)$ where $x_i$ is its element, and assume that it is located at the server. Since we have $K$ local models $f_{1:K}$, we equally feed the single criteria input $\textbf{x}_{\mathcal{N}}$ to $K$ local encoders so that we can generate a set of $K$ multiple personalized transformations $\mathcal{Z}_\mathcal{N}$ as follows:
% \begin{equation}
% \begin{split}
% \mathcal{Z}_\mathcal{N} = \{\textbf{z}_{\mathcal{N},k} |  \textbf{z}_{\mathcal{N},k}=E_{k}(\textbf{z}|\textbf{x}_{\mathcal{N}})  \}^{K}_{k=1}
% \end{split}
% \end{equation}


% Now our remaining task is to compute the relevancy between the transformations $\mathcal{Z}_\mathcal{N}$ which each local knowledge is efficiently encoded. We use the cosine similarity to measure the relevance between target model $f_k$ with the rest of models in $\mathcal{F}$, as follows:
% \begin{equation}
% \label{eq:sim}
% \begin{split}
% \Omega(k, \tau, \mathcal{Z}_\mathcal{N}, \Theta) = \{(\sigma_i, \theta_i) | \sigma_i = \frac{\textbf{z}_k \cdot \textbf{z}_i}{\|\textbf{z}_k\|\|\textbf{z}_i\|}, \\ \sigma_i \geq \tau,\textbf{z}_i\in\mathcal{Z},\theta_i\in\Theta \}_{i=1}^{K},
% \end{split}
% \end{equation}
% where $\Theta$ is a set of $K$ local weights maintained at the server and $\Omega(\cdot)$ is a similarity function that returns a set of pairs of the similarity score $\sigma_i$ and the corresponding local weights $\theta_i$ for all $i \in \{1,2,\dots,K\}$, while satisfying the similarity scores are greater or equal to the given threshold $\tau$. 
\section{Experiments}

\label{Sec:experiments}



\begin{table*}[t]
    \small
    \centering
    \caption{Node classification performance (Accuracy(\%)$\pm$Std) on various types of noisy graphs}
    \vskip -1.5em
    \begin{tabularx}{0.985\textwidth}{|p{0.05\textwidth}|p{0.14\textwidth}|CC>{\centering\arraybackslash}p{0.1\linewidth}C>{\centering\arraybackslash}p{0.1\linewidth}>{\centering\arraybackslash}p{0.08\linewidth}CC|}
    \hline
    Dataset & Graph & GCN & SuperGAT &Self-Training & RGCN & GCN-jaccard & GCN-SVD & Pro-GNN & Ours \\
    \hline
    
    \multirow{4}{*}{Cora}
        &Raw Graph            & 65.5 $\pm 0.5$& 69.0 $\pm 1.7$ & 67.9 $\pm 0.9$ & 63.0 $\pm 0.7$ &65.7 $\pm 0.6$ & 62.9 $\pm 1.1$  & 65.9 $\pm 1.3$ & \textbf{75.3} $\pm \textbf{0.6}$\\
        &Random Noise        & 59.2 $\pm 0.7$ & 58.8 $\pm 0.4$ & 63.1 $\pm 0.5$ &51.5 $\pm 0.7$ & 57.8 $\pm 1.4$ & 51.5 $\pm 0.7$ & 56.1 $\pm 3.0$ & \textbf{71.8} $\pm \textbf{1.5}$\\
        &Non-Targeted Attack  & 26.8 $\pm 2.5$ & 41.5 $\pm 1.6$ & 29.6 $\pm 0.4$ &30.4 $\pm 1.0$ & 48.3 $\pm2.0$ & 37.1 $\pm 1.4$ & 41.7 $\pm 5.7$& \textbf{70.8} $\pm \textbf{0.7}$  \\
        &Targeted Attack      & 45.3 $\pm 1.2$& 44.4 $\pm 1.3$ &46.7 $\pm 2.1$ &40.3 $\pm 1.0$ & 49.5 $\pm 1.0$ & 44.8 $\pm 0.7$ & 49.7 $\pm 0.9$ & \textbf{67.8} $\pm \textbf{1.2}$ \\

    \hline
    \multirow{4}{*}{Cora-ML}
        &Raw Graph         & 72.4 $\pm 0.8$ & 73.8 $\pm 1.4$ & 72.7 $\pm 1.4$ & 72.9 $\pm 0.7$ & 71.0 $\pm 1.2 $ & 71.1 $\pm 1.0$ & 62.0 $\pm 1.5$ & \textbf{75.6} $\pm \textbf{0.4}$\\
        &Random Noise       & 62.3 $\pm 0.6$ & 63.7 $\pm 0.9$ & 62.8 $\pm 1.3$ & 61.4 $\pm 1.1$ & 61.3 $\pm 0.5$ & 62.6 $\pm 0.6$ & 57.1 $\pm 2.1$ & \textbf{72.9} $\pm \textbf{0.7}$\\
        &Non-Targeted Attack & 13.2 $\pm 1.4$ & 18.6 $\pm 1.5$ & 15.0 $\pm 0.7$ & 11.0 $\pm 1.0$ & 48.9 $\pm 5.3$ & 16.3 $\pm 0.6$ & 18.2 $\pm 2.4$ & \textbf{73.2} $ \pm \textbf{1.2}$ \\
        &Targeted Attack     & 55.7 $\pm 0.7$ & 56.5 $\pm 1.7$ & 57.7 $\pm 1.2$ & 54.6 $\pm 0.6$ & 61.2 $\pm 0.9$ & 53.0 $\pm 0.8$ & 55.1 $\pm 1.6$ & \textbf{70.8} $\pm \textbf{0.7}$ \\

    \hline
    \multirow{4}{*}{Citeseer}
        &Raw Graph           & 64.8 $\pm 1.4$ & 64.2 $\pm 1.7$ & 65.7 $\pm 1.1$ & 56.6 $\pm 1.2$ & 62.2 $\pm 2.0$ & 61.3 $\pm 2.0$ & 60.6 $\pm 2.0 $  & \textbf{71.2} $\pm \textbf{1.4}$\\
        &Random Noise       & 57.0 $\pm 1.2$ & 54.6 $\pm 1.3$ & 58.7 $\pm 2.1$ & 48.2 $\pm 1.2$ & 61.1 $\pm 2.8$ & 48.3 $\pm 1.6$ & 54.4 $\pm 2.6$ & \textbf{68.8} $\pm \textbf{1.5}$\\
        &Non-Targeted Attack & 26.6 $\pm 2.5$ & 42.3 $\pm 2.6$ & 28.8 $\pm 2.7$ &26.6 $\pm 1.1$ & 57.9 $\pm 2.7$ & 41.7 $\pm 1.6$ & 41.6 $\pm 3.1$ & \textbf{68.0} $\pm \textbf{0.4}$ \\
        &Targeted Attack     & 43.9 $\pm 1.7$ & 42.9 $\pm 0.4$ & 47.6 $\pm 1.2$ &35.3 $\pm 1.5$ & 52.5 $\pm 2.3$ & 40.5 $\pm 0.7$ & 48.1 $\pm 1.6$ & \textbf{67.2} $\pm \textbf{1.3}$\\
    \hline
    \multirow{4}{*}{Pubmed}
    & Raw Graph          & 85.9 $\pm 0.1$ & 86.0 $\pm 1.2$ & 86.1 $\pm 0.2$ & 85.1 $\pm 0.1$ & 86.0 $\pm 0.1$  & 83.0 $\pm 0.1$ & 86.1 $\pm 0.1$ & \textbf{86.9} $\pm \textbf{0.1}$  \\
    & Random Noise & 80.5 $\pm 0.1$ & 79.8 $\pm 0.1$ & 81.2 $\pm 0.2$ & 79.7 $\pm 0.1$ &  83.0 $\pm 0.1$  & 82.0 $\pm 0.1$ &  85.1 $\pm 0.2$ &  \textbf{86.4} $\pm \textbf{0.1}$\\
    & Non-Targeted Attack & 73.7 $\pm 0.2$ & 73.8 $\pm 0.2$ & 73.5 $\pm 0.3$ & 73.8 $\pm 0.3$ & 84.4 $\pm 0.1$ & 83.0 $\pm 0.1$ & 86.0 $\pm$ 0.1 & \textbf{86.3} $\pm \textbf{0.1}$ \\
    & Targeted Attack & 76.5 $\pm 0.1$ & 75.6 $\pm 0.1$ & 76.8 $\pm 0.2$ & 76.2 $\pm 0.2$  & 82.7 $\pm 0.2$ &78.1 $\pm 1.3$ & 79.1 $\pm 0.1$ & \textbf{84.3} $\pm \textbf{0.2}$ \\
    \hline
    \end{tabularx}
    
    \label{tab:results}
    \vskip -1.2em
\end{table*}

In this section, we evaluate the proposed RS-GNN on noisy graphs with limited labels to answer the following research questions:
\begin{itemize}[leftmargin=*]
    \item \textbf{RQ1} How robust is the proposed framework on various types of noisy graphs with limited labeled nodes?
    \item \textbf{RQ2} How does the proposed framework perform under various label rates and graph sparsity levels?
    \item \textbf{RQ3} What are the contributions of link predictor and label smoothness regularization from predicted edges on RS-GNN?
\end{itemize}
\subsection{Experimental Settings}
\label{Sec:ex_settings}


\subsubsection{Datasets} 
\label{Sec:datasets}
For a fair comparison, we conduct experiments on four widely used benchmark datasets, i.e., Cora, Cora-ML, Citeseer and Pubmed~\cite{sen2008collective}.
The statistics of the datasets are presented in the Table \ref{tab:dataset} in Appendix. Note that the split of validation and testing on all datasets are the same as described in the cited papers to keep consistence. For the training set, we randomly sample 1\% of nodes as the labeled set for Cora, Cora-ML and Citeseer. For Pubmed, we randomly sample 10\% of nodes to compose the labeled set. The training node set doesn't overlap with the validation and test sets. 

\subsubsection{Noisy Graphs}
To show RS-GNN is robust to various structural noises, we evaluate RS-GNN on the following types of noises:
\begin{itemize}[leftmargin=*]
    \item \textbf{Raw Graphs}: They are the original graphs of the benchmark datasets which may contain inherent structural noise.
    \item \textbf{Random Noise}: We randomly inject fake edges and remove normal edges to add random noise to graphs.
    \item \textbf{Non-Targeted Attack}: 
    We adopt \textit{metattack}~\cite{zugner2019adversarial} to poison the graph structures by adding and removing edges, which aims to reduces the overall performance of GNNs on the whole graph. 
    \item \textbf{Targeted Attack}: It aims to lead the GNN to misclassify target nodes. Following~\cite{tang2020transferring}, we randomly select 15\% nodes as target nodes and apply \textit{nettack}~\cite{zugner2018adversarial} to perturb the graph structure. 

\end{itemize}




\subsubsection{Baselines} We compare RS-GNN with the representative and state-of-the-art GNNs, and robust GNNs against adversarial attacks:
\begin{itemize}[leftmargin=*]
    \item \textbf{GCN}~\cite{kipf2016semi}: GCN is a representative GNN which defines Graph convolution with spectral analysis.
     \item \textbf{SuperGAT}~\cite{kim2021find}: This extends GAT~\cite{velivckovic2017graph} with self-supervised learning. Edge prediction is deployed as the pretext task to guide the learning of attention to facilitate the message-passing.
    \item \textbf{Self-Training}~\cite{li2018deeper}: This is a self-supervised learning method. A GCN is firstly trained on given labels. Then, confident pseudo labels would be added to the label set to improve the GCN.
    \item \textbf{RGCN}~\cite{zhu2019robust}: It uses Gaussian distributions as representations to absorb the effects of adversarial edges. 
    \item \textbf{GCN-jaccard}~\cite{wu2019adversarial}: GCN-Jaccard eliminates edges that connect nodes with low Jaccard similarity, then apply GCN on the graph.
    \item \textbf{GCN-SVD}~\cite{entezari2020all}: This preprocessing method is based on low rank assumption. Low-rank approximation of the perturbed graph is used to train GNNs against adversarial attacks.
    \item \textbf{Pro-GNN}~\cite{jin2020graph}: It applies low-rank and sparsity constraints to learn a clean graph structure close to the noisy graph structure. 
\end{itemize}
For all the baselines, we use the implementation from the repository DeepRobust~\cite{li2020deeprobust}. All the hyperparameters of the baselines are tuned on the validation set to make a fair comparison with RS-GNN.

\subsubsection{Implementation Details}
\label{sec:implementation}
\textit{Each experiment is conducted 5 times} and average results with standard deviations are reported. The hyperparameters are tuned based on the performance of validation set. More specifically, for RS-GNN, we vary $\alpha$ as  \{0.003, 0.03, 0.3, 3, 30 \}, and $\beta$ as \{0.01, 0.03, 0.1, 0.3, 1\}. For all experiments, $T_l$, $T_h$, $\sigma$, and $Q$ are fixed as 0.1, 0.8, 100, and 50, respectively. $K$ is set as 100, 300, 400 and 10 for Cora, Cora-ML, Citeseer and Pubmed, respectively. More details about the hyperparameters sensitivity is discussed in Sec. \ref{Sec:para_analysis}.
A one-hidden layer MLP with 64 filters is applied as the link predictor. We use GCN as the backbone of RS-GNN. Various GNNs can be used in RS-GNN and we leave it as a future work. 


\begin{figure}[t]
\centering
\begin{subfigure}{0.49\columnwidth}
    \centering
    \includegraphics[width=0.85\linewidth]{figure/meta_ptb.pdf} 
    \vskip -0.5em
    \caption{Metattack}
    % \label{fig:1_a}
\end{subfigure}
%\vspace{-1em}
\begin{subfigure}{0.49\columnwidth}
    \centering
    \includegraphics[width=0.85\linewidth]{figure/random_ptb.pdf} 
    \vskip -0.5em
    \caption{Random Noise}
    % \label{fig:1_b}
\end{subfigure}
\vspace{-1.2em}
\caption{Robustness under different Ptb rates on Cora.  }
\label{fig:ptb}
\vskip -1.5em
\end{figure}

\subsection{Performance on Noisy Graphs}
To answer \textbf{RQ1}, we first compare RS-GNN with the baselines on various noisy graphs. We then evaluate the performance of RS-GNN on the graphs with different levels of structural noise.



\subsubsection{Comparisons with baselines}
We conduct experiments on four types of noisy graphs, i.e., raw graphs, graphs with random noise, non-targeted attack perturbed graphs and targeted attack perturbed graphs. The perturbation rate of non-targeted attack and targeted attack is 0.15. The perturbation rate of random noise is set as 0.3. Since we focus on noisy graph with sparse labels, we set the label rates as 0.01 for Cora, Cora-ML, Citeseer and 0.1 for Pubmed. The results are reported in Table \ref{tab:results}, where we can observe:

\begin{itemize}[leftmargin=*]
    \item With limited labeled nodes, GCN even hardly performs well on raw graph, which indicates the necessity of investigating method to address the challenge of sparsely labeled graphs. Though recent GNNs such as SuperGAT and Self-Training can improve the performance with self-supervised learning, our RS-GNN still outperforms them by a large margin. This shows the effectiveness of graph densification in dealing with sparsely labeled graphs.
    \item The structural noise further degrades the performance of GCN, but its impact to RS-GNN is negligible. RS-GNN achieves better results than the state-of-the-art robust GNNs. This indicates RS-GNN could eliminate the effects of the noisy edges.
    \item Compared with the preprocessing methods and Pro-GNN, RS-GNN achieves higher accuracy on the sparsely labeled graphs perturbed by attack methods. 
    This is because the baselines only focus on eliminating potential noisy edges, which will even result in less involvement of unlabeled nodes. 
    By contrast, RS-GNN can down-weights/removes the adversarial edges to defend the adversarial attacks and densify the graph to facilitate the message passing for predictions of unlabeled nodes.
    
\end{itemize}



\subsubsection{Robustness Under Different Ptb Rates } 
To show that RS-GNN is resistant to different levels of structural noise, we vary the perturbation rate as $\{0\%, 5\%, 10\%, \dots, 25\%\}$ and compare the performance of RS-GNN with the most effective baselines. The label rate is fixed as 0.01. Since we have similar observations on other datasets, we only report the average accuracy and standard deviation on Cora in Figure \ref{fig:ptb}. From the figure, we make following observations:
\begin{itemize}[leftmargin=*]
    \item As the perturbation rate increases, the performance of all the baselines drop significantly, which is as expected. Though the performance of RS-GNN also drops, it is much stable and consistently outperforms the baselines, which shows the robustness of RS-GNN against various levels of attacks and random noise; and %  the proposed framework RS-GNN  For the noisy graph perturbed by \textit{metattack}, the performance of GCN degrades significantly. Our proposed RS-GNN could achieve high performance under high perturbation rate and consistently outperforms the baselines, which demonstrates RS-GNN could well defend attacks under different perturbation rates.
    \item  Compared with GCN, RS-GNN uses GCN as backbone but significantly outperforms GCN, especially when the perturbation rate is large, which shows the effectiveness of eliminating the effects of noisy edges and densifying the graph to benefit the predictions given limited labels. % consistently achieve high performance on the graph containing various amounts of noisy edges. It demonstrates that RS-GNN is robust when applied to different levels of random structural noises. \suhang{TODO}
\end{itemize}

\begin{figure}[t]
\centering
\begin{subfigure}{0.49\columnwidth}
    \centering
    \includegraphics[width=0.85\linewidth]{figure/random_cora_2.pdf} 
    \vskip -0.8em
    \caption{Cora}
    % \label{fig:1_a}
\end{subfigure}
%\vspace{-1em}
\begin{subfigure}{0.49\columnwidth}
    \centering
    \includegraphics[width=0.85\linewidth]{figure/random_cora_ml_2.pdf} 
    \vskip -0.8em
    \caption{CoraML}
    % \label{fig:1_b}
\end{subfigure}
\vspace{-1.5em}
\caption{ Distributions of the weights of normal and noisy edges on the generated graph.}
\label{fig:weight}
\vskip -1em
\end{figure}

\subsection{Analysis of the Learned Graph} 
To demonstrate that RS-GNN could alleviate negative effects of noisy edges by downweighting the noisy edges, we investigate the distribution of the learned edge weights $\mathbf{S}_{ij}$ of normal and noisy edges in this subsection. The edge weight  distributions of graphs perturbed by random noise with 30\% perturbation rate on Cora and Cora-ML are shown in Fig.~\ref{fig:weight}.  From this figure, we observe: (\textbf{i}) The weights of noisy edges are significantly lower than the weights of normal edges, which indicates RS-GNN manages to reduce the effects of noisy edges for robust GNN; and (\textbf{ii}) Although most normal edges have higher weights, some of their weights are very low, which implies inherent noise exists in the graph and RS-GNN is able to get rid of such inherent structural noise. 

We also provide more details about the number of involved unlabeled nodes with the learned graph in Appendix~\ref{sec:app_graph}, which proves RS-GNN can enhance the involvement of unlabeled nodes.






\begin{figure}[t]
\centering
\begin{subfigure}{0.49\columnwidth}
    \centering
    \includegraphics[width=0.9\linewidth]{figure/clean_label_rate.pdf} 
    \vskip -0.5em
    \caption{Raw Graph}
    % \label{fig:1_a}
\end{subfigure}
%\vspace{-1em}
\begin{subfigure}{0.49\columnwidth}
    \centering
    \includegraphics[width=0.9\linewidth]{figure/ptb_label_rate.pdf} 
    \vskip -0.5em
    \caption{Metattack with 15\% Ptb}
    % \label{fig:1_b}
\end{subfigure}
\vspace{-1.2em}
\caption{Performance on Cora with different label rates.  }
\label{fig:label_rate}
\vskip -0.8em
\end{figure}


\subsection{Impacts of Label Rate and Graph Sparsity}

To answer \textbf{RQ2}, we study the impacts of the number of labeled nodes and sparsity of the graph by varying the label rate and edge rate of the graph. The hyperparameters are selected with the process described in Sec. \ref{sec:implementation}. Each experiment is conducted 5 times and average accuracy with standard deviation are reported.



\subsubsection{Impacts of Label Rate} We vary label rates as \{0.01, 0.02,\dots, 0.06\}. Experiments are conducted on raw graphs and graphs perturbed by \textit{mettack} to study the effectiveness of RS-GNN under various label rates. The results on Cora are shown in Fig.~\ref{fig:label_rate}. We have similar observations on other datasets. From Fig.~\ref{fig:label_rate}, we observe:
\begin{itemize}[leftmargin=*]
    \item Generally, as the increase of label rate, the performances of all the methods increase, which is as expected.
    \item For the raw graph, though RS-GNN consistently outperforms the baselines, as the label rate increases, the improvement of RS-GNN becomes marginal. This is because the raw graph doesn't contain much noise. Thus, as label rate increases to 6\%, there are already adequate labels. Since higher label rates would result in more unlabeled nodes involving in the training, the effects of densifying graphs and label smoothness become less significant; %when the label rate is high enough to involve most of the unlabeled nodes.
    \item For the metattack graph, as the label rate increases, RS-GNN still significantly outperforms baselines. That's because the training graph contains a lot of adversarial edges. Though we have enough training labels, the adversarial edges can still contaminate the message passing of GNNs. But RS-GNN can eliminate noisy edges and densify the graph, thus having better results.
\end{itemize}



\subsubsection{Impacts of Graph Sparsity} As RS-GNN can generate dense graphs, it should have the ability to handle sparse graphs. Thus, we randomly select $x\%$ edges from the raw graph to build graphs of different sparsity levels. We vary edge rate $x\%$ from 20\% to 100\% with a step of 40\%. Since we are interested in how the sparsity of the graph could affect RS-GNN in generating dense graphs, we only focus on the performance on raw graphs. 
The average results of 5 runs on Citeseer are reported in Table~\ref{tab:sparsity}. From the table, we have the following observations:
\begin{itemize}[leftmargin=*]
    \item As the edge rate decreases, the performance of all the methods decrease, which is because message-passing of GNNs becomes ineffective on very sparse graphs;
    \item RS-GNN consistently outperforms the baselines. In particular, when the graph becomes more sparse, the improvement of RS-GNN over the baselines becomes larger. For example, the improvement of RS-GNN over GCN on Citeseer is 6.4\% when Edge Rate is 100\%, and becomes 9.2\% when Edge Rate is 20\%, which shows the importance of generating edges for densifying the graph and smoothing predictions with the learned graph. 
\end{itemize}


\begin{table}[t]
    \small
    \centering
    \caption{Accuracy (\%) on Citeseer in different sparsity levels. }
    \vskip-1.5em
    \begin{tabularx}{0.96\linewidth}{>{\centering\arraybackslash}p{0.20\linewidth}CCC}
    \toprule
    Edge Rate (\%) & GCN & Pro-GNN & RS-GNN\\
    \midrule
    % \multirow{3}{*}{Cora} 
    % & 20 & 51.9 $\pm 2.0$ & 49.5 $\pm 0.8$ & \textbf{64.7} $\pm \textbf{1.7}$\\ 
    % % & 40 & 53.9 $\pm 1.2$ & 51.2 $\pm 0.7$ & \textbf{66.3} $\pm \textbf{1.2}$ \\
    % & 60 & 62.0 $\pm 0.3$ & 62.5 $\pm 0.5$ & \textbf{68.5} $\pm \textbf{1.7}$\\
    % % & 80 & 64.2 $\pm 0.5$ & 64.6 $\pm 0.2$ & \textbf{72.8} $\pm \textbf{1.4}$ \\
    % & 100 & 65.5 $\pm 0.5$ & 65.9 $\pm 1.1$ & \textbf{75.3} $\pm \textbf{0.6}$\\
    % \hline

    20 & 54.5 $\pm 1.2$ & 55.2 $\pm 1.6$ & \textbf{63.7} $\pm \textbf{2.2}$\\
    % & 40 & 56.8 $\pm 1.1$ & 58.6 $\pm 1.5$ & \textbf{67.3} $\pm \textbf{1.8}$\\
    60 & 58.7 $\pm 1.8$ & 58.3 $\pm 2.4$ &
    \textbf{69.8} $\pm \textbf{1.1}$\\
    % & 80 & 60.1 $\pm 2.2$ &60.2 $\pm 1.3$ & \textbf{70.4} $\pm \textbf{0.7}$\\
    100 & 64.8 $\pm 1.4$ & 60.6 $\pm 2.0$ &
    \textbf{71.2} $\pm \textbf{1.4}$\\
    \bottomrule
    \end{tabularx}
    \label{tab:sparsity}
    \vskip -1.em
\end{table}




%\vspace{-1em}

\begin{figure}[h]
\centering
\begin{subfigure}{0.49\columnwidth}
    \centering
    \includegraphics[width=0.85\linewidth]{figure/neta_abl.pdf} 
    \vskip -0.5em
    \caption{Nettack}
    \label{fig:abla_neta}
\end{subfigure}
\begin{subfigure}{0.49\columnwidth}
    \centering
    \includegraphics[width=0.85\linewidth]{figure/abla_meta.pdf} 
    \vskip -0.5em
    \caption{Metattack with 15\% Ptb}
    \label{fig:abla_meta}
\end{subfigure}
\vspace{-1.3em}
\caption{Ablation studies on Cora with different label rates.}
\label{fig:abl}
\vskip -1.8em
\end{figure}
\subsection{Ablation Study}
To answer \textbf{RQ3}, we conduct ablation studies to understand the effects of graph densification, graph purification and label smoothness regularization. In RS-GNN, the link predictor densify the graph to enhance the performance on unlabeled nodes. To demonstrate the effects of adding edges with the link predictor, we remove the process of adding edges and obtain RS-GNN$\backslash$A. 
To testify the effectiveness of the label smoothness regularization based on the generated graph, we eliminate the label smoothness regularization and get RS-GNN$\backslash$U. To show our link predictor can eliminate the effects of noisy edges, we compare a variant named as RS-GNN$\backslash$AU which only use the link predictor to denoise graphs. Graph desification and label smoothness are not applied in RS-GNN$\backslash$AU. 
We also implement a variant named as RS-GNN$_{GCN}$ which uses GCN as link predictor to show that the noisy edges would largely affects the GNNs for link prediction. Hyperparameters selection follows the process in Sec~\ref{sec:implementation}. We only show the results on the Cora graph perturbed with \textit{metattack} and random noise, because similar trends are observed on other datasets. Results are presented in Fig.~\ref{fig:abl}. From this figure, we observe that: 
\begin{itemize}[leftmargin=*]
    \item RS-GNN performs much better than RS-GNN$\backslash$A and RS-GNN$\backslash$U, which shows that densifying graphs and label smoothness with the learned graph can address the label sparsity issue;
    \item With the increase of label rate, the gap between RS-GNN and RS-GNN$\backslash$U will be narrowed. This is consistent with our analysis that higher label rates would involve more unlabeled nodes;
    \item RS-GNN$_{GCN}$ performs much worse than RS-GNN, which indicates adversarial edges would impair GCN and result in a poor link predictor for denoising and densification.
\end{itemize} 


\begin{figure}[t]
\centering
\begin{subfigure}{0.49\columnwidth}
    \centering
    \includegraphics[width=0.98\linewidth]{figure/cora_para_clean_2}
    \vskip -0.5em
    \caption{Raw Graph}
    \label{fig:para_raw}
\end{subfigure}
%\vspace{-1em}
\begin{subfigure}{0.49\columnwidth}
    \centering
    \includegraphics[width=0.98\linewidth]{figure/cora_para_ptb_2}
    \vskip -0.5em
    \caption{Metattack with 15\% Ptb}
    \label{fig:para_meta}
\end{subfigure}
\vspace{-1em}
\caption{Parameter sensitivity analysis on Cora.}
\label{fig:para}
\vskip -1.7em
\end{figure}


\subsection{Parameter Sensitivity Analysis}
\label{Sec:para_analysis}
In this subsection, we explore the sensitivity of the most crucial hyperparameters $\alpha$ and $\beta$ which are in the final objective function of RS-GNN. The analysis about other hyperparameters is presented in the supplementary material. $\alpha$ controls how well the link predictor reconstructs the noisy graph and $\beta$ controls the contribution of label smoothness. To investigate the effects of $\alpha$ and $\beta$, we vary the values of $\alpha$ as $\{0.003, 0.03, 0.3, 3, 30\}$ and $\beta$ as $\{0.01, 0.03, 0.1, 0.3, 1, 3\}$ on Cora. The results are shown in Fig~\ref{fig:para}. In the raw graph, when $\alpha$ is large, the accuracy is stable and high. But if the $\alpha$ is too large in the perturbed graph, the performance would decrease. This difference is due to the noise levels of the raw graph and the perturbed graph. The structural noise in the perturbed graph is severe, faithfully reconstructing the perturbed graph with high $\alpha$ would lead to a poor link predictor. As for the $\beta$, a value between 0.03 to 0.3 generally gives good performance, which eases the parameter selection.
 We propose a novel commonsense reasoning challenge, \textsc{RiddleSense}, which requires complex commonsense skills for reasoning about creative and counterfactual questions, coming with a large multiple-choice QA dataset.  
 We systematically evaluate recent commonsense reasoning methods over the proposed \textsc{RiddleSense} dataset, and find that the best model is still far behind human performance, suggesting that there is still much space for commonsense reasoning methods to improve.
 We hope \textsc{RiddleSense} can serve as a benchmark dataset for future research targeting complex commonsense reasoning and computational creativity.


\section*{Acknowledgements}
This research is supported in part by the Office of the Director of National Intelligence (ODNI), Intelligence Advanced Research Projects Activity (IARPA), via Contract No. 2019-19051600007, the DARPA MCS program under Contract No. N660011924033 with the United States Office Of Naval Research, the Defense Advanced Research Projects Agency with award W911NF-19-20271, and NSF SMA 18-29268. The views and conclusions contained herein are those of the authors and should not be interpreted as necessarily representing the official policies, either expressed or implied, of ODNI, IARPA, or the U.S. Government. We would like to thank all the collaborators in USC INK research lab and the reviewers for their constructive feedback on the work.

\bibliography{citation}
\bibliographystyle{icml2022}
%%%%%%%%%%%%%%%%%%%%%%%%%%%%%%%%%%%%%%%%%%%%%%%%%%%%%%%%%%%%%%%%%%%%%%%%%%%%%%%
%%%%%%%%%%%%%%%%%%%%%%%%%%%%%%%%%%%%%%%%%%%%%%%%%%%%%%%%%%%%%%%%%%%%%%%%%%%%%%%
% APPENDIX
%%%%%%%%%%%%%%%%%%%%%%%%%%%%%%%%%%%%%%%%%%%%%%%%%%%%%%%%%%%%%%%%%%%%%%%%%%%%%%%
%%%%%%%%%%%%%%%%%%%%%%%%%%%%%%%%%%%%%%%%%%%%%%%%%%%%%%%%%%%%%%%%%%%%%%%%%%%%%%%
\newpage
\newpage

\section*{Appendix}
\label{sec:appendix}

% \subsection*{Trackers on \BEFORE and \AFTER Separately for Websites with and without a Consent Banner }

% For the sake of completeness, we here show the number of Trackers per website and websites with at least one Tracker, similar to what is shown in Figure~\ref{fig:ca_country}. Unlike the previous figure, here we show separate bars for websites where \TOOL did or did not find a Consent Banner. Our goal is to show how Tracker number varies on the \BEFORE and \AFTER for those websites implementing the Consent Banner.

% The dark red bars show our measurements on the \BEFORE, only for the 57.3\% of websites where \TOOL \emph{found} a Consent Banner. The blue bars show numbers on \AFTER for the same websites. The figure allows quantifying the (sizeable) impact on tracking of accepting the Consent Banner on websites implementing one. We find that the tracking pervasiveness upon acceptance largely increases, leading to similar conclusions as in Figure~\ref{fig:ca_country}. For reference, the light red bars report the same measure for the 42.7\% of websites where \TOOL \emph{did not} find any Consent Banner.





% \begin{figure}[!h]
%     \centering
%     \begin{subfigure}[t]{0.495\columnwidth}
%         \includegraphics[width=\columnwidth]{figures/cookieaccept_websites_with_trackers_separate.pdf}
%         \caption{Percentage of websites embedding Trackers.}
%         \label{fig:ca_country_one_sep}
%     \end{subfigure}
%     \begin{subfigure}[t]{0.495\columnwidth}
%         \includegraphics[width=\columnwidth]{figures/cookieaccept_trackers_per_website_separate.pdf}
%         \caption{Average number of Trackers per website.}
%         \label{fig:ca_country_avg_sep}
%     \end{subfigure}
% 	\caption{Trackers penetration and number, 
% 	on websites during different phases of a browsing sessions (top 2\,500 websites per country). We separate websites with and without a Consent Banner. }
% 	\label{fig:ca_country_sep}
% \end{figure}

% \newpage

\subsection*{Impact of Repeated Visits on Tracking Measurements}

We here complement the analysis we carried out on the last paragraph of Section~\ref{sec:trackers_country}. Web tracking involves a number of mechanisms ( real-time bidding among all ) that result in the same page containing different Trackers on multiple visits. To obtain a reliable picture, we repeat each test 5 times. In Figure~\ref{fig:visit_nb}, we show how two macroscopic tracking measurements vary with different number of repetited visits for each website. The blue line in the figure shows the fraction of websites that contain at least one Tracker when measured with an increasing number of test repetitions. It is moderately affected by the number of tests, increasing from 69.1\% with a single repetition to 70.0\% with 5 repetitions. Similarly, the average number of Trackers, increases from $6.5$ to $7.8$.

\begin{figure}[!h]
    \centering
    \includegraphics[width=0.6\columnwidth]{figures/cookieaccept_visits_nb.pdf}
    \caption{Variation of tracker number with different numbers of repeated visits. Measurements have sizeable despite moderate variation when repeated.}
	\label{fig:visit_nb}
\end{figure}

\newpage

\subsection*{Trackers per Website (Tranco List)}

We here report the same analyses depicted in Figure~\ref{fig:tranco_tp} and Figure~\ref{fig:ca_perf_tp} showing the number of Trackers instead of the number of Third-Parties. The two pictures lead to similar conclusions.

\begin{figure}[!h]
    \centering
    \includegraphics[width=0.5\columnwidth]{figures/cookieaccept_tranco_rank_eu_tracker_nb.pdf}
    \caption{Average number of Trackers per website (Tranco list).}
    \label{fig:tranco_trackers}
\end{figure}


\begin{figure}[!h]
    \centering
    \includegraphics[width=0.5\textwidth]{figures/cookieaccept_tracker_nb_tranco.pdf}
    \caption{Distribution of the number of Trackers (Tranco list). Notice the log scales.}
    \label{fig:ca_perf_tracker}
\end{figure}


%%%%%%%%%%%%%%%%%%%%%%%%%%%%%%%%%%%%%%%%%%%%%%%%%%%%%%%%%%%%%%%%%%%%%%%%%%%%%%%
%%%%%%%%%%%%%%%%%%%%%%%%%%%%%%%%%%%%%%%%%%%%%%%%%%%%%%%%%%%%%%%%%%%%%%%%%%%%%%%


\end{document}


% This document was modified from the file originally made available by
% Pat Langley and Andrea Danyluk for ICML-2K. This version was created
% by Iain Murray in 2018, and modified by Alexandre Bouchard in
% 2019 and 2021 and by Csaba Szepesvari, Gang Niu and Sivan Sabato in 2022. 
% Previous contributors include Dan Roy, Lise Getoor and Tobias
% Scheffer, which was slightly modified from the 2010 version by
% Thorsten Joachims & Johannes Fuernkranz, slightly modified from the
% 2009 version by Kiri Wagstaff and Sam Roweis's 2008 version, which is
% slightly modified from Prasad Tadepalli's 2007 version which is a
% lightly changed version of the previous year's version by Andrew
% Moore, which was in turn edited from those of Kristian Kersting and
% Codrina Lauth. Alex Smola contributed to the algorithmic style files.
