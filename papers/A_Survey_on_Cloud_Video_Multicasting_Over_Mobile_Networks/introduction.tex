\section{Introduction}
\label{sec:intro}
During the past five years, the demand for multimedia streaming over mobile networks has been steadily increased. According to a recent report published by Cisco \cite{ref1}, the data traffic over mobile networks was equivalent to 240 terabytes per month in 2010. It is expected that this traffic will increase 26 times to reach 6300 terabytes for every month at the end of 2015. In the same report, Cisco predicted around sixty six percent of this traffic will most likely carry videos, whereas almost twenty one percent of this traffic will be caused by data applications.

Since the current 3G cellular networks only support unicast communications, it would not be efficient to transmit simultaneous multimedia streams to a wide number of mobile devices. In order to cope with this issue, cellular service providers may either deploy supplementary base stations within their networks or purchase additional wireless spectrums. Unfortunately, both
approaches are not preferred because they are extremely expensive, which can cost hundreds of million dollars. To end the bandwidth crisis, cellular service providers are recommended to depend on multicast-capable 4G cellular networks. Currently, the WiMAX standard defines Multicast and Broadcast Service (MBS) in the data link layer in order to facilitate the process of initiating multicasting and broadcasting sessions \cite{ref2}. Similarly, Evolved Multimedia Broadcast Multicast Services (eMBMS) allows LTE cellular networks to deliver video streams over multicast groups \cite{ref3}. With these multicastcapable networks, a streaming server may reduce the network load by multicasting its video streams such that mobile devices interested in the same video stream can subscribe to a multicast group.

Someone may ask about the possibility of applying the multimedia multicasting algorithms used in the Internet-based applications. As a matter of fact, these conventional methods cannot easily be implemented on mobile networks for several reasons. For instance, a wireless channel is severely vulnerable to some physical phenomena such as multipath fading and interference. Furthermore, a mobile network always suffers from the instability of the peer-to-peer connections for a long time period due to the dynamic movement of its users. Usually, these factors make clients in a wireless network experience high and variable round trip time, link outage, rate fluctuations, and occasional burst losses. In addition to these challenges, mobile receivers usually have constrained power supplies, low computational abilities, and limited butter spaces. As a consequence, multimedia streaming approaches designed specifically for wired networks are not recommended to be applied over wireless and mobile networks.

This survey aims at investigating the state of art approaches in the multimedia multicasting over mobile networks. Since several schemes have been proposed in this area, we focus just on the algorithms introduced specifically for the video-on-demand applications. Hence, concentrating on video-ondemand applications will cover a wide range of applications without violating the limited scope of this report.