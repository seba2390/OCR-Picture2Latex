\usepackage{listings}
%\usepackage[usenames,dvipsnames]{xcolor}
\usepackage{xcolor}


%%% global pour tous

\lstset{%
    basicstyle       = \ttfamily,
    keywordstyle     = \bfseries\color{blue},
    stringstyle      = \color{magenta},
    commentstyle     = \color{ForestGreen},
    showstringspaces = false,
    literate={á}{{\'a}}1 {ã}{{\~a}}1 {é}{{\'e}}1 {à}{{\`a}}1 {è}{{\`e}}1 {ê}{{\^e}}1 {ï}{{\"i}}1 {î}{{\^i}}1 {ù}{{\`u}}1,
}





%%
%% Julia definition (c) 2014 Jubobs
%%
\lstdefinelanguage{Julia}%
  {morekeywords={abstract,break,case,catch,const,continue,do,else,elseif,%
      begin,end,export,false,for,function,immutable,import,importall,if,in,%
      macro,module,otherwise,quote,return,switch,true,try,type,typealias,%
      using,while},%
   sensitive=true,%
   alsoother={$},%
   morecomment=[l]\#,%
   morecomment=[n]{\#=}{=\#},%
   morestring=[s]{"}{"},%
   morestring=[m]{'}{'},%
}[keywords,comments,strings]%

\newtcblisting{Julia}{enhanced,breakable,listing only,listing options={language=Julia},colback=white}
\newtcbinputlisting{\codeJulia}[1]{enhanced,breakable,listing only,listing options={language=Julia},listing file={#1},colback=white}
\newtcbinputlisting{\extraitcodeJulia}[3]{enhanced,breakable,listing only,listing options={language=Julia,firstline={#2},lastline={#3}},listing file={#1},colback=white}
%\lstnewenvironment{Julia}
%  {\lstset{%
%    language         = Julia,
%}}
%  {}


\lstnewenvironment{Scilab}
  {\lstset{language=Scilab,
}}
  {}
\lstnewenvironment{Bash}
  {\lstset{language=bash,
}}
  {}
