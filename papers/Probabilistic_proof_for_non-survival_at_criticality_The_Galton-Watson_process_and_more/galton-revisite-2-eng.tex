\documentclass[svgnames]{amsart}

% version du 1er avril 2003

%\usepackage{typearea}
\usepackage[utf8]{inputenc}
%\usepackage[T1]{fontenc}
\usepackage{amsmath,amssymb,amsthm}
%\usepackage{showkeys}
\usepackage{graphicx}
\usepackage[english]{babel}
%\usepackage{macros}

%\usepackage{lscape}
\DeclareMathAlphabet{\mathbbo}{U}{bbold}{m}{n}
\usepackage{verbatim}
\newcommand{\ber}{\text{Ber}}
\newcommand{\floor}[1]{\lfloor #1 \rfloor}
\newcommand{\gestoch}{\succeq}
\newcommand{\lestoch}{\preceq}
\newcommand{\N}{\ensuremath{\mathbb{N}}}
\newcommand{\D}{\ensuremath{\mathbb{D}}}
\newcommand{\Z}{\ensuremath{\mathbb{Z}}}
\newcommand{\Zd}{\ensuremath{\mathbb{Z}^d}}
\newcommand{\Q}{\ensuremath{\mathbb{Q}}}
\newcommand{\R}{\ensuremath{\mathbb{R}}}
\newcommand{\Rd}{\ensuremath{\mathbb{R}^d}}
\newcommand{\Rn}{\ensuremath{\mathbb{R}^n}}
\newcommand{\K}{\ensuremath{\mathbb{K}}}
\newcommand{\C}{\ensuremath{\mathbb{C}}}
\newcommand{\T}{\ensuremath{\mathbb{T}}}
\newcommand{\U}{\ensuremath{\mathbb{U}}}
\newcommand{\B}{\ensuremath{\mathbb{B}}}
\newcommand{\E}{\ensuremath{\mathbb{E}}}
\newcommand{\X}{\ensuremath{\mathbb{X}}}
\newcommand{\G}{\ensuremath{\mathfrak{G}}}
\renewcommand{\H}{\ensuremath{\mathcal{H}}}
\newcommand{\tr}{\text{Tr }}
\newcommand{\cl}{\text{cl}}
\newcommand{\ie}{\emph{i.e. }}
\newcommand{\Cov}{\text{Cov}}
\newcommand{\Covar}{\text{Covar }}
\renewcommand{\P}{\ensuremath{\mathbb{P}}}
%{m\hspace{-0.3cm}\diagup}
\newcommand{\resp}{\emph{resp.}\ }
\newcommand{\Sn}{\ensuremath{\mathfrak{S}_{n}}}
\newcommand{\Kn}[1][K]{\ensuremath{K_{n}(\mathbb{#1})}}
\newcommand{\Gln}[1][K]{\ensuremath{Gl_{n}(\mathbb{#1})}}
\newcommand{\Gld}[1][K]{\ensuremath{Gl_{d}(\mathbb{#1})}}
\renewcommand{\epsilon}{\varepsilon}
\renewcommand{\limsup}{\overline{\lim}\quad}
\renewcommand{\liminf}{\underline{\lim}\quad}
\renewcommand{\Re}{\text{Re }}
\renewcommand{\Im}{\text{Im }}


\newcommand{\Remarque}[1][]{\textbf{Remarque#1 : }}
\newcommand{\Exemple}[1][]{\textbf{Exemple#1 : }}
\newcommand{\defi}[1][]{\textbf{Définition#1 : }}
\newcommand{\exo}[1][]{\textbf{Exercice#1 : }}

\newcommand{\miniop}[3]{%
\renewcommand{\arraystretch}{0.6}
\begin{array}{c}
{\scriptstyle #1}\\
#2\\
{\scriptstyle #3}
\end{array}
\renewcommand{\arraystretch}{1}}

\newcommand{\Card}[1]{\vert #1 \vert}
\newcommand{\grad}{\text{grad }}

%\newcommand{\1}{1\hspace{-2.7mm}1}
\newcommand{\1}{\mathbbo{1}}
\newcommand{\Var}{\text{Var }}
\newcommand{\Id}{\text{Id}}

%\newcommand{\Covar}{\text{Covar }}
\newcommand{\supp}{\text{supp }}
\newcommand{\spec}{\text{spec }}

\newtheorem{theo}{Theorem}
\newtheorem{lemme}{Lemma}
\newtheorem{coro}{Corollary}
\newtheorem{rem}{Remark}
\newtheorem{prop}{Proposition}

\setlength{\topmargin}{-0.5in} \setlength{\textheight}{9.25in}
%\setlength{\oddsidemargin}{0in} \setlength{\evensidemargin}{0in}
%\setlength{\textwidth}{6in}

\newcommand{\tribuf}{\mathcal{F}}
\newcommand{\tribubor}{\mathcal{B}(\R)}


\usepackage[most]{tcolorbox}
\tcbuselibrary{listingsutf8}
\usepackage{listings}
%\usepackage[usenames,dvipsnames]{xcolor}
\usepackage{xcolor}


%%% global pour tous

\lstset{%
    basicstyle       = \ttfamily,
    keywordstyle     = \bfseries\color{blue},
    stringstyle      = \color{magenta},
    commentstyle     = \color{ForestGreen},
    showstringspaces = false,
    literate={á}{{\'a}}1 {ã}{{\~a}}1 {é}{{\'e}}1 {à}{{\`a}}1 {è}{{\`e}}1 {ê}{{\^e}}1 {ï}{{\"i}}1 {î}{{\^i}}1 {ù}{{\`u}}1,
}





%%
%% Julia definition (c) 2014 Jubobs
%%
\lstdefinelanguage{Julia}%
  {morekeywords={abstract,break,case,catch,const,continue,do,else,elseif,%
      begin,end,export,false,for,function,immutable,import,importall,if,in,%
      macro,module,otherwise,quote,return,switch,true,try,type,typealias,%
      using,while},%
   sensitive=true,%
   alsoother={$},%
   morecomment=[l]\#,%
   morecomment=[n]{\#=}{=\#},%
   morestring=[s]{"}{"},%
   morestring=[m]{'}{'},%
}[keywords,comments,strings]%

\newtcblisting{Julia}{enhanced,breakable,listing only,listing options={language=Julia},colback=white}
\newtcbinputlisting{\codeJulia}[1]{enhanced,breakable,listing only,listing options={language=Julia},listing file={#1},colback=white}
\newtcbinputlisting{\extraitcodeJulia}[3]{enhanced,breakable,listing only,listing options={language=Julia,firstline={#2},lastline={#3}},listing file={#1},colback=white}
%\lstnewenvironment{Julia}
%  {\lstset{%
%    language         = Julia,
%}}
%  {}


\lstnewenvironment{Scilab}
  {\lstset{language=Scilab,
}}
  {}
\lstnewenvironment{Bash}
  {\lstset{language=bash,
}}
  {}
 


\author{Olivier \textsc{Garet}}
%\date{mai 2013}
\title[Probabilistic proof for non--survival at criticality]{Probabilistic proof for non--survival at criticality: the Galton--Watson process and more}%{The critical Galton--Watson process vanishes}
\begin{document}
\subjclass[2000]{60K35, 82B43.}
\keywords{Galton--Watson process, growth model,renormalization}


\maketitle
\begin{abstract}
In a famous paper, Bezuidenhout and Grimmett demonstrated that the contact process dies out at the critical point.Their proof technique has often been used to study the growth of population patterns.
The present text is intended as an introduction to their ideas, with examples of minimal technicality. In particular, we recover the basic theorem about 
Galton--Watson chains: except in a degenerate case, survival is possible only if the fertility rate exceeds $1$. The classical proof that is taught in classrooms is essentially analytic, based on generating functions and convexity arguments. Following the Bezuidenhout--Grimmett way, we propose a proof that is more consistent with probabilistic intuition.
We also study the survival problem for an original model, mixing sexual and asexual reproduction.
\end{abstract}


\section{Introduction}
Inspired by an article by Grimmett and Marstrand on supercritical  percolation in dimension $d\ge 3$, Bezuidenhout and Grimmett have shown in a famous article that the contact process vanishes at the critical point.
Their proof technique has often been used to study various growth  models.

The implementation of their proof technique is usually quite technical, as it relies on a renormalization procedure with quite complicated events as a basic brick.

The purpose of this article is therefore to introduce this technique with growth models for which the implementation is much simpler.


Among the growth models, the most famous is the Galton--Watson process.
The basic theorem concerns the probability of survival as a function of fertility: except in degenerate cases, survival is possible only if the fertility rate exceeds $1$.
The proof that is usually taught  -- see for example  Benaïm--El Karoui~\cite{BEK} or Durrett~\cite{Durrett} --  is essentially analytic. It relies on generating functions and convexity arguments, which may seem rather frustrating or at least quite miraculous.

We propose here, inspired by the work of Bezuidenhout and Grimmett, to give a proof that is more in line  with the probabilistic intuition.

This gives an introduction to the ideas of Bezuidenhout and Grimmett, with a model that is probably the simplest of the models that can be considered.
We then continue with the study of the survival problem on an original model, mixing sexual and asexual reproduction.

%It also gives  an introduction to their ideas, in the frame of the simple example of the Galton--Watson process.


In order to keep our text self-contained (maybe event suitable for a presentation to graduate students), the first section is devoted to the introduction of the Galton--Watson process with all the necessary results.
The new proof of the classical result comes in Section~2.
Section~3 is devoted to the introduction and the study of a new cooperative model, mixing sexual and asexual reproduction.
\section{Galton--Watson processes: definition and first properties}


Let $\nu,\mu$ be two distributions on $\N$.
The distribution $\nu$ is denoted as the offspring distribution, whereas $\mu$ 
is the distribution of the size of the initial population. 

We denote as  the 
Galton--Watson process with initial distribution $\mu$ and offspring distribution $\nu$
the Markov chain that starts with  $\mu$ as initial distribution, and whose transition matrix is given by
\begin{equation*}
p_{i,j}=
\begin{cases}
\nu^{*i}(j)\text{ if }i\ne 0\\
\delta_0(j)\text{ if }i=0
\end{cases}
\end{equation*}

One can build such a chain as follows:
Let $(X_i^n)_{i,j\ge 1}$, $Y_0$ be independent random variables with $Y_0\sim\mu$ and $X_i^n\sim\nu$ for every $i,n$.
Then, the sequence $(Y_n)_{n\ge 1}$ is recursively defined by

$$\forall n\ge 0\quad Y_{n+1}=\sum_{1\le i\le Y_n}X_i^n.$$
Then, $(Y_n)_{n\ge 0}$, is a Galton--Watson process with initial distribution $\mu$ and offspring distribution $\nu$.
The mean number of offspring  $m=\int_{\N} x\ d\nu(x)$ is denoted as the fertility.
If we define  $\mathcal{F}_n=\sigma(X_i^k,i\ge 1,k\le n)$, we have %facilement
%$$\E[Y_{n+1}|\mathcal{F}_n]=mY_n,$$ 
%d'où 
\begin{equation}
\label{puissance}
\E[Y_{n+1}|\mathcal{F}_n]=mY_n,\quad \E[Y_{n+1}]=m\E[Y_n]\text{ and }\E[Y_n]=m^n\E[Y_0]
\end{equation}
We define the time to extinction $\tau$ as follows: $\tau=\inf\{n\ge 0; Y_n=0\}$.


\begin{theo}
\label{mort}
If $m<1$, $\P(\tau>n)=O(m^n)$. Particularly, $\P(\tau<+\infty)=1$.
\end{theo}
\begin{proof}
With~\eqref{puissance}, we have $\P(\tau>n)\le\P(Y_n\ge 1)\le \E[Y_n]=m^n\E[Y_0]$.
\end{proof}


\begin{theo}
\label{galtonindep}
Let $(X_n)_{n\ge 0}$ and  $(Y_n)_{n\ge 0}$ be independent  Galton--Watson processes with the same offspring distribution  $\nu$. Then, 
 $(X_n+Y_n)_{n\ge 0}$ is also a  Galton--Watson process with $\nu$ as offspring distribution.
\end{theo}
\begin{proof}
Since $(X_n)_{n\ge 0}$ and  $(Y_n)_{n\ge 0}$ are independent Markov chains, $((X_n,Y_n))_{n\ge 0}$ is a Markov chain, with the transition matrix
$$p_{(x,a),(y,b)}=\nu^{* x}(a)\nu^{* y}(b).$$ Let us denote by 
$\P^{(x,y)}$ the distributions of the canonically associated Markov chains.
We must prove that if the function $f$ is defined by  $f(x,y)=x+y$, then $(f(X_n,Y_n))_{n\ge 0}$ is still a Markov Chain. To this aim, we apply the Dynkin criterion: it is sufficient to prove that whenever $x+y=r$, then
$\P^{(x,y)}(f(X_1,Y_1)=\ell)$ only depends on $r$ and $\ell$.
Also, under  $\P^{(x,y)}$, $X_1$ and $Y_1$ are independent random variables with  $\nu^{* x}$ and $\nu^{* y}$ as their respective distributions, so the distribution of
 $f(X_1,Y_1)$ is  $\nu^{* x}* \nu^{* y}=\nu^{* (x+y)}=\nu^{* r}$.
Finally, $\P^{(x,y)}(f(X_1,Y_1)=\ell)=\nu^{* r}(\{\ell\})$ and 
$(X_n+Y_n)_{n\ge 0}$ is a Galton--Watson process with $\nu$ as offspring distribution. Since the initial distribution is $\P_{X_0+Y_0}=\P_{X_0}*\P_{Y_0}=\mu_1*\mu_2$, we get the desired result.
\end{proof}


In the sequel, $\P^i$ denotes a probability measure under which
$(Y_n)_{n\ge 0}$ is a Galton--Watson process with initial distribution $\delta_{i}$ and offspring distribution $\nu$.


\begin{coro}
We have
\begin{itemize}
\item For each $n\ge 0$, $\P^n(\tau<+\infty)=\P^{1}(\tau<+\infty)^n$
\item For $n,\ell\ge 0$, $\P^n(\tau<+\infty|\mathcal{F}_{\ell})=\P^{1}(\tau<+\infty)^{Y_{\ell}}$.
\item For $n,\ell\ge 1$, we have $\P^n(\tau=+\infty)>0 \iff \P^{\ell}(\tau=+\infty)>0.$
\end{itemize}
\end{coro}

\begin{proof}
  Thanks to Theorem~\ref{galtonindep}, we have $$\P^{n+1}(\tau<+\infty)=\P^{n}(\tau<+\infty)\P^{1}(\tau<+\infty),$$ then  $\P^{n}(\tau<+\infty)=\P^{1}(\tau<+\infty)^n$ follows by natural induction. This gives the first item.
Then, the second item follows from the Markov property. The last point is obvious.
\end{proof}

\begin{coro}
\label{souschaine}
Let $T\ge 1$.  $(Y_{Tn})_{n\ge 0}$ is a Galton--Watson process with offspring distribution $\P^1_{Y_T}$.
\end{coro}
\begin{proof}
Since $(Y_n)$ is a Markov chain, it is well known that so does $(Y_{Tn})_{n\ge 0}$. Let us compute the transition probabilities.

Let $k\ge 1$.
Applying Theorem~\ref{galtonindep} ($k-1$ times), we see that if the processes $(Y^1_t)_{t\ge 0}$, $(Y^2_t)_{t\ge 0}, $\dots$ (Y^k_t)_{t\ge 0}$ are independent Galton--Watson processes with $\delta_1$ as their common initial distribution and $\nu$ as offspring distribution,
then $(Y^1_t+\dots Y^k_t)_{t\ge 0}$ is a  Galton--Watson process with  $\delta_{k}$ as initial distribution and $\nu$ as  offspring distribution.
Then,
$$\P^k(Y_T=\ell)=\P(Y^1_T+\dots Y^k_T=\ell)=\P_{Y^1_T}^{*k}(\ell).$$
Also, $\P^0(Y_T=\ell)=\delta_0(\ell)$: this gives the desired result.
\end{proof}

\section{A probabilistic proof}

In the first step of the proof, we show that a certain growth process may survive, with the idea that the process that we finally want to study will be compared to the surviving reference process.
In the present paper, the reference process is a Galton--Watson process too.
However in general, the reference process may belong to a related family.
For example, Bezuidenhout and Grimmett compared the contact process to a supercritical oriented percolation process.


\subsection{Survival in the supercritical phase}

\begin{theo}
\label{surcritique}
If $m>1$, then $\P^{1}(\tau=+\infty)>0$.
\end{theo}
\begin{proof}
Let $a$ with $1<a<m$. We have
$$\miniop{}{\lim}{M\to +\infty} \int x\wedge M \ d\nu= \int  x \ d\nu=m,$$ so there exists $M$ with  $\int x\wedge M \ d\nu>a$.  For $k\ge n$, we have

\begin{align*}
\P^k(Y_1<na)&=\P(X_1+\dots X_k<na)\\\le &\P(X_1\wedge M+\dots X_n\wedge M<na)\\& =\P(n\E[X_1\wedge M]-(X_1\wedge M+\dots X_n\wedge M))> (\E[X_1\wedge M]-a)n)\\& \le\frac{\Var X_1\wedge M}{(\E[X_1\wedge M]-a)n},
\end{align*}
by the  Tchebitchef inequality. 
Let us define $\phi(k,x)=\P^k(Y_1<x)$ and consider $n>c=\frac{\Var (X_1\wedge M)}{\E[X_1\wedge M]-a}$.\\
Let $t\ge 0$. By the Markov property, for each
 $A\in \mathcal{F}_t$ with $A\subset\{Y_t\ge n\}$, we can write
\begin{align*}
\P(A\cap \{Y_{t+1}<an\})&=
\E[\1_A  \1_{Y_{t+1}<an\}}]=\E[\1_A\E[\1_{Y_{t+1}<an\}}|\mathcal{F}_t]]\\
&=\E[\1_A \phi(Y_t,an)]\le \E [\1_A c/n]=c/n\P(A),
\end{align*}
so $\P(Y_{t+1}\ge an| A)\ge 1-\frac{c}n$.\\
By natural induction, it follows that for $A_t=\miniop{t}{\cap}{i=1}\{Y_{t}\ge na^t\}$, we have
$$\P^n(A_t)\ge\miniop{t-1}{\prod}{i=0}\left(1-\frac{c}{na^i}\right),$$
then
$\P^n(\tau=+\infty)\ge\P^n(\forall t\ge 0\quad Y_{t}\ge na^t)\ge\prod_{i=0}^{+\infty} \left(1-\frac{c}{na^i}\right)>0.$
\end{proof}

Some remarks:
\begin{itemize}
\item Obviously, the bound $1-\frac{c}n$ is very bad, coming from the Tchebitchev inequality. We were doing better with the Höffding inequality, but that is sufficient for our purpose.
\item  The same pattern can be applied to demonstrate that survival is possible for a multitype Galton--Watson process whose fertility matrix has a spectral radius strictly greater than~1 (see for example~\cite{Garet-livre}).
\end{itemize}
\subsection{Survival is a local property}

\begin{theo}
\label{equivalence}
Let $(Y_n)_{n\ge 0}$ be a  Galton--Watson process with offspring distribution~$\nu$.
Suppose that $\nu(0)>0$. Then there is an equivalence between:
\begin{itemize}
\item $\exists N,T\ge 1\quad \P^N(Y_T\ge 2N)>\frac12$.
\item $\P^1(\tau=+\infty)>0$.
\end{itemize}
\end{theo}

The event $\{Y_T\ge 2N\}$ only depends on what happens in a finite time box. Thus, it can be considered to be a local event, which will be useful to get some continuity with respect to the parameters of the model. \\

Before starting the proof, let us give the main ideas:
\begin{itemize}
\item For the direct implication, the idea is to compare the chain with a supercritical  Galton--Watson process, then conclude with the help of Theorem~\ref{surcritique}.
\item The reverse implication is quite simple, because one essentially has to prove that the number of particles explodes as soon as the process survives.
  However, it must be kept in mind that if the local event is more complicated, this part will actually be the most difficult one.
\end{itemize}

\begin{lemme}
 If there exist $a>0$ and $N\ge 1$ such that $a\P^N(Y_1\ge aN)>1$, then $\P^1(\tau=+\infty)>0$.
\end{lemme}
\begin{proof}
Let $X_i^n$ be i.i.d. with $\nu$ as common distribution.
Let $M_0=1$, $Y_0=N$, and then 
$$\forall n\ge 0\quad Y_{n+1}=\sum_{1\le i\le Y_n}X_i^n\text{ and }M_{n+1}=\sum_{i=1}^{M_n} aB_i^n,$$
with $B_i^n=\1_{\{X^n_{(i-1)N+1}+\dots X^n_{iN}\ge aN\}}$.\\
We prove by natural induction that $Y_n\ge NM_n$ for each $n\ge 0$. Indeed, if $Y_n\ge NM_n$, it follows that
\begin{align*}
  Y_{n+1}=\sum_{1\le i\le Y_n}X_i^n\ge \sum_{1\le i\le NM_n}X_i^n&=\sum_{i=1}^{M_n}(X^n_{(i-1)N+1}+\dots X^n_{iN})\\&\ge \sum_{i=1}^{M_n} aNB_i^n=NM_{n+1}.
  \end{align*}
We note that $(M_n)$ is a Galton--Watson process, and its fertility is given by\\ $m=\E[aB_i^n]=a\P^N(Y_1\ge aN)>1$, then it may survive by Theorem~\ref{surcritique}.
Since  $Y_n\ge NM_n$, the process $(Y_n)$ may survive too.
\end{proof}
Note that the proof of the lemma relies on a coupling argument: we make live on the same space $(Y_n)_{n\ge 0}$ and a Galton--Watson process with offspring distribution $(1-q)\delta_0+q\delta_{a}$, where $q=\P^N(Y_1\ge aN)$.\\
This step can be seen as a static renormalization: with the  help of the local events $\{X^n_{(i-1)N+1}+\dots X^n_{iN}\ge aN\}$, we build a growth process involving Bernoulli variables, in such a way that
\begin{itemize}
\item The process using Bernoulli variables is known to be able to survive;
\item  The process using Bernoulli variables is dominated by the process that we study.
  \end{itemize}
  


\begin{proof}[Proof of Theorem~\ref{equivalence}]
By corollary~\ref{souschaine}, $(Y_{nT})_{n\ge 0}$ is a Galton--Watson process. So we can apply the Lemma with $a=2$: $(Y_{nT})_{n\ge 0}$ may survive, thus  $(Y_{n})_{n\ge 0}$ may survive also.


Conversely, let us suppose that $\nu(0)>0$, and $\P^1(\tau<+\infty)<1$.\\
Since $\P^N(\tau<+\infty)=\P^1(\tau<+\infty)^N$, there exists $N$ with $\P^N(\tau<+\infty)<1/2$.


We have noted that $\P^N(\tau<+\infty|\mathcal{F}_t)=\P^1(\tau<+\infty)^{Y_t}$.\\
Since $\P^1(\tau<+\infty)\ge \P^1(Y_1=0)=\nu(0)>0$, we can write
$$Y_t=\frac{\log \P^N(\tau<+\infty|\mathcal{F}_t)}{\log \P^1(\tau<+\infty)}.$$
Now, the Martingale convergence Theorem ensures that
$$\E^N[\1_{\{\tau<+\infty\}}|\mathcal{F}_t]=\P^N(\tau<+\infty|\mathcal{F}_t)\to \1_{\{\tau<+\infty\}}\quad\P^N\text{ a.s.}$$ when $t$ tends to infinity. \\
Particularly, on the event $\{\tau=+\infty\}$, 
$\P^N(\tau<+\infty|\mathcal{F}_t)$ almost surely tends to  $0$ and
$Y_t$ almost surely tends to infinity. Therefore, the following inequality holds $\P^N$-almost surely:
$$\1_{\{\tau=+\infty\}}\le \miniop{}{\liminf}{t\to +\infty}\1_{\{Y_t\ge 2N\}}.$$
With the Fatou Lemma, it follows that
$$\P^N(\tau=+\infty)=\E^N(\1_{\{\tau=+\infty\}})\le \miniop{}{\liminf}{t\to +\infty}\E^N[\1_{\{Y_t\ge 2N\}}]= \miniop{}{\liminf}{t\to +\infty}\P^N(Y_t\ge 2N).$$
Since $\P^N(\tau=+\infty)>1/2$, there exists $T$ such that $\P^N(Y_T\ge 2N)>1/2$.
\end{proof}

\subsection{The critical case}

\begin{theo}
  If $\nu(0)>0$ and $m=1$, then $\P^1(\tau=+\infty)=0$.
\end{theo}

\begin{proof}[First proof]
It is sufficient to note that for every $N,T\ge 1$, we have $$\P^N(Y_T\ge 2N)\le\frac{\E^N(Y_T)}{2N}=\frac{N}{2N}=\frac12,$$
then apply the converse part in Theorem~\ref{equivalence}.
\end{proof}
We now present another line of proof, somewhat longer, but also more robust.
It was used in Garet--Marchand~\cite{GM-BRW} and Gantert--Junk~\cite{Gantert} for the study of some branching random walks.

The first proof is not robust because it exploits the fact that we exactly know how to characterize the critical parameter for survival.
However, in many growth models, the critical parameter can not be given explicitly.
The idea is then: having shown that survival is characterized by the fact that a local event has a fairly high probability, we reason by contradiction and suppose that there is survival at the critical point for a certain parameter.
Then, with a slight modification of the local event, we can, by continuity, exhibit a model of the same family that is a little weaker, for which the local event still has a probability that is large enough to ensure survival, but which must nevertheless die because its parameter has become subcritical.

\begin{proof}[Second proof]
By contradiction, let us assume that we have $\nu(0)>0$, $m=1$ and also $\P^1(\tau=+\infty)>0$.


By Theorem~\ref{equivalence} (converse implication), one can choose $n$ and $T$ such that  $\P^N(Y_T\ge 2N)>\frac12$.

The idea is to provide a coupling with a subcritical process.
Let $(X_i^n)_{i,j\ge 1}$, $(B_i^n)_{i,j\ge 1}$ be independent variables with $X_i^n\sim \nu$,  and the $(B_i^n)_{i,j\ge 1}$'s are  Bernoulli with parameter $p$. Define $Y_0=N$, $Y^p_0=N$, then 

$$\forall n\ge 0\quad Y_{n+1}=\sum_{1\le i\le Y_n}X_i^n\text{ and }Y^p_{n+1}=\sum_{1\le i\le Y^p_n}B_i^n X_i^n.$$

By monotonicity,
$$\miniop{}{\lim}{M\to +\infty}\P^N( \max(Y_i,0\le i\le T)\le M,Y_T\ge 2N)=\P^N( Y_T\ge 2N)>1/2,$$
so there exists $M$ such that $\P( \max(Y_i,0\le i\le T)\le M,Y_T\ge 2N)>1/2$.
We have then
\begin{align*}
\P(Y^p_T\ge 2N)&\ge \P(Y_T\ge 2N,\forall i\le T\quad Y^p_i=Y_i)\\
& \ge \P\left( \begin{array}{l}\max(Y_i,0\le i\le T)\le M,Y_T\ge 2N,\\\forall (t,i)\in\{0,\dots, T-1\}\times\{1,\dots,M\} \quad B_i^t=1\end{array}\right)\\&=\P( \max(Y_i,0\le i\le T)\le M,Y_T\ge 2N)p^{TM}
\end{align*}
Taking $p<1$ large enough, we have $$\P( \max(Y_i,0\le i\le T)\le M,Y_T\ge 2N)p^{TM}>1/2,$$ so $\P(Y^p_T\ge 2N)>1/2$.
But $(Y^p_t)$ is a Galton--Watson process with offspring distribution $B_1^1X_1^1$ and initial distribution $\delta_N$, so by Theorem~\ref{equivalence} (direct implication), this Galton--Watson process may survive.
However $$\E[B_1^1X_1^1]=\E[B_1^1]\E[X_1^1]=pm=p<1,$$ so by Theorem~\ref{mort}, the process can not survive. This is a contradiction.

\end{proof}

%\input{multi-eng}


The Galton--Watson process has the peculiarity that the survival domain can be described explicitly.
This is obviously very convenient, but it may raise doubts about the generality of the proof technique we present.

In fact, this technique is more often applied to models where the critical value is not known, the most emblematic being the contact process or the directed percolation.
But in these models, the proof of the existence of a large probability for the local event in question is often quite technical, requiring many steps.

We will therefore present a simpler model, which allows us to show the power of the method in a model which is not exactly solvable, and seems to us to be rich enough to be worthy of interest. 



\section {Application to a cooperative model}

We describe a cooperative model with two species by a Markov chain $((X_n,Y_n)_{n\ge 0})$ with values in $\N^2$, given by the conditional laws

\begin{align*}
  \mathcal{L}((X_{n+1},Y_{n+1})|\mathcal{F}_n)&=\mu_{(X_n,Y_n)}\text{ with }\mu_{(x,y)}=\ber(2,q)^{*(x+y)}\otimes \ber(2,p)^{*\min(x,y)},
\end{align*}
where $p$ and $q$ belong to $(0,1)$ and $\mathcal{F}_n=\sigma((X_k,Y_k)_{0\le k\le n})$.

The first component of the pair is the number of elements of type 1, which are athe sexual offspring of representatives of both types; while the second component, the elements of type~2, arise from a meeting between elements of type~1 and elements of type~2.


We note that if at a given moment there are no more particles of type $1$ or no more particles of type $2$, the particles of type $2$ disappear without any possibility of reappearing.

On the other hand, if the type~$2$ particles disappear and some type~1 particles remain, the  process behaves like a Galton-Watson process of reproduction law $\ber(2,q)$:  survival is possible if and only if $q>\frac12$.

In the following, we will note $\P^{(x,y)}_{p,q}$ the law of the process starting from the state $(x,y)$ with parameters $p$ and $q$.


Recall the classical definitions for stochastic order: we say that $f:\R^E\to \R$ is non-decreasing if $f(x)\ge f(x')$ as soon as $x_i\ge x'_i$ for each $i\in E$ and also say that the measure $\mu$ on $\R^E$ is stochastically dominated by $\nu$
if $\int f\ d\mu\le \int f\ d\nu$ holds for any non-decreasing function $f$.

Thus, we can note that our model is super-additive: we have the stochastic inequality on the transition laws
\begin{align*}
  \mu_{(x,y)}* \mu_{(x',y')}&=\ber(2,q)^{*(x+x'+y+y')}\otimes \ber(2,p)^{*\min(x,y)+\min(x',y')}\\
  &\lestoch\ber(2,q)^{*(x+x'+y+y')}\otimes\ber(2,p)^{*\min(x+x',y+y')}= \mu_{(x+x',y+y')},
\end{align*}
where $\lestoch$ denotes the stochastic domination on $\R^2$.
This inequality is classically transported to the laws of processes: for each $p,q$ in $(0,1)$ and each collection of integers $x,x',y,y'$, we have 
\begin{align}
  \label{inegloi}
  \P_{p,q}^{(x,y)}*\P_{p,q}^{(x',y')}\lestoch \P_{p,q}^{(x+y',y+y')}.
\end{align}

In particular, if $x\ge x'$ and $y\ge y'$, then $\P_p^{(x,y)}\gestoch \P_p^{(x',y')}$.


The remarks made so far lead us to focus on the problem of the simultaneous survival of the two types.

We thus define
\begin{align*}
  Z_n=\inf(X_n,Y_n)\text{ and }\tau=\inf\{n\ge 0;Z_n=0\}.
\end{align*}  

\begin{comment}

$p_c=inf(X_n,Y_n)$ and $tau=inf(Z_n=0) $.

Then, we have

\begin{theo}
  \begin{itemize}
  \item $\frac12\le p_c<0,71$
  \item $\P^{(1,1)}_{p_c}(\tau=+\infty)=0$
  \end{itemize}
\end{theo}

Let us note $Z_n=min(X_n,Y_n)$, $Delta=(\{0\times})\cup(\N\times{0\})$ and $\tau'=inf\{n\ge 0; Z_n=0\}$.

It is easy to see that if $(x,y)\in\Delta$, then under $\P_p^{(x,y)}$,
$Y_n=0$ for all $n=1$, while $(X_n)$ is a Galton-Watson chain of reproduction law $\ber(2,p/2)$, which leads to $\tau<+infty$ almost surely.
Thus, with the strong Markov property, the events $\tau<+\infty}$ and 
$${$tau'<+\infty\}$ coincide to within one negligible, whatever the starting point.


In particular, if $\min(x,y)\ge N$, we have
\begin{align}
  \label{small}
  \P^{(x,y)}(\tau<+\infty)&\le \P^{(1,1)}(\tau<+\infty)^N
\end{align}
\end{comment}


We begin by deducing from~\eqref{inegloi} a lemma which will be very useful later:
\begin{lemme}\label{grandpas}
  Let $p\in [0,1]$, $x,y,k$ be natural numbers, $N$ and $T$ be non-zero natural numbers.
  Then
  \begin{align*} \P_{p,q}^{(x,y)}(Z_T\ge kN)&\ge \gamma^{*a}([k,+\infty[),
  \end{align*}
  where $a=\min(\floor{\frac{x}N},\floor{y}N)$ and $\gamma$ 
  is the law of $\floor{\frac{Z_{T}}{N}}$ under $P_{p,q}^{(N,N)}$.
\end{lemme}

\begin{proof}
 Let $(\tilde{X}_1,\tilde{Y_1}),\dots, (\tilde{X}_a,\tilde{Y_a})$ $a$ be independent random vectors following the law of $(X_T,Y_T)$ under $\P_{p,q}^{(N,N)}$.
 Posing $\tilde{Z}_k=\min(\tilde{X}_k,\tilde{Y}_k)$, we have
\begin{align*}
    \P_{p,q}^{(x,y)}(Z_T\ge kN)&\ge \P(\tilde{X}_1+\dots \tilde{X}_a\ge kN,\tilde{Y}_1+\dots \tilde{Y}_a\ge kN)\\
    &\ge \P(\tilde{Z}_1+\dots \tilde{Z}_a\ge kN)\\
    &\ge \P( \floor{\frac{\tilde{Z}_1}N}+\dots \floor{\frac{\tilde{Z}_a}N}\ge k) =\gamma^{*a}([k,+\infty)).
  \end{align*}
  Thus
  \begin{align*}
     \P_{p,q}^{(N,N)}(\floor{\frac{Z_{(n+1)T}}{N}}\ge k|\mathcal{F}_{nT})&= \P_{p,q}^{(X_{nT},Y_{nT})}(\floor{\frac{Z_{T}}{N}}\ge k|\mathcal{F}_{nT})\\ &\ge \gamma^{* \floor{\frac{Z_{nT}}{N}} }([k,+\infty))
  \end{align*}    
  
\end{proof}

We can now state the locality lemma, analogous to Theorem~\ref{equivalence}.

\begin{lemme}
  \label{leqvencore}
  Let $p\in (0,1)$. We have equivalence between
  \begin{itemize}
  \item $\P_{p,q}^{(1,1)}(\tau=+\infty)>0$.
  \item $\exists N\ge 1,T\ge 1\quad \E_{p,q}^{(N,N)}(\floor{\frac{Z_T}N})>1$.  
  \end{itemize}
\end{lemme}


\begin{proof}
  Suppose $p<1$ and $\P_{p,q}^{(1,1)}(\tau=+\infty)>0$.
From~\eqref{inegloi}, we deduce
that if $\min(x,y)\ge N$, we have
\begin{align}
  \label{sibeaucoup}
  \P^{(x,y)}(\tau<+\infty)&\le \P^{(1,1)}(\tau<+\infty)^N
\end{align}
  
  Then, thanks to~\eqref{sibeaucoup}, we can find $N$ such that $\P_{p,q}^{(N,N)}(\tau=+\infty)>\frac12$.
  We have
  \begin{align*}
    \P_{p,q}^{(N,N)}(\tau<+\infty|\mathcal{F}_n)&\ge \P_{p,q}^{(N,N)}(Y_{n+1}=0|\mathcal{F}_n)=(1-p)^{2Z_n}
  \end{align*}
  On the event $\{\tau=+\infty\}$, $\P_{p,q}^{(N,N)}(\tau<+\infty|\mathcal{F}_n)$ converges almost surely to $0$, so $Z_n$ $\P_{p,q}^{(N,N)}$-almost surely tends to infinity. Thus, $\1_{\{\tau=+\infty,Z_n<2N\}}$ almost surely to $0$, and by dominated convergence, $\P_{p,q}^{(N,N)}(\tau=+\infty,Z_n<2N)$ tends to $0$, which implies
  that $P_{p,q}^{(N,N)}(Z_T<2N)>\frac12$ for $T$ sufficiently large.
We deduce that 
\begin{align*}
    \E^{(N,N)}_{p,q}(\floor{\frac{Z_{T}}{N}})&\ge \E^{(N,N)}_{p,q}(2\1_{\{Z_T\ge 2N\}})=2\P^{(N,N)}_{p,q}(Z_T\ge 2N)>1.
\end{align*}

Conversely, suppose now that there exist $N$ and $T$ such that
$\E^{(N,N)}_{p,q}(\floor{\frac{Z_{T}}{N}})>1$.
%$\P_{p,q}(N,N)(Z_T\ge 2N)>\frac12$.

We will show that the process $(\floor{\frac{Z_{nT}}{N}})_{n\ge 0}$ stochastically dominates a supercritical Galton-Watson process.

Let $\gamma$ be the law of $\floor{\frac{Z_{T}}{N}}$ under $\P_{p,q}^{(N,N)}$.

Using Lemma~\ref{grandpas} and the Markov property, we have for any $k\ge 0$ and any natural number $n$:

  \begin{align*}
    \P_{p,q}^{(N,N)}(\floor{\frac{Z_{(n+1)T}}{N}}\ge k|\mathcal{F}_{nT})&= \P_{p,q}^{(X_{nT},Y_{nT})}(\floor{\frac{Z_{T}}{N}}\ge k|\mathcal{F}_{nT})\\ &\ge \gamma^{* \floor{\frac{Z_{nT}}{N}} }([k,+\infty)).
  \end{align*}  

  This shows that $(\floor{\frac{Z_{nT}}{N}})_{n\ge 0}$ stochastically dominates a Galton-Watson process with a $\gamma$ replication law, which is supercritical according to the condition on the expectation
 This results in the process surviving with strictly positive probability. 
\end{proof}  

We deduce the continuity theorem:
\begin{theo}
 The set 
 $$S=\{(p,q)\in(0,1)^2; \P^{(1,1)}_{p,q}(\tau=+\infty)>0\}.$$
 is an open subset of $(0,1)^2$.
\end{theo}

\begin{proof}
  Since $$S=\miniop{}{\cup}{N\ge 1,T\ge 1} \{(p,q)\in (0,1)^2; \E_{p,q}^{(N,N)}(\floor{\frac{Z_T}N})>1\},$$
  it is enough to prove that for $N,T\ge 1$, $(p,q)\mapsto \E_{p,q}^{(N,N)}(\floor{\frac{Z_T}N})$ is a continuous function.
  Since we have the recurrence formula:
  \begin{align*}
    \P^{(N,N)}_{p,q}\left(\begin{array}{c}X_{n+1}=k\\Y_{n+1}=\ell\end{array}\right)&=\miniop{}{\sum}{\substack{\min(i,j)\ge\ell \\ 2(i+j)\ge k}} \binom{2\min(i,j)}{\ell}\binom{2(i+j)}{k}\\ &\quad \quad\times q^k(1-q)^{2(i+j)-k}p^{\ell}(1-p)^{2\min(i,j)-\ell}\\ & \quad \quad\times \P^{(N,N)}_{p,q}(X_{n}=i,Y_{n}=j),
  \end{align*}
  the probabilities of the different values for $(X_n,Y_n)$ are described by a polynomial in $p$ and $q$, and so
   $(p,q)\mapsto \E_{p,q}^{(N,N)}(\floor{\frac{Z_T}N})$ is a polynomial function, obviously continuous.
  \end{proof}  
  
We can finally exhibit areas where we can prove that survival is possible or impossible.


First, we have
\begin{align*}
  \E_{p,q}[Z_{n+1}|\mathcal{F}_n]&\le \E_{p,q}[Y_{n+1}|\mathcal{F}_n]=2p Z_n,
\end{align*}
so $\E_{p,q}[Z_n]\le (2p)^n \E_{p,q}(Z_0)$ and with the Borel-Cantelli lemma,
$Z_n$ tends almost surely to $0$ for $p<1/2$.
%We deduce that $p_c\ge\frac12$.

To exhibit a survival domain, we apply Lemma~\ref{leqvencore} with $N=1$ and $T=1$. Then, we  have

\begin{align*}
  \E_{p,q}^{(1,1)}(Z_1)&=\int_{\R^2}\min(s,t) \ d\mu_{(1,1)}(s,t)
\end{align*}
In other words, $h(p,q)= \E_{p,q}^{(1,1)}(Z_1)$ is $\E(\min(U,V))$ where
$U$ and $V$ are independent random variables, respectively following
$\ber(4,q)$ and $\ber(p,2)$. A simple calculation gives

$$h(p,q)=4p^2q^4-12p^2q^3+12p^2q^2-4p^2q-2pq^4+8pq^3-12pq^2+8pq.$$

This allows to represent an area where the possibility of survival is demonstrated. 

\begin{figure}[ht]  
  \centering
  \begin{tabular}{cc}
    \includegraphics[scale=0.5]{survie_theor.png}% {survie_theo-4}%&\includegraphics[scale=0.3]{simul_modele}
   \end{tabular} 
  \caption{Set of points where $h(p,q)>1$.}
\end{figure}
%includegraphics[scale=0.5]{survival_theo}

We can compare with the result of the simulations:

\begin{figure}[ht]  
  \centering
  \begin{tabular}{cc}
    % \includegraphics[scale=0.4]{survie_theo}&
%    \includegraphics[scale=0.3]{survival_both_species}%{simul_modele-4}
    \includegraphics[scale=0.3]{survival_both_species_with_dot_and_legende}%{simul_modele-4}
   \end{tabular} 
  \caption{Estimation of $\P^{(1,1)}_{p,q}(\tau>\inf\{n\ge 0;\max(X_n,Y_n)>10^8\})$.}
  
\end{figure}


\section*{Appendix: source code in Julia}

\begin{Julia}
using AbstractAlgebra

function compute_proba(p,q)
 ex=0
 for a=0:1,b=0:1,c=0:1,d=0:1,e=0:1,f=0:1
     s=a+b
     t=c+d+e+f
     z=p^s*(1-p)^(2-s)*q^t*(1-q)^(4-t)
     m=min(s,t)
     ex+=m*z
 end
 return(ex)
end
 
A,(p,q)=PolynomialRing(ZZ,["p"; "q"])
chaine="h(p,q)="*repr(compute_proba(p,q))
println(chaine)
eval(Meta.parse(chaine))

# from now on
# h(p,q)=4*p^2*q^4-12*p^2*q^3+12*p^2*q^2-
#   4*p^2*q-2*p*q^4+8*p*q^3-12*p*q^2+8*p*q

using Plots
using Distributed
using Distributions
using DistributedArrays

@everywhere using Distributions

println(workers())

@everywhere function montecarlo(N,survie,p,q=p/2)
    s=0
    for i=1:N
    a=Integer(1)
    b=Integer(1)
    while (a>0) && (b>0) && (b<survie) && (a<survie)
     distA=Binomial(2*(a+b),q)
     distB=Binomial(2*min(a,b),p)
     a=rand(distA,1)[1]
     b=rand(distB,1)[1]
    end
    s+=(a>0) && (b>0)
end
return(s/N)
end


pas=0.00125
NMC=1000
interv=0:pas:1
survie=@DArray [montecarlo(NMC,10^8,i,j) for i=interv,
  j=interv]
survie_simul=convert(Array,survie)
heatmap(interv,interv,survie_simul,ratio=1,xlabel="q",
ylabel="p",c=reverse(cgrad(:default)),size=(1200,800))
savefig("survival_both_species_without_dot.png")

function q_critique(pp)    
    qmin=0
    qmax=1
    milieu=0.5
    while ((qmax-qmin)>10^(-12))
        milieu=(qmin+qmax)/2
        if (h(pp,milieu)<1)
            qmin=milieu
        else
            qmax=milieu
        end
    end
    return(milieu)
end

y=0.5:0.01:1
plot!(q_critique.(y),y,linewidth=2,linestyle=:dash,
color=:green,label="h(p,q)=1")
savefig("survival_2_species_with_dot_and_legend.png")
\end{Julia}
%\nocite{*} 

\def\refname{References}
\bibliographystyle{plain}
\begin{thebibliography}{1}

\bibitem{BEK}
Michel Benaïm and Nicole El~Karoui.
\newblock {\em Promenade aléatoire: Chaines de Markov et simulations;
  martingales et stratégies}.
\newblock Ecole Polytechnique, 2004.

\bibitem{MR1071804}
Carol Bezuidenhout and Geoffrey Grimmett.
\newblock The critical contact process dies out.
\newblock {\em Ann. Probab.}, 18(4):1462--1482, 1990.

\bibitem{Durrett}
  Rick Durrett
  \newblock {\em Probability: theory and examples}.
  \newblock Cambridge Series in Statistical and Probabilistic Mathematics,
  \newblock Cambridge University Press, 2010.

\bibitem{Gantert}
  Gantert, Nina and Junk, Stefan
  \newblock A branching random walk among disasters.
  \newblock {\em Electron. J. Probab.}, Vol 22, 2017.
  
\bibitem{Garet-livre}
  Olivier Garet.
  \newblock Probabilités et Processus Stochastiques.
  \newblock distributed by Amazon, 2017, 508 p.
  
\bibitem{GM-BRW}
Olivier Garet and R\'egine Marchand.
\newblock The critical branching random walk in a random environment dies out.
\newblock {\em Electron. Comm. Probab.}, 18(9):1--15 (electronic), 2013.

\bibitem{Grimmett-Marstrand}
G.~R. Grimmett and J.~M. Marstrand.
\newblock The supercritical phase of percolation is well behaved.
\newblock {\em Proc. Roy. Soc. London Ser. A}, 430(1879):439--457, 1990.


\end{thebibliography}



\end{document}
