The Galton--Watson process has the peculiarity that the survival domain can be described explicitly.
This is obviously very convenient, but it may raise doubts about the generality of the proof technique we present.

In fact, this technique is more often applied to models where the critical value is not known, the most emblematic being the contact process or the directed percolation.
But in these models, the proof of the existence of a large probability for the local event in question is often quite technical, requiring many steps.

We will therefore present a simpler model, which allows us to show the power of the method in a model which is not exactly solvable, and seems to us to be rich enough to be worthy of interest. 



\section {Application to a cooperative model}

We describe a cooperative model with two species by a Markov chain $((X_n,Y_n)_{n\ge 0})$ with values in $\N^2$, given by the conditional laws

\begin{align*}
  \mathcal{L}((X_{n+1},Y_{n+1})|\mathcal{F}_n)&=\mu_{(X_n,Y_n)}\text{ with }\mu_{(x,y)}=\ber(2,q)^{*(x+y)}\otimes \ber(2,p)^{*\min(x,y)},
\end{align*}
where $p$ and $q$ belong to $(0,1)$ and $\mathcal{F}_n=\sigma((X_k,Y_k)_{0\le k\le n})$.

The first component of the pair is the number of elements of type 1, which are athe sexual offspring of representatives of both types; while the second component, the elements of type~2, arise from a meeting between elements of type~1 and elements of type~2.


We note that if at a given moment there are no more particles of type $1$ or no more particles of type $2$, the particles of type $2$ disappear without any possibility of reappearing.

On the other hand, if the type~$2$ particles disappear and some type~1 particles remain, the  process behaves like a Galton-Watson process of reproduction law $\ber(2,q)$:  survival is possible if and only if $q>\frac12$.

In the following, we will note $\P^{(x,y)}_{p,q}$ the law of the process starting from the state $(x,y)$ with parameters $p$ and $q$.


Recall the classical definitions for stochastic order: we say that $f:\R^E\to \R$ is non-decreasing if $f(x)\ge f(x')$ as soon as $x_i\ge x'_i$ for each $i\in E$ and also say that the measure $\mu$ on $\R^E$ is stochastically dominated by $\nu$
if $\int f\ d\mu\le \int f\ d\nu$ holds for any non-decreasing function $f$.

Thus, we can note that our model is super-additive: we have the stochastic inequality on the transition laws
\begin{align*}
  \mu_{(x,y)}* \mu_{(x',y')}&=\ber(2,q)^{*(x+x'+y+y')}\otimes \ber(2,p)^{*\min(x,y)+\min(x',y')}\\
  &\lestoch\ber(2,q)^{*(x+x'+y+y')}\otimes\ber(2,p)^{*\min(x+x',y+y')}= \mu_{(x+x',y+y')},
\end{align*}
where $\lestoch$ denotes the stochastic domination on $\R^2$.
This inequality is classically transported to the laws of processes: for each $p,q$ in $(0,1)$ and each collection of integers $x,x',y,y'$, we have 
\begin{align}
  \label{inegloi}
  \P_{p,q}^{(x,y)}*\P_{p,q}^{(x',y')}\lestoch \P_{p,q}^{(x+y',y+y')}.
\end{align}

In particular, if $x\ge x'$ and $y\ge y'$, then $\P_p^{(x,y)}\gestoch \P_p^{(x',y')}$.


The remarks made so far lead us to focus on the problem of the simultaneous survival of the two types.

We thus define
\begin{align*}
  Z_n=\inf(X_n,Y_n)\text{ and }\tau=\inf\{n\ge 0;Z_n=0\}.
\end{align*}  

\begin{comment}

$p_c=inf(X_n,Y_n)$ and $tau=inf(Z_n=0) $.

Then, we have

\begin{theo}
  \begin{itemize}
  \item $\frac12\le p_c<0,71$
  \item $\P^{(1,1)}_{p_c}(\tau=+\infty)=0$
  \end{itemize}
\end{theo}

Let us note $Z_n=min(X_n,Y_n)$, $Delta=(\{0\times})\cup(\N\times{0\})$ and $\tau'=inf\{n\ge 0; Z_n=0\}$.

It is easy to see that if $(x,y)\in\Delta$, then under $\P_p^{(x,y)}$,
$Y_n=0$ for all $n=1$, while $(X_n)$ is a Galton-Watson chain of reproduction law $\ber(2,p/2)$, which leads to $\tau<+infty$ almost surely.
Thus, with the strong Markov property, the events $\tau<+\infty}$ and 
$${$tau'<+\infty\}$ coincide to within one negligible, whatever the starting point.


In particular, if $\min(x,y)\ge N$, we have
\begin{align}
  \label{small}
  \P^{(x,y)}(\tau<+\infty)&\le \P^{(1,1)}(\tau<+\infty)^N
\end{align}
\end{comment}


We begin by deducing from~\eqref{inegloi} a lemma which will be very useful later:
\begin{lemme}\label{grandpas}
  Let $p\in [0,1]$, $x,y,k$ be natural numbers, $N$ and $T$ be non-zero natural numbers.
  Then
  \begin{align*} \P_{p,q}^{(x,y)}(Z_T\ge kN)&\ge \gamma^{*a}([k,+\infty[),
  \end{align*}
  where $a=\min(\floor{\frac{x}N},\floor{y}N)$ and $\gamma$ 
  is the law of $\floor{\frac{Z_{T}}{N}}$ under $P_{p,q}^{(N,N)}$.
\end{lemme}

\begin{proof}
 Let $(\tilde{X}_1,\tilde{Y_1}),\dots, (\tilde{X}_a,\tilde{Y_a})$ $a$ be independent random vectors following the law of $(X_T,Y_T)$ under $\P_{p,q}^{(N,N)}$.
 Posing $\tilde{Z}_k=\min(\tilde{X}_k,\tilde{Y}_k)$, we have
\begin{align*}
    \P_{p,q}^{(x,y)}(Z_T\ge kN)&\ge \P(\tilde{X}_1+\dots \tilde{X}_a\ge kN,\tilde{Y}_1+\dots \tilde{Y}_a\ge kN)\\
    &\ge \P(\tilde{Z}_1+\dots \tilde{Z}_a\ge kN)\\
    &\ge \P( \floor{\frac{\tilde{Z}_1}N}+\dots \floor{\frac{\tilde{Z}_a}N}\ge k) =\gamma^{*a}([k,+\infty)).
  \end{align*}
  Thus
  \begin{align*}
     \P_{p,q}^{(N,N)}(\floor{\frac{Z_{(n+1)T}}{N}}\ge k|\mathcal{F}_{nT})&= \P_{p,q}^{(X_{nT},Y_{nT})}(\floor{\frac{Z_{T}}{N}}\ge k|\mathcal{F}_{nT})\\ &\ge \gamma^{* \floor{\frac{Z_{nT}}{N}} }([k,+\infty))
  \end{align*}    
  
\end{proof}

We can now state the locality lemma, analogous to Theorem~\ref{equivalence}.

\begin{lemme}
  \label{leqvencore}
  Let $p\in (0,1)$. We have equivalence between
  \begin{itemize}
  \item $\P_{p,q}^{(1,1)}(\tau=+\infty)>0$.
  \item $\exists N\ge 1,T\ge 1\quad \E_{p,q}^{(N,N)}(\floor{\frac{Z_T}N})>1$.  
  \end{itemize}
\end{lemme}


\begin{proof}
  Suppose $p<1$ and $\P_{p,q}^{(1,1)}(\tau=+\infty)>0$.
From~\eqref{inegloi}, we deduce
that if $\min(x,y)\ge N$, we have
\begin{align}
  \label{sibeaucoup}
  \P^{(x,y)}(\tau<+\infty)&\le \P^{(1,1)}(\tau<+\infty)^N
\end{align}
  
  Then, thanks to~\eqref{sibeaucoup}, we can find $N$ such that $\P_{p,q}^{(N,N)}(\tau=+\infty)>\frac12$.
  We have
  \begin{align*}
    \P_{p,q}^{(N,N)}(\tau<+\infty|\mathcal{F}_n)&\ge \P_{p,q}^{(N,N)}(Y_{n+1}=0|\mathcal{F}_n)=(1-p)^{2Z_n}
  \end{align*}
  On the event $\{\tau=+\infty\}$, $\P_{p,q}^{(N,N)}(\tau<+\infty|\mathcal{F}_n)$ converges almost surely to $0$, so $Z_n$ $\P_{p,q}^{(N,N)}$-almost surely tends to infinity. Thus, $\1_{\{\tau=+\infty,Z_n<2N\}}$ almost surely to $0$, and by dominated convergence, $\P_{p,q}^{(N,N)}(\tau=+\infty,Z_n<2N)$ tends to $0$, which implies
  that $P_{p,q}^{(N,N)}(Z_T<2N)>\frac12$ for $T$ sufficiently large.
We deduce that 
\begin{align*}
    \E^{(N,N)}_{p,q}(\floor{\frac{Z_{T}}{N}})&\ge \E^{(N,N)}_{p,q}(2\1_{\{Z_T\ge 2N\}})=2\P^{(N,N)}_{p,q}(Z_T\ge 2N)>1.
\end{align*}

Conversely, suppose now that there exist $N$ and $T$ such that
$\E^{(N,N)}_{p,q}(\floor{\frac{Z_{T}}{N}})>1$.
%$\P_{p,q}(N,N)(Z_T\ge 2N)>\frac12$.

We will show that the process $(\floor{\frac{Z_{nT}}{N}})_{n\ge 0}$ stochastically dominates a supercritical Galton-Watson process.

Let $\gamma$ be the law of $\floor{\frac{Z_{T}}{N}}$ under $\P_{p,q}^{(N,N)}$.

Using Lemma~\ref{grandpas} and the Markov property, we have for any $k\ge 0$ and any natural number $n$:

  \begin{align*}
    \P_{p,q}^{(N,N)}(\floor{\frac{Z_{(n+1)T}}{N}}\ge k|\mathcal{F}_{nT})&= \P_{p,q}^{(X_{nT},Y_{nT})}(\floor{\frac{Z_{T}}{N}}\ge k|\mathcal{F}_{nT})\\ &\ge \gamma^{* \floor{\frac{Z_{nT}}{N}} }([k,+\infty)).
  \end{align*}  

  This shows that $(\floor{\frac{Z_{nT}}{N}})_{n\ge 0}$ stochastically dominates a Galton-Watson process with a $\gamma$ replication law, which is supercritical according to the condition on the expectation
 This results in the process surviving with strictly positive probability. 
\end{proof}  

We deduce the continuity theorem:
\begin{theo}
 The set 
 $$S=\{(p,q)\in(0,1)^2; \P^{(1,1)}_{p,q}(\tau=+\infty)>0\}.$$
 is an open subset of $(0,1)^2$.
\end{theo}

\begin{proof}
  Since $$S=\miniop{}{\cup}{N\ge 1,T\ge 1} \{(p,q)\in (0,1)^2; \E_{p,q}^{(N,N)}(\floor{\frac{Z_T}N})>1\},$$
  it is enough to prove that for $N,T\ge 1$, $(p,q)\mapsto \E_{p,q}^{(N,N)}(\floor{\frac{Z_T}N})$ is a continuous function.
  Since we have the recurrence formula:
  \begin{align*}
    \P^{(N,N)}_{p,q}\left(\begin{array}{c}X_{n+1}=k\\Y_{n+1}=\ell\end{array}\right)&=\miniop{}{\sum}{\substack{\min(i,j)\ge\ell \\ 2(i+j)\ge k}} \binom{2\min(i,j)}{\ell}\binom{2(i+j)}{k}\\ &\quad \quad\times q^k(1-q)^{2(i+j)-k}p^{\ell}(1-p)^{2\min(i,j)-\ell}\\ & \quad \quad\times \P^{(N,N)}_{p,q}(X_{n}=i,Y_{n}=j),
  \end{align*}
  the probabilities of the different values for $(X_n,Y_n)$ are described by a polynomial in $p$ and $q$, and so
   $(p,q)\mapsto \E_{p,q}^{(N,N)}(\floor{\frac{Z_T}N})$ is a polynomial function, obviously continuous.
  \end{proof}  
  
We can finally exhibit areas where we can prove that survival is possible or impossible.


First, we have
\begin{align*}
  \E_{p,q}[Z_{n+1}|\mathcal{F}_n]&\le \E_{p,q}[Y_{n+1}|\mathcal{F}_n]=2p Z_n,
\end{align*}
so $\E_{p,q}[Z_n]\le (2p)^n \E_{p,q}(Z_0)$ and with the Borel-Cantelli lemma,
$Z_n$ tends almost surely to $0$ for $p<1/2$.
%We deduce that $p_c\ge\frac12$.

To exhibit a survival domain, we apply Lemma~\ref{leqvencore} with $N=1$ and $T=1$. Then, we  have

\begin{align*}
  \E_{p,q}^{(1,1)}(Z_1)&=\int_{\R^2}\min(s,t) \ d\mu_{(1,1)}(s,t)
\end{align*}
In other words, $h(p,q)= \E_{p,q}^{(1,1)}(Z_1)$ is $\E(\min(U,V))$ where
$U$ and $V$ are independent random variables, respectively following
$\ber(4,q)$ and $\ber(p,2)$. A simple calculation gives

$$h(p,q)=4p^2q^4-12p^2q^3+12p^2q^2-4p^2q-2pq^4+8pq^3-12pq^2+8pq.$$

This allows to represent an area where the possibility of survival is demonstrated. 

\begin{figure}[ht]  
  \centering
  \begin{tabular}{cc}
    \includegraphics[scale=0.5]{survie_theor.png}% {survie_theo-4}%&\includegraphics[scale=0.3]{simul_modele}
   \end{tabular} 
  \caption{Set of points where $h(p,q)>1$.}
\end{figure}
%includegraphics[scale=0.5]{survival_theo}

We can compare with the result of the simulations:

\begin{figure}[ht]  
  \centering
  \begin{tabular}{cc}
    % \includegraphics[scale=0.4]{survie_theo}&
%    \includegraphics[scale=0.3]{survival_both_species}%{simul_modele-4}
    \includegraphics[scale=0.3]{survival_both_species_with_dot_and_legende}%{simul_modele-4}
   \end{tabular} 
  \caption{Estimation of $\P^{(1,1)}_{p,q}(\tau>\inf\{n\ge 0;\max(X_n,Y_n)>10^8\})$.}
  
\end{figure}
