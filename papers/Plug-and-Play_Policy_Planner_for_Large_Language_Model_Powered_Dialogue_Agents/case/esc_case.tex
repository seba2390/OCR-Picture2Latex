\begin{table}[]
    \centering
    \begin{tabular}{p{13.5cm}}
    \toprule
    \multicolumn{1}{c}{\textsc{\textbf{Background Information}}}\\
    \midrule
    \textbf{Emotion Type}: Fear \\
    \textbf{Problem Type}: Job Crisis\\
    \textbf{Situation}: I think I will be losing my job soon. I just read an email taking about the need for us to cut cost and also how we have not got any support from the government. \\
    \midrule
    \midrule
    \multicolumn{1}{c}{\textsc{\textbf{Standard}}}\\
    \midrule
    Patient: I think I will be losing my job soon. I just read an email taking about the need for us to cut cost and also how we have not got any support from the government.\\
    Therapist: It sounds like you're feeling anxious and uncertain about your job security due to the current financial situation.\\
    Patient: Yes, I am feeling extremely anxious and uncertain about my job and financial stability.\\
    Therapist: I understand that this situation is causing you a lot of anxiety and uncertainty.\\
    Patient: Yes, it's been keeping me up at night and I can't stop worrying about what will happen if I lose my job.\\
    Therapist: It's understandable that the fear of losing your job is causing you significant distress and affecting your sleep.\\
    Patient: Yes, it's consuming my thoughts and I feel overwhelmed by the fear of not being able to support myself and my family.\\
    Therapist: The fear of not being able to provide for yourself and your family is a heavy burden to carry, and it's important to address these concerns.\\
    Patient: I know, but I just don't know what to do. I feel stuck and helpless in this situation.\\
    Therapist: Feeling stuck and helpless is a common response to uncertainty, but there are steps we can explore together to help you regain a sense of control and explore potential options.\\
    Patient: I'm willing to try anything at this point. I just want to find a way to alleviate this fear and find some stability in my life.\\
    Therapist: I'm glad to hear that you're open to exploring different strategies. Let's start by identifying your strengths and resources that can help you navigate this challenging situation.\\
    Patient: I'm not sure what strengths and resources I have, but I'm willing to explore and discover them.\\
    Therapist: That's a great mindset to have. Together, we can explore and uncover your strengths and resources that will support you in finding stability and managing your fear.\\
    Patient: Thank you, I appreciate your support. I'm ready to begin this journey of self-discovery and finding ways to overcome my fear of job crisis. \\
    \midrule
    \textbf{Turns of Conversations}: 7 \\
    \bottomrule
    \end{tabular}
    \caption{Example conversations with Standard prompts in emotional support dialogues. }
    \label{tab:esc_case_1}
\end{table}



\begin{table}[]
    \centering
    \begin{tabular}{p{13.5cm}}
    \toprule
    \multicolumn{1}{c}{\textsc{\textbf{Background Information}}}\\
    \midrule
    \multicolumn{1}{c}{Same as Table \ref{tab:esc_case_1}}\\
    \midrule
    \midrule
    \multicolumn{1}{c}{\textsc{\textbf{Ask-an-Expert}}}\\
    \midrule
    Patient: I think I will be losing my job soon. I just read an email taking about the need for us to cut cost and also how we have not got any support from the government.\\
    \texttt{[action] The therapist should empathize with the patient's concerns.}\\
    Therapist: I understand that receiving that email has caused you a lot of worry and uncertainty about your job security.\\
    Patient: Yes, it has. I feel anxious and scared about what will happen to me if I lose my job.\\
    \texttt{[action] The therapist should explore the patient's feelings of anxiety and fear related to potential job loss.}\\
    Therapist: It's completely normal to feel anxious and scared in this situation. Remember that you have faced challenges before and have the strength to navigate through this one as well.\\
    Patient: I know, but I can't help but constantly worry about the future and what it holds for me.\\
    \texttt{[action] The therapist should provide reassurance and support.}\\
    Therapist: It's understandable that you're worried about the future, but try to focus on the present moment and take things one step at a time.\\
    Patient: I'll try, but it's hard not to let my fear of job loss consume my thoughts and affect my daily life.\\
    \texttt{[action] The therapist should explore the patient's fears.}\\
    Therapist: I hear you, and it's important to acknowledge those fears.\\
    Patient: Thank you for understanding. I just need some guidance on how to manage this fear and find some peace of mind. \\
    \texttt{[action] }\\
    Therapist: One strategy you can try is practicing mindfulness, which involves focusing on the present moment and accepting your thoughts and emotions without judgment.\\
    Patient: That sounds helpful. How can I start practicing mindfulness?\\
    \texttt{[action] The therapist should provide guidance on managing the fear and finding peace of mind.}\\
    Therapist: You can start by setting aside a few minutes each day to sit quietly, focus on your breath, and observe your thoughts and sensations without getting caught up in them.\\
    Patient: I will give it a try. Thank you for suggesting mindfulness as a way to manage my fear and find some peace.\\
    \texttt{[action] The therapist should support and encourage the patient in their mindfulness practice for managing fear and finding peace.}\\
    Therapist: You're welcome! Remember, it takes time and practice, so be patient with yourself. I'm here to support you along the way.\\
    Patient: Thank you, I appreciate your support. It means a lot to me.\\
    \midrule
    \textbf{Turns of Conversations}: 7\\
    \bottomrule
    \end{tabular}
    \caption{Example conversations with Ask-an-Expert in emotional support dialogues. }
    \label{tab:esc_case_2}
\end{table}



\begin{table}[]
    \centering
    \begin{tabular}{p{13.5cm}}
    \toprule
    \multicolumn{1}{c}{\textsc{\textbf{Background Information}}}\\
    \midrule
    \multicolumn{1}{c}{Same as Table \ref{tab:esc_case_1}}\\
    \midrule
    \midrule
    \multicolumn{1}{c}{\textsc{\textbf{ProCoT}}}\\
    \midrule
    Patient: I think I will be losing my job soon. I just read an email taking about the need for us to cut cost and also how we have not got any support from the government.\\
    \texttt{[action] The Therapist provides affirmation and reassurance to the Patient on the situation they just described.} \\
    Therapist: It sounds like you're feeling anxious and uncertain about your job security due to the current financial situation.\\
    Patient: Yes, I am feeling extremely anxious and uncertain about my job and financial stability.\\
    \texttt{[action] The Therapist provides a statement relating to the Patient about the situation they just described.} \\
    Therapist: It's completely understandable to feel anxious and uncertain in this situation, but remember that you have the ability to adapt and find new opportunities.\\
    Patient: I know, but the fear of not being able to support myself and my family is overwhelming.\\
    \texttt{[action] The Therapist provides a statement relating to the Patient about the situation they just described.} \\
    Therapist: I can understand how overwhelming that fear can be, but it's important to remember that you have faced challenges before and have the strength to navigate through this one as.\\
    Patient: I'm trying to remind myself of that, but the fear keeps consuming me and I can't seem to shake it off.\\
    \texttt{[action] The Therapist provides suggestions to the Patient on the situation they just described.} \\
    Therapist: It's normal for fear to consume us at times, but we can work together to develop strategies to manage and reduce that fear, allowing you to regain a sense.\\
    Patient: I would really appreciate your help in developing those strategies to manage my fear and regain a sense of control.\\
    \midrule
    \textbf{Turns of Conversations}: 4 \\
    \bottomrule
    \end{tabular}
    \caption{Example conversations with PPDPP in emotional support dialogues. }
    \label{tab:esc_case_3}
\end{table}



\begin{table}[]
    \centering
    \begin{tabular}{p{13.5cm}}
    \toprule
    \multicolumn{1}{c}{\textsc{\textbf{Background Information}}}\\
    \midrule
    \multicolumn{1}{c}{Same as Table \ref{tab:esc_case_1}}\\
    \midrule
    \midrule
    \multicolumn{1}{c}{\textsc{\textbf{ICL-AIF}}}\\
    \midrule
    \texttt{[action] 1. Validate the patient's emotions and acknowledge the significance of their distress.}\\
    \texttt{2. Encourage the patient to explore and express their fears and concerns openly.}\\
    \texttt{3. Offer to help the patient develop healthy coping strategies to manage their fear and anxiety.}\\
    Patient: I think I will be losing my job soon. I just read an email taking about the need for us to cut cost and also how we have not got any support from the government.\\
    Therapist: It sounds like you're feeling worried and uncertain about your job security.\\
    Patient: Yes, I am extremely worried about what will happen if I lose my job. I have bills to pay and a family to support.\\
    Therapist: I can understand why you would feel so concerned about the potential impact losing your job could have on your financial stability and your ability to provide for your family.\\
    Patient: Yes, it's a constant source of anxiety for me. I feel like I'm constantly on edge, waiting for the other shoe to drop.\\
    Therapist: It's completely understandable that you would feel on edge and anxious given the uncertainty surrounding your job situation. It's important to acknowledge and address these feelings.\\
    Patient: I just don't know how to cope with this fear and uncertainty. It's consuming my thoughts and affecting my overall well-being.\\
    Therapist: I hear you, and it's important to find healthy coping strategies to manage your fear and uncertainty. I'm here to support you in developing those strategies.\\
    Patient: Thank you, I appreciate your support. I just need some guidance on how to navigate through this difficult time and find some peace of mind.\\
    Therapist: I'm here to help you navigate through this difficult time. Together, we can explore strategies to find peace of mind and manage the challenges you're facing.\\
    Patient: That means a lot to me. I'm ready to work on finding peace of mind and managing my fears about the job crisis.\\
    \midrule
    \textbf{Turns of Conversations}: 5 \\
    \midrule
    \midrule
    \multicolumn{1}{c}{\textsc{\textbf{PPDPP}}}\\
    \midrule
    Patient: I think I will be losing my job soon. I just read an email taking about the need for us to cut cost and also how we have not got any support from the government.\\
    \texttt{[action] The Therapist acknowledges the Patient’s feelings about the situation they described.} \\
    Therapist: I understand that receiving that email about potential job loss is causing you a lot of distress.\\
    Patient: Yes, it's been causing me a lot of anxiety and fear about my future.\\
    \texttt{[action] The Therapist provides affirmation and reassurance to the Patient on the situation they just described.} \\
    Therapist: It's completely normal to feel anxious and fearful in this situation, but remember that you have the strength and resilience to navigate through it.\\
    Patient: Thank you, I appreciate your reassurance.\\
    \texttt{[action] The Therapist provides suggestions to the Patient on the situation they just described.} \\
    Therapist: You're welcome. Remember, you are not alone in this, and we will work together to find ways to cope and navigate through this challenging time.\\
    Patient: That's comforting to hear. I'm ready to work on finding ways to cope with my fear and uncertainty about the job crisis.\\
    \midrule
    \textbf{Turns of Conversations}: 3 \\
    \bottomrule
    \end{tabular}
    \caption{Example conversations with ICL-AIF and PPDPP methods in emotional support dialogues. }
    \label{tab:esc_case_4}
\end{table}