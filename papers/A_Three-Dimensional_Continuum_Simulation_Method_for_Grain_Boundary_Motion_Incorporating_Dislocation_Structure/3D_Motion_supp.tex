%\documentclass[preprint,review,12pt,3p]{elsarticle}
\documentclass[smallextended]{svjour3}

\usepackage{lineno,hyperref}
%\modulolinenumbers[5]
\usepackage{amssymb,amsmath}
%\usepackage{amsthm}
\usepackage{bm}
\usepackage{lipsum}
\usepackage{amsfonts}
\usepackage{graphicx,subfigure}
\usepackage{epstopdf}
\usepackage{algorithmic}
\usepackage{color}
\usepackage{tikz}
%\usepackage{comment}
%\journal{Journal of computatianl physics}

%%%%%%%%%%%%%%%%%%%%%%%
%% Elsevier bibliography styles
%%%%%%%%%%%%%%%%%%%%%%%
%% To change the style, put a % in front of the second line of the current style and
%% remove the % from the second line of the style you would like to use.
%%%%%%%%%%%%%%%%%%%%%%%

%% Numbered
%\bibliographystyle{model1-num-names}

%% Numbered without titles
%\bibliographystyle{model1a-num-names}

%% Harvard
%\bibliographystyle{model2-names.bst}\biboptions{authoryear}

%% Vancouver numbered
%\usepackage{numcompress}\bibliographystyle{model3-num-names}

%% Vancouver name/year
%\usepackage{numcompress}\bibliographystyle{model4-names}\biboptions{authoryear}

%% APA style
%\bibliographystyle{model5-names}\biboptions{authoryear}

%% AMA style
%\usepackage{numcompress}\bibliographystyle{model6-num-names}

%% `Elsevier LaTeX' style
\bibliographystyle{spmpsci}
%%%%%%%%%%%%%%%%%%%%%%%

\begin{document}

%\begin{frontmatter}

\title{A Three-Dimensional Continuum Simulation Method for Grain Boundary Motion Incorporating Dislocation Structure\\
{\bf Supplementary Materials}}%\tnoteref{mytitlenote}}
%\tnotetext[mytitlenote]{Fully documented templates are available in the elsarticle package on %\href{http://www.ctan.org/tex-archive/macros/latex/contrib/elsarticle}{CTAN}.}

%%% Group authors per affiliation:
%\author{Xiaoxue Qin}
%\author{Luchan Zhang\corref{mycorrespondingauthor}}
%\ead{malczhang@ust.hk}
%\author{Yang Xiang\corref{mycorrespondingauthor}}
%\ead{maxiang@ust.hk}
%\address{Department of Mathematics, Hong Kong University of Science and Technology, Clear Water Bay, Kowloon, Hong Kong}
%%% or include affiliations in footnotes:
%%\author[mymainaddress,mysecondaryaddress]{Department of Mathematics, Hong Kong University of Science and Technology}
%%\ead[url]{www.elsevier.com}
%
%%\author[mysecondaryaddress]{Global Customer Service\corref{mycorrespondingauthor}}
%%\cortext[mycorrespondingauthor]{Corresponding author}
%%\ead{xqinac@connect.ust.hk, lzhangas@connect.ust.hk, maxiang@ust.hk}
%%
%%\address[mymainaddress]{Department of Mathematics, Hong Kong University of Science and Technology}
%%\address[mysecondaryaddress]{Clearwater Bay, Kowloon, Hong Kong}
%
%\cortext[mycorrespondingauthor]{Corresponding author}
%%\ead{malczhang@ust.hk,maxiang@ust.hk}

\titlerunning{A Three-Dimensional Continuum Simulation Method for Grain Boundary Motion}

%% Group authors per affiliation:
\author{Xiaoxue Qin \and Luchan Zhang \and Yang Xiang}

\authorrunning{X. X. Qin, L. C. Zhang, Y. Xiang}

\institute{X. X. Qin \at Department of Mathematics, Hong Kong University of Science and Technology, Clear Water Bay, Kowloon, Hong Kong. \email{maxqin@ust.hk}
\and L. C. Zhang \at College of Mathematics and Statistics, Shenzhen University, Shenzhen, 518060,  China. Corresponding author. \email{zhanglc@szu.edu.cn}
\and Y. Xiang \at Department of Mathematics, Hong Kong University of Science and Technology, Clear Water Bay, Kowloon, Hong Kong. Corresponding author. \email{maxiang@ust.hk} }

\date{}
%\date{Received: date / Accepted: date}
% The correct dates will be entered by the editor


\maketitle

%\begin{abstract}
%We develop a continuum model for the dynamics of grain boundaries in three dimensions that incorporates the motion and reaction of the constituent dislocations. The continuum model includes evolution equations for both the motion of the grain boundary and the evolution of dislocation structure on the grain boundary. The critical but computationally expensive long-range elastic interaction of dislocations is replaced by a projection formulation that maintains the constraint of the Frank's formula describing the equilibrium of the strong long-range interaction. This continuum model is able to describe the grain boundary motion and grain rotation due to both coupling and sliding effects, to which the classical motion by mean curvature model does not apply. Comparisons with atomistic simulation results show that our continuum model is able to give excellent predictions of evolutions of low angle grain boundaries and their dislocation structures.
%\end{abstract}
%
%\begin{keyword}
%  Grain boundary dynamics, Coupling and sliding motions,
%Dislocation dynamics, Frank's formula, Projection method
%\end{keyword}

%\end{frontmatter}

%\linenumbers
%
%\section{Introduction}
%Grain boundaries are indispensable components in polycrystalline materials. The energy and dynamics of grain boundaries play essential roles in the mechanical and plastic behaviors of the materials \cite{Sutton1995}. %There are extensive studies in the literature on this motion of grain boundaries by using molecular dynamics or continuum simulations, e.g. \cite{chen1994computer,upmanyu1998atomistic,kazaryan2000generalized,liuchun2001,upmanyu2002boundary,feng2003numerical,zhang2005curvature,kinderlehrer2006variational,zhang2009numerical,Selim2009,lazar2010more,elsey2013simulations,dai2018convergence}.
%Most of the available continuum models for the dynamics of grain boundaries are mainly based on the motion driven by the capillary force which is proportional to the local mean curvature of the grain boundary \cite{Sutton1995,Herring1951,mullins1956two}. This motion is a process to reduce the interfacial energy $\int_S\gamma dS$, where $S$ is the grain boundary and $\gamma$ is the grain boundary energy density. If the energy density $\gamma$ is fixed, the driving force based on variation of the total energy  is in the normal direction of the grain boundary and is proportional to its mean curvature. There are many
%atomistic simulations and continuum models in the literature for the grain boundary motion driven by the capillary force (by mean curvature), e.g., \cite{chen1994computer,kazaryan2000generalized,liuchun2001,upmanyu2002boundary,zhang2005curvature,Kirch2006,
%Selim2009,lazar2010more}.
%
%
%%For low angle grain boundaries, the grain boundary energy density $\gamma(\theta)$ is an increasing function of the misorientation angle $\theta$.
%The decreasing of grain boundary energy density $\gamma(\theta)$ can also reduce the total energy. For a low angle grain boundary,
%this implies the decreasing the misorientation angle $\theta$.
%In this case, the two grains on different sides of the grain boundary will rotate and cause a relatively rigid-body translation of the two grains along the boundary. This process is called sliding motion of grain boundaries \cite{li1962possibility,shewmon1966energy,harris1998grain,Kobayashi2000,upmanyu2006simultaneous,
%esedoglu2016grain,epshteyn2019motion}.
%
%There is a different type of grain boundary motion which is called coupling motion \cite{li1953stress,srinivasan2002challenging,cahn2004unified}, in which
% the normal motion of the grain boundary induces a tangential motion proportionally. In the coupling motion, the energy density $\gamma(\theta)$ can increase although the total energy $\int_S\gamma dS $  is decreasing. Cahn and Taylor \cite{cahn2004unified} proposed a unified theory for the coupling and sliding motions of the grain boundary and demonstrated the theory based on dislocation mechanisms for a circular low angle grain boundary in two dimensions. Especially, the coupling motion of the grain boundary is associated with dislocation conservation during the motion of the grain boundary.  The  Cahn-Taylor theory and mechanisms of motion and reaction of the constituent dislocations have been examined by atomistic simulations and experiments \cite{srinivasan2002challenging,cahn2006coupling,molodov2007low,Molodov2009,trautt2012grain,wu2012phase,
% mcreynolds2016grain,yamanaka2017phase,salvalaglio2018defects}. It has been shown in Ref.~\cite{Rath2007} by a dislocation model and experimental observations that conservation and annihilation of the constituent dislocations may lead to cancelation of the coupling and sliding motions of the grain boundary, leading to the classical motion by curvature.    A continuum model  has been developed based on the motion and reaction of the constituent dislocations  for the dynamics of low angle grain boundaries in two dimensions \cite{zhang2018motion}. Their model can describe both the coupling  and sliding motions of low angle grain boundaries. Recently, they have derived a more efficient formulation \cite{zhang2019new}.
%A continuum model that generalizes the Cahn-Taylor theory based on mass transfer by diffusion confined on the grain boundary has been proposed \citep{Taylor2007}, and numerical simulations based on this generalization were performed using the level set method \citep{Gupta2014}. Crystal plasticity models that include shear-coupled grain boundary motion in the phase field framework of Ref.~\cite{Kobayashi2000} have been developed \cite{AdmalIJP2018,AskJMPS2018}, in which the geometric necessary dislocation (GND) tensor/lattice curvature tensor was used to represent the actual dislocation distributions on the grain boundaries. All these continuum models are for grain boundaries in two dimensions.
%
%There are only limited studies in the literature for the three-dimensional coupling and sliding motions of grain boundaries.
%Grain boundary motion and grain rotation in bcc and fcc bicrystals composed of
%a spherical grain embedded in a single crystal matrix were studied by using three-dimensional phase field crystal models \cite{yamanaka2017phase,salvalaglio2018defects}, and  properties of grain boundaries and their dislocation structures as the grain boundary evolves have been examined.
%Although these atomistic-level phase field crystal simulations are able to provide detailed information associated with the coupling and sliding motions of grain boundaries in three dimensions, three-dimensional continuum models of the dynamics of grain boundaries incorporating their dislocation structures are still desired for larger scale simulations.
%
%
%In this paper, we generalize the two-dimensional continuum model for grain boundary dynamics in Ref.~\cite{zhang2018motion,zhang2019new} to three dimensions, where grain boundaries and their constituent dislocations are curved in general. The three-dimensional continuum model for the dynamics of grain boundaries incorporates the motion and reaction of the constituent dislocations, and is able to describe the grain boundary motion and grain rotation due to both coupling and sliding effects, to which the classical motion by mean curvature model does not apply. The continuum model includes evolution equations for both the motion of the grain boundary and the evolution of dislocation structure on the grain boundary. The evolution of orientation-dependent continuous distributions of dislocation lines on the grain boundary is based on the simple representation using dislocation density potential functions \cite{zhu2014continuum}.
%%This simple representation method also guarantees continuity of the dislocation lines on the grain boundaries during the evolution. This model describes the distribution and dynamics of curves on a curved surfaces, and can apply to not only the dislocations and grain boundaries.
%
%
%The continuum model contains a long-range force in the form of singular integrals, whose evaluation is time-consuming. With the fact that the long-range force is so strong that an equilibrium state described by the Frank ’s formula~\cite{Frank1950,Bilby1955,zhu2014continuum} is quickly reached during the evolution, we replace the long-range force by a constraint of the Frank’s formula. We further solve the constraint evolution problem using the projection method, leading to a new, efficient continuum formulation.  Using the obtained continuum model, we perform numerical simulations for the evolution of initially spherical low angle grain boundaries in fcc and bcc crystals. Simulation results are compared with atomistic simulation results using phase field crystal models \cite{yamanaka2017phase,salvalaglio2018defects}. In particular, we explain the anisotropic motion of the grain boundary in the directions along and normal the rotation axis in terms of the constraint of the Frank's formula.
%
%This paper is organized as follows. In Sec.~\ref{sec:des}, we review  the description of a grain boundary in three dimensions and the continuum framework for grain boundary dislocation structure in Ref.~\cite{zhu2014continuum}. We also briefly review the Frank's formula for the equilibrium dislocation structure of a grain boundary. In Sec.~\ref{sec:cm}, we present our continuum model for grain boundary dynamics in three dimensions incorporating the evolution of the underlying dislocation structure, and develop a new, efficient formulation based on constrained evolution.
%Simulations for the evolution of initially spherical low angle grain boundaries in fcc and bcc crystals
% using our continuum model and comparisons with results of phase field crystal simulations~\cite{yamanaka2017phase,salvalaglio2018defects} are presented in Sec.~\ref{sec:nr}.
%
%\section{Geometric descriptions and continuum framework for grain boundaries and their dislocation structures}\label{sec:des}
%
%\subsection{Geometric descriptions for grain boundaries}
%
%Consider a grain boundary $S$ in three dimensions parameterized by $(u,v)$. That is,
%%two parameters $u$ and $v$:
%%\begin{equation}
%%\begin{split}
%%x & = x(u,v), \\
%%y & = y(u,v), \\
%%z & = z(u,v).
%%\end{split}
%%\end{equation}
%a point on grain boundary $S$ is
%\begin{equation}
% \mathbf r=\mathbf r(u,v) =(x(u,v),y(u,v),z(u,v)).
% \end{equation}
% With this parametrization, we have the following formulas for two unit tangent vectors
%  $ \mathbf{T^{(1)}}$, $\mathbf{T^{(2)}}$ and the unit normal vector $\mathbf{n}$:
%\begin{equation}
%%\begin{split}
%\mathbf{T^{(1)}}  = \frac{\mathbf r_u}{\|\mathbf r_u \|},
%%=(\frac{\partial x}{\partial u},\frac{\partial y}{\partial u},\frac{\partial z}{\partial u} )\frac{1}{\sqrt{(\frac{\partial x}{\partial u})^2+(\frac{\partial y}{\partial u})^2+(\frac{\partial z}{\partial u})^2}},
% \ \ \
%\mathbf{T^{(2)}}  =\frac{\mathbf r_v}{\|\mathbf r_v \|},
%%= (\frac{\partial x}{\partial v},\frac{\partial y}{\partial v},\frac{\partial z}{\partial v} )\frac{1}{\sqrt{(\frac{\partial x}{\partial v})^2+(\frac{\partial y}{\partial v})^2+(\frac{\partial z}{\partial v})^2}}.
%\ \ \
%\mathbf{n}  = \frac{\mathbf{T^{(1)}}\times\mathbf{T^{(2)}}}{\|\mathbf{T^{(1)}}\times\mathbf{T^{(2)}}\|}=\frac{\mathbf r_u\times\mathbf r_v }{\|\mathbf r_u\times\mathbf r_v\|},
%%\end{split}
%\end{equation}
%where $\mathbf r_u=\left(\frac{\partial x}{\partial u},\frac{\partial y}{\partial u},\frac{\partial z}{\partial u}\right)$ and $\mathbf r_v=\left(\frac{\partial x}{\partial v},\frac{\partial y}{\partial v},\frac{\partial z}{\partial v}\right)$. The area of the grain boundary is
%\begin{equation}\label{eqn:area0}
%S_A=\int_S\|\mathbf r_u\times\mathbf r_v\|\mathrm{d}u\mathrm{d}v.
%\end{equation}
%
%%%\begin{comment}
%%with $\frac{D(y,z)}{D(u,v)}$ are the determinant of the Jacobian matrix
%%\begin{equation}
%%\begin{split}
%%&\mathbf{n}  = (\frac{D(y,z)}{D(u,v)},\frac{D(z,x)}{D(u,v)},\frac{D(x,y)}{D(u,v)})\frac{1}{\sqrt{\frac{D(y,z)}{D(u,v)}^2+\frac{D(z,x)}{D(u,v)}^2+\frac{D(x,y)}{D(u,v)}^2}}, \\
%%&\mathbf{T^{(1)}}  = (\frac{\partial x}{\partial u},\frac{\partial y}{\partial u},\frac{\partial z}{\partial u} )\frac{1}{\sqrt{(\frac{\partial x}{\partial u})^2+(\frac{\partial y}{\partial u})^2+(\frac{\partial z}{\partial u})^2}}, \\
%%&\mathbf{T^{(2)}}  = (\frac{\partial x}{\partial v},\frac{\partial y}{\partial v},\frac{\partial z}{\partial v} )\frac{1}{\sqrt{(\frac{\partial x}{\partial v})^2+(\frac{\partial y}{\partial v})^2+(\frac{\partial z}{\partial v})^2}}.
%%\end{split}
%%\end{equation}
%%%\end{comment}
%Given a function $f$ on the grain boundary $S$, its surface gradient is
%\begin{equation}
%\begin{split}
%\nabla_S f &=\left(\nabla-\mathbf n(\mathbf n\cdot\nabla)\right)f\\
%&= \left(\frac{\|\mathbf r_v\|^2}{\|\mathbf r_u\times\mathbf r_v\|^2}f_{u}
%-\frac{\mathbf r_u\cdot\mathbf r_v}{\|\mathbf r_u\times\mathbf r_v\|^2}f_v
%\right)\mathbf r_u+\left(\frac{\|\mathbf r_u\|^2}{\|\mathbf r_u\times\mathbf r_v\|^2}f_v
%-\frac{\mathbf r_u\cdot\mathbf r_v}{\|\mathbf r_u\times\mathbf r_v\|^2}f_u
%\right)\mathbf r_v.
%\end{split}\label{eqn:surgra}
%\end{equation}
%
%
%\subsection{Continuum framework for grain boundary dislocation structure}
%In this subsection, we review the continuum framework for the dislocation structure of low angle grain boundaries proposed in Ref.~\cite{zhu2014continuum}. In this continuum framework,  dislocation density potential functions (scalar functions) defined on the grain boundary are employed to describe the dislocation structure on the grain boundary. This provides a simple and accurate description of the orientation dependent densities of dislocations which are connected lines. Energy of the grain boundary and driving forces for the dynamics are given based on the dislocation structure.
%
%
%\begin{figure}[htbp]
%	\centering
%	\includegraphics[width=3.5in]{figures/ddpf.eps}
%	\caption{A dislocation density potential function $\eta$ defined on a grain boundary $S$. Its integer-value contour lines represents the array of dislocations with the same Burgers vector. }
%	\label{fig:dislocation}
%\end{figure}
%
%For a dislocation density potential function $\eta$ defined on a grain boundary $S$, the constituent dislocations of Burgers vector $\mathbf b$ are given by the contour lines of $\eta$: $\eta=j$, for integer j. See Fig.~\ref{fig:dislocation} for an example of dislocation structure on a spherical grain boundary and $\eta$ defined on it.  From the dislocation density potential function $\eta$, the inter-dislocation distance $D$ can be calculated by
%\begin{equation}
%D=\frac{1}{ \| \nabla_S\eta\|},
%\end{equation}
%and the dislocation direction is given by
%\begin{equation}
%\mathbf t=\frac{\nabla_S\eta \times \mathbf n }{\| \nabla_S\eta\| },
%\end{equation}
%where $\nabla_S\eta$ is the surface gradient of $\eta$ on $S$:
%\begin{equation}
%\nabla_S\eta=\left(\nabla-\mathbf n (\mathbf n \cdot \nabla)\right)\eta.
%\end{equation}
%Multiple dislocation density potential functions are used for dislocations with different Burgers vectors.
%
%
%
%Suppose there are $J$ arrays of dislocations with Burgers vectors $\mathbf b^{(j)}, j=1,2,\cdots,J$, respectively, on the grain boundary  $S$, and they are described by the dislocation density potential functions $\eta_j$,  $j=1,2,\cdots,J$, respectively. The continuum formulation of the elastic energy is
%\begin{equation}
%E_{\rm tot}=E_{\rm long}+E_{\rm local},\label{eqn:energy}
%\end{equation}
%where  $E_{\rm long}$ is the long-range interaction energy of dislocations and $E_{\rm local}$ is the local dislocation line energy:
%\begin{eqnarray}
%&&\begin{aligned}
%E_{\rm long}& = \frac{1}{2}\sum_{i=1}^J\sum_{j=1}^J \int_S  \mathrm{d}S_i \int_S  \mathrm{d}S_j \left[ \right. \frac{\mu}{4\pi}\frac{(\nabla_S \eta_i\! \times \!\mathbf{n}_i\cdot\mathbf b^{(i)})(\nabla_S \eta_j\! \times \!\mathbf{n}_j\cdot\mathbf b^{(j)}) }{r_{ij}}\\
%&  -\frac{\mu}{2\pi}\frac{(\nabla_S \eta_i\! \times \!\mathbf{n}_i )\times (\nabla_S \eta_j\! \times \!\mathbf{n}_j ) \cdot(\mathbf b^{(i)}\times \mathbf b^{(j)} ) }{r_{ij}}\\
%&\left.+ \frac{\mu}{4\pi(1-\nu)} (\nabla_S \eta_i\! \times \!\mathbf{n}_i\cdot\mathbf b^{(i)})\cdot (\nabla\otimes \nabla r_{ij})\cdot (\nabla_S \eta_j\! \times \!\mathbf{n}_j\cdot\mathbf b^{(j)})\right],   \\
%\end{aligned}  \vspace{1ex}\\
%&& E_{\rm local } = {\displaystyle \sum_{j=1}^J \int_S {\displaystyle \frac{\mu(b^{(j)})^2}{4\pi(1-\nu)}\!\left[1-\nu\frac{(\nabla_S \eta_j\! \times \!\mathbf{n} \!\cdot\! \mathbf{b}^{(j)})^2}{(b^{(j)})^2 {\|\nabla_S \eta_j\|}^2}\right]\|\nabla_S \eta_j\| \log\! \frac{1}{r_g\|\nabla_S \eta_j\| }}  \mathrm{d}S_j}. \hspace{0.2in}\label{eqn:localenergy}
%\end{eqnarray}
%Here $r_{ij}=\|\mathbf X_i-\mathbf X_j\|$, where $\mathbf X_i$ and $\mathbf X_j$ are the points varying on the grain boundary $S$ and are associated with the surface integral $dS_i$ and $dS_j$, respectively. $\mathbf n_j$ is the normal direction of the surface S associated with the surface integral $dS_j$, notation $\otimes$ is the tensor product operator, the gradient in the term $\nabla \otimes \nabla r_{ij}$ is taken with respect to $\mathbf X_i$, and $b^{(j)}=\|\mathbf b^{(j)}\|$. The elastic constants $\mu$ is the shear modulus and $\nu$ is the Poisson's ratio.
%
%The driving forces for the dynamics of the grain boundary and the dislocation structure are associated with the variations of the total energy in Eqs.~\eqref{eqn:energy}--\eqref{eqn:localenergy}:
%\begin{eqnarray}
%&&\frac{\delta E_{\rm tot}}{\delta r}=-\sum_{j=1}^{J}\| \nabla_S\eta_j\|(\mathbf f_{\rm long} ^{(j)}+\mathbf f_{\rm local} ^{(j)} )\cdot \mathbf n,\label{eqn:var_r}\\
%&&\frac{\delta E_{\rm tot}}{\delta \eta_j}=(\mathbf f_{\rm long} ^{(j)}+\mathbf f_{\rm local} ^{(j)} )\cdot \frac{\nabla_S\eta_j }{\| \nabla_S\eta_j\|},
%\end{eqnarray}
%where $\mathbf f_{\rm long} ^{(j)} $ is the continuum long-range force, $\mathbf f_{\rm local} ^{(j)}$ the local force on the $\mathbf b ^{(j)}$-dislocations (i.e., the array of dislocations with Burgers vector $\mathbf b ^{(j)}$), and $r$ is the signed distance in the normal direction of the grain boundary. In fact, the total force $\mathbf f_{\rm long} ^{(j)}+\mathbf f_{\rm local} ^{(j)} $ is the continuum limit \cite{zhu2014continuum} from the Peach-Koehler force on dislocations in discrete dislocation dynamics~\cite{HirthLothe1982}.
%
%The long-range force $\mathbf f_{\rm long} ^{(j)} $ and the local force $\mathbf f_{\rm local} ^{(j)}$ have the following formulations:
%\begin{flalign}
%\mathbf f^{(j)}_{\rm long}=&(\pmb \sigma^{\rm tot}\cdot \mathbf b^{(j)})\times \left(\frac{\nabla_S \eta_j}{\|\nabla_S \eta_j\|}\times \mathbf n\right), \label{eqn:c3flong}\\
%\mathbf f_{\rm local} ^{(j)}=&\mathbf f_{lt}^{(j)}+ \mathbf f_{p}^{(j)}, \label{eqn:localf}\\
%\mathbf f_{lt}^{(j)}=&\frac{\mu}{4\pi(1-\nu)}\kappa_d \left[(1+\nu)(b^{(j)}_t)^2+(1-2\nu)(b^{(j)}_N)^2+(b^{(j)}_B)^2\right]\log \frac{1}{r_g \| \nabla_S\eta_j\|}\mathbf n_d^{(j)}  \nonumber \\
%&-\frac{\mu\nu}{2\pi(1-\nu) }\kappa_db^{(j)}_Nb^{(j)}_B\mathbf t^{(j)}\times \mathbf n_d^{(j)},
%\label{eqn:flt}\\
% \mathbf f_{p}^{(j)}=&\frac{\mu}{4\pi(1-\nu)}\kappa_p^{(j)}\left[(b^{(j)})^2-\nu(b_t^{(j)})^2 \right]\mathbf n_p^{(j)}.\label{eqn:fp}
%\end{flalign}
%Here $\pmb \sigma^{\rm tot}$ is the total stress field, which includes the
%long-range stress field generated by the dislocation arrays on $S$ and other stress fields. The
%long-range stress field generated by the dislocation arrays is
%\begin{equation}
%\begin{aligned}
%\pmb \sigma(\mathbf X)=&\sum_{j=1}^J\frac{\mu}{4\pi}\int_S\left[ \left(\nabla\frac{1}{r}\times\mathbf b^{(j)}\right)\otimes (\nabla_S\eta_j\times\mathbf n)+(\nabla_S\eta_j\times\mathbf n)\times\left(\nabla\frac{1}{r}\times\mathbf b^{(j)} \right) \right. \\
%&\left.+\frac{1}{1-\nu}\left(\mathbf b^{(j)}\times (\nabla_S\eta_j\times\mathbf n)\cdot \nabla\right)(\nabla\otimes\nabla-I\Delta)r   \right]\mathrm{d}S,
%\end{aligned} \label{eqn:lsig}
%\end{equation}
%where $r=\|\mathbf X-\mathbf X_S \| $ with points $ \mathbf X_S$ varying on the grain boundary S and $\nabla_S\eta_j$ and $\mathbf n$ being evaluated at $\mathbf X_S$. The local force $\mathbf f_{\rm local} ^{(j)}$ consists of the dislocation line tension force $\mathbf f_{lt}^{(j)}$ and the curvature force $\mathbf f_{p}^{(j)}$ associated the curve on $S$ that is normal to the local dislocation,
%where $\kappa_d$ is the curvature of dislocation line, $\mathbf n_d$ is the normal direction of dislocation, $\kappa_p$ and $\mathbf n_p$ are the curvature and normal direction of the curve on $S$ that is normal to the location dislocation, respectively, and $\kappa_d \mathbf n_d=(\nabla_S\mathbf t )\cdot \mathbf t$, $\kappa_p \mathbf n_p=\left(\nabla_S \frac{\nabla_S \eta_j}{\|\nabla_S \eta_j\|} \right)\cdot \frac{\nabla_S \eta_j}{\|\nabla_S \eta_j\|} $, $b_t=\mathbf b\cdot \mathbf t$, $b_N=\mathbf b\cdot \mathbf n_d$, $b_B=\mathbf b\cdot (\mathbf t \times \mathbf n_d ) $.
%
%\subsection{Frank's formula for an equilibrium grain boundary}
%
%The equilibrium dislocation structure on a grain boundary is governed by the Frank's formula~\cite{Frank1950,Bilby1955,zhu2014continuum}:
%\begin{equation}
%\label{Frank}
%\theta(\mathbf{V}\times\mathbf{a})-\sum_{j=1}^{J}\mathbf{b}^{(j)}(\nabla_S \eta_j{\cdot}\mathbf{V})=\mathbf{0},
%\end{equation}
%where $\theta$ is the misorientation angle, $\mathbf{a}=(a_1,a_2,a_3)$ is the rotation axis, $\mathbf{V}$ is any vector in the grain boundary's tangent plane,  and there are $J$ arrays of dislocations with Burgers vectors $\mathbf b^{(j)}, j=1,2,\cdots,J$, respectively, on the grain boundary.
%
%The Frank's formula holds if and only if the long-range elastic fields generated by the grain boundary cancel out~\cite{Frank1950,Bilby1955,zhu2014continuum}. The classical Frank's formula is based on planar grain boundaries~\cite{Frank1950,Bilby1955}. It has been shown in Ref.~\cite{zhu2014continuum} that the equivalence of the Frank's formula to the cancelation of the long-range elastic fields also holds for curved grain boundaries. Eq.~\eqref{Frank} is the Frank's formula for a curved grain boundary, where the dislocation structure is represented by the dislocation density potential functions $\{\eta_j\}$. Eq.~\eqref{Frank} holds  for any vector $\mathbf{V}$ in the grain boundary's tangent plane if and only if it holds for the two tangent vectors
%$\mathbf V=\mathbf r_u$ and $\mathbf V=\mathbf r_v$, where $u$ and $v$ are the two parameters of the grain boundary.
%
%\section{Continuum model for grain boundary dynamics in three dimensions}\label{sec:cm}
%
%In this section, we present our continuum model for the dynamics of grain boundaries in three dimensions that incorporates the coupling and sliding motions. The continuum model consists of an equation for the motion of the grain boundary and an equation for the evolution of its dislocation structure. The continuum model is based on the simple representation of dislocation structure on grain boundaries using dislocation density potential functions and the continuum energy formulation in Ref.~\cite{zhu2014continuum}. The continuum model is able to describe both the coupling and sliding motions of the grain boundary. The misorientation angle associated with the grain boundary is calculated by the grain boundary profile and its dislocation structure.
%
%This continuum model has a long-range force in the form of a singular integral over the entire grain boundary. Simulations directly using this continuum formulation will be time-consuming. Moreover, when the source term of dislocations (for annihilation and generation of dislocations)  is directly expressed in terms of variation of the grain boundary energy, the resulting evolution equation of the dislocation structure is not well-posed due to the fact that the (local) energy of the grain boundary is a concave function of dislocation densities.
%Modifications of the continuum formulation will be developed to address these limitations.
%
%
%
%\subsection{Continuum model with long range force}\label{sec:long-range}
%
%We first present the three-dimensional continuum model with long range force for the dynamics of a grain boundary and its dislocation structure.
%
%\vspace{0.1in}
%\noindent
%\underline{\bf Continuum model with long range force}
%\vspace{0.05in}
%\begin{eqnarray}
%&&v_n= M_\mathrm{d}\sum_{j=1}^J\frac{\|\nabla_S \eta_j\|}{\sum_{k=1}^J\|\nabla_S \eta_k\|}(\mathbf{f}^{(j)}_{\mathrm{long}}+\mathbf{f}^{(j)}_{\mathrm{local}})\cdot\mathbf{n}, \label{eqn:mod1v}\\
%&& \frac{\partial \eta_j}{\partial t}=-M_{\rm d} \, \mathbf{f}^{(j)}_{\mathrm{long}}\cdot\nabla_S\eta_j+s_j.\label{eqn:mod1e}
%\end{eqnarray}
%
%
%Eq.~\eqref{eqn:mod1v} gives the normal velocity of the grain boundary. In this equation,  $\mathbf{f}^{(j)}_{\mathrm{long}}$ (given in Eqs.~\eqref{eqn:c3flong} and \eqref{eqn:lsig}) and $\mathbf{f}^{(j)}_{\mathrm{local}}$ (given in Eqs.~\eqref{eqn:localf}--\eqref{eqn:fp}) are the long-range force and the local force, respectively. The velocity of a point on a grain boundary is the weighted average of the velocities of  dislocations on the grain boundaries with different Burgers vectors, and $M_{\rm d}$ is the mobility of dislocations.
%
%Eq.~\eqref{eqn:mod1e} describes the evolution of the dislocations structure on the grain boundary. The first term on the right-hand side describes the motion of dislocations along the grain boundary, which is in the form of  conservation law driven by  the continuum long-range elastic force.
%The continuum long-range force $\mathbf{f}_{\mathrm{long}}$ is the leading order continuum force,
%and it  maintains a stable dislocation structure~\cite{xiang2018stability}. Recall that the cancelation of the long-range force is equivalent to the Frank's formula for an equilibrium dislocation structure. The second term $s_j$ on the right hand side of Eq.~\eqref{eqn:mod1e} is the source term that accounts for dislocation reaction \cite{srinivasan2002challenging,trautt2012grain,yamanaka2017phase}, which on the continuum level may be modeled based on the variation of the local energy of the grain boundary and whose exact formulation will be discussed in Sec.~\ref{sec:well-posed}.
%
%
%%Recall that the equilibrium dislocation structure, which is reached if and only if the long-range elastic fields cancel out, is governed by the Frank's formula \cite{Frank1950,Bilby1955,zhu2014continuum}. The second term on the right-hand side of Eq.  \eqref{eqn:mod1e} comes from the driving force due to variation of the grain boundary energy density $\gamma$ (when the long-range elastic interaction vanishes) with respect to the change of dislocation density $\eta_j$ on the fixed grain boundary, and $M_r$ is the mobility associated with this driving force. If we choose $M_r=0$ , this means there is no dislocation reaction or the number of dislocations on the grain boundary is conserved. When $M_r\neq 0$, the dislocation will react to reduce the total energy, this is the case shown in \cite{yamanaka2017phase,salvalaglio2018defects}.
%
%The misorientation angle $\theta$ during the evolution can be calculated using Frank's formula in Eq.~\eqref{Frank} based on the grain boundary profile and the dislocation structure, and the formula is
%\begin{equation}
%\theta=\frac{1}{S_A}\int_S \sum_{j=1}^J \frac{ ( \eta_{ju}+  \eta_{jv})( \mathbf r_u + \mathbf r_v )\times\mathbf{a}{\cdot}\mathbf{b}^{(j)} }{\|(  \mathbf r_u+ \mathbf r_v) \times\mathbf{a}\|^2}  \mathrm{d}S.\label{eqn:mod1t}
%\end{equation}
%where $\eta_{ju}=\frac{\partial \eta_j}{\partial u}$, $\eta_{jv}=\frac{\partial \eta_j}{\partial v}$, and recall that $S_A$ is the area of the grain boundary $S$ that can be calculated using Eq.~\eqref{eqn:area0}.
%Note that although the dislocation structure on the grain boundary is driven into equilibrium by the long-range force generated by the dislocations, at any moment during the evolution, the dislocation structure on the grain boundary is not necessarily in equilibrium exactly, i.e., there may not be a constant value $\theta$ that makes the Frank's formula in Eq.~\eqref{Frank} holds at every point on the grain boundary. The formula of $\theta$ in Eq.~\eqref{eqn:mod1t} is an estimation for the misorientation angle based on some average over the grain boundary. Its derivation is as follows.
%
%Substituting $\mathbf V_1= \mathbf r_u$ and $\mathbf V_2= \mathbf r_v$ into Frank's formula in Eq.~\eqref{Frank},
%% and using the surface gradient formula in Eq. \eqref{eqn:surgra},
%we have
%%
%%%\begin{comment}
%%\begin{equation}
%%\label{Frank2}
%%\begin{split}
%%&\theta(\mathbf r_u\times\mathbf{a})-\sum_{j=1}^{J}\mathbf{b}^{(j)}(\bigtriangledown _s \eta_j\bm{\cdot}\mathbf r_u)\\
%%=&\theta(\mathbf r_u\times\mathbf{a})-\sum_{j=1}^{J}\mathbf{b}^{(j)}( (\bigtriangledown-\mathbf{n}(\mathbf{n}\bm{\cdot}\bigtriangledown))\eta_j\bm{\cdot}\mathbf r_u)\\
%%=&\theta(\mathbf r_u\times\mathbf{a})-\sum_{j=1}^{J}\mathbf{b}^{(j)}( \bigtriangledown\eta_j\bm{\cdot}\mathbf r_u)\\
%%=&\theta(\mathbf r_u\times\mathbf{a})-\sum_{j=1}^{J}\mathbf{b}^{(j)}\eta_{ju}\\
%%=&0.
%%\end{split}
%%\end{equation}
%%%\end{comment}
%\begin{flalign}
%\theta(\mathbf r_u\times\mathbf{a})-\sum_{j=1}^{J}\mathbf{b}^{(j)}\eta_{ju}=&0,\label{eqn:F1} \\
%\theta(\mathbf r_v\times\mathbf{a})-\sum_{j=1}^{J}\mathbf{b}^{(j)}\eta_{jv}=&0.\label{eqn:F2}
%\end{flalign}
%Here we have used $\nabla_S \eta_j\bm{\cdot}\mathbf r_u=\eta_{ju}$ and $\nabla_S \eta_j\bm{\cdot}\mathbf r_v=\eta_{jv}$. Adding the two equations \eqref{eqn:F1} and \eqref{eqn:F2}, multiplying both size of the summation by $( \mathbf r_u+ \mathbf r_v) \times\mathbf{a} $, we have
%\begin{equation}
%\theta \|(  \mathbf r_u+ \mathbf r_v) \times\mathbf{a}\|^2=
% \sum_{j=1}^J  ( \eta_{ju}+  \eta_{jv})( \mathbf r_u + \mathbf r_v )\times\mathbf{a}{\cdot}\mathbf{b}^{(j)}.
%\end{equation}
%Integrating over the entire grain boundary $S$, we obtain the formula of $\theta$ in Eq.~\eqref{eqn:mod1t}, which is exact if the Franks' formula holds and serves as an estimate formula otherwise.
%
%
%
%
%
%\subsection{Continuum model based on projection}\label{sec:projection}
%
%The continuum formulation given by Eqs.~\eqref {eqn:mod1v} and \eqref{eqn:mod1e} contains the long-range elastic force
%$\mathbf{f}^{(j)}_{\mathrm{long}}$ (given in Eqs.~\eqref{eqn:c3flong} and \eqref{eqn:lsig}), which is a singular integral over the entire grain boundary surface. Numerically,
%computation of such long-range force with reasonable accuracy is complicated and time-consuming
%even in two-dimensional cases \cite{zhang2018motion,zhang2019new}.
% Actually, the long-range interaction between the grain boundary dislocations is so strong that an equilibrium state described by the Frank's formula \cite{Frank1950,Bilby1955,zhu2014continuum} is quickly reached during the evolution of the grain boundaries.
% We simply assume that the Frank's formula always holds during the evolution of the grain boundary, which will not lead to significant change in grain boundary motion. Under on this assumption, the long-range elastic force is replaced by the constraint of Frank’s formula in the continuum dynamics model.  This assumption has been validated  in the two-dimensional model~\cite{zhang2018motion,zhang2019new}, and here we apply this assumption to the continuum model in three dimensions. The new formulation is:
%
%\vspace{0.1in}
%\noindent
%\underline{\bf Continuum model with constraint}
%\vspace{0.05in}
%\begin{flalign}
%&v_n= M_\mathrm{d}\sum_{j=1}^J\frac{\|\nabla_S \eta_j\|}{\sum_{k=1}^J\|\nabla_S \eta_k\|}\mathbf{f}^{(j)}_{\mathrm{local}}\cdot\mathbf{n}, \label{eqn:mod2v}\\
%& \frac{\partial \eta_j}{\partial t}=s_j,\label{eqn:mod2e}\\
%%&& \frac{\delta\eta_j}{\delta t}= -M_r\frac{\delta \gamma_{gb}}{\delta \eta_j},\label{eqn:mod2e}\\
%&\text{subject to} \ \
%\mathbf{h}=\theta(\mathbf{V}\times \mathbf{a}) - {\displaystyle \sum_{j=1}^J} \mathbf{b}^{(j)}(\nabla_S\eta_j\cdot\mathbf{V})=\mathbf 0.\label{eqn:mod2frank}
%%&&\theta(\mathbf r_u ,\mathbf r_v,\eta_{ju},\eta_{jv})=\frac{1}{S}\int\int_S \sum_{j=1}^J \frac{ ( \eta_{ju}+  \eta_{jv})( \mathbf r_u + \mathbf r_v )\times\mathbf{a}{\cdot}\mathbf{b}^{(j)} }{\|(  \mathbf r_u+ \mathbf r_v) \times\mathbf{a}\|^2}  \mathrm{d}S.\hspace{0.1in}\label{eqn:mod2t}
%\end{flalign}
%Here, the constraint in Eq. \eqref{eqn:mod2frank} is the Frank's formula.
%
%%Since the equilibrium dislocation structure that satisfies the Frank's formula is stable \cite{xiang2018stability}. Thus, $M_d=0$ in the step of $\eta$ evolution \eqref{eqn:mod2e} is a good approximation.
%
%The constrained evolution model in Eqs. \eqref{eqn:mod2v}-\eqref{eqn:mod2frank} can be implemented using the projection method. We evolve the grain boundary and dislocation without the constraint first, and then project the grain boundary and its dislocation structure to a new configuration that satisfies the constraint (i.e. the Frank's formula). The projection procedure can be solved, leading to the following formulation. Here without loss of generality, suppose that the rotation axis is in the $+z$ direction, i.e., $\mathbf a = (0,0,1)$.
%
%
%\vspace{0.1in}
%\noindent
%\underline{\bf Continuum model using projection method}
%\vspace{0.05in}
%
%
%\hspace{0.3in}From $t_n$ to $t_{n+1}=t_n+\delta t$,
%%\begin{align}
%\begin{eqnarray}
%&&\mathbf v^*=\left(M_\mathrm{d}\sum_{j=1}^J\frac{\|\nabla_S\eta_j\|}{\sum_{k=1}^J \|\nabla_S\eta_k\| }\mathbf f_{\rm local}^{(j)}\cdot \mathbf n \right) \mathbf n,\label{eqn:v1}\\
%&&\mathbf r^*=\mathbf r^n +\mathbf v^* \delta t,\label{eqn:e11}\\
%&&\eta^{n+1}_{j}=\eta^{n}_{j} +s_j^n \delta t,\label{eqn:e2}\\
%&&\delta \theta=\theta(\mathbf r^{*}_u ,\mathbf r^{*}_v,\eta^{n+1}_{ju},\eta^{n+1}_{jv}  )-\theta(\mathbf r^{n}_u ,\mathbf r^{n}_v,\eta^{n}_{ju},\eta^{n}_{jv}) ,\label{eqn:dt}\\
%&&\mathbf v=\left(-\frac{\delta \theta}{\theta \delta t}(x-c_1), \ -\frac{\delta \theta}{\theta \delta t}(y-c_2),  \ v^*_3\right)+\left(-\frac{1}{\theta}\sum_{j=1}^J  b^{(j)}_2s_j, \ \frac{1}{\theta}\sum_{j=1}^J  b^{(j)}_1s_j,\ 0\right), \label{eqn:v2}\\
%%&&\mathbf v=(-\frac{d \theta}{\theta d t}x-\frac{1}{\theta}\sum_{j=1}^J \mathbf b^{(j)}_2\frac{d\eta_{j}}{dt},-\frac{d \theta}{\theta d t}y+\frac{1}{\theta}\sum_{j=1}^J \mathbf b^{(j)}_1\frac{d\eta_{j}}{dt},  v^*_3), \hspace{0.4in}\label{eqn:v2}\\
%&&\mathbf r^{n+1}=\mathbf r^n +\mathbf v \delta t. \label{eqn:v3}
%\end{eqnarray}
% where  constanta $c_1$ and $c_2$ can be determined numerically by the condition that the projection procedure alone does not lead to extra rigid translation of the grain boundary. When the top point of the grain boundary in the $+z$ direction always has a velocity in the $z$ direction during the evolution due to some symmetry,  we simply have $c_1=c_2=0$ if  the $z$ axis is set passing through that point. The values of misoreintation angle $\theta$ in Eq.~\eqref{eqn:dt} are calculated by using Eq.~\eqref{eqn:mod1t}.
%
%Eq.~\eqref{eqn:e11} is the virtual evolution of the grain boundary after the time step $\delta t$, using the velocity $\mathbf v^*=(v_1^*,v_2^*,v_3^*)$ due to the local force without the constraint as given in Eq.~\eqref{eqn:v1}. Eq.~\eqref{eqn:e2} is the evolution of dislocation structure due to dislocation reaction without the constraint.
%The change of misorientation angle $\delta \theta$ during this time step is determined by this virtual evolution as given in Eq.~\eqref{eqn:dt}.
%Based on this obtained $\delta \theta$, the grain boundary  velocity is adjusted by projection of the new configuration of the grain boundary to a state that satisfies the constraint of Frank's formula, and the obtained velocity $\mathbf v$ is given in Eq.~\eqref{eqn:v2}, following which the grain boundary is actually evolved.
%
%Note that in the projection procedure, we essentially adjust the local value of $\theta$ determined by the  Frank's formula in Eq.~\eqref{eqn:mod2frank} to achieve a uniform misorientation angle $\theta$ over the entire grain boundary. This procedure should not lead to additional rigid translation of the grain boundary. The two constants  $c_1$ and $c_2$ in the  projected velocity formula in Eq.~\eqref{eqn:v2} can be determined by this condition. For some symmetric configuration of the grain boundary, we have $c_1=c_2=0$ as described above.
%Also note that the Frank's formula in Eq.~\eqref{eqn:mod2frank} does not impose any constraint in the direction parallel to the rotation axis $\mathbf a$, which is in $+z$ direction. As a result, we simply keep the $z$-component $v^*_3$ of the virtual velocity in the projected velocity formula in Eq.~\eqref{eqn:v2}.
%
%In the projected velocity formula in Eq.~\eqref{eqn:v2},  the first term describes the pure coupling motion of the grain boundary, the second term describes the additional effect of the sliding motion  of the grain boundary due to dislocation reaction.
%%Here we consider the rotation axis $\mathbf a =(0,0,1)$. The Frank's formula gives the velocity in $x,y$ direction. Denote $\mathbf v^*=(v^*_1,v^*_2,v^*_3)$,  $v^*_3$ gives the velocity in $z$ direction. In Eq. \eqref{eqn:v2}, $d \eta_j$ is the change of dislocation density $\eta_j$ which is given in Eq.\eqref{eqn:e2}.
%Derivation of this formulation is given below.
%
%
%%Eq. \eqref{eqn:v1} give an intermediate normal velocity to get the difference between new misorentation angle and old misorentation angle, which is given in Eq. \eqref{eqn:dt}.
%%
%%Eqs. \eqref{eqn:e2} give the dislocation evolution equation due to dislocation reaction. Eq. \eqref{eqn:dt} gives the evolution of misorentation angle. In this equation, the values of function $\theta$ are calculated using Eq. \eqref{eqn:mod1t}, which depend on the location of grain boundary $\mathbf r$ and the components of dislocation density $\eta_{ju}, \eta_{jv}$. Before the virtual evolution of grain boundary using $\mathbf v^*$ and the dislocation evolution \eqref{eqn:e2}, the grain boundary location is $\mathbf r$, the dislocation density components are $\eta^{\rm old}_{ju}, \eta^{\rm old}_{jv}$. After the virtual evolution of grain boundary \eqref{eqn:v1}, and the dislocation evolution, the grain boundary location is $\mathbf r^*$, the dislocation density components are $\eta_{ju}, \eta_{jv}$.
%
%
%
%\vspace{0.1in}
%\noindent
%\underline{\bf Derivation of the solution of projection method}
%\vspace{0.05in}
%
%
%Suppose that the grain boundary velocity is $\mathbf v$, and the Frank's formula in Eqs.~\eqref{eqn:F1} and \eqref{eqn:F2} hold at the time $t_n$. After a small time step $\delta t$, if the Frank's formula still holds, we have
%\begin{flalign}
%(\delta\theta \mathbf r_u+\theta \delta t \mathbf v_u)\times\mathbf{a}-\sum_{j=1}^{J}\mathbf{b}^{(j)}\delta\eta_{ju}=&0,\label{eqn:Fu}\\
%(\delta\theta \mathbf r_v+\theta \delta t \mathbf v_v)\times\mathbf{a}-\sum_{j=1}^{J}\mathbf{b}^{(j)}\delta\eta_{jv}=&0.\label{eqn:Fv}
%\end{flalign}
%Here we have used  $\delta \mathbf r_u=\delta t \mathbf v_u$ and $\delta \mathbf r_v=\delta t \mathbf v_v$, where $\mathbf v_u=\frac{\partial \mathbf v}{\partial u}$ and $\mathbf v_v=\frac{\partial \mathbf v}{\partial v}$.
%
%%The dynamics of dislocation arrays:
%%\begin{equation}
%%    \frac{\delta\eta_j}{\delta t}+\mathbf v^{(j)}\cdot\nabla_S\eta_j=0
%%\end{equation}
%%$\mathbf v^{(j)}$ is the velocity of the dislocation array surface $S_j$.
%
%%we have
%%\begin{equation}
%%    \delta\eta_j=-\Delta t\mathbf v^{(j)}\cdot\nabla_S\eta_j+s_j
%%\end{equation}
%Integrating Eq.~\eqref{eqn:Fu} with respect to $u$, we have
%%\begin{equation}
%%\begin{split}
%%   &\int_{u=0}^u(\delta\theta \frac{\partial\mathbf r}{\partial u}+\theta \Delta t \frac{\partial \mathbf v_P}{\partial u})\times\mathbf{a}\mathrm{d}u-\sum_{j=1}^{J}\mathbf{b}^{(j)}\int_{u=0}^u\delta\eta_{ju}\mathrm{d}u\\
%%   &=\left[\delta\theta \mathbf r(u,v)+\theta \Delta t  \mathbf v_P(u,v)\right]\times\mathbf{a}-\left[\delta\theta \mathbf r(0,v)+\theta \Delta t  \mathbf v_P(0,v))\right]\times\mathbf{a}\\&-\sum_{j=1}^{J}\mathbf{b}^{(j)}[\delta\eta_{j}(u,v)-\delta\eta_{j}(0,v)]\\
%%   &=0
%%\end{split}
%%\end{equation}
%\begin{equation}
%\label{Frank1}
%\begin{split}
%&\left(\delta\theta \mathbf r(u,v)+\theta \delta t  \mathbf v(u,v)\right)\times\mathbf{a}-\sum_{j=1}^{J}\mathbf{b}^{(j)}\delta\eta_{j}(u,v)\\
%=&\left(\delta\theta \mathbf r(0,v)+\theta \delta t  \mathbf v(0,v))\right)\times\mathbf{a}-\sum_{j=1}^{J}\mathbf{b}^{(j)}\delta\eta_{j}(0,v).
%\end{split}
%\end{equation}
%Similarly, integrating Eq.~\eqref{eqn:Fv} with respect to $v$, we have
%\begin{equation}
%\label{Frank22}
%\begin{split}
%&\left(\delta\theta \mathbf r(u,v)+\theta \delta t  \mathbf v(u,v)\right)\times\mathbf{a}-\sum_{j=1}^{J}\mathbf{b}^{(j)}\delta\eta_{j}(u,v)\\
%=&\left(\delta\theta \mathbf r(u,0)+\theta \delta t  \mathbf v(u,0))\right)\times\mathbf{a}-\sum_{j=1}^{J}\mathbf{b}^{(j)}\delta\eta_{j}(u,0).
%\end{split}
%\end{equation}
%
%Notice that the left-hand sides of Eqs.~\eqref{Frank1} and \eqref{Frank22} are equal, whereas the right-hand side of  Eq. \eqref{Frank1} depends only on $v$ and the right-hand side of  Eq. \eqref{Frank22} depends only on $u$. Thus the right-hand sides of Eqs.~\eqref{Frank1} and \eqref{Frank22} must equal to the same constant independent of $u$ and $v$, denoted by $\mathbf C$. That is,
%\begin{equation}
%\left(\delta\theta \mathbf r(u,v)+\theta \delta t  \mathbf v(u,v)\right)\times\mathbf{a}-\sum_{j=1}^{J}\mathbf{b}^{(j)}\delta\eta_{j}(u,v)=\mathbf C.
%\end{equation}
%As discussed above, the constant $C$ can be determined by the condition that the projection procedure alone does not lead to extra rigid translation of the grain boundary, and can be calculated numerically during the evolution. Note that the dislocation structure depends only on $\{\nabla_S \eta_j\}$, and does not change by adding a constant.
%%where $C$ is some constant.
%%we have
%%\begin{equation}
%%    \delta\eta_j=-\Delta t\mathbf v^{(j)}\cdot\nabla_S\eta_j+\Delta t s_j
%%\end{equation}
%
% In the case when the top point of the grain boundary in the $+z$ direction always has a velocity in the $z$ direction (e.g., due to some symmetry), we set the $z$ axis passing through that point, i.e., that point is $\mathbf r=(0,0,z)$ during the evolution. In this case, at that point, we have $(\delta\theta \mathbf r+\theta \delta t  \mathbf v)\times\mathbf{a}=\mathbf 0$.
%Thus, we have $\mathbf C=\mathbf 0$, and
%\begin{flalign}
%v_1 =&-\frac{\delta \theta}{\theta\delta t}x-\frac{1}{\theta}\sum_{j=1}^J  b^{(j)}_2\frac{\delta\eta_{j}}{\delta t},
%\label{velocity0}
%\\
%v_2 =&-\frac{\delta \theta}{\theta\delta t}y+\frac{1}{\theta}\sum_{j=1}^J  b^{(j)}_1\frac{\delta\eta_{j}}{\delta t},
%\label{velocity}
%\end{flalign}
%where $\mathbf v=(v_1,v_2,v_3)$.
%
%The Frank's formula in Eqs.~\eqref{eqn:F1} and \eqref{eqn:F2}  does not impose any condition on the velocity in the direction of the rotation axis, i.e., the $z$ direction. Thus we keep the projection of the
%velocity due to the local force in the direction of the rotation axis, which is  $v^*_3$ where $\mathbf v^*=(v^*_1,v^*_2,v^*_3)$ is the
%velocity due to the local force
%given in Eq. \eqref{eqn:v1}.
%
%In summary, the grain boundary velocity after the projection based on the constraint of Frank's formula is
%$\mathbf v=(v_1,v_2, v^*_3)$,
%where $v_1$ and $v_2$ are given in Eqs.~\eqref{velocity0} and \eqref{velocity}, and $v^*_3$ is the $z$-component of the
%velocity $\mathbf v^*$ due to the local force
%given in Eq. \eqref{eqn:v1}. This is the projected velocity formula in Eq.~\eqref{eqn:v2}.
%
%%%\begin{comment}
%%We use the force formula in \cite{zhu2014continuum}.
%%\begin{equation}
%%\mathbf v^{(j)}=M_1\mathbf f_{lt}^{(j)}+M_2\mathbf f_{p}^{(j)}
%%\label{vj}
%%\end{equation}
%%\begin{eqnarray}
%%&&\mathbf  f_{lt}^{(j)}=f^{(j)}_N\mathbf n_d+f^{(j)}_B\mathbf t\times \mathbf n_d\\
%%&&\mathbf f_{p}^{(j)}=\frac{\mu}{4\pi(1-\nu)}\kappa_p^{(j)}\left[(b^{(j)})^2-\nu(b_t^{(j)})^2 \right]\mathbf n_p^{(j)}
%%\end{eqnarray}
%%
%%\begin{equation}
%%\mathbf v_{local}=\sum_{j=1}^J\frac{\|\nabla_S\eta_j\|}{\sum_{k=1}^J |\nabla_S\eta_k\| }(\mathbf f_{lt}^{(j)}+\mathbf f_{p}^{(j)})
%%\label{vlocal}
%%\end{equation}
%%combine the Franks' formula results, we have the projection velocity formula:
%%\begin{equation}
%%\mathbf v=(v_1,v_2, \mathbf v_{local}(3))
%%\end{equation}
%%\begin{equation}
%%\frac{\delta\eta_j}{\delta t}=-\mathbf v^{(j)}\cdot\nabla_S\eta_j+ s_j
%%\end{equation}
%%%\end{comment}
%
%
%
%\subsection{Well-posed formulation with dislocation reaction}\label{sec:well-posed}
%
%A commonly adopted formulation for the dislocation reaction term $s_j$ in the evolution of dislocation structure in Eq.~\eqref{eqn:mod1e} or Eq.~\eqref{eqn:mod2e} is based on variation of the grain boundary energy, i.e., $E_{\rm local}$ in Eq.~\eqref{eqn:localenergy}. (Recall that equilibrium with respect to the long-range energy is almost reached during the evolution of the grain boundary.)
%However,
%the (local) grain boundary energy density $\gamma_{\rm gb}$ is not convex as a function of $\{\nabla_S \eta_j\}$, where
%\begin{equation}
%\gamma_{\rm gb}=\sum_{j=1}^J\frac{\mu(b^{(j)})^2}{4\pi(1-\nu)}\!\left(1-\nu\frac{(\nabla_S \eta_j\! \times \!\mathbf{n} \!\cdot\! \mathbf{b}^{(j)})^2}{(b^{(j)})^2 {\|\nabla_S \eta_j\|}^2}\right)\|\nabla_S \eta_j\| \log\! \frac{1}{r_g \|\nabla_S \eta_j\|}  \mathrm{d}S_j.\label{eqn:gamma_gb}
%\end{equation}
%This nonconvexity will lead to an ill-posed formulation when using gradient flow of the grain boundary energy  for the evolution of dislocation structure in Eq.~\eqref{eqn:mod1e} or Eq.~\eqref{eqn:mod2e}. This can be understood as follows. Neglecting the orientation dependence factor, the contribution of $\mathbf b^{(j)}$-dislocations in the energy density $\gamma_{\rm gb}$ is essentially $-\|\nabla_S \eta_j\|\log\|\nabla_S \eta_j\|$, which is a concave function of $\|\nabla_S \eta_j\|$. Recall that from the definition, we always have $\|\nabla_S \eta_j\|\leq 1$.  As a result, the gradient flow will give a backward-diffusion like ill-posed equation of $\eta_j$.
%
% In order to obtain a well-posed gradient flow formulation, we use the components of $\nabla_S \eta_j$, i.e., $\eta_{ju}$ and $\eta_{jv}$ as independent variables instead of $\eta_j$ itself in the evolution equation of dislocation structure. Variations of the grain boundary energy are taken with respect to $\eta_{ju}$ and $\eta_{jv}$ instead of $\eta_j$.
% This leads to a new problem that
% $\eta_{ju}$ and $\eta_{jv}$ are not independent, and they are related by
% $ \frac{\partial \eta_{ju}}{\partial v}-\frac{\partial \eta_{jv}}{\partial u}=0$.
% The physical meaning of these relations is that the dislocations are connected lines on the grain boundary.
% We including these relations as further constraints in our continuum model. Using these treatments,  the continuum formulation with constraint in Eqs.~\eqref{eqn:mod2v}--\eqref{eqn:mod2frank} can be rewritten as:
%
% \vspace{0.1in}
%\noindent
%\underline{\bf Well-posed continuum model with constraints}
%\vspace{0.05in}
%\begin{flalign}
%&v_n= M_\mathrm{d}\sum_{j=1}^J\frac{\|\nabla_S \eta_j\|}{\sum_{k=1}^J\|\nabla_S \eta_k\|}\mathbf{f}^{(j)}_{\mathrm{local}}\cdot\mathbf{n}, \label{eqn:mod2vn}\\
%& \frac{\partial \eta_{ju}}{\partial t}=-M_\mathrm{r}\frac{\partial \gamma_{\rm gb}}{\partial \eta_{ju}},  \ \ \frac{\partial \eta_{jv}}{\partial t}=-M_\mathrm{r}\frac{\partial \gamma_{\rm gb}}{\partial \eta_{jv}},\label{eqn:mod2en}\\
%&\text{subject to} \ \
%\mathbf{h}=\theta(\mathbf{V}\times \mathbf{a}) - {\displaystyle \sum_{j=1}^J} \mathbf{b}^{(j)}(\nabla_S\eta_j\cdot\mathbf{V})=\mathbf 0,\label{eqn:mod2frankn}\\
%& \hspace{0.8in} \frac{\partial \eta_{ju}}{\partial v}-\frac{\partial \eta_{jv}}{\partial u}=0.\label{eqn:gcon}
%%&&\theta(\mathbf r_u ,\mathbf r_v,\eta_{ju},\eta_{jv})=\frac{1}{S}\int\int_S \sum_{j=1}^J \frac{ ( \eta_{ju}+  \eta_{jv})( \mathbf r_u + \mathbf r_v )\times\mathbf{a}{\cdot}\mathbf{b}^{(j)} }{\|(  \mathbf r_u+ \mathbf r_v) \times\mathbf{a}\|^2}  \mathrm{d}S.\hspace{0.1in}\label{eqn:mod2t}
%\end{flalign}
%Here $M_\mathrm{r}$ is the mobility associated with dislocation reaction.
%
%%\vspace{0.1in}
%%\noindent
%%\underline{\bf Formulation using gradient component}
%%\vspace{0.05in}
%%
%%\hspace{0.005in}From $t_n$ to $t_{n+1}=t_n+d t$,
%%\begin{eqnarray*}
%%	&&\mathbf v^*=\left[\sum_{j=1}^J\frac{\|\nabla_S\eta_j\|}{\sum_{k=1}^J \|\nabla_S\eta_k\| }\mathbf f_{\rm local}^{(j)}\cdot \mathbf n \right] \mathbf n\label{eqn:gv1},\\
%%	&&\mathbf r^*=\mathbf r^n +\mathbf v^* d t,\\
%%	&&\eta^*_{ju}=\eta^{n}_{ju} -\left.M_r\frac{\delta\gamma_{gb}}{\delta\eta_{ju}} \right|_{t_n} d t, \eta^*_{jv}=\eta^{n}_{jv} -\left.M_r\frac{\delta\gamma_{gb}}{\delta\eta_{jv}}\right|_{t_n} d t, \label{eqn:ge1}\\
%%	&& \eta^{n+1}_{ju} ,\eta^{n+1}_{jv} \hspace{0.1in}\text{satisfies}\hspace{0.1in}
%%	{\textstyle \frac{\partial \eta_{ju}}{\partial v}-\frac{\partial \eta_{jv}}{\partial u}=0}, \ \ j=1,2,\cdots,J, \label{eqn:gcon}\\
%%	&&d \theta=\theta(\mathbf r^*_u ,\mathbf r^*_v,\eta_{ju},\eta_{jv}  )-\theta(\mathbf r_u ,\mathbf r_v,\eta^{\rm old}_{ju},\eta^{\rm old}_{jv}),\\
%%	&&\mathbf v=(-\frac{d \theta}{\theta d t}x,-\frac{d \theta}{\theta d t}y,  v^*_3)+(-\frac{1}{\theta}\sum_{j=1}^J  b^{(j)}_2\frac{d\eta_{j}}{dt},\frac{1}{\theta}\sum_{j=1}^J  b^{(j)}_1\frac{d\eta_{j}}{dt},0), \hspace{0.4in} \hspace{0.4in}\label{eqn:gv2}\\
%%	&&\mathbf r^{n+1}=\mathbf r^n +\mathbf v\rm d t.
%%\end{eqnarray*}
%%Here the velocity shown in \eqref{eqn:gv2} include the the dislocation density variance $d\eta_j$, we compute $d\eta_j$ by $d \eta_j(u,v)=\int_{0}^{u}\eta_{ju}du+\int_{0}^{v}\eta_{jv}dv+C^j$, $C^j$ are some chosen constants such that $\int_S \delta\eta_jdS=0$.  The evolution $\eta_{ju}$ and $\eta_{jv}$ in Eqs. \eqref{eqn:ge1}  is subject to the constraint Eq. \eqref{eqn:gcon}.
%
%Numerically, we implement the additional
% constraint in Eq.~\eqref{eqn:gcon}  using a projection method similar to that for  fluid dynamics problems \cite{Chorin1968}. Recall that the evolution of  $\eta_{ju}$ and $\eta_{jv}$ in Eq.~\eqref{eqn:mod2en} is to minimize the energy $E_{\rm local} = \int_S \gamma_{\rm gb} dS$ by gradient flow. In order to implement the  constraint in Eq.~\eqref{eqn:gcon}, we introduce a Lagrangian function:
%\begin{equation}
%L=\int_S \left(\gamma_{\rm gb}+\sum_{j=1}^{J}\lambda_j\left(\frac{\partial \eta_{ju}}{\partial v}-\frac{\partial \eta_{jv}}{\partial u}\right)\right)  \mathrm{d}S,
%\end{equation}
%where $\lambda_j$, $j=1,2,\cdots,J$, are Lagrange multipliers associated with the constraints.
%The evolution of dislocation structure in  Eq.~\eqref{eqn:mod2en} now becomes
%\begin{flalign}
%&\frac{\partial\eta_{ju} }{\partial t}= -M_\mathrm{r}\frac{\partial \gamma_{\rm gb}}{\partial \eta_{ju}}+\frac{\partial\lambda_{j}}{\partial v},\\
%&\frac{\partial\eta_{jv} }{\partial t}=-M_\mathrm{r}\frac{\partial \gamma_{\rm gb}}{\partial \eta_{jv}}-\frac{\partial\lambda_{j}}{\partial u}.
%\end{flalign}
%Here the coefficients of $\frac{\partial\lambda_{j}}{\partial v}$ and $\frac{\partial\lambda_{j}}{\partial u}$ in these equations are set to be $1$.
%
%During the evolution in the time step from $t_n$ to $t_{n+1}=t_n+\delta t$, we separate the contributions from $\gamma_{\rm gb}$ and $\lambda_j$ into two steps:
%\begin{flalign}
%&\eta_{ju}^* = \eta_{ju}^n-M_{\rm r}\left.\frac{\partial \gamma_{\rm gb}}{\partial \eta_{ju}}\right|_{t_n}\cdot \delta t, \ \ \eta_{jv}^* = \eta_{jv}^n-M_{\rm r}\left.\frac{\partial \gamma_{\rm gb}}{\partial \eta_{jv}}\right|_{t_n}\cdot \delta t, \vspace{1ex} \label{eqn:dp1}\\
%&  \eta_{ju}^{n+1}=\eta_{ju}^*+\frac{\partial\lambda_{j}^{n+1}}{\partial v}\delta t, \ \ \eta_{jv}^{n+1}=\eta_{jv}^*-\frac{\partial\lambda_{j}^{n+1}}{\partial u}\delta t. \label{ddd}
%\end{flalign}
%In order to satisfy the constraint $\frac{\partial \eta_{ju}^{n+1}}{\partial v}-\frac{\partial \eta_{jv}^{n+1}}{\partial u}=0 $, using Eq.~\eqref{ddd}, we have the formula for updating $\lambda_j$:
%\begin{equation}
%\bigtriangleup \lambda_{j}^{n+1}=\frac{1}{dt}\left(\frac{\partial \eta_{jv}^{*}}{\partial u}- \frac{\partial \eta_{ju}^{*}}{\partial v}\right), \label{eqn:dp2}
%\end{equation}
%where $\bigtriangleup$ is the Laplace operator.
%
%Combining this numerical algorithm with that for the implementation of the continuum model with the constraint of Frank's formula given in Eqs.~\eqref{eqn:v1}--\eqref{eqn:v3},
%we have the numerical algorithm for the well-posed continuum model with constraints in Eqs.~\eqref{eqn:mod2vn}--\eqref{eqn:gcon}:
%
%
%\vspace{0.1in}
%\noindent
%\underline{\bf Numerical Algorithm}
%\vspace{0.05in}
%
%From $t_n$ to $t_{n+1}=t_n+\delta t$,
%\begin{flalign}
%\mathbf v^*=&\left(M_{\rm d}\sum_{j=1}^J\frac{\|\nabla_S\eta_j\|}{\sum_{k=1}^J \|\nabla_S\eta_k\| }\mathbf f_{\rm local}^{(j)}\cdot \mathbf n \right) \mathbf n,\label{eqn:nav1}\\
%\mathbf r^*=&\mathbf r^n +\mathbf v^* \delta t,\\
%\eta^*_{ju}=&\eta^{n}_{ju} -\left.M_{\rm r}\frac{\partial \gamma_{\rm gb}}{\partial \eta_{ju}} \right|_{t_n} \delta t, \ \ \eta^*_{jv}=\eta^{n}_{jv} -\left.M_{\rm r}\frac{\partial\gamma_{\rm gb}}{\partial\eta_{jv}}\right|_{t_n} \delta t, \label{eqn:mod3en}\\
%\bigtriangleup \lambda_{j}^{n+1}=&\frac{1}{\delta t}\left(\frac{\partial \eta_{jv}^{*}}{\partial u}- \frac{\partial \eta_{ju}^{*}}{\partial v}\right),\label{eqn:Poisson}\\
%  \eta_{ju}^{n+1}=&\eta_{ju}^*+\frac{\partial\lambda_{j}^{n+1}}{\partial v}\delta t, \ \ \eta_{jv}^{n+1}=\eta_{jv}^*-\frac{\partial\lambda_{j}^{n+1}}{\partial u}\delta t,\\
%\delta \theta=&\theta(\mathbf r^{*}_u ,\mathbf r^{*}_v,\eta^{n+1}_{ju},\eta^{n+1}_{jv}  )-\theta(\mathbf r^{n}_u ,\mathbf r^{n}_v,\eta^{n}_{ju},\eta^{n}_{jv}),\label{eqn:deltatheta3}\\
%\mathbf v=&\left(-\frac{\delta \theta}{\theta \delta t}(x-c_1),-\frac{\delta \theta}{\theta \delta t}(y-c_2),  v^*_3\right)+\left(-\frac{1}{\theta}\sum_{j=1}^J  b^{(j)}_2\frac{\delta \eta_{j}}{\delta t},\frac{1}{\theta}\sum_{j=1}^J  b^{(j)}_1\frac{\delta \eta_{j}}{\delta t},0\right), \label{eqn:fv2}\\
%\mathbf r^{n+1}=&\mathbf r^n +\mathbf v\rm \delta t.\label{eqn:nav2}
%\end{flalign}
%
%Recall that in this numerical formulation, $\mathbf{f}^{(j)}_{\mathrm{local}}$ is the local force on the grain boundary given in Eqs.~\eqref{eqn:localf}--\eqref{eqn:fp}, $\gamma_{\rm gb}$ is the local grain boundary energy given in Eq.~\eqref{eqn:gamma_gb}, the misorientation angle $\theta$ is calculated using Eq.~\eqref{eqn:mod1t}, and the constants $c_1$ and $c_2$ in the projected velocity $\mathbf v$ is determined numerically by the condition that the projection procedure alone does not lead to extra rigid translation of the grain boundary as discussed before (after Eq.~\eqref{eqn:v3}). The Poisson equation in \eqref{eqn:Poisson} can be solved using a finite difference method.
%
%\vspace{0.1in}
%\noindent
%\underline{\bf Initial dislocation structure}
%\vspace{0.05in}
%
%
%We assume that the initial grain boundary has an
%equilibrium dislocation structure that satisfies the Frank's formula and has the lowest energy. See Ref.~\cite{Qin2020} for  the method based on constrained energy minimization to find the equilibrium dislocation structure on a curved low angle grain boundary, which is a generalization of the model for planar low angle grain boundaries \cite{zhang2017energy} examined extensively by comparisons with atomistic simulation results.
%
%\subsection{Comparison with motion by mean curvature}
%The classical grain boundary dynamics models are based upon the motion by mean curvature (or curvature flow) to reduce the total grain boundary energy, with fixed misorientation angle and energy density of the grain boundary \cite{Herring1951,mullins1956two,Sutton1995,Giga2006}. Using the notation $E_{\rm local}$ for the total (local) grain boundary energy, a general, anisotropic curvature flow is
%\begin{equation}
%v_n=-M\frac{\delta E_{\rm local}}{\delta r},
%\end{equation}
%where $v_n$ is the velocity in the normal direction of the grain boundary, $r$ is the signed distance in the normal direction, and $M>0$.
%
%On the other hand, consider  our continuum model in the form of constrained evolution in Eqs.~\eqref{eqn:mod2v}--\eqref{eqn:mod2frank}  (which has a general form for the source term due to dislocation reaction). It can be seen that our model has the same grain boundary normal velocity as that in the curvature flow (if $M=\frac{M_{\rm d}}{\sum_{k=1}^J \|\nabla_S\eta_k\| }$, using Eq.~\eqref{eqn:var_r}), and in addition to that, our model also has a constraint due to Frank's formula and evolution of the dislocation structure on the grain boundary. The misorientation angle $\theta$ is fixed during the evolution of the grain boundary by curvature flow; whereas in our model, $\theta$ also evolves and is determined by the grain boundary profile and dislocation structure at any moment (by Eq.~\eqref{eqn:mod1t}), and especially, $\theta$ and the grain boundary energy density $\gamma_{\rm gb}$ may increase during the evolution. (See numerical examples in the next section.)
%
%Our continuum model can be reduced to the classical curvature flow in a special case where evolutions of the grain boundary and the dislocation structure are balanced to keep a constant misorientation angle $\theta$. We use the numerical formulation using projection method in Eqs.~\eqref{eqn:v1}--\eqref{eqn:v3} for this comparison. Now misorientation angle $\theta$ is fixed, i.e., $\delta \theta=0$. Under this condition, the projected velocity formulation in Eq.~\eqref{eqn:v2} becomes
%\begin{equation}
%\mathbf v=\left(-\frac{1}{\theta}\sum_{j=1}^J  b^{(j)}_2s_j, \ \frac{1}{\theta}\sum_{j=1}^J  b^{(j)}_1s_j, \ v^*_3 \right).
%\end{equation}
%It reduces to the velocity of the curvature flow, i.e., $\mathbf v^*$, if the dislocation reaction terms $\{s_j\}$ satisfy $-\frac{1}{\theta}\sum_{j=1}^J  b^{(j)}_2s_j=v_1^*$ and $\frac{1}{\theta}\sum_{j=1}^J  b^{(j)}_1s_j=v_2^*$.




\section{Numerical simulations of bcc spherical grain boundaries with rotation axis [011]}\label{sec:nr}
In this section, we perform numerical simulations of grain boundary dynamics using our numerical algorithm presented in the main text.

%\subsection{fcc spherical grain boundaries with rotation axis [111]}\label{sec:fcc}
%We first consider grain boundaries in fcc Al. We choose the directions $[\bar{1}10]$, $[\bar{1}\bar{1}2]$, $[111]$ to be the $x$, $y$ and $z$ directions, respectively. In this coordinate system, the six Burgers vectors are $\mathbf{b}^{(1)}=(1,0,0)b$, $\mathbf{b}^{(2)}=\left(\frac{1}{2},\frac{\sqrt{3}}{2},0\right)b$, $\mathbf{b}^{(3)}=\left(\frac{1}{2},-\frac{\sqrt{3}}{2},0\right)b$,
%$\mathbf{b}^{(4)}=\left(0,\frac{\sqrt{3}}{3},-\frac{\sqrt{6}}{3}\right)b$,
%$\mathbf{b}^{(5)}=\left(\frac{1}{2},\frac{\sqrt{3}}{6},\frac{\sqrt{6}}{3}\right)b$, and
%$\mathbf{b}^{(6)}=\left(-\frac{1}{2},\frac{\sqrt{3}}{6},\frac{\sqrt{6}}{3}\right)b$, where $b$ is the magnitude of the Burgers vectors. In Al, $b=0.286\rm nm$ and the Poisson ratio is $\nu=0.347$. The rotation axis $\mathbf a$ is in the $[111]$ direction, i.e., $+z$ direction.
%
%We study the evolution of an initially spherical grain boundary, whose radius is $R=20b$ and misorientation angle is $\theta = 5^\circ $.
%There are three sets of dislocations with Burgers vectors $\mathbf b^{(1)}$, $\mathbf{b}^{(2)}$, and $\mathbf{b}^{(3)}$, respectively, in the equilibrium dislocation structure on this initial, spherical grain boundary; see the top image in Fig.~\ref{fig:fccfigure0}(a).
%
%%\begin{figure}[htbp]
%%	\centering
%%	\includegraphics[width=3in]{figures/fcc3D.eps}
%%	\caption{The equilibrium dislocation structure on the initial, spherical grain boundary. The radius of this grain boundary is $R=20b$, and the misorientation angle is $\theta = 5^\circ $. Dislocations with Burgers vector $\mathbf b^{(1)}, \mathbf b^{(2)}, \mathbf b^{(3)}$ are shown by blue, black and red lines.  Length unit: $b$.}\label{fig:fcc3D}
%%\end{figure}
%
%
%In the dynamics simulation, the grain boundary is parameterized using spherical coordinates $R=R(\alpha,\beta)$, for $0\leq \alpha < 2\pi$ and $0\leq \beta\leq \pi$. Here $\alpha$ is the angle between the position vector of a point on the grain boundary and the $x$ axis, and $\beta$ is the angle between the position vector of the point  and the $z$ axis.
%Initially, $R(\alpha,\beta)=20b$. The $(\alpha,\beta)$ domain is discretized into $40\times20$ uniform grids during the evolution. The center of the spherical grain boundary is the origin $(0,0,0)$ in the coordinate system. Due to symmetry, the two constants $c_1=c_2=0$ in the projected velocity formula in Eq.~\eqref{eqn:fv2}.
%%we update the points $R(\alpha,\beta)$ on the grain boundary with this uniform mesh of the $(\alpha, \beta)$ domain fixed.
%
%\subsubsection{Pure coupling motion} \label{subsec:coup}
%We first consider the grain boundary motion without dislocation reaction, i.e. the reaction mobility $M_{\rm r}=0$ in Eq.~\eqref{eqn:mod3en} (and Eq.~\eqref{eqn:mod2en}), and accordingly $\delta \eta_j=0$ in Eq.~\eqref{eqn:fv2}. This is the pure coupling motion.
%
%
%\begin{figure}[htbp]
%	\centering
%	\centering
%	\subfigure[]{
%		\begin{minipage}{0.21\linewidth}
%			\includegraphics[width=\linewidth]{figures/refcc3D1.eps}
%			\includegraphics[width=\linewidth]{figures/refcc11.eps}
%			\includegraphics[width=\linewidth]{figures/refcc21.eps}
%		\end{minipage}
%	}
%	\subfigure[]{
%		\begin{minipage}{0.21\linewidth}			
%			\includegraphics[width=\linewidth]{figures/fcc3D200.eps}
%			\includegraphics[width=\linewidth]{figures/fcc1200.eps}
%			\includegraphics[width=\linewidth]{figures/fcc2200.eps}
%		\end{minipage}
%	}	
%	\subfigure[]{
%		\begin{minipage}{0.21\linewidth}			
%			\includegraphics[width=\linewidth]{figures/fcc3D300.eps}
%			\includegraphics[width=\linewidth]{figures/fcc1300.eps}
%			\includegraphics[width=\linewidth]{figures/fcc2300.eps}
%		\end{minipage}
%	}	
%	\subfigure[]{
%		\begin{minipage}{0.21\linewidth}			
%			\includegraphics[width=\linewidth]{figures/fcc3D400.eps}
%			\includegraphics[width=\linewidth]{figures/fcc1400.eps}
%			\includegraphics[width=\linewidth]{figures/fcc2400.eps}
%		\end{minipage}
%	}
%	\caption{Shrinkage of an initially spherical grain boundary in fcc under pure coupling motion, i.e., without dislocation reaction. The rotation axis is the $z$ direction ($[111]$), and the initial misorientation angle $\theta=5^\circ$. The upper panel of images show the three-dimensional view of the grain boundary during evolution. The middle panel of images show the grain boundary during evolution viewed from the $+z$ direction ($[111]$), and the lower panel of images show the grain boundary during evolution viewed from the $+x$ direction ($[\bar{1}10]$). Dislocations with Burgers vectors $\mathbf b^{(1)}$, $\mathbf b^{(2)}$ and $\mathbf b^{(3)}$ are shown by blue, black and red lines, respectively. Length unit: $b$. (a) The initial spherical grain boundary. (b), (c), and (d) Configurations at time $t=10/M_{\rm d}\mu, 15/M_{\rm d}\mu, 20/M_{\rm d}\mu$, respectively.}\label{fig:fccfigure0}
%\end{figure}
%
%
%Fig.~\ref{fig:fccfigure0} shows the shrinkage of the spherical grain boundary under this pure coupling motion. The grain boundary eventually disappears. In this case, since $M_{\rm r}=0$ and $\delta \eta_j=0$, the grain boundary velocity in Eq.~\eqref{eqn:fv2} becomes $\mathbf v =-\frac{\delta \theta}{\theta\delta t}(x,y,0)+(0,0, v^*_3)$. In the direction normal the rotation axis, i.e., in the $xy$ plane, the velocity component is in the inward radial direction, as in the two-dimensional model \cite{zhang2018motion,zhang2019new}; this is adjusted from the velocity component due to curvature flow in order to satisfy the Frank's formula. Whereas in the direction of the rotation axis, i.e., the $z$ direction, there is no constraint imposed by the Frank's formula, and the velocity component is
%the same as that in the curvature flow.
%
%As an example, we consider the cross-section of the grain boundary with the $z=0$ plane (i.e., cross-section normal to the $[111]$ rotation axis), which is the equator of the grain boundary in the three dimensional view in the upper panel in Fig.~\ref{fig:fccfigure0} and is a circle (the outer circle) as shown  in the second panel in Fig.~\ref{fig:fccfigure0} for the view from $+z$ direction. The grain boundary along this circular cross-section is pure tilt, which is similar to the two-dimensional grain boundary discussed in Ref.~\cite{zhang2018motion,zhang2019new}. Along this circle, during the evolution, we have $\mathbf v^*_3=0$, and the grain boundary velocity is $\mathbf v =-\frac{\delta \theta}{\theta\delta t}(x,y,0)$, which is completely in the inward radial direction in the $z=0$ plane. Thus the cross-section keeps the circular shape as it shrinks during the evolution, as shown in the  second panel in Fig.~\ref{fig:fccfigure0}.
%This shape-preserving evolution agrees with the results of the two-dimensional grain boundary dynamics models \cite{Taylor2007,zhang2018motion,zhang2019new} and shrinkage of circular grain boundaries in two dimensions by molecular dynamics \cite{srinivasan2002challenging} and phase field crystal \cite{wu2012phase} simulations. However, here the changing rate of misorientation angle $\frac{\delta \theta}{\delta t}$ in the velocity formula is depending on the entire grain boundary in three dimensions by Eq.~\eqref{eqn:deltatheta3}, and is not just depending on the circular cross-section itself as in the two dimensional continuum model \cite{zhang2019new}.
%
%Next, we consider the cross-section of the grain boundary with the $x=0$ plane (i.e., cross-section normal to the $[\bar{1}10]$ direction); see the lower panel in Fig.~\ref{fig:fccfigure0} (the outer boundary of the projected grain boundary surface).
%Initially, the cross-section is a circle, and  it gradually changes to an ellipse as it shrinks during the evolution. This shows that the velocity in the rotation axis direction,  i.e. $z$ direction is larger than that in the $x$ and $y$ directions. The reason for this is that there is no constraint of Frank's formula in the $z$ direction which is the direction of the rotation axis, and the velocity at the  two poles on the grain boundary with respect to the $z$ direction (where the grain boundary is pure twist) is the same as that in the curvature flow; whereas the velocity components in the $x$ and $y$ directions are adjusted from those in the curvature flow by the constraint of the Frank's formula, and the resulting velocity in the $xy$ plane are depending on the entire grain boundary through the coefficient $\delta \theta$, as discussed above.
%
%Evolution of this initially spherical grain boundary and its dislocation structure, especially the property that the shrinkage of the grain boundary is faster in the direction of the rotation axis than in other directions, agree with the results of atomistic simulations using phase field crystal models \cite{yamanaka2017phase,salvalaglio2018defects}.
%Notice that
% with respect to the rotation axis of the $z$ direction,  at the two poles on the grain boundary,  the grain boundary is pure twist, and the constituent dislocations are screw dislocations;
%along the equator of the grain boundary, the grain boundary is pure tilt, and the dislocations are edge dislocations. It was suggested in Ref.~\cite{yamanaka2017phase} that such an anisotropic motion may be due to the difference in dislocation densities or that in the mobilities of screw and edge dislocations.  Here, our continuum model provides a further explanation that this anisotropic motion of the grain boundary (and accordingly, the anisotropic motion of the screw and edge portions of the constituent dislocations) is due to the constraint of Frank's formula in order to maintain an equilibrium dislocation structure.
%
%%  This results agrees with the high temperature case of bcc iron spherical grain boundary in \cite{yamanaka2017phase} and the fcc, bcc spherical grain boundaries in \cite{salvalaglio2018defects} obtained using phase-field crystal model.
%
%Fig. \ref{fig:fcctheta0}(a) shows the change of misorientation angle $\theta$ during the evolution, which is continuously  increasing. This behavior agrees with Cahn-Taylor theory \cite{cahn2004unified}, three dimensional phase-field crystal simulations \cite{yamanaka2017phase}, and  two-dimensional atomistic \cite{srinivasan2002challenging,trautt2012grain}, phase field crystal \cite{wu2012phase}, and continuum \cite{zhang2018motion,zhang2019new} simulations. Such increasing  of misorientation angle cannot be obtained by the classical motion by mean curvature models or pure sliding models, in which the misorientation angle is constant or is decreasing during the evolution.
%
%
%
%\begin{figure}[htbp]
%	\centering
%	\subfigure[]{\includegraphics[width=2.5in]{figures/fccangle.eps}}
%	\subfigure[]{\includegraphics[width=2.5in]{figures/fccarea.eps}}
%	\subfigure[]{\includegraphics[width=2.5in]{figures/fccdensity.eps}}
%	\subfigure[]{\includegraphics[width=2.5in]{figures/fcclength.eps}}
%	\caption{Shrinkage of an initially spherical grain boundary in fcc under pure coupling motion. The rotation axis is the $z$ direction ($[111]$), and the initial misorientation angle $\theta=5^\circ$.  (a) Evolution of  misorientation angle $\theta$. (b) Evolution of grain boundary area $S_A$, where $S_A^0$ is the area of the initial grain boundary. (c) Evolution of density of dislocations with Burgers vector $\mathbf b^{(1)}$/ $\mathbf b^{(2)}$/$\mathbf b^{(3)}$ on the grain boundary. (d) Evolution of the total length of dislocations with Burgers vector $\mathbf b^{(1)}$/ $\mathbf b^{(2)}$/$\mathbf b^{(3)}$ on the grain boundary. In (c) and (d), the densities and total lengths of dislocations with these three Burgers vectors are almost identical. }\label{fig:fcctheta0}
%\end{figure}	
%
%
%Fig.~\ref{fig:fcctheta0}(b) shows evolution of the area of the grain boundary, which  reveals the relation:
%\begin{equation}
%\frac{S_A(t)}{S_A^0}=1-At,
%\label{eqn:area}
%\end{equation}
%where $A$ is some constant, and $S_A(t)$ and $S_A^0$ are the grain boundary area at time $t$ and that of the initial configuration, respectively. This agrees with the results of nearly linear decrease of the grain boundary area using phase field crystal model for grain boundaries in both fcc and bcc crystals \cite{salvalaglio2018defects}.
%The phase field crystal simulations in Ref.~\cite{yamanaka2017phase} showed that the decrease of the volume of the grain enclosed by an initially spherical low angle grain boundary in a bcc crystal approximately follows the relation $\frac{V^{2/3}(t)}{V_0^{2/3}}=1- A_1t$, where $A_1$ is some constant,   and $V(t)$ and $V_0$ are the volume of the grain enclosed by the grain boundary at time $t$ and that of the initial configuration, respectively. It was argued in  Ref.~\cite{yamanaka2017phase} that their results are consistent with the result of classical Von Neumann-Mullins relation \cite{mullins1956two} for a two dimensional grain boundary driven by curvature with constant energy density, i.e., Eq.~\eqref{eqn:area} if $S_A$ denotes the area enclosed by the grain boundary in two dimensions, considering the approximate relation $V^{2/3}\sim S_A$.
%In this sense, simulation results using our continuum model and the phase field crystal simulations in \cite{salvalaglio2018defects} are consistent with the results in  Ref.~\cite{yamanaka2017phase} as well as the result of the classical Von Neumann-Mullins relation. The nearly linear decrease of the grain boundary area in Eq.~\eqref{eqn:area} obtained by our continuum model and the phase field crystal model in Ref.~\cite{salvalaglio2018defects} is also in consistent with the result that the area enclosed by a two dimensional grain boundary is linearly decreasing in the two dimensional phase field crystal simulations for circular grain boundaries \cite{wu2012phase} and continuum model simulations for circular \cite{zhang2018motion} and general shape \cite{zhang2019new} grain boundaries in two dimensions.
%
%
%
%%%\begin{comment}
%%We approximately rewrite the area formula in Eq. \eqref{eqn:area} as $\frac{dR}{dt} \simeq -\frac{A}{2R}\simeq v_{\perp} $. A the grain boundary keeps shrinkage, the radius $R$ is decreasing which implies the normal velocity is increasing. This relation explains the increasing tangential velocity in Fig. \ref{fig:fcctheta0} (a).
%%%\end{comment}
%
%Evolutions of dislocation densities on the grain boundary and total length of dislocations are shown in Figs.~\ref{fig:fcctheta0}(c) and (d). It can be seen from Fig.~\ref{fig:fcctheta0}(c) that the densities of the dislocations with all the three Burgers vectors are increasing during the evolution. This is consistent with the increase of misorientation angle $\theta$ during the evolution. Recall that in the two dimensional case with dislocation conservation, the inter-dislocation distance is decreasing as the grain boundary shrinks,  which leads to the increases of both the dislocation densities and the misorientation angle during the evolution \cite{cahn2004unified}.
%The total length of dislocations is decreasing during the evolution as shown in Fig.~\ref{fig:fcctheta0}(d). This is in agreement with the phase field crystal simulation results in Ref.~\cite{yamanaka2017phase}.
%Unlike in the two dimensional case with dislocation conservation \cite{srinivasan2002challenging,cahn2004unified,trautt2012grain,wu2012phase,zhang2018motion,zhang2019new}, in three dimensions without dislocation reaction, the grain boundary and its constituent dislocations (which are closed loops instead of infinite, straight lines in the two dimensional case) are moving freely in the direction of the rotation axis, although the motion is subject to Frank's formula in a direction normal to the rotation axis. As a result, although along the equator of the grain boundary, the total number of dislocations does not change during the evolution, all the dislocation loops are shrinking and the total length of dislocations is decreasing as the grain boundary shrinks.
%
%%\newpage
%\subsubsection{Motion with dislocation reaction}\label{subsec:fccreact}
%
%Now we perform simulations using our continuum model considering dislocation reaction, i.e. $M_{\rm r}\neq 0$. Dislocation reaction leads to removal of dislocations,  resulting in the coupling motion of the grain boundary \cite{srinivasan2002challenging,cahn2004unified,trautt2012grain,yamanaka2017phase,zhang2018motion,zhang2019new}. We use the same initial spherical grain boundary as in Sec.~\ref{subsec:coup} without dislocation reaction.
%
%\begin{figure}[htbp]
%	\centering
%	\centering
%	\subfigure[]{
%		\begin{minipage}{0.21\linewidth}
%			\includegraphics[width=\linewidth]{figures/refcc3D1.eps}
%			\includegraphics[width=\linewidth]{figures/refcc11.eps}
%			\includegraphics[width=\linewidth]{figures/refcc21.eps}
%		\end{minipage}
%	}
%	\subfigure[]{
%		\begin{minipage}{0.21\linewidth}			
%			\includegraphics[width=\linewidth]{figures/refcc3D200.eps}
%			\includegraphics[width=\linewidth]{figures/refcc1200.eps}
%			\includegraphics[width=\linewidth]{figures/refcc2200.eps}
%		\end{minipage}
%	}	
%	\subfigure[]{
%		\begin{minipage}{0.21\linewidth}			
%			\includegraphics[width=\linewidth]{figures/refcc3D300.eps}
%			\includegraphics[width=\linewidth]{figures/refcc1300.eps}
%			\includegraphics[width=\linewidth]{figures/refcc2300.eps}
%		\end{minipage}
%	}	
%	\subfigure[]{
%		\begin{minipage}{0.21\linewidth}			
%			\includegraphics[width=\linewidth]{figures/refcc3D400.eps}
%			\includegraphics[width=\linewidth]{figures/refcc1400.eps}
%			\includegraphics[width=\linewidth]{figures/refcc2400.eps}
%		\end{minipage}
%	}
%	\caption{Shrinkage of an initially spherical grain boundary in fcc with dislocation reaction: $M_{\rm r}b^3/M_{\rm d}=1.83\times10^{-4}$. The rotation axis is the $z$ direction ($[111]$), and the initial misorientation angle $\theta=5^\circ$. The upper panel of images show the three-dimensional view of the grain boundary during evolution. The middle panel of images show the grain boundary during evolution viewed from the $+z$ direction ($[111]$), and the lower panel of images show the grain boundary during evolution viewed from the $+x$ direction ($[\bar{1}10]$). Dislocations with Burgers vectors $\mathbf b^{(1)}$, $\mathbf b^{(2)}$ and $\mathbf b^{(3)}$ are shown by blue, black and red lines, respectively. Length unit: $b$. (a) The initial spherical grain boundary. (b), (c), and (d) Configurations at time $t=10/M_{\rm d}\mu, 15/M_{\rm d}\mu, 20/M_{\rm d}\mu$, respectively.}\label{fig:fccfigure}
%\end{figure}
%
%
%
%Fig.~\ref{fig:fccfigure} shows the shrinkage of the initially spherical grain boundary with dislocation reaction, where the reaction mobility  $M_{\rm r}b^3/M_{\rm d}=1.83\times10^{-4}$.
%We consider the cross-section of the grain boundary with the $z=0$ plane (i.e., cross-section normal to the $[111]$ rotation axis), which is the equator of the grain boundary in the three dimensional view in the upper panel in Fig.~\ref{fig:fccfigure} and the outer curve in the view from the $+z$ axis in
% the second panel in Fig.~\ref{fig:fccfigure}.
%Along this curve, the grain boundary is pure tilt everywhere, and we have $ v^*_3=0$, i.e., the velocity is always in the $z=0$ plane during the evolution.  The evolution of this curve is similar to that of the two-dimensional grain boundary discussed in \cite{zhang2018motion,zhang2019new}.  The initial circular cross-section gradually changes to a hexagonal shape as it shrinks. Each edge in this hexagon is pure tilt that consists of dislocations of only one Burgers vector. This behavior is consistent with the fact that the energy density of the grain boundary is anisotropic and the pure tilt boundary has the minimum energy of all tilt boundaries, and is the same as the evolution of two dimensional grain boundary with dislocation reaction obtained in \cite{zhang2018motion,zhang2019new}.
%
%The lower panel of Fig.~\ref{fig:fccfigure} shows the evolution of the grain boundary in the view from the $+x$ direction ($[\bar{1}10]$ direction). The cross-section of the grain boundary with the $x=0$ plane gradually changes to an ellipse as it shrinks, meaning that the grain boundary moves faster in the rotation axis direction than in a direction normal to the rotation axis. This behavior and the underlying reason are the same as those in the simulation without dislocation reaction shown and discussed in Sec.~\ref{subsec:coup}.
%
%
%These behaviors of the evolution of the initially spherical grain boundary with dislocation reaction in an fcc crystals are similar to the phase field crystal simulation results of an initially spherical  grain boundary in a bcc crystal~\cite{yamanaka2017phase}.
%
%
%\begin{figure}[htbp]
%	\centering
%	\subfigure[]{\includegraphics[width=2.5in]{figures/fccreangle.eps}}
%	\subfigure[]{\includegraphics[width=2.5in]{figures/fccrearea.eps}}
%	\subfigure[]{\includegraphics[width=2.5in]{figures/fccredensity.eps}}
%	\subfigure[]{\includegraphics[width=2.5in]{figures/fccrelength.eps}}
%	\caption{Shrinkage of an initially spherical grain boundary in fcc with different values of reaction mobility $M_{\rm r}$. The rotation axis is the $z$ direction ($[111]$), and the initial misorientation angle $\theta=5^\circ$.   The reaction mobility $M_{\rm r}b^3/M_{\rm d}=0$, $9.16\times10^{-5}$, $1.83\times10^{-4}$, and $3.74\times10^{-4}$ from the top curve to the bottom one in (a), (c) and (d), and from the bottom to the top ones in (b).
%(a) Evolution of  misorientation angle $\theta$. (b) Evolution of grain boundary area $S_A$, where $S_A^0$ is the area of the initial grain boundary. (c) Evolution of density of dislocations with Burgers vector $\mathbf b^{(1)}$/ $\mathbf b^{(2)}$/$\mathbf b^{(3)}$ on the grain boundary. (d) Evolution of the total length of dislocations with Burgers vector $\mathbf b^{(1)}$/ $\mathbf b^{(2)}$/$\mathbf b^{(3)}$ on the grain boundary. In (c) and (d), the densities and total lengths of dislocations with these three Burgers vectors are almost identical.  }\label{fig:fcctheta}
%\end{figure}
%
%
%Evolution of the misorietation angle $\theta$ with different values of reaction mobility $M_{\rm r}$ is shown in Fig.~\ref{fig:fcctheta}(a). When $M_{\rm r}\neq 0$, the evolution of misorientation angle is controlled by both the coupling effect and sliding effect. As can be seen from  Fig.~\ref{fig:fcctheta}(a), the misorientation angle $\theta$ is increasing during the evolution except for the case with very high dislocation reaction mobility;
%  as the dislocation reaction mobility $M_{\rm r}$ increases, meaning the sliding effect due to dislocation reaction is becoming stronger, the increase rate of $\theta$ decreases, and when the sliding effect is strong enough, the misorientation angle $\theta$ is decreasing.  These properties are the same as those in the two-dimensional cases \cite{zhang2018motion,zhang2019new}: the coupling motion of grain boundary associated with the conservation of dislocations will increase the misorentation angle $\theta$ during the evolution, and the sliding motion generated by dislocation reaction will decrease $\theta$.
%
%Fig.~\ref{fig:fcctheta}(b) shows the evolution of grain boundary area with different values of dislocation reaction mobility $M_{\rm r}$. It can be seen that except for the case with very high dislocation reaction mobility, the decrease of grain boundary area still follows the linear law in Eq.~\eqref{eqn:area}, and is almost unchanged with different values of dislocation reaction mobility. In the case with very high dislocation reaction mobility $M_{\rm r}=  3.74\times10^{-4}M_{\rm d}/b^3$, the decrease of grain boundary area starts to deviate from the linear law with slower deceasing rate, which is due to the resulting significant decrease in the grain boundary energy density that slows down the shrinking of the grain boundary. Again, the linear decrease of grain boundary area is consistent with the available phase field crystal simulation results \cite{yamanaka2017phase,salvalaglio2018defects} and results of two dimensional simulations using different methods as discussed in Sec.~\ref{subsec:coup} for the case without dislocation reaction.
%
%
%
%
%Evolutions of dislocation densities on the grain boundary and total length of dislocations with different values of reaction mobility $M_{\rm r}$ are shown in Figs.~\ref{fig:fcctheta}(c) and (d). As can be seen from  Fig.~\ref{fig:fcctheta}(c),  the densities of the dislocations with all the three Burgers vectors are increasing during the evolution except for the case with very high dislocation reaction mobility;
%as the dislocation reaction mobility $M_{\rm r}$ increases, the increase rate of dislocation densities decreases, and when the dislocation reaction mobility is high enough, the dislocation densities are decreasing.
% These behaviors are consistent with the increase of misorientation angle $\theta$ during the evolution shown in Fig.~\ref{fig:fcctheta}(a).
%Fig.~\ref{fig:fcctheta}(d) shows that
%the total length of dislocations is decreasing as the grain boundary shrinks, and the decrease rate is higher for higher dislocation reaction mobility  $M_{\rm r}$.
% The decrease of the total length of dislocations is in agreement with the phase field crystal simulation results in Ref.~\cite{yamanaka2017phase}.
%
%Note that detailed dislocation reaction mechanisms on a curved grain boundary in three dimensions have been analyzed in Ref.~\cite{yamanaka2017phase} based on their phase field crystal simulations.  There are also mechanisms observed/proposed  based two dimensional molecular dynamics simulations \cite{srinivasan2002challenging,trautt2012grain}. In general, the small dislocation loops and segments of dislocation networks on a curved grain boundary in three dimensions are easier to react than the infinite, straight dislocations in the two dimensional cases.



%\subsection{bcc spherical grain boundaries with rotation axis [011]}

Now we consider grain boundaries in bcc Fe. We choose the directions  $[100]$, $[01\bar 1]$ and $[011]$ to be the $x$, $y$ and $z$ directions, respectively. There seven possible Burgers vectors, and in this coordinate system, they are $\mathbf{b}^{(1)}=\left(\frac{1}{2},0,\frac{\sqrt{2}}{2}\right)a$, $\mathbf{b}^{(2)}=\left(\frac{1}{2},\frac{\sqrt{2}}{2},0\right)a$,
$\mathbf{b}^{(3)}=\left(\frac{1}{2},-\frac{\sqrt{2}}{2},0\right)a$,
$\mathbf{b}^{(4)}=\left(-\frac{1}{2},0,\frac{\sqrt{2}}{2}\right)a$,
$\mathbf{b}^{(5)}=(1,0,0)a$, $\mathbf{b}^{(6)}=\left(0,\frac{\sqrt{2}}{2},\frac{\sqrt{2}}{2}\right)a$, and
$\mathbf{b}^{(7)}=\left(0,-\frac{\sqrt{2}}{2},\frac{\sqrt{2}}{2}\right)a$, where $a$ is the lattice constant.  For bcc Fe, a$=0.2856 \rm nm$  and the Poisson ratio is $\nu=0.29$.  The rotation axis is $\mathbf{a}=(0,0,1)$, i.e., in the $[011]$ direction.

We also study the evolution of an initially spherical grain boundary, whose radius is $R=20b$ and misorientation angle  is $\theta = 4^\circ $.  There are three sets of dislocations with Burgers vectors $\mathbf b^{(2)}$, $\mathbf{b}^{(3)}$, $\mathbf{b}^{(5)}$, respectively,  in the equilibrium dislocation structure on this initial, spherical grain boundary; see the top image in Fig.~\ref{fig:bccfigure0}(a).
Parametrization and discretization of the grain boundary are the same as those for the grain boundary in fcc in the main text.


%
%The grain boundary is parameterized using spherical coordinates $R=R(\alpha,\beta)$ for $0\leq \alpha < 2\pi$, and $0\leq \beta\leq \pi$. Initially, $R$ is constant. The $(\alpha,\beta)$ domain is discretized into $40\times20$ grids with uniform. Here $\alpha$ is the angle between the position vector of points on grain boundary and the $x$ axis and $\beta$ is the angle between the position vector of points on grain boundary and the $z$ axis. During the evolution steps, we update the points $R(\alpha,\beta)$ on the grain boundary with this uniform mesh of the $(\alpha, \beta)$ domain fixed.
%%
%\begin{figure}[htbp]
%	\centering
%	\includegraphics[width=3in]{figures/bcc3D.eps}
%	\caption{The equilibrium dislocation structure with three sets of dislocations.}\label{fig:bcc3D}
%\end{figure}

\subsection{Pure coupling motion} \label{subsec:coupbcc}

We first consider the motion of this grain boundary  without dislocation reaction, i.e. the reaction mobility $M_{\rm r}=0$ in Eq.~(54) (and Eq.~(43)). This is the pure coupling motion.

%We firstly focus on the grain boundary motion with the conservation of the number of grain boundary dislocations. This means that the reaction mobility $M_r=0$, and the grain boundary motion is pure coupling motion. The initial misorientation angle is $\theta = 4^\circ $, the initial radius of the spherical grain boundary is $R=20b$.

\begin{figure}[htbp]
\centering
\subfigure[]{
	\begin{minipage}{0.21\linewidth}
		\includegraphics[width=\linewidth]{bcc3D1.pdf}
		\includegraphics[width=\linewidth]{bcc11.pdf}
		\includegraphics[width=\linewidth]{bcc21.pdf}
	\end{minipage}
}
\subfigure[]{
	\begin{minipage}{0.21\linewidth}			
		\includegraphics[width=\linewidth]{bcc3D300.pdf}
		\includegraphics[width=\linewidth]{bcc1300.pdf}
		\includegraphics[width=\linewidth]{bcc2300.pdf}
	\end{minipage}
}	
\subfigure[]{
	\begin{minipage}{0.21\linewidth}			
		\includegraphics[width=\linewidth]{bcc3D600.pdf}
		\includegraphics[width=\linewidth]{bcc1600.pdf}
		\includegraphics[width=\linewidth]{bcc2600.pdf}
	\end{minipage}
}	
\subfigure[]{
	\begin{minipage}{0.21\linewidth}			
		\includegraphics[width=\linewidth]{bcc3D900.pdf}
		\includegraphics[width=\linewidth]{bcc1900.pdf}
		\includegraphics[width=\linewidth]{bcc2900.pdf}
	\end{minipage}
}
	\caption{Shrinkage of an initially spherical grain boundary in bcc under pure coupling motion, i.e., without dislocation reaction. The rotation axis is the $z$ direction ($[011]$), and the initial misorientation angle $\theta=4^\circ$. The upper panel of images show the three-dimensional view of the grain boundary during evolution. The middle panel of images show the grain boundary during evolution viewed from the $+z$ direction ($[011]$), and the lower panel of images show the grain boundary during evolution viewed from the $+x$ direction ($[100]$). Dislocations with Burgers vectors $\mathbf b^{(2)}$, $\mathbf b^{(3)}$ and $\mathbf b^{(5)}$ are shown by blue, black and red lines, respectively. Length unit: $b$. (a) The initial spherical grain boundary. (b), (c), and (d) Configurations at time $t=3/M_{\rm d}\mu, 6/M_{\rm d}\mu, 9/M_{\rm d}\mu$, respectively.
}\label{fig:bccfigure0}
\end{figure}

Fig.~\ref{fig:bccfigure0} shows the shrinkage of this spherical grain boundary in bcc under the pure coupling motion.
Evolutions of the grain boundary and its dislocation structure are similar to those of the grain boundary in fcc discussed in the main text. Especially, the shrinkage of the grain boundary is faster in the direction of the rotation axis than in other directions, which agrees with the results of atomistic simulations using phase field crystal models \cite{yamanaka2017phase,salvalaglio2018defects}. Again, this anisotropic motion can be explained based on our continuum model by the constraint of Frank's formula in any direction normal to the rotation axis. The shape-preserving evolution of the equator of the grain boundary (with respect to the rotation axis) agrees with the results of the two-dimensional grain boundary dynamics models \cite{srinivasan2002challenging,Taylor2007,wu2012phase,zhang2018motion,zhang2019new}.




%Fig. \ref{fig:bccfigure0} shows the shrinkage of a spherical grain boundary under these conditions. In this case, since $M_r=0$, the grain boundary velocity is $\mathbf v =-\frac{\delta \theta}{\theta\delta t}(x,y,0)+(0,0, v^*_3)$ from Eq. \eqref{eqn:fv2}. If we focus on the circular  at $z=0$, which is the outer circular seen on the top column in Fig. \ref{fig:bccfigure0}, the grain boundary is pure tilt. On the spherical grain boundary, at $z=0$, we have $\mathbf v^*_3=0$, and the velocity is $\mathbf v =-\frac{\delta \theta}{\theta\delta t}(x,y,0)$, which is in the inward radial direction.
%
%Now we focus on the circle at $x=0$ on the grain boundary, and the lower column in Fig. \ref{fig:bccfigure0}. The circle whose evolution is shown in gradually changes to an ellipse during the evolution. This is because the velocity in the rotation axis direction,  i.e. $+z$ direction is much higher than the other two direction. If we view from $+z$ direction, the grain boundary at $x=0,y=0$ is pure twist, and the dislocations are screw dislocations. If we view from $+x$ direction, the grain boundary at $y=0,z=0$ is pure tilt,  and the dislocations are edge dislocations. The results indicate that the velocity of screw dislocations tends to be higher than that of the edge dislocation. This results agrees with the high temperature case of bcc iron spherical grain boundary in \cite{yamanaka2017phase} and the fcc, bcc spherical grain boundaries in \cite{salvalaglio2018defects} obtained using phase-field crystal model.

\begin{figure}[htbp]
	\centering
	\subfigure[]{\includegraphics[width=2.3in]{bccangle.pdf}}
	\subfigure[]{\includegraphics[width=2.3in]{bccarea.pdf}}
	\subfigure[]{\includegraphics[width=2.3in]{bccdensity.pdf}}
	\subfigure[]{\includegraphics[width=2.3in]{bcclength.pdf}}
	\caption{Shrinkage of an initially spherical grain boundary in bcc under pure coupling motion. The rotation axis is the $z$ direction ($[011]$), and the initial misorientation angle $\theta=4^\circ$.  (a) Evolution of  misorientation angle $\theta$. (b) Evolution of grain boundary area $S_A$, where $S_A^0$ is the area of the initial grain boundary. (c) Evolution of densities of dislocations on the grain boundary. These dislocations have Burgers vectors
 $\mathbf b^{(2)}$, $\mathbf b^{(3)}$, and $\mathbf b^{(5)}$ on the grain boundary. (d) Evolution of the total lengths of dislocations with these Burgers vectors.
 The densities and total lengths of dislocations with the Burgers vectors $\mathbf b^{(2)}$, $\mathbf b^{(3)}$, and $\mathbf b^{(5)}$  are shown by blue lines, black dots, and red lines, respectively, in (c) and (d).
}\label{fig:bcctheta0}
\end{figure}

Fig.~\ref{fig:bcctheta0}  shows the increase of the misorientation angle $\theta$, linear decrease of the grain boundary area, increase of densities of dislocations, and decrease of the total lengths of dislocations during the shrinkage of the grain boundary. These results are similar to those of the grain boundary in fcc discussed in the main text and are consistent with available phase field crystal and two dimensional simulation results; see the discussion there. Note that for the dynamics of this grain boundary in bcc, the densities and total lengths of dislocations with Burgers vectors $\mathbf b^{(2)}$ and $\mathbf b^{(3)}$ are almost identical, and are greater than those of dislocations with Burgers vector $\mathbf b^{(5)}$. Recall that the lengths of Burgers vectors  $\mathbf b^{(2)}$ and $\mathbf b^{(3)}$ are equal, and are smaller than the length of Burgers vector $\mathbf b^{(5)}$.



\subsection{Motion with dislocation reaction}

Now we perform simulations using our continuum model considering dislocation reaction, i.e. $M_{\rm r}\neq 0$, for the motion of a grain boundary in bcc Fe.  We use the same initial spherical grain boundary as in Sec.~\ref{subsec:coupbcc} without dislocation reaction.

\begin{figure}[htbp]
\centering
\subfigure[]{
	\begin{minipage}{0.21\linewidth}
		\includegraphics[width=\linewidth]{rebcc3D1.pdf}
		\includegraphics[width=\linewidth]{rebcc11.pdf}
		\includegraphics[width=\linewidth]{rebcc21.pdf}
	\end{minipage}
}
\subfigure[]{
	\begin{minipage}{0.21\linewidth}			
		\includegraphics[width=\linewidth]{rebcc3D300.pdf}
		\includegraphics[width=\linewidth]{rebcc1300.pdf}
		\includegraphics[width=\linewidth]{rebcc2300.pdf}
	\end{minipage}
}	
\subfigure[]{
	\begin{minipage}{0.21\linewidth}			
		\includegraphics[width=\linewidth]{rebcc3D600.pdf}
		\includegraphics[width=\linewidth]{rebcc1600.pdf}
		\includegraphics[width=\linewidth]{rebcc2600.pdf}
	\end{minipage}
}	
\subfigure[]{
	\begin{minipage}{0.21\linewidth}			
		\includegraphics[width=\linewidth]{rebcc3D900.pdf}
		\includegraphics[width=\linewidth]{rebcc1900.pdf}
		\includegraphics[width=\linewidth]{rebcc2900.pdf}
	\end{minipage}
}
	\caption{Shrinkage of an initially spherical grain boundary in bcc with dislocation reaction: $M_{\rm r}b^3/M_{\rm d}=7.01\times10^{-5}$. The rotation axis is the $z$ direction ($[011]$), and the initial misorientation angle $\theta=4^\circ$. The upper panel of images show the three-dimensional view of the grain boundary during evolution. The middle panel of images show the grain boundary during evolution viewed from the $+z$ direction ($[011]$), and the lower panel of images show the grain boundary during evolution viewed from the $+x$ direction ($[100]$). Dislocations with Burgers vectors $\mathbf b^{(2)}$, $\mathbf b^{(3)}$ and $\mathbf b^{(5)}$ are shown by blue, black and red lines, respectively. Length unit: $b$. (a) The initial spherical grain boundary. (b), (c), and (d) Configurations at time $t=3/M_{\rm d}\mu, 6/M_{\rm d}\mu, 9/M_{\rm d}\mu$, respectively.}\label{fig:bccfigure}
\end{figure}


Fig.~\ref{fig:bccfigure} shows the shrinkage of the initially spherical grain boundary with dislocation reaction, where the reaction mobility  $M_{\rm r}b^3/M_{\rm d}=7.01\times10^{-5}$. As in the evolution of the grain boundary in fcc with dislocation reaction shown in the main text, the equator of the grain boundary (with respect to the rotation axis) gradually evolves from a circle into a hexagon. However, unlike  in fcc, here the grain boundary in bcc evolves into a non-regular hexagon. This is due to the fact that here the lengths of Burgers vectors  $\mathbf b^{(2)}$ and $\mathbf b^{(3)}$ are smaller than the length of Burgers vector $\mathbf b^{(5)}$, while in the grain boundary in fcc, the three Burgers vectors have the same length.
Again, we can see that the shrinkage of the grain boundary is faster in the direction of the rotation axis than in other directions. These results  agree with the those of phase field crystal simulations \cite{yamanaka2017phase}.


\begin{figure}[htbp]
	\centering
	\subfigure[]{\includegraphics[width=2.3in]{bccreangle.pdf}}
	\subfigure[]{\includegraphics[width=2.3in]{bccrearea.pdf}}
	\subfigure[]{\includegraphics[width=2.3in]{bccredensity.pdf}}
	\subfigure[]{\includegraphics[width=2.3in]{bccrelength.pdf}}
	\caption{ Shrinkage of an initially spherical grain boundary in bcc with different values of reaction mobility $M_{\rm r}$. The rotation axis is the $z$ direction ($[011]$), and the initial misorientation angle $\theta=4^\circ$.   The reaction mobility $M_{\rm r}b^3/M_{\rm d}=0$, $7.01\times10^{-5}$, $1.40\times10^{-4}$, and $2.11\times10^{-4}$ from the top curve to the bottom one in (a), (c) and (d), and from the bottom to the top ones in (b).
(a) Evolution of  misorientation angle $\theta$. (b) Evolution of grain boundary area $S_A$, where $S_A^0$ is the area of the initial grain boundary. (c) Evolution of the density of all the dislocations on the grain boundary. (d) Evolution of the total length of all the dislocations on the grain boundary. }\label{fig:bcctheta}
\end{figure}

Evolution of the misorietation angle $\theta$ with different values of reaction mobility $M_{\rm r}$ is shown in Fig.~\ref{fig:bcctheta}(a). When $M_{\rm r}\neq 0$, the evolution of misorientation angle is controlled by both the coupling effect (which is associated with the conservation of dislocations and increases $\theta$) and sliding effect (which is associated with dislocation reaction and decreases $\theta$). While the misorientation angle $\theta$ is increasing during the evolution, the increase rate of $\theta$ decreases as
  the dislocation reaction mobility $M_{\rm r}$ increases.
Fig.~\ref{fig:bcctheta}(b) shows the evolution of grain boundary area with different values of dislocation reaction mobility $M_{\rm r}$. The decrease of grain boundary area still follows the linear law in Eq.~(62) in the main text except in the later stage of the evolution with high dislocation reaction mobility,  and is slower for higher dislocation reaction mobility.
Figs.~\ref{fig:bcctheta}(c) and (d) show the evolutions of dislocation densities on the grain boundary and total length of dislocations with different values of reaction mobility $M_{\rm r}$. The density of the dislocations on the grain boundary is increasing whereas the total length of dislocations is decreasing for these values of dislocation reaction mobility $M_{\rm r}$.
As the dislocation reaction mobility $M_{\rm r}$ increases, the increase rate of dislocation density  decreases, and the decrease rate of the total length of dislocations increases except for the later stage of evolution without dislocation reaction. These behaviors are similar to those of the grain boundary in fcc shown in  Fig.~5 in the main text, and more discussion can be found there. These results also agree with the available phase field crystal simulation results \cite{yamanaka2017phase,salvalaglio2018defects}.




%
%
%%\newpage
%\section{Conclusions}\label{sec:con}
%We have developed a continuum model for the dynamics of grain boundaries in three dimensions that incorporates the motion and reaction of the constituent dislocations. The continuum model includes evolution equations for both the motion of the grain boundary and the evolution of dislocation structure on the grain boundary. The evolution of orientation-dependent continuous distributions of dislocation lines on the grain boundary is based on the simple representation using dislocation density potential functions. This simple representation method also guarantees continuity of the dislocation lines on the grain boundaries during the evolution. The critical but computationally expensive long-range elastic interaction of dislocations is replaced by a projection formulation that maintains the constraint of the Frank's formula describing the equilibrium of the strong long-range interaction. This continuum model is able to describe the grain boundary motion and grain rotation due to both coupling and sliding effects, to which the classical motion by mean curvature model does not apply.
%
%In order to overcome the ill-posedness in formulation that comes from the nonconvexity of the energy density, we use the components of the surface gradients of the dislocation density potential functions instead of these functions directly. Relationship between the components of these surface gradients (i.e. continuity of dislocation lines) is maintained by the projection method during the evolution.
%
%Using the obtained continuum model, simulations are performed for the dynamics of initially spherical low angle grain boundaries in fcc Al and bcc Fe, under the conditions without dislocation reaction (pure coupling motion) and with dislocation reaction (with sliding motion). The simulations have shown increase of the misorientation angle as the grain boundary shrinks under the effect of conservation of dislocations, anisotropic motion in the directions along and normal the rotation axis, anisotropic motion in the normal plane with respect to the rotation axis due to dislocation reaction, and linear decrease of grain boundary area.  These results
%agree well with those of atomistic simulations (phase-field crystal simulations) \cite{yamanaka2017phase,salvalaglio2018defects}. The simulation results are also consistent with previously obtained results using continuum model in two dimensions \cite{zhang2018motion,zhang2019new}. In particular, we explain the anisotropic motion in the directions along and normal the rotation axis by the fact that the constraint of Frank's formula only has effect in a direction normal to the rotation axis, and the motion is free in the direction of the rotation axis.
%
%
%
%\section*{Acknowledgement}
%This work was supported by the Hong Kong Research Grants Council General Research
%Fund 16301720 (LCZ) and 16302818 (YX).

%\section*{References}

\bibliography{ref}

\end{document}
