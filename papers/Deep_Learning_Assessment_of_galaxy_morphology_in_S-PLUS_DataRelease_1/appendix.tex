\section{Notes on the Ambiguous sample}
\label{appendix}
The debiased classification published in Table 2 of the GZ1 data release provides a debiased likelihood of a galaxy being a spiral or an elliptical and a morphology flag \citet{lintott2011}, using a binary classification (0=false,1=true) that represents a final catalogue of ETG and Sp likelihood\footnote{For further information we also refer to the data release notes in \url{https://data.galaxyzoo.org/}}. The deep learning technique described in this contribution used the morphology flag instead of the debiased likelihood from Galaxy Zoo. We discussed this choice in Section \ref{subsec:GZ1_training}. While all galaxies in Table 2 of GZ1 data release have a debiased likelihood of being elliptical or spiral galaxies, they might have a flag zero in both elliptical and spiral morphology, see \citet{bamford2009galaxy,lintott2011} for more details.
Nevertheless, we can compare the probability of being a spiral or a lenticular galaxy obtained in this work using a DL method with the debiased likelihood provided in \citet{lintott2011}, as shown in Figure \ref{fig:amb_comp_prob}.

\begin{figure}
\centering
\includegraphics[width=0.9\linewidth]{Figures/appendix/scores_compare_E_amb.pdf}\\
\includegraphics[width=0.9\linewidth]{Figures/appendix/scores_compare_S_amb.pdf}
    \caption{
    \label{fig:amb_comp_prob}
    Distribution of probabilities (scores) for galaxies to be ETG or Sp provided by the Neural Network (blue) in comparison with the debiased likelihood from GZ1 (red).
    }
\end{figure}

The debiased likelihood of the ambiguous set presents a smooth distribution. It is worth noticing that the threshold cut of $0.8$ presented in \citep{bamford2009galaxy,lintott2011} is not equivalent to the flag type used in this work, which includes objects with probability higher than $0.8$ of being either Sp or E according to GZ1, resulting in a morphology flag equal to 0 in both classes. 
On the other side, the result of the DL method (the blue histogram in Figure \ref{fig:amb_comp}) shows a clear bimodality and a net classification between elliptical and spiral galaxies. To check if the Deep Learning method is overconfident, we inspected by eye all the galaxies belonging to the sub-sample II, comparing them with the deepest Legacy Survey \citep{dey2019overview,zou2017project} data. We present an example of such a comparison in Figure \ref{fig:amb_comp}.


\begin{figure}
\centering
\includegraphics[width=0.95\linewidth]{Figures/appendix/examples_Clecio_geferson.pdf}\\
    \caption{
    \label{fig:amb_comp}
    Example of color images showing  ETG galaxies {\it (left panels)} and a Sp galaxy {\it(right panels)} for S-PLUS and Legacy Survey. \\
    }
\end{figure}

It is interesting to see that, in general, galaxies classified as spiral by the DL method present clear spiral arms in the Legacy data, even if these arms are barely visible in S-PLUS images. Note that the depth of the S-PLUS survey is slightly higher than the standard SDSS depth, which characterises GZ1 galaxies. 
On the other side, some galaxies classified as ETGs by the DL method present hints of spiral arms in the Legacy data. Clearly, the S-PLUS imaging only detects the central part of the galaxy light, where the bulge is dominant, revealing the limitations due to the data.