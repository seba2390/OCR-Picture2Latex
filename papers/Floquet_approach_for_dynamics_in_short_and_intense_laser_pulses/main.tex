%%%%%%%%%% esta es la aps cabecera %%%%%%%%%%%
\documentclass[
pra%
,preprint%
%,twocolumn%
,amssymb, nobibnotes, aps, superscriptaddress, floatfix]{revtex4}


\usepackage{amsmath}
\usepackage{enumitem}
\usepackage{euscript}
\usepackage{graphicx}
\usepackage{amsfonts}
\usepackage{bm}

%\usepackage[usenames]{color}
%\usepackage{hyperref}
%\definecolor{rltred}{rgb}{0.75,0,0}
%\definecolor{rltgreen}{rgb}{0,0.5,0}
%\hypersetup{colorlinks,linkcolor=rltred,citecolor=rltgreen}
%\usepackage[dvipsnames]{xcolor}
%
%\graphicspath{{./}{figures/}}
%
%\newcommand{\sumint}{\ensuremath{\sum \!\!\!\!\!\!\!\! \int}}
%\renewcommand{\equationautorefname}{Eq.}
%\renewcommand{\figureautorefname}{Fig.}
%%%%%%%%%%% aqui acaba la aps cabecera %%%%%%%%%%% 
%%%%%%%%%%%%%%%%%%%%%%%%%%%%%%%%%%%%%%%%%%%%%%%%%%%%%%%%%%%%%%%

\newcommand{\fig}{Fig.\,}

\renewcommand{\imath}{\mathrm{i}}
\newcommand{\dd}{\mathrm{d}}
\newcommand{\rv}{\mathbf{r}}
\newcommand{\av}{\boldsymbol{\alpha}}
\newcommand{\as}{\alpha}
\newcommand{\kv}{\mathbf{k}}
%\newcommand{\e}[1]{\mathrm{e}^{#1}}

\newcommand{\CAH}{$\mathcal{H}$}
\newcommand{\CAHt}{$\mathcal{H}(t)$}
\newcommand{\KH}{KH}
\newcommand{\khsub}{\mathrm{KH}}
\newcommand{\phase}{\phi}

\newcommand{\psit}{\widetilde{\psi}}
\newcommand{\Vt}{\widetilde{V}}
\newcommand{\Vb}{\overline{V}}

\newcommand{\Tmat}{\mathbf{T}}
\newcommand{\Vmat}{\mathbf{V}}
\newcommand{\Pvec}{\boldsymbol{\psi}}
\newcommand{\Cvec}{\mathbf{c}}

%\newcommand{\KNORM}{\frac{(\Delta K)^3}{(2 \pi)^3}}
\newcommand{\KNORM}{\frac{1}{(2 \pi)^3}}
%\newcommand{\WNORM}{\frac{\omega}{2 \pi}}
\newcommand{\WNORM}{\frac{1}{T_\omega}}

\newcommand{\intK}{\int \dd^3 k}
\newcommand{\dk}{\Delta k}

%
% Coefficients
%

% Continuous versions
\newcommand{\Vidx}[1]{\Vt_{#1}}
\newcommand{\Vmk}{\Vidx{m}(\kv, t)} %\newcommand{\Vmk}{\Vidx{\kv m}(t)}
\newcommand{\Vmkp}{\Vidx{m'}(\kv-\kv', t)} %\newcommand{\Vmkp}{\Vidx{\kv-\kv', m'}(t)}
\newcommand{\Vmpkp}{\Vidx{m-m'}(\kv-\kv', t)}
\newcommand{\Vk}{\Vt(\kv)} %\newcommand{\Vk}{\Vidx{\kv}}
\newcommand{\Vkp}{\Vt(\kv-\kv')} %\newcommand{\Vkp}{\Vidx{\kv-\kv'}}
\newcommand{\Vkpo}{\Vidx{0}(\kv-\kv', t)} %\newcommand{\Vkpo}{\Vidx{\kv-\kv',0}(t)}
\newcommand{\Vko}{\Vidx{0}(\kv, t)} %\newcommand{\Vko}{\Vidx{\kv0}(t)}

\newcommand{\Pidx}[1]{\psit_{#1}}
\newcommand{\Pmk}{\Pidx{m}(\kv, t)} %\newcommand{\Pmk}{\Pidx{\kv m}(t)}
\newcommand{\Pmkp}{\Pidx{m}(\kv', t)}
\newcommand{\Pmpkp}{\Pidx{m'}(\kv', t)} %\newcommand{\Pmkp}{\Pidx{\kv' m'}(t)}

% Discrete versions
\newcommand{\Vdmk}{\Vidx{\kv m}(t)}
\newcommand{\Vdmkp}{\Vidx{\kv-\kv', m'}(t)}
\newcommand{\Vdk}{\Vidx{\kv}}
\newcommand{\Vdkp}{\Vidx{\kv-\kv'}}
\newcommand{\Vdkpo}{\Vidx{\kv-\kv',0}(t)}
\newcommand{\Vdko}{\Vidx{\kv0}(t)}

\newcommand{\Pdmk}{\Pidx{\kv m}(t)}
\newcommand{\Pdmkp}{\Pidx{\kv' m'}(t)}

%
%
%

%\def\i{\mathrm{i}}
%\def\d{\mathrm{d}}
\def\e#1{\mathrm{e}^{#1}}
\def\vec#1{\mathbf{#1}}
%\def\balpha{\boldsymbol{\alpha}}

\usepackage{color}
\def\ulf#1{{\color{blue}\it#1}}

%
%
%

\usepackage{color}

%\newcommand{\edit}[1]{\textbf{#1}}
\newcommand{\edit}[1]{\textcolor{black}{#1}}
\newcommand{\editt}[1]{\textcolor{black}{#1}}
\newcommand{\tmp}[1]{\textcolor{black}{#1}}
\newcommand{\hide}[1]{}


% Adds a star to the left of the line outside the page margins
\newcommand{\impmark}{\strut\vadjust{\domark}}
\newcommand{\domark}{%
	\vbox to 0pt{
		\kern-\dp\strutbox
		\smash{\llap{*\kern1em}}
		\vss
	}%
}

%
\begin{document}

\title[]{Floquet approach for dynamics in short and intense laser pulses}

 % % % % % % % % % % % % % % % % % % % % % % % % % % % % % % % % % % % % % % % % %
\author{Lukas Medi\v sauskas}
\author{Ulf Saalmann}
\author{Jan-Michael Rost }
\affiliation{Max Planck Institute for the Physics of Complex Systems, N\"{o}thnitzer Stra{\ss}e 38, D-01187 Dresden, Germany}
%%%%%%%%%%%%%%%%%%%%%%%%%%%%%%%%%%%%%%%%%%%%%%%%%

\begin{abstract}
We present a time-dependent Floquet method that allows one to use the cycle-averaged Kramers-Henneberger basis for short pulses and arbitrary laser frequencies.
By means of a particular plane-wave expansion we arrive at a time-dependent Schr\"{o}dinger equation that consists of convolutions of momentum and Floquet components. A dedicated numerical treatment of these convolutions, based on Toeplitz matrices and fast Fourier transformations, allows for an efficient time-propagation of large Floquet expansions. 
%An efficient numerical procedure to propagate the Floquet Hamiltonian is provided that relies on the Toeplitz matrix formalism and Fast Fourier Transformations.
%It enables efficient time propagation with large Floquet expansions, while still taking advantage of the cycle-averaged Kramers-Henneberger basis. 
Three illustrative cases of ionization with different photon energies are analyzed, where the envelope of a short and intense pulse is crucial to the underlying dynamics.

%The approach is used to illustrate the role of a short and intense laser pulse in the ionization of one electron atom in 1D using three different frequencies.
%
%For large frequencies, we show that the dynamics stays adiabatic up to very short pulses. For ultra short pulses, non-adiabatic transitions to the excited states take place, that are induced by the time variation of the field cycle averaged Hamiltonian.
%
%For small frequencies, the Kramers-Henneberger approach fails to represented adiabatic states of the laser driven system leading to a rapid distribution of population over many excited states and Floquet manifolds.
%
%For intermediate frequencies, the population is transferred from the ground to the excited states at the crossings of states belonging to different Floquet manifolds. The interference of transitions at the begining and ending of the pulse lead to a high sensitivity to the duration of a short pulse.

\end{abstract}

\maketitle

%
%
%

\section{Introduction}

Non-perturbative laser-matter interaction provides a rich yet challenging area for theoretical studies. While numerical methods have to deal with large energy bandwidths required to fully account for the dynamics, analytical methods are faced with the challenge of finding an appropriate description of non-perturbative light-matter interaction.

A successful analytical approach to non-perturbative laser-matter interaction is the Kramers-Henneberger (KH) approximation \cite{Henneberger1968a,Gersten1974}. It describes the dynamics in the Kramers-Henneberger reference frame co-moving with the laser-driven electron(s). Traditionally, the time-dependence of the problem is eliminated by using a Hamiltonian, averaged over one optical cycle. The corresponding potential and eigenenergies are often referred to as the KH approach or the KH ``atom." For sufficiently large field strengths and high frequencies, the cycle-averaged Hamiltonian (\CAH) largely determines the properties of the coupled light-matter system, while all higher-order corrections remain small and can be treated perturbatively. 

Since its introduction, the KH approach was thoroughly examined and the properties of \CAH\ are very well known \cite{Gavrila2002, Popov2003}. It was applied to a large variety of problems in atomic, molecular \cite{Pont1988} and solid state physics \cite{Niculescu2008,Lima2008} and in particular, was indispensable in the study of ionization suppression phenomena for atoms in strong and high-frequency fields. Nevertheless, most of the theoretical predictions were not tested experimentally (see \cite{VanDruten1997,DeBoer1993,DeBoer1994} for application for Rydberg state ionization) because high-intensity and high-frequency lasers were not available at that time.

The situation has, however, changed due to the free-electron lasers (FEL) \cite{McNeil2010} that are already able to provide pulses of sufficiently high-frequency and intensity to enable the observation of non-perturbative phenomena. The first experimental studies of Raman processes in the VUV and XUV frequency range, which require coherent multiple photon absorption/emission, have been carried out \cite{Weninger2013}. It can be expected that FEL’s will soon reveal high-frequency non-perturbative phenomena which were proposed theoretically, such as adiabatic stabilization \cite{Gavrila2002}, dynamic interferences \cite{Toyota2008,Toyota2007,Tolstikhin2008,Demekhin2012,Baghery2017}, Rabi oscillations between core-hole states \cite{Demekhin2011}, to name a few.

The KH approach is ideally suited to describe strong-field high-frequency physics to be realized in FEL facilities apart from one crucial aspect: studies so far were mostly limited to continuous-wave laser radiation. Indeed, for a continuous-wave field, a perturbative expansion in Floquet orders can be readily developed. On the other hand, for short (FEL) pulses many Floquet channels have to be included, rendering the time-dependent treatment prohibitively demanding. 

Clearly, the time-dependent aspect is crucial since the short pulses created by FEL sources can lead to additional dynamics driven by the pulse envelope as was recently predicted \cite{Toyota,Simonsen2016}, or be necessary to account for phenomena like impulsive Raman scattering \cite{Miyabe2015}. Hence, in order to apply the KH approach to the dynamics involving intense and short pulses, a formulation different from the ones so far known appears to be necessary. 

Here we propose a numerical approach for short-pulse non-perturbative laser-matter interaction that is based on a time-dependent Floquet formalism in the KH reference frame. It uses \CAHt\ which depends on the instantaneous intensity of the laser pulse and relies on time-propagation using the full Floquet Hamiltonian, which is performed with an efficient Fast-Fourier-Transformation based algorithm.
Combining these two approaches allows us to obtain both a qualitative and quantitative understanding of the light-matter interaction \emph{during} the laser pulse, despite treating short laser pulses produced by FEL facilities non-perturbatively.
%The use of the two-timescale \CAHt\ allows us to obtain a qualitative understanding of the light-matter interaction in the non-perturbative high-frequency regime while the use of full Floquet Hamiltonian allows treating short laser pulses produced by FEL facilities. 

In Section \ref{sec:theory} we will present the time-dependent Floquet approach, followed in Section \ref{sec:num} by the introduction of the novel algorithm to solve the Floquet problem in momentum space. The approach is illustrated in Section \ref{sec:results}, where the role of the envelope of a short and intense laser pulse is investigated for the ionization in 1D potential. By varying the laser frequency, while keeping \CAHt\ invariant, three parameter ranges are explored: high, intermediate, and low-frequency regimes. We show in Section \ref{sec:high_freq} that \CAHt\ provides an excellent approximation of the laser-driven dynamics for frequencies higher than the binding energy of the potential. 
For intermediate frequencies close to the ionization threshold, discussed in Section \ref{sec:mid_freq}, the pulse envelope plays a crucial role in determining the channels involved in the ionization.
Finally, the low-frequency regime is discussed in Section \ref{sec:low_freq}; in this case, the photon energy is much smaller than the binding energy of the field-free potential and several hundred Floquet channels are required to fully account for the dynamics. 
We show that the population is rapidly distributed over many excited states of \CAHt\ during the rising part of the laser pulse, which has to be considered if one wants to use the KH approach for low-frequency fields.


%
%
%


\pagebreak
\section{Time-dependent Kramers--Henneberger--Floquet approach in momentum representation} \label{sec:theory}

%\cite{Peskin1993,Telnov1995,Chu2010a,Halasz2012,Toyota}
In this section a time-dependent Floquet approach is derived for efficiently computing laser-driven dynamics with \CAHt\ in the KH reference frame, i.e., using the cycle averaged Hamiltonian that depends on instantaneous laser intensity, and is a generalization of the Envelope Hamiltonian approach introduced in \cite{Toyota}. The formalism allows one to explore the transformation of the wave function from the field-free to the ``field-dressed" picture while still fully accounting for the effects of a short laser pulse.

\subsection{Kramers--Henneberger transformation}

The time-dependent Schr\"{o}dinger equation (TDSE) within the single active electron approximation (in the following, atomic units will be used, unless stated otherwise)
\begin{equation}
\imath \frac{\partial}{\partial t}\Psi(\rv, t) = \Big[-\frac{1}{2} \big( \boldsymbol{\nabla} - \imath \mathbf{A}(t) \big)^2 + V(\rv) \Big] \Psi(\rv, t), \label{eq:tdse}
\end{equation}
where $\mathbf{A}(t)$ relate to the laser electric field $\mathbf{E}(t)$ by $\mathbf{A}(t) = -\int^t\mathrm{d}\tau \, \mathbf{E}(\tau)$, can be transformed into a space-translated frame of reference, the so-called KH frame, by applying the unitary transformation
\begin{align}
\hat{U} = \exp\Big( -\int^t \dd \tau \mathbf{A}(\tau) \cdot \boldsymbol{\nabla}  \Big) \,
\exp\Big( \frac{\imath}{2} \int^t \mathrm{d}\tau \mathbf{A}^2(\tau) \Big).
\end{align}
In the KH reference frame the TDSE acquires the form
\begin{align}
	\imath \frac{\partial}{\partial t} \Psi_{\khsub}(\rv, t) = \Big[-\frac{\boldsymbol{\nabla}^2}{2} + V\big(\rv + \av(t)\big) \Big] \Psi_{\khsub}(\rv, t), \label{eq:tdse_kh}
\end{align}
where the coupling with the laser field is reduced to the time-dependent shift $\av(t)$ of the binding potential $V\big(\rv + \av(t)\big)$. For simplicity we assume this shift to be of the form
	\begin{align}
	%\av(t) = -\frac{1}{M}\int^t \mathbf{A}(\tau) \, \mathrm{d}\tau = \av_0 \, \cos(\omega \, t)
	\av(t) = -\int^t \mathbf{A}(\tau) \, \mathrm{d}\tau = \av_0(t) \, \cos(\omega \, t + \phase)
	%= \frac{\mathbf{A}_0(t)}{M\omega^2} \, \cos(\omega \, t) 
	\label{eq:excursion}
	\end{align}
corresponding to the classical trajectory of a charged particle in a laser field. 
%At first, however, the Floquet formalism will be introduced assuming a continuous-wave laser field, for which the envelope of the shift is constant $ \av_0(t) \rightarrow \av_0$. Later, the treatment will be extended to laser pulses using the two-timescale formalism.

%\subsubsection{The Kramers--Henneberger potential}

\begin{figure}
\centering
%\includegraphics{Pictures/KH_potentials_b}
\includegraphics{Figure1}
%\includegraphics{/home/medisauskas/Work/Solids/Code/Bloch-Floquet/KH_potentials_c}
\caption{Cycle-averaged soft-core atomic potential (see Eq.~(\ref{eq:V0}) and Eq.~(\ref{eq:V_soft}) ) for different excursion lengths $\as_0$. Shaded area indicates binding part of the potential.}
\label{fig:KH_potentials}
\end{figure}


The KH transformation describes the laser--atom interaction in a frame of reference, where the electron can be considered to be ``stationary," while the binding potential of the ``atom" is time-dependent. In other words, the electron ``sees the nucleus oscillating back and forth". The oscillating potential $V\big(\rv + \av(t)\big)$ can be integrated over a single cycle $T_\omega$ of the oscillation $\av(t)$ to obtain the averaged potential
\begin{equation}
V_{0}(\rv, \av_0) = \frac{1}{T_\omega}\int_0^{T_\omega} \, \dd t \, V\big(\rv + \av(t)\big), \label{eq:V0}
\end{equation}
which is also called the ``KH potential" and has been used to describe the properties of atoms in strong and high-frequency laser fields \cite{Pont1988,Pont1990}. The average potential strongly depends on the electron excursion length $\av_0$, as illustrated in \fig \ref{fig:KH_potentials}, and, for sufficiently large excursion lengths $\av_0$, transforms from a single-well to a dichotomous double-well shape.
In this work, we are going to use the cycle-averaged potential that adjusts to time-variation of $\av_0(t)$ to describe the atom-laser interaction with pulsed laser fields.

\subsection{Time-dependent Floquet approach for short laser pulses}

The time-independent KH potential and its properties are analyzed in great detail in the literature using a variety of methods \cite{Gavrila1984,Reed1990,Pont1990,Su1990,Dorr1991,Yao1992,Scrinzi1993,Atabek1997,Smirnova2000,Morales2011}.
In practice, however, one needs to deal with finite and often short pulses and considering a static KH potential is not sufficient.
Here, we consider a cycle-averaged potential that adjusts to the laser pulse envelope, while still providing an exact description of the dynamics. At first glance it looks cumbersome to perform for each instance of time a full cycle average. However, when switching simultaneously to momentum space one arrives at a compact and distinct form of the TDSE (cf. Eq.~(\ref{eq:tdse_tt}) below), as we will show briefly here and in detail in Appendix \ref{sec:app_expansions}.

The potential in the KH reference frame can be written in a plane-wave expansion
\begin{align}
V(\rv,t) & = \int\dd^{3}r'\!\int\dd t'\,V(\rv',t')\,\delta(\rv'{-}\rv)\,\delta(t'{-}t)
\notag\\
&= \KNORM \WNORM \sum_{m} \intK \;
\Big[\int \dd^{3}r'\!\int_{0}^{T_\omega}\dd t'\,V(\rv',t')\e{+\imath m\omega t'}\e{-\imath \kv \cdot \rv'}\Big]
\e{-\imath m\omega t}\e{+\imath \kv \cdot \rv}
\label{eq:expa}
\end{align}
with integer $m$, anticipating that the potential oscillates with frequency $\omega$.
%, and discrete momenta $\kv$ defined by a finite box of size $L=2 \pi / \Delta K$ in position space. Of course, by enlarging this box one may perform the transition to continuous momenta $\kv$. Here, without loss of generality, we stick to the discrete form.

In order to efficiently treat short pulses we will not apply the standard expansion, but rather extract the envelope of the laser pulse.
This is done by splitting the electron displacement $\av(t)$ into the non-periodic envelope $\av_0(t)$, described by the time variable $t$ and the periodic oscillation $\cos(\omega t'+\phase)$, described by time $t'$. Thereby, the KH potential becomes a ``two-time potential''
\begin{equation}
\label{eq:ttpot}
\Vb(\rv, t, t') = V\big(\rv + \av_{0}(t)\cos(\omega t'{+}\phase)\big),
\end{equation}
which can be used straightaway in expansion \eqref{eq:expa} to give
\begin{subequations}\label{eq:ttexpa}
\begin{align}
V(\rv,t) &= \sum_{m} \intK \; \Vmk \,
\e{-\imath m\omega t}\e{+\imath \kv{\cdot}\rv}
\intertext{with}
\Vmk &\equiv 
\KNORM \WNORM
\int \dd^{3}r'\!\int_{0}^{T_\omega}\dd t'\,\Vb(\rv', t, t')\e{+\imath m\omega t'}\e{-\imath \kv{\cdot}\rv'}.
\end{align}
\end{subequations}
Thereby, the components $\Vmk$ depend on time through the pulse envelope.
Needless to say that expansion \eqref{eq:ttexpa} is exact.
%Note that although the use of two-times is conceptually similar to other approaches \cite{Peskin1993,Telnov1995,Chu2010a,Halasz2012}, the ansatz we use is different from the one used in previous works.
There are other approaches that adopt two times \cite{Peskin1993,Telnov1995,Chu2010a,Halasz2012}. Here the two times are used to straightforwardly derive expansion (\ref{eq:ttexpa}), which turn out to be very convenient for short pulses.

Now it is essential that in the KH reference frame, by means of a translation in space and the Jacobi-Anger expansion, the components $\Vmk$ can be rewritten as products (see Appendix~\ref{sec:app_expansions} and Ref.~\cite{Yao1992} for the derivation)
\begin{subequations}\label{eq:VJexp}
\begin{align}
\Vmk &=\Vk \,\imath^{|m|} J_{|m|} \big(\kv\,{\cdot}\,\av_{0}(t)\big)\e{-\imath m\phi}
\intertext{with the momentum components of the field-free potential}
\Vk &\equiv \KNORM \int \dd^{3}r\,V(\rv)\,\e{-\imath\kv{\cdot}\rv}
\end{align}
\end{subequations}
and $J_{m}$ denoting the ordinary Bessel functions of the 1st kind. 

%
%%To derive the time-dependent Floquet formulation of TDSE, 
%%we split the electron displacement  $\av(t)$ into the non-periodic envelope $\av_0(t)$, described by the time variable $t$ and the periodic oscillation $\cos(\omega t' + \phase)$, described by the time $t'$
%%\begin{equation}
%%\av(t) \rightarrow \av(t, t') = \av_0(t) \, \cos(\omega t' + \phase). \label{eq:excursion_two_timescale}
%%\end{equation}
%The potential in the KH reference frame is described by expanding it into Fourier modes and plane-waves
%\begin{subequations}
%\begin{equation}
%\mathrm{V}(\rv, t) = \sum_{\kv m} \, \Vmk \, \mathrm{e}^{-\imath m \omega t} \, 
%\mathrm{e}^{\imath \kv \cdot \rv}. \label{eq:Fourier_expansion_V_a}
%\end{equation} 
%In contrast contrast to the standard expansion the Fourier terms $\Vmk(t)$ depend on time through the envelope of the laser pulse. This is incorporated by splitting the electron displacement $\av(t)$ into the non-periodic envelope $\av_0(t)$, described by the time variable $t$ and the periodic oscillation $\cos(\omega t' + \phase)$, described by time $t'$. Thereby, the KH potential becomes a ``two-time potential"
%\begin{equation}
%\mathrm{V}(\rv, t, t') = \mathrm{V}\big(\rv + \av_0(t)\cos(\omega t'+\phase)\big),\label{eq:Fourier_expansion_V_c}
%\end{equation}
%which is used to define the Fourier components in Eq.~(\ref{eq:Fourier_expansion_V_a}) as
%\begin{equation}
%\Vmk = \frac{1}{\sqrt{2 \pi}} \int\dd \rv \, \, \mathrm{e}^{\imath \kv \cdot \rv} \,
%\frac{1}{T_{\omega}} \, \int_0^{T_\omega} \, \dd t' \, \mathrm{V}(\rv, t, t') \, 
%\mathrm{e}^{\imath m \omega t'},\label{eq:Fourier_expansion_V_b}
%\end{equation}\label{eq:Fourier_expansion_V}
%\end{subequations}
%The above expansion is \emph{exact}  as can be seen by means of $\int \, \mathrm{d}^3 r \, \mathrm{e}^{\imath(\kv-kv')\rv}=\delta_{\kv \kv'}$ and $\sum_m \, \mathrm{e}^{\imath m (t-t')} = \delta(t-t')$. 
%In the KH reference frame, the Fourier terms in (\ref{eq:Fourier_expansion_V_b}) can be rewritten as a product \cite{Yao1992}
%\begin{equation}
%\Vmk = \Vk \,  (\imath)^{|m|} \, J_{|m|}\big( \kv \cdot \av_0(t)\big) \, \mathrm{e}^{-\imath m \phase}, 
%\label{eq:plane_wave_fourier_components}
%\end{equation}
%with the spatial components of the field-free potential
%\begin{equation}
%\Vk = \frac{1}{\sqrt{2 \pi}} \int\dd \rv \, \mathrm{V}(\rv) \, \mathrm{e}^{\imath \kv \cdot \rv}.
%\end{equation}
%
%%representation of the full time-dependent potential $\mathrm{V}(\rv, t)$, as shown in Appendix \ref{sec:app_expansions}.
%%
%%The KH potential in momentum representation acquires a convenient form that allows to factorize it into the time-dependent and time-independent parts, and to write Eq.~(\ref{eq:Fourier_expansion_V_b}) as (see Appendix \ref{sec:app_expansions} for details)
%%	\begin{align}
%%	\Vmk = \frac{1}{T_{\omega}} \, \int_0^{T_\omega} \, \dd t' \, 
%%	\Vk \, \mathrm{e}^{\imath \kv \cdot \av(t, t')} \,
%%	\mathrm{e}^{\imath m \omega t'}\label{eq:PL_expansion},
%%	\end{align}
%%where the time-dependence is fully contained in the exponential factor and
%%\begin{equation}
%%\Vk = \frac{1}{\sqrt{2 \pi}} \int\dd \rv \, \mathrm{V}(\rv) \, \mathrm{e}^{\imath \kv \cdot \rv}
%%\end{equation}
%%are the spacial Fourier component of the time-independent (field free) potential.
%%Using the definition of excursion length $\av(t, t')$ in Eq.~(\ref{eq:excursion_two_timescale}) and the Jacobi-Anger expansion, the plane-wave Fourier components $\Vmk$ are written as \cite{Yao1992}
%%\begin{equation}
%%\Vmk = \Vk \,  (\imath)^{|m|} \, J_{|m|}\big( \kv \cdot \av_0(t)\big) \, \mathrm{e}^{-\imath m \phase}, 
%%\label{eq:plane_wave_fourier_components}
%%\end{equation}
%%where $J_m$ is the ordinary Bessel function of the 1st kind of order $m$. Eq.~(\ref{eq:plane_wave_fourier_components}) together with Eq.~(\ref{eq:Fourier_expansion_V_a}) provide an exact representation of the time-dependent potential in the KH reference frame, with the time-dependence of the $\Vmk$ terms accounting for the effect of the laser pulse envelope and the expansion in Fourier modes $m$ taking care of the periodic part of the oscillation.

Having rewritten the potential as a sum of products we use a similar expansion in terms of Fourier modes and plane-waves for the wave function
\begin{equation}
\Psi_{\khsub}(\rv, t) = \sum_{m} \intK \; \Pmk \, \mathrm{e}^{-\imath m \omega t} \, \mathrm{e}^{\imath \kv \cdot \rv}. \label{eq:ansatz_wf_fourier}
\end{equation}
Just like in the Eq.~(\ref{eq:ttexpa}), this ansatz does not imply any restriction on the total wave function. The expansion coefficients $\Pmk$ are determined by inserting Eqs.~(\ref{eq:ttexpa},\ref{eq:VJexp}) and Eq.~(\ref{eq:ansatz_wf_fourier}) into the TDSE. This allows one to derive an equation for the $\kv$-th plane-wave and $m$-th Fourier components of the wave function (see Appendix \ref{sec:app_expansions} for details)
\begin{multline}
\imath \frac{\partial}{\partial t}\Pmk = \Big[ \frac{\kv^2}{2} - m \omega \Big] \Pmk +\\
+ \sum_{m'} \intK' \; \Vkp \, \Pmpkp \;  
\imath^{|m-m'|} \, J_{|m-m'|}\big( (\kv-\kv') \cdot \av_0(t) \big) \; \mathrm{e}^{-\imath (m-m')\phase}.\label{eq:tdse_tt}
\end{multline}
Eq.~(\ref{eq:tdse_tt}) is the main equation used in this work and provides an \emph{exact} description of the laser driven dynamics in the KH reference frame. 

The accuracy of its numerical implementation is limited only by the basis and propagation routines, see Sec.~\ref{sec:accuracy} for more extended discussion.
The momentum representation used here is particularly suited to describe the dynamics in the KH reference frame as it reduces the TDSE to a convenient form that allows one to use efficient numerical propagation methods, as described in Section \ref{sec:num}.

%To solve the TDSE with the potential in Eq.~(\ref{eq:Fourier_expansion_V}), the wavefunction is expanded in the basis of Fourier modes and plane-waves
%\begin{equation}
%\Psi_{\khsub}(\rv, t) = \sum_{\kv m} \, \Pmk \, \mathrm{e}^{-\imath m \omega t} \, \mathrm{e}^{\imath \kv \cdot \rv}. \label{eq:ansatz_wf_fourier}
%\end{equation}
%Just like in the Eq.~(\ref{eq:Fourier_expansion_V}), this ansatz does not imply any restrictions on the total wavefunction. the expansion coefficients $\Pmk$ are determined by inserting Eqs.~(\ref{eq:Fourier_expansion_V},\ref{eq:plane_wave_fourier_components}) and Eq.~(\ref{eq:ansatz_wf_fourier}) into the TDSE. This allows to derive an equation for the $\kv$-th plane-wave and $m$-th Fourier components $m$ of the wavefunction (see Appendix \ref{sec:app_expansions} for details)
%\begin{multline}
%\imath \frac{\partial}{\partial t}\Pmk = \Big[ \frac{\kv^2}{2} - m \omega \Big] \Pmk +\\
%+ \sum_{\mathbf{K}'m'} \Vkp \, \Pmkp \times  
%(\imath)^{|m-m'|} \, J_{|m-m'|}\big( (\kv-\kv') \cdot \av_0(t) \big) \times \mathrm{e}^{-\imath (m-m')\phase}.\label{eq:tdse_tt}
%\end{multline}
%Eq.~(\ref{eq:tdse_tt}) is the main equation used in this work and provides an \emph{exact} description of the laser driven dynamics in the KH reference frame. The accuracy of this approach is limited only by the basis and propagation routines, see Sec.~\ref{sec:accuracy} for more extended discussion.
%The momentum representation used here is particularly suited to describe the dynamics in the KH reference frame as it reduces the TDSE to a convenient form that allows to use an efficient numerical propagation methods described in Section \ref{sec:num}.

\subsubsection{Physical interpretation of the wave function in the Fourier basis}

The physical significance of the index $m$ becomes apparent, if we consider an isolated Fourier subspace $m$, i.e., ignore the coupling between the wave function coefficients $\Pmk$ with different $m$. In such a case, the only remaining potential coupling terms in Eq.~(\ref{eq:tdse_tt}) are
%\begin{equation}
%\imath \frac{\partial}{\partial t}\Pmk = \Big[ \frac{\kv^2}{2} - m \omega \Big] \Pmk
%+ \sum_{\mathbf{K}'} \Vkpo \, \Pidx{\kv' m}(t),
%\end{equation}
%where  
\begin{equation}
\Vkpo = \Vkp \, J_{0}\big( (\kv-\kv') \cdot \av_0(t) \big),
\end{equation}
which is just the momentum-representation of the cycle-averaged potential
\begin{equation}
\intK \; \Vko \, \e{\imath \kv \cdot \rv} =  \frac{1}{T_\omega}\int_0^{T_\omega} \, \dd t' \, V\big(\rv + \av(t,t')\big). 
%= \mathrm{V}_0(\rv, \av_0(t)).
\end{equation}
Therefore, considering a single Fourier subspace in isolation is similar to the original KH approach \cite{Henneberger1968a}, where only the cycle-averaged potential is considered.

The components $\Vmk$ with $|m|>0$ couple different Fourier subspaces and lead to transitions between the states of the cycle-averaged Hamiltonian \CAHt. Therefore, the index $m$ can be interpreted as the number of absorbed/emitted photons. For example, population initially created in the $m=0$ subspace and ending up in the $m$-th subspace after the pulse represents $m$-photon absorption, therefore, when its physical meaning will be important the $m$-th Fourier subspace will be referred to as Floquet channel.
%These processes are considered up to the lowest order in the high-frequency Floquet theory ($m-m'=0, \pm 1$) \cite{Gavrila1984,Marinescu1996}. 
In this work, enough Floquet channels are included to achieve numerical convergence. Therefore fields of arbitrary frequency can be considered. 

%Finally note, that each Fourier subspace $m$ contains the full spectra of the cycle-averaged Hamiltonian \CAHt, shifted by the energy $m \omega$, as is evident from Eq.~(\ref{eq:tdse_tt}). Therefore, in connection with the Floquet theory, the Fourier subspaces labeled by $m$ will be referred to as Floquet zones.

%, which in our case the also adjusts to the laser pulse envelope via $\av_0(t)$.
%High-frequency Floquet theory normally considers only the coupling to the neighboring Fourier subspaces ($m-m'=0, \pm 1$) \cite{Marinescu1996}. In this work, as many Fourier subspaces as are required to achieve numerical convergence are included.
%Finally note, that all Fourier manifolds $m$ in Eq.~(\ref{eq:tdse_tt}) are coupled. This is in contrast to Floquet expansions in a frame of reference fixed to the atom, for which only adjacent manifolds couple together. 


%Therefore, when their physical meaning will be important, all states sharing the same index $m$ will be referred to as a Floquet zones.

%Once the coefficients  $\Pmk$ are obtained by solving the TDSE in Eq.~(\ref{eq:tdse_tt}), the full wavefunction can be recovered by summing over all $\kv$ and $m$ components following Eq.~(\ref{eq:ansatz_wf_fourier}). Alternatively, a partial sum over $\kv$ components for a fixed $m$ provides the wavefunction in the $m$-th Floquet zone.
%This partial wavefunction can be projected on the eigenstate of the KH potential $\mathrm{V}_0(\rv, \av_0(t))$ to obtain their populations at time $t$.

%
%
%




%%
%%$\subsubsection*{Physical meaning of $\psi_m$}
%%
%
%It is instructive to consider an isolated Fourier manifold in Eq.~(\ref{eq:tdse_Fourier_components}), i.e., $\mathrm{v}_{m-m'} = \mathrm{v}_{m-m'} \, \delta_{m, m'}$. Eq.~(\ref{eq:tdse_Fourier_components}) then reduces to
%\begin{equation}
%\imath \frac{\partial}{\partial t} \psi_m(\rv, t)  =\Big(-\frac{\nabla_\rv}{2} -m \omega \Big) \psi_m(\rv, t) + \mathrm{v}_{0}(\rv, t) \, \psi_{m}(\rv, t),
%\end{equation}
%where $\mathrm{v}_{0}(\rv, t)$ describes the cycle averaged potential 
%\begin{equation}
%\mathrm{v}_0(\rv, t) = \frac{1}{T_\omega}\int_0^{T_\omega} \, \mathrm{d}t' \, \mathrm{V}\big(\rv + \av(t,t')\big) 
%= \mathrm{V}_0(\rv, \av_0(t))
%\end{equation}
%defined in Eq.~(\ref{eq:V0}). %, which adjusts to the envelope of the laser pulse, described by $\av_0(t)$.
%Hence, each Fourier manifold in isolation (i.e., if $\mathrm{v}_{|m|>0}(\rv, t)$ are negligible) describes the dynamics governed only by the cycle averaged potential, which adjusts to the laser pulse envelope via $\av_0(t)$.
%The index $m$ can be interpreted as the number of absorbed or emitted photons. For example, population initially created in the $m=0$ manifold and ending up in the $m$-th manifold after the pulse represent $m$-photon absorption. 
%The original KH approach as in \cite{Henneberger1968a} is obtained by considering only a single Fourier manifold ($m-m'=0$).
%High-frequency Floquet theory normally considers only the coupling to the neighboring Fourier manifolds ($m-m'=0, \pm 1$) \cite{Marinescu1996}. In this work, as many manifolds as being required to achieve numerical convergence will be included. 
%%Finally note, that all Fourier manifolds $m$ in Eq.~(\ref{eq:tdse_tt}) are coupled. This is in contrast to Floquet expansions in a frame of reference fixed to the atom, for which only adjacent manifolds couple together. 
%From here on, when their physical meaning will be important, all states sharing the same index $m$ will be refer to as a Floquet manifold.


%\subsection{Momentum-space representation}
%
%%Inserting the ansatz in Eq.~(\ref{eq:Ansatz}) into the TDSE in Eq.~(\ref{eq:Floquet_hamiltonian_TDSE}) a system of equations for the dynamics of coupled Floquet modes can be obtained \cite{Drese1999,Fleischer2005}. However, such an approach requires to diagonalize the Floquet Hamiltonian at each time $t$ to obtain the quasi energies and Floquet modes.
%%Furthermore, these states can be meaningfully used only when the adiabatic theorem holds.
%%%, which is time consuming, especially when high-energy continuum states have to be included. 
%%Here we use a different approach and represent the Floquet Hamiltonian $\mathfrak{H}(\rv, t, t')$ and Floquet modes $\tilde{\Phi}_n(\rv, t'; t)$ into plane-waves basis. 
%%This is similar to the so-called crude adiabatic approximation \cite{longuet1961some}, where the use of coordinate independent basis (in our case $t'$-independent) allows to fully account for the non-adiabatic effects.
%%This allows to solve the TDSE in Eq.~(\ref{eq:Floquet_hamiltonian_TDSE}) efficiently avoiding the need to determine the quasi-energies and Floquet modes at each $t$.
%
%
%Expanding the Fourier components of the wavefunction $\psi_m(\rv, t)$ and the potential $\mathrm{v}_m(\rv, t)$ into (discrete) plane-wave components 
%\begin{subequations}
%\begin{align}
%\psi_m(\rv, t) &= \sum_{\kv} \, u_{\kv,m}(t) \, \mathrm{e}^{\imath \kv \cdot \rv}, \\
%\mathrm{v}_m(\rv,t) &= \sum_\kv v_{\kv,m}(t) \, \mathrm{e}^{\imath \kv \cdot \rv},
%\end{align}\label{eq:plane_wave_expansion}
%\end{subequations}
%inserting them into Eq.~(\ref{eq:tdse_Fourier_components}) and using Eq.~(\ref{eq:plane_wave_fourier_components}) leads to
%\begin{multline}
%\imath \frac{\partial}{\partial t}u_{\mathbf{K}m}(t) = \Big[ \frac{\kv^2}{2} - m \omega \Big] u_{\mathbf{K}m}(t) +\\
%+ \sum_{\mathbf{K}'m'} v_{\mathbf{K}-\mathbf{K}'} u_{\mathbf{K}'m'}(t) \times  
%(\imath)^{|m-m'|} \, J_{|m-m'|}\big( (\kv-\kv') \cdot \av_0(t) \big) \times \mathrm{e}^{-\imath (m-m')\phase}. \label{eq:tdse_tt}
%\end{multline}
%Eq.~(\ref{eq:tdse_tt}) is the main equation used in this work. It allows one to obtain the expansion coefficients $u_{\kv m}(t)$ of the full wavefunction $\Psi_{\khsub}(\rv, t)$, while still using the cycle-averaged potential \CAHt\ that depends only on the laser pulse envelope via $\av_0(t)$. This allows to clearly separate the dynamics due to photon absorption from the dynamics governed by the time variation of \CAHt. %\, are captured by Eq.~(\ref{eq:tdse_tt}).
%The method is exact and its accuracy was carefully tested and found to be limited only by the plane-wave basis and propagation routines, see Sec.~\ref{sec:accuracy} for more extended discussion.
%Such a momentum representation is particularly suited to describe the dynamics in the KH reference frame as it reduces the TDSE to a convenient form that also enables efficient numerical propagation, as will be discussed in Section \ref{sec:num} below.
%
%Once the coefficients  $u_{\kv m}(t)$ are obtained by solving the TDSE in Eq.~(\ref{eq:tdse_tt}), the full wavefunction can be recovered by summing over all $\kv$ and $m$ components following Eqs.~(\ref{eq:ansatz_wf_fourier}) and (\ref{eq:plane_wave_expansion}). Alternatively, one can sum over $\kv$ components for a fixed $m$ to obtain $\psi_m(\rv, t)$, i.e., the wavefunction in a $m$-th Floquet manifold.
%By projecting $\psi_m(\rv, t)$ on the eigenstates of $\mathrm{v}_{m=0}(\rv, t)$, the probability amplitude in an eigenstate of the KH potential at time $t$ can be obtained.
%%By projecting  $\psi_m(\rv, t)$ onto the eigenstates of $\mathrm{v}_{m=0}(\rv, t)$, which corresponds to the cycle-averaged KH potential at time $t$, the probability amplitude in an eigenstate of the KH potential can be obtained.
%%Since $\mathrm{v}_{m=0}(\rv, t)$ corresponds to the cycle-averaged KH potential at time $t$,
%%%, that depends on the instantaneousness electron excursion amplitude $\av_0(t)$, 
%%we can  project $\psi_m(\rv, t)$ onto the eigenstates of $\mathrm{v}_{m=0}(\rv, t)$ to obtain the probability amplitude in an eigenstate of the KH potential at time $t$.
%
%%they Fourier components $c_{m,n}$ of the amplitude in the eigenstates $\chi_n(\rv, t)$ of the cycle-averaged potential $V_0(\rv, \av_0(t))$ can be constructed
%%\begin{subequations}
%%	\begin{align}
%%	\Psi_{\khsub}(\rv, t) &= \sum_{n, m} \, c_{n, m} \, \chi_n(\rv, t) \,  \exp(-\imath m \omega t),\\
%%	c_{m, n}(t) &= 
%%	\sum_{\kv} \, u_{\kv m}(t) \, \int \mathrm{d}\rv \, \chi_n(\rv, t)^* \, 
%%	\exp(\imath \kv \cdot \rv),
%%	\end{align}
%%\end{subequations}
%%where the eigenstates  $\chi_n(\rv, t)$ can be found by diagonalizing the Hamiltonian defined by Eq.~(\ref{eq:tdse_tt}) for time $t$ and a single $m=0$ component.

\section{Numerical implementation} \label{sec:num}

\editt{
To numerically solve Eq.~(\ref{eq:tdse_tt})  we first rewrite it for a discrete momentum $\kv$ grid, yielding
\begin{multline}
\imath \frac{\partial}{\partial t}\Pdmk = \Big[ \frac{\kv^2}{2} - m \omega \Big] \Pdmk +\\
+ \sum_{\kv' m'} \Vdkp \, \Pdmkp \;  
\imath^{|m-m'|} \, J_{|m-m'|}\big( (\kv-\kv') \cdot \av_0(t) \big) \; \mathrm{e}^{-\imath (m-m')\phase}, \label{eq:tdse_tt_disc}
\end{multline}
where for D dimensions the field-free potential and the wave function are renormalized according to $\Vdk {=} \widetilde{V}(\kv) \, (\dk)^D$ and $\Pdmk=\Pmk (\dk)^{D/2}$
implying a box discretization with a box of size $L^D=(2\pi/\dk)^D$.
}
\edit{
The right-hand side of Eq.~(\ref{eq:tdse_tt_disc}) can be split into two parts. The first part, which using matrix notation is defined by
\begin{equation}
[ \Tmat \cdot \Pvec ]_{\kv m} \equiv \Big[ \frac{\kv^2}{2} - m \omega \Big] \Pdmk,
\end{equation}
is diagonal and can be easily computed numerically. The computation of the sum
\begin{equation}
[ \Vmat \cdot \Pvec ]_{\kv m} \equiv \sum_{\kv' m'} \Vdkp \Pdmkp \times  
\imath^{|m-m'|} \, J_{|m-m'|}\big( (\kv-\kv') \cdot \av_0(t) \big) \times \mathrm{e}^{-\imath (m-m')\phase} \label{eq:potential_operator}
\end{equation}
requires the main numerical effort as it is associated with the non-diagonal elements of $\Vmat$.
%The kinetic energy plus photon energy terms of TDSE in Eq.~(\ref{eq:tdse_tt}) 
%\begin{equation}
%\mathbf{T}_{\kv m}(t) = \Big[ \frac{\kv^2}{2} - m \omega \Big] \Pmk
%\end{equation}
%are diagonal and therefore
%the main numerical effort required to solve Eq.~(\ref{eq:tdse_tt}) stems from evaluation of the potential energy terms 
%\begin{equation}
%\mathbf{V}_{\kv\kv' mn}(t) = v_{\mathbf{K}-\mathbf{K}'} \tilde{u}_{\mathbf{K}'n}(t) \times  
%(\imath)^{|m-n|} \, J_{|m-n|}\big( (\kv-\kv') \cdot \av_0(t) \big) \times \exp(-\imath (m-n)\delta). \label{eq:potential_operator}
%\end{equation}
%\begin{equation}
%\mathbf{V}_{\kv\kv' mn}(t) = \sum_{\kv \kv'} \Big[ \, v_{\mathbf{K}-\mathbf{K}'}  \times  
%(\imath)^{|m-n|} \, J_{|m-n|}\big( (\kv-\kv') \cdot \av_0(t) \big) \times \exp(-\imath (m-n)\delta) \Big]. \label{eq:potential_operator}
%\end{equation}
%\begin{equation}
%\mathbf{V}_{\kv m}(t) = \sum_{\kv' m'} \Vkp \Pmkp \times  
%(\imath)^{|m-m'|} \, J_{|m-m'|}\big( (\kv-\kv') \cdot \av_0(t) \big) \times \mathrm{e}^{-\imath (m-m')\phase}. \label{eq:potential_operator}
%\end{equation}
In the field-free case ($\av_0=0$), the part (\ref{eq:potential_operator}) describes a convolution between momentum components $\kv$ of the wave function  and the potential. If the laser field is present ($\av_0 \neq 0$), additional terms proportional to $ J_{|m-m'|}\big( (\kv-\kv') \cdot \av_0(t) \big)$ enter the sum (\ref{eq:potential_operator}). They couple different Floquet channels $m$ and also modify the coupling between momentum components $\kv$. Nevertheless, the convolution form of the matrix $\Vmat$ in (\ref{eq:potential_operator}) is preserved, since the couplings depend only on the differences $\kv-\kv'$ and $m-m'$. Note, that $\Vmat$ and $\Pvec$ depend on time, which will be kept implicit for the brevity of notation. 
}

\edit{
%The numerical cost of solving a Floquet system increases rapidly with the number of Floquet orders $m$. 
%This scaling was often the reason, that Floquet-like approaches were mainly limited to analytical treatments or to just a few Floquet orders. 
%However, 
%the momentum space representation used here implies a 
The convolution form of the matrix $\Vmat$ allows one to apply the convolution theorem and to replace the convolution between potential and wave function,
%\tmp{in momentum and plane-wave representation}
described by (\ref{eq:potential_operator}), by their product in the Fourier domain. This greatly increases the speed of computation, in particular if a fast Fourier transformation (FFT) algorithm is used to convert to and from the Fourier domain.
}

\edit{
The convolution theorem strictly holds only for infinite or periodic vectors, which implies an expansion in $\kv$ and $m$ to infinite order. In practical numerical calculations, the necessity to use a finite size basis will normally violate the conditions for validity of the convolution theorem, consequently causing numerical errors.
Therefore,
we use an alternative approach that is based on the theory of Toeplitz matrices \cite{doi:10.1137/1.9780898718850}. It takes advantage of the convolution form of the matrix $\Vmat$ and allows one to use the FFT algorithm to accelerate the calculations. However, unlike the direct application of convolution theorem, the method based on the Toeplitz matrix theory is exact for vectors of finite size. 
%With the algorithm presented below one can truncate the Floquet basis without introducing additional \emph{numerical} errors due to the truncation. 
This approach is particularly useful to study Floquet systems, as it allows one to truncate the basis to only a few Floquet channels. 
}

\subsubsection{Description of the algorithm}

For a single Floquet channel, e.g., $m=m'$, the elements along the diagonal of the matrix $\Vmat$ in Eq.~(\ref{eq:potential_operator}) are equal, which follows directly from the momentum representation. Such a matrix is called a Toeplitz matrix and its properties are well known in the literature, see, e.g., \cite{doi:10.1137/1.9780898718850}. It can be fully described by a single row and column only. Furthermore, a product of a finite Toeplitz matrix with any vector can be performed exactly using the FFT algorithm. 


The algorithm to multiply a Toeplitz matrix $\Vmat$ with a vector $\Pvec$ is as follows (see Appendix \ref{sec:app_T} for a more detailed description): 
\begin{enumerate}
	\item A circulant vector $\Cvec$ is formed from the first column and the first row of the matrix $\Vmat$.$\bf$
	\item Zeros are appended to the vector $\Pvec$ to match the length of $\Cvec$.
	\item A Fourier transformation of both the circulant vector $\Cvec$ and the extended coefficient vector $\Pvec$ is performed.
	\item The two transformed vectors are multiplied and an inverse Fourier transformation is applied to the product.
\end{enumerate}
The first half of the final vector now stores the matrix--vector multiplication result, while the second half is discarded.

If the couplings between different Floquet channels are considered, i.e., $m-m' \neq 0$, then the matrix $\Vmat$ is of block form with all equal blocks on the same diagonal. Additionally, each block is of Toeplitz form. Such a matrix is called a Block Toeplitz matrix with Toeplitz Blocks (BTTB). The product of a BTTB matrix and any vector can be performed using a two-dimensional Fourier transformation algorithm in a similar way as a Toeplitz matrix--vector product, see Appendix \ref{sec:app_BTTB} for a detailed description. The approach can be further extended to an arbitrary number of dimensions. 

The algorithm to calculate the Toeplitz matrix-vector product can be considered as generalization of the well-known split operator technique (transformation to Fourier domain, multiplication and inverse transformation), see, e.g., \cite{Kosloff1988}, which is widely used to solve the TDSE. On the other hand, the algorithm presented here cannot be used to directly evaluate the product of a vector with a function of Toeplitz matrix, e.g., $ \exp(-\imath \Vmat \Delta t) \Pvec$. \hide{Nevertheless, our implementation reduces the number of Fourier and plane-wave components required to achieve high numerical accuracy. Therefore, for finite systems or truncated bases the Toeplitz matrix-vector multiplication algorithm outperforms the traditional split-operator technique.}
\editt{Nevertheless, the Toeplitz matrix-vector multiplication algorithm allows us to reduce the number of Fourier and plane-wave components required to achieve high numerical accuracy and allows it to outperform the traditional split-operator technique.}

\subsubsection{Time propagation}

Many different numerical methods to solve Eq.~(\ref{eq:tdse_tt_disc}) could be used, for example explicit Runge-Kutta or Arnoldi-Krylov algorithms. 
However, to take advantage of the BTTB symmetry of the potential matrix $\Vmat$, the matrix-vector multiplications involving $\Vmat$ must be implemented using efficient FFT routines with the method described above.
In this work the Taylor expansion propagator is used. This method relies on the expansion of the propagator over a discrete time-step $\Delta t$ in a Taylor series up to the desired order, so that the wave function expansion coefficients can be computed as
\begin{align}
\Pvec(t+\Delta t) = \exp\big[ -\imath ( \Tmat + \Vmat )\Delta t \big] \Pvec(t) = \nonumber\\
[ \mathbf{1} - 
\imath ( \Tmat + \Vmat )\Delta t 
-  \frac{1}{2}( \Tmat + \Vmat )^2 \Delta t^2 + \ldots] \Pvec(t),\label{eq:Taylor_expansion_propagator}
\end{align}
where each term in the expansion is evaluated iteratively. Hence, the numerical problem reduces to the evaluation of products $\Tmat\cdot\Pvec$, where $\mathbf{T}$ is diagonal, and $\Vmat\cdot\Pvec$, which is evaluated using the Toeplitz matrix--vector multiplication algorithm presented above. 
%The desired accuracy of the propagation can be easily controlled via the order of expansion, which can be dynamically chosen at each time step.
The accuracy can be controlled by choosing  the order of expansion at each time-step.
Although the propagator is not norm-conserving, if enough expansion orders are included norm conservation up to a desired numerical accuracy can be easily achieved.
In this work, the expansion was truncated once the norm of the corrections to the wave function coefficients dropped below $10^{-16}$.
The Taylor expansion propagator thus provides an accurate and reliable method to obtain a numerical solution to Eq.~(\ref{eq:tdse_tt}). Importantly, combined with the FFT algorithm for matrix-vector multiplication operations, large Fourier expansion orders $m$ can be treated explicitly.
More sophisticated propagation methods that also rely on matrix-vector products like Arnoldi-Krylov-propagators may be easily implemented.

%The momentum space TDSE in Eq. \ref{eq:tdse_tt} can be written using the matrix notation as
%\begin{equation}
%\imath \dot{\mathbf{u}} = \mathbf{T}\mathbf{u} + \mathbf{V}\mathbf{u}, \label{eq:tdse_matrix_form}
%\end{equation}
%where $\mathbf{u}$ is the wavefunction coefficient vector, $\mathbf{T}$ is the kinetic energy operator  which is diagonal in both momentum component $\kv$ and the Fourier index $m$ and $\mathbf{V}$ is the potential matrix operator that is dense and couples both $\kv$ and $m$.
%
%A single time-step $\Delta t$ in the time-propagation of Eq. \ref{eq:tdse_matrix_form} can be calculated by evaluating
%\begin{equation}
%\mathbf{u}(t+\Delta t) = \exp\big[ -\imath ( \mathbf{T} + \mathbf{V} )\Delta t \big] \mathbf{u}(t). \label{eq:time_step}
%\end{equation}
%which requires to calculate the exponent of the matrix $( \mathbf{T} + \mathbf{V} )$. 
%
%
%
%The use of FFT methods for coupling between different Floquet states is possible due to the Toeplitz symmetry of the coupling matrix. However, Toeplitz symmetry is not unique to the Fourier expansions used here, but can emerge in other important cases. For example, in solid state physics the interaction between different lattice sites depends only on the distance between sites and not on the absolute position of the sites, and therefore also lead to a Toeplitz symmetry for the coupling matrix. 

%\subsubsection*{Comparison to Split-Operator FFT methods}
%
%A convolution operator can be easily calculated by taking advantage of the convolution theorem. Indeed, the popular "split-operator FFT" method is based on this theorem. However, formally the convolution theorem is only valid when applied to infinite vectors. Therefore, in practical numerical applications that employ the convolution theorem, such as "split-operator FFT" method, the calculation "box" needs to be extended far outside the region of interest, where the wave function is effectively zero. However, if the calculation "box" is finite or if a truncation of the Hamiltonian is performed, the convolution theorem becomes inapplicable.
%
%In the following, we outline a general approach, that takes advantage of the FFT algorithm to calculate the convolution with the potential operator for vectors of any arbitrary length. The advantage of this approach is that it allows to introduce truncation of the full Hamiltonian up to desired Fourier order, without introducing numerical errors. That means that dynamics in a system with a few Fourier components can be calculated efficiently.
%
%One should note that in principle it is possible to apply the split operator technique and calculate the product  $ \exp(-\imath \mathbf{V} \Delta t) \mathbf{u}$ separately. However, 
%This is in contrast to the split operator FFT method despite its apparent similarity to the algorithm to calculate the product Toeplitz matrix-vector $\mathbf{Vu}$ (transformation to Fourier domain, multiplication and inverse transformation). The reason for this is that the algorithm used here takes into account the finite non-periodic nature of the matrices.


\subsection{Accuracy} \label{sec:accuracy}

\begin{figure}
\centering
%\includegraphics{../../Work/Solids/Code/Calculations/Benchmarking/spectra_benchmark_dos}
\includegraphics{Figure2}
%\includegraphics{spectra_benchmark_dos}
\caption{Energy resolved photoionization spectra after the pulse obtained by solving the TDSE in velocity gauge (black lines) compared with the spectra obtained using the time-dependent Floquet approach (a) for each Floquet channel $m$; (b) combined spectra from all Floquet channels.}
\label{fig:spectra_benchmark}
\end{figure}


The accuracy of the time-dependent Floquet approach developed in this work is verified by comparing the wave function obtained by directly solving the TDSE in velocity gauge in Eq.~(\ref{eq:tdse}) with the solution of the TDSE defined in Eq.~(\ref{eq:tdse_tt_disc}). In both cases, identical plane-wave basis and propagator routines of the TDSE were used. 
%Therefore, the any differences between the final wave functions were due to the accuracy of the two-timescale approach.

For all laser pulse parameters that were used in this work, the wave functions obtained from the time-dependent Floquet approach and by directly solving the TDSE in velocity gauge were found to match up to numerical accuracy, if sufficiently many Floquet channels $m$ were considered.
The accuracy of the time-propagation procedure is determined by the time-step and Taylor expansion order. The required number of Fourier components $m_{max}$ can be determined from the plane-wave basis set by requiring that $m_{max} \, \omega > |\kv|^2_{max}/2$, where $|\kv|_{max}$ is maximum momenta described by the plane-wave basis. Note, however, that in the Floquet formulation of TDSE in Eq.~(\ref{eq:tdse_tt}) both positive and negative Floquet channels have to be considered, i.e., $-m_{max} < m < m_{max}$.

An illustrative example is provided in Fig.~\ref{fig:spectra_benchmark} for ionization from a soft-core potential, which is defined in Sec.~\ref{sec:model}, with $\omega=1$ a.u. photon energy, $I=2.4 \times 10^{18} \, \text{W/cm}^2$ intensity and 5 fs full-width at half-maximum duration pulse. The spectra under similar laser pulse parameters were extensively investigated in previous works \cite{Toyota2008,Toyota2007,Tolstikhin2008,Demekhin2012,Baghery2017} and the calculation is further discussed in Sec.~\ref{sec:high_freq}, therefore here we only note that each Floquet channel provides the $m$-photon absorption channel, see Fig.~\ref{fig:spectra_benchmark}a. The final spectra, obtained by summation over all Floquet channels $m$, is indistinguishable from the spectra obtained by the direct solution of the TDSE in velocity gauge, see  Fig.~\ref{fig:spectra_benchmark}b. 

The approach was tested to be accurate for photon energies ranging from 0.05 to 1 a.u. Furthermore, it was accurate for pulses down to single cycle duration for both low and high frequencies. 
%It demonstrates that the ansatz used for the two-timescale wavefunction in Eq.~(\ref{eq:Ansatz}) fully accounts for the dynamics in the extended Hilbert space, governed by the two-time Floquet Hamiltonian. 
Therefore, the time-dependent Floquet formalism is capable of fully describing the dynamics driven by intense and short laser pulses using the cycle integrated Hamiltonian for arbitrary laser parameters.

The accuracy of the numerical procedure is further dictated by the quality of the plane-wave basis set. In all the calculations presented in this work, a converged basis set in terms of maximum momenta $|\kv|_{max}$ and spacing between momenta components $\dk$ is used. % However, the discussion on the properties of different basis sets to describe the photoionization dynamics is out of scope of this work. 
%It is, however, widely used and well documented in the literature, see, e.g., \cite{}, and therefore will not be further discussed here. 

Finally, note that atomic potentials with a long-range tail lead to a singularity at the origin in the momentum representation, i.e., $\Vt(\kv=0) \rightarrow -\infty$. This singularity could be treated by, for example, a Land\`{e} subtraction  procedure \cite{Norbury1994,Shvetsov-Shilovski2014}. Alternatively, the potential can be considered in a ``finite box", as in \cite{Jiang2008}. Finally, since adding a delta function to a momentum representation of a potential leads only to a trivial energy shift of the spectra, the singularity can be removed by introducing a new potential $\Vdkp + \xi \delta_{\kv-\kv'}$ into Eq.~(\ref{eq:tdse_tt_disc}) with $\xi \rightarrow \infty$, where $\delta_{\kv-\kv'}$ is the Kronecker delta function. Numerically this corresponds to setting all elements  $ \Vidx{\kv-\kv'=0}$ to zero.

\subsection{Performance}

\begin{figure}
	\centering
%	\includegraphics{../../Work/Solids/Code/Calculations/Benchmarking/scaling_element_number_II}
	\includegraphics{Figure3}
	\caption{Numerical effort as a function of the total number of basis elements, evaluated in terms of processor cycles spent per single time-step for calculations with different basis sizes and pulse durations. Black dots indicate the effort required to compute a single time-step; green crosses -- time effort required to update the Hamiltonian. Gray dashed line indicates the scaling $N\log(N)$.}
	\label{fig:scaling_element_number}
\end{figure}


\begin{figure}
\centering
%\includegraphics{../../Work/Solids/Code/Calculations/Benchmarking/scaling_basis_size}
\includegraphics{Figure4}
\caption{Numerical effort as a function of the number of Fourier components for different plane-wave basis sizes, evaluated in terms of processor cycles spent per single time-step; gray dashed line indicates the scaling $N\log(N)$.}
\label{fig:scaling_basis_size}
\end{figure}


The Toeplitz matrix approach described above allows to efficiently solve the time-dependent Floquet formulation of TDSE in Eq.~(\ref{eq:tdse_tt_disc}) using the FFT based matrix-vector multiplication. The use of FFT algorithm allows to achieve scaling proportional to $N\log(N)$ with respect to the size of the basis $N=N_K \times N_F$, where $N_K$ is the number of plane-waves and $N_F$ is the number of Fourier components. 
This scaling is illustrated as a function of the total number of basis elements $N$ in Fig.~\ref{fig:scaling_element_number} for different calculations with laser pulse parameters used in this work. The size of the plane-waves basis is varied between 512 and 4096 with the maximum momenta kept fixed. The number of Fourier components is varied between 1 and 401. The numerical effort is measured in terms of processor cycles spent solving the TDSE. The number of cycles is then divided by the total number of time-steps used, so that calculations using different pulse lengths could be directly compared. In addition, the expansion order of the Taylor propagator in Eq.~(\ref{eq:Taylor_expansion_propagator}) is kept fixed at 8. In an adaptive expansion scheme, the expansion order mainly depends on $|\kv|_{max} \times \Delta t$.
%depends weakly on the laser intensity and laser pulse duration and strongly on the maximum momentum in the plane-wave basis. 

The main effort required to solve the TDSE stems from updating the wave function at each time-step using the Taylor expansion method, which is illustrated by the black dots in Fig.~\ref{fig:scaling_element_number}. It scales proportionally to $N \, \log(N)$ as expected. The numerical effort required for different sizes of the plane-wave basis is shown in Fig.~\ref{fig:scaling_basis_size}. Again, the effort scales proportionally to $~N_F \, \log(N_F)$ with the number of Fourier components.
Additional numerical effort is required to update the elements of potential energy operator at every time step, since they depend on the laser field. This effort is illustrated by green crosses in Fig.~\ref{fig:scaling_element_number}. It can take up to 40\% of the total effort. However, it scales linearly with the total number of basis elements since only $N_K \times N_F$ elements are stored in memory.

The time-dependent Floquet method presented here cannot compete in efficiency with conventional approaches to solve TDSE that do not use Floquet expansion. The numerical effort required for the latter would be comparable to using just a single Floquet channel, see Fig.~\ref{fig:scaling_basis_size}. However, the time-dependent Floquet method does not aim to compete with the established approaches in terms of speed or accuracy, but rather to provide an efficient way to tackle large-scale Floquet problems. Therefore, the strength of the current approach is its ability to provide insight into the dynamics during the laser pulse, which is possible due to Floquet-like approach only. 


\subsubsection{Extension to more spatial dimensions}

Although this work is limited to one-dimensional potentials, the generalization to more dimensions $D$ for a  plane-wave basis in Cartesian coordinates is straightforward. However, such an approach does not take advantage of the symmetry of the potential and therefore in general requires a large number of basis elements to be included into the Hamiltonian, which scales as $N_K^D \times N_F$. The corresponding increase of numerical effort can be extrapolated from Fig.~\ref{fig:scaling_element_number} and  Fig.~\ref{fig:scaling_basis_size}. 

A direct extension of the method to, e.g., a spherical coordinate system is not straightforward. The advantage of the plane-wave basis in Cartesian coordinates is the separation of any arbitrary potential in the KH reference frame into time-independent and time-dependent parts, as can be seen from Eq.~(\ref{eq:VJexp}), which allows us to calculate the coupling between the plane-wave components at each time-step efficiently. 
We did not find such a simple form for the expansion of the KH potential into spherical harmonics for linearly polarized fields. 

A possible alternative approach to describe atoms in linearly polarized laser field beyond a single dimension is to use cylindrical coordinate system (see, e.g., \cite{Pont1990a}). Since the KH potential is symmetric around the laser polarization axis, a plane-wave expansion can be applied along this direction. 
The KH approach can also be formulated for a circularly polarized field, see, e.g., \cite{Pont1990a}.
%Another alternative is to formulate the KH approach for circularly polarized laser fields
Finally, multi-pole expansion of the KH potential can be used, which allows for an efficient description using conventional quantum chemistry methods \cite{Atabek1997}. 



%
%
%
\pagebreak
\section{Dynamics driven by short laser pulses using the Kramers--Henneberger--Floquet representation} \label{sec:results}

\subsection{Model system} \label{sec:model}

\begin{figure*}
	\centering
	%\includegraphics[]{Ladder_plus_Qdefect_05_twoScales_II}
	\includegraphics[]{Figure5}
	\caption{
	\editt{
	Eigenenergies (a) and effective quantum numbers (b) of the cycle-averaged soft-core potential (see Eqs.~(\ref{eq:V0}) and (\ref{eq:V_soft})) as a function of time along the pulse envelope (bottom axis) and classical excursion length $\as_0$ (top axis) for a maximal excursion length of $\as_0=10$ a.u. and a pulse defined in Eq.~(\ref{eq:pulse}).
	}
	}% occuring at $t=T_0/2$.}
	\label{fig:Ladder}
\end{figure*}


%With the numerical approach presented in this work one can explore the dynamics of atoms and molecules in terms of  \CAHt\ in the KH reference frame for short laser pulses. In particular, the explicit treatment of the Floquet Hamiltonian up to all orders required for numerical convergence allows us to test the KH approach for different driving frequencies and pulse durations. This is particularly important for driving frequencies, for which a direct analytical treatment of the Floquet problem is not possible. 

The time-dependent Floquet approach is illustrated using the example of a 1D model atom, described by a soft-core potential
\begin{equation}
V(x) = -\frac{1}{\sqrt{x^2+x_0^2}}, \label{eq:V_soft}
\end{equation}
which has been widely used to study the dynamics of atoms in high intensity laser fields both analytically and numerically \cite{Su1990,Su1991}. In this work the softening parameter is chosen to be $x_0^2=2$, which leads to a binding energy equal to that of a hydrogen atom $I_p$=0.5 a.u. 

The laser pulse with a peak electric field $F_0$ is defined in terms of the classical electron trajectory introduced in Eq.~(\ref{eq:excursion})  with a Gaussian envelope function \cite{Toyota}
%\begin{equation}
%\alpha(t) = \frac{F_0}{\omega^2} \, \frac{1}{1+8\ln(2)/(T\omega)^2}  \, \exp\left( -4 \ln(2) \left(\frac{t}{T}\right)^2 \right) \, \cos(\omega \, t).
%\end{equation}
\begin{subequations}\label{eq:pulse}
\begin{equation}
\alpha(t) = \frac{F_0}{\omega^2} \, P\left( T\omega \right)  \, \exp\left( -4 \ln2 \left(t/T\right)^2 \right) \, \cos(\omega \, t + \phase),
\end{equation}
\begin{equation}
P\left( T\omega \right) =  \frac{1}{1+8\ln2/(T\omega)^2}.
\end{equation}
\end{subequations}
An envelope of $T=5$ fs full-width at half-maximum (FWHM) is used, unless specified otherwise. For all except few cycle pulses $ P\left( T\omega \right) \sim 1$ holds.
%A corresponding vector potential of the laser pulse is given by
%\begin{equation}
%A(t) = 
%\end{equation}

Furthermore, laser intensity and frequency are chosen such that the maximum classical excursion length $\as_0$ is equal to 
\begin{equation}
\alpha_0 = \frac{F_0}{\omega^2} = 10 \, \text{a.u.}
\end{equation}
for all laser frequencies investigated. Since the KH potential depends only on the classical excursion length $\as_0$  the eigenenergies of the cycle-averaged potential will have identical time-dependence. Nevertheless, the dynamics will still depend on the frequency via the spacing between Floquet channels.  Therefore, the choice of a constant maximal $\as_0$ will allow one to clearly separate the role of the cycle-averaged potential from the role of the laser frequency.



%The KH potential $V_0(r, \av_0)$ in Eq. \ref{eq:V0} depends only on the classical excursion length $\av_0$. Hence, different laser frequencies may lead to the same value of $\av_0$, if $F_0$ is adjusted accordingly. Nevertheless, the dynamics will still depend on the frequency via the spacing between Floquet manifolds. Hence, in this work laser intensity and frequency are chosen such, that the maximum classical excursion length $\av_0$ remains constant and equal to 
%\begin{equation}
%	\alpha_0 = \frac{F_0}{\omega^2} = 10 \, \text{a.u.}
%\end{equation}
%for all laser frequencies investigated. This allows to clearly separate the role of the cycle averaged potential from the role of the laser frequency. Hence, in all cases presented below, the spectra of the cycle averaged potential will have identical behavior over time and only the energy spacing between different Floquet manifolds will be different.

The typical eigenenergy spectra of the 1D cycle-averaged potential $V_0(x, \as_0)$ are depicted in \fig \ref{fig:Ladder}a as a function of time during the pulse. They are obtained by diagonalizing the Hamiltonian in a single Floquet channel. %\textbf{ A strong modification of the spectra along the pulse takes place. The binding energy of the ground state strongly decreases with increasing pulse intensity. The binding energy of the excited states, on the other hand, tends to increase at low intensities, and starts decreasing at higher field strengths. This change of behavior happens at the value of $\av_0$ for which the cycle averaged potential becomes dichotomous (see \fig \ref{fig:KH_potentials}).}
The eigenenergies of \CAHt\ strongly depend on the instantaneous intensity of the laser pulse due to the widening and the formation of the dichotomy of the cycle-averaged potential in \fig \ref{fig:KH_potentials} for increasing electron excursion $\as_0$. 

\editt{
In \fig \ref{fig:Ladder}b the effective quantum numbers $n^*=\sqrt{-0.5/E_n}$ are plotted for bound states of \CAHt\ as a function of time along the laser pulse. Bound states of a hydrogenic potential would lead to an infinite series of equally spaced $n^*$, which is the case at the initial time in \fig \ref{fig:Ladder}b. Deviation from a pure hydrogenic potential lead to an uneven spacing of $n^*$, referred to as quantum defect. In the case of the cycle-averaged potential, the quantum defect is a result of the dichotomy of the potential and is clearly visible for the lowest eigenstates. On the other hand, although the energy of the higher $(n>3)$ eigenstates is lowered due to the widening of the cycle-averaged potential, $n^*$ stay approximately equidistant, indicating that these states are determined by the long-range Coulomb tail of the cycle-averaged potential and are not strongly influenced by its dichotomy.}
%More details on the spectra of the CAH for different potentials are available in, i.e., \cite{}.

All the calculations presented further in this work were done using 2048 plane-wave basis states with momenta equidistantly spaced by $\dk = 2\pi / 2000$ a.u. A time-step of $\Delta t = 0.1$ a.u. was used for the propagation, which we found sufficient to obtain converged ionization probabilities. The ground state was obtained by diagonalizing the Hamiltonian defined by Eq.~(\ref{eq:tdse_tt_disc}) with $\as_0=0$ for a single $m=0$ Floquet channel. Finally, the carrier-envelope phase was set to $\phase=0$ in all calculations.

\subsection{High-frequency} \label{sec:high_freq}

\begin{figure*}
	\centering
	%\includegraphics{Pictures/KH_and_Floquet_w1_5fs_a10}
	\includegraphics{Figure6}
	\caption{(a) Population in $m=0$ and $m=1$ Floquet channels as a function of time and
	(b) population in the ground state and the continuum states of the KH potential as a function of time for a laser pulse with frequency $\omega=1$ a.u., intensity $I=2.4 \times 10^{18} \, \text{W/cm}^2$ and duration $T=5$~fs FWHM pulse.}
	\label{fig:KH_and_Floquet_w1_5fs_a10}
\end{figure*}

%\begin{figure}
%\centering
%\includegraphics{Pictures/Floquet_w1_5fs_a10}
%\caption{Population in $m=0$ and $m=1$ Floquet manifolds as a function of time for $\omega=1$ a.u. frequency and $I=2.4 \times 10^{18} \, \text{W/cm}^2$ intensity laser pulse.}
%\label{fig:Floquet_w1_5fs_a10}
%\end{figure}
%
%\begin{figure}
%\centering
%\includegraphics{Pictures/KH_w1_5fs_a10}
%\caption{Population in the ground state and the continuum states of the cycle averaged potential as a function of time for $\omega=1$ a.u. frequency and $I=2.4 \times 10^{18} \, \text{W/cm}^2$ intensity laser pulse.}
%\label{fig:KH_w1_5fs_a10}
%\end{figure}


The KH approach was originally proposed in the context of high-laser frequencies, for which the underlying dynamics is by now mostly well understood, see \cite{Gavrila2002} and \cite{Popov2003} for comprehensive reviews. Therefore, the high-frequency case provides a good reference point to illustrate the influence of the pulse envelope on the dynamics induced by high-intensity laser fields using the time-dependent Floquet approach developed here. We choose a laser frequency of $\omega = 1 \, \text{a.u.} \sim 27 \, \text{eV}$, substantially larger than the field-free ionization potential. Accordingly, the peak laser intensity is set to $I=2.4 \times 10^{18} \, \text{W/cm}^2$, so that the maximum electron excursion length is $\as_0(t=0)=10$ a.u. Note that for these laser parameters, non-dipole effects contribute negligibly to the dynamics \cite{Førre2005}. Hence, we safely work in the dipole approximation.
%, in order to have a direct comparison with lower frequency cases presented later.

The population in the Floquet channels $m=0$ and $1$ as a function of time is depicted in \fig \ref{fig:KH_and_Floquet_w1_5fs_a10}a. During the initial part of the pulse around 30\% of the population is transferred from $m=0$ to the $m=1$ Floquet channel, indicating a one-photon absorption process. The population transfer stops, when the adiabatic stabilization regime is reached. Around the peak of the pulse, despite the rapid increase of field strength, population in each Floquet channel $m $ stays approximately constant, implying that the Floquet channels are decoupled as predicted by the high-frequency Floquet theory \cite{Marinescu1996}. As the field intensity decreases at the end of the pulse, another 20\% of the population is transferred to the $m=1$ Floquet channel by single photon absorption. The remaining Floquet channels $(m>1)$ contain $\ll 1$\% of the population after the pulse. 

\subsubsection{Non-adiabatic excitations}

\begin{figure}
\centering
%\includegraphics{Pictures/KH_w1_1fs_a10}
\includegraphics{Figure7}
\caption{Population in the ground state and $n=2, 4, 6$ excited states of the cycle-averaged potential in the $m=0$ Floquet channel as a function of time for the same pulse as in \fig \ref{fig:KH_and_Floquet_w1_5fs_a10} but with $T=1$~fs FWHM pulse.}
\label{fig:KH_w1_1fs_a10}
\end{figure}

Projecting the population in each Floquet channel $m$ onto the eigenstates of \CAHt\ reveals that the ionization process is adiabatic, see \fig \ref{fig:KH_and_Floquet_w1_5fs_a10}b.
%The field-free ground state smoothly transforms into the ground state of the cycle-averaged Hamiltonian and no substantial transitions due to the time-dependence of the eigenenergies of the cycle-averaged Hamiltonian occur.
The population in the $m=0$ subspace stays in the ground state throughout the dynamics and no substantial transitions due to the time-dependence of \CAHt\ takes place.
In this case, the ionization process can be described in terms of a discrete state that belongs to  the $m=0$ Floquet channel embedded into the continuum of states that belong to the $m=1$ channel, as assumed within the high-frequency approximation \cite{Marinescu1996}. Furthermore, once adiabatic stabilization sets in, the envelope of the laser pulse plays a minor role.

The adiabatic picture is not applicable for a shorter laser pulse with the same peak intensity. In this case, due to the rapid change of the eigenstates, non-adiabatic excitations from the ground to the excited states occur as is shown in \fig \ref{fig:KH_w1_1fs_a10} for a $T=1$~fs FWHM pulse. However, the population stays in the $m$=0 Floquet channels, i.e., no photons are absorbed from the field indicating that the excitations are induced by the envelope of the pulse. This is confirmed by excitations of even-parity states only, as absorption of a photon would lead to the excitation of odd-parity states. At the end of the pulse, the excited states of \CAHt\ transform to the corresponding field-free states. Such non-adiabatic transitions \cite{Toyota2009} were investigated in \cite{Toyota} using the envelope Hamiltonian formalism, where it was shown that they can be quantified using time-dependent perturbation theory.

%\begin{figure}
%\centering
%\includegraphics[width=\linewidth]{Pictures/high_freq_Floquet}
%\caption{Population in the ground state of LH atom in the m=0 Floquet manifold (blue line) together with the total population in the higher Floquet manifolds m$>$0 (green line). The pulse envelope is indicated by the gray line.}
%\label{fig:high_freq_Floquet}
%\end{figure}

\subsection{Intermediate-frequency} \label{sec:mid_freq}

\begin{figure}
	\centering
	%\includegraphics[height=0.8\textheight]{Pictures/Triplet_w040_5fs_a10}
	\includegraphics[height=0.8\textheight]{Figure8}
	\caption{(a) Population in $m=0, 1$ and $m=2$ Floquet channels as a function of time; 
	(b) population in the ground state of $m=0$ Floquet channel, $n=3, 5, 7$ excited states of $m=1$ Floquet channel and continuum states of \CAHt\ as a function of time for $\omega=0.4$ a.u. frequency and $I=9 \times 10^{16} \, \text{W/cm}^2$ intensity laser pulse; (c) energies of the $n=3, 5, 7$ excited states from $m=1$ Floquet channel as a function of time, with the ground state energy of $m=0$ Floquet channel.}
	\label{fig:Triplet_w041_5fs_a10}
\end{figure}

%\begin{figure}
%\centering
%\includegraphics{Pictures/Floquet_w041_5fs_a10}
%\caption{Population in $m=0, 1$ and $m=2$ Floquet manifolds as a function of time for $\omega=0.41$ a.u. frequency and $I=1 \times 10^{17} \, \text{W/cm}^2$ intensity laser pulse.}
%\label{fig:Floquet_w041_5fs_a10}
%\end{figure}
%
%
%\begin{figure}
%\centering
%\includegraphics{Pictures/KH_w041_5fs_a10}
%\caption{Population in the ground state, $n=3, 5, 7$ excited states and continuum states of the cycle averaged potential as a function of time for $\omega=0.41$ a.u. frequency and $I=9.9 \times 10^{16} \, \text{W/cm}^2$ intensity laser pulse.}
%\label{fig:KH_w041_5fs_a10}
%\end{figure}

%Formally, the KH approach requires  high frequency \cite{Gavrila2002}, i.e., the laser frequency should be larger than the binding energy of the cycle-averaged potential. 
A range of ``intermediate" frequencies can be defined, where the laser frequency is smaller than the binding energy of the field-free potential, but larger than the binding energy of the cycle-averaged potential at peak intensity. We will show, that for such ``intermediate" frequencies, the field-free ground state does not simply adiabatically connect to the ground state of \CAHt, as in the high-frequency case. Instead, it undergoes a series of crossings with excited states that belong to higher Floquet channels. 
%At these crossings, the population is distributed over the excited states leading to dynamics that is very sensitive to the pulse duration. 

We choose a laser frequency of $\omega = 0.4 \, \text{a.u.} \sim 10.9 \, \text{eV}$ and an intensity of $I= 9 \times 10^{16} \, \text{W/cm}^2$. Therefore, two photons are required for ionization, however the photon energy is still twice the binding energy of the cycle-averaged potential at the peak of the pulse, see \fig \ref{fig:Ladder}a.

The time-dependent populations in the $m=0, 1$ and $2$ Floquet channels, which are shown  in \fig \ref{fig:Triplet_w041_5fs_a10}a, immediately suggest that ionization proceeds in a sequential manner via the intermediate $m=1$ channel.
This is confirmed by the time-dependent population in the eigenstates of \CAHt\ depicted in \fig \ref{fig:Triplet_w041_5fs_a10}b. While the ground state of the $m=0$ Floquet channel is rapidly depopulated, the population is transferred to the odd-parity excited states of the $m=1$ channel, i.e., via one-photon transition. From these states, the population is slowly transferred to the $m=2$ channel, i.e., ionized via absorption of a further photon. 
%The ground and the lowest excited states strongly couple to many higher energy states, leading to the small oscillations seen in \fig \ref{fig:Triplet_w041_5fs_a10}b.
% and which cannot be easily explained using only few state dynamics. 

%Although the binding energy of the excited states is smaller then the photon frequency, the ionization of the excited states is still substantial, and no clear stabilization occurs. Furthermore, although the laser frequency was chosen to be in resonance with the $n=3$ excited state, for the 5 fs FWHM pulse duration this does not play a significant role, as the state are quickly shifted out of resonance.

\begin{figure*}
	\centering
	%\includegraphics{Pictures/KH_w040_scan_and_freq_pops_fwhm_a10}
	\includegraphics{Figure9}
	\caption{(a) Final population in the  ground state and $n=3, 5, 7$ excited states of \CAHt\ as a function of pulse duration for $\omega=0.4$ a.u. frequency and $I=9 \times 10^{16} \, \text{W/cm}^2$ maximum intensity laser pulse; the vertical dashed line indicates the pulse duration used in \fig \ref{fig:Triplet_w041_5fs_a10} and pulse durations when $n=0$ and $n=3$ state dominate; (b) frequency of population oscillations for each excited state;  (c) and (d) shows time-dependent population in the excited states for pulse durations when $n=0$ and $n=3$ state are dominantly populated after the pulse (indicated by vertical dashed lines in (a)).}
	\label{fig:KH_w041_scan_fwhm_a10}
\end{figure*}

%The dynamics depicted in \fig \ref{fig:Triplet_w041_5fs_a10}a and \ref{fig:Triplet_w041_5fs_a10}b can be understood in terms of avoided crossings of the ground and the ``photon-dressed" excited states of the $m=1$ Floquet manifold as their energies change during the pulse as illustrated in \fig \ref{fig:Triplet_w041_5fs_a10}c. It is conceptually similar to Stark-induced multi photon resonances discussed in, i.e., \cite{Schafer1997a}. The wavefunction amplitudes transferred between the excited states during the rising and falling parts of the pulse are expected to interfere leading to high sensitive to pulse duration, frequency and intensity. This is indeed the case, as is illustrated in \fig \ref{fig:KH_w041_scan_fwhm_a10}a. The excited state population after the pulse shows a rapid oscillation with changing pulse duration with clear state-dependent frequencies shown in \fig \ref{fig:KH_w041_scan_fwhm_a10}b. 
%%and resemble the dynamics of strongly driven few-level system.
%%\textbf{
%%The oscillation of population between the states with changing pulse duration shows an interplay between Rabi oscillations, interferences and effects due to coupling between multiple states, similar to those reported in \cite{Demekhin2011}. A detail analysis of this dynamics beyond of the scope of this work.
%%These frequencies increase linearly with photon frequency and electric field of the laser pulse with a much stronger dependence for high-frequencies $\nu > 0.1$ a.u. than for low frequencies $\nu \leq 0.1$ a.u.. However, the difference between oscillation frequencies associated to different excited states stays approximately fixed as field parameters are varied.}

%The origin of oscillations is further elucidated by the evolution of excited state population during the pulse in \fig \ref{fig:KH_w041_scan_fwhm_a10}c and \fig \ref{fig:KH_w041_scan_fwhm_a10}d for pulse durations, when either $n=0$ or $n=3$  are maximized. Initially the population dynamics in both cases is very similar. Only around the time when the ``field-dressed" states experience the second avoided crossings at the end of the pulse, clear difference emerge, indicating that interference of wavefunction amplitude created in different states determines the final populations. This is further confirmed by the disappearance of the high-frequency peaks in \fig \ref{fig:KH_w041_scan_fwhm_a10}b for longer pulses, for which the population is completely depleted from bound states by the end of the pulse.
%%\textbf{However, due to strong couplings and a large number of states involved, the oscillating behavior is not regular and cannot be easily explained with a simple two state model, as was done in \cite{Schafer1997a}.}
%%These crossings between the ground and the $n=3, 5, 7$ excited states are depicted in \fig \ref{}. Comparing their positions with population in the corresponding states in \fig \ref{fig:KH_w041_5fs_a10}, it is evident that they are responsible for the depopulation from the ground state. This can also be seen  in \fig \ref{}, where the depicted energies are obtained by diagonalizing the Hamiltonian that includes only $m=0$ and $m=1$ Floquet manifolds. 
%%In this representation, the dynamics can be described as Landau-Zener transitions over a series of avoided crossings. 

%
%
%

The dynamics in \fig \ref{fig:Triplet_w041_5fs_a10}a and \ref{fig:Triplet_w041_5fs_a10}b can be understood in terms of the evolution of \CAHt\ eigenstates during the pulse, shown in \fig \ref{fig:Triplet_w041_5fs_a10}c.
At the beginning of the pulse the energy of the ground states rapidly increases and undergoes a series of crossings with the excited states that belong to the $m=1$ Floquet channel. At each of these crossings, a fraction of ground state population is transferred to the excited state. As the energy of the ground state decreases at the end of the pulse, the population is exchanged again at the second crossing. Between these crossings, a small but significant coupling of the states leads to small Rabi oscillations that are seen around $t=0$ in \fig \ref{fig:Triplet_w041_5fs_a10}b.

%The two transitions that occur at the crossings of the ground and excited states of \CAHt\ lead to Landau-Zener-St\"{u}ckelberg interference \cite{Shevchenko2010} and are also similar to the Stark-induced multi photon resonances discussed in \cite{Schafer1997a}. 

The two transitions that occur at the crossings of the ground and excited states of \CAHt\ lead to interference that depends on the phase accumulated in each state in between. This phase in turn depends on both the energy differences and the couplings between the states. Since the time between two crossings depends on the pulse duration it strongly influences the final population after the pulse, as seen in \fig \ref{fig:KH_w041_scan_fwhm_a10}a. 

\editt{
The origin of oscillations in \fig \ref{fig:KH_w041_scan_fwhm_a10}a is further elucidated by the evolution of excited-state populations during the pulse. In \fig \ref{fig:KH_w041_scan_fwhm_a10}c and \fig \ref{fig:KH_w041_scan_fwhm_a10}d these populations are shown for pulse durations, when either $n=0$ or $n=3$ state is predominantly populated after the pulse. Initially the dynamics in both cases is very similar. Clear differences emerge only after the second crossing between the states, indicating that interference effects determine the final populations. }

The final populations oscillate with well-defined frequencies as the pulse duration changes, see \fig \ref{fig:KH_w041_scan_fwhm_a10}b. 
The biggest amplitude oscillation is between the ground and $n=3$ state, which is to be expected since the coupling between these states is at least a factor of two larger than between any other states and $n=3$ state is the first one to undergo a crossing with the ground state. The time-window of strong interaction with the ground state is also longest for the $n=3$ state, since the crossing occurs at the beginning of the pulse, where the energy--time gradient is not as steep as for higher-energy states.

Unlike the final population of each state in \fig  \ref{fig:KH_w041_scan_fwhm_a10}a, which requires one to consider all interactions, we found that the frequencies of oscillation of final populations in \fig \ref{fig:KH_w041_scan_fwhm_a10}b are determined mainly by the dynamics of the ground and a single excited state.
The presence of higher-energy states does not significantly perturb these frequencies, since they mainly depend on  the phase difference accumulated between the times of crossing of the two states. 
\editt{
These times in turn depend on their energy difference -- the higher the energy of the excited state, the later the first crossing will occur. Higher-energy states contribute to smaller frequencies reducing their influence. 
}

Interaction of the ground state with any individual excited state can be readily described by a Landau-Zener-St\"{u}ckelberg (LZS) interference process \cite{Schafer1997a, Shevchenko2010}. In case of multiple states with non-trivial time-dependence of energies and couplings, the interconnected LZS transitions lead to complicated and rich dynamics. Nevertheless, characteristic features prevail. The ground state will be depleted sequentially transferring population to higher excited states at later times. Therefore later crossings will become less important due to weaker couplings and the smaller population available for transfer. Hence, the traces of single state dynamics show up in \fig \ref{fig:KH_w041_scan_fwhm_a10}b even for very high laser intensities.


\subsection{Low frequencies} \label{sec:low_freq}

\begin{figure}
\centering
%\includegraphics{Pictures/Floquet_w005_30fs_a10}
\includegraphics{Figure10}
\caption{Population in $|m| \leq 12$ Floquet channels as a function of time for $\omega=0.057$ a.u.  frequency and $I=3.7 \times 10^{13} \, \text{W/cm}^2$  intensity laser pulse. The gray line indicates the envelope of the laser pulse.}
\label{fig:Floquet_w005_30fs_a10}
\end{figure}



\begin{figure*}
\centering
%\includegraphics{Pictures/KH_bound_and_Floquet_w005_30fs_a10}
\includegraphics{Figure11}
\caption{Population in (a) the bound states of \CAHt\ and (b) in the Floquet channels at the peak of the laser pulse for $\omega=0.057$ a.u. frequency and $I=3.7 \times 10^{13} \, \text{W/cm}^2$ intensity laser pulse.}
\label{fig:low_freq_combined}
\end{figure*}


Although the KH approach was originally proposed to study the interaction of atoms with high-frequency laser fields, it was speculated that it could also be applicable for low-frequency and high intensity radiation \cite{Popov1999,Smirnova2000,Popov2011,Morales2011}. More recently, the KH approach and in particular the properties of the cycle-averaged potential was used to explain the nonlinear Kerr effect in laser filamentation \cite{Patchkovskii2013} and acceleration of neutral atoms in laser fields \cite{Wei2017}.
%Therefore, it is interesting to apply the two-timescale Floquet formalism developed here to investigate the applicability of the KH approach for frequencies much smaller then the ionization potential.
The time-dependent Floquet formalism developed here allows one to directly investigate the KH approach for frequencies that are much smaller than the ionization potential. 

We choose the laser frequency of $\omega=0.057$ a.u., which corresponds to $\lambda=800$ nm wavelength radiation, and $I=3.7 \times 10^{13}$W/cm$^2$ intensity, so that the maximal electron excursion length is again $\as_0=$10 a.u. Therefore, the eigenenergies of \CAHt\ and its dependence on the pulse shape shown in \fig \ref{fig:Ladder} is identical to the high and intermediate-frequency cases analyzed above. The FWHM duration of the pulses is set to $T=30$~fs. %, to assure smooth variation of the pulse envelope within a single cycle of the field. 
%However, the essential results presented here are not sensitive the pulse envelope. 
However, the essential results presented here do not depend sensitively on the duration of the pulse.
Note that in order to obtain converged results, 201 Floquet channels ($\pm 100 \, \omega$) are treated explicitly in the numerical calculation.

The population in Floquet channels as a function of time is plotted in \fig \ref{fig:Floquet_w005_30fs_a10}. 
Almost all of the population is transferred to higher Floquet channels at the peak of the pulse and then returns to the $m=0$ channel at the end of the pulse.
Therefore, these transitions are virtual, which is in contrast to high and intermediate-frequency cases. This is not surprising, however, since for the given laser field parameters the total ionization is less than $1\%$ and most of the population is expected to stay in the ground state after the pulse. 
%Importantly, during the peak of the pulse, the population of the system is distributed over many Floquet manifolds, see \fig \ref{}.

Projecting the population at the peak of the pulse onto the instantaneous eigenstates of \CAHt\ in \fig \ref{fig:low_freq_combined}a
%for the bound part of the spectrum. It
reveals that the population is distributed over many states. Crucially, no single state is dominating. 
Also, many Floquet channels are populated during the peak of the pulse, as is shown in \fig \ref{fig:low_freq_combined}b.
An increase of the field intensity leads to a broadening of the distribution over both the excited states $n $ and also over Floquet channels $m$.

Virtual excitations created in multiple Floquet channels can be understood qualitatively within the Floquet picture. Since the interaction strength between states that belong to different channels is much larger than the energy spacing between them, a quasi-continuum of states is created. 
In this situation, the eigenstates of \CAHt\ do not correspond to any adiabatic or nearly adiabatic states of the  field driven system. Therefore, the wave function, which stays nearly identical to that of the field-free ground state, is distributed over many excited states in the KH reference frame. 
The redistribution occurs at avoided crossings between states that belong to different Floquet channels, similarly as in the intermediate-frequency case. However, for low frequencies many more avoided crossings become important.

For sufficiently large peak field strengths a regime may exist, where \CAHt\ becomes applicable \cite{Popov1999,Smirnova2000,Popov2011}. However, by the time this intensity of the laser pulse is reached, the wave function is already distributed over the excited states of \CAHt. \hide{Therefore, the transformation of the field-free ground state to the ``field-dressed" KH picture, which is determined by the switching on of the laser pulse, is an essential step in order to apply the KH approach at low laser frequencies.}
\editt{Therefore, in order to apply the KH approach at low laser frequencies, it is essential to consider the transformation of the field-free ground state wave function to the ``field-dressed" KH picture during the switching-on of the pulse.}
%However, this spectra is discrete leading to a reversible population dynamics. Non-reversible transitions take pace only into states that belong to the ionization  continuum. However, for laser parameters studied here the fraction of population in the state is small.



%
%
%
%\pagebreak
%\section{Conclusion}
%
%%Floquet basis is only formally defined for plane waves, however in practice we have to deal finite and often very short laser pulses. In addition, although Floquet states provide an exact basis for the field-dressed system, they are also not bound in energy. Hence, the system formally does not have a ground state and there is no easy way to determine, which state the system will occupy once the laser field is switched on. Hence, of crucial importance is the question of how a field-free system transforms into a Floquet system as the laser field is switched on. 
%
%An approach to study the dynamics of atoms exposed to high intensity laser fields is proposed. By combining the  in the KH frame with a two-timescale Floquet approach, numerical calculations for short laser pulses can be performed while using the cycle-averaged basis that is particularly suited, but not limited to, high frequency fields.
%
%A numerical method is also presented that is ideally suited to Floquet Hamiltonian in the KH reference frame and allows to effectively treat a large number of coupled Floquet manifolds. It is based on the generalization of the split-operator method and relies on the Fast Fourier Transform algorithm. However, it can be applied to finite or truncated systems without inducing additional numerical errors. 
%
%%The approach is applied to study the ionization of a model potential in different frequency regimes with a particular focus on the role of a short laser pulse in the underlying dynamics.Tree different regimes are investigated, that differ by the ratio of the laser frequency to the binding energy of the field free and cycle averaged potentials and lead to very different dynamics. 
%The approach is used to illustrate the role of a short and intense laser pulse in the ionization of an atom from the perspective of the KH approach. Three different frequencies are investigated that differ by their ratio to the ionization potential and reflect different roles of the envelope of a short laser pulse in ionization.
%
%In the case of frequencies larger than the binding energy of the field free potential, the eigenstates of the CAH in the KH reference frame provide an excellent approximation to the adiabatic of the field-driven system. On the other hand, for short pulses, non-adiabatic transitions to the excited states take place, that are induced by the time variation of the CAH along the laser pulse.
%
%For laser frequencies much smaller than the binding energy, the eigenstates of the CAH fail to represented adiabatic states of the laser driven system and the population is quickly distributed over many excited states and Floquet manifolds as the pulse is switched on.
%
%The envelope of the laser pulse is particularly important for laser frequencies, that are between the binding energies of the field free potential and the cycle-averaged potential. For these frequencies, the population is transferred from the ground to the excited states of the CAH. The excitation occurs at crossings of states belonging to different Floquet manifolds and are very sensitive to the pulse duration.
%
%Free Electron Lasers are capable to provide short and intense laser pulses that open the path to study high-frequency non-perturbative physics. Although the KH approach is particularly suited to study high frequency physics, up till now its consistent extension to short pulses was lacing. We believed the methods presented here close this gap and are therefore ideally suited to study interaction for atoms with short and intense pulses provided by the FEL facilities.
%
%
%
%%The Kramers-Henneberger picture and the model of Kramers-Henneberger atom, is a particular case of a Floquet theory. Since a Floquet Hamiltonian formally does not have a ground state, as the Floquet states form a infinite energy ``ladder", of crucial importance is the question, how a field-free eigenstates of a potential transform into field-dressed Floquet states as the laser field is switched on. We  have address this question for the special case of Kramers-Henneberger atom model for which we have demonstrated how the shape of the pulse can significantly influence the dynamics of the Kramers-Henneberger atom model.

\pagebreak
\section{Summary and conclusions}

We have developed a time-dependent Floquet approach formulated in the KH reference frame to study dynamics driven by short and intense laser pulses. It constitutes a systematic and flexible extension of the Envelope Hamiltonian \cite{Toyota} applicable for arbitrary frequencies and provide a convenient and efficient way to propagate Floquet Hamiltonians. 

\editt{Numerical application of Floquet approaches is often hampered by the rapid increase of the number of Fourier components required to describe the Hamiltonian.}
\hide{Numerical application of Floquet approaches is often hampered by the rapid growth of the Hamiltonian which requires an increasing number of Fourier components.} Therefore, we have devised an efficient numerical procedure to propagate the Floquet Hamiltonian which is able to overcome the hurdle of large expansions. Indeed, we have performed calculations for laser parameters, for which several hundred Floquet channels had to be considered explicitly. 
Key element is the formulation of the problem in the momentum representation, which is particularly suited for the KH reference frame as it allows us to separate the components of the field-free potential from the field-dependent ones.
We further use the formalism of Toeplitz matrices and the FFT algorithm to achieve a favorable scaling with Floquet channels. However, unlike the split-operator methods that also rely on FFT, the Toeplitz approach is numerically exact for finite-size matrices. For Floquet problems it allows us to truncate the basis to only several Floquet channels. Yet, the method can be applied to any other Fourier basis to achieve an efficient and accurate propagation.

The main advantage of the method is its ability to extend the KH approach, which is particularly suited for high-frequency and high-intensity fields, to the limit of very short pulses. Thereby, we can investigate physical effects that emerge at high intensities and can only be understood by explicitly considering the time-evolution of the pulse envelope. 

We have shown that the pulse envelope exerts control over two types of dynamics. For very short pulses, the rapid 
change of the eigenstates of \CAHt\ over time leads to non-adiabatic excitations. They are induced by the pulse envelope and therefore do not involve the absorption of any photons from the field. Thus, even for very high frequencies they can lead to significant population in low lying bound states. 

The second type of transitions, sensitive to the pulse envelope, occurs at the crossings between the discrete eigenstates of \CAHt\ that belong to different Floquet channels, as their energy changes along the pulse. 
Although dynamics at each crossing can be easily understood in terms of Landau-Zener-St\"{u}ckelberg theory, in our case multiple states are strongly coupled evading simple interpretations. Nevertheless, strong features due to the coupling between individual states can be discerned, which is quite remarkable, considering the high intensities used. They lead to a large sensitivity of final state populations to the pulse duration providing a possible route for their coherent control.

An extreme case for our KH approach is the low-frequency limit, when the photon energy is much smaller than the binding energy of the electron. In this case, the population is transferred between the bound states of \CAHt\ at their crossings, which are very dense due to the small energy spacing between the Floquet channels, resulting in a rapid distribution of the population over many eigenstates of \CAHt\ before any populations has a chance to reach continuum states.

To summarize, the time-dependent Floquet approach presented here provides a convenient basis for short laser pulses and for all but the smallest photon energies. In all investigated cases, including few-cycle pulses, the approach was able to provide accurate numerical results indistinguishable from the ones obtained using conventional techniques of propagating the TDSE. However, unlike the conventional TDSE propagators, the time-dependent Floquet method allows one to obtain insight into dynamics that crucially depend on the pulse envelope. Such dynamics  will become particularly important for short and intense pulses generated by FEL facilities, which often devise unusual pulse
shapes.

The richest envelope-dependent dynamics is observed in the intermediate-frequency range. For multi-electron systems, this energy range will be much more extended due to the ubiquitous presence of core and double excitations, which lead to a rich energy structure even for very high photon energies. Therefore, we expect that for such systems the time-dependent Floquet formalism presented here will be even more valuable.




%%%%%%%%%%%%%%%%%%%%%%%%%%%%%%%%%%%%%%%%%%%%%%%
%
%
% APPENDIX
%
%
%%%%%%%%%%%%%%%%%%%%%%%%%%%%%%%%%%%%%%%%%%%%%%%
\pagebreak
\appendix
\section{Derivation of the time-dependent Floquet formalism}\label{sec:app_expansions}

\subsection{Expansion of the Kramers--Henneberger potential into Fourier components}

%Splitting the electron displacement  $\av(t)$ into the non-periodic envelope $\av_0(t)$, described by the time variable $t$ and the periodic oscillation $\cos(\omega t' + \phase)$, described by the time $t'$
%\begin{equation}
%\av(t) \rightarrow \av(t, t') = \av_0(t) \, \cos(\omega t' + \phase). \label{eq:app_excursion_two_timescale}
%\end{equation}

In this work, the potential in the KH reference frame is expanded into Fourier and plane-wave components as
\begin{equation}
V(\rv, t) = \sum_{m} \intK \; \Vmk \, \mathrm{e}^{-\imath m \omega t} \, 
\e{\imath \kv \cdot \rv}. \label{eq:app_Fourier_expansion_V_a}
\end{equation} 
To determine the expansion coefficients $\Vmk$, let us first consider only the expansion into the Fourier components
\begin{equation}
V(\rv, t) = \sum_{m} \, \Vidx{m}(\rv, t) \, \e{-\imath m \omega t},\label{eq:app_Fourier_only_expansion_V} 
\end{equation}
which are determined by
\begin{equation}
\Vidx{m}(\rv, t) = \frac{1}{T_{\omega}} \, \int_0^{T_\omega} \, \dd t' \, \Vb(\rv, t, t') \, 
\e{\imath m \omega t'}.\label{eq:app_Fourier_expansion_V_b}
\end{equation}
The time-integration in Eq.~(\ref{eq:app_Fourier_expansion_V_b}) is performed only over $t'$, i.e., the periodic oscillation of the two-time potential
\begin{equation}
\Vb(\rv, t, t') = V\big(\rv + \av_0(t)\cos(\omega t'+\phase)\big).\label{eq:app_Fourier_expansion_V_c} 
\end{equation}
Nevertheless, this provides an \emph{exact} representation of the full time-dependent potential $V(\rv, t)$, as is easily verified 
by inserting Eq.~(\ref{eq:app_Fourier_expansion_V_b}) into Eq.~(\ref{eq:app_Fourier_only_expansion_V}):
\begin{multline}
V(\rv, t) =
\sum_m \Big( 
\frac{1}{T_\omega} \int_0^{T_\omega} \, \dd t' \, \Vb(\rv, t, t') \, \e{\imath m \omega t'}
\Big) \e{-\imath m \omega t} = \\
=\frac{1}{T_\omega} \int_0^{T_\omega} \, \dd t' \, \Vb(\rv, t, t') \, \sum_m \e{\imath m \omega (t'-t)} = \\
=\int_0^{T_\omega} \, \dd t' \, \Vb(\rv, t, t') \, \delta_\omega(t'-t) = V(\rv, t), \label{eq:V_tt_to_t}
\end{multline}
where the definition of the Dirac delta function
\begin{equation}
\delta_\omega(t'-t) = \frac{\omega}{2\pi} \sum_m \e{\imath m \omega (t'-t)} = 
\frac{1}{T_\omega} \sum_m \e{\imath m \omega (t'-t)} \label{eq:delta_def}
\end{equation}
was used.
Note that $\delta_\omega(t'-t)$ is periodic, with the period $T_\omega=2\pi/\omega$. However, its use is justified since $\Vb(\rv, t, t'+T_\omega)=\Vb(\rv, t, t')$ and the integration in Eq.~(\ref{eq:V_tt_to_t}) can be limited to the range $t' \in [0, T_\omega)$.

%
%
%

\subsection{Expansion of the Kramers--Henneberger potential into plane-wave components}

The Fourier components $\Vidx{m}(\rv, t)$ are further expanded into the basis of plane-waves. Using the definition of the potential in Eq.~(\ref{eq:app_Fourier_expansion_V_b}) and Eq.~(\ref{eq:app_Fourier_expansion_V_c}) the expansion coefficients can be written as (see also \cite{Yao1992} for a similar derivation)
\begin{subequations}
	\begin{align}
	\Vmk &= \KNORM \int \dd^3 r \, \Vidx{m}(\rv, t) \, \e{-\imath \kv \cdot \rv} \\
	&= \KNORM \frac{1}{T_\omega} \, \int_0^{T_\omega} \, \dd t' \, 
	\Big(\int \dd^3 r \, \Vb(\rv, t, t') \, \e{-\imath \kv \cdot \rv} \,\Big)
	\e{\imath m \omega t'} \\
	&=\KNORM \frac{1}{T_\omega} \, \int_0^{T_\omega} \, \dd t' \, 
	\Big(\int \dd^3 r \, V\big(\rv +\av(t, t')\big) \, \e{-\imath \kv \cdot \rv} \,\Big)
	\e{\imath m \omega t'} \\
	%&= \KNORM \frac{1}{T_\omega} \, \int_0^{T_\omega} \, \dd t' \, 
	%\Big(\int \dd^3 r' \, V(\rv') \, \e{-\imath \kv \cdot \rv'} \, \e{\imath \kv \cdot \av(t, t')} \, \Big)
	%\e{\imath m \omega t'} \\
	&= \frac{1}{T_\omega} \, \int_0^{T_\omega} \, \dd t' \, 
	\Vk \, \e{\imath \kv \cdot \av(t, t')} \,
	\e{\imath m \omega t'},
	\end{align}
\end{subequations}
where $\Vk$ is the projection of the field-free potential on the $\kv$-th plane-wave
\begin{equation}
\Vk = \KNORM \int_{-\infty}^{\infty} \dd^3 r' \, \mathrm{V}(\rv') \, \e{-\imath \kv \cdot \rv'},
%\Vk = 
%\KNORM \int \dd^3 r' \, \mathrm{V}(\rv') \, \e{-\imath \kv \cdot \rv'}
%\xrightarrow{L \rightarrow \infty} 
%\frac{1}{(2 \pi)^3} \int_{-\infty}^{\infty} \dd^3 r \, \mathrm{V}(\rv) \, \e{-\imath \kv \cdot \rv}
\end{equation}
with $\rv' = \rv + \av(t, t')$.
%In the above, box discretization of the plane-waves with position-space size $L=2\pi/\Delta K$ was assumed.
Using
\begin{equation}
\av(t, t') = \av_0(t) \, \cos(\omega t' + \phase)
\end{equation}
and 
applying the Jacobi-Anger expression, the plane-wave Fourier components $\Vmk$ can be further expressed as
\begin{equation}
\Vmk = \Vk \, \imath^{|m|} \, J_{|m|}\big( \kv \cdot \av_0(t)\big) \, \e{-\imath m \phase},\label{eq:app_Vmk_equation} 
\end{equation}
where $J_m$ is the ordinary Bessel function of the 1st kind of order $m$.

\subsection{Derivation of the time-dependent Schr\"{o}dinger equation for the coupled Fourier and plane-wave components}

After expanding the wave function in terms of Fourier and plane-wave components
\begin{equation}
\Psi_{\khsub}(\rv, t) = \sum_{m} \intK \; \Pmk \, \e{-\imath m \omega t} \, \e{\imath \kv \cdot \rv} \label{eq:app_ansatz_wf_fourier}
\end{equation}
the expansion coefficients $\Pmk$ are determined by inserting Eqs.~(\ref{eq:app_Fourier_expansion_V_a}) and Eq.~(\ref{eq:app_ansatz_wf_fourier}) into the TDSE
\begin{equation}
\imath \frac{\partial}{\partial t} \Psi_\khsub(\rv, t) = -\frac{1}{2} \boldsymbol{\nabla} \Psi_\khsub(\rv, t) + V(\rv, t) \Psi_\khsub(\rv, t).
\end{equation}
After projecting on the $\kv$-th plane-wave component, the TDSE becomes
\begin{multline}
\sum_m \e{-\imath m \omega t} \Big( \imath \frac{\partial}{\partial t} \Pmk +m \omega \, \Pmk \Big) = \\
\sum_m \e{-\imath m \omega t} \Big( \frac{\kv^2}{2} \Pmk + \sum_{m'} \intK' \; \Vmkp \, \Pmkp \e{-\imath m' \omega t} \Big).
\end{multline}
Collecting the terms proportional to $\e{-\imath m \omega t}$ we obtain
\begin{multline}
\sum_m \e{-\imath m \omega t} \Bigg[  \Big(\frac{\kv^2}{2} -m \omega \Big) \Pmk + \sum_{m'} \intK' \; \Vmpkp \, \Pmpkp - \imath \frac{\partial}{\partial t} \Pmk \Bigg] = 0, \label{eq:collected_coefficients}
\end{multline}
where we have used that
\begin{multline}
\sum_{m,m'} \e{-\imath m \omega t} \e{-\imath m' \omega t} \, \Vmkp \, \Pmkp = \\
= \sum_{m''} \e{-\imath m'' \omega t}  \sum_{m} \Vt_{m''-m}(\kv-\kv', t) \, \Pmkp \\
= \sum_{m} \e{-\imath m \omega t} \sum_{m'} \Vmpkp \, \Pmpkp
\end{multline}
and where $m'' = m+m'$ was used in the second line together with relabeling of the indexes in the last line.
Eq.~(\ref{eq:collected_coefficients}) is satisfied, if the expression in brackets is zero for all times $t$ and for all $m$ and $\kv$. This condition leads to the coupled system of equations for the Fourier and plane-wave components of the wave function
\begin{equation}
\imath \frac{\partial}{\partial t} \Pmk  = \Big(\frac{\kv^2}{2} -m \omega \Big) \Pmk + \sum_{m'} \intK' \; \Vmpkp \, \Pmpkp. \label{eq:tdse_Fourier_components}
\end{equation}
Inserting the definition for $\Vmpkp$ from Eq.~(\ref{eq:app_Vmk_equation}) leads to the Eq.~(\ref{eq:tdse_tt}), which is the main equation used in this work.

%\pagebreak
%\section{Connection to the two-timescale approach}\label{sec:app_tt}
%
%Let us define a scalar time-dependent potential $\mathrm{V}(\rv, t)$ that depends on position $\rv$ and time $t$. $\mathrm{V}(\rv, t)$ could describe a general single electron potential coupled to an electromagnetic field in the length gauge description. In this work, we focus on the KH reference frame \cite{Henneberger1968a,Gersten1974}, where $\mathrm{V}(\rv, t)$ describes an oscillating atomic potential 
%\begin{equation}
%\mathrm{V}(\rv, t) \equiv \mathrm{V}(\rv + \av(t))
%\end{equation}
%and where the time dependence enters via the electron displacement 
%\begin{equation}
%\av(t)=\av_0(t)\,\cos(\omega t + \phase).
%\end{equation}
%Non-periodic part of the electron displacement $\av_0(t)$ can be separated from the periodically oscillating part $\cos(\omega t + \phase)$ by introducing two time variables $t$ and $t'$:
%\begin{equation}
%\av(t, t') = \av_0(t) \, \cos(\omega t' + \phase).
%\end{equation}
%Furthermore, we can define a potential that depends on both $t$ and $t'$:
%\begin{equation}
%\mathrm{V}(\rv, t, t') = \mathrm{V}(\rv + \av(t, t')).
%\end{equation}
%It immediately follows that 
%\begin{subequations}
%	\begin{align}
%	\av(t) \equiv \av(t, t') \Big|_{t=t'}, \\
%	\mathrm{V}(\rv, t) \equiv \mathrm{V}(\rv, t, t') \Big|_{t=t'}.
%	\end{align}
%\end{subequations}
%We can now write $\mathrm{V}(\rv, t)$ as
%\begin{equation}
%\mathrm{V}(\rv, t) = \int_{-\infty}^{\infty} \dd t' \, \mathrm{V}(\rv, t, t') \, \delta(t'-t).
%\end{equation}
%%where $t'$ is integrated over a single period $T$ of the periodic oscillation.
%Since $\mathrm{V}(\rv, t, t'+T)=\mathrm{V}(\rv, t, t')$, the integration domain can be limited to $t' \in [0, T)$ and a periodic Dirac delta function
%\begin{equation}
%\delta_\omega(t'-t) = \frac{\omega}{2 \pi} \sum_m \, \mathrm{e}^{\imath m \omega (t'-t)},
%\end{equation}
%can be used
%so that the expression for the potential $\mathrm{V}(\rv, t)$ becomes
%\begin{subequations}
%\begin{align}
%\mathrm{V}(\rv, t) 
%&= \frac{1}{T} \, \int_0^T \dd t' \, \mathrm{V}(\rv, t, t') \, \sum_m \, \mathrm{e}^{\imath m \omega (t'-t)} \\
%&= \sum_m \, \mathrm{e}^{-\imath m \omega t} \, \Bigg( \frac{1}{T} \, \int_0^T \dd t' \, \mathrm{V}(\rv, t, t') \, \mathrm{e}^{\imath m \omega t'} \Bigg) \\
%&= \sum_m \, \mathrm{e}^{-\imath m \omega t} \, \mathrm{v}_m(\rv, t),
%\end{align}
%where
%\begin{equation}
%\mathrm{v}_m(\rv, t) = \frac{1}{T} \, \int_0^T \dd t' \, \mathrm{V}(\rv, t, t') \, \mathrm{e}^{\imath m \omega t'}.
%\end{equation}
%\end{subequations}
%
%\subsection*{Connection to the (t, t') approach}
%
%Let us now introduce a two-time wavefunction $\Psi(\rv, t, t')$ that depends on both $t$ and $t'$ and satisfies the TDSE with the potential $\mathrm{V}(\rv, t, t')$
%\begin{subequations}
%\begin{align}
%\imath \frac{\dd}{\dd t}\Psi &= \nonumber\\ 
%&= \imath \frac{\partial}{\partial t}\Psi(\rv, t, t') + \imath \frac{\partial t'}{\partial t}\frac{\partial}{\partial t'}\Psi(\rv, t, t') \\
%&= \imath \frac{\partial}{\partial t}\Psi(\rv, t, t') + \imath \frac{\partial}{\partial t'}\Psi(\rv, t, t') \\
%&= \Big(-\frac{1}{2} \nabla_\rv + \mathrm{V}(\rv, t, t') \Big) \Psi(\rv, t, t'),
%\end{align}\label{eq:app1_tdse_tt}
%\end{subequations}
%where we choose that $\frac{\partial t'}{\partial t}=1$.
%Let us further assume that we have solved Eq.~(\ref{eq:app1_tdse_tt}) and obtained $\Psi(\rv, t, t')$. We can show that 
%\begin{subequations}
%\begin{align}
%\Psi(\rv, t) \equiv \Psi(\rv, t, t')|_{t=t'} = \int_{-\infty}^{\infty}\dd t' \, \Psi(\rv, t, t') \, \delta(t'-t) \\
%=  \frac{1}{T} \, \int_0^T \dd t' \, \Psi(\rv, t, t') \, \sum_m \, \mathrm{e}^{\imath m \omega (t'-t)} \\
%= \sum_m \, \psi_m(\rv, t) \, \mathrm{e}^{-\imath m \omega t}
%\end{align} \label{eq:app_Psi_t_to_tt}
%\end{subequations}
%satisfies the single-time TDSE. Note, that Eq.~(\ref{eq:app_Psi_t_to_tt}) implies that $\Psi(\rv, t, t'+T) = \Psi(\rv, t, t')$, which, as we will show later, is justified.
%Thus, let us first evaluate the left-hand side of the TDSE
%\begin{subequations}
%\begin{align}
%\imath \frac{\dd}{\dd t}\Psi(\rv, t) =&\nonumber\\
%=&  \frac{\imath}{T} \, \int_0^T \dd t' \, \frac{\dd}{\dd t} \Big( \Psi(\rv, t, t') \, \sum_m \, \mathrm{e}^{\imath m \omega (t'-t)}\Big) \\
%=& \frac{\imath}{T} \, \int_0^T \dd t' \, \Big( \frac{\partial}{\partial t} + \frac{\partial}{\partial t'} \Big)  \Big( \Psi(\rv, t, t') \, \sum_m \, \mathrm{e}^{\imath m \omega (t'-t)}\Big) \\
%=& \frac{\imath}{T} \, \int_0^T \dd t' \, \sum_m \, \mathrm{e}^{\imath m \omega (t'-t)} \Big( \frac{\partial}{\partial t} \Psi(\rv, t, t') - \imath m \omega \Big) \,  \nonumber\\
%&+ \frac{\imath}{T} \, \int_0^T \dd t' \, \sum_m \, \mathrm{e}^{\imath m \omega (t'-t)} \Big( \frac{\partial}{\partial t'} \Psi(\rv, t, t') + \imath m \omega \Big) \\
%=& \frac{1}{T} \, \int_0^T \dd t' \, \sum_m \, \mathrm{e}^{\imath m \omega (t'-t)} \Big( \imath \frac{\partial}{\partial t} \Psi(\rv, t, t') + \imath \frac{\partial}{\partial t'} \Psi(\rv, t, t')  \Big).
%\end{align} \label{eq:app_dtPsi}
%\end{subequations}
%The right-hand side of the TDSE is
%\begin{subequations}
%\begin{align}
%\Big( -\frac{1}{2}\nabla_\rv &+ \mathrm{V}(\rv, t) \Big)\Psi(\rv, t) = \nonumber\\
%=& -\frac{1}{2} \frac{1}{T} \, \int_0^T \dd t' \, \nabla_\rv \Psi(\rv, t, t') \, \sum_m \, \mathrm{e}^{\imath m \omega (t'-t)} + \nonumber\\
%&+ \frac{1}{T^2} \, \int_0^T \dd t'' \, \mathrm{V}(\rv, t, t'') \, \sum_{m'} \, \mathrm{e}^{\imath m' \omega (t''-t)} \, 
%\int_0^T \dd t' \, \Psi(\rv, t, t') \, \sum_m \, \mathrm{e}^{\imath m \omega (t'-t)} \\
%=& -\frac{1}{2} \frac{1}{T} \, \int_0^T \dd t' \, \nabla_\rv \Psi(\rv, t, t') \, \sum_m \, \mathrm{e}^{\imath m \omega (t'-t)} + \nonumber\\
%&+ \frac{1}{T^2} \, \int_0^T \dd t'' \int_0^T \dd t' \, \mathrm{V}(\rv, t, t'') \, \Psi(\rv, t, t') 
%\sum_{m,m'} \mathrm{e}^{\imath m' \omega (t''-t)} \mathrm{e}^{\imath m \omega (t'-t)} \label{eq:app_VPsi_b} \\ 
%=& -\frac{1}{2} \frac{1}{T} \, \int_0^T \dd t' \, \nabla_\rv \Psi(\rv, t, t') \, \sum_m \, \mathrm{e}^{\imath m \omega (t'-t)} + \nonumber\\
%&+ \frac{1}{T} \, \int_0^T \dd t' \, \mathrm{V}(\rv, t, t') \, \Psi(\rv, t, t') 
%\sum_{m} \mathrm{e}^{\imath m \omega (t'-t)}  \label{eq:app_VPsi_c} \\
%=& \frac{1}{T} \, \int_0^T \dd t' \, \sum_{m} \mathrm{e}^{\imath m \omega (t'-t)}  
%\Big( -\frac{1}{2} \nabla_\rv + \mathrm{V}(\rv, t, t') \Big) \Psi(\rv, t, t'),
%\end{align} \label{eq:app_VPsi}
%\end{subequations}
%where to go from Eq.~(\ref{eq:app_VPsi_b}) to Eq.~(\ref{eq:app_VPsi_c}) we have used 
%\begin{multline}
%\sum_{m,m'} \mathrm{e}^{\imath m' \omega (t''-t)} \mathrm{e}^{\imath m \omega (t'-t)} =
%\sum_{m,m'} \mathrm{e}^{-\imath (m+m') \omega t} \mathrm{e}^{\imath m' \omega t''} \mathrm{e}^{\imath m \omega t'} = \\
%= \sum_{m',m''} \mathrm{e}^{-\imath m'' \omega t} \mathrm{e}^{\imath m' \omega t''} \mathrm{e}^{\imath (m''-m') \omega t'} = \sum_{m''} \mathrm{e}^{\imath m'' \omega (t'-t)} \sum_{m'} \mathrm{e}^{\imath m' \omega (t''-t')} = \\
%=  T \, \delta(t''-t') \, \sum_{m''} \mathrm{e}^{\imath m'' \omega (t'-t)} 
%= T \, \delta(t''-t') \, \sum_{m} \mathrm{e}^{\imath m \omega (t'-t)}
%\end{multline}
%in which the definition $m'' = m+m'$ was used in the second line and indices relabeled in the last line.
%Equating Eqs.~(\ref{eq:app_dtPsi}) and (\ref{eq:app_VPsi}) it is evident that, if $\Psi(\rv, t, t')$ satisfies the two-time TDSE in Eq.~(\ref{eq:app1_tdse_tt}), than $\Psi(\rv, t, t')|_{t=t'}$, defined in Eq.~(\ref{eq:app_Psi_t_to_tt}), satisfies the single time TDSE. 
%On the other hand, solving the single time TDSE using the ansatz
%\begin{subequations}
%	\begin{align}
%	\Psi(\rv, t) &= %\frac{1}{T} \, \int_0^T \dd t' \, \Psi(\rv, t, t') \, \sum_m \, \mathrm{e}^{\imath m \omega (t'-t)} 
%	\sum_m \psi_m(\rv, t) \, \mathrm{e}^{-\imath m \omega t} \\
%	\psi_m(\rv, t) &= \frac{1}{T} \, \int_0^T \dd t' \, \Psi(\rv, t, t') \, \mathrm{e}^{\imath m \omega t'}
%	\end{align} 
%\end{subequations}
%is equivalent to solving the two-time TDSE and taking the solution at $t=t'$.
%
%\subsubsection*{Proof for $\Psi(\rv, t, t'+T)=\Psi(\rv, t, t')$}
%
%Let us write Eq.~(\ref{eq:app1_tdse_tt}) as
%\begin{subequations}
%	\begin{align}
%	%\imath \frac{\partial}{\partial t}\Psi(\rv, t, t') + \imath \frac{\partial}{\partial t'}\Psi(\rv, t, t') \\
%	%	&= \Big(-\frac{1}{2} \nabla_\rv + \mathrm{V}(\rv, t, t') \Big) \Psi(\rv, t, t'),
%	\imath \frac{\partial}{\partial t}\Psi(\rv, t, t') &= \mathfrak{H}(\rv, t, t') \Psi(\rv, t, t'), \\
%	\mathfrak{H}(\rv, t, t') &= -\frac{1}{2} \nabla_\rv + \mathrm{V}(\rv, t, t') -  \imath \frac{\partial}{\partial t'}.
%	\end{align}
%\end{subequations}
%Since the Floquet Hamiltonian $\mathfrak{H}(\rv, t, t')$ is periodic in $t'$, i.e., $\mathfrak{H}(\rv, t, t'+T)=\mathfrak{H}(\rv, t, t')$, from the Floquet theory follows that the complete set of eigen-functions of $\mathfrak{H}(\rv, t, t')$ at any fixed $t$ consists of Floquet modes $\Phi_m(\rv, t'; t) \, \mathrm{e}^{-\imath m \omega t'}$, which are also periodic in $t'$, i.e., $\Phi_m(\rv, t'+T; t) = \Phi_m(\rv, t'; t)$, see, e.g., \cite{Drese1999}.
%Therefore, the wavefunction $\Psi(\rv, t, t')$ at any time $t$ can be expanded in terms of the Floquet modes
%\begin{equation}
%\Psi(\rv, t, t') = \sum_{n,m}a_{n,m}(t) \, \Phi_m(\rv, t'; t) \, \mathrm{e}^{-\imath m \omega t'}.
%\end{equation}
%From the periodicity of Floquet modes $\Phi_m(\rv, t'; t)$ it follows, that $\Psi(\rv, t, t')$ must also be periodic in $t'$ for any fixed $t$, i.e., $\Psi(\rv, t, t'+T) = \Psi(\rv, t, t')$.

%
%
%
%
%
\pagebreak

\section{Matrix-vector multiplication with Toeplitz and BTTB (Block Toeplitz with Toeplitz Blocks) matrices}

\subsection{Toeplitz matrix}\label{sec:app_T}

A matrix is Toeplitz if each of its diagonals is formed of equal elements. To describe a Toeplitz matrix only the knowledge of its 1st column and 1st row are required. Any $N \times N $ Toeplitz matrix can be  cast into a $2N \times 2N$ circulant matrix, which has identical rows and where each row is shifted to the right by one element with respect to the previous row, with rightmost element of the row transferred to the leftmost position. For example, a $3 \times 3$ Toeplitz matrix $\mathbf{T}$ can be transformed into a circulant matrix $\mathbf{C}$ as
\begin{equation}
\mathbf{T} =
\begin{bmatrix}
\mathsf{T}_{0} & \mathsf{T}_{-1} & \mathsf{T}_{-2} \\
\mathsf{T}_{1} & \mathsf{T}_{0} & \mathsf{T}_{-1}  \\
\mathsf{T}_{2} & \mathsf{T}_{1} & \mathsf{T}_{0} 
\end{bmatrix}
\Rightarrow
\mathbf{C} =
\begin{bmatrix}
\mathsf{T}_{0} & \mathsf{T}_{-1} & \mathsf{T}_{-2}  & 0 &  \mathsf{T}_{2} & \mathsf{T}_{1} \\
\mathsf{T}_{1} & \mathsf{T}_{0} & \mathsf{T}_{-1}  &  \mathsf{T}_{-2} & 0 & \mathsf{T}_{2} \\
\mathsf{T}_{2} & \mathsf{T}_{1} & \mathsf{T}_{0}  &  \mathsf{T}_{-1} & \mathsf{T}_{-2} & 0 \\
0 & \mathsf{T}_{2} & \mathsf{T}_{1} & \mathsf{T}_{0}  &  \mathsf{T}_{-1} & \mathsf{T}_{-2} \\
\mathsf{T}_{-2} & 0 & \mathsf{T}_{2} & \mathsf{T}_{1} & \mathsf{T}_{0}  &  \mathsf{T}_{-1} \\
\mathsf{T}_{-1} & \mathsf{T}_{-2} &  0 & \mathsf{T}_{2} & \mathsf{T}_{1} & \mathsf{T}_{0}  
\end{bmatrix} \label{eq:toeplitz_example},
\end{equation}
where zero was appended in each row to concatenate a Toeplitz matrix row with its column, and an arbitrary number of zeros can be used when forming a circulant matrix.

Multiplication of a circulant matrix $\mathbf{C}$ and a vector $\mathbf{\tilde{x}}$ is equal to a convolution between the first column of the circulant matrix $\mathbf{C}_{n0} \equiv \mathbf{c}$ and the vector $\mathbf{\tilde{x}}$. Hence, from the convolution theorem it follows that
\begin{align}
\mathbf{\tilde{b}} & = \mathbf{C} \cdot \mathbf{\tilde{x}} = \mathbf{c} \star \mathbf{\tilde{x}} \nonumber,\\
\mathcal{FT}(\mathbf{c} \star \mathbf{\tilde{x}}) &= \mathcal{FT}(\mathbf{c}) \, \mathcal{FT}(\mathbf{\tilde{x}}) = \mathcal{FT}(\mathbf{\tilde{b}}),
\end{align}
where $\star$ denotes the convolution operation and $\mathcal{FT}$ is the discrete Fourier transformation. 

To efficiently multiply a vector $\mathbf{x}$ by a general Toeplitz matrix $\mathbf{T}$ we need to: (i) form a column of the circulant matrix $\mathbf{c}$ from the 1st column and the 1st row of the Toeplitz matrix $\mathbf{T}$; (ii) append the vector $\mathbf{x}$ with zeros forming an extended vector $\mathbf{\tilde{x}}$ that has the same length as the vector $\mathbf{c}$; (iii) perform the Discrete Fourier Transformation of the vectors $\mathbf{c}$ and $\mathbf{\tilde{x}}$, multiply them together element-by-element and perform the inverse discrete Fourier transformation of the product. The result will be stored in the first $N$ elements, where $N$ is the size of the original vector. For the example in Eq.~(\ref{eq:toeplitz_example}):
\begin{align}
&\mathbf{b} = \mathbf{T} \cdot \mathbf{x} \Rightarrow 
\mathbf{\tilde{b}} = \mathbf{c} \cdot \mathbf{\tilde{x}}, \nonumber \\
\intertext{where} \nonumber\\
%&\text{where} \nonumber \\
&\mathbf{T} \Rightarrow
\mathbf{c} = 
\begin{bmatrix}
\mathsf{T}_{0} \\
\mathsf{T}_{1} \\
\mathsf{T}_{2} \\
0 \\
\mathsf{T}_{-2} \\
\mathsf{T}_{-1}
\end{bmatrix},
\hspace{20pt}
\mathbf{x}=
\begin{bmatrix}
X_{0} \\
X_{1} \\
X_{2}
\end{bmatrix}
\Rightarrow
\mathbf{\tilde{x}}=
\begin{bmatrix}
X_{0} \\
X_{1} \\
X_{2} \\
0 \\
0 \\
0 \\
\end{bmatrix}, \nonumber \\ \nonumber \\
\intertext{and} \nonumber\\
&\mathbf{\tilde{b}} =  \mathcal{FT}^{-1}\Big(  \mathcal{FT}(\mathbf{c}) \, \mathcal{FT}(\mathbf{\tilde{x}}) \Big), \nonumber \\
&\mathbf{b}_i = \mathbf{\tilde{b}}_i \; \text{ for } i < 3.
\end{align}

\subsection{BTTB matrix}\label{sec:app_BTTB}

A Block Toeplitz with Toeplitz Block, or BTTB matrix, is a block matrix where the blocks are of Toeplitz form and the blocks on each diagonal are identical. For example, assume a BTTB matrix $\mathbf{B}$ with $3 \times 3$ blocks, where each block is a Toeplitz matrix $\mathbf{T}_m$ of the from in Eq.~(\ref{eq:toeplitz_example})
%. The corresponding BTTB matrix $\mathbf{B}$ is
\begin{equation}
\mathbf{B} = 
\begin{bmatrix}
\mathbf{T}_{0} & \mathbf{T}_{-1} & \mathbf{T}_{-2} \\
\mathbf{T}_{1} & \mathbf{T}_{0} & \mathbf{T}_{-1}  \\
\mathbf{T}_{2} & \mathbf{T}_{1} & \mathbf{T}_{0} 
\end{bmatrix}
\text{  with  }
\mathbf{T}_m = 
\begin{bmatrix}
\mathsf{T}_{m,0} & \mathsf{T}_{m,-1} & \mathsf{T}_{m,-2} \\
\mathsf{T}_{m,1} & \mathsf{T}_{m,0} & \mathsf{T}_{m,-1}  \\
\mathsf{T}_{m,2} & \mathsf{T}_{m,1} & \mathsf{T}_{m,0} 
\end{bmatrix} \label{eq:example_bttb}.
\end{equation}

To calculate the dot product of a BTTB matrix $\mathbf{B}$ and a vector $\mathbf{x}$ one has to (i) form a 2D circulant matrix $\mathbf{C}$, where each column is the circulant vector for a Toeplitz block $\mathbf{T}_m$; (ii) reshape the vector $\mathbf{x}$ into a matrix $\mathbf{\tilde{X}}$ that has the same shape as the circulant matrix $\mathbf{C}$ by filling the lower half and the right half of $\mathbf{\tilde{X}}$ with zeros; (iii) perform the 2D Discrete Fourier Transformation of the matrices $\mathbf{C}$ and $\mathbf{\tilde{X}}$, multiply them together element-by-element and perform inverse 2D Discrete Fourier Transformation of the product. The result will be stored in the upper left corner of size $N \times M$, where $N$ is the size of each Toeplitz block $\mathbf{T}_m$ and $M$ is the number of block. For the example in Eq.~(\ref{eq:example_bttb}) the procedure is
\begin{align}
&\mathbf{d} = \mathbf{B} \cdot \mathbf{x} \Rightarrow \mathbf{\tilde{D}} 
= \mathbf{C} \cdot \mathbf{\tilde{X}}, \nonumber
%&\text{where} \nonumber \\
\end{align}
where
\begin{align}
&\mathbf{B} \Rightarrow
\mathbf{C} = 
\begin{bmatrix}
\mathbf{c}_{0} & \mathbf{c}_{1} & \mathbf{c}_{2} & \mathbf{0} & \mathbf{c}_{-2} & \mathbf{c}_{-1}
\end{bmatrix}
= \nonumber \\
& =\begin{bmatrix}
\mathsf{T}_{0, 0} & \mathsf{T}_{1, 0} & \mathsf{T}_{2, 0} & 0 & \mathsf{T}_{-2, 0} & \mathsf{T}_{-1, 0} \\
\mathsf{T}_{0, 1} & \mathsf{T}_{1, 1} & \mathsf{T}_{2, 1} & 0 & \mathsf{T}_{-2, 1} & \mathsf{T}_{-1, 1} \\
\mathsf{T}_{0, 2} & \mathsf{T}_{1, 2} & \mathsf{T}_{2, 2} & 0 & \mathsf{T}_{-2, 2} & \mathsf{T}_{-1, 2} \\
0 & 0 & 0 & 0 & 0 & 0 \\
\mathsf{T}_{0, -2} & \mathsf{T}_{1, -2} & \mathsf{T}_{2, -2} & 0 & \mathsf{T}_{-2, -2} & \mathsf{T}_{-1, -2} \\
\mathsf{T}_{0, -1} & \mathsf{T}_{1, -1} & \mathsf{T}_{2, -1} & 0 & \mathsf{T}_{-2, -1} & \mathsf{T}_{-1, -1} 
\end{bmatrix}, \nonumber \\ \nonumber \\
&\mathbf{x}=
\begin{bmatrix}
X_{0} \\ X_{1} \\ X_{2} \\ X_{3} \\ X_{4} \\
X_{5} \\ X_{6} \\ X_{7} \\ X_{8} 
\end{bmatrix}
\Rightarrow
\mathbf{\tilde{X}}=
\begin{bmatrix}
X_{0} & X_{3} & X_{6} & 0 & 0 & 0 \\
X_{1} & X_{4} & X_{7} & 0 & 0 & 0 \\
X_{2} & X_{5} & X_{8} & 0 & 0 & 0 \\
0 & 0 & 0 & 0 & 0 & 0 \\
0 & 0 & 0 & 0 & 0 & 0 \\
0 & 0 & 0 & 0 & 0 & 0 
\end{bmatrix}, \nonumber \\ \nonumber
\end{align}
and
\begin{align}
%&\text{and} \nonumber \\
&\mathbf{\tilde{D}} = \mathcal{FT}^{-1}_{2D}\Big(  \mathcal{FT}_{2D}(\mathbf{C}) \, \mathcal{FT}_{2D}(\mathbf{\tilde{X}}) \Big), \nonumber \\
& \mathbf{d}_{i+3j} = \mathbf{\tilde{D}}_{i,j} \; \text{ for } i,j < 3. 
\end{align}


%\subsection{Time propagation algorithm based on the Taylor expansion propagator}
%
%The time-dependent Schr\"{o}dinger equation can be solved by first defining the time propagation operator $\mathbf{U(t, t_0)}$
%\begin{subequations}
%	\begin{equation}
%	\mathbf{C}(t) = \mathbf{U}(t, t_0) \, \mathbf{C}(t_0),
%	\end{equation}
%	\begin{equation}
%	\mathbf{U}(t, t_0) = \exp\Big[ -\imath \int_{t_0}^t \dd \tau \, (\mathbf{T} + \mathbf{V(\tau)})\Big],
%	\end{equation}
%\end{subequations}
%and continuously applying it for small time steps $\Delta t$ during which the time dependent potential $\mathbf{V}(t)$ is kept fixed
%\begin{align}
%\mathbf{C}(t+\Delta t) &= \mathbf{U}(t+\Delta t, t) \, \mathbf{C}(t) = \nonumber\\
%&= \exp\Big[ -\imath \Delta t \, \big(\mathbf{T} + \mathbf{V}(t)\big)\Big] \, \mathbf{C}(t).
%\end{align}
%
%To evaluate the exponent of the matrix $\mathbf{T} + \mathbf{V}(t)$ one Taylor-expands the exponent up to the desired order. The evolution of the wavefunction $\mathbf{C}$ over a single time step is then calculated by
%\begin{equation}
%\mathbf{C}(t+\Delta t) = \Big[ 1 -\imath \Delta t \big( \mathbf{T} + \mathbf{V}(t) \big)   -(\Delta t)^2 \big( \mathbf{T} + \mathbf{V}(t) \big)^2 + \cdots \Big] \, \mathbf{C}(t),
%\end{equation}
%which can be easily calculated by repeatedly evaluating
%\begin{equation}
%\mathbf{C}_{n+1} = -\imath \Delta t \big( \mathbf{T} + \mathbf{V}(t) \big) \cdot \mathbf{C}_n.
%\end{equation}
%Since $\mathbf{T}$ is a diagonal matrix, the $\mathbf{T} \cdot \mathbf(C)_n$ product can be easily evaluated requiring of order $N \times M$ operations, where $N \times M$ is the size of the matrix $\mathbf{T}$. $\mathbf{V}$ matrix has a BTTB structure and with the FFT algorithm can be evaluated using the order of $N M\log(N M)$ operations.

%%%%%%%%%%%%%%%%%%%%%%%%%%%%%%%%%%%%%%%%%%%%%%%

%\begin{acknowledgments}
%\end{acknowledgments}



%%%%%%%%%%%%%%%%%%%%%%%%%%%%%%%%%%%%%%%%%%%%%%%
\pagebreak
\section*{References}
%\bibliographystyle{plain}
\bibliographystyle{ieeetr}
%\bibliography{library}
\bibliography{bibliography}
%
\end{document}
