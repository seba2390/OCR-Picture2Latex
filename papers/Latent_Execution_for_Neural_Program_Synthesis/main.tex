\documentclass{article}
% Recommended, but optional, packages for figures and better typesetting:
\usepackage[final,nonatbib]{neurips_2021}
\usepackage{microtype}
\usepackage{graphicx}
\usepackage{bmpsize}
% \usepackage{subfigure}
\usepackage{booktabs} % for professional tables
\usepackage{amsmath}
\usepackage{amssymb}
\usepackage[colorinlistoftodos,textsize=scriptsize]{todonotes}
\usepackage{marginnote}
\usepackage{amsthm}
\usepackage{booktabs}
\usepackage{multirow}
\usepackage{float}
\usepackage{subcaption}
\usepackage{wrapfig}
% \usepackage{xcolor}
\graphicspath{{./Figures/}}
\usepackage{hyperref}
\usepackage[utf8]{inputenc}
\usepackage{bbm}
\usepackage{enumitem}
\usepackage{pifont}
\usepackage{xcolor}
\usepackage[frozencache=true,cachedir=.]{minted}
\newif\ifarxivsubmit
\arxivsubmittrue

\title{Latent Execution for Neural Program Synthesis}
\author{
    Xinyun Chen \\
    UC Berkeley \\ 
    \texttt{xinyun.chen@berkeley.edu} 
    \And 
    Dawn Song \\
    UC Berkeley \\ 
    \texttt{dawnsong@cs.berkeley.edu}
    \And
    Yuandong Tian \\
    Facebook AI Research \\
    \texttt{yuandong@fb.com}
}

\begin{document}
\newtheorem{lemma}{Lemma}
\newtheorem{proposition}{Proposition}
\newtheorem{theorem}{Theorem}
\newtheorem{definition}{Definition}
\newtheorem{corollary}{Corollary}
\newtheorem{remark}{Remark}

\newcommand{\argmin}{\mathop{^\rm argmin}}
\newcommand{\argmax}{\mathop{\rm argmax}}
\newcommand{\rank}{\mathop{\sf rank}}
\newcommand{\iprod}[2]{\langle #1, #2 \rangle}
\newcommand{\lmin}{\lambda_{\min}}
\newcommand{\norm}[1]{\left\|#1\right\|}
\newcommand{\mge}{\succeq}
\newcommand{\mi}{{-1}}
\newcommand{\R}{\mathbb{R}}\newcommand{\B}{\mathbb{B}}\newcommand{\Id}{\mathbf{I}}
\newcommand{\zerovec}{\mathbf{0}}
\newcommand{\onevec}{\mathbf{1}}
\newcommand{\defeq}{:=}
\newcommand{\secref}[1]{Section~\ref{#1}}
\newcommand{\lemref}[1]{Lemma~\ref{#1}}
\newcommand{\proref}[1]{Proposition~\ref{#1}}
\newcommand{\thmref}[1]{Theorem~\ref{#1}}
\newcommand{\assref}[1]{Assumption~\ref{#1}}
\newcommand{\eqnref}[1]{(\ref{#1})}
\newcommand{\algref}[1]{Algorithm~\ref{#1}}

\newcommand{\Prob}{\mathbb{P}}
\newcommand{\E}{\mathbb{E}}
\newcommand{\V}{\mathbb{V}}
\newcommand{\tr}[1]{\textrm{tr}\left(#1\right)}
%\newcommand{\dt}[1]{\textrm{det}\left(#1\right)}

\newcommand{\Evt}{\mathcal{E}}
\newcommand{\EvtG}{\Evt_G}
\newcommand{\EvtD}{\Evt_\Delta}
\newcommand{\EvtX}{\Evt_X}

\newcommand{\lihong}[1]{[[\textbf{LL:} #1]]}
\renewcommand{\lihong}[1]{}

\newcommand{\RN}[1]{%
  \textup{\uppercase\expandafter{\romannumeral#1}}%
}

\newtheorem{assumption}{Assumption}


\maketitle

\iffalse
\begin{abstract}
Program synthesis from input-output examples has been a long-standing challenge, and recent works have demonstrated some success in designing deep neural networks for program synthesis. However, existing efforts in input-output neural program synthesis have been focusing on domain-specific languages, thus the applicability of previous approaches to synthesize code in full-fledged popular programming languages, such as C, remains a question. The main challenges lie in two folds. On the one hand, the program search space grows exponentially when the syntax and semantics of the programming language become more complex, which poses higher requirements on the synthesis algorithm. On the other hand, increasing the complexity of the programming language also imposes more difficulties on data collection, since building a large-scale training set for input-output program synthesis require random program generators to sample programs and input-output examples. In this work, we take the first step to synthesize C programs from input-output examples. In particular, we propose {\ours}, which learns the latent representation to approximate the execution of partially generated programs, even if their semantics are not well-defined. We demonstrate the possibility of synthesizing elementary C code from input-output examples, and leveraging learned execution significantly improves the prediction performance over existing approaches. Meanwhile, compared to the randomly generated ground-truth programs, {\ours} synthesizes more concise programs that resemble human-written code. We show that training on these synthesized programs further improves the prediction performance for both Karel and C program synthesis, indicating the promise of leveraging the learned program synthesizer to improve the dataset quality for input-output program synthesis.
\end{abstract}
\fi

\iffalse 
in designing deep neural networks for program synthesis.
existing efforts in input-output neural program synthesis 
The main challenges lie in two folds. 

search space grows exponentially when become more complex, which poses higher requirements on the synthesis algorithm. On the other hand, increasing the complexity of the programming language also imposes more difficulties on data collection, since building a large-scale training set for input-output program synthesis require random program generators to sample programs and input-output examples. 

we take the first step to synthesize C programs from input-output examples. In particular, We demonstrate the possibility of synthesizing elementary C code from input-output examples, 
\fi

\begin{abstract}
Program synthesis from input-output (IO) examples has been a long-standing challenge. While recent works demonstrated limited success on domain-specific languages (DSL), it remains highly challenging to apply them to real-world programming languages, such as C. Due to complicated syntax and token variation, there are three major challenges: \textbf{(1)} unlike many DSLs, programs in languages like C need to compile first and are not executed via interpreters; \textbf{(2)} the program search space grows exponentially when the syntax and semantics of the programming language become more complex; and \textbf{(3)} collecting a large-scale dataset of real-world programs is non-trivial. As a first step to address these challenges, we propose {\ours} and show its efficacy in a \emph{restricted-C} domain (i.e., C code with tens of tokens, with sequential, branching, loop and simple arithmetic operations but no library call). More specifically, \ours{} learns the latent representation to approximate the execution of partially generated programs, even if they are incomplete in syntax (addressing \textbf{(1)}). The learned execution significantly improves the performance of next token prediction over existing approaches, facilitating search (addressing \textbf{(2)}). Finally, once trained with randomly generated ground-truth programs and their IO pairs, \ours{} can synthesize more concise programs that resemble human-written code. Furthermore, retraining our model with these synthesized programs yields better performance with fewer samples for both Karel and C program synthesis, indicating the promise of leveraging the learned program synthesizer to improve the dataset quality for input-output program synthesis (addressing \textbf{(3)}). When evaluating on whether the program execution outputs match the IO pairs, \ours{} achieves 55.2\% accuracy on generating simple C code with tens of tokens including loops and branches, outperforming existing approaches without executors by around 20\%.~\footnote{The code is available at \url{https://github.com/Jungyhuk/latent-execution}.}
\end{abstract}

Reinforcement learning has achieved great success in areas such as Game-playing \citep{silver2018general,vinyals2019grandmaster}, robotics \cite{kober2013reinforcement}, large language models \citep{ouyang2022training}, etc.
However, due to safety concerns or physical limitations, in some real-world reinforcement learning problems, we must consider additional constraints that may influence the optimal policy and the learning process \citep{garcia2015comprehensive}.
% For example, a robotic arm must not take actions that may cause harm to itself or the environments.
A standard framework to handle such cases is the constrained Markov Decision Process (CMDP) \citep{altman1999constrained}.
Within the CMDP framework, the agent has to maximize
the expected cumulative reward while
obeying a finite number of constraints, which are usually in the form of expected cumulative cost criteria.

However, we are sometimes concerned with the problem with a continuum of constraints.
For example,
the constraints we meet might be time-evolving or subject to uncertain parameters, which
cannot be formulated as an ordinary CMDP
(see Examples \ref{Example_Time_Evolving} and  \ref{Example_Uncertain}).
In this paper we would study a generalized CMDP  
to address the above problem.  Because the constraints are not only infinite-number but also lie
in a continuous set,
the generalization is not trivial. Fortunately, we find that we can borrow the idea behind semi-infinite programming (SIP) \citep{remez1934determination, hettich1993semi} to deal with the semi-infinite constraints.
Accordingly, we propose \emph{semi-infinitely constrained Markov decision processes} (SICMDPs)
as a novel complement to the ordinary CMDP framework.
%More specifically,  an SICMDP model %, we consider 
%contains a continuum of constraints whereas an ordinary CMDP contains a finite number of constraints. 

%This generalization is natural but not trivial. However, we can brows the idea  
%The idea is quite natural and can be backtracked
%to the practice of extending linear programming to linear semi-infinite programming (LSIP) %\cite{remez1934determination, GobernaLSIO1998}.
%In addition, 
%As a complementary approach to the ordinary CMDP framework, 
%SICMDP can be used to model these problems  which cannot be described by a finite number of constraints
%that are not covered by .
%For example,
%the restrictions we consider can be time-evolving or subject to uncertain parameters
%, thus
%cannot be described by a finite number of constraints but a continuum of constraints 
%(see Examples \ref{Example_Time_Evolving} and  \ref{Example_Uncertain}).

We also present two reinforcement learning algorithms to solve SICMDPs called SI-CRL and SI-CPO, respectively.
SI-CRL is a model-based reinforcement learning algorithm designed for tabular cases, and SI-CPO is a policy optimization algorithm for non-tabular cases.
% and analyze its performance both theoretically and empirically.
The main challenge is that we need to deal with a continuum of constraints, thus reinforcement learning algorithms for ordinary CMDPs do not work anymore.
In SI-CRL, we tackle this difficulty by first transforming the reinforcement learning problem to an equivalent LSIP problem, which can then be solved using methods in the LSIP literature like the dual exchange methods \citep{Hu1990,reemtsen1998numerical}.
In SI-CPO, we resort to the idea of cooperative stochastic approximation developed in \cite{lan2020algorithms, wei2020comirror}.
As far as we know, we are the first to introduce tools from semi-infinitely programming (SIP) into the reinforcement learning community for solving constrained reinforcement learning problems.

% To the best of our knowledge, we are the first to apply tools from semi-infinitely programming (SIP) to solve reinforcement learning problems.
Furthermore, we give theoretical analysis for both SI-CRL and SI-CPO.
We decompose the error of SI-CRL into two parts: the statistical error from approximating the true SICMDP with an offline dataset and the optimization error due to the fact that the solution of the LSIP problem obtained by the dual exchange method is inexact.
On the optimization side, we show that the iteration complexity of SI-CRL is $O\left(\left\{\mathrm{diam}(Y)L\sqrt{|\gS|^2|\gA|m}/\left[(1-\gamma)\epsilon\right]\right\}^m\right)$.
On the statistical side, we show that the sample complexity of SI-CRL is $\widetilde O\left(\frac{|S|^2|A|^2}{\epsilon^2(1-\gamma)^3}\right)$ if the offline dataset is generated by a generative model, and $\widetilde O\left(\frac{|S||A|}{\nu_{\min} \epsilon^2(1-\gamma)^3}\right)$ if the dataset is generated by a probability measure $\nu$ as considered in \cite{chen2019information}.
Here $\widetilde O$ means that all logarithm terms are discarded.
For SI-CPO, things become a little more complicated because other than the statistical error and the optimization error, we also need to consider the function approximation error, which comes from imperfect policy parametrizations.
It is shown if the function approximation error can be controlled to $O(\epsilon)$ order, the iteration complexity of SI-CPO is $\widetilde{O}\left(\frac{1}{\epsilon^2(1-\gamma)^6}\right)$ and the sample complexity of SI-CPO is $\widetilde{O}(\frac{1}{\epsilon^4(1-\gamma)^{10}})$.
Here our iteration complexity bound is equivalent to a typical $\widetilde O(1/\sqrt{T})$ global convergence rate.

We perform a set of numerical experiments to illustrate the SICMDP model and validate our proposed algorithms.
Specifically, we examine two numerical examples, namely the discharge of sewage and ship route planning.
Through the discharge of sewage example, we show the advantage of the SICMDP framework over the CMDP baseline obtained by naive discretization in modeling realistic sequential decision-making problems.
Moreover, we demonstrate the effectiveness of the SI-CRL and SI-CPO algorithms in such tabular environments. 
In the ship route planning example, we illustrate the benefits of the SICMDP framework and the ability of the SI-CPO algorithm to address complex continuous control tasks involving continuous state spaces with modern deep reinforcement learning techniques.

% In summary, our contributions are listed as follows.
% First, we present the SICMDP model, which can be viewed as a generalization of the ordinary CMDP model.
% Second, we propose an algorithm to perform reinforcement learning for SICMDPs, which is called SI-CRL, and we believe that we are the first to apply tools from SIP
% to solve reinforcement learning problems.
% Third, we give a theoretical analysis of SI-CRL and identify both its sample complexity and iteration complexity.
% In addition, we perform numerical experiments to illustrate the SICMDP model and validate the SI-CRL algorithm.
% \{This paragraph can be removed!!! \}





\vspace{-0.1in}
\section{Neural Program Synthesis from Input-Output Examples}
\vspace{-0.1in}
In programming by example tasks, the program specification is a set of input-output examples~\cite{devlin2017robustfill,bunel2018leveraging}. Specifically, we provide the synthesizer with a set of $K$ input-output pairs $\{(I^{(k)}, O^{(k)})\}_{k=1}^K$ ($\{IO\}^K$ in short). These input-output pairs are annotated with a ground truth program $P^\star$, so that $P^\star(I^{(k)})=O^{(k)}$ for any $k \in \{1, 2, ..., K\}$. To measure the program correctness, we include another set of held-out test cases $\{IO\}_{test}^{K_{test}}$ that differs from $\{IO\}^K$. The goal of the program synthesizer is to predict a program $P$ from $\{IO\}^K$, so that $P(I)=P^\star(I)=O$ for any $(I, O) \in \{IO\}^K + \{IO\}_{test}^{K_{test}}$.

%\label{sec:c-data}
\textbf{C Program Synthesis}. In this work, we make the first attempt of synthesizing C code in a restricted domain from input-output examples only, and we focus on programs for list processing. List processing tasks have been studied in some prior works on input-output program synthesis, but they synthesize programs in restricted domain-specific languages instead of full-fledged popular programming languages~\cite{balog2016deepcoder,odena2020learning,odena2020bustle}. 

Our C code synthesis problem brings new challenges for programming by example. Compared to domain-specific languages, the syntax and semantics of C are much more complicated, which significantly enlarges the program search space. Meanwhile, learning good representations for partially decoded programs also becomes more difficult. In particular, prior neural program synthesizers that utilize per-line interpreters for the programming language to guide the synthesis and representation learning~\cite{chen2018execution,shin2018improving,nye2020representing,Ellis2019WriteEAExtendExecution,odena2020bustle} are not directly applicable to C. Although it is possible to dump some intermediate variable states during C code execution~\cite{campbell2012executable}, since partial C programs are not executable, we are able to obtain all the execution states only until a full C code is generated, which is too late to include them in the program decoding process. In particular, the intermediate execution state is not available when the partial program is syntactically invalid, and this happens more frequently for C due to its syntax design.
\begin{figure}
    \centering
    \includegraphics[width=\textwidth]{fig/c-program-synthesis-crop.pdf}
\caption{\small Illustration of the C program synthesis pipeline. For dataset construction, we develop a random program generator to sample random C programs, then execute the program over randomly generated inputs and obtain the outputs. The input-output pairs are fed into the neural program synthesizer to predict the programs. Note that the synthesized program can be more concise than the original random program.}
\label{fig:ex-c}
\end{figure}


\section{Approach}
\begin{figure}[t]
\centering
\resizebox{0.48\textwidth}{!}{ 
  \includegraphics[width=\textwidth]{figures/workflow.PNG}
}
  \caption{Workflow of \system}
  \label{fig:workflow}
\end{figure}

Figure ~\ref{fig:workflow} shows the overall workflow of \system. The triggers for using \system are usually alert(s) from automated anomaly detection, or sometimes an SRE engineer's suspicion. There are three major steps: constructing the service  dependency graph, constructing the event causality graph,  and root cause ranking. The outputs are the root causes ranked by the likelihood. To support fast human investigation experience, we build an interactive UI as shown in  Figure~\ref{fig:UI}: the service dependency, events with causal links and additional details such as raw metrics or the developer contact (of a code deployment event) are presented to the user for next steps. As an  offline part of human investigation, we label/collect a data set, perform validation, and summarize the knowledge for further improvement on all incidents on a daily basis. %as validations and heterogeneous graph learning (HGL)~\cite{qiao2020heterogeneous} to synthesize the knowledge from existing cases in order to further improve the system.

\subsection{Constructing Service Dependency Graph}
\label{sec:appgraph}

The construction of the service dependency graph starts with the initial alerted or suspicious service(s), denoted as $I$. For example, in Figure ~\ref{fig:ex1_dep}, $I=\{\textit{Checkout}\}$. $I$ can contain multiple services based on the range of the trigger alerts or suspicions. We maintain domain service lists where domain-level alerts can be triggered because there is no clear service-level indication.

At the back end, \system maintains a global service dependency graph $G_{global}$ via distributed tracing and log analysis. The directed edge from nodes $A$ to $B$ (two services or system components) in the dependency graph indicates a service invocation or other forms of dependency. In Figure~\ref{fig:ex1_dep}, the black arrows indicate such edges. Bi-directional edges and cycles between the services can be possible and exist. In this work, the global dependency graph is updated daily.%by extracting from one day's total site traffic.

The service dependency (sub)graph $G$ is constructed using $G_{global}$ and $I$. An extended service list $L$ is first constructed by traversing each service in $I$ over $G_{global}$ for a radius range $r$. Each service $u \in L$ can be traversed by at least one service $v \in I$ within $r$ steps: $L=\{u|\exists v\in I, \ dist(u,v)\le r\ or\ dist(v,u)\le r\}$. Then, the service dependency subgraph $G$ is constructed by the nodes in $L$ and the edges between them in $G_{global}$. In our current implementation, $r$ is set to $2$, since this dependency graph may be dynamically extended in the next steps based on events' detail for longer issue chains or additional dependencies.

\subsection{Constructing Event Causality Graph}
\label{sec:causality}

In the second step, \system collects all supported events for each service in $G$ and constructs the causal links between events. 

\subsubsection{Collecting Events}

Table~\ref{tab:events} presents some example event types and detection techniques for \system's production implementation. For detection techniques, ``De Facto'' indicates that the event can be directly collected via a specific API or storage. %The detection can be done passively at the back end continuously then store anomaly events in different databases; or done actively by pulling data and run detection on the fly to save compute resources. 
The detection either runs passively in the back end to reduce delay and improve accuracy, or runs actively for only the services within the dependency graph range to save resources. %For example, low-level error signals or logs are detected actively since they are too many to scale. 

There are three major categories of events: performance metrics, status logs, and developer activities:
\begin{itemize}
    \item \emph{Performance metrics} represent an anomaly of monitored time series metrics. For example, high CPU usage indicates that the service is causing high CPU usage on a certain machine. In this category, most events are continuously and passively detected and stored. %For high CPU usage, threshold indicates the event is created when CPU usage is higher than certain predefined value. TPS spike indicates a spike in transaction per second, since TPS is a moving average value, we use some statistical model learned from historical data to detect such events.
    \item \emph{Status logs} are caused by abnormal system status, such as spike of HTTP error code metrics while accessing other services' endpoints. Different types of error metrics are important and supported in \system, including third-party APIs. For example, Bad Host indicates abnormal patterns on some machines running the service, and can be detected by a  clustering-based ML approach.%Markdown indicates that the whole service is down. 
    \item \emph{Developer activities} are the events generated when a certain activity of developers is triggered, such as code deployment and config change.
\end{itemize}

\begin{table}[t]
\centering
\caption{List of example event types used in \system}
\resizebox{0.4\textwidth}{!}{ 
\begin{tabular}{|c|c|c|}
\hline
Type                                & Event Type                  & Detection Technique  \\ \hline
\multirow{6}{*}{Performance Metrics} & High GC (Overhead)      & Rule-based        \\ \cline{2-3} 
                                    & High CPU Usage          & Rule-based        \\ \cline{2-3} 
%                                    & Out of Memory           & Rule-based        \\ \cline{2-3} 
%                                    & LB Connection Stacking  & Statistical Model \\ \cline{2-3} 
                                    & Latency Spike           & Statistical Model \\ \cline{2-3} 
                                    & TPS Spike               & Statistical Model \\ \cline{2-3} 
                                    & Database Anomaly        & ML Model          \\ \cline{2-3} 
                                    & Business Metric Anomaly & ML Model          \\ \hline
\multirow{4}{*}{Status Logs}        & WebAPI Error            & Statistical Model \\ \cline{2-3} 
                                    & Internal Error          & Statistical Model \\ \cline{2-3} 
                                    & ServiceClient Error     & Statistical Model \\ \cline{2-3} 
                                    & Bad Host                & ML Model          \\ \hline %\cline{2-3} 
%                                    & Hystrix Circuit Break   & De Facto          \\ \hline
\multirow{3}{*}{Developer Activities} & Code Deployment         & De Facto          \\ \cline{2-3} 
                                    & Configuration Change    & De Facto          \\ \cline{2-3} 
                                    & Execute URL             & De Facto          \\ \hline
\end{tabular}
}
\label{tab:events}
\end{table}

In Groot, there are more than a dozen event types such as \emph{Latency Spike} as listed in the column 2 of Table~\ref{tab:events}. 
Each event type is characterized by three aspects: $Name$ indicates the name of this event type; $Lookback Period$ %\footnote{In Figure~\ref{fig:ex2_n1}, there are two periods, 1 day indicates the look-back range if the model has already finished deployment, 4 days indicates the range if the model deployment is still ongoing(incremental deployment).} 
indicates the time range to look back (from the time when the use of \system is triggered) for collecting events of this event type;  $PropertyType$ indicates the types of the properties that an event of this event type should hold. 
$PropertType$  is characterized by a vector of pairs, each of which indicates the string type for a property's name and the primitive type for the property's value such as string, integer, and float. 
Formally, an event type is defined as a tuple: 
$ET = <Name, Lookback Period, PropertyType>$ 
where 
$PropertyType = <(string, \textit{type}_1), ..., (string, \textit{type}_{n})>$ ($n$ is the number of properties that an event of this event type holds). 
%

Each event of a certain event type $ET$ is characterized by four aspects:
$\textit{Service}$ indicates the service name that the event belongs to; $\textit{Type}$ indicates $ET$'s $\textit{Name}$;  $\textit{StartTime}$ indicates the time when the event happens; $\textit{Properties}$ indicates the properties that the event  holds.
Formally, an event is defined as a tuple: 
$e = <Service, Type, StartTime, Properties>$ 
where $Properties$ is an instantiation of $ET$'s  $PropertyType$. 


%and each event is defined as $e = \{<\textit{Property}_i, \textit{value}_i>\}$. Each event type serves as a template for the event instantiation. such as a string, an integer, a float or a set of primitive types while $\textit{value}$ is limited to primitive data types. 
%
%Each event is defined as a sequence of property-value pairs where the set size is $n$.

For example, in Figure~\ref{fig:example1}, the generated event for \emph{Latency Spike in DataCenter-A} in \emph{Service-C} would be $<``\textit{Service-C}'', ``\textit{Latency\ Spike}'', \textit{2021/08/01-12:36:04}, <(``\textit{DataCenter}'',``\textit{DC-1}''),  ...>>$. %So for each service in $G$, we detect/collect and filter the events within specified time range of the alert.

\subsubsection{Constructing Causal Link}

After collecting all events on all services in $G$, in this step, causal links between these events are constructed for RCA ranking. The causal links (red arrows) in Figure~\ref{fig:ex1_cas} are such examples. A causal link represents that the source event can possibly be caused by the target event. SRE knowledge is engineered into rules and used to create causal links between the pairs of events. %As shown in Figure~\ref{fig:example2}, there are two categories of rules: basic rules and conditional rules. 

A rule for constructing a causal link is defined as a tuple:  $Rule = <Target\mbox{-}Type,  Source\mbox{-}Events, Target\mbox{-}Events, Direction,\\ Target\mbox{-}Service,  Condition>$  ($Condition$ can be optionally specified). $Target\mbox{-}Type$ indicates the type of the rule, being either $Static$ or $Dynamic$ (explained further later). $Source\mbox{-}Events$ indicates the type of the causal link's source event ($Source\mbox{-}Events$ are listed in the names of the rules shown in Figures~\ref{fig:ex2_n1},~\ref{fig:ex2_n2} and~\ref{fig:dynamic_example}).   $Target\mbox{-}Events$ indicates the type of the causal link's target event. $Direction$ indicates the direction of the casual link between the target event and source event. $Target\mbox{-}Service$ indicates the service that the target event should belong to. Note that $Target\mbox{-}Service$ in $Static$ rules can be  $Self$, which indicates that the target event would be within the same service as the source event, or $Outgoing$/$Incoming$, which indicates that the target event would belong to the downstream/upstream services of the service that the source event belongs to in $G$.

\begin{figure}[t]
\centering
\includegraphics[width=0.56\columnwidth]{figures/example3.png}
\caption{Example of dynamic rule}
\label{fig:dynamic_example}
\end{figure}

There are two categories of special rules. The first category is \emph{dynamic} rules (i.e., rules whose $Target\mbox{-}Type$  is set to $Dynamic$) to support dynamic dependencies. Here $Target\mbox{-}Service$ does not indicate any of the three possible options listed earlier but indicates the name of the target service that \system would need to create. For example, live DB dependencies are not available due to different tech stacks and high volume. In Figure~\ref{fig:dynamic_example}, a DB issue (DB Markdown) is shown in \emph{Service-A}. Based on the listed \emph{dynamic} rule, \system creates a new ``service'' \emph{DB-1} in $G$, a new event ``Issues'' that belongs to \emph{DB-1}, and a causal link between the two events.  In practice, the SRE teams use dynamic rules to cover a lot of third-party services and database issues since the live dependencies are not easy to maintain.  %However through the internal error messages and dynamic rules, \system is still able to handle these dependencies. %we can still support external inferences. 

The second category of special rules is \emph{conditional} rules. \emph{Conditional} rules are used when some prerequisite conditions should be satisfied before a certain causal link is created. In these rules, $Condition$ is specified with a boolean predicate. As shown in Figure~\ref{fig:ex2_n2}, the SRE teams believe \emph{Latency Spike} events from different services are related only when both events happen within the same data center. Based on this observation, \system would first evaluate the predicate in $Condition$ and build only the causal link when the predicate is true. A conditional rule overwrites the basic rule on the same source-target event pair.

When constructing causal links, \system first applies the \emph{dynamic} rules so that dynamic dependencies and events are first created at once. Then for every event in the initial services (denoted as $I$), if the rule conditions are satisfied, one or many causal links are created from this event to other events from the same or upstream/downstream services. When a causal link is created, the step is repeated recursively for the target event (as a new origin) to create new causal links. After no new causal links are created, the construction of the event causality graph is finished.

% When \system constructs the causal links, \system first processes all dynamic rules as they may create new event nodes in the graph. %\system enumerates the dynamic rules on each existing event node and also on the newly added nodes (There could also be rules applicable to the newly added nodes) until no new event nodes can be created. 


%Each rule is defined as a predicate containing both events' property-value pair. If the predicate evaluates to be true between two events, then we would add the edge in the causality graph. For example, in Figure~\ref{fig:example1}, the rule used to establish the edge between \emph{GC overhead in RNO} and \emph{Latency increase in LVS, RNO, SLC} would be like this: Suppose we are now determining whether there should be a link from event $u$ to event $v$, then this rule would be $u.\text{pool} = v.\text{pool}\ and\ u.\text{type} = ``\text{High GC Overhead}"\ and\ v.\text{type} = ``\text{Latency increase}"\ and\ u.\text{center} \cap v.\text{center} \ne \emptyset$ which holds true for these two events. Each causality link is also associated with a weight which represents the likelihood of causality - we set all initial values as $1.0$. Overtime these value are updated by the statistical analysis result of the collected data set.


\subsection{Root Cause Ranking}
Finally, \system ranks and recommends the most probable root causes from the event causality graph. Similar to how search engines infer the importance of pages by page links, we customize the PageRank \cite{manning2010introduction} algorithm to calculate the root cause ranking; the customized algorithm is named as GrootRank. The input is the event causality graph from the previous step. Each edge is associated with a weighted score for weighted propagation. The default value is set as $1$, and is set lower for alerts with high false-positive rates. 

Based on the observation that dangling nodes are more likely to be the root cause, we customize the personalization vector as $P_n = f_n $ or $P_d = 1$, where $P_d$ is the personalization score for dangling nodes, and $P_n$ is for the remaining nodes; and $f_n$ is a value smaller than 1 to enhance the propagation between dangling nodes. In our work, the parameter setting is $f_n = 0.5$, $\alpha = 0.85$, $max_{iter} = 100$ (which are parameters for the PageRank algorithm). Figure \ref{fig:person} illustrates an example. The grey circles are the events collected from three services and one database. The grey arrows are the dependency links and the red ones are the causal links with the weight of $1$. Both of the PageRank and GrootRank algorithms detect $event 5$ (DB issue) as the root cause, which is expected and correct. However, the PageRank algorithm ranks $event 4$ higher than $event 3$. But $event 3$ of $\textit{Service-C}$ is more likely to be the second most possible root cause (besides $event 5$), because the scores on dangling nodes are propagated to all others equally in each iteration. We can see that $event 3$ is correctly ranked as second using the GrootRank algorithm.

The second step of GrootRank is to break the tied results from the previous step. The tied results are due to the fact that the event graph can contain multiple disconnected sub-graphs with the same shape. We design two techniques to untie the ranking: 
\begin{figure}[t]
\centering
  \includegraphics[width=0.8\columnwidth]{figures/personalvector.png}
  \caption{Example of personalization vector customization}
  \label{fig:person}
\end{figure}

\begin{figure}[t]
\centering
  \includegraphics[width=0.8\columnwidth]{figures/accessdistance.png}
  \caption{Example of using access distance to untie the ranking results}
  \label{fig:untie}
\end{figure}
\begin{enumerate}
\item For each joint event, the access distance (sum) is calculated from the initial anomaly service(s) to the service where the event belongs to. If any ``access'' is not reachable, the distance is set as $d_m+1$ where $d_m$ is the maximum possible distance. The one with shorter access distance (sum) would be ranked higher and vice versa. Figure \ref{fig:untie} presents an example, where \emph{Service-A} and \emph{Service-B} are both initial anomaly services. Since \system suspects that $event 2$ is caused by either $event 3$ or $event 1$ with the same weight. The scores of $event 3$ and $event 1$ are tied. Then, $event 3$ has a score of $1$ (i.e., $0+1$) and $event 1$ has a score of 2 (i.e., $0+2$), since it is not reachable by \emph{Service-B}). Therefore, $event 3$ is ranked first and logical. 
\item For the remaining joint results with the same access distances, \system continues to untie by using the historical root cause frequency of the event types under the same trigger conditions (e.g., checkout domain alerts). This frequency information is generated from the manually labeled dataset. A more frequently occurred root cause type is ranked higher.% than the less frequent ones.
\end{enumerate}


\subsection{Rule Customization Management}

While \system users create or update the rules,  there could be overlaps, inconsistencies, or even conflicts being introduced such as the example in Figure~\ref{fig:ex2_n2}. \system uses two graphs to manage the rule relationships and avoid conflicts for users. One graph is to represent the link rules between events in the same service (\emph{Same-Graph}) while the other is to represent links between different services (\emph{Diff-Graph}). The nodes in these two graphs are the event types defined in Section~\ref{sec:causality}. There are three statuses between each (directional) pair of event types: (1) no rule, (2) only basic rule, and (3) conditional rule (since it overwrites the basic rule). In \emph{Same-Graph}, \system does not allow self-loop as it does not build links between an event and itself.
% but it is possible that we build links between different services with the same event type.

When rule change happens, existing rules are enumerated to build edges in \emph{Same-Graph} and \emph{Diff-Graph} based on $Target\mbox{-}Events$ and $Target\mbox{-}Service$. Based on the users' operation of 
% \begin{itemize}
%     \item 
    (1) ``remove a rule'',  \system removes the corresponding edge on the graphs;
    % \item 
    (2) ``add/update a rule'',  \system checks whether there are existing edges between the given event types, and then warns the users for possible overwrites. 
    % The users can also combine the conditional rules.   % while users are adding basic rules between event types if there are existing conditional rules between them.
    If there are no conflicts, \system just adds/updates edges between the event types.
    % \item Add conditional rules. We would first alert the possible overwrite. Then if users are about to add new conditional rules on the top of existing conditional rules, we would ask the users to combine these two conditions to add a new one. We then build or change all corresponding edges to status 3.
% \end{itemize} 

After all changes, \system extracts the rules from the graphs by converting each edge to a single rule. These rules are automatically implemented, and then tested against our labeled data set. The \system users need to review the changes with validation reports before the changes go online.

% Note that currently we don't check the consistencies between dynamic rules as we cannot process the dynamic event types, but this could be solved in the future by using nodes with symbolic values to represent such event types. 
\subsection{Unsupervised Grammar Induction}

\subsubsection{Setup}\label{sec:LM_setup}
\paragraph{Baselines and Evaluation.} 
For comparison, we include six recent strong models for unsupervised parsing with available open source implementations: StructFormer \cite{DBLP:conf/acl/ShenTZBMC20}, Ordered Neurons~\cite{DBLP:conf/iclr/ShenTSC19}, URNNG~\cite{dblp:conf/naacl/kimrykdm19}, DIORA~\cite{dblp:conf/naacl/drozdovvyim19}, C-PCFG~\cite{kim-etal-2019-compound}, and R2D2~\cite{hu-etal-2021-r2d2}. 
To observe the marginal gain from pretraining, we also include Fast-R2D2 without pretraining denoted as Fast-R2D2$_{\rm w/o}$.
Following~\newcite{htut-etal-2018-grammar}, we train all systems on a training set consisting only of raw text, and evaluate and report the results on an annotated test set. 
As an evaluation metric, we adopt sentence-level unlabeled $F_1$ computed using the script from \newcite{kim-etal-2019-compound}.
We compare against the non-binarized gold trees per convention.
The results of Fast-R2D2 are obtained from 3 runs of each model with different random seeds in pre-training.
The best checkpoint for each system is picked based on scores on the validation set. 
Fast-R2D2 is pretrained with span constraints for the word level but without span constraints for the word-piece level.
To support word-piece level evaluation, 
we convert gold trees to word-piece level trees 
by simply breaking each terminal node into a non-terminal node with its word-pieces as terminals, e.g., (NN discrepancy) into (NN (WP disc) (WP \#\#re) (WP \#\#pan) (WP \#\#cy)).

\paragraph{Environment.} EFLOPS~\cite{DBLP:conf/hpca/DongCZYWFZLSPGJ20} is a highly scalable distributed training system designed by Alibaba. With its optimized hardware architecture and co-designed supporting software tools, including ACCL~\cite{DBLP:journals/micro/DongWFCPTLLRGGL21} and KSpeed (the high-speed data-loading service), it could easily be extended to 10K nodes (GPUs) with linear scalability.

\paragraph{Hyperparameters.} The tree encoder of our model uses 4-layer Transformers with 768-dimensional embeddings, 
3,072-dimensional hidden layer representations, and 12 attention heads. 
The top-down parser of our model uses a 4-layer bidirectional LSTM with 128-dimensional embeddings and 256-dimensional hidden layer. The sampling number $K$ is set to be 256.
Training is conducted using Adam optimization with weight decay using a learning rate of $5 \times 10^{-5}$ for the tree encoder and $1 \times 10^{-2}$ for the top-down parser.
The batch size is set to 64 per GPU for $m$=$4$, though we also limit the maximum total length for each batch, such that excess sentences are moved to the next batch. The limit is set to 1,536. It takes about 120 hours for 60 epochs of training with $m$=$4$ on 8 A100 GPUs.

\paragraph{Data.}  For English, to fully leverage the scalability of Fast-R2D2, we pretrain Fast-R2D2 on WikiText103~\cite{DBLP:conf/iclr/MerityX0S17}
and then fine-tune the model on the Penn Treebank (PTB)~\cite{marcus-etal-1993-building}
for 10 epochs with the same objective.
WikiText103 is split at the sentence level, and sentences longer than 200 after tokenization are discarded (about 0.04‰ of the original data). 
The total number of sentences is 4,089,500, and the average sentence length is 26.97.
For Chinese, we use a subset of Chinese Wikipedia (Simplified Characters) for pretraining, specifically the first 10,000,000 sentences shorter than 150 characters and then fine-tune on Chinese Penn Treebank (CTB) 8.0~\cite{ctb8}.
We test our approach on PTB WSJ data with the standard splits (2--21 for training, 22 for validation, 23 for test) and the same preprocessing as in recent work \cite{kim-etal-2019-compound}, where we discard punctuation and lower-case all tokens. 
To explore the universality of the model across languages, we further evaluate using the CTB,
on which we also remove punctuation.
Note that in all settings, the training and fine-tuning is conducted entirely on raw unannotated text.

\subsubsection{Results and Discussion}

\begin{table}
\newcommand{\invzero}{\hphantom{0}}
\begin{center}
\setlength{\tabcolsep}{3.pt}
\resizebox{0.45\textwidth}{!}{
\begin{tabular}{@{}l|l|l|l|l@{}}
                    &  eval & mem. & \multicolumn{1}{c|}{WSJ}  & \multicolumn{1}{c}{CTB}  \\
Model               & gran. & cplx  &  $F_1(\mu)$ & $F_1(\mu)$\\ \hline \hline
Left Branching (W)  & WD & $O(n)$& \invzero 8.15  & 11.28 \\
Right Branching (W) & WD & $O(n)$& 39.62 & 27.53 \\
Random Trees (W)    & WD & $O(n)$ & 17.76 & 20.17 \\
\hline
URNNG (W)           & WD & $O(n^3)$& 45.4$^\dag$ & ~~--- \\
ON-LSTM (W)         & WD & $O(n)$  & 47.7$^\dag$ & 24.73 \\
DIORA (W)           & WD & $O(n^3)$& 51.4 & ~~---  \\
StructFormer (W)    & WD & $O(n^2)$& 54.0$^\ddagger$ & ~~--- \\
C-PCFG (W)          & WD & $O(n^3)$& 55.2$^\dag$ & 49.95 \\ \hline
R2D2 (WP)           & WD & $O(n)$ & 48.11 & 44.85  \\
Fast-R2D2$^*$(W)$_{\rm w/o}$ & WD & $O(n)$ & 48.24 & 45.24 \\
Fast-R2D2$^*$(WP)$_{\rm w/o}$ & WD & $O(n)$ & 48.89 & 45.26 \\
Fast-R2D2$^*$(WP)  & WD & $O(n)$ & \textbf{57.22} & \textbf{53.13} \\
\hline \hline
R2D2 (WP)           & WP & $O(n)$  & 52.28 & 63.94 \\ 
Fast-R2D2(WP)      & WP & $O(n)$ & 50.20 & \textbf{67.79} \\
Fast-R2D2$^*$(WP)  & WP & $O(n)$& \textbf{53.88} & 67.74 \\ \hline
\end{tabular}
}
\end{center}
\caption{Unsupervised parsing results with words (W) or word-pieces (WP) as input. ``eval gran." is short for evaluation granularity.
        Values marked with $^{\dag}$ are taken from \newcite{kim-etal-2019-compound}, while $^{\ddagger}$ denotes values taken from \newcite{DBLP:conf/acl/ShenTZBMC20}.
        The bottom three systems are all pre-trained or trained 
        at the word-piece level \textbf{without} span constraints and are measured against word-piece level golden trees. ${\rm w/o}$ means without pretraining.}
\label{tbl:constituency_parsing}
\end{table}


Table~\ref{tbl:constituency_parsing} shows the results of all systems with words (W) and word-pieces (WP) as input on the WSJ and CTB test sets. 
When we evaluate all systems on word-level golden trees, 
our Fast-R2D2 performs substantially better than R2D2 across both datasets.
We denote as Fast-R2D2 the method of using the parser to guide the pruning and selecting the best tree using the chart table and as Fast-R2D2$^*$ the system that uses the top-down parser for tree induction with subsequent R2D2 encoding.
Interestingly, the results suggest that Fast-R2D2$^*$ outperforms Fast-R2D2, especially on the WSJ test set.
Additionally, pretrained Fast-R2D2$^*$
outperforms the models specifically designed for grammar induction.

\begin{table}[!htb]
\small
\begin{center}
\setlength{\tabcolsep}{3.5pt}
\resizebox{0.48\textwidth}{!}{ %
\begin{tabular}{@{}ll| l l l l l l@{}}
 & Model  & WD & NNP & VP & SBAR\\\hline \hline
\multirow{5}{*}{\rotatebox[origin=c]{90}{WSJ}} & DIORA (WP)  & 94.63 & 77.83 & 17.30 & 22.16\\
& C-PCFG (W)                  & ~~--- & ~~--- & 41.7$^\dag$ & 56.1$^\dag$ \\
& C-PCFG (WP)                  & 87.35 & 66.44 & 23.63 & 40.40 \\
& R2D2 (WP)    & \textbf{99.76} & \textbf{86.76} & 24.74 & 39.81\\
& Fast-R2D2$^*$ (WP) & 97.67 & 83.44 & \textbf{63.80} & \textbf{65.68} \\ \hline \hline
\multirow{3}{*}{\rotatebox[origin=c]{90}{CTB}} & C-PCFG(WP) &89.34 & 46.74 & 39.53 & ~~---\\
 & R2D2 (WP) & 97.16 & 67.19 & 37.90 & ~~---\\
 & Fast-R2D2$^*$ (WP) & \textbf{97.80} & \textbf{68.57} & \textbf{46.59} & ~~---
 \\ \hline \hline
\end{tabular}
}
\end{center}
\caption{Recall of constituents and words. WD means word.  Values with $^{\dag}$ are taken from \newcite{kim-etal-2019-compound}.}
\label{tbl:unsupervised_chunking}
\end{table}

Following \newcite{dblp:conf/naacl/kimrykdm19} and \newcite{drozdov-etal-2020-unsupervised},
we also compute the recall of constituents when evaluating on word-piece level golden trees.
Besides standard constituents, we also compare the recall of word-piece chunks and proper noun chunks. 
Proper noun chunks are extracted by finding adjacent unary nodes with the same parent and tag NNP. 
Table~\ref{tbl:unsupervised_chunking} reports the recall scores for constituents and words on the WSJ and CTB test sets. 
Compared with the R2D2 baseline, 
our Fast-R2D2 performs slightly worse for small semantic units, 
but significantly better over larger semantic units (such as VP and SBAR) on the WSJ test set.
On the CTB test set, our Fast-R2D2 outperforms R2D2 on all constituents. 

From Tables~\ref{tbl:constituency_parsing}~and~\ref{tbl:unsupervised_chunking}, 
we conclude that Fast-R2D2 overall obtains better results than R2D2 on CTB, while faring slightly worse than R2D2 only for small semantic units on WSJ. We conjecture that this difference stems from differences in  tokenization between Chinese and English. 
Chinese is a character-based language without complex morphology, where collocations of characters are consistent with the language, making it easier for the top-down parser to learn them well. 
In contrast, word-pieces for English are built based on statistics, and individual word-pieces are not necessarily natural semantic units. Thus, there may not be sufficient semantic self-consistency, such that it is harder for a top-down parser with a small number of parameters to fit it well.

\subsection{Downstream Tasks}
We next consider the effectiveness of Fast-R2D2 in downstream tasks. This experiment is not intended to advance the state-of-the-art on the GLUE benchmark but rather to assess to what extent our approach performs respectably against the dominant inductive bias as in conventional sequential Transformers.

\subsubsection{Setup}
\paragraph{Data and Baseline.}
We fine-tune pretrained models on several datasets,
including SST-2, CoLA, QQP, and MNLI from the GLUE benchmark~\cite{wang2018glue}.
As sequential Transformers with their dominant inductive bias remain the norm for numerous NLP tasks, 
we mainly compare Fast-R2D2 with \bert~\cite{devlin2018} as a representative pretrained model based on a sequential Transformer. 
We did not include recursive models such as Gumbel-Tree-LSTMs~\cite{DBLP:conf/aaai/ChoiYL18} and CRvNN~\cite{DBLP:conf/icml/ChowdhuryC21} among our baselines, as they are not pretrained models.
In order to compare the two forms of inductive bias fairly and efficiently,
we pretrain \bert models with 4 layers and 12 layers as well as our Fast-R2D2 from scratch on the WikiText103 corpus following Section~\ref{sec:LM_setup}. 
Considering that longer inputs in the pre-training stage are helpful for BERT’s downstream task performance, we use the original corpus that is not split into sentences as inputs.
For simplicity, Fast-R2D2 is fine-tuned without span constraints.
Following the common settings, we add an MLP layer over the root representation of the R2D2 encoder for single-sentence classification. 
For cross-sentence tasks such as QQP and MNLI, we feed the root representations of the two sentences into the pretrained tree encoder of R2D2 as left and right inputs, 
and also add a new task ID as another input term to the R2D2 encoder. 
Then we feed the hidden output of the new task ID into another MLP layer to predict the final label.
We train all systems across the four datasets for 10 epochs 
with a learning rate of $5\times 10^{-5}$, batch size $64$, and maximum input length $200$.
We validate each model in each epoch and report the best results on development sets.

\begin{table}
\begin{center}
\setlength{\tabcolsep}{1.5pt}
\resizebox{0.48\textwidth}{!}{
\begin{tabular}{l|c|r r|r r}
\multirow{4}{*}{Model} & \multirow{4}{*}{Para.} & \multicolumn{2}{c|}{Single sent.} & \multicolumn{2}{c}{Cross sent.} \\
 &  & \begin{tabular}[c]{@{}l@{}}SST-2\\ (Acc.)\end{tabular} & \begin{tabular}[c]{@{}l@{}}CoLA\\ (Mcc.)\end{tabular} & \begin{tabular}[c]{@{}l@{}}QQP\\ (F1)\end{tabular} & \begin{tabular}[c]{@{}l@{}}MNLI\\m/mm\\ (Acc.)\end{tabular}            \\ \hline \hline
\bert (4L)  & 52M & 84.98 & 17.07 & 84.01 & 73.73/74.63 \\
\bert (12L) & 116M & 90.25 & 40.72 & 87.13 & 80.00/80.41 \\ \hline
R2D2        & 52M & 89.33 & 34.79 & 84.27 &  69.35/68.72 \\ \hline
Fast-R2D2$^\dag$& {\multirow{2}{*}{\begin{tabular}[c]{@{}c@{}}\\52M/\\ 10M\end{tabular}}} & 87.50 & 8.67 & 83.97 & 69.53/69.50 \\
Fast-R2D2$^*\dag$& {} & 88.30 & 10.14 & 84.07 & 69.36/69.11 \\
Fast-R2D2  & {} & 90.25 & 38.45 & 84.35 & 69.36/68.80 \\ 
Fast-R2D2$^*$& {} & 90.71 & 40.11 & 84.32 & 69.64/69.57\\
\hline \hline
\end{tabular}
}
\end{center}
\caption{Downstream results. All systems are pretrained from scratch on WikiText103.
        Para.\ describes the number of parameters for each model. Fast-R2D2 contains the R2D2 encoder and top-down parser, two components with 52M and 10M parameters, respectively.
        Mcc.\ stands for Matthew's correlation coefficient.
        Fast-R2D2 with $\dag$ are models fine-tuned without $\mathcal{L}_\mathrm{bilm}$ for an ablation study.
    }\vspace{-10pt}
\label{tbl:classification}
\end{table}
\subsubsection{Results and Discussion}
Table~\ref{tbl:classification} shows the corresponding scores on SST-2, CoLA, QQPl, and MNLI. 
In terms of the parameter size, our Fast-R2D2 model has 52M and 10M parameters for the R2D2 encoder and top-down parser, respectively.
It is clear that 12-layer \bert is significantly better than 4-layer \bert.
As mentioned in Section~\ref{sec:downstream}, Fast-R2D2 has two options to construct the final tree and representation for a given input sentence:
Fast-R2D2$^*$ uses the output tree from the top-down parser, while Fast-R2D2 uses the best tree inferred by the R2D2 encoder.
Similar to the results for unsupervised parsing, Fast-R2D2$^*$ in classification tasks again outperforms Fast-R2D2.
We hypothesize that trees generated by the top-down parser without Gumbel noise are more stable and reasonable.
Fast-R2D2 significantly outperforms 4-layer \bert and achieves competitive results compared to 12-layer \bert in single sentence classification tasks such as SST-2 and CoLA, but still performs significantly worse in the cross-sentence tasks. 
We believe this is an expected result, as there is no cross-attention mechanism in the inductive bias of Fast-R2D2. 
However, the performance of Fast-R2D2 on classification tasks shows that the inductive bias of R2D2 has higher parameter utilization than sequentially applied Transformers.
Importantly, we demonstrate that a Recursive Neural Network variant with an unsupervised parser can achieve comparable results to pretrained sequential Transformers even with fewer parameters and interpretable intermediate results, 
Hence, our Fast-R2D2 framework provides an alternative for NLP tasks.

\subsection{Speed Evaluation}
To assess the time cost, we mainly compare sequential Transformers and Fast-R2D2 in forced encoding on various sequence length ranges. We randomly select 1,000 sentences for each range from WikiText103 and report the average time consumption on a single A100 GPU. \bert is based on the open source Transformers library\footnote{\url{https://github.com/huggingface/transformers}} and R2D2 is based on the official code in \newcite{hu-etal-2021-r2d2}.\footnote{\url{https://github.com/alipay/StructuredLM_RTDT/tree/r2d2}}

\begin{table}% [htb!]
\small
\begin{center}
\setlength{\tabcolsep}{3.pt}
\resizebox{0.45\textwidth}{!}{
\begin{tabular}{l|rrrr}
\multirow{2}{*}{Model} & \multicolumn{4}{c}{Sequence Length Ranges} \\\cline{2-5}
      & \multicolumn{1}{c|}{0--50} & \multicolumn{1}{l|}{50--100} & \multicolumn{1}{l|}{100--200} & 200--500 \\ 
\hline
\bert (12L) & \multicolumn{1}{r|}{1.36}     & \multicolumn{1}{r|}{1.46}       & \multicolumn{1}{r|}{1.62}        & 2.38 \\ \hline
R2D2  & \multicolumn{1}{r|}{38.06}     & \multicolumn{1}{r|}{173.74}       & \multicolumn{1}{r|}{555.95}        &    ---     \\
Fast-R2D2  & \multicolumn{1}{r|}{4.67} & \multicolumn{1}{r|}{14.91} & \multicolumn{1}{r|}{39.73} & 150.26 \\
Fast-R2D2* & \multicolumn{1}{r|}{1.28} & \multicolumn{1}{r|}{2.96}  & \multicolumn{1}{r|}{5.56}  & 10.70 \\ 
\hline \hline
\end{tabular}
}
\end{center}
\caption{Inference time in seconds for various systems to process 1,000 sentences with a batch size of 50.}
\label{tbl:speed_test}
\end{table}

Table~\ref{tbl:speed_test} shows the inference time in seconds for different systems to process 1,000 sentences with a batch size of 50.
Running R2D2 is time-consuming, since the heuristic pruning method involves substantial memory exchanges between GPU and CPU. 
In Fast-R2D2, we alleviate this problem by using model-guided pruning to accelerate the chart table processing,
in conjunction with a code implementation in CUDA, Fast-R2D2 reduces the inference time significantly. 
Fast-R2D2$^{*}$ further improves the inference speed by running forced encoding in parallel over the binary tree generated by the parser, which is about 30--50 times faster than R2D2 in various ranges. 
Although there is still a gap in speed compared to sequential Transformers, Fast-R2D2$^{*}$ is sufficiently fast for most NLP tasks while producing interpretable intermediate representations.

\section{Related Work}
\label{sec:work}

\textbf{Programming by example.} Programming by example problems have been widely studied with various applications, and recent works have developed deep neural networks as program synthesizers~\cite{gulwani2012spreadsheet,parisotto2016neuro,devlin2017robustfill,bunel2018leveraging}. Most prior works focus on synthesizing programs in domain-specific languages, such as FlashFill~\cite{parisotto2016neuro,devlin2017robustfill,vijayakumar2018neural} for string transformation, Karel~\cite{bunel2018leveraging,shin2018improving,chen2018execution,gupta2020synthesize} for simulated robot navigation, and LISP-style languages for list manipulation~\cite{balog2016deepcoder,polosukhin2018neural,Zohar2018AutomaticPSExtendExecution,nye2019learning}. In this work, we make the first attempt of synthesizing C code in a restricted domain from input-output examples only, and we focus on the list manipulation domain.

Some recent works investigate the limitations of synthetic datasets and the ambiguity in program specifications for neural program synthesis~\cite{shin2019synthetic,clymo2020data,suh2020creating,laich2019guiding}. These works focus on reducing the bias of data distributions and generating more diverse input-output pairs, while our data regeneration aims to improve the quality of programs. We consider incorporating both lines of work to further improve the dataset quality as future work. In addition, drawing the inspiration from self-training and bootstrapping techniques developed for other applications~\cite{mooney1993bootstrapping,abney2002bootstrapping,mcclosky2006effective,xie2020self} to extend our iterative retraining scheme is also another future direction.

\textbf{Execution-guided program synthesis.} To learn better program representations, some recent works incorporate the execution information to guide the synthesis process~\cite{sun2018neural,Zohar2018AutomaticPSExtendExecution,shin2018improving,chen2018execution,Ellis2019WriteEAExtendExecution,tian2019learning,balog2020neural,gupta2020synthesize,odena2020bustle,nye2020representing,mandal2021learning}. In particular, leveraging partial program execution states improves the performance for several program synthesis tasks~\cite{chen2018execution,Zohar2018AutomaticPSExtendExecution,Ellis2019WriteEAExtendExecution,nye2020representing}. However, existing approaches rely on program interpreters to provide the intermediate execution results whenever applicable. In contrast, we demonstrate that our latent executor learns the latent execution traces (\laet{}) without such a requirement. Besides program synthesis, execution traces have also been utilized for other software engineering applications~\cite{alam2019zero,mendis2019ithemal}.

\textbf{Neural execution.} Our latent executor is related to prior works on learning to execute algorithms~\cite{zaremba2014learning,velivckovic2019neural,yan2020neural} and programs~\cite{bieber2020learning}. They focus on predicting execution results for full algorithms and programs, but do not utilize them for program synthesis. Latent state prediction has also been studied in other applications such as task-oriented dialogue systems~\cite{min2020dsi,zhang2020probabilistic} and robotics~\cite{paxton2019prospection}.



\vspace{-3mm}
\section{Conclusion}
\label{sec:conc}
\vspace{-2mm}

%In this paper, we bring a human in the loop to  facilitate the learning process of an image captioning model. 
In this paper, we enable a human teacher to provide feedback to the learning agent in the form of natural language. We focused on the problem of image captioning. 
We proposed a hierarchical phrase-based RNN as our captioning model, which allowed natural integration with human feedback. We crowd-sourced feedback for a snapshot of our model, and showed how to incorporate it in Policy Gradient optimization. We showed  that by exploiting descriptive feedback our model learns to perform better than when given independently written captions. %In the future, we aim to explicitly deal with the annotator 

%There are several exciting avenues for future work. In particular, when dealing with crowd-sourced human reward one needs to take into account annotator noise. We further plan to explore self-criticism, where the agent automatically decides when to seek human advice, possibly in a dialog-like setting. 

\vspace{-3mm}
\section*{Acknowledgment}
\vspace{-3mm}
\begin{small}
We gratefully acknowledge the support from NVIDIA for their donation of the GPUs used for this research. This work was partially  supported by NSERC. We also thank Relu Patrascu for infrastructure support. \end{small}



{%\small
\bibliographystyle{abbrv}
\bibliography{ref}
}

%\clearpage
%\section*{Checklist}


\begin{enumerate}

% \answerTODO{}, \answerYes{}, \answerNo{}, \answerNA{}

\item For all authors...
\begin{enumerate}
  \item Do the main claims made in the abstract and introduction accurately reflect the paper's contributions and scope?
    \answerYes{}
  \item Did you describe the limitations of your work?
    \answerYes{}\\
    \textcolor{blue}{See Section~\ref{sec:conclusion}.}
  \item Did you discuss any potential negative societal impacts of your work?
    \answerYes{}\\
    \textcolor{blue}{See Section~\ref{sec:conclusion}.}
  \item Have you read the ethics review guidelines and ensured that your paper conforms to them?
    \answerYes{}
\end{enumerate}

\item If you are including theoretical results...
\begin{enumerate}
  \item Did you state the full set of assumptions of all theoretical results?
    \answerNA{}
	\item Did you include complete proofs of all theoretical results?
    \answerNA{}
\end{enumerate}

\item If you ran experiments...
\begin{enumerate}
  \item Did you include the code, data, and instructions needed to reproduce the main experimental results (either in the supplemental material or as a URL)?
    \answerYes{}\\
    \textcolor{blue}{We provide our code in the supplementary materials.}
  \item Did you specify all the training details (e.g., data splits, hyperparameters, how they were chosen)?
    \answerYes{}\\
    \textcolor{blue}{We state the details in Section~\ref{sec:exp} and Appendix~\ref{sec:details}.}
	\item Did you report error bars (e.g., with respect to the random seed after running experiments multiple times)?
    \answerYes{}\\
    \textcolor{blue}{We report the error bars for the experiments with larger variances (e.g., Table~\ref{tab:bg-main}).}
	\item Did you include the total amount of compute and the type of resources used (e.g., type of GPUs, internal cluster, or cloud provider)?
    \answerYes{}\\
    \textcolor{blue}{See common setup in Section~\ref{sec:exp}.}
\end{enumerate}

\item If you are using existing assets (e.g., code, data, models) or curating/releasing new assets...
\begin{enumerate}
  \item If your work uses existing assets, did you cite the creators?
    \answerYes{}\\
    \textcolor{blue}{We cite the datasets and libraries we use.}
  \item Did you mention the license of the assets?
    \answerYes{}\\
    \textcolor{blue}{We only use the public datasets and open source libraries.}
  \item Did you include any new assets either in the supplemental material or as a URL?
    \answerNA{}
  \item Did you discuss whether and how consent was obtained from people whose data you're using/curating?
    \answerNA{}
  \item Did you discuss whether the data you are using/curating contains personally identifiable information or offensive content?
    \answerNA{}
\end{enumerate}

\item If you used crowdsourcing or conducted research with human subjects...
\begin{enumerate}
  \item Did you include the full text of instructions given to participants and screenshots, if applicable?
    \answerNA{}
  \item Did you describe any potential participant risks, with links to Institutional Review Board (IRB) approvals, if applicable?
    \answerNA{}
  \item Did you include the estimated hourly wage paid to participants and the total amount spent on participant compensation?
    \answerNA{}
\end{enumerate}

\end{enumerate}

\clearpage
\appendix
%\appendices
\section{Pseudocode for Algorithm of Section~\ref{subsec:instantly_decodable}}
\label{app:pseudocode}

\algdef{SE}[EVENT]{Event}{EndEvent}[1]{\textbf{upon event}\ #1\ \algorithmicdo}{\algorithmicend\ \textbf{event}}%
\algtext*{EndEvent}

\begin{algorithm}
\caption{Coding for Three Users with Feedback}\label{alg:three_users}
\begin{algorithmic}[1]
\State \textbf{Initialize}: $r_i \gets 0, \forall i \in \mathcal{U}$  % \Comment{Packets received by each user}
\ForAll {$t \in [N]$}
    \While {$\nexists \ i \in \mathcal{U} \ s.t.\ Z_i = 0$}
%    \State Send $S(t)$ until at least one user receives it
    	\State Transmit $S(t)$
    \EndWhile
%    \If {$\exists \ i \in \mathcal{U}, T \in \mathbb{N} \ s.t.\ N_i(T) = 0$}
%    \If {$\exists \ i \in \mathcal{U} \ s.t.\ N_i = 0$}
        \State $Q_{\mathcal{E}} \gets Q_{\mathcal{E}} \cup \{S(t)\}$
        \State $r_j\gets r_j + 1 \qquad \forall \ j \ s.t.\  Z_j = 0$
%    \EndIf
\EndFor

%\While{$\exists \ \textrm{distinct} i, j, k \in \mathcal{U}, i\neq j \neq k \ s.t.\ Q_i \neq \varnothing \ \textbf{and} \ \Q{j,k} \neq \varnothing$}
\While{$\exists \ \textrm{distinct} \ i, j, k \in \mathcal{U}, \ s.t.\ Q_i \neq \varnothing \ \textbf{and} \ \Q{j,k} \neq \varnothing$}
    \While {$\nexists \ l \in \mathcal{U} \ s.t.\ Z_l = 0$}
%	\State Let $q_i \in Q_i$, $q_{j,k} \in \Q{j,k}$ 
%       \State Transmit $q_i \oplus q_{j,k}$ until at least one user receives it  
       \State Transmit $q_i \oplus q_{j,k}$ where $q_i \in Q_i$, $q_{j,k} \in \Q{j,k}$ 
    \EndWhile
    \State $r_l\gets r_l + 1 \qquad \forall \ l \ s.t.\  Z_l = 0$
    \If {$Z_i = 0$}
    	\State $Q_i \gets Q_i \setminus \{q_i\}$
    \EndIf
    \If {$Z_j = 0 \ \textbf{and} \ Z_k = 1$}
    	\State $Q_k \gets Q_k \cup \{q_{j, k}\}$
       \State $\Q{j,k} \gets \Q{j, k} \setminus \{q_{j,k}\}$
    \ElsIf {$Z_j = 1 \ \textbf{and} \ Z_k = 0$}
    	\State $Q_j \gets Q_j \cup \{q_{j, k}\}$
	\State $\Q{j,k} \gets \Q{j, k} \setminus \{q_{j,k}\}$
    \ElsIf {$Z_j = 0 \ \textbf{and} \ Z_k = 0$}
    	\State $\Q{j,k} \gets \Q{j, k} \setminus \{q_{j,k}\}$
    \EndIf    
\EndWhile

\While{$Q_i \neq \varnothing \  \forall \ i \in \mathcal{U}$}
	\While {$\nexists \ l \in \mathcal{U} \ s.t.\ Z_l = 0$}
		\State Transmit $q_1 \oplus q_{2} \oplus q_{3}$, where $q_i \in Q_i \ \forall i \in \mathcal{U}$
	\EndWhile
	\ForAll {$l \in \mathcal{U} \ s.t.\ Z_l = 0$}
		\State $Q_l \gets Q_l \setminus \{q_l\}$
		\State $z_l \gets z_l + 1$
	\EndFor
\EndWhile

\Event{$ \exists \ i \ s.t.\ r_i \geq 1- d_i  $} %\Comment{One user finished}
%	\State Let $j, k \in \mathcal{U} \setminus \{i\}, \ s.t.\ j \neq k$
	\State $Q_j \gets Q_j \cup \Q{i,j} \qquad \forall \ j \in \mathcal{U} \setminus \{i\}$
	\State \textbf{run} two-user algorithm of Section~\ref{sec:two_users}
\EndEvent
\end{algorithmic}
\end{algorithm}

\end{document}
