\section{Spallation cuts for the solar analysis}
\label{sec:spacuts}
Here, we present a new spallation cut that improves on the reduction strategy described in Sec.~\ref{sec:spaprevious}. We take advantage of several improvements and studies that took place within the last decade. First, the muon track reconstruction was replaced. Previously we used a simple, fast muon track fitter developed at the beginning of SK. It assumes through-going single muons and misreconstructs or fails on other muons. The more complex muon track reconstruction of this analysis categorizes as described in Sec.~\ref{sec:muons}, and reconstructs all categories (up to ten tracks). It was used to reject spallation background by dE/dx reconstruction in the search for diffuse supernova neutrino interactions in SK-I, II, III~\cite{bib:sksrn123}, which
inspired the development of
FLUKA-based simulation studies~\cite{BLi_1} and highlighted the importance of muon-induced hadronic showers for isotope identification, and allowed to characterize their shapes and sizes. Finally, the improvements in the detector electronics associated with the SK-IV phase allowed to raise the PMT saturation rate and detect higher values of the total charge (and therefore $Q_{res}$), as well as identify neutron clouds as described in Sec.~\ref{sec:neutspall}.

The new spallation reduction strategy proceeds as follows. First, we apply two sets of preselection cuts in order to remove a sizable fraction of spallation events with minimal harm to the signal efficiency. These cuts aim at removing events close in space and time to neutron clouds, as well as clusters of low energy events, typically associated with the decays of multiple isotopes produced by the same muons. Then, we remove most of the remaining spallation events using an updated version of the likelihood cut described in Sec.~\ref{sec:spaprevious}. 

\subsection{Neutron Cloud Spallation Cut for Solar Analysis}
\label{sec:cloudcut}
Using the observables defined in Sec.~\ref{sec:vardef}, we define a set of cuts to reliably identify neutron clouds and investigate their space and time correlation with solar event candidates (most of which are spallation events before cuts). First, we define neutron candidates as WIT events found less than 500$~\mu$s after a muon and within 5~m of its track. The number of these candidates gives the neutron cloud multiplicity. Then, in order to compute the cloud barycenter, we consider a high-purity subsample of these neutron candidates, requiring them to verify $\Delta t >  20~\mu$s and $E_{rec} < 5$~MeV. We then assign weights to these candidates depending on their vertex and direction goodness $g_t$ and $g_p$. Specifically, we consider three regions of weights $0$, $1$, and $2$ in the $g_t-g_p$ space, that are shown in Fig.~\ref{fig:mcovaq}.

Once clouds are identified and their barycenter is defined, their positions and detection times can be compared to the location and times of solar event candidates. For this analysis we consider SK-IV low energy events that passed all the cuts defined for the solar analysis in Sec.~\ref{sec:solaranalysis} except spallation cuts. This sample is expected to be largely dominated by spallation isotope decays. We then invert the time sequence similar to the procedure described in Sec.~\ref{sec:spaprevious}, this time considering neutron clouds observed up to 60~s before each low energy event. In order to take advantage of the expected shower shape of neutron clouds, we then need to define a specific coordinate system for each cloud. Here, we consider three possible options, shown in Fig.~\ref{fig:coordinate_systems}. First, the axes of the new coordinate system could align with the axes of the best-fit ellipsoid of neutron cloud. This option is not practical, however, due to the large shape uncertainties for the low multiplicity clouds. A second possibility is to use the muon track as the $z$ axis of our coordinate system and the center of the neutron cloud as its origin. Finally, the third option also uses the muon track as a $z$ axis but sets the projection of the cloud center on the muon track as the origin. In order to assess the discriminating power of these last two options, we compute the transverse and longitudinal distances of low energy events, $l_t$ and $\Delta l_{\mbox{\tiny LONG}}=l_{\mbox{\tiny LONG}}^{\mbox{\tiny isotope}}-l_{\mbox{\tiny LONG}}^{\mbox{\tiny n-cloud}}$, defined in Fig.~\ref{fig:coordinate_systems}, to the origins of their respective coordinate systems. The distribution of $l_t$ and $\Delta l_{\mbox{\tiny LONG}}$ is shown in Fig.~\ref{fig:neutmuoncomp} for all neutron cloud multiplicities. We notice that using the projection of the neutron cloud center on the muon track as an origin significantly reduces the spread of this distribution in $l_t$, a spread that is primarily driven by contributions from low multiplicity clouds. We hence choose this definition of the origin and set the $z$ axis to be along the muon track for our analysis. 

\begin{figure}
    \centering
    \includegraphics[width=9cm]{neutron_cuts/figures/neutron_cloud_coordinate_systems.eps} %LLMM path expansion needed for arXiv
    \caption{Possible options for the neutron cloud coordinate systems. Left: the origin is the neutron cloud center and the axes are the neutron cloud axes. $\theta_n$ is the angle with respect to the muon track. Center: the origin is the neutron cloud center and the $z$ axis is aligned with the muon track. Right: the origin is the projection of the neutron cloud center along the muon track and the $z$ axis is aligned with the muon track. We also show the definition of the transverse and longitudinal distances $l_t$ and $\Delta l_{\mbox{\tiny LONG}}$, between a spallation isotope and the origin.}
    \label{fig:coordinate_systems}
\end{figure}

Using the coordinate system defined on the right panel of Fig.~\ref{fig:coordinate_systems} we then define cuts on $\Delta t$, $l_t$ and $\Delta l_{\mbox{\tiny LONG}}$ for each low energy event--neutron cloud pair, with $\Delta t$ defined as the time difference between the low energy event and the muon associated with the cloud. We first define spherical cuts, removing events within either $0.2$~s and $7.5$~m or $2$~s and $5$~m of clouds with 2 or more neutrons. Then, we define multiplicity bins of 2, 3, 4--5, 6--9, and $\geq$10~neutrons candidates, and, for each bin, define a specific ellipsoidal cut on {\it lt} and {\it ln}. Since clouds with only 2~candidates are often not associated with hadronic showers, as shown in Sec.~\ref{sec:simucompare}, we require $\Delta t < 30$~s. For higher multiplicities we consider all muons up to 60~s before the low energy event. The different cuts are summarized in Table~\ref{tab:cloudtab}. If a low energy event--neutron cloud pair correlation is within the required $\Delta t$, $l_t$ and $l_{\mbox{tiny LONG}}$ values, the corresponding low energy event is rejected.

\begin{figure*}
    \centering
    \includegraphics[width=\textwidth]{./neutron_cuts/figures/compare_types_cont.eps}
    \caption{Tagged spallation for all neutron cloud multiplicities using a neutron cloud centered~(left) and muon track centered~(right) orientation. Greater uncertainty in lower neutron multiplicity reconstruction compared to muon track reconstruction drives the difference in ability to accurately tag spallation. Shown contour levels are arbitrary but consistent between the left and the right panel.}
    \label{fig:neutmuoncomp}
\end{figure*}

\begin{table}[t]
    \centering
        \caption{Table showing the different cut conditions for the cloud cut. The symbol `+' is used to represent showers of at least that multiplicity.}
    \begin{tabular}{ c*{6}{c}c }
    \toprule Multiplicity  & 2+ & 2+ & 2 & 3 & 4--5 & 6--9 & 10+ \\\hline
    \% of Showers & 100 & 100 & 73 & 15 & 6 & 3 & 3  \\\hline
    $\Delta t$  & 0.2 & 2 & 30 & 60 & 60 & 60 & 60 \\\hline
    $\Delta l_{\mbox{\tiny LONG}}$ [cm] & 750 & 500 & 350 & 500 & 550 & 650 & 700  \\\hline
    %$lt^2$ [cm2] & 40000 & 60000 & 120000 & 200000 & 250000 & 562500 & 250000 \\\hline
    $l_t$ [cm] & 750 & 500 & 200 & 245 & 346 & 447 & 500 \\\bottomrule        % LLMM botrule -> bottomrule
    \end{tabular}
    \label{tab:cloudtab}
\end{table}

Since the sample used for this analysis is largely dominated by spallation isotope decays, the background rejection rate of the neutron cloud cut can be readily estimated. To evaluate the deadtime, we use a sample of events with reconstructed energies between 3.5~MeV and 5~MeV whose vertices have been replaced by randomly generated ones. We then pair these events with muons observed up to 60~s after them and apply the reduction steps outlined above, with the sign of $\Delta t$ inverted. The deadtime is then given by the fraction of remaining low energy events and has been found to 1.3\%. 
 
