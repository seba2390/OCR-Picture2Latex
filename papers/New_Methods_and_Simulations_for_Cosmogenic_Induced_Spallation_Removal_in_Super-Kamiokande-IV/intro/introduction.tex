\section{introduction}
Spallation from cosmic-ray muons produces radioactive isotopes and induces one of the largest backgrounds for the Super-Kamiokande (SK) neutrino signal between ${\sim}6$ and ${\sim}25$~MeV. Reducing this background is pivotal for the success of many different analyses in this energy range, and has major implications in solar, reactor, and supernova relic neutrino searches. Specifically in SK, cosmic ray muons and the showers they produce sometimes interact with $^{16}\mbox{O}$ nuclei within the detector volume, producing radioactive isotopes. 

Showers induced by the muons are primarily electromagnetic in nature ($\gamma$-rays and electrons) as a result of delta-ray production, pair production, and bremsstrahlung. However, there is also the possibility for muons to produce secondary particles in the form of neutrons, pions, and others. Recent simulation studies have shown that most spallation isotopes are produced by these secondary particles, with only $11\%$ of isotopes being made directly from muons, and hence these isotopes can be found up to several meters away from the muon track~\cite{BLi_1,BLi_2,BLi_3}. These isotopes then undergo mostly $\beta$ or $\beta\gamma$ decays, mimicking the expected signal for neutrino interactions. Their half-lives extend from milliseconds to seconds, and thus can be much larger than the time interval between two muons in SK, where the muon rate is about $2$~Hz. Identifying spallation isotopes by pairing them with their parent muons is therefore particularly challenging.

In previous SK analyses, spallation reduction algorithms characterized muon signatures solely by considering their reconstructed tracks and prompt light deposition patterns. This method had important limitations due to its reliance on the muon track reconstruction quality. Moreover, while the muon prompt light contains information about produced showers, the shower signatures are partially obscured by muon Cherenkov light. As a consequence, cylindrical cuts around the entire muon track are often necessary.

For the solar neutrino analysis in SK, applying a likelihood cut based on time difference, distance to the muon track and muon light yield  removed $20\%$ of the signal while rejecting $90$\% of the background in the $6.0$-$19.5$~MeV kinetic energy range. The remaining background is dominated by decays of $^{16}\mbox{N}$, which is not only the most abundantly produced isotope~\cite{SKspall_zhang}, but also is particularly difficult to identify \cite{skivsolar}. 

$^{16}\mbox{N}$ is primarily produced through $(n, p)$ interactions on $^{16}$O involving neutrons from muon-induced hadronic showers. Since such neutrons can reach GeV-scale energies,  $^{16}\mbox{N}$ can be found up to several meters away from the muon track. This large distance, together with the long half-life of this isotope, $7.3$~s, makes it particularly difficult to correlate $^{16}\mbox{N}$ decays with their parent muon. Moreover, these decays occur either through the $\beta\gamma$ (66\%) or the $\beta$ (28\%) channel, producing particles with energies ranging between $3.8$ and $10.4$~MeV, well within the solar neutrino energy range. Hence, removing $^{16}$N using only muon track information is particularly difficult and results in significant reduction in neutrino signal efficiency.

Showers producing $^{16}\mbox{N}$ typically contain many neutrons ($\mathcal{O}(100)$) \cite{BLi_comm} which capture on H after thermalizing, as follows:
\begin{equation}
    n + \mathrm{^1}\mbox{H} \rightarrow \mathrm{^{2}}\mbox{H} + \gamma\ (2.2\ \mbox{MeV})
\end{equation}
A single 2.2 MeV $\gamma$ is difficult to be seen in SK, but the large neutron multiplicity makes it possible to directly tag the showers and use them for spallation identification purposes. 

In this paper, we develop a framework to characterize muon-induced spallation processes. In particular we describe new methods to improve spallation identification at SK by tagging the hadronic shower components, identifying clusters of spallation isotopes, and expanding the previously developed spallation cut.  Additionally we present a complete simulation of cosmic muon spallation in SK, inspired by the FLUKA simulations~\cite{fluka_paper,Battistoni:2015epi} developed by~\cite{BLi_1}. 

 The characteristics of the SK detector and its trigger system is described in Sec.~\ref{sec:SKdetector}, and the reconstruction algorithms targeting muons and low energy events are explained in Sec.~\ref{sec:eventreco}. Then, in Sec.~\ref{sec:simulation} we present a FLUKA-based simulation framework modeling muon propagation and shower generation in water.  In Sec.~\ref{sec:neutspall} we then show how to take advantage of a recently deployed trigger system at SK to characterize muon-induced hadronic showers and use them to reduce spallation backgrounds. We demonstrate the ability of our simulation to accurately model the shapes and sizes of these hadronic showers, and thus be used to design future analysis strategies, in Secc~\ref{sec:simucompare}. Finally, in Sec.~\ref{sec:solaranalysis} and Sec.~\ref{sec:spacuts}, we discuss how to use the insight gained through our approach to design new spallation reduction tools for the solar neutrino analysis. We first describe the solar neutrino analysis strategy and the spallation cut used for previous searches. Then we propose an update of this spallation cut and discuss how this new approach leads to a significant increase in the signal acceptance. Although we focus on the solar neutrino analysis for this paper, the techniques we present here can be readily applied to a wide range of low energy neutrino searches, targeting e.~g.~astrophysical transients and the diffuse supernova neutrino background. 
 We measure the yields of spallation isotopes in Sec.~\ref{sec:yields} and demonstrate the FLUKA-based simulation's ability to successfully predict these yields within hadronic model uncertainties.
 Sec.~\ref{sec:conclusion} discusses the paper's conclusions.