\section{Event Reconstruction}
\label{sec:eventreco}

\subsection{Cosmic Rays and Track Fits}
\label{sec:muons}
Muons pass through SK at a rate of approximately 2~Hz, and their tracks are reconstructed from the PMTs hit within the detector. The muon reconstruction used for this analysis (outlined in \cite{muboy1, muboy2}) accurately fits tracks as well as categorizing muons. After removing PMT noise hits, the fitter makes an initial guess on the track using the earliest hit PMT with at least three neighboring hits as an entry point and time, and the largest cluster of charge as the exit point. The track parameters are then varied and a likelihood dependent on the expected Cherenkov light pattern is maximized to get a final track fit. Muons are categorized based on characteristics of the observed light, with the four different muon categories described by:
\begin{enumerate}
    \item Single Through Going ($\sim$82\%): Single muons that pass entirely through the detector and is the default fit category for a muon. 
    \item Stopping ($\sim$7\%): Single muons that enter the ID but do not exit it. Identified by low light observed near the exit point of the muon and nearby OD information. 
    \item Multiple ($\sim$7\%): Bundles of muons passing through the detector simultaneously. Identified by light inconsistent with a single Cherenkov cone. 
    \item Corner Clipping ($\sim$4\%): Single through going muons found to have a track length of less than 7~m inside the ID, while also occurring near the top or bottom of the detector. 
\end{enumerate}
To check the fitter accuracy, $\sim$2000~events were fit by this method and by hand. For the categories found by the fitter, $\sim$0.5\% of single through going, $\sim$1.4\% of corner clipping, $\sim$13\% of multiple, and $\sim$30\% of stopping muons were found to be something else by eye-scan. The difference in the mistagged stopping muons were hand fit to be through going muons. Resolution studies found the entry point resolution to be 100~cm for all types of muons, except multiple muons with more than 3 tracks, and a directional resolution of 6$^\circ$. In this analysis, if multiple tracks were fit, the principal track is used to identify subsequent events. 