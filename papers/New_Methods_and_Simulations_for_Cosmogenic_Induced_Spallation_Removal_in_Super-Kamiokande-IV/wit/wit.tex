%\subsection{Software Triggers}
\label{sec:softtrig}


SK-IV introduced QTC Based Electronics with Ethernet (QBEE) and new data acquisition (DAQ) computers allowing for enhanced data processing capability~\cite{Nishino:2009zu}. The new DAQ substantially increase the bandwidth of the information collected from the PMTs: each PMT triggers by itself and then its integrated charge and trigger time is digitized and processed. The QBEEs are equipped with three different amplifiers for the charge digitization. These amplifiers enable a dynamic charge detection range, allowing $\sim$5 times increase in the maximum detected charge for an individual PMT. 

Due to these upgrades the former hardware trigger is replaced by software triggers: Data are acquired in 17~$\mu$s segments (hardware triggers) controlled by a 60~kHz clock. The 17~$\mu$s segments connect seamlessly to each other. The standard software trigger applies a simple coincidence criterion: at least 31 triggered PMTs (or hits) within about 200~ns of each other. The dark noise rate per 200~ns of all PMTs is about 11, so a coincident signal from about 20~PMTs is needed. PMT data are saved in a window starting from 500~ns before and 1000~ns after trigger time. If the coincidence is larger than 47 triggered PMTs, the trigger window is expanded to stretch from 5~$\mu$s before and 35~$\mu$s after the trigger time. While the standard SK trigger is well-suited for events with kinetic energies down to $\sim$3.5~MeV, it is not sensitive to neutron captures on hydrogen which are the key to identifying hadronic showers in SK-IV.

A neutron capture on hydrogen emits a single 2.2~MeV $\gamma$-ray. Such events produce little Cherenkov light; in SK-IV 2.2~MeV $\gamma$s result in only about seven detected photo-electrons on average, so the standard software trigger efficiency is very small. After an event of at least $\sim$8~MeV electron-equivalent energy, the standard software trigger will automatically issue after-triggers (AFT) that record all hit PMTs within the next 500~$\mu$s. A cosmic muon easily fulfills this condition and the AFT triggers would catch most 2.2~MeV $\gamma$-rays from subsequent neutron captures, although reliable identification of the 2.2~MeV $\gamma$ signal is possible for only $\sim$20\% of them. However, to not unduly strain the standard software trigger, these AFT triggers are disabled after cosmic muons and thus do not allow to identify neutrons from muon-induced showers. 

The Wideband Intelligent Trigger (WIT) receives the full copy of all PMT data before triggering and runs in parallel to the standard software trigger~\cite{wit,Elnimr:2017nzi}. It is designed to trigger on electrons of at least 2.5~MeV kinetic energy and has some efficiency to stand-alone triggering of 2.2~MeV $\gamma$-rays. WIT is implemented on a computer farm consisting of ten machines connected via 10~GBit Ethernet lines. Seven machines are dual-CPU with eight cores per CPU (16~hyperthreaded cores per CPU) while three newer machines have 28~cores per CPU (56~hyperthreaded cores). WIT receives blocks of 1344~consecutive hardware triggers ($\sim$23~ms). These blocks overlap by 64 hardware triggers (1.1~ms) and are distributed among nine online machines where they are processed separately in one of the hyperthreaded cores. The remaining one of the ten computers sorts all processed files and assembles ``subruns": 4,000~consecutive blocks gathered into one file ($\sim$90~s of data). All data processing and sorting is handled without disk write operations, fast RAM memory of the WIT computers replaces the usual hard drives.

The trigger criterion of WIT is more complex than the standard software trigger as coincidence is applied to PMT hit time residuals
$\Delta t_i$ with respect to a list of possible vertices $\vec{v}_\alpha$:
$\Delta t_i=t_i-|\vec{v}_\alpha-\vec{p}_i|/c_{\mbox{\tiny water}}-t_0$ with the PMT hit vectors $\vec{p}_i$ and the light emission time $t_0$ and the speed of light in water $c_{water}$.
The applied trigger condition is:
\[
sg=\mbox{Max} \sum_{\mbox{\tiny all PMT hits } i} e^{-\frac{1}{2}\left( \frac{\Delta t_i}{\sigma}\right)^2}>6.6.
\]
The time uncertainty is chosen to be $\sigma=5$~ns. ``Max'' refers to the list of possible vertices. In order to create this list, four hit combinations among a set of selected hit PMTs are chosen, and these four hit time residuals are required to be exactly zero in order to define a possible vertex. To improve the speed of the algorithm, a pre-trigger condition of $>$~dark noise~$+11$ hit coincidence within an absolute time window of 230~ns is applied, and the hits in that window are required to obey relations $\delta x_{ij}>c_{\mbox{\tiny water}} \delta t_{ij}$ here $\delta t_{ij}$ ($\delta x_{ij}$) are the time (spatial) difference between hit $i$ and $j$. The raw PMT times and pulse heights are converted from the digitized counts to calibrated times and photo-electrons in real time.

If a trigger is found, a fast vertex fit to the set of hits used for the construction of four-hit combinations is done. Only events reconstructing further than 1.5~m from any PMT are passed to the online version of the standard SK vertex fitter. For that, a 1.5~$\mu$s window is formed around the trigger time ($500$~ns before and $1000$~ns after). If the event reconstructs at least 2~m from any PMT, and if the number of hits with time residuals between $-6$~ns and $+12~$ns is larger than ten, the event is saved. The trigger efficiency of 2.2~MeV $\gamma$-rays is 13\% averaged over the entire detector and 17\% averaged over the fiducial volume.

The $1.5~\mu$s events are stored in ROOT format output files \cite{root} containing raw PMT time and pulse height digitized counts, calibrated PMT times and pulse heights, the trigger time and position, the reconstructed vertices by both vertex fits, and the the number of PMT hits within 18~ns. Also stored are GPS time stamps and pedestal and calibration measurements. Storing these low-energy events will allow the resolution of the structure of muon-induced hadronic showers and the design of an efficient spallation reduction strategy at SK-IV.