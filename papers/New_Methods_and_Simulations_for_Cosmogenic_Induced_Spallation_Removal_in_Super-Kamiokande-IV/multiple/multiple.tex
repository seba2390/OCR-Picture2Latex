\subsection{Multiple Spallation}
\label{sec:multispa}
Since neutrons in a hadronic shower are a good indicator of
the production of spallation nuclei, then the observation of the decay of a spallation nucleus is likewise an indicator for a hadronic shower. We therefore apply
(in addition to the neutron cloud cut), a preselection removing clusters of isotopes produced by the same muon. Instead of pairing candidate events with possible parent muons, this multiple spallation cut identifies clusters of low energy events observed within a few tens of seconds and a few meters of each other. Here, we consider a sample composed of all SK-IV events passing the first reduction and quality cuts defined for the solar analysis, as discussed in Sec.~\ref{sec:solaranalysis}. Since we need to take all spallation isotope decays into account we apply neither the old spallation cut nor the pattern likelihood cut, as the latter targets $\mathrm{^{16}N}$ $\beta\gamma$ decays. We find that the optimal cut removes candidates found within $4$~m and $60$~s of any event from this sample. This cut allows to remove $45\%$ of spallation background events with a deadtime of $1.3\%$. The solar angle distribution of the rejected and the remaining events is shown in Fig.\ref{fig:multispa}. The absence of a peak around $\cos\theta_{sun} = 1$ for the rejected events confirms the low deadtime for this cut.  

\begin{figure}
        \includegraphics[width=\linewidth]{./multiple/figures/hmulti.eps} 
    \caption{Comparison of the events removed~(dashed) and remaining~(solid) in SK-IV solar sample using multiple spallation cut above 5.99~MeV . The sample above uses the final sample criteria from \cite{skivsolar}.}
    \label{fig:multispa}
\end{figure}

\iffalse
\begin{figure}
    \centering
    \begin{subfigure}{0.5\textwidth}
    \includegraphics[width=0.5\textwidth{multiple/figures/multicut.png}
    \end{subfigure}
    \begin{subfigure}{0.5\textwidth}
    \includegraphics{multiple/figures/nomulti.png}
    \end{subfigure}
    \caption{Comparison of the events removed~(left) and remaining~(right) in SK-IV solar sample using multiple spallation cut above 5.5~MeV. The sample above uses the final sample criteria from~\cite{skivsolar}.}
    \label{fig:multispa}
\end{figure}
\fi

