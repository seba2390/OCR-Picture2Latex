\section{Spallation Tagging with Hadronic Showers}
\label{sec:neutspall}
This section describes how to identify neutron clouds associated with the hadronic showers induced by cosmic muons, and exploit their space and time correlations with spallation isotopes to reduce spallation backgrounds. Efficient reduction requires not only reliably identifying neutrons but also accurately reconstructing their positions in order to locate the shower and characterize its shape. As discussed in Sec.~\ref{sec:softtrig}, however, neutrons produced in SK-IV can be identified only through their capture signal on hydrogen: a single $2.2$~MeV $\gamma$-ray. Since this signal is below the threshold of standard triggers and the AFT trigger is disabled for OD signals~(all muons except for fully contained atmospheric $\nu_{\mu}$ CC interactions), we use the WIT trigger described in Sec.~\ref{sec:softtrig} to select neutron candidates. In this section, we discuss the relevant observables needed to refine this initial selection and reliably characterize neutron clouds, and present several case studies illustrating the validity of our approach. In Sec.~\ref{sec:simucompare}, the characteristics of the observed neutron cloud will be compared to predictions from the simulation described in Sec.~\ref{sec:simulation}. Neutron identification will additionally be used to build spallation-rich samples and refine the computation of the spallation isotope yields at SK in Sec.~\ref{sec:yields}. Finally, an example of incorporating neutron cloud cuts into a full spallation analysis at SK will be shown in Sec.~\ref{sec:spacuts}.

\subsection{Observables for neutron cloud identification}
\label{sec:vardef}
Locating and characterizing neutron clouds requires applying quality cuts on the neutron candidates selected using WIT. These cuts are necessary to select well-reconstructed neutron capture vertices, as well as to remove events associated with, e.~g.~radioactivity, flashing PMTs, and dark noise fluctuations. To this end, we consider two categories of observables, based either on the neutron candidate hit time and light pattern, or on the space and time correlations between neutron candidates and muons. 

\subsubsection{Neutron candidate hit time and light pattern}
The amount of Cherenkov light deposited by an event in SK, as well as the shape of the associated ring, allows to identify well-reconstructed neutron capture events. Using the tools outlined in Sec.~\ref{sec:eventreco}, cuts are placed on neutron capture candidates based on these variables to increase sample purity. 

\paragraph{Reconstructed vertex:} intrinsic radioactivity (except for radon daughters) events typically occur on the walls of the detector or in the surrounding rock. Hence, their reconstructed vertices will typically lie either outside the ID or near its walls. In what follows we will therefore require the reconstructed vertices of the candidate neutrons to lie inside the ID. Note that for events with energies lower than ${\sim}3.5$~MeV the WIT trigger already requires online reconstructed vertices to lie in the FV. WIT triggers above that energy only need to reconstruct inside the detector.

\paragraph{Reconstructed energy $E_{rec}$:} when a neutron is captured on hydrogen, a single 2.2~MeV $\gamma$ is released. The SK energy reconstruction for this analysis assumes an interaction resulting in an electron. Thus, the reconstructed energy is the equivalent to that of an electron interaction with the same amount of Cherenkov light production. Since the energy is calculated based on the number of effective hit PMTs, we will typically require $E_{rec} < 5-6$~MeV to account for fluctuations in the light yield. 

\paragraph{Reconstruction goodness $g_t$, $g_p$: } these observables have been defined in Sec.~\ref{sec:lowereco} and measure the goodness of the position and direction reconstruction of the neutron candidate. In order to determine suitable spallation cuts, 2.2~MeV $\gamma$-rays were generated using SKDetSim. The events were simulated without dark noise, overlaid with real SK online raw data, and then processed with WIT. The resulting sample was then split into ``good'' and ``bad'' events, whose reconstructed vertices lie less and more than $5$~m away from the true vertices respectively. Figure~\ref{fig:mcovaq} shows the two-dimensional distributions of $g_t$ and $g_p$ for both event categories, with ``good'' events being associated with higher goodness values. In the simple case studies presented in this section, we will only require $g_t > 0.5$. In the more refined strategy discussed in Sec.~\ref{sec:spacuts} we will impose cuts on both $g_t$ and $g_p$ in order to improve the neutron cloud localization. The resulting refined cut is overlaid on the goodness distributions shown in Fig.~\ref{fig:mcovaq}.
 
 \begin{figure*}
    \centering
    \includegraphics[width=\textwidth]{./neutrons_wit/figures/mcovaq.eps}
    \caption{MC generated 2.2~MeV $\gamma$-rays processed by WIT software. The left distribution shows ``poorly" reconstructed events as defined as reconstructed vertex being more then 5~m from MC truth and the right distribution shows the ``good" reconstructions. The lines separate the different weight regions used to parameterize the neutron cloud cuts described in Sec.~\ref{sec:cloudcut}.}
    \label{fig:mcovaq}
\end{figure*}
 
\subsubsection{Correlations with muons}
\label{sec:corrmu}
Neutrons produced in muon-induced showers will be found close in time and space to muon tracks. The observables we describe here allow to identify these space and time correlations. Moving forward we make the reasonable assumption that muons follow a straight path inside the detector. We then parameterize their tracks using an entry point and a direction, that are determined using the fitter described in Sec.~\ref{sec:muons}. We then define the following observables: the time difference $\Delta t$ between a neutron candidate and a muon, and the transverse and longitudinal (with respect to the muon entry point) distances from the candidate to the muon track, $l_t$ and $l_{\mbox{\tiny \it LONG}}$.

\paragraph{Time difference $\Delta t$:} the characteristic neutron capture time in pure water is $\tau_{cap}{\sim}205~\mu$s. We hence expect to find most neutrons within about $500~\mu$s from their parent muon crossing time. Since the muon rate in SK is about $2$~Hz, this time window alone allows to unambiguously link a neutron capture event to its parent muon. Additionally, for high-quality neutron cloud samples, we will require $\Delta t$ to be larger than 20~$\mu$s to account for PMT afterpulsing \footnote{Muons creating hadronic showers also deposit a significant amount of light within the detector, and ionization of residual air within the PMT creates a delayed pulse after drifting to the principal dynode. Although there is a relatively small chance for a PMT to experience this, bright muons have hits at most, if not all PMTs, within the detector and these afterpulsings cause problems. The extra hits create false/poorly reconstructed events in the 10--20~$\mu$s time range and this time region is therefore avoided.} The afterpulsing features are shown in Fig.~\ref{fig:afterpulsing} for events found within 5~m of a muon track.

\paragraph{Transverse and longitudinal distances, $l_t$ and $l_{\mbox{\tiny \it LONG}}$: }  $l_t$ defines the distance between a neutron candidate and the closest point on the muon track while $l_{\mbox{\tiny LONG}}$ is the longitudinal distance, the distance between that closest point on the muon track and the muon track entry point. The definitions of these two observables are shown in Fig.~\ref{fig:vardia}. Their two-dimensional distribution, shown in Fig.~\ref{fig:ltvln}, allows to characterize the shapes and sizes of the neutron clouds. In this figure, $\langle l_{\mbox{\tiny LONG}}\rangle$ refers to the average $l_{\mbox{\tiny LONG}}$ of all the neutrons in the cloud. Note that neutron clouds have an elongated shape along the muon track and can extend up to about 5~m. Here, we showed $l_t^2$ instead of $l_t$ to reflect the amount of phase space available.



\begin{figure}
    \centering
    \includegraphics[width=\linewidth]{./neutrons_wit/figures/dtshort.eps}
    \caption{Time distribution of WIT-triggered events with $l_t < 5$~m. The goodness cuts included in the WIT trigger lead to a deficit of events in the 5--10~$\mu$s time range while the bumps are contributions from fake neutrons associated with PMT afterpulsing. In order to remove these fake neutrons, quality events are required to have $\Delta t > 20~\mu$s.}
    \label{fig:afterpulsing}
\end{figure}

\begin{figure}
    \centering
    \includegraphics[width=0.57\linewidth]{neutrons_wit/figures/vardiagram3.eps} %LLMM path expansion needed for arXiv
    \caption{Diagram showing the $l_t$ and $l_{\mbox{\tiny LONG}}$ observables associated with neutron identification, as described in Sec.~\ref{sec:corrmu}.}
    \label{fig:vardia}
\end{figure}
\begin{figure}
    \centering
    \includegraphics[width=\linewidth]{neutrons_wit/figures/ltvsln.eps}    %LLMM path expansion needed for arXiv
    \caption{Data Vertex correlation of neutron capture events. The vertical axis shows the squared distance to the muon track, the vertical axis the distance along the muon track with respect to the average. Solid contours indicate levels of multiple 1,000 events/bin, the thick contours are for 5,000, 10,000, 15,000, and 20,000. The levels of the dotted line contours are marked in the Figure.}
    \label{fig:ltvln}
\end{figure}

Incorporating the observables defined above in spallation analyses thus allow to identify neutrons and accurately locate neutron clouds and the showers that generated them. In what follows we will show case studies highlighting the validity of our neutron identification and localization procedure. 

\subsection{Neutron cloud identification: case studies}
\label{sec:hadronic_data}
Here, we present examples of how to identify neutrons and reconstruct clouds using the observables defined above. We first demonstrate our ability to identify individual neutrons by associating the WIT trigger with simple goodness and position cuts. Then we use well-reconstructed neutron candidates to build neutron clouds, and exploit their spatial correlation with spallation isotopes.

\subsubsection{Identifying individual neutrons}
\label{subsec:hadronic_data}
 We build a high-purity neutron sample using WIT-triggered events verifying $g_t > 0.5$, $l_t < 1.5$~m, and $E_{rec} <  5$~MeV. We estimate the number of neutrons in this sample by comparing the distribution of time differences between neutron candidates and their parent muons, $\Delta t$, to the one expected from calibration studies using an americium beryllium (AmBe) source~\cite{Super-Kamiokande:2008mmn}. This $\Delta t$ distribution is shown in Fig.~\ref{fig:neutdt} and was fitted from 50~$\mu$s to 500~$\mu$s by the following equation
\begin{equation}
    N(\Delta t) = A\cdot e^{-\Delta t/\tau} + C
    \label{eq:expfit}
\end{equation}
where $N$ is the number of events, $\tau$ is the exponential decay time constant, and the constant {\it C} absorbs remaining background contributions. For the WIT data the time constant was measured to be $\tau = 211.8 \pm 1.7~\mu$s. In comparison, in AmBe calibration studies, the neutron capture time on hydrogen was measured to be $\tau = 203.7 \pm 2.8~\mu$s. This results in about a 2.5$\sigma$ difference between the two measurements. This discrepancy is believed to be due to missing neutrons within higher multiplicity showers from pile up. WIT defines a 1.5~$\mu$s window around a triggered event (500~ns before and 1,000~ns after the trigger time). If another neutron capture happens within that window, no new trigger will be issued. This introduces a form of deadtime~\footnote{We call the signal loss of the spallation cut due to accidental coincidence with a muon ``deadtime''.} for captures close together in time, biasing for a longer capture time constant.

\begin{figure}
    \centering
    \includegraphics[width=\linewidth]{./neutrons_wit/figures/dtgood5.eps}
    \caption{$\Delta t$ (black dots) and resulting fit (grey solid line) using the function from Eq.~(\ref{eq:expfit}), for neutron events detected after muons by WIT. The time constant obtained from the fit is $\tau = 211.8 \pm 1.7~\mu$s.}
    \label{fig:neutdt}
\end{figure}

\subsubsection{Identifying neutron clouds}
\label{sec:hadronic_spall}
After having successfully identified neutrons using WIT triggers, we will now investigate correlations between neutron clouds and spallation isotopes. Here, we define neutron clouds as groups of two or more WIT events observed within $500~\mu$s after a muon. Additionally, we require at least one of these WIT events to have $\Delta t > 20~\mu$s, $g_t > 0.5$, and $E < 5$~MeV. Note that fewer events reconstruct outside the ID (and even the fiducial volume) due to the WIT trigger conditions on the online event reconstruction.
%Note that for this energy range, the WIT trigger condition ensures that this event is reconstructed within the ID. 
As detailed at the beginning of this section, these cuts allow to discriminate neutrons against noise fluctuations, radioactivity, and afterpulsing events.

If a cloud was found using the conditions listed above, spallation candidates were searched for in close proximity of the center of a neutron cloud. Spallation candidate events were preselected using the noise reduction and quality cuts described in Sec.~\ref{sec:solaranalysis} and in Ref.~\cite{skivsolar} for the SK-IV solar neutrino analysis ---without the spallation and pattern likelihood cuts. Spallation candidates found within less than 5~m and 60~s \emph{after} an observed neutron cloud are selected. 
60~s was chosen to contain more than 99\% of the $^{16}$N decays and the 5~m value reflects the general size of muon-induced showers, as seen in Fig.~\ref{fig:ltvln}, %accounts for the vertex resolution of neutron clouds. 
In addition to this signal sample, we build a background sample using candidates found within 5~m of and up to 60~s \emph{before} neutron clouds. This background sample allows to estimate the fraction of spurious pairings between spallation candidates and uncorrelated neutron clouds in the signal sample, and subtract off the corresponding effects.

A strong spatial correlation between spallation candidate events and the centers of neutron clouds was found. As shown in Fig.~\ref{fig:spallvertex}, this correlation increases with the multiplicity (the number of WIT triggered events) of the neutron cloud. The slow decrease of the number of spallation candidates relative to the others at low multiplicities is due to accidental pairings between candidates and neutron clouds. Conversely, high-multiplicity neutron clouds can be more easily located and allow to reliably identify spallation products. These high-multiplicity clouds are also likely to be associated with multiple isotopes in large hadronic showers, as can be observed from Fig.~\ref{fig:neutvspall}. 

\begin{figure}
    \centering
    \includegraphics[width=\linewidth]{neutrons_wit/figures/vertexcorrelation.eps}    %LLMM path expansion needed for arXiv
    \caption{Signal distribution for vertex correlation by neutron cloud multiplicity. The figure is normalized by the number of spallation events for each multiplicity (Total events in signal $-$ background distribution). As multiplicity increases, the steepness in the tail of the distribution increases as there is more accuracy in parameterization of the cloud and more spallation is expected in larger hadronic showers. Note that the vertical axis is on a logarithmic scale.}
    \label{fig:spallvertex}
\end{figure}


\begin{figure}
    \centering
    \includegraphics[width=\linewidth]{./neutrons_wit/figures/spallvneut.eps}    %LLMM path expansion needed for arXiv 
    \caption{Two-dimensional distribution showing the correlation between neutron capture candidate multiplicity and spallation candidate multiplicity. Note the color scale is on a logarithmic scale.}
    \label{fig:neutvspall}
\end{figure}

Identifying neutron clouds and correlating them with spallation candidates using the criteria outlined above allows to remove $55\%$ of spallation events with a little more than $4\%$ deadtime. Due to the low detection efficiency for the 2.2~MeV $\gamma$s resulting from neutron capture on hydrogen, small showers are missed, and therefore this procedure alone does not suffice to remove spallation in an SK analysis. In Sec.~\ref{sec:spacuts} we will show how to optimize neutron cloud reconstruction and associate it to traditional spallation cuts for the SK solar analysis. There, in addition to the observables used for this case study, we will notably use the directional goodness $g_p$ to better identify well-reconstructed neutrons, and will account for the elongated shape of neutron clouds when investigating spatial correlations with isotope vertices. 

Although the neutron cloud cuts presented here cannot be used as a standalone spallation reduction technique in SK-IV, their impact is expected to drastically increase in the new SK-Gd phase of the detector. Estimating this new impact and redesigning spallation cuts accordingly requires extensive simulation studies. In the next section, we compare neutron cloud measurements in pure water to predictions from the simulation described in Sec.~\ref{sec:simulation}, and show that this simulation can be reliably used to optimize future neutron cloud cuts.