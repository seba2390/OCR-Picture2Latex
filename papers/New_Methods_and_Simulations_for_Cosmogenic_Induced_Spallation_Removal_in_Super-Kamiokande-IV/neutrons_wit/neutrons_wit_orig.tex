\section{Spallation Tagging with Hadronic Showers}
\label{sec:neutspall}
The following section goes over the process for successfully directly tagging hadronic showers by identifying the capture of 2.2 MeV gammas from the interactions as well as tagging subsequent spallation. This process was performed in three steps, beginning with a search for neutrons immediately after muons passing trough SK (Sec.\ref{sec:hadronic_data}). Next, with neutron clouds identified, a search for spallation in nearby proximity was done (Sec.\ref{sec:hadronic_spall}). Finally, a cut was tuned from the information gathered in the first two steps as well as from MC data to be used for the solar neutrino analysis (Sec.\ref{sec:cloudcut}). 

\subsection{Variable Definition}
\label{sec:vardef}
Before progressing further, major variables used throughout for neutron capture and spallation candidates are defined as follows, with some shown in Fig~\ref{fig:vardia}:
\begin{itemize}
    \item Muon Track: The reconstructed path of the muon passing through the detector. Muons pass through the detector with negligible change in direction so tracks are parameterized by an entry point and direction. 
    \item Time Difference ($\Delta t$): Time difference between two events, generally taken as $t_{cand} - t_\mu$.(Fig.~\ref{fig:neutdt})
    \item Transverse Distance ($lt$): The distance of closest approach of a candidate event to a muon. Typically shown as $lt^2$ to flatten the phase space of the variable. 
    \item Longitudinal Distance ($l_{LONG}$): Parameter used for distances along the muon track of both neutron captures and spallation candidates. $l_{LONG}$ is defined as the distance from the muon entry point to the candidate point of closest approach to the muon track.
    \item Reconstructed Energy: when a neutron is captured on H, a single 2.2 MeV $\gamma$ is released. The SK energy reconstruction for this analysis assumes an interaction resulting in an electron. Thus, the reconstructed energy is the equivalent to that of an electron interaction with the same amount of Cherenkov light production.
    
\end{itemize}

\begin{figure}
    \centering
    \includegraphics[width=0.8\linewidth]{spallpaper/neutrons_wit/figures/vardiagram2.png}
    \caption{Diagram of common variables}
    \label{fig:vardia}
\end{figure}

\subsection{Search for Neutron Captures from Hadronic Showers}
\label{sec:hadronic_data}
In order to look for the expected neutron captures WIT was used to search for events immediately after a muon. As described in Sec.~\ref{sec:softtrig}, the AFT trigger is disabled for the standard triggering scheme after muons so data was unavailable from there. When looking for whether or not these events were neutrons a high purity cut was placed to reduce minimize background noise. The initial sample of potential neutrons, events were required to have $g_{VT}$ greater than 0.5, $lt$ less than 1.5m, and a reconstructed energy of less than 5 MeV. 

To conclude whether these events were indeed neutrons, capture times for the events after muons to previously AmBe calibration source studies.  The $\Delta t$ between WIT triggered events with the above cuts to preceding muon is shown in Fig.~\ref{fig:neutdt} and was fitted from 50 $\mu$s to 500 $\mu$s by the following equation
\begin{equation}
    N(\Delta t) = A\cdot e^{-\Delta t/\tau} + C
\end{equation}
with {\itN} being the rate of events, $\tau$ is the exponential decay time constant, and {\it C} is a constant to remove background contributions. For the WIT data the time constant was measured to be $\tau = 211.8 \pm 1.7 \mu$s. In comparison, the AmBe data measured the capture time for neutrons on H to be $\tau = 203.7 \pm 2.8\mu$s. This results in about a $2.5\sigma$ difference between the two measurements, and the difference in values are believed to originate in the difference in their energy, nature of the neutron generation, and statistics.

\begin{figure}
    \centering
    \includegraphics[width=\linewidth]{spallpaper/neutrons_wit/figures/dtgood5.png}
    \caption{$\Delta t$ and resulting fit for events seen in WIT after muons.}
    \label{fig:neutdt}
\end{figure}

\subsection{Identifying Spallation from Tagged Neutrons}
\label{sec:hadronic_spall}
With the successful identification of neutrons from muon induced hadronic showers, the next step was to look for the subsequent spallation decays generated by these showers. For this section, a neutron cloud's location was defined from ``Quality" neutron capture candidates. The criteria for these quality events are defined as events with 20 $\mu\mbox{s}< \Delta t < 500 \mu$s, $g_{VT} > 0.5$, reconstructed energy \textless $6$~MeV, a reconstructed vertex within the ID, and $lt < 5m$. Also, two or more event triggers and at least one event meeting these listed conditions were required to be classified as a neutron cloud for this exploratory search. 

A few notes on the choice for these cuts is outlined here. Although the full ID is allowed, the trigger conditions in this energy range have a FV cut placed on the events. The $\Delta t$ has a lower bound of 20 $\mu$s to account for PMT afterpulsing. Muons creating hadronic showers also deposit a significant amount of light within the detector, and ionization of residual air within the PMT can create a delayed pulse after drifting to the principal dynode. Although there is a relatively small chance for a PMT to experience this, bright muons cause hits at most, if not all PMTs, within the detector and these afterpulsings cause problems. The extra hits create false/poorly reconstructed events in the 10-20 $\mu$s time range and is therefore the time region is avoided. These features are shown in Fig.~\ref{fig:afterpulsing} for events with only an $lt$ cut of 5m.
\begin{figure}
    \centering
    \includegraphics[width=\linewidth]{spallpaper/neutrons_wit/figures/dtshort.png}
    \caption{Afterpulsing pile up seen when only a 5m $lt$ cut is placed on the data. Goodness cuts create a deficit of events in the 10-20 $\mu$s time range. Quality events are taken to have a $\Delta t > 20 \mu$s to avoid this on other issues.}
    \label{fig:afterpulsing}
\end{figure}

The maximum allowed energy is significantly greater than $2.2$~MeV to account for fluctuations in the light yield measured by the effective number of hit PMTs. The energy is calculated based on the number of effective hit PMTs. Neutron showers were found to be mostly contained within 4-5 m in transverse and longitudinal distances. \textless$l_{LONG}$\textgreater here is taken in reference to the average position along the track of all quality captures. Fig.~\ref{fig:ltvln} shows the relationship of $lt^2$ and $ln$ for the neutron clouds. 

\begin{figure}
    \centering
    \includegraphics[width=\linewidth]{spallpaper/neutrons_wit/figures/lt2vln_updated.png}
    \caption{Caption}
    \label{fig:ltvln}
\end{figure}

If a cloud was found using the listed conditions from above, spallation candidates were searched for in close proximity of the center of a neutron cloud. Spallation candidate events were selected using the cut criteria for the final solar sample outlined in \ref{sec:solar} and \cite{skivpaper}, with the removal of the spallation cut and pattern likelihood cut. Spallation candidates were required to be within within 5m and 60s of the observed neutron cloud. 60s was chosen to contain nearly all $^{16}$N decays and 5m chosen to account for vertex resolution of neutron clouds as well as being the general size of showers as seen in \ref{fig:ltvln}. A background sample was also created with an inverse $\Delta t$ correlation performing the same search in the 60s window prior to the muon. The resulting distributions are subtracted from the signal (normal time ordering) distributions where applicable for a pure signal distribution. 

Strong correlation between spallation candidate events and neutron cloud center was found. Splitting the sample up by multiplicity (number of triggered events in WIT) displays a trend of better spallation tracking as the multiplicity of the neutron cloud increases. The flat tail in the lower multiplicities is a result of accidentally tagging spallation along the track regardless of the neutron fitting. Decay times for the tagged events match the decay times of the dominant isotopes. For reference, fitting an exponential from 10-45s looking for $^{16}N$ was $1.4\sigma$ agreement of expectation at $9.9\pm.24$s. 

\subsection{Neutron Cloud Spallation Cut for Solar Analysis}
\label{sec:cloudcut}
Using the cut criteria from before, with a minimum of two events seen in WIT, ${\sim}55\%$ of spallation is removed with a little more than $4\%$ deadtime.\footnote{We call the signal loss of the spallation cut due to accidental coincidence with a muon ``deadtime".} Due to the low tagging efficiency to see 2.2 MeV $\gamma$s from n capture on H, small showers are missed and therefore this cannot be used as a standalone cut for spallation removal in the solar analysis. The cut was tuned from the initial search outlined above to maintain as much spallation removal efficiency while minimizing deadtime. 

First to improve the event reconstruction quality criteria, $2.2$~MeV $\gamma$s were generated in the SK Monte Carlo detector simulation (SKDETSIM,\cite{skdetsim}) and processed by the WIT software. A comparison between ``good" and ``bad" events was made, with ``good" events being defined with a reconstructed vertex within 5m of the true vertex, and ``bad" events with a reconstructed vertex more than 5m away. The event direction goodness based on azimuthal symmetry of the Cherenkov ring, $g_{AD}$ was examined in addition to $g_{VT}$ to check event quality. A cut formed based on these distributions, with the phase space split into three regions of weights of 0, 1, and 2 used for calculating the center of the neutron cloud. Fig~\ref{fig:mcovaq} shows the resulting distributions, with the regions outlined.

\begin{figure}
    \centering
    \includegraphics[width=\linewidth]{spallpaper/neutrons_wit/figures/mcovaq_compare.png}
    \caption{MC generated 2.2 MeV $\gamma$s processed by WIT software. The left distribution shows ``poorly" reconstructed events as defined as reconstructed vertex being more then 5m from MC truth and the right distribution shows the ``good" reconstructions. The lines separate the different weight regions used to parameterize the neutron cloud.}
    \label{fig:mcovaq}
\end{figure}

Next, parameterization of the shower was decided. Three potential methods were compared using cylindrical coordinates with varying origin and $z$-axis direction, with the weighted events used for the parameterization. The first option used the cloud shape to make a coordinate system with origin at the cloud center and the overall shape of the shower to align the $z$-axis with the principal axis of the shower. The lack of sufficient neutrons to reliably characterize the shower's multiplicity prevented this approach. The second option used the center of the neutron cloud as the origin, and maintained the muon direction as the shower direction, and even this method would need higher neutron multiplicity to work reliably. The third projected the shower center to the muon track and maintained the muon direction as the principal axis. Clouds were also separated by multiplicity, with multiplicity based on the number of events low energy events found in WIT with only the 500 $\mu$s and 5m to the muon track requirement. Multiplicity bins used were 2, 3, 4-5, 6-9, and 10+ events. 

The parent muon track fit is more reliable than the fit on few neutron capture events, while at large multiplicity events showed minimal difference in effectiveness. Fig~\ref{fig:neutmuoncomp} shows the lowest multiplicity and highest multiplicity bin for the two methods. Therefore, the third parameterization was chosen. The asymmetry of the shape of the clouds was also seen to be minimal, so an ellipsoidal cut was implemented for the different cloud multiplicities cutting all events within the ellipsoid for 60s. Due to the unreliability of the lowest multiplicity bin, only 30s of events are cut within its ellipsoid. Additionally, since minimal deadtime is accrued for very short times, a $7.5$~m and 5m spherical cut is placed for 0.2 and 2 seconds respectively. This reduces the deadtime from the previous cut to $1.3\%$ deadtime while maintaining $53\%$ spallation removal effectiveness. The full parameterization of the cloud cut can be found in Table~\ref{tab:cloudtab}. 

\begin{figure}
    \centering
    \includegraphics[width=\linewidth]{spallpaper/neutrons_wit/figures/compare_types_updated.png}
    \caption{Tagged spallation for all neutron cloud multiplicities using a neutron cloud centered (left) and muon track centered (right) orientation. Greater uncertainty in lower neutron multiplicity reconstruction compared to muon track reconstruction drives the difference in ability to accurately tag spallation.}
    \label{fig:neutmuoncomp}
\end{figure}

\begin{table}[t]
    \centering
    \begin{tabular}{ | *{14}{c|} }
    \hline Multiplicity  & 2+ & 2+ & 2 & 3 & 4-5 & 6-9 & 10+ \\
    \% of Showers & 100 & 100 & 73 & 15 & 6 & 3 & 3  \\\hline
    $\Delta t$  & 0.2 & 2 & 30 & 60 & 60 & 60 & 60 \\\hline
    $l_{LONG}$ [cm] & 750 & 500 & 350 & 500 & 550 & 650 & 700  \\\hline
    %$lt^2$ [cm2] & 40000 & 60000 & 120000 & 200000 & 250000 & 562500 & 250000 \\\hline
    $lt$ [cm] & 750 & 500 & 200 & 245 & 346 & 447 & 500 \\\hline
    \end{tabular}
    \caption{Table showing the different cut conditions for the cloud cut. + is used to represent showers of at least that multiplicity.}
    \label{tab:cloudtab}
\end{table}

The deadtime estimation for all of the spallation cuts is performed by taking a sample of uncorrelated events. Events with a reconstruction energy between 3.5-5~MeV are taken and their vertices within the detector replaced by randomly generated ones, while maintaining their event timing. Using an inverted time cut~(event time is before muon) the various spallation cuts are performed and a comparison between cut events and total events is made. The quoted deadtimes are the ratio of cut events to total events for the FV. 