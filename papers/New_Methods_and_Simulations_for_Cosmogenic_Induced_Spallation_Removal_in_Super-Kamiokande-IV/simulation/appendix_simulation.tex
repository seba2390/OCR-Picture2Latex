\section{Simulation settings}
\label{appendix_sim}
In this appendix  the main settings chosen to build the FLUKA simulation are described in detail. 
FLUKA code fully integrates the most relevant physics models and libraries; it is not possible for the user to modify or adjust them according to their needs. Several default settings are available and must be chosen at the beginning of the simulation depending on the general physics problem the user is dealing with. In addition to this, FLUKA offers several options to customize the default settings enabling or disabling a certain type of processes or changing the treatment of specific type of interactions. For this work, FLUKA simulation was built with the default setting PRECISIO(n)~\cite{fluka_manual}. All the specifics that are particularly important for the scope of this paper are summarized below.

Low-energy neutron, which are defined to have less than 20~MeV energy, are transported down to thermal energies.

The absorption is fully analogue for low energy neutrons: in a fully analogue run, each interaction is simulated by sampling each exclusive reaction channel with its actual physical probability, this allows for event-by-event analysis. In general, this is not always the case, especially concerning low-energy neutron interactions.

Muon photonuclear interactions are activated with explicit generation of secondaries.

Several options are used to complement the default setting.

PHOTONUC option: photon and electron interactions with nuclei are activated at all energies.
 
MUPHOTON option: controls the full simulation of muon nuclear interactions at all energies and the production of secondary hadrons.

EVAPORAT(ion) and COALESCE(nce) options: these two are set to give a more detailed treatment of nuclear de-excitations. Despite the related large CPU penalty, it is fundamental to activate these options when isotope production want to be studied. EVAPORAT enables the production of heavy nuclear fragments ($A>1$) while COALESCE sets the emission of energetic light-fragments. 

IONSPLIT option: used for activating ion splitting into nucleons.

IONTRANS option: full transport of all light and heavy ions and activation of nuclear interactions.

RADDECAY option: activate radioactive decay calculations.

Settings are specified in the main input file: FLUKA, unlike other Monte Carlo particle transport codes, is built to get the basic running conditions from a single standard code. However, due to the complexity of spallation mechanism, standard options do not satisfy to retrieve the problem-specific informations we need to score: customized input and output routines (SOURCE and MGDRAW) are required to be written in order to incorporate non-standard primary particle distributions, the ones calculated with MUSIC simulation, and to extract event-by-event informations for the shower reconstruction. In particular, only primary muons inducing the production of at least one hadron or of an isotope are selected and recorded; the rest are not interesting for this study and are discarded to save computational time.