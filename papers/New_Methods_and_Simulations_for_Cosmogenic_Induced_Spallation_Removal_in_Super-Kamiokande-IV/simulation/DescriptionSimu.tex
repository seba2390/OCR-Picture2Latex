\section{Spallation simulation}
\label{sec:simulation}

\subsection{Cosmic muon simulation}
In order to understand and optimize spallation event removal techniques we simulate the interactions of cosmic muons and the subsequent production of neutrons and isotopes in the SK water. %In order to build a framework that can be used for future spallation analyses we also simulate the interactions of cosmic muons and the subsequent production of isotopes and neutrons inside the SK tank.
This muon simulation is composed of five parts. We first model the muon flux at the surface of the Earth using a modified Gaisser parameterization described in Ref. \cite{muonsimulation}, and propagate muons through the rock to SK using a dedicated transport simulation code.
%Then, we propagate it through the rock to SK using MUSIC (MUon SImulation Code)~\cite{music,music2,music3}, a three-dimensional simulation code for muon transport through matter of large thickness, especially designed for underground physics experiments. 
Second, we simulate the production of hadronic showers and radioactive isotopes inside SK using FLUKA \cite{fluka_manual, fluka_paper}. The FLUKA results are then injected into SKDetSim, the official GEANT-3 \cite{geant3} based detector simulation for SK, that will model detector effects as well as minimum ionization around the muon track. Finally, we reconstruct muon tracks, neutron captures, and isotope decays using standard SK reconstruction software as well as the procedure described in Sec.~\ref{sec:neutspall}. The simulation pipeline is summarized in Fig.~\ref{fig:simupipeline}.
\begin{figure}
    \centering
    \fbox{\begin{minipage}{4cm}
    Muon flux at surface\\
    {\bf Modified Gaisser}
    \end{minipage}}\\
    $\Downarrow$\\
    \fbox{\begin{minipage}{4cm}
    Muon travel through rock\\
    {\bf MUSIC}
    \end{minipage}}\\
    $\Downarrow$\\
    \fbox{\begin{minipage}{4cm}
    Hadronic shower and isotope generation\\
    {\bf FLUKA}
    \end{minipage}}\\
    $\Downarrow$\\
    \fbox{\begin{minipage}{4cm}
    Detector simulation\\
    {\bf SKDetSim--GEANT3}
    \end{minipage}}\\
    $\Downarrow$\\
    \fbox{\begin{minipage}{4cm}
    Event reconstruction
    \end{minipage}}\\
    %\includegraphics[width=4cm]{spallpaper/simulation/figures/simu.png}
    \caption{Simulation steps, from the modeling of the muon flux at the surface to event reconstruction in SK.}
    \label{fig:simupipeline}
\end{figure}
\subsubsection{Muon generation and travel}
Most of the muons reaching the Earth's surface are produced at an altitude of around $15$~km, from the interactions of primary cosmic rays in the atmosphere, principally the decays of charged mesons~\cite{gaisser_book}. The shape of the meson production energy and angular distributions reflects a convolution of the production spectra, the energy loss, and the decay probability in the atmosphere. In this study we model the muon flux at the surface using a modified Gaisser parameterization, that has been optimized for detectors at shallow depth such as SK~\cite{muonsimulation}. While muons contributing most to spallation background can cross the detector either alone or as part of muon ``bundles'' --caused by meson decays within cosmic ray showers in the atmosphere-- differences in spallation observables between these two configurations will be entirely due to track reconstruction issues. In this paper, our primary goal is to evaluate FLUKA's ability to model muon-induced showers and isotope production in SK. We will therefore consider only single muons and will leave the subject of bundles to a future study.
%and neglect effects associated to muon bundles. Indeed, while muon bundles sizably contribute to the total spallation rate, in this paper we will use the simulation to predict the relative rates of spallation isotopes as well as the makeup of muon-induced showers; these quantities are not significantly affected by the muon multiplicity.

In order to obtain the muon flux entering the SK detector we now need to propagate muons through the rock surrounding the detector. We simulate muon propagation using the MUSIC \cite{music,music2,music3} propagation code. MUSIC integrates models for all the different types of muon interactions with matter leading to energy losses and deflections, such as pair production, bremsstrahlung, ionization and muon-nucleus inelastic scattering. Angular and lateral displacements due to multiple scattering are also taken into account. Muons are transported with energies up to 10$^7$ GeV. Here, we used the rock composition model described in~\cite{muonsimulation}, with an average density of $\rho = 2.70~\mathrm{g\,cm^{-3}}$. We compute the muon travel distance within the rock as a function of the incident angle using a topological map of the SK area from 1997~\cite{muonsimuKAMLAND,geoinstitute}.

Figure~\ref{fig:musi_final_distrib} shows the energy spectrum of the muons that reach SK. The rock above the detector constitutes a particularly efficient shield, effectively blocking muons with energies lower than $600$~GeV. The cosmic muon flux is reduced from $6.5\times10^{5}~\mu \mathrm{m^{-2}\,h^{-1}}$~\cite{borexino} to $1.54 \times 10^{-7}~\mathrm{cm^{-2}\,s^{-1}}$ which corresponds to a muon rate of 1.87~Hz, as expected from previous measurements~\cite{muonrate1, muonrate2, muonrate3}.
\begin{figure}[tbph]
	\centering
	\includegraphics[width=\linewidth]{simulation/figures/spectra}
	\caption{Muon energy spectrum at the location of SK detector in the mine inside Mt. Ikenoyama.}
	\label{fig:musi_final_distrib}
\end{figure}

Although MUSIC allows to evaluate the effect of muon transport through rock on the muon directional distribution, it does not account for the detector's cylindrical geometry. We account for these effects by assuming that cosmic muons are isotropically produced in the atmosphere and that all muons produced in the same area of the sky have quasi-parallel trajectories inside SK. For each muon generated by MUSIC with a given direction $(\theta,\phi)$, with origin of the coordinates at the center of SK, $\theta$ representing the zenith angle and $\phi$ the azimuthal one set to zero when the final muon travels from east to west, we generate a set of parallel tracks with uniformly distributed intersection points in the plane perpendicular to the $(\theta,\phi)$ vector, as shown in Fig.~\ref{fig:muentrypoints}. We then reject all the tracks that do not cross the detector, thus straightforwardly accounting for geometrical effects. This procedure allows us to convert the directions generated by MUSIC into a sample of entry points distributed on the surface of the SK inner detector. 
\begin{figure}
    \centering
    \includegraphics[width=\linewidth]{simulation/figures/samplingmethod_2.eps}  % LLMM path expansion needed for arXiv
    \caption{Spatial distribution of trajectories for muons produced in the same area of the sky. These muons can be considered almost parallel when reaching SK and the intersection of their trajectories with a plane perpendicular to their direction will be uniformly distributed. Here, $R$ and $H$ are the radius and the total height of SK's inner detector while $\theta$ and $\phi$ define the direction of the muons. Here, the stars indicate the muon entry points and the crosses indicate the intersections of the muon trajectories with the plane.}
    \label{fig:muentrypoints}
\end{figure}
\subsubsection{Muon interactions in water}
Propagation and interactions of muons in water are simulated with FLUKA, taking as input MUSIC energy and angular distributions. FLUKA \cite{fluka_manual, fluka_paper} is a general purpose Monte Carlo code for the description of interactions and transport of particles in matter. It simulates hadrons, ions, and electromagnetic particles, from few keV to cosmic ray energies. It is built and frequently upgraded with the aim of maintaining implementations and improvements of modern physical models. FLUKA version 2011.2x.7 is used for this work, together with FLAIR (version 2.3-0), an advanced user interface to facilitate the editing of FLUKA input files, execution of the code and visualization of the output files \cite{flair}. FLUKA propagates muons into the SK detector, simulating all the relevant physics processes that lead to energy losses and creation of secondary particles: ionization and bremsstrahlung, gamma-ray pair production, Compton scattering and muon photonuclear interactions.  Hadronic processes such as pion production and interactions, low energy neutron interactions with nuclei and photo-desintegration are also modeled.

FLUKA code fully integrates the most relevant physics models and libraries. For this work, the simulation was built with the default setting PRECISIO(n). All the specifics related to this setting can be found in \cite{fluka_manual}.  More detail about the models and settings used in this paper can be found in appendix~\ref{appendix_sim}. In particular, low-energy neutrons, which are defined to have less than 20~MeV energy, are transported down to thermal energies, a setting that is critical for our study. 

Crucial options complement the default setting: EVAPORAT(ion) and COALESCE(nce) give a detailed treatment of nuclear de-excitations while nucleus-nucleus interactions are enabled for all energies via the option IONTRANS.
\vspace{.3cm}\\
The SK detector is modeled as a cylindric volume of pure water, as described in Sec.~\ref{sec:SKdetector}. The PMT structure is not simulated in FLUKA, given that it is fully incorporated in SKDetSim. Since muons can induce showers outside the water tank and secondary products may reach the active part of the detector, previous studies~\cite{BLi_1} examined the effect of a 2~m thickness of rock surrounding the OD:  it was proven that this has a minor effect on the results. Thus the rock,  as well as the tank and the support structure, are not simulated in this work. Both negative and positive muons are generated assuming a muon charge ratio, defined as the number of positive over negative charged muons, of $N_{\mu^+}/N_{\mu^-}$ = 1.27 \cite{CMSchargeratio}. Note that the measured values of the charge ratio at SK depth  can vary by about 20\%, with the highest value (1.37$\pm$0.06) measured at Kamiokande~\cite{Yamada:1991aq}. However, since the isotope yield depends only weakly on the muon charge, these variations have a negligible impact on this analysis, with an effect on the predicted yields of less than 1\%.

\subsubsection{Detector response and event reconstruction}


We model the detector response using the official detector simulation for Super-Kamiokande, referred to as SKDetSim. This simulation is based on GEANT 3.21~\cite{geant3} for detector modeling and uses a customized model for light propagation. It covers all aspects of event detection, from the initial interaction to the light collection on the PMTs and event reconstruction.

For this purpose, SKDetSim models in detail the entire geometry of the detector, the particle propagation in water, the emission of Cherenkov photons, reflection and absorption of light on materials, photo-electron production, and electronic response. Simulated data and real data are processed similarly. For this reason, particles simulated in SKDetSim are recorded in trigger windows corresponding to the ones applied to data. For each event, the detector dark rate is also simulated and can be added or not to the outputs depending on the specific needs.


Since FLUKA already models the shower development in water, special care must be taken when interfacing with SKDetSim, which is used for the detector response, including light propagation and collection. It is important to avoid SKDetSim doing a parallel generation of muon-induced showers, which would lead to a double counting of isotopes. In this study, we therefore focus on observables that are particularly robust against possible mismodeling of the shower Cherenkov light pattern: the yields of the spallation-produced isotopes, and the characteristic neutron clouds discussed in Sec.~\ref{sec:hadronic_spall}. Isotope decays and neutron captures can indeed be simulated in isolation from their parent muons in SKDetSim, and will hence not lead to unwanted interactions. Muon-induced showers still need to be simulated, as they affect the reconstruction of the muon track and hence neutron identification; however this effect on reconstruction is limited. The impact of shower mismodeling can therefore be mitigated using a few simple steps. The following points describe the interfacing between FLUKA and SKDetSim for a typical event, which consists in a  primary muon, a hadronic shower, potential neutron captures and isotopes decays.

An essential change in the input card is the deactivation of muon-nucleus interactions (GEANT-MUNU is set to zero) to prevent the muon from inducing extra showers. 
Thus, the muon will averagely behave only as an ionizing particle. Radiative losses through interactions with atomic nuclei are deactivated but processes such as bremsstrahlung and pair production are still possible and their importance increases with muon energy. 


Together with the muon, particles produced in hadronic and electromagnetic showers in FLUKA are injected at $t = d/c_{\mbox{\tiny vac}}$, where $d$ is the distance from the muon entry point to the shower particle. This typically corresponds to a few nanoseconds. 
Here again, in order to avoid double-counting, we deactivate photofission as well as secondary particle generation for all hadronic interactions in SKDetSim, that is used only to model electromagnetic~(EM) processes, including the emission of Cherenkov light. Conversely, we only inject particles from FLUKA that are commonly produced in hadronic showers and can lead to the prompt emission of Cherenkov light, that is, $\gamma$-rays, pions, and kaons. The only exception occurs if a radioactive isotope is produced in a shower initiated by electromagnetic processes, e.g. by interactions involving electrons, photons, or positrons. In this case, the particles initiating the shower are also injected into SKDetSim, provided their energy is larger than 0.1~GeV since no isotope production has been observed below this threshold in previous simulation studies~\cite{BLi_2}. This scenario is however particularly rare and the impact of these extra particles on the muon light pattern will be extremely limited. Table \ref{tab:processes_FLUKA_SKDETSIM} summarizes the processes treated in FLUKA and SKDetSim respectively. 
\begin{table}
	\begin{center}
		\caption{\label{tab:processes_FLUKA_SKDETSIM}The Table summarizes the main processes activated and deactivated in SKDetSim and the particles generated or ignored in FLUKA and injected in SKDetSim. More description in the text.}
		\begin{tabular}{cc}
			
			
			\toprule
			\multicolumn{2}{c@{\quad}}{SKDetSim} \\
			Deactivated & Activated    \\
			\hline
			Photofission & EM interactions \\
			Secondary part.  & Cherenkov radiation  \\
			generation in inel. int.& \\
			& Decays \\
	
			\toprule
			\multicolumn{2}{c@{\quad}}{FLUKA}\\
		     Generated & Ignored      \\
			\hline
			 EM show. w/ spall. & EM show. w/o spall. \\
			 (if $E>0.1$ GeV) &  \\
		 $\gamma, \pi, K$ from inel. int. &  \\
			 $\gamma$ from $n$ captures &\\
			 Isotope decay prod. & \\
			\bottomrule        % LLMM botrule -> bottomrule
			
		\end{tabular}
	\end{center}
\end{table} 


While the Cherenkov light emission from pions and $\gamma$-rays coincides with the one from muon ionization, neutron capture typically occurs over much larger time scales. For a given muon, we therefore treat each neutron capture separately in SKDetSim. Since neutrons have already been propagated by FLUKA, we directly simulate a $2.2$~MeV $\gamma$-ray at each capture vertex. 


Finally, using the isotope production vertices and decay times given by FLUKA, we simulate isotope decay products in SKDetSim.

Each isotope decay can produce either a $\gamma$ or a $\gamma$- and a $\beta$-ray, sometimes followed by a neutron. In what follows we will consider only products leading to prompt Cherenkov light emission, as their reconstructed energy distribution will affect isotope yield measurements.


Each of the steps described above requires modeling not only the signal, but also the noise in the detector. For muons, shower particles, and isotope decays, we use the modeling of the PMT dark noise from SKDetSim, based on regular measurements made over the SK-IV period. The treatment of neutron capture signals is more complex; this signal in water is particularly weak and the predicted performance of the neutron tagging algorithm is highly sensitive to dark noise modeling. Therefore, we inject noise samples from data into the signal simulation results. These samples were taken at different times over the whole SK-IV using a random trigger, and can hence be used to reflect the time variations of the noise in the detector. Finally, the muon, neutron capture, and radioactivity events undergo the same reconstruction and reduction steps as the ones described in Sec.~\ref{sec:SKdetector} and \ref{sec:hadronic_spall} for data. WIT triggers were used to reconstruct neutron captures.


