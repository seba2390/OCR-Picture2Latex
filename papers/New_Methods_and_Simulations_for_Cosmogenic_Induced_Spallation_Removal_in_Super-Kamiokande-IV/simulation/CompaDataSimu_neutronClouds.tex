\subsection{Neutron clouds}
Using the cuts described in Sec.~\ref{sec:neutspall}, we identify neutron clouds in the SK-IV data and in the simulation. We use the simulation to evaluate the performance of the neutron tagging algorithm, then characterize the shapes of the neutron clouds as well as their multiplicities.

\subsubsection{Neutron tagging algorithm performance}
We use the MC true neutron capture times from showering muons to determine the neutron trigger efficiency and reconstruction accuracy. We consider a neutron to be correctly reconstructed if a signal is found within $50$~ns of the simulated capture time. The tagging efficiencies associated with the WIT trigger and the cuts described in Sec.~\ref{sec:softtrig} and Sec.~\ref{sec:neutspall} are shown in Table~\ref{tab:tab_efficiency}. The final neutron identification efficiency is around $6.5\%$. This low efficiency is due to the weakness of the neutron capture signal ---that is often indistinguishable from dark noise--- and to our lack of a dedicated trigger to save PMT hits following muons. The associated mistag rate can be estimated using random trigger data, and evaluating the efficiency of the $l_t$ cut using the minimum ionizing muon sample. We find a rate of $0.044 \pm 0.001$ tagged fake neutrons per EM shower muon. Conversely, the rate of real neutrons tagged per hadron-producing muon is $0.240 \pm 0.005$.

\begin{table}
	\begin{center}
	\caption{\label{tab:tab_efficiency}Neutron tagging efficiencies for different selection criteria applied sequentially: WIT trigger, proximity to the muon track, timing cut to remove the primary muon signal and the associated after-pulse, goodness-of-fit cut, and reconstructed energy ($E_n$) cut. We consider a simulated neutron to be tagged if its true capture time lies within 50~ns of the reconstructed capture time of a neutron candidate.}
\begin{tabular}{cc}
	\toprule
%	& \\
	Selection requirements & Efficiency values ($\%$)      \\ 
%	&\\
	\hline
	WIT trigger & 13.3 $\pm$ 0.1 \\ 
	%	\hline 
	20~$\mu\mathrm{s}<  \Delta t < 500~\mu\mathrm{s}$ & 89.4 $\pm$ 0.4 \\ 
%	\hline 
	$l_t<500$~cm & 76.7 $\pm$ 0.3 \\ 
%	\hline 
	FV  & 93.3 $\pm$ 0.5 \\ 
%	\hline 
	Fit quality & 76.9 $\pm$ 0.4 \\ 
%	\hline 
	$E_n <$ 5~MeV & 99.9 $\pm$ 0.6 \\ 
	\bottomrule        % LLMM botrule -> bottomrule
\end{tabular}
\end{center}
%true neutrons from the Monte Carlo simulation with candidates which are reconstructed within 200 ns from the true capture time in order to remove candidates uncorrelated to the simulated muon.}
\end{table} 
\subsubsection{Neutron cloud shapes}
We define the shape of a neutron cloud using the $l_t$ and $l_{\mbox{\tiny LONG}}$ observables defined in Sec.~\ref{sec:vardef} and shown in Fig.~\ref{fig:vardia}. For the latter, in order to compare multiple clouds, we use the average $l_{\mbox{\tiny LONG}}$ of the cloud as a reference and consider $\Delta l_{\mbox{\tiny LONG}} = l_{\mbox{\tiny LONG}} - \langle l_{\mbox{\tiny LONG}}\rangle$. In what follows we define a neutron cloud as a cluster of two or more reconstructed neutrons. In the simulation these reconstructed clouds represent only about $5\%$ of all real neutron clouds, due to the low neutron identification efficiency.

In spite of the low neutron mistag rate, due to the large fraction of minimum ionizing muons the data sample will be contaminated by non-negligible contributions from fake neutrons. Estimating the contribution from these fake neutrons requires determining the fraction of muons not leading to hadronic showers (referred to as electromagnetic-only muons at the beginning of this section), which is determined by nuclear effects that are difficult to model accurately~\footnote{For example, as will be shown in Sec.~\ref{sec:yields}, we expect $\mathcal{O}(100\%)$ uncertainties on the predicted isotope yields.}. This fraction, however, can be readily extracted from data since fake neutrons will populate the tails of the $l_t$ and $\Delta l_{\mbox{\tiny LONG}}$ distributions while contributions from real neutrons, namely neutrons produced in muon induced showers in the simulation, will dominate at small distances. At SK, with a muon rate of about 2~Hz, only a few months of data taking are needed for this estimate. 

For this study, we evaluate the fraction of EM muons by fitting the predicted $l_t$ and $\Delta l_{\mbox{\tiny LONG}}$ distributions to the SK-IV data. We perform a separate $\chi^2$ fit for each observable in order to evaluate the robustness of our model and find that the best-fit fractions of muons without hadronic showers are $96$\% ($\chi^2/NDF = 1.1$, with NDF being the number of degrees of freedom) and $97\%$ ($\chi^2/NDF = 1.6$) for $l_t$ and $\Delta l_{\mbox{\tiny LONG}}$ respectively. These values are compatible with each other, but larger than the FLUKA prediction of $89\%$. For this analysis we use a fraction of $96.5\%$ and treat the difference between the fit results for the $l_t$ and $\Delta l_{\mbox{\tiny LONG}}$ as a systematic uncertainty. Table~\ref{tab:tab_ratesmu} shows the resulting muon rates for different tagged neutron multiplicities.
%associated with different numbers of tagged neutrons, if $96.5\%$ of EM muons is assumed. 
\begin{table}
	\begin{center}
		\caption{\label{tab:tab_ratesmu}Table with muon rates in SK if 0 neutrons are tagged, at least 1 neutron is tagged and at least 2 neutrons are tagged (namely a neutron cloud is found). Muon rates are expressed in Hz and expressed separately for hadron producing and electromagnetic only muons, where the fraction of the second sample is extracted from $\chi^2$ as described in the main text. Only through going muons are considered.}
		\begin{tabular}{cccc}
			\toprule 
			&  & Muon rate [Hz] if:  &  \\  
			& 0 $n$ tagged & $\ge$ 1 $n$ tagged  & $\ge$ 2 $n$ tagged  \\ 
			\hline
			Hadr. muons & 0.051 & 0.0088 & 0.0023 \\ 
			EM muons & 1.48 & 0.066 & 0.0018 \\ 
			\bottomrule        % LLMM botrule -> bottomrule
		\end{tabular}
	\end{center}
\end{table} 

\begin{tabular}{|c|c|c|c|}
 
\end{tabular} 

The distributions of $l_t$ and $\Delta l_{\mbox{\tiny LONG}}$ for all neutrons are shown in Fig.~\ref{fig:ncloudsimu} for the simulation. Here the uncertainties combine both the statistical uncertainties and the systematics associated with the different optimal EM muon fractions described in the previous paragraph. As expected, fake neutrons are associated with larger distances as they can be observed anywhere within $5$~m of the muon track. The real neutron component overwhelmingly dominates for $\Delta l_{\mbox{\tiny LONG}} < 5$~m and $l_t < 3$~m.

\begin{figure}
    \centering
    %\includegraphics[width=\linewidth]{spallpaper/simulation/figures/NEUTCLOUD_SHOWMIP_compare_Ll_i-Ll_av_fractions_forpaper2.pdf}
    %\includegraphics[width=\linewidth]{spallpaper/simulation/figures/NEUTCLOUD_SHOWMIP_compare_Lt_vs_ScottData_fraction_forpaper2.pdf}
    \includegraphics[width=\linewidth]{./simulation/figures/Lt_fractions_bw.eps}
    \includegraphics[width=\linewidth]{./simulation/figures/Ll_fractions_bw_new.eps}
    \caption{$l_t$ (top) and $\Delta l_{\mbox{\tiny LONG}}$ (bottom) distributions for neutrons belonging to a cluster of multiplicity larger than or equal to two, for simulation results. The fraction of muons that do not produce hadronic showers is set to $96.5\%$. The black histograms with solid line represent contributions from all reconstructed neutrons while the gray filled and black filled distributions show the true and fake neutron components respectively.}
    \label{fig:ncloudsimu}
\end{figure}

Figure~\ref{fig:neutroncloudcompare} shows the $l_t^2$ and $\Delta l_{\mbox{\tiny LONG}}$ distributions for the SK-IV data and the simulation. The observed neutron clouds have an elongated elliptical shape, with average transverse and longitudinal extensions of $3$~m and $5$~m respectively. For distances of less than about 5~m, where contributions from real neutrons dominate, the predictions differ from the data by at most $15\%$. This excellent agreement motivates the use of a FLUKA-based simulation to predict neutron cloud shapes and optimize the associated cuts for future spallation analyses, notably at SK-Gd and Hyper-Kamiokande~\cite{HyperK}. At SK-Gd in particular, these simulation-based studies will allow to significantly reshape the spallation reduction procedure, as gadolinium doping will sizably increase the neutron identification efficiency.


\begin{figure}[htb]
    \centering
    %\includegraphics[width=\linewidth]{spallpaper/simulation/figures/NEUTCLOUD_SHOWMIP_compare_Ll_i-Ll_av_withresiduals2.pdf}
    %\includegraphics[width=\linewidth]{spallpaper/simulation/figures/NEUTCLOUD_SHOWMIP_compare_Ltsq_vs_ScottData_withresiduals2.pdf}
   \includegraphics[width=\linewidth]{./simulation/figures/Ltsq_MCData.eps}
    \includegraphics[width=\linewidth]{./simulation/figures/Ll_MCData_new.eps}
    
    
    \caption{$l_t^2$ (top) and $\Delta l_{\mbox{\tiny LONG}}$ (bottom) distributions for neutrons belonging to a cluster of multiplicity larger than or equal to two, for simulations (black) and data (red). The fraction of muons that do not produce hadronic showers is set to $96.5\%$. The dotted lines the average transverse and longitudinal cloud extensions of $3$~m and $5$~m respectively.}
    \label{fig:neutroncloudcompare}
\end{figure}

\subsubsection{Neutron multiplicity}
We finally estimate the number of reconstructed neutrons associated with the muons in both the data and simulation samples. Here, as in the previous section we use an EM muon fraction of $96.5\%$ and treat possible mismodeling of this fraction as a systematic uncertainty. The neutron cloud multiplicities for both simulation and data are shown in Fig.~\ref{fig:neutmult}. The abundance of low-multiplicity clouds is due to both fake neutron contributions and the low efficiency of the neutron tagging algorithm. For neutron clouds with multiplicities lower than $10$, simulation and data show reasonable agreement. %Conversely, the amount of neutron clouds with multiplicities larger than $10$ is strongly overestimated in the simulation. 
Conversely, for multiplicities larger than $10$, FLUKA fails to accurately simulate the tails of the data distribution. Note, however, that such large clouds are typically associated with shower-producing hundreds, sometimes up to thousands of neutrons. Muons associated with these high-multiplicity showers are not only rare but also deposit a high amount of light in the detector and are hence easier to identify using other observables, such as for example the residual charge $Q_{res}$ that will be introduced in Sec.~\ref{sec:spaloglike}. In any case, neutron multiplicity distributions from data will be used to improve the simulation in future.
%They are hence easy to identify by considering more low-level observables such as the charge deposited in the detector. 
Our results hence demonstrate the ability of FLUKA-based simulations to accurately model hadronic showers for the types of muons that need most to be studied in future SK analyses.
\begin{figure}
    \centering
    %\includegraphics[width=\linewidth]{spallpaper/simulation/figures/NEUTCLOUD_SHOWMIP_compare_multiplicity_bestchisq_witherrors_and systematicerr_forpaper.pdf}
    %\includegraphics[width=\linewidth]{spallpaper/simulation/figures/NEUTCLOUD_SHOWMIP_compare_multiplicity_bestchisq_witherrors_andsystematicerr_500_forpaper.pdf}
    \includegraphics[width=\linewidth]{./simulation/figures/mult_2_10_log}
    \includegraphics[width=\linewidth]{./simulation/figures/mult_11_45}
    \caption{Neutron multiplicity distributions for simulation~(black) and data~(red) with the EM muon contribution shown for the simulation~(filled gray), this contributes only below 3 neutrons multiplicity. We show these distributions separately for neutron multiplicities from 2 up to $10$, in log scale~(top) and from 11 to $45$ in linear scale~(bottom).}
    \label{fig:neutmult}
\end{figure}
