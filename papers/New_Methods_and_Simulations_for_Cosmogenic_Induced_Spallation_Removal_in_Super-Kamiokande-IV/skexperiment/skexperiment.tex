\section{Super-Kamiokande Experiment}
\label{sec:SKdetector}

Located in Gifu Prefecture, Japan, the SK experiment is a 50~kton cylindrical water Cherenkov detector, with $2700$~m water equivalent overburden. Major changes in the detector or its electronics define the phases of the experiment~\cite{Super-Kamiokande:2002weg}. This analysis is based on the SK-IV data-taking phase, the longest running phase of SK (August 2008--May 2018). The experiment is composed of an optically separated inner detector (ID) with 11,129 20-inch PMTs and an outer detector (OD) with 1885 8-inch PMTs from Hamamatsu, with the inner detector having a height of 36.2~m and radius of 16.9~m (32.5~kton). The OD acts as a buffer for backgrounds emanating from the surrounding rock and a cosmic-ray veto. The information from the OD is sometimes used to fit and categorize cosmic-rays as well. Inside the ID a fiducial volume (FV) is defined by selecting events with vertices located at least 2~m away from the ID boundary in order to reduce backgrounds arising from radioactivity in the surrounding rock, the PMTs, their stainless steel support structure as well as the tank.

The detector response has been modeled using various calibration procedures, a detailed description of which can be found in~\cite{skcalib}. The two sources most relevant for low energy analyses are a linear accelerator~(LINAC) and a
Deuterium-Tritium~(DT) neutron generator. 
The LINAC calibration checks the absolute energy scale as well as vertex and direction reconstruction, by injecting mono-energetic electrons into SK at various locations, as described in~\cite{16Ncalibrationsource}. 
The DT generator produces 14~MeV neutrons from DT fusion; they in turn produce $^{16}$N isotopes from $\mathrm{^{16}O}$ via $(n, p)$ reactions, as described in~\cite{16Ncalibrationsource}. The $^{16}$N decays are used to check the absolute energy scale. The DT calibration is much easier to perform compared to LINAC calibration and therefore the DT calibrations are done at regular time intervals while LINAC calibrations are only done once every few years. Also, LINAC electrons are limited to the downward direction while the $^{16}$N decays isotropically.