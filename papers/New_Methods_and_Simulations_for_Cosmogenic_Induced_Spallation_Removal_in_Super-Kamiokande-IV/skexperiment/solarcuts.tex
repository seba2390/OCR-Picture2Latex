\subsection{Reconstruction of low energy events}
\label{sec:lowereco}
Events are reconstructed from their hit timing (vertex), hit pattern (direction), and the effective number of hits observed (energy). Events below 19.5 MeV only travel several cm within SK and are treated as a point source. The vertex is reconstructed by maximizing a likelihood dependent on the timing residuals of the hit PMTs, $\tau_i$:
\begin{equation}
    \tau_i = t_i - t_{TOF} - t_0
\label{eq:timeres}
\end{equation}

where $t_i$ is the time of the PMT hit, $t_{TOF}$ is the time of flight from the test vertex to the PMT, and $t_0$ is the event time. This likelihood is unbinned, and therefore it cannot be used to evaluate the goodness of the vertex fit~($g_t$). 

To create a goodness of fit~($g_t$) from the timing residuals $\tau_i$ the weighted sum of Gaussian functions $G(\tau_i|\sigma)=\exp\left[-0.5\left(\tau_i/\sigma\right)^2\right]$ is used:
\begin{equation}
    g_t=\sum_i^{N_{hit}}W_i G(\tau_i|\sigma)
\label{eq:gvt}
\end{equation}
where $\sigma=5$~ns is the width of the Gaussian, obtained by combining the single photoelectron PMT timing resolution of 3~ns to the effective time smearing due to light scattering and reflection. The weights $W_i$ are $W_i=G(\tau_i|\omega)/(\sum_j G(\tau_j|\omega))$
with a ``weight width" of $\omega=60$~ns.

Event direction reconstruction is a maximum likelihood method comparing data and MC simulation of the PMT hit pattern caused by Cherenkov cones. This likelihood is dependent on reconstructed energy and the angle between the event direction and the direction to individual PMTs. Using the reconstructed direction, the azimuthal symmetry of the PMT hit pattern is probed with the goodness $g_p$, a KS test:
\begin{equation}
\begin{gathered}
g_p = \frac{\max[\phi^{uni}_i-\phi^{data}_i] - \min[\phi^{uni}_i-\phi^{data}_i]}{2\pi}, \\
\phi^{uni}_i = \frac{2\pi\cdot i}{N_{hit}},
\end{gathered}
\label{eq:gdir}
\end{equation}

\noindent where $\phi^{uni}$ is the angle for evenly spaced hits around the event direction and $\phi^{data}$ is the actual hit angle around event direction. Like $g_t$, $g_p$ also tests the quality of the vertex reconstruction: a badly misplaced vertex often presents the direction fitter with a Cherenkov cone pattern appearing too small (or too big), which implies an accumulation of hit PMTs on ``one side" of the best-fit direction.

Finally, for the energy reconstruction, we evaluate the photons' times of flight from the reconstructed vertex to the hit PMTs and subtract them from the measured arrival times. We then define the effective number of hits $N_{\mbox{\tiny eff}}$ as the maximal number of hits in a 50~ns coincidence window.
This number is then corrected for water transparency, the angle of incidence to the PMTs and photocathode coverage, dark noise rate, PMT gain over the course of SK-IV, PMT occupancy around a hit, the PMT quantum efficiency, and the fraction of live PMTs. The energy is then calculated from a fifth order polynomial dependent on $N_{\mbox{\tiny eff}}$ for energies in the solar neutrino range. The energy reconstruction assumes an electron interaction. This is important to note as neutron captures on hydrogen have a single $\gamma$ which creates less light than a 2.2 MeV electron would. Within this paper the energy of events will be given in terms of the kinetic energy of an electron with equivalent light yield.

