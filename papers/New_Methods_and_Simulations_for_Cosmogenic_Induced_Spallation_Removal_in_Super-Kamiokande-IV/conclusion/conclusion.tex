\section{conclusion}
\label{sec:conclusion}
In this paper, we presented new techniques to reduce spallation backgrounds for low energy analyses at SK, as well as the first realistic spallation simulation in the detector. We notably developed algorithms locating muon-induced hadronic showers, both by improving the reconstruction of the energy deposited along the muon track, and by identifying neutrons using a recently-implemented low energy trigger. New spallation cuts based on these algorithms allow to reduce the deadtime of the solar neutrino analysis by a factor of two, allowing a gain of the equivalent of one year of exposure at SK-IV. Moreover, the profiles of the neutron clouds produced in muon-induced hadronic showers are well reproduced by the spallation simulation, motivating its use to develop spallation reduction algorithms for future analyses. 

In addition to developing new spallation cuts, we computed the yields of the most abundant spallation isotopes at SK-IV, updating the study presented in~\cite{SKspall_zhang} with a 50\% increase in exposure. For the isotopes with the highest production rates, the yields can be determined with a precision of a few percent. Overall, the yields predicted by our spallation simulation lie within a factor of two of the observed values, well within the uncertainties associated with hadron production models. This study also demonstrated that identifying neutron captures associated with muons allows to build spallation-rich samples while keeping the relative fractions of the most abundant isotopes stable.

A central piece of the spallation studies described in this paper is the identification of neutrons produced in muon-induced showers. At SK-IV, the performance of our neutron tagging algorithm has been limited by the low livetime of the associated trigger, and the weakness of the neutron capture signal. At SK-Gd, however, the neutron capture visibility will be significantly increased due to gadolinium doping. Hence, algorithms based on neutron clouds will become a key component of the upcoming spallation reduction algorithms. In this context, the simulation presented in this paper will be instrumental in designing future analysis strategies. This paper thus demonstrates that, beyond neutrino-antineutrino discrimination, neutron tagging will impact significantly all low energy neutrino searches at SK.

