\section{Isotopes Yields}
\label{sec:yields}
In this section, we estimate the spallation isotope yields at SK and compare them with the MC simulation from Sec.~\ref{sec:simulation}. To this end, we update a previous study performed using 1890 days of SK-IV data~\cite{SKspall_zhang}, which determined the yield of each isotope using their different half-lives, by fitting the distribution of the $\Delta t$ observable defined in Sec.~\ref{sec:spaloglike}. In addition to an increased livetime of $2970$ days, we update the selection criteria for the spallation event samples, as well as the computation of the cut efficiencies and isotope decay spectra.  


\subsection{Selection of spallation isotopes}
\label{sec:preselyield}
To build a spallation-rich sample, we select events passing the noise reduction steps outlined in~\cite{newsolarpaper} and apply the following quality cuts:
\begin{itemize} 
    \item Events with reconstructed vertex more than 200~cm from the ID wall~(FV cut)
    \item Events less than 50~$\mu$s of the muon were rejected to remove cosmic $\mu$-$e$ decays as well as PMT afterpulsing. 
\end{itemize}

In addition to the noise reduction cuts, we apply cuts designed to increase the fraction of spallation events in the sample. First, we require the event kinetic energies to lie between 6 and 24.5~MeV. This window ranges from the energy at which spallation starts dominating over intrinsic radioactivity to the highest possible energy of the isotope decay products. 


To fit the time difference between pairs of spallation isotopes and their parent muons, we pair spallation candidates with muons observed up to 100~s before them. Here, we extended the $\Delta t$ range compared to the solar analysis from Sec.~\ref{sec:solaranalysis} in order to include all long-lived isotopes. Additionally, we consider only muons with a reconstructed track of at least 200~cm. For stopping muons, we compute $l_t$ by extending the track through the entire detector; for multiple muons we considered only the primary track. Due to the 200~cm requirement, corner clipping muons will not be taken into account. Their contribution to the spallation background in the fiducial volume is however negligible; the total number of events found within 500~cm of a corner clipping muon track is about $10^{-5}$ of the total number of spallation events.

After pairing spallation candidates with suitable muons as described above, we build two separate samples with a high fraction of spallation pairs. In the first sample, we apply the same selection cut as in the previous study by Ref.~\cite{SKspall_zhang}, requiring $lt<200$~cm. In the second sample, we select muons associated with three or more tagged neutrons. In Sec.~\ref{sec:hadronic_data} events were required to have at least two events triggered in WIT within 500~$\mu$s and 500~cm, and one quality event. For this study, we instead require at least three events triggered by WIT to further increase purity. The cloud location is still only defined by the weighted center of quality neutron cloud events. Then, we pair the selected neutron clouds with spallation candidates verifying the conditions listed in table~\ref{tab:cloudtab}. Within this table, all cuts requiring $\Delta t < 60$~s were used for pairs with $\Delta t$ up to 100~s.

\subsection{Spallation Yield Fit}
\label{sec:yieldfit}
To extract isotope yields from data, we fit the decay time distributions of the 10 most abundant isotopes found in~\cite{SKspall_zhang,BLi_1} to time difference $\Delta t$ distribution using the spallation sample described above. Specifically, we parameterized the time dependence of the total event rate $R_{TOT}$ by a sum of exponentials as follows:

\begin{equation}
    R_{TOT} = \sum_i^n R_i\cdot e^{(-\Delta t/\tau_i)} + const.
\label{eq:allfit}
\end{equation}
where we keep the isotope decay constants $\tau_i$ fixed and fit the production rates $R_i$. For this study, we perform a $\chi^2$ fit analytically over the whole $\Delta t = $ 0--100~s range.  

To mitigate possible degeneracies, we grouped isotopes together when their decay constants were within $10\%$ of each other. For this study, $^8$Li and $^8$B as well as $^9$C and $^8$He were paired together. The decay constant associated with each of these pairs was taken as a weighted average between the two isotopes, using the yields predicted by~\cite{BLi_1}.
The $\tau_i$s used were 1.18~s and 0.181~s for the two pairs respectively. For the case of stopping muons, looking at the individual fit contributions, most spallation products were from those which is normally produced from a neutron interacting with nucleus. The fits are dominated by $^{16}$N, with the next largest contribution being $^{12}$B with an observed raw rate roughly a factor of six smaller. For single through going muons, $^{16}$N is observed roughly twice as often. 

In contrast to the the previous paper from~\cite{SKspall_zhang}, the rate distribution ($\Delta t$ distribution divided by time bin width) was made with logarithmic bins. Also, as mentioned earlier, the maximum $\Delta t$ considered for the fit was extended from 30~s to 100~s in order to account for long-lived isotopes and better constrain the distribution of accidentally coincident muon/spallation isotope candidate pairs. The result of this fit for all 10 isotopes is shown in Fig.~\ref{fig:fitandresidual}. The $\chi^2/NDF$ was 231.5/243 resulting in a p-value of 69\%. Figure~\ref{fig:fitandresidual} shows the fit to the entire time range as well as the residual of the fits. Overall, an excellent agreement between the data and the fit can be observed and the rates of the most abundant isotopes can be determined with percent-level precision.

\begin{figure*}
    \centering
    \includegraphics[width=\textwidth]{./yields/figures/fit_v_resid.eps}
    \caption{Left: Full $\Delta t$ distribution from 50~$\mu$s to 100~s. The solid red line corresponds to the entire fit, and the colored lines correspond to the individual isotope fits and background. Right: Fit residuals shown using unscaled data minus the fit value at the bin center multiplied by the with of the corresponding time bin.}
    \label{fig:fitandresidual}
\end{figure*}

\subsection{Spallation Efficiencies}
\label{sec:productionrates}
Obtaining isotope yields from the rates estimated above requires computing the efficiencies of the selection cuts described in~Sec.\ref{sec:preselyield}. These efficiencies were computed using both MC simulations and data. 

The efficiencies of the noise reduction, quality requirements, and energy cuts for the different spallation isotopes have been determined by simulating isotope decays using the SK detector simulation, based on GEANT 3.21~\cite{geant3}. In this simulation, we accounted for the time dependence of the detector properties using measurements of the PMT and water properties taken almost daily. The decay spectra are simulated using Geant~4.10.7~\cite{GEANT41,GEANT42} for most isotopes, taking into account both $\beta$ and $\gamma$ emission. Since the $^8$B decay was mismodeled in Geant4 we modeled its $\beta$ spectrum using tabulated values from~\cite{Winter:2004kf}. The final efficiencies are shown in Table~\ref{tab:spal_eff_simu}.

To determine the efficiency of the $l_t < 200$~cm cut introduced in Sec.~\ref{sec:preselyield}, we combine estimates from the data and from the spallation simulation described in Sec.~\ref{sec:simulation}. The simulation readily gives us efficiencies for individual isotopes paired with single through-going muons, that are shown in Table~\ref{tab:spal_eff_simu} but does not currently allow to study other muon categories. 


\begin{table}[t]
    \centering
        \caption{Spallation sample efficiencies from simulation, for single through-going muons. The second column shows the isotope decay identification efficiencies for the FV, first reduction, and 6 MeV cuts, and the third column shows the efficiencies for the $lt \leq 200$~cm cut. The total $l_t$ efficiency has been obtained by averaging the efficiencies associated with the different isotopes, weighted by their predicted yields.}
    \begin{tabular}{ ccc }
    \toprule \multirow{2}{*}{Isotope}  & FV + 1st red & \multirow{2}{*}{$l_t \leq 200$~cm (\%)}\\
    & + E $\leq$ 6 MeV (\%) &  \\\hline
$\mathrm{^{12}N}$ & 69.0 & 94.9 \\ 
$\mathrm{^{12}B}$ & 54.3 & 89.9 \\ 
$\mathrm{^{8}He}$ & 23.9 & 94.6 \\ 
$\mathrm{^{9}C}$ & 67.4 & 94.0 \\ 
$\mathrm{^{9}Li}$ & 44.4 & 91.8 \\ 
$\mathrm{^{8}Li}$ & 51.8 & 91.4 \\ 
$\mathrm{^{8}B}$ & 48.1 & 92.9 \\ 
$\mathrm{^{15}C}$ & 37.1 & 87.7 \\ 
$\mathrm{^{16}N}$ & 54.8 & 88.9 \\ 
$\mathrm{^{11}Be}$ & 44.7 & 86.5 \\
\hline
Total & & 90.3 $\pm$ 2.8\\
   \bottomrule        % LLMM botrule -> bottomrule
   \end{tabular}
    \label{tab:spal_eff_simu}
\end{table}

We complement this simulation study by extracting the $l_t$ cut efficiency from data for each muon type; we later take the total efficiency to be a weighted average using the observed relative contributions of each muon category.
We obtain the efficiencies for each muon type by fitting the $\Delta t$ distributions observed in 50~cm $l_t$ bins using the method described in Sec.~\ref{sec:yieldfit}. The 50~cm bin $\Delta t$ distribution were scanned to determine the largest $l_t$ where evidence for spallation could be found (the “endpoint”). The endpoints for each muon category, the $l_t$ efficiency and the spallation proportion is found in Table~\ref{tab:typeeff}. To combine the final $l_t$ efficiencies we then compute the relative contributions from each muon category to the spallation sample by considering all pairs within $l_t < 2000$~cm ---the maximal distance between spallation isotopes and their reconstructed parent muon track. The resulting total $l_t$ efficiency is calculated to be 73.6\%.

\begin{table}[t]
    \centering
        \caption{This table summarizes the proportion of total spallation production for the different muon categories and their respective efficiencies. 
    The proportion of spallation is measured as the ratio of the integral of the category fits at 2000 cm to the sum of all categories. }
    \begin{tabular}{>{\centering\arraybackslash}m{0.2\linewidth}>{\centering\arraybackslash}m{0.2\linewidth}>{\centering\arraybackslash}m{0.2\linewidth}>{\centering\arraybackslash}m{0.2\linewidth}}
         \toprule
         Category & Endpoint~[cm] & Efficiency~[\%] & Proportion~[\%]\\ \hline
         Single & 500 & 92.3 & 54.6 \\ \hline
         Multiple & 3000 & 49.3 & 43.5 \\ \hline
         Stopping & 400 & 93.4 & 1.9 \\ \hline
         Corner Clipping & N/A & N/A & $<10^{-3}$ \\ \hline 
         All & N/A & 73.6 & 100 \\ \bottomrule        % LLMM botrule -> bottomrule
    \end{tabular}
    \label{tab:typeeff}
\end{table}

A second method identical to the previous results was implemented to validate: 
After applying a pre-cut on $\Delta t$ the efficiency is the ratio of the number of events within the $l_t$ cut value over all events passing the pre-cut. Eight different pre-cut values ranging from 10~ms to 30~s were chosen.
To estimate these efficiencies, a background sample was made using muons found after spallation candidates, as described in Sec.~\ref{sec:spaloglike}, and the $\Delta t$ distribution from this sample was subtracted from the spallation sample distribution. This procedure yields an $l_t$ efficiency of $74.2\pm0.5\%$. The total $l_t$ efficiency is then taken to be $74.0\pm0.7\%$ to account for the discrepancies between the two methods. The $l_t$ efficiency for single through-going muons obtained from data was compared to the average efficiency from the MC simulation and was found to be about $1\sigma$ away. 

\begin{table*}[t]
    \centering
        \caption{Observed and calculated spallation rates and yields using the $l_t$ cut. For the two sets of isotopes that could not be separated, the systematic error covers the range of yields corresponding to changing the relative fraction of an isotope from 0 to 1. The upper limit uses a 90\% confidence level~(C.L.) using the positive systematic error. Compared to the previous results, discovery for $^{15}\mbox{C}$ has been made, and much stronger constraint on the $^{11}\mbox{Be}$ measurement has been made. $^{11}\mbox{Be}$ was at a 1.5$\sigma$ excess. The $^{9}\mbox{C}$/$^{8}\mbox{He}$ and $^{9}\mbox{Li}$ fits for neutron clouds were 98.2\% anti-correlated. The total number of events for the two fit contributions is 65\% the corresponding $l_t^2$ contributions. The calculated relative fractions of $^{8}\mbox{Li}$ and $^{8}\mbox{B}$ in the  $^{8}\mbox{Li}+ {^{8}\mbox{B}}$ sample are 70.1\% and 29.9\% respectively. The calculated relative fractions of $^{9}\mbox{C}$ and $^{8}\mbox{He}$ in the  $^{9}\mbox{C} + {^{8}\mbox{He}}$ sample are 78.8\% and 21.2\% respectively. %*Individual contributions within the simulation
    }
    %\begin{tabular}{|*{14}{c|}}
    \begin{tabular}{c|cc|ccc|cc}
    \toprule   & \multicolumn{2}{c|}{\makecell{Yields \\ $[10^{-7}\mbox{cm}^2\mu^{-1}\mbox{g}^{-1}]$}} & \multicolumn{3}{c|}{Rates [$\mbox{kton}^{-1}\mbox{day}^{-1}$]} & \multirow{2}{*}{\makecell{Neutron data\\
    Fraction of \\ $l_t$ data}} & \multirow{2}{*}{\makecell{Calculated\\Isotope\\Fractions}}  \\%\cline{1-6}
    Isotope  & Calculated & Observed & \makecell{Total Rate \\ ($l_t$ data)} & \makecell{Raw Rate \\ ($l_t$ data)} & \makecell{ Raw Rate \\ (neutron data)} & &\\\hline
$^{12}\mbox{N}$ & 0.92 &1.72 & $3.04\pm0.06\pm0.028$ &1.55  & 1.08 &  70\% & 2.3\% \\
$^{12}\mbox{B}$ & 8.6 & 12.9 & $22.86\pm0.11\pm0.21$  & 9.19 & 5.95 & 65\%  & 21.1\% \\
$^{9}\mbox{C}$/$^{8}\mbox{He}$ & 0.8 & \textless0.61 & \textless1.08 & 0.11 & 0.20 & 176\% & -- \\
$^{9}\mbox{Li}$ & 1.5 & 0.67 & $1.19\pm0.33\pm0.010$ &0.39  & 0.13 & 34\% & 3.7\% \\
$^{8}\mbox{Li}$/$^{8}\mbox{B}$ & 13.4 & 5.11 & $9.04\pm0.17^{+0.60}_{-1.1}$ & 3.69 & 2.59 & 70\% & 32.8\% \\
$^{15}\mbox{C}$ & 0.55 & 1.57 & $2.78\pm0.45\pm0.032$ & 0.76 & 0.37 & 49\% & 1.3\% \\
$^{16}\mbox{N}$ & 14.5 & 27.3 & $48.43\pm0.60\pm0.49$  & 19.64 & 12.01 & 61\% & 35.3\% \\
$^{11}\mbox{Be}$ & 0.61 & \textless1.05 & \textless1.9  & 0.33 & 0.19 & 56\% & -- \\
   \bottomrule        % LLMM botrule -> bottomrule
   \end{tabular}
    \label{tab:isotopeyield}
\end{table*}
Finally, the isotope-dependence of the $l_t$ distribution was included to the measurement by scaling each isotope efficiency from MC by the weighted average of all isotopes. This factor was then used to scale the single through going measurement from data, and half of its effect was used to scale the multiple and stopping muon case. Since the multiple muon case was not performed in MC, this allowed for the difference to full or no isotope dependence to be covered in an isotope dependent systematic error. This error was relatively small for most isotopes, with only $^{15}\mbox{C}$ and $^{11}\mbox{Be}$ having an effect greater than the full $l_t$ systematic error.  
Here, using the simulation allows to refine the $l_t$ efficiency estimate performed in~\cite{SKspall_zhang}, where the isotope dependence was covered by a $\sim 4\%$ systematic uncertainties.


\subsection{Rate and Yields Calculation}
To calculate the total rates of the individual isotopes, the $l_t$ efficiency obtained from the data is combined with the efficiency associated with the noise, quality, and energy cuts. Raw rates for each isotope obtained from the fits in Sec~\ref{sec:yieldfit} are then corrected by these efficiencies to obtain the total production rates at SK-IV. For the isotopes that were paired together for the $\Delta t$ fit, the contributions of the isotopes were varied from 0 to 1 and used as a systematic error. These results are shown in Table~\ref{tab:isotopeyield}. 

The rates extracted from the data are compared to the FLUKA-based simulation described in Sec.~\ref{sec:simulation}. A simulated spallation sample of $1.362\times~10^8$ initial muons is generated in order to accumulate enough statistics for the low yield isotopes. The predicted rates are shown alongside the observed results in Table~\ref{tab:isotopeyield}. 


Finally, the isotope yields are obtained by rescaling the total rates computed above for each isotope as follows:
\begin{equation}
    Y_i = \frac{R_i\cdot FV}{R_\mu\cdot\rho\cdot L_\mu}
\end{equation}
where $\rho$ is the density of water, $R_i$ is the total rate of the $i^{th}$ isotope at SK-IV, $R_\mu$ is the muon rate~(2.00~Hz), $L_\mu$ is the average length of reconstructed muon tracks, and $FV$ is the fiducial volume of the detector. Table~\ref{tab:isotopeyield} shows the final isotope yields for the data.


\subsection{Isotope study with neutron clouds}
\label{sec:neuteff}
In addition to updating the study performed in~\cite{SKspall_zhang}, we investigate the impact of neutron cloud cuts on isotope rates. Here, as stated in Sec.~\ref{sec:preselyield}, we consider a sample of spallation candidates paired with muons associated with at least three tagged neutrons. The fitting procedure described in Sec.~\ref{sec:yieldfit} is then performed for all isotopes, giving a $\chi^2$/dof of 252.6/243, which corresponds to a p-value of 0.323. Then, the final rates are scaled to account for the lower live time of the WIT trigger and allow a comparison with the $lt< 200$~cm sample. The scaled rates without efficiency corrections (also called raw rates) are shown in Table~\ref{tab:isotopeyield} for both the neutron cloud and the $l_t$ sample.

As shown in Table~\ref{tab:isotopeyield}, the fitted rates for the neutron cloud sample range between 33\% and 193\% of the rates found for the $l_t$ sample. The largest discrepancies between the two rates are seen for $^8$He/$^9$C and $^9$Li. For these subdominant isotopes, however, the precision of the fit is limited by statistics. Moreover, these isotopes have similar half-lives, which leads to degeneracies; the rates for the $^8$He/$^9$C and for $^9$Li are in fact found to be 98.3\% anti-correlated when considering the covariance of the fit parameters. For the most abundant isotopes, on the other hand, the ratio between the rates in the neutron cloud and $l_t$ sample remains around 60--70\%. The stability of this ratio highlights the correlation between the neutron cloud and $l_t$ cuts, as neutron cloud cuts also make use of the isotope distance to the muon track. 




