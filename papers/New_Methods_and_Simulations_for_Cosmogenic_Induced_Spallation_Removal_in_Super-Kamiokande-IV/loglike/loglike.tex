\subsection{Spallation Log Likelihood}
\label{sec:spaloglike}
Multiple spallation in tandem with tagging neutrons from hadronic showers allow to remove 65\% of spallation events with 2.4\% deadtime. To identify the remaining spallation background events, we update the likelihood cut defined in Sec.~\ref{sec:spaprevious}. In particular, in addition to the $dt$, $l_t$, and $Q_{\mbox{\tiny res}}$ observables, we also consider the difference in longidudinal distance $\Delta l_{\mbox{\tiny LONG}}$ between an isotope and the segment of the muon track associated with the highest amount of light deposited in the detector. This new observable allows to estimate the distance between isotopes and muon-induced showers. In what follows, we describe how we build and parameterize PDFs using low energy events passing the multiple spallation and neutron cloud cuts described above. Similarly to the procedure described in Sec.~\ref{sec:spaprevious} we pair these events with muons within $60$~s~(inverting the time sequence to separate the random coincidence component from the spallation component) to obtain the PDFs distributions for spallation and random pairs. The choices of parameters of the PDFs for all observables are shown in appendix~\ref{appendix_fitparam}.

\subsubsection{Time difference $\Delta t$}
We obtain the $\Delta t$ PDFs following a procedure similar to the one described in Sec.~\ref{sec:yields}, fitting the $\Delta t$ distribution for low energy events found within 2~m of a muon track. The $\Delta t$ distribution for uncorrelated random pairs is flat in this instance is subtracted statistically. The functional form for the spallation PDF is as follows:
\begin{equation}
    \mbox{\it PDF}_{sig}(\Delta t) = \sum_i^7 A_i e^{-\Delta t/\tau_i}
\end{equation}
where $\tau_i$ is the decay constant for the isotope and $A_i$ is the fitted amplitude. 

\subsubsection{Transverse distance $l_t$}
To account for correlations between $l_t$ and other observables we compute PDFs for $l_t$ in 7 $Q_{\mbox{\tiny res}}$ and 3 $\Delta t$ bins. These $\Delta t$ bins are taken to be 0--100~ms, 100~ms--3~s, and 3--60~s, in order to account for the different half-lives of the isotopes. To reflect the amount of phase space available, we express PDFs as a function of $l_t^2$. Their functional forms are as follows:
\begin{gather}
\label{fitltsig}
    PDF_{SIG,l_t}(l_t^2) = \sum_{i=1}^3e^{c_i - p_i\cdot l_t^2}\\
    \label{fitltbg}
    PDF_{BG,l_t}(l_t^2) = 
    \begin{cases}
       p_0 & l_t^2 \leq l_{t0}^2 \\ 
       p_0e^{-p_1(l_t^2 - l_{t0}^2) + p_2(l_t^2-l_{t0}^2)^2}          & l_t^2 > l_{t0}^2 
    \end{cases}
\end{gather}

\noindent where $c_i$ and $p_i$ are the fit parameters. Due to the finite size of the detector, at large $l_t$ the allowed region is no longer cylindrical, so a ``piecewise'' function was defined for the background PDFs, with $l_{t0}^2$ being the point where the function changes. As an example, the $l_t$ distributions and PDFs are shown for the 3--30~s $\Delta t$ and 0.5--1.0~Mpe $Q_{\mbox{\tiny res}}$ bin in Fig.~\ref{fig:lt2both}, for the spallation and random coincidence samples.
%No additional cuts were placed on the events since the PDFs were binned in time and charge. 

\begin{figure*}
    \centering
    \includegraphics[width=\textwidth]{./loglike/figures/ltcompare.eps}
    \caption{Distributions with PDF for the spallation (left, normal time sequence) and random coincidence~(right, inverted time sequence) samples with corresponding fits (solid lines). The distributions shown here are for the 3--30~s $\Delta t$ and 0.5--1.0~Mpe $Q_{\mbox{\tiny res}}$ bin. The analytical forms for the fits are shown in Equations~\ref{fitltsig} and~\ref{fitltbg}.}
    \label{fig:lt2both}
\end{figure*}

\subsubsection{Residual charge $Q_{\mbox{\tiny res}}$}
The residual charge ($Q_{\mbox{\tiny res}}$) is the excess charge observed for a muon event compared to a minimum ionizing particle (MIP) traveling the same distance inside the detector. The MIP muon charge per unit of track length is defined as the peak value of the distribution of the amount of charge deposited by unit track length for single through-going muons. It is evaluated for each run time period and typically lies around $26.78$~photo-electrons~(p.e.) per cm. We obtain the PDFs for this observable by using a sample of low energy events within 2~m and 10~s of a muon. The distributions for spallation and random uncorrelated pairs are shown in Fig.~\ref{fig:qres}. Both signal and background PDFs in the positive $Q_{\mbox{\tiny res}}$ region are a sum of exponential functions:
\begin{equation}
\mbox{\it PDF}(Q_{\mbox{\tiny res}}) =\sum^5_{i=1} e^{c_i - p_i\cdot Q_{\mbox{\tiny res}}}, \ \ Q_{\mbox{\tiny res}} > 0
\end{equation}
where $c_i$ and $p_i$ are the fit parameters for exponential functions. For negative $Q_{\mbox{\tiny res}}$, no analytical form was assumed and a linear interpolation of the sample bins was used. 

\begin{figure}
    \centering
    \includegraphics[width=\linewidth]{./loglike/figures/resq_bw.eps}
    \caption{Signal (red triangles) and Background (blue circles) distributions for the sample used for the $Q_{\mbox{\tiny res}}$ PDF construction.}
    \label{fig:qres}
\end{figure}

\subsubsection{Longitudinal distance $\Delta l_{\mbox{\tiny LONG}}$}
Defining PDFs for the difference in longitudinal length $\Delta l_{\mbox{\tiny LONG}}=l_{\mbox{\tiny LONG}}^{\mbox{\tiny isotope}}-l_{\mbox{\tiny LONG}}^{\mbox{\tiny dE/dx peak}}$ allows to include information about the muon-induced hadronic shower into the likelihood cut. Here, we developed a new method to reconstruct the energy loss of the muon along its track (dE/dx) based on a previously published method~\cite{bib:sksrn123}. Defects and possible improvements to this method were suggested in~Ref.\cite{BLi_3}. The method presented here remedies those defects, although the improvements differ from those suggested by~Ref.~\cite{BLi_3}. Using the entry time of the muon in the detector and the PMT hit pattern, dE/dx is estimated by identifying the points of the muon track verifying the time correlation equation for each PMT hit:
\begin{equation}
t_{PMT} - t_{entry} = l\cdot c_{vac} + d \cdot c_{water} 
\label{eq:dedxsimp}
\end{equation}
where $t_{PMT}$ and $t_{entry}$ are the PMT time and muon entry time respectively, $l$ is the distance from the muon entry point to the point along the track where the light is emitted from, $d$ is the distance from the emission point to the PMT, and $c_{vac}$ and $c_{water}$ are the speeds of light in vacuum and water respectively. The dE/dx is computed for 50~cm segments of the muon track, corresponding roughly to the vertex resolution for events in the energy region of 3.49--19.5~MeV in SK. The simplest approach to estimating dE/dx is to add the charge of each PMT to each bin containing a solution of Eq.~(\ref{eq:dedxsimp}). Here, using the method proposed in~\cite{BLi_1}, we spread the charge from each hit across multiple bins to account for the PMT timing resolution. More specifically, we take the contribution, $g_{ij}$, of the $i^{th}$ PMT to the $j^{th}$ bin to be:
\begin{equation}
g_{ij} = Q_i \cdot \frac{e^{d(l_j + 25)/\lambda}}{S(\theta_{ij},\phi_{ij})} \cdot \frac{f_{ij}}{\sum_k f_{ik}}
\end{equation}
where
\begin{equation}
f_{ij} = \left|\mbox{Erf}\left(\frac{\tau(l_j) - t_i}{\sqrt{2}\sigma_i}\right)- \mbox{Erf}\left(\frac{\tau(l_j+50) - t_i}{\sqrt{2}\sigma_i}\right)\right|, 
\end{equation}
$\sigma_i$ is the timing resolution for the observed charge by the $i^{th}$ PMT, $\tau(l_j)$ is the $t_{PMT}$ that solves Eq.~(\ref{eq:dedxsimp}). For the $j^{th}$ bin boundary's $d$ and $l$, Erf is the standard error function, $Q_j$ is the charge observed by the PMT, the exponential function is water attenuation correction, $S$ is the photocathode coverage correction, and the integral of the sum is normalized to one. This procedure ensures that the charge of a given PMT hit is only counted once. 
Although error functions are used for the integral, since $\tau_j(l)$ is non-linear, the integral is not easily normalized. Care also has to be taken for the shape of $\tau_j(l)$ as it is not monotonic, therefore special cases are implemented to handle scenarios where $\tau(l) - t_i = 0$ and when d$\tau$/dl = 0.

For muons that induce particle showers in addition to minimum ionization, the segment of the muon track associated with the largest dE/dx can indicate the location of these showers. We hence define $\Delta l_{\mbox{\tiny LONG}}=l_{\mbox{\tiny LONG}}^{\mbox{\tiny isotope}}-l_{\mbox{\tiny LONG}}^{\mbox{\tiny dE/dx peak}}$ as the longitudinal distance of an isotope to this segment along the muon track. Distributions of this observable for spallation and uncorrelated pairs are shown in Fig.~\ref{fig:ln}. To define the corresponding PDFs, the $\Delta l_{\mbox{\tiny LONG}}$ distribution for low energy events found within $2$~m and $10$~s of a muon is fitted by a sum of three Gaussians:
\begin{equation}
\mbox{\it PDF}(\Delta l_{\mbox{\tiny long}}) = \sum^3_{i=1} A_ie^{\frac{-(l_{\mbox{\tiny long}} - x_i)^2}{2\sigma_i^2}}
\end{equation}
where $A_i$, $x$, and $\sigma_i$ are the fit parameters for each PDF. For uncorrelated pairs, one of the Gaussian fits was degenerate and dropped from the final form. For minimum ionizing muons, the dE/dx peak is more likely to be at the end of the track, resulting in the background distribution being slightly shifted away from 0.

\begin{figure}
    \centering
    \includegraphics[width=\linewidth]{./loglike/figures/lnlike.eps}
    \caption{$\Delta l_{\mbox{\tiny LONG}}$ distribution used for PDF fit, for the spallation signal (blue triangles) and for spallation accidentals (black circles). The peak of the background distribution is shifted to negative $\Delta l_{\mbox{\tiny LONG}}$ as a result of non-showering muons being more likely to have a dE/dx peak later in the track.}
    \label{fig:ln}
\end{figure}

Using the PDFs defined above, we define a log likelihood function as shown in Eq.~(\ref{eq:loglike}). The distributions of this log likelihood for signal and background are shown in Fig.~\ref{fig:loglike}. To estimate the impact of the cut on this function, we take advantage of the fact that the low energy event sample that we are considering is dominated by spallation and solar events. We can hence readily estimate the background rejection rate of our algorithm by computing the fraction of events with $\cos\theta_{sun}<0$ removed by the likelihood cuts. Conversely, the signal efficiency can be computed by applying cuts on a random sample, where low energy events are paired with muons observed after them. We use these techniques to tune the cut point for the log likelihood, maximizing the signal efficiency for a background rejection rate of 90\%, similar to the one obtained with the previous spallation cut described in Sec.~\ref{sec:spaprevious}. Since WIT was running only during a small fraction of the SK-IV period, we apply different likelihood cuts depending on whether the neutron cloud information is available. The availability of neutron cloud information notably allows to loosen the likelihood cut. 

\begin{figure}
    \centering
    \includegraphics[width=\linewidth]{./loglike/figures/loglike_fixed.eps}
    \caption{Comparison of the $\log\mathcal{L}$ for the signal and background of the non-neutron data period. This figure shows the distributions for the spallation signal (dashed) and spallation accidentals (solid). The vertical dashed line shows the tuned cut value. Since the multiple spallation cut is already applied, only 82\% of remaining spallation is removed to achieve 90\% overall spallation removal effectiveness.}
    \label{fig:loglike}
\end{figure}