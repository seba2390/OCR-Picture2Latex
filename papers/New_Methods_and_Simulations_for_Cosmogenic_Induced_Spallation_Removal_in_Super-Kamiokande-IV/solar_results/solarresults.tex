\subsection{Total Spallation Cut Results}
After tuning the neutron cloud, multiple spallation, and likelihood cuts described throughout this section, we apply spallation reduction to a sample of SK-IV events passing the noise reduction, quality cuts, and pattern likelihood cut described in Sec.~\ref{sec:solaranalysis}.

The total dead times for the periods with and without neutron cloud information are 8.9\% and 10.8\%. Figure~\ref{fig:deadvol} displays the position dependence of the dead time with neutron cloud information. The new spallation cut hence allows to reduce the dead time by up to 55\% compared to the previous analysis.

\begin{figure}
    \centering
    \includegraphics[width=\linewidth]{solar_results/figures/deadvol.eps} % LLMM path expansion needed for arXiv
    \caption{Spallation cut dead time as a function of radius and height. The origin is taken at the center of the detector~\cite{skdetector}. This function combines neutron cloud, multiple spallation and likelihood dead time. The fiducial volume is indicated by the shaded area.}
    \label{fig:deadvol}
\end{figure}

The effect of this new reduction on the SK-IV solar analysis is shown in Fig.~\ref{fig:coscompdiff}, that shows the $\cos\theta_{sun}$ distribution for events passing the new spallation cut and failing the one described in Sec.~\ref{sec:solaranalysis}. The clear peak around $\cos\theta_{sun} = 1$ shows that the new procedure allows to retrieve a sizable number of solar events. In the final sample, this cut allows for an increase of 12.6\% solar neutrino signal events, with a reduction in the relative error on the number of solar events of 6.6\%. Compared to the total SK-IV exposure, retrieving this signal corresponds to an increase of roughly a year of detector running.


\begin{figure}
    \centering
    \includegraphics[width=\linewidth]{./solar_results/figures/cossun_65.eps}
    \caption{Figure showing the difference in events between final solar samples using new spallation cut compared to the previous cut for 5.99 to 19.5 MeV. The peak at the right contains the additional solar neutrino interactions gained with the change in cut. The small excess in the flat region reflects changes to the other solar neutrino cuts and differing interaction with the previous and new spallation cut. The intrinsic radioactivity in the energy region below 6 MeV was rejected by the previous spallation cut due to accidental coincidence.}
    \label{fig:coscompdiff}
\end{figure}
